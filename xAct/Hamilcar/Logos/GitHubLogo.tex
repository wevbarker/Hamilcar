\documentclass[tikz]{standalone}
\usepackage{tikz}
\usetikzlibrary{3d}
\usepackage{etoolbox}
\usepackage{tensor}
\usepackage{xspace}
\usepackage{amsmath}
\usepackage[dvipsnames]{xcolor}
\usepackage{calc}
% Macros

\newcommand{\wb}[1]{{\color[RGB]{0,155,0}{\textbf{\textit{[WB: #1]}}}}}

\newrobustcmd{\NPhy}{%
	\tensor{N}{_{\text{Phy}}}%
}
\newrobustcmd{\NCan}{%
	\tensor{N}{_{\text{Can}}}%
}
\newrobustcmd{\NFirst}{%
	\tensor{N}{_{\text{1st}}}%
}
\newrobustcmd{\NSecond}{%
	\tensor{N}{_{\text{2nd}}}%
}

\newrobustcmd{\ScaleFactor}{%
	a
}
\newrobustcmd{\HubbleNumber}{%
	H
}
\newrobustcmd{\HubbleNumberDot}{%
	\dot{H}
}

\newrobustcmd{\Primary}[1]{\tensor{\Phi}{_{#1}}}
\newrobustcmd{\FirstOrderPrimary}[1]{\tensor{\vphantom{l^{l^2}}\smash{\stackrel{\raisebox{-3pt}{\scalebox{0.55}{$(1)$}}}{\scalebox{0.99}{$\Phi$}}}}{_{#1}}}
\newrobustcmd{\SecondOrderPrimary}[1]{\tensor{\vphantom{l^{l^2}}\smash{\stackrel{\raisebox{-3pt}{\scalebox{0.55}{$(2)$}}}{\scalebox{0.99}{$\Phi$}}}}{_{#1}}}
\newrobustcmd{\Secondary}[1]{\tensor{\Psi}{_{#1}}}
\newrobustcmd{\FirstOrderSecondary}[1]{\tensor{\vphantom{l^{l^2}}\smash{\stackrel{\raisebox{-3pt}{\scalebox{0.55}{$(1)$}}}{\scalebox{0.99}{$\Psi$}}}}{_{#1}}}
\newrobustcmd{\SecondOrderSecondary}[1]{\tensor{\vphantom{l^{l^2}}\smash{\stackrel{\raisebox{-3pt}{\scalebox{0.55}{$(2)$}}}{\scalebox{0.99}{$\Psi$}}}}{_{#1}}}
\newrobustcmd{\ActualSecondary}[1]{\tensor{\Theta}{_{#1}}}
\newrobustcmd{\FirstOrderActualSecondary}[1]{\tensor{\vphantom{l^{l^2}}\smash{\stackrel{\raisebox{-3pt}{\scalebox{0.55}{$(1)$}}}{\scalebox{0.99}{$\Theta$}}}}{_{#1}}}
\newrobustcmd{\SecondOrderActualSecondary}[1]{\tensor{\vphantom{l^{l^2}}\smash{\stackrel{\raisebox{-3pt}{\scalebox{0.55}{$(2)$}}}{\scalebox{0.99}{$\Theta$}}}}{_{#1}}}
\newrobustcmd{\QVar}[1]{\tensor{q}{_{#1}}}
\newrobustcmd{\BackgroundQVar}[1]{\tensor*{\overline{q}}{_{#1}}}
\newrobustcmd{\DeltaQVar}[1]{\tensor*{\delta q}{_{#1}}}
\newrobustcmd{\QVarDot}[1]{\tensor{\dot{q}}{_{#1}}}
\newrobustcmd{\DeltaQVarDot}[1]{\tensor*{\delta \dot{q}}{_{#1}}}
\newrobustcmd{\QVarDash}[1]{\tensor{q'}{_{#1}}}
\newrobustcmd{\DeltaQVarDash}[1]{\tensor*{\delta q'}{_{#1}}}
\newrobustcmd{\PVar}[1]{\tensor{p}{_{#1}}}
\newrobustcmd{\BackgroundPVar}[1]{\tensor*{\overline{p}}{_{#1}}}
\newrobustcmd{\DeltaPVar}[1]{\tensor*{\delta p}{_{#1}}}
\newrobustcmd{\PVarDot}[1]{\tensor{\dot{p}}{_{#1}}}
\newrobustcmd{\DeltaPVarDot}[1]{\tensor*{\delta \dot{p}}{_{#1}}}
\newrobustcmd{\PVarDash}[1]{\tensor{p'}{_{#1}}}
\newrobustcmd{\DeltaPVarDash}[1]{\tensor*{\delta p'}{_{#1}}}
\newrobustcmd{\Mul}[1]{\tensor{\ell}{_{#1}}}
\newrobustcmd{\BackgroundMul}[1]{\tensor*{\overline{\ell}}{_{#1}}}
\newrobustcmd{\DeltaMul}[1]{\tensor*{\delta\ell}{_{#1}}}
\newrobustcmd{\MulDot}[1]{\tensor{\dot{\ell}}{_{#1}}}
\newrobustcmd{\DeltaMulDot}[1]{\tensor*{\delta \dot{\ell}}{_{#1}}}
\newrobustcmd{\MulDash}[1]{\tensor{\ell'}{_{#1}}}
\newrobustcmd{\DeltaMulDash}[1]{\tensor*{\delta \ell'}{_{#1}}}

\newrobustcmd{\Action}{%
	\mathscr{S}%
}

\newrobustcmd{\FirstOrderAction}[1]{%
	\tensor{\vphantom{l^{l^2}}\smash{\stackrel{\raisebox{-3pt}{\scalebox{0.55}{$(1)$}}}{\scalebox{0.99}{$\mathscr{S}$}}}}{#1}%
}
\newrobustcmd{\SecondOrderAction}[1]{%
	\tensor{\vphantom{l^{l^2}}\smash{\stackrel{\raisebox{-3pt}{\scalebox{0.55}{$(2)$}}}{\scalebox{0.99}{$\mathscr{S}$}}}}{#1}%
}

%	The covariant derivative 
\newrobustcmd{\CovD}[1]{%
	\tensor{D}{#1}
}
%	The background covariant derivative
\newrobustcmd{\BackgroundCovD}[1]{%
	\tensor*{\overline{D}}{#1}
}
%	The covariant derivative induced on the foliation
\newrobustcmd{\CD}[1]{%
	\tensor{\nabla}{#1}%
}
%	The background covariant derivative induced on the foliation
\newrobustcmd{\BackgroundCD}[1]{%
	\tensor{\overline{\nabla}}{#1}%
}
% 	The Bach tensor
\newrobustcmd{\Bach}[1]{%
	\tensor*{B}{#1}%
}
%	The Weyl tensor
\newrobustcmd{\Weyl}[1]{%
	\tensor{C}{#1}%
}
%	The Schouten tensor
\newrobustcmd{\Schouten}[1]{%
	\tensor{S}{#1}%
}
%	The Riemann, Ricci or Ricci scalar 
\newrobustcmd{\Curvature}[1]{%
	\tensor{R}{#1}
}
%	The velocity of the Riemann, Ricci or Ricci scalar
\newrobustcmd{\DotCurvature}[1]{%
	\tensor*{\dot{R}}{#1}
}
%	The acceleration of the Riemann, Ricci or Ricci scalar
\newrobustcmd{\DDotCurvature}[1]{%
	\tensor*{\ddot{R}}{#1}
}
%	The background Riemann, Ricci or Ricci scalar
\newrobustcmd{\BackgroundCurvature}[1]{%
	\tensor{\overline{R}}{#1}
}
%	The induced Riemann, Ricci or Ricci scalar on the foliation
\newrobustcmd{\R}[1]{%
	\tensor{\mathcal{R}}{#1}%
}
%	The background induced Riemann, Ricci or Ricci scalar on the foliation
\newrobustcmd{\BackgroundR}[1]{%
	\tensor*{\overline{\mathcal{R}}}{#1}%
}
%	The induced Riemann, Ricci or Ricci scalar on the foliation
\newrobustcmd{\CurvatureFoliation}[1]{%
	\tensor{\mathcal{R}}{#1}
}
%	The background induced Riemann, Ricci or Ricci scalar on the foliation
\newrobustcmd{\BackgroundCurvatureFoliation}[1]{%
	\tensor*{\overline{\mathcal{R}}}{#1}
}
%	The primary constraint
\newrobustcmd{\PrimaryConstraint}[1]{%
	\tensor*{\Phi}{#1}%
}
\newrobustcmd{\FirstOrderPrimaryConstraint}[1]{%
	\tensor{\vphantom{l^{l^2}}\smash{\stackrel{\raisebox{-3pt}{\scalebox{0.55}{$(1)$}}}{\scalebox{0.99}{$\Phi$}}}}{#1}%
}
\newrobustcmd{\SecondOrderPrimaryConstraint}[1]{%
	\tensor{\vphantom{l^{l^2}}\smash{\stackrel{\raisebox{-3pt}{\scalebox{0.55}{$(2)$}}}{\scalebox{0.99}{$\Phi$}}}}{#1}%
}
%	The multiplier 
\newrobustcmd{\Multiplier}[1]{%
	\tensor*{\lambda}{#1}%
}
%	The background multiplier
\newrobustcmd{\BackgroundMultiplier}[1]{%
	\tensor*{\overline{\lambda}}{#1}%
}
%	The perturbation to the multiplier
\newrobustcmd{\DeltaMultiplier}[1]{%
	\tensor*{\delta\lambda}{#1}%
}
%	The secondary constraint
\newrobustcmd{\SecondaryConstraint}[1]{%
	\tensor*{\Psi}{#1}%
}
\newrobustcmd{\FirstOrderSecondaryConstraint}[1]{%
	\tensor{\vphantom{l^{l^2}}\smash{\stackrel{\raisebox{-3pt}{\scalebox{0.55}{$(1)$}}}{\scalebox{0.99}{$\Psi$}}}}{#1}%
}
\newrobustcmd{\SecondOrderSecondaryConstraint}[1]{%
	\tensor{\vphantom{l^{l^2}}\smash{\stackrel{\raisebox{-3pt}{\scalebox{0.55}{$(2)$}}}{\scalebox{0.99}{$\Psi$}}}}{#1}%
}
\newrobustcmd{\FirstOrderActualSecondaryConstraint}[1]{%
	\tensor{\vphantom{l^{l^2}}\smash{\stackrel{\raisebox{-3pt}{\scalebox{0.55}{$(1)$}}}{\scalebox{0.99}{$\Theta$}}}}{#1}%
}
\newrobustcmd{\SecondOrderActualSecondaryConstraint}[1]{%
	\tensor{\vphantom{l^{l^2}}\smash{\stackrel{\raisebox{-3pt}{\scalebox{0.55}{$(2)$}}}{\scalebox{0.99}{$\Theta$}}}}{#1}%
}
%	A smearing function
\newrobustcmd{\SmearingS}[1]{%
	\tensor*{s}{#1}%
}
%	A smearing function
\newrobustcmd{\SmearingF}[1]{%
	\tensor*{f}{#1}%
}
%	The generic primary constraint
\newrobustcmd{\GenericPrimaryConstraint}[1]{%
	\tensor*{\Phi}{}\left[{#1}\right]%
}
%	The first order generic primary constraint
\newrobustcmd{\FirstOrderGenericPrimaryConstraint}[1]{%
	\tensor*{\vphantom{l^{l^2}}\smash{\stackrel{\raisebox{-3pt}{\scalebox{0.55}{$(1)$}}}{\scalebox{0.99}{$\Phi$}}}}{}\left[{#1}\right]%
}
%	The second order generic primary constraint
\newrobustcmd{\SecondOrderGenericPrimaryConstraint}[1]{%
	\tensor*{\vphantom{l^{l^2}}\smash{\stackrel{\raisebox{-3pt}{\scalebox{0.55}{$(2)$}}}{\scalebox{0.99}{$\Phi$}}}}{}\left[{#1}\right]%
}
%	The total Hamiltonian
\newrobustcmd{\TotalHamiltonian}{%
	\tensor*{\mathscr{H}}{}%
}
%	The total Hamiltonian
\newrobustcmd{\TotalHamiltonianGen}[1]{%
	\tensor*{\mathscr{H}}{}\left[{#1}\right]%
}
%	The first order total Hamiltonian
\newrobustcmd{\FirstOrderTotalHamiltonian}{%
	\tensor*{\vphantom{l^{l^2}}\smash{\stackrel{\raisebox{-3pt}{\scalebox{0.55}{$(1)$}}}{\scalebox{0.99}{$\mathscr{H}$}}}}{}%
}
%	The second order total Hamiltonian
\newrobustcmd{\SecondOrderTotalHamiltonian}[1]{%
	\tensor*{\vphantom{l^{l^2}}\smash{\stackrel{\raisebox{-3pt}{\scalebox{0.55}{$(2)$}}}{\scalebox{0.99}{$\mathscr{H}$}}}}{}\left[{#1}\right]%
}
%	The third order total Hamiltonian
\newrobustcmd{\ThirdOrderTotalHamiltonian}[1]{%
	\tensor*{\vphantom{l^{l^2}}\smash{\stackrel{\raisebox{-3pt}{\scalebox{0.55}{$(3)$}}}{\scalebox{0.99}{$\mathscr{H}$}}}}{}\left[{#1}\right]%
}
%	The super-Hamiltonian or super-momentum constraint
\newrobustcmd{\SuperConstraint}[1]{%
	\tensor{\mathcal{H}}{#1}%
}
%	The reduced super-Hamiltonian or super-momentum constraint
\newrobustcmd{\ReducedSuperConstraint}[1]{%
	\tensor{\mathbb{H}}{#1}%
}
\newrobustcmd{\ZerothOrderReducedSuperConstraint}[1]{%
	\tensor{\vphantom{l^{l^2}}\smash{\stackrel{\raisebox{-3pt}{\scalebox{0.55}{$(0)$}}}{\scalebox{0.99}{$\mathbb{H}$}}}}{#1}%
}
\newrobustcmd{\FirstOrderReducedSuperConstraint}[1]{%
	\tensor{\vphantom{l^{l^2}}\smash{\stackrel{\raisebox{-3pt}{\scalebox{0.55}{$(1)$}}}{\scalebox{0.99}{$\mathbb{H}$}}}}{#1}%
}
\newrobustcmd{\SuperConstraintDot}[1]{%
	\tensor{\dot{\mathcal{H}}}{#1}%
}
%	The dressed super-Hamiltonian or super-momentum constraint
\newrobustcmd{\DressedSuperConstraint}[1]{%
	\tensor{\mathfrak{H}}{#1}%
}
\newrobustcmd{\FirstOrderSuperConstraint}[1]{%
	\tensor{\vphantom{l^{l^2}}\smash{\stackrel{\raisebox{-3pt}{\scalebox{0.55}{$(1)$}}}{\scalebox{0.99}{$\mathcal{H}$}}}}{#1}%
}
\newrobustcmd{\SecondOrderSuperConstraint}[1]{%
	\tensor{\vphantom{l^{l^2}}\smash{\stackrel{\raisebox{-3pt}{\scalebox{0.55}{$(2)$}}}{\scalebox{0.99}{$\mathcal{H}$}}}}{#1}%
}
%	The lapse function 
\newrobustcmd{\Lapse}{%
	N	
}
%	The background lapse function
\newrobustcmd{\BackgroundLapse}{%
	\overline{N}	
}
%	The perturbation to the lapse function
\newrobustcmd{\DeltaLapse}{%
	\delta N	
}
%	The shift vector
\newrobustcmd{\Shift}[1]{%
	\tensor{N}{#1}
}
%	The background shift vector
\newrobustcmd{\BackgroundShift}[1]{%
	\tensor*{\overline{N}}{#1}
}
%	The perturbation to the shift vector
\newrobustcmd{\DeltaShift}[1]{%
	\tensor*{\delta N}{#1}
}
%	The Kronecker delta
\newrobustcmd{\Kronecker}[1]{%
	\tensor*{\delta}{#1}
}
%	The metric tensor
\newrobustcmd{\G}[1]{%
	\tensor*{g}{#1}%
}
%	The metric tensor
\newrobustcmd{\Metric}[1]{%
	\tensor{g}{#1}
}
%	The background metric tensor
\newrobustcmd{\BackgroundG}[1]{%
	\tensor*{\overline{g}}{#1}%
}
%	The perturbation to the metric tensor
\newrobustcmd{\DeltaMetric}[1]{%
	\tensor*{\delta g}{#1}%
}
%	The perturbation of the stress-energy tensor
\newrobustcmd{\DeltaStressEnergy}[1]{%
	\tensor*{\delta T}{#1}%
}
%	The perturbation to the metric tensor
\newrobustcmd{\DeltaG}[1]{%
	\tensor*{\delta h}{#1}%
}
%	The perturbation to the metric tensor
\newrobustcmd{\DeltaMetricFoliation}[1]{%
	\tensor*{\delta h}{#1}%
}
%	The velocity of the perturbation to the metric tensor
\newrobustcmd{\DotDeltaMetricFoliation}[1]{%
	\tensor*{\delta \dot{h}}{#1}%
}
%	The momentum conjugate to the metric tensor
\newrobustcmd{\ConjugateMomentumG}[1]{%
	\tensor*{\pi}{#1}%
}
%	The velocity of the momentum conjugate to the metric tensor
\newrobustcmd{\ConjugateMomentumGDot}[1]{%
	\tensor*{\dot{\pi}}{#1}%
}
%	The background momentum conjugate to the metric tensor
\newrobustcmd{\BackgroundConjugateMomentumG}[1]{%
	\tensor*{\overline{\pi}}{#1}%
}
%	The perturbation to the momentum conjugate to the metric tensor
\newrobustcmd{\ConjugateMomentumDeltaG}[1]{%
	\tensor*{\delta\pi}{#1}%
}
%	The second perturbation to the momentum conjugate to the metric tensor
\newrobustcmd{\ConjugateMomentumDDeltaG}[1]{%
	\tensor*{\delta\delta\pi}{#1}%
}
%	The actual F-function 
\newrobustcmd{\FFunctionActual}[1]{%
	\tensor*{F}{#1}%
}
%	The F-function
\newrobustcmd{\FFunction}[1]{%
	\tensor*{\mathcal{F}}{#1}%
}
%	The actual extrinsic curvature tensor
\newrobustcmd{\ExtrinsicCurvatureActual}[1]{%
	\tensor*{K}{#1}%
}
%	The gauge projector
\newrobustcmd{\GaugeProjector}[1]{%
	\tensor{\mathscr{P}}{#1}%
}
%	The Schouten projector
\newrobustcmd{\SchoutenProjector}[1]{%
	\tensor{\mathfrak{P}}{#1}%
}
%	The Killing vector
\newrobustcmd{\Killing}[1]{%
	\tensor*{\xi}{#1}%
}
%	The actual extrinsic curvature tensor
\newrobustcmd{\ExtrinsicCurvatureActualDot}[1]{%
	\tensor*{\dot{K}}{#1}%
}
%	The extrinsic curvature tensor
\newrobustcmd{\ExtrinsicCurvature}[1]{%
	\tensor*{\mathcal{K}}{#1}%
}
%	The background extrinsic curvature tensor
\newrobustcmd{\BackgroundExtrinsicCurvature}[1]{%
	\tensor*{\overline{\mathcal{K}}}{#1}%
}
% 	The perturbation to the extrinsic curvature tensor
\newrobustcmd{\DeltaExtrinsicCurvature}[1]{%
	\tensor*{\delta\mathcal{K}}{#1}%
}
%	The velocity of the perturbation of the extrinsic curvature tensor
\newrobustcmd{\DotDeltaExtrinsicCurvature}[1]{%
	\tensor*{\delta\dot{\mathcal{K}}}{#1}%
}
%	The conjugate momentum to the extrinsic curvature tensor
\newrobustcmd{\ConjugateMomentumExtrinsicCurvature}[1]{%
	\tensor*{\rho}{#1}%
}
%	The second perturbation to the conjugate momentum to the extrinsic curvature tensor
\newrobustcmd{\ConjugateMomentumDDeltaExtrinsicCurvature}[1]{%
	\tensor*{\delta\delta\rho}{#1}%
}
%	The velocity of the conjugate momentum to the extrinsic curvature tensor
\newrobustcmd{\ConjugateMomentumExtrinsicCurvatureDot}[1]{%
	\tensor*{\dot{\rho}}{#1}%
}
%	The background conjugate momentum to the extrinsic curvature tensor
\newrobustcmd{\BackgroundConjugateMomentumExtrinsicCurvature}[1]{%
	\tensor*{\overline{\rho}}{#1}%
}
%	The perturbation to the conjugate momentum to the extrinsic curvature tensor
\newrobustcmd{\ConjugateMomentumDeltaExtrinsicCurvature}[1]{%
	\tensor*{\delta\rho}{#1}%
}
\newrobustcmd{\Cadabra}{\textit{Cadabra}\xspace}



% from the old version
%	The unit-timelike normal vector 
\newrobustcmd{\UnitTimelike}[1]{%
	\tensor{n}{#1}
}
%	The coordinate functions
\newrobustcmd{\Coordinate}[1]{%
	\tensor{x}{#1}
}
%	The general vector 
\newrobustcmd{\Vector}[1]{%
	\tensor{A}{#1}
}
%	The partial derivative 
\newrobustcmd{\PD}[1]{%
	\tensor*{\partial}{#1}
}
%	The Christoffel symbols 
\newrobustcmd{\Christoffel}[1]{%
	\tensor*{\Gamma}{#1}
}
%	The generic field 
\newrobustcmd{\GenQ}{%
	\tensor*{q}{}
}
%	The generic momentum 
\newrobustcmd{\GenP}{%
	\tensor*{p}{}
}
%	The perturbation to the generic field
\newrobustcmd{\DeltaGenQ}{%
	\tensor*{\delta q}{}
}
%	The perturbation to the generic momentum
\newrobustcmd{\DeltaGenP}{%
	\tensor*{\delta p}{}
}
%	The velocity of the perturbation of the generic field
\newrobustcmd{\DotDeltaGenQ}{%
	\tensor*{\delta\dot{q}}{}
}
%	The velocity of the perturbation of the generic momentum
\newrobustcmd{\DotDeltaGenP}{%
	\tensor*{\delta\dot{p}}{}
}
%	The second perturbation to the generic field
\newrobustcmd{\DDeltaGenQ}{%
	\tensor*{\delta\delta q}{}
}
%	The second perturbation to the generic momentum
\newrobustcmd{\DDeltaGenP}{%
	\tensor*{\delta\delta p}{}
}
%	The velocity of the second perturbation of the generic field
\newrobustcmd{\DotDDeltaGenQ}{%
	\tensor*{\delta\delta \dot{q}}{}
}
%	The velocity of the second perturbation of the generic momentum
\newrobustcmd{\DotDDeltaGenP}{%
	\tensor*{\delta\delta \dot{p}}{}
}
%	The momentum conjugate to the metric on the foliation
\newrobustcmd{\ConjugateMomentumMetricFoliation}[1]{%
	\tensor*{\pi}{#1}
}
%	The velocity of the momentum conjugate to the metric on the foliation
\newrobustcmd{\ConjugateMomentumMetricFoliationDot}[1]{%
	\tensor*{\dot{\pi}}{#1}
}
%	The metric on the foliation
\newrobustcmd{\MetricFoliation}[1]{%
	\tensor{h}{#1}
}
%	The background metric on the foliation
\newrobustcmd{\BackgroundMetricFoliation}[1]{%
	\tensor*{\overline{h}}{#1}
}
%	The velocity of the background metric on the foliation
\newrobustcmd{\BackgroundMetricFoliationDot}[1]{%
	\tensor*{\dot{\overline{h}}}{#1}
}
%	The velocity of the metric on the foliation
\newrobustcmd{\MetricFoliationDot}[1]{%
	\tensor{\dot{h}}{#1}
}
%	The timelike Killing vector field
\newrobustcmd{\KillingVector}[1]{%
	\tensor{\xi}{#1}
}
%	The velocity of the extrinsic curvature
\newrobustcmd{\ExtrinsicCurvatureDot}[1]{%
	\tensor*{\dot{\mathcal{K}}}{#1}
}
%	The Brinkmann function
\newrobustcmd{\Brinkmann}{%
	{\mathcal{G}}
}
%	The Schwarzschild radius
\newrobustcmd{\SchwarzschildRadius}{%
	{r_{\text{g}}}
}
%	The first Weyl exponent 
\newrobustcmd{\PsiFunc}{%
	{\psi}
}
%	The second Weyl exponent
\newrobustcmd{\GammaFunc}{%
	{\gamma}
}
%	The U function
\newrobustcmd{\UFunc}{%
	{\mathcal{U}}
}
%	The U function
\newrobustcmd{\DotUFunc}{%
	{\mathcal{\dot{U}}}
}
%	The angular momentum per unit mass
\newrobustcmd{\AngularMomentumPerUnitMass}{%
	{a}
}
%	The first Kasner exponent
\newrobustcmd{\KasnerA}{%
	{\alpha}
}
%	The second Kasner exponent
\newrobustcmd{\KasnerB}{%
	{\beta}
}
%	The third Kasner exponent
\newrobustcmd{\KasnerC}{%
	{\gamma}
}
%	Shorthand K tensor for GR at two loops
\newrobustcmd{\ShorthandK}[1]{%
	\tensor*{\mathscr{K}}{#1}%
}
%	Shorthand L tensor for GR at two loops
\newrobustcmd{\ShorthandL}[1]{%
	\tensor{\mathscr{L}}{#1}%
}
%	Shorthand Q tensor for GR at two loops
\newrobustcmd{\ShorthandQ}[1]{%
	\tensor{\mathscr{Q}}{#1}%
}
%	Trace of shorthand Q tensor (2-index, symmetric)
\newrobustcmd{\TraceShorthandQ}[1]{%
	\tensor*{\mathscr{Q}}{#1}%
}
%	The 0+ spin-parity component of the vector field
\newrobustcmd{\VectorZeroPlus}[1]{%
	\tensor{A^{\#1}_{0^+}}{#1}%
}
%	The 1- spin-parity component of the vector field
\newrobustcmd{\VectorOneMinus}[1]{%
	\tensor{A^{\#1}_{1^-}}{#1}%
}
%	The conjugate momentum to the 0+ spin-parity component of the vector field
\newrobustcmd{\ConjugateMomentumVectorZeroPlus}[1]{%
	\tensor{\pi(A)^{\#1}_{0^+}}{#1}%
}
%	The conjugate momentum to the 1- spin-parity component of the vector field
\newrobustcmd{\ConjugateMomentumVectorOneMinus}[1]{%
	\tensor{\pi(A)^{\#1}_{1^-}}{#1}%
}
%	The Lagrange multiplier for the 0+ spin-parity component of the vector field
\newrobustcmd{\LagrangeMultiplierVectorZeroPlus}[1]{%
	\tensor{\lambda(A)^{\#1}_{0^+}}{#1}%
}
%	The Lagrange multiplier for the 1- spin-parity component of the vector field
\newrobustcmd{\LagrangeMultiplierVectorOneMinus}[1]{%
	\tensor{\lambda(A)^{\#1}_{1^-}}{#1}%
}
%	The primary constraint for the 1- spin-parity component of the vector field
\newrobustcmd{\PrimaryConstraintVectorOneMinus}[1]{%
	\tensor{\phi(A)^{\#1}_{1^-}}{#1}%
}
%	The time derivative of the primary constraint for the 1- spin-parity component of the vector field
\newrobustcmd{\PrimaryConstraintVectorOneMinusDot}[1]{%
	\tensor{\dot{\phi}(A)^{\#1}_{1^-}}{#1}%
}
%	The secondary constraint for the 0+ spin-parity component of the vector field
\newrobustcmd{\SecondaryConstraintVectorZeroPlus}[1]{%
	\tensor{\chi(A)^{\#1}_{0^+}}{#1}%
}
%	The secondary constraint for the 1- spin-parity component of the vector field
\newrobustcmd{\SecondaryConstraintVectorOneMinus}[1]{%
	\tensor{\chi(A)^{\#1}_{1^-}}{#1}%
}

%MacrosNames

\usepackage{xspace}

\newrobustcmd{\Hamilcar}{\textit{Hamilcar}\xspace}
\newrobustcmd{\Hasdrubal}{\textit{Hasdrubal}\xspace}
\newrobustcmd{\PSALTer}{\textit{PSALTer}\xspace}
\newrobustcmd{\xAct}{\textit{xAct}\xspace}
\newrobustcmd{\xTensor}{\textit{xTensor}\xspace}
\newrobustcmd{\xCoba}{\textit{xCoba}\xspace}
\newrobustcmd{\xPerm}{\textit{xPerm}\xspace}
\newrobustcmd{\xCore}{\textit{xCore}\xspace}
\newrobustcmd{\xTras}{\textit{xTras}\xspace}
\newrobustcmd{\xPlain}{\textit{xPlain}\xspace}
\newrobustcmd{\SymManipulator}{\textit{SymManipulator}\xspace}
\newrobustcmd{\Mathematica}{\textit{Mathematica}\xspace}
\newrobustcmd{\Wolfram}{\textit{Wolfram}\xspace}
\newrobustcmd{\WolframLanguage}{\textit{Wolfram Language}\xspace}
\newrobustcmd{\GitHub}{\textit{GitHub}\xspace}
\newrobustcmd{\GitLab}{\textit{GitLab}\xspace}
\newrobustcmd{\Bash}{\textit{Bash}\xspace}
\newrobustcmd{\Linux}{\textit{GNU/Linux}\xspace}
\newrobustcmd{\Windows}{\textit{Windows}\xspace}
\newrobustcmd{\Mac}{\textit{macOS}\xspace}
\newrobustcmd{\OpenAI}{\textit{OpenAI}\xspace}
\newrobustcmd{\AgentsSDK}{\textit{Agents SDK}\xspace}
\newrobustcmd{\WolframResearch}{\textit{Wolfram Research}\xspace}
\newrobustcmd{\Python}{\textit{Python}\xspace}

\begin{document}
\begin{tikzpicture}[
    x={(1cm,0cm)},
    y={(0.4cm,0.3cm)},
    z={(0cm,1cm)}
]
% Padding (in screen coordinates)
\path[use as bounding box] (-0.5cm,-0.5cm) rectangle (20cm,5.5cm);

% Pale blue background
\fill[blue!5] (-0.5cm,-0.5cm) rectangle (20cm,5.5cm);

\foreach \xoffset in {0, 14} {
\begin{scope}[shift={(\xoffset,0,0)}]
% Foliated spacetime block - 3D
\def\nslices{3}
\def\cubesize{3}
\def\xsize{\cubesize}
\def\ysize{\cubesize}
\def\zsize{\cubesize}
\pgfmathsetmacro{\sliceheight}{\zsize/\nslices}
\def\curvature{0.12}

% Bounding cube
\pgfmathsetmacro{\zmax}{\nslices*\sliceheight}

% Annular Gaussian parameters (origin at sheet center)
\def\maxringradius{1.0}
\def\minsigma{0.2}
\def\maxsigma{0.6}
\def\maxamplitude{1.0}
\def\minamplitude{0.4}
\def\patchsize{0.5}
% Spiral modulation: r0 -> r0 + spiralamp * sin(spiralarms * theta + spiralradial * r)
\def\spiralamp{0.5}
\def\spiralarms{2}
\def\spiralradial{15}

% Vertical faces - colored for left figure, plain for right figure
\def\vpatchsize{0.5}

\ifdim\xoffset pt=0pt
% Precomputed colored patches for left figure (front face, right face, top slice)
% Auto-generated by generate_logo_patches.py
% Front face patches
\fill[blue!15.0, opacity=0.5] (0.0000, -0.0000, 0.0000) -- (0.0500, -0.0000, 0.0063) -- (0.0501, -0.0001, 0.0563) -- (0.0001, -0.0001, 0.0500) -- cycle;
\fill[blue!15.0, opacity=0.5] (0.0001, -0.0001, 0.0500) -- (0.0501, -0.0001, 0.0563) -- (0.0503, -0.0003, 0.1063) -- (0.0003, -0.0003, 0.1000) -- cycle;
\fill[blue!15.0, opacity=0.5] (0.0003, -0.0003, 0.1000) -- (0.0503, -0.0003, 0.1063) -- (0.0507, -0.0007, 0.1563) -- (0.0007, -0.0007, 0.1500) -- cycle;
\fill[blue!15.0, opacity=0.5] (0.0007, -0.0007, 0.1500) -- (0.0507, -0.0007, 0.1563) -- (0.0513, -0.0013, 0.2063) -- (0.0013, -0.0013, 0.2000) -- cycle;
\fill[blue!15.0, opacity=0.5] (0.0013, -0.0013, 0.2000) -- (0.0513, -0.0013, 0.2063) -- (0.0520, -0.0020, 0.2563) -- (0.0020, -0.0020, 0.2500) -- cycle;
\fill[blue!15.0, opacity=0.5] (0.0020, -0.0020, 0.2500) -- (0.0520, -0.0020, 0.2563) -- (0.0528, -0.0029, 0.3063) -- (0.0029, -0.0029, 0.3000) -- cycle;
\fill[blue!15.0, opacity=0.5] (0.0029, -0.0029, 0.3000) -- (0.0528, -0.0029, 0.3063) -- (0.0539, -0.0040, 0.3563) -- (0.0040, -0.0040, 0.3500) -- cycle;
\fill[blue!15.0, opacity=0.5] (0.0040, -0.0040, 0.3500) -- (0.0539, -0.0040, 0.3563) -- (0.0550, -0.0052, 0.4063) -- (0.0052, -0.0052, 0.4000) -- cycle;
\fill[blue!15.0, opacity=0.5] (0.0052, -0.0052, 0.4000) -- (0.0550, -0.0052, 0.4063) -- (0.0563, -0.0065, 0.4563) -- (0.0065, -0.0065, 0.4500) -- cycle;
\fill[blue!15.0, opacity=0.5] (0.0065, -0.0065, 0.4500) -- (0.0563, -0.0065, 0.4563) -- (0.0578, -0.0080, 0.5063) -- (0.0080, -0.0080, 0.5000) -- cycle;
\fill[blue!15.0, opacity=0.5] (0.0080, -0.0080, 0.5000) -- (0.0578, -0.0080, 0.5063) -- (0.0594, -0.0097, 0.5563) -- (0.0097, -0.0097, 0.5500) -- cycle;
\fill[blue!15.0, opacity=0.5] (0.0097, -0.0097, 0.5500) -- (0.0594, -0.0097, 0.5563) -- (0.0611, -0.0115, 0.6063) -- (0.0115, -0.0115, 0.6000) -- cycle;
\fill[blue!15.0, opacity=0.5] (0.0115, -0.0115, 0.6000) -- (0.0611, -0.0115, 0.6063) -- (0.0629, -0.0134, 0.6563) -- (0.0134, -0.0134, 0.6500) -- cycle;
\fill[blue!15.0, opacity=0.5] (0.0134, -0.0134, 0.6500) -- (0.0629, -0.0134, 0.6563) -- (0.0649, -0.0154, 0.7063) -- (0.0154, -0.0154, 0.7000) -- cycle;
\fill[blue!15.0, opacity=0.5] (0.0154, -0.0154, 0.7000) -- (0.0649, -0.0154, 0.7063) -- (0.0670, -0.0176, 0.7563) -- (0.0176, -0.0176, 0.7500) -- cycle;
\fill[blue!15.0, opacity=0.5] (0.0176, -0.0176, 0.7500) -- (0.0670, -0.0176, 0.7563) -- (0.0692, -0.0199, 0.8063) -- (0.0199, -0.0199, 0.8000) -- cycle;
\fill[blue!15.0, opacity=0.5] (0.0199, -0.0199, 0.8000) -- (0.0692, -0.0199, 0.8063) -- (0.0715, -0.0222, 0.8563) -- (0.0222, -0.0222, 0.8500) -- cycle;
\fill[blue!15.0, opacity=0.5] (0.0222, -0.0222, 0.8500) -- (0.0715, -0.0222, 0.8563) -- (0.0739, -0.0247, 0.9063) -- (0.0247, -0.0247, 0.9000) -- cycle;
\fill[blue!15.0, opacity=0.5] (0.0247, -0.0247, 0.9000) -- (0.0739, -0.0247, 0.9063) -- (0.0764, -0.0273, 0.9563) -- (0.0273, -0.0273, 0.9500) -- cycle;
\fill[blue!15.0, opacity=0.5] (0.0273, -0.0273, 0.9500) -- (0.0764, -0.0273, 0.9563) -- (0.0790, -0.0300, 1.0063) -- (0.0300, -0.0300, 1.0000) -- cycle;
\fill[blue!15.0, opacity=0.5] (0.0300, -0.0300, 1.0000) -- (0.0790, -0.0300, 1.0063) -- (0.0817, -0.0328, 1.0563) -- (0.0328, -0.0328, 1.0500) -- cycle;
\fill[blue!15.0, opacity=0.5] (0.0328, -0.0328, 1.0500) -- (0.0817, -0.0328, 1.0563) -- (0.0844, -0.0356, 1.1063) -- (0.0356, -0.0356, 1.1000) -- cycle;
\fill[blue!15.0, opacity=0.5] (0.0356, -0.0356, 1.1000) -- (0.0844, -0.0356, 1.1063) -- (0.0872, -0.0385, 1.1563) -- (0.0385, -0.0385, 1.1500) -- cycle;
\fill[blue!15.0, opacity=0.5] (0.0385, -0.0385, 1.1500) -- (0.0872, -0.0385, 1.1563) -- (0.0901, -0.0415, 1.2063) -- (0.0415, -0.0415, 1.2000) -- cycle;
\fill[blue!15.0, opacity=0.5] (0.0415, -0.0415, 1.2000) -- (0.0901, -0.0415, 1.2063) -- (0.0930, -0.0445, 1.2563) -- (0.0445, -0.0445, 1.2500) -- cycle;
\fill[blue!15.0, opacity=0.5] (0.0445, -0.0445, 1.2500) -- (0.0930, -0.0445, 1.2563) -- (0.0959, -0.0475, 1.3063) -- (0.0475, -0.0475, 1.3000) -- cycle;
\fill[blue!15.0, opacity=0.5] (0.0475, -0.0475, 1.3000) -- (0.0959, -0.0475, 1.3063) -- (0.0989, -0.0506, 1.3563) -- (0.0506, -0.0506, 1.3500) -- cycle;
\fill[blue!15.0, opacity=0.5] (0.0506, -0.0506, 1.3500) -- (0.0989, -0.0506, 1.3563) -- (0.1019, -0.0537, 1.4063) -- (0.0537, -0.0537, 1.4000) -- cycle;
\fill[blue!15.0, opacity=0.5] (0.0537, -0.0537, 1.4000) -- (0.1019, -0.0537, 1.4063) -- (0.1050, -0.0569, 1.4563) -- (0.0569, -0.0569, 1.4500) -- cycle;
\fill[blue!15.0, opacity=0.5] (0.0569, -0.0569, 1.4500) -- (0.1050, -0.0569, 1.4563) -- (0.1080, -0.0600, 1.5063) -- (0.0600, -0.0600, 1.5000) -- cycle;
\fill[blue!15.0, opacity=0.5] (0.0600, -0.0600, 1.5000) -- (0.1080, -0.0600, 1.5063) -- (0.1110, -0.0631, 1.5563) -- (0.0631, -0.0631, 1.5500) -- cycle;
\fill[blue!15.1, opacity=0.5] (0.0631, -0.0631, 1.5500) -- (0.1110, -0.0631, 1.5563) -- (0.1141, -0.0663, 1.6063) -- (0.0663, -0.0663, 1.6000) -- cycle;
\fill[blue!15.1, opacity=0.5] (0.0663, -0.0663, 1.6000) -- (0.1141, -0.0663, 1.6063) -- (0.1171, -0.0694, 1.6563) -- (0.0694, -0.0694, 1.6500) -- cycle;
\fill[blue!15.2, opacity=0.5] (0.0694, -0.0694, 1.6500) -- (0.1171, -0.0694, 1.6563) -- (0.1201, -0.0725, 1.7063) -- (0.0725, -0.0725, 1.7000) -- cycle;
\fill[blue!15.2, opacity=0.5] (0.0725, -0.0725, 1.7000) -- (0.1201, -0.0725, 1.7063) -- (0.1230, -0.0755, 1.7563) -- (0.0755, -0.0755, 1.7500) -- cycle;
\fill[blue!15.3, opacity=0.5] (0.0755, -0.0755, 1.7500) -- (0.1230, -0.0755, 1.7563) -- (0.1259, -0.0785, 1.8063) -- (0.0785, -0.0785, 1.8000) -- cycle;
\fill[blue!15.5, opacity=0.5] (0.0785, -0.0785, 1.8000) -- (0.1259, -0.0785, 1.8063) -- (0.1288, -0.0815, 1.8563) -- (0.0815, -0.0815, 1.8500) -- cycle;
\fill[blue!15.6, opacity=0.5] (0.0815, -0.0815, 1.8500) -- (0.1288, -0.0815, 1.8563) -- (0.1316, -0.0844, 1.9063) -- (0.0844, -0.0844, 1.9000) -- cycle;
\fill[blue!15.8, opacity=0.5] (0.0844, -0.0844, 1.9000) -- (0.1316, -0.0844, 1.9063) -- (0.1343, -0.0872, 1.9563) -- (0.0872, -0.0872, 1.9500) -- cycle;
\fill[blue!16.1, opacity=0.5] (0.0872, -0.0872, 1.9500) -- (0.1343, -0.0872, 1.9563) -- (0.1370, -0.0900, 2.0063) -- (0.0900, -0.0900, 2.0000) -- cycle;
\fill[blue!16.4, opacity=0.5] (0.0900, -0.0900, 2.0000) -- (0.1370, -0.0900, 2.0063) -- (0.1396, -0.0927, 2.0563) -- (0.0927, -0.0927, 2.0500) -- cycle;
\fill[blue!16.7, opacity=0.5] (0.0927, -0.0927, 2.0500) -- (0.1396, -0.0927, 2.0563) -- (0.1421, -0.0953, 2.1063) -- (0.0953, -0.0953, 2.1000) -- cycle;
\fill[blue!17.2, opacity=0.5] (0.0953, -0.0953, 2.1000) -- (0.1421, -0.0953, 2.1063) -- (0.1445, -0.0978, 2.1563) -- (0.0978, -0.0978, 2.1500) -- cycle;
\fill[blue!17.6, opacity=0.5] (0.0978, -0.0978, 2.1500) -- (0.1445, -0.0978, 2.1563) -- (0.1468, -0.1001, 2.2063) -- (0.1001, -0.1001, 2.2000) -- cycle;
\fill[blue!18.2, opacity=0.5] (0.1001, -0.1001, 2.2000) -- (0.1468, -0.1001, 2.2063) -- (0.1490, -0.1024, 2.2563) -- (0.1024, -0.1024, 2.2500) -- cycle;
\fill[blue!18.8, opacity=0.5] (0.1024, -0.1024, 2.2500) -- (0.1490, -0.1024, 2.2563) -- (0.1511, -0.1046, 2.3063) -- (0.1046, -0.1046, 2.3000) -- cycle;
\fill[blue!19.5, opacity=0.5] (0.1046, -0.1046, 2.3000) -- (0.1511, -0.1046, 2.3063) -- (0.1531, -0.1066, 2.3563) -- (0.1066, -0.1066, 2.3500) -- cycle;
\fill[blue!20.2, opacity=0.5] (0.1066, -0.1066, 2.3500) -- (0.1531, -0.1066, 2.3563) -- (0.1549, -0.1085, 2.4063) -- (0.1085, -0.1085, 2.4000) -- cycle;
\fill[blue!21.0, opacity=0.5] (0.1085, -0.1085, 2.4000) -- (0.1549, -0.1085, 2.4063) -- (0.1566, -0.1103, 2.4563) -- (0.1103, -0.1103, 2.4500) -- cycle;
\fill[blue!21.8, opacity=0.5] (0.1103, -0.1103, 2.4500) -- (0.1566, -0.1103, 2.4563) -- (0.1582, -0.1120, 2.5063) -- (0.1120, -0.1120, 2.5000) -- cycle;
\fill[blue!22.7, opacity=0.5] (0.1120, -0.1120, 2.5000) -- (0.1582, -0.1120, 2.5063) -- (0.1597, -0.1135, 2.5563) -- (0.1135, -0.1135, 2.5500) -- cycle;
\fill[blue!23.7, opacity=0.5] (0.1135, -0.1135, 2.5500) -- (0.1597, -0.1135, 2.5563) -- (0.1610, -0.1148, 2.6063) -- (0.1148, -0.1148, 2.6000) -- cycle;
\fill[blue!24.7, opacity=0.5] (0.1148, -0.1148, 2.6000) -- (0.1610, -0.1148, 2.6063) -- (0.1621, -0.1160, 2.6563) -- (0.1160, -0.1160, 2.6500) -- cycle;
\fill[blue!25.7, opacity=0.5] (0.1160, -0.1160, 2.6500) -- (0.1621, -0.1160, 2.6563) -- (0.1632, -0.1171, 2.7063) -- (0.1171, -0.1171, 2.7000) -- cycle;
\fill[blue!26.7, opacity=0.5] (0.1171, -0.1171, 2.7000) -- (0.1632, -0.1171, 2.7063) -- (0.1640, -0.1180, 2.7563) -- (0.1180, -0.1180, 2.7500) -- cycle;
\fill[blue!27.8, opacity=0.5] (0.1180, -0.1180, 2.7500) -- (0.1640, -0.1180, 2.7563) -- (0.1647, -0.1187, 2.8063) -- (0.1187, -0.1187, 2.8000) -- cycle;
\fill[blue!28.8, opacity=0.5] (0.1187, -0.1187, 2.8000) -- (0.1647, -0.1187, 2.8063) -- (0.1653, -0.1193, 2.8563) -- (0.1193, -0.1193, 2.8500) -- cycle;
\fill[blue!29.8, opacity=0.5] (0.1193, -0.1193, 2.8500) -- (0.1653, -0.1193, 2.8563) -- (0.1657, -0.1197, 2.9063) -- (0.1197, -0.1197, 2.9000) -- cycle;
\fill[blue!30.8, opacity=0.5] (0.1197, -0.1197, 2.9000) -- (0.1657, -0.1197, 2.9063) -- (0.1659, -0.1199, 2.9563) -- (0.1199, -0.1199, 2.9500) -- cycle;
\fill[blue!31.8, opacity=0.5] (0.1199, -0.1199, 2.9500) -- (0.1659, -0.1199, 2.9563) -- (0.1660, -0.1200, 3.0063) -- (0.1200, -0.1200, 3.0000) -- cycle;
\fill[blue!15.0, opacity=0.5] (0.0500, -0.0000, 0.0063) -- (0.1000, -0.0000, 0.0125) -- (0.1001, -0.0001, 0.0625) -- (0.0501, -0.0001, 0.0563) -- cycle;
\fill[blue!15.0, opacity=0.5] (0.0501, -0.0001, 0.0563) -- (0.1001, -0.0001, 0.0625) -- (0.1003, -0.0003, 0.1125) -- (0.0503, -0.0003, 0.1063) -- cycle;
\fill[blue!15.0, opacity=0.5] (0.0503, -0.0003, 0.1063) -- (0.1003, -0.0003, 0.1125) -- (0.1007, -0.0007, 0.1625) -- (0.0507, -0.0007, 0.1563) -- cycle;
\fill[blue!15.0, opacity=0.5] (0.0507, -0.0007, 0.1563) -- (0.1007, -0.0007, 0.1625) -- (0.1012, -0.0013, 0.2125) -- (0.0513, -0.0013, 0.2063) -- cycle;
\fill[blue!15.0, opacity=0.5] (0.0513, -0.0013, 0.2063) -- (0.1012, -0.0013, 0.2125) -- (0.1019, -0.0020, 0.2625) -- (0.0520, -0.0020, 0.2563) -- cycle;
\fill[blue!15.0, opacity=0.5] (0.0520, -0.0020, 0.2563) -- (0.1019, -0.0020, 0.2625) -- (0.1027, -0.0029, 0.3125) -- (0.0528, -0.0029, 0.3063) -- cycle;
\fill[blue!15.0, opacity=0.5] (0.0528, -0.0029, 0.3063) -- (0.1027, -0.0029, 0.3125) -- (0.1037, -0.0040, 0.3625) -- (0.0539, -0.0040, 0.3563) -- cycle;
\fill[blue!15.0, opacity=0.5] (0.0539, -0.0040, 0.3563) -- (0.1037, -0.0040, 0.3625) -- (0.1048, -0.0052, 0.4125) -- (0.0550, -0.0052, 0.4063) -- cycle;
\fill[blue!15.0, opacity=0.5] (0.0550, -0.0052, 0.4063) -- (0.1048, -0.0052, 0.4125) -- (0.1061, -0.0065, 0.4625) -- (0.0563, -0.0065, 0.4563) -- cycle;
\fill[blue!15.0, opacity=0.5] (0.0563, -0.0065, 0.4563) -- (0.1061, -0.0065, 0.4625) -- (0.1075, -0.0080, 0.5125) -- (0.0578, -0.0080, 0.5063) -- cycle;
\fill[blue!15.0, opacity=0.5] (0.0578, -0.0080, 0.5063) -- (0.1075, -0.0080, 0.5125) -- (0.1090, -0.0097, 0.5625) -- (0.0594, -0.0097, 0.5563) -- cycle;
\fill[blue!15.0, opacity=0.5] (0.0594, -0.0097, 0.5563) -- (0.1090, -0.0097, 0.5625) -- (0.1107, -0.0115, 0.6125) -- (0.0611, -0.0115, 0.6063) -- cycle;
\fill[blue!15.0, opacity=0.5] (0.0611, -0.0115, 0.6063) -- (0.1107, -0.0115, 0.6125) -- (0.1125, -0.0134, 0.6625) -- (0.0629, -0.0134, 0.6563) -- cycle;
\fill[blue!15.0, opacity=0.5] (0.0629, -0.0134, 0.6563) -- (0.1125, -0.0134, 0.6625) -- (0.1144, -0.0154, 0.7125) -- (0.0649, -0.0154, 0.7063) -- cycle;
\fill[blue!15.0, opacity=0.5] (0.0649, -0.0154, 0.7063) -- (0.1144, -0.0154, 0.7125) -- (0.1164, -0.0176, 0.7625) -- (0.0670, -0.0176, 0.7563) -- cycle;
\fill[blue!15.0, opacity=0.5] (0.0670, -0.0176, 0.7563) -- (0.1164, -0.0176, 0.7625) -- (0.1185, -0.0199, 0.8125) -- (0.0692, -0.0199, 0.8063) -- cycle;
\fill[blue!15.0, opacity=0.5] (0.0692, -0.0199, 0.8063) -- (0.1185, -0.0199, 0.8125) -- (0.1208, -0.0222, 0.8625) -- (0.0715, -0.0222, 0.8563) -- cycle;
\fill[blue!15.0, opacity=0.5] (0.0715, -0.0222, 0.8563) -- (0.1208, -0.0222, 0.8625) -- (0.1231, -0.0247, 0.9125) -- (0.0739, -0.0247, 0.9063) -- cycle;
\fill[blue!15.0, opacity=0.5] (0.0739, -0.0247, 0.9063) -- (0.1231, -0.0247, 0.9125) -- (0.1255, -0.0273, 0.9625) -- (0.0764, -0.0273, 0.9563) -- cycle;
\fill[blue!15.0, opacity=0.5] (0.0764, -0.0273, 0.9563) -- (0.1255, -0.0273, 0.9625) -- (0.1280, -0.0300, 1.0125) -- (0.0790, -0.0300, 1.0063) -- cycle;
\fill[blue!15.0, opacity=0.5] (0.0790, -0.0300, 1.0063) -- (0.1280, -0.0300, 1.0125) -- (0.1306, -0.0328, 1.0625) -- (0.0817, -0.0328, 1.0563) -- cycle;
\fill[blue!15.0, opacity=0.5] (0.0817, -0.0328, 1.0563) -- (0.1306, -0.0328, 1.0625) -- (0.1332, -0.0356, 1.1125) -- (0.0844, -0.0356, 1.1063) -- cycle;
\fill[blue!15.0, opacity=0.5] (0.0844, -0.0356, 1.1063) -- (0.1332, -0.0356, 1.1125) -- (0.1359, -0.0385, 1.1625) -- (0.0872, -0.0385, 1.1563) -- cycle;
\fill[blue!15.0, opacity=0.5] (0.0872, -0.0385, 1.1563) -- (0.1359, -0.0385, 1.1625) -- (0.1387, -0.0415, 1.2125) -- (0.0901, -0.0415, 1.2063) -- cycle;
\fill[blue!15.0, opacity=0.5] (0.0901, -0.0415, 1.2063) -- (0.1387, -0.0415, 1.2125) -- (0.1415, -0.0445, 1.2625) -- (0.0930, -0.0445, 1.2563) -- cycle;
\fill[blue!15.0, opacity=0.5] (0.0930, -0.0445, 1.2563) -- (0.1415, -0.0445, 1.2625) -- (0.1444, -0.0475, 1.3125) -- (0.0959, -0.0475, 1.3063) -- cycle;
\fill[blue!15.0, opacity=0.5] (0.0959, -0.0475, 1.3063) -- (0.1444, -0.0475, 1.3125) -- (0.1472, -0.0506, 1.3625) -- (0.0989, -0.0506, 1.3563) -- cycle;
\fill[blue!15.0, opacity=0.5] (0.0989, -0.0506, 1.3563) -- (0.1472, -0.0506, 1.3625) -- (0.1501, -0.0537, 1.4125) -- (0.1019, -0.0537, 1.4063) -- cycle;
\fill[blue!15.0, opacity=0.5] (0.1019, -0.0537, 1.4063) -- (0.1501, -0.0537, 1.4125) -- (0.1531, -0.0569, 1.4625) -- (0.1050, -0.0569, 1.4563) -- cycle;
\fill[blue!15.0, opacity=0.5] (0.1050, -0.0569, 1.4563) -- (0.1531, -0.0569, 1.4625) -- (0.1560, -0.0600, 1.5125) -- (0.1080, -0.0600, 1.5063) -- cycle;
\fill[blue!15.1, opacity=0.5] (0.1080, -0.0600, 1.5063) -- (0.1560, -0.0600, 1.5125) -- (0.1589, -0.0631, 1.5625) -- (0.1110, -0.0631, 1.5563) -- cycle;
\fill[blue!15.1, opacity=0.5] (0.1110, -0.0631, 1.5563) -- (0.1589, -0.0631, 1.5625) -- (0.1619, -0.0663, 1.6125) -- (0.1141, -0.0663, 1.6063) -- cycle;
\fill[blue!15.1, opacity=0.5] (0.1141, -0.0663, 1.6063) -- (0.1619, -0.0663, 1.6125) -- (0.1648, -0.0694, 1.6625) -- (0.1171, -0.0694, 1.6563) -- cycle;
\fill[blue!15.2, opacity=0.5] (0.1171, -0.0694, 1.6563) -- (0.1648, -0.0694, 1.6625) -- (0.1676, -0.0725, 1.7125) -- (0.1201, -0.0725, 1.7063) -- cycle;
\fill[blue!15.3, opacity=0.5] (0.1201, -0.0725, 1.7063) -- (0.1676, -0.0725, 1.7125) -- (0.1705, -0.0755, 1.7625) -- (0.1230, -0.0755, 1.7563) -- cycle;
\fill[blue!15.4, opacity=0.5] (0.1230, -0.0755, 1.7563) -- (0.1705, -0.0755, 1.7625) -- (0.1733, -0.0785, 1.8125) -- (0.1259, -0.0785, 1.8063) -- cycle;
\fill[blue!15.5, opacity=0.5] (0.1259, -0.0785, 1.8063) -- (0.1733, -0.0785, 1.8125) -- (0.1761, -0.0815, 1.8625) -- (0.1288, -0.0815, 1.8563) -- cycle;
\fill[blue!15.7, opacity=0.5] (0.1288, -0.0815, 1.8563) -- (0.1761, -0.0815, 1.8625) -- (0.1788, -0.0844, 1.9125) -- (0.1316, -0.0844, 1.9063) -- cycle;
\fill[blue!15.9, opacity=0.5] (0.1316, -0.0844, 1.9063) -- (0.1788, -0.0844, 1.9125) -- (0.1814, -0.0872, 1.9625) -- (0.1343, -0.0872, 1.9563) -- cycle;
\fill[blue!16.2, opacity=0.5] (0.1343, -0.0872, 1.9563) -- (0.1814, -0.0872, 1.9625) -- (0.1840, -0.0900, 2.0125) -- (0.1370, -0.0900, 2.0063) -- cycle;
\fill[blue!16.6, opacity=0.5] (0.1370, -0.0900, 2.0063) -- (0.1840, -0.0900, 2.0125) -- (0.1865, -0.0927, 2.0625) -- (0.1396, -0.0927, 2.0563) -- cycle;
\fill[blue!16.9, opacity=0.5] (0.1396, -0.0927, 2.0563) -- (0.1865, -0.0927, 2.0625) -- (0.1889, -0.0953, 2.1125) -- (0.1421, -0.0953, 2.1063) -- cycle;
\fill[blue!17.4, opacity=0.5] (0.1421, -0.0953, 2.1063) -- (0.1889, -0.0953, 2.1125) -- (0.1912, -0.0978, 2.1625) -- (0.1445, -0.0978, 2.1563) -- cycle;
\fill[blue!17.9, opacity=0.5] (0.1445, -0.0978, 2.1563) -- (0.1912, -0.0978, 2.1625) -- (0.1935, -0.1001, 2.2125) -- (0.1468, -0.1001, 2.2063) -- cycle;
\fill[blue!18.5, opacity=0.5] (0.1468, -0.1001, 2.2063) -- (0.1935, -0.1001, 2.2125) -- (0.1956, -0.1024, 2.2625) -- (0.1490, -0.1024, 2.2563) -- cycle;
\fill[blue!19.1, opacity=0.5] (0.1490, -0.1024, 2.2563) -- (0.1956, -0.1024, 2.2625) -- (0.1976, -0.1046, 2.3125) -- (0.1511, -0.1046, 2.3063) -- cycle;
\fill[blue!19.9, opacity=0.5] (0.1511, -0.1046, 2.3063) -- (0.1976, -0.1046, 2.3125) -- (0.1995, -0.1066, 2.3625) -- (0.1531, -0.1066, 2.3563) -- cycle;
\fill[blue!20.6, opacity=0.5] (0.1531, -0.1066, 2.3563) -- (0.1995, -0.1066, 2.3625) -- (0.2013, -0.1085, 2.4125) -- (0.1549, -0.1085, 2.4063) -- cycle;
\fill[blue!21.5, opacity=0.5] (0.1549, -0.1085, 2.4063) -- (0.2013, -0.1085, 2.4125) -- (0.2030, -0.1103, 2.4625) -- (0.1566, -0.1103, 2.4563) -- cycle;
\fill[blue!22.4, opacity=0.5] (0.1566, -0.1103, 2.4563) -- (0.2030, -0.1103, 2.4625) -- (0.2045, -0.1120, 2.5125) -- (0.1582, -0.1120, 2.5063) -- cycle;
\fill[blue!23.3, opacity=0.5] (0.1582, -0.1120, 2.5063) -- (0.2045, -0.1120, 2.5125) -- (0.2059, -0.1135, 2.5625) -- (0.1597, -0.1135, 2.5563) -- cycle;
\fill[blue!24.3, opacity=0.5] (0.1597, -0.1135, 2.5563) -- (0.2059, -0.1135, 2.5625) -- (0.2072, -0.1148, 2.6125) -- (0.1610, -0.1148, 2.6063) -- cycle;
\fill[blue!25.3, opacity=0.5] (0.1610, -0.1148, 2.6063) -- (0.2072, -0.1148, 2.6125) -- (0.2083, -0.1160, 2.6625) -- (0.1621, -0.1160, 2.6563) -- cycle;
\fill[blue!26.4, opacity=0.5] (0.1621, -0.1160, 2.6563) -- (0.2083, -0.1160, 2.6625) -- (0.2093, -0.1171, 2.7125) -- (0.1632, -0.1171, 2.7063) -- cycle;
\fill[blue!27.5, opacity=0.5] (0.1632, -0.1171, 2.7063) -- (0.2093, -0.1171, 2.7125) -- (0.2101, -0.1180, 2.7625) -- (0.1640, -0.1180, 2.7563) -- cycle;
\fill[blue!28.5, opacity=0.5] (0.1640, -0.1180, 2.7563) -- (0.2101, -0.1180, 2.7625) -- (0.2108, -0.1187, 2.8125) -- (0.1647, -0.1187, 2.8063) -- cycle;
\fill[blue!29.6, opacity=0.5] (0.1647, -0.1187, 2.8063) -- (0.2108, -0.1187, 2.8125) -- (0.2113, -0.1193, 2.8625) -- (0.1653, -0.1193, 2.8563) -- cycle;
\fill[blue!30.6, opacity=0.5] (0.1653, -0.1193, 2.8563) -- (0.2113, -0.1193, 2.8625) -- (0.2117, -0.1197, 2.9125) -- (0.1657, -0.1197, 2.9063) -- cycle;
\fill[blue!31.6, opacity=0.5] (0.1657, -0.1197, 2.9063) -- (0.2117, -0.1197, 2.9125) -- (0.2119, -0.1199, 2.9625) -- (0.1659, -0.1199, 2.9563) -- cycle;
\fill[blue!32.6, opacity=0.5] (0.1659, -0.1199, 2.9563) -- (0.2119, -0.1199, 2.9625) -- (0.2120, -0.1200, 3.0125) -- (0.1660, -0.1200, 3.0063) -- cycle;
\fill[blue!15.0, opacity=0.5] (0.1000, -0.0000, 0.0125) -- (0.1500, -0.0000, 0.0188) -- (0.1501, -0.0001, 0.0688) -- (0.1001, -0.0001, 0.0625) -- cycle;
\fill[blue!15.0, opacity=0.5] (0.1001, -0.0001, 0.0625) -- (0.1501, -0.0001, 0.0688) -- (0.1503, -0.0003, 0.1188) -- (0.1003, -0.0003, 0.1125) -- cycle;
\fill[blue!15.0, opacity=0.5] (0.1003, -0.0003, 0.1125) -- (0.1503, -0.0003, 0.1188) -- (0.1507, -0.0007, 0.1688) -- (0.1007, -0.0007, 0.1625) -- cycle;
\fill[blue!15.0, opacity=0.5] (0.1007, -0.0007, 0.1625) -- (0.1507, -0.0007, 0.1688) -- (0.1512, -0.0013, 0.2188) -- (0.1012, -0.0013, 0.2125) -- cycle;
\fill[blue!15.0, opacity=0.5] (0.1012, -0.0013, 0.2125) -- (0.1512, -0.0013, 0.2188) -- (0.1518, -0.0020, 0.2688) -- (0.1019, -0.0020, 0.2625) -- cycle;
\fill[blue!15.0, opacity=0.5] (0.1019, -0.0020, 0.2625) -- (0.1518, -0.0020, 0.2688) -- (0.1526, -0.0029, 0.3188) -- (0.1027, -0.0029, 0.3125) -- cycle;
\fill[blue!15.0, opacity=0.5] (0.1027, -0.0029, 0.3125) -- (0.1526, -0.0029, 0.3188) -- (0.1536, -0.0040, 0.3688) -- (0.1037, -0.0040, 0.3625) -- cycle;
\fill[blue!15.0, opacity=0.5] (0.1037, -0.0040, 0.3625) -- (0.1536, -0.0040, 0.3688) -- (0.1547, -0.0052, 0.4188) -- (0.1048, -0.0052, 0.4125) -- cycle;
\fill[blue!15.0, opacity=0.5] (0.1048, -0.0052, 0.4125) -- (0.1547, -0.0052, 0.4188) -- (0.1559, -0.0065, 0.4688) -- (0.1061, -0.0065, 0.4625) -- cycle;
\fill[blue!15.0, opacity=0.5] (0.1061, -0.0065, 0.4625) -- (0.1559, -0.0065, 0.4688) -- (0.1572, -0.0080, 0.5188) -- (0.1075, -0.0080, 0.5125) -- cycle;
\fill[blue!15.0, opacity=0.5] (0.1075, -0.0080, 0.5125) -- (0.1572, -0.0080, 0.5188) -- (0.1587, -0.0097, 0.5688) -- (0.1090, -0.0097, 0.5625) -- cycle;
\fill[blue!15.0, opacity=0.5] (0.1090, -0.0097, 0.5625) -- (0.1587, -0.0097, 0.5688) -- (0.1603, -0.0115, 0.6188) -- (0.1107, -0.0115, 0.6125) -- cycle;
\fill[blue!15.0, opacity=0.5] (0.1107, -0.0115, 0.6125) -- (0.1603, -0.0115, 0.6188) -- (0.1620, -0.0134, 0.6688) -- (0.1125, -0.0134, 0.6625) -- cycle;
\fill[blue!15.0, opacity=0.5] (0.1125, -0.0134, 0.6625) -- (0.1620, -0.0134, 0.6688) -- (0.1639, -0.0154, 0.7188) -- (0.1144, -0.0154, 0.7125) -- cycle;
\fill[blue!15.0, opacity=0.5] (0.1144, -0.0154, 0.7125) -- (0.1639, -0.0154, 0.7188) -- (0.1658, -0.0176, 0.7688) -- (0.1164, -0.0176, 0.7625) -- cycle;
\fill[blue!15.0, opacity=0.5] (0.1164, -0.0176, 0.7625) -- (0.1658, -0.0176, 0.7688) -- (0.1679, -0.0199, 0.8188) -- (0.1185, -0.0199, 0.8125) -- cycle;
\fill[blue!15.0, opacity=0.5] (0.1185, -0.0199, 0.8125) -- (0.1679, -0.0199, 0.8188) -- (0.1700, -0.0222, 0.8688) -- (0.1208, -0.0222, 0.8625) -- cycle;
\fill[blue!15.0, opacity=0.5] (0.1208, -0.0222, 0.8625) -- (0.1700, -0.0222, 0.8688) -- (0.1723, -0.0247, 0.9188) -- (0.1231, -0.0247, 0.9125) -- cycle;
\fill[blue!15.0, opacity=0.5] (0.1231, -0.0247, 0.9125) -- (0.1723, -0.0247, 0.9188) -- (0.1746, -0.0273, 0.9688) -- (0.1255, -0.0273, 0.9625) -- cycle;
\fill[blue!15.0, opacity=0.5] (0.1255, -0.0273, 0.9625) -- (0.1746, -0.0273, 0.9688) -- (0.1770, -0.0300, 1.0188) -- (0.1280, -0.0300, 1.0125) -- cycle;
\fill[blue!15.0, opacity=0.5] (0.1280, -0.0300, 1.0125) -- (0.1770, -0.0300, 1.0188) -- (0.1795, -0.0328, 1.0688) -- (0.1306, -0.0328, 1.0625) -- cycle;
\fill[blue!15.0, opacity=0.5] (0.1306, -0.0328, 1.0625) -- (0.1795, -0.0328, 1.0688) -- (0.1820, -0.0356, 1.1188) -- (0.1332, -0.0356, 1.1125) -- cycle;
\fill[blue!15.0, opacity=0.5] (0.1332, -0.0356, 1.1125) -- (0.1820, -0.0356, 1.1188) -- (0.1846, -0.0385, 1.1688) -- (0.1359, -0.0385, 1.1625) -- cycle;
\fill[blue!15.0, opacity=0.5] (0.1359, -0.0385, 1.1625) -- (0.1846, -0.0385, 1.1688) -- (0.1873, -0.0415, 1.2188) -- (0.1387, -0.0415, 1.2125) -- cycle;
\fill[blue!15.0, opacity=0.5] (0.1387, -0.0415, 1.2125) -- (0.1873, -0.0415, 1.2188) -- (0.1900, -0.0445, 1.2688) -- (0.1415, -0.0445, 1.2625) -- cycle;
\fill[blue!15.0, opacity=0.5] (0.1415, -0.0445, 1.2625) -- (0.1900, -0.0445, 1.2688) -- (0.1928, -0.0475, 1.3188) -- (0.1444, -0.0475, 1.3125) -- cycle;
\fill[blue!15.0, opacity=0.5] (0.1444, -0.0475, 1.3125) -- (0.1928, -0.0475, 1.3188) -- (0.1956, -0.0506, 1.3688) -- (0.1472, -0.0506, 1.3625) -- cycle;
\fill[blue!15.0, opacity=0.5] (0.1472, -0.0506, 1.3625) -- (0.1956, -0.0506, 1.3688) -- (0.1984, -0.0537, 1.4188) -- (0.1501, -0.0537, 1.4125) -- cycle;
\fill[blue!15.0, opacity=0.5] (0.1501, -0.0537, 1.4125) -- (0.1984, -0.0537, 1.4188) -- (0.2012, -0.0569, 1.4688) -- (0.1531, -0.0569, 1.4625) -- cycle;
\fill[blue!15.0, opacity=0.5] (0.1531, -0.0569, 1.4625) -- (0.2012, -0.0569, 1.4688) -- (0.2040, -0.0600, 1.5188) -- (0.1560, -0.0600, 1.5125) -- cycle;
\fill[blue!15.0, opacity=0.5] (0.1560, -0.0600, 1.5125) -- (0.2040, -0.0600, 1.5188) -- (0.2068, -0.0631, 1.5688) -- (0.1589, -0.0631, 1.5625) -- cycle;
\fill[blue!15.0, opacity=0.5] (0.1589, -0.0631, 1.5625) -- (0.2068, -0.0631, 1.5688) -- (0.2096, -0.0663, 1.6188) -- (0.1619, -0.0663, 1.6125) -- cycle;
\fill[blue!15.0, opacity=0.5] (0.1619, -0.0663, 1.6125) -- (0.2096, -0.0663, 1.6188) -- (0.2124, -0.0694, 1.6688) -- (0.1648, -0.0694, 1.6625) -- cycle;
\fill[blue!15.1, opacity=0.5] (0.1648, -0.0694, 1.6625) -- (0.2124, -0.0694, 1.6688) -- (0.2152, -0.0725, 1.7188) -- (0.1676, -0.0725, 1.7125) -- cycle;
\fill[blue!15.1, opacity=0.5] (0.1676, -0.0725, 1.7125) -- (0.2152, -0.0725, 1.7188) -- (0.2180, -0.0755, 1.7688) -- (0.1705, -0.0755, 1.7625) -- cycle;
\fill[blue!15.1, opacity=0.5] (0.1705, -0.0755, 1.7625) -- (0.2180, -0.0755, 1.7688) -- (0.2207, -0.0785, 1.8188) -- (0.1733, -0.0785, 1.8125) -- cycle;
\fill[blue!15.2, opacity=0.5] (0.1733, -0.0785, 1.8125) -- (0.2207, -0.0785, 1.8188) -- (0.2234, -0.0815, 1.8688) -- (0.1761, -0.0815, 1.8625) -- cycle;
\fill[blue!15.3, opacity=0.5] (0.1761, -0.0815, 1.8625) -- (0.2234, -0.0815, 1.8688) -- (0.2260, -0.0844, 1.9188) -- (0.1788, -0.0844, 1.9125) -- cycle;
\fill[blue!15.4, opacity=0.5] (0.1788, -0.0844, 1.9125) -- (0.2260, -0.0844, 1.9188) -- (0.2285, -0.0872, 1.9688) -- (0.1814, -0.0872, 1.9625) -- cycle;
\fill[blue!15.5, opacity=0.5] (0.1814, -0.0872, 1.9625) -- (0.2285, -0.0872, 1.9688) -- (0.2310, -0.0900, 2.0188) -- (0.1840, -0.0900, 2.0125) -- cycle;
\fill[blue!15.7, opacity=0.5] (0.1840, -0.0900, 2.0125) -- (0.2310, -0.0900, 2.0188) -- (0.2334, -0.0927, 2.0688) -- (0.1865, -0.0927, 2.0625) -- cycle;
\fill[blue!15.9, opacity=0.5] (0.1865, -0.0927, 2.0625) -- (0.2334, -0.0927, 2.0688) -- (0.2357, -0.0953, 2.1188) -- (0.1889, -0.0953, 2.1125) -- cycle;
\fill[blue!16.1, opacity=0.5] (0.1889, -0.0953, 2.1125) -- (0.2357, -0.0953, 2.1188) -- (0.2380, -0.0978, 2.1688) -- (0.1912, -0.0978, 2.1625) -- cycle;
\fill[blue!16.4, opacity=0.5] (0.1912, -0.0978, 2.1625) -- (0.2380, -0.0978, 2.1688) -- (0.2401, -0.1001, 2.2188) -- (0.1935, -0.1001, 2.2125) -- cycle;
\fill[blue!16.8, opacity=0.5] (0.1935, -0.1001, 2.2125) -- (0.2401, -0.1001, 2.2188) -- (0.2422, -0.1024, 2.2688) -- (0.1956, -0.1024, 2.2625) -- cycle;
\fill[blue!17.2, opacity=0.5] (0.1956, -0.1024, 2.2625) -- (0.2422, -0.1024, 2.2688) -- (0.2441, -0.1046, 2.3188) -- (0.1976, -0.1046, 2.3125) -- cycle;
\fill[blue!17.6, opacity=0.5] (0.1976, -0.1046, 2.3125) -- (0.2441, -0.1046, 2.3188) -- (0.2460, -0.1066, 2.3688) -- (0.1995, -0.1066, 2.3625) -- cycle;
\fill[blue!18.1, opacity=0.5] (0.1995, -0.1066, 2.3625) -- (0.2460, -0.1066, 2.3688) -- (0.2477, -0.1085, 2.4188) -- (0.2013, -0.1085, 2.4125) -- cycle;
\fill[blue!18.7, opacity=0.5] (0.2013, -0.1085, 2.4125) -- (0.2477, -0.1085, 2.4188) -- (0.2493, -0.1103, 2.4688) -- (0.2030, -0.1103, 2.4625) -- cycle;
\fill[blue!19.3, opacity=0.5] (0.2030, -0.1103, 2.4625) -- (0.2493, -0.1103, 2.4688) -- (0.2508, -0.1120, 2.5188) -- (0.2045, -0.1120, 2.5125) -- cycle;
\fill[blue!20.0, opacity=0.5] (0.2045, -0.1120, 2.5125) -- (0.2508, -0.1120, 2.5188) -- (0.2521, -0.1135, 2.5688) -- (0.2059, -0.1135, 2.5625) -- cycle;
\fill[blue!20.7, opacity=0.5] (0.2059, -0.1135, 2.5625) -- (0.2521, -0.1135, 2.5688) -- (0.2533, -0.1148, 2.6188) -- (0.2072, -0.1148, 2.6125) -- cycle;
\fill[blue!21.5, opacity=0.5] (0.2072, -0.1148, 2.6125) -- (0.2533, -0.1148, 2.6188) -- (0.2544, -0.1160, 2.6688) -- (0.2083, -0.1160, 2.6625) -- cycle;
\fill[blue!22.3, opacity=0.5] (0.2083, -0.1160, 2.6625) -- (0.2544, -0.1160, 2.6688) -- (0.2554, -0.1171, 2.7188) -- (0.2093, -0.1171, 2.7125) -- cycle;
\fill[blue!23.1, opacity=0.5] (0.2093, -0.1171, 2.7125) -- (0.2554, -0.1171, 2.7188) -- (0.2562, -0.1180, 2.7688) -- (0.2101, -0.1180, 2.7625) -- cycle;
\fill[blue!24.0, opacity=0.5] (0.2101, -0.1180, 2.7625) -- (0.2562, -0.1180, 2.7688) -- (0.2568, -0.1187, 2.8188) -- (0.2108, -0.1187, 2.8125) -- cycle;
\fill[blue!24.9, opacity=0.5] (0.2108, -0.1187, 2.8125) -- (0.2568, -0.1187, 2.8188) -- (0.2573, -0.1193, 2.8688) -- (0.2113, -0.1193, 2.8625) -- cycle;
\fill[blue!25.8, opacity=0.5] (0.2113, -0.1193, 2.8625) -- (0.2573, -0.1193, 2.8688) -- (0.2577, -0.1197, 2.9188) -- (0.2117, -0.1197, 2.9125) -- cycle;
\fill[blue!26.7, opacity=0.5] (0.2117, -0.1197, 2.9125) -- (0.2577, -0.1197, 2.9188) -- (0.2579, -0.1199, 2.9688) -- (0.2119, -0.1199, 2.9625) -- cycle;
\fill[blue!27.6, opacity=0.5] (0.2119, -0.1199, 2.9625) -- (0.2579, -0.1199, 2.9688) -- (0.2580, -0.1200, 3.0188) -- (0.2120, -0.1200, 3.0125) -- cycle;
\fill[blue!15.0, opacity=0.5] (0.1500, -0.0000, 0.0188) -- (0.2000, -0.0000, 0.0249) -- (0.2001, -0.0001, 0.0749) -- (0.1501, -0.0001, 0.0688) -- cycle;
\fill[blue!15.0, opacity=0.5] (0.1501, -0.0001, 0.0688) -- (0.2001, -0.0001, 0.0749) -- (0.2003, -0.0003, 0.1249) -- (0.1503, -0.0003, 0.1188) -- cycle;
\fill[blue!15.0, opacity=0.5] (0.1503, -0.0003, 0.1188) -- (0.2003, -0.0003, 0.1249) -- (0.2006, -0.0007, 0.1749) -- (0.1507, -0.0007, 0.1688) -- cycle;
\fill[blue!15.0, opacity=0.5] (0.1507, -0.0007, 0.1688) -- (0.2006, -0.0007, 0.1749) -- (0.2011, -0.0013, 0.2249) -- (0.1512, -0.0013, 0.2188) -- cycle;
\fill[blue!15.0, opacity=0.5] (0.1512, -0.0013, 0.2188) -- (0.2011, -0.0013, 0.2249) -- (0.2018, -0.0020, 0.2749) -- (0.1518, -0.0020, 0.2688) -- cycle;
\fill[blue!15.0, opacity=0.5] (0.1518, -0.0020, 0.2688) -- (0.2018, -0.0020, 0.2749) -- (0.2025, -0.0029, 0.3249) -- (0.1526, -0.0029, 0.3188) -- cycle;
\fill[blue!15.0, opacity=0.5] (0.1526, -0.0029, 0.3188) -- (0.2025, -0.0029, 0.3249) -- (0.2035, -0.0040, 0.3749) -- (0.1536, -0.0040, 0.3688) -- cycle;
\fill[blue!15.0, opacity=0.5] (0.1536, -0.0040, 0.3688) -- (0.2035, -0.0040, 0.3749) -- (0.2045, -0.0052, 0.4249) -- (0.1547, -0.0052, 0.4188) -- cycle;
\fill[blue!15.0, opacity=0.5] (0.1547, -0.0052, 0.4188) -- (0.2045, -0.0052, 0.4249) -- (0.2057, -0.0065, 0.4749) -- (0.1559, -0.0065, 0.4688) -- cycle;
\fill[blue!15.0, opacity=0.5] (0.1559, -0.0065, 0.4688) -- (0.2057, -0.0065, 0.4749) -- (0.2070, -0.0080, 0.5249) -- (0.1572, -0.0080, 0.5188) -- cycle;
\fill[blue!15.0, opacity=0.5] (0.1572, -0.0080, 0.5188) -- (0.2070, -0.0080, 0.5249) -- (0.2084, -0.0097, 0.5749) -- (0.1587, -0.0097, 0.5688) -- cycle;
\fill[blue!15.0, opacity=0.5] (0.1587, -0.0097, 0.5688) -- (0.2084, -0.0097, 0.5749) -- (0.2099, -0.0115, 0.6249) -- (0.1603, -0.0115, 0.6188) -- cycle;
\fill[blue!15.0, opacity=0.5] (0.1603, -0.0115, 0.6188) -- (0.2099, -0.0115, 0.6249) -- (0.2116, -0.0134, 0.6749) -- (0.1620, -0.0134, 0.6688) -- cycle;
\fill[blue!15.0, opacity=0.5] (0.1620, -0.0134, 0.6688) -- (0.2116, -0.0134, 0.6749) -- (0.2134, -0.0154, 0.7249) -- (0.1639, -0.0154, 0.7188) -- cycle;
\fill[blue!15.0, opacity=0.5] (0.1639, -0.0154, 0.7188) -- (0.2134, -0.0154, 0.7249) -- (0.2152, -0.0176, 0.7749) -- (0.1658, -0.0176, 0.7688) -- cycle;
\fill[blue!15.0, opacity=0.5] (0.1658, -0.0176, 0.7688) -- (0.2152, -0.0176, 0.7749) -- (0.2172, -0.0199, 0.8249) -- (0.1679, -0.0199, 0.8188) -- cycle;
\fill[blue!15.0, opacity=0.5] (0.1679, -0.0199, 0.8188) -- (0.2172, -0.0199, 0.8249) -- (0.2193, -0.0222, 0.8749) -- (0.1700, -0.0222, 0.8688) -- cycle;
\fill[blue!15.0, opacity=0.5] (0.1700, -0.0222, 0.8688) -- (0.2193, -0.0222, 0.8749) -- (0.2214, -0.0247, 0.9249) -- (0.1723, -0.0247, 0.9188) -- cycle;
\fill[blue!15.0, opacity=0.5] (0.1723, -0.0247, 0.9188) -- (0.2214, -0.0247, 0.9249) -- (0.2237, -0.0273, 0.9749) -- (0.1746, -0.0273, 0.9688) -- cycle;
\fill[blue!15.0, opacity=0.5] (0.1746, -0.0273, 0.9688) -- (0.2237, -0.0273, 0.9749) -- (0.2260, -0.0300, 1.0249) -- (0.1770, -0.0300, 1.0188) -- cycle;
\fill[blue!15.0, opacity=0.5] (0.1770, -0.0300, 1.0188) -- (0.2260, -0.0300, 1.0249) -- (0.2284, -0.0328, 1.0749) -- (0.1795, -0.0328, 1.0688) -- cycle;
\fill[blue!15.0, opacity=0.5] (0.1795, -0.0328, 1.0688) -- (0.2284, -0.0328, 1.0749) -- (0.2308, -0.0356, 1.1249) -- (0.1820, -0.0356, 1.1188) -- cycle;
\fill[blue!15.0, opacity=0.5] (0.1820, -0.0356, 1.1188) -- (0.2308, -0.0356, 1.1249) -- (0.2334, -0.0385, 1.1749) -- (0.1846, -0.0385, 1.1688) -- cycle;
\fill[blue!15.0, opacity=0.5] (0.1846, -0.0385, 1.1688) -- (0.2334, -0.0385, 1.1749) -- (0.2359, -0.0415, 1.2249) -- (0.1873, -0.0415, 1.2188) -- cycle;
\fill[blue!15.0, opacity=0.5] (0.1873, -0.0415, 1.2188) -- (0.2359, -0.0415, 1.2249) -- (0.2385, -0.0445, 1.2749) -- (0.1900, -0.0445, 1.2688) -- cycle;
\fill[blue!15.0, opacity=0.5] (0.1900, -0.0445, 1.2688) -- (0.2385, -0.0445, 1.2749) -- (0.2412, -0.0475, 1.3249) -- (0.1928, -0.0475, 1.3188) -- cycle;
\fill[blue!15.0, opacity=0.5] (0.1928, -0.0475, 1.3188) -- (0.2412, -0.0475, 1.3249) -- (0.2439, -0.0506, 1.3749) -- (0.1956, -0.0506, 1.3688) -- cycle;
\fill[blue!15.0, opacity=0.5] (0.1956, -0.0506, 1.3688) -- (0.2439, -0.0506, 1.3749) -- (0.2466, -0.0537, 1.4249) -- (0.1984, -0.0537, 1.4188) -- cycle;
\fill[blue!15.0, opacity=0.5] (0.1984, -0.0537, 1.4188) -- (0.2466, -0.0537, 1.4249) -- (0.2493, -0.0569, 1.4749) -- (0.2012, -0.0569, 1.4688) -- cycle;
\fill[blue!15.0, opacity=0.5] (0.2012, -0.0569, 1.4688) -- (0.2493, -0.0569, 1.4749) -- (0.2520, -0.0600, 1.5249) -- (0.2040, -0.0600, 1.5188) -- cycle;
\fill[blue!15.0, opacity=0.5] (0.2040, -0.0600, 1.5188) -- (0.2520, -0.0600, 1.5249) -- (0.2547, -0.0631, 1.5749) -- (0.2068, -0.0631, 1.5688) -- cycle;
\fill[blue!15.0, opacity=0.5] (0.2068, -0.0631, 1.5688) -- (0.2547, -0.0631, 1.5749) -- (0.2574, -0.0663, 1.6249) -- (0.2096, -0.0663, 1.6188) -- cycle;
\fill[blue!15.0, opacity=0.5] (0.2096, -0.0663, 1.6188) -- (0.2574, -0.0663, 1.6249) -- (0.2601, -0.0694, 1.6749) -- (0.2124, -0.0694, 1.6688) -- cycle;
\fill[blue!15.0, opacity=0.5] (0.2124, -0.0694, 1.6688) -- (0.2601, -0.0694, 1.6749) -- (0.2628, -0.0725, 1.7249) -- (0.2152, -0.0725, 1.7188) -- cycle;
\fill[blue!15.0, opacity=0.5] (0.2152, -0.0725, 1.7188) -- (0.2628, -0.0725, 1.7249) -- (0.2655, -0.0755, 1.7749) -- (0.2180, -0.0755, 1.7688) -- cycle;
\fill[blue!15.0, opacity=0.5] (0.2180, -0.0755, 1.7688) -- (0.2655, -0.0755, 1.7749) -- (0.2681, -0.0785, 1.8249) -- (0.2207, -0.0785, 1.8188) -- cycle;
\fill[blue!15.0, opacity=0.5] (0.2207, -0.0785, 1.8188) -- (0.2681, -0.0785, 1.8249) -- (0.2706, -0.0815, 1.8749) -- (0.2234, -0.0815, 1.8688) -- cycle;
\fill[blue!15.0, opacity=0.5] (0.2234, -0.0815, 1.8688) -- (0.2706, -0.0815, 1.8749) -- (0.2732, -0.0844, 1.9249) -- (0.2260, -0.0844, 1.9188) -- cycle;
\fill[blue!15.1, opacity=0.5] (0.2260, -0.0844, 1.9188) -- (0.2732, -0.0844, 1.9249) -- (0.2756, -0.0872, 1.9749) -- (0.2285, -0.0872, 1.9688) -- cycle;
\fill[blue!15.1, opacity=0.5] (0.2285, -0.0872, 1.9688) -- (0.2756, -0.0872, 1.9749) -- (0.2780, -0.0900, 2.0249) -- (0.2310, -0.0900, 2.0188) -- cycle;
\fill[blue!15.1, opacity=0.5] (0.2310, -0.0900, 2.0188) -- (0.2780, -0.0900, 2.0249) -- (0.2803, -0.0927, 2.0749) -- (0.2334, -0.0927, 2.0688) -- cycle;
\fill[blue!15.2, opacity=0.5] (0.2334, -0.0927, 2.0688) -- (0.2803, -0.0927, 2.0749) -- (0.2826, -0.0953, 2.1249) -- (0.2357, -0.0953, 2.1188) -- cycle;
\fill[blue!15.3, opacity=0.5] (0.2357, -0.0953, 2.1188) -- (0.2826, -0.0953, 2.1249) -- (0.2847, -0.0978, 2.1749) -- (0.2380, -0.0978, 2.1688) -- cycle;
\fill[blue!15.3, opacity=0.5] (0.2380, -0.0978, 2.1688) -- (0.2847, -0.0978, 2.1749) -- (0.2868, -0.1001, 2.2249) -- (0.2401, -0.1001, 2.2188) -- cycle;
\fill[blue!15.4, opacity=0.5] (0.2401, -0.1001, 2.2188) -- (0.2868, -0.1001, 2.2249) -- (0.2888, -0.1024, 2.2749) -- (0.2422, -0.1024, 2.2688) -- cycle;
\fill[blue!15.6, opacity=0.5] (0.2422, -0.1024, 2.2688) -- (0.2888, -0.1024, 2.2749) -- (0.2906, -0.1046, 2.3249) -- (0.2441, -0.1046, 2.3188) -- cycle;
\fill[blue!15.7, opacity=0.5] (0.2441, -0.1046, 2.3188) -- (0.2906, -0.1046, 2.3249) -- (0.2924, -0.1066, 2.3749) -- (0.2460, -0.1066, 2.3688) -- cycle;
\fill[blue!15.9, opacity=0.5] (0.2460, -0.1066, 2.3688) -- (0.2924, -0.1066, 2.3749) -- (0.2941, -0.1085, 2.4249) -- (0.2477, -0.1085, 2.4188) -- cycle;
\fill[blue!16.2, opacity=0.5] (0.2477, -0.1085, 2.4188) -- (0.2941, -0.1085, 2.4249) -- (0.2956, -0.1103, 2.4749) -- (0.2493, -0.1103, 2.4688) -- cycle;
\fill[blue!16.4, opacity=0.5] (0.2493, -0.1103, 2.4688) -- (0.2956, -0.1103, 2.4749) -- (0.2970, -0.1120, 2.5249) -- (0.2508, -0.1120, 2.5188) -- cycle;
\fill[blue!16.7, opacity=0.5] (0.2508, -0.1120, 2.5188) -- (0.2970, -0.1120, 2.5249) -- (0.2983, -0.1135, 2.5749) -- (0.2521, -0.1135, 2.5688) -- cycle;
\fill[blue!17.1, opacity=0.5] (0.2521, -0.1135, 2.5688) -- (0.2983, -0.1135, 2.5749) -- (0.2995, -0.1148, 2.6249) -- (0.2533, -0.1148, 2.6188) -- cycle;
\fill[blue!17.4, opacity=0.5] (0.2533, -0.1148, 2.6188) -- (0.2995, -0.1148, 2.6249) -- (0.3005, -0.1160, 2.6749) -- (0.2544, -0.1160, 2.6688) -- cycle;
\fill[blue!17.8, opacity=0.5] (0.2544, -0.1160, 2.6688) -- (0.3005, -0.1160, 2.6749) -- (0.3015, -0.1171, 2.7249) -- (0.2554, -0.1171, 2.7188) -- cycle;
\fill[blue!18.3, opacity=0.5] (0.2554, -0.1171, 2.7188) -- (0.3015, -0.1171, 2.7249) -- (0.3022, -0.1180, 2.7749) -- (0.2562, -0.1180, 2.7688) -- cycle;
\fill[blue!18.8, opacity=0.5] (0.2562, -0.1180, 2.7688) -- (0.3022, -0.1180, 2.7749) -- (0.3029, -0.1187, 2.8249) -- (0.2568, -0.1187, 2.8188) -- cycle;
\fill[blue!19.3, opacity=0.5] (0.2568, -0.1187, 2.8188) -- (0.3029, -0.1187, 2.8249) -- (0.3034, -0.1193, 2.8749) -- (0.2573, -0.1193, 2.8688) -- cycle;
\fill[blue!19.9, opacity=0.5] (0.2573, -0.1193, 2.8688) -- (0.3034, -0.1193, 2.8749) -- (0.3037, -0.1197, 2.9249) -- (0.2577, -0.1197, 2.9188) -- cycle;
\fill[blue!20.5, opacity=0.5] (0.2577, -0.1197, 2.9188) -- (0.3037, -0.1197, 2.9249) -- (0.3039, -0.1199, 2.9749) -- (0.2579, -0.1199, 2.9688) -- cycle;
\fill[blue!21.1, opacity=0.5] (0.2579, -0.1199, 2.9688) -- (0.3039, -0.1199, 2.9749) -- (0.3040, -0.1200, 3.0249) -- (0.2580, -0.1200, 3.0188) -- cycle;
\fill[blue!15.0, opacity=0.5] (0.2000, -0.0000, 0.0249) -- (0.2500, -0.0000, 0.0311) -- (0.2501, -0.0001, 0.0811) -- (0.2001, -0.0001, 0.0749) -- cycle;
\fill[blue!15.0, opacity=0.5] (0.2001, -0.0001, 0.0749) -- (0.2501, -0.0001, 0.0811) -- (0.2503, -0.0003, 0.1311) -- (0.2003, -0.0003, 0.1249) -- cycle;
\fill[blue!15.0, opacity=0.5] (0.2003, -0.0003, 0.1249) -- (0.2503, -0.0003, 0.1311) -- (0.2506, -0.0007, 0.1811) -- (0.2006, -0.0007, 0.1749) -- cycle;
\fill[blue!15.0, opacity=0.5] (0.2006, -0.0007, 0.1749) -- (0.2506, -0.0007, 0.1811) -- (0.2511, -0.0013, 0.2311) -- (0.2011, -0.0013, 0.2249) -- cycle;
\fill[blue!15.0, opacity=0.5] (0.2011, -0.0013, 0.2249) -- (0.2511, -0.0013, 0.2311) -- (0.2517, -0.0020, 0.2811) -- (0.2018, -0.0020, 0.2749) -- cycle;
\fill[blue!15.0, opacity=0.5] (0.2018, -0.0020, 0.2749) -- (0.2517, -0.0020, 0.2811) -- (0.2524, -0.0029, 0.3311) -- (0.2025, -0.0029, 0.3249) -- cycle;
\fill[blue!15.0, opacity=0.5] (0.2025, -0.0029, 0.3249) -- (0.2524, -0.0029, 0.3311) -- (0.2533, -0.0040, 0.3811) -- (0.2035, -0.0040, 0.3749) -- cycle;
\fill[blue!15.0, opacity=0.5] (0.2035, -0.0040, 0.3749) -- (0.2533, -0.0040, 0.3811) -- (0.2543, -0.0052, 0.4311) -- (0.2045, -0.0052, 0.4249) -- cycle;
\fill[blue!15.0, opacity=0.5] (0.2045, -0.0052, 0.4249) -- (0.2543, -0.0052, 0.4311) -- (0.2554, -0.0065, 0.4811) -- (0.2057, -0.0065, 0.4749) -- cycle;
\fill[blue!15.0, opacity=0.5] (0.2057, -0.0065, 0.4749) -- (0.2554, -0.0065, 0.4811) -- (0.2567, -0.0080, 0.5311) -- (0.2070, -0.0080, 0.5249) -- cycle;
\fill[blue!15.0, opacity=0.5] (0.2070, -0.0080, 0.5249) -- (0.2567, -0.0080, 0.5311) -- (0.2581, -0.0097, 0.5811) -- (0.2084, -0.0097, 0.5749) -- cycle;
\fill[blue!15.0, opacity=0.5] (0.2084, -0.0097, 0.5749) -- (0.2581, -0.0097, 0.5811) -- (0.2595, -0.0115, 0.6311) -- (0.2099, -0.0115, 0.6249) -- cycle;
\fill[blue!15.0, opacity=0.5] (0.2099, -0.0115, 0.6249) -- (0.2595, -0.0115, 0.6311) -- (0.2611, -0.0134, 0.6811) -- (0.2116, -0.0134, 0.6749) -- cycle;
\fill[blue!15.0, opacity=0.5] (0.2116, -0.0134, 0.6749) -- (0.2611, -0.0134, 0.6811) -- (0.2628, -0.0154, 0.7311) -- (0.2134, -0.0154, 0.7249) -- cycle;
\fill[blue!15.0, opacity=0.5] (0.2134, -0.0154, 0.7249) -- (0.2628, -0.0154, 0.7311) -- (0.2646, -0.0176, 0.7811) -- (0.2152, -0.0176, 0.7749) -- cycle;
\fill[blue!15.0, opacity=0.5] (0.2152, -0.0176, 0.7749) -- (0.2646, -0.0176, 0.7811) -- (0.2665, -0.0199, 0.8311) -- (0.2172, -0.0199, 0.8249) -- cycle;
\fill[blue!15.0, opacity=0.5] (0.2172, -0.0199, 0.8249) -- (0.2665, -0.0199, 0.8311) -- (0.2685, -0.0222, 0.8811) -- (0.2193, -0.0222, 0.8749) -- cycle;
\fill[blue!15.0, opacity=0.5] (0.2193, -0.0222, 0.8749) -- (0.2685, -0.0222, 0.8811) -- (0.2706, -0.0247, 0.9311) -- (0.2214, -0.0247, 0.9249) -- cycle;
\fill[blue!15.0, opacity=0.5] (0.2214, -0.0247, 0.9249) -- (0.2706, -0.0247, 0.9311) -- (0.2728, -0.0273, 0.9811) -- (0.2237, -0.0273, 0.9749) -- cycle;
\fill[blue!15.0, opacity=0.5] (0.2237, -0.0273, 0.9749) -- (0.2728, -0.0273, 0.9811) -- (0.2750, -0.0300, 1.0311) -- (0.2260, -0.0300, 1.0249) -- cycle;
\fill[blue!15.0, opacity=0.5] (0.2260, -0.0300, 1.0249) -- (0.2750, -0.0300, 1.0311) -- (0.2773, -0.0328, 1.0811) -- (0.2284, -0.0328, 1.0749) -- cycle;
\fill[blue!15.0, opacity=0.5] (0.2284, -0.0328, 1.0749) -- (0.2773, -0.0328, 1.0811) -- (0.2797, -0.0356, 1.1311) -- (0.2308, -0.0356, 1.1249) -- cycle;
\fill[blue!15.0, opacity=0.5] (0.2308, -0.0356, 1.1249) -- (0.2797, -0.0356, 1.1311) -- (0.2821, -0.0385, 1.1811) -- (0.2334, -0.0385, 1.1749) -- cycle;
\fill[blue!15.0, opacity=0.5] (0.2334, -0.0385, 1.1749) -- (0.2821, -0.0385, 1.1811) -- (0.2845, -0.0415, 1.2311) -- (0.2359, -0.0415, 1.2249) -- cycle;
\fill[blue!15.0, opacity=0.5] (0.2359, -0.0415, 1.2249) -- (0.2845, -0.0415, 1.2311) -- (0.2871, -0.0445, 1.2811) -- (0.2385, -0.0445, 1.2749) -- cycle;
\fill[blue!15.0, opacity=0.5] (0.2385, -0.0445, 1.2749) -- (0.2871, -0.0445, 1.2811) -- (0.2896, -0.0475, 1.3311) -- (0.2412, -0.0475, 1.3249) -- cycle;
\fill[blue!15.0, opacity=0.5] (0.2412, -0.0475, 1.3249) -- (0.2896, -0.0475, 1.3311) -- (0.2922, -0.0506, 1.3811) -- (0.2439, -0.0506, 1.3749) -- cycle;
\fill[blue!15.0, opacity=0.5] (0.2439, -0.0506, 1.3749) -- (0.2922, -0.0506, 1.3811) -- (0.2948, -0.0537, 1.4311) -- (0.2466, -0.0537, 1.4249) -- cycle;
\fill[blue!15.0, opacity=0.5] (0.2466, -0.0537, 1.4249) -- (0.2948, -0.0537, 1.4311) -- (0.2974, -0.0569, 1.4811) -- (0.2493, -0.0569, 1.4749) -- cycle;
\fill[blue!15.0, opacity=0.5] (0.2493, -0.0569, 1.4749) -- (0.2974, -0.0569, 1.4811) -- (0.3000, -0.0600, 1.5311) -- (0.2520, -0.0600, 1.5249) -- cycle;
\fill[blue!15.0, opacity=0.5] (0.2520, -0.0600, 1.5249) -- (0.3000, -0.0600, 1.5311) -- (0.3026, -0.0631, 1.5811) -- (0.2547, -0.0631, 1.5749) -- cycle;
\fill[blue!15.0, opacity=0.5] (0.2547, -0.0631, 1.5749) -- (0.3026, -0.0631, 1.5811) -- (0.3052, -0.0663, 1.6311) -- (0.2574, -0.0663, 1.6249) -- cycle;
\fill[blue!15.0, opacity=0.5] (0.2574, -0.0663, 1.6249) -- (0.3052, -0.0663, 1.6311) -- (0.3078, -0.0694, 1.6811) -- (0.2601, -0.0694, 1.6749) -- cycle;
\fill[blue!15.0, opacity=0.5] (0.2601, -0.0694, 1.6749) -- (0.3078, -0.0694, 1.6811) -- (0.3104, -0.0725, 1.7311) -- (0.2628, -0.0725, 1.7249) -- cycle;
\fill[blue!15.0, opacity=0.5] (0.2628, -0.0725, 1.7249) -- (0.3104, -0.0725, 1.7311) -- (0.3129, -0.0755, 1.7811) -- (0.2655, -0.0755, 1.7749) -- cycle;
\fill[blue!15.0, opacity=0.5] (0.2655, -0.0755, 1.7749) -- (0.3129, -0.0755, 1.7811) -- (0.3155, -0.0785, 1.8311) -- (0.2681, -0.0785, 1.8249) -- cycle;
\fill[blue!15.0, opacity=0.5] (0.2681, -0.0785, 1.8249) -- (0.3155, -0.0785, 1.8311) -- (0.3179, -0.0815, 1.8811) -- (0.2706, -0.0815, 1.8749) -- cycle;
\fill[blue!15.0, opacity=0.5] (0.2706, -0.0815, 1.8749) -- (0.3179, -0.0815, 1.8811) -- (0.3203, -0.0844, 1.9311) -- (0.2732, -0.0844, 1.9249) -- cycle;
\fill[blue!15.0, opacity=0.5] (0.2732, -0.0844, 1.9249) -- (0.3203, -0.0844, 1.9311) -- (0.3227, -0.0872, 1.9811) -- (0.2756, -0.0872, 1.9749) -- cycle;
\fill[blue!15.0, opacity=0.5] (0.2756, -0.0872, 1.9749) -- (0.3227, -0.0872, 1.9811) -- (0.3250, -0.0900, 2.0311) -- (0.2780, -0.0900, 2.0249) -- cycle;
\fill[blue!15.0, opacity=0.5] (0.2780, -0.0900, 2.0249) -- (0.3250, -0.0900, 2.0311) -- (0.3272, -0.0927, 2.0811) -- (0.2803, -0.0927, 2.0749) -- cycle;
\fill[blue!15.0, opacity=0.5] (0.2803, -0.0927, 2.0749) -- (0.3272, -0.0927, 2.0811) -- (0.3294, -0.0953, 2.1311) -- (0.2826, -0.0953, 2.1249) -- cycle;
\fill[blue!15.0, opacity=0.5] (0.2826, -0.0953, 2.1249) -- (0.3294, -0.0953, 2.1311) -- (0.3315, -0.0978, 2.1811) -- (0.2847, -0.0978, 2.1749) -- cycle;
\fill[blue!15.0, opacity=0.5] (0.2847, -0.0978, 2.1749) -- (0.3315, -0.0978, 2.1811) -- (0.3335, -0.1001, 2.2311) -- (0.2868, -0.1001, 2.2249) -- cycle;
\fill[blue!15.1, opacity=0.5] (0.2868, -0.1001, 2.2249) -- (0.3335, -0.1001, 2.2311) -- (0.3354, -0.1024, 2.2811) -- (0.2888, -0.1024, 2.2749) -- cycle;
\fill[blue!15.1, opacity=0.5] (0.2888, -0.1024, 2.2749) -- (0.3354, -0.1024, 2.2811) -- (0.3372, -0.1046, 2.3311) -- (0.2906, -0.1046, 2.3249) -- cycle;
\fill[blue!15.1, opacity=0.5] (0.2906, -0.1046, 2.3249) -- (0.3372, -0.1046, 2.3311) -- (0.3389, -0.1066, 2.3811) -- (0.2924, -0.1066, 2.3749) -- cycle;
\fill[blue!15.2, opacity=0.5] (0.2924, -0.1066, 2.3749) -- (0.3389, -0.1066, 2.3811) -- (0.3405, -0.1085, 2.4311) -- (0.2941, -0.1085, 2.4249) -- cycle;
\fill[blue!15.2, opacity=0.5] (0.2941, -0.1085, 2.4249) -- (0.3405, -0.1085, 2.4311) -- (0.3419, -0.1103, 2.4811) -- (0.2956, -0.1103, 2.4749) -- cycle;
\fill[blue!15.3, opacity=0.5] (0.2956, -0.1103, 2.4749) -- (0.3419, -0.1103, 2.4811) -- (0.3433, -0.1120, 2.5311) -- (0.2970, -0.1120, 2.5249) -- cycle;
\fill[blue!15.4, opacity=0.5] (0.2970, -0.1120, 2.5249) -- (0.3433, -0.1120, 2.5311) -- (0.3446, -0.1135, 2.5811) -- (0.2983, -0.1135, 2.5749) -- cycle;
\fill[blue!15.5, opacity=0.5] (0.2983, -0.1135, 2.5749) -- (0.3446, -0.1135, 2.5811) -- (0.3457, -0.1148, 2.6311) -- (0.2995, -0.1148, 2.6249) -- cycle;
\fill[blue!15.6, opacity=0.5] (0.2995, -0.1148, 2.6249) -- (0.3457, -0.1148, 2.6311) -- (0.3467, -0.1160, 2.6811) -- (0.3005, -0.1160, 2.6749) -- cycle;
\fill[blue!15.8, opacity=0.5] (0.3005, -0.1160, 2.6749) -- (0.3467, -0.1160, 2.6811) -- (0.3476, -0.1171, 2.7311) -- (0.3015, -0.1171, 2.7249) -- cycle;
\fill[blue!15.9, opacity=0.5] (0.3015, -0.1171, 2.7249) -- (0.3476, -0.1171, 2.7311) -- (0.3483, -0.1180, 2.7811) -- (0.3022, -0.1180, 2.7749) -- cycle;
\fill[blue!16.1, opacity=0.5] (0.3022, -0.1180, 2.7749) -- (0.3483, -0.1180, 2.7811) -- (0.3489, -0.1187, 2.8311) -- (0.3029, -0.1187, 2.8249) -- cycle;
\fill[blue!16.4, opacity=0.5] (0.3029, -0.1187, 2.8249) -- (0.3489, -0.1187, 2.8311) -- (0.3494, -0.1193, 2.8811) -- (0.3034, -0.1193, 2.8749) -- cycle;
\fill[blue!16.6, opacity=0.5] (0.3034, -0.1193, 2.8749) -- (0.3494, -0.1193, 2.8811) -- (0.3497, -0.1197, 2.9311) -- (0.3037, -0.1197, 2.9249) -- cycle;
\fill[blue!16.9, opacity=0.5] (0.3037, -0.1197, 2.9249) -- (0.3497, -0.1197, 2.9311) -- (0.3499, -0.1199, 2.9811) -- (0.3039, -0.1199, 2.9749) -- cycle;
\fill[blue!17.2, opacity=0.5] (0.3039, -0.1199, 2.9749) -- (0.3499, -0.1199, 2.9811) -- (0.3500, -0.1200, 3.0311) -- (0.3040, -0.1200, 3.0249) -- cycle;
\fill[blue!15.0, opacity=0.5] (0.2500, -0.0000, 0.0311) -- (0.3000, -0.0000, 0.0371) -- (0.3001, -0.0001, 0.0871) -- (0.2501, -0.0001, 0.0811) -- cycle;
\fill[blue!15.0, opacity=0.5] (0.2501, -0.0001, 0.0811) -- (0.3001, -0.0001, 0.0871) -- (0.3003, -0.0003, 0.1371) -- (0.2503, -0.0003, 0.1311) -- cycle;
\fill[blue!15.0, opacity=0.5] (0.2503, -0.0003, 0.1311) -- (0.3003, -0.0003, 0.1371) -- (0.3006, -0.0007, 0.1871) -- (0.2506, -0.0007, 0.1811) -- cycle;
\fill[blue!15.0, opacity=0.5] (0.2506, -0.0007, 0.1811) -- (0.3006, -0.0007, 0.1871) -- (0.3010, -0.0013, 0.2371) -- (0.2511, -0.0013, 0.2311) -- cycle;
\fill[blue!15.0, opacity=0.5] (0.2511, -0.0013, 0.2311) -- (0.3010, -0.0013, 0.2371) -- (0.3016, -0.0020, 0.2871) -- (0.2517, -0.0020, 0.2811) -- cycle;
\fill[blue!15.0, opacity=0.5] (0.2517, -0.0020, 0.2811) -- (0.3016, -0.0020, 0.2871) -- (0.3023, -0.0029, 0.3371) -- (0.2524, -0.0029, 0.3311) -- cycle;
\fill[blue!15.0, opacity=0.5] (0.2524, -0.0029, 0.3311) -- (0.3023, -0.0029, 0.3371) -- (0.3032, -0.0040, 0.3871) -- (0.2533, -0.0040, 0.3811) -- cycle;
\fill[blue!15.0, opacity=0.5] (0.2533, -0.0040, 0.3811) -- (0.3032, -0.0040, 0.3871) -- (0.3041, -0.0052, 0.4371) -- (0.2543, -0.0052, 0.4311) -- cycle;
\fill[blue!15.0, opacity=0.5] (0.2543, -0.0052, 0.4311) -- (0.3041, -0.0052, 0.4371) -- (0.3052, -0.0065, 0.4871) -- (0.2554, -0.0065, 0.4811) -- cycle;
\fill[blue!15.0, opacity=0.5] (0.2554, -0.0065, 0.4811) -- (0.3052, -0.0065, 0.4871) -- (0.3064, -0.0080, 0.5371) -- (0.2567, -0.0080, 0.5311) -- cycle;
\fill[blue!15.0, opacity=0.5] (0.2567, -0.0080, 0.5311) -- (0.3064, -0.0080, 0.5371) -- (0.3077, -0.0097, 0.5871) -- (0.2581, -0.0097, 0.5811) -- cycle;
\fill[blue!15.0, opacity=0.5] (0.2581, -0.0097, 0.5811) -- (0.3077, -0.0097, 0.5871) -- (0.3092, -0.0115, 0.6371) -- (0.2595, -0.0115, 0.6311) -- cycle;
\fill[blue!15.0, opacity=0.5] (0.2595, -0.0115, 0.6311) -- (0.3092, -0.0115, 0.6371) -- (0.3107, -0.0134, 0.6871) -- (0.2611, -0.0134, 0.6811) -- cycle;
\fill[blue!15.0, opacity=0.5] (0.2611, -0.0134, 0.6811) -- (0.3107, -0.0134, 0.6871) -- (0.3123, -0.0154, 0.7371) -- (0.2628, -0.0154, 0.7311) -- cycle;
\fill[blue!15.0, opacity=0.5] (0.2628, -0.0154, 0.7311) -- (0.3123, -0.0154, 0.7371) -- (0.3141, -0.0176, 0.7871) -- (0.2646, -0.0176, 0.7811) -- cycle;
\fill[blue!15.0, opacity=0.5] (0.2646, -0.0176, 0.7811) -- (0.3141, -0.0176, 0.7871) -- (0.3159, -0.0199, 0.8371) -- (0.2665, -0.0199, 0.8311) -- cycle;
\fill[blue!15.0, opacity=0.5] (0.2665, -0.0199, 0.8311) -- (0.3159, -0.0199, 0.8371) -- (0.3178, -0.0222, 0.8871) -- (0.2685, -0.0222, 0.8811) -- cycle;
\fill[blue!15.0, opacity=0.5] (0.2685, -0.0222, 0.8811) -- (0.3178, -0.0222, 0.8871) -- (0.3198, -0.0247, 0.9371) -- (0.2706, -0.0247, 0.9311) -- cycle;
\fill[blue!15.0, opacity=0.5] (0.2706, -0.0247, 0.9311) -- (0.3198, -0.0247, 0.9371) -- (0.3219, -0.0273, 0.9871) -- (0.2728, -0.0273, 0.9811) -- cycle;
\fill[blue!15.0, opacity=0.5] (0.2728, -0.0273, 0.9811) -- (0.3219, -0.0273, 0.9871) -- (0.3240, -0.0300, 1.0371) -- (0.2750, -0.0300, 1.0311) -- cycle;
\fill[blue!15.0, opacity=0.5] (0.2750, -0.0300, 1.0311) -- (0.3240, -0.0300, 1.0371) -- (0.3262, -0.0328, 1.0871) -- (0.2773, -0.0328, 1.0811) -- cycle;
\fill[blue!15.0, opacity=0.5] (0.2773, -0.0328, 1.0811) -- (0.3262, -0.0328, 1.0871) -- (0.3285, -0.0356, 1.1371) -- (0.2797, -0.0356, 1.1311) -- cycle;
\fill[blue!15.0, opacity=0.5] (0.2797, -0.0356, 1.1311) -- (0.3285, -0.0356, 1.1371) -- (0.3308, -0.0385, 1.1871) -- (0.2821, -0.0385, 1.1811) -- cycle;
\fill[blue!15.0, opacity=0.5] (0.2821, -0.0385, 1.1811) -- (0.3308, -0.0385, 1.1871) -- (0.3332, -0.0415, 1.2371) -- (0.2845, -0.0415, 1.2311) -- cycle;
\fill[blue!15.0, opacity=0.5] (0.2845, -0.0415, 1.2311) -- (0.3332, -0.0415, 1.2371) -- (0.3356, -0.0445, 1.2871) -- (0.2871, -0.0445, 1.2811) -- cycle;
\fill[blue!15.0, opacity=0.5] (0.2871, -0.0445, 1.2811) -- (0.3356, -0.0445, 1.2871) -- (0.3380, -0.0475, 1.3371) -- (0.2896, -0.0475, 1.3311) -- cycle;
\fill[blue!15.0, opacity=0.5] (0.2896, -0.0475, 1.3311) -- (0.3380, -0.0475, 1.3371) -- (0.3405, -0.0506, 1.3871) -- (0.2922, -0.0506, 1.3811) -- cycle;
\fill[blue!15.0, opacity=0.5] (0.2922, -0.0506, 1.3811) -- (0.3405, -0.0506, 1.3871) -- (0.3430, -0.0537, 1.4371) -- (0.2948, -0.0537, 1.4311) -- cycle;
\fill[blue!15.0, opacity=0.5] (0.2948, -0.0537, 1.4311) -- (0.3430, -0.0537, 1.4371) -- (0.3455, -0.0569, 1.4871) -- (0.2974, -0.0569, 1.4811) -- cycle;
\fill[blue!15.0, opacity=0.5] (0.2974, -0.0569, 1.4811) -- (0.3455, -0.0569, 1.4871) -- (0.3480, -0.0600, 1.5371) -- (0.3000, -0.0600, 1.5311) -- cycle;
\fill[blue!15.0, opacity=0.5] (0.3000, -0.0600, 1.5311) -- (0.3480, -0.0600, 1.5371) -- (0.3505, -0.0631, 1.5871) -- (0.3026, -0.0631, 1.5811) -- cycle;
\fill[blue!15.0, opacity=0.5] (0.3026, -0.0631, 1.5811) -- (0.3505, -0.0631, 1.5871) -- (0.3530, -0.0663, 1.6371) -- (0.3052, -0.0663, 1.6311) -- cycle;
\fill[blue!15.0, opacity=0.5] (0.3052, -0.0663, 1.6311) -- (0.3530, -0.0663, 1.6371) -- (0.3555, -0.0694, 1.6871) -- (0.3078, -0.0694, 1.6811) -- cycle;
\fill[blue!15.0, opacity=0.5] (0.3078, -0.0694, 1.6811) -- (0.3555, -0.0694, 1.6871) -- (0.3580, -0.0725, 1.7371) -- (0.3104, -0.0725, 1.7311) -- cycle;
\fill[blue!15.0, opacity=0.5] (0.3104, -0.0725, 1.7311) -- (0.3580, -0.0725, 1.7371) -- (0.3604, -0.0755, 1.7871) -- (0.3129, -0.0755, 1.7811) -- cycle;
\fill[blue!15.0, opacity=0.5] (0.3129, -0.0755, 1.7811) -- (0.3604, -0.0755, 1.7871) -- (0.3628, -0.0785, 1.8371) -- (0.3155, -0.0785, 1.8311) -- cycle;
\fill[blue!15.0, opacity=0.5] (0.3155, -0.0785, 1.8311) -- (0.3628, -0.0785, 1.8371) -- (0.3652, -0.0815, 1.8871) -- (0.3179, -0.0815, 1.8811) -- cycle;
\fill[blue!15.0, opacity=0.5] (0.3179, -0.0815, 1.8811) -- (0.3652, -0.0815, 1.8871) -- (0.3675, -0.0844, 1.9371) -- (0.3203, -0.0844, 1.9311) -- cycle;
\fill[blue!15.0, opacity=0.5] (0.3203, -0.0844, 1.9311) -- (0.3675, -0.0844, 1.9371) -- (0.3698, -0.0872, 1.9871) -- (0.3227, -0.0872, 1.9811) -- cycle;
\fill[blue!15.0, opacity=0.5] (0.3227, -0.0872, 1.9811) -- (0.3698, -0.0872, 1.9871) -- (0.3720, -0.0900, 2.0371) -- (0.3250, -0.0900, 2.0311) -- cycle;
\fill[blue!15.0, opacity=0.5] (0.3250, -0.0900, 2.0311) -- (0.3720, -0.0900, 2.0371) -- (0.3741, -0.0927, 2.0871) -- (0.3272, -0.0927, 2.0811) -- cycle;
\fill[blue!15.0, opacity=0.5] (0.3272, -0.0927, 2.0811) -- (0.3741, -0.0927, 2.0871) -- (0.3762, -0.0953, 2.1371) -- (0.3294, -0.0953, 2.1311) -- cycle;
\fill[blue!15.0, opacity=0.5] (0.3294, -0.0953, 2.1311) -- (0.3762, -0.0953, 2.1371) -- (0.3782, -0.0978, 2.1871) -- (0.3315, -0.0978, 2.1811) -- cycle;
\fill[blue!15.0, opacity=0.5] (0.3315, -0.0978, 2.1811) -- (0.3782, -0.0978, 2.1871) -- (0.3801, -0.1001, 2.2371) -- (0.3335, -0.1001, 2.2311) -- cycle;
\fill[blue!15.0, opacity=0.5] (0.3335, -0.1001, 2.2311) -- (0.3801, -0.1001, 2.2371) -- (0.3819, -0.1024, 2.2871) -- (0.3354, -0.1024, 2.2811) -- cycle;
\fill[blue!15.0, opacity=0.5] (0.3354, -0.1024, 2.2811) -- (0.3819, -0.1024, 2.2871) -- (0.3837, -0.1046, 2.3371) -- (0.3372, -0.1046, 2.3311) -- cycle;
\fill[blue!15.0, opacity=0.5] (0.3372, -0.1046, 2.3311) -- (0.3837, -0.1046, 2.3371) -- (0.3853, -0.1066, 2.3871) -- (0.3389, -0.1066, 2.3811) -- cycle;
\fill[blue!15.0, opacity=0.5] (0.3389, -0.1066, 2.3811) -- (0.3853, -0.1066, 2.3871) -- (0.3868, -0.1085, 2.4371) -- (0.3405, -0.1085, 2.4311) -- cycle;
\fill[blue!15.0, opacity=0.5] (0.3405, -0.1085, 2.4311) -- (0.3868, -0.1085, 2.4371) -- (0.3883, -0.1103, 2.4871) -- (0.3419, -0.1103, 2.4811) -- cycle;
\fill[blue!15.1, opacity=0.5] (0.3419, -0.1103, 2.4811) -- (0.3883, -0.1103, 2.4871) -- (0.3896, -0.1120, 2.5371) -- (0.3433, -0.1120, 2.5311) -- cycle;
\fill[blue!15.1, opacity=0.5] (0.3433, -0.1120, 2.5311) -- (0.3896, -0.1120, 2.5371) -- (0.3908, -0.1135, 2.5871) -- (0.3446, -0.1135, 2.5811) -- cycle;
\fill[blue!15.1, opacity=0.5] (0.3446, -0.1135, 2.5811) -- (0.3908, -0.1135, 2.5871) -- (0.3919, -0.1148, 2.6371) -- (0.3457, -0.1148, 2.6311) -- cycle;
\fill[blue!15.2, opacity=0.5] (0.3457, -0.1148, 2.6311) -- (0.3919, -0.1148, 2.6371) -- (0.3928, -0.1160, 2.6871) -- (0.3467, -0.1160, 2.6811) -- cycle;
\fill[blue!15.2, opacity=0.5] (0.3467, -0.1160, 2.6811) -- (0.3928, -0.1160, 2.6871) -- (0.3937, -0.1171, 2.7371) -- (0.3476, -0.1171, 2.7311) -- cycle;
\fill[blue!15.3, opacity=0.5] (0.3476, -0.1171, 2.7311) -- (0.3937, -0.1171, 2.7371) -- (0.3944, -0.1180, 2.7871) -- (0.3483, -0.1180, 2.7811) -- cycle;
\fill[blue!15.3, opacity=0.5] (0.3483, -0.1180, 2.7811) -- (0.3944, -0.1180, 2.7871) -- (0.3950, -0.1187, 2.8371) -- (0.3489, -0.1187, 2.8311) -- cycle;
\fill[blue!15.4, opacity=0.5] (0.3489, -0.1187, 2.8311) -- (0.3950, -0.1187, 2.8371) -- (0.3954, -0.1193, 2.8871) -- (0.3494, -0.1193, 2.8811) -- cycle;
\fill[blue!15.5, opacity=0.5] (0.3494, -0.1193, 2.8811) -- (0.3954, -0.1193, 2.8871) -- (0.3957, -0.1197, 2.9371) -- (0.3497, -0.1197, 2.9311) -- cycle;
\fill[blue!15.6, opacity=0.5] (0.3497, -0.1197, 2.9311) -- (0.3957, -0.1197, 2.9371) -- (0.3959, -0.1199, 2.9871) -- (0.3499, -0.1199, 2.9811) -- cycle;
\fill[blue!15.7, opacity=0.5] (0.3499, -0.1199, 2.9811) -- (0.3959, -0.1199, 2.9871) -- (0.3960, -0.1200, 3.0371) -- (0.3500, -0.1200, 3.0311) -- cycle;
\fill[blue!15.0, opacity=0.5] (0.3000, -0.0000, 0.0371) -- (0.3500, -0.0000, 0.0430) -- (0.3501, -0.0001, 0.0930) -- (0.3001, -0.0001, 0.0871) -- cycle;
\fill[blue!15.0, opacity=0.5] (0.3001, -0.0001, 0.0871) -- (0.3501, -0.0001, 0.0930) -- (0.3503, -0.0003, 0.1430) -- (0.3003, -0.0003, 0.1371) -- cycle;
\fill[blue!15.0, opacity=0.5] (0.3003, -0.0003, 0.1371) -- (0.3503, -0.0003, 0.1430) -- (0.3506, -0.0007, 0.1930) -- (0.3006, -0.0007, 0.1871) -- cycle;
\fill[blue!15.0, opacity=0.5] (0.3006, -0.0007, 0.1871) -- (0.3506, -0.0007, 0.1930) -- (0.3510, -0.0013, 0.2430) -- (0.3010, -0.0013, 0.2371) -- cycle;
\fill[blue!15.0, opacity=0.5] (0.3010, -0.0013, 0.2371) -- (0.3510, -0.0013, 0.2430) -- (0.3516, -0.0020, 0.2930) -- (0.3016, -0.0020, 0.2871) -- cycle;
\fill[blue!15.0, opacity=0.5] (0.3016, -0.0020, 0.2871) -- (0.3516, -0.0020, 0.2930) -- (0.3523, -0.0029, 0.3430) -- (0.3023, -0.0029, 0.3371) -- cycle;
\fill[blue!15.0, opacity=0.5] (0.3023, -0.0029, 0.3371) -- (0.3523, -0.0029, 0.3430) -- (0.3531, -0.0040, 0.3930) -- (0.3032, -0.0040, 0.3871) -- cycle;
\fill[blue!15.0, opacity=0.5] (0.3032, -0.0040, 0.3871) -- (0.3531, -0.0040, 0.3930) -- (0.3540, -0.0052, 0.4430) -- (0.3041, -0.0052, 0.4371) -- cycle;
\fill[blue!15.0, opacity=0.5] (0.3041, -0.0052, 0.4371) -- (0.3540, -0.0052, 0.4430) -- (0.3550, -0.0065, 0.4930) -- (0.3052, -0.0065, 0.4871) -- cycle;
\fill[blue!15.0, opacity=0.5] (0.3052, -0.0065, 0.4871) -- (0.3550, -0.0065, 0.4930) -- (0.3562, -0.0080, 0.5430) -- (0.3064, -0.0080, 0.5371) -- cycle;
\fill[blue!15.0, opacity=0.5] (0.3064, -0.0080, 0.5371) -- (0.3562, -0.0080, 0.5430) -- (0.3574, -0.0097, 0.5930) -- (0.3077, -0.0097, 0.5871) -- cycle;
\fill[blue!15.0, opacity=0.5] (0.3077, -0.0097, 0.5871) -- (0.3574, -0.0097, 0.5930) -- (0.3588, -0.0115, 0.6430) -- (0.3092, -0.0115, 0.6371) -- cycle;
\fill[blue!15.0, opacity=0.5] (0.3092, -0.0115, 0.6371) -- (0.3588, -0.0115, 0.6430) -- (0.3603, -0.0134, 0.6930) -- (0.3107, -0.0134, 0.6871) -- cycle;
\fill[blue!15.0, opacity=0.5] (0.3107, -0.0134, 0.6871) -- (0.3603, -0.0134, 0.6930) -- (0.3618, -0.0154, 0.7430) -- (0.3123, -0.0154, 0.7371) -- cycle;
\fill[blue!15.0, opacity=0.5] (0.3123, -0.0154, 0.7371) -- (0.3618, -0.0154, 0.7430) -- (0.3635, -0.0176, 0.7930) -- (0.3141, -0.0176, 0.7871) -- cycle;
\fill[blue!15.0, opacity=0.5] (0.3141, -0.0176, 0.7871) -- (0.3635, -0.0176, 0.7930) -- (0.3652, -0.0199, 0.8430) -- (0.3159, -0.0199, 0.8371) -- cycle;
\fill[blue!15.0, opacity=0.5] (0.3159, -0.0199, 0.8371) -- (0.3652, -0.0199, 0.8430) -- (0.3671, -0.0222, 0.8930) -- (0.3178, -0.0222, 0.8871) -- cycle;
\fill[blue!15.0, opacity=0.5] (0.3178, -0.0222, 0.8871) -- (0.3671, -0.0222, 0.8930) -- (0.3690, -0.0247, 0.9430) -- (0.3198, -0.0247, 0.9371) -- cycle;
\fill[blue!15.0, opacity=0.5] (0.3198, -0.0247, 0.9371) -- (0.3690, -0.0247, 0.9430) -- (0.3709, -0.0273, 0.9930) -- (0.3219, -0.0273, 0.9871) -- cycle;
\fill[blue!15.0, opacity=0.5] (0.3219, -0.0273, 0.9871) -- (0.3709, -0.0273, 0.9930) -- (0.3730, -0.0300, 1.0430) -- (0.3240, -0.0300, 1.0371) -- cycle;
\fill[blue!15.0, opacity=0.5] (0.3240, -0.0300, 1.0371) -- (0.3730, -0.0300, 1.0430) -- (0.3751, -0.0328, 1.0930) -- (0.3262, -0.0328, 1.0871) -- cycle;
\fill[blue!15.0, opacity=0.5] (0.3262, -0.0328, 1.0871) -- (0.3751, -0.0328, 1.0930) -- (0.3773, -0.0356, 1.1430) -- (0.3285, -0.0356, 1.1371) -- cycle;
\fill[blue!15.0, opacity=0.5] (0.3285, -0.0356, 1.1371) -- (0.3773, -0.0356, 1.1430) -- (0.3795, -0.0385, 1.1930) -- (0.3308, -0.0385, 1.1871) -- cycle;
\fill[blue!15.0, opacity=0.5] (0.3308, -0.0385, 1.1871) -- (0.3795, -0.0385, 1.1930) -- (0.3818, -0.0415, 1.2430) -- (0.3332, -0.0415, 1.2371) -- cycle;
\fill[blue!15.0, opacity=0.5] (0.3332, -0.0415, 1.2371) -- (0.3818, -0.0415, 1.2430) -- (0.3841, -0.0445, 1.2930) -- (0.3356, -0.0445, 1.2871) -- cycle;
\fill[blue!15.0, opacity=0.5] (0.3356, -0.0445, 1.2871) -- (0.3841, -0.0445, 1.2930) -- (0.3864, -0.0475, 1.3430) -- (0.3380, -0.0475, 1.3371) -- cycle;
\fill[blue!15.0, opacity=0.5] (0.3380, -0.0475, 1.3371) -- (0.3864, -0.0475, 1.3430) -- (0.3888, -0.0506, 1.3930) -- (0.3405, -0.0506, 1.3871) -- cycle;
\fill[blue!15.0, opacity=0.5] (0.3405, -0.0506, 1.3871) -- (0.3888, -0.0506, 1.3930) -- (0.3912, -0.0537, 1.4430) -- (0.3430, -0.0537, 1.4371) -- cycle;
\fill[blue!15.0, opacity=0.5] (0.3430, -0.0537, 1.4371) -- (0.3912, -0.0537, 1.4430) -- (0.3936, -0.0569, 1.4930) -- (0.3455, -0.0569, 1.4871) -- cycle;
\fill[blue!15.0, opacity=0.5] (0.3455, -0.0569, 1.4871) -- (0.3936, -0.0569, 1.4930) -- (0.3960, -0.0600, 1.5430) -- (0.3480, -0.0600, 1.5371) -- cycle;
\fill[blue!15.0, opacity=0.5] (0.3480, -0.0600, 1.5371) -- (0.3960, -0.0600, 1.5430) -- (0.3984, -0.0631, 1.5930) -- (0.3505, -0.0631, 1.5871) -- cycle;
\fill[blue!15.0, opacity=0.5] (0.3505, -0.0631, 1.5871) -- (0.3984, -0.0631, 1.5930) -- (0.4008, -0.0663, 1.6430) -- (0.3530, -0.0663, 1.6371) -- cycle;
\fill[blue!15.0, opacity=0.5] (0.3530, -0.0663, 1.6371) -- (0.4008, -0.0663, 1.6430) -- (0.4032, -0.0694, 1.6930) -- (0.3555, -0.0694, 1.6871) -- cycle;
\fill[blue!15.0, opacity=0.5] (0.3555, -0.0694, 1.6871) -- (0.4032, -0.0694, 1.6930) -- (0.4056, -0.0725, 1.7430) -- (0.3580, -0.0725, 1.7371) -- cycle;
\fill[blue!15.0, opacity=0.5] (0.3580, -0.0725, 1.7371) -- (0.4056, -0.0725, 1.7430) -- (0.4079, -0.0755, 1.7930) -- (0.3604, -0.0755, 1.7871) -- cycle;
\fill[blue!15.0, opacity=0.5] (0.3604, -0.0755, 1.7871) -- (0.4079, -0.0755, 1.7930) -- (0.4102, -0.0785, 1.8430) -- (0.3628, -0.0785, 1.8371) -- cycle;
\fill[blue!15.0, opacity=0.5] (0.3628, -0.0785, 1.8371) -- (0.4102, -0.0785, 1.8430) -- (0.4125, -0.0815, 1.8930) -- (0.3652, -0.0815, 1.8871) -- cycle;
\fill[blue!15.0, opacity=0.5] (0.3652, -0.0815, 1.8871) -- (0.4125, -0.0815, 1.8930) -- (0.4147, -0.0844, 1.9430) -- (0.3675, -0.0844, 1.9371) -- cycle;
\fill[blue!15.0, opacity=0.5] (0.3675, -0.0844, 1.9371) -- (0.4147, -0.0844, 1.9430) -- (0.4169, -0.0872, 1.9930) -- (0.3698, -0.0872, 1.9871) -- cycle;
\fill[blue!15.0, opacity=0.5] (0.3698, -0.0872, 1.9871) -- (0.4169, -0.0872, 1.9930) -- (0.4190, -0.0900, 2.0430) -- (0.3720, -0.0900, 2.0371) -- cycle;
\fill[blue!15.0, opacity=0.5] (0.3720, -0.0900, 2.0371) -- (0.4190, -0.0900, 2.0430) -- (0.4211, -0.0927, 2.0930) -- (0.3741, -0.0927, 2.0871) -- cycle;
\fill[blue!15.0, opacity=0.5] (0.3741, -0.0927, 2.0871) -- (0.4211, -0.0927, 2.0930) -- (0.4230, -0.0953, 2.1430) -- (0.3762, -0.0953, 2.1371) -- cycle;
\fill[blue!15.0, opacity=0.5] (0.3762, -0.0953, 2.1371) -- (0.4230, -0.0953, 2.1430) -- (0.4249, -0.0978, 2.1930) -- (0.3782, -0.0978, 2.1871) -- cycle;
\fill[blue!15.0, opacity=0.5] (0.3782, -0.0978, 2.1871) -- (0.4249, -0.0978, 2.1930) -- (0.4268, -0.1001, 2.2430) -- (0.3801, -0.1001, 2.2371) -- cycle;
\fill[blue!15.0, opacity=0.5] (0.3801, -0.1001, 2.2371) -- (0.4268, -0.1001, 2.2430) -- (0.4285, -0.1024, 2.2930) -- (0.3819, -0.1024, 2.2871) -- cycle;
\fill[blue!15.0, opacity=0.5] (0.3819, -0.1024, 2.2871) -- (0.4285, -0.1024, 2.2930) -- (0.4302, -0.1046, 2.3430) -- (0.3837, -0.1046, 2.3371) -- cycle;
\fill[blue!15.0, opacity=0.5] (0.3837, -0.1046, 2.3371) -- (0.4302, -0.1046, 2.3430) -- (0.4317, -0.1066, 2.3930) -- (0.3853, -0.1066, 2.3871) -- cycle;
\fill[blue!15.0, opacity=0.5] (0.3853, -0.1066, 2.3871) -- (0.4317, -0.1066, 2.3930) -- (0.4332, -0.1085, 2.4430) -- (0.3868, -0.1085, 2.4371) -- cycle;
\fill[blue!15.0, opacity=0.5] (0.3868, -0.1085, 2.4371) -- (0.4332, -0.1085, 2.4430) -- (0.4346, -0.1103, 2.4930) -- (0.3883, -0.1103, 2.4871) -- cycle;
\fill[blue!15.0, opacity=0.5] (0.3883, -0.1103, 2.4871) -- (0.4346, -0.1103, 2.4930) -- (0.4358, -0.1120, 2.5430) -- (0.3896, -0.1120, 2.5371) -- cycle;
\fill[blue!15.0, opacity=0.5] (0.3896, -0.1120, 2.5371) -- (0.4358, -0.1120, 2.5430) -- (0.4370, -0.1135, 2.5930) -- (0.3908, -0.1135, 2.5871) -- cycle;
\fill[blue!15.0, opacity=0.5] (0.3908, -0.1135, 2.5871) -- (0.4370, -0.1135, 2.5930) -- (0.4380, -0.1148, 2.6430) -- (0.3919, -0.1148, 2.6371) -- cycle;
\fill[blue!15.1, opacity=0.5] (0.3919, -0.1148, 2.6371) -- (0.4380, -0.1148, 2.6430) -- (0.4389, -0.1160, 2.6930) -- (0.3928, -0.1160, 2.6871) -- cycle;
\fill[blue!15.1, opacity=0.5] (0.3928, -0.1160, 2.6871) -- (0.4389, -0.1160, 2.6930) -- (0.4397, -0.1171, 2.7430) -- (0.3937, -0.1171, 2.7371) -- cycle;
\fill[blue!15.1, opacity=0.5] (0.3937, -0.1171, 2.7371) -- (0.4397, -0.1171, 2.7430) -- (0.4404, -0.1180, 2.7930) -- (0.3944, -0.1180, 2.7871) -- cycle;
\fill[blue!15.1, opacity=0.5] (0.3944, -0.1180, 2.7871) -- (0.4404, -0.1180, 2.7930) -- (0.4410, -0.1187, 2.8430) -- (0.3950, -0.1187, 2.8371) -- cycle;
\fill[blue!15.2, opacity=0.5] (0.3950, -0.1187, 2.8371) -- (0.4410, -0.1187, 2.8430) -- (0.4414, -0.1193, 2.8930) -- (0.3954, -0.1193, 2.8871) -- cycle;
\fill[blue!15.2, opacity=0.5] (0.3954, -0.1193, 2.8871) -- (0.4414, -0.1193, 2.8930) -- (0.4417, -0.1197, 2.9430) -- (0.3957, -0.1197, 2.9371) -- cycle;
\fill[blue!15.3, opacity=0.5] (0.3957, -0.1197, 2.9371) -- (0.4417, -0.1197, 2.9430) -- (0.4419, -0.1199, 2.9930) -- (0.3959, -0.1199, 2.9871) -- cycle;
\fill[blue!15.3, opacity=0.5] (0.3959, -0.1199, 2.9871) -- (0.4419, -0.1199, 2.9930) -- (0.4420, -0.1200, 3.0430) -- (0.3960, -0.1200, 3.0371) -- cycle;
\fill[blue!15.0, opacity=0.5] (0.3500, -0.0000, 0.0430) -- (0.4000, -0.0000, 0.0488) -- (0.4001, -0.0001, 0.0988) -- (0.3501, -0.0001, 0.0930) -- cycle;
\fill[blue!15.0, opacity=0.5] (0.3501, -0.0001, 0.0930) -- (0.4001, -0.0001, 0.0988) -- (0.4002, -0.0003, 0.1488) -- (0.3503, -0.0003, 0.1430) -- cycle;
\fill[blue!15.0, opacity=0.5] (0.3503, -0.0003, 0.1430) -- (0.4002, -0.0003, 0.1488) -- (0.4005, -0.0007, 0.1988) -- (0.3506, -0.0007, 0.1930) -- cycle;
\fill[blue!15.0, opacity=0.5] (0.3506, -0.0007, 0.1930) -- (0.4005, -0.0007, 0.1988) -- (0.4010, -0.0013, 0.2488) -- (0.3510, -0.0013, 0.2430) -- cycle;
\fill[blue!15.0, opacity=0.5] (0.3510, -0.0013, 0.2430) -- (0.4010, -0.0013, 0.2488) -- (0.4015, -0.0020, 0.2988) -- (0.3516, -0.0020, 0.2930) -- cycle;
\fill[blue!15.0, opacity=0.5] (0.3516, -0.0020, 0.2930) -- (0.4015, -0.0020, 0.2988) -- (0.4022, -0.0029, 0.3488) -- (0.3523, -0.0029, 0.3430) -- cycle;
\fill[blue!15.0, opacity=0.5] (0.3523, -0.0029, 0.3430) -- (0.4022, -0.0029, 0.3488) -- (0.4029, -0.0040, 0.3988) -- (0.3531, -0.0040, 0.3930) -- cycle;
\fill[blue!15.0, opacity=0.5] (0.3531, -0.0040, 0.3930) -- (0.4029, -0.0040, 0.3988) -- (0.4038, -0.0052, 0.4488) -- (0.3540, -0.0052, 0.4430) -- cycle;
\fill[blue!15.0, opacity=0.5] (0.3540, -0.0052, 0.4430) -- (0.4038, -0.0052, 0.4488) -- (0.4048, -0.0065, 0.4988) -- (0.3550, -0.0065, 0.4930) -- cycle;
\fill[blue!15.0, opacity=0.5] (0.3550, -0.0065, 0.4930) -- (0.4048, -0.0065, 0.4988) -- (0.4059, -0.0080, 0.5488) -- (0.3562, -0.0080, 0.5430) -- cycle;
\fill[blue!15.0, opacity=0.5] (0.3562, -0.0080, 0.5430) -- (0.4059, -0.0080, 0.5488) -- (0.4071, -0.0097, 0.5988) -- (0.3574, -0.0097, 0.5930) -- cycle;
\fill[blue!15.0, opacity=0.5] (0.3574, -0.0097, 0.5930) -- (0.4071, -0.0097, 0.5988) -- (0.4084, -0.0115, 0.6488) -- (0.3588, -0.0115, 0.6430) -- cycle;
\fill[blue!15.0, opacity=0.5] (0.3588, -0.0115, 0.6430) -- (0.4084, -0.0115, 0.6488) -- (0.4098, -0.0134, 0.6988) -- (0.3603, -0.0134, 0.6930) -- cycle;
\fill[blue!15.0, opacity=0.5] (0.3603, -0.0134, 0.6930) -- (0.4098, -0.0134, 0.6988) -- (0.4113, -0.0154, 0.7488) -- (0.3618, -0.0154, 0.7430) -- cycle;
\fill[blue!15.0, opacity=0.5] (0.3618, -0.0154, 0.7430) -- (0.4113, -0.0154, 0.7488) -- (0.4129, -0.0176, 0.7988) -- (0.3635, -0.0176, 0.7930) -- cycle;
\fill[blue!15.0, opacity=0.5] (0.3635, -0.0176, 0.7930) -- (0.4129, -0.0176, 0.7988) -- (0.4146, -0.0199, 0.8488) -- (0.3652, -0.0199, 0.8430) -- cycle;
\fill[blue!15.0, opacity=0.5] (0.3652, -0.0199, 0.8430) -- (0.4146, -0.0199, 0.8488) -- (0.4163, -0.0222, 0.8988) -- (0.3671, -0.0222, 0.8930) -- cycle;
\fill[blue!15.0, opacity=0.5] (0.3671, -0.0222, 0.8930) -- (0.4163, -0.0222, 0.8988) -- (0.4181, -0.0247, 0.9488) -- (0.3690, -0.0247, 0.9430) -- cycle;
\fill[blue!15.0, opacity=0.5] (0.3690, -0.0247, 0.9430) -- (0.4181, -0.0247, 0.9488) -- (0.4200, -0.0273, 0.9988) -- (0.3709, -0.0273, 0.9930) -- cycle;
\fill[blue!15.0, opacity=0.5] (0.3709, -0.0273, 0.9930) -- (0.4200, -0.0273, 0.9988) -- (0.4220, -0.0300, 1.0488) -- (0.3730, -0.0300, 1.0430) -- cycle;
\fill[blue!15.0, opacity=0.5] (0.3730, -0.0300, 1.0430) -- (0.4220, -0.0300, 1.0488) -- (0.4240, -0.0328, 1.0988) -- (0.3751, -0.0328, 1.0930) -- cycle;
\fill[blue!15.0, opacity=0.5] (0.3751, -0.0328, 1.0930) -- (0.4240, -0.0328, 1.0988) -- (0.4261, -0.0356, 1.1488) -- (0.3773, -0.0356, 1.1430) -- cycle;
\fill[blue!15.0, opacity=0.5] (0.3773, -0.0356, 1.1430) -- (0.4261, -0.0356, 1.1488) -- (0.4282, -0.0385, 1.1988) -- (0.3795, -0.0385, 1.1930) -- cycle;
\fill[blue!15.0, opacity=0.5] (0.3795, -0.0385, 1.1930) -- (0.4282, -0.0385, 1.1988) -- (0.4304, -0.0415, 1.2488) -- (0.3818, -0.0415, 1.2430) -- cycle;
\fill[blue!15.0, opacity=0.5] (0.3818, -0.0415, 1.2430) -- (0.4304, -0.0415, 1.2488) -- (0.4326, -0.0445, 1.2988) -- (0.3841, -0.0445, 1.2930) -- cycle;
\fill[blue!15.0, opacity=0.5] (0.3841, -0.0445, 1.2930) -- (0.4326, -0.0445, 1.2988) -- (0.4349, -0.0475, 1.3488) -- (0.3864, -0.0475, 1.3430) -- cycle;
\fill[blue!15.0, opacity=0.5] (0.3864, -0.0475, 1.3430) -- (0.4349, -0.0475, 1.3488) -- (0.4371, -0.0506, 1.3988) -- (0.3888, -0.0506, 1.3930) -- cycle;
\fill[blue!15.0, opacity=0.5] (0.3888, -0.0506, 1.3930) -- (0.4371, -0.0506, 1.3988) -- (0.4394, -0.0537, 1.4488) -- (0.3912, -0.0537, 1.4430) -- cycle;
\fill[blue!15.0, opacity=0.5] (0.3912, -0.0537, 1.4430) -- (0.4394, -0.0537, 1.4488) -- (0.4417, -0.0569, 1.4988) -- (0.3936, -0.0569, 1.4930) -- cycle;
\fill[blue!15.0, opacity=0.5] (0.3936, -0.0569, 1.4930) -- (0.4417, -0.0569, 1.4988) -- (0.4440, -0.0600, 1.5488) -- (0.3960, -0.0600, 1.5430) -- cycle;
\fill[blue!15.0, opacity=0.5] (0.3960, -0.0600, 1.5430) -- (0.4440, -0.0600, 1.5488) -- (0.4463, -0.0631, 1.5988) -- (0.3984, -0.0631, 1.5930) -- cycle;
\fill[blue!15.0, opacity=0.5] (0.3984, -0.0631, 1.5930) -- (0.4463, -0.0631, 1.5988) -- (0.4486, -0.0663, 1.6488) -- (0.4008, -0.0663, 1.6430) -- cycle;
\fill[blue!15.0, opacity=0.5] (0.4008, -0.0663, 1.6430) -- (0.4486, -0.0663, 1.6488) -- (0.4509, -0.0694, 1.6988) -- (0.4032, -0.0694, 1.6930) -- cycle;
\fill[blue!15.0, opacity=0.5] (0.4032, -0.0694, 1.6930) -- (0.4509, -0.0694, 1.6988) -- (0.4531, -0.0725, 1.7488) -- (0.4056, -0.0725, 1.7430) -- cycle;
\fill[blue!15.0, opacity=0.5] (0.4056, -0.0725, 1.7430) -- (0.4531, -0.0725, 1.7488) -- (0.4554, -0.0755, 1.7988) -- (0.4079, -0.0755, 1.7930) -- cycle;
\fill[blue!15.0, opacity=0.5] (0.4079, -0.0755, 1.7930) -- (0.4554, -0.0755, 1.7988) -- (0.4576, -0.0785, 1.8488) -- (0.4102, -0.0785, 1.8430) -- cycle;
\fill[blue!15.0, opacity=0.5] (0.4102, -0.0785, 1.8430) -- (0.4576, -0.0785, 1.8488) -- (0.4598, -0.0815, 1.8988) -- (0.4125, -0.0815, 1.8930) -- cycle;
\fill[blue!15.0, opacity=0.5] (0.4125, -0.0815, 1.8930) -- (0.4598, -0.0815, 1.8988) -- (0.4619, -0.0844, 1.9488) -- (0.4147, -0.0844, 1.9430) -- cycle;
\fill[blue!15.0, opacity=0.5] (0.4147, -0.0844, 1.9430) -- (0.4619, -0.0844, 1.9488) -- (0.4640, -0.0872, 1.9988) -- (0.4169, -0.0872, 1.9930) -- cycle;
\fill[blue!15.0, opacity=0.5] (0.4169, -0.0872, 1.9930) -- (0.4640, -0.0872, 1.9988) -- (0.4660, -0.0900, 2.0488) -- (0.4190, -0.0900, 2.0430) -- cycle;
\fill[blue!15.0, opacity=0.5] (0.4190, -0.0900, 2.0430) -- (0.4660, -0.0900, 2.0488) -- (0.4680, -0.0927, 2.0988) -- (0.4211, -0.0927, 2.0930) -- cycle;
\fill[blue!15.0, opacity=0.5] (0.4211, -0.0927, 2.0930) -- (0.4680, -0.0927, 2.0988) -- (0.4699, -0.0953, 2.1488) -- (0.4230, -0.0953, 2.1430) -- cycle;
\fill[blue!15.0, opacity=0.5] (0.4230, -0.0953, 2.1430) -- (0.4699, -0.0953, 2.1488) -- (0.4717, -0.0978, 2.1988) -- (0.4249, -0.0978, 2.1930) -- cycle;
\fill[blue!15.0, opacity=0.5] (0.4249, -0.0978, 2.1930) -- (0.4717, -0.0978, 2.1988) -- (0.4734, -0.1001, 2.2488) -- (0.4268, -0.1001, 2.2430) -- cycle;
\fill[blue!15.0, opacity=0.5] (0.4268, -0.1001, 2.2430) -- (0.4734, -0.1001, 2.2488) -- (0.4751, -0.1024, 2.2988) -- (0.4285, -0.1024, 2.2930) -- cycle;
\fill[blue!15.0, opacity=0.5] (0.4285, -0.1024, 2.2930) -- (0.4751, -0.1024, 2.2988) -- (0.4767, -0.1046, 2.3488) -- (0.4302, -0.1046, 2.3430) -- cycle;
\fill[blue!15.0, opacity=0.5] (0.4302, -0.1046, 2.3430) -- (0.4767, -0.1046, 2.3488) -- (0.4782, -0.1066, 2.3988) -- (0.4317, -0.1066, 2.3930) -- cycle;
\fill[blue!15.0, opacity=0.5] (0.4317, -0.1066, 2.3930) -- (0.4782, -0.1066, 2.3988) -- (0.4796, -0.1085, 2.4488) -- (0.4332, -0.1085, 2.4430) -- cycle;
\fill[blue!15.0, opacity=0.5] (0.4332, -0.1085, 2.4430) -- (0.4796, -0.1085, 2.4488) -- (0.4809, -0.1103, 2.4988) -- (0.4346, -0.1103, 2.4930) -- cycle;
\fill[blue!15.0, opacity=0.5] (0.4346, -0.1103, 2.4930) -- (0.4809, -0.1103, 2.4988) -- (0.4821, -0.1120, 2.5488) -- (0.4358, -0.1120, 2.5430) -- cycle;
\fill[blue!15.0, opacity=0.5] (0.4358, -0.1120, 2.5430) -- (0.4821, -0.1120, 2.5488) -- (0.4832, -0.1135, 2.5988) -- (0.4370, -0.1135, 2.5930) -- cycle;
\fill[blue!15.0, opacity=0.5] (0.4370, -0.1135, 2.5930) -- (0.4832, -0.1135, 2.5988) -- (0.4842, -0.1148, 2.6488) -- (0.4380, -0.1148, 2.6430) -- cycle;
\fill[blue!15.0, opacity=0.5] (0.4380, -0.1148, 2.6430) -- (0.4842, -0.1148, 2.6488) -- (0.4851, -0.1160, 2.6988) -- (0.4389, -0.1160, 2.6930) -- cycle;
\fill[blue!15.0, opacity=0.5] (0.4389, -0.1160, 2.6930) -- (0.4851, -0.1160, 2.6988) -- (0.4858, -0.1171, 2.7488) -- (0.4397, -0.1171, 2.7430) -- cycle;
\fill[blue!15.1, opacity=0.5] (0.4397, -0.1171, 2.7430) -- (0.4858, -0.1171, 2.7488) -- (0.4865, -0.1180, 2.7988) -- (0.4404, -0.1180, 2.7930) -- cycle;
\fill[blue!15.1, opacity=0.5] (0.4404, -0.1180, 2.7930) -- (0.4865, -0.1180, 2.7988) -- (0.4870, -0.1187, 2.8488) -- (0.4410, -0.1187, 2.8430) -- cycle;
\fill[blue!15.1, opacity=0.5] (0.4410, -0.1187, 2.8430) -- (0.4870, -0.1187, 2.8488) -- (0.4875, -0.1193, 2.8988) -- (0.4414, -0.1193, 2.8930) -- cycle;
\fill[blue!15.1, opacity=0.5] (0.4414, -0.1193, 2.8930) -- (0.4875, -0.1193, 2.8988) -- (0.4878, -0.1197, 2.9488) -- (0.4417, -0.1197, 2.9430) -- cycle;
\fill[blue!15.2, opacity=0.5] (0.4417, -0.1197, 2.9430) -- (0.4878, -0.1197, 2.9488) -- (0.4879, -0.1199, 2.9988) -- (0.4419, -0.1199, 2.9930) -- cycle;
\fill[blue!15.2, opacity=0.5] (0.4419, -0.1199, 2.9930) -- (0.4879, -0.1199, 2.9988) -- (0.4880, -0.1200, 3.0488) -- (0.4420, -0.1200, 3.0430) -- cycle;
\fill[blue!15.0, opacity=0.5] (0.4000, -0.0000, 0.0488) -- (0.4500, -0.0000, 0.0545) -- (0.4501, -0.0001, 0.1045) -- (0.4001, -0.0001, 0.0988) -- cycle;
\fill[blue!15.0, opacity=0.5] (0.4001, -0.0001, 0.0988) -- (0.4501, -0.0001, 0.1045) -- (0.4502, -0.0003, 0.1545) -- (0.4002, -0.0003, 0.1488) -- cycle;
\fill[blue!15.0, opacity=0.5] (0.4002, -0.0003, 0.1488) -- (0.4502, -0.0003, 0.1545) -- (0.4505, -0.0007, 0.2045) -- (0.4005, -0.0007, 0.1988) -- cycle;
\fill[blue!15.0, opacity=0.5] (0.4005, -0.0007, 0.1988) -- (0.4505, -0.0007, 0.2045) -- (0.4509, -0.0013, 0.2545) -- (0.4010, -0.0013, 0.2488) -- cycle;
\fill[blue!15.0, opacity=0.5] (0.4010, -0.0013, 0.2488) -- (0.4509, -0.0013, 0.2545) -- (0.4514, -0.0020, 0.3045) -- (0.4015, -0.0020, 0.2988) -- cycle;
\fill[blue!15.0, opacity=0.5] (0.4015, -0.0020, 0.2988) -- (0.4514, -0.0020, 0.3045) -- (0.4521, -0.0029, 0.3545) -- (0.4022, -0.0029, 0.3488) -- cycle;
\fill[blue!15.0, opacity=0.5] (0.4022, -0.0029, 0.3488) -- (0.4521, -0.0029, 0.3545) -- (0.4528, -0.0040, 0.4045) -- (0.4029, -0.0040, 0.3988) -- cycle;
\fill[blue!15.0, opacity=0.5] (0.4029, -0.0040, 0.3988) -- (0.4528, -0.0040, 0.4045) -- (0.4536, -0.0052, 0.4545) -- (0.4038, -0.0052, 0.4488) -- cycle;
\fill[blue!15.0, opacity=0.5] (0.4038, -0.0052, 0.4488) -- (0.4536, -0.0052, 0.4545) -- (0.4546, -0.0065, 0.5045) -- (0.4048, -0.0065, 0.4988) -- cycle;
\fill[blue!15.0, opacity=0.5] (0.4048, -0.0065, 0.4988) -- (0.4546, -0.0065, 0.5045) -- (0.4556, -0.0080, 0.5545) -- (0.4059, -0.0080, 0.5488) -- cycle;
\fill[blue!15.0, opacity=0.5] (0.4059, -0.0080, 0.5488) -- (0.4556, -0.0080, 0.5545) -- (0.4568, -0.0097, 0.6045) -- (0.4071, -0.0097, 0.5988) -- cycle;
\fill[blue!15.0, opacity=0.5] (0.4071, -0.0097, 0.5988) -- (0.4568, -0.0097, 0.6045) -- (0.4580, -0.0115, 0.6545) -- (0.4084, -0.0115, 0.6488) -- cycle;
\fill[blue!15.0, opacity=0.5] (0.4084, -0.0115, 0.6488) -- (0.4580, -0.0115, 0.6545) -- (0.4594, -0.0134, 0.7045) -- (0.4098, -0.0134, 0.6988) -- cycle;
\fill[blue!15.0, opacity=0.5] (0.4098, -0.0134, 0.6988) -- (0.4594, -0.0134, 0.7045) -- (0.4608, -0.0154, 0.7545) -- (0.4113, -0.0154, 0.7488) -- cycle;
\fill[blue!15.0, opacity=0.5] (0.4113, -0.0154, 0.7488) -- (0.4608, -0.0154, 0.7545) -- (0.4623, -0.0176, 0.8045) -- (0.4129, -0.0176, 0.7988) -- cycle;
\fill[blue!15.0, opacity=0.5] (0.4129, -0.0176, 0.7988) -- (0.4623, -0.0176, 0.8045) -- (0.4639, -0.0199, 0.8545) -- (0.4146, -0.0199, 0.8488) -- cycle;
\fill[blue!15.0, opacity=0.5] (0.4146, -0.0199, 0.8488) -- (0.4639, -0.0199, 0.8545) -- (0.4656, -0.0222, 0.9045) -- (0.4163, -0.0222, 0.8988) -- cycle;
\fill[blue!15.0, opacity=0.5] (0.4163, -0.0222, 0.8988) -- (0.4656, -0.0222, 0.9045) -- (0.4673, -0.0247, 0.9545) -- (0.4181, -0.0247, 0.9488) -- cycle;
\fill[blue!15.0, opacity=0.5] (0.4181, -0.0247, 0.9488) -- (0.4673, -0.0247, 0.9545) -- (0.4691, -0.0273, 1.0045) -- (0.4200, -0.0273, 0.9988) -- cycle;
\fill[blue!15.0, opacity=0.5] (0.4200, -0.0273, 0.9988) -- (0.4691, -0.0273, 1.0045) -- (0.4710, -0.0300, 1.0545) -- (0.4220, -0.0300, 1.0488) -- cycle;
\fill[blue!15.0, opacity=0.5] (0.4220, -0.0300, 1.0488) -- (0.4710, -0.0300, 1.0545) -- (0.4729, -0.0328, 1.1045) -- (0.4240, -0.0328, 1.0988) -- cycle;
\fill[blue!15.0, opacity=0.5] (0.4240, -0.0328, 1.0988) -- (0.4729, -0.0328, 1.1045) -- (0.4749, -0.0356, 1.1545) -- (0.4261, -0.0356, 1.1488) -- cycle;
\fill[blue!15.0, opacity=0.5] (0.4261, -0.0356, 1.1488) -- (0.4749, -0.0356, 1.1545) -- (0.4769, -0.0385, 1.2045) -- (0.4282, -0.0385, 1.1988) -- cycle;
\fill[blue!15.0, opacity=0.5] (0.4282, -0.0385, 1.1988) -- (0.4769, -0.0385, 1.2045) -- (0.4790, -0.0415, 1.2545) -- (0.4304, -0.0415, 1.2488) -- cycle;
\fill[blue!15.0, opacity=0.5] (0.4304, -0.0415, 1.2488) -- (0.4790, -0.0415, 1.2545) -- (0.4811, -0.0445, 1.3045) -- (0.4326, -0.0445, 1.2988) -- cycle;
\fill[blue!15.0, opacity=0.5] (0.4326, -0.0445, 1.2988) -- (0.4811, -0.0445, 1.3045) -- (0.4833, -0.0475, 1.3545) -- (0.4349, -0.0475, 1.3488) -- cycle;
\fill[blue!15.0, opacity=0.5] (0.4349, -0.0475, 1.3488) -- (0.4833, -0.0475, 1.3545) -- (0.4854, -0.0506, 1.4045) -- (0.4371, -0.0506, 1.3988) -- cycle;
\fill[blue!15.0, opacity=0.5] (0.4371, -0.0506, 1.3988) -- (0.4854, -0.0506, 1.4045) -- (0.4876, -0.0537, 1.4545) -- (0.4394, -0.0537, 1.4488) -- cycle;
\fill[blue!15.0, opacity=0.5] (0.4394, -0.0537, 1.4488) -- (0.4876, -0.0537, 1.4545) -- (0.4898, -0.0569, 1.5045) -- (0.4417, -0.0569, 1.4988) -- cycle;
\fill[blue!15.0, opacity=0.5] (0.4417, -0.0569, 1.4988) -- (0.4898, -0.0569, 1.5045) -- (0.4920, -0.0600, 1.5545) -- (0.4440, -0.0600, 1.5488) -- cycle;
\fill[blue!15.0, opacity=0.5] (0.4440, -0.0600, 1.5488) -- (0.4920, -0.0600, 1.5545) -- (0.4942, -0.0631, 1.6045) -- (0.4463, -0.0631, 1.5988) -- cycle;
\fill[blue!15.0, opacity=0.5] (0.4463, -0.0631, 1.5988) -- (0.4942, -0.0631, 1.6045) -- (0.4964, -0.0663, 1.6545) -- (0.4486, -0.0663, 1.6488) -- cycle;
\fill[blue!15.0, opacity=0.5] (0.4486, -0.0663, 1.6488) -- (0.4964, -0.0663, 1.6545) -- (0.4986, -0.0694, 1.7045) -- (0.4509, -0.0694, 1.6988) -- cycle;
\fill[blue!15.0, opacity=0.5] (0.4509, -0.0694, 1.6988) -- (0.4986, -0.0694, 1.7045) -- (0.5007, -0.0725, 1.7545) -- (0.4531, -0.0725, 1.7488) -- cycle;
\fill[blue!15.0, opacity=0.5] (0.4531, -0.0725, 1.7488) -- (0.5007, -0.0725, 1.7545) -- (0.5029, -0.0755, 1.8045) -- (0.4554, -0.0755, 1.7988) -- cycle;
\fill[blue!15.0, opacity=0.5] (0.4554, -0.0755, 1.7988) -- (0.5029, -0.0755, 1.8045) -- (0.5050, -0.0785, 1.8545) -- (0.4576, -0.0785, 1.8488) -- cycle;
\fill[blue!15.0, opacity=0.5] (0.4576, -0.0785, 1.8488) -- (0.5050, -0.0785, 1.8545) -- (0.5071, -0.0815, 1.9045) -- (0.4598, -0.0815, 1.8988) -- cycle;
\fill[blue!15.0, opacity=0.5] (0.4598, -0.0815, 1.8988) -- (0.5071, -0.0815, 1.9045) -- (0.5091, -0.0844, 1.9545) -- (0.4619, -0.0844, 1.9488) -- cycle;
\fill[blue!15.0, opacity=0.5] (0.4619, -0.0844, 1.9488) -- (0.5091, -0.0844, 1.9545) -- (0.5111, -0.0872, 2.0045) -- (0.4640, -0.0872, 1.9988) -- cycle;
\fill[blue!15.0, opacity=0.5] (0.4640, -0.0872, 1.9988) -- (0.5111, -0.0872, 2.0045) -- (0.5130, -0.0900, 2.0545) -- (0.4660, -0.0900, 2.0488) -- cycle;
\fill[blue!15.0, opacity=0.5] (0.4660, -0.0900, 2.0488) -- (0.5130, -0.0900, 2.0545) -- (0.5149, -0.0927, 2.1045) -- (0.4680, -0.0927, 2.0988) -- cycle;
\fill[blue!15.0, opacity=0.5] (0.4680, -0.0927, 2.0988) -- (0.5149, -0.0927, 2.1045) -- (0.5167, -0.0953, 2.1545) -- (0.4699, -0.0953, 2.1488) -- cycle;
\fill[blue!15.0, opacity=0.5] (0.4699, -0.0953, 2.1488) -- (0.5167, -0.0953, 2.1545) -- (0.5184, -0.0978, 2.2045) -- (0.4717, -0.0978, 2.1988) -- cycle;
\fill[blue!15.0, opacity=0.5] (0.4717, -0.0978, 2.1988) -- (0.5184, -0.0978, 2.2045) -- (0.5201, -0.1001, 2.2545) -- (0.4734, -0.1001, 2.2488) -- cycle;
\fill[blue!15.0, opacity=0.5] (0.4734, -0.1001, 2.2488) -- (0.5201, -0.1001, 2.2545) -- (0.5217, -0.1024, 2.3045) -- (0.4751, -0.1024, 2.2988) -- cycle;
\fill[blue!15.0, opacity=0.5] (0.4751, -0.1024, 2.2988) -- (0.5217, -0.1024, 2.3045) -- (0.5232, -0.1046, 2.3545) -- (0.4767, -0.1046, 2.3488) -- cycle;
\fill[blue!15.0, opacity=0.5] (0.4767, -0.1046, 2.3488) -- (0.5232, -0.1046, 2.3545) -- (0.5246, -0.1066, 2.4045) -- (0.4782, -0.1066, 2.3988) -- cycle;
\fill[blue!15.0, opacity=0.5] (0.4782, -0.1066, 2.3988) -- (0.5246, -0.1066, 2.4045) -- (0.5260, -0.1085, 2.4545) -- (0.4796, -0.1085, 2.4488) -- cycle;
\fill[blue!15.0, opacity=0.5] (0.4796, -0.1085, 2.4488) -- (0.5260, -0.1085, 2.4545) -- (0.5272, -0.1103, 2.5045) -- (0.4809, -0.1103, 2.4988) -- cycle;
\fill[blue!15.0, opacity=0.5] (0.4809, -0.1103, 2.4988) -- (0.5272, -0.1103, 2.5045) -- (0.5284, -0.1120, 2.5545) -- (0.4821, -0.1120, 2.5488) -- cycle;
\fill[blue!15.0, opacity=0.5] (0.4821, -0.1120, 2.5488) -- (0.5284, -0.1120, 2.5545) -- (0.5294, -0.1135, 2.6045) -- (0.4832, -0.1135, 2.5988) -- cycle;
\fill[blue!15.0, opacity=0.5] (0.4832, -0.1135, 2.5988) -- (0.5294, -0.1135, 2.6045) -- (0.5304, -0.1148, 2.6545) -- (0.4842, -0.1148, 2.6488) -- cycle;
\fill[blue!15.1, opacity=0.5] (0.4842, -0.1148, 2.6488) -- (0.5304, -0.1148, 2.6545) -- (0.5312, -0.1160, 2.7045) -- (0.4851, -0.1160, 2.6988) -- cycle;
\fill[blue!15.1, opacity=0.5] (0.4851, -0.1160, 2.6988) -- (0.5312, -0.1160, 2.7045) -- (0.5319, -0.1171, 2.7545) -- (0.4858, -0.1171, 2.7488) -- cycle;
\fill[blue!15.1, opacity=0.5] (0.4858, -0.1171, 2.7488) -- (0.5319, -0.1171, 2.7545) -- (0.5326, -0.1180, 2.8045) -- (0.4865, -0.1180, 2.7988) -- cycle;
\fill[blue!15.1, opacity=0.5] (0.4865, -0.1180, 2.7988) -- (0.5326, -0.1180, 2.8045) -- (0.5331, -0.1187, 2.8545) -- (0.4870, -0.1187, 2.8488) -- cycle;
\fill[blue!15.2, opacity=0.5] (0.4870, -0.1187, 2.8488) -- (0.5331, -0.1187, 2.8545) -- (0.5335, -0.1193, 2.9045) -- (0.4875, -0.1193, 2.8988) -- cycle;
\fill[blue!15.2, opacity=0.5] (0.4875, -0.1193, 2.8988) -- (0.5335, -0.1193, 2.9045) -- (0.5338, -0.1197, 2.9545) -- (0.4878, -0.1197, 2.9488) -- cycle;
\fill[blue!15.3, opacity=0.5] (0.4878, -0.1197, 2.9488) -- (0.5338, -0.1197, 2.9545) -- (0.5339, -0.1199, 3.0045) -- (0.4879, -0.1199, 2.9988) -- cycle;
\fill[blue!15.3, opacity=0.5] (0.4879, -0.1199, 2.9988) -- (0.5339, -0.1199, 3.0045) -- (0.5340, -0.1200, 3.0545) -- (0.4880, -0.1200, 3.0488) -- cycle;
\fill[blue!15.0, opacity=0.5] (0.4500, -0.0000, 0.0545) -- (0.5000, -0.0000, 0.0600) -- (0.5001, -0.0001, 0.1100) -- (0.4501, -0.0001, 0.1045) -- cycle;
\fill[blue!15.0, opacity=0.5] (0.4501, -0.0001, 0.1045) -- (0.5001, -0.0001, 0.1100) -- (0.5002, -0.0003, 0.1600) -- (0.4502, -0.0003, 0.1545) -- cycle;
\fill[blue!15.0, opacity=0.5] (0.4502, -0.0003, 0.1545) -- (0.5002, -0.0003, 0.1600) -- (0.5005, -0.0007, 0.2100) -- (0.4505, -0.0007, 0.2045) -- cycle;
\fill[blue!15.0, opacity=0.5] (0.4505, -0.0007, 0.2045) -- (0.5005, -0.0007, 0.2100) -- (0.5009, -0.0013, 0.2600) -- (0.4509, -0.0013, 0.2545) -- cycle;
\fill[blue!15.0, opacity=0.5] (0.4509, -0.0013, 0.2545) -- (0.5009, -0.0013, 0.2600) -- (0.5014, -0.0020, 0.3100) -- (0.4514, -0.0020, 0.3045) -- cycle;
\fill[blue!15.0, opacity=0.5] (0.4514, -0.0020, 0.3045) -- (0.5014, -0.0020, 0.3100) -- (0.5020, -0.0029, 0.3600) -- (0.4521, -0.0029, 0.3545) -- cycle;
\fill[blue!15.0, opacity=0.5] (0.4521, -0.0029, 0.3545) -- (0.5020, -0.0029, 0.3600) -- (0.5027, -0.0040, 0.4100) -- (0.4528, -0.0040, 0.4045) -- cycle;
\fill[blue!15.0, opacity=0.5] (0.4528, -0.0040, 0.4045) -- (0.5027, -0.0040, 0.4100) -- (0.5035, -0.0052, 0.4600) -- (0.4536, -0.0052, 0.4545) -- cycle;
\fill[blue!15.0, opacity=0.5] (0.4536, -0.0052, 0.4545) -- (0.5035, -0.0052, 0.4600) -- (0.5044, -0.0065, 0.5100) -- (0.4546, -0.0065, 0.5045) -- cycle;
\fill[blue!15.0, opacity=0.5] (0.4546, -0.0065, 0.5045) -- (0.5044, -0.0065, 0.5100) -- (0.5054, -0.0080, 0.5600) -- (0.4556, -0.0080, 0.5545) -- cycle;
\fill[blue!15.0, opacity=0.5] (0.4556, -0.0080, 0.5545) -- (0.5054, -0.0080, 0.5600) -- (0.5065, -0.0097, 0.6100) -- (0.4568, -0.0097, 0.6045) -- cycle;
\fill[blue!15.0, opacity=0.5] (0.4568, -0.0097, 0.6045) -- (0.5065, -0.0097, 0.6100) -- (0.5076, -0.0115, 0.6600) -- (0.4580, -0.0115, 0.6545) -- cycle;
\fill[blue!15.0, opacity=0.5] (0.4580, -0.0115, 0.6545) -- (0.5076, -0.0115, 0.6600) -- (0.5089, -0.0134, 0.7100) -- (0.4594, -0.0134, 0.7045) -- cycle;
\fill[blue!15.0, opacity=0.5] (0.4594, -0.0134, 0.7045) -- (0.5089, -0.0134, 0.7100) -- (0.5103, -0.0154, 0.7600) -- (0.4608, -0.0154, 0.7545) -- cycle;
\fill[blue!15.0, opacity=0.5] (0.4608, -0.0154, 0.7545) -- (0.5103, -0.0154, 0.7600) -- (0.5117, -0.0176, 0.8100) -- (0.4623, -0.0176, 0.8045) -- cycle;
\fill[blue!15.0, opacity=0.5] (0.4623, -0.0176, 0.8045) -- (0.5117, -0.0176, 0.8100) -- (0.5132, -0.0199, 0.8600) -- (0.4639, -0.0199, 0.8545) -- cycle;
\fill[blue!15.0, opacity=0.5] (0.4639, -0.0199, 0.8545) -- (0.5132, -0.0199, 0.8600) -- (0.5148, -0.0222, 0.9100) -- (0.4656, -0.0222, 0.9045) -- cycle;
\fill[blue!15.0, opacity=0.5] (0.4656, -0.0222, 0.9045) -- (0.5148, -0.0222, 0.9100) -- (0.5165, -0.0247, 0.9600) -- (0.4673, -0.0247, 0.9545) -- cycle;
\fill[blue!15.0, opacity=0.5] (0.4673, -0.0247, 0.9545) -- (0.5165, -0.0247, 0.9600) -- (0.5182, -0.0273, 1.0100) -- (0.4691, -0.0273, 1.0045) -- cycle;
\fill[blue!15.0, opacity=0.5] (0.4691, -0.0273, 1.0045) -- (0.5182, -0.0273, 1.0100) -- (0.5200, -0.0300, 1.0600) -- (0.4710, -0.0300, 1.0545) -- cycle;
\fill[blue!15.0, opacity=0.5] (0.4710, -0.0300, 1.0545) -- (0.5200, -0.0300, 1.0600) -- (0.5218, -0.0328, 1.1100) -- (0.4729, -0.0328, 1.1045) -- cycle;
\fill[blue!15.0, opacity=0.5] (0.4729, -0.0328, 1.1045) -- (0.5218, -0.0328, 1.1100) -- (0.5237, -0.0356, 1.1600) -- (0.4749, -0.0356, 1.1545) -- cycle;
\fill[blue!15.0, opacity=0.5] (0.4749, -0.0356, 1.1545) -- (0.5237, -0.0356, 1.1600) -- (0.5257, -0.0385, 1.2100) -- (0.4769, -0.0385, 1.2045) -- cycle;
\fill[blue!15.0, opacity=0.5] (0.4769, -0.0385, 1.2045) -- (0.5257, -0.0385, 1.2100) -- (0.5276, -0.0415, 1.2600) -- (0.4790, -0.0415, 1.2545) -- cycle;
\fill[blue!15.0, opacity=0.5] (0.4790, -0.0415, 1.2545) -- (0.5276, -0.0415, 1.2600) -- (0.5296, -0.0445, 1.3100) -- (0.4811, -0.0445, 1.3045) -- cycle;
\fill[blue!15.0, opacity=0.5] (0.4811, -0.0445, 1.3045) -- (0.5296, -0.0445, 1.3100) -- (0.5317, -0.0475, 1.3600) -- (0.4833, -0.0475, 1.3545) -- cycle;
\fill[blue!15.0, opacity=0.5] (0.4833, -0.0475, 1.3545) -- (0.5317, -0.0475, 1.3600) -- (0.5337, -0.0506, 1.4100) -- (0.4854, -0.0506, 1.4045) -- cycle;
\fill[blue!15.0, opacity=0.5] (0.4854, -0.0506, 1.4045) -- (0.5337, -0.0506, 1.4100) -- (0.5358, -0.0537, 1.4600) -- (0.4876, -0.0537, 1.4545) -- cycle;
\fill[blue!15.0, opacity=0.5] (0.4876, -0.0537, 1.4545) -- (0.5358, -0.0537, 1.4600) -- (0.5379, -0.0569, 1.5100) -- (0.4898, -0.0569, 1.5045) -- cycle;
\fill[blue!15.0, opacity=0.5] (0.4898, -0.0569, 1.5045) -- (0.5379, -0.0569, 1.5100) -- (0.5400, -0.0600, 1.5600) -- (0.4920, -0.0600, 1.5545) -- cycle;
\fill[blue!15.0, opacity=0.5] (0.4920, -0.0600, 1.5545) -- (0.5400, -0.0600, 1.5600) -- (0.5421, -0.0631, 1.6100) -- (0.4942, -0.0631, 1.6045) -- cycle;
\fill[blue!15.0, opacity=0.5] (0.4942, -0.0631, 1.6045) -- (0.5421, -0.0631, 1.6100) -- (0.5442, -0.0663, 1.6600) -- (0.4964, -0.0663, 1.6545) -- cycle;
\fill[blue!15.0, opacity=0.5] (0.4964, -0.0663, 1.6545) -- (0.5442, -0.0663, 1.6600) -- (0.5463, -0.0694, 1.7100) -- (0.4986, -0.0694, 1.7045) -- cycle;
\fill[blue!15.0, opacity=0.5] (0.4986, -0.0694, 1.7045) -- (0.5463, -0.0694, 1.7100) -- (0.5483, -0.0725, 1.7600) -- (0.5007, -0.0725, 1.7545) -- cycle;
\fill[blue!15.0, opacity=0.5] (0.5007, -0.0725, 1.7545) -- (0.5483, -0.0725, 1.7600) -- (0.5504, -0.0755, 1.8100) -- (0.5029, -0.0755, 1.8045) -- cycle;
\fill[blue!15.0, opacity=0.5] (0.5029, -0.0755, 1.8045) -- (0.5504, -0.0755, 1.8100) -- (0.5524, -0.0785, 1.8600) -- (0.5050, -0.0785, 1.8545) -- cycle;
\fill[blue!15.0, opacity=0.5] (0.5050, -0.0785, 1.8545) -- (0.5524, -0.0785, 1.8600) -- (0.5543, -0.0815, 1.9100) -- (0.5071, -0.0815, 1.9045) -- cycle;
\fill[blue!15.0, opacity=0.5] (0.5071, -0.0815, 1.9045) -- (0.5543, -0.0815, 1.9100) -- (0.5563, -0.0844, 1.9600) -- (0.5091, -0.0844, 1.9545) -- cycle;
\fill[blue!15.0, opacity=0.5] (0.5091, -0.0844, 1.9545) -- (0.5563, -0.0844, 1.9600) -- (0.5582, -0.0872, 2.0100) -- (0.5111, -0.0872, 2.0045) -- cycle;
\fill[blue!15.0, opacity=0.5] (0.5111, -0.0872, 2.0045) -- (0.5582, -0.0872, 2.0100) -- (0.5600, -0.0900, 2.0600) -- (0.5130, -0.0900, 2.0545) -- cycle;
\fill[blue!15.0, opacity=0.5] (0.5130, -0.0900, 2.0545) -- (0.5600, -0.0900, 2.0600) -- (0.5618, -0.0927, 2.1100) -- (0.5149, -0.0927, 2.1045) -- cycle;
\fill[blue!15.0, opacity=0.5] (0.5149, -0.0927, 2.1045) -- (0.5618, -0.0927, 2.1100) -- (0.5635, -0.0953, 2.1600) -- (0.5167, -0.0953, 2.1545) -- cycle;
\fill[blue!15.0, opacity=0.5] (0.5167, -0.0953, 2.1545) -- (0.5635, -0.0953, 2.1600) -- (0.5652, -0.0978, 2.2100) -- (0.5184, -0.0978, 2.2045) -- cycle;
\fill[blue!15.0, opacity=0.5] (0.5184, -0.0978, 2.2045) -- (0.5652, -0.0978, 2.2100) -- (0.5668, -0.1001, 2.2600) -- (0.5201, -0.1001, 2.2545) -- cycle;
\fill[blue!15.0, opacity=0.5] (0.5201, -0.1001, 2.2545) -- (0.5668, -0.1001, 2.2600) -- (0.5683, -0.1024, 2.3100) -- (0.5217, -0.1024, 2.3045) -- cycle;
\fill[blue!15.0, opacity=0.5] (0.5217, -0.1024, 2.3045) -- (0.5683, -0.1024, 2.3100) -- (0.5697, -0.1046, 2.3600) -- (0.5232, -0.1046, 2.3545) -- cycle;
\fill[blue!15.0, opacity=0.5] (0.5232, -0.1046, 2.3545) -- (0.5697, -0.1046, 2.3600) -- (0.5711, -0.1066, 2.4100) -- (0.5246, -0.1066, 2.4045) -- cycle;
\fill[blue!15.0, opacity=0.5] (0.5246, -0.1066, 2.4045) -- (0.5711, -0.1066, 2.4100) -- (0.5724, -0.1085, 2.4600) -- (0.5260, -0.1085, 2.4545) -- cycle;
\fill[blue!15.0, opacity=0.5] (0.5260, -0.1085, 2.4545) -- (0.5724, -0.1085, 2.4600) -- (0.5735, -0.1103, 2.5100) -- (0.5272, -0.1103, 2.5045) -- cycle;
\fill[blue!15.1, opacity=0.5] (0.5272, -0.1103, 2.5045) -- (0.5735, -0.1103, 2.5100) -- (0.5746, -0.1120, 2.5600) -- (0.5284, -0.1120, 2.5545) -- cycle;
\fill[blue!15.1, opacity=0.5] (0.5284, -0.1120, 2.5545) -- (0.5746, -0.1120, 2.5600) -- (0.5756, -0.1135, 2.6100) -- (0.5294, -0.1135, 2.6045) -- cycle;
\fill[blue!15.1, opacity=0.5] (0.5294, -0.1135, 2.6045) -- (0.5756, -0.1135, 2.6100) -- (0.5765, -0.1148, 2.6600) -- (0.5304, -0.1148, 2.6545) -- cycle;
\fill[blue!15.2, opacity=0.5] (0.5304, -0.1148, 2.6545) -- (0.5765, -0.1148, 2.6600) -- (0.5773, -0.1160, 2.7100) -- (0.5312, -0.1160, 2.7045) -- cycle;
\fill[blue!15.2, opacity=0.5] (0.5312, -0.1160, 2.7045) -- (0.5773, -0.1160, 2.7100) -- (0.5780, -0.1171, 2.7600) -- (0.5319, -0.1171, 2.7545) -- cycle;
\fill[blue!15.3, opacity=0.5] (0.5319, -0.1171, 2.7545) -- (0.5780, -0.1171, 2.7600) -- (0.5786, -0.1180, 2.8100) -- (0.5326, -0.1180, 2.8045) -- cycle;
\fill[blue!15.3, opacity=0.5] (0.5326, -0.1180, 2.8045) -- (0.5786, -0.1180, 2.8100) -- (0.5791, -0.1187, 2.8600) -- (0.5331, -0.1187, 2.8545) -- cycle;
\fill[blue!15.4, opacity=0.5] (0.5331, -0.1187, 2.8545) -- (0.5791, -0.1187, 2.8600) -- (0.5795, -0.1193, 2.9100) -- (0.5335, -0.1193, 2.9045) -- cycle;
\fill[blue!15.5, opacity=0.5] (0.5335, -0.1193, 2.9045) -- (0.5795, -0.1193, 2.9100) -- (0.5798, -0.1197, 2.9600) -- (0.5338, -0.1197, 2.9545) -- cycle;
\fill[blue!15.6, opacity=0.5] (0.5338, -0.1197, 2.9545) -- (0.5798, -0.1197, 2.9600) -- (0.5799, -0.1199, 3.0100) -- (0.5339, -0.1199, 3.0045) -- cycle;
\fill[blue!15.7, opacity=0.5] (0.5339, -0.1199, 3.0045) -- (0.5799, -0.1199, 3.0100) -- (0.5800, -0.1200, 3.0600) -- (0.5340, -0.1200, 3.0545) -- cycle;
\fill[blue!15.0, opacity=0.5] (0.5000, -0.0000, 0.0600) -- (0.5500, -0.0000, 0.0654) -- (0.5501, -0.0001, 0.1154) -- (0.5001, -0.0001, 0.1100) -- cycle;
\fill[blue!15.0, opacity=0.5] (0.5001, -0.0001, 0.1100) -- (0.5501, -0.0001, 0.1154) -- (0.5502, -0.0003, 0.1654) -- (0.5002, -0.0003, 0.1600) -- cycle;
\fill[blue!15.0, opacity=0.5] (0.5002, -0.0003, 0.1600) -- (0.5502, -0.0003, 0.1654) -- (0.5505, -0.0007, 0.2154) -- (0.5005, -0.0007, 0.2100) -- cycle;
\fill[blue!15.0, opacity=0.5] (0.5005, -0.0007, 0.2100) -- (0.5505, -0.0007, 0.2154) -- (0.5508, -0.0013, 0.2654) -- (0.5009, -0.0013, 0.2600) -- cycle;
\fill[blue!15.0, opacity=0.5] (0.5009, -0.0013, 0.2600) -- (0.5508, -0.0013, 0.2654) -- (0.5513, -0.0020, 0.3154) -- (0.5014, -0.0020, 0.3100) -- cycle;
\fill[blue!15.0, opacity=0.5] (0.5014, -0.0020, 0.3100) -- (0.5513, -0.0020, 0.3154) -- (0.5519, -0.0029, 0.3654) -- (0.5020, -0.0029, 0.3600) -- cycle;
\fill[blue!15.0, opacity=0.5] (0.5020, -0.0029, 0.3600) -- (0.5519, -0.0029, 0.3654) -- (0.5525, -0.0040, 0.4154) -- (0.5027, -0.0040, 0.4100) -- cycle;
\fill[blue!15.0, opacity=0.5] (0.5027, -0.0040, 0.4100) -- (0.5525, -0.0040, 0.4154) -- (0.5533, -0.0052, 0.4654) -- (0.5035, -0.0052, 0.4600) -- cycle;
\fill[blue!15.0, opacity=0.5] (0.5035, -0.0052, 0.4600) -- (0.5533, -0.0052, 0.4654) -- (0.5541, -0.0065, 0.5154) -- (0.5044, -0.0065, 0.5100) -- cycle;
\fill[blue!15.0, opacity=0.5] (0.5044, -0.0065, 0.5100) -- (0.5541, -0.0065, 0.5154) -- (0.5551, -0.0080, 0.5654) -- (0.5054, -0.0080, 0.5600) -- cycle;
\fill[blue!15.0, opacity=0.5] (0.5054, -0.0080, 0.5600) -- (0.5551, -0.0080, 0.5654) -- (0.5561, -0.0097, 0.6154) -- (0.5065, -0.0097, 0.6100) -- cycle;
\fill[blue!15.0, opacity=0.5] (0.5065, -0.0097, 0.6100) -- (0.5561, -0.0097, 0.6154) -- (0.5573, -0.0115, 0.6654) -- (0.5076, -0.0115, 0.6600) -- cycle;
\fill[blue!15.0, opacity=0.5] (0.5076, -0.0115, 0.6600) -- (0.5573, -0.0115, 0.6654) -- (0.5585, -0.0134, 0.7154) -- (0.5089, -0.0134, 0.7100) -- cycle;
\fill[blue!15.0, opacity=0.5] (0.5089, -0.0134, 0.7100) -- (0.5585, -0.0134, 0.7154) -- (0.5598, -0.0154, 0.7654) -- (0.5103, -0.0154, 0.7600) -- cycle;
\fill[blue!15.0, opacity=0.5] (0.5103, -0.0154, 0.7600) -- (0.5598, -0.0154, 0.7654) -- (0.5611, -0.0176, 0.8154) -- (0.5117, -0.0176, 0.8100) -- cycle;
\fill[blue!15.0, opacity=0.5] (0.5117, -0.0176, 0.8100) -- (0.5611, -0.0176, 0.8154) -- (0.5626, -0.0199, 0.8654) -- (0.5132, -0.0199, 0.8600) -- cycle;
\fill[blue!15.0, opacity=0.5] (0.5132, -0.0199, 0.8600) -- (0.5626, -0.0199, 0.8654) -- (0.5641, -0.0222, 0.9154) -- (0.5148, -0.0222, 0.9100) -- cycle;
\fill[blue!15.0, opacity=0.5] (0.5148, -0.0222, 0.9100) -- (0.5641, -0.0222, 0.9154) -- (0.5657, -0.0247, 0.9654) -- (0.5165, -0.0247, 0.9600) -- cycle;
\fill[blue!15.0, opacity=0.5] (0.5165, -0.0247, 0.9600) -- (0.5657, -0.0247, 0.9654) -- (0.5673, -0.0273, 1.0154) -- (0.5182, -0.0273, 1.0100) -- cycle;
\fill[blue!15.0, opacity=0.5] (0.5182, -0.0273, 1.0100) -- (0.5673, -0.0273, 1.0154) -- (0.5690, -0.0300, 1.0654) -- (0.5200, -0.0300, 1.0600) -- cycle;
\fill[blue!15.0, opacity=0.5] (0.5200, -0.0300, 1.0600) -- (0.5690, -0.0300, 1.0654) -- (0.5707, -0.0328, 1.1154) -- (0.5218, -0.0328, 1.1100) -- cycle;
\fill[blue!15.0, opacity=0.5] (0.5218, -0.0328, 1.1100) -- (0.5707, -0.0328, 1.1154) -- (0.5725, -0.0356, 1.1654) -- (0.5237, -0.0356, 1.1600) -- cycle;
\fill[blue!15.0, opacity=0.5] (0.5237, -0.0356, 1.1600) -- (0.5725, -0.0356, 1.1654) -- (0.5744, -0.0385, 1.2154) -- (0.5257, -0.0385, 1.2100) -- cycle;
\fill[blue!15.0, opacity=0.5] (0.5257, -0.0385, 1.2100) -- (0.5744, -0.0385, 1.2154) -- (0.5763, -0.0415, 1.2654) -- (0.5276, -0.0415, 1.2600) -- cycle;
\fill[blue!15.0, opacity=0.5] (0.5276, -0.0415, 1.2600) -- (0.5763, -0.0415, 1.2654) -- (0.5782, -0.0445, 1.3154) -- (0.5296, -0.0445, 1.3100) -- cycle;
\fill[blue!15.0, opacity=0.5] (0.5296, -0.0445, 1.3100) -- (0.5782, -0.0445, 1.3154) -- (0.5801, -0.0475, 1.3654) -- (0.5317, -0.0475, 1.3600) -- cycle;
\fill[blue!15.0, opacity=0.5] (0.5317, -0.0475, 1.3600) -- (0.5801, -0.0475, 1.3654) -- (0.5821, -0.0506, 1.4154) -- (0.5337, -0.0506, 1.4100) -- cycle;
\fill[blue!15.0, opacity=0.5] (0.5337, -0.0506, 1.4100) -- (0.5821, -0.0506, 1.4154) -- (0.5840, -0.0537, 1.4654) -- (0.5358, -0.0537, 1.4600) -- cycle;
\fill[blue!15.0, opacity=0.5] (0.5358, -0.0537, 1.4600) -- (0.5840, -0.0537, 1.4654) -- (0.5860, -0.0569, 1.5154) -- (0.5379, -0.0569, 1.5100) -- cycle;
\fill[blue!15.0, opacity=0.5] (0.5379, -0.0569, 1.5100) -- (0.5860, -0.0569, 1.5154) -- (0.5880, -0.0600, 1.5654) -- (0.5400, -0.0600, 1.5600) -- cycle;
\fill[blue!15.0, opacity=0.5] (0.5400, -0.0600, 1.5600) -- (0.5880, -0.0600, 1.5654) -- (0.5900, -0.0631, 1.6154) -- (0.5421, -0.0631, 1.6100) -- cycle;
\fill[blue!15.0, opacity=0.5] (0.5421, -0.0631, 1.6100) -- (0.5900, -0.0631, 1.6154) -- (0.5920, -0.0663, 1.6654) -- (0.5442, -0.0663, 1.6600) -- cycle;
\fill[blue!15.0, opacity=0.5] (0.5442, -0.0663, 1.6600) -- (0.5920, -0.0663, 1.6654) -- (0.5939, -0.0694, 1.7154) -- (0.5463, -0.0694, 1.7100) -- cycle;
\fill[blue!15.0, opacity=0.5] (0.5463, -0.0694, 1.7100) -- (0.5939, -0.0694, 1.7154) -- (0.5959, -0.0725, 1.7654) -- (0.5483, -0.0725, 1.7600) -- cycle;
\fill[blue!15.0, opacity=0.5] (0.5483, -0.0725, 1.7600) -- (0.5959, -0.0725, 1.7654) -- (0.5978, -0.0755, 1.8154) -- (0.5504, -0.0755, 1.8100) -- cycle;
\fill[blue!15.0, opacity=0.5] (0.5504, -0.0755, 1.8100) -- (0.5978, -0.0755, 1.8154) -- (0.5997, -0.0785, 1.8654) -- (0.5524, -0.0785, 1.8600) -- cycle;
\fill[blue!15.0, opacity=0.5] (0.5524, -0.0785, 1.8600) -- (0.5997, -0.0785, 1.8654) -- (0.6016, -0.0815, 1.9154) -- (0.5543, -0.0815, 1.9100) -- cycle;
\fill[blue!15.0, opacity=0.5] (0.5543, -0.0815, 1.9100) -- (0.6016, -0.0815, 1.9154) -- (0.6035, -0.0844, 1.9654) -- (0.5563, -0.0844, 1.9600) -- cycle;
\fill[blue!15.0, opacity=0.5] (0.5563, -0.0844, 1.9600) -- (0.6035, -0.0844, 1.9654) -- (0.6053, -0.0872, 2.0154) -- (0.5582, -0.0872, 2.0100) -- cycle;
\fill[blue!15.0, opacity=0.5] (0.5582, -0.0872, 2.0100) -- (0.6053, -0.0872, 2.0154) -- (0.6070, -0.0900, 2.0654) -- (0.5600, -0.0900, 2.0600) -- cycle;
\fill[blue!15.0, opacity=0.5] (0.5600, -0.0900, 2.0600) -- (0.6070, -0.0900, 2.0654) -- (0.6087, -0.0927, 2.1154) -- (0.5618, -0.0927, 2.1100) -- cycle;
\fill[blue!15.0, opacity=0.5] (0.5618, -0.0927, 2.1100) -- (0.6087, -0.0927, 2.1154) -- (0.6103, -0.0953, 2.1654) -- (0.5635, -0.0953, 2.1600) -- cycle;
\fill[blue!15.0, opacity=0.5] (0.5635, -0.0953, 2.1600) -- (0.6103, -0.0953, 2.1654) -- (0.6119, -0.0978, 2.2154) -- (0.5652, -0.0978, 2.2100) -- cycle;
\fill[blue!15.0, opacity=0.5] (0.5652, -0.0978, 2.2100) -- (0.6119, -0.0978, 2.2154) -- (0.6134, -0.1001, 2.2654) -- (0.5668, -0.1001, 2.2600) -- cycle;
\fill[blue!15.1, opacity=0.5] (0.5668, -0.1001, 2.2600) -- (0.6134, -0.1001, 2.2654) -- (0.6149, -0.1024, 2.3154) -- (0.5683, -0.1024, 2.3100) -- cycle;
\fill[blue!15.1, opacity=0.5] (0.5683, -0.1024, 2.3100) -- (0.6149, -0.1024, 2.3154) -- (0.6162, -0.1046, 2.3654) -- (0.5697, -0.1046, 2.3600) -- cycle;
\fill[blue!15.1, opacity=0.5] (0.5697, -0.1046, 2.3600) -- (0.6162, -0.1046, 2.3654) -- (0.6175, -0.1066, 2.4154) -- (0.5711, -0.1066, 2.4100) -- cycle;
\fill[blue!15.2, opacity=0.5] (0.5711, -0.1066, 2.4100) -- (0.6175, -0.1066, 2.4154) -- (0.6187, -0.1085, 2.4654) -- (0.5724, -0.1085, 2.4600) -- cycle;
\fill[blue!15.2, opacity=0.5] (0.5724, -0.1085, 2.4600) -- (0.6187, -0.1085, 2.4654) -- (0.6199, -0.1103, 2.5154) -- (0.5735, -0.1103, 2.5100) -- cycle;
\fill[blue!15.3, opacity=0.5] (0.5735, -0.1103, 2.5100) -- (0.6199, -0.1103, 2.5154) -- (0.6209, -0.1120, 2.5654) -- (0.5746, -0.1120, 2.5600) -- cycle;
\fill[blue!15.4, opacity=0.5] (0.5746, -0.1120, 2.5600) -- (0.6209, -0.1120, 2.5654) -- (0.6219, -0.1135, 2.6154) -- (0.5756, -0.1135, 2.6100) -- cycle;
\fill[blue!15.5, opacity=0.5] (0.5756, -0.1135, 2.6100) -- (0.6219, -0.1135, 2.6154) -- (0.6227, -0.1148, 2.6654) -- (0.5765, -0.1148, 2.6600) -- cycle;
\fill[blue!15.6, opacity=0.5] (0.5765, -0.1148, 2.6600) -- (0.6227, -0.1148, 2.6654) -- (0.6235, -0.1160, 2.7154) -- (0.5773, -0.1160, 2.7100) -- cycle;
\fill[blue!15.7, opacity=0.5] (0.5773, -0.1160, 2.7100) -- (0.6235, -0.1160, 2.7154) -- (0.6241, -0.1171, 2.7654) -- (0.5780, -0.1171, 2.7600) -- cycle;
\fill[blue!15.9, opacity=0.5] (0.5780, -0.1171, 2.7600) -- (0.6241, -0.1171, 2.7654) -- (0.6247, -0.1180, 2.8154) -- (0.5786, -0.1180, 2.8100) -- cycle;
\fill[blue!16.1, opacity=0.5] (0.5786, -0.1180, 2.8100) -- (0.6247, -0.1180, 2.8154) -- (0.6252, -0.1187, 2.8654) -- (0.5791, -0.1187, 2.8600) -- cycle;
\fill[blue!16.3, opacity=0.5] (0.5791, -0.1187, 2.8600) -- (0.6252, -0.1187, 2.8654) -- (0.6255, -0.1193, 2.9154) -- (0.5795, -0.1193, 2.9100) -- cycle;
\fill[blue!16.6, opacity=0.5] (0.5795, -0.1193, 2.9100) -- (0.6255, -0.1193, 2.9154) -- (0.6258, -0.1197, 2.9654) -- (0.5798, -0.1197, 2.9600) -- cycle;
\fill[blue!16.8, opacity=0.5] (0.5798, -0.1197, 2.9600) -- (0.6258, -0.1197, 2.9654) -- (0.6259, -0.1199, 3.0154) -- (0.5799, -0.1199, 3.0100) -- cycle;
\fill[blue!17.1, opacity=0.5] (0.5799, -0.1199, 3.0100) -- (0.6259, -0.1199, 3.0154) -- (0.6260, -0.1200, 3.0654) -- (0.5800, -0.1200, 3.0600) -- cycle;
\fill[blue!15.0, opacity=0.5] (0.5500, -0.0000, 0.0654) -- (0.6000, -0.0000, 0.0705) -- (0.6000, -0.0001, 0.1205) -- (0.5501, -0.0001, 0.1154) -- cycle;
\fill[blue!15.0, opacity=0.5] (0.5501, -0.0001, 0.1154) -- (0.6000, -0.0001, 0.1205) -- (0.6002, -0.0003, 0.1705) -- (0.5502, -0.0003, 0.1654) -- cycle;
\fill[blue!15.0, opacity=0.5] (0.5502, -0.0003, 0.1654) -- (0.6002, -0.0003, 0.1705) -- (0.6004, -0.0007, 0.2205) -- (0.5505, -0.0007, 0.2154) -- cycle;
\fill[blue!15.0, opacity=0.5] (0.5505, -0.0007, 0.2154) -- (0.6004, -0.0007, 0.2205) -- (0.6008, -0.0013, 0.2705) -- (0.5508, -0.0013, 0.2654) -- cycle;
\fill[blue!15.0, opacity=0.5] (0.5508, -0.0013, 0.2654) -- (0.6008, -0.0013, 0.2705) -- (0.6012, -0.0020, 0.3205) -- (0.5513, -0.0020, 0.3154) -- cycle;
\fill[blue!15.0, opacity=0.5] (0.5513, -0.0020, 0.3154) -- (0.6012, -0.0020, 0.3205) -- (0.6018, -0.0029, 0.3705) -- (0.5519, -0.0029, 0.3654) -- cycle;
\fill[blue!15.0, opacity=0.5] (0.5519, -0.0029, 0.3654) -- (0.6018, -0.0029, 0.3705) -- (0.6024, -0.0040, 0.4205) -- (0.5525, -0.0040, 0.4154) -- cycle;
\fill[blue!15.0, opacity=0.5] (0.5525, -0.0040, 0.4154) -- (0.6024, -0.0040, 0.4205) -- (0.6031, -0.0052, 0.4705) -- (0.5533, -0.0052, 0.4654) -- cycle;
\fill[blue!15.0, opacity=0.5] (0.5533, -0.0052, 0.4654) -- (0.6031, -0.0052, 0.4705) -- (0.6039, -0.0065, 0.5205) -- (0.5541, -0.0065, 0.5154) -- cycle;
\fill[blue!15.0, opacity=0.5] (0.5541, -0.0065, 0.5154) -- (0.6039, -0.0065, 0.5205) -- (0.6048, -0.0080, 0.5705) -- (0.5551, -0.0080, 0.5654) -- cycle;
\fill[blue!15.0, opacity=0.5] (0.5551, -0.0080, 0.5654) -- (0.6048, -0.0080, 0.5705) -- (0.6058, -0.0097, 0.6205) -- (0.5561, -0.0097, 0.6154) -- cycle;
\fill[blue!15.0, opacity=0.5] (0.5561, -0.0097, 0.6154) -- (0.6058, -0.0097, 0.6205) -- (0.6069, -0.0115, 0.6705) -- (0.5573, -0.0115, 0.6654) -- cycle;
\fill[blue!15.0, opacity=0.5] (0.5573, -0.0115, 0.6654) -- (0.6069, -0.0115, 0.6705) -- (0.6080, -0.0134, 0.7205) -- (0.5585, -0.0134, 0.7154) -- cycle;
\fill[blue!15.0, opacity=0.5] (0.5585, -0.0134, 0.7154) -- (0.6080, -0.0134, 0.7205) -- (0.6092, -0.0154, 0.7705) -- (0.5598, -0.0154, 0.7654) -- cycle;
\fill[blue!15.0, opacity=0.5] (0.5598, -0.0154, 0.7654) -- (0.6092, -0.0154, 0.7705) -- (0.6105, -0.0176, 0.8205) -- (0.5611, -0.0176, 0.8154) -- cycle;
\fill[blue!15.0, opacity=0.5] (0.5611, -0.0176, 0.8154) -- (0.6105, -0.0176, 0.8205) -- (0.6119, -0.0199, 0.8705) -- (0.5626, -0.0199, 0.8654) -- cycle;
\fill[blue!15.0, opacity=0.5] (0.5626, -0.0199, 0.8654) -- (0.6119, -0.0199, 0.8705) -- (0.6133, -0.0222, 0.9205) -- (0.5641, -0.0222, 0.9154) -- cycle;
\fill[blue!15.0, opacity=0.5] (0.5641, -0.0222, 0.9154) -- (0.6133, -0.0222, 0.9205) -- (0.6148, -0.0247, 0.9705) -- (0.5657, -0.0247, 0.9654) -- cycle;
\fill[blue!15.0, opacity=0.5] (0.5657, -0.0247, 0.9654) -- (0.6148, -0.0247, 0.9705) -- (0.6164, -0.0273, 1.0205) -- (0.5673, -0.0273, 1.0154) -- cycle;
\fill[blue!15.0, opacity=0.5] (0.5673, -0.0273, 1.0154) -- (0.6164, -0.0273, 1.0205) -- (0.6180, -0.0300, 1.0705) -- (0.5690, -0.0300, 1.0654) -- cycle;
\fill[blue!15.0, opacity=0.5] (0.5690, -0.0300, 1.0654) -- (0.6180, -0.0300, 1.0705) -- (0.6197, -0.0328, 1.1205) -- (0.5707, -0.0328, 1.1154) -- cycle;
\fill[blue!15.0, opacity=0.5] (0.5707, -0.0328, 1.1154) -- (0.6197, -0.0328, 1.1205) -- (0.6214, -0.0356, 1.1705) -- (0.5725, -0.0356, 1.1654) -- cycle;
\fill[blue!15.0, opacity=0.5] (0.5725, -0.0356, 1.1654) -- (0.6214, -0.0356, 1.1705) -- (0.6231, -0.0385, 1.2205) -- (0.5744, -0.0385, 1.2154) -- cycle;
\fill[blue!15.0, opacity=0.5] (0.5744, -0.0385, 1.2154) -- (0.6231, -0.0385, 1.2205) -- (0.6249, -0.0415, 1.2705) -- (0.5763, -0.0415, 1.2654) -- cycle;
\fill[blue!15.0, opacity=0.5] (0.5763, -0.0415, 1.2654) -- (0.6249, -0.0415, 1.2705) -- (0.6267, -0.0445, 1.3205) -- (0.5782, -0.0445, 1.3154) -- cycle;
\fill[blue!15.0, opacity=0.5] (0.5782, -0.0445, 1.3154) -- (0.6267, -0.0445, 1.3205) -- (0.6285, -0.0475, 1.3705) -- (0.5801, -0.0475, 1.3654) -- cycle;
\fill[blue!15.0, opacity=0.5] (0.5801, -0.0475, 1.3654) -- (0.6285, -0.0475, 1.3705) -- (0.6304, -0.0506, 1.4205) -- (0.5821, -0.0506, 1.4154) -- cycle;
\fill[blue!15.0, opacity=0.5] (0.5821, -0.0506, 1.4154) -- (0.6304, -0.0506, 1.4205) -- (0.6322, -0.0537, 1.4705) -- (0.5840, -0.0537, 1.4654) -- cycle;
\fill[blue!15.0, opacity=0.5] (0.5840, -0.0537, 1.4654) -- (0.6322, -0.0537, 1.4705) -- (0.6341, -0.0569, 1.5205) -- (0.5860, -0.0569, 1.5154) -- cycle;
\fill[blue!15.0, opacity=0.5] (0.5860, -0.0569, 1.5154) -- (0.6341, -0.0569, 1.5205) -- (0.6360, -0.0600, 1.5705) -- (0.5880, -0.0600, 1.5654) -- cycle;
\fill[blue!15.0, opacity=0.5] (0.5880, -0.0600, 1.5654) -- (0.6360, -0.0600, 1.5705) -- (0.6379, -0.0631, 1.6205) -- (0.5900, -0.0631, 1.6154) -- cycle;
\fill[blue!15.0, opacity=0.5] (0.5900, -0.0631, 1.6154) -- (0.6379, -0.0631, 1.6205) -- (0.6398, -0.0663, 1.6705) -- (0.5920, -0.0663, 1.6654) -- cycle;
\fill[blue!15.0, opacity=0.5] (0.5920, -0.0663, 1.6654) -- (0.6398, -0.0663, 1.6705) -- (0.6416, -0.0694, 1.7205) -- (0.5939, -0.0694, 1.7154) -- cycle;
\fill[blue!15.0, opacity=0.5] (0.5939, -0.0694, 1.7154) -- (0.6416, -0.0694, 1.7205) -- (0.6435, -0.0725, 1.7705) -- (0.5959, -0.0725, 1.7654) -- cycle;
\fill[blue!15.0, opacity=0.5] (0.5959, -0.0725, 1.7654) -- (0.6435, -0.0725, 1.7705) -- (0.6453, -0.0755, 1.8205) -- (0.5978, -0.0755, 1.8154) -- cycle;
\fill[blue!15.0, opacity=0.5] (0.5978, -0.0755, 1.8154) -- (0.6453, -0.0755, 1.8205) -- (0.6471, -0.0785, 1.8705) -- (0.5997, -0.0785, 1.8654) -- cycle;
\fill[blue!15.0, opacity=0.5] (0.5997, -0.0785, 1.8654) -- (0.6471, -0.0785, 1.8705) -- (0.6489, -0.0815, 1.9205) -- (0.6016, -0.0815, 1.9154) -- cycle;
\fill[blue!15.0, opacity=0.5] (0.6016, -0.0815, 1.9154) -- (0.6489, -0.0815, 1.9205) -- (0.6506, -0.0844, 1.9705) -- (0.6035, -0.0844, 1.9654) -- cycle;
\fill[blue!15.1, opacity=0.5] (0.6035, -0.0844, 1.9654) -- (0.6506, -0.0844, 1.9705) -- (0.6523, -0.0872, 2.0205) -- (0.6053, -0.0872, 2.0154) -- cycle;
\fill[blue!15.1, opacity=0.5] (0.6053, -0.0872, 2.0154) -- (0.6523, -0.0872, 2.0205) -- (0.6540, -0.0900, 2.0705) -- (0.6070, -0.0900, 2.0654) -- cycle;
\fill[blue!15.1, opacity=0.5] (0.6070, -0.0900, 2.0654) -- (0.6540, -0.0900, 2.0705) -- (0.6556, -0.0927, 2.1205) -- (0.6087, -0.0927, 2.1154) -- cycle;
\fill[blue!15.2, opacity=0.5] (0.6087, -0.0927, 2.1154) -- (0.6556, -0.0927, 2.1205) -- (0.6572, -0.0953, 2.1705) -- (0.6103, -0.0953, 2.1654) -- cycle;
\fill[blue!15.2, opacity=0.5] (0.6103, -0.0953, 2.1654) -- (0.6572, -0.0953, 2.1705) -- (0.6587, -0.0978, 2.2205) -- (0.6119, -0.0978, 2.2154) -- cycle;
\fill[blue!15.3, opacity=0.5] (0.6119, -0.0978, 2.2154) -- (0.6587, -0.0978, 2.2205) -- (0.6601, -0.1001, 2.2705) -- (0.6134, -0.1001, 2.2654) -- cycle;
\fill[blue!15.4, opacity=0.5] (0.6134, -0.1001, 2.2654) -- (0.6601, -0.1001, 2.2705) -- (0.6615, -0.1024, 2.3205) -- (0.6149, -0.1024, 2.3154) -- cycle;
\fill[blue!15.6, opacity=0.5] (0.6149, -0.1024, 2.3154) -- (0.6615, -0.1024, 2.3205) -- (0.6628, -0.1046, 2.3705) -- (0.6162, -0.1046, 2.3654) -- cycle;
\fill[blue!15.7, opacity=0.5] (0.6162, -0.1046, 2.3654) -- (0.6628, -0.1046, 2.3705) -- (0.6640, -0.1066, 2.4205) -- (0.6175, -0.1066, 2.4154) -- cycle;
\fill[blue!15.9, opacity=0.5] (0.6175, -0.1066, 2.4154) -- (0.6640, -0.1066, 2.4205) -- (0.6651, -0.1085, 2.4705) -- (0.6187, -0.1085, 2.4654) -- cycle;
\fill[blue!16.1, opacity=0.5] (0.6187, -0.1085, 2.4654) -- (0.6651, -0.1085, 2.4705) -- (0.6662, -0.1103, 2.5205) -- (0.6199, -0.1103, 2.5154) -- cycle;
\fill[blue!16.4, opacity=0.5] (0.6199, -0.1103, 2.5154) -- (0.6662, -0.1103, 2.5205) -- (0.6672, -0.1120, 2.5705) -- (0.6209, -0.1120, 2.5654) -- cycle;
\fill[blue!16.6, opacity=0.5] (0.6209, -0.1120, 2.5654) -- (0.6672, -0.1120, 2.5705) -- (0.6681, -0.1135, 2.6205) -- (0.6219, -0.1135, 2.6154) -- cycle;
\fill[blue!17.0, opacity=0.5] (0.6219, -0.1135, 2.6154) -- (0.6681, -0.1135, 2.6205) -- (0.6689, -0.1148, 2.6705) -- (0.6227, -0.1148, 2.6654) -- cycle;
\fill[blue!17.3, opacity=0.5] (0.6227, -0.1148, 2.6654) -- (0.6689, -0.1148, 2.6705) -- (0.6696, -0.1160, 2.7205) -- (0.6235, -0.1160, 2.7154) -- cycle;
\fill[blue!17.7, opacity=0.5] (0.6235, -0.1160, 2.7154) -- (0.6696, -0.1160, 2.7205) -- (0.6702, -0.1171, 2.7705) -- (0.6241, -0.1171, 2.7654) -- cycle;
\fill[blue!18.2, opacity=0.5] (0.6241, -0.1171, 2.7654) -- (0.6702, -0.1171, 2.7705) -- (0.6708, -0.1180, 2.8205) -- (0.6247, -0.1180, 2.8154) -- cycle;
\fill[blue!18.7, opacity=0.5] (0.6247, -0.1180, 2.8154) -- (0.6708, -0.1180, 2.8205) -- (0.6712, -0.1187, 2.8705) -- (0.6252, -0.1187, 2.8654) -- cycle;
\fill[blue!19.2, opacity=0.5] (0.6252, -0.1187, 2.8654) -- (0.6712, -0.1187, 2.8705) -- (0.6716, -0.1193, 2.9205) -- (0.6255, -0.1193, 2.9154) -- cycle;
\fill[blue!19.7, opacity=0.5] (0.6255, -0.1193, 2.9154) -- (0.6716, -0.1193, 2.9205) -- (0.6718, -0.1197, 2.9705) -- (0.6258, -0.1197, 2.9654) -- cycle;
\fill[blue!20.3, opacity=0.5] (0.6258, -0.1197, 2.9654) -- (0.6718, -0.1197, 2.9705) -- (0.6720, -0.1199, 3.0205) -- (0.6259, -0.1199, 3.0154) -- cycle;
\fill[blue!20.9, opacity=0.5] (0.6259, -0.1199, 3.0154) -- (0.6720, -0.1199, 3.0205) -- (0.6720, -0.1200, 3.0705) -- (0.6260, -0.1200, 3.0654) -- cycle;
\fill[blue!15.0, opacity=0.5] (0.6000, -0.0000, 0.0705) -- (0.6500, -0.0000, 0.0755) -- (0.6500, -0.0001, 0.1255) -- (0.6000, -0.0001, 0.1205) -- cycle;
\fill[blue!15.0, opacity=0.5] (0.6000, -0.0001, 0.1205) -- (0.6500, -0.0001, 0.1255) -- (0.6502, -0.0003, 0.1755) -- (0.6002, -0.0003, 0.1705) -- cycle;
\fill[blue!15.0, opacity=0.5] (0.6002, -0.0003, 0.1705) -- (0.6502, -0.0003, 0.1755) -- (0.6504, -0.0007, 0.2255) -- (0.6004, -0.0007, 0.2205) -- cycle;
\fill[blue!15.0, opacity=0.5] (0.6004, -0.0007, 0.2205) -- (0.6504, -0.0007, 0.2255) -- (0.6507, -0.0013, 0.2755) -- (0.6008, -0.0013, 0.2705) -- cycle;
\fill[blue!15.0, opacity=0.5] (0.6008, -0.0013, 0.2705) -- (0.6507, -0.0013, 0.2755) -- (0.6512, -0.0020, 0.3255) -- (0.6012, -0.0020, 0.3205) -- cycle;
\fill[blue!15.0, opacity=0.5] (0.6012, -0.0020, 0.3205) -- (0.6512, -0.0020, 0.3255) -- (0.6517, -0.0029, 0.3755) -- (0.6018, -0.0029, 0.3705) -- cycle;
\fill[blue!15.0, opacity=0.5] (0.6018, -0.0029, 0.3705) -- (0.6517, -0.0029, 0.3755) -- (0.6523, -0.0040, 0.4255) -- (0.6024, -0.0040, 0.4205) -- cycle;
\fill[blue!15.0, opacity=0.5] (0.6024, -0.0040, 0.4205) -- (0.6523, -0.0040, 0.4255) -- (0.6529, -0.0052, 0.4755) -- (0.6031, -0.0052, 0.4705) -- cycle;
\fill[blue!15.0, opacity=0.5] (0.6031, -0.0052, 0.4705) -- (0.6529, -0.0052, 0.4755) -- (0.6537, -0.0065, 0.5255) -- (0.6039, -0.0065, 0.5205) -- cycle;
\fill[blue!15.0, opacity=0.5] (0.6039, -0.0065, 0.5205) -- (0.6537, -0.0065, 0.5255) -- (0.6546, -0.0080, 0.5755) -- (0.6048, -0.0080, 0.5705) -- cycle;
\fill[blue!15.0, opacity=0.5] (0.6048, -0.0080, 0.5705) -- (0.6546, -0.0080, 0.5755) -- (0.6555, -0.0097, 0.6255) -- (0.6058, -0.0097, 0.6205) -- cycle;
\fill[blue!15.0, opacity=0.5] (0.6058, -0.0097, 0.6205) -- (0.6555, -0.0097, 0.6255) -- (0.6565, -0.0115, 0.6755) -- (0.6069, -0.0115, 0.6705) -- cycle;
\fill[blue!15.0, opacity=0.5] (0.6069, -0.0115, 0.6705) -- (0.6565, -0.0115, 0.6755) -- (0.6576, -0.0134, 0.7255) -- (0.6080, -0.0134, 0.7205) -- cycle;
\fill[blue!15.0, opacity=0.5] (0.6080, -0.0134, 0.7205) -- (0.6576, -0.0134, 0.7255) -- (0.6587, -0.0154, 0.7755) -- (0.6092, -0.0154, 0.7705) -- cycle;
\fill[blue!15.0, opacity=0.5] (0.6092, -0.0154, 0.7705) -- (0.6587, -0.0154, 0.7755) -- (0.6600, -0.0176, 0.8255) -- (0.6105, -0.0176, 0.8205) -- cycle;
\fill[blue!15.0, opacity=0.5] (0.6105, -0.0176, 0.8205) -- (0.6600, -0.0176, 0.8255) -- (0.6612, -0.0199, 0.8755) -- (0.6119, -0.0199, 0.8705) -- cycle;
\fill[blue!15.0, opacity=0.5] (0.6119, -0.0199, 0.8705) -- (0.6612, -0.0199, 0.8755) -- (0.6626, -0.0222, 0.9255) -- (0.6133, -0.0222, 0.9205) -- cycle;
\fill[blue!15.0, opacity=0.5] (0.6133, -0.0222, 0.9205) -- (0.6626, -0.0222, 0.9255) -- (0.6640, -0.0247, 0.9755) -- (0.6148, -0.0247, 0.9705) -- cycle;
\fill[blue!15.0, opacity=0.5] (0.6148, -0.0247, 0.9705) -- (0.6640, -0.0247, 0.9755) -- (0.6655, -0.0273, 1.0255) -- (0.6164, -0.0273, 1.0205) -- cycle;
\fill[blue!15.0, opacity=0.5] (0.6164, -0.0273, 1.0205) -- (0.6655, -0.0273, 1.0255) -- (0.6670, -0.0300, 1.0755) -- (0.6180, -0.0300, 1.0705) -- cycle;
\fill[blue!15.0, opacity=0.5] (0.6180, -0.0300, 1.0705) -- (0.6670, -0.0300, 1.0755) -- (0.6686, -0.0328, 1.1255) -- (0.6197, -0.0328, 1.1205) -- cycle;
\fill[blue!15.0, opacity=0.5] (0.6197, -0.0328, 1.1205) -- (0.6686, -0.0328, 1.1255) -- (0.6702, -0.0356, 1.1755) -- (0.6214, -0.0356, 1.1705) -- cycle;
\fill[blue!15.0, opacity=0.5] (0.6214, -0.0356, 1.1705) -- (0.6702, -0.0356, 1.1755) -- (0.6718, -0.0385, 1.2255) -- (0.6231, -0.0385, 1.2205) -- cycle;
\fill[blue!15.0, opacity=0.5] (0.6231, -0.0385, 1.2205) -- (0.6718, -0.0385, 1.2255) -- (0.6735, -0.0415, 1.2755) -- (0.6249, -0.0415, 1.2705) -- cycle;
\fill[blue!15.0, opacity=0.5] (0.6249, -0.0415, 1.2705) -- (0.6735, -0.0415, 1.2755) -- (0.6752, -0.0445, 1.3255) -- (0.6267, -0.0445, 1.3205) -- cycle;
\fill[blue!15.0, opacity=0.5] (0.6267, -0.0445, 1.3205) -- (0.6752, -0.0445, 1.3255) -- (0.6769, -0.0475, 1.3755) -- (0.6285, -0.0475, 1.3705) -- cycle;
\fill[blue!15.0, opacity=0.5] (0.6285, -0.0475, 1.3705) -- (0.6769, -0.0475, 1.3755) -- (0.6787, -0.0506, 1.4255) -- (0.6304, -0.0506, 1.4205) -- cycle;
\fill[blue!15.0, opacity=0.5] (0.6304, -0.0506, 1.4205) -- (0.6787, -0.0506, 1.4255) -- (0.6804, -0.0537, 1.4755) -- (0.6322, -0.0537, 1.4705) -- cycle;
\fill[blue!15.0, opacity=0.5] (0.6322, -0.0537, 1.4705) -- (0.6804, -0.0537, 1.4755) -- (0.6822, -0.0569, 1.5255) -- (0.6341, -0.0569, 1.5205) -- cycle;
\fill[blue!15.0, opacity=0.5] (0.6341, -0.0569, 1.5205) -- (0.6822, -0.0569, 1.5255) -- (0.6840, -0.0600, 1.5755) -- (0.6360, -0.0600, 1.5705) -- cycle;
\fill[blue!15.0, opacity=0.5] (0.6360, -0.0600, 1.5705) -- (0.6840, -0.0600, 1.5755) -- (0.6858, -0.0631, 1.6255) -- (0.6379, -0.0631, 1.6205) -- cycle;
\fill[blue!15.0, opacity=0.5] (0.6379, -0.0631, 1.6205) -- (0.6858, -0.0631, 1.6255) -- (0.6876, -0.0663, 1.6755) -- (0.6398, -0.0663, 1.6705) -- cycle;
\fill[blue!15.1, opacity=0.5] (0.6398, -0.0663, 1.6705) -- (0.6876, -0.0663, 1.6755) -- (0.6893, -0.0694, 1.7255) -- (0.6416, -0.0694, 1.7205) -- cycle;
\fill[blue!15.1, opacity=0.5] (0.6416, -0.0694, 1.7205) -- (0.6893, -0.0694, 1.7255) -- (0.6911, -0.0725, 1.7755) -- (0.6435, -0.0725, 1.7705) -- cycle;
\fill[blue!15.1, opacity=0.5] (0.6435, -0.0725, 1.7705) -- (0.6911, -0.0725, 1.7755) -- (0.6928, -0.0755, 1.8255) -- (0.6453, -0.0755, 1.8205) -- cycle;
\fill[blue!15.2, opacity=0.5] (0.6453, -0.0755, 1.8205) -- (0.6928, -0.0755, 1.8255) -- (0.6945, -0.0785, 1.8755) -- (0.6471, -0.0785, 1.8705) -- cycle;
\fill[blue!15.3, opacity=0.5] (0.6471, -0.0785, 1.8705) -- (0.6945, -0.0785, 1.8755) -- (0.6962, -0.0815, 1.9255) -- (0.6489, -0.0815, 1.9205) -- cycle;
\fill[blue!15.4, opacity=0.5] (0.6489, -0.0815, 1.9205) -- (0.6962, -0.0815, 1.9255) -- (0.6978, -0.0844, 1.9755) -- (0.6506, -0.0844, 1.9705) -- cycle;
\fill[blue!15.5, opacity=0.5] (0.6506, -0.0844, 1.9705) -- (0.6978, -0.0844, 1.9755) -- (0.6994, -0.0872, 2.0255) -- (0.6523, -0.0872, 2.0205) -- cycle;
\fill[blue!15.6, opacity=0.5] (0.6523, -0.0872, 2.0205) -- (0.6994, -0.0872, 2.0255) -- (0.7010, -0.0900, 2.0755) -- (0.6540, -0.0900, 2.0705) -- cycle;
\fill[blue!15.8, opacity=0.5] (0.6540, -0.0900, 2.0705) -- (0.7010, -0.0900, 2.0755) -- (0.7025, -0.0927, 2.1255) -- (0.6556, -0.0927, 2.1205) -- cycle;
\fill[blue!16.1, opacity=0.5] (0.6556, -0.0927, 2.1205) -- (0.7025, -0.0927, 2.1255) -- (0.7040, -0.0953, 2.1755) -- (0.6572, -0.0953, 2.1705) -- cycle;
\fill[blue!16.4, opacity=0.5] (0.6572, -0.0953, 2.1705) -- (0.7040, -0.0953, 2.1755) -- (0.7054, -0.0978, 2.2255) -- (0.6587, -0.0978, 2.2205) -- cycle;
\fill[blue!16.7, opacity=0.5] (0.6587, -0.0978, 2.2205) -- (0.7054, -0.0978, 2.2255) -- (0.7068, -0.1001, 2.2755) -- (0.6601, -0.1001, 2.2705) -- cycle;
\fill[blue!17.1, opacity=0.5] (0.6601, -0.1001, 2.2705) -- (0.7068, -0.1001, 2.2755) -- (0.7080, -0.1024, 2.3255) -- (0.6615, -0.1024, 2.3205) -- cycle;
\fill[blue!17.6, opacity=0.5] (0.6615, -0.1024, 2.3205) -- (0.7080, -0.1024, 2.3255) -- (0.7093, -0.1046, 2.3755) -- (0.6628, -0.1046, 2.3705) -- cycle;
\fill[blue!18.1, opacity=0.5] (0.6628, -0.1046, 2.3705) -- (0.7093, -0.1046, 2.3755) -- (0.7104, -0.1066, 2.4255) -- (0.6640, -0.1066, 2.4205) -- cycle;
\fill[blue!18.6, opacity=0.5] (0.6640, -0.1066, 2.4205) -- (0.7104, -0.1066, 2.4255) -- (0.7115, -0.1085, 2.4755) -- (0.6651, -0.1085, 2.4705) -- cycle;
\fill[blue!19.2, opacity=0.5] (0.6651, -0.1085, 2.4705) -- (0.7115, -0.1085, 2.4755) -- (0.7125, -0.1103, 2.5255) -- (0.6662, -0.1103, 2.5205) -- cycle;
\fill[blue!19.9, opacity=0.5] (0.6662, -0.1103, 2.5205) -- (0.7125, -0.1103, 2.5255) -- (0.7134, -0.1120, 2.5755) -- (0.6672, -0.1120, 2.5705) -- cycle;
\fill[blue!20.7, opacity=0.5] (0.6672, -0.1120, 2.5705) -- (0.7134, -0.1120, 2.5755) -- (0.7143, -0.1135, 2.6255) -- (0.6681, -0.1135, 2.6205) -- cycle;
\fill[blue!21.4, opacity=0.5] (0.6681, -0.1135, 2.6205) -- (0.7143, -0.1135, 2.6255) -- (0.7151, -0.1148, 2.6755) -- (0.6689, -0.1148, 2.6705) -- cycle;
\fill[blue!22.3, opacity=0.5] (0.6689, -0.1148, 2.6705) -- (0.7151, -0.1148, 2.6755) -- (0.7157, -0.1160, 2.7255) -- (0.6696, -0.1160, 2.7205) -- cycle;
\fill[blue!23.1, opacity=0.5] (0.6696, -0.1160, 2.7205) -- (0.7157, -0.1160, 2.7255) -- (0.7163, -0.1171, 2.7755) -- (0.6702, -0.1171, 2.7705) -- cycle;
\fill[blue!24.0, opacity=0.5] (0.6702, -0.1171, 2.7705) -- (0.7163, -0.1171, 2.7755) -- (0.7168, -0.1180, 2.8255) -- (0.6708, -0.1180, 2.8205) -- cycle;
\fill[blue!24.9, opacity=0.5] (0.6708, -0.1180, 2.8205) -- (0.7168, -0.1180, 2.8255) -- (0.7173, -0.1187, 2.8755) -- (0.6712, -0.1187, 2.8705) -- cycle;
\fill[blue!25.9, opacity=0.5] (0.6712, -0.1187, 2.8705) -- (0.7173, -0.1187, 2.8755) -- (0.7176, -0.1193, 2.9255) -- (0.6716, -0.1193, 2.9205) -- cycle;
\fill[blue!26.8, opacity=0.5] (0.6716, -0.1193, 2.9205) -- (0.7176, -0.1193, 2.9255) -- (0.7178, -0.1197, 2.9755) -- (0.6718, -0.1197, 2.9705) -- cycle;
\fill[blue!27.8, opacity=0.5] (0.6718, -0.1197, 2.9705) -- (0.7178, -0.1197, 2.9755) -- (0.7180, -0.1199, 3.0255) -- (0.6720, -0.1199, 3.0205) -- cycle;
\fill[blue!28.7, opacity=0.5] (0.6720, -0.1199, 3.0205) -- (0.7180, -0.1199, 3.0255) -- (0.7180, -0.1200, 3.0755) -- (0.6720, -0.1200, 3.0705) -- cycle;
\fill[blue!15.0, opacity=0.5] (0.6500, -0.0000, 0.0755) -- (0.7000, -0.0000, 0.0803) -- (0.7000, -0.0001, 0.1303) -- (0.6500, -0.0001, 0.1255) -- cycle;
\fill[blue!15.0, opacity=0.5] (0.6500, -0.0001, 0.1255) -- (0.7000, -0.0001, 0.1303) -- (0.7002, -0.0003, 0.1803) -- (0.6502, -0.0003, 0.1755) -- cycle;
\fill[blue!15.0, opacity=0.5] (0.6502, -0.0003, 0.1755) -- (0.7002, -0.0003, 0.1803) -- (0.7004, -0.0007, 0.2303) -- (0.6504, -0.0007, 0.2255) -- cycle;
\fill[blue!15.0, opacity=0.5] (0.6504, -0.0007, 0.2255) -- (0.7004, -0.0007, 0.2303) -- (0.7007, -0.0013, 0.2803) -- (0.6507, -0.0013, 0.2755) -- cycle;
\fill[blue!15.0, opacity=0.5] (0.6507, -0.0013, 0.2755) -- (0.7007, -0.0013, 0.2803) -- (0.7011, -0.0020, 0.3303) -- (0.6512, -0.0020, 0.3255) -- cycle;
\fill[blue!15.0, opacity=0.5] (0.6512, -0.0020, 0.3255) -- (0.7011, -0.0020, 0.3303) -- (0.7016, -0.0029, 0.3803) -- (0.6517, -0.0029, 0.3755) -- cycle;
\fill[blue!15.0, opacity=0.5] (0.6517, -0.0029, 0.3755) -- (0.7016, -0.0029, 0.3803) -- (0.7021, -0.0040, 0.4303) -- (0.6523, -0.0040, 0.4255) -- cycle;
\fill[blue!15.0, opacity=0.5] (0.6523, -0.0040, 0.4255) -- (0.7021, -0.0040, 0.4303) -- (0.7028, -0.0052, 0.4803) -- (0.6529, -0.0052, 0.4755) -- cycle;
\fill[blue!15.0, opacity=0.5] (0.6529, -0.0052, 0.4755) -- (0.7028, -0.0052, 0.4803) -- (0.7035, -0.0065, 0.5303) -- (0.6537, -0.0065, 0.5255) -- cycle;
\fill[blue!15.0, opacity=0.5] (0.6537, -0.0065, 0.5255) -- (0.7035, -0.0065, 0.5303) -- (0.7043, -0.0080, 0.5803) -- (0.6546, -0.0080, 0.5755) -- cycle;
\fill[blue!15.0, opacity=0.5] (0.6546, -0.0080, 0.5755) -- (0.7043, -0.0080, 0.5803) -- (0.7052, -0.0097, 0.6303) -- (0.6555, -0.0097, 0.6255) -- cycle;
\fill[blue!15.0, opacity=0.5] (0.6555, -0.0097, 0.6255) -- (0.7052, -0.0097, 0.6303) -- (0.7061, -0.0115, 0.6803) -- (0.6565, -0.0115, 0.6755) -- cycle;
\fill[blue!15.0, opacity=0.5] (0.6565, -0.0115, 0.6755) -- (0.7061, -0.0115, 0.6803) -- (0.7071, -0.0134, 0.7303) -- (0.6576, -0.0134, 0.7255) -- cycle;
\fill[blue!15.0, opacity=0.5] (0.6576, -0.0134, 0.7255) -- (0.7071, -0.0134, 0.7303) -- (0.7082, -0.0154, 0.7803) -- (0.6587, -0.0154, 0.7755) -- cycle;
\fill[blue!15.0, opacity=0.5] (0.6587, -0.0154, 0.7755) -- (0.7082, -0.0154, 0.7803) -- (0.7094, -0.0176, 0.8303) -- (0.6600, -0.0176, 0.8255) -- cycle;
\fill[blue!15.0, opacity=0.5] (0.6600, -0.0176, 0.8255) -- (0.7094, -0.0176, 0.8303) -- (0.7106, -0.0199, 0.8803) -- (0.6612, -0.0199, 0.8755) -- cycle;
\fill[blue!15.0, opacity=0.5] (0.6612, -0.0199, 0.8755) -- (0.7106, -0.0199, 0.8803) -- (0.7119, -0.0222, 0.9303) -- (0.6626, -0.0222, 0.9255) -- cycle;
\fill[blue!15.0, opacity=0.5] (0.6626, -0.0222, 0.9255) -- (0.7119, -0.0222, 0.9303) -- (0.7132, -0.0247, 0.9803) -- (0.6640, -0.0247, 0.9755) -- cycle;
\fill[blue!15.0, opacity=0.5] (0.6640, -0.0247, 0.9755) -- (0.7132, -0.0247, 0.9803) -- (0.7146, -0.0273, 1.0303) -- (0.6655, -0.0273, 1.0255) -- cycle;
\fill[blue!15.0, opacity=0.5] (0.6655, -0.0273, 1.0255) -- (0.7146, -0.0273, 1.0303) -- (0.7160, -0.0300, 1.0803) -- (0.6670, -0.0300, 1.0755) -- cycle;
\fill[blue!15.0, opacity=0.5] (0.6670, -0.0300, 1.0755) -- (0.7160, -0.0300, 1.0803) -- (0.7175, -0.0328, 1.1303) -- (0.6686, -0.0328, 1.1255) -- cycle;
\fill[blue!15.0, opacity=0.5] (0.6686, -0.0328, 1.1255) -- (0.7175, -0.0328, 1.1303) -- (0.7190, -0.0356, 1.1803) -- (0.6702, -0.0356, 1.1755) -- cycle;
\fill[blue!15.0, opacity=0.5] (0.6702, -0.0356, 1.1755) -- (0.7190, -0.0356, 1.1803) -- (0.7205, -0.0385, 1.2303) -- (0.6718, -0.0385, 1.2255) -- cycle;
\fill[blue!15.0, opacity=0.5] (0.6718, -0.0385, 1.2255) -- (0.7205, -0.0385, 1.2303) -- (0.7221, -0.0415, 1.2803) -- (0.6735, -0.0415, 1.2755) -- cycle;
\fill[blue!15.0, opacity=0.5] (0.6735, -0.0415, 1.2755) -- (0.7221, -0.0415, 1.2803) -- (0.7237, -0.0445, 1.3303) -- (0.6752, -0.0445, 1.3255) -- cycle;
\fill[blue!15.0, opacity=0.5] (0.6752, -0.0445, 1.3255) -- (0.7237, -0.0445, 1.3303) -- (0.7253, -0.0475, 1.3803) -- (0.6769, -0.0475, 1.3755) -- cycle;
\fill[blue!15.0, opacity=0.5] (0.6769, -0.0475, 1.3755) -- (0.7253, -0.0475, 1.3803) -- (0.7270, -0.0506, 1.4303) -- (0.6787, -0.0506, 1.4255) -- cycle;
\fill[blue!15.1, opacity=0.5] (0.6787, -0.0506, 1.4255) -- (0.7270, -0.0506, 1.4303) -- (0.7287, -0.0537, 1.4803) -- (0.6804, -0.0537, 1.4755) -- cycle;
\fill[blue!15.1, opacity=0.5] (0.6804, -0.0537, 1.4755) -- (0.7287, -0.0537, 1.4803) -- (0.7303, -0.0569, 1.5303) -- (0.6822, -0.0569, 1.5255) -- cycle;
\fill[blue!15.2, opacity=0.5] (0.6822, -0.0569, 1.5255) -- (0.7303, -0.0569, 1.5303) -- (0.7320, -0.0600, 1.5803) -- (0.6840, -0.0600, 1.5755) -- cycle;
\fill[blue!15.2, opacity=0.5] (0.6840, -0.0600, 1.5755) -- (0.7320, -0.0600, 1.5803) -- (0.7337, -0.0631, 1.6303) -- (0.6858, -0.0631, 1.6255) -- cycle;
\fill[blue!15.3, opacity=0.5] (0.6858, -0.0631, 1.6255) -- (0.7337, -0.0631, 1.6303) -- (0.7353, -0.0663, 1.6803) -- (0.6876, -0.0663, 1.6755) -- cycle;
\fill[blue!15.5, opacity=0.5] (0.6876, -0.0663, 1.6755) -- (0.7353, -0.0663, 1.6803) -- (0.7370, -0.0694, 1.7303) -- (0.6893, -0.0694, 1.7255) -- cycle;
\fill[blue!15.6, opacity=0.5] (0.6893, -0.0694, 1.7255) -- (0.7370, -0.0694, 1.7303) -- (0.7387, -0.0725, 1.7803) -- (0.6911, -0.0725, 1.7755) -- cycle;
\fill[blue!15.9, opacity=0.5] (0.6911, -0.0725, 1.7755) -- (0.7387, -0.0725, 1.7803) -- (0.7403, -0.0755, 1.8303) -- (0.6928, -0.0755, 1.8255) -- cycle;
\fill[blue!16.1, opacity=0.5] (0.6928, -0.0755, 1.8255) -- (0.7403, -0.0755, 1.8303) -- (0.7419, -0.0785, 1.8803) -- (0.6945, -0.0785, 1.8755) -- cycle;
\fill[blue!16.5, opacity=0.5] (0.6945, -0.0785, 1.8755) -- (0.7419, -0.0785, 1.8803) -- (0.7435, -0.0815, 1.9303) -- (0.6962, -0.0815, 1.9255) -- cycle;
\fill[blue!16.9, opacity=0.5] (0.6962, -0.0815, 1.9255) -- (0.7435, -0.0815, 1.9303) -- (0.7450, -0.0844, 1.9803) -- (0.6978, -0.0844, 1.9755) -- cycle;
\fill[blue!17.4, opacity=0.5] (0.6978, -0.0844, 1.9755) -- (0.7450, -0.0844, 1.9803) -- (0.7465, -0.0872, 2.0303) -- (0.6994, -0.0872, 2.0255) -- cycle;
\fill[blue!17.9, opacity=0.5] (0.6994, -0.0872, 2.0255) -- (0.7465, -0.0872, 2.0303) -- (0.7480, -0.0900, 2.0803) -- (0.7010, -0.0900, 2.0755) -- cycle;
\fill[blue!18.6, opacity=0.5] (0.7010, -0.0900, 2.0755) -- (0.7480, -0.0900, 2.0803) -- (0.7494, -0.0927, 2.1303) -- (0.7025, -0.0927, 2.1255) -- cycle;
\fill[blue!19.3, opacity=0.5] (0.7025, -0.0927, 2.1255) -- (0.7494, -0.0927, 2.1303) -- (0.7508, -0.0953, 2.1803) -- (0.7040, -0.0953, 2.1755) -- cycle;
\fill[blue!20.1, opacity=0.5] (0.7040, -0.0953, 2.1755) -- (0.7508, -0.0953, 2.1803) -- (0.7521, -0.0978, 2.2303) -- (0.7054, -0.0978, 2.2255) -- cycle;
\fill[blue!21.0, opacity=0.5] (0.7054, -0.0978, 2.2255) -- (0.7521, -0.0978, 2.2303) -- (0.7534, -0.1001, 2.2803) -- (0.7068, -0.1001, 2.2755) -- cycle;
\fill[blue!22.0, opacity=0.5] (0.7068, -0.1001, 2.2755) -- (0.7534, -0.1001, 2.2803) -- (0.7546, -0.1024, 2.3303) -- (0.7080, -0.1024, 2.3255) -- cycle;
\fill[blue!23.0, opacity=0.5] (0.7080, -0.1024, 2.3255) -- (0.7546, -0.1024, 2.3303) -- (0.7558, -0.1046, 2.3803) -- (0.7093, -0.1046, 2.3755) -- cycle;
\fill[blue!24.1, opacity=0.5] (0.7093, -0.1046, 2.3755) -- (0.7558, -0.1046, 2.3803) -- (0.7569, -0.1066, 2.4303) -- (0.7104, -0.1066, 2.4255) -- cycle;
\fill[blue!25.3, opacity=0.5] (0.7104, -0.1066, 2.4255) -- (0.7569, -0.1066, 2.4303) -- (0.7579, -0.1085, 2.4803) -- (0.7115, -0.1085, 2.4755) -- cycle;
\fill[blue!26.5, opacity=0.5] (0.7115, -0.1085, 2.4755) -- (0.7579, -0.1085, 2.4803) -- (0.7588, -0.1103, 2.5303) -- (0.7125, -0.1103, 2.5255) -- cycle;
\fill[blue!27.7, opacity=0.5] (0.7125, -0.1103, 2.5255) -- (0.7588, -0.1103, 2.5303) -- (0.7597, -0.1120, 2.5803) -- (0.7134, -0.1120, 2.5755) -- cycle;
\fill[blue!29.0, opacity=0.5] (0.7134, -0.1120, 2.5755) -- (0.7597, -0.1120, 2.5803) -- (0.7605, -0.1135, 2.6303) -- (0.7143, -0.1135, 2.6255) -- cycle;
\fill[blue!30.2, opacity=0.5] (0.7143, -0.1135, 2.6255) -- (0.7605, -0.1135, 2.6303) -- (0.7612, -0.1148, 2.6803) -- (0.7151, -0.1148, 2.6755) -- cycle;
\fill[blue!31.5, opacity=0.5] (0.7151, -0.1148, 2.6755) -- (0.7612, -0.1148, 2.6803) -- (0.7619, -0.1160, 2.7303) -- (0.7157, -0.1160, 2.7255) -- cycle;
\fill[blue!32.8, opacity=0.5] (0.7157, -0.1160, 2.7255) -- (0.7619, -0.1160, 2.7303) -- (0.7624, -0.1171, 2.7803) -- (0.7163, -0.1171, 2.7755) -- cycle;
\fill[blue!34.0, opacity=0.5] (0.7163, -0.1171, 2.7755) -- (0.7624, -0.1171, 2.7803) -- (0.7629, -0.1180, 2.8303) -- (0.7168, -0.1180, 2.8255) -- cycle;
\fill[blue!35.2, opacity=0.5] (0.7168, -0.1180, 2.8255) -- (0.7629, -0.1180, 2.8303) -- (0.7633, -0.1187, 2.8803) -- (0.7173, -0.1187, 2.8755) -- cycle;
\fill[blue!36.4, opacity=0.5] (0.7173, -0.1187, 2.8755) -- (0.7633, -0.1187, 2.8803) -- (0.7636, -0.1193, 2.9303) -- (0.7176, -0.1193, 2.9255) -- cycle;
\fill[blue!37.5, opacity=0.5] (0.7176, -0.1193, 2.9255) -- (0.7636, -0.1193, 2.9303) -- (0.7638, -0.1197, 2.9803) -- (0.7178, -0.1197, 2.9755) -- cycle;
\fill[blue!38.5, opacity=0.5] (0.7178, -0.1197, 2.9755) -- (0.7638, -0.1197, 2.9803) -- (0.7640, -0.1199, 3.0303) -- (0.7180, -0.1199, 3.0255) -- cycle;
\fill[blue!39.5, opacity=0.5] (0.7180, -0.1199, 3.0255) -- (0.7640, -0.1199, 3.0303) -- (0.7640, -0.1200, 3.0803) -- (0.7180, -0.1200, 3.0755) -- cycle;
\fill[blue!15.0, opacity=0.5] (0.7000, -0.0000, 0.0803) -- (0.7500, -0.0000, 0.0849) -- (0.7500, -0.0001, 0.1349) -- (0.7000, -0.0001, 0.1303) -- cycle;
\fill[blue!15.0, opacity=0.5] (0.7000, -0.0001, 0.1303) -- (0.7500, -0.0001, 0.1349) -- (0.7502, -0.0003, 0.1849) -- (0.7002, -0.0003, 0.1803) -- cycle;
\fill[blue!15.0, opacity=0.5] (0.7002, -0.0003, 0.1803) -- (0.7502, -0.0003, 0.1849) -- (0.7504, -0.0007, 0.2349) -- (0.7004, -0.0007, 0.2303) -- cycle;
\fill[blue!15.0, opacity=0.5] (0.7004, -0.0007, 0.2303) -- (0.7504, -0.0007, 0.2349) -- (0.7507, -0.0013, 0.2849) -- (0.7007, -0.0013, 0.2803) -- cycle;
\fill[blue!15.0, opacity=0.5] (0.7007, -0.0013, 0.2803) -- (0.7507, -0.0013, 0.2849) -- (0.7510, -0.0020, 0.3349) -- (0.7011, -0.0020, 0.3303) -- cycle;
\fill[blue!15.0, opacity=0.5] (0.7011, -0.0020, 0.3303) -- (0.7510, -0.0020, 0.3349) -- (0.7515, -0.0029, 0.3849) -- (0.7016, -0.0029, 0.3803) -- cycle;
\fill[blue!15.0, opacity=0.5] (0.7016, -0.0029, 0.3803) -- (0.7515, -0.0029, 0.3849) -- (0.7520, -0.0040, 0.4349) -- (0.7021, -0.0040, 0.4303) -- cycle;
\fill[blue!15.0, opacity=0.5] (0.7021, -0.0040, 0.4303) -- (0.7520, -0.0040, 0.4349) -- (0.7526, -0.0052, 0.4849) -- (0.7028, -0.0052, 0.4803) -- cycle;
\fill[blue!15.0, opacity=0.5] (0.7028, -0.0052, 0.4803) -- (0.7526, -0.0052, 0.4849) -- (0.7533, -0.0065, 0.5349) -- (0.7035, -0.0065, 0.5303) -- cycle;
\fill[blue!15.0, opacity=0.5] (0.7035, -0.0065, 0.5303) -- (0.7533, -0.0065, 0.5349) -- (0.7540, -0.0080, 0.5849) -- (0.7043, -0.0080, 0.5803) -- cycle;
\fill[blue!15.0, opacity=0.5] (0.7043, -0.0080, 0.5803) -- (0.7540, -0.0080, 0.5849) -- (0.7548, -0.0097, 0.6349) -- (0.7052, -0.0097, 0.6303) -- cycle;
\fill[blue!15.0, opacity=0.5] (0.7052, -0.0097, 0.6303) -- (0.7548, -0.0097, 0.6349) -- (0.7557, -0.0115, 0.6849) -- (0.7061, -0.0115, 0.6803) -- cycle;
\fill[blue!15.0, opacity=0.5] (0.7061, -0.0115, 0.6803) -- (0.7557, -0.0115, 0.6849) -- (0.7567, -0.0134, 0.7349) -- (0.7071, -0.0134, 0.7303) -- cycle;
\fill[blue!15.0, opacity=0.5] (0.7071, -0.0134, 0.7303) -- (0.7567, -0.0134, 0.7349) -- (0.7577, -0.0154, 0.7849) -- (0.7082, -0.0154, 0.7803) -- cycle;
\fill[blue!15.0, opacity=0.5] (0.7082, -0.0154, 0.7803) -- (0.7577, -0.0154, 0.7849) -- (0.7588, -0.0176, 0.8349) -- (0.7094, -0.0176, 0.8303) -- cycle;
\fill[blue!15.0, opacity=0.5] (0.7094, -0.0176, 0.8303) -- (0.7588, -0.0176, 0.8349) -- (0.7599, -0.0199, 0.8849) -- (0.7106, -0.0199, 0.8803) -- cycle;
\fill[blue!15.0, opacity=0.5] (0.7106, -0.0199, 0.8803) -- (0.7599, -0.0199, 0.8849) -- (0.7611, -0.0222, 0.9349) -- (0.7119, -0.0222, 0.9303) -- cycle;
\fill[blue!15.0, opacity=0.5] (0.7119, -0.0222, 0.9303) -- (0.7611, -0.0222, 0.9349) -- (0.7624, -0.0247, 0.9849) -- (0.7132, -0.0247, 0.9803) -- cycle;
\fill[blue!15.0, opacity=0.5] (0.7132, -0.0247, 0.9803) -- (0.7624, -0.0247, 0.9849) -- (0.7637, -0.0273, 1.0349) -- (0.7146, -0.0273, 1.0303) -- cycle;
\fill[blue!15.0, opacity=0.5] (0.7146, -0.0273, 1.0303) -- (0.7637, -0.0273, 1.0349) -- (0.7650, -0.0300, 1.0849) -- (0.7160, -0.0300, 1.0803) -- cycle;
\fill[blue!15.0, opacity=0.5] (0.7160, -0.0300, 1.0803) -- (0.7650, -0.0300, 1.0849) -- (0.7664, -0.0328, 1.1349) -- (0.7175, -0.0328, 1.1303) -- cycle;
\fill[blue!15.0, opacity=0.5] (0.7175, -0.0328, 1.1303) -- (0.7664, -0.0328, 1.1349) -- (0.7678, -0.0356, 1.1849) -- (0.7190, -0.0356, 1.1803) -- cycle;
\fill[blue!15.0, opacity=0.5] (0.7190, -0.0356, 1.1803) -- (0.7678, -0.0356, 1.1849) -- (0.7692, -0.0385, 1.2349) -- (0.7205, -0.0385, 1.2303) -- cycle;
\fill[blue!15.1, opacity=0.5] (0.7205, -0.0385, 1.2303) -- (0.7692, -0.0385, 1.2349) -- (0.7707, -0.0415, 1.2849) -- (0.7221, -0.0415, 1.2803) -- cycle;
\fill[blue!15.1, opacity=0.5] (0.7221, -0.0415, 1.2803) -- (0.7707, -0.0415, 1.2849) -- (0.7722, -0.0445, 1.3349) -- (0.7237, -0.0445, 1.3303) -- cycle;
\fill[blue!15.2, opacity=0.5] (0.7237, -0.0445, 1.3303) -- (0.7722, -0.0445, 1.3349) -- (0.7738, -0.0475, 1.3849) -- (0.7253, -0.0475, 1.3803) -- cycle;
\fill[blue!15.3, opacity=0.5] (0.7253, -0.0475, 1.3803) -- (0.7738, -0.0475, 1.3849) -- (0.7753, -0.0506, 1.4349) -- (0.7270, -0.0506, 1.4303) -- cycle;
\fill[blue!15.4, opacity=0.5] (0.7270, -0.0506, 1.4303) -- (0.7753, -0.0506, 1.4349) -- (0.7769, -0.0537, 1.4849) -- (0.7287, -0.0537, 1.4803) -- cycle;
\fill[blue!15.6, opacity=0.5] (0.7287, -0.0537, 1.4803) -- (0.7769, -0.0537, 1.4849) -- (0.7784, -0.0569, 1.5349) -- (0.7303, -0.0569, 1.5303) -- cycle;
\fill[blue!15.8, opacity=0.5] (0.7303, -0.0569, 1.5303) -- (0.7784, -0.0569, 1.5349) -- (0.7800, -0.0600, 1.5849) -- (0.7320, -0.0600, 1.5803) -- cycle;
\fill[blue!16.2, opacity=0.5] (0.7320, -0.0600, 1.5803) -- (0.7800, -0.0600, 1.5849) -- (0.7816, -0.0631, 1.6349) -- (0.7337, -0.0631, 1.6303) -- cycle;
\fill[blue!16.5, opacity=0.5] (0.7337, -0.0631, 1.6303) -- (0.7816, -0.0631, 1.6349) -- (0.7831, -0.0663, 1.6849) -- (0.7353, -0.0663, 1.6803) -- cycle;
\fill[blue!17.0, opacity=0.5] (0.7353, -0.0663, 1.6803) -- (0.7831, -0.0663, 1.6849) -- (0.7847, -0.0694, 1.7349) -- (0.7370, -0.0694, 1.7303) -- cycle;
\fill[blue!17.6, opacity=0.5] (0.7370, -0.0694, 1.7303) -- (0.7847, -0.0694, 1.7349) -- (0.7862, -0.0725, 1.7849) -- (0.7387, -0.0725, 1.7803) -- cycle;
\fill[blue!18.3, opacity=0.5] (0.7387, -0.0725, 1.7803) -- (0.7862, -0.0725, 1.7849) -- (0.7878, -0.0755, 1.8349) -- (0.7403, -0.0755, 1.8303) -- cycle;
\fill[blue!19.0, opacity=0.5] (0.7403, -0.0755, 1.8303) -- (0.7878, -0.0755, 1.8349) -- (0.7893, -0.0785, 1.8849) -- (0.7419, -0.0785, 1.8803) -- cycle;
\fill[blue!19.9, opacity=0.5] (0.7419, -0.0785, 1.8803) -- (0.7893, -0.0785, 1.8849) -- (0.7908, -0.0815, 1.9349) -- (0.7435, -0.0815, 1.9303) -- cycle;
\fill[blue!20.9, opacity=0.5] (0.7435, -0.0815, 1.9303) -- (0.7908, -0.0815, 1.9349) -- (0.7922, -0.0844, 1.9849) -- (0.7450, -0.0844, 1.9803) -- cycle;
\fill[blue!22.0, opacity=0.5] (0.7450, -0.0844, 1.9803) -- (0.7922, -0.0844, 1.9849) -- (0.7936, -0.0872, 2.0349) -- (0.7465, -0.0872, 2.0303) -- cycle;
\fill[blue!23.2, opacity=0.5] (0.7465, -0.0872, 2.0303) -- (0.7936, -0.0872, 2.0349) -- (0.7950, -0.0900, 2.0849) -- (0.7480, -0.0900, 2.0803) -- cycle;
\fill[blue!24.5, opacity=0.5] (0.7480, -0.0900, 2.0803) -- (0.7950, -0.0900, 2.0849) -- (0.7963, -0.0927, 2.1349) -- (0.7494, -0.0927, 2.1303) -- cycle;
\fill[blue!25.9, opacity=0.5] (0.7494, -0.0927, 2.1303) -- (0.7963, -0.0927, 2.1349) -- (0.7976, -0.0953, 2.1849) -- (0.7508, -0.0953, 2.1803) -- cycle;
\fill[blue!27.4, opacity=0.5] (0.7508, -0.0953, 2.1803) -- (0.7976, -0.0953, 2.1849) -- (0.7989, -0.0978, 2.2349) -- (0.7521, -0.0978, 2.2303) -- cycle;
\fill[blue!28.9, opacity=0.5] (0.7521, -0.0978, 2.2303) -- (0.7989, -0.0978, 2.2349) -- (0.8001, -0.1001, 2.2849) -- (0.7534, -0.1001, 2.2803) -- cycle;
\fill[blue!30.5, opacity=0.5] (0.7534, -0.1001, 2.2803) -- (0.8001, -0.1001, 2.2849) -- (0.8012, -0.1024, 2.3349) -- (0.7546, -0.1024, 2.3303) -- cycle;
\fill[blue!32.1, opacity=0.5] (0.7546, -0.1024, 2.3303) -- (0.8012, -0.1024, 2.3349) -- (0.8023, -0.1046, 2.3849) -- (0.7558, -0.1046, 2.3803) -- cycle;
\fill[blue!33.7, opacity=0.5] (0.7558, -0.1046, 2.3803) -- (0.8023, -0.1046, 2.3849) -- (0.8033, -0.1066, 2.4349) -- (0.7569, -0.1066, 2.4303) -- cycle;
\fill[blue!35.4, opacity=0.5] (0.7569, -0.1066, 2.4303) -- (0.8033, -0.1066, 2.4349) -- (0.8043, -0.1085, 2.4849) -- (0.7579, -0.1085, 2.4803) -- cycle;
\fill[blue!37.0, opacity=0.5] (0.7579, -0.1085, 2.4803) -- (0.8043, -0.1085, 2.4849) -- (0.8052, -0.1103, 2.5349) -- (0.7588, -0.1103, 2.5303) -- cycle;
\fill[blue!38.5, opacity=0.5] (0.7588, -0.1103, 2.5303) -- (0.8052, -0.1103, 2.5349) -- (0.8060, -0.1120, 2.5849) -- (0.7597, -0.1120, 2.5803) -- cycle;
\fill[blue!40.1, opacity=0.5] (0.7597, -0.1120, 2.5803) -- (0.8060, -0.1120, 2.5849) -- (0.8067, -0.1135, 2.6349) -- (0.7605, -0.1135, 2.6303) -- cycle;
\fill[blue!41.5, opacity=0.5] (0.7605, -0.1135, 2.6303) -- (0.8067, -0.1135, 2.6349) -- (0.8074, -0.1148, 2.6849) -- (0.7612, -0.1148, 2.6803) -- cycle;
\fill[blue!42.9, opacity=0.5] (0.7612, -0.1148, 2.6803) -- (0.8074, -0.1148, 2.6849) -- (0.8080, -0.1160, 2.7349) -- (0.7619, -0.1160, 2.7303) -- cycle;
\fill[blue!44.2, opacity=0.5] (0.7619, -0.1160, 2.7303) -- (0.8080, -0.1160, 2.7349) -- (0.8085, -0.1171, 2.7849) -- (0.7624, -0.1171, 2.7803) -- cycle;
\fill[blue!45.4, opacity=0.5] (0.7624, -0.1171, 2.7803) -- (0.8085, -0.1171, 2.7849) -- (0.8090, -0.1180, 2.8349) -- (0.7629, -0.1180, 2.8303) -- cycle;
\fill[blue!46.5, opacity=0.5] (0.7629, -0.1180, 2.8303) -- (0.8090, -0.1180, 2.8349) -- (0.8093, -0.1187, 2.8849) -- (0.7633, -0.1187, 2.8803) -- cycle;
\fill[blue!47.5, opacity=0.5] (0.7633, -0.1187, 2.8803) -- (0.8093, -0.1187, 2.8849) -- (0.8096, -0.1193, 2.9349) -- (0.7636, -0.1193, 2.9303) -- cycle;
\fill[blue!48.4, opacity=0.5] (0.7636, -0.1193, 2.9303) -- (0.8096, -0.1193, 2.9349) -- (0.8098, -0.1197, 2.9849) -- (0.7638, -0.1197, 2.9803) -- cycle;
\fill[blue!49.1, opacity=0.5] (0.7638, -0.1197, 2.9803) -- (0.8098, -0.1197, 2.9849) -- (0.8100, -0.1199, 3.0349) -- (0.7640, -0.1199, 3.0303) -- cycle;
\fill[blue!49.7, opacity=0.5] (0.7640, -0.1199, 3.0303) -- (0.8100, -0.1199, 3.0349) -- (0.8100, -0.1200, 3.0849) -- (0.7640, -0.1200, 3.0803) -- cycle;
\fill[blue!15.0, opacity=0.5] (0.7500, -0.0000, 0.0849) -- (0.8000, -0.0000, 0.0892) -- (0.8000, -0.0001, 0.1392) -- (0.7500, -0.0001, 0.1349) -- cycle;
\fill[blue!15.0, opacity=0.5] (0.7500, -0.0001, 0.1349) -- (0.8000, -0.0001, 0.1392) -- (0.8002, -0.0003, 0.1892) -- (0.7502, -0.0003, 0.1849) -- cycle;
\fill[blue!15.0, opacity=0.5] (0.7502, -0.0003, 0.1849) -- (0.8002, -0.0003, 0.1892) -- (0.8003, -0.0007, 0.2392) -- (0.7504, -0.0007, 0.2349) -- cycle;
\fill[blue!15.0, opacity=0.5] (0.7504, -0.0007, 0.2349) -- (0.8003, -0.0007, 0.2392) -- (0.8006, -0.0013, 0.2892) -- (0.7507, -0.0013, 0.2849) -- cycle;
\fill[blue!15.0, opacity=0.5] (0.7507, -0.0013, 0.2849) -- (0.8006, -0.0013, 0.2892) -- (0.8010, -0.0020, 0.3392) -- (0.7510, -0.0020, 0.3349) -- cycle;
\fill[blue!15.0, opacity=0.5] (0.7510, -0.0020, 0.3349) -- (0.8010, -0.0020, 0.3392) -- (0.8014, -0.0029, 0.3892) -- (0.7515, -0.0029, 0.3849) -- cycle;
\fill[blue!15.0, opacity=0.5] (0.7515, -0.0029, 0.3849) -- (0.8014, -0.0029, 0.3892) -- (0.8019, -0.0040, 0.4392) -- (0.7520, -0.0040, 0.4349) -- cycle;
\fill[blue!15.0, opacity=0.5] (0.7520, -0.0040, 0.4349) -- (0.8019, -0.0040, 0.4392) -- (0.8024, -0.0052, 0.4892) -- (0.7526, -0.0052, 0.4849) -- cycle;
\fill[blue!15.0, opacity=0.5] (0.7526, -0.0052, 0.4849) -- (0.8024, -0.0052, 0.4892) -- (0.8031, -0.0065, 0.5392) -- (0.7533, -0.0065, 0.5349) -- cycle;
\fill[blue!15.0, opacity=0.5] (0.7533, -0.0065, 0.5349) -- (0.8031, -0.0065, 0.5392) -- (0.8038, -0.0080, 0.5892) -- (0.7540, -0.0080, 0.5849) -- cycle;
\fill[blue!15.0, opacity=0.5] (0.7540, -0.0080, 0.5849) -- (0.8038, -0.0080, 0.5892) -- (0.8045, -0.0097, 0.6392) -- (0.7548, -0.0097, 0.6349) -- cycle;
\fill[blue!15.0, opacity=0.5] (0.7548, -0.0097, 0.6349) -- (0.8045, -0.0097, 0.6392) -- (0.8053, -0.0115, 0.6892) -- (0.7557, -0.0115, 0.6849) -- cycle;
\fill[blue!15.0, opacity=0.5] (0.7557, -0.0115, 0.6849) -- (0.8053, -0.0115, 0.6892) -- (0.8062, -0.0134, 0.7392) -- (0.7567, -0.0134, 0.7349) -- cycle;
\fill[blue!15.0, opacity=0.5] (0.7567, -0.0134, 0.7349) -- (0.8062, -0.0134, 0.7392) -- (0.8072, -0.0154, 0.7892) -- (0.7577, -0.0154, 0.7849) -- cycle;
\fill[blue!15.0, opacity=0.5] (0.7577, -0.0154, 0.7849) -- (0.8072, -0.0154, 0.7892) -- (0.8082, -0.0176, 0.8392) -- (0.7588, -0.0176, 0.8349) -- cycle;
\fill[blue!15.0, opacity=0.5] (0.7588, -0.0176, 0.8349) -- (0.8082, -0.0176, 0.8392) -- (0.8093, -0.0199, 0.8892) -- (0.7599, -0.0199, 0.8849) -- cycle;
\fill[blue!15.0, opacity=0.5] (0.7599, -0.0199, 0.8849) -- (0.8093, -0.0199, 0.8892) -- (0.8104, -0.0222, 0.9392) -- (0.7611, -0.0222, 0.9349) -- cycle;
\fill[blue!15.0, opacity=0.5] (0.7611, -0.0222, 0.9349) -- (0.8104, -0.0222, 0.9392) -- (0.8115, -0.0247, 0.9892) -- (0.7624, -0.0247, 0.9849) -- cycle;
\fill[blue!15.0, opacity=0.5] (0.7624, -0.0247, 0.9849) -- (0.8115, -0.0247, 0.9892) -- (0.8128, -0.0273, 1.0392) -- (0.7637, -0.0273, 1.0349) -- cycle;
\fill[blue!15.0, opacity=0.5] (0.7637, -0.0273, 1.0349) -- (0.8128, -0.0273, 1.0392) -- (0.8140, -0.0300, 1.0892) -- (0.7650, -0.0300, 1.0849) -- cycle;
\fill[blue!15.1, opacity=0.5] (0.7650, -0.0300, 1.0849) -- (0.8140, -0.0300, 1.0892) -- (0.8153, -0.0328, 1.1392) -- (0.7664, -0.0328, 1.1349) -- cycle;
\fill[blue!15.1, opacity=0.5] (0.7664, -0.0328, 1.1349) -- (0.8153, -0.0328, 1.1392) -- (0.8166, -0.0356, 1.1892) -- (0.7678, -0.0356, 1.1849) -- cycle;
\fill[blue!15.2, opacity=0.5] (0.7678, -0.0356, 1.1849) -- (0.8166, -0.0356, 1.1892) -- (0.8180, -0.0385, 1.2392) -- (0.7692, -0.0385, 1.2349) -- cycle;
\fill[blue!15.4, opacity=0.5] (0.7692, -0.0385, 1.2349) -- (0.8180, -0.0385, 1.2392) -- (0.8193, -0.0415, 1.2892) -- (0.7707, -0.0415, 1.2849) -- cycle;
\fill[blue!15.5, opacity=0.5] (0.7707, -0.0415, 1.2849) -- (0.8193, -0.0415, 1.2892) -- (0.8208, -0.0445, 1.3392) -- (0.7722, -0.0445, 1.3349) -- cycle;
\fill[blue!15.8, opacity=0.5] (0.7722, -0.0445, 1.3349) -- (0.8208, -0.0445, 1.3392) -- (0.8222, -0.0475, 1.3892) -- (0.7738, -0.0475, 1.3849) -- cycle;
\fill[blue!16.1, opacity=0.5] (0.7738, -0.0475, 1.3849) -- (0.8222, -0.0475, 1.3892) -- (0.8236, -0.0506, 1.4392) -- (0.7753, -0.0506, 1.4349) -- cycle;
\fill[blue!16.5, opacity=0.5] (0.7753, -0.0506, 1.4349) -- (0.8236, -0.0506, 1.4392) -- (0.8251, -0.0537, 1.4892) -- (0.7769, -0.0537, 1.4849) -- cycle;
\fill[blue!17.0, opacity=0.5] (0.7769, -0.0537, 1.4849) -- (0.8251, -0.0537, 1.4892) -- (0.8265, -0.0569, 1.5392) -- (0.7784, -0.0569, 1.5349) -- cycle;
\fill[blue!17.6, opacity=0.5] (0.7784, -0.0569, 1.5349) -- (0.8265, -0.0569, 1.5392) -- (0.8280, -0.0600, 1.5892) -- (0.7800, -0.0600, 1.5849) -- cycle;
\fill[blue!18.3, opacity=0.5] (0.7800, -0.0600, 1.5849) -- (0.8280, -0.0600, 1.5892) -- (0.8295, -0.0631, 1.6392) -- (0.7816, -0.0631, 1.6349) -- cycle;
\fill[blue!19.2, opacity=0.5] (0.7816, -0.0631, 1.6349) -- (0.8295, -0.0631, 1.6392) -- (0.8309, -0.0663, 1.6892) -- (0.7831, -0.0663, 1.6849) -- cycle;
\fill[blue!20.2, opacity=0.5] (0.7831, -0.0663, 1.6849) -- (0.8309, -0.0663, 1.6892) -- (0.8324, -0.0694, 1.7392) -- (0.7847, -0.0694, 1.7349) -- cycle;
\fill[blue!21.3, opacity=0.5] (0.7847, -0.0694, 1.7349) -- (0.8324, -0.0694, 1.7392) -- (0.8338, -0.0725, 1.7892) -- (0.7862, -0.0725, 1.7849) -- cycle;
\fill[blue!22.6, opacity=0.5] (0.7862, -0.0725, 1.7849) -- (0.8338, -0.0725, 1.7892) -- (0.8352, -0.0755, 1.8392) -- (0.7878, -0.0755, 1.8349) -- cycle;
\fill[blue!24.0, opacity=0.5] (0.7878, -0.0755, 1.8349) -- (0.8352, -0.0755, 1.8392) -- (0.8367, -0.0785, 1.8892) -- (0.7893, -0.0785, 1.8849) -- cycle;
\fill[blue!25.5, opacity=0.5] (0.7893, -0.0785, 1.8849) -- (0.8367, -0.0785, 1.8892) -- (0.8380, -0.0815, 1.9392) -- (0.7908, -0.0815, 1.9349) -- cycle;
\fill[blue!27.1, opacity=0.5] (0.7908, -0.0815, 1.9349) -- (0.8380, -0.0815, 1.9392) -- (0.8394, -0.0844, 1.9892) -- (0.7922, -0.0844, 1.9849) -- cycle;
\fill[blue!28.9, opacity=0.5] (0.7922, -0.0844, 1.9849) -- (0.8394, -0.0844, 1.9892) -- (0.8407, -0.0872, 2.0392) -- (0.7936, -0.0872, 2.0349) -- cycle;
\fill[blue!30.7, opacity=0.5] (0.7936, -0.0872, 2.0349) -- (0.8407, -0.0872, 2.0392) -- (0.8420, -0.0900, 2.0892) -- (0.7950, -0.0900, 2.0849) -- cycle;
\fill[blue!32.6, opacity=0.5] (0.7950, -0.0900, 2.0849) -- (0.8420, -0.0900, 2.0892) -- (0.8432, -0.0927, 2.1392) -- (0.7963, -0.0927, 2.1349) -- cycle;
\fill[blue!34.5, opacity=0.5] (0.7963, -0.0927, 2.1349) -- (0.8432, -0.0927, 2.1392) -- (0.8445, -0.0953, 2.1892) -- (0.7976, -0.0953, 2.1849) -- cycle;
\fill[blue!36.4, opacity=0.5] (0.7976, -0.0953, 2.1849) -- (0.8445, -0.0953, 2.1892) -- (0.8456, -0.0978, 2.2392) -- (0.7989, -0.0978, 2.2349) -- cycle;
\fill[blue!38.3, opacity=0.5] (0.7989, -0.0978, 2.2349) -- (0.8456, -0.0978, 2.2392) -- (0.8467, -0.1001, 2.2892) -- (0.8001, -0.1001, 2.2849) -- cycle;
\fill[blue!40.2, opacity=0.5] (0.8001, -0.1001, 2.2849) -- (0.8467, -0.1001, 2.2892) -- (0.8478, -0.1024, 2.3392) -- (0.8012, -0.1024, 2.3349) -- cycle;
\fill[blue!42.1, opacity=0.5] (0.8012, -0.1024, 2.3349) -- (0.8478, -0.1024, 2.3392) -- (0.8488, -0.1046, 2.3892) -- (0.8023, -0.1046, 2.3849) -- cycle;
\fill[blue!43.9, opacity=0.5] (0.8023, -0.1046, 2.3849) -- (0.8488, -0.1046, 2.3892) -- (0.8498, -0.1066, 2.4392) -- (0.8033, -0.1066, 2.4349) -- cycle;
\fill[blue!45.6, opacity=0.5] (0.8033, -0.1066, 2.4349) -- (0.8498, -0.1066, 2.4392) -- (0.8507, -0.1085, 2.4892) -- (0.8043, -0.1085, 2.4849) -- cycle;
\fill[blue!47.3, opacity=0.5] (0.8043, -0.1085, 2.4849) -- (0.8507, -0.1085, 2.4892) -- (0.8515, -0.1103, 2.5392) -- (0.8052, -0.1103, 2.5349) -- cycle;
\fill[blue!48.8, opacity=0.5] (0.8052, -0.1103, 2.5349) -- (0.8515, -0.1103, 2.5392) -- (0.8522, -0.1120, 2.5892) -- (0.8060, -0.1120, 2.5849) -- cycle;
\fill[blue!50.2, opacity=0.5] (0.8060, -0.1120, 2.5849) -- (0.8522, -0.1120, 2.5892) -- (0.8529, -0.1135, 2.6392) -- (0.8067, -0.1135, 2.6349) -- cycle;
\fill[blue!51.5, opacity=0.5] (0.8067, -0.1135, 2.6349) -- (0.8529, -0.1135, 2.6392) -- (0.8536, -0.1148, 2.6892) -- (0.8074, -0.1148, 2.6849) -- cycle;
\fill[blue!52.7, opacity=0.5] (0.8074, -0.1148, 2.6849) -- (0.8536, -0.1148, 2.6892) -- (0.8541, -0.1160, 2.7392) -- (0.8080, -0.1160, 2.7349) -- cycle;
\fill[blue!53.7, opacity=0.5] (0.8080, -0.1160, 2.7349) -- (0.8541, -0.1160, 2.7392) -- (0.8546, -0.1171, 2.7892) -- (0.8085, -0.1171, 2.7849) -- cycle;
\fill[blue!54.5, opacity=0.5] (0.8085, -0.1171, 2.7849) -- (0.8546, -0.1171, 2.7892) -- (0.8550, -0.1180, 2.8392) -- (0.8090, -0.1180, 2.8349) -- cycle;
\fill[blue!55.3, opacity=0.5] (0.8090, -0.1180, 2.8349) -- (0.8550, -0.1180, 2.8392) -- (0.8554, -0.1187, 2.8892) -- (0.8093, -0.1187, 2.8849) -- cycle;
\fill[blue!55.8, opacity=0.5] (0.8093, -0.1187, 2.8849) -- (0.8554, -0.1187, 2.8892) -- (0.8557, -0.1193, 2.9392) -- (0.8096, -0.1193, 2.9349) -- cycle;
\fill[blue!56.3, opacity=0.5] (0.8096, -0.1193, 2.9349) -- (0.8557, -0.1193, 2.9392) -- (0.8558, -0.1197, 2.9892) -- (0.8098, -0.1197, 2.9849) -- cycle;
\fill[blue!56.6, opacity=0.5] (0.8098, -0.1197, 2.9849) -- (0.8558, -0.1197, 2.9892) -- (0.8560, -0.1199, 3.0392) -- (0.8100, -0.1199, 3.0349) -- cycle;
\fill[blue!56.7, opacity=0.5] (0.8100, -0.1199, 3.0349) -- (0.8560, -0.1199, 3.0392) -- (0.8560, -0.1200, 3.0892) -- (0.8100, -0.1200, 3.0849) -- cycle;
\fill[blue!15.0, opacity=0.5] (0.8000, -0.0000, 0.0892) -- (0.8500, -0.0000, 0.0933) -- (0.8500, -0.0001, 0.1433) -- (0.8000, -0.0001, 0.1392) -- cycle;
\fill[blue!15.0, opacity=0.5] (0.8000, -0.0001, 0.1392) -- (0.8500, -0.0001, 0.1433) -- (0.8501, -0.0003, 0.1933) -- (0.8002, -0.0003, 0.1892) -- cycle;
\fill[blue!15.0, opacity=0.5] (0.8002, -0.0003, 0.1892) -- (0.8501, -0.0003, 0.1933) -- (0.8503, -0.0007, 0.2433) -- (0.8003, -0.0007, 0.2392) -- cycle;
\fill[blue!15.0, opacity=0.5] (0.8003, -0.0007, 0.2392) -- (0.8503, -0.0007, 0.2433) -- (0.8506, -0.0013, 0.2933) -- (0.8006, -0.0013, 0.2892) -- cycle;
\fill[blue!15.0, opacity=0.5] (0.8006, -0.0013, 0.2892) -- (0.8506, -0.0013, 0.2933) -- (0.8509, -0.0020, 0.3433) -- (0.8010, -0.0020, 0.3392) -- cycle;
\fill[blue!15.0, opacity=0.5] (0.8010, -0.0020, 0.3392) -- (0.8509, -0.0020, 0.3433) -- (0.8513, -0.0029, 0.3933) -- (0.8014, -0.0029, 0.3892) -- cycle;
\fill[blue!15.0, opacity=0.5] (0.8014, -0.0029, 0.3892) -- (0.8513, -0.0029, 0.3933) -- (0.8517, -0.0040, 0.4433) -- (0.8019, -0.0040, 0.4392) -- cycle;
\fill[blue!15.0, opacity=0.5] (0.8019, -0.0040, 0.4392) -- (0.8517, -0.0040, 0.4433) -- (0.8522, -0.0052, 0.4933) -- (0.8024, -0.0052, 0.4892) -- cycle;
\fill[blue!15.0, opacity=0.5] (0.8024, -0.0052, 0.4892) -- (0.8522, -0.0052, 0.4933) -- (0.8528, -0.0065, 0.5433) -- (0.8031, -0.0065, 0.5392) -- cycle;
\fill[blue!15.0, opacity=0.5] (0.8031, -0.0065, 0.5392) -- (0.8528, -0.0065, 0.5433) -- (0.8535, -0.0080, 0.5933) -- (0.8038, -0.0080, 0.5892) -- cycle;
\fill[blue!15.0, opacity=0.5] (0.8038, -0.0080, 0.5892) -- (0.8535, -0.0080, 0.5933) -- (0.8542, -0.0097, 0.6433) -- (0.8045, -0.0097, 0.6392) -- cycle;
\fill[blue!15.0, opacity=0.5] (0.8045, -0.0097, 0.6392) -- (0.8542, -0.0097, 0.6433) -- (0.8550, -0.0115, 0.6933) -- (0.8053, -0.0115, 0.6892) -- cycle;
\fill[blue!15.0, opacity=0.5] (0.8053, -0.0115, 0.6892) -- (0.8550, -0.0115, 0.6933) -- (0.8558, -0.0134, 0.7433) -- (0.8062, -0.0134, 0.7392) -- cycle;
\fill[blue!15.0, opacity=0.5] (0.8062, -0.0134, 0.7392) -- (0.8558, -0.0134, 0.7433) -- (0.8567, -0.0154, 0.7933) -- (0.8072, -0.0154, 0.7892) -- cycle;
\fill[blue!15.0, opacity=0.5] (0.8072, -0.0154, 0.7892) -- (0.8567, -0.0154, 0.7933) -- (0.8576, -0.0176, 0.8433) -- (0.8082, -0.0176, 0.8392) -- cycle;
\fill[blue!15.0, opacity=0.5] (0.8082, -0.0176, 0.8392) -- (0.8576, -0.0176, 0.8433) -- (0.8586, -0.0199, 0.8933) -- (0.8093, -0.0199, 0.8892) -- cycle;
\fill[blue!15.0, opacity=0.5] (0.8093, -0.0199, 0.8892) -- (0.8586, -0.0199, 0.8933) -- (0.8596, -0.0222, 0.9433) -- (0.8104, -0.0222, 0.9392) -- cycle;
\fill[blue!15.1, opacity=0.5] (0.8104, -0.0222, 0.9392) -- (0.8596, -0.0222, 0.9433) -- (0.8607, -0.0247, 0.9933) -- (0.8115, -0.0247, 0.9892) -- cycle;
\fill[blue!15.1, opacity=0.5] (0.8115, -0.0247, 0.9892) -- (0.8607, -0.0247, 0.9933) -- (0.8618, -0.0273, 1.0433) -- (0.8128, -0.0273, 1.0392) -- cycle;
\fill[blue!15.2, opacity=0.5] (0.8128, -0.0273, 1.0392) -- (0.8618, -0.0273, 1.0433) -- (0.8630, -0.0300, 1.0933) -- (0.8140, -0.0300, 1.0892) -- cycle;
\fill[blue!15.3, opacity=0.5] (0.8140, -0.0300, 1.0892) -- (0.8630, -0.0300, 1.0933) -- (0.8642, -0.0328, 1.1433) -- (0.8153, -0.0328, 1.1392) -- cycle;
\fill[blue!15.4, opacity=0.5] (0.8153, -0.0328, 1.1392) -- (0.8642, -0.0328, 1.1433) -- (0.8654, -0.0356, 1.1933) -- (0.8166, -0.0356, 1.1892) -- cycle;
\fill[blue!15.6, opacity=0.5] (0.8166, -0.0356, 1.1892) -- (0.8654, -0.0356, 1.1933) -- (0.8667, -0.0385, 1.2433) -- (0.8180, -0.0385, 1.2392) -- cycle;
\fill[blue!15.9, opacity=0.5] (0.8180, -0.0385, 1.2392) -- (0.8667, -0.0385, 1.2433) -- (0.8680, -0.0415, 1.2933) -- (0.8193, -0.0415, 1.2892) -- cycle;
\fill[blue!16.2, opacity=0.5] (0.8193, -0.0415, 1.2892) -- (0.8680, -0.0415, 1.2933) -- (0.8693, -0.0445, 1.3433) -- (0.8208, -0.0445, 1.3392) -- cycle;
\fill[blue!16.7, opacity=0.5] (0.8208, -0.0445, 1.3392) -- (0.8693, -0.0445, 1.3433) -- (0.8706, -0.0475, 1.3933) -- (0.8222, -0.0475, 1.3892) -- cycle;
\fill[blue!17.3, opacity=0.5] (0.8222, -0.0475, 1.3892) -- (0.8706, -0.0475, 1.3933) -- (0.8719, -0.0506, 1.4433) -- (0.8236, -0.0506, 1.4392) -- cycle;
\fill[blue!18.0, opacity=0.5] (0.8236, -0.0506, 1.4392) -- (0.8719, -0.0506, 1.4433) -- (0.8733, -0.0537, 1.4933) -- (0.8251, -0.0537, 1.4892) -- cycle;
\fill[blue!18.9, opacity=0.5] (0.8251, -0.0537, 1.4892) -- (0.8733, -0.0537, 1.4933) -- (0.8746, -0.0569, 1.5433) -- (0.8265, -0.0569, 1.5392) -- cycle;
\fill[blue!19.9, opacity=0.5] (0.8265, -0.0569, 1.5392) -- (0.8746, -0.0569, 1.5433) -- (0.8760, -0.0600, 1.5933) -- (0.8280, -0.0600, 1.5892) -- cycle;
\fill[blue!21.1, opacity=0.5] (0.8280, -0.0600, 1.5892) -- (0.8760, -0.0600, 1.5933) -- (0.8774, -0.0631, 1.6433) -- (0.8295, -0.0631, 1.6392) -- cycle;
\fill[blue!22.4, opacity=0.5] (0.8295, -0.0631, 1.6392) -- (0.8774, -0.0631, 1.6433) -- (0.8787, -0.0663, 1.6933) -- (0.8309, -0.0663, 1.6892) -- cycle;
\fill[blue!23.9, opacity=0.5] (0.8309, -0.0663, 1.6892) -- (0.8787, -0.0663, 1.6933) -- (0.8801, -0.0694, 1.7433) -- (0.8324, -0.0694, 1.7392) -- cycle;
\fill[blue!25.5, opacity=0.5] (0.8324, -0.0694, 1.7392) -- (0.8801, -0.0694, 1.7433) -- (0.8814, -0.0725, 1.7933) -- (0.8338, -0.0725, 1.7892) -- cycle;
\fill[blue!27.3, opacity=0.5] (0.8338, -0.0725, 1.7892) -- (0.8814, -0.0725, 1.7933) -- (0.8827, -0.0755, 1.8433) -- (0.8352, -0.0755, 1.8392) -- cycle;
\fill[blue!29.2, opacity=0.5] (0.8352, -0.0755, 1.8392) -- (0.8827, -0.0755, 1.8433) -- (0.8840, -0.0785, 1.8933) -- (0.8367, -0.0785, 1.8892) -- cycle;
\fill[blue!31.2, opacity=0.5] (0.8367, -0.0785, 1.8892) -- (0.8840, -0.0785, 1.8933) -- (0.8853, -0.0815, 1.9433) -- (0.8380, -0.0815, 1.9392) -- cycle;
\fill[blue!33.2, opacity=0.5] (0.8380, -0.0815, 1.9392) -- (0.8853, -0.0815, 1.9433) -- (0.8866, -0.0844, 1.9933) -- (0.8394, -0.0844, 1.9892) -- cycle;
\fill[blue!35.3, opacity=0.5] (0.8394, -0.0844, 1.9892) -- (0.8866, -0.0844, 1.9933) -- (0.8878, -0.0872, 2.0433) -- (0.8407, -0.0872, 2.0392) -- cycle;
\fill[blue!37.5, opacity=0.5] (0.8407, -0.0872, 2.0392) -- (0.8878, -0.0872, 2.0433) -- (0.8890, -0.0900, 2.0933) -- (0.8420, -0.0900, 2.0892) -- cycle;
\fill[blue!39.6, opacity=0.5] (0.8420, -0.0900, 2.0892) -- (0.8890, -0.0900, 2.0933) -- (0.8902, -0.0927, 2.1433) -- (0.8432, -0.0927, 2.1392) -- cycle;
\fill[blue!41.8, opacity=0.5] (0.8432, -0.0927, 2.1392) -- (0.8902, -0.0927, 2.1433) -- (0.8913, -0.0953, 2.1933) -- (0.8445, -0.0953, 2.1892) -- cycle;
\fill[blue!43.9, opacity=0.5] (0.8445, -0.0953, 2.1892) -- (0.8913, -0.0953, 2.1933) -- (0.8924, -0.0978, 2.2433) -- (0.8456, -0.0978, 2.2392) -- cycle;
\fill[blue!45.9, opacity=0.5] (0.8456, -0.0978, 2.2392) -- (0.8924, -0.0978, 2.2433) -- (0.8934, -0.1001, 2.2933) -- (0.8467, -0.1001, 2.2892) -- cycle;
\fill[blue!47.9, opacity=0.5] (0.8467, -0.1001, 2.2892) -- (0.8934, -0.1001, 2.2933) -- (0.8944, -0.1024, 2.3433) -- (0.8478, -0.1024, 2.3392) -- cycle;
\fill[blue!49.7, opacity=0.5] (0.8478, -0.1024, 2.3392) -- (0.8944, -0.1024, 2.3433) -- (0.8953, -0.1046, 2.3933) -- (0.8488, -0.1046, 2.3892) -- cycle;
\fill[blue!51.5, opacity=0.5] (0.8488, -0.1046, 2.3892) -- (0.8953, -0.1046, 2.3933) -- (0.8962, -0.1066, 2.4433) -- (0.8498, -0.1066, 2.4392) -- cycle;
\fill[blue!53.1, opacity=0.5] (0.8498, -0.1066, 2.4392) -- (0.8962, -0.1066, 2.4433) -- (0.8970, -0.1085, 2.4933) -- (0.8507, -0.1085, 2.4892) -- cycle;
\fill[blue!54.6, opacity=0.5] (0.8507, -0.1085, 2.4892) -- (0.8970, -0.1085, 2.4933) -- (0.8978, -0.1103, 2.5433) -- (0.8515, -0.1103, 2.5392) -- cycle;
\fill[blue!55.9, opacity=0.5] (0.8515, -0.1103, 2.5392) -- (0.8978, -0.1103, 2.5433) -- (0.8985, -0.1120, 2.5933) -- (0.8522, -0.1120, 2.5892) -- cycle;
\fill[blue!57.1, opacity=0.5] (0.8522, -0.1120, 2.5892) -- (0.8985, -0.1120, 2.5933) -- (0.8992, -0.1135, 2.6433) -- (0.8529, -0.1135, 2.6392) -- cycle;
\fill[blue!58.1, opacity=0.5] (0.8529, -0.1135, 2.6392) -- (0.8992, -0.1135, 2.6433) -- (0.8998, -0.1148, 2.6933) -- (0.8536, -0.1148, 2.6892) -- cycle;
\fill[blue!58.9, opacity=0.5] (0.8536, -0.1148, 2.6892) -- (0.8998, -0.1148, 2.6933) -- (0.9003, -0.1160, 2.7433) -- (0.8541, -0.1160, 2.7392) -- cycle;
\fill[blue!59.6, opacity=0.5] (0.8541, -0.1160, 2.7392) -- (0.9003, -0.1160, 2.7433) -- (0.9007, -0.1171, 2.7933) -- (0.8546, -0.1171, 2.7892) -- cycle;
\fill[blue!60.1, opacity=0.5] (0.8546, -0.1171, 2.7892) -- (0.9007, -0.1171, 2.7933) -- (0.9011, -0.1180, 2.8433) -- (0.8550, -0.1180, 2.8392) -- cycle;
\fill[blue!60.4, opacity=0.5] (0.8550, -0.1180, 2.8392) -- (0.9011, -0.1180, 2.8433) -- (0.9014, -0.1187, 2.8933) -- (0.8554, -0.1187, 2.8892) -- cycle;
\fill[blue!60.6, opacity=0.5] (0.8554, -0.1187, 2.8892) -- (0.9014, -0.1187, 2.8933) -- (0.9017, -0.1193, 2.9433) -- (0.8557, -0.1193, 2.9392) -- cycle;
\fill[blue!60.7, opacity=0.5] (0.8557, -0.1193, 2.9392) -- (0.9017, -0.1193, 2.9433) -- (0.9019, -0.1197, 2.9933) -- (0.8558, -0.1197, 2.9892) -- cycle;
\fill[blue!60.6, opacity=0.5] (0.8558, -0.1197, 2.9892) -- (0.9019, -0.1197, 2.9933) -- (0.9020, -0.1199, 3.0433) -- (0.8560, -0.1199, 3.0392) -- cycle;
\fill[blue!60.3, opacity=0.5] (0.8560, -0.1199, 3.0392) -- (0.9020, -0.1199, 3.0433) -- (0.9020, -0.1200, 3.0933) -- (0.8560, -0.1200, 3.0892) -- cycle;
\fill[blue!15.0, opacity=0.5] (0.8500, -0.0000, 0.0933) -- (0.9000, -0.0000, 0.0971) -- (0.9000, -0.0001, 0.1471) -- (0.8500, -0.0001, 0.1433) -- cycle;
\fill[blue!15.0, opacity=0.5] (0.8500, -0.0001, 0.1433) -- (0.9000, -0.0001, 0.1471) -- (0.9001, -0.0003, 0.1971) -- (0.8501, -0.0003, 0.1933) -- cycle;
\fill[blue!15.0, opacity=0.5] (0.8501, -0.0003, 0.1933) -- (0.9001, -0.0003, 0.1971) -- (0.9003, -0.0007, 0.2471) -- (0.8503, -0.0007, 0.2433) -- cycle;
\fill[blue!15.0, opacity=0.5] (0.8503, -0.0007, 0.2433) -- (0.9003, -0.0007, 0.2471) -- (0.9005, -0.0013, 0.2971) -- (0.8506, -0.0013, 0.2933) -- cycle;
\fill[blue!15.0, opacity=0.5] (0.8506, -0.0013, 0.2933) -- (0.9005, -0.0013, 0.2971) -- (0.9008, -0.0020, 0.3471) -- (0.8509, -0.0020, 0.3433) -- cycle;
\fill[blue!15.0, opacity=0.5] (0.8509, -0.0020, 0.3433) -- (0.9008, -0.0020, 0.3471) -- (0.9012, -0.0029, 0.3971) -- (0.8513, -0.0029, 0.3933) -- cycle;
\fill[blue!15.0, opacity=0.5] (0.8513, -0.0029, 0.3933) -- (0.9012, -0.0029, 0.3971) -- (0.9016, -0.0040, 0.4471) -- (0.8517, -0.0040, 0.4433) -- cycle;
\fill[blue!15.0, opacity=0.5] (0.8517, -0.0040, 0.4433) -- (0.9016, -0.0040, 0.4471) -- (0.9021, -0.0052, 0.4971) -- (0.8522, -0.0052, 0.4933) -- cycle;
\fill[blue!15.0, opacity=0.5] (0.8522, -0.0052, 0.4933) -- (0.9021, -0.0052, 0.4971) -- (0.9026, -0.0065, 0.5471) -- (0.8528, -0.0065, 0.5433) -- cycle;
\fill[blue!15.0, opacity=0.5] (0.8528, -0.0065, 0.5433) -- (0.9026, -0.0065, 0.5471) -- (0.9032, -0.0080, 0.5971) -- (0.8535, -0.0080, 0.5933) -- cycle;
\fill[blue!15.0, opacity=0.5] (0.8535, -0.0080, 0.5933) -- (0.9032, -0.0080, 0.5971) -- (0.9039, -0.0097, 0.6471) -- (0.8542, -0.0097, 0.6433) -- cycle;
\fill[blue!15.0, opacity=0.5] (0.8542, -0.0097, 0.6433) -- (0.9039, -0.0097, 0.6471) -- (0.9046, -0.0115, 0.6971) -- (0.8550, -0.0115, 0.6933) -- cycle;
\fill[blue!15.0, opacity=0.5] (0.8550, -0.0115, 0.6933) -- (0.9046, -0.0115, 0.6971) -- (0.9053, -0.0134, 0.7471) -- (0.8558, -0.0134, 0.7433) -- cycle;
\fill[blue!15.0, opacity=0.5] (0.8558, -0.0134, 0.7433) -- (0.9053, -0.0134, 0.7471) -- (0.9062, -0.0154, 0.7971) -- (0.8567, -0.0154, 0.7933) -- cycle;
\fill[blue!15.0, opacity=0.5] (0.8567, -0.0154, 0.7933) -- (0.9062, -0.0154, 0.7971) -- (0.9070, -0.0176, 0.8471) -- (0.8576, -0.0176, 0.8433) -- cycle;
\fill[blue!15.0, opacity=0.5] (0.8576, -0.0176, 0.8433) -- (0.9070, -0.0176, 0.8471) -- (0.9079, -0.0199, 0.8971) -- (0.8586, -0.0199, 0.8933) -- cycle;
\fill[blue!15.1, opacity=0.5] (0.8586, -0.0199, 0.8933) -- (0.9079, -0.0199, 0.8971) -- (0.9089, -0.0222, 0.9471) -- (0.8596, -0.0222, 0.9433) -- cycle;
\fill[blue!15.1, opacity=0.5] (0.8596, -0.0222, 0.9433) -- (0.9089, -0.0222, 0.9471) -- (0.9099, -0.0247, 0.9971) -- (0.8607, -0.0247, 0.9933) -- cycle;
\fill[blue!15.2, opacity=0.5] (0.8607, -0.0247, 0.9933) -- (0.9099, -0.0247, 0.9971) -- (0.9109, -0.0273, 1.0471) -- (0.8618, -0.0273, 1.0433) -- cycle;
\fill[blue!15.3, opacity=0.5] (0.8618, -0.0273, 1.0433) -- (0.9109, -0.0273, 1.0471) -- (0.9120, -0.0300, 1.0971) -- (0.8630, -0.0300, 1.0933) -- cycle;
\fill[blue!15.4, opacity=0.5] (0.8630, -0.0300, 1.0933) -- (0.9120, -0.0300, 1.0971) -- (0.9131, -0.0328, 1.1471) -- (0.8642, -0.0328, 1.1433) -- cycle;
\fill[blue!15.6, opacity=0.5] (0.8642, -0.0328, 1.1433) -- (0.9131, -0.0328, 1.1471) -- (0.9142, -0.0356, 1.1971) -- (0.8654, -0.0356, 1.1933) -- cycle;
\fill[blue!16.0, opacity=0.5] (0.8654, -0.0356, 1.1933) -- (0.9142, -0.0356, 1.1971) -- (0.9154, -0.0385, 1.2471) -- (0.8667, -0.0385, 1.2433) -- cycle;
\fill[blue!16.4, opacity=0.5] (0.8667, -0.0385, 1.2433) -- (0.9154, -0.0385, 1.2471) -- (0.9166, -0.0415, 1.2971) -- (0.8680, -0.0415, 1.2933) -- cycle;
\fill[blue!16.9, opacity=0.5] (0.8680, -0.0415, 1.2933) -- (0.9166, -0.0415, 1.2971) -- (0.9178, -0.0445, 1.3471) -- (0.8693, -0.0445, 1.3433) -- cycle;
\fill[blue!17.5, opacity=0.5] (0.8693, -0.0445, 1.3433) -- (0.9178, -0.0445, 1.3471) -- (0.9190, -0.0475, 1.3971) -- (0.8706, -0.0475, 1.3933) -- cycle;
\fill[blue!18.3, opacity=0.5] (0.8706, -0.0475, 1.3933) -- (0.9190, -0.0475, 1.3971) -- (0.9202, -0.0506, 1.4471) -- (0.8719, -0.0506, 1.4433) -- cycle;
\fill[blue!19.3, opacity=0.5] (0.8719, -0.0506, 1.4433) -- (0.9202, -0.0506, 1.4471) -- (0.9215, -0.0537, 1.4971) -- (0.8733, -0.0537, 1.4933) -- cycle;
\fill[blue!20.4, opacity=0.5] (0.8733, -0.0537, 1.4933) -- (0.9215, -0.0537, 1.4971) -- (0.9227, -0.0569, 1.5471) -- (0.8746, -0.0569, 1.5433) -- cycle;
\fill[blue!21.7, opacity=0.5] (0.8746, -0.0569, 1.5433) -- (0.9227, -0.0569, 1.5471) -- (0.9240, -0.0600, 1.5971) -- (0.8760, -0.0600, 1.5933) -- cycle;
\fill[blue!23.1, opacity=0.5] (0.8760, -0.0600, 1.5933) -- (0.9240, -0.0600, 1.5971) -- (0.9253, -0.0631, 1.6471) -- (0.8774, -0.0631, 1.6433) -- cycle;
\fill[blue!24.8, opacity=0.5] (0.8774, -0.0631, 1.6433) -- (0.9253, -0.0631, 1.6471) -- (0.9265, -0.0663, 1.6971) -- (0.8787, -0.0663, 1.6933) -- cycle;
\fill[blue!26.5, opacity=0.5] (0.8787, -0.0663, 1.6933) -- (0.9265, -0.0663, 1.6971) -- (0.9278, -0.0694, 1.7471) -- (0.8801, -0.0694, 1.7433) -- cycle;
\fill[blue!28.4, opacity=0.5] (0.8801, -0.0694, 1.7433) -- (0.9278, -0.0694, 1.7471) -- (0.9290, -0.0725, 1.7971) -- (0.8814, -0.0725, 1.7933) -- cycle;
\fill[blue!30.5, opacity=0.5] (0.8814, -0.0725, 1.7933) -- (0.9290, -0.0725, 1.7971) -- (0.9302, -0.0755, 1.8471) -- (0.8827, -0.0755, 1.8433) -- cycle;
\fill[blue!32.6, opacity=0.5] (0.8827, -0.0755, 1.8433) -- (0.9302, -0.0755, 1.8471) -- (0.9314, -0.0785, 1.8971) -- (0.8840, -0.0785, 1.8933) -- cycle;
\fill[blue!34.8, opacity=0.5] (0.8840, -0.0785, 1.8933) -- (0.9314, -0.0785, 1.8971) -- (0.9326, -0.0815, 1.9471) -- (0.8853, -0.0815, 1.9433) -- cycle;
\fill[blue!37.1, opacity=0.5] (0.8853, -0.0815, 1.9433) -- (0.9326, -0.0815, 1.9471) -- (0.9338, -0.0844, 1.9971) -- (0.8866, -0.0844, 1.9933) -- cycle;
\fill[blue!39.3, opacity=0.5] (0.8866, -0.0844, 1.9933) -- (0.9338, -0.0844, 1.9971) -- (0.9349, -0.0872, 2.0471) -- (0.8878, -0.0872, 2.0433) -- cycle;
\fill[blue!41.6, opacity=0.5] (0.8878, -0.0872, 2.0433) -- (0.9349, -0.0872, 2.0471) -- (0.9360, -0.0900, 2.0971) -- (0.8890, -0.0900, 2.0933) -- cycle;
\fill[blue!43.9, opacity=0.5] (0.8890, -0.0900, 2.0933) -- (0.9360, -0.0900, 2.0971) -- (0.9371, -0.0927, 2.1471) -- (0.8902, -0.0927, 2.1433) -- cycle;
\fill[blue!46.1, opacity=0.5] (0.8902, -0.0927, 2.1433) -- (0.9371, -0.0927, 2.1471) -- (0.9381, -0.0953, 2.1971) -- (0.8913, -0.0953, 2.1933) -- cycle;
\fill[blue!48.2, opacity=0.5] (0.8913, -0.0953, 2.1933) -- (0.9381, -0.0953, 2.1971) -- (0.9391, -0.0978, 2.2471) -- (0.8924, -0.0978, 2.2433) -- cycle;
\fill[blue!50.2, opacity=0.5] (0.8924, -0.0978, 2.2433) -- (0.9391, -0.0978, 2.2471) -- (0.9401, -0.1001, 2.2971) -- (0.8934, -0.1001, 2.2933) -- cycle;
\fill[blue!52.1, opacity=0.5] (0.8934, -0.1001, 2.2933) -- (0.9401, -0.1001, 2.2971) -- (0.9410, -0.1024, 2.3471) -- (0.8944, -0.1024, 2.3433) -- cycle;
\fill[blue!53.9, opacity=0.5] (0.8944, -0.1024, 2.3433) -- (0.9410, -0.1024, 2.3471) -- (0.9418, -0.1046, 2.3971) -- (0.8953, -0.1046, 2.3933) -- cycle;
\fill[blue!55.5, opacity=0.5] (0.8953, -0.1046, 2.3933) -- (0.9418, -0.1046, 2.3971) -- (0.9427, -0.1066, 2.4471) -- (0.8962, -0.1066, 2.4433) -- cycle;
\fill[blue!57.0, opacity=0.5] (0.8962, -0.1066, 2.4433) -- (0.9427, -0.1066, 2.4471) -- (0.9434, -0.1085, 2.4971) -- (0.8970, -0.1085, 2.4933) -- cycle;
\fill[blue!58.3, opacity=0.5] (0.8970, -0.1085, 2.4933) -- (0.9434, -0.1085, 2.4971) -- (0.9441, -0.1103, 2.5471) -- (0.8978, -0.1103, 2.5433) -- cycle;
\fill[blue!59.5, opacity=0.5] (0.8978, -0.1103, 2.5433) -- (0.9441, -0.1103, 2.5471) -- (0.9448, -0.1120, 2.5971) -- (0.8985, -0.1120, 2.5933) -- cycle;
\fill[blue!60.5, opacity=0.5] (0.8985, -0.1120, 2.5933) -- (0.9448, -0.1120, 2.5971) -- (0.9454, -0.1135, 2.6471) -- (0.8992, -0.1135, 2.6433) -- cycle;
\fill[blue!61.3, opacity=0.5] (0.8992, -0.1135, 2.6433) -- (0.9454, -0.1135, 2.6471) -- (0.9459, -0.1148, 2.6971) -- (0.8998, -0.1148, 2.6933) -- cycle;
\fill[blue!61.9, opacity=0.5] (0.8998, -0.1148, 2.6933) -- (0.9459, -0.1148, 2.6971) -- (0.9464, -0.1160, 2.7471) -- (0.9003, -0.1160, 2.7433) -- cycle;
\fill[blue!62.4, opacity=0.5] (0.9003, -0.1160, 2.7433) -- (0.9464, -0.1160, 2.7471) -- (0.9468, -0.1171, 2.7971) -- (0.9007, -0.1171, 2.7933) -- cycle;
\fill[blue!62.6, opacity=0.5] (0.9007, -0.1171, 2.7933) -- (0.9468, -0.1171, 2.7971) -- (0.9472, -0.1180, 2.8471) -- (0.9011, -0.1180, 2.8433) -- cycle;
\fill[blue!62.8, opacity=0.5] (0.9011, -0.1180, 2.8433) -- (0.9472, -0.1180, 2.8471) -- (0.9475, -0.1187, 2.8971) -- (0.9014, -0.1187, 2.8933) -- cycle;
\fill[blue!62.7, opacity=0.5] (0.9014, -0.1187, 2.8933) -- (0.9475, -0.1187, 2.8971) -- (0.9477, -0.1193, 2.9471) -- (0.9017, -0.1193, 2.9433) -- cycle;
\fill[blue!62.6, opacity=0.5] (0.9017, -0.1193, 2.9433) -- (0.9477, -0.1193, 2.9471) -- (0.9479, -0.1197, 2.9971) -- (0.9019, -0.1197, 2.9933) -- cycle;
\fill[blue!62.2, opacity=0.5] (0.9019, -0.1197, 2.9933) -- (0.9479, -0.1197, 2.9971) -- (0.9480, -0.1199, 3.0471) -- (0.9020, -0.1199, 3.0433) -- cycle;
\fill[blue!61.8, opacity=0.5] (0.9020, -0.1199, 3.0433) -- (0.9480, -0.1199, 3.0471) -- (0.9480, -0.1200, 3.0971) -- (0.9020, -0.1200, 3.0933) -- cycle;
\fill[blue!15.0, opacity=0.5] (0.9000, -0.0000, 0.0971) -- (0.9500, -0.0000, 0.1006) -- (0.9500, -0.0001, 0.1506) -- (0.9000, -0.0001, 0.1471) -- cycle;
\fill[blue!15.0, opacity=0.5] (0.9000, -0.0001, 0.1471) -- (0.9500, -0.0001, 0.1506) -- (0.9501, -0.0003, 0.2006) -- (0.9001, -0.0003, 0.1971) -- cycle;
\fill[blue!15.0, opacity=0.5] (0.9001, -0.0003, 0.1971) -- (0.9501, -0.0003, 0.2006) -- (0.9503, -0.0007, 0.2506) -- (0.9003, -0.0007, 0.2471) -- cycle;
\fill[blue!15.0, opacity=0.5] (0.9003, -0.0007, 0.2471) -- (0.9503, -0.0007, 0.2506) -- (0.9505, -0.0013, 0.3006) -- (0.9005, -0.0013, 0.2971) -- cycle;
\fill[blue!15.0, opacity=0.5] (0.9005, -0.0013, 0.2971) -- (0.9505, -0.0013, 0.3006) -- (0.9507, -0.0020, 0.3506) -- (0.9008, -0.0020, 0.3471) -- cycle;
\fill[blue!15.0, opacity=0.5] (0.9008, -0.0020, 0.3471) -- (0.9507, -0.0020, 0.3506) -- (0.9511, -0.0029, 0.4006) -- (0.9012, -0.0029, 0.3971) -- cycle;
\fill[blue!15.0, opacity=0.5] (0.9012, -0.0029, 0.3971) -- (0.9511, -0.0029, 0.4006) -- (0.9515, -0.0040, 0.4506) -- (0.9016, -0.0040, 0.4471) -- cycle;
\fill[blue!15.0, opacity=0.5] (0.9016, -0.0040, 0.4471) -- (0.9515, -0.0040, 0.4506) -- (0.9519, -0.0052, 0.5006) -- (0.9021, -0.0052, 0.4971) -- cycle;
\fill[blue!15.0, opacity=0.5] (0.9021, -0.0052, 0.4971) -- (0.9519, -0.0052, 0.5006) -- (0.9524, -0.0065, 0.5506) -- (0.9026, -0.0065, 0.5471) -- cycle;
\fill[blue!15.0, opacity=0.5] (0.9026, -0.0065, 0.5471) -- (0.9524, -0.0065, 0.5506) -- (0.9529, -0.0080, 0.6006) -- (0.9032, -0.0080, 0.5971) -- cycle;
\fill[blue!15.0, opacity=0.5] (0.9032, -0.0080, 0.5971) -- (0.9529, -0.0080, 0.6006) -- (0.9535, -0.0097, 0.6506) -- (0.9039, -0.0097, 0.6471) -- cycle;
\fill[blue!15.0, opacity=0.5] (0.9039, -0.0097, 0.6471) -- (0.9535, -0.0097, 0.6506) -- (0.9542, -0.0115, 0.7006) -- (0.9046, -0.0115, 0.6971) -- cycle;
\fill[blue!15.0, opacity=0.5] (0.9046, -0.0115, 0.6971) -- (0.9542, -0.0115, 0.7006) -- (0.9549, -0.0134, 0.7506) -- (0.9053, -0.0134, 0.7471) -- cycle;
\fill[blue!15.0, opacity=0.5] (0.9053, -0.0134, 0.7471) -- (0.9549, -0.0134, 0.7506) -- (0.9557, -0.0154, 0.8006) -- (0.9062, -0.0154, 0.7971) -- cycle;
\fill[blue!15.0, opacity=0.5] (0.9062, -0.0154, 0.7971) -- (0.9557, -0.0154, 0.8006) -- (0.9564, -0.0176, 0.8506) -- (0.9070, -0.0176, 0.8471) -- cycle;
\fill[blue!15.0, opacity=0.5] (0.9070, -0.0176, 0.8471) -- (0.9564, -0.0176, 0.8506) -- (0.9573, -0.0199, 0.9006) -- (0.9079, -0.0199, 0.8971) -- cycle;
\fill[blue!15.1, opacity=0.5] (0.9079, -0.0199, 0.8971) -- (0.9573, -0.0199, 0.9006) -- (0.9582, -0.0222, 0.9506) -- (0.9089, -0.0222, 0.9471) -- cycle;
\fill[blue!15.1, opacity=0.5] (0.9089, -0.0222, 0.9471) -- (0.9582, -0.0222, 0.9506) -- (0.9591, -0.0247, 1.0006) -- (0.9099, -0.0247, 0.9971) -- cycle;
\fill[blue!15.2, opacity=0.5] (0.9099, -0.0247, 0.9971) -- (0.9591, -0.0247, 1.0006) -- (0.9600, -0.0273, 1.0506) -- (0.9109, -0.0273, 1.0471) -- cycle;
\fill[blue!15.3, opacity=0.5] (0.9109, -0.0273, 1.0471) -- (0.9600, -0.0273, 1.0506) -- (0.9610, -0.0300, 1.1006) -- (0.9120, -0.0300, 1.0971) -- cycle;
\fill[blue!15.5, opacity=0.5] (0.9120, -0.0300, 1.0971) -- (0.9610, -0.0300, 1.1006) -- (0.9620, -0.0328, 1.1506) -- (0.9131, -0.0328, 1.1471) -- cycle;
\fill[blue!15.7, opacity=0.5] (0.9131, -0.0328, 1.1471) -- (0.9620, -0.0328, 1.1506) -- (0.9631, -0.0356, 1.2006) -- (0.9142, -0.0356, 1.1971) -- cycle;
\fill[blue!16.1, opacity=0.5] (0.9142, -0.0356, 1.1971) -- (0.9631, -0.0356, 1.2006) -- (0.9641, -0.0385, 1.2506) -- (0.9154, -0.0385, 1.2471) -- cycle;
\fill[blue!16.5, opacity=0.5] (0.9154, -0.0385, 1.2471) -- (0.9641, -0.0385, 1.2506) -- (0.9652, -0.0415, 1.3006) -- (0.9166, -0.0415, 1.2971) -- cycle;
\fill[blue!17.1, opacity=0.5] (0.9166, -0.0415, 1.2971) -- (0.9652, -0.0415, 1.3006) -- (0.9663, -0.0445, 1.3506) -- (0.9178, -0.0445, 1.3471) -- cycle;
\fill[blue!17.8, opacity=0.5] (0.9178, -0.0445, 1.3471) -- (0.9663, -0.0445, 1.3506) -- (0.9674, -0.0475, 1.4006) -- (0.9190, -0.0475, 1.3971) -- cycle;
\fill[blue!18.6, opacity=0.5] (0.9190, -0.0475, 1.3971) -- (0.9674, -0.0475, 1.4006) -- (0.9686, -0.0506, 1.4506) -- (0.9202, -0.0506, 1.4471) -- cycle;
\fill[blue!19.6, opacity=0.5] (0.9202, -0.0506, 1.4471) -- (0.9686, -0.0506, 1.4506) -- (0.9697, -0.0537, 1.5006) -- (0.9215, -0.0537, 1.4971) -- cycle;
\fill[blue!20.8, opacity=0.5] (0.9215, -0.0537, 1.4971) -- (0.9697, -0.0537, 1.5006) -- (0.9708, -0.0569, 1.5506) -- (0.9227, -0.0569, 1.5471) -- cycle;
\fill[blue!22.2, opacity=0.5] (0.9227, -0.0569, 1.5471) -- (0.9708, -0.0569, 1.5506) -- (0.9720, -0.0600, 1.6006) -- (0.9240, -0.0600, 1.5971) -- cycle;
\fill[blue!23.7, opacity=0.5] (0.9240, -0.0600, 1.5971) -- (0.9720, -0.0600, 1.6006) -- (0.9732, -0.0631, 1.6506) -- (0.9253, -0.0631, 1.6471) -- cycle;
\fill[blue!25.4, opacity=0.5] (0.9253, -0.0631, 1.6471) -- (0.9732, -0.0631, 1.6506) -- (0.9743, -0.0663, 1.7006) -- (0.9265, -0.0663, 1.6971) -- cycle;
\fill[blue!27.2, opacity=0.5] (0.9265, -0.0663, 1.6971) -- (0.9743, -0.0663, 1.7006) -- (0.9754, -0.0694, 1.7506) -- (0.9278, -0.0694, 1.7471) -- cycle;
\fill[blue!29.2, opacity=0.5] (0.9278, -0.0694, 1.7471) -- (0.9754, -0.0694, 1.7506) -- (0.9766, -0.0725, 1.8006) -- (0.9290, -0.0725, 1.7971) -- cycle;
\fill[blue!31.3, opacity=0.5] (0.9290, -0.0725, 1.7971) -- (0.9766, -0.0725, 1.8006) -- (0.9777, -0.0755, 1.8506) -- (0.9302, -0.0755, 1.8471) -- cycle;
\fill[blue!33.5, opacity=0.5] (0.9302, -0.0755, 1.8471) -- (0.9777, -0.0755, 1.8506) -- (0.9788, -0.0785, 1.9006) -- (0.9314, -0.0785, 1.8971) -- cycle;
\fill[blue!35.8, opacity=0.5] (0.9314, -0.0785, 1.8971) -- (0.9788, -0.0785, 1.9006) -- (0.9799, -0.0815, 1.9506) -- (0.9326, -0.0815, 1.9471) -- cycle;
\fill[blue!38.1, opacity=0.5] (0.9326, -0.0815, 1.9471) -- (0.9799, -0.0815, 1.9506) -- (0.9809, -0.0844, 2.0006) -- (0.9338, -0.0844, 1.9971) -- cycle;
\fill[blue!40.4, opacity=0.5] (0.9338, -0.0844, 1.9971) -- (0.9809, -0.0844, 2.0006) -- (0.9820, -0.0872, 2.0506) -- (0.9349, -0.0872, 2.0471) -- cycle;
\fill[blue!42.7, opacity=0.5] (0.9349, -0.0872, 2.0471) -- (0.9820, -0.0872, 2.0506) -- (0.9830, -0.0900, 2.1006) -- (0.9360, -0.0900, 2.0971) -- cycle;
\fill[blue!44.9, opacity=0.5] (0.9360, -0.0900, 2.0971) -- (0.9830, -0.0900, 2.1006) -- (0.9840, -0.0927, 2.1506) -- (0.9371, -0.0927, 2.1471) -- cycle;
\fill[blue!47.1, opacity=0.5] (0.9371, -0.0927, 2.1471) -- (0.9840, -0.0927, 2.1506) -- (0.9849, -0.0953, 2.2006) -- (0.9381, -0.0953, 2.1971) -- cycle;
\fill[blue!49.3, opacity=0.5] (0.9381, -0.0953, 2.1971) -- (0.9849, -0.0953, 2.2006) -- (0.9858, -0.0978, 2.2506) -- (0.9391, -0.0978, 2.2471) -- cycle;
\fill[blue!51.3, opacity=0.5] (0.9391, -0.0978, 2.2471) -- (0.9858, -0.0978, 2.2506) -- (0.9867, -0.1001, 2.3006) -- (0.9401, -0.1001, 2.2971) -- cycle;
\fill[blue!53.2, opacity=0.5] (0.9401, -0.1001, 2.2971) -- (0.9867, -0.1001, 2.3006) -- (0.9876, -0.1024, 2.3506) -- (0.9410, -0.1024, 2.3471) -- cycle;
\fill[blue!54.9, opacity=0.5] (0.9410, -0.1024, 2.3471) -- (0.9876, -0.1024, 2.3506) -- (0.9883, -0.1046, 2.4006) -- (0.9418, -0.1046, 2.3971) -- cycle;
\fill[blue!56.5, opacity=0.5] (0.9418, -0.1046, 2.3971) -- (0.9883, -0.1046, 2.4006) -- (0.9891, -0.1066, 2.4506) -- (0.9427, -0.1066, 2.4471) -- cycle;
\fill[blue!58.0, opacity=0.5] (0.9427, -0.1066, 2.4471) -- (0.9891, -0.1066, 2.4506) -- (0.9898, -0.1085, 2.5006) -- (0.9434, -0.1085, 2.4971) -- cycle;
\fill[blue!59.2, opacity=0.5] (0.9434, -0.1085, 2.4971) -- (0.9898, -0.1085, 2.5006) -- (0.9905, -0.1103, 2.5506) -- (0.9441, -0.1103, 2.5471) -- cycle;
\fill[blue!60.3, opacity=0.5] (0.9441, -0.1103, 2.5471) -- (0.9905, -0.1103, 2.5506) -- (0.9911, -0.1120, 2.6006) -- (0.9448, -0.1120, 2.5971) -- cycle;
\fill[blue!61.3, opacity=0.5] (0.9448, -0.1120, 2.5971) -- (0.9911, -0.1120, 2.6006) -- (0.9916, -0.1135, 2.6506) -- (0.9454, -0.1135, 2.6471) -- cycle;
\fill[blue!62.0, opacity=0.5] (0.9454, -0.1135, 2.6471) -- (0.9916, -0.1135, 2.6506) -- (0.9921, -0.1148, 2.7006) -- (0.9459, -0.1148, 2.6971) -- cycle;
\fill[blue!62.6, opacity=0.5] (0.9459, -0.1148, 2.6971) -- (0.9921, -0.1148, 2.7006) -- (0.9925, -0.1160, 2.7506) -- (0.9464, -0.1160, 2.7471) -- cycle;
\fill[blue!63.0, opacity=0.5] (0.9464, -0.1160, 2.7471) -- (0.9925, -0.1160, 2.7506) -- (0.9929, -0.1171, 2.8006) -- (0.9468, -0.1171, 2.7971) -- cycle;
\fill[blue!63.2, opacity=0.5] (0.9468, -0.1171, 2.7971) -- (0.9929, -0.1171, 2.8006) -- (0.9933, -0.1180, 2.8506) -- (0.9472, -0.1180, 2.8471) -- cycle;
\fill[blue!63.3, opacity=0.5] (0.9472, -0.1180, 2.8471) -- (0.9933, -0.1180, 2.8506) -- (0.9935, -0.1187, 2.9006) -- (0.9475, -0.1187, 2.8971) -- cycle;
\fill[blue!63.2, opacity=0.5] (0.9475, -0.1187, 2.8971) -- (0.9935, -0.1187, 2.9006) -- (0.9937, -0.1193, 2.9506) -- (0.9477, -0.1193, 2.9471) -- cycle;
\fill[blue!63.0, opacity=0.5] (0.9477, -0.1193, 2.9471) -- (0.9937, -0.1193, 2.9506) -- (0.9939, -0.1197, 3.0006) -- (0.9479, -0.1197, 2.9971) -- cycle;
\fill[blue!62.6, opacity=0.5] (0.9479, -0.1197, 2.9971) -- (0.9939, -0.1197, 3.0006) -- (0.9940, -0.1199, 3.0506) -- (0.9480, -0.1199, 3.0471) -- cycle;
\fill[blue!62.1, opacity=0.5] (0.9480, -0.1199, 3.0471) -- (0.9940, -0.1199, 3.0506) -- (0.9940, -0.1200, 3.1006) -- (0.9480, -0.1200, 3.0971) -- cycle;
\fill[blue!15.0, opacity=0.5] (0.9500, -0.0000, 0.1006) -- (1.0000, -0.0000, 0.1039) -- (1.0000, -0.0001, 0.1539) -- (0.9500, -0.0001, 0.1506) -- cycle;
\fill[blue!15.0, opacity=0.5] (0.9500, -0.0001, 0.1506) -- (1.0000, -0.0001, 0.1539) -- (1.0001, -0.0003, 0.2039) -- (0.9501, -0.0003, 0.2006) -- cycle;
\fill[blue!15.0, opacity=0.5] (0.9501, -0.0003, 0.2006) -- (1.0001, -0.0003, 0.2039) -- (1.0002, -0.0007, 0.2539) -- (0.9503, -0.0007, 0.2506) -- cycle;
\fill[blue!15.0, opacity=0.5] (0.9503, -0.0007, 0.2506) -- (1.0002, -0.0007, 0.2539) -- (1.0004, -0.0013, 0.3039) -- (0.9505, -0.0013, 0.3006) -- cycle;
\fill[blue!15.0, opacity=0.5] (0.9505, -0.0013, 0.3006) -- (1.0004, -0.0013, 0.3039) -- (1.0007, -0.0020, 0.3539) -- (0.9507, -0.0020, 0.3506) -- cycle;
\fill[blue!15.0, opacity=0.5] (0.9507, -0.0020, 0.3506) -- (1.0007, -0.0020, 0.3539) -- (1.0010, -0.0029, 0.4039) -- (0.9511, -0.0029, 0.4006) -- cycle;
\fill[blue!15.0, opacity=0.5] (0.9511, -0.0029, 0.4006) -- (1.0010, -0.0029, 0.4039) -- (1.0013, -0.0040, 0.4539) -- (0.9515, -0.0040, 0.4506) -- cycle;
\fill[blue!15.0, opacity=0.5] (0.9515, -0.0040, 0.4506) -- (1.0013, -0.0040, 0.4539) -- (1.0017, -0.0052, 0.5039) -- (0.9519, -0.0052, 0.5006) -- cycle;
\fill[blue!15.0, opacity=0.5] (0.9519, -0.0052, 0.5006) -- (1.0017, -0.0052, 0.5039) -- (1.0022, -0.0065, 0.5539) -- (0.9524, -0.0065, 0.5506) -- cycle;
\fill[blue!15.0, opacity=0.5] (0.9524, -0.0065, 0.5506) -- (1.0022, -0.0065, 0.5539) -- (1.0027, -0.0080, 0.6039) -- (0.9529, -0.0080, 0.6006) -- cycle;
\fill[blue!15.0, opacity=0.5] (0.9529, -0.0080, 0.6006) -- (1.0027, -0.0080, 0.6039) -- (1.0032, -0.0097, 0.6539) -- (0.9535, -0.0097, 0.6506) -- cycle;
\fill[blue!15.0, opacity=0.5] (0.9535, -0.0097, 0.6506) -- (1.0032, -0.0097, 0.6539) -- (1.0038, -0.0115, 0.7039) -- (0.9542, -0.0115, 0.7006) -- cycle;
\fill[blue!15.0, opacity=0.5] (0.9542, -0.0115, 0.7006) -- (1.0038, -0.0115, 0.7039) -- (1.0045, -0.0134, 0.7539) -- (0.9549, -0.0134, 0.7506) -- cycle;
\fill[blue!15.0, opacity=0.5] (0.9549, -0.0134, 0.7506) -- (1.0045, -0.0134, 0.7539) -- (1.0051, -0.0154, 0.8039) -- (0.9557, -0.0154, 0.8006) -- cycle;
\fill[blue!15.0, opacity=0.5] (0.9557, -0.0154, 0.8006) -- (1.0051, -0.0154, 0.8039) -- (1.0059, -0.0176, 0.8539) -- (0.9564, -0.0176, 0.8506) -- cycle;
\fill[blue!15.0, opacity=0.5] (0.9564, -0.0176, 0.8506) -- (1.0059, -0.0176, 0.8539) -- (1.0066, -0.0199, 0.9039) -- (0.9573, -0.0199, 0.9006) -- cycle;
\fill[blue!15.1, opacity=0.5] (0.9573, -0.0199, 0.9006) -- (1.0066, -0.0199, 0.9039) -- (1.0074, -0.0222, 0.9539) -- (0.9582, -0.0222, 0.9506) -- cycle;
\fill[blue!15.1, opacity=0.5] (0.9582, -0.0222, 0.9506) -- (1.0074, -0.0222, 0.9539) -- (1.0082, -0.0247, 1.0039) -- (0.9591, -0.0247, 1.0006) -- cycle;
\fill[blue!15.2, opacity=0.5] (0.9591, -0.0247, 1.0006) -- (1.0082, -0.0247, 1.0039) -- (1.0091, -0.0273, 1.0539) -- (0.9600, -0.0273, 1.0506) -- cycle;
\fill[blue!15.3, opacity=0.5] (0.9600, -0.0273, 1.0506) -- (1.0091, -0.0273, 1.0539) -- (1.0100, -0.0300, 1.1039) -- (0.9610, -0.0300, 1.1006) -- cycle;
\fill[blue!15.4, opacity=0.5] (0.9610, -0.0300, 1.1006) -- (1.0100, -0.0300, 1.1039) -- (1.0109, -0.0328, 1.1539) -- (0.9620, -0.0328, 1.1506) -- cycle;
\fill[blue!15.6, opacity=0.5] (0.9620, -0.0328, 1.1506) -- (1.0109, -0.0328, 1.1539) -- (1.0119, -0.0356, 1.2039) -- (0.9631, -0.0356, 1.2006) -- cycle;
\fill[blue!15.9, opacity=0.5] (0.9631, -0.0356, 1.2006) -- (1.0119, -0.0356, 1.2039) -- (1.0128, -0.0385, 1.2539) -- (0.9641, -0.0385, 1.2506) -- cycle;
\fill[blue!16.3, opacity=0.5] (0.9641, -0.0385, 1.2506) -- (1.0128, -0.0385, 1.2539) -- (1.0138, -0.0415, 1.3039) -- (0.9652, -0.0415, 1.3006) -- cycle;
\fill[blue!16.8, opacity=0.5] (0.9652, -0.0415, 1.3006) -- (1.0138, -0.0415, 1.3039) -- (1.0148, -0.0445, 1.3539) -- (0.9663, -0.0445, 1.3506) -- cycle;
\fill[blue!17.4, opacity=0.5] (0.9663, -0.0445, 1.3506) -- (1.0148, -0.0445, 1.3539) -- (1.0158, -0.0475, 1.4039) -- (0.9674, -0.0475, 1.4006) -- cycle;
\fill[blue!18.2, opacity=0.5] (0.9674, -0.0475, 1.4006) -- (1.0158, -0.0475, 1.4039) -- (1.0169, -0.0506, 1.4539) -- (0.9686, -0.0506, 1.4506) -- cycle;
\fill[blue!19.2, opacity=0.5] (0.9686, -0.0506, 1.4506) -- (1.0169, -0.0506, 1.4539) -- (1.0179, -0.0537, 1.5039) -- (0.9697, -0.0537, 1.5006) -- cycle;
\fill[blue!20.3, opacity=0.5] (0.9697, -0.0537, 1.5006) -- (1.0179, -0.0537, 1.5039) -- (1.0190, -0.0569, 1.5539) -- (0.9708, -0.0569, 1.5506) -- cycle;
\fill[blue!21.5, opacity=0.5] (0.9708, -0.0569, 1.5506) -- (1.0190, -0.0569, 1.5539) -- (1.0200, -0.0600, 1.6039) -- (0.9720, -0.0600, 1.6006) -- cycle;
\fill[blue!23.0, opacity=0.5] (0.9720, -0.0600, 1.6006) -- (1.0200, -0.0600, 1.6039) -- (1.0210, -0.0631, 1.6539) -- (0.9732, -0.0631, 1.6506) -- cycle;
\fill[blue!24.6, opacity=0.5] (0.9732, -0.0631, 1.6506) -- (1.0210, -0.0631, 1.6539) -- (1.0221, -0.0663, 1.7039) -- (0.9743, -0.0663, 1.7006) -- cycle;
\fill[blue!26.3, opacity=0.5] (0.9743, -0.0663, 1.7006) -- (1.0221, -0.0663, 1.7039) -- (1.0231, -0.0694, 1.7539) -- (0.9754, -0.0694, 1.7506) -- cycle;
\fill[blue!28.2, opacity=0.5] (0.9754, -0.0694, 1.7506) -- (1.0231, -0.0694, 1.7539) -- (1.0242, -0.0725, 1.8039) -- (0.9766, -0.0725, 1.8006) -- cycle;
\fill[blue!30.2, opacity=0.5] (0.9766, -0.0725, 1.8006) -- (1.0242, -0.0725, 1.8039) -- (1.0252, -0.0755, 1.8539) -- (0.9777, -0.0755, 1.8506) -- cycle;
\fill[blue!32.3, opacity=0.5] (0.9777, -0.0755, 1.8506) -- (1.0252, -0.0755, 1.8539) -- (1.0262, -0.0785, 1.9039) -- (0.9788, -0.0785, 1.9006) -- cycle;
\fill[blue!34.5, opacity=0.5] (0.9788, -0.0785, 1.9006) -- (1.0262, -0.0785, 1.9039) -- (1.0272, -0.0815, 1.9539) -- (0.9799, -0.0815, 1.9506) -- cycle;
\fill[blue!36.7, opacity=0.5] (0.9799, -0.0815, 1.9506) -- (1.0272, -0.0815, 1.9539) -- (1.0281, -0.0844, 2.0039) -- (0.9809, -0.0844, 2.0006) -- cycle;
\fill[blue!39.0, opacity=0.5] (0.9809, -0.0844, 2.0006) -- (1.0281, -0.0844, 2.0039) -- (1.0291, -0.0872, 2.0539) -- (0.9820, -0.0872, 2.0506) -- cycle;
\fill[blue!41.3, opacity=0.5] (0.9820, -0.0872, 2.0506) -- (1.0291, -0.0872, 2.0539) -- (1.0300, -0.0900, 2.1039) -- (0.9830, -0.0900, 2.1006) -- cycle;
\fill[blue!43.5, opacity=0.5] (0.9830, -0.0900, 2.1006) -- (1.0300, -0.0900, 2.1039) -- (1.0309, -0.0927, 2.1539) -- (0.9840, -0.0927, 2.1506) -- cycle;
\fill[blue!45.7, opacity=0.5] (0.9840, -0.0927, 2.1506) -- (1.0309, -0.0927, 2.1539) -- (1.0318, -0.0953, 2.2039) -- (0.9849, -0.0953, 2.2006) -- cycle;
\fill[blue!47.8, opacity=0.5] (0.9849, -0.0953, 2.2006) -- (1.0318, -0.0953, 2.2039) -- (1.0326, -0.0978, 2.2539) -- (0.9858, -0.0978, 2.2506) -- cycle;
\fill[blue!49.9, opacity=0.5] (0.9858, -0.0978, 2.2506) -- (1.0326, -0.0978, 2.2539) -- (1.0334, -0.1001, 2.3039) -- (0.9867, -0.1001, 2.3006) -- cycle;
\fill[blue!51.8, opacity=0.5] (0.9867, -0.1001, 2.3006) -- (1.0334, -0.1001, 2.3039) -- (1.0341, -0.1024, 2.3539) -- (0.9876, -0.1024, 2.3506) -- cycle;
\fill[blue!53.6, opacity=0.5] (0.9876, -0.1024, 2.3506) -- (1.0341, -0.1024, 2.3539) -- (1.0349, -0.1046, 2.4039) -- (0.9883, -0.1046, 2.4006) -- cycle;
\fill[blue!55.2, opacity=0.5] (0.9883, -0.1046, 2.4006) -- (1.0349, -0.1046, 2.4039) -- (1.0355, -0.1066, 2.4539) -- (0.9891, -0.1066, 2.4506) -- cycle;
\fill[blue!56.7, opacity=0.5] (0.9891, -0.1066, 2.4506) -- (1.0355, -0.1066, 2.4539) -- (1.0362, -0.1085, 2.5039) -- (0.9898, -0.1085, 2.5006) -- cycle;
\fill[blue!58.1, opacity=0.5] (0.9898, -0.1085, 2.5006) -- (1.0362, -0.1085, 2.5039) -- (1.0368, -0.1103, 2.5539) -- (0.9905, -0.1103, 2.5506) -- cycle;
\fill[blue!59.2, opacity=0.5] (0.9905, -0.1103, 2.5506) -- (1.0368, -0.1103, 2.5539) -- (1.0373, -0.1120, 2.6039) -- (0.9911, -0.1120, 2.6006) -- cycle;
\fill[blue!60.2, opacity=0.5] (0.9911, -0.1120, 2.6006) -- (1.0373, -0.1120, 2.6039) -- (1.0378, -0.1135, 2.6539) -- (0.9916, -0.1135, 2.6506) -- cycle;
\fill[blue!61.0, opacity=0.5] (0.9916, -0.1135, 2.6506) -- (1.0378, -0.1135, 2.6539) -- (1.0383, -0.1148, 2.7039) -- (0.9921, -0.1148, 2.7006) -- cycle;
\fill[blue!61.7, opacity=0.5] (0.9921, -0.1148, 2.7006) -- (1.0383, -0.1148, 2.7039) -- (1.0387, -0.1160, 2.7539) -- (0.9925, -0.1160, 2.7506) -- cycle;
\fill[blue!62.1, opacity=0.5] (0.9925, -0.1160, 2.7506) -- (1.0387, -0.1160, 2.7539) -- (1.0390, -0.1171, 2.8039) -- (0.9929, -0.1171, 2.8006) -- cycle;
\fill[blue!62.4, opacity=0.5] (0.9929, -0.1171, 2.8006) -- (1.0390, -0.1171, 2.8039) -- (1.0393, -0.1180, 2.8539) -- (0.9933, -0.1180, 2.8506) -- cycle;
\fill[blue!62.6, opacity=0.5] (0.9933, -0.1180, 2.8506) -- (1.0393, -0.1180, 2.8539) -- (1.0396, -0.1187, 2.9039) -- (0.9935, -0.1187, 2.9006) -- cycle;
\fill[blue!62.6, opacity=0.5] (0.9935, -0.1187, 2.9006) -- (1.0396, -0.1187, 2.9039) -- (1.0398, -0.1193, 2.9539) -- (0.9937, -0.1193, 2.9506) -- cycle;
\fill[blue!62.4, opacity=0.5] (0.9937, -0.1193, 2.9506) -- (1.0398, -0.1193, 2.9539) -- (1.0399, -0.1197, 3.0039) -- (0.9939, -0.1197, 3.0006) -- cycle;
\fill[blue!62.1, opacity=0.5] (0.9939, -0.1197, 3.0006) -- (1.0399, -0.1197, 3.0039) -- (1.0400, -0.1199, 3.0539) -- (0.9940, -0.1199, 3.0506) -- cycle;
\fill[blue!61.7, opacity=0.5] (0.9940, -0.1199, 3.0506) -- (1.0400, -0.1199, 3.0539) -- (1.0400, -0.1200, 3.1039) -- (0.9940, -0.1200, 3.1006) -- cycle;
\fill[blue!15.0, opacity=0.5] (1.0000, -0.0000, 0.1039) -- (1.0500, -0.0000, 0.1069) -- (1.0500, -0.0001, 0.1569) -- (1.0000, -0.0001, 0.1539) -- cycle;
\fill[blue!15.0, opacity=0.5] (1.0000, -0.0001, 0.1539) -- (1.0500, -0.0001, 0.1569) -- (1.0501, -0.0003, 0.2069) -- (1.0001, -0.0003, 0.2039) -- cycle;
\fill[blue!15.0, opacity=0.5] (1.0001, -0.0003, 0.2039) -- (1.0501, -0.0003, 0.2069) -- (1.0502, -0.0007, 0.2569) -- (1.0002, -0.0007, 0.2539) -- cycle;
\fill[blue!15.0, opacity=0.5] (1.0002, -0.0007, 0.2539) -- (1.0502, -0.0007, 0.2569) -- (1.0504, -0.0013, 0.3069) -- (1.0004, -0.0013, 0.3039) -- cycle;
\fill[blue!15.0, opacity=0.5] (1.0004, -0.0013, 0.3039) -- (1.0504, -0.0013, 0.3069) -- (1.0506, -0.0020, 0.3569) -- (1.0007, -0.0020, 0.3539) -- cycle;
\fill[blue!15.0, opacity=0.5] (1.0007, -0.0020, 0.3539) -- (1.0506, -0.0020, 0.3569) -- (1.0509, -0.0029, 0.4069) -- (1.0010, -0.0029, 0.4039) -- cycle;
\fill[blue!15.0, opacity=0.5] (1.0010, -0.0029, 0.4039) -- (1.0509, -0.0029, 0.4069) -- (1.0512, -0.0040, 0.4569) -- (1.0013, -0.0040, 0.4539) -- cycle;
\fill[blue!15.0, opacity=0.5] (1.0013, -0.0040, 0.4539) -- (1.0512, -0.0040, 0.4569) -- (1.0516, -0.0052, 0.5069) -- (1.0017, -0.0052, 0.5039) -- cycle;
\fill[blue!15.0, opacity=0.5] (1.0017, -0.0052, 0.5039) -- (1.0516, -0.0052, 0.5069) -- (1.0520, -0.0065, 0.5569) -- (1.0022, -0.0065, 0.5539) -- cycle;
\fill[blue!15.0, opacity=0.5] (1.0022, -0.0065, 0.5539) -- (1.0520, -0.0065, 0.5569) -- (1.0524, -0.0080, 0.6069) -- (1.0027, -0.0080, 0.6039) -- cycle;
\fill[blue!15.0, opacity=0.5] (1.0027, -0.0080, 0.6039) -- (1.0524, -0.0080, 0.6069) -- (1.0529, -0.0097, 0.6569) -- (1.0032, -0.0097, 0.6539) -- cycle;
\fill[blue!15.0, opacity=0.5] (1.0032, -0.0097, 0.6539) -- (1.0529, -0.0097, 0.6569) -- (1.0534, -0.0115, 0.7069) -- (1.0038, -0.0115, 0.7039) -- cycle;
\fill[blue!15.0, opacity=0.5] (1.0038, -0.0115, 0.7039) -- (1.0534, -0.0115, 0.7069) -- (1.0540, -0.0134, 0.7569) -- (1.0045, -0.0134, 0.7539) -- cycle;
\fill[blue!15.0, opacity=0.5] (1.0045, -0.0134, 0.7539) -- (1.0540, -0.0134, 0.7569) -- (1.0546, -0.0154, 0.8069) -- (1.0051, -0.0154, 0.8039) -- cycle;
\fill[blue!15.0, opacity=0.5] (1.0051, -0.0154, 0.8039) -- (1.0546, -0.0154, 0.8069) -- (1.0553, -0.0176, 0.8569) -- (1.0059, -0.0176, 0.8539) -- cycle;
\fill[blue!15.0, opacity=0.5] (1.0059, -0.0176, 0.8539) -- (1.0553, -0.0176, 0.8569) -- (1.0560, -0.0199, 0.9069) -- (1.0066, -0.0199, 0.9039) -- cycle;
\fill[blue!15.0, opacity=0.5] (1.0066, -0.0199, 0.9039) -- (1.0560, -0.0199, 0.9069) -- (1.0567, -0.0222, 0.9569) -- (1.0074, -0.0222, 0.9539) -- cycle;
\fill[blue!15.1, opacity=0.5] (1.0074, -0.0222, 0.9539) -- (1.0567, -0.0222, 0.9569) -- (1.0574, -0.0247, 1.0069) -- (1.0082, -0.0247, 1.0039) -- cycle;
\fill[blue!15.1, opacity=0.5] (1.0082, -0.0247, 1.0039) -- (1.0574, -0.0247, 1.0069) -- (1.0582, -0.0273, 1.0569) -- (1.0091, -0.0273, 1.0539) -- cycle;
\fill[blue!15.2, opacity=0.5] (1.0091, -0.0273, 1.0539) -- (1.0582, -0.0273, 1.0569) -- (1.0590, -0.0300, 1.1069) -- (1.0100, -0.0300, 1.1039) -- cycle;
\fill[blue!15.3, opacity=0.5] (1.0100, -0.0300, 1.1039) -- (1.0590, -0.0300, 1.1069) -- (1.0598, -0.0328, 1.1569) -- (1.0109, -0.0328, 1.1539) -- cycle;
\fill[blue!15.4, opacity=0.5] (1.0109, -0.0328, 1.1539) -- (1.0598, -0.0328, 1.1569) -- (1.0607, -0.0356, 1.2069) -- (1.0119, -0.0356, 1.2039) -- cycle;
\fill[blue!15.7, opacity=0.5] (1.0119, -0.0356, 1.2039) -- (1.0607, -0.0356, 1.2069) -- (1.0615, -0.0385, 1.2569) -- (1.0128, -0.0385, 1.2539) -- cycle;
\fill[blue!16.0, opacity=0.5] (1.0128, -0.0385, 1.2539) -- (1.0615, -0.0385, 1.2569) -- (1.0624, -0.0415, 1.3069) -- (1.0138, -0.0415, 1.3039) -- cycle;
\fill[blue!16.4, opacity=0.5] (1.0138, -0.0415, 1.3039) -- (1.0624, -0.0415, 1.3069) -- (1.0633, -0.0445, 1.3569) -- (1.0148, -0.0445, 1.3539) -- cycle;
\fill[blue!16.9, opacity=0.5] (1.0148, -0.0445, 1.3539) -- (1.0633, -0.0445, 1.3569) -- (1.0643, -0.0475, 1.4069) -- (1.0158, -0.0475, 1.4039) -- cycle;
\fill[blue!17.5, opacity=0.5] (1.0158, -0.0475, 1.4039) -- (1.0643, -0.0475, 1.4069) -- (1.0652, -0.0506, 1.4569) -- (1.0169, -0.0506, 1.4539) -- cycle;
\fill[blue!18.3, opacity=0.5] (1.0169, -0.0506, 1.4539) -- (1.0652, -0.0506, 1.4569) -- (1.0661, -0.0537, 1.5069) -- (1.0179, -0.0537, 1.5039) -- cycle;
\fill[blue!19.2, opacity=0.5] (1.0179, -0.0537, 1.5039) -- (1.0661, -0.0537, 1.5069) -- (1.0671, -0.0569, 1.5569) -- (1.0190, -0.0569, 1.5539) -- cycle;
\fill[blue!20.3, opacity=0.5] (1.0190, -0.0569, 1.5539) -- (1.0671, -0.0569, 1.5569) -- (1.0680, -0.0600, 1.6069) -- (1.0200, -0.0600, 1.6039) -- cycle;
\fill[blue!21.5, opacity=0.5] (1.0200, -0.0600, 1.6039) -- (1.0680, -0.0600, 1.6069) -- (1.0689, -0.0631, 1.6569) -- (1.0210, -0.0631, 1.6539) -- cycle;
\fill[blue!22.9, opacity=0.5] (1.0210, -0.0631, 1.6539) -- (1.0689, -0.0631, 1.6569) -- (1.0699, -0.0663, 1.7069) -- (1.0221, -0.0663, 1.7039) -- cycle;
\fill[blue!24.5, opacity=0.5] (1.0221, -0.0663, 1.7039) -- (1.0699, -0.0663, 1.7069) -- (1.0708, -0.0694, 1.7569) -- (1.0231, -0.0694, 1.7539) -- cycle;
\fill[blue!26.2, opacity=0.5] (1.0231, -0.0694, 1.7539) -- (1.0708, -0.0694, 1.7569) -- (1.0717, -0.0725, 1.8069) -- (1.0242, -0.0725, 1.8039) -- cycle;
\fill[blue!28.0, opacity=0.5] (1.0242, -0.0725, 1.8039) -- (1.0717, -0.0725, 1.8069) -- (1.0727, -0.0755, 1.8569) -- (1.0252, -0.0755, 1.8539) -- cycle;
\fill[blue!30.0, opacity=0.5] (1.0252, -0.0755, 1.8539) -- (1.0727, -0.0755, 1.8569) -- (1.0736, -0.0785, 1.9069) -- (1.0262, -0.0785, 1.9039) -- cycle;
\fill[blue!32.0, opacity=0.5] (1.0262, -0.0785, 1.9039) -- (1.0736, -0.0785, 1.9069) -- (1.0745, -0.0815, 1.9569) -- (1.0272, -0.0815, 1.9539) -- cycle;
\fill[blue!34.1, opacity=0.5] (1.0272, -0.0815, 1.9539) -- (1.0745, -0.0815, 1.9569) -- (1.0753, -0.0844, 2.0069) -- (1.0281, -0.0844, 2.0039) -- cycle;
\fill[blue!36.3, opacity=0.5] (1.0281, -0.0844, 2.0039) -- (1.0753, -0.0844, 2.0069) -- (1.0762, -0.0872, 2.0569) -- (1.0291, -0.0872, 2.0539) -- cycle;
\fill[blue!38.4, opacity=0.5] (1.0291, -0.0872, 2.0539) -- (1.0762, -0.0872, 2.0569) -- (1.0770, -0.0900, 2.1069) -- (1.0300, -0.0900, 2.1039) -- cycle;
\fill[blue!40.6, opacity=0.5] (1.0300, -0.0900, 2.1039) -- (1.0770, -0.0900, 2.1069) -- (1.0778, -0.0927, 2.1569) -- (1.0309, -0.0927, 2.1539) -- cycle;
\fill[blue!42.8, opacity=0.5] (1.0309, -0.0927, 2.1539) -- (1.0778, -0.0927, 2.1569) -- (1.0786, -0.0953, 2.2069) -- (1.0318, -0.0953, 2.2039) -- cycle;
\fill[blue!44.9, opacity=0.5] (1.0318, -0.0953, 2.2039) -- (1.0786, -0.0953, 2.2069) -- (1.0793, -0.0978, 2.2569) -- (1.0326, -0.0978, 2.2539) -- cycle;
\fill[blue!46.9, opacity=0.5] (1.0326, -0.0978, 2.2539) -- (1.0793, -0.0978, 2.2569) -- (1.0800, -0.1001, 2.3069) -- (1.0334, -0.1001, 2.3039) -- cycle;
\fill[blue!48.9, opacity=0.5] (1.0334, -0.1001, 2.3039) -- (1.0800, -0.1001, 2.3069) -- (1.0807, -0.1024, 2.3569) -- (1.0341, -0.1024, 2.3539) -- cycle;
\fill[blue!50.7, opacity=0.5] (1.0341, -0.1024, 2.3539) -- (1.0807, -0.1024, 2.3569) -- (1.0814, -0.1046, 2.4069) -- (1.0349, -0.1046, 2.4039) -- cycle;
\fill[blue!52.4, opacity=0.5] (1.0349, -0.1046, 2.4039) -- (1.0814, -0.1046, 2.4069) -- (1.0820, -0.1066, 2.4569) -- (1.0355, -0.1066, 2.4539) -- cycle;
\fill[blue!54.0, opacity=0.5] (1.0355, -0.1066, 2.4539) -- (1.0820, -0.1066, 2.4569) -- (1.0826, -0.1085, 2.5069) -- (1.0362, -0.1085, 2.5039) -- cycle;
\fill[blue!55.5, opacity=0.5] (1.0362, -0.1085, 2.5039) -- (1.0826, -0.1085, 2.5069) -- (1.0831, -0.1103, 2.5569) -- (1.0368, -0.1103, 2.5539) -- cycle;
\fill[blue!56.8, opacity=0.5] (1.0368, -0.1103, 2.5539) -- (1.0831, -0.1103, 2.5569) -- (1.0836, -0.1120, 2.6069) -- (1.0373, -0.1120, 2.6039) -- cycle;
\fill[blue!57.9, opacity=0.5] (1.0373, -0.1120, 2.6039) -- (1.0836, -0.1120, 2.6069) -- (1.0840, -0.1135, 2.6569) -- (1.0378, -0.1135, 2.6539) -- cycle;
\fill[blue!58.8, opacity=0.5] (1.0378, -0.1135, 2.6539) -- (1.0840, -0.1135, 2.6569) -- (1.0844, -0.1148, 2.7069) -- (1.0383, -0.1148, 2.7039) -- cycle;
\fill[blue!59.6, opacity=0.5] (1.0383, -0.1148, 2.7039) -- (1.0844, -0.1148, 2.7069) -- (1.0848, -0.1160, 2.7569) -- (1.0387, -0.1160, 2.7539) -- cycle;
\fill[blue!60.3, opacity=0.5] (1.0387, -0.1160, 2.7539) -- (1.0848, -0.1160, 2.7569) -- (1.0851, -0.1171, 2.8069) -- (1.0390, -0.1171, 2.8039) -- cycle;
\fill[blue!60.7, opacity=0.5] (1.0390, -0.1171, 2.8039) -- (1.0851, -0.1171, 2.8069) -- (1.0854, -0.1180, 2.8569) -- (1.0393, -0.1180, 2.8539) -- cycle;
\fill[blue!61.0, opacity=0.5] (1.0393, -0.1180, 2.8539) -- (1.0854, -0.1180, 2.8569) -- (1.0856, -0.1187, 2.9069) -- (1.0396, -0.1187, 2.9039) -- cycle;
\fill[blue!61.1, opacity=0.5] (1.0396, -0.1187, 2.9039) -- (1.0856, -0.1187, 2.9069) -- (1.0858, -0.1193, 2.9569) -- (1.0398, -0.1193, 2.9539) -- cycle;
\fill[blue!61.1, opacity=0.5] (1.0398, -0.1193, 2.9539) -- (1.0858, -0.1193, 2.9569) -- (1.0859, -0.1197, 3.0069) -- (1.0399, -0.1197, 3.0039) -- cycle;
\fill[blue!61.0, opacity=0.5] (1.0399, -0.1197, 3.0039) -- (1.0859, -0.1197, 3.0069) -- (1.0860, -0.1199, 3.0569) -- (1.0400, -0.1199, 3.0539) -- cycle;
\fill[blue!60.7, opacity=0.5] (1.0400, -0.1199, 3.0539) -- (1.0860, -0.1199, 3.0569) -- (1.0860, -0.1200, 3.1069) -- (1.0400, -0.1200, 3.1039) -- cycle;
\fill[blue!15.0, opacity=0.5] (1.0500, -0.0000, 0.1069) -- (1.1000, -0.0000, 0.1096) -- (1.1000, -0.0001, 0.1596) -- (1.0500, -0.0001, 0.1569) -- cycle;
\fill[blue!15.0, opacity=0.5] (1.0500, -0.0001, 0.1569) -- (1.1000, -0.0001, 0.1596) -- (1.1001, -0.0003, 0.2096) -- (1.0501, -0.0003, 0.2069) -- cycle;
\fill[blue!15.0, opacity=0.5] (1.0501, -0.0003, 0.2069) -- (1.1001, -0.0003, 0.2096) -- (1.1002, -0.0007, 0.2596) -- (1.0502, -0.0007, 0.2569) -- cycle;
\fill[blue!15.0, opacity=0.5] (1.0502, -0.0007, 0.2569) -- (1.1002, -0.0007, 0.2596) -- (1.1003, -0.0013, 0.3096) -- (1.0504, -0.0013, 0.3069) -- cycle;
\fill[blue!15.0, opacity=0.5] (1.0504, -0.0013, 0.3069) -- (1.1003, -0.0013, 0.3096) -- (1.1005, -0.0020, 0.3596) -- (1.0506, -0.0020, 0.3569) -- cycle;
\fill[blue!15.0, opacity=0.5] (1.0506, -0.0020, 0.3569) -- (1.1005, -0.0020, 0.3596) -- (1.1008, -0.0029, 0.4096) -- (1.0509, -0.0029, 0.4069) -- cycle;
\fill[blue!15.0, opacity=0.5] (1.0509, -0.0029, 0.4069) -- (1.1008, -0.0029, 0.4096) -- (1.1011, -0.0040, 0.4596) -- (1.0512, -0.0040, 0.4569) -- cycle;
\fill[blue!15.0, opacity=0.5] (1.0512, -0.0040, 0.4569) -- (1.1011, -0.0040, 0.4596) -- (1.1014, -0.0052, 0.5096) -- (1.0516, -0.0052, 0.5069) -- cycle;
\fill[blue!15.0, opacity=0.5] (1.0516, -0.0052, 0.5069) -- (1.1014, -0.0052, 0.5096) -- (1.1017, -0.0065, 0.5596) -- (1.0520, -0.0065, 0.5569) -- cycle;
\fill[blue!15.0, opacity=0.5] (1.0520, -0.0065, 0.5569) -- (1.1017, -0.0065, 0.5596) -- (1.1021, -0.0080, 0.6096) -- (1.0524, -0.0080, 0.6069) -- cycle;
\fill[blue!15.0, opacity=0.5] (1.0524, -0.0080, 0.6069) -- (1.1021, -0.0080, 0.6096) -- (1.1026, -0.0097, 0.6596) -- (1.0529, -0.0097, 0.6569) -- cycle;
\fill[blue!15.0, opacity=0.5] (1.0529, -0.0097, 0.6569) -- (1.1026, -0.0097, 0.6596) -- (1.1031, -0.0115, 0.7096) -- (1.0534, -0.0115, 0.7069) -- cycle;
\fill[blue!15.0, opacity=0.5] (1.0534, -0.0115, 0.7069) -- (1.1031, -0.0115, 0.7096) -- (1.1036, -0.0134, 0.7596) -- (1.0540, -0.0134, 0.7569) -- cycle;
\fill[blue!15.0, opacity=0.5] (1.0540, -0.0134, 0.7569) -- (1.1036, -0.0134, 0.7596) -- (1.1041, -0.0154, 0.8096) -- (1.0546, -0.0154, 0.8069) -- cycle;
\fill[blue!15.0, opacity=0.5] (1.0546, -0.0154, 0.8069) -- (1.1041, -0.0154, 0.8096) -- (1.1047, -0.0176, 0.8596) -- (1.0553, -0.0176, 0.8569) -- cycle;
\fill[blue!15.0, opacity=0.5] (1.0553, -0.0176, 0.8569) -- (1.1047, -0.0176, 0.8596) -- (1.1053, -0.0199, 0.9096) -- (1.0560, -0.0199, 0.9069) -- cycle;
\fill[blue!15.0, opacity=0.5] (1.0560, -0.0199, 0.9069) -- (1.1053, -0.0199, 0.9096) -- (1.1059, -0.0222, 0.9596) -- (1.0567, -0.0222, 0.9569) -- cycle;
\fill[blue!15.0, opacity=0.5] (1.0567, -0.0222, 0.9569) -- (1.1059, -0.0222, 0.9596) -- (1.1066, -0.0247, 1.0096) -- (1.0574, -0.0247, 1.0069) -- cycle;
\fill[blue!15.1, opacity=0.5] (1.0574, -0.0247, 1.0069) -- (1.1066, -0.0247, 1.0096) -- (1.1073, -0.0273, 1.0596) -- (1.0582, -0.0273, 1.0569) -- cycle;
\fill[blue!15.1, opacity=0.5] (1.0582, -0.0273, 1.0569) -- (1.1073, -0.0273, 1.0596) -- (1.1080, -0.0300, 1.1096) -- (1.0590, -0.0300, 1.1069) -- cycle;
\fill[blue!15.2, opacity=0.5] (1.0590, -0.0300, 1.1069) -- (1.1080, -0.0300, 1.1096) -- (1.1087, -0.0328, 1.1596) -- (1.0598, -0.0328, 1.1569) -- cycle;
\fill[blue!15.3, opacity=0.5] (1.0598, -0.0328, 1.1569) -- (1.1087, -0.0328, 1.1596) -- (1.1095, -0.0356, 1.2096) -- (1.0607, -0.0356, 1.2069) -- cycle;
\fill[blue!15.4, opacity=0.5] (1.0607, -0.0356, 1.2069) -- (1.1095, -0.0356, 1.2096) -- (1.1103, -0.0385, 1.2596) -- (1.0615, -0.0385, 1.2569) -- cycle;
\fill[blue!15.7, opacity=0.5] (1.0615, -0.0385, 1.2569) -- (1.1103, -0.0385, 1.2596) -- (1.1111, -0.0415, 1.3096) -- (1.0624, -0.0415, 1.3069) -- cycle;
\fill[blue!15.9, opacity=0.5] (1.0624, -0.0415, 1.3069) -- (1.1111, -0.0415, 1.3096) -- (1.1119, -0.0445, 1.3596) -- (1.0633, -0.0445, 1.3569) -- cycle;
\fill[blue!16.3, opacity=0.5] (1.0633, -0.0445, 1.3569) -- (1.1119, -0.0445, 1.3596) -- (1.1127, -0.0475, 1.4096) -- (1.0643, -0.0475, 1.4069) -- cycle;
\fill[blue!16.8, opacity=0.5] (1.0643, -0.0475, 1.4069) -- (1.1127, -0.0475, 1.4096) -- (1.1135, -0.0506, 1.4596) -- (1.0652, -0.0506, 1.4569) -- cycle;
\fill[blue!17.4, opacity=0.5] (1.0652, -0.0506, 1.4569) -- (1.1135, -0.0506, 1.4596) -- (1.1143, -0.0537, 1.5096) -- (1.0661, -0.0537, 1.5069) -- cycle;
\fill[blue!18.1, opacity=0.5] (1.0661, -0.0537, 1.5069) -- (1.1143, -0.0537, 1.5096) -- (1.1152, -0.0569, 1.5596) -- (1.0671, -0.0569, 1.5569) -- cycle;
\fill[blue!19.0, opacity=0.5] (1.0671, -0.0569, 1.5569) -- (1.1152, -0.0569, 1.5596) -- (1.1160, -0.0600, 1.6096) -- (1.0680, -0.0600, 1.6069) -- cycle;
\fill[blue!20.0, opacity=0.5] (1.0680, -0.0600, 1.6069) -- (1.1160, -0.0600, 1.6096) -- (1.1168, -0.0631, 1.6596) -- (1.0689, -0.0631, 1.6569) -- cycle;
\fill[blue!21.2, opacity=0.5] (1.0689, -0.0631, 1.6569) -- (1.1168, -0.0631, 1.6596) -- (1.1177, -0.0663, 1.7096) -- (1.0699, -0.0663, 1.7069) -- cycle;
\fill[blue!22.5, opacity=0.5] (1.0699, -0.0663, 1.7069) -- (1.1177, -0.0663, 1.7096) -- (1.1185, -0.0694, 1.7596) -- (1.0708, -0.0694, 1.7569) -- cycle;
\fill[blue!24.0, opacity=0.5] (1.0708, -0.0694, 1.7569) -- (1.1185, -0.0694, 1.7596) -- (1.1193, -0.0725, 1.8096) -- (1.0717, -0.0725, 1.8069) -- cycle;
\fill[blue!25.6, opacity=0.5] (1.0717, -0.0725, 1.8069) -- (1.1193, -0.0725, 1.8096) -- (1.1201, -0.0755, 1.8596) -- (1.0727, -0.0755, 1.8569) -- cycle;
\fill[blue!27.3, opacity=0.5] (1.0727, -0.0755, 1.8569) -- (1.1201, -0.0755, 1.8596) -- (1.1209, -0.0785, 1.9096) -- (1.0736, -0.0785, 1.9069) -- cycle;
\fill[blue!29.1, opacity=0.5] (1.0736, -0.0785, 1.9069) -- (1.1209, -0.0785, 1.9096) -- (1.1217, -0.0815, 1.9596) -- (1.0745, -0.0815, 1.9569) -- cycle;
\fill[blue!31.0, opacity=0.5] (1.0745, -0.0815, 1.9569) -- (1.1217, -0.0815, 1.9596) -- (1.1225, -0.0844, 2.0096) -- (1.0753, -0.0844, 2.0069) -- cycle;
\fill[blue!33.0, opacity=0.5] (1.0753, -0.0844, 2.0069) -- (1.1225, -0.0844, 2.0096) -- (1.1233, -0.0872, 2.0596) -- (1.0762, -0.0872, 2.0569) -- cycle;
\fill[blue!35.1, opacity=0.5] (1.0762, -0.0872, 2.0569) -- (1.1233, -0.0872, 2.0596) -- (1.1240, -0.0900, 2.1096) -- (1.0770, -0.0900, 2.1069) -- cycle;
\fill[blue!37.2, opacity=0.5] (1.0770, -0.0900, 2.1069) -- (1.1240, -0.0900, 2.1096) -- (1.1247, -0.0927, 2.1596) -- (1.0778, -0.0927, 2.1569) -- cycle;
\fill[blue!39.2, opacity=0.5] (1.0778, -0.0927, 2.1569) -- (1.1247, -0.0927, 2.1596) -- (1.1254, -0.0953, 2.2096) -- (1.0786, -0.0953, 2.2069) -- cycle;
\fill[blue!41.3, opacity=0.5] (1.0786, -0.0953, 2.2069) -- (1.1254, -0.0953, 2.2096) -- (1.1261, -0.0978, 2.2596) -- (1.0793, -0.0978, 2.2569) -- cycle;
\fill[blue!43.3, opacity=0.5] (1.0793, -0.0978, 2.2569) -- (1.1261, -0.0978, 2.2596) -- (1.1267, -0.1001, 2.3096) -- (1.0800, -0.1001, 2.3069) -- cycle;
\fill[blue!45.3, opacity=0.5] (1.0800, -0.1001, 2.3069) -- (1.1267, -0.1001, 2.3096) -- (1.1273, -0.1024, 2.3596) -- (1.0807, -0.1024, 2.3569) -- cycle;
\fill[blue!47.2, opacity=0.5] (1.0807, -0.1024, 2.3569) -- (1.1273, -0.1024, 2.3596) -- (1.1279, -0.1046, 2.4096) -- (1.0814, -0.1046, 2.4069) -- cycle;
\fill[blue!48.9, opacity=0.5] (1.0814, -0.1046, 2.4069) -- (1.1279, -0.1046, 2.4096) -- (1.1284, -0.1066, 2.4596) -- (1.0820, -0.1066, 2.4569) -- cycle;
\fill[blue!50.6, opacity=0.5] (1.0820, -0.1066, 2.4569) -- (1.1284, -0.1066, 2.4596) -- (1.1289, -0.1085, 2.5096) -- (1.0826, -0.1085, 2.5069) -- cycle;
\fill[blue!52.2, opacity=0.5] (1.0826, -0.1085, 2.5069) -- (1.1289, -0.1085, 2.5096) -- (1.1294, -0.1103, 2.5596) -- (1.0831, -0.1103, 2.5569) -- cycle;
\fill[blue!53.6, opacity=0.5] (1.0831, -0.1103, 2.5569) -- (1.1294, -0.1103, 2.5596) -- (1.1299, -0.1120, 2.6096) -- (1.0836, -0.1120, 2.6069) -- cycle;
\fill[blue!54.8, opacity=0.5] (1.0836, -0.1120, 2.6069) -- (1.1299, -0.1120, 2.6096) -- (1.1303, -0.1135, 2.6596) -- (1.0840, -0.1135, 2.6569) -- cycle;
\fill[blue!55.9, opacity=0.5] (1.0840, -0.1135, 2.6569) -- (1.1303, -0.1135, 2.6596) -- (1.1306, -0.1148, 2.7096) -- (1.0844, -0.1148, 2.7069) -- cycle;
\fill[blue!56.9, opacity=0.5] (1.0844, -0.1148, 2.7069) -- (1.1306, -0.1148, 2.7096) -- (1.1309, -0.1160, 2.7596) -- (1.0848, -0.1160, 2.7569) -- cycle;
\fill[blue!57.7, opacity=0.5] (1.0848, -0.1160, 2.7569) -- (1.1309, -0.1160, 2.7596) -- (1.1312, -0.1171, 2.8096) -- (1.0851, -0.1171, 2.8069) -- cycle;
\fill[blue!58.3, opacity=0.5] (1.0851, -0.1171, 2.8069) -- (1.1312, -0.1171, 2.8096) -- (1.1315, -0.1180, 2.8596) -- (1.0854, -0.1180, 2.8569) -- cycle;
\fill[blue!58.8, opacity=0.5] (1.0854, -0.1180, 2.8569) -- (1.1315, -0.1180, 2.8596) -- (1.1317, -0.1187, 2.9096) -- (1.0856, -0.1187, 2.9069) -- cycle;
\fill[blue!59.1, opacity=0.5] (1.0856, -0.1187, 2.9069) -- (1.1317, -0.1187, 2.9096) -- (1.1318, -0.1193, 2.9596) -- (1.0858, -0.1193, 2.9569) -- cycle;
\fill[blue!59.3, opacity=0.5] (1.0858, -0.1193, 2.9569) -- (1.1318, -0.1193, 2.9596) -- (1.1319, -0.1197, 3.0096) -- (1.0859, -0.1197, 3.0069) -- cycle;
\fill[blue!59.3, opacity=0.5] (1.0859, -0.1197, 3.0069) -- (1.1319, -0.1197, 3.0096) -- (1.1320, -0.1199, 3.0596) -- (1.0860, -0.1199, 3.0569) -- cycle;
\fill[blue!59.2, opacity=0.5] (1.0860, -0.1199, 3.0569) -- (1.1320, -0.1199, 3.0596) -- (1.1320, -0.1200, 3.1096) -- (1.0860, -0.1200, 3.1069) -- cycle;
\fill[blue!15.0, opacity=0.5] (1.1000, -0.0000, 0.1096) -- (1.1500, -0.0000, 0.1120) -- (1.1500, -0.0001, 0.1620) -- (1.1000, -0.0001, 0.1596) -- cycle;
\fill[blue!15.0, opacity=0.5] (1.1000, -0.0001, 0.1596) -- (1.1500, -0.0001, 0.1620) -- (1.1501, -0.0003, 0.2120) -- (1.1001, -0.0003, 0.2096) -- cycle;
\fill[blue!15.0, opacity=0.5] (1.1001, -0.0003, 0.2096) -- (1.1501, -0.0003, 0.2120) -- (1.1502, -0.0007, 0.2620) -- (1.1002, -0.0007, 0.2596) -- cycle;
\fill[blue!15.0, opacity=0.5] (1.1002, -0.0007, 0.2596) -- (1.1502, -0.0007, 0.2620) -- (1.1503, -0.0013, 0.3120) -- (1.1003, -0.0013, 0.3096) -- cycle;
\fill[blue!15.0, opacity=0.5] (1.1003, -0.0013, 0.3096) -- (1.1503, -0.0013, 0.3120) -- (1.1505, -0.0020, 0.3620) -- (1.1005, -0.0020, 0.3596) -- cycle;
\fill[blue!15.0, opacity=0.5] (1.1005, -0.0020, 0.3596) -- (1.1505, -0.0020, 0.3620) -- (1.1507, -0.0029, 0.4120) -- (1.1008, -0.0029, 0.4096) -- cycle;
\fill[blue!15.0, opacity=0.5] (1.1008, -0.0029, 0.4096) -- (1.1507, -0.0029, 0.4120) -- (1.1509, -0.0040, 0.4620) -- (1.1011, -0.0040, 0.4596) -- cycle;
\fill[blue!15.0, opacity=0.5] (1.1011, -0.0040, 0.4596) -- (1.1509, -0.0040, 0.4620) -- (1.1512, -0.0052, 0.5120) -- (1.1014, -0.0052, 0.5096) -- cycle;
\fill[blue!15.0, opacity=0.5] (1.1014, -0.0052, 0.5096) -- (1.1512, -0.0052, 0.5120) -- (1.1515, -0.0065, 0.5620) -- (1.1017, -0.0065, 0.5596) -- cycle;
\fill[blue!15.0, opacity=0.5] (1.1017, -0.0065, 0.5596) -- (1.1515, -0.0065, 0.5620) -- (1.1519, -0.0080, 0.6120) -- (1.1021, -0.0080, 0.6096) -- cycle;
\fill[blue!15.0, opacity=0.5] (1.1021, -0.0080, 0.6096) -- (1.1519, -0.0080, 0.6120) -- (1.1523, -0.0097, 0.6620) -- (1.1026, -0.0097, 0.6596) -- cycle;
\fill[blue!15.0, opacity=0.5] (1.1026, -0.0097, 0.6596) -- (1.1523, -0.0097, 0.6620) -- (1.1527, -0.0115, 0.7120) -- (1.1031, -0.0115, 0.7096) -- cycle;
\fill[blue!15.0, opacity=0.5] (1.1031, -0.0115, 0.7096) -- (1.1527, -0.0115, 0.7120) -- (1.1531, -0.0134, 0.7620) -- (1.1036, -0.0134, 0.7596) -- cycle;
\fill[blue!15.0, opacity=0.5] (1.1036, -0.0134, 0.7596) -- (1.1531, -0.0134, 0.7620) -- (1.1536, -0.0154, 0.8120) -- (1.1041, -0.0154, 0.8096) -- cycle;
\fill[blue!15.0, opacity=0.5] (1.1041, -0.0154, 0.8096) -- (1.1536, -0.0154, 0.8120) -- (1.1541, -0.0176, 0.8620) -- (1.1047, -0.0176, 0.8596) -- cycle;
\fill[blue!15.0, opacity=0.5] (1.1047, -0.0176, 0.8596) -- (1.1541, -0.0176, 0.8620) -- (1.1546, -0.0199, 0.9120) -- (1.1053, -0.0199, 0.9096) -- cycle;
\fill[blue!15.0, opacity=0.5] (1.1053, -0.0199, 0.9096) -- (1.1546, -0.0199, 0.9120) -- (1.1552, -0.0222, 0.9620) -- (1.1059, -0.0222, 0.9596) -- cycle;
\fill[blue!15.0, opacity=0.5] (1.1059, -0.0222, 0.9596) -- (1.1552, -0.0222, 0.9620) -- (1.1558, -0.0247, 1.0120) -- (1.1066, -0.0247, 1.0096) -- cycle;
\fill[blue!15.0, opacity=0.5] (1.1066, -0.0247, 1.0096) -- (1.1558, -0.0247, 1.0120) -- (1.1564, -0.0273, 1.0620) -- (1.1073, -0.0273, 1.0596) -- cycle;
\fill[blue!15.1, opacity=0.5] (1.1073, -0.0273, 1.0596) -- (1.1564, -0.0273, 1.0620) -- (1.1570, -0.0300, 1.1120) -- (1.1080, -0.0300, 1.1096) -- cycle;
\fill[blue!15.1, opacity=0.5] (1.1080, -0.0300, 1.1096) -- (1.1570, -0.0300, 1.1120) -- (1.1576, -0.0328, 1.1620) -- (1.1087, -0.0328, 1.1596) -- cycle;
\fill[blue!15.2, opacity=0.5] (1.1087, -0.0328, 1.1596) -- (1.1576, -0.0328, 1.1620) -- (1.1583, -0.0356, 1.2120) -- (1.1095, -0.0356, 1.2096) -- cycle;
\fill[blue!15.3, opacity=0.5] (1.1095, -0.0356, 1.2096) -- (1.1583, -0.0356, 1.2120) -- (1.1590, -0.0385, 1.2620) -- (1.1103, -0.0385, 1.2596) -- cycle;
\fill[blue!15.4, opacity=0.5] (1.1103, -0.0385, 1.2596) -- (1.1590, -0.0385, 1.2620) -- (1.1597, -0.0415, 1.3120) -- (1.1111, -0.0415, 1.3096) -- cycle;
\fill[blue!15.6, opacity=0.5] (1.1111, -0.0415, 1.3096) -- (1.1597, -0.0415, 1.3120) -- (1.1604, -0.0445, 1.3620) -- (1.1119, -0.0445, 1.3596) -- cycle;
\fill[blue!15.9, opacity=0.5] (1.1119, -0.0445, 1.3596) -- (1.1604, -0.0445, 1.3620) -- (1.1611, -0.0475, 1.4120) -- (1.1127, -0.0475, 1.4096) -- cycle;
\fill[blue!16.3, opacity=0.5] (1.1127, -0.0475, 1.4096) -- (1.1611, -0.0475, 1.4120) -- (1.1618, -0.0506, 1.4620) -- (1.1135, -0.0506, 1.4596) -- cycle;
\fill[blue!16.7, opacity=0.5] (1.1135, -0.0506, 1.4596) -- (1.1618, -0.0506, 1.4620) -- (1.1625, -0.0537, 1.5120) -- (1.1143, -0.0537, 1.5096) -- cycle;
\fill[blue!17.3, opacity=0.5] (1.1143, -0.0537, 1.5096) -- (1.1625, -0.0537, 1.5120) -- (1.1633, -0.0569, 1.5620) -- (1.1152, -0.0569, 1.5596) -- cycle;
\fill[blue!17.9, opacity=0.5] (1.1152, -0.0569, 1.5596) -- (1.1633, -0.0569, 1.5620) -- (1.1640, -0.0600, 1.6120) -- (1.1160, -0.0600, 1.6096) -- cycle;
\fill[blue!18.8, opacity=0.5] (1.1160, -0.0600, 1.6096) -- (1.1640, -0.0600, 1.6120) -- (1.1647, -0.0631, 1.6620) -- (1.1168, -0.0631, 1.6596) -- cycle;
\fill[blue!19.7, opacity=0.5] (1.1168, -0.0631, 1.6596) -- (1.1647, -0.0631, 1.6620) -- (1.1655, -0.0663, 1.7120) -- (1.1177, -0.0663, 1.7096) -- cycle;
\fill[blue!20.8, opacity=0.5] (1.1177, -0.0663, 1.7096) -- (1.1655, -0.0663, 1.7120) -- (1.1662, -0.0694, 1.7620) -- (1.1185, -0.0694, 1.7596) -- cycle;
\fill[blue!22.0, opacity=0.5] (1.1185, -0.0694, 1.7596) -- (1.1662, -0.0694, 1.7620) -- (1.1669, -0.0725, 1.8120) -- (1.1193, -0.0725, 1.8096) -- cycle;
\fill[blue!23.4, opacity=0.5] (1.1193, -0.0725, 1.8096) -- (1.1669, -0.0725, 1.8120) -- (1.1676, -0.0755, 1.8620) -- (1.1201, -0.0755, 1.8596) -- cycle;
\fill[blue!24.9, opacity=0.5] (1.1201, -0.0755, 1.8596) -- (1.1676, -0.0755, 1.8620) -- (1.1683, -0.0785, 1.9120) -- (1.1209, -0.0785, 1.9096) -- cycle;
\fill[blue!26.5, opacity=0.5] (1.1209, -0.0785, 1.9096) -- (1.1683, -0.0785, 1.9120) -- (1.1690, -0.0815, 1.9620) -- (1.1217, -0.0815, 1.9596) -- cycle;
\fill[blue!28.2, opacity=0.5] (1.1217, -0.0815, 1.9596) -- (1.1690, -0.0815, 1.9620) -- (1.1697, -0.0844, 2.0120) -- (1.1225, -0.0844, 2.0096) -- cycle;
\fill[blue!30.0, opacity=0.5] (1.1225, -0.0844, 2.0096) -- (1.1697, -0.0844, 2.0120) -- (1.1704, -0.0872, 2.0620) -- (1.1233, -0.0872, 2.0596) -- cycle;
\fill[blue!31.9, opacity=0.5] (1.1233, -0.0872, 2.0596) -- (1.1704, -0.0872, 2.0620) -- (1.1710, -0.0900, 2.1120) -- (1.1240, -0.0900, 2.1096) -- cycle;
\fill[blue!33.9, opacity=0.5] (1.1240, -0.0900, 2.1096) -- (1.1710, -0.0900, 2.1120) -- (1.1716, -0.0927, 2.1620) -- (1.1247, -0.0927, 2.1596) -- cycle;
\fill[blue!35.8, opacity=0.5] (1.1247, -0.0927, 2.1596) -- (1.1716, -0.0927, 2.1620) -- (1.1722, -0.0953, 2.2120) -- (1.1254, -0.0953, 2.2096) -- cycle;
\fill[blue!37.8, opacity=0.5] (1.1254, -0.0953, 2.2096) -- (1.1722, -0.0953, 2.2120) -- (1.1728, -0.0978, 2.2620) -- (1.1261, -0.0978, 2.2596) -- cycle;
\fill[blue!39.8, opacity=0.5] (1.1261, -0.0978, 2.2596) -- (1.1728, -0.0978, 2.2620) -- (1.1734, -0.1001, 2.3120) -- (1.1267, -0.1001, 2.3096) -- cycle;
\fill[blue!41.7, opacity=0.5] (1.1267, -0.1001, 2.3096) -- (1.1734, -0.1001, 2.3120) -- (1.1739, -0.1024, 2.3620) -- (1.1273, -0.1024, 2.3596) -- cycle;
\fill[blue!43.6, opacity=0.5] (1.1273, -0.1024, 2.3596) -- (1.1739, -0.1024, 2.3620) -- (1.1744, -0.1046, 2.4120) -- (1.1279, -0.1046, 2.4096) -- cycle;
\fill[blue!45.4, opacity=0.5] (1.1279, -0.1046, 2.4096) -- (1.1744, -0.1046, 2.4120) -- (1.1749, -0.1066, 2.4620) -- (1.1284, -0.1066, 2.4596) -- cycle;
\fill[blue!47.1, opacity=0.5] (1.1284, -0.1066, 2.4596) -- (1.1749, -0.1066, 2.4620) -- (1.1753, -0.1085, 2.5120) -- (1.1289, -0.1085, 2.5096) -- cycle;
\fill[blue!48.7, opacity=0.5] (1.1289, -0.1085, 2.5096) -- (1.1753, -0.1085, 2.5120) -- (1.1757, -0.1103, 2.5620) -- (1.1294, -0.1103, 2.5596) -- cycle;
\fill[blue!50.2, opacity=0.5] (1.1294, -0.1103, 2.5596) -- (1.1757, -0.1103, 2.5620) -- (1.1761, -0.1120, 2.6120) -- (1.1299, -0.1120, 2.6096) -- cycle;
\fill[blue!51.6, opacity=0.5] (1.1299, -0.1120, 2.6096) -- (1.1761, -0.1120, 2.6120) -- (1.1765, -0.1135, 2.6620) -- (1.1303, -0.1135, 2.6596) -- cycle;
\fill[blue!52.8, opacity=0.5] (1.1303, -0.1135, 2.6596) -- (1.1765, -0.1135, 2.6620) -- (1.1768, -0.1148, 2.7120) -- (1.1306, -0.1148, 2.7096) -- cycle;
\fill[blue!53.9, opacity=0.5] (1.1306, -0.1148, 2.7096) -- (1.1768, -0.1148, 2.7120) -- (1.1771, -0.1160, 2.7620) -- (1.1309, -0.1160, 2.7596) -- cycle;
\fill[blue!54.9, opacity=0.5] (1.1309, -0.1160, 2.7596) -- (1.1771, -0.1160, 2.7620) -- (1.1773, -0.1171, 2.8120) -- (1.1312, -0.1171, 2.8096) -- cycle;
\fill[blue!55.7, opacity=0.5] (1.1312, -0.1171, 2.8096) -- (1.1773, -0.1171, 2.8120) -- (1.1775, -0.1180, 2.8620) -- (1.1315, -0.1180, 2.8596) -- cycle;
\fill[blue!56.4, opacity=0.5] (1.1315, -0.1180, 2.8596) -- (1.1775, -0.1180, 2.8620) -- (1.1777, -0.1187, 2.9120) -- (1.1317, -0.1187, 2.9096) -- cycle;
\fill[blue!56.9, opacity=0.5] (1.1317, -0.1187, 2.9096) -- (1.1777, -0.1187, 2.9120) -- (1.1778, -0.1193, 2.9620) -- (1.1318, -0.1193, 2.9596) -- cycle;
\fill[blue!57.2, opacity=0.5] (1.1318, -0.1193, 2.9596) -- (1.1778, -0.1193, 2.9620) -- (1.1779, -0.1197, 3.0120) -- (1.1319, -0.1197, 3.0096) -- cycle;
\fill[blue!57.4, opacity=0.5] (1.1319, -0.1197, 3.0096) -- (1.1779, -0.1197, 3.0120) -- (1.1780, -0.1199, 3.0620) -- (1.1320, -0.1199, 3.0596) -- cycle;
\fill[blue!57.5, opacity=0.5] (1.1320, -0.1199, 3.0596) -- (1.1780, -0.1199, 3.0620) -- (1.1780, -0.1200, 3.1120) -- (1.1320, -0.1200, 3.1096) -- cycle;
\fill[blue!15.0, opacity=0.5] (1.1500, -0.0000, 0.1120) -- (1.2000, -0.0000, 0.1141) -- (1.2000, -0.0001, 0.1641) -- (1.1500, -0.0001, 0.1620) -- cycle;
\fill[blue!15.0, opacity=0.5] (1.1500, -0.0001, 0.1620) -- (1.2000, -0.0001, 0.1641) -- (1.2001, -0.0003, 0.2141) -- (1.1501, -0.0003, 0.2120) -- cycle;
\fill[blue!15.0, opacity=0.5] (1.1501, -0.0003, 0.2120) -- (1.2001, -0.0003, 0.2141) -- (1.2001, -0.0007, 0.2641) -- (1.1502, -0.0007, 0.2620) -- cycle;
\fill[blue!15.0, opacity=0.5] (1.1502, -0.0007, 0.2620) -- (1.2001, -0.0007, 0.2641) -- (1.2003, -0.0013, 0.3141) -- (1.1503, -0.0013, 0.3120) -- cycle;
\fill[blue!15.0, opacity=0.5] (1.1503, -0.0013, 0.3120) -- (1.2003, -0.0013, 0.3141) -- (1.2004, -0.0020, 0.3641) -- (1.1505, -0.0020, 0.3620) -- cycle;
\fill[blue!15.0, opacity=0.5] (1.1505, -0.0020, 0.3620) -- (1.2004, -0.0020, 0.3641) -- (1.2006, -0.0029, 0.4141) -- (1.1507, -0.0029, 0.4120) -- cycle;
\fill[blue!15.0, opacity=0.5] (1.1507, -0.0029, 0.4120) -- (1.2006, -0.0029, 0.4141) -- (1.2008, -0.0040, 0.4641) -- (1.1509, -0.0040, 0.4620) -- cycle;
\fill[blue!15.0, opacity=0.5] (1.1509, -0.0040, 0.4620) -- (1.2008, -0.0040, 0.4641) -- (1.2010, -0.0052, 0.5141) -- (1.1512, -0.0052, 0.5120) -- cycle;
\fill[blue!15.0, opacity=0.5] (1.1512, -0.0052, 0.5120) -- (1.2010, -0.0052, 0.5141) -- (1.2013, -0.0065, 0.5641) -- (1.1515, -0.0065, 0.5620) -- cycle;
\fill[blue!15.0, opacity=0.5] (1.1515, -0.0065, 0.5620) -- (1.2013, -0.0065, 0.5641) -- (1.2016, -0.0080, 0.6141) -- (1.1519, -0.0080, 0.6120) -- cycle;
\fill[blue!15.0, opacity=0.5] (1.1519, -0.0080, 0.6120) -- (1.2016, -0.0080, 0.6141) -- (1.2019, -0.0097, 0.6641) -- (1.1523, -0.0097, 0.6620) -- cycle;
\fill[blue!15.0, opacity=0.5] (1.1523, -0.0097, 0.6620) -- (1.2019, -0.0097, 0.6641) -- (1.2023, -0.0115, 0.7141) -- (1.1527, -0.0115, 0.7120) -- cycle;
\fill[blue!15.0, opacity=0.5] (1.1527, -0.0115, 0.7120) -- (1.2023, -0.0115, 0.7141) -- (1.2027, -0.0134, 0.7641) -- (1.1531, -0.0134, 0.7620) -- cycle;
\fill[blue!15.0, opacity=0.5] (1.1531, -0.0134, 0.7620) -- (1.2027, -0.0134, 0.7641) -- (1.2031, -0.0154, 0.8141) -- (1.1536, -0.0154, 0.8120) -- cycle;
\fill[blue!15.0, opacity=0.5] (1.1536, -0.0154, 0.8120) -- (1.2031, -0.0154, 0.8141) -- (1.2035, -0.0176, 0.8641) -- (1.1541, -0.0176, 0.8620) -- cycle;
\fill[blue!15.0, opacity=0.5] (1.1541, -0.0176, 0.8620) -- (1.2035, -0.0176, 0.8641) -- (1.2040, -0.0199, 0.9141) -- (1.1546, -0.0199, 0.9120) -- cycle;
\fill[blue!15.0, opacity=0.5] (1.1546, -0.0199, 0.9120) -- (1.2040, -0.0199, 0.9141) -- (1.2044, -0.0222, 0.9641) -- (1.1552, -0.0222, 0.9620) -- cycle;
\fill[blue!15.0, opacity=0.5] (1.1552, -0.0222, 0.9620) -- (1.2044, -0.0222, 0.9641) -- (1.2049, -0.0247, 1.0141) -- (1.1558, -0.0247, 1.0120) -- cycle;
\fill[blue!15.0, opacity=0.5] (1.1558, -0.0247, 1.0120) -- (1.2049, -0.0247, 1.0141) -- (1.2055, -0.0273, 1.0641) -- (1.1564, -0.0273, 1.0620) -- cycle;
\fill[blue!15.0, opacity=0.5] (1.1564, -0.0273, 1.0620) -- (1.2055, -0.0273, 1.0641) -- (1.2060, -0.0300, 1.1141) -- (1.1570, -0.0300, 1.1120) -- cycle;
\fill[blue!15.1, opacity=0.5] (1.1570, -0.0300, 1.1120) -- (1.2060, -0.0300, 1.1141) -- (1.2066, -0.0328, 1.1641) -- (1.1576, -0.0328, 1.1620) -- cycle;
\fill[blue!15.1, opacity=0.5] (1.1576, -0.0328, 1.1620) -- (1.2066, -0.0328, 1.1641) -- (1.2071, -0.0356, 1.2141) -- (1.1583, -0.0356, 1.2120) -- cycle;
\fill[blue!15.2, opacity=0.5] (1.1583, -0.0356, 1.2120) -- (1.2071, -0.0356, 1.2141) -- (1.2077, -0.0385, 1.2641) -- (1.1590, -0.0385, 1.2620) -- cycle;
\fill[blue!15.3, opacity=0.5] (1.1590, -0.0385, 1.2620) -- (1.2077, -0.0385, 1.2641) -- (1.2083, -0.0415, 1.3141) -- (1.1597, -0.0415, 1.3120) -- cycle;
\fill[blue!15.4, opacity=0.5] (1.1597, -0.0415, 1.3120) -- (1.2083, -0.0415, 1.3141) -- (1.2089, -0.0445, 1.3641) -- (1.1604, -0.0445, 1.3620) -- cycle;
\fill[blue!15.6, opacity=0.5] (1.1604, -0.0445, 1.3620) -- (1.2089, -0.0445, 1.3641) -- (1.2095, -0.0475, 1.4141) -- (1.1611, -0.0475, 1.4120) -- cycle;
\fill[blue!15.9, opacity=0.5] (1.1611, -0.0475, 1.4120) -- (1.2095, -0.0475, 1.4141) -- (1.2101, -0.0506, 1.4641) -- (1.1618, -0.0506, 1.4620) -- cycle;
\fill[blue!16.2, opacity=0.5] (1.1618, -0.0506, 1.4620) -- (1.2101, -0.0506, 1.4641) -- (1.2107, -0.0537, 1.5141) -- (1.1625, -0.0537, 1.5120) -- cycle;
\fill[blue!16.7, opacity=0.5] (1.1625, -0.0537, 1.5120) -- (1.2107, -0.0537, 1.5141) -- (1.2114, -0.0569, 1.5641) -- (1.1633, -0.0569, 1.5620) -- cycle;
\fill[blue!17.2, opacity=0.5] (1.1633, -0.0569, 1.5620) -- (1.2114, -0.0569, 1.5641) -- (1.2120, -0.0600, 1.6141) -- (1.1640, -0.0600, 1.6120) -- cycle;
\fill[blue!17.8, opacity=0.5] (1.1640, -0.0600, 1.6120) -- (1.2120, -0.0600, 1.6141) -- (1.2126, -0.0631, 1.6641) -- (1.1647, -0.0631, 1.6620) -- cycle;
\fill[blue!18.6, opacity=0.5] (1.1647, -0.0631, 1.6620) -- (1.2126, -0.0631, 1.6641) -- (1.2133, -0.0663, 1.7141) -- (1.1655, -0.0663, 1.7120) -- cycle;
\fill[blue!19.5, opacity=0.5] (1.1655, -0.0663, 1.7120) -- (1.2133, -0.0663, 1.7141) -- (1.2139, -0.0694, 1.7641) -- (1.1662, -0.0694, 1.7620) -- cycle;
\fill[blue!20.6, opacity=0.5] (1.1662, -0.0694, 1.7620) -- (1.2139, -0.0694, 1.7641) -- (1.2145, -0.0725, 1.8141) -- (1.1669, -0.0725, 1.8120) -- cycle;
\fill[blue!21.7, opacity=0.5] (1.1669, -0.0725, 1.8120) -- (1.2145, -0.0725, 1.8141) -- (1.2151, -0.0755, 1.8641) -- (1.1676, -0.0755, 1.8620) -- cycle;
\fill[blue!23.0, opacity=0.5] (1.1676, -0.0755, 1.8620) -- (1.2151, -0.0755, 1.8641) -- (1.2157, -0.0785, 1.9141) -- (1.1683, -0.0785, 1.9120) -- cycle;
\fill[blue!24.4, opacity=0.5] (1.1683, -0.0785, 1.9120) -- (1.2157, -0.0785, 1.9141) -- (1.2163, -0.0815, 1.9641) -- (1.1690, -0.0815, 1.9620) -- cycle;
\fill[blue!26.0, opacity=0.5] (1.1690, -0.0815, 1.9620) -- (1.2163, -0.0815, 1.9641) -- (1.2169, -0.0844, 2.0141) -- (1.1697, -0.0844, 2.0120) -- cycle;
\fill[blue!27.6, opacity=0.5] (1.1697, -0.0844, 2.0120) -- (1.2169, -0.0844, 2.0141) -- (1.2174, -0.0872, 2.0641) -- (1.1704, -0.0872, 2.0620) -- cycle;
\fill[blue!29.3, opacity=0.5] (1.1704, -0.0872, 2.0620) -- (1.2174, -0.0872, 2.0641) -- (1.2180, -0.0900, 2.1141) -- (1.1710, -0.0900, 2.1120) -- cycle;
\fill[blue!31.1, opacity=0.5] (1.1710, -0.0900, 2.1120) -- (1.2180, -0.0900, 2.1141) -- (1.2185, -0.0927, 2.1641) -- (1.1716, -0.0927, 2.1620) -- cycle;
\fill[blue!32.9, opacity=0.5] (1.1716, -0.0927, 2.1620) -- (1.2185, -0.0927, 2.1641) -- (1.2191, -0.0953, 2.2141) -- (1.1722, -0.0953, 2.2120) -- cycle;
\fill[blue!34.8, opacity=0.5] (1.1722, -0.0953, 2.2120) -- (1.2191, -0.0953, 2.2141) -- (1.2196, -0.0978, 2.2641) -- (1.1728, -0.0978, 2.2620) -- cycle;
\fill[blue!36.7, opacity=0.5] (1.1728, -0.0978, 2.2620) -- (1.2196, -0.0978, 2.2641) -- (1.2200, -0.1001, 2.3141) -- (1.1734, -0.1001, 2.3120) -- cycle;
\fill[blue!38.6, opacity=0.5] (1.1734, -0.1001, 2.3120) -- (1.2200, -0.1001, 2.3141) -- (1.2205, -0.1024, 2.3641) -- (1.1739, -0.1024, 2.3620) -- cycle;
\fill[blue!40.4, opacity=0.5] (1.1739, -0.1024, 2.3620) -- (1.2205, -0.1024, 2.3641) -- (1.2209, -0.1046, 2.4141) -- (1.1744, -0.1046, 2.4120) -- cycle;
\fill[blue!42.2, opacity=0.5] (1.1744, -0.1046, 2.4120) -- (1.2209, -0.1046, 2.4141) -- (1.2213, -0.1066, 2.4641) -- (1.1749, -0.1066, 2.4620) -- cycle;
\fill[blue!43.9, opacity=0.5] (1.1749, -0.1066, 2.4620) -- (1.2213, -0.1066, 2.4641) -- (1.2217, -0.1085, 2.5141) -- (1.1753, -0.1085, 2.5120) -- cycle;
\fill[blue!45.6, opacity=0.5] (1.1753, -0.1085, 2.5120) -- (1.2217, -0.1085, 2.5141) -- (1.2221, -0.1103, 2.5641) -- (1.1757, -0.1103, 2.5620) -- cycle;
\fill[blue!47.2, opacity=0.5] (1.1757, -0.1103, 2.5620) -- (1.2221, -0.1103, 2.5641) -- (1.2224, -0.1120, 2.6141) -- (1.1761, -0.1120, 2.6120) -- cycle;
\fill[blue!48.6, opacity=0.5] (1.1761, -0.1120, 2.6120) -- (1.2224, -0.1120, 2.6141) -- (1.2227, -0.1135, 2.6641) -- (1.1765, -0.1135, 2.6620) -- cycle;
\fill[blue!50.0, opacity=0.5] (1.1765, -0.1135, 2.6620) -- (1.2227, -0.1135, 2.6641) -- (1.2230, -0.1148, 2.7141) -- (1.1768, -0.1148, 2.7120) -- cycle;
\fill[blue!51.2, opacity=0.5] (1.1768, -0.1148, 2.7120) -- (1.2230, -0.1148, 2.7141) -- (1.2232, -0.1160, 2.7641) -- (1.1771, -0.1160, 2.7620) -- cycle;
\fill[blue!52.2, opacity=0.5] (1.1771, -0.1160, 2.7620) -- (1.2232, -0.1160, 2.7641) -- (1.2234, -0.1171, 2.8141) -- (1.1773, -0.1171, 2.8120) -- cycle;
\fill[blue!53.2, opacity=0.5] (1.1773, -0.1171, 2.8120) -- (1.2234, -0.1171, 2.8141) -- (1.2236, -0.1180, 2.8641) -- (1.1775, -0.1180, 2.8620) -- cycle;
\fill[blue!54.0, opacity=0.5] (1.1775, -0.1180, 2.8620) -- (1.2236, -0.1180, 2.8641) -- (1.2237, -0.1187, 2.9141) -- (1.1777, -0.1187, 2.9120) -- cycle;
\fill[blue!54.6, opacity=0.5] (1.1777, -0.1187, 2.9120) -- (1.2237, -0.1187, 2.9141) -- (1.2239, -0.1193, 2.9641) -- (1.1778, -0.1193, 2.9620) -- cycle;
\fill[blue!55.1, opacity=0.5] (1.1778, -0.1193, 2.9620) -- (1.2239, -0.1193, 2.9641) -- (1.2239, -0.1197, 3.0141) -- (1.1779, -0.1197, 3.0120) -- cycle;
\fill[blue!55.5, opacity=0.5] (1.1779, -0.1197, 3.0120) -- (1.2239, -0.1197, 3.0141) -- (1.2240, -0.1199, 3.0641) -- (1.1780, -0.1199, 3.0620) -- cycle;
\fill[blue!55.7, opacity=0.5] (1.1780, -0.1199, 3.0620) -- (1.2240, -0.1199, 3.0641) -- (1.2240, -0.1200, 3.1141) -- (1.1780, -0.1200, 3.1120) -- cycle;
\fill[blue!15.0, opacity=0.5] (1.2000, -0.0000, 0.1141) -- (1.2500, -0.0000, 0.1159) -- (1.2500, -0.0001, 0.1659) -- (1.2000, -0.0001, 0.1641) -- cycle;
\fill[blue!15.0, opacity=0.5] (1.2000, -0.0001, 0.1641) -- (1.2500, -0.0001, 0.1659) -- (1.2501, -0.0003, 0.2159) -- (1.2001, -0.0003, 0.2141) -- cycle;
\fill[blue!15.0, opacity=0.5] (1.2001, -0.0003, 0.2141) -- (1.2501, -0.0003, 0.2159) -- (1.2501, -0.0007, 0.2659) -- (1.2001, -0.0007, 0.2641) -- cycle;
\fill[blue!15.0, opacity=0.5] (1.2001, -0.0007, 0.2641) -- (1.2501, -0.0007, 0.2659) -- (1.2502, -0.0013, 0.3159) -- (1.2003, -0.0013, 0.3141) -- cycle;
\fill[blue!15.0, opacity=0.5] (1.2003, -0.0013, 0.3141) -- (1.2502, -0.0013, 0.3159) -- (1.2503, -0.0020, 0.3659) -- (1.2004, -0.0020, 0.3641) -- cycle;
\fill[blue!15.0, opacity=0.5] (1.2004, -0.0020, 0.3641) -- (1.2503, -0.0020, 0.3659) -- (1.2505, -0.0029, 0.4159) -- (1.2006, -0.0029, 0.4141) -- cycle;
\fill[blue!15.0, opacity=0.5] (1.2006, -0.0029, 0.4141) -- (1.2505, -0.0029, 0.4159) -- (1.2507, -0.0040, 0.4659) -- (1.2008, -0.0040, 0.4641) -- cycle;
\fill[blue!15.0, opacity=0.5] (1.2008, -0.0040, 0.4641) -- (1.2507, -0.0040, 0.4659) -- (1.2509, -0.0052, 0.5159) -- (1.2010, -0.0052, 0.5141) -- cycle;
\fill[blue!15.0, opacity=0.5] (1.2010, -0.0052, 0.5141) -- (1.2509, -0.0052, 0.5159) -- (1.2511, -0.0065, 0.5659) -- (1.2013, -0.0065, 0.5641) -- cycle;
\fill[blue!15.0, opacity=0.5] (1.2013, -0.0065, 0.5641) -- (1.2511, -0.0065, 0.5659) -- (1.2513, -0.0080, 0.6159) -- (1.2016, -0.0080, 0.6141) -- cycle;
\fill[blue!15.0, opacity=0.5] (1.2016, -0.0080, 0.6141) -- (1.2513, -0.0080, 0.6159) -- (1.2516, -0.0097, 0.6659) -- (1.2019, -0.0097, 0.6641) -- cycle;
\fill[blue!15.0, opacity=0.5] (1.2019, -0.0097, 0.6641) -- (1.2516, -0.0097, 0.6659) -- (1.2519, -0.0115, 0.7159) -- (1.2023, -0.0115, 0.7141) -- cycle;
\fill[blue!15.0, opacity=0.5] (1.2023, -0.0115, 0.7141) -- (1.2519, -0.0115, 0.7159) -- (1.2522, -0.0134, 0.7659) -- (1.2027, -0.0134, 0.7641) -- cycle;
\fill[blue!15.0, opacity=0.5] (1.2027, -0.0134, 0.7641) -- (1.2522, -0.0134, 0.7659) -- (1.2526, -0.0154, 0.8159) -- (1.2031, -0.0154, 0.8141) -- cycle;
\fill[blue!15.0, opacity=0.5] (1.2031, -0.0154, 0.8141) -- (1.2526, -0.0154, 0.8159) -- (1.2529, -0.0176, 0.8659) -- (1.2035, -0.0176, 0.8641) -- cycle;
\fill[blue!15.0, opacity=0.5] (1.2035, -0.0176, 0.8641) -- (1.2529, -0.0176, 0.8659) -- (1.2533, -0.0199, 0.9159) -- (1.2040, -0.0199, 0.9141) -- cycle;
\fill[blue!15.0, opacity=0.5] (1.2040, -0.0199, 0.9141) -- (1.2533, -0.0199, 0.9159) -- (1.2537, -0.0222, 0.9659) -- (1.2044, -0.0222, 0.9641) -- cycle;
\fill[blue!15.0, opacity=0.5] (1.2044, -0.0222, 0.9641) -- (1.2537, -0.0222, 0.9659) -- (1.2541, -0.0247, 1.0159) -- (1.2049, -0.0247, 1.0141) -- cycle;
\fill[blue!15.0, opacity=0.5] (1.2049, -0.0247, 1.0141) -- (1.2541, -0.0247, 1.0159) -- (1.2546, -0.0273, 1.0659) -- (1.2055, -0.0273, 1.0641) -- cycle;
\fill[blue!15.0, opacity=0.5] (1.2055, -0.0273, 1.0641) -- (1.2546, -0.0273, 1.0659) -- (1.2550, -0.0300, 1.1159) -- (1.2060, -0.0300, 1.1141) -- cycle;
\fill[blue!15.0, opacity=0.5] (1.2060, -0.0300, 1.1141) -- (1.2550, -0.0300, 1.1159) -- (1.2555, -0.0328, 1.1659) -- (1.2066, -0.0328, 1.1641) -- cycle;
\fill[blue!15.1, opacity=0.5] (1.2066, -0.0328, 1.1641) -- (1.2555, -0.0328, 1.1659) -- (1.2559, -0.0356, 1.2159) -- (1.2071, -0.0356, 1.2141) -- cycle;
\fill[blue!15.1, opacity=0.5] (1.2071, -0.0356, 1.2141) -- (1.2559, -0.0356, 1.2159) -- (1.2564, -0.0385, 1.2659) -- (1.2077, -0.0385, 1.2641) -- cycle;
\fill[blue!15.2, opacity=0.5] (1.2077, -0.0385, 1.2641) -- (1.2564, -0.0385, 1.2659) -- (1.2569, -0.0415, 1.3159) -- (1.2083, -0.0415, 1.3141) -- cycle;
\fill[blue!15.3, opacity=0.5] (1.2083, -0.0415, 1.3141) -- (1.2569, -0.0415, 1.3159) -- (1.2574, -0.0445, 1.3659) -- (1.2089, -0.0445, 1.3641) -- cycle;
\fill[blue!15.5, opacity=0.5] (1.2089, -0.0445, 1.3641) -- (1.2574, -0.0445, 1.3659) -- (1.2579, -0.0475, 1.4159) -- (1.2095, -0.0475, 1.4141) -- cycle;
\fill[blue!15.7, opacity=0.5] (1.2095, -0.0475, 1.4141) -- (1.2579, -0.0475, 1.4159) -- (1.2584, -0.0506, 1.4659) -- (1.2101, -0.0506, 1.4641) -- cycle;
\fill[blue!15.9, opacity=0.5] (1.2101, -0.0506, 1.4641) -- (1.2584, -0.0506, 1.4659) -- (1.2590, -0.0537, 1.5159) -- (1.2107, -0.0537, 1.5141) -- cycle;
\fill[blue!16.3, opacity=0.5] (1.2107, -0.0537, 1.5141) -- (1.2590, -0.0537, 1.5159) -- (1.2595, -0.0569, 1.5659) -- (1.2114, -0.0569, 1.5641) -- cycle;
\fill[blue!16.7, opacity=0.5] (1.2114, -0.0569, 1.5641) -- (1.2595, -0.0569, 1.5659) -- (1.2600, -0.0600, 1.6159) -- (1.2120, -0.0600, 1.6141) -- cycle;
\fill[blue!17.3, opacity=0.5] (1.2120, -0.0600, 1.6141) -- (1.2600, -0.0600, 1.6159) -- (1.2605, -0.0631, 1.6659) -- (1.2126, -0.0631, 1.6641) -- cycle;
\fill[blue!17.9, opacity=0.5] (1.2126, -0.0631, 1.6641) -- (1.2605, -0.0631, 1.6659) -- (1.2610, -0.0663, 1.7159) -- (1.2133, -0.0663, 1.7141) -- cycle;
\fill[blue!18.7, opacity=0.5] (1.2133, -0.0663, 1.7141) -- (1.2610, -0.0663, 1.7159) -- (1.2616, -0.0694, 1.7659) -- (1.2139, -0.0694, 1.7641) -- cycle;
\fill[blue!19.6, opacity=0.5] (1.2139, -0.0694, 1.7641) -- (1.2616, -0.0694, 1.7659) -- (1.2621, -0.0725, 1.8159) -- (1.2145, -0.0725, 1.8141) -- cycle;
\fill[blue!20.6, opacity=0.5] (1.2145, -0.0725, 1.8141) -- (1.2621, -0.0725, 1.8159) -- (1.2626, -0.0755, 1.8659) -- (1.2151, -0.0755, 1.8641) -- cycle;
\fill[blue!21.7, opacity=0.5] (1.2151, -0.0755, 1.8641) -- (1.2626, -0.0755, 1.8659) -- (1.2631, -0.0785, 1.9159) -- (1.2157, -0.0785, 1.9141) -- cycle;
\fill[blue!23.0, opacity=0.5] (1.2157, -0.0785, 1.9141) -- (1.2631, -0.0785, 1.9159) -- (1.2636, -0.0815, 1.9659) -- (1.2163, -0.0815, 1.9641) -- cycle;
\fill[blue!24.4, opacity=0.5] (1.2163, -0.0815, 1.9641) -- (1.2636, -0.0815, 1.9659) -- (1.2641, -0.0844, 2.0159) -- (1.2169, -0.0844, 2.0141) -- cycle;
\fill[blue!25.8, opacity=0.5] (1.2169, -0.0844, 2.0141) -- (1.2641, -0.0844, 2.0159) -- (1.2645, -0.0872, 2.0659) -- (1.2174, -0.0872, 2.0641) -- cycle;
\fill[blue!27.4, opacity=0.5] (1.2174, -0.0872, 2.0641) -- (1.2645, -0.0872, 2.0659) -- (1.2650, -0.0900, 2.1159) -- (1.2180, -0.0900, 2.1141) -- cycle;
\fill[blue!29.1, opacity=0.5] (1.2180, -0.0900, 2.1141) -- (1.2650, -0.0900, 2.1159) -- (1.2654, -0.0927, 2.1659) -- (1.2185, -0.0927, 2.1641) -- cycle;
\fill[blue!30.8, opacity=0.5] (1.2185, -0.0927, 2.1641) -- (1.2654, -0.0927, 2.1659) -- (1.2659, -0.0953, 2.2159) -- (1.2191, -0.0953, 2.2141) -- cycle;
\fill[blue!32.6, opacity=0.5] (1.2191, -0.0953, 2.2141) -- (1.2659, -0.0953, 2.2159) -- (1.2663, -0.0978, 2.2659) -- (1.2196, -0.0978, 2.2641) -- cycle;
\fill[blue!34.4, opacity=0.5] (1.2196, -0.0978, 2.2641) -- (1.2663, -0.0978, 2.2659) -- (1.2667, -0.1001, 2.3159) -- (1.2200, -0.1001, 2.3141) -- cycle;
\fill[blue!36.2, opacity=0.5] (1.2200, -0.1001, 2.3141) -- (1.2667, -0.1001, 2.3159) -- (1.2671, -0.1024, 2.3659) -- (1.2205, -0.1024, 2.3641) -- cycle;
\fill[blue!38.0, opacity=0.5] (1.2205, -0.1024, 2.3641) -- (1.2671, -0.1024, 2.3659) -- (1.2674, -0.1046, 2.4159) -- (1.2209, -0.1046, 2.4141) -- cycle;
\fill[blue!39.7, opacity=0.5] (1.2209, -0.1046, 2.4141) -- (1.2674, -0.1046, 2.4159) -- (1.2678, -0.1066, 2.4659) -- (1.2213, -0.1066, 2.4641) -- cycle;
\fill[blue!41.5, opacity=0.5] (1.2213, -0.1066, 2.4641) -- (1.2678, -0.1066, 2.4659) -- (1.2681, -0.1085, 2.5159) -- (1.2217, -0.1085, 2.5141) -- cycle;
\fill[blue!43.1, opacity=0.5] (1.2217, -0.1085, 2.5141) -- (1.2681, -0.1085, 2.5159) -- (1.2684, -0.1103, 2.5659) -- (1.2221, -0.1103, 2.5641) -- cycle;
\fill[blue!44.7, opacity=0.5] (1.2221, -0.1103, 2.5641) -- (1.2684, -0.1103, 2.5659) -- (1.2687, -0.1120, 2.6159) -- (1.2224, -0.1120, 2.6141) -- cycle;
\fill[blue!46.2, opacity=0.5] (1.2224, -0.1120, 2.6141) -- (1.2687, -0.1120, 2.6159) -- (1.2689, -0.1135, 2.6659) -- (1.2227, -0.1135, 2.6641) -- cycle;
\fill[blue!47.6, opacity=0.5] (1.2227, -0.1135, 2.6641) -- (1.2689, -0.1135, 2.6659) -- (1.2691, -0.1148, 2.7159) -- (1.2230, -0.1148, 2.7141) -- cycle;
\fill[blue!48.9, opacity=0.5] (1.2230, -0.1148, 2.7141) -- (1.2691, -0.1148, 2.7159) -- (1.2693, -0.1160, 2.7659) -- (1.2232, -0.1160, 2.7641) -- cycle;
\fill[blue!50.1, opacity=0.5] (1.2232, -0.1160, 2.7641) -- (1.2693, -0.1160, 2.7659) -- (1.2695, -0.1171, 2.8159) -- (1.2234, -0.1171, 2.8141) -- cycle;
\fill[blue!51.1, opacity=0.5] (1.2234, -0.1171, 2.8141) -- (1.2695, -0.1171, 2.8159) -- (1.2697, -0.1180, 2.8659) -- (1.2236, -0.1180, 2.8641) -- cycle;
\fill[blue!52.0, opacity=0.5] (1.2236, -0.1180, 2.8641) -- (1.2697, -0.1180, 2.8659) -- (1.2698, -0.1187, 2.9159) -- (1.2237, -0.1187, 2.9141) -- cycle;
\fill[blue!52.7, opacity=0.5] (1.2237, -0.1187, 2.9141) -- (1.2698, -0.1187, 2.9159) -- (1.2699, -0.1193, 2.9659) -- (1.2239, -0.1193, 2.9641) -- cycle;
\fill[blue!53.4, opacity=0.5] (1.2239, -0.1193, 2.9641) -- (1.2699, -0.1193, 2.9659) -- (1.2699, -0.1197, 3.0159) -- (1.2239, -0.1197, 3.0141) -- cycle;
\fill[blue!53.8, opacity=0.5] (1.2239, -0.1197, 3.0141) -- (1.2699, -0.1197, 3.0159) -- (1.2700, -0.1199, 3.0659) -- (1.2240, -0.1199, 3.0641) -- cycle;
\fill[blue!54.2, opacity=0.5] (1.2240, -0.1199, 3.0641) -- (1.2700, -0.1199, 3.0659) -- (1.2700, -0.1200, 3.1159) -- (1.2240, -0.1200, 3.1141) -- cycle;
\fill[blue!15.0, opacity=0.5] (1.2500, -0.0000, 0.1159) -- (1.3000, -0.0000, 0.1174) -- (1.3000, -0.0001, 0.1674) -- (1.2500, -0.0001, 0.1659) -- cycle;
\fill[blue!15.0, opacity=0.5] (1.2500, -0.0001, 0.1659) -- (1.3000, -0.0001, 0.1674) -- (1.3000, -0.0003, 0.2174) -- (1.2501, -0.0003, 0.2159) -- cycle;
\fill[blue!15.0, opacity=0.5] (1.2501, -0.0003, 0.2159) -- (1.3000, -0.0003, 0.2174) -- (1.3001, -0.0007, 0.2674) -- (1.2501, -0.0007, 0.2659) -- cycle;
\fill[blue!15.0, opacity=0.5] (1.2501, -0.0007, 0.2659) -- (1.3001, -0.0007, 0.2674) -- (1.3002, -0.0013, 0.3174) -- (1.2502, -0.0013, 0.3159) -- cycle;
\fill[blue!15.0, opacity=0.5] (1.2502, -0.0013, 0.3159) -- (1.3002, -0.0013, 0.3174) -- (1.3003, -0.0020, 0.3674) -- (1.2503, -0.0020, 0.3659) -- cycle;
\fill[blue!15.0, opacity=0.5] (1.2503, -0.0020, 0.3659) -- (1.3003, -0.0020, 0.3674) -- (1.3004, -0.0029, 0.4174) -- (1.2505, -0.0029, 0.4159) -- cycle;
\fill[blue!15.0, opacity=0.5] (1.2505, -0.0029, 0.4159) -- (1.3004, -0.0029, 0.4174) -- (1.3005, -0.0040, 0.4674) -- (1.2507, -0.0040, 0.4659) -- cycle;
\fill[blue!15.0, opacity=0.5] (1.2507, -0.0040, 0.4659) -- (1.3005, -0.0040, 0.4674) -- (1.3007, -0.0052, 0.5174) -- (1.2509, -0.0052, 0.5159) -- cycle;
\fill[blue!15.0, opacity=0.5] (1.2509, -0.0052, 0.5159) -- (1.3007, -0.0052, 0.5174) -- (1.3009, -0.0065, 0.5674) -- (1.2511, -0.0065, 0.5659) -- cycle;
\fill[blue!15.0, opacity=0.5] (1.2511, -0.0065, 0.5659) -- (1.3009, -0.0065, 0.5674) -- (1.3011, -0.0080, 0.6174) -- (1.2513, -0.0080, 0.6159) -- cycle;
\fill[blue!15.0, opacity=0.5] (1.2513, -0.0080, 0.6159) -- (1.3011, -0.0080, 0.6174) -- (1.3013, -0.0097, 0.6674) -- (1.2516, -0.0097, 0.6659) -- cycle;
\fill[blue!15.0, opacity=0.5] (1.2516, -0.0097, 0.6659) -- (1.3013, -0.0097, 0.6674) -- (1.3015, -0.0115, 0.7174) -- (1.2519, -0.0115, 0.7159) -- cycle;
\fill[blue!15.0, opacity=0.5] (1.2519, -0.0115, 0.7159) -- (1.3015, -0.0115, 0.7174) -- (1.3018, -0.0134, 0.7674) -- (1.2522, -0.0134, 0.7659) -- cycle;
\fill[blue!15.0, opacity=0.5] (1.2522, -0.0134, 0.7659) -- (1.3018, -0.0134, 0.7674) -- (1.3021, -0.0154, 0.8174) -- (1.2526, -0.0154, 0.8159) -- cycle;
\fill[blue!15.0, opacity=0.5] (1.2526, -0.0154, 0.8159) -- (1.3021, -0.0154, 0.8174) -- (1.3023, -0.0176, 0.8674) -- (1.2529, -0.0176, 0.8659) -- cycle;
\fill[blue!15.0, opacity=0.5] (1.2529, -0.0176, 0.8659) -- (1.3023, -0.0176, 0.8674) -- (1.3026, -0.0199, 0.9174) -- (1.2533, -0.0199, 0.9159) -- cycle;
\fill[blue!15.0, opacity=0.5] (1.2533, -0.0199, 0.9159) -- (1.3026, -0.0199, 0.9174) -- (1.3030, -0.0222, 0.9674) -- (1.2537, -0.0222, 0.9659) -- cycle;
\fill[blue!15.0, opacity=0.5] (1.2537, -0.0222, 0.9659) -- (1.3030, -0.0222, 0.9674) -- (1.3033, -0.0247, 1.0174) -- (1.2541, -0.0247, 1.0159) -- cycle;
\fill[blue!15.0, opacity=0.5] (1.2541, -0.0247, 1.0159) -- (1.3033, -0.0247, 1.0174) -- (1.3036, -0.0273, 1.0674) -- (1.2546, -0.0273, 1.0659) -- cycle;
\fill[blue!15.0, opacity=0.5] (1.2546, -0.0273, 1.0659) -- (1.3036, -0.0273, 1.0674) -- (1.3040, -0.0300, 1.1174) -- (1.2550, -0.0300, 1.1159) -- cycle;
\fill[blue!15.0, opacity=0.5] (1.2550, -0.0300, 1.1159) -- (1.3040, -0.0300, 1.1174) -- (1.3044, -0.0328, 1.1674) -- (1.2555, -0.0328, 1.1659) -- cycle;
\fill[blue!15.1, opacity=0.5] (1.2555, -0.0328, 1.1659) -- (1.3044, -0.0328, 1.1674) -- (1.3047, -0.0356, 1.2174) -- (1.2559, -0.0356, 1.2159) -- cycle;
\fill[blue!15.1, opacity=0.5] (1.2559, -0.0356, 1.2159) -- (1.3047, -0.0356, 1.2174) -- (1.3051, -0.0385, 1.2674) -- (1.2564, -0.0385, 1.2659) -- cycle;
\fill[blue!15.2, opacity=0.5] (1.2564, -0.0385, 1.2659) -- (1.3051, -0.0385, 1.2674) -- (1.3055, -0.0415, 1.3174) -- (1.2569, -0.0415, 1.3159) -- cycle;
\fill[blue!15.2, opacity=0.5] (1.2569, -0.0415, 1.3159) -- (1.3055, -0.0415, 1.3174) -- (1.3059, -0.0445, 1.3674) -- (1.2574, -0.0445, 1.3659) -- cycle;
\fill[blue!15.4, opacity=0.5] (1.2574, -0.0445, 1.3659) -- (1.3059, -0.0445, 1.3674) -- (1.3063, -0.0475, 1.4174) -- (1.2579, -0.0475, 1.4159) -- cycle;
\fill[blue!15.5, opacity=0.5] (1.2579, -0.0475, 1.4159) -- (1.3063, -0.0475, 1.4174) -- (1.3067, -0.0506, 1.4674) -- (1.2584, -0.0506, 1.4659) -- cycle;
\fill[blue!15.8, opacity=0.5] (1.2584, -0.0506, 1.4659) -- (1.3067, -0.0506, 1.4674) -- (1.3072, -0.0537, 1.5174) -- (1.2590, -0.0537, 1.5159) -- cycle;
\fill[blue!16.1, opacity=0.5] (1.2590, -0.0537, 1.5159) -- (1.3072, -0.0537, 1.5174) -- (1.3076, -0.0569, 1.5674) -- (1.2595, -0.0569, 1.5659) -- cycle;
\fill[blue!16.4, opacity=0.5] (1.2595, -0.0569, 1.5659) -- (1.3076, -0.0569, 1.5674) -- (1.3080, -0.0600, 1.6174) -- (1.2600, -0.0600, 1.6159) -- cycle;
\fill[blue!16.9, opacity=0.5] (1.2600, -0.0600, 1.6159) -- (1.3080, -0.0600, 1.6174) -- (1.3084, -0.0631, 1.6674) -- (1.2605, -0.0631, 1.6659) -- cycle;
\fill[blue!17.5, opacity=0.5] (1.2605, -0.0631, 1.6659) -- (1.3084, -0.0631, 1.6674) -- (1.3088, -0.0663, 1.7174) -- (1.2610, -0.0663, 1.7159) -- cycle;
\fill[blue!18.2, opacity=0.5] (1.2610, -0.0663, 1.7159) -- (1.3088, -0.0663, 1.7174) -- (1.3093, -0.0694, 1.7674) -- (1.2616, -0.0694, 1.7659) -- cycle;
\fill[blue!19.0, opacity=0.5] (1.2616, -0.0694, 1.7659) -- (1.3093, -0.0694, 1.7674) -- (1.3097, -0.0725, 1.8174) -- (1.2621, -0.0725, 1.8159) -- cycle;
\fill[blue!19.9, opacity=0.5] (1.2621, -0.0725, 1.8159) -- (1.3097, -0.0725, 1.8174) -- (1.3101, -0.0755, 1.8674) -- (1.2626, -0.0755, 1.8659) -- cycle;
\fill[blue!20.9, opacity=0.5] (1.2626, -0.0755, 1.8659) -- (1.3101, -0.0755, 1.8674) -- (1.3105, -0.0785, 1.9174) -- (1.2631, -0.0785, 1.9159) -- cycle;
\fill[blue!22.1, opacity=0.5] (1.2631, -0.0785, 1.9159) -- (1.3105, -0.0785, 1.9174) -- (1.3109, -0.0815, 1.9674) -- (1.2636, -0.0815, 1.9659) -- cycle;
\fill[blue!23.3, opacity=0.5] (1.2636, -0.0815, 1.9659) -- (1.3109, -0.0815, 1.9674) -- (1.3113, -0.0844, 2.0174) -- (1.2641, -0.0844, 2.0159) -- cycle;
\fill[blue!24.7, opacity=0.5] (1.2641, -0.0844, 2.0159) -- (1.3113, -0.0844, 2.0174) -- (1.3116, -0.0872, 2.0674) -- (1.2645, -0.0872, 2.0659) -- cycle;
\fill[blue!26.2, opacity=0.5] (1.2645, -0.0872, 2.0659) -- (1.3116, -0.0872, 2.0674) -- (1.3120, -0.0900, 2.1174) -- (1.2650, -0.0900, 2.1159) -- cycle;
\fill[blue!27.8, opacity=0.5] (1.2650, -0.0900, 2.1159) -- (1.3120, -0.0900, 2.1174) -- (1.3124, -0.0927, 2.1674) -- (1.2654, -0.0927, 2.1659) -- cycle;
\fill[blue!29.4, opacity=0.5] (1.2654, -0.0927, 2.1659) -- (1.3124, -0.0927, 2.1674) -- (1.3127, -0.0953, 2.2174) -- (1.2659, -0.0953, 2.2159) -- cycle;
\fill[blue!31.1, opacity=0.5] (1.2659, -0.0953, 2.2159) -- (1.3127, -0.0953, 2.2174) -- (1.3130, -0.0978, 2.2674) -- (1.2663, -0.0978, 2.2659) -- cycle;
\fill[blue!32.9, opacity=0.5] (1.2663, -0.0978, 2.2659) -- (1.3130, -0.0978, 2.2674) -- (1.3134, -0.1001, 2.3174) -- (1.2667, -0.1001, 2.3159) -- cycle;
\fill[blue!34.6, opacity=0.5] (1.2667, -0.1001, 2.3159) -- (1.3134, -0.1001, 2.3174) -- (1.3137, -0.1024, 2.3674) -- (1.2671, -0.1024, 2.3659) -- cycle;
\fill[blue!36.4, opacity=0.5] (1.2671, -0.1024, 2.3659) -- (1.3137, -0.1024, 2.3674) -- (1.3139, -0.1046, 2.4174) -- (1.2674, -0.1046, 2.4159) -- cycle;
\fill[blue!38.1, opacity=0.5] (1.2674, -0.1046, 2.4159) -- (1.3139, -0.1046, 2.4174) -- (1.3142, -0.1066, 2.4674) -- (1.2678, -0.1066, 2.4659) -- cycle;
\fill[blue!39.8, opacity=0.5] (1.2678, -0.1066, 2.4659) -- (1.3142, -0.1066, 2.4674) -- (1.3145, -0.1085, 2.5174) -- (1.2681, -0.1085, 2.5159) -- cycle;
\fill[blue!41.5, opacity=0.5] (1.2681, -0.1085, 2.5159) -- (1.3145, -0.1085, 2.5174) -- (1.3147, -0.1103, 2.5674) -- (1.2684, -0.1103, 2.5659) -- cycle;
\fill[blue!43.1, opacity=0.5] (1.2684, -0.1103, 2.5659) -- (1.3147, -0.1103, 2.5674) -- (1.3149, -0.1120, 2.6174) -- (1.2687, -0.1120, 2.6159) -- cycle;
\fill[blue!44.6, opacity=0.5] (1.2687, -0.1120, 2.6159) -- (1.3149, -0.1120, 2.6174) -- (1.3151, -0.1135, 2.6674) -- (1.2689, -0.1135, 2.6659) -- cycle;
\fill[blue!46.0, opacity=0.5] (1.2689, -0.1135, 2.6659) -- (1.3151, -0.1135, 2.6674) -- (1.3153, -0.1148, 2.7174) -- (1.2691, -0.1148, 2.7159) -- cycle;
\fill[blue!47.3, opacity=0.5] (1.2691, -0.1148, 2.7159) -- (1.3153, -0.1148, 2.7174) -- (1.3155, -0.1160, 2.7674) -- (1.2693, -0.1160, 2.7659) -- cycle;
\fill[blue!48.5, opacity=0.5] (1.2693, -0.1160, 2.7659) -- (1.3155, -0.1160, 2.7674) -- (1.3156, -0.1171, 2.8174) -- (1.2695, -0.1171, 2.8159) -- cycle;
\fill[blue!49.6, opacity=0.5] (1.2695, -0.1171, 2.8159) -- (1.3156, -0.1171, 2.8174) -- (1.3157, -0.1180, 2.8674) -- (1.2697, -0.1180, 2.8659) -- cycle;
\fill[blue!50.6, opacity=0.5] (1.2697, -0.1180, 2.8659) -- (1.3157, -0.1180, 2.8674) -- (1.3158, -0.1187, 2.9174) -- (1.2698, -0.1187, 2.9159) -- cycle;
\fill[blue!51.4, opacity=0.5] (1.2698, -0.1187, 2.9159) -- (1.3158, -0.1187, 2.9174) -- (1.3159, -0.1193, 2.9674) -- (1.2699, -0.1193, 2.9659) -- cycle;
\fill[blue!52.1, opacity=0.5] (1.2699, -0.1193, 2.9659) -- (1.3159, -0.1193, 2.9674) -- (1.3160, -0.1197, 3.0174) -- (1.2699, -0.1197, 3.0159) -- cycle;
\fill[blue!52.6, opacity=0.5] (1.2699, -0.1197, 3.0159) -- (1.3160, -0.1197, 3.0174) -- (1.3160, -0.1199, 3.0674) -- (1.2700, -0.1199, 3.0659) -- cycle;
\fill[blue!53.1, opacity=0.5] (1.2700, -0.1199, 3.0659) -- (1.3160, -0.1199, 3.0674) -- (1.3160, -0.1200, 3.1174) -- (1.2700, -0.1200, 3.1159) -- cycle;
\fill[blue!15.0, opacity=0.5] (1.3000, -0.0000, 0.1174) -- (1.3500, -0.0000, 0.1185) -- (1.3500, -0.0001, 0.1685) -- (1.3000, -0.0001, 0.1674) -- cycle;
\fill[blue!15.0, opacity=0.5] (1.3000, -0.0001, 0.1674) -- (1.3500, -0.0001, 0.1685) -- (1.3500, -0.0003, 0.2185) -- (1.3000, -0.0003, 0.2174) -- cycle;
\fill[blue!15.0, opacity=0.5] (1.3000, -0.0003, 0.2174) -- (1.3500, -0.0003, 0.2185) -- (1.3501, -0.0007, 0.2685) -- (1.3001, -0.0007, 0.2674) -- cycle;
\fill[blue!15.0, opacity=0.5] (1.3001, -0.0007, 0.2674) -- (1.3501, -0.0007, 0.2685) -- (1.3501, -0.0013, 0.3185) -- (1.3002, -0.0013, 0.3174) -- cycle;
\fill[blue!15.0, opacity=0.5] (1.3002, -0.0013, 0.3174) -- (1.3501, -0.0013, 0.3185) -- (1.3502, -0.0020, 0.3685) -- (1.3003, -0.0020, 0.3674) -- cycle;
\fill[blue!15.0, opacity=0.5] (1.3003, -0.0020, 0.3674) -- (1.3502, -0.0020, 0.3685) -- (1.3503, -0.0029, 0.4185) -- (1.3004, -0.0029, 0.4174) -- cycle;
\fill[blue!15.0, opacity=0.5] (1.3004, -0.0029, 0.4174) -- (1.3503, -0.0029, 0.4185) -- (1.3504, -0.0040, 0.4685) -- (1.3005, -0.0040, 0.4674) -- cycle;
\fill[blue!15.0, opacity=0.5] (1.3005, -0.0040, 0.4674) -- (1.3504, -0.0040, 0.4685) -- (1.3505, -0.0052, 0.5185) -- (1.3007, -0.0052, 0.5174) -- cycle;
\fill[blue!15.0, opacity=0.5] (1.3007, -0.0052, 0.5174) -- (1.3505, -0.0052, 0.5185) -- (1.3507, -0.0065, 0.5685) -- (1.3009, -0.0065, 0.5674) -- cycle;
\fill[blue!15.0, opacity=0.5] (1.3009, -0.0065, 0.5674) -- (1.3507, -0.0065, 0.5685) -- (1.3508, -0.0080, 0.6185) -- (1.3011, -0.0080, 0.6174) -- cycle;
\fill[blue!15.0, opacity=0.5] (1.3011, -0.0080, 0.6174) -- (1.3508, -0.0080, 0.6185) -- (1.3510, -0.0097, 0.6685) -- (1.3013, -0.0097, 0.6674) -- cycle;
\fill[blue!15.0, opacity=0.5] (1.3013, -0.0097, 0.6674) -- (1.3510, -0.0097, 0.6685) -- (1.3511, -0.0115, 0.7185) -- (1.3015, -0.0115, 0.7174) -- cycle;
\fill[blue!15.0, opacity=0.5] (1.3015, -0.0115, 0.7174) -- (1.3511, -0.0115, 0.7185) -- (1.3513, -0.0134, 0.7685) -- (1.3018, -0.0134, 0.7674) -- cycle;
\fill[blue!15.0, opacity=0.5] (1.3018, -0.0134, 0.7674) -- (1.3513, -0.0134, 0.7685) -- (1.3515, -0.0154, 0.8185) -- (1.3021, -0.0154, 0.8174) -- cycle;
\fill[blue!15.0, opacity=0.5] (1.3021, -0.0154, 0.8174) -- (1.3515, -0.0154, 0.8185) -- (1.3518, -0.0176, 0.8685) -- (1.3023, -0.0176, 0.8674) -- cycle;
\fill[blue!15.0, opacity=0.5] (1.3023, -0.0176, 0.8674) -- (1.3518, -0.0176, 0.8685) -- (1.3520, -0.0199, 0.9185) -- (1.3026, -0.0199, 0.9174) -- cycle;
\fill[blue!15.0, opacity=0.5] (1.3026, -0.0199, 0.9174) -- (1.3520, -0.0199, 0.9185) -- (1.3522, -0.0222, 0.9685) -- (1.3030, -0.0222, 0.9674) -- cycle;
\fill[blue!15.0, opacity=0.5] (1.3030, -0.0222, 0.9674) -- (1.3522, -0.0222, 0.9685) -- (1.3525, -0.0247, 1.0185) -- (1.3033, -0.0247, 1.0174) -- cycle;
\fill[blue!15.0, opacity=0.5] (1.3033, -0.0247, 1.0174) -- (1.3525, -0.0247, 1.0185) -- (1.3527, -0.0273, 1.0685) -- (1.3036, -0.0273, 1.0674) -- cycle;
\fill[blue!15.0, opacity=0.5] (1.3036, -0.0273, 1.0674) -- (1.3527, -0.0273, 1.0685) -- (1.3530, -0.0300, 1.1185) -- (1.3040, -0.0300, 1.1174) -- cycle;
\fill[blue!15.0, opacity=0.5] (1.3040, -0.0300, 1.1174) -- (1.3530, -0.0300, 1.1185) -- (1.3533, -0.0328, 1.1685) -- (1.3044, -0.0328, 1.1674) -- cycle;
\fill[blue!15.0, opacity=0.5] (1.3044, -0.0328, 1.1674) -- (1.3533, -0.0328, 1.1685) -- (1.3536, -0.0356, 1.2185) -- (1.3047, -0.0356, 1.2174) -- cycle;
\fill[blue!15.1, opacity=0.5] (1.3047, -0.0356, 1.2174) -- (1.3536, -0.0356, 1.2185) -- (1.3538, -0.0385, 1.2685) -- (1.3051, -0.0385, 1.2674) -- cycle;
\fill[blue!15.1, opacity=0.5] (1.3051, -0.0385, 1.2674) -- (1.3538, -0.0385, 1.2685) -- (1.3541, -0.0415, 1.3185) -- (1.3055, -0.0415, 1.3174) -- cycle;
\fill[blue!15.2, opacity=0.5] (1.3055, -0.0415, 1.3174) -- (1.3541, -0.0415, 1.3185) -- (1.3544, -0.0445, 1.3685) -- (1.3059, -0.0445, 1.3674) -- cycle;
\fill[blue!15.3, opacity=0.5] (1.3059, -0.0445, 1.3674) -- (1.3544, -0.0445, 1.3685) -- (1.3548, -0.0475, 1.4185) -- (1.3063, -0.0475, 1.4174) -- cycle;
\fill[blue!15.5, opacity=0.5] (1.3063, -0.0475, 1.4174) -- (1.3548, -0.0475, 1.4185) -- (1.3551, -0.0506, 1.4685) -- (1.3067, -0.0506, 1.4674) -- cycle;
\fill[blue!15.7, opacity=0.5] (1.3067, -0.0506, 1.4674) -- (1.3551, -0.0506, 1.4685) -- (1.3554, -0.0537, 1.5185) -- (1.3072, -0.0537, 1.5174) -- cycle;
\fill[blue!16.0, opacity=0.5] (1.3072, -0.0537, 1.5174) -- (1.3554, -0.0537, 1.5185) -- (1.3557, -0.0569, 1.5685) -- (1.3076, -0.0569, 1.5674) -- cycle;
\fill[blue!16.3, opacity=0.5] (1.3076, -0.0569, 1.5674) -- (1.3557, -0.0569, 1.5685) -- (1.3560, -0.0600, 1.6185) -- (1.3080, -0.0600, 1.6174) -- cycle;
\fill[blue!16.8, opacity=0.5] (1.3080, -0.0600, 1.6174) -- (1.3560, -0.0600, 1.6185) -- (1.3563, -0.0631, 1.6685) -- (1.3084, -0.0631, 1.6674) -- cycle;
\fill[blue!17.3, opacity=0.5] (1.3084, -0.0631, 1.6674) -- (1.3563, -0.0631, 1.6685) -- (1.3566, -0.0663, 1.7185) -- (1.3088, -0.0663, 1.7174) -- cycle;
\fill[blue!17.9, opacity=0.5] (1.3088, -0.0663, 1.7174) -- (1.3566, -0.0663, 1.7185) -- (1.3569, -0.0694, 1.7685) -- (1.3093, -0.0694, 1.7674) -- cycle;
\fill[blue!18.7, opacity=0.5] (1.3093, -0.0694, 1.7674) -- (1.3569, -0.0694, 1.7685) -- (1.3572, -0.0725, 1.8185) -- (1.3097, -0.0725, 1.8174) -- cycle;
\fill[blue!19.6, opacity=0.5] (1.3097, -0.0725, 1.8174) -- (1.3572, -0.0725, 1.8185) -- (1.3576, -0.0755, 1.8685) -- (1.3101, -0.0755, 1.8674) -- cycle;
\fill[blue!20.5, opacity=0.5] (1.3101, -0.0755, 1.8674) -- (1.3576, -0.0755, 1.8685) -- (1.3579, -0.0785, 1.9185) -- (1.3105, -0.0785, 1.9174) -- cycle;
\fill[blue!21.7, opacity=0.5] (1.3105, -0.0785, 1.9174) -- (1.3579, -0.0785, 1.9185) -- (1.3582, -0.0815, 1.9685) -- (1.3109, -0.0815, 1.9674) -- cycle;
\fill[blue!22.9, opacity=0.5] (1.3109, -0.0815, 1.9674) -- (1.3582, -0.0815, 1.9685) -- (1.3584, -0.0844, 2.0185) -- (1.3113, -0.0844, 2.0174) -- cycle;
\fill[blue!24.2, opacity=0.5] (1.3113, -0.0844, 2.0174) -- (1.3584, -0.0844, 2.0185) -- (1.3587, -0.0872, 2.0685) -- (1.3116, -0.0872, 2.0674) -- cycle;
\fill[blue!25.7, opacity=0.5] (1.3116, -0.0872, 2.0674) -- (1.3587, -0.0872, 2.0685) -- (1.3590, -0.0900, 2.1185) -- (1.3120, -0.0900, 2.1174) -- cycle;
\fill[blue!27.2, opacity=0.5] (1.3120, -0.0900, 2.1174) -- (1.3590, -0.0900, 2.1185) -- (1.3593, -0.0927, 2.1685) -- (1.3124, -0.0927, 2.1674) -- cycle;
\fill[blue!28.8, opacity=0.5] (1.3124, -0.0927, 2.1674) -- (1.3593, -0.0927, 2.1685) -- (1.3595, -0.0953, 2.2185) -- (1.3127, -0.0953, 2.2174) -- cycle;
\fill[blue!30.4, opacity=0.5] (1.3127, -0.0953, 2.2174) -- (1.3595, -0.0953, 2.2185) -- (1.3598, -0.0978, 2.2685) -- (1.3130, -0.0978, 2.2674) -- cycle;
\fill[blue!32.1, opacity=0.5] (1.3130, -0.0978, 2.2674) -- (1.3598, -0.0978, 2.2685) -- (1.3600, -0.1001, 2.3185) -- (1.3134, -0.1001, 2.3174) -- cycle;
\fill[blue!33.9, opacity=0.5] (1.3134, -0.1001, 2.3174) -- (1.3600, -0.1001, 2.3185) -- (1.3602, -0.1024, 2.3685) -- (1.3137, -0.1024, 2.3674) -- cycle;
\fill[blue!35.6, opacity=0.5] (1.3137, -0.1024, 2.3674) -- (1.3602, -0.1024, 2.3685) -- (1.3605, -0.1046, 2.4185) -- (1.3139, -0.1046, 2.4174) -- cycle;
\fill[blue!37.3, opacity=0.5] (1.3139, -0.1046, 2.4174) -- (1.3605, -0.1046, 2.4185) -- (1.3607, -0.1066, 2.4685) -- (1.3142, -0.1066, 2.4674) -- cycle;
\fill[blue!39.0, opacity=0.5] (1.3142, -0.1066, 2.4674) -- (1.3607, -0.1066, 2.4685) -- (1.3609, -0.1085, 2.5185) -- (1.3145, -0.1085, 2.5174) -- cycle;
\fill[blue!40.7, opacity=0.5] (1.3145, -0.1085, 2.5174) -- (1.3609, -0.1085, 2.5185) -- (1.3610, -0.1103, 2.5685) -- (1.3147, -0.1103, 2.5674) -- cycle;
\fill[blue!42.3, opacity=0.5] (1.3147, -0.1103, 2.5674) -- (1.3610, -0.1103, 2.5685) -- (1.3612, -0.1120, 2.6185) -- (1.3149, -0.1120, 2.6174) -- cycle;
\fill[blue!43.8, opacity=0.5] (1.3149, -0.1120, 2.6174) -- (1.3612, -0.1120, 2.6185) -- (1.3613, -0.1135, 2.6685) -- (1.3151, -0.1135, 2.6674) -- cycle;
\fill[blue!45.2, opacity=0.5] (1.3151, -0.1135, 2.6674) -- (1.3613, -0.1135, 2.6685) -- (1.3615, -0.1148, 2.7185) -- (1.3153, -0.1148, 2.7174) -- cycle;
\fill[blue!46.6, opacity=0.5] (1.3153, -0.1148, 2.7174) -- (1.3615, -0.1148, 2.7185) -- (1.3616, -0.1160, 2.7685) -- (1.3155, -0.1160, 2.7674) -- cycle;
\fill[blue!47.8, opacity=0.5] (1.3155, -0.1160, 2.7674) -- (1.3616, -0.1160, 2.7685) -- (1.3617, -0.1171, 2.8185) -- (1.3156, -0.1171, 2.8174) -- cycle;
\fill[blue!48.9, opacity=0.5] (1.3156, -0.1171, 2.8174) -- (1.3617, -0.1171, 2.8185) -- (1.3618, -0.1180, 2.8685) -- (1.3157, -0.1180, 2.8674) -- cycle;
\fill[blue!49.9, opacity=0.5] (1.3157, -0.1180, 2.8674) -- (1.3618, -0.1180, 2.8685) -- (1.3619, -0.1187, 2.9185) -- (1.3158, -0.1187, 2.9174) -- cycle;
\fill[blue!50.7, opacity=0.5] (1.3158, -0.1187, 2.9174) -- (1.3619, -0.1187, 2.9185) -- (1.3619, -0.1193, 2.9685) -- (1.3159, -0.1193, 2.9674) -- cycle;
\fill[blue!51.5, opacity=0.5] (1.3159, -0.1193, 2.9674) -- (1.3619, -0.1193, 2.9685) -- (1.3620, -0.1197, 3.0185) -- (1.3160, -0.1197, 3.0174) -- cycle;
\fill[blue!52.1, opacity=0.5] (1.3160, -0.1197, 3.0174) -- (1.3620, -0.1197, 3.0185) -- (1.3620, -0.1199, 3.0685) -- (1.3160, -0.1199, 3.0674) -- cycle;
\fill[blue!52.5, opacity=0.5] (1.3160, -0.1199, 3.0674) -- (1.3620, -0.1199, 3.0685) -- (1.3620, -0.1200, 3.1185) -- (1.3160, -0.1200, 3.1174) -- cycle;
\fill[blue!15.0, opacity=0.5] (1.3500, -0.0000, 0.1185) -- (1.4000, -0.0000, 0.1193) -- (1.4000, -0.0001, 0.1693) -- (1.3500, -0.0001, 0.1685) -- cycle;
\fill[blue!15.0, opacity=0.5] (1.3500, -0.0001, 0.1685) -- (1.4000, -0.0001, 0.1693) -- (1.4000, -0.0003, 0.2193) -- (1.3500, -0.0003, 0.2185) -- cycle;
\fill[blue!15.0, opacity=0.5] (1.3500, -0.0003, 0.2185) -- (1.4000, -0.0003, 0.2193) -- (1.4000, -0.0007, 0.2693) -- (1.3501, -0.0007, 0.2685) -- cycle;
\fill[blue!15.0, opacity=0.5] (1.3501, -0.0007, 0.2685) -- (1.4000, -0.0007, 0.2693) -- (1.4001, -0.0013, 0.3193) -- (1.3501, -0.0013, 0.3185) -- cycle;
\fill[blue!15.0, opacity=0.5] (1.3501, -0.0013, 0.3185) -- (1.4001, -0.0013, 0.3193) -- (1.4001, -0.0020, 0.3693) -- (1.3502, -0.0020, 0.3685) -- cycle;
\fill[blue!15.0, opacity=0.5] (1.3502, -0.0020, 0.3685) -- (1.4001, -0.0020, 0.3693) -- (1.4002, -0.0029, 0.4193) -- (1.3503, -0.0029, 0.4185) -- cycle;
\fill[blue!15.0, opacity=0.5] (1.3503, -0.0029, 0.4185) -- (1.4002, -0.0029, 0.4193) -- (1.4003, -0.0040, 0.4693) -- (1.3504, -0.0040, 0.4685) -- cycle;
\fill[blue!15.0, opacity=0.5] (1.3504, -0.0040, 0.4685) -- (1.4003, -0.0040, 0.4693) -- (1.4003, -0.0052, 0.5193) -- (1.3505, -0.0052, 0.5185) -- cycle;
\fill[blue!15.0, opacity=0.5] (1.3505, -0.0052, 0.5185) -- (1.4003, -0.0052, 0.5193) -- (1.4004, -0.0065, 0.5693) -- (1.3507, -0.0065, 0.5685) -- cycle;
\fill[blue!15.0, opacity=0.5] (1.3507, -0.0065, 0.5685) -- (1.4004, -0.0065, 0.5693) -- (1.4005, -0.0080, 0.6193) -- (1.3508, -0.0080, 0.6185) -- cycle;
\fill[blue!15.0, opacity=0.5] (1.3508, -0.0080, 0.6185) -- (1.4005, -0.0080, 0.6193) -- (1.4006, -0.0097, 0.6693) -- (1.3510, -0.0097, 0.6685) -- cycle;
\fill[blue!15.0, opacity=0.5] (1.3510, -0.0097, 0.6685) -- (1.4006, -0.0097, 0.6693) -- (1.4008, -0.0115, 0.7193) -- (1.3511, -0.0115, 0.7185) -- cycle;
\fill[blue!15.0, opacity=0.5] (1.3511, -0.0115, 0.7185) -- (1.4008, -0.0115, 0.7193) -- (1.4009, -0.0134, 0.7693) -- (1.3513, -0.0134, 0.7685) -- cycle;
\fill[blue!15.0, opacity=0.5] (1.3513, -0.0134, 0.7685) -- (1.4009, -0.0134, 0.7693) -- (1.4010, -0.0154, 0.8193) -- (1.3515, -0.0154, 0.8185) -- cycle;
\fill[blue!15.0, opacity=0.5] (1.3515, -0.0154, 0.8185) -- (1.4010, -0.0154, 0.8193) -- (1.4012, -0.0176, 0.8693) -- (1.3518, -0.0176, 0.8685) -- cycle;
\fill[blue!15.0, opacity=0.5] (1.3518, -0.0176, 0.8685) -- (1.4012, -0.0176, 0.8693) -- (1.4013, -0.0199, 0.9193) -- (1.3520, -0.0199, 0.9185) -- cycle;
\fill[blue!15.0, opacity=0.5] (1.3520, -0.0199, 0.9185) -- (1.4013, -0.0199, 0.9193) -- (1.4015, -0.0222, 0.9693) -- (1.3522, -0.0222, 0.9685) -- cycle;
\fill[blue!15.0, opacity=0.5] (1.3522, -0.0222, 0.9685) -- (1.4015, -0.0222, 0.9693) -- (1.4016, -0.0247, 1.0193) -- (1.3525, -0.0247, 1.0185) -- cycle;
\fill[blue!15.0, opacity=0.5] (1.3525, -0.0247, 1.0185) -- (1.4016, -0.0247, 1.0193) -- (1.4018, -0.0273, 1.0693) -- (1.3527, -0.0273, 1.0685) -- cycle;
\fill[blue!15.0, opacity=0.5] (1.3527, -0.0273, 1.0685) -- (1.4018, -0.0273, 1.0693) -- (1.4020, -0.0300, 1.1193) -- (1.3530, -0.0300, 1.1185) -- cycle;
\fill[blue!15.0, opacity=0.5] (1.3530, -0.0300, 1.1185) -- (1.4020, -0.0300, 1.1193) -- (1.4022, -0.0328, 1.1693) -- (1.3533, -0.0328, 1.1685) -- cycle;
\fill[blue!15.1, opacity=0.5] (1.3533, -0.0328, 1.1685) -- (1.4022, -0.0328, 1.1693) -- (1.4024, -0.0356, 1.2193) -- (1.3536, -0.0356, 1.2185) -- cycle;
\fill[blue!15.1, opacity=0.5] (1.3536, -0.0356, 1.2185) -- (1.4024, -0.0356, 1.2193) -- (1.4026, -0.0385, 1.2693) -- (1.3538, -0.0385, 1.2685) -- cycle;
\fill[blue!15.1, opacity=0.5] (1.3538, -0.0385, 1.2685) -- (1.4026, -0.0385, 1.2693) -- (1.4028, -0.0415, 1.3193) -- (1.3541, -0.0415, 1.3185) -- cycle;
\fill[blue!15.2, opacity=0.5] (1.3541, -0.0415, 1.3185) -- (1.4028, -0.0415, 1.3193) -- (1.4030, -0.0445, 1.3693) -- (1.3544, -0.0445, 1.3685) -- cycle;
\fill[blue!15.3, opacity=0.5] (1.3544, -0.0445, 1.3685) -- (1.4030, -0.0445, 1.3693) -- (1.4032, -0.0475, 1.4193) -- (1.3548, -0.0475, 1.4185) -- cycle;
\fill[blue!15.5, opacity=0.5] (1.3548, -0.0475, 1.4185) -- (1.4032, -0.0475, 1.4193) -- (1.4034, -0.0506, 1.4693) -- (1.3551, -0.0506, 1.4685) -- cycle;
\fill[blue!15.7, opacity=0.5] (1.3551, -0.0506, 1.4685) -- (1.4034, -0.0506, 1.4693) -- (1.4036, -0.0537, 1.5193) -- (1.3554, -0.0537, 1.5185) -- cycle;
\fill[blue!16.0, opacity=0.5] (1.3554, -0.0537, 1.5185) -- (1.4036, -0.0537, 1.5193) -- (1.4038, -0.0569, 1.5693) -- (1.3557, -0.0569, 1.5685) -- cycle;
\fill[blue!16.3, opacity=0.5] (1.3557, -0.0569, 1.5685) -- (1.4038, -0.0569, 1.5693) -- (1.4040, -0.0600, 1.6193) -- (1.3560, -0.0600, 1.6185) -- cycle;
\fill[blue!16.8, opacity=0.5] (1.3560, -0.0600, 1.6185) -- (1.4040, -0.0600, 1.6193) -- (1.4042, -0.0631, 1.6693) -- (1.3563, -0.0631, 1.6685) -- cycle;
\fill[blue!17.3, opacity=0.5] (1.3563, -0.0631, 1.6685) -- (1.4042, -0.0631, 1.6693) -- (1.4044, -0.0663, 1.7193) -- (1.3566, -0.0663, 1.7185) -- cycle;
\fill[blue!18.0, opacity=0.5] (1.3566, -0.0663, 1.7185) -- (1.4044, -0.0663, 1.7193) -- (1.4046, -0.0694, 1.7693) -- (1.3569, -0.0694, 1.7685) -- cycle;
\fill[blue!18.7, opacity=0.5] (1.3569, -0.0694, 1.7685) -- (1.4046, -0.0694, 1.7693) -- (1.4048, -0.0725, 1.8193) -- (1.3572, -0.0725, 1.8185) -- cycle;
\fill[blue!19.6, opacity=0.5] (1.3572, -0.0725, 1.8185) -- (1.4048, -0.0725, 1.8193) -- (1.4050, -0.0755, 1.8693) -- (1.3576, -0.0755, 1.8685) -- cycle;
\fill[blue!20.6, opacity=0.5] (1.3576, -0.0755, 1.8685) -- (1.4050, -0.0755, 1.8693) -- (1.4052, -0.0785, 1.9193) -- (1.3579, -0.0785, 1.9185) -- cycle;
\fill[blue!21.7, opacity=0.5] (1.3579, -0.0785, 1.9185) -- (1.4052, -0.0785, 1.9193) -- (1.4054, -0.0815, 1.9693) -- (1.3582, -0.0815, 1.9685) -- cycle;
\fill[blue!22.9, opacity=0.5] (1.3582, -0.0815, 1.9685) -- (1.4054, -0.0815, 1.9693) -- (1.4056, -0.0844, 2.0193) -- (1.3584, -0.0844, 2.0185) -- cycle;
\fill[blue!24.3, opacity=0.5] (1.3584, -0.0844, 2.0185) -- (1.4056, -0.0844, 2.0193) -- (1.4058, -0.0872, 2.0693) -- (1.3587, -0.0872, 2.0685) -- cycle;
\fill[blue!25.7, opacity=0.5] (1.3587, -0.0872, 2.0685) -- (1.4058, -0.0872, 2.0693) -- (1.4060, -0.0900, 2.1193) -- (1.3590, -0.0900, 2.1185) -- cycle;
\fill[blue!27.3, opacity=0.5] (1.3590, -0.0900, 2.1185) -- (1.4060, -0.0900, 2.1193) -- (1.4062, -0.0927, 2.1693) -- (1.3593, -0.0927, 2.1685) -- cycle;
\fill[blue!28.9, opacity=0.5] (1.3593, -0.0927, 2.1685) -- (1.4062, -0.0927, 2.1693) -- (1.4064, -0.0953, 2.2193) -- (1.3595, -0.0953, 2.2185) -- cycle;
\fill[blue!30.5, opacity=0.5] (1.3595, -0.0953, 2.2185) -- (1.4064, -0.0953, 2.2193) -- (1.4065, -0.0978, 2.2693) -- (1.3598, -0.0978, 2.2685) -- cycle;
\fill[blue!32.2, opacity=0.5] (1.3598, -0.0978, 2.2685) -- (1.4065, -0.0978, 2.2693) -- (1.4067, -0.1001, 2.3193) -- (1.3600, -0.1001, 2.3185) -- cycle;
\fill[blue!34.0, opacity=0.5] (1.3600, -0.1001, 2.3185) -- (1.4067, -0.1001, 2.3193) -- (1.4068, -0.1024, 2.3693) -- (1.3602, -0.1024, 2.3685) -- cycle;
\fill[blue!35.7, opacity=0.5] (1.3602, -0.1024, 2.3685) -- (1.4068, -0.1024, 2.3693) -- (1.4070, -0.1046, 2.4193) -- (1.3605, -0.1046, 2.4185) -- cycle;
\fill[blue!37.4, opacity=0.5] (1.3605, -0.1046, 2.4185) -- (1.4070, -0.1046, 2.4193) -- (1.4071, -0.1066, 2.4693) -- (1.3607, -0.1066, 2.4685) -- cycle;
\fill[blue!39.1, opacity=0.5] (1.3607, -0.1066, 2.4685) -- (1.4071, -0.1066, 2.4693) -- (1.4072, -0.1085, 2.5193) -- (1.3609, -0.1085, 2.5185) -- cycle;
\fill[blue!40.8, opacity=0.5] (1.3609, -0.1085, 2.5185) -- (1.4072, -0.1085, 2.5193) -- (1.4074, -0.1103, 2.5693) -- (1.3610, -0.1103, 2.5685) -- cycle;
\fill[blue!42.4, opacity=0.5] (1.3610, -0.1103, 2.5685) -- (1.4074, -0.1103, 2.5693) -- (1.4075, -0.1120, 2.6193) -- (1.3612, -0.1120, 2.6185) -- cycle;
\fill[blue!43.9, opacity=0.5] (1.3612, -0.1120, 2.6185) -- (1.4075, -0.1120, 2.6193) -- (1.4076, -0.1135, 2.6693) -- (1.3613, -0.1135, 2.6685) -- cycle;
\fill[blue!45.3, opacity=0.5] (1.3613, -0.1135, 2.6685) -- (1.4076, -0.1135, 2.6693) -- (1.4077, -0.1148, 2.7193) -- (1.3615, -0.1148, 2.7185) -- cycle;
\fill[blue!46.7, opacity=0.5] (1.3615, -0.1148, 2.7185) -- (1.4077, -0.1148, 2.7193) -- (1.4077, -0.1160, 2.7693) -- (1.3616, -0.1160, 2.7685) -- cycle;
\fill[blue!47.9, opacity=0.5] (1.3616, -0.1160, 2.7685) -- (1.4077, -0.1160, 2.7693) -- (1.4078, -0.1171, 2.8193) -- (1.3617, -0.1171, 2.8185) -- cycle;
\fill[blue!49.0, opacity=0.5] (1.3617, -0.1171, 2.8185) -- (1.4078, -0.1171, 2.8193) -- (1.4079, -0.1180, 2.8693) -- (1.3618, -0.1180, 2.8685) -- cycle;
\fill[blue!50.0, opacity=0.5] (1.3618, -0.1180, 2.8685) -- (1.4079, -0.1180, 2.8693) -- (1.4079, -0.1187, 2.9193) -- (1.3619, -0.1187, 2.9185) -- cycle;
\fill[blue!50.8, opacity=0.5] (1.3619, -0.1187, 2.9185) -- (1.4079, -0.1187, 2.9193) -- (1.4080, -0.1193, 2.9693) -- (1.3619, -0.1193, 2.9685) -- cycle;
\fill[blue!51.5, opacity=0.5] (1.3619, -0.1193, 2.9685) -- (1.4080, -0.1193, 2.9693) -- (1.4080, -0.1197, 3.0193) -- (1.3620, -0.1197, 3.0185) -- cycle;
\fill[blue!52.1, opacity=0.5] (1.3620, -0.1197, 3.0185) -- (1.4080, -0.1197, 3.0193) -- (1.4080, -0.1199, 3.0693) -- (1.3620, -0.1199, 3.0685) -- cycle;
\fill[blue!52.6, opacity=0.5] (1.3620, -0.1199, 3.0685) -- (1.4080, -0.1199, 3.0693) -- (1.4080, -0.1200, 3.1193) -- (1.3620, -0.1200, 3.1185) -- cycle;
\fill[blue!15.0, opacity=0.5] (1.4000, -0.0000, 0.1193) -- (1.4500, -0.0000, 0.1198) -- (1.4500, -0.0001, 0.1698) -- (1.4000, -0.0001, 0.1693) -- cycle;
\fill[blue!15.0, opacity=0.5] (1.4000, -0.0001, 0.1693) -- (1.4500, -0.0001, 0.1698) -- (1.4500, -0.0003, 0.2198) -- (1.4000, -0.0003, 0.2193) -- cycle;
\fill[blue!15.0, opacity=0.5] (1.4000, -0.0003, 0.2193) -- (1.4500, -0.0003, 0.2198) -- (1.4500, -0.0007, 0.2698) -- (1.4000, -0.0007, 0.2693) -- cycle;
\fill[blue!15.0, opacity=0.5] (1.4000, -0.0007, 0.2693) -- (1.4500, -0.0007, 0.2698) -- (1.4500, -0.0013, 0.3198) -- (1.4001, -0.0013, 0.3193) -- cycle;
\fill[blue!15.0, opacity=0.5] (1.4001, -0.0013, 0.3193) -- (1.4500, -0.0013, 0.3198) -- (1.4501, -0.0020, 0.3698) -- (1.4001, -0.0020, 0.3693) -- cycle;
\fill[blue!15.0, opacity=0.5] (1.4001, -0.0020, 0.3693) -- (1.4501, -0.0020, 0.3698) -- (1.4501, -0.0029, 0.4198) -- (1.4002, -0.0029, 0.4193) -- cycle;
\fill[blue!15.0, opacity=0.5] (1.4002, -0.0029, 0.4193) -- (1.4501, -0.0029, 0.4198) -- (1.4501, -0.0040, 0.4698) -- (1.4003, -0.0040, 0.4693) -- cycle;
\fill[blue!15.0, opacity=0.5] (1.4003, -0.0040, 0.4693) -- (1.4501, -0.0040, 0.4698) -- (1.4502, -0.0052, 0.5198) -- (1.4003, -0.0052, 0.5193) -- cycle;
\fill[blue!15.0, opacity=0.5] (1.4003, -0.0052, 0.5193) -- (1.4502, -0.0052, 0.5198) -- (1.4502, -0.0065, 0.5698) -- (1.4004, -0.0065, 0.5693) -- cycle;
\fill[blue!15.0, opacity=0.5] (1.4004, -0.0065, 0.5693) -- (1.4502, -0.0065, 0.5698) -- (1.4503, -0.0080, 0.6198) -- (1.4005, -0.0080, 0.6193) -- cycle;
\fill[blue!15.0, opacity=0.5] (1.4005, -0.0080, 0.6193) -- (1.4503, -0.0080, 0.6198) -- (1.4503, -0.0097, 0.6698) -- (1.4006, -0.0097, 0.6693) -- cycle;
\fill[blue!15.0, opacity=0.5] (1.4006, -0.0097, 0.6693) -- (1.4503, -0.0097, 0.6698) -- (1.4504, -0.0115, 0.7198) -- (1.4008, -0.0115, 0.7193) -- cycle;
\fill[blue!15.0, opacity=0.5] (1.4008, -0.0115, 0.7193) -- (1.4504, -0.0115, 0.7198) -- (1.4504, -0.0134, 0.7698) -- (1.4009, -0.0134, 0.7693) -- cycle;
\fill[blue!15.0, opacity=0.5] (1.4009, -0.0134, 0.7693) -- (1.4504, -0.0134, 0.7698) -- (1.4505, -0.0154, 0.8198) -- (1.4010, -0.0154, 0.8193) -- cycle;
\fill[blue!15.0, opacity=0.5] (1.4010, -0.0154, 0.8193) -- (1.4505, -0.0154, 0.8198) -- (1.4506, -0.0176, 0.8698) -- (1.4012, -0.0176, 0.8693) -- cycle;
\fill[blue!15.0, opacity=0.5] (1.4012, -0.0176, 0.8693) -- (1.4506, -0.0176, 0.8698) -- (1.4507, -0.0199, 0.9198) -- (1.4013, -0.0199, 0.9193) -- cycle;
\fill[blue!15.0, opacity=0.5] (1.4013, -0.0199, 0.9193) -- (1.4507, -0.0199, 0.9198) -- (1.4507, -0.0222, 0.9698) -- (1.4015, -0.0222, 0.9693) -- cycle;
\fill[blue!15.0, opacity=0.5] (1.4015, -0.0222, 0.9693) -- (1.4507, -0.0222, 0.9698) -- (1.4508, -0.0247, 1.0198) -- (1.4016, -0.0247, 1.0193) -- cycle;
\fill[blue!15.0, opacity=0.5] (1.4016, -0.0247, 1.0193) -- (1.4508, -0.0247, 1.0198) -- (1.4509, -0.0273, 1.0698) -- (1.4018, -0.0273, 1.0693) -- cycle;
\fill[blue!15.0, opacity=0.5] (1.4018, -0.0273, 1.0693) -- (1.4509, -0.0273, 1.0698) -- (1.4510, -0.0300, 1.1198) -- (1.4020, -0.0300, 1.1193) -- cycle;
\fill[blue!15.0, opacity=0.5] (1.4020, -0.0300, 1.1193) -- (1.4510, -0.0300, 1.1198) -- (1.4511, -0.0328, 1.1698) -- (1.4022, -0.0328, 1.1693) -- cycle;
\fill[blue!15.1, opacity=0.5] (1.4022, -0.0328, 1.1693) -- (1.4511, -0.0328, 1.1698) -- (1.4512, -0.0356, 1.2198) -- (1.4024, -0.0356, 1.2193) -- cycle;
\fill[blue!15.1, opacity=0.5] (1.4024, -0.0356, 1.2193) -- (1.4512, -0.0356, 1.2198) -- (1.4513, -0.0385, 1.2698) -- (1.4026, -0.0385, 1.2693) -- cycle;
\fill[blue!15.2, opacity=0.5] (1.4026, -0.0385, 1.2693) -- (1.4513, -0.0385, 1.2698) -- (1.4514, -0.0415, 1.3198) -- (1.4028, -0.0415, 1.3193) -- cycle;
\fill[blue!15.3, opacity=0.5] (1.4028, -0.0415, 1.3193) -- (1.4514, -0.0415, 1.3198) -- (1.4515, -0.0445, 1.3698) -- (1.4030, -0.0445, 1.3693) -- cycle;
\fill[blue!15.4, opacity=0.5] (1.4030, -0.0445, 1.3693) -- (1.4515, -0.0445, 1.3698) -- (1.4516, -0.0475, 1.4198) -- (1.4032, -0.0475, 1.4193) -- cycle;
\fill[blue!15.6, opacity=0.5] (1.4032, -0.0475, 1.4193) -- (1.4516, -0.0475, 1.4198) -- (1.4517, -0.0506, 1.4698) -- (1.4034, -0.0506, 1.4693) -- cycle;
\fill[blue!15.8, opacity=0.5] (1.4034, -0.0506, 1.4693) -- (1.4517, -0.0506, 1.4698) -- (1.4518, -0.0537, 1.5198) -- (1.4036, -0.0537, 1.5193) -- cycle;
\fill[blue!16.1, opacity=0.5] (1.4036, -0.0537, 1.5193) -- (1.4518, -0.0537, 1.5198) -- (1.4519, -0.0569, 1.5698) -- (1.4038, -0.0569, 1.5693) -- cycle;
\fill[blue!16.5, opacity=0.5] (1.4038, -0.0569, 1.5693) -- (1.4519, -0.0569, 1.5698) -- (1.4520, -0.0600, 1.6198) -- (1.4040, -0.0600, 1.6193) -- cycle;
\fill[blue!17.0, opacity=0.5] (1.4040, -0.0600, 1.6193) -- (1.4520, -0.0600, 1.6198) -- (1.4521, -0.0631, 1.6698) -- (1.4042, -0.0631, 1.6693) -- cycle;
\fill[blue!17.6, opacity=0.5] (1.4042, -0.0631, 1.6693) -- (1.4521, -0.0631, 1.6698) -- (1.4522, -0.0663, 1.7198) -- (1.4044, -0.0663, 1.7193) -- cycle;
\fill[blue!18.2, opacity=0.5] (1.4044, -0.0663, 1.7193) -- (1.4522, -0.0663, 1.7198) -- (1.4523, -0.0694, 1.7698) -- (1.4046, -0.0694, 1.7693) -- cycle;
\fill[blue!19.1, opacity=0.5] (1.4046, -0.0694, 1.7693) -- (1.4523, -0.0694, 1.7698) -- (1.4524, -0.0725, 1.8198) -- (1.4048, -0.0725, 1.8193) -- cycle;
\fill[blue!20.0, opacity=0.5] (1.4048, -0.0725, 1.8193) -- (1.4524, -0.0725, 1.8198) -- (1.4525, -0.0755, 1.8698) -- (1.4050, -0.0755, 1.8693) -- cycle;
\fill[blue!21.1, opacity=0.5] (1.4050, -0.0755, 1.8693) -- (1.4525, -0.0755, 1.8698) -- (1.4526, -0.0785, 1.9198) -- (1.4052, -0.0785, 1.9193) -- cycle;
\fill[blue!22.2, opacity=0.5] (1.4052, -0.0785, 1.9193) -- (1.4526, -0.0785, 1.9198) -- (1.4527, -0.0815, 1.9698) -- (1.4054, -0.0815, 1.9693) -- cycle;
\fill[blue!23.5, opacity=0.5] (1.4054, -0.0815, 1.9693) -- (1.4527, -0.0815, 1.9698) -- (1.4528, -0.0844, 2.0198) -- (1.4056, -0.0844, 2.0193) -- cycle;
\fill[blue!24.9, opacity=0.5] (1.4056, -0.0844, 2.0193) -- (1.4528, -0.0844, 2.0198) -- (1.4529, -0.0872, 2.0698) -- (1.4058, -0.0872, 2.0693) -- cycle;
\fill[blue!26.4, opacity=0.5] (1.4058, -0.0872, 2.0693) -- (1.4529, -0.0872, 2.0698) -- (1.4530, -0.0900, 2.1198) -- (1.4060, -0.0900, 2.1193) -- cycle;
\fill[blue!28.0, opacity=0.5] (1.4060, -0.0900, 2.1193) -- (1.4530, -0.0900, 2.1198) -- (1.4531, -0.0927, 2.1698) -- (1.4062, -0.0927, 2.1693) -- cycle;
\fill[blue!29.7, opacity=0.5] (1.4062, -0.0927, 2.1693) -- (1.4531, -0.0927, 2.1698) -- (1.4532, -0.0953, 2.2198) -- (1.4064, -0.0953, 2.2193) -- cycle;
\fill[blue!31.4, opacity=0.5] (1.4064, -0.0953, 2.2193) -- (1.4532, -0.0953, 2.2198) -- (1.4533, -0.0978, 2.2698) -- (1.4065, -0.0978, 2.2693) -- cycle;
\fill[blue!33.1, opacity=0.5] (1.4065, -0.0978, 2.2693) -- (1.4533, -0.0978, 2.2698) -- (1.4533, -0.1001, 2.3198) -- (1.4067, -0.1001, 2.3193) -- cycle;
\fill[blue!34.9, opacity=0.5] (1.4067, -0.1001, 2.3193) -- (1.4533, -0.1001, 2.3198) -- (1.4534, -0.1024, 2.3698) -- (1.4068, -0.1024, 2.3693) -- cycle;
\fill[blue!36.7, opacity=0.5] (1.4068, -0.1024, 2.3693) -- (1.4534, -0.1024, 2.3698) -- (1.4535, -0.1046, 2.4198) -- (1.4070, -0.1046, 2.4193) -- cycle;
\fill[blue!38.4, opacity=0.5] (1.4070, -0.1046, 2.4193) -- (1.4535, -0.1046, 2.4198) -- (1.4536, -0.1066, 2.4698) -- (1.4071, -0.1066, 2.4693) -- cycle;
\fill[blue!40.1, opacity=0.5] (1.4071, -0.1066, 2.4693) -- (1.4536, -0.1066, 2.4698) -- (1.4536, -0.1085, 2.5198) -- (1.4072, -0.1085, 2.5193) -- cycle;
\fill[blue!41.8, opacity=0.5] (1.4072, -0.1085, 2.5193) -- (1.4536, -0.1085, 2.5198) -- (1.4537, -0.1103, 2.5698) -- (1.4074, -0.1103, 2.5693) -- cycle;
\fill[blue!43.4, opacity=0.5] (1.4074, -0.1103, 2.5693) -- (1.4537, -0.1103, 2.5698) -- (1.4537, -0.1120, 2.6198) -- (1.4075, -0.1120, 2.6193) -- cycle;
\fill[blue!44.9, opacity=0.5] (1.4075, -0.1120, 2.6193) -- (1.4537, -0.1120, 2.6198) -- (1.4538, -0.1135, 2.6698) -- (1.4076, -0.1135, 2.6693) -- cycle;
\fill[blue!46.3, opacity=0.5] (1.4076, -0.1135, 2.6693) -- (1.4538, -0.1135, 2.6698) -- (1.4538, -0.1148, 2.7198) -- (1.4077, -0.1148, 2.7193) -- cycle;
\fill[blue!47.6, opacity=0.5] (1.4077, -0.1148, 2.7193) -- (1.4538, -0.1148, 2.7198) -- (1.4539, -0.1160, 2.7698) -- (1.4077, -0.1160, 2.7693) -- cycle;
\fill[blue!48.8, opacity=0.5] (1.4077, -0.1160, 2.7693) -- (1.4539, -0.1160, 2.7698) -- (1.4539, -0.1171, 2.8198) -- (1.4078, -0.1171, 2.8193) -- cycle;
\fill[blue!49.9, opacity=0.5] (1.4078, -0.1171, 2.8193) -- (1.4539, -0.1171, 2.8198) -- (1.4539, -0.1180, 2.8698) -- (1.4079, -0.1180, 2.8693) -- cycle;
\fill[blue!50.8, opacity=0.5] (1.4079, -0.1180, 2.8693) -- (1.4539, -0.1180, 2.8698) -- (1.4540, -0.1187, 2.9198) -- (1.4079, -0.1187, 2.9193) -- cycle;
\fill[blue!51.7, opacity=0.5] (1.4079, -0.1187, 2.9193) -- (1.4540, -0.1187, 2.9198) -- (1.4540, -0.1193, 2.9698) -- (1.4080, -0.1193, 2.9693) -- cycle;
\fill[blue!52.3, opacity=0.5] (1.4080, -0.1193, 2.9693) -- (1.4540, -0.1193, 2.9698) -- (1.4540, -0.1197, 3.0198) -- (1.4080, -0.1197, 3.0193) -- cycle;
\fill[blue!52.9, opacity=0.5] (1.4080, -0.1197, 3.0193) -- (1.4540, -0.1197, 3.0198) -- (1.4540, -0.1199, 3.0698) -- (1.4080, -0.1199, 3.0693) -- cycle;
\fill[blue!53.3, opacity=0.5] (1.4080, -0.1199, 3.0693) -- (1.4540, -0.1199, 3.0698) -- (1.4540, -0.1200, 3.1198) -- (1.4080, -0.1200, 3.1193) -- cycle;
\fill[blue!15.0, opacity=0.5] (1.4500, -0.0000, 0.1198) -- (1.5000, -0.0000, 0.1200) -- (1.5000, -0.0001, 0.1700) -- (1.4500, -0.0001, 0.1698) -- cycle;
\fill[blue!15.0, opacity=0.5] (1.4500, -0.0001, 0.1698) -- (1.5000, -0.0001, 0.1700) -- (1.5000, -0.0003, 0.2200) -- (1.4500, -0.0003, 0.2198) -- cycle;
\fill[blue!15.0, opacity=0.5] (1.4500, -0.0003, 0.2198) -- (1.5000, -0.0003, 0.2200) -- (1.5000, -0.0007, 0.2700) -- (1.4500, -0.0007, 0.2698) -- cycle;
\fill[blue!15.0, opacity=0.5] (1.4500, -0.0007, 0.2698) -- (1.5000, -0.0007, 0.2700) -- (1.5000, -0.0013, 0.3200) -- (1.4500, -0.0013, 0.3198) -- cycle;
\fill[blue!15.0, opacity=0.5] (1.4500, -0.0013, 0.3198) -- (1.5000, -0.0013, 0.3200) -- (1.5000, -0.0020, 0.3700) -- (1.4501, -0.0020, 0.3698) -- cycle;
\fill[blue!15.0, opacity=0.5] (1.4501, -0.0020, 0.3698) -- (1.5000, -0.0020, 0.3700) -- (1.5000, -0.0029, 0.4200) -- (1.4501, -0.0029, 0.4198) -- cycle;
\fill[blue!15.0, opacity=0.5] (1.4501, -0.0029, 0.4198) -- (1.5000, -0.0029, 0.4200) -- (1.5000, -0.0040, 0.4700) -- (1.4501, -0.0040, 0.4698) -- cycle;
\fill[blue!15.0, opacity=0.5] (1.4501, -0.0040, 0.4698) -- (1.5000, -0.0040, 0.4700) -- (1.5000, -0.0052, 0.5200) -- (1.4502, -0.0052, 0.5198) -- cycle;
\fill[blue!15.0, opacity=0.5] (1.4502, -0.0052, 0.5198) -- (1.5000, -0.0052, 0.5200) -- (1.5000, -0.0065, 0.5700) -- (1.4502, -0.0065, 0.5698) -- cycle;
\fill[blue!15.0, opacity=0.5] (1.4502, -0.0065, 0.5698) -- (1.5000, -0.0065, 0.5700) -- (1.5000, -0.0080, 0.6200) -- (1.4503, -0.0080, 0.6198) -- cycle;
\fill[blue!15.0, opacity=0.5] (1.4503, -0.0080, 0.6198) -- (1.5000, -0.0080, 0.6200) -- (1.5000, -0.0097, 0.6700) -- (1.4503, -0.0097, 0.6698) -- cycle;
\fill[blue!15.0, opacity=0.5] (1.4503, -0.0097, 0.6698) -- (1.5000, -0.0097, 0.6700) -- (1.5000, -0.0115, 0.7200) -- (1.4504, -0.0115, 0.7198) -- cycle;
\fill[blue!15.0, opacity=0.5] (1.4504, -0.0115, 0.7198) -- (1.5000, -0.0115, 0.7200) -- (1.5000, -0.0134, 0.7700) -- (1.4504, -0.0134, 0.7698) -- cycle;
\fill[blue!15.0, opacity=0.5] (1.4504, -0.0134, 0.7698) -- (1.5000, -0.0134, 0.7700) -- (1.5000, -0.0154, 0.8200) -- (1.4505, -0.0154, 0.8198) -- cycle;
\fill[blue!15.0, opacity=0.5] (1.4505, -0.0154, 0.8198) -- (1.5000, -0.0154, 0.8200) -- (1.5000, -0.0176, 0.8700) -- (1.4506, -0.0176, 0.8698) -- cycle;
\fill[blue!15.0, opacity=0.5] (1.4506, -0.0176, 0.8698) -- (1.5000, -0.0176, 0.8700) -- (1.5000, -0.0199, 0.9200) -- (1.4507, -0.0199, 0.9198) -- cycle;
\fill[blue!15.0, opacity=0.5] (1.4507, -0.0199, 0.9198) -- (1.5000, -0.0199, 0.9200) -- (1.5000, -0.0222, 0.9700) -- (1.4507, -0.0222, 0.9698) -- cycle;
\fill[blue!15.0, opacity=0.5] (1.4507, -0.0222, 0.9698) -- (1.5000, -0.0222, 0.9700) -- (1.5000, -0.0247, 1.0200) -- (1.4508, -0.0247, 1.0198) -- cycle;
\fill[blue!15.0, opacity=0.5] (1.4508, -0.0247, 1.0198) -- (1.5000, -0.0247, 1.0200) -- (1.5000, -0.0273, 1.0700) -- (1.4509, -0.0273, 1.0698) -- cycle;
\fill[blue!15.0, opacity=0.5] (1.4509, -0.0273, 1.0698) -- (1.5000, -0.0273, 1.0700) -- (1.5000, -0.0300, 1.1200) -- (1.4510, -0.0300, 1.1198) -- cycle;
\fill[blue!15.0, opacity=0.5] (1.4510, -0.0300, 1.1198) -- (1.5000, -0.0300, 1.1200) -- (1.5000, -0.0328, 1.1700) -- (1.4511, -0.0328, 1.1698) -- cycle;
\fill[blue!15.1, opacity=0.5] (1.4511, -0.0328, 1.1698) -- (1.5000, -0.0328, 1.1700) -- (1.5000, -0.0356, 1.2200) -- (1.4512, -0.0356, 1.2198) -- cycle;
\fill[blue!15.1, opacity=0.5] (1.4512, -0.0356, 1.2198) -- (1.5000, -0.0356, 1.2200) -- (1.5000, -0.0385, 1.2700) -- (1.4513, -0.0385, 1.2698) -- cycle;
\fill[blue!15.2, opacity=0.5] (1.4513, -0.0385, 1.2698) -- (1.5000, -0.0385, 1.2700) -- (1.5000, -0.0415, 1.3200) -- (1.4514, -0.0415, 1.3198) -- cycle;
\fill[blue!15.3, opacity=0.5] (1.4514, -0.0415, 1.3198) -- (1.5000, -0.0415, 1.3200) -- (1.5000, -0.0445, 1.3700) -- (1.4515, -0.0445, 1.3698) -- cycle;
\fill[blue!15.5, opacity=0.5] (1.4515, -0.0445, 1.3698) -- (1.5000, -0.0445, 1.3700) -- (1.5000, -0.0475, 1.4200) -- (1.4516, -0.0475, 1.4198) -- cycle;
\fill[blue!15.7, opacity=0.5] (1.4516, -0.0475, 1.4198) -- (1.5000, -0.0475, 1.4200) -- (1.5000, -0.0506, 1.4700) -- (1.4517, -0.0506, 1.4698) -- cycle;
\fill[blue!16.0, opacity=0.5] (1.4517, -0.0506, 1.4698) -- (1.5000, -0.0506, 1.4700) -- (1.5000, -0.0537, 1.5200) -- (1.4518, -0.0537, 1.5198) -- cycle;
\fill[blue!16.4, opacity=0.5] (1.4518, -0.0537, 1.5198) -- (1.5000, -0.0537, 1.5200) -- (1.5000, -0.0569, 1.5700) -- (1.4519, -0.0569, 1.5698) -- cycle;
\fill[blue!16.8, opacity=0.5] (1.4519, -0.0569, 1.5698) -- (1.5000, -0.0569, 1.5700) -- (1.5000, -0.0600, 1.6200) -- (1.4520, -0.0600, 1.6198) -- cycle;
\fill[blue!17.4, opacity=0.5] (1.4520, -0.0600, 1.6198) -- (1.5000, -0.0600, 1.6200) -- (1.5000, -0.0631, 1.6700) -- (1.4521, -0.0631, 1.6698) -- cycle;
\fill[blue!18.1, opacity=0.5] (1.4521, -0.0631, 1.6698) -- (1.5000, -0.0631, 1.6700) -- (1.5000, -0.0663, 1.7200) -- (1.4522, -0.0663, 1.7198) -- cycle;
\fill[blue!18.9, opacity=0.5] (1.4522, -0.0663, 1.7198) -- (1.5000, -0.0663, 1.7200) -- (1.5000, -0.0694, 1.7700) -- (1.4523, -0.0694, 1.7698) -- cycle;
\fill[blue!19.8, opacity=0.5] (1.4523, -0.0694, 1.7698) -- (1.5000, -0.0694, 1.7700) -- (1.5000, -0.0725, 1.8200) -- (1.4524, -0.0725, 1.8198) -- cycle;
\fill[blue!20.8, opacity=0.5] (1.4524, -0.0725, 1.8198) -- (1.5000, -0.0725, 1.8200) -- (1.5000, -0.0755, 1.8700) -- (1.4525, -0.0755, 1.8698) -- cycle;
\fill[blue!22.0, opacity=0.5] (1.4525, -0.0755, 1.8698) -- (1.5000, -0.0755, 1.8700) -- (1.5000, -0.0785, 1.9200) -- (1.4526, -0.0785, 1.9198) -- cycle;
\fill[blue!23.3, opacity=0.5] (1.4526, -0.0785, 1.9198) -- (1.5000, -0.0785, 1.9200) -- (1.5000, -0.0815, 1.9700) -- (1.4527, -0.0815, 1.9698) -- cycle;
\fill[blue!24.7, opacity=0.5] (1.4527, -0.0815, 1.9698) -- (1.5000, -0.0815, 1.9700) -- (1.5000, -0.0844, 2.0200) -- (1.4528, -0.0844, 2.0198) -- cycle;
\fill[blue!26.3, opacity=0.5] (1.4528, -0.0844, 2.0198) -- (1.5000, -0.0844, 2.0200) -- (1.5000, -0.0872, 2.0700) -- (1.4529, -0.0872, 2.0698) -- cycle;
\fill[blue!27.9, opacity=0.5] (1.4529, -0.0872, 2.0698) -- (1.5000, -0.0872, 2.0700) -- (1.5000, -0.0900, 2.1200) -- (1.4530, -0.0900, 2.1198) -- cycle;
\fill[blue!29.6, opacity=0.5] (1.4530, -0.0900, 2.1198) -- (1.5000, -0.0900, 2.1200) -- (1.5000, -0.0927, 2.1700) -- (1.4531, -0.0927, 2.1698) -- cycle;
\fill[blue!31.3, opacity=0.5] (1.4531, -0.0927, 2.1698) -- (1.5000, -0.0927, 2.1700) -- (1.5000, -0.0953, 2.2200) -- (1.4532, -0.0953, 2.2198) -- cycle;
\fill[blue!33.1, opacity=0.5] (1.4532, -0.0953, 2.2198) -- (1.5000, -0.0953, 2.2200) -- (1.5000, -0.0978, 2.2700) -- (1.4533, -0.0978, 2.2698) -- cycle;
\fill[blue!34.9, opacity=0.5] (1.4533, -0.0978, 2.2698) -- (1.5000, -0.0978, 2.2700) -- (1.5000, -0.1001, 2.3200) -- (1.4533, -0.1001, 2.3198) -- cycle;
\fill[blue!36.7, opacity=0.5] (1.4533, -0.1001, 2.3198) -- (1.5000, -0.1001, 2.3200) -- (1.5000, -0.1024, 2.3700) -- (1.4534, -0.1024, 2.3698) -- cycle;
\fill[blue!38.6, opacity=0.5] (1.4534, -0.1024, 2.3698) -- (1.5000, -0.1024, 2.3700) -- (1.5000, -0.1046, 2.4200) -- (1.4535, -0.1046, 2.4198) -- cycle;
\fill[blue!40.3, opacity=0.5] (1.4535, -0.1046, 2.4198) -- (1.5000, -0.1046, 2.4200) -- (1.5000, -0.1066, 2.4700) -- (1.4536, -0.1066, 2.4698) -- cycle;
\fill[blue!42.1, opacity=0.5] (1.4536, -0.1066, 2.4698) -- (1.5000, -0.1066, 2.4700) -- (1.5000, -0.1085, 2.5200) -- (1.4536, -0.1085, 2.5198) -- cycle;
\fill[blue!43.7, opacity=0.5] (1.4536, -0.1085, 2.5198) -- (1.5000, -0.1085, 2.5200) -- (1.5000, -0.1103, 2.5700) -- (1.4537, -0.1103, 2.5698) -- cycle;
\fill[blue!45.3, opacity=0.5] (1.4537, -0.1103, 2.5698) -- (1.5000, -0.1103, 2.5700) -- (1.5000, -0.1120, 2.6200) -- (1.4537, -0.1120, 2.6198) -- cycle;
\fill[blue!46.8, opacity=0.5] (1.4537, -0.1120, 2.6198) -- (1.5000, -0.1120, 2.6200) -- (1.5000, -0.1135, 2.6700) -- (1.4538, -0.1135, 2.6698) -- cycle;
\fill[blue!48.2, opacity=0.5] (1.4538, -0.1135, 2.6698) -- (1.5000, -0.1135, 2.6700) -- (1.5000, -0.1148, 2.7200) -- (1.4538, -0.1148, 2.7198) -- cycle;
\fill[blue!49.4, opacity=0.5] (1.4538, -0.1148, 2.7198) -- (1.5000, -0.1148, 2.7200) -- (1.5000, -0.1160, 2.7700) -- (1.4539, -0.1160, 2.7698) -- cycle;
\fill[blue!50.6, opacity=0.5] (1.4539, -0.1160, 2.7698) -- (1.5000, -0.1160, 2.7700) -- (1.5000, -0.1171, 2.8200) -- (1.4539, -0.1171, 2.8198) -- cycle;
\fill[blue!51.6, opacity=0.5] (1.4539, -0.1171, 2.8198) -- (1.5000, -0.1171, 2.8200) -- (1.5000, -0.1180, 2.8700) -- (1.4539, -0.1180, 2.8698) -- cycle;
\fill[blue!52.5, opacity=0.5] (1.4539, -0.1180, 2.8698) -- (1.5000, -0.1180, 2.8700) -- (1.5000, -0.1187, 2.9200) -- (1.4540, -0.1187, 2.9198) -- cycle;
\fill[blue!53.2, opacity=0.5] (1.4540, -0.1187, 2.9198) -- (1.5000, -0.1187, 2.9200) -- (1.5000, -0.1193, 2.9700) -- (1.4540, -0.1193, 2.9698) -- cycle;
\fill[blue!53.8, opacity=0.5] (1.4540, -0.1193, 2.9698) -- (1.5000, -0.1193, 2.9700) -- (1.5000, -0.1197, 3.0200) -- (1.4540, -0.1197, 3.0198) -- cycle;
\fill[blue!54.3, opacity=0.5] (1.4540, -0.1197, 3.0198) -- (1.5000, -0.1197, 3.0200) -- (1.5000, -0.1199, 3.0700) -- (1.4540, -0.1199, 3.0698) -- cycle;
\fill[blue!54.6, opacity=0.5] (1.4540, -0.1199, 3.0698) -- (1.5000, -0.1199, 3.0700) -- (1.5000, -0.1200, 3.1200) -- (1.4540, -0.1200, 3.1198) -- cycle;
\fill[blue!15.0, opacity=0.5] (1.5000, -0.0000, 0.1200) -- (1.5500, -0.0000, 0.1198) -- (1.5500, -0.0001, 0.1698) -- (1.5000, -0.0001, 0.1700) -- cycle;
\fill[blue!15.0, opacity=0.5] (1.5000, -0.0001, 0.1700) -- (1.5500, -0.0001, 0.1698) -- (1.5500, -0.0003, 0.2198) -- (1.5000, -0.0003, 0.2200) -- cycle;
\fill[blue!15.0, opacity=0.5] (1.5000, -0.0003, 0.2200) -- (1.5500, -0.0003, 0.2198) -- (1.5500, -0.0007, 0.2698) -- (1.5000, -0.0007, 0.2700) -- cycle;
\fill[blue!15.0, opacity=0.5] (1.5000, -0.0007, 0.2700) -- (1.5500, -0.0007, 0.2698) -- (1.5500, -0.0013, 0.3198) -- (1.5000, -0.0013, 0.3200) -- cycle;
\fill[blue!15.0, opacity=0.5] (1.5000, -0.0013, 0.3200) -- (1.5500, -0.0013, 0.3198) -- (1.5499, -0.0020, 0.3698) -- (1.5000, -0.0020, 0.3700) -- cycle;
\fill[blue!15.0, opacity=0.5] (1.5000, -0.0020, 0.3700) -- (1.5499, -0.0020, 0.3698) -- (1.5499, -0.0029, 0.4198) -- (1.5000, -0.0029, 0.4200) -- cycle;
\fill[blue!15.0, opacity=0.5] (1.5000, -0.0029, 0.4200) -- (1.5499, -0.0029, 0.4198) -- (1.5499, -0.0040, 0.4698) -- (1.5000, -0.0040, 0.4700) -- cycle;
\fill[blue!15.0, opacity=0.5] (1.5000, -0.0040, 0.4700) -- (1.5499, -0.0040, 0.4698) -- (1.5498, -0.0052, 0.5198) -- (1.5000, -0.0052, 0.5200) -- cycle;
\fill[blue!15.0, opacity=0.5] (1.5000, -0.0052, 0.5200) -- (1.5498, -0.0052, 0.5198) -- (1.5498, -0.0065, 0.5698) -- (1.5000, -0.0065, 0.5700) -- cycle;
\fill[blue!15.0, opacity=0.5] (1.5000, -0.0065, 0.5700) -- (1.5498, -0.0065, 0.5698) -- (1.5497, -0.0080, 0.6198) -- (1.5000, -0.0080, 0.6200) -- cycle;
\fill[blue!15.0, opacity=0.5] (1.5000, -0.0080, 0.6200) -- (1.5497, -0.0080, 0.6198) -- (1.5497, -0.0097, 0.6698) -- (1.5000, -0.0097, 0.6700) -- cycle;
\fill[blue!15.0, opacity=0.5] (1.5000, -0.0097, 0.6700) -- (1.5497, -0.0097, 0.6698) -- (1.5496, -0.0115, 0.7198) -- (1.5000, -0.0115, 0.7200) -- cycle;
\fill[blue!15.0, opacity=0.5] (1.5000, -0.0115, 0.7200) -- (1.5496, -0.0115, 0.7198) -- (1.5496, -0.0134, 0.7698) -- (1.5000, -0.0134, 0.7700) -- cycle;
\fill[blue!15.0, opacity=0.5] (1.5000, -0.0134, 0.7700) -- (1.5496, -0.0134, 0.7698) -- (1.5495, -0.0154, 0.8198) -- (1.5000, -0.0154, 0.8200) -- cycle;
\fill[blue!15.0, opacity=0.5] (1.5000, -0.0154, 0.8200) -- (1.5495, -0.0154, 0.8198) -- (1.5494, -0.0176, 0.8698) -- (1.5000, -0.0176, 0.8700) -- cycle;
\fill[blue!15.0, opacity=0.5] (1.5000, -0.0176, 0.8700) -- (1.5494, -0.0176, 0.8698) -- (1.5493, -0.0199, 0.9198) -- (1.5000, -0.0199, 0.9200) -- cycle;
\fill[blue!15.0, opacity=0.5] (1.5000, -0.0199, 0.9200) -- (1.5493, -0.0199, 0.9198) -- (1.5493, -0.0222, 0.9698) -- (1.5000, -0.0222, 0.9700) -- cycle;
\fill[blue!15.0, opacity=0.5] (1.5000, -0.0222, 0.9700) -- (1.5493, -0.0222, 0.9698) -- (1.5492, -0.0247, 1.0198) -- (1.5000, -0.0247, 1.0200) -- cycle;
\fill[blue!15.0, opacity=0.5] (1.5000, -0.0247, 1.0200) -- (1.5492, -0.0247, 1.0198) -- (1.5491, -0.0273, 1.0698) -- (1.5000, -0.0273, 1.0700) -- cycle;
\fill[blue!15.0, opacity=0.5] (1.5000, -0.0273, 1.0700) -- (1.5491, -0.0273, 1.0698) -- (1.5490, -0.0300, 1.1198) -- (1.5000, -0.0300, 1.1200) -- cycle;
\fill[blue!15.1, opacity=0.5] (1.5000, -0.0300, 1.1200) -- (1.5490, -0.0300, 1.1198) -- (1.5489, -0.0328, 1.1698) -- (1.5000, -0.0328, 1.1700) -- cycle;
\fill[blue!15.1, opacity=0.5] (1.5000, -0.0328, 1.1700) -- (1.5489, -0.0328, 1.1698) -- (1.5488, -0.0356, 1.2198) -- (1.5000, -0.0356, 1.2200) -- cycle;
\fill[blue!15.2, opacity=0.5] (1.5000, -0.0356, 1.2200) -- (1.5488, -0.0356, 1.2198) -- (1.5487, -0.0385, 1.2698) -- (1.5000, -0.0385, 1.2700) -- cycle;
\fill[blue!15.3, opacity=0.5] (1.5000, -0.0385, 1.2700) -- (1.5487, -0.0385, 1.2698) -- (1.5486, -0.0415, 1.3198) -- (1.5000, -0.0415, 1.3200) -- cycle;
\fill[blue!15.5, opacity=0.5] (1.5000, -0.0415, 1.3200) -- (1.5486, -0.0415, 1.3198) -- (1.5485, -0.0445, 1.3698) -- (1.5000, -0.0445, 1.3700) -- cycle;
\fill[blue!15.7, opacity=0.5] (1.5000, -0.0445, 1.3700) -- (1.5485, -0.0445, 1.3698) -- (1.5484, -0.0475, 1.4198) -- (1.5000, -0.0475, 1.4200) -- cycle;
\fill[blue!16.0, opacity=0.5] (1.5000, -0.0475, 1.4200) -- (1.5484, -0.0475, 1.4198) -- (1.5483, -0.0506, 1.4698) -- (1.5000, -0.0506, 1.4700) -- cycle;
\fill[blue!16.4, opacity=0.5] (1.5000, -0.0506, 1.4700) -- (1.5483, -0.0506, 1.4698) -- (1.5482, -0.0537, 1.5198) -- (1.5000, -0.0537, 1.5200) -- cycle;
\fill[blue!16.8, opacity=0.5] (1.5000, -0.0537, 1.5200) -- (1.5482, -0.0537, 1.5198) -- (1.5481, -0.0569, 1.5698) -- (1.5000, -0.0569, 1.5700) -- cycle;
\fill[blue!17.4, opacity=0.5] (1.5000, -0.0569, 1.5700) -- (1.5481, -0.0569, 1.5698) -- (1.5480, -0.0600, 1.6198) -- (1.5000, -0.0600, 1.6200) -- cycle;
\fill[blue!18.1, opacity=0.5] (1.5000, -0.0600, 1.6200) -- (1.5480, -0.0600, 1.6198) -- (1.5479, -0.0631, 1.6698) -- (1.5000, -0.0631, 1.6700) -- cycle;
\fill[blue!18.9, opacity=0.5] (1.5000, -0.0631, 1.6700) -- (1.5479, -0.0631, 1.6698) -- (1.5478, -0.0663, 1.7198) -- (1.5000, -0.0663, 1.7200) -- cycle;
\fill[blue!19.9, opacity=0.5] (1.5000, -0.0663, 1.7200) -- (1.5478, -0.0663, 1.7198) -- (1.5477, -0.0694, 1.7698) -- (1.5000, -0.0694, 1.7700) -- cycle;
\fill[blue!21.0, opacity=0.5] (1.5000, -0.0694, 1.7700) -- (1.5477, -0.0694, 1.7698) -- (1.5476, -0.0725, 1.8198) -- (1.5000, -0.0725, 1.8200) -- cycle;
\fill[blue!22.2, opacity=0.5] (1.5000, -0.0725, 1.8200) -- (1.5476, -0.0725, 1.8198) -- (1.5475, -0.0755, 1.8698) -- (1.5000, -0.0755, 1.8700) -- cycle;
\fill[blue!23.6, opacity=0.5] (1.5000, -0.0755, 1.8700) -- (1.5475, -0.0755, 1.8698) -- (1.5474, -0.0785, 1.9198) -- (1.5000, -0.0785, 1.9200) -- cycle;
\fill[blue!25.1, opacity=0.5] (1.5000, -0.0785, 1.9200) -- (1.5474, -0.0785, 1.9198) -- (1.5473, -0.0815, 1.9698) -- (1.5000, -0.0815, 1.9700) -- cycle;
\fill[blue!26.7, opacity=0.5] (1.5000, -0.0815, 1.9700) -- (1.5473, -0.0815, 1.9698) -- (1.5472, -0.0844, 2.0198) -- (1.5000, -0.0844, 2.0200) -- cycle;
\fill[blue!28.4, opacity=0.5] (1.5000, -0.0844, 2.0200) -- (1.5472, -0.0844, 2.0198) -- (1.5471, -0.0872, 2.0698) -- (1.5000, -0.0872, 2.0700) -- cycle;
\fill[blue!30.1, opacity=0.5] (1.5000, -0.0872, 2.0700) -- (1.5471, -0.0872, 2.0698) -- (1.5470, -0.0900, 2.1198) -- (1.5000, -0.0900, 2.1200) -- cycle;
\fill[blue!32.0, opacity=0.5] (1.5000, -0.0900, 2.1200) -- (1.5470, -0.0900, 2.1198) -- (1.5469, -0.0927, 2.1698) -- (1.5000, -0.0927, 2.1700) -- cycle;
\fill[blue!33.8, opacity=0.5] (1.5000, -0.0927, 2.1700) -- (1.5469, -0.0927, 2.1698) -- (1.5468, -0.0953, 2.2198) -- (1.5000, -0.0953, 2.2200) -- cycle;
\fill[blue!35.7, opacity=0.5] (1.5000, -0.0953, 2.2200) -- (1.5468, -0.0953, 2.2198) -- (1.5467, -0.0978, 2.2698) -- (1.5000, -0.0978, 2.2700) -- cycle;
\fill[blue!37.6, opacity=0.5] (1.5000, -0.0978, 2.2700) -- (1.5467, -0.0978, 2.2698) -- (1.5467, -0.1001, 2.3198) -- (1.5000, -0.1001, 2.3200) -- cycle;
\fill[blue!39.5, opacity=0.5] (1.5000, -0.1001, 2.3200) -- (1.5467, -0.1001, 2.3198) -- (1.5466, -0.1024, 2.3698) -- (1.5000, -0.1024, 2.3700) -- cycle;
\fill[blue!41.4, opacity=0.5] (1.5000, -0.1024, 2.3700) -- (1.5466, -0.1024, 2.3698) -- (1.5465, -0.1046, 2.4198) -- (1.5000, -0.1046, 2.4200) -- cycle;
\fill[blue!43.2, opacity=0.5] (1.5000, -0.1046, 2.4200) -- (1.5465, -0.1046, 2.4198) -- (1.5464, -0.1066, 2.4698) -- (1.5000, -0.1066, 2.4700) -- cycle;
\fill[blue!44.9, opacity=0.5] (1.5000, -0.1066, 2.4700) -- (1.5464, -0.1066, 2.4698) -- (1.5464, -0.1085, 2.5198) -- (1.5000, -0.1085, 2.5200) -- cycle;
\fill[blue!46.6, opacity=0.5] (1.5000, -0.1085, 2.5200) -- (1.5464, -0.1085, 2.5198) -- (1.5463, -0.1103, 2.5698) -- (1.5000, -0.1103, 2.5700) -- cycle;
\fill[blue!48.1, opacity=0.5] (1.5000, -0.1103, 2.5700) -- (1.5463, -0.1103, 2.5698) -- (1.5463, -0.1120, 2.6198) -- (1.5000, -0.1120, 2.6200) -- cycle;
\fill[blue!49.6, opacity=0.5] (1.5000, -0.1120, 2.6200) -- (1.5463, -0.1120, 2.6198) -- (1.5462, -0.1135, 2.6698) -- (1.5000, -0.1135, 2.6700) -- cycle;
\fill[blue!50.9, opacity=0.5] (1.5000, -0.1135, 2.6700) -- (1.5462, -0.1135, 2.6698) -- (1.5462, -0.1148, 2.7198) -- (1.5000, -0.1148, 2.7200) -- cycle;
\fill[blue!52.1, opacity=0.5] (1.5000, -0.1148, 2.7200) -- (1.5462, -0.1148, 2.7198) -- (1.5461, -0.1160, 2.7698) -- (1.5000, -0.1160, 2.7700) -- cycle;
\fill[blue!53.1, opacity=0.5] (1.5000, -0.1160, 2.7700) -- (1.5461, -0.1160, 2.7698) -- (1.5461, -0.1171, 2.8198) -- (1.5000, -0.1171, 2.8200) -- cycle;
\fill[blue!54.0, opacity=0.5] (1.5000, -0.1171, 2.8200) -- (1.5461, -0.1171, 2.8198) -- (1.5461, -0.1180, 2.8698) -- (1.5000, -0.1180, 2.8700) -- cycle;
\fill[blue!54.7, opacity=0.5] (1.5000, -0.1180, 2.8700) -- (1.5461, -0.1180, 2.8698) -- (1.5460, -0.1187, 2.9198) -- (1.5000, -0.1187, 2.9200) -- cycle;
\fill[blue!55.4, opacity=0.5] (1.5000, -0.1187, 2.9200) -- (1.5460, -0.1187, 2.9198) -- (1.5460, -0.1193, 2.9698) -- (1.5000, -0.1193, 2.9700) -- cycle;
\fill[blue!55.8, opacity=0.5] (1.5000, -0.1193, 2.9700) -- (1.5460, -0.1193, 2.9698) -- (1.5460, -0.1197, 3.0198) -- (1.5000, -0.1197, 3.0200) -- cycle;
\fill[blue!56.1, opacity=0.5] (1.5000, -0.1197, 3.0200) -- (1.5460, -0.1197, 3.0198) -- (1.5460, -0.1199, 3.0698) -- (1.5000, -0.1199, 3.0700) -- cycle;
\fill[blue!56.3, opacity=0.5] (1.5000, -0.1199, 3.0700) -- (1.5460, -0.1199, 3.0698) -- (1.5460, -0.1200, 3.1198) -- (1.5000, -0.1200, 3.1200) -- cycle;
\fill[blue!15.0, opacity=0.5] (1.5500, -0.0000, 0.1198) -- (1.6000, -0.0000, 0.1193) -- (1.6000, -0.0001, 0.1693) -- (1.5500, -0.0001, 0.1698) -- cycle;
\fill[blue!15.0, opacity=0.5] (1.5500, -0.0001, 0.1698) -- (1.6000, -0.0001, 0.1693) -- (1.6000, -0.0003, 0.2193) -- (1.5500, -0.0003, 0.2198) -- cycle;
\fill[blue!15.0, opacity=0.5] (1.5500, -0.0003, 0.2198) -- (1.6000, -0.0003, 0.2193) -- (1.6000, -0.0007, 0.2693) -- (1.5500, -0.0007, 0.2698) -- cycle;
\fill[blue!15.0, opacity=0.5] (1.5500, -0.0007, 0.2698) -- (1.6000, -0.0007, 0.2693) -- (1.5999, -0.0013, 0.3193) -- (1.5500, -0.0013, 0.3198) -- cycle;
\fill[blue!15.0, opacity=0.5] (1.5500, -0.0013, 0.3198) -- (1.5999, -0.0013, 0.3193) -- (1.5999, -0.0020, 0.3693) -- (1.5499, -0.0020, 0.3698) -- cycle;
\fill[blue!15.0, opacity=0.5] (1.5499, -0.0020, 0.3698) -- (1.5999, -0.0020, 0.3693) -- (1.5998, -0.0029, 0.4193) -- (1.5499, -0.0029, 0.4198) -- cycle;
\fill[blue!15.0, opacity=0.5] (1.5499, -0.0029, 0.4198) -- (1.5998, -0.0029, 0.4193) -- (1.5997, -0.0040, 0.4693) -- (1.5499, -0.0040, 0.4698) -- cycle;
\fill[blue!15.0, opacity=0.5] (1.5499, -0.0040, 0.4698) -- (1.5997, -0.0040, 0.4693) -- (1.5997, -0.0052, 0.5193) -- (1.5498, -0.0052, 0.5198) -- cycle;
\fill[blue!15.0, opacity=0.5] (1.5498, -0.0052, 0.5198) -- (1.5997, -0.0052, 0.5193) -- (1.5996, -0.0065, 0.5693) -- (1.5498, -0.0065, 0.5698) -- cycle;
\fill[blue!15.0, opacity=0.5] (1.5498, -0.0065, 0.5698) -- (1.5996, -0.0065, 0.5693) -- (1.5995, -0.0080, 0.6193) -- (1.5497, -0.0080, 0.6198) -- cycle;
\fill[blue!15.0, opacity=0.5] (1.5497, -0.0080, 0.6198) -- (1.5995, -0.0080, 0.6193) -- (1.5994, -0.0097, 0.6693) -- (1.5497, -0.0097, 0.6698) -- cycle;
\fill[blue!15.0, opacity=0.5] (1.5497, -0.0097, 0.6698) -- (1.5994, -0.0097, 0.6693) -- (1.5992, -0.0115, 0.7193) -- (1.5496, -0.0115, 0.7198) -- cycle;
\fill[blue!15.0, opacity=0.5] (1.5496, -0.0115, 0.7198) -- (1.5992, -0.0115, 0.7193) -- (1.5991, -0.0134, 0.7693) -- (1.5496, -0.0134, 0.7698) -- cycle;
\fill[blue!15.0, opacity=0.5] (1.5496, -0.0134, 0.7698) -- (1.5991, -0.0134, 0.7693) -- (1.5990, -0.0154, 0.8193) -- (1.5495, -0.0154, 0.8198) -- cycle;
\fill[blue!15.0, opacity=0.5] (1.5495, -0.0154, 0.8198) -- (1.5990, -0.0154, 0.8193) -- (1.5988, -0.0176, 0.8693) -- (1.5494, -0.0176, 0.8698) -- cycle;
\fill[blue!15.0, opacity=0.5] (1.5494, -0.0176, 0.8698) -- (1.5988, -0.0176, 0.8693) -- (1.5987, -0.0199, 0.9193) -- (1.5493, -0.0199, 0.9198) -- cycle;
\fill[blue!15.0, opacity=0.5] (1.5493, -0.0199, 0.9198) -- (1.5987, -0.0199, 0.9193) -- (1.5985, -0.0222, 0.9693) -- (1.5493, -0.0222, 0.9698) -- cycle;
\fill[blue!15.0, opacity=0.5] (1.5493, -0.0222, 0.9698) -- (1.5985, -0.0222, 0.9693) -- (1.5984, -0.0247, 1.0193) -- (1.5492, -0.0247, 1.0198) -- cycle;
\fill[blue!15.0, opacity=0.5] (1.5492, -0.0247, 1.0198) -- (1.5984, -0.0247, 1.0193) -- (1.5982, -0.0273, 1.0693) -- (1.5491, -0.0273, 1.0698) -- cycle;
\fill[blue!15.1, opacity=0.5] (1.5491, -0.0273, 1.0698) -- (1.5982, -0.0273, 1.0693) -- (1.5980, -0.0300, 1.1193) -- (1.5490, -0.0300, 1.1198) -- cycle;
\fill[blue!15.1, opacity=0.5] (1.5490, -0.0300, 1.1198) -- (1.5980, -0.0300, 1.1193) -- (1.5978, -0.0328, 1.1693) -- (1.5489, -0.0328, 1.1698) -- cycle;
\fill[blue!15.2, opacity=0.5] (1.5489, -0.0328, 1.1698) -- (1.5978, -0.0328, 1.1693) -- (1.5976, -0.0356, 1.2193) -- (1.5488, -0.0356, 1.2198) -- cycle;
\fill[blue!15.3, opacity=0.5] (1.5488, -0.0356, 1.2198) -- (1.5976, -0.0356, 1.2193) -- (1.5974, -0.0385, 1.2693) -- (1.5487, -0.0385, 1.2698) -- cycle;
\fill[blue!15.5, opacity=0.5] (1.5487, -0.0385, 1.2698) -- (1.5974, -0.0385, 1.2693) -- (1.5972, -0.0415, 1.3193) -- (1.5486, -0.0415, 1.3198) -- cycle;
\fill[blue!15.8, opacity=0.5] (1.5486, -0.0415, 1.3198) -- (1.5972, -0.0415, 1.3193) -- (1.5970, -0.0445, 1.3693) -- (1.5485, -0.0445, 1.3698) -- cycle;
\fill[blue!16.1, opacity=0.5] (1.5485, -0.0445, 1.3698) -- (1.5970, -0.0445, 1.3693) -- (1.5968, -0.0475, 1.4193) -- (1.5484, -0.0475, 1.4198) -- cycle;
\fill[blue!16.5, opacity=0.5] (1.5484, -0.0475, 1.4198) -- (1.5968, -0.0475, 1.4193) -- (1.5966, -0.0506, 1.4693) -- (1.5483, -0.0506, 1.4698) -- cycle;
\fill[blue!17.0, opacity=0.5] (1.5483, -0.0506, 1.4698) -- (1.5966, -0.0506, 1.4693) -- (1.5964, -0.0537, 1.5193) -- (1.5482, -0.0537, 1.5198) -- cycle;
\fill[blue!17.6, opacity=0.5] (1.5482, -0.0537, 1.5198) -- (1.5964, -0.0537, 1.5193) -- (1.5962, -0.0569, 1.5693) -- (1.5481, -0.0569, 1.5698) -- cycle;
\fill[blue!18.4, opacity=0.5] (1.5481, -0.0569, 1.5698) -- (1.5962, -0.0569, 1.5693) -- (1.5960, -0.0600, 1.6193) -- (1.5480, -0.0600, 1.6198) -- cycle;
\fill[blue!19.3, opacity=0.5] (1.5480, -0.0600, 1.6198) -- (1.5960, -0.0600, 1.6193) -- (1.5958, -0.0631, 1.6693) -- (1.5479, -0.0631, 1.6698) -- cycle;
\fill[blue!20.3, opacity=0.5] (1.5479, -0.0631, 1.6698) -- (1.5958, -0.0631, 1.6693) -- (1.5956, -0.0663, 1.7193) -- (1.5478, -0.0663, 1.7198) -- cycle;
\fill[blue!21.5, opacity=0.5] (1.5478, -0.0663, 1.7198) -- (1.5956, -0.0663, 1.7193) -- (1.5954, -0.0694, 1.7693) -- (1.5477, -0.0694, 1.7698) -- cycle;
\fill[blue!22.9, opacity=0.5] (1.5477, -0.0694, 1.7698) -- (1.5954, -0.0694, 1.7693) -- (1.5952, -0.0725, 1.8193) -- (1.5476, -0.0725, 1.8198) -- cycle;
\fill[blue!24.3, opacity=0.5] (1.5476, -0.0725, 1.8198) -- (1.5952, -0.0725, 1.8193) -- (1.5950, -0.0755, 1.8693) -- (1.5475, -0.0755, 1.8698) -- cycle;
\fill[blue!25.9, opacity=0.5] (1.5475, -0.0755, 1.8698) -- (1.5950, -0.0755, 1.8693) -- (1.5948, -0.0785, 1.9193) -- (1.5474, -0.0785, 1.9198) -- cycle;
\fill[blue!27.6, opacity=0.5] (1.5474, -0.0785, 1.9198) -- (1.5948, -0.0785, 1.9193) -- (1.5946, -0.0815, 1.9693) -- (1.5473, -0.0815, 1.9698) -- cycle;
\fill[blue!29.5, opacity=0.5] (1.5473, -0.0815, 1.9698) -- (1.5946, -0.0815, 1.9693) -- (1.5944, -0.0844, 2.0193) -- (1.5472, -0.0844, 2.0198) -- cycle;
\fill[blue!31.3, opacity=0.5] (1.5472, -0.0844, 2.0198) -- (1.5944, -0.0844, 2.0193) -- (1.5942, -0.0872, 2.0693) -- (1.5471, -0.0872, 2.0698) -- cycle;
\fill[blue!33.3, opacity=0.5] (1.5471, -0.0872, 2.0698) -- (1.5942, -0.0872, 2.0693) -- (1.5940, -0.0900, 2.1193) -- (1.5470, -0.0900, 2.1198) -- cycle;
\fill[blue!35.3, opacity=0.5] (1.5470, -0.0900, 2.1198) -- (1.5940, -0.0900, 2.1193) -- (1.5938, -0.0927, 2.1693) -- (1.5469, -0.0927, 2.1698) -- cycle;
\fill[blue!37.3, opacity=0.5] (1.5469, -0.0927, 2.1698) -- (1.5938, -0.0927, 2.1693) -- (1.5936, -0.0953, 2.2193) -- (1.5468, -0.0953, 2.2198) -- cycle;
\fill[blue!39.3, opacity=0.5] (1.5468, -0.0953, 2.2198) -- (1.5936, -0.0953, 2.2193) -- (1.5935, -0.0978, 2.2693) -- (1.5467, -0.0978, 2.2698) -- cycle;
\fill[blue!41.3, opacity=0.5] (1.5467, -0.0978, 2.2698) -- (1.5935, -0.0978, 2.2693) -- (1.5933, -0.1001, 2.3193) -- (1.5467, -0.1001, 2.3198) -- cycle;
\fill[blue!43.3, opacity=0.5] (1.5467, -0.1001, 2.3198) -- (1.5933, -0.1001, 2.3193) -- (1.5932, -0.1024, 2.3693) -- (1.5466, -0.1024, 2.3698) -- cycle;
\fill[blue!45.2, opacity=0.5] (1.5466, -0.1024, 2.3698) -- (1.5932, -0.1024, 2.3693) -- (1.5930, -0.1046, 2.4193) -- (1.5465, -0.1046, 2.4198) -- cycle;
\fill[blue!47.0, opacity=0.5] (1.5465, -0.1046, 2.4198) -- (1.5930, -0.1046, 2.4193) -- (1.5929, -0.1066, 2.4693) -- (1.5464, -0.1066, 2.4698) -- cycle;
\fill[blue!48.7, opacity=0.5] (1.5464, -0.1066, 2.4698) -- (1.5929, -0.1066, 2.4693) -- (1.5928, -0.1085, 2.5193) -- (1.5464, -0.1085, 2.5198) -- cycle;
\fill[blue!50.3, opacity=0.5] (1.5464, -0.1085, 2.5198) -- (1.5928, -0.1085, 2.5193) -- (1.5926, -0.1103, 2.5693) -- (1.5463, -0.1103, 2.5698) -- cycle;
\fill[blue!51.7, opacity=0.5] (1.5463, -0.1103, 2.5698) -- (1.5926, -0.1103, 2.5693) -- (1.5925, -0.1120, 2.6193) -- (1.5463, -0.1120, 2.6198) -- cycle;
\fill[blue!53.1, opacity=0.5] (1.5463, -0.1120, 2.6198) -- (1.5925, -0.1120, 2.6193) -- (1.5924, -0.1135, 2.6693) -- (1.5462, -0.1135, 2.6698) -- cycle;
\fill[blue!54.2, opacity=0.5] (1.5462, -0.1135, 2.6698) -- (1.5924, -0.1135, 2.6693) -- (1.5923, -0.1148, 2.7193) -- (1.5462, -0.1148, 2.7198) -- cycle;
\fill[blue!55.3, opacity=0.5] (1.5462, -0.1148, 2.7198) -- (1.5923, -0.1148, 2.7193) -- (1.5923, -0.1160, 2.7693) -- (1.5461, -0.1160, 2.7698) -- cycle;
\fill[blue!56.2, opacity=0.5] (1.5461, -0.1160, 2.7698) -- (1.5923, -0.1160, 2.7693) -- (1.5922, -0.1171, 2.8193) -- (1.5461, -0.1171, 2.8198) -- cycle;
\fill[blue!56.9, opacity=0.5] (1.5461, -0.1171, 2.8198) -- (1.5922, -0.1171, 2.8193) -- (1.5921, -0.1180, 2.8693) -- (1.5461, -0.1180, 2.8698) -- cycle;
\fill[blue!57.5, opacity=0.5] (1.5461, -0.1180, 2.8698) -- (1.5921, -0.1180, 2.8693) -- (1.5921, -0.1187, 2.9193) -- (1.5460, -0.1187, 2.9198) -- cycle;
\fill[blue!57.9, opacity=0.5] (1.5460, -0.1187, 2.9198) -- (1.5921, -0.1187, 2.9193) -- (1.5920, -0.1193, 2.9693) -- (1.5460, -0.1193, 2.9698) -- cycle;
\fill[blue!58.2, opacity=0.5] (1.5460, -0.1193, 2.9698) -- (1.5920, -0.1193, 2.9693) -- (1.5920, -0.1197, 3.0193) -- (1.5460, -0.1197, 3.0198) -- cycle;
\fill[blue!58.3, opacity=0.5] (1.5460, -0.1197, 3.0198) -- (1.5920, -0.1197, 3.0193) -- (1.5920, -0.1199, 3.0693) -- (1.5460, -0.1199, 3.0698) -- cycle;
\fill[blue!58.3, opacity=0.5] (1.5460, -0.1199, 3.0698) -- (1.5920, -0.1199, 3.0693) -- (1.5920, -0.1200, 3.1193) -- (1.5460, -0.1200, 3.1198) -- cycle;
\fill[blue!15.0, opacity=0.5] (1.6000, -0.0000, 0.1193) -- (1.6500, -0.0000, 0.1185) -- (1.6500, -0.0001, 0.1685) -- (1.6000, -0.0001, 0.1693) -- cycle;
\fill[blue!15.0, opacity=0.5] (1.6000, -0.0001, 0.1693) -- (1.6500, -0.0001, 0.1685) -- (1.6500, -0.0003, 0.2185) -- (1.6000, -0.0003, 0.2193) -- cycle;
\fill[blue!15.0, opacity=0.5] (1.6000, -0.0003, 0.2193) -- (1.6500, -0.0003, 0.2185) -- (1.6499, -0.0007, 0.2685) -- (1.6000, -0.0007, 0.2693) -- cycle;
\fill[blue!15.0, opacity=0.5] (1.6000, -0.0007, 0.2693) -- (1.6499, -0.0007, 0.2685) -- (1.6499, -0.0013, 0.3185) -- (1.5999, -0.0013, 0.3193) -- cycle;
\fill[blue!15.0, opacity=0.5] (1.5999, -0.0013, 0.3193) -- (1.6499, -0.0013, 0.3185) -- (1.6498, -0.0020, 0.3685) -- (1.5999, -0.0020, 0.3693) -- cycle;
\fill[blue!15.0, opacity=0.5] (1.5999, -0.0020, 0.3693) -- (1.6498, -0.0020, 0.3685) -- (1.6497, -0.0029, 0.4185) -- (1.5998, -0.0029, 0.4193) -- cycle;
\fill[blue!15.0, opacity=0.5] (1.5998, -0.0029, 0.4193) -- (1.6497, -0.0029, 0.4185) -- (1.6496, -0.0040, 0.4685) -- (1.5997, -0.0040, 0.4693) -- cycle;
\fill[blue!15.0, opacity=0.5] (1.5997, -0.0040, 0.4693) -- (1.6496, -0.0040, 0.4685) -- (1.6495, -0.0052, 0.5185) -- (1.5997, -0.0052, 0.5193) -- cycle;
\fill[blue!15.0, opacity=0.5] (1.5997, -0.0052, 0.5193) -- (1.6495, -0.0052, 0.5185) -- (1.6493, -0.0065, 0.5685) -- (1.5996, -0.0065, 0.5693) -- cycle;
\fill[blue!15.0, opacity=0.5] (1.5996, -0.0065, 0.5693) -- (1.6493, -0.0065, 0.5685) -- (1.6492, -0.0080, 0.6185) -- (1.5995, -0.0080, 0.6193) -- cycle;
\fill[blue!15.0, opacity=0.5] (1.5995, -0.0080, 0.6193) -- (1.6492, -0.0080, 0.6185) -- (1.6490, -0.0097, 0.6685) -- (1.5994, -0.0097, 0.6693) -- cycle;
\fill[blue!15.0, opacity=0.5] (1.5994, -0.0097, 0.6693) -- (1.6490, -0.0097, 0.6685) -- (1.6489, -0.0115, 0.7185) -- (1.5992, -0.0115, 0.7193) -- cycle;
\fill[blue!15.0, opacity=0.5] (1.5992, -0.0115, 0.7193) -- (1.6489, -0.0115, 0.7185) -- (1.6487, -0.0134, 0.7685) -- (1.5991, -0.0134, 0.7693) -- cycle;
\fill[blue!15.0, opacity=0.5] (1.5991, -0.0134, 0.7693) -- (1.6487, -0.0134, 0.7685) -- (1.6485, -0.0154, 0.8185) -- (1.5990, -0.0154, 0.8193) -- cycle;
\fill[blue!15.0, opacity=0.5] (1.5990, -0.0154, 0.8193) -- (1.6485, -0.0154, 0.8185) -- (1.6482, -0.0176, 0.8685) -- (1.5988, -0.0176, 0.8693) -- cycle;
\fill[blue!15.0, opacity=0.5] (1.5988, -0.0176, 0.8693) -- (1.6482, -0.0176, 0.8685) -- (1.6480, -0.0199, 0.9185) -- (1.5987, -0.0199, 0.9193) -- cycle;
\fill[blue!15.0, opacity=0.5] (1.5987, -0.0199, 0.9193) -- (1.6480, -0.0199, 0.9185) -- (1.6478, -0.0222, 0.9685) -- (1.5985, -0.0222, 0.9693) -- cycle;
\fill[blue!15.1, opacity=0.5] (1.5985, -0.0222, 0.9693) -- (1.6478, -0.0222, 0.9685) -- (1.6475, -0.0247, 1.0185) -- (1.5984, -0.0247, 1.0193) -- cycle;
\fill[blue!15.1, opacity=0.5] (1.5984, -0.0247, 1.0193) -- (1.6475, -0.0247, 1.0185) -- (1.6473, -0.0273, 1.0685) -- (1.5982, -0.0273, 1.0693) -- cycle;
\fill[blue!15.2, opacity=0.5] (1.5982, -0.0273, 1.0693) -- (1.6473, -0.0273, 1.0685) -- (1.6470, -0.0300, 1.1185) -- (1.5980, -0.0300, 1.1193) -- cycle;
\fill[blue!15.3, opacity=0.5] (1.5980, -0.0300, 1.1193) -- (1.6470, -0.0300, 1.1185) -- (1.6467, -0.0328, 1.1685) -- (1.5978, -0.0328, 1.1693) -- cycle;
\fill[blue!15.4, opacity=0.5] (1.5978, -0.0328, 1.1693) -- (1.6467, -0.0328, 1.1685) -- (1.6464, -0.0356, 1.2185) -- (1.5976, -0.0356, 1.2193) -- cycle;
\fill[blue!15.6, opacity=0.5] (1.5976, -0.0356, 1.2193) -- (1.6464, -0.0356, 1.2185) -- (1.6462, -0.0385, 1.2685) -- (1.5974, -0.0385, 1.2693) -- cycle;
\fill[blue!15.9, opacity=0.5] (1.5974, -0.0385, 1.2693) -- (1.6462, -0.0385, 1.2685) -- (1.6459, -0.0415, 1.3185) -- (1.5972, -0.0415, 1.3193) -- cycle;
\fill[blue!16.2, opacity=0.5] (1.5972, -0.0415, 1.3193) -- (1.6459, -0.0415, 1.3185) -- (1.6456, -0.0445, 1.3685) -- (1.5970, -0.0445, 1.3693) -- cycle;
\fill[blue!16.7, opacity=0.5] (1.5970, -0.0445, 1.3693) -- (1.6456, -0.0445, 1.3685) -- (1.6452, -0.0475, 1.4185) -- (1.5968, -0.0475, 1.4193) -- cycle;
\fill[blue!17.3, opacity=0.5] (1.5968, -0.0475, 1.4193) -- (1.6452, -0.0475, 1.4185) -- (1.6449, -0.0506, 1.4685) -- (1.5966, -0.0506, 1.4693) -- cycle;
\fill[blue!18.0, opacity=0.5] (1.5966, -0.0506, 1.4693) -- (1.6449, -0.0506, 1.4685) -- (1.6446, -0.0537, 1.5185) -- (1.5964, -0.0537, 1.5193) -- cycle;
\fill[blue!18.9, opacity=0.5] (1.5964, -0.0537, 1.5193) -- (1.6446, -0.0537, 1.5185) -- (1.6443, -0.0569, 1.5685) -- (1.5962, -0.0569, 1.5693) -- cycle;
\fill[blue!19.9, opacity=0.5] (1.5962, -0.0569, 1.5693) -- (1.6443, -0.0569, 1.5685) -- (1.6440, -0.0600, 1.6185) -- (1.5960, -0.0600, 1.6193) -- cycle;
\fill[blue!21.0, opacity=0.5] (1.5960, -0.0600, 1.6193) -- (1.6440, -0.0600, 1.6185) -- (1.6437, -0.0631, 1.6685) -- (1.5958, -0.0631, 1.6693) -- cycle;
\fill[blue!22.4, opacity=0.5] (1.5958, -0.0631, 1.6693) -- (1.6437, -0.0631, 1.6685) -- (1.6434, -0.0663, 1.7185) -- (1.5956, -0.0663, 1.7193) -- cycle;
\fill[blue!23.9, opacity=0.5] (1.5956, -0.0663, 1.7193) -- (1.6434, -0.0663, 1.7185) -- (1.6431, -0.0694, 1.7685) -- (1.5954, -0.0694, 1.7693) -- cycle;
\fill[blue!25.5, opacity=0.5] (1.5954, -0.0694, 1.7693) -- (1.6431, -0.0694, 1.7685) -- (1.6428, -0.0725, 1.8185) -- (1.5952, -0.0725, 1.8193) -- cycle;
\fill[blue!27.2, opacity=0.5] (1.5952, -0.0725, 1.8193) -- (1.6428, -0.0725, 1.8185) -- (1.6424, -0.0755, 1.8685) -- (1.5950, -0.0755, 1.8693) -- cycle;
\fill[blue!29.1, opacity=0.5] (1.5950, -0.0755, 1.8693) -- (1.6424, -0.0755, 1.8685) -- (1.6421, -0.0785, 1.9185) -- (1.5948, -0.0785, 1.9193) -- cycle;
\fill[blue!31.1, opacity=0.5] (1.5948, -0.0785, 1.9193) -- (1.6421, -0.0785, 1.9185) -- (1.6418, -0.0815, 1.9685) -- (1.5946, -0.0815, 1.9693) -- cycle;
\fill[blue!33.2, opacity=0.5] (1.5946, -0.0815, 1.9693) -- (1.6418, -0.0815, 1.9685) -- (1.6416, -0.0844, 2.0185) -- (1.5944, -0.0844, 2.0193) -- cycle;
\fill[blue!35.3, opacity=0.5] (1.5944, -0.0844, 2.0193) -- (1.6416, -0.0844, 2.0185) -- (1.6413, -0.0872, 2.0685) -- (1.5942, -0.0872, 2.0693) -- cycle;
\fill[blue!37.4, opacity=0.5] (1.5942, -0.0872, 2.0693) -- (1.6413, -0.0872, 2.0685) -- (1.6410, -0.0900, 2.1185) -- (1.5940, -0.0900, 2.1193) -- cycle;
\fill[blue!39.6, opacity=0.5] (1.5940, -0.0900, 2.1193) -- (1.6410, -0.0900, 2.1185) -- (1.6407, -0.0927, 2.1685) -- (1.5938, -0.0927, 2.1693) -- cycle;
\fill[blue!41.7, opacity=0.5] (1.5938, -0.0927, 2.1693) -- (1.6407, -0.0927, 2.1685) -- (1.6405, -0.0953, 2.2185) -- (1.5936, -0.0953, 2.2193) -- cycle;
\fill[blue!43.8, opacity=0.5] (1.5936, -0.0953, 2.2193) -- (1.6405, -0.0953, 2.2185) -- (1.6402, -0.0978, 2.2685) -- (1.5935, -0.0978, 2.2693) -- cycle;
\fill[blue!45.8, opacity=0.5] (1.5935, -0.0978, 2.2693) -- (1.6402, -0.0978, 2.2685) -- (1.6400, -0.1001, 2.3185) -- (1.5933, -0.1001, 2.3193) -- cycle;
\fill[blue!47.8, opacity=0.5] (1.5933, -0.1001, 2.3193) -- (1.6400, -0.1001, 2.3185) -- (1.6398, -0.1024, 2.3685) -- (1.5932, -0.1024, 2.3693) -- cycle;
\fill[blue!49.6, opacity=0.5] (1.5932, -0.1024, 2.3693) -- (1.6398, -0.1024, 2.3685) -- (1.6395, -0.1046, 2.4185) -- (1.5930, -0.1046, 2.4193) -- cycle;
\fill[blue!51.4, opacity=0.5] (1.5930, -0.1046, 2.4193) -- (1.6395, -0.1046, 2.4185) -- (1.6393, -0.1066, 2.4685) -- (1.5929, -0.1066, 2.4693) -- cycle;
\fill[blue!53.0, opacity=0.5] (1.5929, -0.1066, 2.4693) -- (1.6393, -0.1066, 2.4685) -- (1.6391, -0.1085, 2.5185) -- (1.5928, -0.1085, 2.5193) -- cycle;
\fill[blue!54.5, opacity=0.5] (1.5928, -0.1085, 2.5193) -- (1.6391, -0.1085, 2.5185) -- (1.6390, -0.1103, 2.5685) -- (1.5926, -0.1103, 2.5693) -- cycle;
\fill[blue!55.8, opacity=0.5] (1.5926, -0.1103, 2.5693) -- (1.6390, -0.1103, 2.5685) -- (1.6388, -0.1120, 2.6185) -- (1.5925, -0.1120, 2.6193) -- cycle;
\fill[blue!57.0, opacity=0.5] (1.5925, -0.1120, 2.6193) -- (1.6388, -0.1120, 2.6185) -- (1.6387, -0.1135, 2.6685) -- (1.5924, -0.1135, 2.6693) -- cycle;
\fill[blue!58.0, opacity=0.5] (1.5924, -0.1135, 2.6693) -- (1.6387, -0.1135, 2.6685) -- (1.6385, -0.1148, 2.7185) -- (1.5923, -0.1148, 2.7193) -- cycle;
\fill[blue!58.8, opacity=0.5] (1.5923, -0.1148, 2.7193) -- (1.6385, -0.1148, 2.7185) -- (1.6384, -0.1160, 2.7685) -- (1.5923, -0.1160, 2.7693) -- cycle;
\fill[blue!59.5, opacity=0.5] (1.5923, -0.1160, 2.7693) -- (1.6384, -0.1160, 2.7685) -- (1.6383, -0.1171, 2.8185) -- (1.5922, -0.1171, 2.8193) -- cycle;
\fill[blue!60.0, opacity=0.5] (1.5922, -0.1171, 2.8193) -- (1.6383, -0.1171, 2.8185) -- (1.6382, -0.1180, 2.8685) -- (1.5921, -0.1180, 2.8693) -- cycle;
\fill[blue!60.4, opacity=0.5] (1.5921, -0.1180, 2.8693) -- (1.6382, -0.1180, 2.8685) -- (1.6381, -0.1187, 2.9185) -- (1.5921, -0.1187, 2.9193) -- cycle;
\fill[blue!60.6, opacity=0.5] (1.5921, -0.1187, 2.9193) -- (1.6381, -0.1187, 2.9185) -- (1.6381, -0.1193, 2.9685) -- (1.5920, -0.1193, 2.9693) -- cycle;
\fill[blue!60.6, opacity=0.5] (1.5920, -0.1193, 2.9693) -- (1.6381, -0.1193, 2.9685) -- (1.6380, -0.1197, 3.0185) -- (1.5920, -0.1197, 3.0193) -- cycle;
\fill[blue!60.5, opacity=0.5] (1.5920, -0.1197, 3.0193) -- (1.6380, -0.1197, 3.0185) -- (1.6380, -0.1199, 3.0685) -- (1.5920, -0.1199, 3.0693) -- cycle;
\fill[blue!60.3, opacity=0.5] (1.5920, -0.1199, 3.0693) -- (1.6380, -0.1199, 3.0685) -- (1.6380, -0.1200, 3.1185) -- (1.5920, -0.1200, 3.1193) -- cycle;
\fill[blue!15.0, opacity=0.5] (1.6500, -0.0000, 0.1185) -- (1.7000, -0.0000, 0.1174) -- (1.7000, -0.0001, 0.1674) -- (1.6500, -0.0001, 0.1685) -- cycle;
\fill[blue!15.0, opacity=0.5] (1.6500, -0.0001, 0.1685) -- (1.7000, -0.0001, 0.1674) -- (1.7000, -0.0003, 0.2174) -- (1.6500, -0.0003, 0.2185) -- cycle;
\fill[blue!15.0, opacity=0.5] (1.6500, -0.0003, 0.2185) -- (1.7000, -0.0003, 0.2174) -- (1.6999, -0.0007, 0.2674) -- (1.6499, -0.0007, 0.2685) -- cycle;
\fill[blue!15.0, opacity=0.5] (1.6499, -0.0007, 0.2685) -- (1.6999, -0.0007, 0.2674) -- (1.6998, -0.0013, 0.3174) -- (1.6499, -0.0013, 0.3185) -- cycle;
\fill[blue!15.0, opacity=0.5] (1.6499, -0.0013, 0.3185) -- (1.6998, -0.0013, 0.3174) -- (1.6997, -0.0020, 0.3674) -- (1.6498, -0.0020, 0.3685) -- cycle;
\fill[blue!15.0, opacity=0.5] (1.6498, -0.0020, 0.3685) -- (1.6997, -0.0020, 0.3674) -- (1.6996, -0.0029, 0.4174) -- (1.6497, -0.0029, 0.4185) -- cycle;
\fill[blue!15.0, opacity=0.5] (1.6497, -0.0029, 0.4185) -- (1.6996, -0.0029, 0.4174) -- (1.6995, -0.0040, 0.4674) -- (1.6496, -0.0040, 0.4685) -- cycle;
\fill[blue!15.0, opacity=0.5] (1.6496, -0.0040, 0.4685) -- (1.6995, -0.0040, 0.4674) -- (1.6993, -0.0052, 0.5174) -- (1.6495, -0.0052, 0.5185) -- cycle;
\fill[blue!15.0, opacity=0.5] (1.6495, -0.0052, 0.5185) -- (1.6993, -0.0052, 0.5174) -- (1.6991, -0.0065, 0.5674) -- (1.6493, -0.0065, 0.5685) -- cycle;
\fill[blue!15.0, opacity=0.5] (1.6493, -0.0065, 0.5685) -- (1.6991, -0.0065, 0.5674) -- (1.6989, -0.0080, 0.6174) -- (1.6492, -0.0080, 0.6185) -- cycle;
\fill[blue!15.0, opacity=0.5] (1.6492, -0.0080, 0.6185) -- (1.6989, -0.0080, 0.6174) -- (1.6987, -0.0097, 0.6674) -- (1.6490, -0.0097, 0.6685) -- cycle;
\fill[blue!15.0, opacity=0.5] (1.6490, -0.0097, 0.6685) -- (1.6987, -0.0097, 0.6674) -- (1.6985, -0.0115, 0.7174) -- (1.6489, -0.0115, 0.7185) -- cycle;
\fill[blue!15.0, opacity=0.5] (1.6489, -0.0115, 0.7185) -- (1.6985, -0.0115, 0.7174) -- (1.6982, -0.0134, 0.7674) -- (1.6487, -0.0134, 0.7685) -- cycle;
\fill[blue!15.0, opacity=0.5] (1.6487, -0.0134, 0.7685) -- (1.6982, -0.0134, 0.7674) -- (1.6979, -0.0154, 0.8174) -- (1.6485, -0.0154, 0.8185) -- cycle;
\fill[blue!15.0, opacity=0.5] (1.6485, -0.0154, 0.8185) -- (1.6979, -0.0154, 0.8174) -- (1.6977, -0.0176, 0.8674) -- (1.6482, -0.0176, 0.8685) -- cycle;
\fill[blue!15.0, opacity=0.5] (1.6482, -0.0176, 0.8685) -- (1.6977, -0.0176, 0.8674) -- (1.6974, -0.0199, 0.9174) -- (1.6480, -0.0199, 0.9185) -- cycle;
\fill[blue!15.1, opacity=0.5] (1.6480, -0.0199, 0.9185) -- (1.6974, -0.0199, 0.9174) -- (1.6970, -0.0222, 0.9674) -- (1.6478, -0.0222, 0.9685) -- cycle;
\fill[blue!15.1, opacity=0.5] (1.6478, -0.0222, 0.9685) -- (1.6970, -0.0222, 0.9674) -- (1.6967, -0.0247, 1.0174) -- (1.6475, -0.0247, 1.0185) -- cycle;
\fill[blue!15.2, opacity=0.5] (1.6475, -0.0247, 1.0185) -- (1.6967, -0.0247, 1.0174) -- (1.6964, -0.0273, 1.0674) -- (1.6473, -0.0273, 1.0685) -- cycle;
\fill[blue!15.3, opacity=0.5] (1.6473, -0.0273, 1.0685) -- (1.6964, -0.0273, 1.0674) -- (1.6960, -0.0300, 1.1174) -- (1.6470, -0.0300, 1.1185) -- cycle;
\fill[blue!15.5, opacity=0.5] (1.6470, -0.0300, 1.1185) -- (1.6960, -0.0300, 1.1174) -- (1.6956, -0.0328, 1.1674) -- (1.6467, -0.0328, 1.1685) -- cycle;
\fill[blue!15.7, opacity=0.5] (1.6467, -0.0328, 1.1685) -- (1.6956, -0.0328, 1.1674) -- (1.6953, -0.0356, 1.2174) -- (1.6464, -0.0356, 1.2185) -- cycle;
\fill[blue!16.0, opacity=0.5] (1.6464, -0.0356, 1.2185) -- (1.6953, -0.0356, 1.2174) -- (1.6949, -0.0385, 1.2674) -- (1.6462, -0.0385, 1.2685) -- cycle;
\fill[blue!16.4, opacity=0.5] (1.6462, -0.0385, 1.2685) -- (1.6949, -0.0385, 1.2674) -- (1.6945, -0.0415, 1.3174) -- (1.6459, -0.0415, 1.3185) -- cycle;
\fill[blue!17.0, opacity=0.5] (1.6459, -0.0415, 1.3185) -- (1.6945, -0.0415, 1.3174) -- (1.6941, -0.0445, 1.3674) -- (1.6456, -0.0445, 1.3685) -- cycle;
\fill[blue!17.6, opacity=0.5] (1.6456, -0.0445, 1.3685) -- (1.6941, -0.0445, 1.3674) -- (1.6937, -0.0475, 1.4174) -- (1.6452, -0.0475, 1.4185) -- cycle;
\fill[blue!18.5, opacity=0.5] (1.6452, -0.0475, 1.4185) -- (1.6937, -0.0475, 1.4174) -- (1.6933, -0.0506, 1.4674) -- (1.6449, -0.0506, 1.4685) -- cycle;
\fill[blue!19.5, opacity=0.5] (1.6449, -0.0506, 1.4685) -- (1.6933, -0.0506, 1.4674) -- (1.6928, -0.0537, 1.5174) -- (1.6446, -0.0537, 1.5185) -- cycle;
\fill[blue!20.6, opacity=0.5] (1.6446, -0.0537, 1.5185) -- (1.6928, -0.0537, 1.5174) -- (1.6924, -0.0569, 1.5674) -- (1.6443, -0.0569, 1.5685) -- cycle;
\fill[blue!21.9, opacity=0.5] (1.6443, -0.0569, 1.5685) -- (1.6924, -0.0569, 1.5674) -- (1.6920, -0.0600, 1.6174) -- (1.6440, -0.0600, 1.6185) -- cycle;
\fill[blue!23.4, opacity=0.5] (1.6440, -0.0600, 1.6185) -- (1.6920, -0.0600, 1.6174) -- (1.6916, -0.0631, 1.6674) -- (1.6437, -0.0631, 1.6685) -- cycle;
\fill[blue!25.1, opacity=0.5] (1.6437, -0.0631, 1.6685) -- (1.6916, -0.0631, 1.6674) -- (1.6912, -0.0663, 1.7174) -- (1.6434, -0.0663, 1.7185) -- cycle;
\fill[blue!26.9, opacity=0.5] (1.6434, -0.0663, 1.7185) -- (1.6912, -0.0663, 1.7174) -- (1.6907, -0.0694, 1.7674) -- (1.6431, -0.0694, 1.7685) -- cycle;
\fill[blue!28.8, opacity=0.5] (1.6431, -0.0694, 1.7685) -- (1.6907, -0.0694, 1.7674) -- (1.6903, -0.0725, 1.8174) -- (1.6428, -0.0725, 1.8185) -- cycle;
\fill[blue!30.9, opacity=0.5] (1.6428, -0.0725, 1.8185) -- (1.6903, -0.0725, 1.8174) -- (1.6899, -0.0755, 1.8674) -- (1.6424, -0.0755, 1.8685) -- cycle;
\fill[blue!33.1, opacity=0.5] (1.6424, -0.0755, 1.8685) -- (1.6899, -0.0755, 1.8674) -- (1.6895, -0.0785, 1.9174) -- (1.6421, -0.0785, 1.9185) -- cycle;
\fill[blue!35.3, opacity=0.5] (1.6421, -0.0785, 1.9185) -- (1.6895, -0.0785, 1.9174) -- (1.6891, -0.0815, 1.9674) -- (1.6418, -0.0815, 1.9685) -- cycle;
\fill[blue!37.6, opacity=0.5] (1.6418, -0.0815, 1.9685) -- (1.6891, -0.0815, 1.9674) -- (1.6887, -0.0844, 2.0174) -- (1.6416, -0.0844, 2.0185) -- cycle;
\fill[blue!39.9, opacity=0.5] (1.6416, -0.0844, 2.0185) -- (1.6887, -0.0844, 2.0174) -- (1.6884, -0.0872, 2.0674) -- (1.6413, -0.0872, 2.0685) -- cycle;
\fill[blue!42.2, opacity=0.5] (1.6413, -0.0872, 2.0685) -- (1.6884, -0.0872, 2.0674) -- (1.6880, -0.0900, 2.1174) -- (1.6410, -0.0900, 2.1185) -- cycle;
\fill[blue!44.4, opacity=0.5] (1.6410, -0.0900, 2.1185) -- (1.6880, -0.0900, 2.1174) -- (1.6876, -0.0927, 2.1674) -- (1.6407, -0.0927, 2.1685) -- cycle;
\fill[blue!46.6, opacity=0.5] (1.6407, -0.0927, 2.1685) -- (1.6876, -0.0927, 2.1674) -- (1.6873, -0.0953, 2.2174) -- (1.6405, -0.0953, 2.2185) -- cycle;
\fill[blue!48.7, opacity=0.5] (1.6405, -0.0953, 2.2185) -- (1.6873, -0.0953, 2.2174) -- (1.6870, -0.0978, 2.2674) -- (1.6402, -0.0978, 2.2685) -- cycle;
\fill[blue!50.8, opacity=0.5] (1.6402, -0.0978, 2.2685) -- (1.6870, -0.0978, 2.2674) -- (1.6866, -0.1001, 2.3174) -- (1.6400, -0.1001, 2.3185) -- cycle;
\fill[blue!52.7, opacity=0.5] (1.6400, -0.1001, 2.3185) -- (1.6866, -0.1001, 2.3174) -- (1.6863, -0.1024, 2.3674) -- (1.6398, -0.1024, 2.3685) -- cycle;
\fill[blue!54.4, opacity=0.5] (1.6398, -0.1024, 2.3685) -- (1.6863, -0.1024, 2.3674) -- (1.6861, -0.1046, 2.4174) -- (1.6395, -0.1046, 2.4185) -- cycle;
\fill[blue!56.1, opacity=0.5] (1.6395, -0.1046, 2.4185) -- (1.6861, -0.1046, 2.4174) -- (1.6858, -0.1066, 2.4674) -- (1.6393, -0.1066, 2.4685) -- cycle;
\fill[blue!57.5, opacity=0.5] (1.6393, -0.1066, 2.4685) -- (1.6858, -0.1066, 2.4674) -- (1.6855, -0.1085, 2.5174) -- (1.6391, -0.1085, 2.5185) -- cycle;
\fill[blue!58.8, opacity=0.5] (1.6391, -0.1085, 2.5185) -- (1.6855, -0.1085, 2.5174) -- (1.6853, -0.1103, 2.5674) -- (1.6390, -0.1103, 2.5685) -- cycle;
\fill[blue!59.9, opacity=0.5] (1.6390, -0.1103, 2.5685) -- (1.6853, -0.1103, 2.5674) -- (1.6851, -0.1120, 2.6174) -- (1.6388, -0.1120, 2.6185) -- cycle;
\fill[blue!60.9, opacity=0.5] (1.6388, -0.1120, 2.6185) -- (1.6851, -0.1120, 2.6174) -- (1.6849, -0.1135, 2.6674) -- (1.6387, -0.1135, 2.6685) -- cycle;
\fill[blue!61.7, opacity=0.5] (1.6387, -0.1135, 2.6685) -- (1.6849, -0.1135, 2.6674) -- (1.6847, -0.1148, 2.7174) -- (1.6385, -0.1148, 2.7185) -- cycle;
\fill[blue!62.3, opacity=0.5] (1.6385, -0.1148, 2.7185) -- (1.6847, -0.1148, 2.7174) -- (1.6845, -0.1160, 2.7674) -- (1.6384, -0.1160, 2.7685) -- cycle;
\fill[blue!62.7, opacity=0.5] (1.6384, -0.1160, 2.7685) -- (1.6845, -0.1160, 2.7674) -- (1.6844, -0.1171, 2.8174) -- (1.6383, -0.1171, 2.8185) -- cycle;
\fill[blue!62.9, opacity=0.5] (1.6383, -0.1171, 2.8185) -- (1.6844, -0.1171, 2.8174) -- (1.6843, -0.1180, 2.8674) -- (1.6382, -0.1180, 2.8685) -- cycle;
\fill[blue!63.0, opacity=0.5] (1.6382, -0.1180, 2.8685) -- (1.6843, -0.1180, 2.8674) -- (1.6842, -0.1187, 2.9174) -- (1.6381, -0.1187, 2.9185) -- cycle;
\fill[blue!63.0, opacity=0.5] (1.6381, -0.1187, 2.9185) -- (1.6842, -0.1187, 2.9174) -- (1.6841, -0.1193, 2.9674) -- (1.6381, -0.1193, 2.9685) -- cycle;
\fill[blue!62.8, opacity=0.5] (1.6381, -0.1193, 2.9685) -- (1.6841, -0.1193, 2.9674) -- (1.6840, -0.1197, 3.0174) -- (1.6380, -0.1197, 3.0185) -- cycle;
\fill[blue!62.4, opacity=0.5] (1.6380, -0.1197, 3.0185) -- (1.6840, -0.1197, 3.0174) -- (1.6840, -0.1199, 3.0674) -- (1.6380, -0.1199, 3.0685) -- cycle;
\fill[blue!61.9, opacity=0.5] (1.6380, -0.1199, 3.0685) -- (1.6840, -0.1199, 3.0674) -- (1.6840, -0.1200, 3.1174) -- (1.6380, -0.1200, 3.1185) -- cycle;
\fill[blue!15.0, opacity=0.5] (1.7000, -0.0000, 0.1174) -- (1.7500, -0.0000, 0.1159) -- (1.7500, -0.0001, 0.1659) -- (1.7000, -0.0001, 0.1674) -- cycle;
\fill[blue!15.0, opacity=0.5] (1.7000, -0.0001, 0.1674) -- (1.7500, -0.0001, 0.1659) -- (1.7499, -0.0003, 0.2159) -- (1.7000, -0.0003, 0.2174) -- cycle;
\fill[blue!15.0, opacity=0.5] (1.7000, -0.0003, 0.2174) -- (1.7499, -0.0003, 0.2159) -- (1.7499, -0.0007, 0.2659) -- (1.6999, -0.0007, 0.2674) -- cycle;
\fill[blue!15.0, opacity=0.5] (1.6999, -0.0007, 0.2674) -- (1.7499, -0.0007, 0.2659) -- (1.7498, -0.0013, 0.3159) -- (1.6998, -0.0013, 0.3174) -- cycle;
\fill[blue!15.0, opacity=0.5] (1.6998, -0.0013, 0.3174) -- (1.7498, -0.0013, 0.3159) -- (1.7497, -0.0020, 0.3659) -- (1.6997, -0.0020, 0.3674) -- cycle;
\fill[blue!15.0, opacity=0.5] (1.6997, -0.0020, 0.3674) -- (1.7497, -0.0020, 0.3659) -- (1.7495, -0.0029, 0.4159) -- (1.6996, -0.0029, 0.4174) -- cycle;
\fill[blue!15.0, opacity=0.5] (1.6996, -0.0029, 0.4174) -- (1.7495, -0.0029, 0.4159) -- (1.7493, -0.0040, 0.4659) -- (1.6995, -0.0040, 0.4674) -- cycle;
\fill[blue!15.0, opacity=0.5] (1.6995, -0.0040, 0.4674) -- (1.7493, -0.0040, 0.4659) -- (1.7491, -0.0052, 0.5159) -- (1.6993, -0.0052, 0.5174) -- cycle;
\fill[blue!15.0, opacity=0.5] (1.6993, -0.0052, 0.5174) -- (1.7491, -0.0052, 0.5159) -- (1.7489, -0.0065, 0.5659) -- (1.6991, -0.0065, 0.5674) -- cycle;
\fill[blue!15.0, opacity=0.5] (1.6991, -0.0065, 0.5674) -- (1.7489, -0.0065, 0.5659) -- (1.7487, -0.0080, 0.6159) -- (1.6989, -0.0080, 0.6174) -- cycle;
\fill[blue!15.0, opacity=0.5] (1.6989, -0.0080, 0.6174) -- (1.7487, -0.0080, 0.6159) -- (1.7484, -0.0097, 0.6659) -- (1.6987, -0.0097, 0.6674) -- cycle;
\fill[blue!15.0, opacity=0.5] (1.6987, -0.0097, 0.6674) -- (1.7484, -0.0097, 0.6659) -- (1.7481, -0.0115, 0.7159) -- (1.6985, -0.0115, 0.7174) -- cycle;
\fill[blue!15.0, opacity=0.5] (1.6985, -0.0115, 0.7174) -- (1.7481, -0.0115, 0.7159) -- (1.7478, -0.0134, 0.7659) -- (1.6982, -0.0134, 0.7674) -- cycle;
\fill[blue!15.0, opacity=0.5] (1.6982, -0.0134, 0.7674) -- (1.7478, -0.0134, 0.7659) -- (1.7474, -0.0154, 0.8159) -- (1.6979, -0.0154, 0.8174) -- cycle;
\fill[blue!15.0, opacity=0.5] (1.6979, -0.0154, 0.8174) -- (1.7474, -0.0154, 0.8159) -- (1.7471, -0.0176, 0.8659) -- (1.6977, -0.0176, 0.8674) -- cycle;
\fill[blue!15.1, opacity=0.5] (1.6977, -0.0176, 0.8674) -- (1.7471, -0.0176, 0.8659) -- (1.7467, -0.0199, 0.9159) -- (1.6974, -0.0199, 0.9174) -- cycle;
\fill[blue!15.1, opacity=0.5] (1.6974, -0.0199, 0.9174) -- (1.7467, -0.0199, 0.9159) -- (1.7463, -0.0222, 0.9659) -- (1.6970, -0.0222, 0.9674) -- cycle;
\fill[blue!15.2, opacity=0.5] (1.6970, -0.0222, 0.9674) -- (1.7463, -0.0222, 0.9659) -- (1.7459, -0.0247, 1.0159) -- (1.6967, -0.0247, 1.0174) -- cycle;
\fill[blue!15.3, opacity=0.5] (1.6967, -0.0247, 1.0174) -- (1.7459, -0.0247, 1.0159) -- (1.7454, -0.0273, 1.0659) -- (1.6964, -0.0273, 1.0674) -- cycle;
\fill[blue!15.5, opacity=0.5] (1.6964, -0.0273, 1.0674) -- (1.7454, -0.0273, 1.0659) -- (1.7450, -0.0300, 1.1159) -- (1.6960, -0.0300, 1.1174) -- cycle;
\fill[blue!15.8, opacity=0.5] (1.6960, -0.0300, 1.1174) -- (1.7450, -0.0300, 1.1159) -- (1.7445, -0.0328, 1.1659) -- (1.6956, -0.0328, 1.1674) -- cycle;
\fill[blue!16.1, opacity=0.5] (1.6956, -0.0328, 1.1674) -- (1.7445, -0.0328, 1.1659) -- (1.7441, -0.0356, 1.2159) -- (1.6953, -0.0356, 1.2174) -- cycle;
\fill[blue!16.6, opacity=0.5] (1.6953, -0.0356, 1.2174) -- (1.7441, -0.0356, 1.2159) -- (1.7436, -0.0385, 1.2659) -- (1.6949, -0.0385, 1.2674) -- cycle;
\fill[blue!17.2, opacity=0.5] (1.6949, -0.0385, 1.2674) -- (1.7436, -0.0385, 1.2659) -- (1.7431, -0.0415, 1.3159) -- (1.6945, -0.0415, 1.3174) -- cycle;
\fill[blue!18.0, opacity=0.5] (1.6945, -0.0415, 1.3174) -- (1.7431, -0.0415, 1.3159) -- (1.7426, -0.0445, 1.3659) -- (1.6941, -0.0445, 1.3674) -- cycle;
\fill[blue!18.9, opacity=0.5] (1.6941, -0.0445, 1.3674) -- (1.7426, -0.0445, 1.3659) -- (1.7421, -0.0475, 1.4159) -- (1.6937, -0.0475, 1.4174) -- cycle;
\fill[blue!20.0, opacity=0.5] (1.6937, -0.0475, 1.4174) -- (1.7421, -0.0475, 1.4159) -- (1.7416, -0.0506, 1.4659) -- (1.6933, -0.0506, 1.4674) -- cycle;
\fill[blue!21.3, opacity=0.5] (1.6933, -0.0506, 1.4674) -- (1.7416, -0.0506, 1.4659) -- (1.7410, -0.0537, 1.5159) -- (1.6928, -0.0537, 1.5174) -- cycle;
\fill[blue!22.8, opacity=0.5] (1.6928, -0.0537, 1.5174) -- (1.7410, -0.0537, 1.5159) -- (1.7405, -0.0569, 1.5659) -- (1.6924, -0.0569, 1.5674) -- cycle;
\fill[blue!24.4, opacity=0.5] (1.6924, -0.0569, 1.5674) -- (1.7405, -0.0569, 1.5659) -- (1.7400, -0.0600, 1.6159) -- (1.6920, -0.0600, 1.6174) -- cycle;
\fill[blue!26.2, opacity=0.5] (1.6920, -0.0600, 1.6174) -- (1.7400, -0.0600, 1.6159) -- (1.7395, -0.0631, 1.6659) -- (1.6916, -0.0631, 1.6674) -- cycle;
\fill[blue!28.2, opacity=0.5] (1.6916, -0.0631, 1.6674) -- (1.7395, -0.0631, 1.6659) -- (1.7390, -0.0663, 1.7159) -- (1.6912, -0.0663, 1.7174) -- cycle;
\fill[blue!30.3, opacity=0.5] (1.6912, -0.0663, 1.7174) -- (1.7390, -0.0663, 1.7159) -- (1.7384, -0.0694, 1.7659) -- (1.6907, -0.0694, 1.7674) -- cycle;
\fill[blue!32.6, opacity=0.5] (1.6907, -0.0694, 1.7674) -- (1.7384, -0.0694, 1.7659) -- (1.7379, -0.0725, 1.8159) -- (1.6903, -0.0725, 1.8174) -- cycle;
\fill[blue!34.9, opacity=0.5] (1.6903, -0.0725, 1.8174) -- (1.7379, -0.0725, 1.8159) -- (1.7374, -0.0755, 1.8659) -- (1.6899, -0.0755, 1.8674) -- cycle;
\fill[blue!37.3, opacity=0.5] (1.6899, -0.0755, 1.8674) -- (1.7374, -0.0755, 1.8659) -- (1.7369, -0.0785, 1.9159) -- (1.6895, -0.0785, 1.9174) -- cycle;
\fill[blue!39.7, opacity=0.5] (1.6895, -0.0785, 1.9174) -- (1.7369, -0.0785, 1.9159) -- (1.7364, -0.0815, 1.9659) -- (1.6891, -0.0815, 1.9674) -- cycle;
\fill[blue!42.2, opacity=0.5] (1.6891, -0.0815, 1.9674) -- (1.7364, -0.0815, 1.9659) -- (1.7359, -0.0844, 2.0159) -- (1.6887, -0.0844, 2.0174) -- cycle;
\fill[blue!44.6, opacity=0.5] (1.6887, -0.0844, 2.0174) -- (1.7359, -0.0844, 2.0159) -- (1.7355, -0.0872, 2.0659) -- (1.6884, -0.0872, 2.0674) -- cycle;
\fill[blue!47.0, opacity=0.5] (1.6884, -0.0872, 2.0674) -- (1.7355, -0.0872, 2.0659) -- (1.7350, -0.0900, 2.1159) -- (1.6880, -0.0900, 2.1174) -- cycle;
\fill[blue!49.3, opacity=0.5] (1.6880, -0.0900, 2.1174) -- (1.7350, -0.0900, 2.1159) -- (1.7346, -0.0927, 2.1659) -- (1.6876, -0.0927, 2.1674) -- cycle;
\fill[blue!51.5, opacity=0.5] (1.6876, -0.0927, 2.1674) -- (1.7346, -0.0927, 2.1659) -- (1.7341, -0.0953, 2.2159) -- (1.6873, -0.0953, 2.2174) -- cycle;
\fill[blue!53.5, opacity=0.5] (1.6873, -0.0953, 2.2174) -- (1.7341, -0.0953, 2.2159) -- (1.7337, -0.0978, 2.2659) -- (1.6870, -0.0978, 2.2674) -- cycle;
\fill[blue!55.5, opacity=0.5] (1.6870, -0.0978, 2.2674) -- (1.7337, -0.0978, 2.2659) -- (1.7333, -0.1001, 2.3159) -- (1.6866, -0.1001, 2.3174) -- cycle;
\fill[blue!57.3, opacity=0.5] (1.6866, -0.1001, 2.3174) -- (1.7333, -0.1001, 2.3159) -- (1.7329, -0.1024, 2.3659) -- (1.6863, -0.1024, 2.3674) -- cycle;
\fill[blue!58.9, opacity=0.5] (1.6863, -0.1024, 2.3674) -- (1.7329, -0.1024, 2.3659) -- (1.7326, -0.1046, 2.4159) -- (1.6861, -0.1046, 2.4174) -- cycle;
\fill[blue!60.3, opacity=0.5] (1.6861, -0.1046, 2.4174) -- (1.7326, -0.1046, 2.4159) -- (1.7322, -0.1066, 2.4659) -- (1.6858, -0.1066, 2.4674) -- cycle;
\fill[blue!61.6, opacity=0.5] (1.6858, -0.1066, 2.4674) -- (1.7322, -0.1066, 2.4659) -- (1.7319, -0.1085, 2.5159) -- (1.6855, -0.1085, 2.5174) -- cycle;
\fill[blue!62.7, opacity=0.5] (1.6855, -0.1085, 2.5174) -- (1.7319, -0.1085, 2.5159) -- (1.7316, -0.1103, 2.5659) -- (1.6853, -0.1103, 2.5674) -- cycle;
\fill[blue!63.6, opacity=0.5] (1.6853, -0.1103, 2.5674) -- (1.7316, -0.1103, 2.5659) -- (1.7313, -0.1120, 2.6159) -- (1.6851, -0.1120, 2.6174) -- cycle;
\fill[blue!64.3, opacity=0.5] (1.6851, -0.1120, 2.6174) -- (1.7313, -0.1120, 2.6159) -- (1.7311, -0.1135, 2.6659) -- (1.6849, -0.1135, 2.6674) -- cycle;
\fill[blue!64.8, opacity=0.5] (1.6849, -0.1135, 2.6674) -- (1.7311, -0.1135, 2.6659) -- (1.7309, -0.1148, 2.7159) -- (1.6847, -0.1148, 2.7174) -- cycle;
\fill[blue!65.1, opacity=0.5] (1.6847, -0.1148, 2.7174) -- (1.7309, -0.1148, 2.7159) -- (1.7307, -0.1160, 2.7659) -- (1.6845, -0.1160, 2.7674) -- cycle;
\fill[blue!65.3, opacity=0.5] (1.6845, -0.1160, 2.7674) -- (1.7307, -0.1160, 2.7659) -- (1.7305, -0.1171, 2.8159) -- (1.6844, -0.1171, 2.8174) -- cycle;
\fill[blue!65.3, opacity=0.5] (1.6844, -0.1171, 2.8174) -- (1.7305, -0.1171, 2.8159) -- (1.7303, -0.1180, 2.8659) -- (1.6843, -0.1180, 2.8674) -- cycle;
\fill[blue!65.1, opacity=0.5] (1.6843, -0.1180, 2.8674) -- (1.7303, -0.1180, 2.8659) -- (1.7302, -0.1187, 2.9159) -- (1.6842, -0.1187, 2.9174) -- cycle;
\fill[blue!64.8, opacity=0.5] (1.6842, -0.1187, 2.9174) -- (1.7302, -0.1187, 2.9159) -- (1.7301, -0.1193, 2.9659) -- (1.6841, -0.1193, 2.9674) -- cycle;
\fill[blue!64.4, opacity=0.5] (1.6841, -0.1193, 2.9674) -- (1.7301, -0.1193, 2.9659) -- (1.7301, -0.1197, 3.0159) -- (1.6840, -0.1197, 3.0174) -- cycle;
\fill[blue!63.8, opacity=0.5] (1.6840, -0.1197, 3.0174) -- (1.7301, -0.1197, 3.0159) -- (1.7300, -0.1199, 3.0659) -- (1.6840, -0.1199, 3.0674) -- cycle;
\fill[blue!63.1, opacity=0.5] (1.6840, -0.1199, 3.0674) -- (1.7300, -0.1199, 3.0659) -- (1.7300, -0.1200, 3.1159) -- (1.6840, -0.1200, 3.1174) -- cycle;
\fill[blue!15.0, opacity=0.5] (1.7500, -0.0000, 0.1159) -- (1.8000, -0.0000, 0.1141) -- (1.8000, -0.0001, 0.1641) -- (1.7500, -0.0001, 0.1659) -- cycle;
\fill[blue!15.0, opacity=0.5] (1.7500, -0.0001, 0.1659) -- (1.8000, -0.0001, 0.1641) -- (1.7999, -0.0003, 0.2141) -- (1.7499, -0.0003, 0.2159) -- cycle;
\fill[blue!15.0, opacity=0.5] (1.7499, -0.0003, 0.2159) -- (1.7999, -0.0003, 0.2141) -- (1.7999, -0.0007, 0.2641) -- (1.7499, -0.0007, 0.2659) -- cycle;
\fill[blue!15.0, opacity=0.5] (1.7499, -0.0007, 0.2659) -- (1.7999, -0.0007, 0.2641) -- (1.7997, -0.0013, 0.3141) -- (1.7498, -0.0013, 0.3159) -- cycle;
\fill[blue!15.0, opacity=0.5] (1.7498, -0.0013, 0.3159) -- (1.7997, -0.0013, 0.3141) -- (1.7996, -0.0020, 0.3641) -- (1.7497, -0.0020, 0.3659) -- cycle;
\fill[blue!15.0, opacity=0.5] (1.7497, -0.0020, 0.3659) -- (1.7996, -0.0020, 0.3641) -- (1.7994, -0.0029, 0.4141) -- (1.7495, -0.0029, 0.4159) -- cycle;
\fill[blue!15.0, opacity=0.5] (1.7495, -0.0029, 0.4159) -- (1.7994, -0.0029, 0.4141) -- (1.7992, -0.0040, 0.4641) -- (1.7493, -0.0040, 0.4659) -- cycle;
\fill[blue!15.0, opacity=0.5] (1.7493, -0.0040, 0.4659) -- (1.7992, -0.0040, 0.4641) -- (1.7990, -0.0052, 0.5141) -- (1.7491, -0.0052, 0.5159) -- cycle;
\fill[blue!15.0, opacity=0.5] (1.7491, -0.0052, 0.5159) -- (1.7990, -0.0052, 0.5141) -- (1.7987, -0.0065, 0.5641) -- (1.7489, -0.0065, 0.5659) -- cycle;
\fill[blue!15.0, opacity=0.5] (1.7489, -0.0065, 0.5659) -- (1.7987, -0.0065, 0.5641) -- (1.7984, -0.0080, 0.6141) -- (1.7487, -0.0080, 0.6159) -- cycle;
\fill[blue!15.0, opacity=0.5] (1.7487, -0.0080, 0.6159) -- (1.7984, -0.0080, 0.6141) -- (1.7981, -0.0097, 0.6641) -- (1.7484, -0.0097, 0.6659) -- cycle;
\fill[blue!15.0, opacity=0.5] (1.7484, -0.0097, 0.6659) -- (1.7981, -0.0097, 0.6641) -- (1.7977, -0.0115, 0.7141) -- (1.7481, -0.0115, 0.7159) -- cycle;
\fill[blue!15.0, opacity=0.5] (1.7481, -0.0115, 0.7159) -- (1.7977, -0.0115, 0.7141) -- (1.7973, -0.0134, 0.7641) -- (1.7478, -0.0134, 0.7659) -- cycle;
\fill[blue!15.0, opacity=0.5] (1.7478, -0.0134, 0.7659) -- (1.7973, -0.0134, 0.7641) -- (1.7969, -0.0154, 0.8141) -- (1.7474, -0.0154, 0.8159) -- cycle;
\fill[blue!15.1, opacity=0.5] (1.7474, -0.0154, 0.8159) -- (1.7969, -0.0154, 0.8141) -- (1.7965, -0.0176, 0.8641) -- (1.7471, -0.0176, 0.8659) -- cycle;
\fill[blue!15.1, opacity=0.5] (1.7471, -0.0176, 0.8659) -- (1.7965, -0.0176, 0.8641) -- (1.7960, -0.0199, 0.9141) -- (1.7467, -0.0199, 0.9159) -- cycle;
\fill[blue!15.2, opacity=0.5] (1.7467, -0.0199, 0.9159) -- (1.7960, -0.0199, 0.9141) -- (1.7956, -0.0222, 0.9641) -- (1.7463, -0.0222, 0.9659) -- cycle;
\fill[blue!15.3, opacity=0.5] (1.7463, -0.0222, 0.9659) -- (1.7956, -0.0222, 0.9641) -- (1.7951, -0.0247, 1.0141) -- (1.7459, -0.0247, 1.0159) -- cycle;
\fill[blue!15.5, opacity=0.5] (1.7459, -0.0247, 1.0159) -- (1.7951, -0.0247, 1.0141) -- (1.7945, -0.0273, 1.0641) -- (1.7454, -0.0273, 1.0659) -- cycle;
\fill[blue!15.8, opacity=0.5] (1.7454, -0.0273, 1.0659) -- (1.7945, -0.0273, 1.0641) -- (1.7940, -0.0300, 1.1141) -- (1.7450, -0.0300, 1.1159) -- cycle;
\fill[blue!16.1, opacity=0.5] (1.7450, -0.0300, 1.1159) -- (1.7940, -0.0300, 1.1141) -- (1.7934, -0.0328, 1.1641) -- (1.7445, -0.0328, 1.1659) -- cycle;
\fill[blue!16.6, opacity=0.5] (1.7445, -0.0328, 1.1659) -- (1.7934, -0.0328, 1.1641) -- (1.7929, -0.0356, 1.2141) -- (1.7441, -0.0356, 1.2159) -- cycle;
\fill[blue!17.3, opacity=0.5] (1.7441, -0.0356, 1.2159) -- (1.7929, -0.0356, 1.2141) -- (1.7923, -0.0385, 1.2641) -- (1.7436, -0.0385, 1.2659) -- cycle;
\fill[blue!18.0, opacity=0.5] (1.7436, -0.0385, 1.2659) -- (1.7923, -0.0385, 1.2641) -- (1.7917, -0.0415, 1.3141) -- (1.7431, -0.0415, 1.3159) -- cycle;
\fill[blue!19.0, opacity=0.5] (1.7431, -0.0415, 1.3159) -- (1.7917, -0.0415, 1.3141) -- (1.7911, -0.0445, 1.3641) -- (1.7426, -0.0445, 1.3659) -- cycle;
\fill[blue!20.2, opacity=0.5] (1.7426, -0.0445, 1.3659) -- (1.7911, -0.0445, 1.3641) -- (1.7905, -0.0475, 1.4141) -- (1.7421, -0.0475, 1.4159) -- cycle;
\fill[blue!21.5, opacity=0.5] (1.7421, -0.0475, 1.4159) -- (1.7905, -0.0475, 1.4141) -- (1.7899, -0.0506, 1.4641) -- (1.7416, -0.0506, 1.4659) -- cycle;
\fill[blue!23.1, opacity=0.5] (1.7416, -0.0506, 1.4659) -- (1.7899, -0.0506, 1.4641) -- (1.7893, -0.0537, 1.5141) -- (1.7410, -0.0537, 1.5159) -- cycle;
\fill[blue!24.8, opacity=0.5] (1.7410, -0.0537, 1.5159) -- (1.7893, -0.0537, 1.5141) -- (1.7886, -0.0569, 1.5641) -- (1.7405, -0.0569, 1.5659) -- cycle;
\fill[blue!26.7, opacity=0.5] (1.7405, -0.0569, 1.5659) -- (1.7886, -0.0569, 1.5641) -- (1.7880, -0.0600, 1.6141) -- (1.7400, -0.0600, 1.6159) -- cycle;
\fill[blue!28.8, opacity=0.5] (1.7400, -0.0600, 1.6159) -- (1.7880, -0.0600, 1.6141) -- (1.7874, -0.0631, 1.6641) -- (1.7395, -0.0631, 1.6659) -- cycle;
\fill[blue!31.0, opacity=0.5] (1.7395, -0.0631, 1.6659) -- (1.7874, -0.0631, 1.6641) -- (1.7867, -0.0663, 1.7141) -- (1.7390, -0.0663, 1.7159) -- cycle;
\fill[blue!33.4, opacity=0.5] (1.7390, -0.0663, 1.7159) -- (1.7867, -0.0663, 1.7141) -- (1.7861, -0.0694, 1.7641) -- (1.7384, -0.0694, 1.7659) -- cycle;
\fill[blue!35.8, opacity=0.5] (1.7384, -0.0694, 1.7659) -- (1.7861, -0.0694, 1.7641) -- (1.7855, -0.0725, 1.8141) -- (1.7379, -0.0725, 1.8159) -- cycle;
\fill[blue!38.3, opacity=0.5] (1.7379, -0.0725, 1.8159) -- (1.7855, -0.0725, 1.8141) -- (1.7849, -0.0755, 1.8641) -- (1.7374, -0.0755, 1.8659) -- cycle;
\fill[blue!40.9, opacity=0.5] (1.7374, -0.0755, 1.8659) -- (1.7849, -0.0755, 1.8641) -- (1.7843, -0.0785, 1.9141) -- (1.7369, -0.0785, 1.9159) -- cycle;
\fill[blue!43.4, opacity=0.5] (1.7369, -0.0785, 1.9159) -- (1.7843, -0.0785, 1.9141) -- (1.7837, -0.0815, 1.9641) -- (1.7364, -0.0815, 1.9659) -- cycle;
\fill[blue!46.0, opacity=0.5] (1.7364, -0.0815, 1.9659) -- (1.7837, -0.0815, 1.9641) -- (1.7831, -0.0844, 2.0141) -- (1.7359, -0.0844, 2.0159) -- cycle;
\fill[blue!48.4, opacity=0.5] (1.7359, -0.0844, 2.0159) -- (1.7831, -0.0844, 2.0141) -- (1.7826, -0.0872, 2.0641) -- (1.7355, -0.0872, 2.0659) -- cycle;
\fill[blue!50.8, opacity=0.5] (1.7355, -0.0872, 2.0659) -- (1.7826, -0.0872, 2.0641) -- (1.7820, -0.0900, 2.1141) -- (1.7350, -0.0900, 2.1159) -- cycle;
\fill[blue!53.1, opacity=0.5] (1.7350, -0.0900, 2.1159) -- (1.7820, -0.0900, 2.1141) -- (1.7815, -0.0927, 2.1641) -- (1.7346, -0.0927, 2.1659) -- cycle;
\fill[blue!55.2, opacity=0.5] (1.7346, -0.0927, 2.1659) -- (1.7815, -0.0927, 2.1641) -- (1.7809, -0.0953, 2.2141) -- (1.7341, -0.0953, 2.2159) -- cycle;
\fill[blue!57.2, opacity=0.5] (1.7341, -0.0953, 2.2159) -- (1.7809, -0.0953, 2.2141) -- (1.7804, -0.0978, 2.2641) -- (1.7337, -0.0978, 2.2659) -- cycle;
\fill[blue!59.1, opacity=0.5] (1.7337, -0.0978, 2.2659) -- (1.7804, -0.0978, 2.2641) -- (1.7800, -0.1001, 2.3141) -- (1.7333, -0.1001, 2.3159) -- cycle;
\fill[blue!60.7, opacity=0.5] (1.7333, -0.1001, 2.3159) -- (1.7800, -0.1001, 2.3141) -- (1.7795, -0.1024, 2.3641) -- (1.7329, -0.1024, 2.3659) -- cycle;
\fill[blue!62.2, opacity=0.5] (1.7329, -0.1024, 2.3659) -- (1.7795, -0.1024, 2.3641) -- (1.7791, -0.1046, 2.4141) -- (1.7326, -0.1046, 2.4159) -- cycle;
\fill[blue!63.5, opacity=0.5] (1.7326, -0.1046, 2.4159) -- (1.7791, -0.1046, 2.4141) -- (1.7787, -0.1066, 2.4641) -- (1.7322, -0.1066, 2.4659) -- cycle;
\fill[blue!64.5, opacity=0.5] (1.7322, -0.1066, 2.4659) -- (1.7787, -0.1066, 2.4641) -- (1.7783, -0.1085, 2.5141) -- (1.7319, -0.1085, 2.5159) -- cycle;
\fill[blue!65.4, opacity=0.5] (1.7319, -0.1085, 2.5159) -- (1.7783, -0.1085, 2.5141) -- (1.7779, -0.1103, 2.5641) -- (1.7316, -0.1103, 2.5659) -- cycle;
\fill[blue!66.1, opacity=0.5] (1.7316, -0.1103, 2.5659) -- (1.7779, -0.1103, 2.5641) -- (1.7776, -0.1120, 2.6141) -- (1.7313, -0.1120, 2.6159) -- cycle;
\fill[blue!66.6, opacity=0.5] (1.7313, -0.1120, 2.6159) -- (1.7776, -0.1120, 2.6141) -- (1.7773, -0.1135, 2.6641) -- (1.7311, -0.1135, 2.6659) -- cycle;
\fill[blue!66.9, opacity=0.5] (1.7311, -0.1135, 2.6659) -- (1.7773, -0.1135, 2.6641) -- (1.7770, -0.1148, 2.7141) -- (1.7309, -0.1148, 2.7159) -- cycle;
\fill[blue!67.1, opacity=0.5] (1.7309, -0.1148, 2.7159) -- (1.7770, -0.1148, 2.7141) -- (1.7768, -0.1160, 2.7641) -- (1.7307, -0.1160, 2.7659) -- cycle;
\fill[blue!67.0, opacity=0.5] (1.7307, -0.1160, 2.7659) -- (1.7768, -0.1160, 2.7641) -- (1.7766, -0.1171, 2.8141) -- (1.7305, -0.1171, 2.8159) -- cycle;
\fill[blue!66.8, opacity=0.5] (1.7305, -0.1171, 2.8159) -- (1.7766, -0.1171, 2.8141) -- (1.7764, -0.1180, 2.8641) -- (1.7303, -0.1180, 2.8659) -- cycle;
\fill[blue!66.5, opacity=0.5] (1.7303, -0.1180, 2.8659) -- (1.7764, -0.1180, 2.8641) -- (1.7763, -0.1187, 2.9141) -- (1.7302, -0.1187, 2.9159) -- cycle;
\fill[blue!66.0, opacity=0.5] (1.7302, -0.1187, 2.9159) -- (1.7763, -0.1187, 2.9141) -- (1.7761, -0.1193, 2.9641) -- (1.7301, -0.1193, 2.9659) -- cycle;
\fill[blue!65.3, opacity=0.5] (1.7301, -0.1193, 2.9659) -- (1.7761, -0.1193, 2.9641) -- (1.7761, -0.1197, 3.0141) -- (1.7301, -0.1197, 3.0159) -- cycle;
\fill[blue!64.6, opacity=0.5] (1.7301, -0.1197, 3.0159) -- (1.7761, -0.1197, 3.0141) -- (1.7760, -0.1199, 3.0641) -- (1.7300, -0.1199, 3.0659) -- cycle;
\fill[blue!63.7, opacity=0.5] (1.7300, -0.1199, 3.0659) -- (1.7760, -0.1199, 3.0641) -- (1.7760, -0.1200, 3.1141) -- (1.7300, -0.1200, 3.1159) -- cycle;
\fill[blue!15.0, opacity=0.5] (1.8000, -0.0000, 0.1141) -- (1.8500, -0.0000, 0.1120) -- (1.8500, -0.0001, 0.1620) -- (1.8000, -0.0001, 0.1641) -- cycle;
\fill[blue!15.0, opacity=0.5] (1.8000, -0.0001, 0.1641) -- (1.8500, -0.0001, 0.1620) -- (1.8499, -0.0003, 0.2120) -- (1.7999, -0.0003, 0.2141) -- cycle;
\fill[blue!15.0, opacity=0.5] (1.7999, -0.0003, 0.2141) -- (1.8499, -0.0003, 0.2120) -- (1.8498, -0.0007, 0.2620) -- (1.7999, -0.0007, 0.2641) -- cycle;
\fill[blue!15.0, opacity=0.5] (1.7999, -0.0007, 0.2641) -- (1.8498, -0.0007, 0.2620) -- (1.8497, -0.0013, 0.3120) -- (1.7997, -0.0013, 0.3141) -- cycle;
\fill[blue!15.0, opacity=0.5] (1.7997, -0.0013, 0.3141) -- (1.8497, -0.0013, 0.3120) -- (1.8495, -0.0020, 0.3620) -- (1.7996, -0.0020, 0.3641) -- cycle;
\fill[blue!15.0, opacity=0.5] (1.7996, -0.0020, 0.3641) -- (1.8495, -0.0020, 0.3620) -- (1.8493, -0.0029, 0.4120) -- (1.7994, -0.0029, 0.4141) -- cycle;
\fill[blue!15.0, opacity=0.5] (1.7994, -0.0029, 0.4141) -- (1.8493, -0.0029, 0.4120) -- (1.8491, -0.0040, 0.4620) -- (1.7992, -0.0040, 0.4641) -- cycle;
\fill[blue!15.0, opacity=0.5] (1.7992, -0.0040, 0.4641) -- (1.8491, -0.0040, 0.4620) -- (1.8488, -0.0052, 0.5120) -- (1.7990, -0.0052, 0.5141) -- cycle;
\fill[blue!15.0, opacity=0.5] (1.7990, -0.0052, 0.5141) -- (1.8488, -0.0052, 0.5120) -- (1.8485, -0.0065, 0.5620) -- (1.7987, -0.0065, 0.5641) -- cycle;
\fill[blue!15.0, opacity=0.5] (1.7987, -0.0065, 0.5641) -- (1.8485, -0.0065, 0.5620) -- (1.8481, -0.0080, 0.6120) -- (1.7984, -0.0080, 0.6141) -- cycle;
\fill[blue!15.0, opacity=0.5] (1.7984, -0.0080, 0.6141) -- (1.8481, -0.0080, 0.6120) -- (1.8477, -0.0097, 0.6620) -- (1.7981, -0.0097, 0.6641) -- cycle;
\fill[blue!15.0, opacity=0.5] (1.7981, -0.0097, 0.6641) -- (1.8477, -0.0097, 0.6620) -- (1.8473, -0.0115, 0.7120) -- (1.7977, -0.0115, 0.7141) -- cycle;
\fill[blue!15.0, opacity=0.5] (1.7977, -0.0115, 0.7141) -- (1.8473, -0.0115, 0.7120) -- (1.8469, -0.0134, 0.7620) -- (1.7973, -0.0134, 0.7641) -- cycle;
\fill[blue!15.0, opacity=0.5] (1.7973, -0.0134, 0.7641) -- (1.8469, -0.0134, 0.7620) -- (1.8464, -0.0154, 0.8120) -- (1.7969, -0.0154, 0.8141) -- cycle;
\fill[blue!15.1, opacity=0.5] (1.7969, -0.0154, 0.8141) -- (1.8464, -0.0154, 0.8120) -- (1.8459, -0.0176, 0.8620) -- (1.7965, -0.0176, 0.8641) -- cycle;
\fill[blue!15.1, opacity=0.5] (1.7965, -0.0176, 0.8641) -- (1.8459, -0.0176, 0.8620) -- (1.8454, -0.0199, 0.9120) -- (1.7960, -0.0199, 0.9141) -- cycle;
\fill[blue!15.2, opacity=0.5] (1.7960, -0.0199, 0.9141) -- (1.8454, -0.0199, 0.9120) -- (1.8448, -0.0222, 0.9620) -- (1.7956, -0.0222, 0.9641) -- cycle;
\fill[blue!15.4, opacity=0.5] (1.7956, -0.0222, 0.9641) -- (1.8448, -0.0222, 0.9620) -- (1.8442, -0.0247, 1.0120) -- (1.7951, -0.0247, 1.0141) -- cycle;
\fill[blue!15.6, opacity=0.5] (1.7951, -0.0247, 1.0141) -- (1.8442, -0.0247, 1.0120) -- (1.8436, -0.0273, 1.0620) -- (1.7945, -0.0273, 1.0641) -- cycle;
\fill[blue!15.9, opacity=0.5] (1.7945, -0.0273, 1.0641) -- (1.8436, -0.0273, 1.0620) -- (1.8430, -0.0300, 1.1120) -- (1.7940, -0.0300, 1.1141) -- cycle;
\fill[blue!16.3, opacity=0.5] (1.7940, -0.0300, 1.1141) -- (1.8430, -0.0300, 1.1120) -- (1.8424, -0.0328, 1.1620) -- (1.7934, -0.0328, 1.1641) -- cycle;
\fill[blue!16.9, opacity=0.5] (1.7934, -0.0328, 1.1641) -- (1.8424, -0.0328, 1.1620) -- (1.8417, -0.0356, 1.2120) -- (1.7929, -0.0356, 1.2141) -- cycle;
\fill[blue!17.6, opacity=0.5] (1.7929, -0.0356, 1.2141) -- (1.8417, -0.0356, 1.2120) -- (1.8410, -0.0385, 1.2620) -- (1.7923, -0.0385, 1.2641) -- cycle;
\fill[blue!18.5, opacity=0.5] (1.7923, -0.0385, 1.2641) -- (1.8410, -0.0385, 1.2620) -- (1.8403, -0.0415, 1.3120) -- (1.7917, -0.0415, 1.3141) -- cycle;
\fill[blue!19.5, opacity=0.5] (1.7917, -0.0415, 1.3141) -- (1.8403, -0.0415, 1.3120) -- (1.8396, -0.0445, 1.3620) -- (1.7911, -0.0445, 1.3641) -- cycle;
\fill[blue!20.8, opacity=0.5] (1.7911, -0.0445, 1.3641) -- (1.8396, -0.0445, 1.3620) -- (1.8389, -0.0475, 1.4120) -- (1.7905, -0.0475, 1.4141) -- cycle;
\fill[blue!22.3, opacity=0.5] (1.7905, -0.0475, 1.4141) -- (1.8389, -0.0475, 1.4120) -- (1.8382, -0.0506, 1.4620) -- (1.7899, -0.0506, 1.4641) -- cycle;
\fill[blue!23.9, opacity=0.5] (1.7899, -0.0506, 1.4641) -- (1.8382, -0.0506, 1.4620) -- (1.8375, -0.0537, 1.5120) -- (1.7893, -0.0537, 1.5141) -- cycle;
\fill[blue!25.8, opacity=0.5] (1.7893, -0.0537, 1.5141) -- (1.8375, -0.0537, 1.5120) -- (1.8367, -0.0569, 1.5620) -- (1.7886, -0.0569, 1.5641) -- cycle;
\fill[blue!27.8, opacity=0.5] (1.7886, -0.0569, 1.5641) -- (1.8367, -0.0569, 1.5620) -- (1.8360, -0.0600, 1.6120) -- (1.7880, -0.0600, 1.6141) -- cycle;
\fill[blue!30.0, opacity=0.5] (1.7880, -0.0600, 1.6141) -- (1.8360, -0.0600, 1.6120) -- (1.8353, -0.0631, 1.6620) -- (1.7874, -0.0631, 1.6641) -- cycle;
\fill[blue!32.4, opacity=0.5] (1.7874, -0.0631, 1.6641) -- (1.8353, -0.0631, 1.6620) -- (1.8345, -0.0663, 1.7120) -- (1.7867, -0.0663, 1.7141) -- cycle;
\fill[blue!34.8, opacity=0.5] (1.7867, -0.0663, 1.7141) -- (1.8345, -0.0663, 1.7120) -- (1.8338, -0.0694, 1.7620) -- (1.7861, -0.0694, 1.7641) -- cycle;
\fill[blue!37.4, opacity=0.5] (1.7861, -0.0694, 1.7641) -- (1.8338, -0.0694, 1.7620) -- (1.8331, -0.0725, 1.8120) -- (1.7855, -0.0725, 1.8141) -- cycle;
\fill[blue!39.9, opacity=0.5] (1.7855, -0.0725, 1.8141) -- (1.8331, -0.0725, 1.8120) -- (1.8324, -0.0755, 1.8620) -- (1.7849, -0.0755, 1.8641) -- cycle;
\fill[blue!42.6, opacity=0.5] (1.7849, -0.0755, 1.8641) -- (1.8324, -0.0755, 1.8620) -- (1.8317, -0.0785, 1.9120) -- (1.7843, -0.0785, 1.9141) -- cycle;
\fill[blue!45.1, opacity=0.5] (1.7843, -0.0785, 1.9141) -- (1.8317, -0.0785, 1.9120) -- (1.8310, -0.0815, 1.9620) -- (1.7837, -0.0815, 1.9641) -- cycle;
\fill[blue!47.7, opacity=0.5] (1.7837, -0.0815, 1.9641) -- (1.8310, -0.0815, 1.9620) -- (1.8303, -0.0844, 2.0120) -- (1.7831, -0.0844, 2.0141) -- cycle;
\fill[blue!50.2, opacity=0.5] (1.7831, -0.0844, 2.0141) -- (1.8303, -0.0844, 2.0120) -- (1.8296, -0.0872, 2.0620) -- (1.7826, -0.0872, 2.0641) -- cycle;
\fill[blue!52.5, opacity=0.5] (1.7826, -0.0872, 2.0641) -- (1.8296, -0.0872, 2.0620) -- (1.8290, -0.0900, 2.1120) -- (1.7820, -0.0900, 2.1141) -- cycle;
\fill[blue!54.8, opacity=0.5] (1.7820, -0.0900, 2.1141) -- (1.8290, -0.0900, 2.1120) -- (1.8284, -0.0927, 2.1620) -- (1.7815, -0.0927, 2.1641) -- cycle;
\fill[blue!56.9, opacity=0.5] (1.7815, -0.0927, 2.1641) -- (1.8284, -0.0927, 2.1620) -- (1.8278, -0.0953, 2.2120) -- (1.7809, -0.0953, 2.2141) -- cycle;
\fill[blue!58.8, opacity=0.5] (1.7809, -0.0953, 2.2141) -- (1.8278, -0.0953, 2.2120) -- (1.8272, -0.0978, 2.2620) -- (1.7804, -0.0978, 2.2641) -- cycle;
\fill[blue!60.6, opacity=0.5] (1.7804, -0.0978, 2.2641) -- (1.8272, -0.0978, 2.2620) -- (1.8266, -0.1001, 2.3120) -- (1.7800, -0.1001, 2.3141) -- cycle;
\fill[blue!62.2, opacity=0.5] (1.7800, -0.1001, 2.3141) -- (1.8266, -0.1001, 2.3120) -- (1.8261, -0.1024, 2.3620) -- (1.7795, -0.1024, 2.3641) -- cycle;
\fill[blue!63.6, opacity=0.5] (1.7795, -0.1024, 2.3641) -- (1.8261, -0.1024, 2.3620) -- (1.8256, -0.1046, 2.4120) -- (1.7791, -0.1046, 2.4141) -- cycle;
\fill[blue!64.8, opacity=0.5] (1.7791, -0.1046, 2.4141) -- (1.8256, -0.1046, 2.4120) -- (1.8251, -0.1066, 2.4620) -- (1.7787, -0.1066, 2.4641) -- cycle;
\fill[blue!65.8, opacity=0.5] (1.7787, -0.1066, 2.4641) -- (1.8251, -0.1066, 2.4620) -- (1.8247, -0.1085, 2.5120) -- (1.7783, -0.1085, 2.5141) -- cycle;
\fill[blue!66.6, opacity=0.5] (1.7783, -0.1085, 2.5141) -- (1.8247, -0.1085, 2.5120) -- (1.8243, -0.1103, 2.5620) -- (1.7779, -0.1103, 2.5641) -- cycle;
\fill[blue!67.2, opacity=0.5] (1.7779, -0.1103, 2.5641) -- (1.8243, -0.1103, 2.5620) -- (1.8239, -0.1120, 2.6120) -- (1.7776, -0.1120, 2.6141) -- cycle;
\fill[blue!67.6, opacity=0.5] (1.7776, -0.1120, 2.6141) -- (1.8239, -0.1120, 2.6120) -- (1.8235, -0.1135, 2.6620) -- (1.7773, -0.1135, 2.6641) -- cycle;
\fill[blue!67.8, opacity=0.5] (1.7773, -0.1135, 2.6641) -- (1.8235, -0.1135, 2.6620) -- (1.8232, -0.1148, 2.7120) -- (1.7770, -0.1148, 2.7141) -- cycle;
\fill[blue!67.8, opacity=0.5] (1.7770, -0.1148, 2.7141) -- (1.8232, -0.1148, 2.7120) -- (1.8229, -0.1160, 2.7620) -- (1.7768, -0.1160, 2.7641) -- cycle;
\fill[blue!67.7, opacity=0.5] (1.7768, -0.1160, 2.7641) -- (1.8229, -0.1160, 2.7620) -- (1.8227, -0.1171, 2.8120) -- (1.7766, -0.1171, 2.8141) -- cycle;
\fill[blue!67.4, opacity=0.5] (1.7766, -0.1171, 2.8141) -- (1.8227, -0.1171, 2.8120) -- (1.8225, -0.1180, 2.8620) -- (1.7764, -0.1180, 2.8641) -- cycle;
\fill[blue!67.0, opacity=0.5] (1.7764, -0.1180, 2.8641) -- (1.8225, -0.1180, 2.8620) -- (1.8223, -0.1187, 2.9120) -- (1.7763, -0.1187, 2.9141) -- cycle;
\fill[blue!66.4, opacity=0.5] (1.7763, -0.1187, 2.9141) -- (1.8223, -0.1187, 2.9120) -- (1.8222, -0.1193, 2.9620) -- (1.7761, -0.1193, 2.9641) -- cycle;
\fill[blue!65.7, opacity=0.5] (1.7761, -0.1193, 2.9641) -- (1.8222, -0.1193, 2.9620) -- (1.8221, -0.1197, 3.0120) -- (1.7761, -0.1197, 3.0141) -- cycle;
\fill[blue!64.8, opacity=0.5] (1.7761, -0.1197, 3.0141) -- (1.8221, -0.1197, 3.0120) -- (1.8220, -0.1199, 3.0620) -- (1.7760, -0.1199, 3.0641) -- cycle;
\fill[blue!63.9, opacity=0.5] (1.7760, -0.1199, 3.0641) -- (1.8220, -0.1199, 3.0620) -- (1.8220, -0.1200, 3.1120) -- (1.7760, -0.1200, 3.1141) -- cycle;
\fill[blue!15.0, opacity=0.5] (1.8500, -0.0000, 0.1120) -- (1.9000, -0.0000, 0.1096) -- (1.9000, -0.0001, 0.1596) -- (1.8500, -0.0001, 0.1620) -- cycle;
\fill[blue!15.0, opacity=0.5] (1.8500, -0.0001, 0.1620) -- (1.9000, -0.0001, 0.1596) -- (1.8999, -0.0003, 0.2096) -- (1.8499, -0.0003, 0.2120) -- cycle;
\fill[blue!15.0, opacity=0.5] (1.8499, -0.0003, 0.2120) -- (1.8999, -0.0003, 0.2096) -- (1.8998, -0.0007, 0.2596) -- (1.8498, -0.0007, 0.2620) -- cycle;
\fill[blue!15.0, opacity=0.5] (1.8498, -0.0007, 0.2620) -- (1.8998, -0.0007, 0.2596) -- (1.8997, -0.0013, 0.3096) -- (1.8497, -0.0013, 0.3120) -- cycle;
\fill[blue!15.0, opacity=0.5] (1.8497, -0.0013, 0.3120) -- (1.8997, -0.0013, 0.3096) -- (1.8995, -0.0020, 0.3596) -- (1.8495, -0.0020, 0.3620) -- cycle;
\fill[blue!15.0, opacity=0.5] (1.8495, -0.0020, 0.3620) -- (1.8995, -0.0020, 0.3596) -- (1.8992, -0.0029, 0.4096) -- (1.8493, -0.0029, 0.4120) -- cycle;
\fill[blue!15.0, opacity=0.5] (1.8493, -0.0029, 0.4120) -- (1.8992, -0.0029, 0.4096) -- (1.8989, -0.0040, 0.4596) -- (1.8491, -0.0040, 0.4620) -- cycle;
\fill[blue!15.0, opacity=0.5] (1.8491, -0.0040, 0.4620) -- (1.8989, -0.0040, 0.4596) -- (1.8986, -0.0052, 0.5096) -- (1.8488, -0.0052, 0.5120) -- cycle;
\fill[blue!15.0, opacity=0.5] (1.8488, -0.0052, 0.5120) -- (1.8986, -0.0052, 0.5096) -- (1.8983, -0.0065, 0.5596) -- (1.8485, -0.0065, 0.5620) -- cycle;
\fill[blue!15.0, opacity=0.5] (1.8485, -0.0065, 0.5620) -- (1.8983, -0.0065, 0.5596) -- (1.8979, -0.0080, 0.6096) -- (1.8481, -0.0080, 0.6120) -- cycle;
\fill[blue!15.0, opacity=0.5] (1.8481, -0.0080, 0.6120) -- (1.8979, -0.0080, 0.6096) -- (1.8974, -0.0097, 0.6596) -- (1.8477, -0.0097, 0.6620) -- cycle;
\fill[blue!15.0, opacity=0.5] (1.8477, -0.0097, 0.6620) -- (1.8974, -0.0097, 0.6596) -- (1.8969, -0.0115, 0.7096) -- (1.8473, -0.0115, 0.7120) -- cycle;
\fill[blue!15.0, opacity=0.5] (1.8473, -0.0115, 0.7120) -- (1.8969, -0.0115, 0.7096) -- (1.8964, -0.0134, 0.7596) -- (1.8469, -0.0134, 0.7620) -- cycle;
\fill[blue!15.0, opacity=0.5] (1.8469, -0.0134, 0.7620) -- (1.8964, -0.0134, 0.7596) -- (1.8959, -0.0154, 0.8096) -- (1.8464, -0.0154, 0.8120) -- cycle;
\fill[blue!15.1, opacity=0.5] (1.8464, -0.0154, 0.8120) -- (1.8959, -0.0154, 0.8096) -- (1.8953, -0.0176, 0.8596) -- (1.8459, -0.0176, 0.8620) -- cycle;
\fill[blue!15.1, opacity=0.5] (1.8459, -0.0176, 0.8620) -- (1.8953, -0.0176, 0.8596) -- (1.8947, -0.0199, 0.9096) -- (1.8454, -0.0199, 0.9120) -- cycle;
\fill[blue!15.2, opacity=0.5] (1.8454, -0.0199, 0.9120) -- (1.8947, -0.0199, 0.9096) -- (1.8941, -0.0222, 0.9596) -- (1.8448, -0.0222, 0.9620) -- cycle;
\fill[blue!15.3, opacity=0.5] (1.8448, -0.0222, 0.9620) -- (1.8941, -0.0222, 0.9596) -- (1.8934, -0.0247, 1.0096) -- (1.8442, -0.0247, 1.0120) -- cycle;
\fill[blue!15.5, opacity=0.5] (1.8442, -0.0247, 1.0120) -- (1.8934, -0.0247, 1.0096) -- (1.8927, -0.0273, 1.0596) -- (1.8436, -0.0273, 1.0620) -- cycle;
\fill[blue!15.8, opacity=0.5] (1.8436, -0.0273, 1.0620) -- (1.8927, -0.0273, 1.0596) -- (1.8920, -0.0300, 1.1096) -- (1.8430, -0.0300, 1.1120) -- cycle;
\fill[blue!16.2, opacity=0.5] (1.8430, -0.0300, 1.1120) -- (1.8920, -0.0300, 1.1096) -- (1.8913, -0.0328, 1.1596) -- (1.8424, -0.0328, 1.1620) -- cycle;
\fill[blue!16.7, opacity=0.5] (1.8424, -0.0328, 1.1620) -- (1.8913, -0.0328, 1.1596) -- (1.8905, -0.0356, 1.2096) -- (1.8417, -0.0356, 1.2120) -- cycle;
\fill[blue!17.3, opacity=0.5] (1.8417, -0.0356, 1.2120) -- (1.8905, -0.0356, 1.2096) -- (1.8897, -0.0385, 1.2596) -- (1.8410, -0.0385, 1.2620) -- cycle;
\fill[blue!18.1, opacity=0.5] (1.8410, -0.0385, 1.2620) -- (1.8897, -0.0385, 1.2596) -- (1.8889, -0.0415, 1.3096) -- (1.8403, -0.0415, 1.3120) -- cycle;
\fill[blue!19.1, opacity=0.5] (1.8403, -0.0415, 1.3120) -- (1.8889, -0.0415, 1.3096) -- (1.8881, -0.0445, 1.3596) -- (1.8396, -0.0445, 1.3620) -- cycle;
\fill[blue!20.2, opacity=0.5] (1.8396, -0.0445, 1.3620) -- (1.8881, -0.0445, 1.3596) -- (1.8873, -0.0475, 1.4096) -- (1.8389, -0.0475, 1.4120) -- cycle;
\fill[blue!21.6, opacity=0.5] (1.8389, -0.0475, 1.4120) -- (1.8873, -0.0475, 1.4096) -- (1.8865, -0.0506, 1.4596) -- (1.8382, -0.0506, 1.4620) -- cycle;
\fill[blue!23.1, opacity=0.5] (1.8382, -0.0506, 1.4620) -- (1.8865, -0.0506, 1.4596) -- (1.8857, -0.0537, 1.5096) -- (1.8375, -0.0537, 1.5120) -- cycle;
\fill[blue!24.9, opacity=0.5] (1.8375, -0.0537, 1.5120) -- (1.8857, -0.0537, 1.5096) -- (1.8848, -0.0569, 1.5596) -- (1.8367, -0.0569, 1.5620) -- cycle;
\fill[blue!26.8, opacity=0.5] (1.8367, -0.0569, 1.5620) -- (1.8848, -0.0569, 1.5596) -- (1.8840, -0.0600, 1.6096) -- (1.8360, -0.0600, 1.6120) -- cycle;
\fill[blue!28.9, opacity=0.5] (1.8360, -0.0600, 1.6120) -- (1.8840, -0.0600, 1.6096) -- (1.8832, -0.0631, 1.6596) -- (1.8353, -0.0631, 1.6620) -- cycle;
\fill[blue!31.2, opacity=0.5] (1.8353, -0.0631, 1.6620) -- (1.8832, -0.0631, 1.6596) -- (1.8823, -0.0663, 1.7096) -- (1.8345, -0.0663, 1.7120) -- cycle;
\fill[blue!33.5, opacity=0.5] (1.8345, -0.0663, 1.7120) -- (1.8823, -0.0663, 1.7096) -- (1.8815, -0.0694, 1.7596) -- (1.8338, -0.0694, 1.7620) -- cycle;
\fill[blue!36.0, opacity=0.5] (1.8338, -0.0694, 1.7620) -- (1.8815, -0.0694, 1.7596) -- (1.8807, -0.0725, 1.8096) -- (1.8331, -0.0725, 1.8120) -- cycle;
\fill[blue!38.5, opacity=0.5] (1.8331, -0.0725, 1.8120) -- (1.8807, -0.0725, 1.8096) -- (1.8799, -0.0755, 1.8596) -- (1.8324, -0.0755, 1.8620) -- cycle;
\fill[blue!41.1, opacity=0.5] (1.8324, -0.0755, 1.8620) -- (1.8799, -0.0755, 1.8596) -- (1.8791, -0.0785, 1.9096) -- (1.8317, -0.0785, 1.9120) -- cycle;
\fill[blue!43.6, opacity=0.5] (1.8317, -0.0785, 1.9120) -- (1.8791, -0.0785, 1.9096) -- (1.8783, -0.0815, 1.9596) -- (1.8310, -0.0815, 1.9620) -- cycle;
\fill[blue!46.1, opacity=0.5] (1.8310, -0.0815, 1.9620) -- (1.8783, -0.0815, 1.9596) -- (1.8775, -0.0844, 2.0096) -- (1.8303, -0.0844, 2.0120) -- cycle;
\fill[blue!48.6, opacity=0.5] (1.8303, -0.0844, 2.0120) -- (1.8775, -0.0844, 2.0096) -- (1.8767, -0.0872, 2.0596) -- (1.8296, -0.0872, 2.0620) -- cycle;
\fill[blue!51.0, opacity=0.5] (1.8296, -0.0872, 2.0620) -- (1.8767, -0.0872, 2.0596) -- (1.8760, -0.0900, 2.1096) -- (1.8290, -0.0900, 2.1120) -- cycle;
\fill[blue!53.3, opacity=0.5] (1.8290, -0.0900, 2.1120) -- (1.8760, -0.0900, 2.1096) -- (1.8753, -0.0927, 2.1596) -- (1.8284, -0.0927, 2.1620) -- cycle;
\fill[blue!55.4, opacity=0.5] (1.8284, -0.0927, 2.1620) -- (1.8753, -0.0927, 2.1596) -- (1.8746, -0.0953, 2.2096) -- (1.8278, -0.0953, 2.2120) -- cycle;
\fill[blue!57.4, opacity=0.5] (1.8278, -0.0953, 2.2120) -- (1.8746, -0.0953, 2.2096) -- (1.8739, -0.0978, 2.2596) -- (1.8272, -0.0978, 2.2620) -- cycle;
\fill[blue!59.2, opacity=0.5] (1.8272, -0.0978, 2.2620) -- (1.8739, -0.0978, 2.2596) -- (1.8733, -0.1001, 2.3096) -- (1.8266, -0.1001, 2.3120) -- cycle;
\fill[blue!60.9, opacity=0.5] (1.8266, -0.1001, 2.3120) -- (1.8733, -0.1001, 2.3096) -- (1.8727, -0.1024, 2.3596) -- (1.8261, -0.1024, 2.3620) -- cycle;
\fill[blue!62.3, opacity=0.5] (1.8261, -0.1024, 2.3620) -- (1.8727, -0.1024, 2.3596) -- (1.8721, -0.1046, 2.4096) -- (1.8256, -0.1046, 2.4120) -- cycle;
\fill[blue!63.6, opacity=0.5] (1.8256, -0.1046, 2.4120) -- (1.8721, -0.1046, 2.4096) -- (1.8716, -0.1066, 2.4596) -- (1.8251, -0.1066, 2.4620) -- cycle;
\fill[blue!64.7, opacity=0.5] (1.8251, -0.1066, 2.4620) -- (1.8716, -0.1066, 2.4596) -- (1.8711, -0.1085, 2.5096) -- (1.8247, -0.1085, 2.5120) -- cycle;
\fill[blue!65.5, opacity=0.5] (1.8247, -0.1085, 2.5120) -- (1.8711, -0.1085, 2.5096) -- (1.8706, -0.1103, 2.5596) -- (1.8243, -0.1103, 2.5620) -- cycle;
\fill[blue!66.2, opacity=0.5] (1.8243, -0.1103, 2.5620) -- (1.8706, -0.1103, 2.5596) -- (1.8701, -0.1120, 2.6096) -- (1.8239, -0.1120, 2.6120) -- cycle;
\fill[blue!66.7, opacity=0.5] (1.8239, -0.1120, 2.6120) -- (1.8701, -0.1120, 2.6096) -- (1.8697, -0.1135, 2.6596) -- (1.8235, -0.1135, 2.6620) -- cycle;
\fill[blue!67.0, opacity=0.5] (1.8235, -0.1135, 2.6620) -- (1.8697, -0.1135, 2.6596) -- (1.8694, -0.1148, 2.7096) -- (1.8232, -0.1148, 2.7120) -- cycle;
\fill[blue!67.1, opacity=0.5] (1.8232, -0.1148, 2.7120) -- (1.8694, -0.1148, 2.7096) -- (1.8691, -0.1160, 2.7596) -- (1.8229, -0.1160, 2.7620) -- cycle;
\fill[blue!67.1, opacity=0.5] (1.8229, -0.1160, 2.7620) -- (1.8691, -0.1160, 2.7596) -- (1.8688, -0.1171, 2.8096) -- (1.8227, -0.1171, 2.8120) -- cycle;
\fill[blue!66.9, opacity=0.5] (1.8227, -0.1171, 2.8120) -- (1.8688, -0.1171, 2.8096) -- (1.8685, -0.1180, 2.8596) -- (1.8225, -0.1180, 2.8620) -- cycle;
\fill[blue!66.5, opacity=0.5] (1.8225, -0.1180, 2.8620) -- (1.8685, -0.1180, 2.8596) -- (1.8683, -0.1187, 2.9096) -- (1.8223, -0.1187, 2.9120) -- cycle;
\fill[blue!66.0, opacity=0.5] (1.8223, -0.1187, 2.9120) -- (1.8683, -0.1187, 2.9096) -- (1.8682, -0.1193, 2.9596) -- (1.8222, -0.1193, 2.9620) -- cycle;
\fill[blue!65.4, opacity=0.5] (1.8222, -0.1193, 2.9620) -- (1.8682, -0.1193, 2.9596) -- (1.8681, -0.1197, 3.0096) -- (1.8221, -0.1197, 3.0120) -- cycle;
\fill[blue!64.6, opacity=0.5] (1.8221, -0.1197, 3.0120) -- (1.8681, -0.1197, 3.0096) -- (1.8680, -0.1199, 3.0596) -- (1.8220, -0.1199, 3.0620) -- cycle;
\fill[blue!63.7, opacity=0.5] (1.8220, -0.1199, 3.0620) -- (1.8680, -0.1199, 3.0596) -- (1.8680, -0.1200, 3.1096) -- (1.8220, -0.1200, 3.1120) -- cycle;
\fill[blue!15.0, opacity=0.5] (1.9000, -0.0000, 0.1096) -- (1.9500, -0.0000, 0.1069) -- (1.9500, -0.0001, 0.1569) -- (1.9000, -0.0001, 0.1596) -- cycle;
\fill[blue!15.0, opacity=0.5] (1.9000, -0.0001, 0.1596) -- (1.9500, -0.0001, 0.1569) -- (1.9499, -0.0003, 0.2069) -- (1.8999, -0.0003, 0.2096) -- cycle;
\fill[blue!15.0, opacity=0.5] (1.8999, -0.0003, 0.2096) -- (1.9499, -0.0003, 0.2069) -- (1.9498, -0.0007, 0.2569) -- (1.8998, -0.0007, 0.2596) -- cycle;
\fill[blue!15.0, opacity=0.5] (1.8998, -0.0007, 0.2596) -- (1.9498, -0.0007, 0.2569) -- (1.9496, -0.0013, 0.3069) -- (1.8997, -0.0013, 0.3096) -- cycle;
\fill[blue!15.0, opacity=0.5] (1.8997, -0.0013, 0.3096) -- (1.9496, -0.0013, 0.3069) -- (1.9494, -0.0020, 0.3569) -- (1.8995, -0.0020, 0.3596) -- cycle;
\fill[blue!15.0, opacity=0.5] (1.8995, -0.0020, 0.3596) -- (1.9494, -0.0020, 0.3569) -- (1.9491, -0.0029, 0.4069) -- (1.8992, -0.0029, 0.4096) -- cycle;
\fill[blue!15.0, opacity=0.5] (1.8992, -0.0029, 0.4096) -- (1.9491, -0.0029, 0.4069) -- (1.9488, -0.0040, 0.4569) -- (1.8989, -0.0040, 0.4596) -- cycle;
\fill[blue!15.0, opacity=0.5] (1.8989, -0.0040, 0.4596) -- (1.9488, -0.0040, 0.4569) -- (1.9484, -0.0052, 0.5069) -- (1.8986, -0.0052, 0.5096) -- cycle;
\fill[blue!15.0, opacity=0.5] (1.8986, -0.0052, 0.5096) -- (1.9484, -0.0052, 0.5069) -- (1.9480, -0.0065, 0.5569) -- (1.8983, -0.0065, 0.5596) -- cycle;
\fill[blue!15.0, opacity=0.5] (1.8983, -0.0065, 0.5596) -- (1.9480, -0.0065, 0.5569) -- (1.9476, -0.0080, 0.6069) -- (1.8979, -0.0080, 0.6096) -- cycle;
\fill[blue!15.0, opacity=0.5] (1.8979, -0.0080, 0.6096) -- (1.9476, -0.0080, 0.6069) -- (1.9471, -0.0097, 0.6569) -- (1.8974, -0.0097, 0.6596) -- cycle;
\fill[blue!15.0, opacity=0.5] (1.8974, -0.0097, 0.6596) -- (1.9471, -0.0097, 0.6569) -- (1.9466, -0.0115, 0.7069) -- (1.8969, -0.0115, 0.7096) -- cycle;
\fill[blue!15.0, opacity=0.5] (1.8969, -0.0115, 0.7096) -- (1.9466, -0.0115, 0.7069) -- (1.9460, -0.0134, 0.7569) -- (1.8964, -0.0134, 0.7596) -- cycle;
\fill[blue!15.0, opacity=0.5] (1.8964, -0.0134, 0.7596) -- (1.9460, -0.0134, 0.7569) -- (1.9454, -0.0154, 0.8069) -- (1.8959, -0.0154, 0.8096) -- cycle;
\fill[blue!15.0, opacity=0.5] (1.8959, -0.0154, 0.8096) -- (1.9454, -0.0154, 0.8069) -- (1.9447, -0.0176, 0.8569) -- (1.8953, -0.0176, 0.8596) -- cycle;
\fill[blue!15.1, opacity=0.5] (1.8953, -0.0176, 0.8596) -- (1.9447, -0.0176, 0.8569) -- (1.9440, -0.0199, 0.9069) -- (1.8947, -0.0199, 0.9096) -- cycle;
\fill[blue!15.1, opacity=0.5] (1.8947, -0.0199, 0.9096) -- (1.9440, -0.0199, 0.9069) -- (1.9433, -0.0222, 0.9569) -- (1.8941, -0.0222, 0.9596) -- cycle;
\fill[blue!15.2, opacity=0.5] (1.8941, -0.0222, 0.9596) -- (1.9433, -0.0222, 0.9569) -- (1.9426, -0.0247, 1.0069) -- (1.8934, -0.0247, 1.0096) -- cycle;
\fill[blue!15.3, opacity=0.5] (1.8934, -0.0247, 1.0096) -- (1.9426, -0.0247, 1.0069) -- (1.9418, -0.0273, 1.0569) -- (1.8927, -0.0273, 1.0596) -- cycle;
\fill[blue!15.4, opacity=0.5] (1.8927, -0.0273, 1.0596) -- (1.9418, -0.0273, 1.0569) -- (1.9410, -0.0300, 1.1069) -- (1.8920, -0.0300, 1.1096) -- cycle;
\fill[blue!15.7, opacity=0.5] (1.8920, -0.0300, 1.1096) -- (1.9410, -0.0300, 1.1069) -- (1.9402, -0.0328, 1.1569) -- (1.8913, -0.0328, 1.1596) -- cycle;
\fill[blue!16.0, opacity=0.5] (1.8913, -0.0328, 1.1596) -- (1.9402, -0.0328, 1.1569) -- (1.9393, -0.0356, 1.2069) -- (1.8905, -0.0356, 1.2096) -- cycle;
\fill[blue!16.4, opacity=0.5] (1.8905, -0.0356, 1.2096) -- (1.9393, -0.0356, 1.2069) -- (1.9385, -0.0385, 1.2569) -- (1.8897, -0.0385, 1.2596) -- cycle;
\fill[blue!16.9, opacity=0.5] (1.8897, -0.0385, 1.2596) -- (1.9385, -0.0385, 1.2569) -- (1.9376, -0.0415, 1.3069) -- (1.8889, -0.0415, 1.3096) -- cycle;
\fill[blue!17.6, opacity=0.5] (1.8889, -0.0415, 1.3096) -- (1.9376, -0.0415, 1.3069) -- (1.9367, -0.0445, 1.3569) -- (1.8881, -0.0445, 1.3596) -- cycle;
\fill[blue!18.4, opacity=0.5] (1.8881, -0.0445, 1.3596) -- (1.9367, -0.0445, 1.3569) -- (1.9357, -0.0475, 1.4069) -- (1.8873, -0.0475, 1.4096) -- cycle;
\fill[blue!19.4, opacity=0.5] (1.8873, -0.0475, 1.4096) -- (1.9357, -0.0475, 1.4069) -- (1.9348, -0.0506, 1.4569) -- (1.8865, -0.0506, 1.4596) -- cycle;
\fill[blue!20.6, opacity=0.5] (1.8865, -0.0506, 1.4596) -- (1.9348, -0.0506, 1.4569) -- (1.9339, -0.0537, 1.5069) -- (1.8857, -0.0537, 1.5096) -- cycle;
\fill[blue!22.0, opacity=0.5] (1.8857, -0.0537, 1.5096) -- (1.9339, -0.0537, 1.5069) -- (1.9329, -0.0569, 1.5569) -- (1.8848, -0.0569, 1.5596) -- cycle;
\fill[blue!23.5, opacity=0.5] (1.8848, -0.0569, 1.5596) -- (1.9329, -0.0569, 1.5569) -- (1.9320, -0.0600, 1.6069) -- (1.8840, -0.0600, 1.6096) -- cycle;
\fill[blue!25.2, opacity=0.5] (1.8840, -0.0600, 1.6096) -- (1.9320, -0.0600, 1.6069) -- (1.9311, -0.0631, 1.6569) -- (1.8832, -0.0631, 1.6596) -- cycle;
\fill[blue!27.1, opacity=0.5] (1.8832, -0.0631, 1.6596) -- (1.9311, -0.0631, 1.6569) -- (1.9301, -0.0663, 1.7069) -- (1.8823, -0.0663, 1.7096) -- cycle;
\fill[blue!29.1, opacity=0.5] (1.8823, -0.0663, 1.7096) -- (1.9301, -0.0663, 1.7069) -- (1.9292, -0.0694, 1.7569) -- (1.8815, -0.0694, 1.7596) -- cycle;
\fill[blue!31.3, opacity=0.5] (1.8815, -0.0694, 1.7596) -- (1.9292, -0.0694, 1.7569) -- (1.9283, -0.0725, 1.8069) -- (1.8807, -0.0725, 1.8096) -- cycle;
\fill[blue!33.5, opacity=0.5] (1.8807, -0.0725, 1.8096) -- (1.9283, -0.0725, 1.8069) -- (1.9273, -0.0755, 1.8569) -- (1.8799, -0.0755, 1.8596) -- cycle;
\fill[blue!35.9, opacity=0.5] (1.8799, -0.0755, 1.8596) -- (1.9273, -0.0755, 1.8569) -- (1.9264, -0.0785, 1.9069) -- (1.8791, -0.0785, 1.9096) -- cycle;
\fill[blue!38.2, opacity=0.5] (1.8791, -0.0785, 1.9096) -- (1.9264, -0.0785, 1.9069) -- (1.9255, -0.0815, 1.9569) -- (1.8783, -0.0815, 1.9596) -- cycle;
\fill[blue!40.6, opacity=0.5] (1.8783, -0.0815, 1.9596) -- (1.9255, -0.0815, 1.9569) -- (1.9247, -0.0844, 2.0069) -- (1.8775, -0.0844, 2.0096) -- cycle;
\fill[blue!43.0, opacity=0.5] (1.8775, -0.0844, 2.0096) -- (1.9247, -0.0844, 2.0069) -- (1.9238, -0.0872, 2.0569) -- (1.8767, -0.0872, 2.0596) -- cycle;
\fill[blue!45.4, opacity=0.5] (1.8767, -0.0872, 2.0596) -- (1.9238, -0.0872, 2.0569) -- (1.9230, -0.0900, 2.1069) -- (1.8760, -0.0900, 2.1096) -- cycle;
\fill[blue!47.6, opacity=0.5] (1.8760, -0.0900, 2.1096) -- (1.9230, -0.0900, 2.1069) -- (1.9222, -0.0927, 2.1569) -- (1.8753, -0.0927, 2.1596) -- cycle;
\fill[blue!49.8, opacity=0.5] (1.8753, -0.0927, 2.1596) -- (1.9222, -0.0927, 2.1569) -- (1.9214, -0.0953, 2.2069) -- (1.8746, -0.0953, 2.2096) -- cycle;
\fill[blue!51.9, opacity=0.5] (1.8746, -0.0953, 2.2096) -- (1.9214, -0.0953, 2.2069) -- (1.9207, -0.0978, 2.2569) -- (1.8739, -0.0978, 2.2596) -- cycle;
\fill[blue!53.9, opacity=0.5] (1.8739, -0.0978, 2.2596) -- (1.9207, -0.0978, 2.2569) -- (1.9200, -0.1001, 2.3069) -- (1.8733, -0.1001, 2.3096) -- cycle;
\fill[blue!55.7, opacity=0.5] (1.8733, -0.1001, 2.3096) -- (1.9200, -0.1001, 2.3069) -- (1.9193, -0.1024, 2.3569) -- (1.8727, -0.1024, 2.3596) -- cycle;
\fill[blue!57.4, opacity=0.5] (1.8727, -0.1024, 2.3596) -- (1.9193, -0.1024, 2.3569) -- (1.9186, -0.1046, 2.4069) -- (1.8721, -0.1046, 2.4096) -- cycle;
\fill[blue!58.9, opacity=0.5] (1.8721, -0.1046, 2.4096) -- (1.9186, -0.1046, 2.4069) -- (1.9180, -0.1066, 2.4569) -- (1.8716, -0.1066, 2.4596) -- cycle;
\fill[blue!60.3, opacity=0.5] (1.8716, -0.1066, 2.4596) -- (1.9180, -0.1066, 2.4569) -- (1.9174, -0.1085, 2.5069) -- (1.8711, -0.1085, 2.5096) -- cycle;
\fill[blue!61.4, opacity=0.5] (1.8711, -0.1085, 2.5096) -- (1.9174, -0.1085, 2.5069) -- (1.9169, -0.1103, 2.5569) -- (1.8706, -0.1103, 2.5596) -- cycle;
\fill[blue!62.4, opacity=0.5] (1.8706, -0.1103, 2.5596) -- (1.9169, -0.1103, 2.5569) -- (1.9164, -0.1120, 2.6069) -- (1.8701, -0.1120, 2.6096) -- cycle;
\fill[blue!63.2, opacity=0.5] (1.8701, -0.1120, 2.6096) -- (1.9164, -0.1120, 2.6069) -- (1.9160, -0.1135, 2.6569) -- (1.8697, -0.1135, 2.6596) -- cycle;
\fill[blue!63.8, opacity=0.5] (1.8697, -0.1135, 2.6596) -- (1.9160, -0.1135, 2.6569) -- (1.9156, -0.1148, 2.7069) -- (1.8694, -0.1148, 2.7096) -- cycle;
\fill[blue!64.2, opacity=0.5] (1.8694, -0.1148, 2.7096) -- (1.9156, -0.1148, 2.7069) -- (1.9152, -0.1160, 2.7569) -- (1.8691, -0.1160, 2.7596) -- cycle;
\fill[blue!64.5, opacity=0.5] (1.8691, -0.1160, 2.7596) -- (1.9152, -0.1160, 2.7569) -- (1.9149, -0.1171, 2.8069) -- (1.8688, -0.1171, 2.8096) -- cycle;
\fill[blue!64.6, opacity=0.5] (1.8688, -0.1171, 2.8096) -- (1.9149, -0.1171, 2.8069) -- (1.9146, -0.1180, 2.8569) -- (1.8685, -0.1180, 2.8596) -- cycle;
\fill[blue!64.5, opacity=0.5] (1.8685, -0.1180, 2.8596) -- (1.9146, -0.1180, 2.8569) -- (1.9144, -0.1187, 2.9069) -- (1.8683, -0.1187, 2.9096) -- cycle;
\fill[blue!64.3, opacity=0.5] (1.8683, -0.1187, 2.9096) -- (1.9144, -0.1187, 2.9069) -- (1.9142, -0.1193, 2.9569) -- (1.8682, -0.1193, 2.9596) -- cycle;
\fill[blue!63.9, opacity=0.5] (1.8682, -0.1193, 2.9596) -- (1.9142, -0.1193, 2.9569) -- (1.9141, -0.1197, 3.0069) -- (1.8681, -0.1197, 3.0096) -- cycle;
\fill[blue!63.4, opacity=0.5] (1.8681, -0.1197, 3.0096) -- (1.9141, -0.1197, 3.0069) -- (1.9140, -0.1199, 3.0569) -- (1.8680, -0.1199, 3.0596) -- cycle;
\fill[blue!62.8, opacity=0.5] (1.8680, -0.1199, 3.0596) -- (1.9140, -0.1199, 3.0569) -- (1.9140, -0.1200, 3.1069) -- (1.8680, -0.1200, 3.1096) -- cycle;
\fill[blue!15.0, opacity=0.5] (1.9500, -0.0000, 0.1069) -- (2.0000, -0.0000, 0.1039) -- (2.0000, -0.0001, 0.1539) -- (1.9500, -0.0001, 0.1569) -- cycle;
\fill[blue!15.0, opacity=0.5] (1.9500, -0.0001, 0.1569) -- (2.0000, -0.0001, 0.1539) -- (1.9999, -0.0003, 0.2039) -- (1.9499, -0.0003, 0.2069) -- cycle;
\fill[blue!15.0, opacity=0.5] (1.9499, -0.0003, 0.2069) -- (1.9999, -0.0003, 0.2039) -- (1.9998, -0.0007, 0.2539) -- (1.9498, -0.0007, 0.2569) -- cycle;
\fill[blue!15.0, opacity=0.5] (1.9498, -0.0007, 0.2569) -- (1.9998, -0.0007, 0.2539) -- (1.9996, -0.0013, 0.3039) -- (1.9496, -0.0013, 0.3069) -- cycle;
\fill[blue!15.0, opacity=0.5] (1.9496, -0.0013, 0.3069) -- (1.9996, -0.0013, 0.3039) -- (1.9993, -0.0020, 0.3539) -- (1.9494, -0.0020, 0.3569) -- cycle;
\fill[blue!15.0, opacity=0.5] (1.9494, -0.0020, 0.3569) -- (1.9993, -0.0020, 0.3539) -- (1.9990, -0.0029, 0.4039) -- (1.9491, -0.0029, 0.4069) -- cycle;
\fill[blue!15.0, opacity=0.5] (1.9491, -0.0029, 0.4069) -- (1.9990, -0.0029, 0.4039) -- (1.9987, -0.0040, 0.4539) -- (1.9488, -0.0040, 0.4569) -- cycle;
\fill[blue!15.0, opacity=0.5] (1.9488, -0.0040, 0.4569) -- (1.9987, -0.0040, 0.4539) -- (1.9983, -0.0052, 0.5039) -- (1.9484, -0.0052, 0.5069) -- cycle;
\fill[blue!15.0, opacity=0.5] (1.9484, -0.0052, 0.5069) -- (1.9983, -0.0052, 0.5039) -- (1.9978, -0.0065, 0.5539) -- (1.9480, -0.0065, 0.5569) -- cycle;
\fill[blue!15.0, opacity=0.5] (1.9480, -0.0065, 0.5569) -- (1.9978, -0.0065, 0.5539) -- (1.9973, -0.0080, 0.6039) -- (1.9476, -0.0080, 0.6069) -- cycle;
\fill[blue!15.0, opacity=0.5] (1.9476, -0.0080, 0.6069) -- (1.9973, -0.0080, 0.6039) -- (1.9968, -0.0097, 0.6539) -- (1.9471, -0.0097, 0.6569) -- cycle;
\fill[blue!15.0, opacity=0.5] (1.9471, -0.0097, 0.6569) -- (1.9968, -0.0097, 0.6539) -- (1.9962, -0.0115, 0.7039) -- (1.9466, -0.0115, 0.7069) -- cycle;
\fill[blue!15.0, opacity=0.5] (1.9466, -0.0115, 0.7069) -- (1.9962, -0.0115, 0.7039) -- (1.9955, -0.0134, 0.7539) -- (1.9460, -0.0134, 0.7569) -- cycle;
\fill[blue!15.0, opacity=0.5] (1.9460, -0.0134, 0.7569) -- (1.9955, -0.0134, 0.7539) -- (1.9949, -0.0154, 0.8039) -- (1.9454, -0.0154, 0.8069) -- cycle;
\fill[blue!15.0, opacity=0.5] (1.9454, -0.0154, 0.8069) -- (1.9949, -0.0154, 0.8039) -- (1.9941, -0.0176, 0.8539) -- (1.9447, -0.0176, 0.8569) -- cycle;
\fill[blue!15.0, opacity=0.5] (1.9447, -0.0176, 0.8569) -- (1.9941, -0.0176, 0.8539) -- (1.9934, -0.0199, 0.9039) -- (1.9440, -0.0199, 0.9069) -- cycle;
\fill[blue!15.0, opacity=0.5] (1.9440, -0.0199, 0.9069) -- (1.9934, -0.0199, 0.9039) -- (1.9926, -0.0222, 0.9539) -- (1.9433, -0.0222, 0.9569) -- cycle;
\fill[blue!15.0, opacity=0.5] (1.9433, -0.0222, 0.9569) -- (1.9926, -0.0222, 0.9539) -- (1.9918, -0.0247, 1.0039) -- (1.9426, -0.0247, 1.0069) -- cycle;
\fill[blue!15.1, opacity=0.5] (1.9426, -0.0247, 1.0069) -- (1.9918, -0.0247, 1.0039) -- (1.9909, -0.0273, 1.0539) -- (1.9418, -0.0273, 1.0569) -- cycle;
\fill[blue!15.1, opacity=0.5] (1.9418, -0.0273, 1.0569) -- (1.9909, -0.0273, 1.0539) -- (1.9900, -0.0300, 1.1039) -- (1.9410, -0.0300, 1.1069) -- cycle;
\fill[blue!15.2, opacity=0.5] (1.9410, -0.0300, 1.1069) -- (1.9900, -0.0300, 1.1039) -- (1.9891, -0.0328, 1.1539) -- (1.9402, -0.0328, 1.1569) -- cycle;
\fill[blue!15.3, opacity=0.5] (1.9402, -0.0328, 1.1569) -- (1.9891, -0.0328, 1.1539) -- (1.9881, -0.0356, 1.2039) -- (1.9393, -0.0356, 1.2069) -- cycle;
\fill[blue!15.5, opacity=0.5] (1.9393, -0.0356, 1.2069) -- (1.9881, -0.0356, 1.2039) -- (1.9872, -0.0385, 1.2539) -- (1.9385, -0.0385, 1.2569) -- cycle;
\fill[blue!15.7, opacity=0.5] (1.9385, -0.0385, 1.2569) -- (1.9872, -0.0385, 1.2539) -- (1.9862, -0.0415, 1.3039) -- (1.9376, -0.0415, 1.3069) -- cycle;
\fill[blue!16.1, opacity=0.5] (1.9376, -0.0415, 1.3069) -- (1.9862, -0.0415, 1.3039) -- (1.9852, -0.0445, 1.3539) -- (1.9367, -0.0445, 1.3569) -- cycle;
\fill[blue!16.5, opacity=0.5] (1.9367, -0.0445, 1.3569) -- (1.9852, -0.0445, 1.3539) -- (1.9842, -0.0475, 1.4039) -- (1.9357, -0.0475, 1.4069) -- cycle;
\fill[blue!17.0, opacity=0.5] (1.9357, -0.0475, 1.4069) -- (1.9842, -0.0475, 1.4039) -- (1.9831, -0.0506, 1.4539) -- (1.9348, -0.0506, 1.4569) -- cycle;
\fill[blue!17.6, opacity=0.5] (1.9348, -0.0506, 1.4569) -- (1.9831, -0.0506, 1.4539) -- (1.9821, -0.0537, 1.5039) -- (1.9339, -0.0537, 1.5069) -- cycle;
\fill[blue!18.4, opacity=0.5] (1.9339, -0.0537, 1.5069) -- (1.9821, -0.0537, 1.5039) -- (1.9810, -0.0569, 1.5539) -- (1.9329, -0.0569, 1.5569) -- cycle;
\fill[blue!19.4, opacity=0.5] (1.9329, -0.0569, 1.5569) -- (1.9810, -0.0569, 1.5539) -- (1.9800, -0.0600, 1.6039) -- (1.9320, -0.0600, 1.6069) -- cycle;
\fill[blue!20.4, opacity=0.5] (1.9320, -0.0600, 1.6069) -- (1.9800, -0.0600, 1.6039) -- (1.9790, -0.0631, 1.6539) -- (1.9311, -0.0631, 1.6569) -- cycle;
\fill[blue!21.7, opacity=0.5] (1.9311, -0.0631, 1.6569) -- (1.9790, -0.0631, 1.6539) -- (1.9779, -0.0663, 1.7039) -- (1.9301, -0.0663, 1.7069) -- cycle;
\fill[blue!23.1, opacity=0.5] (1.9301, -0.0663, 1.7069) -- (1.9779, -0.0663, 1.7039) -- (1.9769, -0.0694, 1.7539) -- (1.9292, -0.0694, 1.7569) -- cycle;
\fill[blue!24.6, opacity=0.5] (1.9292, -0.0694, 1.7569) -- (1.9769, -0.0694, 1.7539) -- (1.9758, -0.0725, 1.8039) -- (1.9283, -0.0725, 1.8069) -- cycle;
\fill[blue!26.3, opacity=0.5] (1.9283, -0.0725, 1.8069) -- (1.9758, -0.0725, 1.8039) -- (1.9748, -0.0755, 1.8539) -- (1.9273, -0.0755, 1.8569) -- cycle;
\fill[blue!28.1, opacity=0.5] (1.9273, -0.0755, 1.8569) -- (1.9748, -0.0755, 1.8539) -- (1.9738, -0.0785, 1.9039) -- (1.9264, -0.0785, 1.9069) -- cycle;
\fill[blue!30.0, opacity=0.5] (1.9264, -0.0785, 1.9069) -- (1.9738, -0.0785, 1.9039) -- (1.9728, -0.0815, 1.9539) -- (1.9255, -0.0815, 1.9569) -- cycle;
\fill[blue!31.9, opacity=0.5] (1.9255, -0.0815, 1.9569) -- (1.9728, -0.0815, 1.9539) -- (1.9719, -0.0844, 2.0039) -- (1.9247, -0.0844, 2.0069) -- cycle;
\fill[blue!34.0, opacity=0.5] (1.9247, -0.0844, 2.0069) -- (1.9719, -0.0844, 2.0039) -- (1.9709, -0.0872, 2.0539) -- (1.9238, -0.0872, 2.0569) -- cycle;
\fill[blue!36.1, opacity=0.5] (1.9238, -0.0872, 2.0569) -- (1.9709, -0.0872, 2.0539) -- (1.9700, -0.0900, 2.1039) -- (1.9230, -0.0900, 2.1069) -- cycle;
\fill[blue!38.2, opacity=0.5] (1.9230, -0.0900, 2.1069) -- (1.9700, -0.0900, 2.1039) -- (1.9691, -0.0927, 2.1539) -- (1.9222, -0.0927, 2.1569) -- cycle;
\fill[blue!40.3, opacity=0.5] (1.9222, -0.0927, 2.1569) -- (1.9691, -0.0927, 2.1539) -- (1.9682, -0.0953, 2.2039) -- (1.9214, -0.0953, 2.2069) -- cycle;
\fill[blue!42.4, opacity=0.5] (1.9214, -0.0953, 2.2069) -- (1.9682, -0.0953, 2.2039) -- (1.9674, -0.0978, 2.2539) -- (1.9207, -0.0978, 2.2569) -- cycle;
\fill[blue!44.4, opacity=0.5] (1.9207, -0.0978, 2.2569) -- (1.9674, -0.0978, 2.2539) -- (1.9666, -0.1001, 2.3039) -- (1.9200, -0.1001, 2.3069) -- cycle;
\fill[blue!46.4, opacity=0.5] (1.9200, -0.1001, 2.3069) -- (1.9666, -0.1001, 2.3039) -- (1.9659, -0.1024, 2.3539) -- (1.9193, -0.1024, 2.3569) -- cycle;
\fill[blue!48.2, opacity=0.5] (1.9193, -0.1024, 2.3569) -- (1.9659, -0.1024, 2.3539) -- (1.9651, -0.1046, 2.4039) -- (1.9186, -0.1046, 2.4069) -- cycle;
\fill[blue!50.0, opacity=0.5] (1.9186, -0.1046, 2.4069) -- (1.9651, -0.1046, 2.4039) -- (1.9645, -0.1066, 2.4539) -- (1.9180, -0.1066, 2.4569) -- cycle;
\fill[blue!51.6, opacity=0.5] (1.9180, -0.1066, 2.4569) -- (1.9645, -0.1066, 2.4539) -- (1.9638, -0.1085, 2.5039) -- (1.9174, -0.1085, 2.5069) -- cycle;
\fill[blue!53.2, opacity=0.5] (1.9174, -0.1085, 2.5069) -- (1.9638, -0.1085, 2.5039) -- (1.9632, -0.1103, 2.5539) -- (1.9169, -0.1103, 2.5569) -- cycle;
\fill[blue!54.5, opacity=0.5] (1.9169, -0.1103, 2.5569) -- (1.9632, -0.1103, 2.5539) -- (1.9627, -0.1120, 2.6039) -- (1.9164, -0.1120, 2.6069) -- cycle;
\fill[blue!55.8, opacity=0.5] (1.9164, -0.1120, 2.6069) -- (1.9627, -0.1120, 2.6039) -- (1.9622, -0.1135, 2.6539) -- (1.9160, -0.1135, 2.6569) -- cycle;
\fill[blue!56.8, opacity=0.5] (1.9160, -0.1135, 2.6569) -- (1.9622, -0.1135, 2.6539) -- (1.9617, -0.1148, 2.7039) -- (1.9156, -0.1148, 2.7069) -- cycle;
\fill[blue!57.7, opacity=0.5] (1.9156, -0.1148, 2.7069) -- (1.9617, -0.1148, 2.7039) -- (1.9613, -0.1160, 2.7539) -- (1.9152, -0.1160, 2.7569) -- cycle;
\fill[blue!58.5, opacity=0.5] (1.9152, -0.1160, 2.7569) -- (1.9613, -0.1160, 2.7539) -- (1.9610, -0.1171, 2.8039) -- (1.9149, -0.1171, 2.8069) -- cycle;
\fill[blue!59.1, opacity=0.5] (1.9149, -0.1171, 2.8069) -- (1.9610, -0.1171, 2.8039) -- (1.9607, -0.1180, 2.8539) -- (1.9146, -0.1180, 2.8569) -- cycle;
\fill[blue!59.5, opacity=0.5] (1.9146, -0.1180, 2.8569) -- (1.9607, -0.1180, 2.8539) -- (1.9604, -0.1187, 2.9039) -- (1.9144, -0.1187, 2.9069) -- cycle;
\fill[blue!59.8, opacity=0.5] (1.9144, -0.1187, 2.9069) -- (1.9604, -0.1187, 2.9039) -- (1.9602, -0.1193, 2.9539) -- (1.9142, -0.1193, 2.9569) -- cycle;
\fill[blue!59.9, opacity=0.5] (1.9142, -0.1193, 2.9569) -- (1.9602, -0.1193, 2.9539) -- (1.9601, -0.1197, 3.0039) -- (1.9141, -0.1197, 3.0069) -- cycle;
\fill[blue!59.9, opacity=0.5] (1.9141, -0.1197, 3.0069) -- (1.9601, -0.1197, 3.0039) -- (1.9600, -0.1199, 3.0539) -- (1.9140, -0.1199, 3.0569) -- cycle;
\fill[blue!59.7, opacity=0.5] (1.9140, -0.1199, 3.0569) -- (1.9600, -0.1199, 3.0539) -- (1.9600, -0.1200, 3.1039) -- (1.9140, -0.1200, 3.1069) -- cycle;
\fill[blue!15.0, opacity=0.5] (2.0000, -0.0000, 0.1039) -- (2.0500, -0.0000, 0.1006) -- (2.0500, -0.0001, 0.1506) -- (2.0000, -0.0001, 0.1539) -- cycle;
\fill[blue!15.0, opacity=0.5] (2.0000, -0.0001, 0.1539) -- (2.0500, -0.0001, 0.1506) -- (2.0499, -0.0003, 0.2006) -- (1.9999, -0.0003, 0.2039) -- cycle;
\fill[blue!15.0, opacity=0.5] (1.9999, -0.0003, 0.2039) -- (2.0499, -0.0003, 0.2006) -- (2.0497, -0.0007, 0.2506) -- (1.9998, -0.0007, 0.2539) -- cycle;
\fill[blue!15.0, opacity=0.5] (1.9998, -0.0007, 0.2539) -- (2.0497, -0.0007, 0.2506) -- (2.0495, -0.0013, 0.3006) -- (1.9996, -0.0013, 0.3039) -- cycle;
\fill[blue!15.0, opacity=0.5] (1.9996, -0.0013, 0.3039) -- (2.0495, -0.0013, 0.3006) -- (2.0493, -0.0020, 0.3506) -- (1.9993, -0.0020, 0.3539) -- cycle;
\fill[blue!15.0, opacity=0.5] (1.9993, -0.0020, 0.3539) -- (2.0493, -0.0020, 0.3506) -- (2.0489, -0.0029, 0.4006) -- (1.9990, -0.0029, 0.4039) -- cycle;
\fill[blue!15.0, opacity=0.5] (1.9990, -0.0029, 0.4039) -- (2.0489, -0.0029, 0.4006) -- (2.0485, -0.0040, 0.4506) -- (1.9987, -0.0040, 0.4539) -- cycle;
\fill[blue!15.0, opacity=0.5] (1.9987, -0.0040, 0.4539) -- (2.0485, -0.0040, 0.4506) -- (2.0481, -0.0052, 0.5006) -- (1.9983, -0.0052, 0.5039) -- cycle;
\fill[blue!15.0, opacity=0.5] (1.9983, -0.0052, 0.5039) -- (2.0481, -0.0052, 0.5006) -- (2.0476, -0.0065, 0.5506) -- (1.9978, -0.0065, 0.5539) -- cycle;
\fill[blue!15.0, opacity=0.5] (1.9978, -0.0065, 0.5539) -- (2.0476, -0.0065, 0.5506) -- (2.0471, -0.0080, 0.6006) -- (1.9973, -0.0080, 0.6039) -- cycle;
\fill[blue!15.0, opacity=0.5] (1.9973, -0.0080, 0.6039) -- (2.0471, -0.0080, 0.6006) -- (2.0465, -0.0097, 0.6506) -- (1.9968, -0.0097, 0.6539) -- cycle;
\fill[blue!15.0, opacity=0.5] (1.9968, -0.0097, 0.6539) -- (2.0465, -0.0097, 0.6506) -- (2.0458, -0.0115, 0.7006) -- (1.9962, -0.0115, 0.7039) -- cycle;
\fill[blue!15.0, opacity=0.5] (1.9962, -0.0115, 0.7039) -- (2.0458, -0.0115, 0.7006) -- (2.0451, -0.0134, 0.7506) -- (1.9955, -0.0134, 0.7539) -- cycle;
\fill[blue!15.0, opacity=0.5] (1.9955, -0.0134, 0.7539) -- (2.0451, -0.0134, 0.7506) -- (2.0443, -0.0154, 0.8006) -- (1.9949, -0.0154, 0.8039) -- cycle;
\fill[blue!15.0, opacity=0.5] (1.9949, -0.0154, 0.8039) -- (2.0443, -0.0154, 0.8006) -- (2.0436, -0.0176, 0.8506) -- (1.9941, -0.0176, 0.8539) -- cycle;
\fill[blue!15.0, opacity=0.5] (1.9941, -0.0176, 0.8539) -- (2.0436, -0.0176, 0.8506) -- (2.0427, -0.0199, 0.9006) -- (1.9934, -0.0199, 0.9039) -- cycle;
\fill[blue!15.0, opacity=0.5] (1.9934, -0.0199, 0.9039) -- (2.0427, -0.0199, 0.9006) -- (2.0418, -0.0222, 0.9506) -- (1.9926, -0.0222, 0.9539) -- cycle;
\fill[blue!15.0, opacity=0.5] (1.9926, -0.0222, 0.9539) -- (2.0418, -0.0222, 0.9506) -- (2.0409, -0.0247, 1.0006) -- (1.9918, -0.0247, 1.0039) -- cycle;
\fill[blue!15.0, opacity=0.5] (1.9918, -0.0247, 1.0039) -- (2.0409, -0.0247, 1.0006) -- (2.0400, -0.0273, 1.0506) -- (1.9909, -0.0273, 1.0539) -- cycle;
\fill[blue!15.0, opacity=0.5] (1.9909, -0.0273, 1.0539) -- (2.0400, -0.0273, 1.0506) -- (2.0390, -0.0300, 1.1006) -- (1.9900, -0.0300, 1.1039) -- cycle;
\fill[blue!15.0, opacity=0.5] (1.9900, -0.0300, 1.1039) -- (2.0390, -0.0300, 1.1006) -- (2.0380, -0.0328, 1.1506) -- (1.9891, -0.0328, 1.1539) -- cycle;
\fill[blue!15.1, opacity=0.5] (1.9891, -0.0328, 1.1539) -- (2.0380, -0.0328, 1.1506) -- (2.0369, -0.0356, 1.2006) -- (1.9881, -0.0356, 1.2039) -- cycle;
\fill[blue!15.1, opacity=0.5] (1.9881, -0.0356, 1.2039) -- (2.0369, -0.0356, 1.2006) -- (2.0359, -0.0385, 1.2506) -- (1.9872, -0.0385, 1.2539) -- cycle;
\fill[blue!15.1, opacity=0.5] (1.9872, -0.0385, 1.2539) -- (2.0359, -0.0385, 1.2506) -- (2.0348, -0.0415, 1.3006) -- (1.9862, -0.0415, 1.3039) -- cycle;
\fill[blue!15.2, opacity=0.5] (1.9862, -0.0415, 1.3039) -- (2.0348, -0.0415, 1.3006) -- (2.0337, -0.0445, 1.3506) -- (1.9852, -0.0445, 1.3539) -- cycle;
\fill[blue!15.3, opacity=0.5] (1.9852, -0.0445, 1.3539) -- (2.0337, -0.0445, 1.3506) -- (2.0326, -0.0475, 1.4006) -- (1.9842, -0.0475, 1.4039) -- cycle;
\fill[blue!15.5, opacity=0.5] (1.9842, -0.0475, 1.4039) -- (2.0326, -0.0475, 1.4006) -- (2.0314, -0.0506, 1.4506) -- (1.9831, -0.0506, 1.4539) -- cycle;
\fill[blue!15.7, opacity=0.5] (1.9831, -0.0506, 1.4539) -- (2.0314, -0.0506, 1.4506) -- (2.0303, -0.0537, 1.5006) -- (1.9821, -0.0537, 1.5039) -- cycle;
\fill[blue!16.0, opacity=0.5] (1.9821, -0.0537, 1.5039) -- (2.0303, -0.0537, 1.5006) -- (2.0292, -0.0569, 1.5506) -- (1.9810, -0.0569, 1.5539) -- cycle;
\fill[blue!16.4, opacity=0.5] (1.9810, -0.0569, 1.5539) -- (2.0292, -0.0569, 1.5506) -- (2.0280, -0.0600, 1.6006) -- (1.9800, -0.0600, 1.6039) -- cycle;
\fill[blue!16.8, opacity=0.5] (1.9800, -0.0600, 1.6039) -- (2.0280, -0.0600, 1.6006) -- (2.0268, -0.0631, 1.6506) -- (1.9790, -0.0631, 1.6539) -- cycle;
\fill[blue!17.4, opacity=0.5] (1.9790, -0.0631, 1.6539) -- (2.0268, -0.0631, 1.6506) -- (2.0257, -0.0663, 1.7006) -- (1.9779, -0.0663, 1.7039) -- cycle;
\fill[blue!18.0, opacity=0.5] (1.9779, -0.0663, 1.7039) -- (2.0257, -0.0663, 1.7006) -- (2.0246, -0.0694, 1.7506) -- (1.9769, -0.0694, 1.7539) -- cycle;
\fill[blue!18.8, opacity=0.5] (1.9769, -0.0694, 1.7539) -- (2.0246, -0.0694, 1.7506) -- (2.0234, -0.0725, 1.8006) -- (1.9758, -0.0725, 1.8039) -- cycle;
\fill[blue!19.7, opacity=0.5] (1.9758, -0.0725, 1.8039) -- (2.0234, -0.0725, 1.8006) -- (2.0223, -0.0755, 1.8506) -- (1.9748, -0.0755, 1.8539) -- cycle;
\fill[blue!20.7, opacity=0.5] (1.9748, -0.0755, 1.8539) -- (2.0223, -0.0755, 1.8506) -- (2.0212, -0.0785, 1.9006) -- (1.9738, -0.0785, 1.9039) -- cycle;
\fill[blue!21.8, opacity=0.5] (1.9738, -0.0785, 1.9039) -- (2.0212, -0.0785, 1.9006) -- (2.0201, -0.0815, 1.9506) -- (1.9728, -0.0815, 1.9539) -- cycle;
\fill[blue!23.1, opacity=0.5] (1.9728, -0.0815, 1.9539) -- (2.0201, -0.0815, 1.9506) -- (2.0191, -0.0844, 2.0006) -- (1.9719, -0.0844, 2.0039) -- cycle;
\fill[blue!24.4, opacity=0.5] (1.9719, -0.0844, 2.0039) -- (2.0191, -0.0844, 2.0006) -- (2.0180, -0.0872, 2.0506) -- (1.9709, -0.0872, 2.0539) -- cycle;
\fill[blue!25.9, opacity=0.5] (1.9709, -0.0872, 2.0539) -- (2.0180, -0.0872, 2.0506) -- (2.0170, -0.0900, 2.1006) -- (1.9700, -0.0900, 2.1039) -- cycle;
\fill[blue!27.4, opacity=0.5] (1.9700, -0.0900, 2.1039) -- (2.0170, -0.0900, 2.1006) -- (2.0160, -0.0927, 2.1506) -- (1.9691, -0.0927, 2.1539) -- cycle;
\fill[blue!29.1, opacity=0.5] (1.9691, -0.0927, 2.1539) -- (2.0160, -0.0927, 2.1506) -- (2.0151, -0.0953, 2.2006) -- (1.9682, -0.0953, 2.2039) -- cycle;
\fill[blue!30.7, opacity=0.5] (1.9682, -0.0953, 2.2039) -- (2.0151, -0.0953, 2.2006) -- (2.0142, -0.0978, 2.2506) -- (1.9674, -0.0978, 2.2539) -- cycle;
\fill[blue!32.5, opacity=0.5] (1.9674, -0.0978, 2.2539) -- (2.0142, -0.0978, 2.2506) -- (2.0133, -0.1001, 2.3006) -- (1.9666, -0.1001, 2.3039) -- cycle;
\fill[blue!34.2, opacity=0.5] (1.9666, -0.1001, 2.3039) -- (2.0133, -0.1001, 2.3006) -- (2.0124, -0.1024, 2.3506) -- (1.9659, -0.1024, 2.3539) -- cycle;
\fill[blue!35.9, opacity=0.5] (1.9659, -0.1024, 2.3539) -- (2.0124, -0.1024, 2.3506) -- (2.0117, -0.1046, 2.4006) -- (1.9651, -0.1046, 2.4039) -- cycle;
\fill[blue!37.7, opacity=0.5] (1.9651, -0.1046, 2.4039) -- (2.0117, -0.1046, 2.4006) -- (2.0109, -0.1066, 2.4506) -- (1.9645, -0.1066, 2.4539) -- cycle;
\fill[blue!39.4, opacity=0.5] (1.9645, -0.1066, 2.4539) -- (2.0109, -0.1066, 2.4506) -- (2.0102, -0.1085, 2.5006) -- (1.9638, -0.1085, 2.5039) -- cycle;
\fill[blue!41.0, opacity=0.5] (1.9638, -0.1085, 2.5039) -- (2.0102, -0.1085, 2.5006) -- (2.0095, -0.1103, 2.5506) -- (1.9632, -0.1103, 2.5539) -- cycle;
\fill[blue!42.6, opacity=0.5] (1.9632, -0.1103, 2.5539) -- (2.0095, -0.1103, 2.5506) -- (2.0089, -0.1120, 2.6006) -- (1.9627, -0.1120, 2.6039) -- cycle;
\fill[blue!44.1, opacity=0.5] (1.9627, -0.1120, 2.6039) -- (2.0089, -0.1120, 2.6006) -- (2.0084, -0.1135, 2.6506) -- (1.9622, -0.1135, 2.6539) -- cycle;
\fill[blue!45.6, opacity=0.5] (1.9622, -0.1135, 2.6539) -- (2.0084, -0.1135, 2.6506) -- (2.0079, -0.1148, 2.7006) -- (1.9617, -0.1148, 2.7039) -- cycle;
\fill[blue!46.9, opacity=0.5] (1.9617, -0.1148, 2.7039) -- (2.0079, -0.1148, 2.7006) -- (2.0075, -0.1160, 2.7506) -- (1.9613, -0.1160, 2.7539) -- cycle;
\fill[blue!48.1, opacity=0.5] (1.9613, -0.1160, 2.7539) -- (2.0075, -0.1160, 2.7506) -- (2.0071, -0.1171, 2.8006) -- (1.9610, -0.1171, 2.8039) -- cycle;
\fill[blue!49.2, opacity=0.5] (1.9610, -0.1171, 2.8039) -- (2.0071, -0.1171, 2.8006) -- (2.0067, -0.1180, 2.8506) -- (1.9607, -0.1180, 2.8539) -- cycle;
\fill[blue!50.2, opacity=0.5] (1.9607, -0.1180, 2.8539) -- (2.0067, -0.1180, 2.8506) -- (2.0065, -0.1187, 2.9006) -- (1.9604, -0.1187, 2.9039) -- cycle;
\fill[blue!51.0, opacity=0.5] (1.9604, -0.1187, 2.9039) -- (2.0065, -0.1187, 2.9006) -- (2.0063, -0.1193, 2.9506) -- (1.9602, -0.1193, 2.9539) -- cycle;
\fill[blue!51.7, opacity=0.5] (1.9602, -0.1193, 2.9539) -- (2.0063, -0.1193, 2.9506) -- (2.0061, -0.1197, 3.0006) -- (1.9601, -0.1197, 3.0039) -- cycle;
\fill[blue!52.3, opacity=0.5] (1.9601, -0.1197, 3.0039) -- (2.0061, -0.1197, 3.0006) -- (2.0060, -0.1199, 3.0506) -- (1.9600, -0.1199, 3.0539) -- cycle;
\fill[blue!52.8, opacity=0.5] (1.9600, -0.1199, 3.0539) -- (2.0060, -0.1199, 3.0506) -- (2.0060, -0.1200, 3.1006) -- (1.9600, -0.1200, 3.1039) -- cycle;
\fill[blue!15.0, opacity=0.5] (2.0500, -0.0000, 0.1006) -- (2.1000, -0.0000, 0.0971) -- (2.1000, -0.0001, 0.1471) -- (2.0500, -0.0001, 0.1506) -- cycle;
\fill[blue!15.0, opacity=0.5] (2.0500, -0.0001, 0.1506) -- (2.1000, -0.0001, 0.1471) -- (2.0999, -0.0003, 0.1971) -- (2.0499, -0.0003, 0.2006) -- cycle;
\fill[blue!15.0, opacity=0.5] (2.0499, -0.0003, 0.2006) -- (2.0999, -0.0003, 0.1971) -- (2.0997, -0.0007, 0.2471) -- (2.0497, -0.0007, 0.2506) -- cycle;
\fill[blue!15.0, opacity=0.5] (2.0497, -0.0007, 0.2506) -- (2.0997, -0.0007, 0.2471) -- (2.0995, -0.0013, 0.2971) -- (2.0495, -0.0013, 0.3006) -- cycle;
\fill[blue!15.0, opacity=0.5] (2.0495, -0.0013, 0.3006) -- (2.0995, -0.0013, 0.2971) -- (2.0992, -0.0020, 0.3471) -- (2.0493, -0.0020, 0.3506) -- cycle;
\fill[blue!15.0, opacity=0.5] (2.0493, -0.0020, 0.3506) -- (2.0992, -0.0020, 0.3471) -- (2.0988, -0.0029, 0.3971) -- (2.0489, -0.0029, 0.4006) -- cycle;
\fill[blue!15.0, opacity=0.5] (2.0489, -0.0029, 0.4006) -- (2.0988, -0.0029, 0.3971) -- (2.0984, -0.0040, 0.4471) -- (2.0485, -0.0040, 0.4506) -- cycle;
\fill[blue!15.0, opacity=0.5] (2.0485, -0.0040, 0.4506) -- (2.0984, -0.0040, 0.4471) -- (2.0979, -0.0052, 0.4971) -- (2.0481, -0.0052, 0.5006) -- cycle;
\fill[blue!15.0, opacity=0.5] (2.0481, -0.0052, 0.5006) -- (2.0979, -0.0052, 0.4971) -- (2.0974, -0.0065, 0.5471) -- (2.0476, -0.0065, 0.5506) -- cycle;
\fill[blue!15.0, opacity=0.5] (2.0476, -0.0065, 0.5506) -- (2.0974, -0.0065, 0.5471) -- (2.0968, -0.0080, 0.5971) -- (2.0471, -0.0080, 0.6006) -- cycle;
\fill[blue!15.0, opacity=0.5] (2.0471, -0.0080, 0.6006) -- (2.0968, -0.0080, 0.5971) -- (2.0961, -0.0097, 0.6471) -- (2.0465, -0.0097, 0.6506) -- cycle;
\fill[blue!15.0, opacity=0.5] (2.0465, -0.0097, 0.6506) -- (2.0961, -0.0097, 0.6471) -- (2.0954, -0.0115, 0.6971) -- (2.0458, -0.0115, 0.7006) -- cycle;
\fill[blue!15.0, opacity=0.5] (2.0458, -0.0115, 0.7006) -- (2.0954, -0.0115, 0.6971) -- (2.0947, -0.0134, 0.7471) -- (2.0451, -0.0134, 0.7506) -- cycle;
\fill[blue!15.0, opacity=0.5] (2.0451, -0.0134, 0.7506) -- (2.0947, -0.0134, 0.7471) -- (2.0938, -0.0154, 0.7971) -- (2.0443, -0.0154, 0.8006) -- cycle;
\fill[blue!15.0, opacity=0.5] (2.0443, -0.0154, 0.8006) -- (2.0938, -0.0154, 0.7971) -- (2.0930, -0.0176, 0.8471) -- (2.0436, -0.0176, 0.8506) -- cycle;
\fill[blue!15.0, opacity=0.5] (2.0436, -0.0176, 0.8506) -- (2.0930, -0.0176, 0.8471) -- (2.0921, -0.0199, 0.8971) -- (2.0427, -0.0199, 0.9006) -- cycle;
\fill[blue!15.0, opacity=0.5] (2.0427, -0.0199, 0.9006) -- (2.0921, -0.0199, 0.8971) -- (2.0911, -0.0222, 0.9471) -- (2.0418, -0.0222, 0.9506) -- cycle;
\fill[blue!15.0, opacity=0.5] (2.0418, -0.0222, 0.9506) -- (2.0911, -0.0222, 0.9471) -- (2.0901, -0.0247, 0.9971) -- (2.0409, -0.0247, 1.0006) -- cycle;
\fill[blue!15.0, opacity=0.5] (2.0409, -0.0247, 1.0006) -- (2.0901, -0.0247, 0.9971) -- (2.0891, -0.0273, 1.0471) -- (2.0400, -0.0273, 1.0506) -- cycle;
\fill[blue!15.0, opacity=0.5] (2.0400, -0.0273, 1.0506) -- (2.0891, -0.0273, 1.0471) -- (2.0880, -0.0300, 1.0971) -- (2.0390, -0.0300, 1.1006) -- cycle;
\fill[blue!15.0, opacity=0.5] (2.0390, -0.0300, 1.1006) -- (2.0880, -0.0300, 1.0971) -- (2.0869, -0.0328, 1.1471) -- (2.0380, -0.0328, 1.1506) -- cycle;
\fill[blue!15.0, opacity=0.5] (2.0380, -0.0328, 1.1506) -- (2.0869, -0.0328, 1.1471) -- (2.0858, -0.0356, 1.1971) -- (2.0369, -0.0356, 1.2006) -- cycle;
\fill[blue!15.0, opacity=0.5] (2.0369, -0.0356, 1.2006) -- (2.0858, -0.0356, 1.1971) -- (2.0846, -0.0385, 1.2471) -- (2.0359, -0.0385, 1.2506) -- cycle;
\fill[blue!15.0, opacity=0.5] (2.0359, -0.0385, 1.2506) -- (2.0846, -0.0385, 1.2471) -- (2.0834, -0.0415, 1.2971) -- (2.0348, -0.0415, 1.3006) -- cycle;
\fill[blue!15.0, opacity=0.5] (2.0348, -0.0415, 1.3006) -- (2.0834, -0.0415, 1.2971) -- (2.0822, -0.0445, 1.3471) -- (2.0337, -0.0445, 1.3506) -- cycle;
\fill[blue!15.0, opacity=0.5] (2.0337, -0.0445, 1.3506) -- (2.0822, -0.0445, 1.3471) -- (2.0810, -0.0475, 1.3971) -- (2.0326, -0.0475, 1.4006) -- cycle;
\fill[blue!15.1, opacity=0.5] (2.0326, -0.0475, 1.4006) -- (2.0810, -0.0475, 1.3971) -- (2.0798, -0.0506, 1.4471) -- (2.0314, -0.0506, 1.4506) -- cycle;
\fill[blue!15.1, opacity=0.5] (2.0314, -0.0506, 1.4506) -- (2.0798, -0.0506, 1.4471) -- (2.0785, -0.0537, 1.4971) -- (2.0303, -0.0537, 1.5006) -- cycle;
\fill[blue!15.2, opacity=0.5] (2.0303, -0.0537, 1.5006) -- (2.0785, -0.0537, 1.4971) -- (2.0773, -0.0569, 1.5471) -- (2.0292, -0.0569, 1.5506) -- cycle;
\fill[blue!15.2, opacity=0.5] (2.0292, -0.0569, 1.5506) -- (2.0773, -0.0569, 1.5471) -- (2.0760, -0.0600, 1.5971) -- (2.0280, -0.0600, 1.6006) -- cycle;
\fill[blue!15.3, opacity=0.5] (2.0280, -0.0600, 1.6006) -- (2.0760, -0.0600, 1.5971) -- (2.0747, -0.0631, 1.6471) -- (2.0268, -0.0631, 1.6506) -- cycle;
\fill[blue!15.5, opacity=0.5] (2.0268, -0.0631, 1.6506) -- (2.0747, -0.0631, 1.6471) -- (2.0735, -0.0663, 1.6971) -- (2.0257, -0.0663, 1.7006) -- cycle;
\fill[blue!15.7, opacity=0.5] (2.0257, -0.0663, 1.7006) -- (2.0735, -0.0663, 1.6971) -- (2.0722, -0.0694, 1.7471) -- (2.0246, -0.0694, 1.7506) -- cycle;
\fill[blue!15.9, opacity=0.5] (2.0246, -0.0694, 1.7506) -- (2.0722, -0.0694, 1.7471) -- (2.0710, -0.0725, 1.7971) -- (2.0234, -0.0725, 1.8006) -- cycle;
\fill[blue!16.2, opacity=0.5] (2.0234, -0.0725, 1.8006) -- (2.0710, -0.0725, 1.7971) -- (2.0698, -0.0755, 1.8471) -- (2.0223, -0.0755, 1.8506) -- cycle;
\fill[blue!16.5, opacity=0.5] (2.0223, -0.0755, 1.8506) -- (2.0698, -0.0755, 1.8471) -- (2.0686, -0.0785, 1.8971) -- (2.0212, -0.0785, 1.9006) -- cycle;
\fill[blue!17.0, opacity=0.5] (2.0212, -0.0785, 1.9006) -- (2.0686, -0.0785, 1.8971) -- (2.0674, -0.0815, 1.9471) -- (2.0201, -0.0815, 1.9506) -- cycle;
\fill[blue!17.5, opacity=0.5] (2.0201, -0.0815, 1.9506) -- (2.0674, -0.0815, 1.9471) -- (2.0662, -0.0844, 1.9971) -- (2.0191, -0.0844, 2.0006) -- cycle;
\fill[blue!18.1, opacity=0.5] (2.0191, -0.0844, 2.0006) -- (2.0662, -0.0844, 1.9971) -- (2.0651, -0.0872, 2.0471) -- (2.0180, -0.0872, 2.0506) -- cycle;
\fill[blue!18.8, opacity=0.5] (2.0180, -0.0872, 2.0506) -- (2.0651, -0.0872, 2.0471) -- (2.0640, -0.0900, 2.0971) -- (2.0170, -0.0900, 2.1006) -- cycle;
\fill[blue!19.5, opacity=0.5] (2.0170, -0.0900, 2.1006) -- (2.0640, -0.0900, 2.0971) -- (2.0629, -0.0927, 2.1471) -- (2.0160, -0.0927, 2.1506) -- cycle;
\fill[blue!20.4, opacity=0.5] (2.0160, -0.0927, 2.1506) -- (2.0629, -0.0927, 2.1471) -- (2.0619, -0.0953, 2.1971) -- (2.0151, -0.0953, 2.2006) -- cycle;
\fill[blue!21.4, opacity=0.5] (2.0151, -0.0953, 2.2006) -- (2.0619, -0.0953, 2.1971) -- (2.0609, -0.0978, 2.2471) -- (2.0142, -0.0978, 2.2506) -- cycle;
\fill[blue!22.4, opacity=0.5] (2.0142, -0.0978, 2.2506) -- (2.0609, -0.0978, 2.2471) -- (2.0599, -0.1001, 2.2971) -- (2.0133, -0.1001, 2.3006) -- cycle;
\fill[blue!23.5, opacity=0.5] (2.0133, -0.1001, 2.3006) -- (2.0599, -0.1001, 2.2971) -- (2.0590, -0.1024, 2.3471) -- (2.0124, -0.1024, 2.3506) -- cycle;
\fill[blue!24.7, opacity=0.5] (2.0124, -0.1024, 2.3506) -- (2.0590, -0.1024, 2.3471) -- (2.0582, -0.1046, 2.3971) -- (2.0117, -0.1046, 2.4006) -- cycle;
\fill[blue!25.9, opacity=0.5] (2.0117, -0.1046, 2.4006) -- (2.0582, -0.1046, 2.3971) -- (2.0573, -0.1066, 2.4471) -- (2.0109, -0.1066, 2.4506) -- cycle;
\fill[blue!27.2, opacity=0.5] (2.0109, -0.1066, 2.4506) -- (2.0573, -0.1066, 2.4471) -- (2.0566, -0.1085, 2.4971) -- (2.0102, -0.1085, 2.5006) -- cycle;
\fill[blue!28.5, opacity=0.5] (2.0102, -0.1085, 2.5006) -- (2.0566, -0.1085, 2.4971) -- (2.0559, -0.1103, 2.5471) -- (2.0095, -0.1103, 2.5506) -- cycle;
\fill[blue!29.8, opacity=0.5] (2.0095, -0.1103, 2.5506) -- (2.0559, -0.1103, 2.5471) -- (2.0552, -0.1120, 2.5971) -- (2.0089, -0.1120, 2.6006) -- cycle;
\fill[blue!31.2, opacity=0.5] (2.0089, -0.1120, 2.6006) -- (2.0552, -0.1120, 2.5971) -- (2.0546, -0.1135, 2.6471) -- (2.0084, -0.1135, 2.6506) -- cycle;
\fill[blue!32.5, opacity=0.5] (2.0084, -0.1135, 2.6506) -- (2.0546, -0.1135, 2.6471) -- (2.0541, -0.1148, 2.6971) -- (2.0079, -0.1148, 2.7006) -- cycle;
\fill[blue!33.8, opacity=0.5] (2.0079, -0.1148, 2.7006) -- (2.0541, -0.1148, 2.6971) -- (2.0536, -0.1160, 2.7471) -- (2.0075, -0.1160, 2.7506) -- cycle;
\fill[blue!35.2, opacity=0.5] (2.0075, -0.1160, 2.7506) -- (2.0536, -0.1160, 2.7471) -- (2.0532, -0.1171, 2.7971) -- (2.0071, -0.1171, 2.8006) -- cycle;
\fill[blue!36.4, opacity=0.5] (2.0071, -0.1171, 2.8006) -- (2.0532, -0.1171, 2.7971) -- (2.0528, -0.1180, 2.8471) -- (2.0067, -0.1180, 2.8506) -- cycle;
\fill[blue!37.6, opacity=0.5] (2.0067, -0.1180, 2.8506) -- (2.0528, -0.1180, 2.8471) -- (2.0525, -0.1187, 2.8971) -- (2.0065, -0.1187, 2.9006) -- cycle;
\fill[blue!38.8, opacity=0.5] (2.0065, -0.1187, 2.9006) -- (2.0525, -0.1187, 2.8971) -- (2.0523, -0.1193, 2.9471) -- (2.0063, -0.1193, 2.9506) -- cycle;
\fill[blue!39.9, opacity=0.5] (2.0063, -0.1193, 2.9506) -- (2.0523, -0.1193, 2.9471) -- (2.0521, -0.1197, 2.9971) -- (2.0061, -0.1197, 3.0006) -- cycle;
\fill[blue!40.9, opacity=0.5] (2.0061, -0.1197, 3.0006) -- (2.0521, -0.1197, 2.9971) -- (2.0520, -0.1199, 3.0471) -- (2.0060, -0.1199, 3.0506) -- cycle;
\fill[blue!41.8, opacity=0.5] (2.0060, -0.1199, 3.0506) -- (2.0520, -0.1199, 3.0471) -- (2.0520, -0.1200, 3.0971) -- (2.0060, -0.1200, 3.1006) -- cycle;
\fill[blue!15.0, opacity=0.5] (2.1000, -0.0000, 0.0971) -- (2.1500, -0.0000, 0.0933) -- (2.1500, -0.0001, 0.1433) -- (2.1000, -0.0001, 0.1471) -- cycle;
\fill[blue!15.0, opacity=0.5] (2.1000, -0.0001, 0.1471) -- (2.1500, -0.0001, 0.1433) -- (2.1499, -0.0003, 0.1933) -- (2.0999, -0.0003, 0.1971) -- cycle;
\fill[blue!15.0, opacity=0.5] (2.0999, -0.0003, 0.1971) -- (2.1499, -0.0003, 0.1933) -- (2.1497, -0.0007, 0.2433) -- (2.0997, -0.0007, 0.2471) -- cycle;
\fill[blue!15.0, opacity=0.5] (2.0997, -0.0007, 0.2471) -- (2.1497, -0.0007, 0.2433) -- (2.1494, -0.0013, 0.2933) -- (2.0995, -0.0013, 0.2971) -- cycle;
\fill[blue!15.0, opacity=0.5] (2.0995, -0.0013, 0.2971) -- (2.1494, -0.0013, 0.2933) -- (2.1491, -0.0020, 0.3433) -- (2.0992, -0.0020, 0.3471) -- cycle;
\fill[blue!15.0, opacity=0.5] (2.0992, -0.0020, 0.3471) -- (2.1491, -0.0020, 0.3433) -- (2.1487, -0.0029, 0.3933) -- (2.0988, -0.0029, 0.3971) -- cycle;
\fill[blue!15.0, opacity=0.5] (2.0988, -0.0029, 0.3971) -- (2.1487, -0.0029, 0.3933) -- (2.1483, -0.0040, 0.4433) -- (2.0984, -0.0040, 0.4471) -- cycle;
\fill[blue!15.0, opacity=0.5] (2.0984, -0.0040, 0.4471) -- (2.1483, -0.0040, 0.4433) -- (2.1478, -0.0052, 0.4933) -- (2.0979, -0.0052, 0.4971) -- cycle;
\fill[blue!15.0, opacity=0.5] (2.0979, -0.0052, 0.4971) -- (2.1478, -0.0052, 0.4933) -- (2.1472, -0.0065, 0.5433) -- (2.0974, -0.0065, 0.5471) -- cycle;
\fill[blue!15.0, opacity=0.5] (2.0974, -0.0065, 0.5471) -- (2.1472, -0.0065, 0.5433) -- (2.1465, -0.0080, 0.5933) -- (2.0968, -0.0080, 0.5971) -- cycle;
\fill[blue!15.0, opacity=0.5] (2.0968, -0.0080, 0.5971) -- (2.1465, -0.0080, 0.5933) -- (2.1458, -0.0097, 0.6433) -- (2.0961, -0.0097, 0.6471) -- cycle;
\fill[blue!15.0, opacity=0.5] (2.0961, -0.0097, 0.6471) -- (2.1458, -0.0097, 0.6433) -- (2.1450, -0.0115, 0.6933) -- (2.0954, -0.0115, 0.6971) -- cycle;
\fill[blue!15.0, opacity=0.5] (2.0954, -0.0115, 0.6971) -- (2.1450, -0.0115, 0.6933) -- (2.1442, -0.0134, 0.7433) -- (2.0947, -0.0134, 0.7471) -- cycle;
\fill[blue!15.0, opacity=0.5] (2.0947, -0.0134, 0.7471) -- (2.1442, -0.0134, 0.7433) -- (2.1433, -0.0154, 0.7933) -- (2.0938, -0.0154, 0.7971) -- cycle;
\fill[blue!15.0, opacity=0.5] (2.0938, -0.0154, 0.7971) -- (2.1433, -0.0154, 0.7933) -- (2.1424, -0.0176, 0.8433) -- (2.0930, -0.0176, 0.8471) -- cycle;
\fill[blue!15.0, opacity=0.5] (2.0930, -0.0176, 0.8471) -- (2.1424, -0.0176, 0.8433) -- (2.1414, -0.0199, 0.8933) -- (2.0921, -0.0199, 0.8971) -- cycle;
\fill[blue!15.0, opacity=0.5] (2.0921, -0.0199, 0.8971) -- (2.1414, -0.0199, 0.8933) -- (2.1404, -0.0222, 0.9433) -- (2.0911, -0.0222, 0.9471) -- cycle;
\fill[blue!15.0, opacity=0.5] (2.0911, -0.0222, 0.9471) -- (2.1404, -0.0222, 0.9433) -- (2.1393, -0.0247, 0.9933) -- (2.0901, -0.0247, 0.9971) -- cycle;
\fill[blue!15.0, opacity=0.5] (2.0901, -0.0247, 0.9971) -- (2.1393, -0.0247, 0.9933) -- (2.1382, -0.0273, 1.0433) -- (2.0891, -0.0273, 1.0471) -- cycle;
\fill[blue!15.0, opacity=0.5] (2.0891, -0.0273, 1.0471) -- (2.1382, -0.0273, 1.0433) -- (2.1370, -0.0300, 1.0933) -- (2.0880, -0.0300, 1.0971) -- cycle;
\fill[blue!15.0, opacity=0.5] (2.0880, -0.0300, 1.0971) -- (2.1370, -0.0300, 1.0933) -- (2.1358, -0.0328, 1.1433) -- (2.0869, -0.0328, 1.1471) -- cycle;
\fill[blue!15.0, opacity=0.5] (2.0869, -0.0328, 1.1471) -- (2.1358, -0.0328, 1.1433) -- (2.1346, -0.0356, 1.1933) -- (2.0858, -0.0356, 1.1971) -- cycle;
\fill[blue!15.0, opacity=0.5] (2.0858, -0.0356, 1.1971) -- (2.1346, -0.0356, 1.1933) -- (2.1333, -0.0385, 1.2433) -- (2.0846, -0.0385, 1.2471) -- cycle;
\fill[blue!15.0, opacity=0.5] (2.0846, -0.0385, 1.2471) -- (2.1333, -0.0385, 1.2433) -- (2.1320, -0.0415, 1.2933) -- (2.0834, -0.0415, 1.2971) -- cycle;
\fill[blue!15.0, opacity=0.5] (2.0834, -0.0415, 1.2971) -- (2.1320, -0.0415, 1.2933) -- (2.1307, -0.0445, 1.3433) -- (2.0822, -0.0445, 1.3471) -- cycle;
\fill[blue!15.0, opacity=0.5] (2.0822, -0.0445, 1.3471) -- (2.1307, -0.0445, 1.3433) -- (2.1294, -0.0475, 1.3933) -- (2.0810, -0.0475, 1.3971) -- cycle;
\fill[blue!15.0, opacity=0.5] (2.0810, -0.0475, 1.3971) -- (2.1294, -0.0475, 1.3933) -- (2.1281, -0.0506, 1.4433) -- (2.0798, -0.0506, 1.4471) -- cycle;
\fill[blue!15.0, opacity=0.5] (2.0798, -0.0506, 1.4471) -- (2.1281, -0.0506, 1.4433) -- (2.1267, -0.0537, 1.4933) -- (2.0785, -0.0537, 1.4971) -- cycle;
\fill[blue!15.0, opacity=0.5] (2.0785, -0.0537, 1.4971) -- (2.1267, -0.0537, 1.4933) -- (2.1254, -0.0569, 1.5433) -- (2.0773, -0.0569, 1.5471) -- cycle;
\fill[blue!15.0, opacity=0.5] (2.0773, -0.0569, 1.5471) -- (2.1254, -0.0569, 1.5433) -- (2.1240, -0.0600, 1.5933) -- (2.0760, -0.0600, 1.5971) -- cycle;
\fill[blue!15.0, opacity=0.5] (2.0760, -0.0600, 1.5971) -- (2.1240, -0.0600, 1.5933) -- (2.1226, -0.0631, 1.6433) -- (2.0747, -0.0631, 1.6471) -- cycle;
\fill[blue!15.1, opacity=0.5] (2.0747, -0.0631, 1.6471) -- (2.1226, -0.0631, 1.6433) -- (2.1213, -0.0663, 1.6933) -- (2.0735, -0.0663, 1.6971) -- cycle;
\fill[blue!15.1, opacity=0.5] (2.0735, -0.0663, 1.6971) -- (2.1213, -0.0663, 1.6933) -- (2.1199, -0.0694, 1.7433) -- (2.0722, -0.0694, 1.7471) -- cycle;
\fill[blue!15.1, opacity=0.5] (2.0722, -0.0694, 1.7471) -- (2.1199, -0.0694, 1.7433) -- (2.1186, -0.0725, 1.7933) -- (2.0710, -0.0725, 1.7971) -- cycle;
\fill[blue!15.2, opacity=0.5] (2.0710, -0.0725, 1.7971) -- (2.1186, -0.0725, 1.7933) -- (2.1173, -0.0755, 1.8433) -- (2.0698, -0.0755, 1.8471) -- cycle;
\fill[blue!15.2, opacity=0.5] (2.0698, -0.0755, 1.8471) -- (2.1173, -0.0755, 1.8433) -- (2.1160, -0.0785, 1.8933) -- (2.0686, -0.0785, 1.8971) -- cycle;
\fill[blue!15.3, opacity=0.5] (2.0686, -0.0785, 1.8971) -- (2.1160, -0.0785, 1.8933) -- (2.1147, -0.0815, 1.9433) -- (2.0674, -0.0815, 1.9471) -- cycle;
\fill[blue!15.5, opacity=0.5] (2.0674, -0.0815, 1.9471) -- (2.1147, -0.0815, 1.9433) -- (2.1134, -0.0844, 1.9933) -- (2.0662, -0.0844, 1.9971) -- cycle;
\fill[blue!15.6, opacity=0.5] (2.0662, -0.0844, 1.9971) -- (2.1134, -0.0844, 1.9933) -- (2.1122, -0.0872, 2.0433) -- (2.0651, -0.0872, 2.0471) -- cycle;
\fill[blue!15.8, opacity=0.5] (2.0651, -0.0872, 2.0471) -- (2.1122, -0.0872, 2.0433) -- (2.1110, -0.0900, 2.0933) -- (2.0640, -0.0900, 2.0971) -- cycle;
\fill[blue!16.1, opacity=0.5] (2.0640, -0.0900, 2.0971) -- (2.1110, -0.0900, 2.0933) -- (2.1098, -0.0927, 2.1433) -- (2.0629, -0.0927, 2.1471) -- cycle;
\fill[blue!16.3, opacity=0.5] (2.0629, -0.0927, 2.1471) -- (2.1098, -0.0927, 2.1433) -- (2.1087, -0.0953, 2.1933) -- (2.0619, -0.0953, 2.1971) -- cycle;
\fill[blue!16.7, opacity=0.5] (2.0619, -0.0953, 2.1971) -- (2.1087, -0.0953, 2.1933) -- (2.1076, -0.0978, 2.2433) -- (2.0609, -0.0978, 2.2471) -- cycle;
\fill[blue!17.1, opacity=0.5] (2.0609, -0.0978, 2.2471) -- (2.1076, -0.0978, 2.2433) -- (2.1066, -0.1001, 2.2933) -- (2.0599, -0.1001, 2.2971) -- cycle;
\fill[blue!17.5, opacity=0.5] (2.0599, -0.1001, 2.2971) -- (2.1066, -0.1001, 2.2933) -- (2.1056, -0.1024, 2.3433) -- (2.0590, -0.1024, 2.3471) -- cycle;
\fill[blue!18.1, opacity=0.5] (2.0590, -0.1024, 2.3471) -- (2.1056, -0.1024, 2.3433) -- (2.1047, -0.1046, 2.3933) -- (2.0582, -0.1046, 2.3971) -- cycle;
\fill[blue!18.6, opacity=0.5] (2.0582, -0.1046, 2.3971) -- (2.1047, -0.1046, 2.3933) -- (2.1038, -0.1066, 2.4433) -- (2.0573, -0.1066, 2.4471) -- cycle;
\fill[blue!19.3, opacity=0.5] (2.0573, -0.1066, 2.4471) -- (2.1038, -0.1066, 2.4433) -- (2.1030, -0.1085, 2.4933) -- (2.0566, -0.1085, 2.4971) -- cycle;
\fill[blue!20.0, opacity=0.5] (2.0566, -0.1085, 2.4971) -- (2.1030, -0.1085, 2.4933) -- (2.1022, -0.1103, 2.5433) -- (2.0559, -0.1103, 2.5471) -- cycle;
\fill[blue!20.7, opacity=0.5] (2.0559, -0.1103, 2.5471) -- (2.1022, -0.1103, 2.5433) -- (2.1015, -0.1120, 2.5933) -- (2.0552, -0.1120, 2.5971) -- cycle;
\fill[blue!21.5, opacity=0.5] (2.0552, -0.1120, 2.5971) -- (2.1015, -0.1120, 2.5933) -- (2.1008, -0.1135, 2.6433) -- (2.0546, -0.1135, 2.6471) -- cycle;
\fill[blue!22.4, opacity=0.5] (2.0546, -0.1135, 2.6471) -- (2.1008, -0.1135, 2.6433) -- (2.1002, -0.1148, 2.6933) -- (2.0541, -0.1148, 2.6971) -- cycle;
\fill[blue!23.3, opacity=0.5] (2.0541, -0.1148, 2.6971) -- (2.1002, -0.1148, 2.6933) -- (2.0997, -0.1160, 2.7433) -- (2.0536, -0.1160, 2.7471) -- cycle;
\fill[blue!24.2, opacity=0.5] (2.0536, -0.1160, 2.7471) -- (2.0997, -0.1160, 2.7433) -- (2.0993, -0.1171, 2.7933) -- (2.0532, -0.1171, 2.7971) -- cycle;
\fill[blue!25.2, opacity=0.5] (2.0532, -0.1171, 2.7971) -- (2.0993, -0.1171, 2.7933) -- (2.0989, -0.1180, 2.8433) -- (2.0528, -0.1180, 2.8471) -- cycle;
\fill[blue!26.2, opacity=0.5] (2.0528, -0.1180, 2.8471) -- (2.0989, -0.1180, 2.8433) -- (2.0986, -0.1187, 2.8933) -- (2.0525, -0.1187, 2.8971) -- cycle;
\fill[blue!27.1, opacity=0.5] (2.0525, -0.1187, 2.8971) -- (2.0986, -0.1187, 2.8933) -- (2.0983, -0.1193, 2.9433) -- (2.0523, -0.1193, 2.9471) -- cycle;
\fill[blue!28.1, opacity=0.5] (2.0523, -0.1193, 2.9471) -- (2.0983, -0.1193, 2.9433) -- (2.0981, -0.1197, 2.9933) -- (2.0521, -0.1197, 2.9971) -- cycle;
\fill[blue!29.1, opacity=0.5] (2.0521, -0.1197, 2.9971) -- (2.0981, -0.1197, 2.9933) -- (2.0980, -0.1199, 3.0433) -- (2.0520, -0.1199, 3.0471) -- cycle;
\fill[blue!30.1, opacity=0.5] (2.0520, -0.1199, 3.0471) -- (2.0980, -0.1199, 3.0433) -- (2.0980, -0.1200, 3.0933) -- (2.0520, -0.1200, 3.0971) -- cycle;
\fill[blue!15.0, opacity=0.5] (2.1500, -0.0000, 0.0933) -- (2.2000, -0.0000, 0.0892) -- (2.2000, -0.0001, 0.1392) -- (2.1500, -0.0001, 0.1433) -- cycle;
\fill[blue!15.0, opacity=0.5] (2.1500, -0.0001, 0.1433) -- (2.2000, -0.0001, 0.1392) -- (2.1998, -0.0003, 0.1892) -- (2.1499, -0.0003, 0.1933) -- cycle;
\fill[blue!15.0, opacity=0.5] (2.1499, -0.0003, 0.1933) -- (2.1998, -0.0003, 0.1892) -- (2.1997, -0.0007, 0.2392) -- (2.1497, -0.0007, 0.2433) -- cycle;
\fill[blue!15.0, opacity=0.5] (2.1497, -0.0007, 0.2433) -- (2.1997, -0.0007, 0.2392) -- (2.1994, -0.0013, 0.2892) -- (2.1494, -0.0013, 0.2933) -- cycle;
\fill[blue!15.0, opacity=0.5] (2.1494, -0.0013, 0.2933) -- (2.1994, -0.0013, 0.2892) -- (2.1990, -0.0020, 0.3392) -- (2.1491, -0.0020, 0.3433) -- cycle;
\fill[blue!15.0, opacity=0.5] (2.1491, -0.0020, 0.3433) -- (2.1990, -0.0020, 0.3392) -- (2.1986, -0.0029, 0.3892) -- (2.1487, -0.0029, 0.3933) -- cycle;
\fill[blue!15.0, opacity=0.5] (2.1487, -0.0029, 0.3933) -- (2.1986, -0.0029, 0.3892) -- (2.1981, -0.0040, 0.4392) -- (2.1483, -0.0040, 0.4433) -- cycle;
\fill[blue!15.0, opacity=0.5] (2.1483, -0.0040, 0.4433) -- (2.1981, -0.0040, 0.4392) -- (2.1976, -0.0052, 0.4892) -- (2.1478, -0.0052, 0.4933) -- cycle;
\fill[blue!15.0, opacity=0.5] (2.1478, -0.0052, 0.4933) -- (2.1976, -0.0052, 0.4892) -- (2.1969, -0.0065, 0.5392) -- (2.1472, -0.0065, 0.5433) -- cycle;
\fill[blue!15.0, opacity=0.5] (2.1472, -0.0065, 0.5433) -- (2.1969, -0.0065, 0.5392) -- (2.1962, -0.0080, 0.5892) -- (2.1465, -0.0080, 0.5933) -- cycle;
\fill[blue!15.0, opacity=0.5] (2.1465, -0.0080, 0.5933) -- (2.1962, -0.0080, 0.5892) -- (2.1955, -0.0097, 0.6392) -- (2.1458, -0.0097, 0.6433) -- cycle;
\fill[blue!15.0, opacity=0.5] (2.1458, -0.0097, 0.6433) -- (2.1955, -0.0097, 0.6392) -- (2.1947, -0.0115, 0.6892) -- (2.1450, -0.0115, 0.6933) -- cycle;
\fill[blue!15.0, opacity=0.5] (2.1450, -0.0115, 0.6933) -- (2.1947, -0.0115, 0.6892) -- (2.1938, -0.0134, 0.7392) -- (2.1442, -0.0134, 0.7433) -- cycle;
\fill[blue!15.0, opacity=0.5] (2.1442, -0.0134, 0.7433) -- (2.1938, -0.0134, 0.7392) -- (2.1928, -0.0154, 0.7892) -- (2.1433, -0.0154, 0.7933) -- cycle;
\fill[blue!15.0, opacity=0.5] (2.1433, -0.0154, 0.7933) -- (2.1928, -0.0154, 0.7892) -- (2.1918, -0.0176, 0.8392) -- (2.1424, -0.0176, 0.8433) -- cycle;
\fill[blue!15.0, opacity=0.5] (2.1424, -0.0176, 0.8433) -- (2.1918, -0.0176, 0.8392) -- (2.1907, -0.0199, 0.8892) -- (2.1414, -0.0199, 0.8933) -- cycle;
\fill[blue!15.0, opacity=0.5] (2.1414, -0.0199, 0.8933) -- (2.1907, -0.0199, 0.8892) -- (2.1896, -0.0222, 0.9392) -- (2.1404, -0.0222, 0.9433) -- cycle;
\fill[blue!15.0, opacity=0.5] (2.1404, -0.0222, 0.9433) -- (2.1896, -0.0222, 0.9392) -- (2.1885, -0.0247, 0.9892) -- (2.1393, -0.0247, 0.9933) -- cycle;
\fill[blue!15.0, opacity=0.5] (2.1393, -0.0247, 0.9933) -- (2.1885, -0.0247, 0.9892) -- (2.1872, -0.0273, 1.0392) -- (2.1382, -0.0273, 1.0433) -- cycle;
\fill[blue!15.0, opacity=0.5] (2.1382, -0.0273, 1.0433) -- (2.1872, -0.0273, 1.0392) -- (2.1860, -0.0300, 1.0892) -- (2.1370, -0.0300, 1.0933) -- cycle;
\fill[blue!15.0, opacity=0.5] (2.1370, -0.0300, 1.0933) -- (2.1860, -0.0300, 1.0892) -- (2.1847, -0.0328, 1.1392) -- (2.1358, -0.0328, 1.1433) -- cycle;
\fill[blue!15.0, opacity=0.5] (2.1358, -0.0328, 1.1433) -- (2.1847, -0.0328, 1.1392) -- (2.1834, -0.0356, 1.1892) -- (2.1346, -0.0356, 1.1933) -- cycle;
\fill[blue!15.0, opacity=0.5] (2.1346, -0.0356, 1.1933) -- (2.1834, -0.0356, 1.1892) -- (2.1820, -0.0385, 1.2392) -- (2.1333, -0.0385, 1.2433) -- cycle;
\fill[blue!15.0, opacity=0.5] (2.1333, -0.0385, 1.2433) -- (2.1820, -0.0385, 1.2392) -- (2.1807, -0.0415, 1.2892) -- (2.1320, -0.0415, 1.2933) -- cycle;
\fill[blue!15.0, opacity=0.5] (2.1320, -0.0415, 1.2933) -- (2.1807, -0.0415, 1.2892) -- (2.1792, -0.0445, 1.3392) -- (2.1307, -0.0445, 1.3433) -- cycle;
\fill[blue!15.0, opacity=0.5] (2.1307, -0.0445, 1.3433) -- (2.1792, -0.0445, 1.3392) -- (2.1778, -0.0475, 1.3892) -- (2.1294, -0.0475, 1.3933) -- cycle;
\fill[blue!15.0, opacity=0.5] (2.1294, -0.0475, 1.3933) -- (2.1778, -0.0475, 1.3892) -- (2.1764, -0.0506, 1.4392) -- (2.1281, -0.0506, 1.4433) -- cycle;
\fill[blue!15.0, opacity=0.5] (2.1281, -0.0506, 1.4433) -- (2.1764, -0.0506, 1.4392) -- (2.1749, -0.0537, 1.4892) -- (2.1267, -0.0537, 1.4933) -- cycle;
\fill[blue!15.0, opacity=0.5] (2.1267, -0.0537, 1.4933) -- (2.1749, -0.0537, 1.4892) -- (2.1735, -0.0569, 1.5392) -- (2.1254, -0.0569, 1.5433) -- cycle;
\fill[blue!15.0, opacity=0.5] (2.1254, -0.0569, 1.5433) -- (2.1735, -0.0569, 1.5392) -- (2.1720, -0.0600, 1.5892) -- (2.1240, -0.0600, 1.5933) -- cycle;
\fill[blue!15.0, opacity=0.5] (2.1240, -0.0600, 1.5933) -- (2.1720, -0.0600, 1.5892) -- (2.1705, -0.0631, 1.6392) -- (2.1226, -0.0631, 1.6433) -- cycle;
\fill[blue!15.0, opacity=0.5] (2.1226, -0.0631, 1.6433) -- (2.1705, -0.0631, 1.6392) -- (2.1691, -0.0663, 1.6892) -- (2.1213, -0.0663, 1.6933) -- cycle;
\fill[blue!15.0, opacity=0.5] (2.1213, -0.0663, 1.6933) -- (2.1691, -0.0663, 1.6892) -- (2.1676, -0.0694, 1.7392) -- (2.1199, -0.0694, 1.7433) -- cycle;
\fill[blue!15.0, opacity=0.5] (2.1199, -0.0694, 1.7433) -- (2.1676, -0.0694, 1.7392) -- (2.1662, -0.0725, 1.7892) -- (2.1186, -0.0725, 1.7933) -- cycle;
\fill[blue!15.0, opacity=0.5] (2.1186, -0.0725, 1.7933) -- (2.1662, -0.0725, 1.7892) -- (2.1648, -0.0755, 1.8392) -- (2.1173, -0.0755, 1.8433) -- cycle;
\fill[blue!15.0, opacity=0.5] (2.1173, -0.0755, 1.8433) -- (2.1648, -0.0755, 1.8392) -- (2.1633, -0.0785, 1.8892) -- (2.1160, -0.0785, 1.8933) -- cycle;
\fill[blue!15.0, opacity=0.5] (2.1160, -0.0785, 1.8933) -- (2.1633, -0.0785, 1.8892) -- (2.1620, -0.0815, 1.9392) -- (2.1147, -0.0815, 1.9433) -- cycle;
\fill[blue!15.1, opacity=0.5] (2.1147, -0.0815, 1.9433) -- (2.1620, -0.0815, 1.9392) -- (2.1606, -0.0844, 1.9892) -- (2.1134, -0.0844, 1.9933) -- cycle;
\fill[blue!15.1, opacity=0.5] (2.1134, -0.0844, 1.9933) -- (2.1606, -0.0844, 1.9892) -- (2.1593, -0.0872, 2.0392) -- (2.1122, -0.0872, 2.0433) -- cycle;
\fill[blue!15.1, opacity=0.5] (2.1122, -0.0872, 2.0433) -- (2.1593, -0.0872, 2.0392) -- (2.1580, -0.0900, 2.0892) -- (2.1110, -0.0900, 2.0933) -- cycle;
\fill[blue!15.2, opacity=0.5] (2.1110, -0.0900, 2.0933) -- (2.1580, -0.0900, 2.0892) -- (2.1568, -0.0927, 2.1392) -- (2.1098, -0.0927, 2.1433) -- cycle;
\fill[blue!15.2, opacity=0.5] (2.1098, -0.0927, 2.1433) -- (2.1568, -0.0927, 2.1392) -- (2.1555, -0.0953, 2.1892) -- (2.1087, -0.0953, 2.1933) -- cycle;
\fill[blue!15.3, opacity=0.5] (2.1087, -0.0953, 2.1933) -- (2.1555, -0.0953, 2.1892) -- (2.1544, -0.0978, 2.2392) -- (2.1076, -0.0978, 2.2433) -- cycle;
\fill[blue!15.4, opacity=0.5] (2.1076, -0.0978, 2.2433) -- (2.1544, -0.0978, 2.2392) -- (2.1533, -0.1001, 2.2892) -- (2.1066, -0.1001, 2.2933) -- cycle;
\fill[blue!15.5, opacity=0.5] (2.1066, -0.1001, 2.2933) -- (2.1533, -0.1001, 2.2892) -- (2.1522, -0.1024, 2.3392) -- (2.1056, -0.1024, 2.3433) -- cycle;
\fill[blue!15.7, opacity=0.5] (2.1056, -0.1024, 2.3433) -- (2.1522, -0.1024, 2.3392) -- (2.1512, -0.1046, 2.3892) -- (2.1047, -0.1046, 2.3933) -- cycle;
\fill[blue!15.9, opacity=0.5] (2.1047, -0.1046, 2.3933) -- (2.1512, -0.1046, 2.3892) -- (2.1502, -0.1066, 2.4392) -- (2.1038, -0.1066, 2.4433) -- cycle;
\fill[blue!16.1, opacity=0.5] (2.1038, -0.1066, 2.4433) -- (2.1502, -0.1066, 2.4392) -- (2.1493, -0.1085, 2.4892) -- (2.1030, -0.1085, 2.4933) -- cycle;
\fill[blue!16.4, opacity=0.5] (2.1030, -0.1085, 2.4933) -- (2.1493, -0.1085, 2.4892) -- (2.1485, -0.1103, 2.5392) -- (2.1022, -0.1103, 2.5433) -- cycle;
\fill[blue!16.6, opacity=0.5] (2.1022, -0.1103, 2.5433) -- (2.1485, -0.1103, 2.5392) -- (2.1478, -0.1120, 2.5892) -- (2.1015, -0.1120, 2.5933) -- cycle;
\fill[blue!17.0, opacity=0.5] (2.1015, -0.1120, 2.5933) -- (2.1478, -0.1120, 2.5892) -- (2.1471, -0.1135, 2.6392) -- (2.1008, -0.1135, 2.6433) -- cycle;
\fill[blue!17.4, opacity=0.5] (2.1008, -0.1135, 2.6433) -- (2.1471, -0.1135, 2.6392) -- (2.1464, -0.1148, 2.6892) -- (2.1002, -0.1148, 2.6933) -- cycle;
\fill[blue!17.8, opacity=0.5] (2.1002, -0.1148, 2.6933) -- (2.1464, -0.1148, 2.6892) -- (2.1459, -0.1160, 2.7392) -- (2.0997, -0.1160, 2.7433) -- cycle;
\fill[blue!18.2, opacity=0.5] (2.0997, -0.1160, 2.7433) -- (2.1459, -0.1160, 2.7392) -- (2.1454, -0.1171, 2.7892) -- (2.0993, -0.1171, 2.7933) -- cycle;
\fill[blue!18.7, opacity=0.5] (2.0993, -0.1171, 2.7933) -- (2.1454, -0.1171, 2.7892) -- (2.1450, -0.1180, 2.8392) -- (2.0989, -0.1180, 2.8433) -- cycle;
\fill[blue!19.3, opacity=0.5] (2.0989, -0.1180, 2.8433) -- (2.1450, -0.1180, 2.8392) -- (2.1446, -0.1187, 2.8892) -- (2.0986, -0.1187, 2.8933) -- cycle;
\fill[blue!19.8, opacity=0.5] (2.0986, -0.1187, 2.8933) -- (2.1446, -0.1187, 2.8892) -- (2.1443, -0.1193, 2.9392) -- (2.0983, -0.1193, 2.9433) -- cycle;
\fill[blue!20.4, opacity=0.5] (2.0983, -0.1193, 2.9433) -- (2.1443, -0.1193, 2.9392) -- (2.1442, -0.1197, 2.9892) -- (2.0981, -0.1197, 2.9933) -- cycle;
\fill[blue!21.1, opacity=0.5] (2.0981, -0.1197, 2.9933) -- (2.1442, -0.1197, 2.9892) -- (2.1440, -0.1199, 3.0392) -- (2.0980, -0.1199, 3.0433) -- cycle;
\fill[blue!21.7, opacity=0.5] (2.0980, -0.1199, 3.0433) -- (2.1440, -0.1199, 3.0392) -- (2.1440, -0.1200, 3.0892) -- (2.0980, -0.1200, 3.0933) -- cycle;
\fill[blue!15.0, opacity=0.5] (2.2000, -0.0000, 0.0892) -- (2.2500, -0.0000, 0.0849) -- (2.2500, -0.0001, 0.1349) -- (2.2000, -0.0001, 0.1392) -- cycle;
\fill[blue!15.0, opacity=0.5] (2.2000, -0.0001, 0.1392) -- (2.2500, -0.0001, 0.1349) -- (2.2498, -0.0003, 0.1849) -- (2.1998, -0.0003, 0.1892) -- cycle;
\fill[blue!15.0, opacity=0.5] (2.1998, -0.0003, 0.1892) -- (2.2498, -0.0003, 0.1849) -- (2.2496, -0.0007, 0.2349) -- (2.1997, -0.0007, 0.2392) -- cycle;
\fill[blue!15.0, opacity=0.5] (2.1997, -0.0007, 0.2392) -- (2.2496, -0.0007, 0.2349) -- (2.2493, -0.0013, 0.2849) -- (2.1994, -0.0013, 0.2892) -- cycle;
\fill[blue!15.0, opacity=0.5] (2.1994, -0.0013, 0.2892) -- (2.2493, -0.0013, 0.2849) -- (2.2490, -0.0020, 0.3349) -- (2.1990, -0.0020, 0.3392) -- cycle;
\fill[blue!15.0, opacity=0.5] (2.1990, -0.0020, 0.3392) -- (2.2490, -0.0020, 0.3349) -- (2.2485, -0.0029, 0.3849) -- (2.1986, -0.0029, 0.3892) -- cycle;
\fill[blue!15.0, opacity=0.5] (2.1986, -0.0029, 0.3892) -- (2.2485, -0.0029, 0.3849) -- (2.2480, -0.0040, 0.4349) -- (2.1981, -0.0040, 0.4392) -- cycle;
\fill[blue!15.0, opacity=0.5] (2.1981, -0.0040, 0.4392) -- (2.2480, -0.0040, 0.4349) -- (2.2474, -0.0052, 0.4849) -- (2.1976, -0.0052, 0.4892) -- cycle;
\fill[blue!15.0, opacity=0.5] (2.1976, -0.0052, 0.4892) -- (2.2474, -0.0052, 0.4849) -- (2.2467, -0.0065, 0.5349) -- (2.1969, -0.0065, 0.5392) -- cycle;
\fill[blue!15.0, opacity=0.5] (2.1969, -0.0065, 0.5392) -- (2.2467, -0.0065, 0.5349) -- (2.2460, -0.0080, 0.5849) -- (2.1962, -0.0080, 0.5892) -- cycle;
\fill[blue!15.0, opacity=0.5] (2.1962, -0.0080, 0.5892) -- (2.2460, -0.0080, 0.5849) -- (2.2452, -0.0097, 0.6349) -- (2.1955, -0.0097, 0.6392) -- cycle;
\fill[blue!15.0, opacity=0.5] (2.1955, -0.0097, 0.6392) -- (2.2452, -0.0097, 0.6349) -- (2.2443, -0.0115, 0.6849) -- (2.1947, -0.0115, 0.6892) -- cycle;
\fill[blue!15.0, opacity=0.5] (2.1947, -0.0115, 0.6892) -- (2.2443, -0.0115, 0.6849) -- (2.2433, -0.0134, 0.7349) -- (2.1938, -0.0134, 0.7392) -- cycle;
\fill[blue!15.0, opacity=0.5] (2.1938, -0.0134, 0.7392) -- (2.2433, -0.0134, 0.7349) -- (2.2423, -0.0154, 0.7849) -- (2.1928, -0.0154, 0.7892) -- cycle;
\fill[blue!15.0, opacity=0.5] (2.1928, -0.0154, 0.7892) -- (2.2423, -0.0154, 0.7849) -- (2.2412, -0.0176, 0.8349) -- (2.1918, -0.0176, 0.8392) -- cycle;
\fill[blue!15.0, opacity=0.5] (2.1918, -0.0176, 0.8392) -- (2.2412, -0.0176, 0.8349) -- (2.2401, -0.0199, 0.8849) -- (2.1907, -0.0199, 0.8892) -- cycle;
\fill[blue!15.0, opacity=0.5] (2.1907, -0.0199, 0.8892) -- (2.2401, -0.0199, 0.8849) -- (2.2389, -0.0222, 0.9349) -- (2.1896, -0.0222, 0.9392) -- cycle;
\fill[blue!15.0, opacity=0.5] (2.1896, -0.0222, 0.9392) -- (2.2389, -0.0222, 0.9349) -- (2.2376, -0.0247, 0.9849) -- (2.1885, -0.0247, 0.9892) -- cycle;
\fill[blue!15.0, opacity=0.5] (2.1885, -0.0247, 0.9892) -- (2.2376, -0.0247, 0.9849) -- (2.2363, -0.0273, 1.0349) -- (2.1872, -0.0273, 1.0392) -- cycle;
\fill[blue!15.0, opacity=0.5] (2.1872, -0.0273, 1.0392) -- (2.2363, -0.0273, 1.0349) -- (2.2350, -0.0300, 1.0849) -- (2.1860, -0.0300, 1.0892) -- cycle;
\fill[blue!15.0, opacity=0.5] (2.1860, -0.0300, 1.0892) -- (2.2350, -0.0300, 1.0849) -- (2.2336, -0.0328, 1.1349) -- (2.1847, -0.0328, 1.1392) -- cycle;
\fill[blue!15.0, opacity=0.5] (2.1847, -0.0328, 1.1392) -- (2.2336, -0.0328, 1.1349) -- (2.2322, -0.0356, 1.1849) -- (2.1834, -0.0356, 1.1892) -- cycle;
\fill[blue!15.0, opacity=0.5] (2.1834, -0.0356, 1.1892) -- (2.2322, -0.0356, 1.1849) -- (2.2308, -0.0385, 1.2349) -- (2.1820, -0.0385, 1.2392) -- cycle;
\fill[blue!15.0, opacity=0.5] (2.1820, -0.0385, 1.2392) -- (2.2308, -0.0385, 1.2349) -- (2.2293, -0.0415, 1.2849) -- (2.1807, -0.0415, 1.2892) -- cycle;
\fill[blue!15.0, opacity=0.5] (2.1807, -0.0415, 1.2892) -- (2.2293, -0.0415, 1.2849) -- (2.2278, -0.0445, 1.3349) -- (2.1792, -0.0445, 1.3392) -- cycle;
\fill[blue!15.0, opacity=0.5] (2.1792, -0.0445, 1.3392) -- (2.2278, -0.0445, 1.3349) -- (2.2262, -0.0475, 1.3849) -- (2.1778, -0.0475, 1.3892) -- cycle;
\fill[blue!15.0, opacity=0.5] (2.1778, -0.0475, 1.3892) -- (2.2262, -0.0475, 1.3849) -- (2.2247, -0.0506, 1.4349) -- (2.1764, -0.0506, 1.4392) -- cycle;
\fill[blue!15.0, opacity=0.5] (2.1764, -0.0506, 1.4392) -- (2.2247, -0.0506, 1.4349) -- (2.2231, -0.0537, 1.4849) -- (2.1749, -0.0537, 1.4892) -- cycle;
\fill[blue!15.0, opacity=0.5] (2.1749, -0.0537, 1.4892) -- (2.2231, -0.0537, 1.4849) -- (2.2216, -0.0569, 1.5349) -- (2.1735, -0.0569, 1.5392) -- cycle;
\fill[blue!15.0, opacity=0.5] (2.1735, -0.0569, 1.5392) -- (2.2216, -0.0569, 1.5349) -- (2.2200, -0.0600, 1.5849) -- (2.1720, -0.0600, 1.5892) -- cycle;
\fill[blue!15.0, opacity=0.5] (2.1720, -0.0600, 1.5892) -- (2.2200, -0.0600, 1.5849) -- (2.2184, -0.0631, 1.6349) -- (2.1705, -0.0631, 1.6392) -- cycle;
\fill[blue!15.0, opacity=0.5] (2.1705, -0.0631, 1.6392) -- (2.2184, -0.0631, 1.6349) -- (2.2169, -0.0663, 1.6849) -- (2.1691, -0.0663, 1.6892) -- cycle;
\fill[blue!15.0, opacity=0.5] (2.1691, -0.0663, 1.6892) -- (2.2169, -0.0663, 1.6849) -- (2.2153, -0.0694, 1.7349) -- (2.1676, -0.0694, 1.7392) -- cycle;
\fill[blue!15.0, opacity=0.5] (2.1676, -0.0694, 1.7392) -- (2.2153, -0.0694, 1.7349) -- (2.2138, -0.0725, 1.7849) -- (2.1662, -0.0725, 1.7892) -- cycle;
\fill[blue!15.0, opacity=0.5] (2.1662, -0.0725, 1.7892) -- (2.2138, -0.0725, 1.7849) -- (2.2122, -0.0755, 1.8349) -- (2.1648, -0.0755, 1.8392) -- cycle;
\fill[blue!15.0, opacity=0.5] (2.1648, -0.0755, 1.8392) -- (2.2122, -0.0755, 1.8349) -- (2.2107, -0.0785, 1.8849) -- (2.1633, -0.0785, 1.8892) -- cycle;
\fill[blue!15.0, opacity=0.5] (2.1633, -0.0785, 1.8892) -- (2.2107, -0.0785, 1.8849) -- (2.2092, -0.0815, 1.9349) -- (2.1620, -0.0815, 1.9392) -- cycle;
\fill[blue!15.0, opacity=0.5] (2.1620, -0.0815, 1.9392) -- (2.2092, -0.0815, 1.9349) -- (2.2078, -0.0844, 1.9849) -- (2.1606, -0.0844, 1.9892) -- cycle;
\fill[blue!15.0, opacity=0.5] (2.1606, -0.0844, 1.9892) -- (2.2078, -0.0844, 1.9849) -- (2.2064, -0.0872, 2.0349) -- (2.1593, -0.0872, 2.0392) -- cycle;
\fill[blue!15.0, opacity=0.5] (2.1593, -0.0872, 2.0392) -- (2.2064, -0.0872, 2.0349) -- (2.2050, -0.0900, 2.0849) -- (2.1580, -0.0900, 2.0892) -- cycle;
\fill[blue!15.0, opacity=0.5] (2.1580, -0.0900, 2.0892) -- (2.2050, -0.0900, 2.0849) -- (2.2037, -0.0927, 2.1349) -- (2.1568, -0.0927, 2.1392) -- cycle;
\fill[blue!15.0, opacity=0.5] (2.1568, -0.0927, 2.1392) -- (2.2037, -0.0927, 2.1349) -- (2.2024, -0.0953, 2.1849) -- (2.1555, -0.0953, 2.1892) -- cycle;
\fill[blue!15.0, opacity=0.5] (2.1555, -0.0953, 2.1892) -- (2.2024, -0.0953, 2.1849) -- (2.2011, -0.0978, 2.2349) -- (2.1544, -0.0978, 2.2392) -- cycle;
\fill[blue!15.1, opacity=0.5] (2.1544, -0.0978, 2.2392) -- (2.2011, -0.0978, 2.2349) -- (2.1999, -0.1001, 2.2849) -- (2.1533, -0.1001, 2.2892) -- cycle;
\fill[blue!15.1, opacity=0.5] (2.1533, -0.1001, 2.2892) -- (2.1999, -0.1001, 2.2849) -- (2.1988, -0.1024, 2.3349) -- (2.1522, -0.1024, 2.3392) -- cycle;
\fill[blue!15.1, opacity=0.5] (2.1522, -0.1024, 2.3392) -- (2.1988, -0.1024, 2.3349) -- (2.1977, -0.1046, 2.3849) -- (2.1512, -0.1046, 2.3892) -- cycle;
\fill[blue!15.2, opacity=0.5] (2.1512, -0.1046, 2.3892) -- (2.1977, -0.1046, 2.3849) -- (2.1967, -0.1066, 2.4349) -- (2.1502, -0.1066, 2.4392) -- cycle;
\fill[blue!15.2, opacity=0.5] (2.1502, -0.1066, 2.4392) -- (2.1967, -0.1066, 2.4349) -- (2.1957, -0.1085, 2.4849) -- (2.1493, -0.1085, 2.4892) -- cycle;
\fill[blue!15.3, opacity=0.5] (2.1493, -0.1085, 2.4892) -- (2.1957, -0.1085, 2.4849) -- (2.1948, -0.1103, 2.5349) -- (2.1485, -0.1103, 2.5392) -- cycle;
\fill[blue!15.4, opacity=0.5] (2.1485, -0.1103, 2.5392) -- (2.1948, -0.1103, 2.5349) -- (2.1940, -0.1120, 2.5849) -- (2.1478, -0.1120, 2.5892) -- cycle;
\fill[blue!15.5, opacity=0.5] (2.1478, -0.1120, 2.5892) -- (2.1940, -0.1120, 2.5849) -- (2.1933, -0.1135, 2.6349) -- (2.1471, -0.1135, 2.6392) -- cycle;
\fill[blue!15.7, opacity=0.5] (2.1471, -0.1135, 2.6392) -- (2.1933, -0.1135, 2.6349) -- (2.1926, -0.1148, 2.6849) -- (2.1464, -0.1148, 2.6892) -- cycle;
\fill[blue!15.8, opacity=0.5] (2.1464, -0.1148, 2.6892) -- (2.1926, -0.1148, 2.6849) -- (2.1920, -0.1160, 2.7349) -- (2.1459, -0.1160, 2.7392) -- cycle;
\fill[blue!16.0, opacity=0.5] (2.1459, -0.1160, 2.7392) -- (2.1920, -0.1160, 2.7349) -- (2.1915, -0.1171, 2.7849) -- (2.1454, -0.1171, 2.7892) -- cycle;
\fill[blue!16.2, opacity=0.5] (2.1454, -0.1171, 2.7892) -- (2.1915, -0.1171, 2.7849) -- (2.1910, -0.1180, 2.8349) -- (2.1450, -0.1180, 2.8392) -- cycle;
\fill[blue!16.4, opacity=0.5] (2.1450, -0.1180, 2.8392) -- (2.1910, -0.1180, 2.8349) -- (2.1907, -0.1187, 2.8849) -- (2.1446, -0.1187, 2.8892) -- cycle;
\fill[blue!16.7, opacity=0.5] (2.1446, -0.1187, 2.8892) -- (2.1907, -0.1187, 2.8849) -- (2.1904, -0.1193, 2.9349) -- (2.1443, -0.1193, 2.9392) -- cycle;
\fill[blue!17.0, opacity=0.5] (2.1443, -0.1193, 2.9392) -- (2.1904, -0.1193, 2.9349) -- (2.1902, -0.1197, 2.9849) -- (2.1442, -0.1197, 2.9892) -- cycle;
\fill[blue!17.3, opacity=0.5] (2.1442, -0.1197, 2.9892) -- (2.1902, -0.1197, 2.9849) -- (2.1900, -0.1199, 3.0349) -- (2.1440, -0.1199, 3.0392) -- cycle;
\fill[blue!17.7, opacity=0.5] (2.1440, -0.1199, 3.0392) -- (2.1900, -0.1199, 3.0349) -- (2.1900, -0.1200, 3.0849) -- (2.1440, -0.1200, 3.0892) -- cycle;
\fill[blue!15.0, opacity=0.5] (2.2500, -0.0000, 0.0849) -- (2.3000, -0.0000, 0.0803) -- (2.3000, -0.0001, 0.1303) -- (2.2500, -0.0001, 0.1349) -- cycle;
\fill[blue!15.0, opacity=0.5] (2.2500, -0.0001, 0.1349) -- (2.3000, -0.0001, 0.1303) -- (2.2998, -0.0003, 0.1803) -- (2.2498, -0.0003, 0.1849) -- cycle;
\fill[blue!15.0, opacity=0.5] (2.2498, -0.0003, 0.1849) -- (2.2998, -0.0003, 0.1803) -- (2.2996, -0.0007, 0.2303) -- (2.2496, -0.0007, 0.2349) -- cycle;
\fill[blue!15.0, opacity=0.5] (2.2496, -0.0007, 0.2349) -- (2.2996, -0.0007, 0.2303) -- (2.2993, -0.0013, 0.2803) -- (2.2493, -0.0013, 0.2849) -- cycle;
\fill[blue!15.0, opacity=0.5] (2.2493, -0.0013, 0.2849) -- (2.2993, -0.0013, 0.2803) -- (2.2989, -0.0020, 0.3303) -- (2.2490, -0.0020, 0.3349) -- cycle;
\fill[blue!15.0, opacity=0.5] (2.2490, -0.0020, 0.3349) -- (2.2989, -0.0020, 0.3303) -- (2.2984, -0.0029, 0.3803) -- (2.2485, -0.0029, 0.3849) -- cycle;
\fill[blue!15.0, opacity=0.5] (2.2485, -0.0029, 0.3849) -- (2.2984, -0.0029, 0.3803) -- (2.2979, -0.0040, 0.4303) -- (2.2480, -0.0040, 0.4349) -- cycle;
\fill[blue!15.0, opacity=0.5] (2.2480, -0.0040, 0.4349) -- (2.2979, -0.0040, 0.4303) -- (2.2972, -0.0052, 0.4803) -- (2.2474, -0.0052, 0.4849) -- cycle;
\fill[blue!15.0, opacity=0.5] (2.2474, -0.0052, 0.4849) -- (2.2972, -0.0052, 0.4803) -- (2.2965, -0.0065, 0.5303) -- (2.2467, -0.0065, 0.5349) -- cycle;
\fill[blue!15.0, opacity=0.5] (2.2467, -0.0065, 0.5349) -- (2.2965, -0.0065, 0.5303) -- (2.2957, -0.0080, 0.5803) -- (2.2460, -0.0080, 0.5849) -- cycle;
\fill[blue!15.0, opacity=0.5] (2.2460, -0.0080, 0.5849) -- (2.2957, -0.0080, 0.5803) -- (2.2948, -0.0097, 0.6303) -- (2.2452, -0.0097, 0.6349) -- cycle;
\fill[blue!15.0, opacity=0.5] (2.2452, -0.0097, 0.6349) -- (2.2948, -0.0097, 0.6303) -- (2.2939, -0.0115, 0.6803) -- (2.2443, -0.0115, 0.6849) -- cycle;
\fill[blue!15.0, opacity=0.5] (2.2443, -0.0115, 0.6849) -- (2.2939, -0.0115, 0.6803) -- (2.2929, -0.0134, 0.7303) -- (2.2433, -0.0134, 0.7349) -- cycle;
\fill[blue!15.0, opacity=0.5] (2.2433, -0.0134, 0.7349) -- (2.2929, -0.0134, 0.7303) -- (2.2918, -0.0154, 0.7803) -- (2.2423, -0.0154, 0.7849) -- cycle;
\fill[blue!15.0, opacity=0.5] (2.2423, -0.0154, 0.7849) -- (2.2918, -0.0154, 0.7803) -- (2.2906, -0.0176, 0.8303) -- (2.2412, -0.0176, 0.8349) -- cycle;
\fill[blue!15.0, opacity=0.5] (2.2412, -0.0176, 0.8349) -- (2.2906, -0.0176, 0.8303) -- (2.2894, -0.0199, 0.8803) -- (2.2401, -0.0199, 0.8849) -- cycle;
\fill[blue!15.0, opacity=0.5] (2.2401, -0.0199, 0.8849) -- (2.2894, -0.0199, 0.8803) -- (2.2881, -0.0222, 0.9303) -- (2.2389, -0.0222, 0.9349) -- cycle;
\fill[blue!15.0, opacity=0.5] (2.2389, -0.0222, 0.9349) -- (2.2881, -0.0222, 0.9303) -- (2.2868, -0.0247, 0.9803) -- (2.2376, -0.0247, 0.9849) -- cycle;
\fill[blue!15.0, opacity=0.5] (2.2376, -0.0247, 0.9849) -- (2.2868, -0.0247, 0.9803) -- (2.2854, -0.0273, 1.0303) -- (2.2363, -0.0273, 1.0349) -- cycle;
\fill[blue!15.0, opacity=0.5] (2.2363, -0.0273, 1.0349) -- (2.2854, -0.0273, 1.0303) -- (2.2840, -0.0300, 1.0803) -- (2.2350, -0.0300, 1.0849) -- cycle;
\fill[blue!15.0, opacity=0.5] (2.2350, -0.0300, 1.0849) -- (2.2840, -0.0300, 1.0803) -- (2.2825, -0.0328, 1.1303) -- (2.2336, -0.0328, 1.1349) -- cycle;
\fill[blue!15.0, opacity=0.5] (2.2336, -0.0328, 1.1349) -- (2.2825, -0.0328, 1.1303) -- (2.2810, -0.0356, 1.1803) -- (2.2322, -0.0356, 1.1849) -- cycle;
\fill[blue!15.0, opacity=0.5] (2.2322, -0.0356, 1.1849) -- (2.2810, -0.0356, 1.1803) -- (2.2795, -0.0385, 1.2303) -- (2.2308, -0.0385, 1.2349) -- cycle;
\fill[blue!15.0, opacity=0.5] (2.2308, -0.0385, 1.2349) -- (2.2795, -0.0385, 1.2303) -- (2.2779, -0.0415, 1.2803) -- (2.2293, -0.0415, 1.2849) -- cycle;
\fill[blue!15.0, opacity=0.5] (2.2293, -0.0415, 1.2849) -- (2.2779, -0.0415, 1.2803) -- (2.2763, -0.0445, 1.3303) -- (2.2278, -0.0445, 1.3349) -- cycle;
\fill[blue!15.0, opacity=0.5] (2.2278, -0.0445, 1.3349) -- (2.2763, -0.0445, 1.3303) -- (2.2747, -0.0475, 1.3803) -- (2.2262, -0.0475, 1.3849) -- cycle;
\fill[blue!15.0, opacity=0.5] (2.2262, -0.0475, 1.3849) -- (2.2747, -0.0475, 1.3803) -- (2.2730, -0.0506, 1.4303) -- (2.2247, -0.0506, 1.4349) -- cycle;
\fill[blue!15.0, opacity=0.5] (2.2247, -0.0506, 1.4349) -- (2.2730, -0.0506, 1.4303) -- (2.2713, -0.0537, 1.4803) -- (2.2231, -0.0537, 1.4849) -- cycle;
\fill[blue!15.0, opacity=0.5] (2.2231, -0.0537, 1.4849) -- (2.2713, -0.0537, 1.4803) -- (2.2697, -0.0569, 1.5303) -- (2.2216, -0.0569, 1.5349) -- cycle;
\fill[blue!15.0, opacity=0.5] (2.2216, -0.0569, 1.5349) -- (2.2697, -0.0569, 1.5303) -- (2.2680, -0.0600, 1.5803) -- (2.2200, -0.0600, 1.5849) -- cycle;
\fill[blue!15.0, opacity=0.5] (2.2200, -0.0600, 1.5849) -- (2.2680, -0.0600, 1.5803) -- (2.2663, -0.0631, 1.6303) -- (2.2184, -0.0631, 1.6349) -- cycle;
\fill[blue!15.0, opacity=0.5] (2.2184, -0.0631, 1.6349) -- (2.2663, -0.0631, 1.6303) -- (2.2647, -0.0663, 1.6803) -- (2.2169, -0.0663, 1.6849) -- cycle;
\fill[blue!15.0, opacity=0.5] (2.2169, -0.0663, 1.6849) -- (2.2647, -0.0663, 1.6803) -- (2.2630, -0.0694, 1.7303) -- (2.2153, -0.0694, 1.7349) -- cycle;
\fill[blue!15.0, opacity=0.5] (2.2153, -0.0694, 1.7349) -- (2.2630, -0.0694, 1.7303) -- (2.2613, -0.0725, 1.7803) -- (2.2138, -0.0725, 1.7849) -- cycle;
\fill[blue!15.0, opacity=0.5] (2.2138, -0.0725, 1.7849) -- (2.2613, -0.0725, 1.7803) -- (2.2597, -0.0755, 1.8303) -- (2.2122, -0.0755, 1.8349) -- cycle;
\fill[blue!15.0, opacity=0.5] (2.2122, -0.0755, 1.8349) -- (2.2597, -0.0755, 1.8303) -- (2.2581, -0.0785, 1.8803) -- (2.2107, -0.0785, 1.8849) -- cycle;
\fill[blue!15.0, opacity=0.5] (2.2107, -0.0785, 1.8849) -- (2.2581, -0.0785, 1.8803) -- (2.2565, -0.0815, 1.9303) -- (2.2092, -0.0815, 1.9349) -- cycle;
\fill[blue!15.0, opacity=0.5] (2.2092, -0.0815, 1.9349) -- (2.2565, -0.0815, 1.9303) -- (2.2550, -0.0844, 1.9803) -- (2.2078, -0.0844, 1.9849) -- cycle;
\fill[blue!15.0, opacity=0.5] (2.2078, -0.0844, 1.9849) -- (2.2550, -0.0844, 1.9803) -- (2.2535, -0.0872, 2.0303) -- (2.2064, -0.0872, 2.0349) -- cycle;
\fill[blue!15.0, opacity=0.5] (2.2064, -0.0872, 2.0349) -- (2.2535, -0.0872, 2.0303) -- (2.2520, -0.0900, 2.0803) -- (2.2050, -0.0900, 2.0849) -- cycle;
\fill[blue!15.0, opacity=0.5] (2.2050, -0.0900, 2.0849) -- (2.2520, -0.0900, 2.0803) -- (2.2506, -0.0927, 2.1303) -- (2.2037, -0.0927, 2.1349) -- cycle;
\fill[blue!15.0, opacity=0.5] (2.2037, -0.0927, 2.1349) -- (2.2506, -0.0927, 2.1303) -- (2.2492, -0.0953, 2.1803) -- (2.2024, -0.0953, 2.1849) -- cycle;
\fill[blue!15.0, opacity=0.5] (2.2024, -0.0953, 2.1849) -- (2.2492, -0.0953, 2.1803) -- (2.2479, -0.0978, 2.2303) -- (2.2011, -0.0978, 2.2349) -- cycle;
\fill[blue!15.0, opacity=0.5] (2.2011, -0.0978, 2.2349) -- (2.2479, -0.0978, 2.2303) -- (2.2466, -0.1001, 2.2803) -- (2.1999, -0.1001, 2.2849) -- cycle;
\fill[blue!15.0, opacity=0.5] (2.1999, -0.1001, 2.2849) -- (2.2466, -0.1001, 2.2803) -- (2.2454, -0.1024, 2.3303) -- (2.1988, -0.1024, 2.3349) -- cycle;
\fill[blue!15.0, opacity=0.5] (2.1988, -0.1024, 2.3349) -- (2.2454, -0.1024, 2.3303) -- (2.2442, -0.1046, 2.3803) -- (2.1977, -0.1046, 2.3849) -- cycle;
\fill[blue!15.0, opacity=0.5] (2.1977, -0.1046, 2.3849) -- (2.2442, -0.1046, 2.3803) -- (2.2431, -0.1066, 2.4303) -- (2.1967, -0.1066, 2.4349) -- cycle;
\fill[blue!15.1, opacity=0.5] (2.1967, -0.1066, 2.4349) -- (2.2431, -0.1066, 2.4303) -- (2.2421, -0.1085, 2.4803) -- (2.1957, -0.1085, 2.4849) -- cycle;
\fill[blue!15.1, opacity=0.5] (2.1957, -0.1085, 2.4849) -- (2.2421, -0.1085, 2.4803) -- (2.2412, -0.1103, 2.5303) -- (2.1948, -0.1103, 2.5349) -- cycle;
\fill[blue!15.1, opacity=0.5] (2.1948, -0.1103, 2.5349) -- (2.2412, -0.1103, 2.5303) -- (2.2403, -0.1120, 2.5803) -- (2.1940, -0.1120, 2.5849) -- cycle;
\fill[blue!15.2, opacity=0.5] (2.1940, -0.1120, 2.5849) -- (2.2403, -0.1120, 2.5803) -- (2.2395, -0.1135, 2.6303) -- (2.1933, -0.1135, 2.6349) -- cycle;
\fill[blue!15.2, opacity=0.5] (2.1933, -0.1135, 2.6349) -- (2.2395, -0.1135, 2.6303) -- (2.2388, -0.1148, 2.6803) -- (2.1926, -0.1148, 2.6849) -- cycle;
\fill[blue!15.3, opacity=0.5] (2.1926, -0.1148, 2.6849) -- (2.2388, -0.1148, 2.6803) -- (2.2381, -0.1160, 2.7303) -- (2.1920, -0.1160, 2.7349) -- cycle;
\fill[blue!15.4, opacity=0.5] (2.1920, -0.1160, 2.7349) -- (2.2381, -0.1160, 2.7303) -- (2.2376, -0.1171, 2.7803) -- (2.1915, -0.1171, 2.7849) -- cycle;
\fill[blue!15.4, opacity=0.5] (2.1915, -0.1171, 2.7849) -- (2.2376, -0.1171, 2.7803) -- (2.2371, -0.1180, 2.8303) -- (2.1910, -0.1180, 2.8349) -- cycle;
\fill[blue!15.6, opacity=0.5] (2.1910, -0.1180, 2.8349) -- (2.2371, -0.1180, 2.8303) -- (2.2367, -0.1187, 2.8803) -- (2.1907, -0.1187, 2.8849) -- cycle;
\fill[blue!15.7, opacity=0.5] (2.1907, -0.1187, 2.8849) -- (2.2367, -0.1187, 2.8803) -- (2.2364, -0.1193, 2.9303) -- (2.1904, -0.1193, 2.9349) -- cycle;
\fill[blue!15.8, opacity=0.5] (2.1904, -0.1193, 2.9349) -- (2.2364, -0.1193, 2.9303) -- (2.2362, -0.1197, 2.9803) -- (2.1902, -0.1197, 2.9849) -- cycle;
\fill[blue!16.0, opacity=0.5] (2.1902, -0.1197, 2.9849) -- (2.2362, -0.1197, 2.9803) -- (2.2360, -0.1199, 3.0303) -- (2.1900, -0.1199, 3.0349) -- cycle;
\fill[blue!16.2, opacity=0.5] (2.1900, -0.1199, 3.0349) -- (2.2360, -0.1199, 3.0303) -- (2.2360, -0.1200, 3.0803) -- (2.1900, -0.1200, 3.0849) -- cycle;
\fill[blue!15.0, opacity=0.5] (2.3000, -0.0000, 0.0803) -- (2.3500, -0.0000, 0.0755) -- (2.3500, -0.0001, 0.1255) -- (2.3000, -0.0001, 0.1303) -- cycle;
\fill[blue!15.0, opacity=0.5] (2.3000, -0.0001, 0.1303) -- (2.3500, -0.0001, 0.1255) -- (2.3498, -0.0003, 0.1755) -- (2.2998, -0.0003, 0.1803) -- cycle;
\fill[blue!15.0, opacity=0.5] (2.2998, -0.0003, 0.1803) -- (2.3498, -0.0003, 0.1755) -- (2.3496, -0.0007, 0.2255) -- (2.2996, -0.0007, 0.2303) -- cycle;
\fill[blue!15.0, opacity=0.5] (2.2996, -0.0007, 0.2303) -- (2.3496, -0.0007, 0.2255) -- (2.3493, -0.0013, 0.2755) -- (2.2993, -0.0013, 0.2803) -- cycle;
\fill[blue!15.0, opacity=0.5] (2.2993, -0.0013, 0.2803) -- (2.3493, -0.0013, 0.2755) -- (2.3488, -0.0020, 0.3255) -- (2.2989, -0.0020, 0.3303) -- cycle;
\fill[blue!15.0, opacity=0.5] (2.2989, -0.0020, 0.3303) -- (2.3488, -0.0020, 0.3255) -- (2.3483, -0.0029, 0.3755) -- (2.2984, -0.0029, 0.3803) -- cycle;
\fill[blue!15.0, opacity=0.5] (2.2984, -0.0029, 0.3803) -- (2.3483, -0.0029, 0.3755) -- (2.3477, -0.0040, 0.4255) -- (2.2979, -0.0040, 0.4303) -- cycle;
\fill[blue!15.0, opacity=0.5] (2.2979, -0.0040, 0.4303) -- (2.3477, -0.0040, 0.4255) -- (2.3471, -0.0052, 0.4755) -- (2.2972, -0.0052, 0.4803) -- cycle;
\fill[blue!15.0, opacity=0.5] (2.2972, -0.0052, 0.4803) -- (2.3471, -0.0052, 0.4755) -- (2.3463, -0.0065, 0.5255) -- (2.2965, -0.0065, 0.5303) -- cycle;
\fill[blue!15.0, opacity=0.5] (2.2965, -0.0065, 0.5303) -- (2.3463, -0.0065, 0.5255) -- (2.3454, -0.0080, 0.5755) -- (2.2957, -0.0080, 0.5803) -- cycle;
\fill[blue!15.0, opacity=0.5] (2.2957, -0.0080, 0.5803) -- (2.3454, -0.0080, 0.5755) -- (2.3445, -0.0097, 0.6255) -- (2.2948, -0.0097, 0.6303) -- cycle;
\fill[blue!15.0, opacity=0.5] (2.2948, -0.0097, 0.6303) -- (2.3445, -0.0097, 0.6255) -- (2.3435, -0.0115, 0.6755) -- (2.2939, -0.0115, 0.6803) -- cycle;
\fill[blue!15.0, opacity=0.5] (2.2939, -0.0115, 0.6803) -- (2.3435, -0.0115, 0.6755) -- (2.3424, -0.0134, 0.7255) -- (2.2929, -0.0134, 0.7303) -- cycle;
\fill[blue!15.0, opacity=0.5] (2.2929, -0.0134, 0.7303) -- (2.3424, -0.0134, 0.7255) -- (2.3413, -0.0154, 0.7755) -- (2.2918, -0.0154, 0.7803) -- cycle;
\fill[blue!15.0, opacity=0.5] (2.2918, -0.0154, 0.7803) -- (2.3413, -0.0154, 0.7755) -- (2.3400, -0.0176, 0.8255) -- (2.2906, -0.0176, 0.8303) -- cycle;
\fill[blue!15.0, opacity=0.5] (2.2906, -0.0176, 0.8303) -- (2.3400, -0.0176, 0.8255) -- (2.3388, -0.0199, 0.8755) -- (2.2894, -0.0199, 0.8803) -- cycle;
\fill[blue!15.0, opacity=0.5] (2.2894, -0.0199, 0.8803) -- (2.3388, -0.0199, 0.8755) -- (2.3374, -0.0222, 0.9255) -- (2.2881, -0.0222, 0.9303) -- cycle;
\fill[blue!15.0, opacity=0.5] (2.2881, -0.0222, 0.9303) -- (2.3374, -0.0222, 0.9255) -- (2.3360, -0.0247, 0.9755) -- (2.2868, -0.0247, 0.9803) -- cycle;
\fill[blue!15.0, opacity=0.5] (2.2868, -0.0247, 0.9803) -- (2.3360, -0.0247, 0.9755) -- (2.3345, -0.0273, 1.0255) -- (2.2854, -0.0273, 1.0303) -- cycle;
\fill[blue!15.0, opacity=0.5] (2.2854, -0.0273, 1.0303) -- (2.3345, -0.0273, 1.0255) -- (2.3330, -0.0300, 1.0755) -- (2.2840, -0.0300, 1.0803) -- cycle;
\fill[blue!15.0, opacity=0.5] (2.2840, -0.0300, 1.0803) -- (2.3330, -0.0300, 1.0755) -- (2.3314, -0.0328, 1.1255) -- (2.2825, -0.0328, 1.1303) -- cycle;
\fill[blue!15.0, opacity=0.5] (2.2825, -0.0328, 1.1303) -- (2.3314, -0.0328, 1.1255) -- (2.3298, -0.0356, 1.1755) -- (2.2810, -0.0356, 1.1803) -- cycle;
\fill[blue!15.0, opacity=0.5] (2.2810, -0.0356, 1.1803) -- (2.3298, -0.0356, 1.1755) -- (2.3282, -0.0385, 1.2255) -- (2.2795, -0.0385, 1.2303) -- cycle;
\fill[blue!15.0, opacity=0.5] (2.2795, -0.0385, 1.2303) -- (2.3282, -0.0385, 1.2255) -- (2.3265, -0.0415, 1.2755) -- (2.2779, -0.0415, 1.2803) -- cycle;
\fill[blue!15.0, opacity=0.5] (2.2779, -0.0415, 1.2803) -- (2.3265, -0.0415, 1.2755) -- (2.3248, -0.0445, 1.3255) -- (2.2763, -0.0445, 1.3303) -- cycle;
\fill[blue!15.0, opacity=0.5] (2.2763, -0.0445, 1.3303) -- (2.3248, -0.0445, 1.3255) -- (2.3231, -0.0475, 1.3755) -- (2.2747, -0.0475, 1.3803) -- cycle;
\fill[blue!15.0, opacity=0.5] (2.2747, -0.0475, 1.3803) -- (2.3231, -0.0475, 1.3755) -- (2.3213, -0.0506, 1.4255) -- (2.2730, -0.0506, 1.4303) -- cycle;
\fill[blue!15.0, opacity=0.5] (2.2730, -0.0506, 1.4303) -- (2.3213, -0.0506, 1.4255) -- (2.3196, -0.0537, 1.4755) -- (2.2713, -0.0537, 1.4803) -- cycle;
\fill[blue!15.0, opacity=0.5] (2.2713, -0.0537, 1.4803) -- (2.3196, -0.0537, 1.4755) -- (2.3178, -0.0569, 1.5255) -- (2.2697, -0.0569, 1.5303) -- cycle;
\fill[blue!15.0, opacity=0.5] (2.2697, -0.0569, 1.5303) -- (2.3178, -0.0569, 1.5255) -- (2.3160, -0.0600, 1.5755) -- (2.2680, -0.0600, 1.5803) -- cycle;
\fill[blue!15.0, opacity=0.5] (2.2680, -0.0600, 1.5803) -- (2.3160, -0.0600, 1.5755) -- (2.3142, -0.0631, 1.6255) -- (2.2663, -0.0631, 1.6303) -- cycle;
\fill[blue!15.0, opacity=0.5] (2.2663, -0.0631, 1.6303) -- (2.3142, -0.0631, 1.6255) -- (2.3124, -0.0663, 1.6755) -- (2.2647, -0.0663, 1.6803) -- cycle;
\fill[blue!15.0, opacity=0.5] (2.2647, -0.0663, 1.6803) -- (2.3124, -0.0663, 1.6755) -- (2.3107, -0.0694, 1.7255) -- (2.2630, -0.0694, 1.7303) -- cycle;
\fill[blue!15.0, opacity=0.5] (2.2630, -0.0694, 1.7303) -- (2.3107, -0.0694, 1.7255) -- (2.3089, -0.0725, 1.7755) -- (2.2613, -0.0725, 1.7803) -- cycle;
\fill[blue!15.0, opacity=0.5] (2.2613, -0.0725, 1.7803) -- (2.3089, -0.0725, 1.7755) -- (2.3072, -0.0755, 1.8255) -- (2.2597, -0.0755, 1.8303) -- cycle;
\fill[blue!15.0, opacity=0.5] (2.2597, -0.0755, 1.8303) -- (2.3072, -0.0755, 1.8255) -- (2.3055, -0.0785, 1.8755) -- (2.2581, -0.0785, 1.8803) -- cycle;
\fill[blue!15.0, opacity=0.5] (2.2581, -0.0785, 1.8803) -- (2.3055, -0.0785, 1.8755) -- (2.3038, -0.0815, 1.9255) -- (2.2565, -0.0815, 1.9303) -- cycle;
\fill[blue!15.0, opacity=0.5] (2.2565, -0.0815, 1.9303) -- (2.3038, -0.0815, 1.9255) -- (2.3022, -0.0844, 1.9755) -- (2.2550, -0.0844, 1.9803) -- cycle;
\fill[blue!15.0, opacity=0.5] (2.2550, -0.0844, 1.9803) -- (2.3022, -0.0844, 1.9755) -- (2.3006, -0.0872, 2.0255) -- (2.2535, -0.0872, 2.0303) -- cycle;
\fill[blue!15.0, opacity=0.5] (2.2535, -0.0872, 2.0303) -- (2.3006, -0.0872, 2.0255) -- (2.2990, -0.0900, 2.0755) -- (2.2520, -0.0900, 2.0803) -- cycle;
\fill[blue!15.0, opacity=0.5] (2.2520, -0.0900, 2.0803) -- (2.2990, -0.0900, 2.0755) -- (2.2975, -0.0927, 2.1255) -- (2.2506, -0.0927, 2.1303) -- cycle;
\fill[blue!15.0, opacity=0.5] (2.2506, -0.0927, 2.1303) -- (2.2975, -0.0927, 2.1255) -- (2.2960, -0.0953, 2.1755) -- (2.2492, -0.0953, 2.1803) -- cycle;
\fill[blue!15.0, opacity=0.5] (2.2492, -0.0953, 2.1803) -- (2.2960, -0.0953, 2.1755) -- (2.2946, -0.0978, 2.2255) -- (2.2479, -0.0978, 2.2303) -- cycle;
\fill[blue!15.0, opacity=0.5] (2.2479, -0.0978, 2.2303) -- (2.2946, -0.0978, 2.2255) -- (2.2932, -0.1001, 2.2755) -- (2.2466, -0.1001, 2.2803) -- cycle;
\fill[blue!15.0, opacity=0.5] (2.2466, -0.1001, 2.2803) -- (2.2932, -0.1001, 2.2755) -- (2.2920, -0.1024, 2.3255) -- (2.2454, -0.1024, 2.3303) -- cycle;
\fill[blue!15.0, opacity=0.5] (2.2454, -0.1024, 2.3303) -- (2.2920, -0.1024, 2.3255) -- (2.2907, -0.1046, 2.3755) -- (2.2442, -0.1046, 2.3803) -- cycle;
\fill[blue!15.0, opacity=0.5] (2.2442, -0.1046, 2.3803) -- (2.2907, -0.1046, 2.3755) -- (2.2896, -0.1066, 2.4255) -- (2.2431, -0.1066, 2.4303) -- cycle;
\fill[blue!15.0, opacity=0.5] (2.2431, -0.1066, 2.4303) -- (2.2896, -0.1066, 2.4255) -- (2.2885, -0.1085, 2.4755) -- (2.2421, -0.1085, 2.4803) -- cycle;
\fill[blue!15.0, opacity=0.5] (2.2421, -0.1085, 2.4803) -- (2.2885, -0.1085, 2.4755) -- (2.2875, -0.1103, 2.5255) -- (2.2412, -0.1103, 2.5303) -- cycle;
\fill[blue!15.1, opacity=0.5] (2.2412, -0.1103, 2.5303) -- (2.2875, -0.1103, 2.5255) -- (2.2866, -0.1120, 2.5755) -- (2.2403, -0.1120, 2.5803) -- cycle;
\fill[blue!15.1, opacity=0.5] (2.2403, -0.1120, 2.5803) -- (2.2866, -0.1120, 2.5755) -- (2.2857, -0.1135, 2.6255) -- (2.2395, -0.1135, 2.6303) -- cycle;
\fill[blue!15.1, opacity=0.5] (2.2395, -0.1135, 2.6303) -- (2.2857, -0.1135, 2.6255) -- (2.2849, -0.1148, 2.6755) -- (2.2388, -0.1148, 2.6803) -- cycle;
\fill[blue!15.2, opacity=0.5] (2.2388, -0.1148, 2.6803) -- (2.2849, -0.1148, 2.6755) -- (2.2843, -0.1160, 2.7255) -- (2.2381, -0.1160, 2.7303) -- cycle;
\fill[blue!15.2, opacity=0.5] (2.2381, -0.1160, 2.7303) -- (2.2843, -0.1160, 2.7255) -- (2.2837, -0.1171, 2.7755) -- (2.2376, -0.1171, 2.7803) -- cycle;
\fill[blue!15.3, opacity=0.5] (2.2376, -0.1171, 2.7803) -- (2.2837, -0.1171, 2.7755) -- (2.2832, -0.1180, 2.8255) -- (2.2371, -0.1180, 2.8303) -- cycle;
\fill[blue!15.3, opacity=0.5] (2.2371, -0.1180, 2.8303) -- (2.2832, -0.1180, 2.8255) -- (2.2827, -0.1187, 2.8755) -- (2.2367, -0.1187, 2.8803) -- cycle;
\fill[blue!15.4, opacity=0.5] (2.2367, -0.1187, 2.8803) -- (2.2827, -0.1187, 2.8755) -- (2.2824, -0.1193, 2.9255) -- (2.2364, -0.1193, 2.9303) -- cycle;
\fill[blue!15.5, opacity=0.5] (2.2364, -0.1193, 2.9303) -- (2.2824, -0.1193, 2.9255) -- (2.2822, -0.1197, 2.9755) -- (2.2362, -0.1197, 2.9803) -- cycle;
\fill[blue!15.6, opacity=0.5] (2.2362, -0.1197, 2.9803) -- (2.2822, -0.1197, 2.9755) -- (2.2820, -0.1199, 3.0255) -- (2.2360, -0.1199, 3.0303) -- cycle;
\fill[blue!15.8, opacity=0.5] (2.2360, -0.1199, 3.0303) -- (2.2820, -0.1199, 3.0255) -- (2.2820, -0.1200, 3.0755) -- (2.2360, -0.1200, 3.0803) -- cycle;
\fill[blue!15.0, opacity=0.5] (2.3500, -0.0000, 0.0755) -- (2.4000, -0.0000, 0.0705) -- (2.4000, -0.0001, 0.1205) -- (2.3500, -0.0001, 0.1255) -- cycle;
\fill[blue!15.0, opacity=0.5] (2.3500, -0.0001, 0.1255) -- (2.4000, -0.0001, 0.1205) -- (2.3998, -0.0003, 0.1705) -- (2.3498, -0.0003, 0.1755) -- cycle;
\fill[blue!15.0, opacity=0.5] (2.3498, -0.0003, 0.1755) -- (2.3998, -0.0003, 0.1705) -- (2.3996, -0.0007, 0.2205) -- (2.3496, -0.0007, 0.2255) -- cycle;
\fill[blue!15.0, opacity=0.5] (2.3496, -0.0007, 0.2255) -- (2.3996, -0.0007, 0.2205) -- (2.3992, -0.0013, 0.2705) -- (2.3493, -0.0013, 0.2755) -- cycle;
\fill[blue!15.0, opacity=0.5] (2.3493, -0.0013, 0.2755) -- (2.3992, -0.0013, 0.2705) -- (2.3988, -0.0020, 0.3205) -- (2.3488, -0.0020, 0.3255) -- cycle;
\fill[blue!15.0, opacity=0.5] (2.3488, -0.0020, 0.3255) -- (2.3988, -0.0020, 0.3205) -- (2.3982, -0.0029, 0.3705) -- (2.3483, -0.0029, 0.3755) -- cycle;
\fill[blue!15.0, opacity=0.5] (2.3483, -0.0029, 0.3755) -- (2.3982, -0.0029, 0.3705) -- (2.3976, -0.0040, 0.4205) -- (2.3477, -0.0040, 0.4255) -- cycle;
\fill[blue!15.0, opacity=0.5] (2.3477, -0.0040, 0.4255) -- (2.3976, -0.0040, 0.4205) -- (2.3969, -0.0052, 0.4705) -- (2.3471, -0.0052, 0.4755) -- cycle;
\fill[blue!15.0, opacity=0.5] (2.3471, -0.0052, 0.4755) -- (2.3969, -0.0052, 0.4705) -- (2.3961, -0.0065, 0.5205) -- (2.3463, -0.0065, 0.5255) -- cycle;
\fill[blue!15.0, opacity=0.5] (2.3463, -0.0065, 0.5255) -- (2.3961, -0.0065, 0.5205) -- (2.3952, -0.0080, 0.5705) -- (2.3454, -0.0080, 0.5755) -- cycle;
\fill[blue!15.0, opacity=0.5] (2.3454, -0.0080, 0.5755) -- (2.3952, -0.0080, 0.5705) -- (2.3942, -0.0097, 0.6205) -- (2.3445, -0.0097, 0.6255) -- cycle;
\fill[blue!15.0, opacity=0.5] (2.3445, -0.0097, 0.6255) -- (2.3942, -0.0097, 0.6205) -- (2.3931, -0.0115, 0.6705) -- (2.3435, -0.0115, 0.6755) -- cycle;
\fill[blue!15.0, opacity=0.5] (2.3435, -0.0115, 0.6755) -- (2.3931, -0.0115, 0.6705) -- (2.3920, -0.0134, 0.7205) -- (2.3424, -0.0134, 0.7255) -- cycle;
\fill[blue!15.0, opacity=0.5] (2.3424, -0.0134, 0.7255) -- (2.3920, -0.0134, 0.7205) -- (2.3908, -0.0154, 0.7705) -- (2.3413, -0.0154, 0.7755) -- cycle;
\fill[blue!15.0, opacity=0.5] (2.3413, -0.0154, 0.7755) -- (2.3908, -0.0154, 0.7705) -- (2.3895, -0.0176, 0.8205) -- (2.3400, -0.0176, 0.8255) -- cycle;
\fill[blue!15.0, opacity=0.5] (2.3400, -0.0176, 0.8255) -- (2.3895, -0.0176, 0.8205) -- (2.3881, -0.0199, 0.8705) -- (2.3388, -0.0199, 0.8755) -- cycle;
\fill[blue!15.0, opacity=0.5] (2.3388, -0.0199, 0.8755) -- (2.3881, -0.0199, 0.8705) -- (2.3867, -0.0222, 0.9205) -- (2.3374, -0.0222, 0.9255) -- cycle;
\fill[blue!15.0, opacity=0.5] (2.3374, -0.0222, 0.9255) -- (2.3867, -0.0222, 0.9205) -- (2.3852, -0.0247, 0.9705) -- (2.3360, -0.0247, 0.9755) -- cycle;
\fill[blue!15.0, opacity=0.5] (2.3360, -0.0247, 0.9755) -- (2.3852, -0.0247, 0.9705) -- (2.3836, -0.0273, 1.0205) -- (2.3345, -0.0273, 1.0255) -- cycle;
\fill[blue!15.0, opacity=0.5] (2.3345, -0.0273, 1.0255) -- (2.3836, -0.0273, 1.0205) -- (2.3820, -0.0300, 1.0705) -- (2.3330, -0.0300, 1.0755) -- cycle;
\fill[blue!15.0, opacity=0.5] (2.3330, -0.0300, 1.0755) -- (2.3820, -0.0300, 1.0705) -- (2.3803, -0.0328, 1.1205) -- (2.3314, -0.0328, 1.1255) -- cycle;
\fill[blue!15.0, opacity=0.5] (2.3314, -0.0328, 1.1255) -- (2.3803, -0.0328, 1.1205) -- (2.3786, -0.0356, 1.1705) -- (2.3298, -0.0356, 1.1755) -- cycle;
\fill[blue!15.0, opacity=0.5] (2.3298, -0.0356, 1.1755) -- (2.3786, -0.0356, 1.1705) -- (2.3769, -0.0385, 1.2205) -- (2.3282, -0.0385, 1.2255) -- cycle;
\fill[blue!15.0, opacity=0.5] (2.3282, -0.0385, 1.2255) -- (2.3769, -0.0385, 1.2205) -- (2.3751, -0.0415, 1.2705) -- (2.3265, -0.0415, 1.2755) -- cycle;
\fill[blue!15.0, opacity=0.5] (2.3265, -0.0415, 1.2755) -- (2.3751, -0.0415, 1.2705) -- (2.3733, -0.0445, 1.3205) -- (2.3248, -0.0445, 1.3255) -- cycle;
\fill[blue!15.0, opacity=0.5] (2.3248, -0.0445, 1.3255) -- (2.3733, -0.0445, 1.3205) -- (2.3715, -0.0475, 1.3705) -- (2.3231, -0.0475, 1.3755) -- cycle;
\fill[blue!15.0, opacity=0.5] (2.3231, -0.0475, 1.3755) -- (2.3715, -0.0475, 1.3705) -- (2.3696, -0.0506, 1.4205) -- (2.3213, -0.0506, 1.4255) -- cycle;
\fill[blue!15.0, opacity=0.5] (2.3213, -0.0506, 1.4255) -- (2.3696, -0.0506, 1.4205) -- (2.3678, -0.0537, 1.4705) -- (2.3196, -0.0537, 1.4755) -- cycle;
\fill[blue!15.0, opacity=0.5] (2.3196, -0.0537, 1.4755) -- (2.3678, -0.0537, 1.4705) -- (2.3659, -0.0569, 1.5205) -- (2.3178, -0.0569, 1.5255) -- cycle;
\fill[blue!15.0, opacity=0.5] (2.3178, -0.0569, 1.5255) -- (2.3659, -0.0569, 1.5205) -- (2.3640, -0.0600, 1.5705) -- (2.3160, -0.0600, 1.5755) -- cycle;
\fill[blue!15.0, opacity=0.5] (2.3160, -0.0600, 1.5755) -- (2.3640, -0.0600, 1.5705) -- (2.3621, -0.0631, 1.6205) -- (2.3142, -0.0631, 1.6255) -- cycle;
\fill[blue!15.0, opacity=0.5] (2.3142, -0.0631, 1.6255) -- (2.3621, -0.0631, 1.6205) -- (2.3602, -0.0663, 1.6705) -- (2.3124, -0.0663, 1.6755) -- cycle;
\fill[blue!15.0, opacity=0.5] (2.3124, -0.0663, 1.6755) -- (2.3602, -0.0663, 1.6705) -- (2.3584, -0.0694, 1.7205) -- (2.3107, -0.0694, 1.7255) -- cycle;
\fill[blue!15.0, opacity=0.5] (2.3107, -0.0694, 1.7255) -- (2.3584, -0.0694, 1.7205) -- (2.3565, -0.0725, 1.7705) -- (2.3089, -0.0725, 1.7755) -- cycle;
\fill[blue!15.0, opacity=0.5] (2.3089, -0.0725, 1.7755) -- (2.3565, -0.0725, 1.7705) -- (2.3547, -0.0755, 1.8205) -- (2.3072, -0.0755, 1.8255) -- cycle;
\fill[blue!15.0, opacity=0.5] (2.3072, -0.0755, 1.8255) -- (2.3547, -0.0755, 1.8205) -- (2.3529, -0.0785, 1.8705) -- (2.3055, -0.0785, 1.8755) -- cycle;
\fill[blue!15.0, opacity=0.5] (2.3055, -0.0785, 1.8755) -- (2.3529, -0.0785, 1.8705) -- (2.3511, -0.0815, 1.9205) -- (2.3038, -0.0815, 1.9255) -- cycle;
\fill[blue!15.0, opacity=0.5] (2.3038, -0.0815, 1.9255) -- (2.3511, -0.0815, 1.9205) -- (2.3494, -0.0844, 1.9705) -- (2.3022, -0.0844, 1.9755) -- cycle;
\fill[blue!15.0, opacity=0.5] (2.3022, -0.0844, 1.9755) -- (2.3494, -0.0844, 1.9705) -- (2.3477, -0.0872, 2.0205) -- (2.3006, -0.0872, 2.0255) -- cycle;
\fill[blue!15.0, opacity=0.5] (2.3006, -0.0872, 2.0255) -- (2.3477, -0.0872, 2.0205) -- (2.3460, -0.0900, 2.0705) -- (2.2990, -0.0900, 2.0755) -- cycle;
\fill[blue!15.0, opacity=0.5] (2.2990, -0.0900, 2.0755) -- (2.3460, -0.0900, 2.0705) -- (2.3444, -0.0927, 2.1205) -- (2.2975, -0.0927, 2.1255) -- cycle;
\fill[blue!15.0, opacity=0.5] (2.2975, -0.0927, 2.1255) -- (2.3444, -0.0927, 2.1205) -- (2.3428, -0.0953, 2.1705) -- (2.2960, -0.0953, 2.1755) -- cycle;
\fill[blue!15.0, opacity=0.5] (2.2960, -0.0953, 2.1755) -- (2.3428, -0.0953, 2.1705) -- (2.3413, -0.0978, 2.2205) -- (2.2946, -0.0978, 2.2255) -- cycle;
\fill[blue!15.0, opacity=0.5] (2.2946, -0.0978, 2.2255) -- (2.3413, -0.0978, 2.2205) -- (2.3399, -0.1001, 2.2705) -- (2.2932, -0.1001, 2.2755) -- cycle;
\fill[blue!15.0, opacity=0.5] (2.2932, -0.1001, 2.2755) -- (2.3399, -0.1001, 2.2705) -- (2.3385, -0.1024, 2.3205) -- (2.2920, -0.1024, 2.3255) -- cycle;
\fill[blue!15.0, opacity=0.5] (2.2920, -0.1024, 2.3255) -- (2.3385, -0.1024, 2.3205) -- (2.3372, -0.1046, 2.3705) -- (2.2907, -0.1046, 2.3755) -- cycle;
\fill[blue!15.0, opacity=0.5] (2.2907, -0.1046, 2.3755) -- (2.3372, -0.1046, 2.3705) -- (2.3360, -0.1066, 2.4205) -- (2.2896, -0.1066, 2.4255) -- cycle;
\fill[blue!15.0, opacity=0.5] (2.2896, -0.1066, 2.4255) -- (2.3360, -0.1066, 2.4205) -- (2.3349, -0.1085, 2.4705) -- (2.2885, -0.1085, 2.4755) -- cycle;
\fill[blue!15.1, opacity=0.5] (2.2885, -0.1085, 2.4755) -- (2.3349, -0.1085, 2.4705) -- (2.3338, -0.1103, 2.5205) -- (2.2875, -0.1103, 2.5255) -- cycle;
\fill[blue!15.1, opacity=0.5] (2.2875, -0.1103, 2.5255) -- (2.3338, -0.1103, 2.5205) -- (2.3328, -0.1120, 2.5705) -- (2.2866, -0.1120, 2.5755) -- cycle;
\fill[blue!15.1, opacity=0.5] (2.2866, -0.1120, 2.5755) -- (2.3328, -0.1120, 2.5705) -- (2.3319, -0.1135, 2.6205) -- (2.2857, -0.1135, 2.6255) -- cycle;
\fill[blue!15.1, opacity=0.5] (2.2857, -0.1135, 2.6255) -- (2.3319, -0.1135, 2.6205) -- (2.3311, -0.1148, 2.6705) -- (2.2849, -0.1148, 2.6755) -- cycle;
\fill[blue!15.2, opacity=0.5] (2.2849, -0.1148, 2.6755) -- (2.3311, -0.1148, 2.6705) -- (2.3304, -0.1160, 2.7205) -- (2.2843, -0.1160, 2.7255) -- cycle;
\fill[blue!15.2, opacity=0.5] (2.2843, -0.1160, 2.7255) -- (2.3304, -0.1160, 2.7205) -- (2.3298, -0.1171, 2.7705) -- (2.2837, -0.1171, 2.7755) -- cycle;
\fill[blue!15.3, opacity=0.5] (2.2837, -0.1171, 2.7755) -- (2.3298, -0.1171, 2.7705) -- (2.3292, -0.1180, 2.8205) -- (2.2832, -0.1180, 2.8255) -- cycle;
\fill[blue!15.4, opacity=0.5] (2.2832, -0.1180, 2.8255) -- (2.3292, -0.1180, 2.8205) -- (2.3288, -0.1187, 2.8705) -- (2.2827, -0.1187, 2.8755) -- cycle;
\fill[blue!15.5, opacity=0.5] (2.2827, -0.1187, 2.8755) -- (2.3288, -0.1187, 2.8705) -- (2.3284, -0.1193, 2.9205) -- (2.2824, -0.1193, 2.9255) -- cycle;
\fill[blue!15.6, opacity=0.5] (2.2824, -0.1193, 2.9255) -- (2.3284, -0.1193, 2.9205) -- (2.3282, -0.1197, 2.9705) -- (2.2822, -0.1197, 2.9755) -- cycle;
\fill[blue!15.7, opacity=0.5] (2.2822, -0.1197, 2.9755) -- (2.3282, -0.1197, 2.9705) -- (2.3280, -0.1199, 3.0205) -- (2.2820, -0.1199, 3.0255) -- cycle;
\fill[blue!15.8, opacity=0.5] (2.2820, -0.1199, 3.0255) -- (2.3280, -0.1199, 3.0205) -- (2.3280, -0.1200, 3.0705) -- (2.2820, -0.1200, 3.0755) -- cycle;
\fill[blue!15.0, opacity=0.5] (2.4000, -0.0000, 0.0705) -- (2.4500, -0.0000, 0.0654) -- (2.4499, -0.0001, 0.1154) -- (2.4000, -0.0001, 0.1205) -- cycle;
\fill[blue!15.0, opacity=0.5] (2.4000, -0.0001, 0.1205) -- (2.4499, -0.0001, 0.1154) -- (2.4498, -0.0003, 0.1654) -- (2.3998, -0.0003, 0.1705) -- cycle;
\fill[blue!15.0, opacity=0.5] (2.3998, -0.0003, 0.1705) -- (2.4498, -0.0003, 0.1654) -- (2.4495, -0.0007, 0.2154) -- (2.3996, -0.0007, 0.2205) -- cycle;
\fill[blue!15.0, opacity=0.5] (2.3996, -0.0007, 0.2205) -- (2.4495, -0.0007, 0.2154) -- (2.4492, -0.0013, 0.2654) -- (2.3992, -0.0013, 0.2705) -- cycle;
\fill[blue!15.0, opacity=0.5] (2.3992, -0.0013, 0.2705) -- (2.4492, -0.0013, 0.2654) -- (2.4487, -0.0020, 0.3154) -- (2.3988, -0.0020, 0.3205) -- cycle;
\fill[blue!15.0, opacity=0.5] (2.3988, -0.0020, 0.3205) -- (2.4487, -0.0020, 0.3154) -- (2.4481, -0.0029, 0.3654) -- (2.3982, -0.0029, 0.3705) -- cycle;
\fill[blue!15.0, opacity=0.5] (2.3982, -0.0029, 0.3705) -- (2.4481, -0.0029, 0.3654) -- (2.4475, -0.0040, 0.4154) -- (2.3976, -0.0040, 0.4205) -- cycle;
\fill[blue!15.0, opacity=0.5] (2.3976, -0.0040, 0.4205) -- (2.4475, -0.0040, 0.4154) -- (2.4467, -0.0052, 0.4654) -- (2.3969, -0.0052, 0.4705) -- cycle;
\fill[blue!15.0, opacity=0.5] (2.3969, -0.0052, 0.4705) -- (2.4467, -0.0052, 0.4654) -- (2.4459, -0.0065, 0.5154) -- (2.3961, -0.0065, 0.5205) -- cycle;
\fill[blue!15.0, opacity=0.5] (2.3961, -0.0065, 0.5205) -- (2.4459, -0.0065, 0.5154) -- (2.4449, -0.0080, 0.5654) -- (2.3952, -0.0080, 0.5705) -- cycle;
\fill[blue!15.0, opacity=0.5] (2.3952, -0.0080, 0.5705) -- (2.4449, -0.0080, 0.5654) -- (2.4439, -0.0097, 0.6154) -- (2.3942, -0.0097, 0.6205) -- cycle;
\fill[blue!15.0, opacity=0.5] (2.3942, -0.0097, 0.6205) -- (2.4439, -0.0097, 0.6154) -- (2.4427, -0.0115, 0.6654) -- (2.3931, -0.0115, 0.6705) -- cycle;
\fill[blue!15.0, opacity=0.5] (2.3931, -0.0115, 0.6705) -- (2.4427, -0.0115, 0.6654) -- (2.4415, -0.0134, 0.7154) -- (2.3920, -0.0134, 0.7205) -- cycle;
\fill[blue!15.0, opacity=0.5] (2.3920, -0.0134, 0.7205) -- (2.4415, -0.0134, 0.7154) -- (2.4402, -0.0154, 0.7654) -- (2.3908, -0.0154, 0.7705) -- cycle;
\fill[blue!15.0, opacity=0.5] (2.3908, -0.0154, 0.7705) -- (2.4402, -0.0154, 0.7654) -- (2.4389, -0.0176, 0.8154) -- (2.3895, -0.0176, 0.8205) -- cycle;
\fill[blue!15.0, opacity=0.5] (2.3895, -0.0176, 0.8205) -- (2.4389, -0.0176, 0.8154) -- (2.4374, -0.0199, 0.8654) -- (2.3881, -0.0199, 0.8705) -- cycle;
\fill[blue!15.0, opacity=0.5] (2.3881, -0.0199, 0.8705) -- (2.4374, -0.0199, 0.8654) -- (2.4359, -0.0222, 0.9154) -- (2.3867, -0.0222, 0.9205) -- cycle;
\fill[blue!15.0, opacity=0.5] (2.3867, -0.0222, 0.9205) -- (2.4359, -0.0222, 0.9154) -- (2.4343, -0.0247, 0.9654) -- (2.3852, -0.0247, 0.9705) -- cycle;
\fill[blue!15.0, opacity=0.5] (2.3852, -0.0247, 0.9705) -- (2.4343, -0.0247, 0.9654) -- (2.4327, -0.0273, 1.0154) -- (2.3836, -0.0273, 1.0205) -- cycle;
\fill[blue!15.0, opacity=0.5] (2.3836, -0.0273, 1.0205) -- (2.4327, -0.0273, 1.0154) -- (2.4310, -0.0300, 1.0654) -- (2.3820, -0.0300, 1.0705) -- cycle;
\fill[blue!15.0, opacity=0.5] (2.3820, -0.0300, 1.0705) -- (2.4310, -0.0300, 1.0654) -- (2.4293, -0.0328, 1.1154) -- (2.3803, -0.0328, 1.1205) -- cycle;
\fill[blue!15.0, opacity=0.5] (2.3803, -0.0328, 1.1205) -- (2.4293, -0.0328, 1.1154) -- (2.4275, -0.0356, 1.1654) -- (2.3786, -0.0356, 1.1705) -- cycle;
\fill[blue!15.0, opacity=0.5] (2.3786, -0.0356, 1.1705) -- (2.4275, -0.0356, 1.1654) -- (2.4256, -0.0385, 1.2154) -- (2.3769, -0.0385, 1.2205) -- cycle;
\fill[blue!15.0, opacity=0.5] (2.3769, -0.0385, 1.2205) -- (2.4256, -0.0385, 1.2154) -- (2.4237, -0.0415, 1.2654) -- (2.3751, -0.0415, 1.2705) -- cycle;
\fill[blue!15.0, opacity=0.5] (2.3751, -0.0415, 1.2705) -- (2.4237, -0.0415, 1.2654) -- (2.4218, -0.0445, 1.3154) -- (2.3733, -0.0445, 1.3205) -- cycle;
\fill[blue!15.0, opacity=0.5] (2.3733, -0.0445, 1.3205) -- (2.4218, -0.0445, 1.3154) -- (2.4199, -0.0475, 1.3654) -- (2.3715, -0.0475, 1.3705) -- cycle;
\fill[blue!15.0, opacity=0.5] (2.3715, -0.0475, 1.3705) -- (2.4199, -0.0475, 1.3654) -- (2.4179, -0.0506, 1.4154) -- (2.3696, -0.0506, 1.4205) -- cycle;
\fill[blue!15.0, opacity=0.5] (2.3696, -0.0506, 1.4205) -- (2.4179, -0.0506, 1.4154) -- (2.4160, -0.0537, 1.4654) -- (2.3678, -0.0537, 1.4705) -- cycle;
\fill[blue!15.0, opacity=0.5] (2.3678, -0.0537, 1.4705) -- (2.4160, -0.0537, 1.4654) -- (2.4140, -0.0569, 1.5154) -- (2.3659, -0.0569, 1.5205) -- cycle;
\fill[blue!15.0, opacity=0.5] (2.3659, -0.0569, 1.5205) -- (2.4140, -0.0569, 1.5154) -- (2.4120, -0.0600, 1.5654) -- (2.3640, -0.0600, 1.5705) -- cycle;
\fill[blue!15.0, opacity=0.5] (2.3640, -0.0600, 1.5705) -- (2.4120, -0.0600, 1.5654) -- (2.4100, -0.0631, 1.6154) -- (2.3621, -0.0631, 1.6205) -- cycle;
\fill[blue!15.0, opacity=0.5] (2.3621, -0.0631, 1.6205) -- (2.4100, -0.0631, 1.6154) -- (2.4080, -0.0663, 1.6654) -- (2.3602, -0.0663, 1.6705) -- cycle;
\fill[blue!15.0, opacity=0.5] (2.3602, -0.0663, 1.6705) -- (2.4080, -0.0663, 1.6654) -- (2.4061, -0.0694, 1.7154) -- (2.3584, -0.0694, 1.7205) -- cycle;
\fill[blue!15.0, opacity=0.5] (2.3584, -0.0694, 1.7205) -- (2.4061, -0.0694, 1.7154) -- (2.4041, -0.0725, 1.7654) -- (2.3565, -0.0725, 1.7705) -- cycle;
\fill[blue!15.0, opacity=0.5] (2.3565, -0.0725, 1.7705) -- (2.4041, -0.0725, 1.7654) -- (2.4022, -0.0755, 1.8154) -- (2.3547, -0.0755, 1.8205) -- cycle;
\fill[blue!15.0, opacity=0.5] (2.3547, -0.0755, 1.8205) -- (2.4022, -0.0755, 1.8154) -- (2.4003, -0.0785, 1.8654) -- (2.3529, -0.0785, 1.8705) -- cycle;
\fill[blue!15.0, opacity=0.5] (2.3529, -0.0785, 1.8705) -- (2.4003, -0.0785, 1.8654) -- (2.3984, -0.0815, 1.9154) -- (2.3511, -0.0815, 1.9205) -- cycle;
\fill[blue!15.0, opacity=0.5] (2.3511, -0.0815, 1.9205) -- (2.3984, -0.0815, 1.9154) -- (2.3965, -0.0844, 1.9654) -- (2.3494, -0.0844, 1.9705) -- cycle;
\fill[blue!15.0, opacity=0.5] (2.3494, -0.0844, 1.9705) -- (2.3965, -0.0844, 1.9654) -- (2.3947, -0.0872, 2.0154) -- (2.3477, -0.0872, 2.0205) -- cycle;
\fill[blue!15.0, opacity=0.5] (2.3477, -0.0872, 2.0205) -- (2.3947, -0.0872, 2.0154) -- (2.3930, -0.0900, 2.0654) -- (2.3460, -0.0900, 2.0705) -- cycle;
\fill[blue!15.0, opacity=0.5] (2.3460, -0.0900, 2.0705) -- (2.3930, -0.0900, 2.0654) -- (2.3913, -0.0927, 2.1154) -- (2.3444, -0.0927, 2.1205) -- cycle;
\fill[blue!15.0, opacity=0.5] (2.3444, -0.0927, 2.1205) -- (2.3913, -0.0927, 2.1154) -- (2.3897, -0.0953, 2.1654) -- (2.3428, -0.0953, 2.1705) -- cycle;
\fill[blue!15.0, opacity=0.5] (2.3428, -0.0953, 2.1705) -- (2.3897, -0.0953, 2.1654) -- (2.3881, -0.0978, 2.2154) -- (2.3413, -0.0978, 2.2205) -- cycle;
\fill[blue!15.0, opacity=0.5] (2.3413, -0.0978, 2.2205) -- (2.3881, -0.0978, 2.2154) -- (2.3866, -0.1001, 2.2654) -- (2.3399, -0.1001, 2.2705) -- cycle;
\fill[blue!15.0, opacity=0.5] (2.3399, -0.1001, 2.2705) -- (2.3866, -0.1001, 2.2654) -- (2.3851, -0.1024, 2.3154) -- (2.3385, -0.1024, 2.3205) -- cycle;
\fill[blue!15.1, opacity=0.5] (2.3385, -0.1024, 2.3205) -- (2.3851, -0.1024, 2.3154) -- (2.3838, -0.1046, 2.3654) -- (2.3372, -0.1046, 2.3705) -- cycle;
\fill[blue!15.1, opacity=0.5] (2.3372, -0.1046, 2.3705) -- (2.3838, -0.1046, 2.3654) -- (2.3825, -0.1066, 2.4154) -- (2.3360, -0.1066, 2.4205) -- cycle;
\fill[blue!15.1, opacity=0.5] (2.3360, -0.1066, 2.4205) -- (2.3825, -0.1066, 2.4154) -- (2.3813, -0.1085, 2.4654) -- (2.3349, -0.1085, 2.4705) -- cycle;
\fill[blue!15.1, opacity=0.5] (2.3349, -0.1085, 2.4705) -- (2.3813, -0.1085, 2.4654) -- (2.3801, -0.1103, 2.5154) -- (2.3338, -0.1103, 2.5205) -- cycle;
\fill[blue!15.2, opacity=0.5] (2.3338, -0.1103, 2.5205) -- (2.3801, -0.1103, 2.5154) -- (2.3791, -0.1120, 2.5654) -- (2.3328, -0.1120, 2.5705) -- cycle;
\fill[blue!15.3, opacity=0.5] (2.3328, -0.1120, 2.5705) -- (2.3791, -0.1120, 2.5654) -- (2.3781, -0.1135, 2.6154) -- (2.3319, -0.1135, 2.6205) -- cycle;
\fill[blue!15.3, opacity=0.5] (2.3319, -0.1135, 2.6205) -- (2.3781, -0.1135, 2.6154) -- (2.3773, -0.1148, 2.6654) -- (2.3311, -0.1148, 2.6705) -- cycle;
\fill[blue!15.4, opacity=0.5] (2.3311, -0.1148, 2.6705) -- (2.3773, -0.1148, 2.6654) -- (2.3765, -0.1160, 2.7154) -- (2.3304, -0.1160, 2.7205) -- cycle;
\fill[blue!15.5, opacity=0.5] (2.3304, -0.1160, 2.7205) -- (2.3765, -0.1160, 2.7154) -- (2.3759, -0.1171, 2.7654) -- (2.3298, -0.1171, 2.7705) -- cycle;
\fill[blue!15.6, opacity=0.5] (2.3298, -0.1171, 2.7705) -- (2.3759, -0.1171, 2.7654) -- (2.3753, -0.1180, 2.8154) -- (2.3292, -0.1180, 2.8205) -- cycle;
\fill[blue!15.8, opacity=0.5] (2.3292, -0.1180, 2.8205) -- (2.3753, -0.1180, 2.8154) -- (2.3748, -0.1187, 2.8654) -- (2.3288, -0.1187, 2.8705) -- cycle;
\fill[blue!16.0, opacity=0.5] (2.3288, -0.1187, 2.8705) -- (2.3748, -0.1187, 2.8654) -- (2.3745, -0.1193, 2.9154) -- (2.3284, -0.1193, 2.9205) -- cycle;
\fill[blue!16.1, opacity=0.5] (2.3284, -0.1193, 2.9205) -- (2.3745, -0.1193, 2.9154) -- (2.3742, -0.1197, 2.9654) -- (2.3282, -0.1197, 2.9705) -- cycle;
\fill[blue!16.3, opacity=0.5] (2.3282, -0.1197, 2.9705) -- (2.3742, -0.1197, 2.9654) -- (2.3741, -0.1199, 3.0154) -- (2.3280, -0.1199, 3.0205) -- cycle;
\fill[blue!16.6, opacity=0.5] (2.3280, -0.1199, 3.0205) -- (2.3741, -0.1199, 3.0154) -- (2.3740, -0.1200, 3.0654) -- (2.3280, -0.1200, 3.0705) -- cycle;
\fill[blue!15.0, opacity=0.5] (2.4500, -0.0000, 0.0654) -- (2.5000, -0.0000, 0.0600) -- (2.4999, -0.0001, 0.1100) -- (2.4499, -0.0001, 0.1154) -- cycle;
\fill[blue!15.0, opacity=0.5] (2.4499, -0.0001, 0.1154) -- (2.4999, -0.0001, 0.1100) -- (2.4998, -0.0003, 0.1600) -- (2.4498, -0.0003, 0.1654) -- cycle;
\fill[blue!15.0, opacity=0.5] (2.4498, -0.0003, 0.1654) -- (2.4998, -0.0003, 0.1600) -- (2.4995, -0.0007, 0.2100) -- (2.4495, -0.0007, 0.2154) -- cycle;
\fill[blue!15.0, opacity=0.5] (2.4495, -0.0007, 0.2154) -- (2.4995, -0.0007, 0.2100) -- (2.4991, -0.0013, 0.2600) -- (2.4492, -0.0013, 0.2654) -- cycle;
\fill[blue!15.0, opacity=0.5] (2.4492, -0.0013, 0.2654) -- (2.4991, -0.0013, 0.2600) -- (2.4986, -0.0020, 0.3100) -- (2.4487, -0.0020, 0.3154) -- cycle;
\fill[blue!15.0, opacity=0.5] (2.4487, -0.0020, 0.3154) -- (2.4986, -0.0020, 0.3100) -- (2.4980, -0.0029, 0.3600) -- (2.4481, -0.0029, 0.3654) -- cycle;
\fill[blue!15.0, opacity=0.5] (2.4481, -0.0029, 0.3654) -- (2.4980, -0.0029, 0.3600) -- (2.4973, -0.0040, 0.4100) -- (2.4475, -0.0040, 0.4154) -- cycle;
\fill[blue!15.0, opacity=0.5] (2.4475, -0.0040, 0.4154) -- (2.4973, -0.0040, 0.4100) -- (2.4965, -0.0052, 0.4600) -- (2.4467, -0.0052, 0.4654) -- cycle;
\fill[blue!15.0, opacity=0.5] (2.4467, -0.0052, 0.4654) -- (2.4965, -0.0052, 0.4600) -- (2.4956, -0.0065, 0.5100) -- (2.4459, -0.0065, 0.5154) -- cycle;
\fill[blue!15.0, opacity=0.5] (2.4459, -0.0065, 0.5154) -- (2.4956, -0.0065, 0.5100) -- (2.4946, -0.0080, 0.5600) -- (2.4449, -0.0080, 0.5654) -- cycle;
\fill[blue!15.0, opacity=0.5] (2.4449, -0.0080, 0.5654) -- (2.4946, -0.0080, 0.5600) -- (2.4935, -0.0097, 0.6100) -- (2.4439, -0.0097, 0.6154) -- cycle;
\fill[blue!15.0, opacity=0.5] (2.4439, -0.0097, 0.6154) -- (2.4935, -0.0097, 0.6100) -- (2.4924, -0.0115, 0.6600) -- (2.4427, -0.0115, 0.6654) -- cycle;
\fill[blue!15.0, opacity=0.5] (2.4427, -0.0115, 0.6654) -- (2.4924, -0.0115, 0.6600) -- (2.4911, -0.0134, 0.7100) -- (2.4415, -0.0134, 0.7154) -- cycle;
\fill[blue!15.0, opacity=0.5] (2.4415, -0.0134, 0.7154) -- (2.4911, -0.0134, 0.7100) -- (2.4897, -0.0154, 0.7600) -- (2.4402, -0.0154, 0.7654) -- cycle;
\fill[blue!15.0, opacity=0.5] (2.4402, -0.0154, 0.7654) -- (2.4897, -0.0154, 0.7600) -- (2.4883, -0.0176, 0.8100) -- (2.4389, -0.0176, 0.8154) -- cycle;
\fill[blue!15.0, opacity=0.5] (2.4389, -0.0176, 0.8154) -- (2.4883, -0.0176, 0.8100) -- (2.4868, -0.0199, 0.8600) -- (2.4374, -0.0199, 0.8654) -- cycle;
\fill[blue!15.0, opacity=0.5] (2.4374, -0.0199, 0.8654) -- (2.4868, -0.0199, 0.8600) -- (2.4852, -0.0222, 0.9100) -- (2.4359, -0.0222, 0.9154) -- cycle;
\fill[blue!15.0, opacity=0.5] (2.4359, -0.0222, 0.9154) -- (2.4852, -0.0222, 0.9100) -- (2.4835, -0.0247, 0.9600) -- (2.4343, -0.0247, 0.9654) -- cycle;
\fill[blue!15.0, opacity=0.5] (2.4343, -0.0247, 0.9654) -- (2.4835, -0.0247, 0.9600) -- (2.4818, -0.0273, 1.0100) -- (2.4327, -0.0273, 1.0154) -- cycle;
\fill[blue!15.0, opacity=0.5] (2.4327, -0.0273, 1.0154) -- (2.4818, -0.0273, 1.0100) -- (2.4800, -0.0300, 1.0600) -- (2.4310, -0.0300, 1.0654) -- cycle;
\fill[blue!15.0, opacity=0.5] (2.4310, -0.0300, 1.0654) -- (2.4800, -0.0300, 1.0600) -- (2.4782, -0.0328, 1.1100) -- (2.4293, -0.0328, 1.1154) -- cycle;
\fill[blue!15.0, opacity=0.5] (2.4293, -0.0328, 1.1154) -- (2.4782, -0.0328, 1.1100) -- (2.4763, -0.0356, 1.1600) -- (2.4275, -0.0356, 1.1654) -- cycle;
\fill[blue!15.0, opacity=0.5] (2.4275, -0.0356, 1.1654) -- (2.4763, -0.0356, 1.1600) -- (2.4743, -0.0385, 1.2100) -- (2.4256, -0.0385, 1.2154) -- cycle;
\fill[blue!15.0, opacity=0.5] (2.4256, -0.0385, 1.2154) -- (2.4743, -0.0385, 1.2100) -- (2.4724, -0.0415, 1.2600) -- (2.4237, -0.0415, 1.2654) -- cycle;
\fill[blue!15.0, opacity=0.5] (2.4237, -0.0415, 1.2654) -- (2.4724, -0.0415, 1.2600) -- (2.4704, -0.0445, 1.3100) -- (2.4218, -0.0445, 1.3154) -- cycle;
\fill[blue!15.0, opacity=0.5] (2.4218, -0.0445, 1.3154) -- (2.4704, -0.0445, 1.3100) -- (2.4683, -0.0475, 1.3600) -- (2.4199, -0.0475, 1.3654) -- cycle;
\fill[blue!15.0, opacity=0.5] (2.4199, -0.0475, 1.3654) -- (2.4683, -0.0475, 1.3600) -- (2.4663, -0.0506, 1.4100) -- (2.4179, -0.0506, 1.4154) -- cycle;
\fill[blue!15.0, opacity=0.5] (2.4179, -0.0506, 1.4154) -- (2.4663, -0.0506, 1.4100) -- (2.4642, -0.0537, 1.4600) -- (2.4160, -0.0537, 1.4654) -- cycle;
\fill[blue!15.0, opacity=0.5] (2.4160, -0.0537, 1.4654) -- (2.4642, -0.0537, 1.4600) -- (2.4621, -0.0569, 1.5100) -- (2.4140, -0.0569, 1.5154) -- cycle;
\fill[blue!15.0, opacity=0.5] (2.4140, -0.0569, 1.5154) -- (2.4621, -0.0569, 1.5100) -- (2.4600, -0.0600, 1.5600) -- (2.4120, -0.0600, 1.5654) -- cycle;
\fill[blue!15.0, opacity=0.5] (2.4120, -0.0600, 1.5654) -- (2.4600, -0.0600, 1.5600) -- (2.4579, -0.0631, 1.6100) -- (2.4100, -0.0631, 1.6154) -- cycle;
\fill[blue!15.0, opacity=0.5] (2.4100, -0.0631, 1.6154) -- (2.4579, -0.0631, 1.6100) -- (2.4558, -0.0663, 1.6600) -- (2.4080, -0.0663, 1.6654) -- cycle;
\fill[blue!15.0, opacity=0.5] (2.4080, -0.0663, 1.6654) -- (2.4558, -0.0663, 1.6600) -- (2.4537, -0.0694, 1.7100) -- (2.4061, -0.0694, 1.7154) -- cycle;
\fill[blue!15.0, opacity=0.5] (2.4061, -0.0694, 1.7154) -- (2.4537, -0.0694, 1.7100) -- (2.4517, -0.0725, 1.7600) -- (2.4041, -0.0725, 1.7654) -- cycle;
\fill[blue!15.0, opacity=0.5] (2.4041, -0.0725, 1.7654) -- (2.4517, -0.0725, 1.7600) -- (2.4496, -0.0755, 1.8100) -- (2.4022, -0.0755, 1.8154) -- cycle;
\fill[blue!15.0, opacity=0.5] (2.4022, -0.0755, 1.8154) -- (2.4496, -0.0755, 1.8100) -- (2.4476, -0.0785, 1.8600) -- (2.4003, -0.0785, 1.8654) -- cycle;
\fill[blue!15.0, opacity=0.5] (2.4003, -0.0785, 1.8654) -- (2.4476, -0.0785, 1.8600) -- (2.4457, -0.0815, 1.9100) -- (2.3984, -0.0815, 1.9154) -- cycle;
\fill[blue!15.0, opacity=0.5] (2.3984, -0.0815, 1.9154) -- (2.4457, -0.0815, 1.9100) -- (2.4437, -0.0844, 1.9600) -- (2.3965, -0.0844, 1.9654) -- cycle;
\fill[blue!15.0, opacity=0.5] (2.3965, -0.0844, 1.9654) -- (2.4437, -0.0844, 1.9600) -- (2.4418, -0.0872, 2.0100) -- (2.3947, -0.0872, 2.0154) -- cycle;
\fill[blue!15.0, opacity=0.5] (2.3947, -0.0872, 2.0154) -- (2.4418, -0.0872, 2.0100) -- (2.4400, -0.0900, 2.0600) -- (2.3930, -0.0900, 2.0654) -- cycle;
\fill[blue!15.1, opacity=0.5] (2.3930, -0.0900, 2.0654) -- (2.4400, -0.0900, 2.0600) -- (2.4382, -0.0927, 2.1100) -- (2.3913, -0.0927, 2.1154) -- cycle;
\fill[blue!15.1, opacity=0.5] (2.3913, -0.0927, 2.1154) -- (2.4382, -0.0927, 2.1100) -- (2.4365, -0.0953, 2.1600) -- (2.3897, -0.0953, 2.1654) -- cycle;
\fill[blue!15.1, opacity=0.5] (2.3897, -0.0953, 2.1654) -- (2.4365, -0.0953, 2.1600) -- (2.4348, -0.0978, 2.2100) -- (2.3881, -0.0978, 2.2154) -- cycle;
\fill[blue!15.2, opacity=0.5] (2.3881, -0.0978, 2.2154) -- (2.4348, -0.0978, 2.2100) -- (2.4332, -0.1001, 2.2600) -- (2.3866, -0.1001, 2.2654) -- cycle;
\fill[blue!15.2, opacity=0.5] (2.3866, -0.1001, 2.2654) -- (2.4332, -0.1001, 2.2600) -- (2.4317, -0.1024, 2.3100) -- (2.3851, -0.1024, 2.3154) -- cycle;
\fill[blue!15.3, opacity=0.5] (2.3851, -0.1024, 2.3154) -- (2.4317, -0.1024, 2.3100) -- (2.4303, -0.1046, 2.3600) -- (2.3838, -0.1046, 2.3654) -- cycle;
\fill[blue!15.4, opacity=0.5] (2.3838, -0.1046, 2.3654) -- (2.4303, -0.1046, 2.3600) -- (2.4289, -0.1066, 2.4100) -- (2.3825, -0.1066, 2.4154) -- cycle;
\fill[blue!15.5, opacity=0.5] (2.3825, -0.1066, 2.4154) -- (2.4289, -0.1066, 2.4100) -- (2.4276, -0.1085, 2.4600) -- (2.3813, -0.1085, 2.4654) -- cycle;
\fill[blue!15.6, opacity=0.5] (2.3813, -0.1085, 2.4654) -- (2.4276, -0.1085, 2.4600) -- (2.4265, -0.1103, 2.5100) -- (2.3801, -0.1103, 2.5154) -- cycle;
\fill[blue!15.8, opacity=0.5] (2.3801, -0.1103, 2.5154) -- (2.4265, -0.1103, 2.5100) -- (2.4254, -0.1120, 2.5600) -- (2.3791, -0.1120, 2.5654) -- cycle;
\fill[blue!16.0, opacity=0.5] (2.3791, -0.1120, 2.5654) -- (2.4254, -0.1120, 2.5600) -- (2.4244, -0.1135, 2.6100) -- (2.3781, -0.1135, 2.6154) -- cycle;
\fill[blue!16.2, opacity=0.5] (2.3781, -0.1135, 2.6154) -- (2.4244, -0.1135, 2.6100) -- (2.4235, -0.1148, 2.6600) -- (2.3773, -0.1148, 2.6654) -- cycle;
\fill[blue!16.5, opacity=0.5] (2.3773, -0.1148, 2.6654) -- (2.4235, -0.1148, 2.6600) -- (2.4227, -0.1160, 2.7100) -- (2.3765, -0.1160, 2.7154) -- cycle;
\fill[blue!16.7, opacity=0.5] (2.3765, -0.1160, 2.7154) -- (2.4227, -0.1160, 2.7100) -- (2.4220, -0.1171, 2.7600) -- (2.3759, -0.1171, 2.7654) -- cycle;
\fill[blue!17.1, opacity=0.5] (2.3759, -0.1171, 2.7654) -- (2.4220, -0.1171, 2.7600) -- (2.4214, -0.1180, 2.8100) -- (2.3753, -0.1180, 2.8154) -- cycle;
\fill[blue!17.4, opacity=0.5] (2.3753, -0.1180, 2.8154) -- (2.4214, -0.1180, 2.8100) -- (2.4209, -0.1187, 2.8600) -- (2.3748, -0.1187, 2.8654) -- cycle;
\fill[blue!17.8, opacity=0.5] (2.3748, -0.1187, 2.8654) -- (2.4209, -0.1187, 2.8600) -- (2.4205, -0.1193, 2.9100) -- (2.3745, -0.1193, 2.9154) -- cycle;
\fill[blue!18.2, opacity=0.5] (2.3745, -0.1193, 2.9154) -- (2.4205, -0.1193, 2.9100) -- (2.4202, -0.1197, 2.9600) -- (2.3742, -0.1197, 2.9654) -- cycle;
\fill[blue!18.7, opacity=0.5] (2.3742, -0.1197, 2.9654) -- (2.4202, -0.1197, 2.9600) -- (2.4201, -0.1199, 3.0100) -- (2.3741, -0.1199, 3.0154) -- cycle;
\fill[blue!19.1, opacity=0.5] (2.3741, -0.1199, 3.0154) -- (2.4201, -0.1199, 3.0100) -- (2.4200, -0.1200, 3.0600) -- (2.3740, -0.1200, 3.0654) -- cycle;
\fill[blue!15.0, opacity=0.5] (2.5000, -0.0000, 0.0600) -- (2.5500, -0.0000, 0.0545) -- (2.5499, -0.0001, 0.1045) -- (2.4999, -0.0001, 0.1100) -- cycle;
\fill[blue!15.0, opacity=0.5] (2.4999, -0.0001, 0.1100) -- (2.5499, -0.0001, 0.1045) -- (2.5498, -0.0003, 0.1545) -- (2.4998, -0.0003, 0.1600) -- cycle;
\fill[blue!15.0, opacity=0.5] (2.4998, -0.0003, 0.1600) -- (2.5498, -0.0003, 0.1545) -- (2.5495, -0.0007, 0.2045) -- (2.4995, -0.0007, 0.2100) -- cycle;
\fill[blue!15.0, opacity=0.5] (2.4995, -0.0007, 0.2100) -- (2.5495, -0.0007, 0.2045) -- (2.5491, -0.0013, 0.2545) -- (2.4991, -0.0013, 0.2600) -- cycle;
\fill[blue!15.0, opacity=0.5] (2.4991, -0.0013, 0.2600) -- (2.5491, -0.0013, 0.2545) -- (2.5486, -0.0020, 0.3045) -- (2.4986, -0.0020, 0.3100) -- cycle;
\fill[blue!15.0, opacity=0.5] (2.4986, -0.0020, 0.3100) -- (2.5486, -0.0020, 0.3045) -- (2.5479, -0.0029, 0.3545) -- (2.4980, -0.0029, 0.3600) -- cycle;
\fill[blue!15.0, opacity=0.5] (2.4980, -0.0029, 0.3600) -- (2.5479, -0.0029, 0.3545) -- (2.5472, -0.0040, 0.4045) -- (2.4973, -0.0040, 0.4100) -- cycle;
\fill[blue!15.0, opacity=0.5] (2.4973, -0.0040, 0.4100) -- (2.5472, -0.0040, 0.4045) -- (2.5464, -0.0052, 0.4545) -- (2.4965, -0.0052, 0.4600) -- cycle;
\fill[blue!15.0, opacity=0.5] (2.4965, -0.0052, 0.4600) -- (2.5464, -0.0052, 0.4545) -- (2.5454, -0.0065, 0.5045) -- (2.4956, -0.0065, 0.5100) -- cycle;
\fill[blue!15.0, opacity=0.5] (2.4956, -0.0065, 0.5100) -- (2.5454, -0.0065, 0.5045) -- (2.5444, -0.0080, 0.5545) -- (2.4946, -0.0080, 0.5600) -- cycle;
\fill[blue!15.0, opacity=0.5] (2.4946, -0.0080, 0.5600) -- (2.5444, -0.0080, 0.5545) -- (2.5432, -0.0097, 0.6045) -- (2.4935, -0.0097, 0.6100) -- cycle;
\fill[blue!15.0, opacity=0.5] (2.4935, -0.0097, 0.6100) -- (2.5432, -0.0097, 0.6045) -- (2.5420, -0.0115, 0.6545) -- (2.4924, -0.0115, 0.6600) -- cycle;
\fill[blue!15.0, opacity=0.5] (2.4924, -0.0115, 0.6600) -- (2.5420, -0.0115, 0.6545) -- (2.5406, -0.0134, 0.7045) -- (2.4911, -0.0134, 0.7100) -- cycle;
\fill[blue!15.0, opacity=0.5] (2.4911, -0.0134, 0.7100) -- (2.5406, -0.0134, 0.7045) -- (2.5392, -0.0154, 0.7545) -- (2.4897, -0.0154, 0.7600) -- cycle;
\fill[blue!15.0, opacity=0.5] (2.4897, -0.0154, 0.7600) -- (2.5392, -0.0154, 0.7545) -- (2.5377, -0.0176, 0.8045) -- (2.4883, -0.0176, 0.8100) -- cycle;
\fill[blue!15.0, opacity=0.5] (2.4883, -0.0176, 0.8100) -- (2.5377, -0.0176, 0.8045) -- (2.5361, -0.0199, 0.8545) -- (2.4868, -0.0199, 0.8600) -- cycle;
\fill[blue!15.0, opacity=0.5] (2.4868, -0.0199, 0.8600) -- (2.5361, -0.0199, 0.8545) -- (2.5344, -0.0222, 0.9045) -- (2.4852, -0.0222, 0.9100) -- cycle;
\fill[blue!15.0, opacity=0.5] (2.4852, -0.0222, 0.9100) -- (2.5344, -0.0222, 0.9045) -- (2.5327, -0.0247, 0.9545) -- (2.4835, -0.0247, 0.9600) -- cycle;
\fill[blue!15.0, opacity=0.5] (2.4835, -0.0247, 0.9600) -- (2.5327, -0.0247, 0.9545) -- (2.5309, -0.0273, 1.0045) -- (2.4818, -0.0273, 1.0100) -- cycle;
\fill[blue!15.0, opacity=0.5] (2.4818, -0.0273, 1.0100) -- (2.5309, -0.0273, 1.0045) -- (2.5290, -0.0300, 1.0545) -- (2.4800, -0.0300, 1.0600) -- cycle;
\fill[blue!15.0, opacity=0.5] (2.4800, -0.0300, 1.0600) -- (2.5290, -0.0300, 1.0545) -- (2.5271, -0.0328, 1.1045) -- (2.4782, -0.0328, 1.1100) -- cycle;
\fill[blue!15.0, opacity=0.5] (2.4782, -0.0328, 1.1100) -- (2.5271, -0.0328, 1.1045) -- (2.5251, -0.0356, 1.1545) -- (2.4763, -0.0356, 1.1600) -- cycle;
\fill[blue!15.0, opacity=0.5] (2.4763, -0.0356, 1.1600) -- (2.5251, -0.0356, 1.1545) -- (2.5231, -0.0385, 1.2045) -- (2.4743, -0.0385, 1.2100) -- cycle;
\fill[blue!15.0, opacity=0.5] (2.4743, -0.0385, 1.2100) -- (2.5231, -0.0385, 1.2045) -- (2.5210, -0.0415, 1.2545) -- (2.4724, -0.0415, 1.2600) -- cycle;
\fill[blue!15.0, opacity=0.5] (2.4724, -0.0415, 1.2600) -- (2.5210, -0.0415, 1.2545) -- (2.5189, -0.0445, 1.3045) -- (2.4704, -0.0445, 1.3100) -- cycle;
\fill[blue!15.0, opacity=0.5] (2.4704, -0.0445, 1.3100) -- (2.5189, -0.0445, 1.3045) -- (2.5167, -0.0475, 1.3545) -- (2.4683, -0.0475, 1.3600) -- cycle;
\fill[blue!15.0, opacity=0.5] (2.4683, -0.0475, 1.3600) -- (2.5167, -0.0475, 1.3545) -- (2.5146, -0.0506, 1.4045) -- (2.4663, -0.0506, 1.4100) -- cycle;
\fill[blue!15.0, opacity=0.5] (2.4663, -0.0506, 1.4100) -- (2.5146, -0.0506, 1.4045) -- (2.5124, -0.0537, 1.4545) -- (2.4642, -0.0537, 1.4600) -- cycle;
\fill[blue!15.0, opacity=0.5] (2.4642, -0.0537, 1.4600) -- (2.5124, -0.0537, 1.4545) -- (2.5102, -0.0569, 1.5045) -- (2.4621, -0.0569, 1.5100) -- cycle;
\fill[blue!15.0, opacity=0.5] (2.4621, -0.0569, 1.5100) -- (2.5102, -0.0569, 1.5045) -- (2.5080, -0.0600, 1.5545) -- (2.4600, -0.0600, 1.5600) -- cycle;
\fill[blue!15.0, opacity=0.5] (2.4600, -0.0600, 1.5600) -- (2.5080, -0.0600, 1.5545) -- (2.5058, -0.0631, 1.6045) -- (2.4579, -0.0631, 1.6100) -- cycle;
\fill[blue!15.0, opacity=0.5] (2.4579, -0.0631, 1.6100) -- (2.5058, -0.0631, 1.6045) -- (2.5036, -0.0663, 1.6545) -- (2.4558, -0.0663, 1.6600) -- cycle;
\fill[blue!15.0, opacity=0.5] (2.4558, -0.0663, 1.6600) -- (2.5036, -0.0663, 1.6545) -- (2.5014, -0.0694, 1.7045) -- (2.4537, -0.0694, 1.7100) -- cycle;
\fill[blue!15.0, opacity=0.5] (2.4537, -0.0694, 1.7100) -- (2.5014, -0.0694, 1.7045) -- (2.4993, -0.0725, 1.7545) -- (2.4517, -0.0725, 1.7600) -- cycle;
\fill[blue!15.1, opacity=0.5] (2.4517, -0.0725, 1.7600) -- (2.4993, -0.0725, 1.7545) -- (2.4971, -0.0755, 1.8045) -- (2.4496, -0.0755, 1.8100) -- cycle;
\fill[blue!15.1, opacity=0.5] (2.4496, -0.0755, 1.8100) -- (2.4971, -0.0755, 1.8045) -- (2.4950, -0.0785, 1.8545) -- (2.4476, -0.0785, 1.8600) -- cycle;
\fill[blue!15.1, opacity=0.5] (2.4476, -0.0785, 1.8600) -- (2.4950, -0.0785, 1.8545) -- (2.4929, -0.0815, 1.9045) -- (2.4457, -0.0815, 1.9100) -- cycle;
\fill[blue!15.2, opacity=0.5] (2.4457, -0.0815, 1.9100) -- (2.4929, -0.0815, 1.9045) -- (2.4909, -0.0844, 1.9545) -- (2.4437, -0.0844, 1.9600) -- cycle;
\fill[blue!15.3, opacity=0.5] (2.4437, -0.0844, 1.9600) -- (2.4909, -0.0844, 1.9545) -- (2.4889, -0.0872, 2.0045) -- (2.4418, -0.0872, 2.0100) -- cycle;
\fill[blue!15.4, opacity=0.5] (2.4418, -0.0872, 2.0100) -- (2.4889, -0.0872, 2.0045) -- (2.4870, -0.0900, 2.0545) -- (2.4400, -0.0900, 2.0600) -- cycle;
\fill[blue!15.5, opacity=0.5] (2.4400, -0.0900, 2.0600) -- (2.4870, -0.0900, 2.0545) -- (2.4851, -0.0927, 2.1045) -- (2.4382, -0.0927, 2.1100) -- cycle;
\fill[blue!15.6, opacity=0.5] (2.4382, -0.0927, 2.1100) -- (2.4851, -0.0927, 2.1045) -- (2.4833, -0.0953, 2.1545) -- (2.4365, -0.0953, 2.1600) -- cycle;
\fill[blue!15.8, opacity=0.5] (2.4365, -0.0953, 2.1600) -- (2.4833, -0.0953, 2.1545) -- (2.4816, -0.0978, 2.2045) -- (2.4348, -0.0978, 2.2100) -- cycle;
\fill[blue!16.0, opacity=0.5] (2.4348, -0.0978, 2.2100) -- (2.4816, -0.0978, 2.2045) -- (2.4799, -0.1001, 2.2545) -- (2.4332, -0.1001, 2.2600) -- cycle;
\fill[blue!16.3, opacity=0.5] (2.4332, -0.1001, 2.2600) -- (2.4799, -0.1001, 2.2545) -- (2.4783, -0.1024, 2.3045) -- (2.4317, -0.1024, 2.3100) -- cycle;
\fill[blue!16.6, opacity=0.5] (2.4317, -0.1024, 2.3100) -- (2.4783, -0.1024, 2.3045) -- (2.4768, -0.1046, 2.3545) -- (2.4303, -0.1046, 2.3600) -- cycle;
\fill[blue!17.0, opacity=0.5] (2.4303, -0.1046, 2.3600) -- (2.4768, -0.1046, 2.3545) -- (2.4754, -0.1066, 2.4045) -- (2.4289, -0.1066, 2.4100) -- cycle;
\fill[blue!17.4, opacity=0.5] (2.4289, -0.1066, 2.4100) -- (2.4754, -0.1066, 2.4045) -- (2.4740, -0.1085, 2.4545) -- (2.4276, -0.1085, 2.4600) -- cycle;
\fill[blue!17.8, opacity=0.5] (2.4276, -0.1085, 2.4600) -- (2.4740, -0.1085, 2.4545) -- (2.4728, -0.1103, 2.5045) -- (2.4265, -0.1103, 2.5100) -- cycle;
\fill[blue!18.3, opacity=0.5] (2.4265, -0.1103, 2.5100) -- (2.4728, -0.1103, 2.5045) -- (2.4716, -0.1120, 2.5545) -- (2.4254, -0.1120, 2.5600) -- cycle;
\fill[blue!18.9, opacity=0.5] (2.4254, -0.1120, 2.5600) -- (2.4716, -0.1120, 2.5545) -- (2.4706, -0.1135, 2.6045) -- (2.4244, -0.1135, 2.6100) -- cycle;
\fill[blue!19.5, opacity=0.5] (2.4244, -0.1135, 2.6100) -- (2.4706, -0.1135, 2.6045) -- (2.4696, -0.1148, 2.6545) -- (2.4235, -0.1148, 2.6600) -- cycle;
\fill[blue!20.2, opacity=0.5] (2.4235, -0.1148, 2.6600) -- (2.4696, -0.1148, 2.6545) -- (2.4688, -0.1160, 2.7045) -- (2.4227, -0.1160, 2.7100) -- cycle;
\fill[blue!20.9, opacity=0.5] (2.4227, -0.1160, 2.7100) -- (2.4688, -0.1160, 2.7045) -- (2.4681, -0.1171, 2.7545) -- (2.4220, -0.1171, 2.7600) -- cycle;
\fill[blue!21.6, opacity=0.5] (2.4220, -0.1171, 2.7600) -- (2.4681, -0.1171, 2.7545) -- (2.4674, -0.1180, 2.8045) -- (2.4214, -0.1180, 2.8100) -- cycle;
\fill[blue!22.4, opacity=0.5] (2.4214, -0.1180, 2.8100) -- (2.4674, -0.1180, 2.8045) -- (2.4669, -0.1187, 2.8545) -- (2.4209, -0.1187, 2.8600) -- cycle;
\fill[blue!23.2, opacity=0.5] (2.4209, -0.1187, 2.8600) -- (2.4669, -0.1187, 2.8545) -- (2.4665, -0.1193, 2.9045) -- (2.4205, -0.1193, 2.9100) -- cycle;
\fill[blue!24.0, opacity=0.5] (2.4205, -0.1193, 2.9100) -- (2.4665, -0.1193, 2.9045) -- (2.4662, -0.1197, 2.9545) -- (2.4202, -0.1197, 2.9600) -- cycle;
\fill[blue!24.8, opacity=0.5] (2.4202, -0.1197, 2.9600) -- (2.4662, -0.1197, 2.9545) -- (2.4661, -0.1199, 3.0045) -- (2.4201, -0.1199, 3.0100) -- cycle;
\fill[blue!25.7, opacity=0.5] (2.4201, -0.1199, 3.0100) -- (2.4661, -0.1199, 3.0045) -- (2.4660, -0.1200, 3.0545) -- (2.4200, -0.1200, 3.0600) -- cycle;
\fill[blue!15.0, opacity=0.5] (2.5500, -0.0000, 0.0545) -- (2.6000, -0.0000, 0.0488) -- (2.5999, -0.0001, 0.0988) -- (2.5499, -0.0001, 0.1045) -- cycle;
\fill[blue!15.0, opacity=0.5] (2.5499, -0.0001, 0.1045) -- (2.5999, -0.0001, 0.0988) -- (2.5998, -0.0003, 0.1488) -- (2.5498, -0.0003, 0.1545) -- cycle;
\fill[blue!15.0, opacity=0.5] (2.5498, -0.0003, 0.1545) -- (2.5998, -0.0003, 0.1488) -- (2.5995, -0.0007, 0.1988) -- (2.5495, -0.0007, 0.2045) -- cycle;
\fill[blue!15.0, opacity=0.5] (2.5495, -0.0007, 0.2045) -- (2.5995, -0.0007, 0.1988) -- (2.5990, -0.0013, 0.2488) -- (2.5491, -0.0013, 0.2545) -- cycle;
\fill[blue!15.0, opacity=0.5] (2.5491, -0.0013, 0.2545) -- (2.5990, -0.0013, 0.2488) -- (2.5985, -0.0020, 0.2988) -- (2.5486, -0.0020, 0.3045) -- cycle;
\fill[blue!15.0, opacity=0.5] (2.5486, -0.0020, 0.3045) -- (2.5985, -0.0020, 0.2988) -- (2.5978, -0.0029, 0.3488) -- (2.5479, -0.0029, 0.3545) -- cycle;
\fill[blue!15.0, opacity=0.5] (2.5479, -0.0029, 0.3545) -- (2.5978, -0.0029, 0.3488) -- (2.5971, -0.0040, 0.3988) -- (2.5472, -0.0040, 0.4045) -- cycle;
\fill[blue!15.0, opacity=0.5] (2.5472, -0.0040, 0.4045) -- (2.5971, -0.0040, 0.3988) -- (2.5962, -0.0052, 0.4488) -- (2.5464, -0.0052, 0.4545) -- cycle;
\fill[blue!15.0, opacity=0.5] (2.5464, -0.0052, 0.4545) -- (2.5962, -0.0052, 0.4488) -- (2.5952, -0.0065, 0.4988) -- (2.5454, -0.0065, 0.5045) -- cycle;
\fill[blue!15.0, opacity=0.5] (2.5454, -0.0065, 0.5045) -- (2.5952, -0.0065, 0.4988) -- (2.5941, -0.0080, 0.5488) -- (2.5444, -0.0080, 0.5545) -- cycle;
\fill[blue!15.0, opacity=0.5] (2.5444, -0.0080, 0.5545) -- (2.5941, -0.0080, 0.5488) -- (2.5929, -0.0097, 0.5988) -- (2.5432, -0.0097, 0.6045) -- cycle;
\fill[blue!15.0, opacity=0.5] (2.5432, -0.0097, 0.6045) -- (2.5929, -0.0097, 0.5988) -- (2.5916, -0.0115, 0.6488) -- (2.5420, -0.0115, 0.6545) -- cycle;
\fill[blue!15.0, opacity=0.5] (2.5420, -0.0115, 0.6545) -- (2.5916, -0.0115, 0.6488) -- (2.5902, -0.0134, 0.6988) -- (2.5406, -0.0134, 0.7045) -- cycle;
\fill[blue!15.0, opacity=0.5] (2.5406, -0.0134, 0.7045) -- (2.5902, -0.0134, 0.6988) -- (2.5887, -0.0154, 0.7488) -- (2.5392, -0.0154, 0.7545) -- cycle;
\fill[blue!15.0, opacity=0.5] (2.5392, -0.0154, 0.7545) -- (2.5887, -0.0154, 0.7488) -- (2.5871, -0.0176, 0.7988) -- (2.5377, -0.0176, 0.8045) -- cycle;
\fill[blue!15.0, opacity=0.5] (2.5377, -0.0176, 0.8045) -- (2.5871, -0.0176, 0.7988) -- (2.5854, -0.0199, 0.8488) -- (2.5361, -0.0199, 0.8545) -- cycle;
\fill[blue!15.0, opacity=0.5] (2.5361, -0.0199, 0.8545) -- (2.5854, -0.0199, 0.8488) -- (2.5837, -0.0222, 0.8988) -- (2.5344, -0.0222, 0.9045) -- cycle;
\fill[blue!15.0, opacity=0.5] (2.5344, -0.0222, 0.9045) -- (2.5837, -0.0222, 0.8988) -- (2.5819, -0.0247, 0.9488) -- (2.5327, -0.0247, 0.9545) -- cycle;
\fill[blue!15.0, opacity=0.5] (2.5327, -0.0247, 0.9545) -- (2.5819, -0.0247, 0.9488) -- (2.5800, -0.0273, 0.9988) -- (2.5309, -0.0273, 1.0045) -- cycle;
\fill[blue!15.0, opacity=0.5] (2.5309, -0.0273, 1.0045) -- (2.5800, -0.0273, 0.9988) -- (2.5780, -0.0300, 1.0488) -- (2.5290, -0.0300, 1.0545) -- cycle;
\fill[blue!15.0, opacity=0.5] (2.5290, -0.0300, 1.0545) -- (2.5780, -0.0300, 1.0488) -- (2.5760, -0.0328, 1.0988) -- (2.5271, -0.0328, 1.1045) -- cycle;
\fill[blue!15.0, opacity=0.5] (2.5271, -0.0328, 1.1045) -- (2.5760, -0.0328, 1.0988) -- (2.5739, -0.0356, 1.1488) -- (2.5251, -0.0356, 1.1545) -- cycle;
\fill[blue!15.0, opacity=0.5] (2.5251, -0.0356, 1.1545) -- (2.5739, -0.0356, 1.1488) -- (2.5718, -0.0385, 1.1988) -- (2.5231, -0.0385, 1.2045) -- cycle;
\fill[blue!15.0, opacity=0.5] (2.5231, -0.0385, 1.2045) -- (2.5718, -0.0385, 1.1988) -- (2.5696, -0.0415, 1.2488) -- (2.5210, -0.0415, 1.2545) -- cycle;
\fill[blue!15.0, opacity=0.5] (2.5210, -0.0415, 1.2545) -- (2.5696, -0.0415, 1.2488) -- (2.5674, -0.0445, 1.2988) -- (2.5189, -0.0445, 1.3045) -- cycle;
\fill[blue!15.0, opacity=0.5] (2.5189, -0.0445, 1.3045) -- (2.5674, -0.0445, 1.2988) -- (2.5651, -0.0475, 1.3488) -- (2.5167, -0.0475, 1.3545) -- cycle;
\fill[blue!15.0, opacity=0.5] (2.5167, -0.0475, 1.3545) -- (2.5651, -0.0475, 1.3488) -- (2.5629, -0.0506, 1.3988) -- (2.5146, -0.0506, 1.4045) -- cycle;
\fill[blue!15.0, opacity=0.5] (2.5146, -0.0506, 1.4045) -- (2.5629, -0.0506, 1.3988) -- (2.5606, -0.0537, 1.4488) -- (2.5124, -0.0537, 1.4545) -- cycle;
\fill[blue!15.0, opacity=0.5] (2.5124, -0.0537, 1.4545) -- (2.5606, -0.0537, 1.4488) -- (2.5583, -0.0569, 1.4988) -- (2.5102, -0.0569, 1.5045) -- cycle;
\fill[blue!15.1, opacity=0.5] (2.5102, -0.0569, 1.5045) -- (2.5583, -0.0569, 1.4988) -- (2.5560, -0.0600, 1.5488) -- (2.5080, -0.0600, 1.5545) -- cycle;
\fill[blue!15.1, opacity=0.5] (2.5080, -0.0600, 1.5545) -- (2.5560, -0.0600, 1.5488) -- (2.5537, -0.0631, 1.5988) -- (2.5058, -0.0631, 1.6045) -- cycle;
\fill[blue!15.2, opacity=0.5] (2.5058, -0.0631, 1.6045) -- (2.5537, -0.0631, 1.5988) -- (2.5514, -0.0663, 1.6488) -- (2.5036, -0.0663, 1.6545) -- cycle;
\fill[blue!15.3, opacity=0.5] (2.5036, -0.0663, 1.6545) -- (2.5514, -0.0663, 1.6488) -- (2.5491, -0.0694, 1.6988) -- (2.5014, -0.0694, 1.7045) -- cycle;
\fill[blue!15.4, opacity=0.5] (2.5014, -0.0694, 1.7045) -- (2.5491, -0.0694, 1.6988) -- (2.5469, -0.0725, 1.7488) -- (2.4993, -0.0725, 1.7545) -- cycle;
\fill[blue!15.5, opacity=0.5] (2.4993, -0.0725, 1.7545) -- (2.5469, -0.0725, 1.7488) -- (2.5446, -0.0755, 1.7988) -- (2.4971, -0.0755, 1.8045) -- cycle;
\fill[blue!15.7, opacity=0.5] (2.4971, -0.0755, 1.8045) -- (2.5446, -0.0755, 1.7988) -- (2.5424, -0.0785, 1.8488) -- (2.4950, -0.0785, 1.8545) -- cycle;
\fill[blue!15.9, opacity=0.5] (2.4950, -0.0785, 1.8545) -- (2.5424, -0.0785, 1.8488) -- (2.5402, -0.0815, 1.8988) -- (2.4929, -0.0815, 1.9045) -- cycle;
\fill[blue!16.2, opacity=0.5] (2.4929, -0.0815, 1.9045) -- (2.5402, -0.0815, 1.8988) -- (2.5381, -0.0844, 1.9488) -- (2.4909, -0.0844, 1.9545) -- cycle;
\fill[blue!16.5, opacity=0.5] (2.4909, -0.0844, 1.9545) -- (2.5381, -0.0844, 1.9488) -- (2.5360, -0.0872, 1.9988) -- (2.4889, -0.0872, 2.0045) -- cycle;
\fill[blue!16.9, opacity=0.5] (2.4889, -0.0872, 2.0045) -- (2.5360, -0.0872, 1.9988) -- (2.5340, -0.0900, 2.0488) -- (2.4870, -0.0900, 2.0545) -- cycle;
\fill[blue!17.4, opacity=0.5] (2.4870, -0.0900, 2.0545) -- (2.5340, -0.0900, 2.0488) -- (2.5320, -0.0927, 2.0988) -- (2.4851, -0.0927, 2.1045) -- cycle;
\fill[blue!17.9, opacity=0.5] (2.4851, -0.0927, 2.1045) -- (2.5320, -0.0927, 2.0988) -- (2.5301, -0.0953, 2.1488) -- (2.4833, -0.0953, 2.1545) -- cycle;
\fill[blue!18.6, opacity=0.5] (2.4833, -0.0953, 2.1545) -- (2.5301, -0.0953, 2.1488) -- (2.5283, -0.0978, 2.1988) -- (2.4816, -0.0978, 2.2045) -- cycle;
\fill[blue!19.2, opacity=0.5] (2.4816, -0.0978, 2.2045) -- (2.5283, -0.0978, 2.1988) -- (2.5266, -0.1001, 2.2488) -- (2.4799, -0.1001, 2.2545) -- cycle;
\fill[blue!20.0, opacity=0.5] (2.4799, -0.1001, 2.2545) -- (2.5266, -0.1001, 2.2488) -- (2.5249, -0.1024, 2.2988) -- (2.4783, -0.1024, 2.3045) -- cycle;
\fill[blue!20.8, opacity=0.5] (2.4783, -0.1024, 2.3045) -- (2.5249, -0.1024, 2.2988) -- (2.5233, -0.1046, 2.3488) -- (2.4768, -0.1046, 2.3545) -- cycle;
\fill[blue!21.7, opacity=0.5] (2.4768, -0.1046, 2.3545) -- (2.5233, -0.1046, 2.3488) -- (2.5218, -0.1066, 2.3988) -- (2.4754, -0.1066, 2.4045) -- cycle;
\fill[blue!22.7, opacity=0.5] (2.4754, -0.1066, 2.4045) -- (2.5218, -0.1066, 2.3988) -- (2.5204, -0.1085, 2.4488) -- (2.4740, -0.1085, 2.4545) -- cycle;
\fill[blue!23.7, opacity=0.5] (2.4740, -0.1085, 2.4545) -- (2.5204, -0.1085, 2.4488) -- (2.5191, -0.1103, 2.4988) -- (2.4728, -0.1103, 2.5045) -- cycle;
\fill[blue!24.8, opacity=0.5] (2.4728, -0.1103, 2.5045) -- (2.5191, -0.1103, 2.4988) -- (2.5179, -0.1120, 2.5488) -- (2.4716, -0.1120, 2.5545) -- cycle;
\fill[blue!25.9, opacity=0.5] (2.4716, -0.1120, 2.5545) -- (2.5179, -0.1120, 2.5488) -- (2.5168, -0.1135, 2.5988) -- (2.4706, -0.1135, 2.6045) -- cycle;
\fill[blue!27.1, opacity=0.5] (2.4706, -0.1135, 2.6045) -- (2.5168, -0.1135, 2.5988) -- (2.5158, -0.1148, 2.6488) -- (2.4696, -0.1148, 2.6545) -- cycle;
\fill[blue!28.2, opacity=0.5] (2.4696, -0.1148, 2.6545) -- (2.5158, -0.1148, 2.6488) -- (2.5149, -0.1160, 2.6988) -- (2.4688, -0.1160, 2.7045) -- cycle;
\fill[blue!29.4, opacity=0.5] (2.4688, -0.1160, 2.7045) -- (2.5149, -0.1160, 2.6988) -- (2.5142, -0.1171, 2.7488) -- (2.4681, -0.1171, 2.7545) -- cycle;
\fill[blue!30.6, opacity=0.5] (2.4681, -0.1171, 2.7545) -- (2.5142, -0.1171, 2.7488) -- (2.5135, -0.1180, 2.7988) -- (2.4674, -0.1180, 2.8045) -- cycle;
\fill[blue!31.7, opacity=0.5] (2.4674, -0.1180, 2.8045) -- (2.5135, -0.1180, 2.7988) -- (2.5130, -0.1187, 2.8488) -- (2.4669, -0.1187, 2.8545) -- cycle;
\fill[blue!32.8, opacity=0.5] (2.4669, -0.1187, 2.8545) -- (2.5130, -0.1187, 2.8488) -- (2.5125, -0.1193, 2.8988) -- (2.4665, -0.1193, 2.9045) -- cycle;
\fill[blue!33.9, opacity=0.5] (2.4665, -0.1193, 2.9045) -- (2.5125, -0.1193, 2.8988) -- (2.5122, -0.1197, 2.9488) -- (2.4662, -0.1197, 2.9545) -- cycle;
\fill[blue!35.0, opacity=0.5] (2.4662, -0.1197, 2.9545) -- (2.5122, -0.1197, 2.9488) -- (2.5121, -0.1199, 2.9988) -- (2.4661, -0.1199, 3.0045) -- cycle;
\fill[blue!36.0, opacity=0.5] (2.4661, -0.1199, 3.0045) -- (2.5121, -0.1199, 2.9988) -- (2.5120, -0.1200, 3.0488) -- (2.4660, -0.1200, 3.0545) -- cycle;
\fill[blue!15.0, opacity=0.5] (2.6000, -0.0000, 0.0488) -- (2.6500, -0.0000, 0.0430) -- (2.6499, -0.0001, 0.0930) -- (2.5999, -0.0001, 0.0988) -- cycle;
\fill[blue!15.0, opacity=0.5] (2.5999, -0.0001, 0.0988) -- (2.6499, -0.0001, 0.0930) -- (2.6497, -0.0003, 0.1430) -- (2.5998, -0.0003, 0.1488) -- cycle;
\fill[blue!15.0, opacity=0.5] (2.5998, -0.0003, 0.1488) -- (2.6497, -0.0003, 0.1430) -- (2.6494, -0.0007, 0.1930) -- (2.5995, -0.0007, 0.1988) -- cycle;
\fill[blue!15.0, opacity=0.5] (2.5995, -0.0007, 0.1988) -- (2.6494, -0.0007, 0.1930) -- (2.6490, -0.0013, 0.2430) -- (2.5990, -0.0013, 0.2488) -- cycle;
\fill[blue!15.0, opacity=0.5] (2.5990, -0.0013, 0.2488) -- (2.6490, -0.0013, 0.2430) -- (2.6484, -0.0020, 0.2930) -- (2.5985, -0.0020, 0.2988) -- cycle;
\fill[blue!15.0, opacity=0.5] (2.5985, -0.0020, 0.2988) -- (2.6484, -0.0020, 0.2930) -- (2.6477, -0.0029, 0.3430) -- (2.5978, -0.0029, 0.3488) -- cycle;
\fill[blue!15.0, opacity=0.5] (2.5978, -0.0029, 0.3488) -- (2.6477, -0.0029, 0.3430) -- (2.6469, -0.0040, 0.3930) -- (2.5971, -0.0040, 0.3988) -- cycle;
\fill[blue!15.0, opacity=0.5] (2.5971, -0.0040, 0.3988) -- (2.6469, -0.0040, 0.3930) -- (2.6460, -0.0052, 0.4430) -- (2.5962, -0.0052, 0.4488) -- cycle;
\fill[blue!15.0, opacity=0.5] (2.5962, -0.0052, 0.4488) -- (2.6460, -0.0052, 0.4430) -- (2.6450, -0.0065, 0.4930) -- (2.5952, -0.0065, 0.4988) -- cycle;
\fill[blue!15.0, opacity=0.5] (2.5952, -0.0065, 0.4988) -- (2.6450, -0.0065, 0.4930) -- (2.6438, -0.0080, 0.5430) -- (2.5941, -0.0080, 0.5488) -- cycle;
\fill[blue!15.0, opacity=0.5] (2.5941, -0.0080, 0.5488) -- (2.6438, -0.0080, 0.5430) -- (2.6426, -0.0097, 0.5930) -- (2.5929, -0.0097, 0.5988) -- cycle;
\fill[blue!15.0, opacity=0.5] (2.5929, -0.0097, 0.5988) -- (2.6426, -0.0097, 0.5930) -- (2.6412, -0.0115, 0.6430) -- (2.5916, -0.0115, 0.6488) -- cycle;
\fill[blue!15.0, opacity=0.5] (2.5916, -0.0115, 0.6488) -- (2.6412, -0.0115, 0.6430) -- (2.6397, -0.0134, 0.6930) -- (2.5902, -0.0134, 0.6988) -- cycle;
\fill[blue!15.0, opacity=0.5] (2.5902, -0.0134, 0.6988) -- (2.6397, -0.0134, 0.6930) -- (2.6382, -0.0154, 0.7430) -- (2.5887, -0.0154, 0.7488) -- cycle;
\fill[blue!15.0, opacity=0.5] (2.5887, -0.0154, 0.7488) -- (2.6382, -0.0154, 0.7430) -- (2.6365, -0.0176, 0.7930) -- (2.5871, -0.0176, 0.7988) -- cycle;
\fill[blue!15.0, opacity=0.5] (2.5871, -0.0176, 0.7988) -- (2.6365, -0.0176, 0.7930) -- (2.6348, -0.0199, 0.8430) -- (2.5854, -0.0199, 0.8488) -- cycle;
\fill[blue!15.0, opacity=0.5] (2.5854, -0.0199, 0.8488) -- (2.6348, -0.0199, 0.8430) -- (2.6329, -0.0222, 0.8930) -- (2.5837, -0.0222, 0.8988) -- cycle;
\fill[blue!15.0, opacity=0.5] (2.5837, -0.0222, 0.8988) -- (2.6329, -0.0222, 0.8930) -- (2.6310, -0.0247, 0.9430) -- (2.5819, -0.0247, 0.9488) -- cycle;
\fill[blue!15.0, opacity=0.5] (2.5819, -0.0247, 0.9488) -- (2.6310, -0.0247, 0.9430) -- (2.6291, -0.0273, 0.9930) -- (2.5800, -0.0273, 0.9988) -- cycle;
\fill[blue!15.0, opacity=0.5] (2.5800, -0.0273, 0.9988) -- (2.6291, -0.0273, 0.9930) -- (2.6270, -0.0300, 1.0430) -- (2.5780, -0.0300, 1.0488) -- cycle;
\fill[blue!15.0, opacity=0.5] (2.5780, -0.0300, 1.0488) -- (2.6270, -0.0300, 1.0430) -- (2.6249, -0.0328, 1.0930) -- (2.5760, -0.0328, 1.0988) -- cycle;
\fill[blue!15.0, opacity=0.5] (2.5760, -0.0328, 1.0988) -- (2.6249, -0.0328, 1.0930) -- (2.6227, -0.0356, 1.1430) -- (2.5739, -0.0356, 1.1488) -- cycle;
\fill[blue!15.0, opacity=0.5] (2.5739, -0.0356, 1.1488) -- (2.6227, -0.0356, 1.1430) -- (2.6205, -0.0385, 1.1930) -- (2.5718, -0.0385, 1.1988) -- cycle;
\fill[blue!15.0, opacity=0.5] (2.5718, -0.0385, 1.1988) -- (2.6205, -0.0385, 1.1930) -- (2.6182, -0.0415, 1.2430) -- (2.5696, -0.0415, 1.2488) -- cycle;
\fill[blue!15.0, opacity=0.5] (2.5696, -0.0415, 1.2488) -- (2.6182, -0.0415, 1.2430) -- (2.6159, -0.0445, 1.2930) -- (2.5674, -0.0445, 1.2988) -- cycle;
\fill[blue!15.1, opacity=0.5] (2.5674, -0.0445, 1.2988) -- (2.6159, -0.0445, 1.2930) -- (2.6136, -0.0475, 1.3430) -- (2.5651, -0.0475, 1.3488) -- cycle;
\fill[blue!15.1, opacity=0.5] (2.5651, -0.0475, 1.3488) -- (2.6136, -0.0475, 1.3430) -- (2.6112, -0.0506, 1.3930) -- (2.5629, -0.0506, 1.3988) -- cycle;
\fill[blue!15.2, opacity=0.5] (2.5629, -0.0506, 1.3988) -- (2.6112, -0.0506, 1.3930) -- (2.6088, -0.0537, 1.4430) -- (2.5606, -0.0537, 1.4488) -- cycle;
\fill[blue!15.2, opacity=0.5] (2.5606, -0.0537, 1.4488) -- (2.6088, -0.0537, 1.4430) -- (2.6064, -0.0569, 1.4930) -- (2.5583, -0.0569, 1.4988) -- cycle;
\fill[blue!15.4, opacity=0.5] (2.5583, -0.0569, 1.4988) -- (2.6064, -0.0569, 1.4930) -- (2.6040, -0.0600, 1.5430) -- (2.5560, -0.0600, 1.5488) -- cycle;
\fill[blue!15.5, opacity=0.5] (2.5560, -0.0600, 1.5488) -- (2.6040, -0.0600, 1.5430) -- (2.6016, -0.0631, 1.5930) -- (2.5537, -0.0631, 1.5988) -- cycle;
\fill[blue!15.7, opacity=0.5] (2.5537, -0.0631, 1.5988) -- (2.6016, -0.0631, 1.5930) -- (2.5992, -0.0663, 1.6430) -- (2.5514, -0.0663, 1.6488) -- cycle;
\fill[blue!16.0, opacity=0.5] (2.5514, -0.0663, 1.6488) -- (2.5992, -0.0663, 1.6430) -- (2.5968, -0.0694, 1.6930) -- (2.5491, -0.0694, 1.6988) -- cycle;
\fill[blue!16.3, opacity=0.5] (2.5491, -0.0694, 1.6988) -- (2.5968, -0.0694, 1.6930) -- (2.5944, -0.0725, 1.7430) -- (2.5469, -0.0725, 1.7488) -- cycle;
\fill[blue!16.7, opacity=0.5] (2.5469, -0.0725, 1.7488) -- (2.5944, -0.0725, 1.7430) -- (2.5921, -0.0755, 1.7930) -- (2.5446, -0.0755, 1.7988) -- cycle;
\fill[blue!17.1, opacity=0.5] (2.5446, -0.0755, 1.7988) -- (2.5921, -0.0755, 1.7930) -- (2.5898, -0.0785, 1.8430) -- (2.5424, -0.0785, 1.8488) -- cycle;
\fill[blue!17.7, opacity=0.5] (2.5424, -0.0785, 1.8488) -- (2.5898, -0.0785, 1.8430) -- (2.5875, -0.0815, 1.8930) -- (2.5402, -0.0815, 1.8988) -- cycle;
\fill[blue!18.3, opacity=0.5] (2.5402, -0.0815, 1.8988) -- (2.5875, -0.0815, 1.8930) -- (2.5853, -0.0844, 1.9430) -- (2.5381, -0.0844, 1.9488) -- cycle;
\fill[blue!19.1, opacity=0.5] (2.5381, -0.0844, 1.9488) -- (2.5853, -0.0844, 1.9430) -- (2.5831, -0.0872, 1.9930) -- (2.5360, -0.0872, 1.9988) -- cycle;
\fill[blue!19.9, opacity=0.5] (2.5360, -0.0872, 1.9988) -- (2.5831, -0.0872, 1.9930) -- (2.5810, -0.0900, 2.0430) -- (2.5340, -0.0900, 2.0488) -- cycle;
\fill[blue!20.8, opacity=0.5] (2.5340, -0.0900, 2.0488) -- (2.5810, -0.0900, 2.0430) -- (2.5789, -0.0927, 2.0930) -- (2.5320, -0.0927, 2.0988) -- cycle;
\fill[blue!21.8, opacity=0.5] (2.5320, -0.0927, 2.0988) -- (2.5789, -0.0927, 2.0930) -- (2.5770, -0.0953, 2.1430) -- (2.5301, -0.0953, 2.1488) -- cycle;
\fill[blue!22.9, opacity=0.5] (2.5301, -0.0953, 2.1488) -- (2.5770, -0.0953, 2.1430) -- (2.5751, -0.0978, 2.1930) -- (2.5283, -0.0978, 2.1988) -- cycle;
\fill[blue!24.1, opacity=0.5] (2.5283, -0.0978, 2.1988) -- (2.5751, -0.0978, 2.1930) -- (2.5732, -0.1001, 2.2430) -- (2.5266, -0.1001, 2.2488) -- cycle;
\fill[blue!25.4, opacity=0.5] (2.5266, -0.1001, 2.2488) -- (2.5732, -0.1001, 2.2430) -- (2.5715, -0.1024, 2.2930) -- (2.5249, -0.1024, 2.2988) -- cycle;
\fill[blue!26.7, opacity=0.5] (2.5249, -0.1024, 2.2988) -- (2.5715, -0.1024, 2.2930) -- (2.5698, -0.1046, 2.3430) -- (2.5233, -0.1046, 2.3488) -- cycle;
\fill[blue!28.0, opacity=0.5] (2.5233, -0.1046, 2.3488) -- (2.5698, -0.1046, 2.3430) -- (2.5683, -0.1066, 2.3930) -- (2.5218, -0.1066, 2.3988) -- cycle;
\fill[blue!29.4, opacity=0.5] (2.5218, -0.1066, 2.3988) -- (2.5683, -0.1066, 2.3930) -- (2.5668, -0.1085, 2.4430) -- (2.5204, -0.1085, 2.4488) -- cycle;
\fill[blue!30.9, opacity=0.5] (2.5204, -0.1085, 2.4488) -- (2.5668, -0.1085, 2.4430) -- (2.5654, -0.1103, 2.4930) -- (2.5191, -0.1103, 2.4988) -- cycle;
\fill[blue!32.3, opacity=0.5] (2.5191, -0.1103, 2.4988) -- (2.5654, -0.1103, 2.4930) -- (2.5642, -0.1120, 2.5430) -- (2.5179, -0.1120, 2.5488) -- cycle;
\fill[blue!33.7, opacity=0.5] (2.5179, -0.1120, 2.5488) -- (2.5642, -0.1120, 2.5430) -- (2.5630, -0.1135, 2.5930) -- (2.5168, -0.1135, 2.5988) -- cycle;
\fill[blue!35.1, opacity=0.5] (2.5168, -0.1135, 2.5988) -- (2.5630, -0.1135, 2.5930) -- (2.5620, -0.1148, 2.6430) -- (2.5158, -0.1148, 2.6488) -- cycle;
\fill[blue!36.5, opacity=0.5] (2.5158, -0.1148, 2.6488) -- (2.5620, -0.1148, 2.6430) -- (2.5611, -0.1160, 2.6930) -- (2.5149, -0.1160, 2.6988) -- cycle;
\fill[blue!37.8, opacity=0.5] (2.5149, -0.1160, 2.6988) -- (2.5611, -0.1160, 2.6930) -- (2.5603, -0.1171, 2.7430) -- (2.5142, -0.1171, 2.7488) -- cycle;
\fill[blue!39.1, opacity=0.5] (2.5142, -0.1171, 2.7488) -- (2.5603, -0.1171, 2.7430) -- (2.5596, -0.1180, 2.7930) -- (2.5135, -0.1180, 2.7988) -- cycle;
\fill[blue!40.3, opacity=0.5] (2.5135, -0.1180, 2.7988) -- (2.5596, -0.1180, 2.7930) -- (2.5590, -0.1187, 2.8430) -- (2.5130, -0.1187, 2.8488) -- cycle;
\fill[blue!41.5, opacity=0.5] (2.5130, -0.1187, 2.8488) -- (2.5590, -0.1187, 2.8430) -- (2.5586, -0.1193, 2.8930) -- (2.5125, -0.1193, 2.8988) -- cycle;
\fill[blue!42.5, opacity=0.5] (2.5125, -0.1193, 2.8988) -- (2.5586, -0.1193, 2.8930) -- (2.5583, -0.1197, 2.9430) -- (2.5122, -0.1197, 2.9488) -- cycle;
\fill[blue!43.4, opacity=0.5] (2.5122, -0.1197, 2.9488) -- (2.5583, -0.1197, 2.9430) -- (2.5581, -0.1199, 2.9930) -- (2.5121, -0.1199, 2.9988) -- cycle;
\fill[blue!44.3, opacity=0.5] (2.5121, -0.1199, 2.9988) -- (2.5581, -0.1199, 2.9930) -- (2.5580, -0.1200, 3.0430) -- (2.5120, -0.1200, 3.0488) -- cycle;
\fill[blue!15.0, opacity=0.5] (2.6500, -0.0000, 0.0430) -- (2.7000, -0.0000, 0.0371) -- (2.6999, -0.0001, 0.0871) -- (2.6499, -0.0001, 0.0930) -- cycle;
\fill[blue!15.0, opacity=0.5] (2.6499, -0.0001, 0.0930) -- (2.6999, -0.0001, 0.0871) -- (2.6997, -0.0003, 0.1371) -- (2.6497, -0.0003, 0.1430) -- cycle;
\fill[blue!15.0, opacity=0.5] (2.6497, -0.0003, 0.1430) -- (2.6997, -0.0003, 0.1371) -- (2.6994, -0.0007, 0.1871) -- (2.6494, -0.0007, 0.1930) -- cycle;
\fill[blue!15.0, opacity=0.5] (2.6494, -0.0007, 0.1930) -- (2.6994, -0.0007, 0.1871) -- (2.6990, -0.0013, 0.2371) -- (2.6490, -0.0013, 0.2430) -- cycle;
\fill[blue!15.0, opacity=0.5] (2.6490, -0.0013, 0.2430) -- (2.6990, -0.0013, 0.2371) -- (2.6984, -0.0020, 0.2871) -- (2.6484, -0.0020, 0.2930) -- cycle;
\fill[blue!15.0, opacity=0.5] (2.6484, -0.0020, 0.2930) -- (2.6984, -0.0020, 0.2871) -- (2.6977, -0.0029, 0.3371) -- (2.6477, -0.0029, 0.3430) -- cycle;
\fill[blue!15.0, opacity=0.5] (2.6477, -0.0029, 0.3430) -- (2.6977, -0.0029, 0.3371) -- (2.6968, -0.0040, 0.3871) -- (2.6469, -0.0040, 0.3930) -- cycle;
\fill[blue!15.0, opacity=0.5] (2.6469, -0.0040, 0.3930) -- (2.6968, -0.0040, 0.3871) -- (2.6959, -0.0052, 0.4371) -- (2.6460, -0.0052, 0.4430) -- cycle;
\fill[blue!15.0, opacity=0.5] (2.6460, -0.0052, 0.4430) -- (2.6959, -0.0052, 0.4371) -- (2.6948, -0.0065, 0.4871) -- (2.6450, -0.0065, 0.4930) -- cycle;
\fill[blue!15.0, opacity=0.5] (2.6450, -0.0065, 0.4930) -- (2.6948, -0.0065, 0.4871) -- (2.6936, -0.0080, 0.5371) -- (2.6438, -0.0080, 0.5430) -- cycle;
\fill[blue!15.0, opacity=0.5] (2.6438, -0.0080, 0.5430) -- (2.6936, -0.0080, 0.5371) -- (2.6923, -0.0097, 0.5871) -- (2.6426, -0.0097, 0.5930) -- cycle;
\fill[blue!15.0, opacity=0.5] (2.6426, -0.0097, 0.5930) -- (2.6923, -0.0097, 0.5871) -- (2.6908, -0.0115, 0.6371) -- (2.6412, -0.0115, 0.6430) -- cycle;
\fill[blue!15.0, opacity=0.5] (2.6412, -0.0115, 0.6430) -- (2.6908, -0.0115, 0.6371) -- (2.6893, -0.0134, 0.6871) -- (2.6397, -0.0134, 0.6930) -- cycle;
\fill[blue!15.0, opacity=0.5] (2.6397, -0.0134, 0.6930) -- (2.6893, -0.0134, 0.6871) -- (2.6877, -0.0154, 0.7371) -- (2.6382, -0.0154, 0.7430) -- cycle;
\fill[blue!15.0, opacity=0.5] (2.6382, -0.0154, 0.7430) -- (2.6877, -0.0154, 0.7371) -- (2.6859, -0.0176, 0.7871) -- (2.6365, -0.0176, 0.7930) -- cycle;
\fill[blue!15.0, opacity=0.5] (2.6365, -0.0176, 0.7930) -- (2.6859, -0.0176, 0.7871) -- (2.6841, -0.0199, 0.8371) -- (2.6348, -0.0199, 0.8430) -- cycle;
\fill[blue!15.0, opacity=0.5] (2.6348, -0.0199, 0.8430) -- (2.6841, -0.0199, 0.8371) -- (2.6822, -0.0222, 0.8871) -- (2.6329, -0.0222, 0.8930) -- cycle;
\fill[blue!15.0, opacity=0.5] (2.6329, -0.0222, 0.8930) -- (2.6822, -0.0222, 0.8871) -- (2.6802, -0.0247, 0.9371) -- (2.6310, -0.0247, 0.9430) -- cycle;
\fill[blue!15.0, opacity=0.5] (2.6310, -0.0247, 0.9430) -- (2.6802, -0.0247, 0.9371) -- (2.6781, -0.0273, 0.9871) -- (2.6291, -0.0273, 0.9930) -- cycle;
\fill[blue!15.0, opacity=0.5] (2.6291, -0.0273, 0.9930) -- (2.6781, -0.0273, 0.9871) -- (2.6760, -0.0300, 1.0371) -- (2.6270, -0.0300, 1.0430) -- cycle;
\fill[blue!15.0, opacity=0.5] (2.6270, -0.0300, 1.0430) -- (2.6760, -0.0300, 1.0371) -- (2.6738, -0.0328, 1.0871) -- (2.6249, -0.0328, 1.0930) -- cycle;
\fill[blue!15.0, opacity=0.5] (2.6249, -0.0328, 1.0930) -- (2.6738, -0.0328, 1.0871) -- (2.6715, -0.0356, 1.1371) -- (2.6227, -0.0356, 1.1430) -- cycle;
\fill[blue!15.0, opacity=0.5] (2.6227, -0.0356, 1.1430) -- (2.6715, -0.0356, 1.1371) -- (2.6692, -0.0385, 1.1871) -- (2.6205, -0.0385, 1.1930) -- cycle;
\fill[blue!15.0, opacity=0.5] (2.6205, -0.0385, 1.1930) -- (2.6692, -0.0385, 1.1871) -- (2.6668, -0.0415, 1.2371) -- (2.6182, -0.0415, 1.2430) -- cycle;
\fill[blue!15.0, opacity=0.5] (2.6182, -0.0415, 1.2430) -- (2.6668, -0.0415, 1.2371) -- (2.6644, -0.0445, 1.2871) -- (2.6159, -0.0445, 1.2930) -- cycle;
\fill[blue!15.1, opacity=0.5] (2.6159, -0.0445, 1.2930) -- (2.6644, -0.0445, 1.2871) -- (2.6620, -0.0475, 1.3371) -- (2.6136, -0.0475, 1.3430) -- cycle;
\fill[blue!15.1, opacity=0.5] (2.6136, -0.0475, 1.3430) -- (2.6620, -0.0475, 1.3371) -- (2.6595, -0.0506, 1.3871) -- (2.6112, -0.0506, 1.3930) -- cycle;
\fill[blue!15.2, opacity=0.5] (2.6112, -0.0506, 1.3930) -- (2.6595, -0.0506, 1.3871) -- (2.6570, -0.0537, 1.4371) -- (2.6088, -0.0537, 1.4430) -- cycle;
\fill[blue!15.3, opacity=0.5] (2.6088, -0.0537, 1.4430) -- (2.6570, -0.0537, 1.4371) -- (2.6545, -0.0569, 1.4871) -- (2.6064, -0.0569, 1.4930) -- cycle;
\fill[blue!15.4, opacity=0.5] (2.6064, -0.0569, 1.4930) -- (2.6545, -0.0569, 1.4871) -- (2.6520, -0.0600, 1.5371) -- (2.6040, -0.0600, 1.5430) -- cycle;
\fill[blue!15.6, opacity=0.5] (2.6040, -0.0600, 1.5430) -- (2.6520, -0.0600, 1.5371) -- (2.6495, -0.0631, 1.5871) -- (2.6016, -0.0631, 1.5930) -- cycle;
\fill[blue!15.8, opacity=0.5] (2.6016, -0.0631, 1.5930) -- (2.6495, -0.0631, 1.5871) -- (2.6470, -0.0663, 1.6371) -- (2.5992, -0.0663, 1.6430) -- cycle;
\fill[blue!16.1, opacity=0.5] (2.5992, -0.0663, 1.6430) -- (2.6470, -0.0663, 1.6371) -- (2.6445, -0.0694, 1.6871) -- (2.5968, -0.0694, 1.6930) -- cycle;
\fill[blue!16.5, opacity=0.5] (2.5968, -0.0694, 1.6930) -- (2.6445, -0.0694, 1.6871) -- (2.6420, -0.0725, 1.7371) -- (2.5944, -0.0725, 1.7430) -- cycle;
\fill[blue!16.9, opacity=0.5] (2.5944, -0.0725, 1.7430) -- (2.6420, -0.0725, 1.7371) -- (2.6396, -0.0755, 1.7871) -- (2.5921, -0.0755, 1.7930) -- cycle;
\fill[blue!17.4, opacity=0.5] (2.5921, -0.0755, 1.7930) -- (2.6396, -0.0755, 1.7871) -- (2.6372, -0.0785, 1.8371) -- (2.5898, -0.0785, 1.8430) -- cycle;
\fill[blue!18.0, opacity=0.5] (2.5898, -0.0785, 1.8430) -- (2.6372, -0.0785, 1.8371) -- (2.6348, -0.0815, 1.8871) -- (2.5875, -0.0815, 1.8930) -- cycle;
\fill[blue!18.7, opacity=0.5] (2.5875, -0.0815, 1.8930) -- (2.6348, -0.0815, 1.8871) -- (2.6325, -0.0844, 1.9371) -- (2.5853, -0.0844, 1.9430) -- cycle;
\fill[blue!19.5, opacity=0.5] (2.5853, -0.0844, 1.9430) -- (2.6325, -0.0844, 1.9371) -- (2.6302, -0.0872, 1.9871) -- (2.5831, -0.0872, 1.9930) -- cycle;
\fill[blue!20.4, opacity=0.5] (2.5831, -0.0872, 1.9930) -- (2.6302, -0.0872, 1.9871) -- (2.6280, -0.0900, 2.0371) -- (2.5810, -0.0900, 2.0430) -- cycle;
\fill[blue!21.4, opacity=0.5] (2.5810, -0.0900, 2.0430) -- (2.6280, -0.0900, 2.0371) -- (2.6259, -0.0927, 2.0871) -- (2.5789, -0.0927, 2.0930) -- cycle;
\fill[blue!22.5, opacity=0.5] (2.5789, -0.0927, 2.0930) -- (2.6259, -0.0927, 2.0871) -- (2.6238, -0.0953, 2.1371) -- (2.5770, -0.0953, 2.1430) -- cycle;
\fill[blue!23.7, opacity=0.5] (2.5770, -0.0953, 2.1430) -- (2.6238, -0.0953, 2.1371) -- (2.6218, -0.0978, 2.1871) -- (2.5751, -0.0978, 2.1930) -- cycle;
\fill[blue!24.9, opacity=0.5] (2.5751, -0.0978, 2.1930) -- (2.6218, -0.0978, 2.1871) -- (2.6199, -0.1001, 2.2371) -- (2.5732, -0.1001, 2.2430) -- cycle;
\fill[blue!26.2, opacity=0.5] (2.5732, -0.1001, 2.2430) -- (2.6199, -0.1001, 2.2371) -- (2.6181, -0.1024, 2.2871) -- (2.5715, -0.1024, 2.2930) -- cycle;
\fill[blue!27.6, opacity=0.5] (2.5715, -0.1024, 2.2930) -- (2.6181, -0.1024, 2.2871) -- (2.6163, -0.1046, 2.3371) -- (2.5698, -0.1046, 2.3430) -- cycle;
\fill[blue!29.0, opacity=0.5] (2.5698, -0.1046, 2.3430) -- (2.6163, -0.1046, 2.3371) -- (2.6147, -0.1066, 2.3871) -- (2.5683, -0.1066, 2.3930) -- cycle;
\fill[blue!30.5, opacity=0.5] (2.5683, -0.1066, 2.3930) -- (2.6147, -0.1066, 2.3871) -- (2.6132, -0.1085, 2.4371) -- (2.5668, -0.1085, 2.4430) -- cycle;
\fill[blue!31.9, opacity=0.5] (2.5668, -0.1085, 2.4430) -- (2.6132, -0.1085, 2.4371) -- (2.6117, -0.1103, 2.4871) -- (2.5654, -0.1103, 2.4930) -- cycle;
\fill[blue!33.4, opacity=0.5] (2.5654, -0.1103, 2.4930) -- (2.6117, -0.1103, 2.4871) -- (2.6104, -0.1120, 2.5371) -- (2.5642, -0.1120, 2.5430) -- cycle;
\fill[blue!34.9, opacity=0.5] (2.5642, -0.1120, 2.5430) -- (2.6104, -0.1120, 2.5371) -- (2.6092, -0.1135, 2.5871) -- (2.5630, -0.1135, 2.5930) -- cycle;
\fill[blue!36.3, opacity=0.5] (2.5630, -0.1135, 2.5930) -- (2.6092, -0.1135, 2.5871) -- (2.6081, -0.1148, 2.6371) -- (2.5620, -0.1148, 2.6430) -- cycle;
\fill[blue!37.7, opacity=0.5] (2.5620, -0.1148, 2.6430) -- (2.6081, -0.1148, 2.6371) -- (2.6072, -0.1160, 2.6871) -- (2.5611, -0.1160, 2.6930) -- cycle;
\fill[blue!39.0, opacity=0.5] (2.5611, -0.1160, 2.6930) -- (2.6072, -0.1160, 2.6871) -- (2.6063, -0.1171, 2.7371) -- (2.5603, -0.1171, 2.7430) -- cycle;
\fill[blue!40.3, opacity=0.5] (2.5603, -0.1171, 2.7430) -- (2.6063, -0.1171, 2.7371) -- (2.6056, -0.1180, 2.7871) -- (2.5596, -0.1180, 2.7930) -- cycle;
\fill[blue!41.5, opacity=0.5] (2.5596, -0.1180, 2.7930) -- (2.6056, -0.1180, 2.7871) -- (2.6050, -0.1187, 2.8371) -- (2.5590, -0.1187, 2.8430) -- cycle;
\fill[blue!42.6, opacity=0.5] (2.5590, -0.1187, 2.8430) -- (2.6050, -0.1187, 2.8371) -- (2.6046, -0.1193, 2.8871) -- (2.5586, -0.1193, 2.8930) -- cycle;
\fill[blue!43.6, opacity=0.5] (2.5586, -0.1193, 2.8930) -- (2.6046, -0.1193, 2.8871) -- (2.6043, -0.1197, 2.9371) -- (2.5583, -0.1197, 2.9430) -- cycle;
\fill[blue!44.5, opacity=0.5] (2.5583, -0.1197, 2.9430) -- (2.6043, -0.1197, 2.9371) -- (2.6041, -0.1199, 2.9871) -- (2.5581, -0.1199, 2.9930) -- cycle;
\fill[blue!45.3, opacity=0.5] (2.5581, -0.1199, 2.9930) -- (2.6041, -0.1199, 2.9871) -- (2.6040, -0.1200, 3.0371) -- (2.5580, -0.1200, 3.0430) -- cycle;
\fill[blue!15.0, opacity=0.5] (2.7000, -0.0000, 0.0371) -- (2.7500, -0.0000, 0.0311) -- (2.7499, -0.0001, 0.0811) -- (2.6999, -0.0001, 0.0871) -- cycle;
\fill[blue!15.0, opacity=0.5] (2.6999, -0.0001, 0.0871) -- (2.7499, -0.0001, 0.0811) -- (2.7497, -0.0003, 0.1311) -- (2.6997, -0.0003, 0.1371) -- cycle;
\fill[blue!15.0, opacity=0.5] (2.6997, -0.0003, 0.1371) -- (2.7497, -0.0003, 0.1311) -- (2.7494, -0.0007, 0.1811) -- (2.6994, -0.0007, 0.1871) -- cycle;
\fill[blue!15.0, opacity=0.5] (2.6994, -0.0007, 0.1871) -- (2.7494, -0.0007, 0.1811) -- (2.7489, -0.0013, 0.2311) -- (2.6990, -0.0013, 0.2371) -- cycle;
\fill[blue!15.0, opacity=0.5] (2.6990, -0.0013, 0.2371) -- (2.7489, -0.0013, 0.2311) -- (2.7483, -0.0020, 0.2811) -- (2.6984, -0.0020, 0.2871) -- cycle;
\fill[blue!15.0, opacity=0.5] (2.6984, -0.0020, 0.2871) -- (2.7483, -0.0020, 0.2811) -- (2.7476, -0.0029, 0.3311) -- (2.6977, -0.0029, 0.3371) -- cycle;
\fill[blue!15.0, opacity=0.5] (2.6977, -0.0029, 0.3371) -- (2.7476, -0.0029, 0.3311) -- (2.7467, -0.0040, 0.3811) -- (2.6968, -0.0040, 0.3871) -- cycle;
\fill[blue!15.0, opacity=0.5] (2.6968, -0.0040, 0.3871) -- (2.7467, -0.0040, 0.3811) -- (2.7457, -0.0052, 0.4311) -- (2.6959, -0.0052, 0.4371) -- cycle;
\fill[blue!15.0, opacity=0.5] (2.6959, -0.0052, 0.4371) -- (2.7457, -0.0052, 0.4311) -- (2.7446, -0.0065, 0.4811) -- (2.6948, -0.0065, 0.4871) -- cycle;
\fill[blue!15.0, opacity=0.5] (2.6948, -0.0065, 0.4871) -- (2.7446, -0.0065, 0.4811) -- (2.7433, -0.0080, 0.5311) -- (2.6936, -0.0080, 0.5371) -- cycle;
\fill[blue!15.0, opacity=0.5] (2.6936, -0.0080, 0.5371) -- (2.7433, -0.0080, 0.5311) -- (2.7419, -0.0097, 0.5811) -- (2.6923, -0.0097, 0.5871) -- cycle;
\fill[blue!15.0, opacity=0.5] (2.6923, -0.0097, 0.5871) -- (2.7419, -0.0097, 0.5811) -- (2.7405, -0.0115, 0.6311) -- (2.6908, -0.0115, 0.6371) -- cycle;
\fill[blue!15.0, opacity=0.5] (2.6908, -0.0115, 0.6371) -- (2.7405, -0.0115, 0.6311) -- (2.7389, -0.0134, 0.6811) -- (2.6893, -0.0134, 0.6871) -- cycle;
\fill[blue!15.0, opacity=0.5] (2.6893, -0.0134, 0.6871) -- (2.7389, -0.0134, 0.6811) -- (2.7372, -0.0154, 0.7311) -- (2.6877, -0.0154, 0.7371) -- cycle;
\fill[blue!15.0, opacity=0.5] (2.6877, -0.0154, 0.7371) -- (2.7372, -0.0154, 0.7311) -- (2.7354, -0.0176, 0.7811) -- (2.6859, -0.0176, 0.7871) -- cycle;
\fill[blue!15.0, opacity=0.5] (2.6859, -0.0176, 0.7871) -- (2.7354, -0.0176, 0.7811) -- (2.7335, -0.0199, 0.8311) -- (2.6841, -0.0199, 0.8371) -- cycle;
\fill[blue!15.0, opacity=0.5] (2.6841, -0.0199, 0.8371) -- (2.7335, -0.0199, 0.8311) -- (2.7315, -0.0222, 0.8811) -- (2.6822, -0.0222, 0.8871) -- cycle;
\fill[blue!15.0, opacity=0.5] (2.6822, -0.0222, 0.8871) -- (2.7315, -0.0222, 0.8811) -- (2.7294, -0.0247, 0.9311) -- (2.6802, -0.0247, 0.9371) -- cycle;
\fill[blue!15.0, opacity=0.5] (2.6802, -0.0247, 0.9371) -- (2.7294, -0.0247, 0.9311) -- (2.7272, -0.0273, 0.9811) -- (2.6781, -0.0273, 0.9871) -- cycle;
\fill[blue!15.0, opacity=0.5] (2.6781, -0.0273, 0.9871) -- (2.7272, -0.0273, 0.9811) -- (2.7250, -0.0300, 1.0311) -- (2.6760, -0.0300, 1.0371) -- cycle;
\fill[blue!15.0, opacity=0.5] (2.6760, -0.0300, 1.0371) -- (2.7250, -0.0300, 1.0311) -- (2.7227, -0.0328, 1.0811) -- (2.6738, -0.0328, 1.0871) -- cycle;
\fill[blue!15.0, opacity=0.5] (2.6738, -0.0328, 1.0871) -- (2.7227, -0.0328, 1.0811) -- (2.7203, -0.0356, 1.1311) -- (2.6715, -0.0356, 1.1371) -- cycle;
\fill[blue!15.0, opacity=0.5] (2.6715, -0.0356, 1.1371) -- (2.7203, -0.0356, 1.1311) -- (2.7179, -0.0385, 1.1811) -- (2.6692, -0.0385, 1.1871) -- cycle;
\fill[blue!15.0, opacity=0.5] (2.6692, -0.0385, 1.1871) -- (2.7179, -0.0385, 1.1811) -- (2.7155, -0.0415, 1.2311) -- (2.6668, -0.0415, 1.2371) -- cycle;
\fill[blue!15.0, opacity=0.5] (2.6668, -0.0415, 1.2371) -- (2.7155, -0.0415, 1.2311) -- (2.7129, -0.0445, 1.2811) -- (2.6644, -0.0445, 1.2871) -- cycle;
\fill[blue!15.0, opacity=0.5] (2.6644, -0.0445, 1.2871) -- (2.7129, -0.0445, 1.2811) -- (2.7104, -0.0475, 1.3311) -- (2.6620, -0.0475, 1.3371) -- cycle;
\fill[blue!15.0, opacity=0.5] (2.6620, -0.0475, 1.3371) -- (2.7104, -0.0475, 1.3311) -- (2.7078, -0.0506, 1.3811) -- (2.6595, -0.0506, 1.3871) -- cycle;
\fill[blue!15.0, opacity=0.5] (2.6595, -0.0506, 1.3871) -- (2.7078, -0.0506, 1.3811) -- (2.7052, -0.0537, 1.4311) -- (2.6570, -0.0537, 1.4371) -- cycle;
\fill[blue!15.1, opacity=0.5] (2.6570, -0.0537, 1.4371) -- (2.7052, -0.0537, 1.4311) -- (2.7026, -0.0569, 1.4811) -- (2.6545, -0.0569, 1.4871) -- cycle;
\fill[blue!15.1, opacity=0.5] (2.6545, -0.0569, 1.4871) -- (2.7026, -0.0569, 1.4811) -- (2.7000, -0.0600, 1.5311) -- (2.6520, -0.0600, 1.5371) -- cycle;
\fill[blue!15.2, opacity=0.5] (2.6520, -0.0600, 1.5371) -- (2.7000, -0.0600, 1.5311) -- (2.6974, -0.0631, 1.5811) -- (2.6495, -0.0631, 1.5871) -- cycle;
\fill[blue!15.2, opacity=0.5] (2.6495, -0.0631, 1.5871) -- (2.6974, -0.0631, 1.5811) -- (2.6948, -0.0663, 1.6311) -- (2.6470, -0.0663, 1.6371) -- cycle;
\fill[blue!15.3, opacity=0.5] (2.6470, -0.0663, 1.6371) -- (2.6948, -0.0663, 1.6311) -- (2.6922, -0.0694, 1.6811) -- (2.6445, -0.0694, 1.6871) -- cycle;
\fill[blue!15.5, opacity=0.5] (2.6445, -0.0694, 1.6871) -- (2.6922, -0.0694, 1.6811) -- (2.6896, -0.0725, 1.7311) -- (2.6420, -0.0725, 1.7371) -- cycle;
\fill[blue!15.7, opacity=0.5] (2.6420, -0.0725, 1.7371) -- (2.6896, -0.0725, 1.7311) -- (2.6871, -0.0755, 1.7811) -- (2.6396, -0.0755, 1.7871) -- cycle;
\fill[blue!15.9, opacity=0.5] (2.6396, -0.0755, 1.7871) -- (2.6871, -0.0755, 1.7811) -- (2.6845, -0.0785, 1.8311) -- (2.6372, -0.0785, 1.8371) -- cycle;
\fill[blue!16.2, opacity=0.5] (2.6372, -0.0785, 1.8371) -- (2.6845, -0.0785, 1.8311) -- (2.6821, -0.0815, 1.8811) -- (2.6348, -0.0815, 1.8871) -- cycle;
\fill[blue!16.5, opacity=0.5] (2.6348, -0.0815, 1.8871) -- (2.6821, -0.0815, 1.8811) -- (2.6797, -0.0844, 1.9311) -- (2.6325, -0.0844, 1.9371) -- cycle;
\fill[blue!16.9, opacity=0.5] (2.6325, -0.0844, 1.9371) -- (2.6797, -0.0844, 1.9311) -- (2.6773, -0.0872, 1.9811) -- (2.6302, -0.0872, 1.9871) -- cycle;
\fill[blue!17.4, opacity=0.5] (2.6302, -0.0872, 1.9871) -- (2.6773, -0.0872, 1.9811) -- (2.6750, -0.0900, 2.0311) -- (2.6280, -0.0900, 2.0371) -- cycle;
\fill[blue!17.9, opacity=0.5] (2.6280, -0.0900, 2.0371) -- (2.6750, -0.0900, 2.0311) -- (2.6728, -0.0927, 2.0811) -- (2.6259, -0.0927, 2.0871) -- cycle;
\fill[blue!18.5, opacity=0.5] (2.6259, -0.0927, 2.0871) -- (2.6728, -0.0927, 2.0811) -- (2.6706, -0.0953, 2.1311) -- (2.6238, -0.0953, 2.1371) -- cycle;
\fill[blue!19.3, opacity=0.5] (2.6238, -0.0953, 2.1371) -- (2.6706, -0.0953, 2.1311) -- (2.6685, -0.0978, 2.1811) -- (2.6218, -0.0978, 2.1871) -- cycle;
\fill[blue!20.0, opacity=0.5] (2.6218, -0.0978, 2.1871) -- (2.6685, -0.0978, 2.1811) -- (2.6665, -0.1001, 2.2311) -- (2.6199, -0.1001, 2.2371) -- cycle;
\fill[blue!20.9, opacity=0.5] (2.6199, -0.1001, 2.2371) -- (2.6665, -0.1001, 2.2311) -- (2.6646, -0.1024, 2.2811) -- (2.6181, -0.1024, 2.2871) -- cycle;
\fill[blue!21.8, opacity=0.5] (2.6181, -0.1024, 2.2871) -- (2.6646, -0.1024, 2.2811) -- (2.6628, -0.1046, 2.3311) -- (2.6163, -0.1046, 2.3371) -- cycle;
\fill[blue!22.8, opacity=0.5] (2.6163, -0.1046, 2.3371) -- (2.6628, -0.1046, 2.3311) -- (2.6611, -0.1066, 2.3811) -- (2.6147, -0.1066, 2.3871) -- cycle;
\fill[blue!23.9, opacity=0.5] (2.6147, -0.1066, 2.3871) -- (2.6611, -0.1066, 2.3811) -- (2.6595, -0.1085, 2.4311) -- (2.6132, -0.1085, 2.4371) -- cycle;
\fill[blue!25.0, opacity=0.5] (2.6132, -0.1085, 2.4371) -- (2.6595, -0.1085, 2.4311) -- (2.6581, -0.1103, 2.4811) -- (2.6117, -0.1103, 2.4871) -- cycle;
\fill[blue!26.2, opacity=0.5] (2.6117, -0.1103, 2.4871) -- (2.6581, -0.1103, 2.4811) -- (2.6567, -0.1120, 2.5311) -- (2.6104, -0.1120, 2.5371) -- cycle;
\fill[blue!27.3, opacity=0.5] (2.6104, -0.1120, 2.5371) -- (2.6567, -0.1120, 2.5311) -- (2.6554, -0.1135, 2.5811) -- (2.6092, -0.1135, 2.5871) -- cycle;
\fill[blue!28.6, opacity=0.5] (2.6092, -0.1135, 2.5871) -- (2.6554, -0.1135, 2.5811) -- (2.6543, -0.1148, 2.6311) -- (2.6081, -0.1148, 2.6371) -- cycle;
\fill[blue!29.8, opacity=0.5] (2.6081, -0.1148, 2.6371) -- (2.6543, -0.1148, 2.6311) -- (2.6533, -0.1160, 2.6811) -- (2.6072, -0.1160, 2.6871) -- cycle;
\fill[blue!31.0, opacity=0.5] (2.6072, -0.1160, 2.6871) -- (2.6533, -0.1160, 2.6811) -- (2.6524, -0.1171, 2.7311) -- (2.6063, -0.1171, 2.7371) -- cycle;
\fill[blue!32.2, opacity=0.5] (2.6063, -0.1171, 2.7371) -- (2.6524, -0.1171, 2.7311) -- (2.6517, -0.1180, 2.7811) -- (2.6056, -0.1180, 2.7871) -- cycle;
\fill[blue!33.4, opacity=0.5] (2.6056, -0.1180, 2.7871) -- (2.6517, -0.1180, 2.7811) -- (2.6511, -0.1187, 2.8311) -- (2.6050, -0.1187, 2.8371) -- cycle;
\fill[blue!34.5, opacity=0.5] (2.6050, -0.1187, 2.8371) -- (2.6511, -0.1187, 2.8311) -- (2.6506, -0.1193, 2.8811) -- (2.6046, -0.1193, 2.8871) -- cycle;
\fill[blue!35.6, opacity=0.5] (2.6046, -0.1193, 2.8871) -- (2.6506, -0.1193, 2.8811) -- (2.6503, -0.1197, 2.9311) -- (2.6043, -0.1197, 2.9371) -- cycle;
\fill[blue!36.7, opacity=0.5] (2.6043, -0.1197, 2.9371) -- (2.6503, -0.1197, 2.9311) -- (2.6501, -0.1199, 2.9811) -- (2.6041, -0.1199, 2.9871) -- cycle;
\fill[blue!37.7, opacity=0.5] (2.6041, -0.1199, 2.9871) -- (2.6501, -0.1199, 2.9811) -- (2.6500, -0.1200, 3.0311) -- (2.6040, -0.1200, 3.0371) -- cycle;
\fill[blue!15.0, opacity=0.5] (2.7500, -0.0000, 0.0311) -- (2.8000, -0.0000, 0.0249) -- (2.7999, -0.0001, 0.0749) -- (2.7499, -0.0001, 0.0811) -- cycle;
\fill[blue!15.0, opacity=0.5] (2.7499, -0.0001, 0.0811) -- (2.7999, -0.0001, 0.0749) -- (2.7997, -0.0003, 0.1249) -- (2.7497, -0.0003, 0.1311) -- cycle;
\fill[blue!15.0, opacity=0.5] (2.7497, -0.0003, 0.1311) -- (2.7997, -0.0003, 0.1249) -- (2.7994, -0.0007, 0.1749) -- (2.7494, -0.0007, 0.1811) -- cycle;
\fill[blue!15.0, opacity=0.5] (2.7494, -0.0007, 0.1811) -- (2.7994, -0.0007, 0.1749) -- (2.7989, -0.0013, 0.2249) -- (2.7489, -0.0013, 0.2311) -- cycle;
\fill[blue!15.0, opacity=0.5] (2.7489, -0.0013, 0.2311) -- (2.7989, -0.0013, 0.2249) -- (2.7982, -0.0020, 0.2749) -- (2.7483, -0.0020, 0.2811) -- cycle;
\fill[blue!15.0, opacity=0.5] (2.7483, -0.0020, 0.2811) -- (2.7982, -0.0020, 0.2749) -- (2.7975, -0.0029, 0.3249) -- (2.7476, -0.0029, 0.3311) -- cycle;
\fill[blue!15.0, opacity=0.5] (2.7476, -0.0029, 0.3311) -- (2.7975, -0.0029, 0.3249) -- (2.7965, -0.0040, 0.3749) -- (2.7467, -0.0040, 0.3811) -- cycle;
\fill[blue!15.0, opacity=0.5] (2.7467, -0.0040, 0.3811) -- (2.7965, -0.0040, 0.3749) -- (2.7955, -0.0052, 0.4249) -- (2.7457, -0.0052, 0.4311) -- cycle;
\fill[blue!15.0, opacity=0.5] (2.7457, -0.0052, 0.4311) -- (2.7955, -0.0052, 0.4249) -- (2.7943, -0.0065, 0.4749) -- (2.7446, -0.0065, 0.4811) -- cycle;
\fill[blue!15.0, opacity=0.5] (2.7446, -0.0065, 0.4811) -- (2.7943, -0.0065, 0.4749) -- (2.7930, -0.0080, 0.5249) -- (2.7433, -0.0080, 0.5311) -- cycle;
\fill[blue!15.0, opacity=0.5] (2.7433, -0.0080, 0.5311) -- (2.7930, -0.0080, 0.5249) -- (2.7916, -0.0097, 0.5749) -- (2.7419, -0.0097, 0.5811) -- cycle;
\fill[blue!15.0, opacity=0.5] (2.7419, -0.0097, 0.5811) -- (2.7916, -0.0097, 0.5749) -- (2.7901, -0.0115, 0.6249) -- (2.7405, -0.0115, 0.6311) -- cycle;
\fill[blue!15.0, opacity=0.5] (2.7405, -0.0115, 0.6311) -- (2.7901, -0.0115, 0.6249) -- (2.7884, -0.0134, 0.6749) -- (2.7389, -0.0134, 0.6811) -- cycle;
\fill[blue!15.0, opacity=0.5] (2.7389, -0.0134, 0.6811) -- (2.7884, -0.0134, 0.6749) -- (2.7866, -0.0154, 0.7249) -- (2.7372, -0.0154, 0.7311) -- cycle;
\fill[blue!15.0, opacity=0.5] (2.7372, -0.0154, 0.7311) -- (2.7866, -0.0154, 0.7249) -- (2.7848, -0.0176, 0.7749) -- (2.7354, -0.0176, 0.7811) -- cycle;
\fill[blue!15.0, opacity=0.5] (2.7354, -0.0176, 0.7811) -- (2.7848, -0.0176, 0.7749) -- (2.7828, -0.0199, 0.8249) -- (2.7335, -0.0199, 0.8311) -- cycle;
\fill[blue!15.0, opacity=0.5] (2.7335, -0.0199, 0.8311) -- (2.7828, -0.0199, 0.8249) -- (2.7807, -0.0222, 0.8749) -- (2.7315, -0.0222, 0.8811) -- cycle;
\fill[blue!15.0, opacity=0.5] (2.7315, -0.0222, 0.8811) -- (2.7807, -0.0222, 0.8749) -- (2.7786, -0.0247, 0.9249) -- (2.7294, -0.0247, 0.9311) -- cycle;
\fill[blue!15.0, opacity=0.5] (2.7294, -0.0247, 0.9311) -- (2.7786, -0.0247, 0.9249) -- (2.7763, -0.0273, 0.9749) -- (2.7272, -0.0273, 0.9811) -- cycle;
\fill[blue!15.0, opacity=0.5] (2.7272, -0.0273, 0.9811) -- (2.7763, -0.0273, 0.9749) -- (2.7740, -0.0300, 1.0249) -- (2.7250, -0.0300, 1.0311) -- cycle;
\fill[blue!15.0, opacity=0.5] (2.7250, -0.0300, 1.0311) -- (2.7740, -0.0300, 1.0249) -- (2.7716, -0.0328, 1.0749) -- (2.7227, -0.0328, 1.0811) -- cycle;
\fill[blue!15.0, opacity=0.5] (2.7227, -0.0328, 1.0811) -- (2.7716, -0.0328, 1.0749) -- (2.7692, -0.0356, 1.1249) -- (2.7203, -0.0356, 1.1311) -- cycle;
\fill[blue!15.0, opacity=0.5] (2.7203, -0.0356, 1.1311) -- (2.7692, -0.0356, 1.1249) -- (2.7666, -0.0385, 1.1749) -- (2.7179, -0.0385, 1.1811) -- cycle;
\fill[blue!15.0, opacity=0.5] (2.7179, -0.0385, 1.1811) -- (2.7666, -0.0385, 1.1749) -- (2.7641, -0.0415, 1.2249) -- (2.7155, -0.0415, 1.2311) -- cycle;
\fill[blue!15.0, opacity=0.5] (2.7155, -0.0415, 1.2311) -- (2.7641, -0.0415, 1.2249) -- (2.7615, -0.0445, 1.2749) -- (2.7129, -0.0445, 1.2811) -- cycle;
\fill[blue!15.0, opacity=0.5] (2.7129, -0.0445, 1.2811) -- (2.7615, -0.0445, 1.2749) -- (2.7588, -0.0475, 1.3249) -- (2.7104, -0.0475, 1.3311) -- cycle;
\fill[blue!15.0, opacity=0.5] (2.7104, -0.0475, 1.3311) -- (2.7588, -0.0475, 1.3249) -- (2.7561, -0.0506, 1.3749) -- (2.7078, -0.0506, 1.3811) -- cycle;
\fill[blue!15.0, opacity=0.5] (2.7078, -0.0506, 1.3811) -- (2.7561, -0.0506, 1.3749) -- (2.7534, -0.0537, 1.4249) -- (2.7052, -0.0537, 1.4311) -- cycle;
\fill[blue!15.0, opacity=0.5] (2.7052, -0.0537, 1.4311) -- (2.7534, -0.0537, 1.4249) -- (2.7507, -0.0569, 1.4749) -- (2.7026, -0.0569, 1.4811) -- cycle;
\fill[blue!15.0, opacity=0.5] (2.7026, -0.0569, 1.4811) -- (2.7507, -0.0569, 1.4749) -- (2.7480, -0.0600, 1.5249) -- (2.7000, -0.0600, 1.5311) -- cycle;
\fill[blue!15.0, opacity=0.5] (2.7000, -0.0600, 1.5311) -- (2.7480, -0.0600, 1.5249) -- (2.7453, -0.0631, 1.5749) -- (2.6974, -0.0631, 1.5811) -- cycle;
\fill[blue!15.0, opacity=0.5] (2.6974, -0.0631, 1.5811) -- (2.7453, -0.0631, 1.5749) -- (2.7426, -0.0663, 1.6249) -- (2.6948, -0.0663, 1.6311) -- cycle;
\fill[blue!15.0, opacity=0.5] (2.6948, -0.0663, 1.6311) -- (2.7426, -0.0663, 1.6249) -- (2.7399, -0.0694, 1.6749) -- (2.6922, -0.0694, 1.6811) -- cycle;
\fill[blue!15.0, opacity=0.5] (2.6922, -0.0694, 1.6811) -- (2.7399, -0.0694, 1.6749) -- (2.7372, -0.0725, 1.7249) -- (2.6896, -0.0725, 1.7311) -- cycle;
\fill[blue!15.1, opacity=0.5] (2.6896, -0.0725, 1.7311) -- (2.7372, -0.0725, 1.7249) -- (2.7345, -0.0755, 1.7749) -- (2.6871, -0.0755, 1.7811) -- cycle;
\fill[blue!15.1, opacity=0.5] (2.6871, -0.0755, 1.7811) -- (2.7345, -0.0755, 1.7749) -- (2.7319, -0.0785, 1.8249) -- (2.6845, -0.0785, 1.8311) -- cycle;
\fill[blue!15.1, opacity=0.5] (2.6845, -0.0785, 1.8311) -- (2.7319, -0.0785, 1.8249) -- (2.7294, -0.0815, 1.8749) -- (2.6821, -0.0815, 1.8811) -- cycle;
\fill[blue!15.2, opacity=0.5] (2.6821, -0.0815, 1.8811) -- (2.7294, -0.0815, 1.8749) -- (2.7268, -0.0844, 1.9249) -- (2.6797, -0.0844, 1.9311) -- cycle;
\fill[blue!15.2, opacity=0.5] (2.6797, -0.0844, 1.9311) -- (2.7268, -0.0844, 1.9249) -- (2.7244, -0.0872, 1.9749) -- (2.6773, -0.0872, 1.9811) -- cycle;
\fill[blue!15.3, opacity=0.5] (2.6773, -0.0872, 1.9811) -- (2.7244, -0.0872, 1.9749) -- (2.7220, -0.0900, 2.0249) -- (2.6750, -0.0900, 2.0311) -- cycle;
\fill[blue!15.4, opacity=0.5] (2.6750, -0.0900, 2.0311) -- (2.7220, -0.0900, 2.0249) -- (2.7197, -0.0927, 2.0749) -- (2.6728, -0.0927, 2.0811) -- cycle;
\fill[blue!15.6, opacity=0.5] (2.6728, -0.0927, 2.0811) -- (2.7197, -0.0927, 2.0749) -- (2.7174, -0.0953, 2.1249) -- (2.6706, -0.0953, 2.1311) -- cycle;
\fill[blue!15.7, opacity=0.5] (2.6706, -0.0953, 2.1311) -- (2.7174, -0.0953, 2.1249) -- (2.7153, -0.0978, 2.1749) -- (2.6685, -0.0978, 2.1811) -- cycle;
\fill[blue!15.9, opacity=0.5] (2.6685, -0.0978, 2.1811) -- (2.7153, -0.0978, 2.1749) -- (2.7132, -0.1001, 2.2249) -- (2.6665, -0.1001, 2.2311) -- cycle;
\fill[blue!16.2, opacity=0.5] (2.6665, -0.1001, 2.2311) -- (2.7132, -0.1001, 2.2249) -- (2.7112, -0.1024, 2.2749) -- (2.6646, -0.1024, 2.2811) -- cycle;
\fill[blue!16.5, opacity=0.5] (2.6646, -0.1024, 2.2811) -- (2.7112, -0.1024, 2.2749) -- (2.7094, -0.1046, 2.3249) -- (2.6628, -0.1046, 2.3311) -- cycle;
\fill[blue!16.8, opacity=0.5] (2.6628, -0.1046, 2.3311) -- (2.7094, -0.1046, 2.3249) -- (2.7076, -0.1066, 2.3749) -- (2.6611, -0.1066, 2.3811) -- cycle;
\fill[blue!17.2, opacity=0.5] (2.6611, -0.1066, 2.3811) -- (2.7076, -0.1066, 2.3749) -- (2.7059, -0.1085, 2.4249) -- (2.6595, -0.1085, 2.4311) -- cycle;
\fill[blue!17.6, opacity=0.5] (2.6595, -0.1085, 2.4311) -- (2.7059, -0.1085, 2.4249) -- (2.7044, -0.1103, 2.4749) -- (2.6581, -0.1103, 2.4811) -- cycle;
\fill[blue!18.1, opacity=0.5] (2.6581, -0.1103, 2.4811) -- (2.7044, -0.1103, 2.4749) -- (2.7030, -0.1120, 2.5249) -- (2.6567, -0.1120, 2.5311) -- cycle;
\fill[blue!18.7, opacity=0.5] (2.6567, -0.1120, 2.5311) -- (2.7030, -0.1120, 2.5249) -- (2.7017, -0.1135, 2.5749) -- (2.6554, -0.1135, 2.5811) -- cycle;
\fill[blue!19.3, opacity=0.5] (2.6554, -0.1135, 2.5811) -- (2.7017, -0.1135, 2.5749) -- (2.7005, -0.1148, 2.6249) -- (2.6543, -0.1148, 2.6311) -- cycle;
\fill[blue!19.9, opacity=0.5] (2.6543, -0.1148, 2.6311) -- (2.7005, -0.1148, 2.6249) -- (2.6995, -0.1160, 2.6749) -- (2.6533, -0.1160, 2.6811) -- cycle;
\fill[blue!20.6, opacity=0.5] (2.6533, -0.1160, 2.6811) -- (2.6995, -0.1160, 2.6749) -- (2.6985, -0.1171, 2.7249) -- (2.6524, -0.1171, 2.7311) -- cycle;
\fill[blue!21.3, opacity=0.5] (2.6524, -0.1171, 2.7311) -- (2.6985, -0.1171, 2.7249) -- (2.6978, -0.1180, 2.7749) -- (2.6517, -0.1180, 2.7811) -- cycle;
\fill[blue!22.0, opacity=0.5] (2.6517, -0.1180, 2.7811) -- (2.6978, -0.1180, 2.7749) -- (2.6971, -0.1187, 2.8249) -- (2.6511, -0.1187, 2.8311) -- cycle;
\fill[blue!22.8, opacity=0.5] (2.6511, -0.1187, 2.8311) -- (2.6971, -0.1187, 2.8249) -- (2.6966, -0.1193, 2.8749) -- (2.6506, -0.1193, 2.8811) -- cycle;
\fill[blue!23.6, opacity=0.5] (2.6506, -0.1193, 2.8811) -- (2.6966, -0.1193, 2.8749) -- (2.6963, -0.1197, 2.9249) -- (2.6503, -0.1197, 2.9311) -- cycle;
\fill[blue!24.4, opacity=0.5] (2.6503, -0.1197, 2.9311) -- (2.6963, -0.1197, 2.9249) -- (2.6961, -0.1199, 2.9749) -- (2.6501, -0.1199, 2.9811) -- cycle;
\fill[blue!25.2, opacity=0.5] (2.6501, -0.1199, 2.9811) -- (2.6961, -0.1199, 2.9749) -- (2.6960, -0.1200, 3.0249) -- (2.6500, -0.1200, 3.0311) -- cycle;
\fill[blue!15.0, opacity=0.5] (2.8000, -0.0000, 0.0249) -- (2.8500, -0.0000, 0.0188) -- (2.8499, -0.0001, 0.0688) -- (2.7999, -0.0001, 0.0749) -- cycle;
\fill[blue!15.0, opacity=0.5] (2.7999, -0.0001, 0.0749) -- (2.8499, -0.0001, 0.0688) -- (2.8497, -0.0003, 0.1188) -- (2.7997, -0.0003, 0.1249) -- cycle;
\fill[blue!15.0, opacity=0.5] (2.7997, -0.0003, 0.1249) -- (2.8497, -0.0003, 0.1188) -- (2.8493, -0.0007, 0.1688) -- (2.7994, -0.0007, 0.1749) -- cycle;
\fill[blue!15.0, opacity=0.5] (2.7994, -0.0007, 0.1749) -- (2.8493, -0.0007, 0.1688) -- (2.8488, -0.0013, 0.2188) -- (2.7989, -0.0013, 0.2249) -- cycle;
\fill[blue!15.0, opacity=0.5] (2.7989, -0.0013, 0.2249) -- (2.8488, -0.0013, 0.2188) -- (2.8482, -0.0020, 0.2688) -- (2.7982, -0.0020, 0.2749) -- cycle;
\fill[blue!15.0, opacity=0.5] (2.7982, -0.0020, 0.2749) -- (2.8482, -0.0020, 0.2688) -- (2.8474, -0.0029, 0.3188) -- (2.7975, -0.0029, 0.3249) -- cycle;
\fill[blue!15.0, opacity=0.5] (2.7975, -0.0029, 0.3249) -- (2.8474, -0.0029, 0.3188) -- (2.8464, -0.0040, 0.3688) -- (2.7965, -0.0040, 0.3749) -- cycle;
\fill[blue!15.0, opacity=0.5] (2.7965, -0.0040, 0.3749) -- (2.8464, -0.0040, 0.3688) -- (2.8453, -0.0052, 0.4188) -- (2.7955, -0.0052, 0.4249) -- cycle;
\fill[blue!15.0, opacity=0.5] (2.7955, -0.0052, 0.4249) -- (2.8453, -0.0052, 0.4188) -- (2.8441, -0.0065, 0.4688) -- (2.7943, -0.0065, 0.4749) -- cycle;
\fill[blue!15.0, opacity=0.5] (2.7943, -0.0065, 0.4749) -- (2.8441, -0.0065, 0.4688) -- (2.8428, -0.0080, 0.5188) -- (2.7930, -0.0080, 0.5249) -- cycle;
\fill[blue!15.0, opacity=0.5] (2.7930, -0.0080, 0.5249) -- (2.8428, -0.0080, 0.5188) -- (2.8413, -0.0097, 0.5688) -- (2.7916, -0.0097, 0.5749) -- cycle;
\fill[blue!15.0, opacity=0.5] (2.7916, -0.0097, 0.5749) -- (2.8413, -0.0097, 0.5688) -- (2.8397, -0.0115, 0.6188) -- (2.7901, -0.0115, 0.6249) -- cycle;
\fill[blue!15.0, opacity=0.5] (2.7901, -0.0115, 0.6249) -- (2.8397, -0.0115, 0.6188) -- (2.8380, -0.0134, 0.6688) -- (2.7884, -0.0134, 0.6749) -- cycle;
\fill[blue!15.0, opacity=0.5] (2.7884, -0.0134, 0.6749) -- (2.8380, -0.0134, 0.6688) -- (2.8361, -0.0154, 0.7188) -- (2.7866, -0.0154, 0.7249) -- cycle;
\fill[blue!15.0, opacity=0.5] (2.7866, -0.0154, 0.7249) -- (2.8361, -0.0154, 0.7188) -- (2.8342, -0.0176, 0.7688) -- (2.7848, -0.0176, 0.7749) -- cycle;
\fill[blue!15.0, opacity=0.5] (2.7848, -0.0176, 0.7749) -- (2.8342, -0.0176, 0.7688) -- (2.8321, -0.0199, 0.8188) -- (2.7828, -0.0199, 0.8249) -- cycle;
\fill[blue!15.0, opacity=0.5] (2.7828, -0.0199, 0.8249) -- (2.8321, -0.0199, 0.8188) -- (2.8300, -0.0222, 0.8688) -- (2.7807, -0.0222, 0.8749) -- cycle;
\fill[blue!15.0, opacity=0.5] (2.7807, -0.0222, 0.8749) -- (2.8300, -0.0222, 0.8688) -- (2.8277, -0.0247, 0.9188) -- (2.7786, -0.0247, 0.9249) -- cycle;
\fill[blue!15.0, opacity=0.5] (2.7786, -0.0247, 0.9249) -- (2.8277, -0.0247, 0.9188) -- (2.8254, -0.0273, 0.9688) -- (2.7763, -0.0273, 0.9749) -- cycle;
\fill[blue!15.0, opacity=0.5] (2.7763, -0.0273, 0.9749) -- (2.8254, -0.0273, 0.9688) -- (2.8230, -0.0300, 1.0188) -- (2.7740, -0.0300, 1.0249) -- cycle;
\fill[blue!15.0, opacity=0.5] (2.7740, -0.0300, 1.0249) -- (2.8230, -0.0300, 1.0188) -- (2.8205, -0.0328, 1.0688) -- (2.7716, -0.0328, 1.0749) -- cycle;
\fill[blue!15.0, opacity=0.5] (2.7716, -0.0328, 1.0749) -- (2.8205, -0.0328, 1.0688) -- (2.8180, -0.0356, 1.1188) -- (2.7692, -0.0356, 1.1249) -- cycle;
\fill[blue!15.0, opacity=0.5] (2.7692, -0.0356, 1.1249) -- (2.8180, -0.0356, 1.1188) -- (2.8154, -0.0385, 1.1688) -- (2.7666, -0.0385, 1.1749) -- cycle;
\fill[blue!15.0, opacity=0.5] (2.7666, -0.0385, 1.1749) -- (2.8154, -0.0385, 1.1688) -- (2.8127, -0.0415, 1.2188) -- (2.7641, -0.0415, 1.2249) -- cycle;
\fill[blue!15.0, opacity=0.5] (2.7641, -0.0415, 1.2249) -- (2.8127, -0.0415, 1.2188) -- (2.8100, -0.0445, 1.2688) -- (2.7615, -0.0445, 1.2749) -- cycle;
\fill[blue!15.0, opacity=0.5] (2.7615, -0.0445, 1.2749) -- (2.8100, -0.0445, 1.2688) -- (2.8072, -0.0475, 1.3188) -- (2.7588, -0.0475, 1.3249) -- cycle;
\fill[blue!15.0, opacity=0.5] (2.7588, -0.0475, 1.3249) -- (2.8072, -0.0475, 1.3188) -- (2.8044, -0.0506, 1.3688) -- (2.7561, -0.0506, 1.3749) -- cycle;
\fill[blue!15.0, opacity=0.5] (2.7561, -0.0506, 1.3749) -- (2.8044, -0.0506, 1.3688) -- (2.8016, -0.0537, 1.4188) -- (2.7534, -0.0537, 1.4249) -- cycle;
\fill[blue!15.0, opacity=0.5] (2.7534, -0.0537, 1.4249) -- (2.8016, -0.0537, 1.4188) -- (2.7988, -0.0569, 1.4688) -- (2.7507, -0.0569, 1.4749) -- cycle;
\fill[blue!15.0, opacity=0.5] (2.7507, -0.0569, 1.4749) -- (2.7988, -0.0569, 1.4688) -- (2.7960, -0.0600, 1.5188) -- (2.7480, -0.0600, 1.5249) -- cycle;
\fill[blue!15.0, opacity=0.5] (2.7480, -0.0600, 1.5249) -- (2.7960, -0.0600, 1.5188) -- (2.7932, -0.0631, 1.5688) -- (2.7453, -0.0631, 1.5749) -- cycle;
\fill[blue!15.0, opacity=0.5] (2.7453, -0.0631, 1.5749) -- (2.7932, -0.0631, 1.5688) -- (2.7904, -0.0663, 1.6188) -- (2.7426, -0.0663, 1.6249) -- cycle;
\fill[blue!15.0, opacity=0.5] (2.7426, -0.0663, 1.6249) -- (2.7904, -0.0663, 1.6188) -- (2.7876, -0.0694, 1.6688) -- (2.7399, -0.0694, 1.6749) -- cycle;
\fill[blue!15.0, opacity=0.5] (2.7399, -0.0694, 1.6749) -- (2.7876, -0.0694, 1.6688) -- (2.7848, -0.0725, 1.7188) -- (2.7372, -0.0725, 1.7249) -- cycle;
\fill[blue!15.0, opacity=0.5] (2.7372, -0.0725, 1.7249) -- (2.7848, -0.0725, 1.7188) -- (2.7820, -0.0755, 1.7688) -- (2.7345, -0.0755, 1.7749) -- cycle;
\fill[blue!15.0, opacity=0.5] (2.7345, -0.0755, 1.7749) -- (2.7820, -0.0755, 1.7688) -- (2.7793, -0.0785, 1.8188) -- (2.7319, -0.0785, 1.8249) -- cycle;
\fill[blue!15.0, opacity=0.5] (2.7319, -0.0785, 1.8249) -- (2.7793, -0.0785, 1.8188) -- (2.7766, -0.0815, 1.8688) -- (2.7294, -0.0815, 1.8749) -- cycle;
\fill[blue!15.0, opacity=0.5] (2.7294, -0.0815, 1.8749) -- (2.7766, -0.0815, 1.8688) -- (2.7740, -0.0844, 1.9188) -- (2.7268, -0.0844, 1.9249) -- cycle;
\fill[blue!15.0, opacity=0.5] (2.7268, -0.0844, 1.9249) -- (2.7740, -0.0844, 1.9188) -- (2.7715, -0.0872, 1.9688) -- (2.7244, -0.0872, 1.9749) -- cycle;
\fill[blue!15.0, opacity=0.5] (2.7244, -0.0872, 1.9749) -- (2.7715, -0.0872, 1.9688) -- (2.7690, -0.0900, 2.0188) -- (2.7220, -0.0900, 2.0249) -- cycle;
\fill[blue!15.0, opacity=0.5] (2.7220, -0.0900, 2.0249) -- (2.7690, -0.0900, 2.0188) -- (2.7666, -0.0927, 2.0688) -- (2.7197, -0.0927, 2.0749) -- cycle;
\fill[blue!15.0, opacity=0.5] (2.7197, -0.0927, 2.0749) -- (2.7666, -0.0927, 2.0688) -- (2.7643, -0.0953, 2.1188) -- (2.7174, -0.0953, 2.1249) -- cycle;
\fill[blue!15.0, opacity=0.5] (2.7174, -0.0953, 2.1249) -- (2.7643, -0.0953, 2.1188) -- (2.7620, -0.0978, 2.1688) -- (2.7153, -0.0978, 2.1749) -- cycle;
\fill[blue!15.1, opacity=0.5] (2.7153, -0.0978, 2.1749) -- (2.7620, -0.0978, 2.1688) -- (2.7599, -0.1001, 2.2188) -- (2.7132, -0.1001, 2.2249) -- cycle;
\fill[blue!15.1, opacity=0.5] (2.7132, -0.1001, 2.2249) -- (2.7599, -0.1001, 2.2188) -- (2.7578, -0.1024, 2.2688) -- (2.7112, -0.1024, 2.2749) -- cycle;
\fill[blue!15.1, opacity=0.5] (2.7112, -0.1024, 2.2749) -- (2.7578, -0.1024, 2.2688) -- (2.7559, -0.1046, 2.3188) -- (2.7094, -0.1046, 2.3249) -- cycle;
\fill[blue!15.2, opacity=0.5] (2.7094, -0.1046, 2.3249) -- (2.7559, -0.1046, 2.3188) -- (2.7540, -0.1066, 2.3688) -- (2.7076, -0.1066, 2.3749) -- cycle;
\fill[blue!15.2, opacity=0.5] (2.7076, -0.1066, 2.3749) -- (2.7540, -0.1066, 2.3688) -- (2.7523, -0.1085, 2.4188) -- (2.7059, -0.1085, 2.4249) -- cycle;
\fill[blue!15.3, opacity=0.5] (2.7059, -0.1085, 2.4249) -- (2.7523, -0.1085, 2.4188) -- (2.7507, -0.1103, 2.4688) -- (2.7044, -0.1103, 2.4749) -- cycle;
\fill[blue!15.4, opacity=0.5] (2.7044, -0.1103, 2.4749) -- (2.7507, -0.1103, 2.4688) -- (2.7492, -0.1120, 2.5188) -- (2.7030, -0.1120, 2.5249) -- cycle;
\fill[blue!15.5, opacity=0.5] (2.7030, -0.1120, 2.5249) -- (2.7492, -0.1120, 2.5188) -- (2.7479, -0.1135, 2.5688) -- (2.7017, -0.1135, 2.5749) -- cycle;
\fill[blue!15.6, opacity=0.5] (2.7017, -0.1135, 2.5749) -- (2.7479, -0.1135, 2.5688) -- (2.7467, -0.1148, 2.6188) -- (2.7005, -0.1148, 2.6249) -- cycle;
\fill[blue!15.7, opacity=0.5] (2.7005, -0.1148, 2.6249) -- (2.7467, -0.1148, 2.6188) -- (2.7456, -0.1160, 2.6688) -- (2.6995, -0.1160, 2.6749) -- cycle;
\fill[blue!15.9, opacity=0.5] (2.6995, -0.1160, 2.6749) -- (2.7456, -0.1160, 2.6688) -- (2.7446, -0.1171, 2.7188) -- (2.6985, -0.1171, 2.7249) -- cycle;
\fill[blue!16.1, opacity=0.5] (2.6985, -0.1171, 2.7249) -- (2.7446, -0.1171, 2.7188) -- (2.7438, -0.1180, 2.7688) -- (2.6978, -0.1180, 2.7749) -- cycle;
\fill[blue!16.3, opacity=0.5] (2.6978, -0.1180, 2.7749) -- (2.7438, -0.1180, 2.7688) -- (2.7432, -0.1187, 2.8188) -- (2.6971, -0.1187, 2.8249) -- cycle;
\fill[blue!16.5, opacity=0.5] (2.6971, -0.1187, 2.8249) -- (2.7432, -0.1187, 2.8188) -- (2.7427, -0.1193, 2.8688) -- (2.6966, -0.1193, 2.8749) -- cycle;
\fill[blue!16.8, opacity=0.5] (2.6966, -0.1193, 2.8749) -- (2.7427, -0.1193, 2.8688) -- (2.7423, -0.1197, 2.9188) -- (2.6963, -0.1197, 2.9249) -- cycle;
\fill[blue!17.1, opacity=0.5] (2.6963, -0.1197, 2.9249) -- (2.7423, -0.1197, 2.9188) -- (2.7421, -0.1199, 2.9688) -- (2.6961, -0.1199, 2.9749) -- cycle;
\fill[blue!17.4, opacity=0.5] (2.6961, -0.1199, 2.9749) -- (2.7421, -0.1199, 2.9688) -- (2.7420, -0.1200, 3.0188) -- (2.6960, -0.1200, 3.0249) -- cycle;
\fill[blue!15.0, opacity=0.5] (2.8500, -0.0000, 0.0188) -- (2.9000, -0.0000, 0.0125) -- (2.8999, -0.0001, 0.0625) -- (2.8499, -0.0001, 0.0688) -- cycle;
\fill[blue!15.0, opacity=0.5] (2.8499, -0.0001, 0.0688) -- (2.8999, -0.0001, 0.0625) -- (2.8997, -0.0003, 0.1125) -- (2.8497, -0.0003, 0.1188) -- cycle;
\fill[blue!15.0, opacity=0.5] (2.8497, -0.0003, 0.1188) -- (2.8997, -0.0003, 0.1125) -- (2.8993, -0.0007, 0.1625) -- (2.8493, -0.0007, 0.1688) -- cycle;
\fill[blue!15.0, opacity=0.5] (2.8493, -0.0007, 0.1688) -- (2.8993, -0.0007, 0.1625) -- (2.8988, -0.0013, 0.2125) -- (2.8488, -0.0013, 0.2188) -- cycle;
\fill[blue!15.0, opacity=0.5] (2.8488, -0.0013, 0.2188) -- (2.8988, -0.0013, 0.2125) -- (2.8981, -0.0020, 0.2625) -- (2.8482, -0.0020, 0.2688) -- cycle;
\fill[blue!15.0, opacity=0.5] (2.8482, -0.0020, 0.2688) -- (2.8981, -0.0020, 0.2625) -- (2.8973, -0.0029, 0.3125) -- (2.8474, -0.0029, 0.3188) -- cycle;
\fill[blue!15.0, opacity=0.5] (2.8474, -0.0029, 0.3188) -- (2.8973, -0.0029, 0.3125) -- (2.8963, -0.0040, 0.3625) -- (2.8464, -0.0040, 0.3688) -- cycle;
\fill[blue!15.0, opacity=0.5] (2.8464, -0.0040, 0.3688) -- (2.8963, -0.0040, 0.3625) -- (2.8952, -0.0052, 0.4125) -- (2.8453, -0.0052, 0.4188) -- cycle;
\fill[blue!15.0, opacity=0.5] (2.8453, -0.0052, 0.4188) -- (2.8952, -0.0052, 0.4125) -- (2.8939, -0.0065, 0.4625) -- (2.8441, -0.0065, 0.4688) -- cycle;
\fill[blue!15.0, opacity=0.5] (2.8441, -0.0065, 0.4688) -- (2.8939, -0.0065, 0.4625) -- (2.8925, -0.0080, 0.5125) -- (2.8428, -0.0080, 0.5188) -- cycle;
\fill[blue!15.0, opacity=0.5] (2.8428, -0.0080, 0.5188) -- (2.8925, -0.0080, 0.5125) -- (2.8910, -0.0097, 0.5625) -- (2.8413, -0.0097, 0.5688) -- cycle;
\fill[blue!15.0, opacity=0.5] (2.8413, -0.0097, 0.5688) -- (2.8910, -0.0097, 0.5625) -- (2.8893, -0.0115, 0.6125) -- (2.8397, -0.0115, 0.6188) -- cycle;
\fill[blue!15.0, opacity=0.5] (2.8397, -0.0115, 0.6188) -- (2.8893, -0.0115, 0.6125) -- (2.8875, -0.0134, 0.6625) -- (2.8380, -0.0134, 0.6688) -- cycle;
\fill[blue!15.0, opacity=0.5] (2.8380, -0.0134, 0.6688) -- (2.8875, -0.0134, 0.6625) -- (2.8856, -0.0154, 0.7125) -- (2.8361, -0.0154, 0.7188) -- cycle;
\fill[blue!15.0, opacity=0.5] (2.8361, -0.0154, 0.7188) -- (2.8856, -0.0154, 0.7125) -- (2.8836, -0.0176, 0.7625) -- (2.8342, -0.0176, 0.7688) -- cycle;
\fill[blue!15.0, opacity=0.5] (2.8342, -0.0176, 0.7688) -- (2.8836, -0.0176, 0.7625) -- (2.8815, -0.0199, 0.8125) -- (2.8321, -0.0199, 0.8188) -- cycle;
\fill[blue!15.0, opacity=0.5] (2.8321, -0.0199, 0.8188) -- (2.8815, -0.0199, 0.8125) -- (2.8792, -0.0222, 0.8625) -- (2.8300, -0.0222, 0.8688) -- cycle;
\fill[blue!15.0, opacity=0.5] (2.8300, -0.0222, 0.8688) -- (2.8792, -0.0222, 0.8625) -- (2.8769, -0.0247, 0.9125) -- (2.8277, -0.0247, 0.9188) -- cycle;
\fill[blue!15.0, opacity=0.5] (2.8277, -0.0247, 0.9188) -- (2.8769, -0.0247, 0.9125) -- (2.8745, -0.0273, 0.9625) -- (2.8254, -0.0273, 0.9688) -- cycle;
\fill[blue!15.0, opacity=0.5] (2.8254, -0.0273, 0.9688) -- (2.8745, -0.0273, 0.9625) -- (2.8720, -0.0300, 1.0125) -- (2.8230, -0.0300, 1.0188) -- cycle;
\fill[blue!15.0, opacity=0.5] (2.8230, -0.0300, 1.0188) -- (2.8720, -0.0300, 1.0125) -- (2.8694, -0.0328, 1.0625) -- (2.8205, -0.0328, 1.0688) -- cycle;
\fill[blue!15.0, opacity=0.5] (2.8205, -0.0328, 1.0688) -- (2.8694, -0.0328, 1.0625) -- (2.8668, -0.0356, 1.1125) -- (2.8180, -0.0356, 1.1188) -- cycle;
\fill[blue!15.0, opacity=0.5] (2.8180, -0.0356, 1.1188) -- (2.8668, -0.0356, 1.1125) -- (2.8641, -0.0385, 1.1625) -- (2.8154, -0.0385, 1.1688) -- cycle;
\fill[blue!15.0, opacity=0.5] (2.8154, -0.0385, 1.1688) -- (2.8641, -0.0385, 1.1625) -- (2.8613, -0.0415, 1.2125) -- (2.8127, -0.0415, 1.2188) -- cycle;
\fill[blue!15.0, opacity=0.5] (2.8127, -0.0415, 1.2188) -- (2.8613, -0.0415, 1.2125) -- (2.8585, -0.0445, 1.2625) -- (2.8100, -0.0445, 1.2688) -- cycle;
\fill[blue!15.0, opacity=0.5] (2.8100, -0.0445, 1.2688) -- (2.8585, -0.0445, 1.2625) -- (2.8556, -0.0475, 1.3125) -- (2.8072, -0.0475, 1.3188) -- cycle;
\fill[blue!15.0, opacity=0.5] (2.8072, -0.0475, 1.3188) -- (2.8556, -0.0475, 1.3125) -- (2.8528, -0.0506, 1.3625) -- (2.8044, -0.0506, 1.3688) -- cycle;
\fill[blue!15.0, opacity=0.5] (2.8044, -0.0506, 1.3688) -- (2.8528, -0.0506, 1.3625) -- (2.8499, -0.0537, 1.4125) -- (2.8016, -0.0537, 1.4188) -- cycle;
\fill[blue!15.0, opacity=0.5] (2.8016, -0.0537, 1.4188) -- (2.8499, -0.0537, 1.4125) -- (2.8469, -0.0569, 1.4625) -- (2.7988, -0.0569, 1.4688) -- cycle;
\fill[blue!15.0, opacity=0.5] (2.7988, -0.0569, 1.4688) -- (2.8469, -0.0569, 1.4625) -- (2.8440, -0.0600, 1.5125) -- (2.7960, -0.0600, 1.5188) -- cycle;
\fill[blue!15.0, opacity=0.5] (2.7960, -0.0600, 1.5188) -- (2.8440, -0.0600, 1.5125) -- (2.8411, -0.0631, 1.5625) -- (2.7932, -0.0631, 1.5688) -- cycle;
\fill[blue!15.0, opacity=0.5] (2.7932, -0.0631, 1.5688) -- (2.8411, -0.0631, 1.5625) -- (2.8381, -0.0663, 1.6125) -- (2.7904, -0.0663, 1.6188) -- cycle;
\fill[blue!15.0, opacity=0.5] (2.7904, -0.0663, 1.6188) -- (2.8381, -0.0663, 1.6125) -- (2.8352, -0.0694, 1.6625) -- (2.7876, -0.0694, 1.6688) -- cycle;
\fill[blue!15.0, opacity=0.5] (2.7876, -0.0694, 1.6688) -- (2.8352, -0.0694, 1.6625) -- (2.8324, -0.0725, 1.7125) -- (2.7848, -0.0725, 1.7188) -- cycle;
\fill[blue!15.0, opacity=0.5] (2.7848, -0.0725, 1.7188) -- (2.8324, -0.0725, 1.7125) -- (2.8295, -0.0755, 1.7625) -- (2.7820, -0.0755, 1.7688) -- cycle;
\fill[blue!15.0, opacity=0.5] (2.7820, -0.0755, 1.7688) -- (2.8295, -0.0755, 1.7625) -- (2.8267, -0.0785, 1.8125) -- (2.7793, -0.0785, 1.8188) -- cycle;
\fill[blue!15.0, opacity=0.5] (2.7793, -0.0785, 1.8188) -- (2.8267, -0.0785, 1.8125) -- (2.8239, -0.0815, 1.8625) -- (2.7766, -0.0815, 1.8688) -- cycle;
\fill[blue!15.0, opacity=0.5] (2.7766, -0.0815, 1.8688) -- (2.8239, -0.0815, 1.8625) -- (2.8212, -0.0844, 1.9125) -- (2.7740, -0.0844, 1.9188) -- cycle;
\fill[blue!15.0, opacity=0.5] (2.7740, -0.0844, 1.9188) -- (2.8212, -0.0844, 1.9125) -- (2.8186, -0.0872, 1.9625) -- (2.7715, -0.0872, 1.9688) -- cycle;
\fill[blue!15.0, opacity=0.5] (2.7715, -0.0872, 1.9688) -- (2.8186, -0.0872, 1.9625) -- (2.8160, -0.0900, 2.0125) -- (2.7690, -0.0900, 2.0188) -- cycle;
\fill[blue!15.0, opacity=0.5] (2.7690, -0.0900, 2.0188) -- (2.8160, -0.0900, 2.0125) -- (2.8135, -0.0927, 2.0625) -- (2.7666, -0.0927, 2.0688) -- cycle;
\fill[blue!15.0, opacity=0.5] (2.7666, -0.0927, 2.0688) -- (2.8135, -0.0927, 2.0625) -- (2.8111, -0.0953, 2.1125) -- (2.7643, -0.0953, 2.1188) -- cycle;
\fill[blue!15.0, opacity=0.5] (2.7643, -0.0953, 2.1188) -- (2.8111, -0.0953, 2.1125) -- (2.8088, -0.0978, 2.1625) -- (2.7620, -0.0978, 2.1688) -- cycle;
\fill[blue!15.0, opacity=0.5] (2.7620, -0.0978, 2.1688) -- (2.8088, -0.0978, 2.1625) -- (2.8065, -0.1001, 2.2125) -- (2.7599, -0.1001, 2.2188) -- cycle;
\fill[blue!15.0, opacity=0.5] (2.7599, -0.1001, 2.2188) -- (2.8065, -0.1001, 2.2125) -- (2.8044, -0.1024, 2.2625) -- (2.7578, -0.1024, 2.2688) -- cycle;
\fill[blue!15.0, opacity=0.5] (2.7578, -0.1024, 2.2688) -- (2.8044, -0.1024, 2.2625) -- (2.8024, -0.1046, 2.3125) -- (2.7559, -0.1046, 2.3188) -- cycle;
\fill[blue!15.0, opacity=0.5] (2.7559, -0.1046, 2.3188) -- (2.8024, -0.1046, 2.3125) -- (2.8005, -0.1066, 2.3625) -- (2.7540, -0.1066, 2.3688) -- cycle;
\fill[blue!15.0, opacity=0.5] (2.7540, -0.1066, 2.3688) -- (2.8005, -0.1066, 2.3625) -- (2.7987, -0.1085, 2.4125) -- (2.7523, -0.1085, 2.4188) -- cycle;
\fill[blue!15.0, opacity=0.5] (2.7523, -0.1085, 2.4188) -- (2.7987, -0.1085, 2.4125) -- (2.7970, -0.1103, 2.4625) -- (2.7507, -0.1103, 2.4688) -- cycle;
\fill[blue!15.0, opacity=0.5] (2.7507, -0.1103, 2.4688) -- (2.7970, -0.1103, 2.4625) -- (2.7955, -0.1120, 2.5125) -- (2.7492, -0.1120, 2.5188) -- cycle;
\fill[blue!15.0, opacity=0.5] (2.7492, -0.1120, 2.5188) -- (2.7955, -0.1120, 2.5125) -- (2.7941, -0.1135, 2.5625) -- (2.7479, -0.1135, 2.5688) -- cycle;
\fill[blue!15.0, opacity=0.5] (2.7479, -0.1135, 2.5688) -- (2.7941, -0.1135, 2.5625) -- (2.7928, -0.1148, 2.6125) -- (2.7467, -0.1148, 2.6188) -- cycle;
\fill[blue!15.1, opacity=0.5] (2.7467, -0.1148, 2.6188) -- (2.7928, -0.1148, 2.6125) -- (2.7917, -0.1160, 2.6625) -- (2.7456, -0.1160, 2.6688) -- cycle;
\fill[blue!15.1, opacity=0.5] (2.7456, -0.1160, 2.6688) -- (2.7917, -0.1160, 2.6625) -- (2.7907, -0.1171, 2.7125) -- (2.7446, -0.1171, 2.7188) -- cycle;
\fill[blue!15.1, opacity=0.5] (2.7446, -0.1171, 2.7188) -- (2.7907, -0.1171, 2.7125) -- (2.7899, -0.1180, 2.7625) -- (2.7438, -0.1180, 2.7688) -- cycle;
\fill[blue!15.1, opacity=0.5] (2.7438, -0.1180, 2.7688) -- (2.7899, -0.1180, 2.7625) -- (2.7892, -0.1187, 2.8125) -- (2.7432, -0.1187, 2.8188) -- cycle;
\fill[blue!15.2, opacity=0.5] (2.7432, -0.1187, 2.8188) -- (2.7892, -0.1187, 2.8125) -- (2.7887, -0.1193, 2.8625) -- (2.7427, -0.1193, 2.8688) -- cycle;
\fill[blue!15.2, opacity=0.5] (2.7427, -0.1193, 2.8688) -- (2.7887, -0.1193, 2.8625) -- (2.7883, -0.1197, 2.9125) -- (2.7423, -0.1197, 2.9188) -- cycle;
\fill[blue!15.3, opacity=0.5] (2.7423, -0.1197, 2.9188) -- (2.7883, -0.1197, 2.9125) -- (2.7881, -0.1199, 2.9625) -- (2.7421, -0.1199, 2.9688) -- cycle;
\fill[blue!15.4, opacity=0.5] (2.7421, -0.1199, 2.9688) -- (2.7881, -0.1199, 2.9625) -- (2.7880, -0.1200, 3.0125) -- (2.7420, -0.1200, 3.0188) -- cycle;
\fill[blue!15.0, opacity=0.5] (2.9000, -0.0000, 0.0125) -- (2.9500, -0.0000, 0.0063) -- (2.9499, -0.0001, 0.0563) -- (2.8999, -0.0001, 0.0625) -- cycle;
\fill[blue!15.0, opacity=0.5] (2.8999, -0.0001, 0.0625) -- (2.9499, -0.0001, 0.0563) -- (2.9497, -0.0003, 0.1063) -- (2.8997, -0.0003, 0.1125) -- cycle;
\fill[blue!15.0, opacity=0.5] (2.8997, -0.0003, 0.1125) -- (2.9497, -0.0003, 0.1063) -- (2.9493, -0.0007, 0.1563) -- (2.8993, -0.0007, 0.1625) -- cycle;
\fill[blue!15.0, opacity=0.5] (2.8993, -0.0007, 0.1625) -- (2.9493, -0.0007, 0.1563) -- (2.9487, -0.0013, 0.2063) -- (2.8988, -0.0013, 0.2125) -- cycle;
\fill[blue!15.0, opacity=0.5] (2.8988, -0.0013, 0.2125) -- (2.9487, -0.0013, 0.2063) -- (2.9480, -0.0020, 0.2563) -- (2.8981, -0.0020, 0.2625) -- cycle;
\fill[blue!15.0, opacity=0.5] (2.8981, -0.0020, 0.2625) -- (2.9480, -0.0020, 0.2563) -- (2.9472, -0.0029, 0.3063) -- (2.8973, -0.0029, 0.3125) -- cycle;
\fill[blue!15.0, opacity=0.5] (2.8973, -0.0029, 0.3125) -- (2.9472, -0.0029, 0.3063) -- (2.9461, -0.0040, 0.3563) -- (2.8963, -0.0040, 0.3625) -- cycle;
\fill[blue!15.0, opacity=0.5] (2.8963, -0.0040, 0.3625) -- (2.9461, -0.0040, 0.3563) -- (2.9450, -0.0052, 0.4063) -- (2.8952, -0.0052, 0.4125) -- cycle;
\fill[blue!15.0, opacity=0.5] (2.8952, -0.0052, 0.4125) -- (2.9450, -0.0052, 0.4063) -- (2.9437, -0.0065, 0.4563) -- (2.8939, -0.0065, 0.4625) -- cycle;
\fill[blue!15.0, opacity=0.5] (2.8939, -0.0065, 0.4625) -- (2.9437, -0.0065, 0.4563) -- (2.9422, -0.0080, 0.5063) -- (2.8925, -0.0080, 0.5125) -- cycle;
\fill[blue!15.0, opacity=0.5] (2.8925, -0.0080, 0.5125) -- (2.9422, -0.0080, 0.5063) -- (2.9406, -0.0097, 0.5563) -- (2.8910, -0.0097, 0.5625) -- cycle;
\fill[blue!15.0, opacity=0.5] (2.8910, -0.0097, 0.5625) -- (2.9406, -0.0097, 0.5563) -- (2.9389, -0.0115, 0.6063) -- (2.8893, -0.0115, 0.6125) -- cycle;
\fill[blue!15.0, opacity=0.5] (2.8893, -0.0115, 0.6125) -- (2.9389, -0.0115, 0.6063) -- (2.9371, -0.0134, 0.6563) -- (2.8875, -0.0134, 0.6625) -- cycle;
\fill[blue!15.0, opacity=0.5] (2.8875, -0.0134, 0.6625) -- (2.9371, -0.0134, 0.6563) -- (2.9351, -0.0154, 0.7063) -- (2.8856, -0.0154, 0.7125) -- cycle;
\fill[blue!15.0, opacity=0.5] (2.8856, -0.0154, 0.7125) -- (2.9351, -0.0154, 0.7063) -- (2.9330, -0.0176, 0.7563) -- (2.8836, -0.0176, 0.7625) -- cycle;
\fill[blue!15.0, opacity=0.5] (2.8836, -0.0176, 0.7625) -- (2.9330, -0.0176, 0.7563) -- (2.9308, -0.0199, 0.8063) -- (2.8815, -0.0199, 0.8125) -- cycle;
\fill[blue!15.0, opacity=0.5] (2.8815, -0.0199, 0.8125) -- (2.9308, -0.0199, 0.8063) -- (2.9285, -0.0222, 0.8563) -- (2.8792, -0.0222, 0.8625) -- cycle;
\fill[blue!15.0, opacity=0.5] (2.8792, -0.0222, 0.8625) -- (2.9285, -0.0222, 0.8563) -- (2.9261, -0.0247, 0.9063) -- (2.8769, -0.0247, 0.9125) -- cycle;
\fill[blue!15.0, opacity=0.5] (2.8769, -0.0247, 0.9125) -- (2.9261, -0.0247, 0.9063) -- (2.9236, -0.0273, 0.9563) -- (2.8745, -0.0273, 0.9625) -- cycle;
\fill[blue!15.0, opacity=0.5] (2.8745, -0.0273, 0.9625) -- (2.9236, -0.0273, 0.9563) -- (2.9210, -0.0300, 1.0063) -- (2.8720, -0.0300, 1.0125) -- cycle;
\fill[blue!15.0, opacity=0.5] (2.8720, -0.0300, 1.0125) -- (2.9210, -0.0300, 1.0063) -- (2.9183, -0.0328, 1.0563) -- (2.8694, -0.0328, 1.0625) -- cycle;
\fill[blue!15.0, opacity=0.5] (2.8694, -0.0328, 1.0625) -- (2.9183, -0.0328, 1.0563) -- (2.9156, -0.0356, 1.1063) -- (2.8668, -0.0356, 1.1125) -- cycle;
\fill[blue!15.0, opacity=0.5] (2.8668, -0.0356, 1.1125) -- (2.9156, -0.0356, 1.1063) -- (2.9128, -0.0385, 1.1563) -- (2.8641, -0.0385, 1.1625) -- cycle;
\fill[blue!15.0, opacity=0.5] (2.8641, -0.0385, 1.1625) -- (2.9128, -0.0385, 1.1563) -- (2.9099, -0.0415, 1.2063) -- (2.8613, -0.0415, 1.2125) -- cycle;
\fill[blue!15.0, opacity=0.5] (2.8613, -0.0415, 1.2125) -- (2.9099, -0.0415, 1.2063) -- (2.9070, -0.0445, 1.2563) -- (2.8585, -0.0445, 1.2625) -- cycle;
\fill[blue!15.0, opacity=0.5] (2.8585, -0.0445, 1.2625) -- (2.9070, -0.0445, 1.2563) -- (2.9041, -0.0475, 1.3063) -- (2.8556, -0.0475, 1.3125) -- cycle;
\fill[blue!15.0, opacity=0.5] (2.8556, -0.0475, 1.3125) -- (2.9041, -0.0475, 1.3063) -- (2.9011, -0.0506, 1.3563) -- (2.8528, -0.0506, 1.3625) -- cycle;
\fill[blue!15.0, opacity=0.5] (2.8528, -0.0506, 1.3625) -- (2.9011, -0.0506, 1.3563) -- (2.8981, -0.0537, 1.4063) -- (2.8499, -0.0537, 1.4125) -- cycle;
\fill[blue!15.0, opacity=0.5] (2.8499, -0.0537, 1.4125) -- (2.8981, -0.0537, 1.4063) -- (2.8950, -0.0569, 1.4563) -- (2.8469, -0.0569, 1.4625) -- cycle;
\fill[blue!15.0, opacity=0.5] (2.8469, -0.0569, 1.4625) -- (2.8950, -0.0569, 1.4563) -- (2.8920, -0.0600, 1.5063) -- (2.8440, -0.0600, 1.5125) -- cycle;
\fill[blue!15.0, opacity=0.5] (2.8440, -0.0600, 1.5125) -- (2.8920, -0.0600, 1.5063) -- (2.8890, -0.0631, 1.5563) -- (2.8411, -0.0631, 1.5625) -- cycle;
\fill[blue!15.0, opacity=0.5] (2.8411, -0.0631, 1.5625) -- (2.8890, -0.0631, 1.5563) -- (2.8859, -0.0663, 1.6063) -- (2.8381, -0.0663, 1.6125) -- cycle;
\fill[blue!15.0, opacity=0.5] (2.8381, -0.0663, 1.6125) -- (2.8859, -0.0663, 1.6063) -- (2.8829, -0.0694, 1.6563) -- (2.8352, -0.0694, 1.6625) -- cycle;
\fill[blue!15.0, opacity=0.5] (2.8352, -0.0694, 1.6625) -- (2.8829, -0.0694, 1.6563) -- (2.8799, -0.0725, 1.7063) -- (2.8324, -0.0725, 1.7125) -- cycle;
\fill[blue!15.0, opacity=0.5] (2.8324, -0.0725, 1.7125) -- (2.8799, -0.0725, 1.7063) -- (2.8770, -0.0755, 1.7563) -- (2.8295, -0.0755, 1.7625) -- cycle;
\fill[blue!15.0, opacity=0.5] (2.8295, -0.0755, 1.7625) -- (2.8770, -0.0755, 1.7563) -- (2.8741, -0.0785, 1.8063) -- (2.8267, -0.0785, 1.8125) -- cycle;
\fill[blue!15.0, opacity=0.5] (2.8267, -0.0785, 1.8125) -- (2.8741, -0.0785, 1.8063) -- (2.8712, -0.0815, 1.8563) -- (2.8239, -0.0815, 1.8625) -- cycle;
\fill[blue!15.0, opacity=0.5] (2.8239, -0.0815, 1.8625) -- (2.8712, -0.0815, 1.8563) -- (2.8684, -0.0844, 1.9063) -- (2.8212, -0.0844, 1.9125) -- cycle;
\fill[blue!15.0, opacity=0.5] (2.8212, -0.0844, 1.9125) -- (2.8684, -0.0844, 1.9063) -- (2.8657, -0.0872, 1.9563) -- (2.8186, -0.0872, 1.9625) -- cycle;
\fill[blue!15.0, opacity=0.5] (2.8186, -0.0872, 1.9625) -- (2.8657, -0.0872, 1.9563) -- (2.8630, -0.0900, 2.0063) -- (2.8160, -0.0900, 2.0125) -- cycle;
\fill[blue!15.0, opacity=0.5] (2.8160, -0.0900, 2.0125) -- (2.8630, -0.0900, 2.0063) -- (2.8604, -0.0927, 2.0563) -- (2.8135, -0.0927, 2.0625) -- cycle;
\fill[blue!15.0, opacity=0.5] (2.8135, -0.0927, 2.0625) -- (2.8604, -0.0927, 2.0563) -- (2.8579, -0.0953, 2.1063) -- (2.8111, -0.0953, 2.1125) -- cycle;
\fill[blue!15.0, opacity=0.5] (2.8111, -0.0953, 2.1125) -- (2.8579, -0.0953, 2.1063) -- (2.8555, -0.0978, 2.1563) -- (2.8088, -0.0978, 2.1625) -- cycle;
\fill[blue!15.0, opacity=0.5] (2.8088, -0.0978, 2.1625) -- (2.8555, -0.0978, 2.1563) -- (2.8532, -0.1001, 2.2063) -- (2.8065, -0.1001, 2.2125) -- cycle;
\fill[blue!15.0, opacity=0.5] (2.8065, -0.1001, 2.2125) -- (2.8532, -0.1001, 2.2063) -- (2.8510, -0.1024, 2.2563) -- (2.8044, -0.1024, 2.2625) -- cycle;
\fill[blue!15.0, opacity=0.5] (2.8044, -0.1024, 2.2625) -- (2.8510, -0.1024, 2.2563) -- (2.8489, -0.1046, 2.3063) -- (2.8024, -0.1046, 2.3125) -- cycle;
\fill[blue!15.0, opacity=0.5] (2.8024, -0.1046, 2.3125) -- (2.8489, -0.1046, 2.3063) -- (2.8469, -0.1066, 2.3563) -- (2.8005, -0.1066, 2.3625) -- cycle;
\fill[blue!15.0, opacity=0.5] (2.8005, -0.1066, 2.3625) -- (2.8469, -0.1066, 2.3563) -- (2.8451, -0.1085, 2.4063) -- (2.7987, -0.1085, 2.4125) -- cycle;
\fill[blue!15.0, opacity=0.5] (2.7987, -0.1085, 2.4125) -- (2.8451, -0.1085, 2.4063) -- (2.8434, -0.1103, 2.4563) -- (2.7970, -0.1103, 2.4625) -- cycle;
\fill[blue!15.0, opacity=0.5] (2.7970, -0.1103, 2.4625) -- (2.8434, -0.1103, 2.4563) -- (2.8418, -0.1120, 2.5063) -- (2.7955, -0.1120, 2.5125) -- cycle;
\fill[blue!15.0, opacity=0.5] (2.7955, -0.1120, 2.5125) -- (2.8418, -0.1120, 2.5063) -- (2.8403, -0.1135, 2.5563) -- (2.7941, -0.1135, 2.5625) -- cycle;
\fill[blue!15.0, opacity=0.5] (2.7941, -0.1135, 2.5625) -- (2.8403, -0.1135, 2.5563) -- (2.8390, -0.1148, 2.6063) -- (2.7928, -0.1148, 2.6125) -- cycle;
\fill[blue!15.0, opacity=0.5] (2.7928, -0.1148, 2.6125) -- (2.8390, -0.1148, 2.6063) -- (2.8379, -0.1160, 2.6563) -- (2.7917, -0.1160, 2.6625) -- cycle;
\fill[blue!15.0, opacity=0.5] (2.7917, -0.1160, 2.6625) -- (2.8379, -0.1160, 2.6563) -- (2.8368, -0.1171, 2.7063) -- (2.7907, -0.1171, 2.7125) -- cycle;
\fill[blue!15.0, opacity=0.5] (2.7907, -0.1171, 2.7125) -- (2.8368, -0.1171, 2.7063) -- (2.8360, -0.1180, 2.7563) -- (2.7899, -0.1180, 2.7625) -- cycle;
\fill[blue!15.0, opacity=0.5] (2.7899, -0.1180, 2.7625) -- (2.8360, -0.1180, 2.7563) -- (2.8353, -0.1187, 2.8063) -- (2.7892, -0.1187, 2.8125) -- cycle;
\fill[blue!15.0, opacity=0.5] (2.7892, -0.1187, 2.8125) -- (2.8353, -0.1187, 2.8063) -- (2.8347, -0.1193, 2.8563) -- (2.7887, -0.1193, 2.8625) -- cycle;
\fill[blue!15.0, opacity=0.5] (2.7887, -0.1193, 2.8625) -- (2.8347, -0.1193, 2.8563) -- (2.8343, -0.1197, 2.9063) -- (2.7883, -0.1197, 2.9125) -- cycle;
\fill[blue!15.1, opacity=0.5] (2.7883, -0.1197, 2.9125) -- (2.8343, -0.1197, 2.9063) -- (2.8341, -0.1199, 2.9563) -- (2.7881, -0.1199, 2.9625) -- cycle;
\fill[blue!15.1, opacity=0.5] (2.7881, -0.1199, 2.9625) -- (2.8341, -0.1199, 2.9563) -- (2.8340, -0.1200, 3.0063) -- (2.7880, -0.1200, 3.0125) -- cycle;
\fill[blue!15.0, opacity=0.5] (2.9500, -0.0000, 0.0063) -- (3.0000, -0.0000, 0.0000) -- (2.9999, -0.0001, 0.0500) -- (2.9499, -0.0001, 0.0563) -- cycle;
\fill[blue!15.0, opacity=0.5] (2.9499, -0.0001, 0.0563) -- (2.9999, -0.0001, 0.0500) -- (2.9997, -0.0003, 0.1000) -- (2.9497, -0.0003, 0.1063) -- cycle;
\fill[blue!15.0, opacity=0.5] (2.9497, -0.0003, 0.1063) -- (2.9997, -0.0003, 0.1000) -- (2.9993, -0.0007, 0.1500) -- (2.9493, -0.0007, 0.1563) -- cycle;
\fill[blue!15.0, opacity=0.5] (2.9493, -0.0007, 0.1563) -- (2.9993, -0.0007, 0.1500) -- (2.9987, -0.0013, 0.2000) -- (2.9487, -0.0013, 0.2063) -- cycle;
\fill[blue!15.0, opacity=0.5] (2.9487, -0.0013, 0.2063) -- (2.9987, -0.0013, 0.2000) -- (2.9980, -0.0020, 0.2500) -- (2.9480, -0.0020, 0.2563) -- cycle;
\fill[blue!15.0, opacity=0.5] (2.9480, -0.0020, 0.2563) -- (2.9980, -0.0020, 0.2500) -- (2.9971, -0.0029, 0.3000) -- (2.9472, -0.0029, 0.3063) -- cycle;
\fill[blue!15.0, opacity=0.5] (2.9472, -0.0029, 0.3063) -- (2.9971, -0.0029, 0.3000) -- (2.9960, -0.0040, 0.3500) -- (2.9461, -0.0040, 0.3563) -- cycle;
\fill[blue!15.0, opacity=0.5] (2.9461, -0.0040, 0.3563) -- (2.9960, -0.0040, 0.3500) -- (2.9948, -0.0052, 0.4000) -- (2.9450, -0.0052, 0.4063) -- cycle;
\fill[blue!15.0, opacity=0.5] (2.9450, -0.0052, 0.4063) -- (2.9948, -0.0052, 0.4000) -- (2.9935, -0.0065, 0.4500) -- (2.9437, -0.0065, 0.4563) -- cycle;
\fill[blue!15.0, opacity=0.5] (2.9437, -0.0065, 0.4563) -- (2.9935, -0.0065, 0.4500) -- (2.9920, -0.0080, 0.5000) -- (2.9422, -0.0080, 0.5063) -- cycle;
\fill[blue!15.0, opacity=0.5] (2.9422, -0.0080, 0.5063) -- (2.9920, -0.0080, 0.5000) -- (2.9903, -0.0097, 0.5500) -- (2.9406, -0.0097, 0.5563) -- cycle;
\fill[blue!15.0, opacity=0.5] (2.9406, -0.0097, 0.5563) -- (2.9903, -0.0097, 0.5500) -- (2.9885, -0.0115, 0.6000) -- (2.9389, -0.0115, 0.6063) -- cycle;
\fill[blue!15.0, opacity=0.5] (2.9389, -0.0115, 0.6063) -- (2.9885, -0.0115, 0.6000) -- (2.9866, -0.0134, 0.6500) -- (2.9371, -0.0134, 0.6563) -- cycle;
\fill[blue!15.0, opacity=0.5] (2.9371, -0.0134, 0.6563) -- (2.9866, -0.0134, 0.6500) -- (2.9846, -0.0154, 0.7000) -- (2.9351, -0.0154, 0.7063) -- cycle;
\fill[blue!15.0, opacity=0.5] (2.9351, -0.0154, 0.7063) -- (2.9846, -0.0154, 0.7000) -- (2.9824, -0.0176, 0.7500) -- (2.9330, -0.0176, 0.7563) -- cycle;
\fill[blue!15.0, opacity=0.5] (2.9330, -0.0176, 0.7563) -- (2.9824, -0.0176, 0.7500) -- (2.9801, -0.0199, 0.8000) -- (2.9308, -0.0199, 0.8063) -- cycle;
\fill[blue!15.0, opacity=0.5] (2.9308, -0.0199, 0.8063) -- (2.9801, -0.0199, 0.8000) -- (2.9778, -0.0222, 0.8500) -- (2.9285, -0.0222, 0.8563) -- cycle;
\fill[blue!15.0, opacity=0.5] (2.9285, -0.0222, 0.8563) -- (2.9778, -0.0222, 0.8500) -- (2.9753, -0.0247, 0.9000) -- (2.9261, -0.0247, 0.9063) -- cycle;
\fill[blue!15.0, opacity=0.5] (2.9261, -0.0247, 0.9063) -- (2.9753, -0.0247, 0.9000) -- (2.9727, -0.0273, 0.9500) -- (2.9236, -0.0273, 0.9563) -- cycle;
\fill[blue!15.0, opacity=0.5] (2.9236, -0.0273, 0.9563) -- (2.9727, -0.0273, 0.9500) -- (2.9700, -0.0300, 1.0000) -- (2.9210, -0.0300, 1.0063) -- cycle;
\fill[blue!15.0, opacity=0.5] (2.9210, -0.0300, 1.0063) -- (2.9700, -0.0300, 1.0000) -- (2.9672, -0.0328, 1.0500) -- (2.9183, -0.0328, 1.0563) -- cycle;
\fill[blue!15.0, opacity=0.5] (2.9183, -0.0328, 1.0563) -- (2.9672, -0.0328, 1.0500) -- (2.9644, -0.0356, 1.1000) -- (2.9156, -0.0356, 1.1063) -- cycle;
\fill[blue!15.0, opacity=0.5] (2.9156, -0.0356, 1.1063) -- (2.9644, -0.0356, 1.1000) -- (2.9615, -0.0385, 1.1500) -- (2.9128, -0.0385, 1.1563) -- cycle;
\fill[blue!15.0, opacity=0.5] (2.9128, -0.0385, 1.1563) -- (2.9615, -0.0385, 1.1500) -- (2.9585, -0.0415, 1.2000) -- (2.9099, -0.0415, 1.2063) -- cycle;
\fill[blue!15.0, opacity=0.5] (2.9099, -0.0415, 1.2063) -- (2.9585, -0.0415, 1.2000) -- (2.9555, -0.0445, 1.2500) -- (2.9070, -0.0445, 1.2563) -- cycle;
\fill[blue!15.0, opacity=0.5] (2.9070, -0.0445, 1.2563) -- (2.9555, -0.0445, 1.2500) -- (2.9525, -0.0475, 1.3000) -- (2.9041, -0.0475, 1.3063) -- cycle;
\fill[blue!15.0, opacity=0.5] (2.9041, -0.0475, 1.3063) -- (2.9525, -0.0475, 1.3000) -- (2.9494, -0.0506, 1.3500) -- (2.9011, -0.0506, 1.3563) -- cycle;
\fill[blue!15.0, opacity=0.5] (2.9011, -0.0506, 1.3563) -- (2.9494, -0.0506, 1.3500) -- (2.9463, -0.0537, 1.4000) -- (2.8981, -0.0537, 1.4063) -- cycle;
\fill[blue!15.0, opacity=0.5] (2.8981, -0.0537, 1.4063) -- (2.9463, -0.0537, 1.4000) -- (2.9431, -0.0569, 1.4500) -- (2.8950, -0.0569, 1.4563) -- cycle;
\fill[blue!15.0, opacity=0.5] (2.8950, -0.0569, 1.4563) -- (2.9431, -0.0569, 1.4500) -- (2.9400, -0.0600, 1.5000) -- (2.8920, -0.0600, 1.5063) -- cycle;
\fill[blue!15.0, opacity=0.5] (2.8920, -0.0600, 1.5063) -- (2.9400, -0.0600, 1.5000) -- (2.9369, -0.0631, 1.5500) -- (2.8890, -0.0631, 1.5563) -- cycle;
\fill[blue!15.0, opacity=0.5] (2.8890, -0.0631, 1.5563) -- (2.9369, -0.0631, 1.5500) -- (2.9337, -0.0663, 1.6000) -- (2.8859, -0.0663, 1.6063) -- cycle;
\fill[blue!15.0, opacity=0.5] (2.8859, -0.0663, 1.6063) -- (2.9337, -0.0663, 1.6000) -- (2.9306, -0.0694, 1.6500) -- (2.8829, -0.0694, 1.6563) -- cycle;
\fill[blue!15.0, opacity=0.5] (2.8829, -0.0694, 1.6563) -- (2.9306, -0.0694, 1.6500) -- (2.9275, -0.0725, 1.7000) -- (2.8799, -0.0725, 1.7063) -- cycle;
\fill[blue!15.0, opacity=0.5] (2.8799, -0.0725, 1.7063) -- (2.9275, -0.0725, 1.7000) -- (2.9245, -0.0755, 1.7500) -- (2.8770, -0.0755, 1.7563) -- cycle;
\fill[blue!15.0, opacity=0.5] (2.8770, -0.0755, 1.7563) -- (2.9245, -0.0755, 1.7500) -- (2.9215, -0.0785, 1.8000) -- (2.8741, -0.0785, 1.8063) -- cycle;
\fill[blue!15.0, opacity=0.5] (2.8741, -0.0785, 1.8063) -- (2.9215, -0.0785, 1.8000) -- (2.9185, -0.0815, 1.8500) -- (2.8712, -0.0815, 1.8563) -- cycle;
\fill[blue!15.0, opacity=0.5] (2.8712, -0.0815, 1.8563) -- (2.9185, -0.0815, 1.8500) -- (2.9156, -0.0844, 1.9000) -- (2.8684, -0.0844, 1.9063) -- cycle;
\fill[blue!15.0, opacity=0.5] (2.8684, -0.0844, 1.9063) -- (2.9156, -0.0844, 1.9000) -- (2.9128, -0.0872, 1.9500) -- (2.8657, -0.0872, 1.9563) -- cycle;
\fill[blue!15.0, opacity=0.5] (2.8657, -0.0872, 1.9563) -- (2.9128, -0.0872, 1.9500) -- (2.9100, -0.0900, 2.0000) -- (2.8630, -0.0900, 2.0063) -- cycle;
\fill[blue!15.0, opacity=0.5] (2.8630, -0.0900, 2.0063) -- (2.9100, -0.0900, 2.0000) -- (2.9073, -0.0927, 2.0500) -- (2.8604, -0.0927, 2.0563) -- cycle;
\fill[blue!15.0, opacity=0.5] (2.8604, -0.0927, 2.0563) -- (2.9073, -0.0927, 2.0500) -- (2.9047, -0.0953, 2.1000) -- (2.8579, -0.0953, 2.1063) -- cycle;
\fill[blue!15.0, opacity=0.5] (2.8579, -0.0953, 2.1063) -- (2.9047, -0.0953, 2.1000) -- (2.9022, -0.0978, 2.1500) -- (2.8555, -0.0978, 2.1563) -- cycle;
\fill[blue!15.0, opacity=0.5] (2.8555, -0.0978, 2.1563) -- (2.9022, -0.0978, 2.1500) -- (2.8999, -0.1001, 2.2000) -- (2.8532, -0.1001, 2.2063) -- cycle;
\fill[blue!15.0, opacity=0.5] (2.8532, -0.1001, 2.2063) -- (2.8999, -0.1001, 2.2000) -- (2.8976, -0.1024, 2.2500) -- (2.8510, -0.1024, 2.2563) -- cycle;
\fill[blue!15.0, opacity=0.5] (2.8510, -0.1024, 2.2563) -- (2.8976, -0.1024, 2.2500) -- (2.8954, -0.1046, 2.3000) -- (2.8489, -0.1046, 2.3063) -- cycle;
\fill[blue!15.0, opacity=0.5] (2.8489, -0.1046, 2.3063) -- (2.8954, -0.1046, 2.3000) -- (2.8934, -0.1066, 2.3500) -- (2.8469, -0.1066, 2.3563) -- cycle;
\fill[blue!15.0, opacity=0.5] (2.8469, -0.1066, 2.3563) -- (2.8934, -0.1066, 2.3500) -- (2.8915, -0.1085, 2.4000) -- (2.8451, -0.1085, 2.4063) -- cycle;
\fill[blue!15.0, opacity=0.5] (2.8451, -0.1085, 2.4063) -- (2.8915, -0.1085, 2.4000) -- (2.8897, -0.1103, 2.4500) -- (2.8434, -0.1103, 2.4563) -- cycle;
\fill[blue!15.0, opacity=0.5] (2.8434, -0.1103, 2.4563) -- (2.8897, -0.1103, 2.4500) -- (2.8880, -0.1120, 2.5000) -- (2.8418, -0.1120, 2.5063) -- cycle;
\fill[blue!15.0, opacity=0.5] (2.8418, -0.1120, 2.5063) -- (2.8880, -0.1120, 2.5000) -- (2.8865, -0.1135, 2.5500) -- (2.8403, -0.1135, 2.5563) -- cycle;
\fill[blue!15.0, opacity=0.5] (2.8403, -0.1135, 2.5563) -- (2.8865, -0.1135, 2.5500) -- (2.8852, -0.1148, 2.6000) -- (2.8390, -0.1148, 2.6063) -- cycle;
\fill[blue!15.0, opacity=0.5] (2.8390, -0.1148, 2.6063) -- (2.8852, -0.1148, 2.6000) -- (2.8840, -0.1160, 2.6500) -- (2.8379, -0.1160, 2.6563) -- cycle;
\fill[blue!15.0, opacity=0.5] (2.8379, -0.1160, 2.6563) -- (2.8840, -0.1160, 2.6500) -- (2.8829, -0.1171, 2.7000) -- (2.8368, -0.1171, 2.7063) -- cycle;
\fill[blue!15.0, opacity=0.5] (2.8368, -0.1171, 2.7063) -- (2.8829, -0.1171, 2.7000) -- (2.8820, -0.1180, 2.7500) -- (2.8360, -0.1180, 2.7563) -- cycle;
\fill[blue!15.0, opacity=0.5] (2.8360, -0.1180, 2.7563) -- (2.8820, -0.1180, 2.7500) -- (2.8813, -0.1187, 2.8000) -- (2.8353, -0.1187, 2.8063) -- cycle;
\fill[blue!15.0, opacity=0.5] (2.8353, -0.1187, 2.8063) -- (2.8813, -0.1187, 2.8000) -- (2.8807, -0.1193, 2.8500) -- (2.8347, -0.1193, 2.8563) -- cycle;
\fill[blue!15.0, opacity=0.5] (2.8347, -0.1193, 2.8563) -- (2.8807, -0.1193, 2.8500) -- (2.8803, -0.1197, 2.9000) -- (2.8343, -0.1197, 2.9063) -- cycle;
\fill[blue!15.0, opacity=0.5] (2.8343, -0.1197, 2.9063) -- (2.8803, -0.1197, 2.9000) -- (2.8801, -0.1199, 2.9500) -- (2.8341, -0.1199, 2.9563) -- cycle;
\fill[blue!15.0, opacity=0.5] (2.8341, -0.1199, 2.9563) -- (2.8801, -0.1199, 2.9500) -- (2.8800, -0.1200, 3.0000) -- (2.8340, -0.1200, 3.0063) -- cycle;
% Right face patches
\fill[blue!15.0, opacity=0.5] (3.0000, 0.0000, 0.0000) -- (3.0000, 0.0500, 0.0063) -- (2.9999, 0.0499, 0.0563) -- (2.9999, -0.0001, 0.0500) -- cycle;
\fill[blue!15.0, opacity=0.5] (2.9999, -0.0001, 0.0500) -- (2.9999, 0.0499, 0.0563) -- (2.9997, 0.0497, 0.1063) -- (2.9997, -0.0003, 0.1000) -- cycle;
\fill[blue!15.0, opacity=0.5] (2.9997, -0.0003, 0.1000) -- (2.9997, 0.0497, 0.1063) -- (2.9993, 0.0493, 0.1563) -- (2.9993, -0.0007, 0.1500) -- cycle;
\fill[blue!15.0, opacity=0.5] (2.9993, -0.0007, 0.1500) -- (2.9993, 0.0493, 0.1563) -- (2.9987, 0.0487, 0.2063) -- (2.9987, -0.0013, 0.2000) -- cycle;
\fill[blue!15.0, opacity=0.5] (2.9987, -0.0013, 0.2000) -- (2.9987, 0.0487, 0.2063) -- (2.9980, 0.0480, 0.2563) -- (2.9980, -0.0020, 0.2500) -- cycle;
\fill[blue!15.0, opacity=0.5] (2.9980, -0.0020, 0.2500) -- (2.9980, 0.0480, 0.2563) -- (2.9971, 0.0472, 0.3063) -- (2.9971, -0.0029, 0.3000) -- cycle;
\fill[blue!15.0, opacity=0.5] (2.9971, -0.0029, 0.3000) -- (2.9971, 0.0472, 0.3063) -- (2.9960, 0.0461, 0.3563) -- (2.9960, -0.0040, 0.3500) -- cycle;
\fill[blue!15.0, opacity=0.5] (2.9960, -0.0040, 0.3500) -- (2.9960, 0.0461, 0.3563) -- (2.9948, 0.0450, 0.4063) -- (2.9948, -0.0052, 0.4000) -- cycle;
\fill[blue!15.0, opacity=0.5] (2.9948, -0.0052, 0.4000) -- (2.9948, 0.0450, 0.4063) -- (2.9935, 0.0437, 0.4563) -- (2.9935, -0.0065, 0.4500) -- cycle;
\fill[blue!15.0, opacity=0.5] (2.9935, -0.0065, 0.4500) -- (2.9935, 0.0437, 0.4563) -- (2.9920, 0.0422, 0.5063) -- (2.9920, -0.0080, 0.5000) -- cycle;
\fill[blue!15.0, opacity=0.5] (2.9920, -0.0080, 0.5000) -- (2.9920, 0.0422, 0.5063) -- (2.9903, 0.0406, 0.5563) -- (2.9903, -0.0097, 0.5500) -- cycle;
\fill[blue!15.0, opacity=0.5] (2.9903, -0.0097, 0.5500) -- (2.9903, 0.0406, 0.5563) -- (2.9885, 0.0389, 0.6063) -- (2.9885, -0.0115, 0.6000) -- cycle;
\fill[blue!15.0, opacity=0.5] (2.9885, -0.0115, 0.6000) -- (2.9885, 0.0389, 0.6063) -- (2.9866, 0.0371, 0.6563) -- (2.9866, -0.0134, 0.6500) -- cycle;
\fill[blue!15.0, opacity=0.5] (2.9866, -0.0134, 0.6500) -- (2.9866, 0.0371, 0.6563) -- (2.9846, 0.0351, 0.7063) -- (2.9846, -0.0154, 0.7000) -- cycle;
\fill[blue!15.0, opacity=0.5] (2.9846, -0.0154, 0.7000) -- (2.9846, 0.0351, 0.7063) -- (2.9824, 0.0330, 0.7563) -- (2.9824, -0.0176, 0.7500) -- cycle;
\fill[blue!15.0, opacity=0.5] (2.9824, -0.0176, 0.7500) -- (2.9824, 0.0330, 0.7563) -- (2.9801, 0.0308, 0.8063) -- (2.9801, -0.0199, 0.8000) -- cycle;
\fill[blue!15.0, opacity=0.5] (2.9801, -0.0199, 0.8000) -- (2.9801, 0.0308, 0.8063) -- (2.9778, 0.0285, 0.8563) -- (2.9778, -0.0222, 0.8500) -- cycle;
\fill[blue!15.0, opacity=0.5] (2.9778, -0.0222, 0.8500) -- (2.9778, 0.0285, 0.8563) -- (2.9753, 0.0261, 0.9063) -- (2.9753, -0.0247, 0.9000) -- cycle;
\fill[blue!15.0, opacity=0.5] (2.9753, -0.0247, 0.9000) -- (2.9753, 0.0261, 0.9063) -- (2.9727, 0.0236, 0.9563) -- (2.9727, -0.0273, 0.9500) -- cycle;
\fill[blue!15.0, opacity=0.5] (2.9727, -0.0273, 0.9500) -- (2.9727, 0.0236, 0.9563) -- (2.9700, 0.0210, 1.0063) -- (2.9700, -0.0300, 1.0000) -- cycle;
\fill[blue!15.0, opacity=0.5] (2.9700, -0.0300, 1.0000) -- (2.9700, 0.0210, 1.0063) -- (2.9672, 0.0183, 1.0563) -- (2.9672, -0.0328, 1.0500) -- cycle;
\fill[blue!15.0, opacity=0.5] (2.9672, -0.0328, 1.0500) -- (2.9672, 0.0183, 1.0563) -- (2.9644, 0.0156, 1.1063) -- (2.9644, -0.0356, 1.1000) -- cycle;
\fill[blue!15.0, opacity=0.5] (2.9644, -0.0356, 1.1000) -- (2.9644, 0.0156, 1.1063) -- (2.9615, 0.0128, 1.1563) -- (2.9615, -0.0385, 1.1500) -- cycle;
\fill[blue!15.0, opacity=0.5] (2.9615, -0.0385, 1.1500) -- (2.9615, 0.0128, 1.1563) -- (2.9585, 0.0099, 1.2063) -- (2.9585, -0.0415, 1.2000) -- cycle;
\fill[blue!15.0, opacity=0.5] (2.9585, -0.0415, 1.2000) -- (2.9585, 0.0099, 1.2063) -- (2.9555, 0.0070, 1.2563) -- (2.9555, -0.0445, 1.2500) -- cycle;
\fill[blue!15.0, opacity=0.5] (2.9555, -0.0445, 1.2500) -- (2.9555, 0.0070, 1.2563) -- (2.9525, 0.0041, 1.3063) -- (2.9525, -0.0475, 1.3000) -- cycle;
\fill[blue!15.0, opacity=0.5] (2.9525, -0.0475, 1.3000) -- (2.9525, 0.0041, 1.3063) -- (2.9494, 0.0011, 1.3563) -- (2.9494, -0.0506, 1.3500) -- cycle;
\fill[blue!15.0, opacity=0.5] (2.9494, -0.0506, 1.3500) -- (2.9494, 0.0011, 1.3563) -- (2.9463, -0.0019, 1.4063) -- (2.9463, -0.0537, 1.4000) -- cycle;
\fill[blue!15.0, opacity=0.5] (2.9463, -0.0537, 1.4000) -- (2.9463, -0.0019, 1.4063) -- (2.9431, -0.0050, 1.4563) -- (2.9431, -0.0569, 1.4500) -- cycle;
\fill[blue!15.0, opacity=0.5] (2.9431, -0.0569, 1.4500) -- (2.9431, -0.0050, 1.4563) -- (2.9400, -0.0080, 1.5063) -- (2.9400, -0.0600, 1.5000) -- cycle;
\fill[blue!15.0, opacity=0.5] (2.9400, -0.0600, 1.5000) -- (2.9400, -0.0080, 1.5063) -- (2.9369, -0.0110, 1.5563) -- (2.9369, -0.0631, 1.5500) -- cycle;
\fill[blue!15.0, opacity=0.5] (2.9369, -0.0631, 1.5500) -- (2.9369, -0.0110, 1.5563) -- (2.9337, -0.0141, 1.6063) -- (2.9337, -0.0663, 1.6000) -- cycle;
\fill[blue!15.0, opacity=0.5] (2.9337, -0.0663, 1.6000) -- (2.9337, -0.0141, 1.6063) -- (2.9306, -0.0171, 1.6563) -- (2.9306, -0.0694, 1.6500) -- cycle;
\fill[blue!15.0, opacity=0.5] (2.9306, -0.0694, 1.6500) -- (2.9306, -0.0171, 1.6563) -- (2.9275, -0.0201, 1.7063) -- (2.9275, -0.0725, 1.7000) -- cycle;
\fill[blue!15.0, opacity=0.5] (2.9275, -0.0725, 1.7000) -- (2.9275, -0.0201, 1.7063) -- (2.9245, -0.0230, 1.7563) -- (2.9245, -0.0755, 1.7500) -- cycle;
\fill[blue!15.0, opacity=0.5] (2.9245, -0.0755, 1.7500) -- (2.9245, -0.0230, 1.7563) -- (2.9215, -0.0259, 1.8063) -- (2.9215, -0.0785, 1.8000) -- cycle;
\fill[blue!15.0, opacity=0.5] (2.9215, -0.0785, 1.8000) -- (2.9215, -0.0259, 1.8063) -- (2.9185, -0.0288, 1.8563) -- (2.9185, -0.0815, 1.8500) -- cycle;
\fill[blue!15.0, opacity=0.5] (2.9185, -0.0815, 1.8500) -- (2.9185, -0.0288, 1.8563) -- (2.9156, -0.0316, 1.9063) -- (2.9156, -0.0844, 1.9000) -- cycle;
\fill[blue!15.0, opacity=0.5] (2.9156, -0.0844, 1.9000) -- (2.9156, -0.0316, 1.9063) -- (2.9128, -0.0343, 1.9563) -- (2.9128, -0.0872, 1.9500) -- cycle;
\fill[blue!15.0, opacity=0.5] (2.9128, -0.0872, 1.9500) -- (2.9128, -0.0343, 1.9563) -- (2.9100, -0.0370, 2.0063) -- (2.9100, -0.0900, 2.0000) -- cycle;
\fill[blue!15.0, opacity=0.5] (2.9100, -0.0900, 2.0000) -- (2.9100, -0.0370, 2.0063) -- (2.9073, -0.0396, 2.0563) -- (2.9073, -0.0927, 2.0500) -- cycle;
\fill[blue!15.0, opacity=0.5] (2.9073, -0.0927, 2.0500) -- (2.9073, -0.0396, 2.0563) -- (2.9047, -0.0421, 2.1063) -- (2.9047, -0.0953, 2.1000) -- cycle;
\fill[blue!15.0, opacity=0.5] (2.9047, -0.0953, 2.1000) -- (2.9047, -0.0421, 2.1063) -- (2.9022, -0.0445, 2.1563) -- (2.9022, -0.0978, 2.1500) -- cycle;
\fill[blue!15.0, opacity=0.5] (2.9022, -0.0978, 2.1500) -- (2.9022, -0.0445, 2.1563) -- (2.8999, -0.0468, 2.2063) -- (2.8999, -0.1001, 2.2000) -- cycle;
\fill[blue!15.0, opacity=0.5] (2.8999, -0.1001, 2.2000) -- (2.8999, -0.0468, 2.2063) -- (2.8976, -0.0490, 2.2563) -- (2.8976, -0.1024, 2.2500) -- cycle;
\fill[blue!15.0, opacity=0.5] (2.8976, -0.1024, 2.2500) -- (2.8976, -0.0490, 2.2563) -- (2.8954, -0.0511, 2.3063) -- (2.8954, -0.1046, 2.3000) -- cycle;
\fill[blue!15.0, opacity=0.5] (2.8954, -0.1046, 2.3000) -- (2.8954, -0.0511, 2.3063) -- (2.8934, -0.0531, 2.3563) -- (2.8934, -0.1066, 2.3500) -- cycle;
\fill[blue!15.0, opacity=0.5] (2.8934, -0.1066, 2.3500) -- (2.8934, -0.0531, 2.3563) -- (2.8915, -0.0549, 2.4063) -- (2.8915, -0.1085, 2.4000) -- cycle;
\fill[blue!15.0, opacity=0.5] (2.8915, -0.1085, 2.4000) -- (2.8915, -0.0549, 2.4063) -- (2.8897, -0.0566, 2.4563) -- (2.8897, -0.1103, 2.4500) -- cycle;
\fill[blue!15.0, opacity=0.5] (2.8897, -0.1103, 2.4500) -- (2.8897, -0.0566, 2.4563) -- (2.8880, -0.0582, 2.5063) -- (2.8880, -0.1120, 2.5000) -- cycle;
\fill[blue!15.0, opacity=0.5] (2.8880, -0.1120, 2.5000) -- (2.8880, -0.0582, 2.5063) -- (2.8865, -0.0597, 2.5563) -- (2.8865, -0.1135, 2.5500) -- cycle;
\fill[blue!15.0, opacity=0.5] (2.8865, -0.1135, 2.5500) -- (2.8865, -0.0597, 2.5563) -- (2.8852, -0.0610, 2.6063) -- (2.8852, -0.1148, 2.6000) -- cycle;
\fill[blue!15.0, opacity=0.5] (2.8852, -0.1148, 2.6000) -- (2.8852, -0.0610, 2.6063) -- (2.8840, -0.0621, 2.6563) -- (2.8840, -0.1160, 2.6500) -- cycle;
\fill[blue!15.0, opacity=0.5] (2.8840, -0.1160, 2.6500) -- (2.8840, -0.0621, 2.6563) -- (2.8829, -0.0632, 2.7063) -- (2.8829, -0.1171, 2.7000) -- cycle;
\fill[blue!15.0, opacity=0.5] (2.8829, -0.1171, 2.7000) -- (2.8829, -0.0632, 2.7063) -- (2.8820, -0.0640, 2.7563) -- (2.8820, -0.1180, 2.7500) -- cycle;
\fill[blue!15.0, opacity=0.5] (2.8820, -0.1180, 2.7500) -- (2.8820, -0.0640, 2.7563) -- (2.8813, -0.0647, 2.8063) -- (2.8813, -0.1187, 2.8000) -- cycle;
\fill[blue!15.0, opacity=0.5] (2.8813, -0.1187, 2.8000) -- (2.8813, -0.0647, 2.8063) -- (2.8807, -0.0653, 2.8563) -- (2.8807, -0.1193, 2.8500) -- cycle;
\fill[blue!15.0, opacity=0.5] (2.8807, -0.1193, 2.8500) -- (2.8807, -0.0653, 2.8563) -- (2.8803, -0.0657, 2.9063) -- (2.8803, -0.1197, 2.9000) -- cycle;
\fill[blue!15.0, opacity=0.5] (2.8803, -0.1197, 2.9000) -- (2.8803, -0.0657, 2.9063) -- (2.8801, -0.0659, 2.9563) -- (2.8801, -0.1199, 2.9500) -- cycle;
\fill[blue!15.0, opacity=0.5] (2.8801, -0.1199, 2.9500) -- (2.8801, -0.0659, 2.9563) -- (2.8800, -0.0660, 3.0063) -- (2.8800, -0.1200, 3.0000) -- cycle;
\fill[blue!15.0, opacity=0.5] (3.0000, 0.0500, 0.0063) -- (3.0000, 0.1000, 0.0125) -- (2.9999, 0.0999, 0.0625) -- (2.9999, 0.0499, 0.0563) -- cycle;
\fill[blue!15.0, opacity=0.5] (2.9999, 0.0499, 0.0563) -- (2.9999, 0.0999, 0.0625) -- (2.9997, 0.0997, 0.1125) -- (2.9997, 0.0497, 0.1063) -- cycle;
\fill[blue!15.0, opacity=0.5] (2.9997, 0.0497, 0.1063) -- (2.9997, 0.0997, 0.1125) -- (2.9993, 0.0993, 0.1625) -- (2.9993, 0.0493, 0.1563) -- cycle;
\fill[blue!15.0, opacity=0.5] (2.9993, 0.0493, 0.1563) -- (2.9993, 0.0993, 0.1625) -- (2.9987, 0.0988, 0.2125) -- (2.9987, 0.0487, 0.2063) -- cycle;
\fill[blue!15.0, opacity=0.5] (2.9987, 0.0487, 0.2063) -- (2.9987, 0.0988, 0.2125) -- (2.9980, 0.0981, 0.2625) -- (2.9980, 0.0480, 0.2563) -- cycle;
\fill[blue!15.0, opacity=0.5] (2.9980, 0.0480, 0.2563) -- (2.9980, 0.0981, 0.2625) -- (2.9971, 0.0973, 0.3125) -- (2.9971, 0.0472, 0.3063) -- cycle;
\fill[blue!15.0, opacity=0.5] (2.9971, 0.0472, 0.3063) -- (2.9971, 0.0973, 0.3125) -- (2.9960, 0.0963, 0.3625) -- (2.9960, 0.0461, 0.3563) -- cycle;
\fill[blue!15.0, opacity=0.5] (2.9960, 0.0461, 0.3563) -- (2.9960, 0.0963, 0.3625) -- (2.9948, 0.0952, 0.4125) -- (2.9948, 0.0450, 0.4063) -- cycle;
\fill[blue!15.0, opacity=0.5] (2.9948, 0.0450, 0.4063) -- (2.9948, 0.0952, 0.4125) -- (2.9935, 0.0939, 0.4625) -- (2.9935, 0.0437, 0.4563) -- cycle;
\fill[blue!15.0, opacity=0.5] (2.9935, 0.0437, 0.4563) -- (2.9935, 0.0939, 0.4625) -- (2.9920, 0.0925, 0.5125) -- (2.9920, 0.0422, 0.5063) -- cycle;
\fill[blue!15.0, opacity=0.5] (2.9920, 0.0422, 0.5063) -- (2.9920, 0.0925, 0.5125) -- (2.9903, 0.0910, 0.5625) -- (2.9903, 0.0406, 0.5563) -- cycle;
\fill[blue!15.0, opacity=0.5] (2.9903, 0.0406, 0.5563) -- (2.9903, 0.0910, 0.5625) -- (2.9885, 0.0893, 0.6125) -- (2.9885, 0.0389, 0.6063) -- cycle;
\fill[blue!15.0, opacity=0.5] (2.9885, 0.0389, 0.6063) -- (2.9885, 0.0893, 0.6125) -- (2.9866, 0.0875, 0.6625) -- (2.9866, 0.0371, 0.6563) -- cycle;
\fill[blue!15.0, opacity=0.5] (2.9866, 0.0371, 0.6563) -- (2.9866, 0.0875, 0.6625) -- (2.9846, 0.0856, 0.7125) -- (2.9846, 0.0351, 0.7063) -- cycle;
\fill[blue!15.0, opacity=0.5] (2.9846, 0.0351, 0.7063) -- (2.9846, 0.0856, 0.7125) -- (2.9824, 0.0836, 0.7625) -- (2.9824, 0.0330, 0.7563) -- cycle;
\fill[blue!15.0, opacity=0.5] (2.9824, 0.0330, 0.7563) -- (2.9824, 0.0836, 0.7625) -- (2.9801, 0.0815, 0.8125) -- (2.9801, 0.0308, 0.8063) -- cycle;
\fill[blue!15.0, opacity=0.5] (2.9801, 0.0308, 0.8063) -- (2.9801, 0.0815, 0.8125) -- (2.9778, 0.0792, 0.8625) -- (2.9778, 0.0285, 0.8563) -- cycle;
\fill[blue!15.0, opacity=0.5] (2.9778, 0.0285, 0.8563) -- (2.9778, 0.0792, 0.8625) -- (2.9753, 0.0769, 0.9125) -- (2.9753, 0.0261, 0.9063) -- cycle;
\fill[blue!15.0, opacity=0.5] (2.9753, 0.0261, 0.9063) -- (2.9753, 0.0769, 0.9125) -- (2.9727, 0.0745, 0.9625) -- (2.9727, 0.0236, 0.9563) -- cycle;
\fill[blue!15.0, opacity=0.5] (2.9727, 0.0236, 0.9563) -- (2.9727, 0.0745, 0.9625) -- (2.9700, 0.0720, 1.0125) -- (2.9700, 0.0210, 1.0063) -- cycle;
\fill[blue!15.0, opacity=0.5] (2.9700, 0.0210, 1.0063) -- (2.9700, 0.0720, 1.0125) -- (2.9672, 0.0694, 1.0625) -- (2.9672, 0.0183, 1.0563) -- cycle;
\fill[blue!15.0, opacity=0.5] (2.9672, 0.0183, 1.0563) -- (2.9672, 0.0694, 1.0625) -- (2.9644, 0.0668, 1.1125) -- (2.9644, 0.0156, 1.1063) -- cycle;
\fill[blue!15.0, opacity=0.5] (2.9644, 0.0156, 1.1063) -- (2.9644, 0.0668, 1.1125) -- (2.9615, 0.0641, 1.1625) -- (2.9615, 0.0128, 1.1563) -- cycle;
\fill[blue!15.0, opacity=0.5] (2.9615, 0.0128, 1.1563) -- (2.9615, 0.0641, 1.1625) -- (2.9585, 0.0613, 1.2125) -- (2.9585, 0.0099, 1.2063) -- cycle;
\fill[blue!15.0, opacity=0.5] (2.9585, 0.0099, 1.2063) -- (2.9585, 0.0613, 1.2125) -- (2.9555, 0.0585, 1.2625) -- (2.9555, 0.0070, 1.2563) -- cycle;
\fill[blue!15.0, opacity=0.5] (2.9555, 0.0070, 1.2563) -- (2.9555, 0.0585, 1.2625) -- (2.9525, 0.0556, 1.3125) -- (2.9525, 0.0041, 1.3063) -- cycle;
\fill[blue!15.0, opacity=0.5] (2.9525, 0.0041, 1.3063) -- (2.9525, 0.0556, 1.3125) -- (2.9494, 0.0528, 1.3625) -- (2.9494, 0.0011, 1.3563) -- cycle;
\fill[blue!15.0, opacity=0.5] (2.9494, 0.0011, 1.3563) -- (2.9494, 0.0528, 1.3625) -- (2.9463, 0.0499, 1.4125) -- (2.9463, -0.0019, 1.4063) -- cycle;
\fill[blue!15.0, opacity=0.5] (2.9463, -0.0019, 1.4063) -- (2.9463, 0.0499, 1.4125) -- (2.9431, 0.0469, 1.4625) -- (2.9431, -0.0050, 1.4563) -- cycle;
\fill[blue!15.0, opacity=0.5] (2.9431, -0.0050, 1.4563) -- (2.9431, 0.0469, 1.4625) -- (2.9400, 0.0440, 1.5125) -- (2.9400, -0.0080, 1.5063) -- cycle;
\fill[blue!15.0, opacity=0.5] (2.9400, -0.0080, 1.5063) -- (2.9400, 0.0440, 1.5125) -- (2.9369, 0.0411, 1.5625) -- (2.9369, -0.0110, 1.5563) -- cycle;
\fill[blue!15.0, opacity=0.5] (2.9369, -0.0110, 1.5563) -- (2.9369, 0.0411, 1.5625) -- (2.9337, 0.0381, 1.6125) -- (2.9337, -0.0141, 1.6063) -- cycle;
\fill[blue!15.0, opacity=0.5] (2.9337, -0.0141, 1.6063) -- (2.9337, 0.0381, 1.6125) -- (2.9306, 0.0352, 1.6625) -- (2.9306, -0.0171, 1.6563) -- cycle;
\fill[blue!15.0, opacity=0.5] (2.9306, -0.0171, 1.6563) -- (2.9306, 0.0352, 1.6625) -- (2.9275, 0.0324, 1.7125) -- (2.9275, -0.0201, 1.7063) -- cycle;
\fill[blue!15.0, opacity=0.5] (2.9275, -0.0201, 1.7063) -- (2.9275, 0.0324, 1.7125) -- (2.9245, 0.0295, 1.7625) -- (2.9245, -0.0230, 1.7563) -- cycle;
\fill[blue!15.0, opacity=0.5] (2.9245, -0.0230, 1.7563) -- (2.9245, 0.0295, 1.7625) -- (2.9215, 0.0267, 1.8125) -- (2.9215, -0.0259, 1.8063) -- cycle;
\fill[blue!15.0, opacity=0.5] (2.9215, -0.0259, 1.8063) -- (2.9215, 0.0267, 1.8125) -- (2.9185, 0.0239, 1.8625) -- (2.9185, -0.0288, 1.8563) -- cycle;
\fill[blue!15.0, opacity=0.5] (2.9185, -0.0288, 1.8563) -- (2.9185, 0.0239, 1.8625) -- (2.9156, 0.0212, 1.9125) -- (2.9156, -0.0316, 1.9063) -- cycle;
\fill[blue!15.0, opacity=0.5] (2.9156, -0.0316, 1.9063) -- (2.9156, 0.0212, 1.9125) -- (2.9128, 0.0186, 1.9625) -- (2.9128, -0.0343, 1.9563) -- cycle;
\fill[blue!15.0, opacity=0.5] (2.9128, -0.0343, 1.9563) -- (2.9128, 0.0186, 1.9625) -- (2.9100, 0.0160, 2.0125) -- (2.9100, -0.0370, 2.0063) -- cycle;
\fill[blue!15.0, opacity=0.5] (2.9100, -0.0370, 2.0063) -- (2.9100, 0.0160, 2.0125) -- (2.9073, 0.0135, 2.0625) -- (2.9073, -0.0396, 2.0563) -- cycle;
\fill[blue!15.0, opacity=0.5] (2.9073, -0.0396, 2.0563) -- (2.9073, 0.0135, 2.0625) -- (2.9047, 0.0111, 2.1125) -- (2.9047, -0.0421, 2.1063) -- cycle;
\fill[blue!15.0, opacity=0.5] (2.9047, -0.0421, 2.1063) -- (2.9047, 0.0111, 2.1125) -- (2.9022, 0.0088, 2.1625) -- (2.9022, -0.0445, 2.1563) -- cycle;
\fill[blue!15.0, opacity=0.5] (2.9022, -0.0445, 2.1563) -- (2.9022, 0.0088, 2.1625) -- (2.8999, 0.0065, 2.2125) -- (2.8999, -0.0468, 2.2063) -- cycle;
\fill[blue!15.0, opacity=0.5] (2.8999, -0.0468, 2.2063) -- (2.8999, 0.0065, 2.2125) -- (2.8976, 0.0044, 2.2625) -- (2.8976, -0.0490, 2.2563) -- cycle;
\fill[blue!15.0, opacity=0.5] (2.8976, -0.0490, 2.2563) -- (2.8976, 0.0044, 2.2625) -- (2.8954, 0.0024, 2.3125) -- (2.8954, -0.0511, 2.3063) -- cycle;
\fill[blue!15.0, opacity=0.5] (2.8954, -0.0511, 2.3063) -- (2.8954, 0.0024, 2.3125) -- (2.8934, 0.0005, 2.3625) -- (2.8934, -0.0531, 2.3563) -- cycle;
\fill[blue!15.0, opacity=0.5] (2.8934, -0.0531, 2.3563) -- (2.8934, 0.0005, 2.3625) -- (2.8915, -0.0013, 2.4125) -- (2.8915, -0.0549, 2.4063) -- cycle;
\fill[blue!15.0, opacity=0.5] (2.8915, -0.0549, 2.4063) -- (2.8915, -0.0013, 2.4125) -- (2.8897, -0.0030, 2.4625) -- (2.8897, -0.0566, 2.4563) -- cycle;
\fill[blue!15.0, opacity=0.5] (2.8897, -0.0566, 2.4563) -- (2.8897, -0.0030, 2.4625) -- (2.8880, -0.0045, 2.5125) -- (2.8880, -0.0582, 2.5063) -- cycle;
\fill[blue!15.0, opacity=0.5] (2.8880, -0.0582, 2.5063) -- (2.8880, -0.0045, 2.5125) -- (2.8865, -0.0059, 2.5625) -- (2.8865, -0.0597, 2.5563) -- cycle;
\fill[blue!15.0, opacity=0.5] (2.8865, -0.0597, 2.5563) -- (2.8865, -0.0059, 2.5625) -- (2.8852, -0.0072, 2.6125) -- (2.8852, -0.0610, 2.6063) -- cycle;
\fill[blue!15.0, opacity=0.5] (2.8852, -0.0610, 2.6063) -- (2.8852, -0.0072, 2.6125) -- (2.8840, -0.0083, 2.6625) -- (2.8840, -0.0621, 2.6563) -- cycle;
\fill[blue!15.0, opacity=0.5] (2.8840, -0.0621, 2.6563) -- (2.8840, -0.0083, 2.6625) -- (2.8829, -0.0093, 2.7125) -- (2.8829, -0.0632, 2.7063) -- cycle;
\fill[blue!15.0, opacity=0.5] (2.8829, -0.0632, 2.7063) -- (2.8829, -0.0093, 2.7125) -- (2.8820, -0.0101, 2.7625) -- (2.8820, -0.0640, 2.7563) -- cycle;
\fill[blue!15.0, opacity=0.5] (2.8820, -0.0640, 2.7563) -- (2.8820, -0.0101, 2.7625) -- (2.8813, -0.0108, 2.8125) -- (2.8813, -0.0647, 2.8063) -- cycle;
\fill[blue!15.0, opacity=0.5] (2.8813, -0.0647, 2.8063) -- (2.8813, -0.0108, 2.8125) -- (2.8807, -0.0113, 2.8625) -- (2.8807, -0.0653, 2.8563) -- cycle;
\fill[blue!15.0, opacity=0.5] (2.8807, -0.0653, 2.8563) -- (2.8807, -0.0113, 2.8625) -- (2.8803, -0.0117, 2.9125) -- (2.8803, -0.0657, 2.9063) -- cycle;
\fill[blue!15.0, opacity=0.5] (2.8803, -0.0657, 2.9063) -- (2.8803, -0.0117, 2.9125) -- (2.8801, -0.0119, 2.9625) -- (2.8801, -0.0659, 2.9563) -- cycle;
\fill[blue!15.1, opacity=0.5] (2.8801, -0.0659, 2.9563) -- (2.8801, -0.0119, 2.9625) -- (2.8800, -0.0120, 3.0125) -- (2.8800, -0.0660, 3.0063) -- cycle;
\fill[blue!15.0, opacity=0.5] (3.0000, 0.1000, 0.0125) -- (3.0000, 0.1500, 0.0188) -- (2.9999, 0.1499, 0.0688) -- (2.9999, 0.0999, 0.0625) -- cycle;
\fill[blue!15.0, opacity=0.5] (2.9999, 0.0999, 0.0625) -- (2.9999, 0.1499, 0.0688) -- (2.9997, 0.1497, 0.1188) -- (2.9997, 0.0997, 0.1125) -- cycle;
\fill[blue!15.0, opacity=0.5] (2.9997, 0.0997, 0.1125) -- (2.9997, 0.1497, 0.1188) -- (2.9993, 0.1493, 0.1688) -- (2.9993, 0.0993, 0.1625) -- cycle;
\fill[blue!15.0, opacity=0.5] (2.9993, 0.0993, 0.1625) -- (2.9993, 0.1493, 0.1688) -- (2.9987, 0.1488, 0.2188) -- (2.9987, 0.0988, 0.2125) -- cycle;
\fill[blue!15.0, opacity=0.5] (2.9987, 0.0988, 0.2125) -- (2.9987, 0.1488, 0.2188) -- (2.9980, 0.1482, 0.2688) -- (2.9980, 0.0981, 0.2625) -- cycle;
\fill[blue!15.0, opacity=0.5] (2.9980, 0.0981, 0.2625) -- (2.9980, 0.1482, 0.2688) -- (2.9971, 0.1474, 0.3188) -- (2.9971, 0.0973, 0.3125) -- cycle;
\fill[blue!15.0, opacity=0.5] (2.9971, 0.0973, 0.3125) -- (2.9971, 0.1474, 0.3188) -- (2.9960, 0.1464, 0.3688) -- (2.9960, 0.0963, 0.3625) -- cycle;
\fill[blue!15.0, opacity=0.5] (2.9960, 0.0963, 0.3625) -- (2.9960, 0.1464, 0.3688) -- (2.9948, 0.1453, 0.4188) -- (2.9948, 0.0952, 0.4125) -- cycle;
\fill[blue!15.0, opacity=0.5] (2.9948, 0.0952, 0.4125) -- (2.9948, 0.1453, 0.4188) -- (2.9935, 0.1441, 0.4688) -- (2.9935, 0.0939, 0.4625) -- cycle;
\fill[blue!15.0, opacity=0.5] (2.9935, 0.0939, 0.4625) -- (2.9935, 0.1441, 0.4688) -- (2.9920, 0.1428, 0.5188) -- (2.9920, 0.0925, 0.5125) -- cycle;
\fill[blue!15.0, opacity=0.5] (2.9920, 0.0925, 0.5125) -- (2.9920, 0.1428, 0.5188) -- (2.9903, 0.1413, 0.5688) -- (2.9903, 0.0910, 0.5625) -- cycle;
\fill[blue!15.0, opacity=0.5] (2.9903, 0.0910, 0.5625) -- (2.9903, 0.1413, 0.5688) -- (2.9885, 0.1397, 0.6188) -- (2.9885, 0.0893, 0.6125) -- cycle;
\fill[blue!15.0, opacity=0.5] (2.9885, 0.0893, 0.6125) -- (2.9885, 0.1397, 0.6188) -- (2.9866, 0.1380, 0.6688) -- (2.9866, 0.0875, 0.6625) -- cycle;
\fill[blue!15.0, opacity=0.5] (2.9866, 0.0875, 0.6625) -- (2.9866, 0.1380, 0.6688) -- (2.9846, 0.1361, 0.7188) -- (2.9846, 0.0856, 0.7125) -- cycle;
\fill[blue!15.0, opacity=0.5] (2.9846, 0.0856, 0.7125) -- (2.9846, 0.1361, 0.7188) -- (2.9824, 0.1342, 0.7688) -- (2.9824, 0.0836, 0.7625) -- cycle;
\fill[blue!15.0, opacity=0.5] (2.9824, 0.0836, 0.7625) -- (2.9824, 0.1342, 0.7688) -- (2.9801, 0.1321, 0.8188) -- (2.9801, 0.0815, 0.8125) -- cycle;
\fill[blue!15.0, opacity=0.5] (2.9801, 0.0815, 0.8125) -- (2.9801, 0.1321, 0.8188) -- (2.9778, 0.1300, 0.8688) -- (2.9778, 0.0792, 0.8625) -- cycle;
\fill[blue!15.0, opacity=0.5] (2.9778, 0.0792, 0.8625) -- (2.9778, 0.1300, 0.8688) -- (2.9753, 0.1277, 0.9188) -- (2.9753, 0.0769, 0.9125) -- cycle;
\fill[blue!15.0, opacity=0.5] (2.9753, 0.0769, 0.9125) -- (2.9753, 0.1277, 0.9188) -- (2.9727, 0.1254, 0.9688) -- (2.9727, 0.0745, 0.9625) -- cycle;
\fill[blue!15.0, opacity=0.5] (2.9727, 0.0745, 0.9625) -- (2.9727, 0.1254, 0.9688) -- (2.9700, 0.1230, 1.0188) -- (2.9700, 0.0720, 1.0125) -- cycle;
\fill[blue!15.0, opacity=0.5] (2.9700, 0.0720, 1.0125) -- (2.9700, 0.1230, 1.0188) -- (2.9672, 0.1205, 1.0688) -- (2.9672, 0.0694, 1.0625) -- cycle;
\fill[blue!15.0, opacity=0.5] (2.9672, 0.0694, 1.0625) -- (2.9672, 0.1205, 1.0688) -- (2.9644, 0.1180, 1.1188) -- (2.9644, 0.0668, 1.1125) -- cycle;
\fill[blue!15.0, opacity=0.5] (2.9644, 0.0668, 1.1125) -- (2.9644, 0.1180, 1.1188) -- (2.9615, 0.1154, 1.1688) -- (2.9615, 0.0641, 1.1625) -- cycle;
\fill[blue!15.0, opacity=0.5] (2.9615, 0.0641, 1.1625) -- (2.9615, 0.1154, 1.1688) -- (2.9585, 0.1127, 1.2188) -- (2.9585, 0.0613, 1.2125) -- cycle;
\fill[blue!15.0, opacity=0.5] (2.9585, 0.0613, 1.2125) -- (2.9585, 0.1127, 1.2188) -- (2.9555, 0.1100, 1.2688) -- (2.9555, 0.0585, 1.2625) -- cycle;
\fill[blue!15.0, opacity=0.5] (2.9555, 0.0585, 1.2625) -- (2.9555, 0.1100, 1.2688) -- (2.9525, 0.1072, 1.3188) -- (2.9525, 0.0556, 1.3125) -- cycle;
\fill[blue!15.0, opacity=0.5] (2.9525, 0.0556, 1.3125) -- (2.9525, 0.1072, 1.3188) -- (2.9494, 0.1044, 1.3688) -- (2.9494, 0.0528, 1.3625) -- cycle;
\fill[blue!15.0, opacity=0.5] (2.9494, 0.0528, 1.3625) -- (2.9494, 0.1044, 1.3688) -- (2.9463, 0.1016, 1.4188) -- (2.9463, 0.0499, 1.4125) -- cycle;
\fill[blue!15.0, opacity=0.5] (2.9463, 0.0499, 1.4125) -- (2.9463, 0.1016, 1.4188) -- (2.9431, 0.0988, 1.4688) -- (2.9431, 0.0469, 1.4625) -- cycle;
\fill[blue!15.0, opacity=0.5] (2.9431, 0.0469, 1.4625) -- (2.9431, 0.0988, 1.4688) -- (2.9400, 0.0960, 1.5188) -- (2.9400, 0.0440, 1.5125) -- cycle;
\fill[blue!15.0, opacity=0.5] (2.9400, 0.0440, 1.5125) -- (2.9400, 0.0960, 1.5188) -- (2.9369, 0.0932, 1.5688) -- (2.9369, 0.0411, 1.5625) -- cycle;
\fill[blue!15.0, opacity=0.5] (2.9369, 0.0411, 1.5625) -- (2.9369, 0.0932, 1.5688) -- (2.9337, 0.0904, 1.6188) -- (2.9337, 0.0381, 1.6125) -- cycle;
\fill[blue!15.0, opacity=0.5] (2.9337, 0.0381, 1.6125) -- (2.9337, 0.0904, 1.6188) -- (2.9306, 0.0876, 1.6688) -- (2.9306, 0.0352, 1.6625) -- cycle;
\fill[blue!15.0, opacity=0.5] (2.9306, 0.0352, 1.6625) -- (2.9306, 0.0876, 1.6688) -- (2.9275, 0.0848, 1.7188) -- (2.9275, 0.0324, 1.7125) -- cycle;
\fill[blue!15.0, opacity=0.5] (2.9275, 0.0324, 1.7125) -- (2.9275, 0.0848, 1.7188) -- (2.9245, 0.0820, 1.7688) -- (2.9245, 0.0295, 1.7625) -- cycle;
\fill[blue!15.0, opacity=0.5] (2.9245, 0.0295, 1.7625) -- (2.9245, 0.0820, 1.7688) -- (2.9215, 0.0793, 1.8188) -- (2.9215, 0.0267, 1.8125) -- cycle;
\fill[blue!15.0, opacity=0.5] (2.9215, 0.0267, 1.8125) -- (2.9215, 0.0793, 1.8188) -- (2.9185, 0.0766, 1.8688) -- (2.9185, 0.0239, 1.8625) -- cycle;
\fill[blue!15.0, opacity=0.5] (2.9185, 0.0239, 1.8625) -- (2.9185, 0.0766, 1.8688) -- (2.9156, 0.0740, 1.9188) -- (2.9156, 0.0212, 1.9125) -- cycle;
\fill[blue!15.0, opacity=0.5] (2.9156, 0.0212, 1.9125) -- (2.9156, 0.0740, 1.9188) -- (2.9128, 0.0715, 1.9688) -- (2.9128, 0.0186, 1.9625) -- cycle;
\fill[blue!15.0, opacity=0.5] (2.9128, 0.0186, 1.9625) -- (2.9128, 0.0715, 1.9688) -- (2.9100, 0.0690, 2.0188) -- (2.9100, 0.0160, 2.0125) -- cycle;
\fill[blue!15.0, opacity=0.5] (2.9100, 0.0160, 2.0125) -- (2.9100, 0.0690, 2.0188) -- (2.9073, 0.0666, 2.0688) -- (2.9073, 0.0135, 2.0625) -- cycle;
\fill[blue!15.0, opacity=0.5] (2.9073, 0.0135, 2.0625) -- (2.9073, 0.0666, 2.0688) -- (2.9047, 0.0643, 2.1188) -- (2.9047, 0.0111, 2.1125) -- cycle;
\fill[blue!15.0, opacity=0.5] (2.9047, 0.0111, 2.1125) -- (2.9047, 0.0643, 2.1188) -- (2.9022, 0.0620, 2.1688) -- (2.9022, 0.0088, 2.1625) -- cycle;
\fill[blue!15.0, opacity=0.5] (2.9022, 0.0088, 2.1625) -- (2.9022, 0.0620, 2.1688) -- (2.8999, 0.0599, 2.2188) -- (2.8999, 0.0065, 2.2125) -- cycle;
\fill[blue!15.0, opacity=0.5] (2.8999, 0.0065, 2.2125) -- (2.8999, 0.0599, 2.2188) -- (2.8976, 0.0578, 2.2688) -- (2.8976, 0.0044, 2.2625) -- cycle;
\fill[blue!15.0, opacity=0.5] (2.8976, 0.0044, 2.2625) -- (2.8976, 0.0578, 2.2688) -- (2.8954, 0.0559, 2.3188) -- (2.8954, 0.0024, 2.3125) -- cycle;
\fill[blue!15.0, opacity=0.5] (2.8954, 0.0024, 2.3125) -- (2.8954, 0.0559, 2.3188) -- (2.8934, 0.0540, 2.3688) -- (2.8934, 0.0005, 2.3625) -- cycle;
\fill[blue!15.0, opacity=0.5] (2.8934, 0.0005, 2.3625) -- (2.8934, 0.0540, 2.3688) -- (2.8915, 0.0523, 2.4188) -- (2.8915, -0.0013, 2.4125) -- cycle;
\fill[blue!15.0, opacity=0.5] (2.8915, -0.0013, 2.4125) -- (2.8915, 0.0523, 2.4188) -- (2.8897, 0.0507, 2.4688) -- (2.8897, -0.0030, 2.4625) -- cycle;
\fill[blue!15.0, opacity=0.5] (2.8897, -0.0030, 2.4625) -- (2.8897, 0.0507, 2.4688) -- (2.8880, 0.0492, 2.5188) -- (2.8880, -0.0045, 2.5125) -- cycle;
\fill[blue!15.0, opacity=0.5] (2.8880, -0.0045, 2.5125) -- (2.8880, 0.0492, 2.5188) -- (2.8865, 0.0479, 2.5688) -- (2.8865, -0.0059, 2.5625) -- cycle;
\fill[blue!15.0, opacity=0.5] (2.8865, -0.0059, 2.5625) -- (2.8865, 0.0479, 2.5688) -- (2.8852, 0.0467, 2.6188) -- (2.8852, -0.0072, 2.6125) -- cycle;
\fill[blue!15.0, opacity=0.5] (2.8852, -0.0072, 2.6125) -- (2.8852, 0.0467, 2.6188) -- (2.8840, 0.0456, 2.6688) -- (2.8840, -0.0083, 2.6625) -- cycle;
\fill[blue!15.0, opacity=0.5] (2.8840, -0.0083, 2.6625) -- (2.8840, 0.0456, 2.6688) -- (2.8829, 0.0446, 2.7188) -- (2.8829, -0.0093, 2.7125) -- cycle;
\fill[blue!15.1, opacity=0.5] (2.8829, -0.0093, 2.7125) -- (2.8829, 0.0446, 2.7188) -- (2.8820, 0.0438, 2.7688) -- (2.8820, -0.0101, 2.7625) -- cycle;
\fill[blue!15.1, opacity=0.5] (2.8820, -0.0101, 2.7625) -- (2.8820, 0.0438, 2.7688) -- (2.8813, 0.0432, 2.8188) -- (2.8813, -0.0108, 2.8125) -- cycle;
\fill[blue!15.1, opacity=0.5] (2.8813, -0.0108, 2.8125) -- (2.8813, 0.0432, 2.8188) -- (2.8807, 0.0427, 2.8688) -- (2.8807, -0.0113, 2.8625) -- cycle;
\fill[blue!15.1, opacity=0.5] (2.8807, -0.0113, 2.8625) -- (2.8807, 0.0427, 2.8688) -- (2.8803, 0.0423, 2.9188) -- (2.8803, -0.0117, 2.9125) -- cycle;
\fill[blue!15.2, opacity=0.5] (2.8803, -0.0117, 2.9125) -- (2.8803, 0.0423, 2.9188) -- (2.8801, 0.0421, 2.9688) -- (2.8801, -0.0119, 2.9625) -- cycle;
\fill[blue!15.2, opacity=0.5] (2.8801, -0.0119, 2.9625) -- (2.8801, 0.0421, 2.9688) -- (2.8800, 0.0420, 3.0188) -- (2.8800, -0.0120, 3.0125) -- cycle;
\fill[blue!15.0, opacity=0.5] (3.0000, 0.1500, 0.0188) -- (3.0000, 0.2000, 0.0249) -- (2.9999, 0.1999, 0.0749) -- (2.9999, 0.1499, 0.0688) -- cycle;
\fill[blue!15.0, opacity=0.5] (2.9999, 0.1499, 0.0688) -- (2.9999, 0.1999, 0.0749) -- (2.9997, 0.1997, 0.1249) -- (2.9997, 0.1497, 0.1188) -- cycle;
\fill[blue!15.0, opacity=0.5] (2.9997, 0.1497, 0.1188) -- (2.9997, 0.1997, 0.1249) -- (2.9993, 0.1994, 0.1749) -- (2.9993, 0.1493, 0.1688) -- cycle;
\fill[blue!15.0, opacity=0.5] (2.9993, 0.1493, 0.1688) -- (2.9993, 0.1994, 0.1749) -- (2.9987, 0.1989, 0.2249) -- (2.9987, 0.1488, 0.2188) -- cycle;
\fill[blue!15.0, opacity=0.5] (2.9987, 0.1488, 0.2188) -- (2.9987, 0.1989, 0.2249) -- (2.9980, 0.1982, 0.2749) -- (2.9980, 0.1482, 0.2688) -- cycle;
\fill[blue!15.0, opacity=0.5] (2.9980, 0.1482, 0.2688) -- (2.9980, 0.1982, 0.2749) -- (2.9971, 0.1975, 0.3249) -- (2.9971, 0.1474, 0.3188) -- cycle;
\fill[blue!15.0, opacity=0.5] (2.9971, 0.1474, 0.3188) -- (2.9971, 0.1975, 0.3249) -- (2.9960, 0.1965, 0.3749) -- (2.9960, 0.1464, 0.3688) -- cycle;
\fill[blue!15.0, opacity=0.5] (2.9960, 0.1464, 0.3688) -- (2.9960, 0.1965, 0.3749) -- (2.9948, 0.1955, 0.4249) -- (2.9948, 0.1453, 0.4188) -- cycle;
\fill[blue!15.0, opacity=0.5] (2.9948, 0.1453, 0.4188) -- (2.9948, 0.1955, 0.4249) -- (2.9935, 0.1943, 0.4749) -- (2.9935, 0.1441, 0.4688) -- cycle;
\fill[blue!15.0, opacity=0.5] (2.9935, 0.1441, 0.4688) -- (2.9935, 0.1943, 0.4749) -- (2.9920, 0.1930, 0.5249) -- (2.9920, 0.1428, 0.5188) -- cycle;
\fill[blue!15.0, opacity=0.5] (2.9920, 0.1428, 0.5188) -- (2.9920, 0.1930, 0.5249) -- (2.9903, 0.1916, 0.5749) -- (2.9903, 0.1413, 0.5688) -- cycle;
\fill[blue!15.0, opacity=0.5] (2.9903, 0.1413, 0.5688) -- (2.9903, 0.1916, 0.5749) -- (2.9885, 0.1901, 0.6249) -- (2.9885, 0.1397, 0.6188) -- cycle;
\fill[blue!15.0, opacity=0.5] (2.9885, 0.1397, 0.6188) -- (2.9885, 0.1901, 0.6249) -- (2.9866, 0.1884, 0.6749) -- (2.9866, 0.1380, 0.6688) -- cycle;
\fill[blue!15.0, opacity=0.5] (2.9866, 0.1380, 0.6688) -- (2.9866, 0.1884, 0.6749) -- (2.9846, 0.1866, 0.7249) -- (2.9846, 0.1361, 0.7188) -- cycle;
\fill[blue!15.0, opacity=0.5] (2.9846, 0.1361, 0.7188) -- (2.9846, 0.1866, 0.7249) -- (2.9824, 0.1848, 0.7749) -- (2.9824, 0.1342, 0.7688) -- cycle;
\fill[blue!15.0, opacity=0.5] (2.9824, 0.1342, 0.7688) -- (2.9824, 0.1848, 0.7749) -- (2.9801, 0.1828, 0.8249) -- (2.9801, 0.1321, 0.8188) -- cycle;
\fill[blue!15.0, opacity=0.5] (2.9801, 0.1321, 0.8188) -- (2.9801, 0.1828, 0.8249) -- (2.9778, 0.1807, 0.8749) -- (2.9778, 0.1300, 0.8688) -- cycle;
\fill[blue!15.0, opacity=0.5] (2.9778, 0.1300, 0.8688) -- (2.9778, 0.1807, 0.8749) -- (2.9753, 0.1786, 0.9249) -- (2.9753, 0.1277, 0.9188) -- cycle;
\fill[blue!15.0, opacity=0.5] (2.9753, 0.1277, 0.9188) -- (2.9753, 0.1786, 0.9249) -- (2.9727, 0.1763, 0.9749) -- (2.9727, 0.1254, 0.9688) -- cycle;
\fill[blue!15.0, opacity=0.5] (2.9727, 0.1254, 0.9688) -- (2.9727, 0.1763, 0.9749) -- (2.9700, 0.1740, 1.0249) -- (2.9700, 0.1230, 1.0188) -- cycle;
\fill[blue!15.0, opacity=0.5] (2.9700, 0.1230, 1.0188) -- (2.9700, 0.1740, 1.0249) -- (2.9672, 0.1716, 1.0749) -- (2.9672, 0.1205, 1.0688) -- cycle;
\fill[blue!15.0, opacity=0.5] (2.9672, 0.1205, 1.0688) -- (2.9672, 0.1716, 1.0749) -- (2.9644, 0.1692, 1.1249) -- (2.9644, 0.1180, 1.1188) -- cycle;
\fill[blue!15.0, opacity=0.5] (2.9644, 0.1180, 1.1188) -- (2.9644, 0.1692, 1.1249) -- (2.9615, 0.1666, 1.1749) -- (2.9615, 0.1154, 1.1688) -- cycle;
\fill[blue!15.0, opacity=0.5] (2.9615, 0.1154, 1.1688) -- (2.9615, 0.1666, 1.1749) -- (2.9585, 0.1641, 1.2249) -- (2.9585, 0.1127, 1.2188) -- cycle;
\fill[blue!15.0, opacity=0.5] (2.9585, 0.1127, 1.2188) -- (2.9585, 0.1641, 1.2249) -- (2.9555, 0.1615, 1.2749) -- (2.9555, 0.1100, 1.2688) -- cycle;
\fill[blue!15.0, opacity=0.5] (2.9555, 0.1100, 1.2688) -- (2.9555, 0.1615, 1.2749) -- (2.9525, 0.1588, 1.3249) -- (2.9525, 0.1072, 1.3188) -- cycle;
\fill[blue!15.0, opacity=0.5] (2.9525, 0.1072, 1.3188) -- (2.9525, 0.1588, 1.3249) -- (2.9494, 0.1561, 1.3749) -- (2.9494, 0.1044, 1.3688) -- cycle;
\fill[blue!15.0, opacity=0.5] (2.9494, 0.1044, 1.3688) -- (2.9494, 0.1561, 1.3749) -- (2.9463, 0.1534, 1.4249) -- (2.9463, 0.1016, 1.4188) -- cycle;
\fill[blue!15.0, opacity=0.5] (2.9463, 0.1016, 1.4188) -- (2.9463, 0.1534, 1.4249) -- (2.9431, 0.1507, 1.4749) -- (2.9431, 0.0988, 1.4688) -- cycle;
\fill[blue!15.0, opacity=0.5] (2.9431, 0.0988, 1.4688) -- (2.9431, 0.1507, 1.4749) -- (2.9400, 0.1480, 1.5249) -- (2.9400, 0.0960, 1.5188) -- cycle;
\fill[blue!15.0, opacity=0.5] (2.9400, 0.0960, 1.5188) -- (2.9400, 0.1480, 1.5249) -- (2.9369, 0.1453, 1.5749) -- (2.9369, 0.0932, 1.5688) -- cycle;
\fill[blue!15.0, opacity=0.5] (2.9369, 0.0932, 1.5688) -- (2.9369, 0.1453, 1.5749) -- (2.9337, 0.1426, 1.6249) -- (2.9337, 0.0904, 1.6188) -- cycle;
\fill[blue!15.0, opacity=0.5] (2.9337, 0.0904, 1.6188) -- (2.9337, 0.1426, 1.6249) -- (2.9306, 0.1399, 1.6749) -- (2.9306, 0.0876, 1.6688) -- cycle;
\fill[blue!15.0, opacity=0.5] (2.9306, 0.0876, 1.6688) -- (2.9306, 0.1399, 1.6749) -- (2.9275, 0.1372, 1.7249) -- (2.9275, 0.0848, 1.7188) -- cycle;
\fill[blue!15.0, opacity=0.5] (2.9275, 0.0848, 1.7188) -- (2.9275, 0.1372, 1.7249) -- (2.9245, 0.1345, 1.7749) -- (2.9245, 0.0820, 1.7688) -- cycle;
\fill[blue!15.0, opacity=0.5] (2.9245, 0.0820, 1.7688) -- (2.9245, 0.1345, 1.7749) -- (2.9215, 0.1319, 1.8249) -- (2.9215, 0.0793, 1.8188) -- cycle;
\fill[blue!15.0, opacity=0.5] (2.9215, 0.0793, 1.8188) -- (2.9215, 0.1319, 1.8249) -- (2.9185, 0.1294, 1.8749) -- (2.9185, 0.0766, 1.8688) -- cycle;
\fill[blue!15.0, opacity=0.5] (2.9185, 0.0766, 1.8688) -- (2.9185, 0.1294, 1.8749) -- (2.9156, 0.1268, 1.9249) -- (2.9156, 0.0740, 1.9188) -- cycle;
\fill[blue!15.0, opacity=0.5] (2.9156, 0.0740, 1.9188) -- (2.9156, 0.1268, 1.9249) -- (2.9128, 0.1244, 1.9749) -- (2.9128, 0.0715, 1.9688) -- cycle;
\fill[blue!15.0, opacity=0.5] (2.9128, 0.0715, 1.9688) -- (2.9128, 0.1244, 1.9749) -- (2.9100, 0.1220, 2.0249) -- (2.9100, 0.0690, 2.0188) -- cycle;
\fill[blue!15.0, opacity=0.5] (2.9100, 0.0690, 2.0188) -- (2.9100, 0.1220, 2.0249) -- (2.9073, 0.1197, 2.0749) -- (2.9073, 0.0666, 2.0688) -- cycle;
\fill[blue!15.0, opacity=0.5] (2.9073, 0.0666, 2.0688) -- (2.9073, 0.1197, 2.0749) -- (2.9047, 0.1174, 2.1249) -- (2.9047, 0.0643, 2.1188) -- cycle;
\fill[blue!15.0, opacity=0.5] (2.9047, 0.0643, 2.1188) -- (2.9047, 0.1174, 2.1249) -- (2.9022, 0.1153, 2.1749) -- (2.9022, 0.0620, 2.1688) -- cycle;
\fill[blue!15.0, opacity=0.5] (2.9022, 0.0620, 2.1688) -- (2.9022, 0.1153, 2.1749) -- (2.8999, 0.1132, 2.2249) -- (2.8999, 0.0599, 2.2188) -- cycle;
\fill[blue!15.0, opacity=0.5] (2.8999, 0.0599, 2.2188) -- (2.8999, 0.1132, 2.2249) -- (2.8976, 0.1112, 2.2749) -- (2.8976, 0.0578, 2.2688) -- cycle;
\fill[blue!15.0, opacity=0.5] (2.8976, 0.0578, 2.2688) -- (2.8976, 0.1112, 2.2749) -- (2.8954, 0.1094, 2.3249) -- (2.8954, 0.0559, 2.3188) -- cycle;
\fill[blue!15.0, opacity=0.5] (2.8954, 0.0559, 2.3188) -- (2.8954, 0.1094, 2.3249) -- (2.8934, 0.1076, 2.3749) -- (2.8934, 0.0540, 2.3688) -- cycle;
\fill[blue!15.1, opacity=0.5] (2.8934, 0.0540, 2.3688) -- (2.8934, 0.1076, 2.3749) -- (2.8915, 0.1059, 2.4249) -- (2.8915, 0.0523, 2.4188) -- cycle;
\fill[blue!15.1, opacity=0.5] (2.8915, 0.0523, 2.4188) -- (2.8915, 0.1059, 2.4249) -- (2.8897, 0.1044, 2.4749) -- (2.8897, 0.0507, 2.4688) -- cycle;
\fill[blue!15.1, opacity=0.5] (2.8897, 0.0507, 2.4688) -- (2.8897, 0.1044, 2.4749) -- (2.8880, 0.1030, 2.5249) -- (2.8880, 0.0492, 2.5188) -- cycle;
\fill[blue!15.2, opacity=0.5] (2.8880, 0.0492, 2.5188) -- (2.8880, 0.1030, 2.5249) -- (2.8865, 0.1017, 2.5749) -- (2.8865, 0.0479, 2.5688) -- cycle;
\fill[blue!15.2, opacity=0.5] (2.8865, 0.0479, 2.5688) -- (2.8865, 0.1017, 2.5749) -- (2.8852, 0.1005, 2.6249) -- (2.8852, 0.0467, 2.6188) -- cycle;
\fill[blue!15.3, opacity=0.5] (2.8852, 0.0467, 2.6188) -- (2.8852, 0.1005, 2.6249) -- (2.8840, 0.0995, 2.6749) -- (2.8840, 0.0456, 2.6688) -- cycle;
\fill[blue!15.3, opacity=0.5] (2.8840, 0.0456, 2.6688) -- (2.8840, 0.0995, 2.6749) -- (2.8829, 0.0985, 2.7249) -- (2.8829, 0.0446, 2.7188) -- cycle;
\fill[blue!15.4, opacity=0.5] (2.8829, 0.0446, 2.7188) -- (2.8829, 0.0985, 2.7249) -- (2.8820, 0.0978, 2.7749) -- (2.8820, 0.0438, 2.7688) -- cycle;
\fill[blue!15.5, opacity=0.5] (2.8820, 0.0438, 2.7688) -- (2.8820, 0.0978, 2.7749) -- (2.8813, 0.0971, 2.8249) -- (2.8813, 0.0432, 2.8188) -- cycle;
\fill[blue!15.7, opacity=0.5] (2.8813, 0.0432, 2.8188) -- (2.8813, 0.0971, 2.8249) -- (2.8807, 0.0966, 2.8749) -- (2.8807, 0.0427, 2.8688) -- cycle;
\fill[blue!15.8, opacity=0.5] (2.8807, 0.0427, 2.8688) -- (2.8807, 0.0966, 2.8749) -- (2.8803, 0.0963, 2.9249) -- (2.8803, 0.0423, 2.9188) -- cycle;
\fill[blue!16.0, opacity=0.5] (2.8803, 0.0423, 2.9188) -- (2.8803, 0.0963, 2.9249) -- (2.8801, 0.0961, 2.9749) -- (2.8801, 0.0421, 2.9688) -- cycle;
\fill[blue!16.1, opacity=0.5] (2.8801, 0.0421, 2.9688) -- (2.8801, 0.0961, 2.9749) -- (2.8800, 0.0960, 3.0249) -- (2.8800, 0.0420, 3.0188) -- cycle;
\fill[blue!15.0, opacity=0.5] (3.0000, 0.2000, 0.0249) -- (3.0000, 0.2500, 0.0311) -- (2.9999, 0.2499, 0.0811) -- (2.9999, 0.1999, 0.0749) -- cycle;
\fill[blue!15.0, opacity=0.5] (2.9999, 0.1999, 0.0749) -- (2.9999, 0.2499, 0.0811) -- (2.9997, 0.2497, 0.1311) -- (2.9997, 0.1997, 0.1249) -- cycle;
\fill[blue!15.0, opacity=0.5] (2.9997, 0.1997, 0.1249) -- (2.9997, 0.2497, 0.1311) -- (2.9993, 0.2494, 0.1811) -- (2.9993, 0.1994, 0.1749) -- cycle;
\fill[blue!15.0, opacity=0.5] (2.9993, 0.1994, 0.1749) -- (2.9993, 0.2494, 0.1811) -- (2.9987, 0.2489, 0.2311) -- (2.9987, 0.1989, 0.2249) -- cycle;
\fill[blue!15.0, opacity=0.5] (2.9987, 0.1989, 0.2249) -- (2.9987, 0.2489, 0.2311) -- (2.9980, 0.2483, 0.2811) -- (2.9980, 0.1982, 0.2749) -- cycle;
\fill[blue!15.0, opacity=0.5] (2.9980, 0.1982, 0.2749) -- (2.9980, 0.2483, 0.2811) -- (2.9971, 0.2476, 0.3311) -- (2.9971, 0.1975, 0.3249) -- cycle;
\fill[blue!15.0, opacity=0.5] (2.9971, 0.1975, 0.3249) -- (2.9971, 0.2476, 0.3311) -- (2.9960, 0.2467, 0.3811) -- (2.9960, 0.1965, 0.3749) -- cycle;
\fill[blue!15.0, opacity=0.5] (2.9960, 0.1965, 0.3749) -- (2.9960, 0.2467, 0.3811) -- (2.9948, 0.2457, 0.4311) -- (2.9948, 0.1955, 0.4249) -- cycle;
\fill[blue!15.0, opacity=0.5] (2.9948, 0.1955, 0.4249) -- (2.9948, 0.2457, 0.4311) -- (2.9935, 0.2446, 0.4811) -- (2.9935, 0.1943, 0.4749) -- cycle;
\fill[blue!15.0, opacity=0.5] (2.9935, 0.1943, 0.4749) -- (2.9935, 0.2446, 0.4811) -- (2.9920, 0.2433, 0.5311) -- (2.9920, 0.1930, 0.5249) -- cycle;
\fill[blue!15.0, opacity=0.5] (2.9920, 0.1930, 0.5249) -- (2.9920, 0.2433, 0.5311) -- (2.9903, 0.2419, 0.5811) -- (2.9903, 0.1916, 0.5749) -- cycle;
\fill[blue!15.0, opacity=0.5] (2.9903, 0.1916, 0.5749) -- (2.9903, 0.2419, 0.5811) -- (2.9885, 0.2405, 0.6311) -- (2.9885, 0.1901, 0.6249) -- cycle;
\fill[blue!15.0, opacity=0.5] (2.9885, 0.1901, 0.6249) -- (2.9885, 0.2405, 0.6311) -- (2.9866, 0.2389, 0.6811) -- (2.9866, 0.1884, 0.6749) -- cycle;
\fill[blue!15.0, opacity=0.5] (2.9866, 0.1884, 0.6749) -- (2.9866, 0.2389, 0.6811) -- (2.9846, 0.2372, 0.7311) -- (2.9846, 0.1866, 0.7249) -- cycle;
\fill[blue!15.0, opacity=0.5] (2.9846, 0.1866, 0.7249) -- (2.9846, 0.2372, 0.7311) -- (2.9824, 0.2354, 0.7811) -- (2.9824, 0.1848, 0.7749) -- cycle;
\fill[blue!15.0, opacity=0.5] (2.9824, 0.1848, 0.7749) -- (2.9824, 0.2354, 0.7811) -- (2.9801, 0.2335, 0.8311) -- (2.9801, 0.1828, 0.8249) -- cycle;
\fill[blue!15.0, opacity=0.5] (2.9801, 0.1828, 0.8249) -- (2.9801, 0.2335, 0.8311) -- (2.9778, 0.2315, 0.8811) -- (2.9778, 0.1807, 0.8749) -- cycle;
\fill[blue!15.0, opacity=0.5] (2.9778, 0.1807, 0.8749) -- (2.9778, 0.2315, 0.8811) -- (2.9753, 0.2294, 0.9311) -- (2.9753, 0.1786, 0.9249) -- cycle;
\fill[blue!15.0, opacity=0.5] (2.9753, 0.1786, 0.9249) -- (2.9753, 0.2294, 0.9311) -- (2.9727, 0.2272, 0.9811) -- (2.9727, 0.1763, 0.9749) -- cycle;
\fill[blue!15.0, opacity=0.5] (2.9727, 0.1763, 0.9749) -- (2.9727, 0.2272, 0.9811) -- (2.9700, 0.2250, 1.0311) -- (2.9700, 0.1740, 1.0249) -- cycle;
\fill[blue!15.0, opacity=0.5] (2.9700, 0.1740, 1.0249) -- (2.9700, 0.2250, 1.0311) -- (2.9672, 0.2227, 1.0811) -- (2.9672, 0.1716, 1.0749) -- cycle;
\fill[blue!15.0, opacity=0.5] (2.9672, 0.1716, 1.0749) -- (2.9672, 0.2227, 1.0811) -- (2.9644, 0.2203, 1.1311) -- (2.9644, 0.1692, 1.1249) -- cycle;
\fill[blue!15.0, opacity=0.5] (2.9644, 0.1692, 1.1249) -- (2.9644, 0.2203, 1.1311) -- (2.9615, 0.2179, 1.1811) -- (2.9615, 0.1666, 1.1749) -- cycle;
\fill[blue!15.0, opacity=0.5] (2.9615, 0.1666, 1.1749) -- (2.9615, 0.2179, 1.1811) -- (2.9585, 0.2155, 1.2311) -- (2.9585, 0.1641, 1.2249) -- cycle;
\fill[blue!15.0, opacity=0.5] (2.9585, 0.1641, 1.2249) -- (2.9585, 0.2155, 1.2311) -- (2.9555, 0.2129, 1.2811) -- (2.9555, 0.1615, 1.2749) -- cycle;
\fill[blue!15.0, opacity=0.5] (2.9555, 0.1615, 1.2749) -- (2.9555, 0.2129, 1.2811) -- (2.9525, 0.2104, 1.3311) -- (2.9525, 0.1588, 1.3249) -- cycle;
\fill[blue!15.0, opacity=0.5] (2.9525, 0.1588, 1.3249) -- (2.9525, 0.2104, 1.3311) -- (2.9494, 0.2078, 1.3811) -- (2.9494, 0.1561, 1.3749) -- cycle;
\fill[blue!15.0, opacity=0.5] (2.9494, 0.1561, 1.3749) -- (2.9494, 0.2078, 1.3811) -- (2.9463, 0.2052, 1.4311) -- (2.9463, 0.1534, 1.4249) -- cycle;
\fill[blue!15.0, opacity=0.5] (2.9463, 0.1534, 1.4249) -- (2.9463, 0.2052, 1.4311) -- (2.9431, 0.2026, 1.4811) -- (2.9431, 0.1507, 1.4749) -- cycle;
\fill[blue!15.0, opacity=0.5] (2.9431, 0.1507, 1.4749) -- (2.9431, 0.2026, 1.4811) -- (2.9400, 0.2000, 1.5311) -- (2.9400, 0.1480, 1.5249) -- cycle;
\fill[blue!15.0, opacity=0.5] (2.9400, 0.1480, 1.5249) -- (2.9400, 0.2000, 1.5311) -- (2.9369, 0.1974, 1.5811) -- (2.9369, 0.1453, 1.5749) -- cycle;
\fill[blue!15.0, opacity=0.5] (2.9369, 0.1453, 1.5749) -- (2.9369, 0.1974, 1.5811) -- (2.9337, 0.1948, 1.6311) -- (2.9337, 0.1426, 1.6249) -- cycle;
\fill[blue!15.0, opacity=0.5] (2.9337, 0.1426, 1.6249) -- (2.9337, 0.1948, 1.6311) -- (2.9306, 0.1922, 1.6811) -- (2.9306, 0.1399, 1.6749) -- cycle;
\fill[blue!15.0, opacity=0.5] (2.9306, 0.1399, 1.6749) -- (2.9306, 0.1922, 1.6811) -- (2.9275, 0.1896, 1.7311) -- (2.9275, 0.1372, 1.7249) -- cycle;
\fill[blue!15.0, opacity=0.5] (2.9275, 0.1372, 1.7249) -- (2.9275, 0.1896, 1.7311) -- (2.9245, 0.1871, 1.7811) -- (2.9245, 0.1345, 1.7749) -- cycle;
\fill[blue!15.0, opacity=0.5] (2.9245, 0.1345, 1.7749) -- (2.9245, 0.1871, 1.7811) -- (2.9215, 0.1845, 1.8311) -- (2.9215, 0.1319, 1.8249) -- cycle;
\fill[blue!15.0, opacity=0.5] (2.9215, 0.1319, 1.8249) -- (2.9215, 0.1845, 1.8311) -- (2.9185, 0.1821, 1.8811) -- (2.9185, 0.1294, 1.8749) -- cycle;
\fill[blue!15.0, opacity=0.5] (2.9185, 0.1294, 1.8749) -- (2.9185, 0.1821, 1.8811) -- (2.9156, 0.1797, 1.9311) -- (2.9156, 0.1268, 1.9249) -- cycle;
\fill[blue!15.0, opacity=0.5] (2.9156, 0.1268, 1.9249) -- (2.9156, 0.1797, 1.9311) -- (2.9128, 0.1773, 1.9811) -- (2.9128, 0.1244, 1.9749) -- cycle;
\fill[blue!15.1, opacity=0.5] (2.9128, 0.1244, 1.9749) -- (2.9128, 0.1773, 1.9811) -- (2.9100, 0.1750, 2.0311) -- (2.9100, 0.1220, 2.0249) -- cycle;
\fill[blue!15.1, opacity=0.5] (2.9100, 0.1220, 2.0249) -- (2.9100, 0.1750, 2.0311) -- (2.9073, 0.1728, 2.0811) -- (2.9073, 0.1197, 2.0749) -- cycle;
\fill[blue!15.1, opacity=0.5] (2.9073, 0.1197, 2.0749) -- (2.9073, 0.1728, 2.0811) -- (2.9047, 0.1706, 2.1311) -- (2.9047, 0.1174, 2.1249) -- cycle;
\fill[blue!15.2, opacity=0.5] (2.9047, 0.1174, 2.1249) -- (2.9047, 0.1706, 2.1311) -- (2.9022, 0.1685, 2.1811) -- (2.9022, 0.1153, 2.1749) -- cycle;
\fill[blue!15.2, opacity=0.5] (2.9022, 0.1153, 2.1749) -- (2.9022, 0.1685, 2.1811) -- (2.8999, 0.1665, 2.2311) -- (2.8999, 0.1132, 2.2249) -- cycle;
\fill[blue!15.3, opacity=0.5] (2.8999, 0.1132, 2.2249) -- (2.8999, 0.1665, 2.2311) -- (2.8976, 0.1646, 2.2811) -- (2.8976, 0.1112, 2.2749) -- cycle;
\fill[blue!15.4, opacity=0.5] (2.8976, 0.1112, 2.2749) -- (2.8976, 0.1646, 2.2811) -- (2.8954, 0.1628, 2.3311) -- (2.8954, 0.1094, 2.3249) -- cycle;
\fill[blue!15.5, opacity=0.5] (2.8954, 0.1094, 2.3249) -- (2.8954, 0.1628, 2.3311) -- (2.8934, 0.1611, 2.3811) -- (2.8934, 0.1076, 2.3749) -- cycle;
\fill[blue!15.7, opacity=0.5] (2.8934, 0.1076, 2.3749) -- (2.8934, 0.1611, 2.3811) -- (2.8915, 0.1595, 2.4311) -- (2.8915, 0.1059, 2.4249) -- cycle;
\fill[blue!15.9, opacity=0.5] (2.8915, 0.1059, 2.4249) -- (2.8915, 0.1595, 2.4311) -- (2.8897, 0.1581, 2.4811) -- (2.8897, 0.1044, 2.4749) -- cycle;
\fill[blue!16.1, opacity=0.5] (2.8897, 0.1044, 2.4749) -- (2.8897, 0.1581, 2.4811) -- (2.8880, 0.1567, 2.5311) -- (2.8880, 0.1030, 2.5249) -- cycle;
\fill[blue!16.3, opacity=0.5] (2.8880, 0.1030, 2.5249) -- (2.8880, 0.1567, 2.5311) -- (2.8865, 0.1554, 2.5811) -- (2.8865, 0.1017, 2.5749) -- cycle;
\fill[blue!16.6, opacity=0.5] (2.8865, 0.1017, 2.5749) -- (2.8865, 0.1554, 2.5811) -- (2.8852, 0.1543, 2.6311) -- (2.8852, 0.1005, 2.6249) -- cycle;
\fill[blue!16.9, opacity=0.5] (2.8852, 0.1005, 2.6249) -- (2.8852, 0.1543, 2.6311) -- (2.8840, 0.1533, 2.6811) -- (2.8840, 0.0995, 2.6749) -- cycle;
\fill[blue!17.2, opacity=0.5] (2.8840, 0.0995, 2.6749) -- (2.8840, 0.1533, 2.6811) -- (2.8829, 0.1524, 2.7311) -- (2.8829, 0.0985, 2.7249) -- cycle;
\fill[blue!17.6, opacity=0.5] (2.8829, 0.0985, 2.7249) -- (2.8829, 0.1524, 2.7311) -- (2.8820, 0.1517, 2.7811) -- (2.8820, 0.0978, 2.7749) -- cycle;
\fill[blue!18.0, opacity=0.5] (2.8820, 0.0978, 2.7749) -- (2.8820, 0.1517, 2.7811) -- (2.8813, 0.1511, 2.8311) -- (2.8813, 0.0971, 2.8249) -- cycle;
\fill[blue!18.5, opacity=0.5] (2.8813, 0.0971, 2.8249) -- (2.8813, 0.1511, 2.8311) -- (2.8807, 0.1506, 2.8811) -- (2.8807, 0.0966, 2.8749) -- cycle;
\fill[blue!19.0, opacity=0.5] (2.8807, 0.0966, 2.8749) -- (2.8807, 0.1506, 2.8811) -- (2.8803, 0.1503, 2.9311) -- (2.8803, 0.0963, 2.9249) -- cycle;
\fill[blue!19.5, opacity=0.5] (2.8803, 0.0963, 2.9249) -- (2.8803, 0.1503, 2.9311) -- (2.8801, 0.1501, 2.9811) -- (2.8801, 0.0961, 2.9749) -- cycle;
\fill[blue!20.0, opacity=0.5] (2.8801, 0.0961, 2.9749) -- (2.8801, 0.1501, 2.9811) -- (2.8800, 0.1500, 3.0311) -- (2.8800, 0.0960, 3.0249) -- cycle;
\fill[blue!15.0, opacity=0.5] (3.0000, 0.2500, 0.0311) -- (3.0000, 0.3000, 0.0371) -- (2.9999, 0.2999, 0.0871) -- (2.9999, 0.2499, 0.0811) -- cycle;
\fill[blue!15.0, opacity=0.5] (2.9999, 0.2499, 0.0811) -- (2.9999, 0.2999, 0.0871) -- (2.9997, 0.2997, 0.1371) -- (2.9997, 0.2497, 0.1311) -- cycle;
\fill[blue!15.0, opacity=0.5] (2.9997, 0.2497, 0.1311) -- (2.9997, 0.2997, 0.1371) -- (2.9993, 0.2994, 0.1871) -- (2.9993, 0.2494, 0.1811) -- cycle;
\fill[blue!15.0, opacity=0.5] (2.9993, 0.2494, 0.1811) -- (2.9993, 0.2994, 0.1871) -- (2.9987, 0.2990, 0.2371) -- (2.9987, 0.2489, 0.2311) -- cycle;
\fill[blue!15.0, opacity=0.5] (2.9987, 0.2489, 0.2311) -- (2.9987, 0.2990, 0.2371) -- (2.9980, 0.2984, 0.2871) -- (2.9980, 0.2483, 0.2811) -- cycle;
\fill[blue!15.0, opacity=0.5] (2.9980, 0.2483, 0.2811) -- (2.9980, 0.2984, 0.2871) -- (2.9971, 0.2977, 0.3371) -- (2.9971, 0.2476, 0.3311) -- cycle;
\fill[blue!15.0, opacity=0.5] (2.9971, 0.2476, 0.3311) -- (2.9971, 0.2977, 0.3371) -- (2.9960, 0.2968, 0.3871) -- (2.9960, 0.2467, 0.3811) -- cycle;
\fill[blue!15.0, opacity=0.5] (2.9960, 0.2467, 0.3811) -- (2.9960, 0.2968, 0.3871) -- (2.9948, 0.2959, 0.4371) -- (2.9948, 0.2457, 0.4311) -- cycle;
\fill[blue!15.0, opacity=0.5] (2.9948, 0.2457, 0.4311) -- (2.9948, 0.2959, 0.4371) -- (2.9935, 0.2948, 0.4871) -- (2.9935, 0.2446, 0.4811) -- cycle;
\fill[blue!15.0, opacity=0.5] (2.9935, 0.2446, 0.4811) -- (2.9935, 0.2948, 0.4871) -- (2.9920, 0.2936, 0.5371) -- (2.9920, 0.2433, 0.5311) -- cycle;
\fill[blue!15.0, opacity=0.5] (2.9920, 0.2433, 0.5311) -- (2.9920, 0.2936, 0.5371) -- (2.9903, 0.2923, 0.5871) -- (2.9903, 0.2419, 0.5811) -- cycle;
\fill[blue!15.0, opacity=0.5] (2.9903, 0.2419, 0.5811) -- (2.9903, 0.2923, 0.5871) -- (2.9885, 0.2908, 0.6371) -- (2.9885, 0.2405, 0.6311) -- cycle;
\fill[blue!15.0, opacity=0.5] (2.9885, 0.2405, 0.6311) -- (2.9885, 0.2908, 0.6371) -- (2.9866, 0.2893, 0.6871) -- (2.9866, 0.2389, 0.6811) -- cycle;
\fill[blue!15.0, opacity=0.5] (2.9866, 0.2389, 0.6811) -- (2.9866, 0.2893, 0.6871) -- (2.9846, 0.2877, 0.7371) -- (2.9846, 0.2372, 0.7311) -- cycle;
\fill[blue!15.0, opacity=0.5] (2.9846, 0.2372, 0.7311) -- (2.9846, 0.2877, 0.7371) -- (2.9824, 0.2859, 0.7871) -- (2.9824, 0.2354, 0.7811) -- cycle;
\fill[blue!15.0, opacity=0.5] (2.9824, 0.2354, 0.7811) -- (2.9824, 0.2859, 0.7871) -- (2.9801, 0.2841, 0.8371) -- (2.9801, 0.2335, 0.8311) -- cycle;
\fill[blue!15.0, opacity=0.5] (2.9801, 0.2335, 0.8311) -- (2.9801, 0.2841, 0.8371) -- (2.9778, 0.2822, 0.8871) -- (2.9778, 0.2315, 0.8811) -- cycle;
\fill[blue!15.0, opacity=0.5] (2.9778, 0.2315, 0.8811) -- (2.9778, 0.2822, 0.8871) -- (2.9753, 0.2802, 0.9371) -- (2.9753, 0.2294, 0.9311) -- cycle;
\fill[blue!15.0, opacity=0.5] (2.9753, 0.2294, 0.9311) -- (2.9753, 0.2802, 0.9371) -- (2.9727, 0.2781, 0.9871) -- (2.9727, 0.2272, 0.9811) -- cycle;
\fill[blue!15.0, opacity=0.5] (2.9727, 0.2272, 0.9811) -- (2.9727, 0.2781, 0.9871) -- (2.9700, 0.2760, 1.0371) -- (2.9700, 0.2250, 1.0311) -- cycle;
\fill[blue!15.0, opacity=0.5] (2.9700, 0.2250, 1.0311) -- (2.9700, 0.2760, 1.0371) -- (2.9672, 0.2738, 1.0871) -- (2.9672, 0.2227, 1.0811) -- cycle;
\fill[blue!15.0, opacity=0.5] (2.9672, 0.2227, 1.0811) -- (2.9672, 0.2738, 1.0871) -- (2.9644, 0.2715, 1.1371) -- (2.9644, 0.2203, 1.1311) -- cycle;
\fill[blue!15.0, opacity=0.5] (2.9644, 0.2203, 1.1311) -- (2.9644, 0.2715, 1.1371) -- (2.9615, 0.2692, 1.1871) -- (2.9615, 0.2179, 1.1811) -- cycle;
\fill[blue!15.0, opacity=0.5] (2.9615, 0.2179, 1.1811) -- (2.9615, 0.2692, 1.1871) -- (2.9585, 0.2668, 1.2371) -- (2.9585, 0.2155, 1.2311) -- cycle;
\fill[blue!15.0, opacity=0.5] (2.9585, 0.2155, 1.2311) -- (2.9585, 0.2668, 1.2371) -- (2.9555, 0.2644, 1.2871) -- (2.9555, 0.2129, 1.2811) -- cycle;
\fill[blue!15.0, opacity=0.5] (2.9555, 0.2129, 1.2811) -- (2.9555, 0.2644, 1.2871) -- (2.9525, 0.2620, 1.3371) -- (2.9525, 0.2104, 1.3311) -- cycle;
\fill[blue!15.0, opacity=0.5] (2.9525, 0.2104, 1.3311) -- (2.9525, 0.2620, 1.3371) -- (2.9494, 0.2595, 1.3871) -- (2.9494, 0.2078, 1.3811) -- cycle;
\fill[blue!15.0, opacity=0.5] (2.9494, 0.2078, 1.3811) -- (2.9494, 0.2595, 1.3871) -- (2.9463, 0.2570, 1.4371) -- (2.9463, 0.2052, 1.4311) -- cycle;
\fill[blue!15.0, opacity=0.5] (2.9463, 0.2052, 1.4311) -- (2.9463, 0.2570, 1.4371) -- (2.9431, 0.2545, 1.4871) -- (2.9431, 0.2026, 1.4811) -- cycle;
\fill[blue!15.0, opacity=0.5] (2.9431, 0.2026, 1.4811) -- (2.9431, 0.2545, 1.4871) -- (2.9400, 0.2520, 1.5371) -- (2.9400, 0.2000, 1.5311) -- cycle;
\fill[blue!15.0, opacity=0.5] (2.9400, 0.2000, 1.5311) -- (2.9400, 0.2520, 1.5371) -- (2.9369, 0.2495, 1.5871) -- (2.9369, 0.1974, 1.5811) -- cycle;
\fill[blue!15.0, opacity=0.5] (2.9369, 0.1974, 1.5811) -- (2.9369, 0.2495, 1.5871) -- (2.9337, 0.2470, 1.6371) -- (2.9337, 0.1948, 1.6311) -- cycle;
\fill[blue!15.1, opacity=0.5] (2.9337, 0.1948, 1.6311) -- (2.9337, 0.2470, 1.6371) -- (2.9306, 0.2445, 1.6871) -- (2.9306, 0.1922, 1.6811) -- cycle;
\fill[blue!15.1, opacity=0.5] (2.9306, 0.1922, 1.6811) -- (2.9306, 0.2445, 1.6871) -- (2.9275, 0.2420, 1.7371) -- (2.9275, 0.1896, 1.7311) -- cycle;
\fill[blue!15.1, opacity=0.5] (2.9275, 0.1896, 1.7311) -- (2.9275, 0.2420, 1.7371) -- (2.9245, 0.2396, 1.7871) -- (2.9245, 0.1871, 1.7811) -- cycle;
\fill[blue!15.2, opacity=0.5] (2.9245, 0.1871, 1.7811) -- (2.9245, 0.2396, 1.7871) -- (2.9215, 0.2372, 1.8371) -- (2.9215, 0.1845, 1.8311) -- cycle;
\fill[blue!15.3, opacity=0.5] (2.9215, 0.1845, 1.8311) -- (2.9215, 0.2372, 1.8371) -- (2.9185, 0.2348, 1.8871) -- (2.9185, 0.1821, 1.8811) -- cycle;
\fill[blue!15.4, opacity=0.5] (2.9185, 0.1821, 1.8811) -- (2.9185, 0.2348, 1.8871) -- (2.9156, 0.2325, 1.9371) -- (2.9156, 0.1797, 1.9311) -- cycle;
\fill[blue!15.5, opacity=0.5] (2.9156, 0.1797, 1.9311) -- (2.9156, 0.2325, 1.9371) -- (2.9128, 0.2302, 1.9871) -- (2.9128, 0.1773, 1.9811) -- cycle;
\fill[blue!15.7, opacity=0.5] (2.9128, 0.1773, 1.9811) -- (2.9128, 0.2302, 1.9871) -- (2.9100, 0.2280, 2.0371) -- (2.9100, 0.1750, 2.0311) -- cycle;
\fill[blue!15.9, opacity=0.5] (2.9100, 0.1750, 2.0311) -- (2.9100, 0.2280, 2.0371) -- (2.9073, 0.2259, 2.0871) -- (2.9073, 0.1728, 2.0811) -- cycle;
\fill[blue!16.2, opacity=0.5] (2.9073, 0.1728, 2.0811) -- (2.9073, 0.2259, 2.0871) -- (2.9047, 0.2238, 2.1371) -- (2.9047, 0.1706, 2.1311) -- cycle;
\fill[blue!16.5, opacity=0.5] (2.9047, 0.1706, 2.1311) -- (2.9047, 0.2238, 2.1371) -- (2.9022, 0.2218, 2.1871) -- (2.9022, 0.1685, 2.1811) -- cycle;
\fill[blue!16.8, opacity=0.5] (2.9022, 0.1685, 2.1811) -- (2.9022, 0.2218, 2.1871) -- (2.8999, 0.2199, 2.2371) -- (2.8999, 0.1665, 2.2311) -- cycle;
\fill[blue!17.3, opacity=0.5] (2.8999, 0.1665, 2.2311) -- (2.8999, 0.2199, 2.2371) -- (2.8976, 0.2181, 2.2871) -- (2.8976, 0.1646, 2.2811) -- cycle;
\fill[blue!17.7, opacity=0.5] (2.8976, 0.1646, 2.2811) -- (2.8976, 0.2181, 2.2871) -- (2.8954, 0.2163, 2.3371) -- (2.8954, 0.1628, 2.3311) -- cycle;
\fill[blue!18.3, opacity=0.5] (2.8954, 0.1628, 2.3311) -- (2.8954, 0.2163, 2.3371) -- (2.8934, 0.2147, 2.3871) -- (2.8934, 0.1611, 2.3811) -- cycle;
\fill[blue!18.9, opacity=0.5] (2.8934, 0.1611, 2.3811) -- (2.8934, 0.2147, 2.3871) -- (2.8915, 0.2132, 2.4371) -- (2.8915, 0.1595, 2.4311) -- cycle;
\fill[blue!19.5, opacity=0.5] (2.8915, 0.1595, 2.4311) -- (2.8915, 0.2132, 2.4371) -- (2.8897, 0.2117, 2.4871) -- (2.8897, 0.1581, 2.4811) -- cycle;
\fill[blue!20.2, opacity=0.5] (2.8897, 0.1581, 2.4811) -- (2.8897, 0.2117, 2.4871) -- (2.8880, 0.2104, 2.5371) -- (2.8880, 0.1567, 2.5311) -- cycle;
\fill[blue!21.0, opacity=0.5] (2.8880, 0.1567, 2.5311) -- (2.8880, 0.2104, 2.5371) -- (2.8865, 0.2092, 2.5871) -- (2.8865, 0.1554, 2.5811) -- cycle;
\fill[blue!21.8, opacity=0.5] (2.8865, 0.1554, 2.5811) -- (2.8865, 0.2092, 2.5871) -- (2.8852, 0.2081, 2.6371) -- (2.8852, 0.1543, 2.6311) -- cycle;
\fill[blue!22.6, opacity=0.5] (2.8852, 0.1543, 2.6311) -- (2.8852, 0.2081, 2.6371) -- (2.8840, 0.2072, 2.6871) -- (2.8840, 0.1533, 2.6811) -- cycle;
\fill[blue!23.5, opacity=0.5] (2.8840, 0.1533, 2.6811) -- (2.8840, 0.2072, 2.6871) -- (2.8829, 0.2063, 2.7371) -- (2.8829, 0.1524, 2.7311) -- cycle;
\fill[blue!24.5, opacity=0.5] (2.8829, 0.1524, 2.7311) -- (2.8829, 0.2063, 2.7371) -- (2.8820, 0.2056, 2.7871) -- (2.8820, 0.1517, 2.7811) -- cycle;
\fill[blue!25.4, opacity=0.5] (2.8820, 0.1517, 2.7811) -- (2.8820, 0.2056, 2.7871) -- (2.8813, 0.2050, 2.8371) -- (2.8813, 0.1511, 2.8311) -- cycle;
\fill[blue!26.4, opacity=0.5] (2.8813, 0.1511, 2.8311) -- (2.8813, 0.2050, 2.8371) -- (2.8807, 0.2046, 2.8871) -- (2.8807, 0.1506, 2.8811) -- cycle;
\fill[blue!27.3, opacity=0.5] (2.8807, 0.1506, 2.8811) -- (2.8807, 0.2046, 2.8871) -- (2.8803, 0.2043, 2.9371) -- (2.8803, 0.1503, 2.9311) -- cycle;
\fill[blue!28.3, opacity=0.5] (2.8803, 0.1503, 2.9311) -- (2.8803, 0.2043, 2.9371) -- (2.8801, 0.2041, 2.9871) -- (2.8801, 0.1501, 2.9811) -- cycle;
\fill[blue!29.2, opacity=0.5] (2.8801, 0.1501, 2.9811) -- (2.8801, 0.2041, 2.9871) -- (2.8800, 0.2040, 3.0371) -- (2.8800, 0.1500, 3.0311) -- cycle;
\fill[blue!15.0, opacity=0.5] (3.0000, 0.3000, 0.0371) -- (3.0000, 0.3500, 0.0430) -- (2.9999, 0.3499, 0.0930) -- (2.9999, 0.2999, 0.0871) -- cycle;
\fill[blue!15.0, opacity=0.5] (2.9999, 0.2999, 0.0871) -- (2.9999, 0.3499, 0.0930) -- (2.9997, 0.3497, 0.1430) -- (2.9997, 0.2997, 0.1371) -- cycle;
\fill[blue!15.0, opacity=0.5] (2.9997, 0.2997, 0.1371) -- (2.9997, 0.3497, 0.1430) -- (2.9993, 0.3494, 0.1930) -- (2.9993, 0.2994, 0.1871) -- cycle;
\fill[blue!15.0, opacity=0.5] (2.9993, 0.2994, 0.1871) -- (2.9993, 0.3494, 0.1930) -- (2.9987, 0.3490, 0.2430) -- (2.9987, 0.2990, 0.2371) -- cycle;
\fill[blue!15.0, opacity=0.5] (2.9987, 0.2990, 0.2371) -- (2.9987, 0.3490, 0.2430) -- (2.9980, 0.3484, 0.2930) -- (2.9980, 0.2984, 0.2871) -- cycle;
\fill[blue!15.0, opacity=0.5] (2.9980, 0.2984, 0.2871) -- (2.9980, 0.3484, 0.2930) -- (2.9971, 0.3477, 0.3430) -- (2.9971, 0.2977, 0.3371) -- cycle;
\fill[blue!15.0, opacity=0.5] (2.9971, 0.2977, 0.3371) -- (2.9971, 0.3477, 0.3430) -- (2.9960, 0.3469, 0.3930) -- (2.9960, 0.2968, 0.3871) -- cycle;
\fill[blue!15.0, opacity=0.5] (2.9960, 0.2968, 0.3871) -- (2.9960, 0.3469, 0.3930) -- (2.9948, 0.3460, 0.4430) -- (2.9948, 0.2959, 0.4371) -- cycle;
\fill[blue!15.0, opacity=0.5] (2.9948, 0.2959, 0.4371) -- (2.9948, 0.3460, 0.4430) -- (2.9935, 0.3450, 0.4930) -- (2.9935, 0.2948, 0.4871) -- cycle;
\fill[blue!15.0, opacity=0.5] (2.9935, 0.2948, 0.4871) -- (2.9935, 0.3450, 0.4930) -- (2.9920, 0.3438, 0.5430) -- (2.9920, 0.2936, 0.5371) -- cycle;
\fill[blue!15.0, opacity=0.5] (2.9920, 0.2936, 0.5371) -- (2.9920, 0.3438, 0.5430) -- (2.9903, 0.3426, 0.5930) -- (2.9903, 0.2923, 0.5871) -- cycle;
\fill[blue!15.0, opacity=0.5] (2.9903, 0.2923, 0.5871) -- (2.9903, 0.3426, 0.5930) -- (2.9885, 0.3412, 0.6430) -- (2.9885, 0.2908, 0.6371) -- cycle;
\fill[blue!15.0, opacity=0.5] (2.9885, 0.2908, 0.6371) -- (2.9885, 0.3412, 0.6430) -- (2.9866, 0.3397, 0.6930) -- (2.9866, 0.2893, 0.6871) -- cycle;
\fill[blue!15.0, opacity=0.5] (2.9866, 0.2893, 0.6871) -- (2.9866, 0.3397, 0.6930) -- (2.9846, 0.3382, 0.7430) -- (2.9846, 0.2877, 0.7371) -- cycle;
\fill[blue!15.0, opacity=0.5] (2.9846, 0.2877, 0.7371) -- (2.9846, 0.3382, 0.7430) -- (2.9824, 0.3365, 0.7930) -- (2.9824, 0.2859, 0.7871) -- cycle;
\fill[blue!15.0, opacity=0.5] (2.9824, 0.2859, 0.7871) -- (2.9824, 0.3365, 0.7930) -- (2.9801, 0.3348, 0.8430) -- (2.9801, 0.2841, 0.8371) -- cycle;
\fill[blue!15.0, opacity=0.5] (2.9801, 0.2841, 0.8371) -- (2.9801, 0.3348, 0.8430) -- (2.9778, 0.3329, 0.8930) -- (2.9778, 0.2822, 0.8871) -- cycle;
\fill[blue!15.0, opacity=0.5] (2.9778, 0.2822, 0.8871) -- (2.9778, 0.3329, 0.8930) -- (2.9753, 0.3310, 0.9430) -- (2.9753, 0.2802, 0.9371) -- cycle;
\fill[blue!15.0, opacity=0.5] (2.9753, 0.2802, 0.9371) -- (2.9753, 0.3310, 0.9430) -- (2.9727, 0.3291, 0.9930) -- (2.9727, 0.2781, 0.9871) -- cycle;
\fill[blue!15.0, opacity=0.5] (2.9727, 0.2781, 0.9871) -- (2.9727, 0.3291, 0.9930) -- (2.9700, 0.3270, 1.0430) -- (2.9700, 0.2760, 1.0371) -- cycle;
\fill[blue!15.0, opacity=0.5] (2.9700, 0.2760, 1.0371) -- (2.9700, 0.3270, 1.0430) -- (2.9672, 0.3249, 1.0930) -- (2.9672, 0.2738, 1.0871) -- cycle;
\fill[blue!15.0, opacity=0.5] (2.9672, 0.2738, 1.0871) -- (2.9672, 0.3249, 1.0930) -- (2.9644, 0.3227, 1.1430) -- (2.9644, 0.2715, 1.1371) -- cycle;
\fill[blue!15.0, opacity=0.5] (2.9644, 0.2715, 1.1371) -- (2.9644, 0.3227, 1.1430) -- (2.9615, 0.3205, 1.1930) -- (2.9615, 0.2692, 1.1871) -- cycle;
\fill[blue!15.0, opacity=0.5] (2.9615, 0.2692, 1.1871) -- (2.9615, 0.3205, 1.1930) -- (2.9585, 0.3182, 1.2430) -- (2.9585, 0.2668, 1.2371) -- cycle;
\fill[blue!15.0, opacity=0.5] (2.9585, 0.2668, 1.2371) -- (2.9585, 0.3182, 1.2430) -- (2.9555, 0.3159, 1.2930) -- (2.9555, 0.2644, 1.2871) -- cycle;
\fill[blue!15.0, opacity=0.5] (2.9555, 0.2644, 1.2871) -- (2.9555, 0.3159, 1.2930) -- (2.9525, 0.3136, 1.3430) -- (2.9525, 0.2620, 1.3371) -- cycle;
\fill[blue!15.0, opacity=0.5] (2.9525, 0.2620, 1.3371) -- (2.9525, 0.3136, 1.3430) -- (2.9494, 0.3112, 1.3930) -- (2.9494, 0.2595, 1.3871) -- cycle;
\fill[blue!15.1, opacity=0.5] (2.9494, 0.2595, 1.3871) -- (2.9494, 0.3112, 1.3930) -- (2.9463, 0.3088, 1.4430) -- (2.9463, 0.2570, 1.4371) -- cycle;
\fill[blue!15.1, opacity=0.5] (2.9463, 0.2570, 1.4371) -- (2.9463, 0.3088, 1.4430) -- (2.9431, 0.3064, 1.4930) -- (2.9431, 0.2545, 1.4871) -- cycle;
\fill[blue!15.2, opacity=0.5] (2.9431, 0.2545, 1.4871) -- (2.9431, 0.3064, 1.4930) -- (2.9400, 0.3040, 1.5430) -- (2.9400, 0.2520, 1.5371) -- cycle;
\fill[blue!15.3, opacity=0.5] (2.9400, 0.2520, 1.5371) -- (2.9400, 0.3040, 1.5430) -- (2.9369, 0.3016, 1.5930) -- (2.9369, 0.2495, 1.5871) -- cycle;
\fill[blue!15.4, opacity=0.5] (2.9369, 0.2495, 1.5871) -- (2.9369, 0.3016, 1.5930) -- (2.9337, 0.2992, 1.6430) -- (2.9337, 0.2470, 1.6371) -- cycle;
\fill[blue!15.5, opacity=0.5] (2.9337, 0.2470, 1.6371) -- (2.9337, 0.2992, 1.6430) -- (2.9306, 0.2968, 1.6930) -- (2.9306, 0.2445, 1.6871) -- cycle;
\fill[blue!15.7, opacity=0.5] (2.9306, 0.2445, 1.6871) -- (2.9306, 0.2968, 1.6930) -- (2.9275, 0.2944, 1.7430) -- (2.9275, 0.2420, 1.7371) -- cycle;
\fill[blue!16.0, opacity=0.5] (2.9275, 0.2420, 1.7371) -- (2.9275, 0.2944, 1.7430) -- (2.9245, 0.2921, 1.7930) -- (2.9245, 0.2396, 1.7871) -- cycle;
\fill[blue!16.3, opacity=0.5] (2.9245, 0.2396, 1.7871) -- (2.9245, 0.2921, 1.7930) -- (2.9215, 0.2898, 1.8430) -- (2.9215, 0.2372, 1.8371) -- cycle;
\fill[blue!16.6, opacity=0.5] (2.9215, 0.2372, 1.8371) -- (2.9215, 0.2898, 1.8430) -- (2.9185, 0.2875, 1.8930) -- (2.9185, 0.2348, 1.8871) -- cycle;
\fill[blue!17.1, opacity=0.5] (2.9185, 0.2348, 1.8871) -- (2.9185, 0.2875, 1.8930) -- (2.9156, 0.2853, 1.9430) -- (2.9156, 0.2325, 1.9371) -- cycle;
\fill[blue!17.6, opacity=0.5] (2.9156, 0.2325, 1.9371) -- (2.9156, 0.2853, 1.9430) -- (2.9128, 0.2831, 1.9930) -- (2.9128, 0.2302, 1.9871) -- cycle;
\fill[blue!18.2, opacity=0.5] (2.9128, 0.2302, 1.9871) -- (2.9128, 0.2831, 1.9930) -- (2.9100, 0.2810, 2.0430) -- (2.9100, 0.2280, 2.0371) -- cycle;
\fill[blue!18.9, opacity=0.5] (2.9100, 0.2280, 2.0371) -- (2.9100, 0.2810, 2.0430) -- (2.9073, 0.2789, 2.0930) -- (2.9073, 0.2259, 2.0871) -- cycle;
\fill[blue!19.6, opacity=0.5] (2.9073, 0.2259, 2.0871) -- (2.9073, 0.2789, 2.0930) -- (2.9047, 0.2770, 2.1430) -- (2.9047, 0.2238, 2.1371) -- cycle;
\fill[blue!20.5, opacity=0.5] (2.9047, 0.2238, 2.1371) -- (2.9047, 0.2770, 2.1430) -- (2.9022, 0.2751, 2.1930) -- (2.9022, 0.2218, 2.1871) -- cycle;
\fill[blue!21.4, opacity=0.5] (2.9022, 0.2218, 2.1871) -- (2.9022, 0.2751, 2.1930) -- (2.8999, 0.2732, 2.2430) -- (2.8999, 0.2199, 2.2371) -- cycle;
\fill[blue!22.4, opacity=0.5] (2.8999, 0.2199, 2.2371) -- (2.8999, 0.2732, 2.2430) -- (2.8976, 0.2715, 2.2930) -- (2.8976, 0.2181, 2.2871) -- cycle;
\fill[blue!23.5, opacity=0.5] (2.8976, 0.2181, 2.2871) -- (2.8976, 0.2715, 2.2930) -- (2.8954, 0.2698, 2.3430) -- (2.8954, 0.2163, 2.3371) -- cycle;
\fill[blue!24.7, opacity=0.5] (2.8954, 0.2163, 2.3371) -- (2.8954, 0.2698, 2.3430) -- (2.8934, 0.2683, 2.3930) -- (2.8934, 0.2147, 2.3871) -- cycle;
\fill[blue!25.9, opacity=0.5] (2.8934, 0.2147, 2.3871) -- (2.8934, 0.2683, 2.3930) -- (2.8915, 0.2668, 2.4430) -- (2.8915, 0.2132, 2.4371) -- cycle;
\fill[blue!27.1, opacity=0.5] (2.8915, 0.2132, 2.4371) -- (2.8915, 0.2668, 2.4430) -- (2.8897, 0.2654, 2.4930) -- (2.8897, 0.2117, 2.4871) -- cycle;
\fill[blue!28.4, opacity=0.5] (2.8897, 0.2117, 2.4871) -- (2.8897, 0.2654, 2.4930) -- (2.8880, 0.2642, 2.5430) -- (2.8880, 0.2104, 2.5371) -- cycle;
\fill[blue!29.7, opacity=0.5] (2.8880, 0.2104, 2.5371) -- (2.8880, 0.2642, 2.5430) -- (2.8865, 0.2630, 2.5930) -- (2.8865, 0.2092, 2.5871) -- cycle;
\fill[blue!31.0, opacity=0.5] (2.8865, 0.2092, 2.5871) -- (2.8865, 0.2630, 2.5930) -- (2.8852, 0.2620, 2.6430) -- (2.8852, 0.2081, 2.6371) -- cycle;
\fill[blue!32.3, opacity=0.5] (2.8852, 0.2081, 2.6371) -- (2.8852, 0.2620, 2.6430) -- (2.8840, 0.2611, 2.6930) -- (2.8840, 0.2072, 2.6871) -- cycle;
\fill[blue!33.6, opacity=0.5] (2.8840, 0.2072, 2.6871) -- (2.8840, 0.2611, 2.6930) -- (2.8829, 0.2603, 2.7430) -- (2.8829, 0.2063, 2.7371) -- cycle;
\fill[blue!34.8, opacity=0.5] (2.8829, 0.2063, 2.7371) -- (2.8829, 0.2603, 2.7430) -- (2.8820, 0.2596, 2.7930) -- (2.8820, 0.2056, 2.7871) -- cycle;
\fill[blue!36.0, opacity=0.5] (2.8820, 0.2056, 2.7871) -- (2.8820, 0.2596, 2.7930) -- (2.8813, 0.2590, 2.8430) -- (2.8813, 0.2050, 2.8371) -- cycle;
\fill[blue!37.2, opacity=0.5] (2.8813, 0.2050, 2.8371) -- (2.8813, 0.2590, 2.8430) -- (2.8807, 0.2586, 2.8930) -- (2.8807, 0.2046, 2.8871) -- cycle;
\fill[blue!38.3, opacity=0.5] (2.8807, 0.2046, 2.8871) -- (2.8807, 0.2586, 2.8930) -- (2.8803, 0.2583, 2.9430) -- (2.8803, 0.2043, 2.9371) -- cycle;
\fill[blue!39.3, opacity=0.5] (2.8803, 0.2043, 2.9371) -- (2.8803, 0.2583, 2.9430) -- (2.8801, 0.2581, 2.9930) -- (2.8801, 0.2041, 2.9871) -- cycle;
\fill[blue!40.2, opacity=0.5] (2.8801, 0.2041, 2.9871) -- (2.8801, 0.2581, 2.9930) -- (2.8800, 0.2580, 3.0430) -- (2.8800, 0.2040, 3.0371) -- cycle;
\fill[blue!15.0, opacity=0.5] (3.0000, 0.3500, 0.0430) -- (3.0000, 0.4000, 0.0488) -- (2.9999, 0.3999, 0.0988) -- (2.9999, 0.3499, 0.0930) -- cycle;
\fill[blue!15.0, opacity=0.5] (2.9999, 0.3499, 0.0930) -- (2.9999, 0.3999, 0.0988) -- (2.9997, 0.3998, 0.1488) -- (2.9997, 0.3497, 0.1430) -- cycle;
\fill[blue!15.0, opacity=0.5] (2.9997, 0.3497, 0.1430) -- (2.9997, 0.3998, 0.1488) -- (2.9993, 0.3995, 0.1988) -- (2.9993, 0.3494, 0.1930) -- cycle;
\fill[blue!15.0, opacity=0.5] (2.9993, 0.3494, 0.1930) -- (2.9993, 0.3995, 0.1988) -- (2.9987, 0.3990, 0.2488) -- (2.9987, 0.3490, 0.2430) -- cycle;
\fill[blue!15.0, opacity=0.5] (2.9987, 0.3490, 0.2430) -- (2.9987, 0.3990, 0.2488) -- (2.9980, 0.3985, 0.2988) -- (2.9980, 0.3484, 0.2930) -- cycle;
\fill[blue!15.0, opacity=0.5] (2.9980, 0.3484, 0.2930) -- (2.9980, 0.3985, 0.2988) -- (2.9971, 0.3978, 0.3488) -- (2.9971, 0.3477, 0.3430) -- cycle;
\fill[blue!15.0, opacity=0.5] (2.9971, 0.3477, 0.3430) -- (2.9971, 0.3978, 0.3488) -- (2.9960, 0.3971, 0.3988) -- (2.9960, 0.3469, 0.3930) -- cycle;
\fill[blue!15.0, opacity=0.5] (2.9960, 0.3469, 0.3930) -- (2.9960, 0.3971, 0.3988) -- (2.9948, 0.3962, 0.4488) -- (2.9948, 0.3460, 0.4430) -- cycle;
\fill[blue!15.0, opacity=0.5] (2.9948, 0.3460, 0.4430) -- (2.9948, 0.3962, 0.4488) -- (2.9935, 0.3952, 0.4988) -- (2.9935, 0.3450, 0.4930) -- cycle;
\fill[blue!15.0, opacity=0.5] (2.9935, 0.3450, 0.4930) -- (2.9935, 0.3952, 0.4988) -- (2.9920, 0.3941, 0.5488) -- (2.9920, 0.3438, 0.5430) -- cycle;
\fill[blue!15.0, opacity=0.5] (2.9920, 0.3438, 0.5430) -- (2.9920, 0.3941, 0.5488) -- (2.9903, 0.3929, 0.5988) -- (2.9903, 0.3426, 0.5930) -- cycle;
\fill[blue!15.0, opacity=0.5] (2.9903, 0.3426, 0.5930) -- (2.9903, 0.3929, 0.5988) -- (2.9885, 0.3916, 0.6488) -- (2.9885, 0.3412, 0.6430) -- cycle;
\fill[blue!15.0, opacity=0.5] (2.9885, 0.3412, 0.6430) -- (2.9885, 0.3916, 0.6488) -- (2.9866, 0.3902, 0.6988) -- (2.9866, 0.3397, 0.6930) -- cycle;
\fill[blue!15.0, opacity=0.5] (2.9866, 0.3397, 0.6930) -- (2.9866, 0.3902, 0.6988) -- (2.9846, 0.3887, 0.7488) -- (2.9846, 0.3382, 0.7430) -- cycle;
\fill[blue!15.0, opacity=0.5] (2.9846, 0.3382, 0.7430) -- (2.9846, 0.3887, 0.7488) -- (2.9824, 0.3871, 0.7988) -- (2.9824, 0.3365, 0.7930) -- cycle;
\fill[blue!15.0, opacity=0.5] (2.9824, 0.3365, 0.7930) -- (2.9824, 0.3871, 0.7988) -- (2.9801, 0.3854, 0.8488) -- (2.9801, 0.3348, 0.8430) -- cycle;
\fill[blue!15.0, opacity=0.5] (2.9801, 0.3348, 0.8430) -- (2.9801, 0.3854, 0.8488) -- (2.9778, 0.3837, 0.8988) -- (2.9778, 0.3329, 0.8930) -- cycle;
\fill[blue!15.0, opacity=0.5] (2.9778, 0.3329, 0.8930) -- (2.9778, 0.3837, 0.8988) -- (2.9753, 0.3819, 0.9488) -- (2.9753, 0.3310, 0.9430) -- cycle;
\fill[blue!15.0, opacity=0.5] (2.9753, 0.3310, 0.9430) -- (2.9753, 0.3819, 0.9488) -- (2.9727, 0.3800, 0.9988) -- (2.9727, 0.3291, 0.9930) -- cycle;
\fill[blue!15.0, opacity=0.5] (2.9727, 0.3291, 0.9930) -- (2.9727, 0.3800, 0.9988) -- (2.9700, 0.3780, 1.0488) -- (2.9700, 0.3270, 1.0430) -- cycle;
\fill[blue!15.0, opacity=0.5] (2.9700, 0.3270, 1.0430) -- (2.9700, 0.3780, 1.0488) -- (2.9672, 0.3760, 1.0988) -- (2.9672, 0.3249, 1.0930) -- cycle;
\fill[blue!15.0, opacity=0.5] (2.9672, 0.3249, 1.0930) -- (2.9672, 0.3760, 1.0988) -- (2.9644, 0.3739, 1.1488) -- (2.9644, 0.3227, 1.1430) -- cycle;
\fill[blue!15.0, opacity=0.5] (2.9644, 0.3227, 1.1430) -- (2.9644, 0.3739, 1.1488) -- (2.9615, 0.3718, 1.1988) -- (2.9615, 0.3205, 1.1930) -- cycle;
\fill[blue!15.0, opacity=0.5] (2.9615, 0.3205, 1.1930) -- (2.9615, 0.3718, 1.1988) -- (2.9585, 0.3696, 1.2488) -- (2.9585, 0.3182, 1.2430) -- cycle;
\fill[blue!15.1, opacity=0.5] (2.9585, 0.3182, 1.2430) -- (2.9585, 0.3696, 1.2488) -- (2.9555, 0.3674, 1.2988) -- (2.9555, 0.3159, 1.2930) -- cycle;
\fill[blue!15.1, opacity=0.5] (2.9555, 0.3159, 1.2930) -- (2.9555, 0.3674, 1.2988) -- (2.9525, 0.3651, 1.3488) -- (2.9525, 0.3136, 1.3430) -- cycle;
\fill[blue!15.2, opacity=0.5] (2.9525, 0.3136, 1.3430) -- (2.9525, 0.3651, 1.3488) -- (2.9494, 0.3629, 1.3988) -- (2.9494, 0.3112, 1.3930) -- cycle;
\fill[blue!15.3, opacity=0.5] (2.9494, 0.3112, 1.3930) -- (2.9494, 0.3629, 1.3988) -- (2.9463, 0.3606, 1.4488) -- (2.9463, 0.3088, 1.4430) -- cycle;
\fill[blue!15.4, opacity=0.5] (2.9463, 0.3088, 1.4430) -- (2.9463, 0.3606, 1.4488) -- (2.9431, 0.3583, 1.4988) -- (2.9431, 0.3064, 1.4930) -- cycle;
\fill[blue!15.6, opacity=0.5] (2.9431, 0.3064, 1.4930) -- (2.9431, 0.3583, 1.4988) -- (2.9400, 0.3560, 1.5488) -- (2.9400, 0.3040, 1.5430) -- cycle;
\fill[blue!15.8, opacity=0.5] (2.9400, 0.3040, 1.5430) -- (2.9400, 0.3560, 1.5488) -- (2.9369, 0.3537, 1.5988) -- (2.9369, 0.3016, 1.5930) -- cycle;
\fill[blue!16.1, opacity=0.5] (2.9369, 0.3016, 1.5930) -- (2.9369, 0.3537, 1.5988) -- (2.9337, 0.3514, 1.6488) -- (2.9337, 0.2992, 1.6430) -- cycle;
\fill[blue!16.4, opacity=0.5] (2.9337, 0.2992, 1.6430) -- (2.9337, 0.3514, 1.6488) -- (2.9306, 0.3491, 1.6988) -- (2.9306, 0.2968, 1.6930) -- cycle;
\fill[blue!16.9, opacity=0.5] (2.9306, 0.2968, 1.6930) -- (2.9306, 0.3491, 1.6988) -- (2.9275, 0.3469, 1.7488) -- (2.9275, 0.2944, 1.7430) -- cycle;
\fill[blue!17.4, opacity=0.5] (2.9275, 0.2944, 1.7430) -- (2.9275, 0.3469, 1.7488) -- (2.9245, 0.3446, 1.7988) -- (2.9245, 0.2921, 1.7930) -- cycle;
\fill[blue!18.0, opacity=0.5] (2.9245, 0.2921, 1.7930) -- (2.9245, 0.3446, 1.7988) -- (2.9215, 0.3424, 1.8488) -- (2.9215, 0.2898, 1.8430) -- cycle;
\fill[blue!18.7, opacity=0.5] (2.9215, 0.2898, 1.8430) -- (2.9215, 0.3424, 1.8488) -- (2.9185, 0.3402, 1.8988) -- (2.9185, 0.2875, 1.8930) -- cycle;
\fill[blue!19.6, opacity=0.5] (2.9185, 0.2875, 1.8930) -- (2.9185, 0.3402, 1.8988) -- (2.9156, 0.3381, 1.9488) -- (2.9156, 0.2853, 1.9430) -- cycle;
\fill[blue!20.5, opacity=0.5] (2.9156, 0.2853, 1.9430) -- (2.9156, 0.3381, 1.9488) -- (2.9128, 0.3360, 1.9988) -- (2.9128, 0.2831, 1.9930) -- cycle;
\fill[blue!21.5, opacity=0.5] (2.9128, 0.2831, 1.9930) -- (2.9128, 0.3360, 1.9988) -- (2.9100, 0.3340, 2.0488) -- (2.9100, 0.2810, 2.0430) -- cycle;
\fill[blue!22.6, opacity=0.5] (2.9100, 0.2810, 2.0430) -- (2.9100, 0.3340, 2.0488) -- (2.9073, 0.3320, 2.0988) -- (2.9073, 0.2789, 2.0930) -- cycle;
\fill[blue!23.9, opacity=0.5] (2.9073, 0.2789, 2.0930) -- (2.9073, 0.3320, 2.0988) -- (2.9047, 0.3301, 2.1488) -- (2.9047, 0.2770, 2.1430) -- cycle;
\fill[blue!25.2, opacity=0.5] (2.9047, 0.2770, 2.1430) -- (2.9047, 0.3301, 2.1488) -- (2.9022, 0.3283, 2.1988) -- (2.9022, 0.2751, 2.1930) -- cycle;
\fill[blue!26.5, opacity=0.5] (2.9022, 0.2751, 2.1930) -- (2.9022, 0.3283, 2.1988) -- (2.8999, 0.3266, 2.2488) -- (2.8999, 0.2732, 2.2430) -- cycle;
\fill[blue!28.0, opacity=0.5] (2.8999, 0.2732, 2.2430) -- (2.8999, 0.3266, 2.2488) -- (2.8976, 0.3249, 2.2988) -- (2.8976, 0.2715, 2.2930) -- cycle;
\fill[blue!29.4, opacity=0.5] (2.8976, 0.2715, 2.2930) -- (2.8976, 0.3249, 2.2988) -- (2.8954, 0.3233, 2.3488) -- (2.8954, 0.2698, 2.3430) -- cycle;
\fill[blue!30.9, opacity=0.5] (2.8954, 0.2698, 2.3430) -- (2.8954, 0.3233, 2.3488) -- (2.8934, 0.3218, 2.3988) -- (2.8934, 0.2683, 2.3930) -- cycle;
\fill[blue!32.5, opacity=0.5] (2.8934, 0.2683, 2.3930) -- (2.8934, 0.3218, 2.3988) -- (2.8915, 0.3204, 2.4488) -- (2.8915, 0.2668, 2.4430) -- cycle;
\fill[blue!34.0, opacity=0.5] (2.8915, 0.2668, 2.4430) -- (2.8915, 0.3204, 2.4488) -- (2.8897, 0.3191, 2.4988) -- (2.8897, 0.2654, 2.4930) -- cycle;
\fill[blue!35.5, opacity=0.5] (2.8897, 0.2654, 2.4930) -- (2.8897, 0.3191, 2.4988) -- (2.8880, 0.3179, 2.5488) -- (2.8880, 0.2642, 2.5430) -- cycle;
\fill[blue!37.0, opacity=0.5] (2.8880, 0.2642, 2.5430) -- (2.8880, 0.3179, 2.5488) -- (2.8865, 0.3168, 2.5988) -- (2.8865, 0.2630, 2.5930) -- cycle;
\fill[blue!38.5, opacity=0.5] (2.8865, 0.2630, 2.5930) -- (2.8865, 0.3168, 2.5988) -- (2.8852, 0.3158, 2.6488) -- (2.8852, 0.2620, 2.6430) -- cycle;
\fill[blue!39.9, opacity=0.5] (2.8852, 0.2620, 2.6430) -- (2.8852, 0.3158, 2.6488) -- (2.8840, 0.3149, 2.6988) -- (2.8840, 0.2611, 2.6930) -- cycle;
\fill[blue!41.2, opacity=0.5] (2.8840, 0.2611, 2.6930) -- (2.8840, 0.3149, 2.6988) -- (2.8829, 0.3142, 2.7488) -- (2.8829, 0.2603, 2.7430) -- cycle;
\fill[blue!42.4, opacity=0.5] (2.8829, 0.2603, 2.7430) -- (2.8829, 0.3142, 2.7488) -- (2.8820, 0.3135, 2.7988) -- (2.8820, 0.2596, 2.7930) -- cycle;
\fill[blue!43.6, opacity=0.5] (2.8820, 0.2596, 2.7930) -- (2.8820, 0.3135, 2.7988) -- (2.8813, 0.3130, 2.8488) -- (2.8813, 0.2590, 2.8430) -- cycle;
\fill[blue!44.7, opacity=0.5] (2.8813, 0.2590, 2.8430) -- (2.8813, 0.3130, 2.8488) -- (2.8807, 0.3125, 2.8988) -- (2.8807, 0.2586, 2.8930) -- cycle;
\fill[blue!45.6, opacity=0.5] (2.8807, 0.2586, 2.8930) -- (2.8807, 0.3125, 2.8988) -- (2.8803, 0.3122, 2.9488) -- (2.8803, 0.2583, 2.9430) -- cycle;
\fill[blue!46.5, opacity=0.5] (2.8803, 0.2583, 2.9430) -- (2.8803, 0.3122, 2.9488) -- (2.8801, 0.3121, 2.9988) -- (2.8801, 0.2581, 2.9930) -- cycle;
\fill[blue!47.2, opacity=0.5] (2.8801, 0.2581, 2.9930) -- (2.8801, 0.3121, 2.9988) -- (2.8800, 0.3120, 3.0488) -- (2.8800, 0.2580, 3.0430) -- cycle;
\fill[blue!15.0, opacity=0.5] (3.0000, 0.4000, 0.0488) -- (3.0000, 0.4500, 0.0545) -- (2.9999, 0.4499, 0.1045) -- (2.9999, 0.3999, 0.0988) -- cycle;
\fill[blue!15.0, opacity=0.5] (2.9999, 0.3999, 0.0988) -- (2.9999, 0.4499, 0.1045) -- (2.9997, 0.4498, 0.1545) -- (2.9997, 0.3998, 0.1488) -- cycle;
\fill[blue!15.0, opacity=0.5] (2.9997, 0.3998, 0.1488) -- (2.9997, 0.4498, 0.1545) -- (2.9993, 0.4495, 0.2045) -- (2.9993, 0.3995, 0.1988) -- cycle;
\fill[blue!15.0, opacity=0.5] (2.9993, 0.3995, 0.1988) -- (2.9993, 0.4495, 0.2045) -- (2.9987, 0.4491, 0.2545) -- (2.9987, 0.3990, 0.2488) -- cycle;
\fill[blue!15.0, opacity=0.5] (2.9987, 0.3990, 0.2488) -- (2.9987, 0.4491, 0.2545) -- (2.9980, 0.4486, 0.3045) -- (2.9980, 0.3985, 0.2988) -- cycle;
\fill[blue!15.0, opacity=0.5] (2.9980, 0.3985, 0.2988) -- (2.9980, 0.4486, 0.3045) -- (2.9971, 0.4479, 0.3545) -- (2.9971, 0.3978, 0.3488) -- cycle;
\fill[blue!15.0, opacity=0.5] (2.9971, 0.3978, 0.3488) -- (2.9971, 0.4479, 0.3545) -- (2.9960, 0.4472, 0.4045) -- (2.9960, 0.3971, 0.3988) -- cycle;
\fill[blue!15.0, opacity=0.5] (2.9960, 0.3971, 0.3988) -- (2.9960, 0.4472, 0.4045) -- (2.9948, 0.4464, 0.4545) -- (2.9948, 0.3962, 0.4488) -- cycle;
\fill[blue!15.0, opacity=0.5] (2.9948, 0.3962, 0.4488) -- (2.9948, 0.4464, 0.4545) -- (2.9935, 0.4454, 0.5045) -- (2.9935, 0.3952, 0.4988) -- cycle;
\fill[blue!15.0, opacity=0.5] (2.9935, 0.3952, 0.4988) -- (2.9935, 0.4454, 0.5045) -- (2.9920, 0.4444, 0.5545) -- (2.9920, 0.3941, 0.5488) -- cycle;
\fill[blue!15.0, opacity=0.5] (2.9920, 0.3941, 0.5488) -- (2.9920, 0.4444, 0.5545) -- (2.9903, 0.4432, 0.6045) -- (2.9903, 0.3929, 0.5988) -- cycle;
\fill[blue!15.0, opacity=0.5] (2.9903, 0.3929, 0.5988) -- (2.9903, 0.4432, 0.6045) -- (2.9885, 0.4420, 0.6545) -- (2.9885, 0.3916, 0.6488) -- cycle;
\fill[blue!15.0, opacity=0.5] (2.9885, 0.3916, 0.6488) -- (2.9885, 0.4420, 0.6545) -- (2.9866, 0.4406, 0.7045) -- (2.9866, 0.3902, 0.6988) -- cycle;
\fill[blue!15.0, opacity=0.5] (2.9866, 0.3902, 0.6988) -- (2.9866, 0.4406, 0.7045) -- (2.9846, 0.4392, 0.7545) -- (2.9846, 0.3887, 0.7488) -- cycle;
\fill[blue!15.0, opacity=0.5] (2.9846, 0.3887, 0.7488) -- (2.9846, 0.4392, 0.7545) -- (2.9824, 0.4377, 0.8045) -- (2.9824, 0.3871, 0.7988) -- cycle;
\fill[blue!15.0, opacity=0.5] (2.9824, 0.3871, 0.7988) -- (2.9824, 0.4377, 0.8045) -- (2.9801, 0.4361, 0.8545) -- (2.9801, 0.3854, 0.8488) -- cycle;
\fill[blue!15.0, opacity=0.5] (2.9801, 0.3854, 0.8488) -- (2.9801, 0.4361, 0.8545) -- (2.9778, 0.4344, 0.9045) -- (2.9778, 0.3837, 0.8988) -- cycle;
\fill[blue!15.0, opacity=0.5] (2.9778, 0.3837, 0.8988) -- (2.9778, 0.4344, 0.9045) -- (2.9753, 0.4327, 0.9545) -- (2.9753, 0.3819, 0.9488) -- cycle;
\fill[blue!15.0, opacity=0.5] (2.9753, 0.3819, 0.9488) -- (2.9753, 0.4327, 0.9545) -- (2.9727, 0.4309, 1.0045) -- (2.9727, 0.3800, 0.9988) -- cycle;
\fill[blue!15.0, opacity=0.5] (2.9727, 0.3800, 0.9988) -- (2.9727, 0.4309, 1.0045) -- (2.9700, 0.4290, 1.0545) -- (2.9700, 0.3780, 1.0488) -- cycle;
\fill[blue!15.0, opacity=0.5] (2.9700, 0.3780, 1.0488) -- (2.9700, 0.4290, 1.0545) -- (2.9672, 0.4271, 1.1045) -- (2.9672, 0.3760, 1.0988) -- cycle;
\fill[blue!15.0, opacity=0.5] (2.9672, 0.3760, 1.0988) -- (2.9672, 0.4271, 1.1045) -- (2.9644, 0.4251, 1.1545) -- (2.9644, 0.3739, 1.1488) -- cycle;
\fill[blue!15.0, opacity=0.5] (2.9644, 0.3739, 1.1488) -- (2.9644, 0.4251, 1.1545) -- (2.9615, 0.4231, 1.2045) -- (2.9615, 0.3718, 1.1988) -- cycle;
\fill[blue!15.1, opacity=0.5] (2.9615, 0.3718, 1.1988) -- (2.9615, 0.4231, 1.2045) -- (2.9585, 0.4210, 1.2545) -- (2.9585, 0.3696, 1.2488) -- cycle;
\fill[blue!15.1, opacity=0.5] (2.9585, 0.3696, 1.2488) -- (2.9585, 0.4210, 1.2545) -- (2.9555, 0.4189, 1.3045) -- (2.9555, 0.3674, 1.2988) -- cycle;
\fill[blue!15.1, opacity=0.5] (2.9555, 0.3674, 1.2988) -- (2.9555, 0.4189, 1.3045) -- (2.9525, 0.4167, 1.3545) -- (2.9525, 0.3651, 1.3488) -- cycle;
\fill[blue!15.2, opacity=0.5] (2.9525, 0.3651, 1.3488) -- (2.9525, 0.4167, 1.3545) -- (2.9494, 0.4146, 1.4045) -- (2.9494, 0.3629, 1.3988) -- cycle;
\fill[blue!15.3, opacity=0.5] (2.9494, 0.3629, 1.3988) -- (2.9494, 0.4146, 1.4045) -- (2.9463, 0.4124, 1.4545) -- (2.9463, 0.3606, 1.4488) -- cycle;
\fill[blue!15.5, opacity=0.5] (2.9463, 0.3606, 1.4488) -- (2.9463, 0.4124, 1.4545) -- (2.9431, 0.4102, 1.5045) -- (2.9431, 0.3583, 1.4988) -- cycle;
\fill[blue!15.7, opacity=0.5] (2.9431, 0.3583, 1.4988) -- (2.9431, 0.4102, 1.5045) -- (2.9400, 0.4080, 1.5545) -- (2.9400, 0.3560, 1.5488) -- cycle;
\fill[blue!15.9, opacity=0.5] (2.9400, 0.3560, 1.5488) -- (2.9400, 0.4080, 1.5545) -- (2.9369, 0.4058, 1.6045) -- (2.9369, 0.3537, 1.5988) -- cycle;
\fill[blue!16.3, opacity=0.5] (2.9369, 0.3537, 1.5988) -- (2.9369, 0.4058, 1.6045) -- (2.9337, 0.4036, 1.6545) -- (2.9337, 0.3514, 1.6488) -- cycle;
\fill[blue!16.7, opacity=0.5] (2.9337, 0.3514, 1.6488) -- (2.9337, 0.4036, 1.6545) -- (2.9306, 0.4014, 1.7045) -- (2.9306, 0.3491, 1.6988) -- cycle;
\fill[blue!17.2, opacity=0.5] (2.9306, 0.3491, 1.6988) -- (2.9306, 0.4014, 1.7045) -- (2.9275, 0.3993, 1.7545) -- (2.9275, 0.3469, 1.7488) -- cycle;
\fill[blue!17.8, opacity=0.5] (2.9275, 0.3469, 1.7488) -- (2.9275, 0.3993, 1.7545) -- (2.9245, 0.3971, 1.8045) -- (2.9245, 0.3446, 1.7988) -- cycle;
\fill[blue!18.5, opacity=0.5] (2.9245, 0.3446, 1.7988) -- (2.9245, 0.3971, 1.8045) -- (2.9215, 0.3950, 1.8545) -- (2.9215, 0.3424, 1.8488) -- cycle;
\fill[blue!19.3, opacity=0.5] (2.9215, 0.3424, 1.8488) -- (2.9215, 0.3950, 1.8545) -- (2.9185, 0.3929, 1.9045) -- (2.9185, 0.3402, 1.8988) -- cycle;
\fill[blue!20.2, opacity=0.5] (2.9185, 0.3402, 1.8988) -- (2.9185, 0.3929, 1.9045) -- (2.9156, 0.3909, 1.9545) -- (2.9156, 0.3381, 1.9488) -- cycle;
\fill[blue!21.2, opacity=0.5] (2.9156, 0.3381, 1.9488) -- (2.9156, 0.3909, 1.9545) -- (2.9128, 0.3889, 2.0045) -- (2.9128, 0.3360, 1.9988) -- cycle;
\fill[blue!22.3, opacity=0.5] (2.9128, 0.3360, 1.9988) -- (2.9128, 0.3889, 2.0045) -- (2.9100, 0.3870, 2.0545) -- (2.9100, 0.3340, 2.0488) -- cycle;
\fill[blue!23.5, opacity=0.5] (2.9100, 0.3340, 2.0488) -- (2.9100, 0.3870, 2.0545) -- (2.9073, 0.3851, 2.1045) -- (2.9073, 0.3320, 2.0988) -- cycle;
\fill[blue!24.8, opacity=0.5] (2.9073, 0.3320, 2.0988) -- (2.9073, 0.3851, 2.1045) -- (2.9047, 0.3833, 2.1545) -- (2.9047, 0.3301, 2.1488) -- cycle;
\fill[blue!26.2, opacity=0.5] (2.9047, 0.3301, 2.1488) -- (2.9047, 0.3833, 2.1545) -- (2.9022, 0.3816, 2.2045) -- (2.9022, 0.3283, 2.1988) -- cycle;
\fill[blue!27.6, opacity=0.5] (2.9022, 0.3283, 2.1988) -- (2.9022, 0.3816, 2.2045) -- (2.8999, 0.3799, 2.2545) -- (2.8999, 0.3266, 2.2488) -- cycle;
\fill[blue!29.1, opacity=0.5] (2.8999, 0.3266, 2.2488) -- (2.8999, 0.3799, 2.2545) -- (2.8976, 0.3783, 2.3045) -- (2.8976, 0.3249, 2.2988) -- cycle;
\fill[blue!30.6, opacity=0.5] (2.8976, 0.3249, 2.2988) -- (2.8976, 0.3783, 2.3045) -- (2.8954, 0.3768, 2.3545) -- (2.8954, 0.3233, 2.3488) -- cycle;
\fill[blue!32.2, opacity=0.5] (2.8954, 0.3233, 2.3488) -- (2.8954, 0.3768, 2.3545) -- (2.8934, 0.3754, 2.4045) -- (2.8934, 0.3218, 2.3988) -- cycle;
\fill[blue!33.8, opacity=0.5] (2.8934, 0.3218, 2.3988) -- (2.8934, 0.3754, 2.4045) -- (2.8915, 0.3740, 2.4545) -- (2.8915, 0.3204, 2.4488) -- cycle;
\fill[blue!35.3, opacity=0.5] (2.8915, 0.3204, 2.4488) -- (2.8915, 0.3740, 2.4545) -- (2.8897, 0.3728, 2.5045) -- (2.8897, 0.3191, 2.4988) -- cycle;
\fill[blue!36.9, opacity=0.5] (2.8897, 0.3191, 2.4988) -- (2.8897, 0.3728, 2.5045) -- (2.8880, 0.3716, 2.5545) -- (2.8880, 0.3179, 2.5488) -- cycle;
\fill[blue!38.4, opacity=0.5] (2.8880, 0.3179, 2.5488) -- (2.8880, 0.3716, 2.5545) -- (2.8865, 0.3706, 2.6045) -- (2.8865, 0.3168, 2.5988) -- cycle;
\fill[blue!39.9, opacity=0.5] (2.8865, 0.3168, 2.5988) -- (2.8865, 0.3706, 2.6045) -- (2.8852, 0.3696, 2.6545) -- (2.8852, 0.3158, 2.6488) -- cycle;
\fill[blue!41.3, opacity=0.5] (2.8852, 0.3158, 2.6488) -- (2.8852, 0.3696, 2.6545) -- (2.8840, 0.3688, 2.7045) -- (2.8840, 0.3149, 2.6988) -- cycle;
\fill[blue!42.6, opacity=0.5] (2.8840, 0.3149, 2.6988) -- (2.8840, 0.3688, 2.7045) -- (2.8829, 0.3681, 2.7545) -- (2.8829, 0.3142, 2.7488) -- cycle;
\fill[blue!43.8, opacity=0.5] (2.8829, 0.3142, 2.7488) -- (2.8829, 0.3681, 2.7545) -- (2.8820, 0.3674, 2.8045) -- (2.8820, 0.3135, 2.7988) -- cycle;
\fill[blue!44.9, opacity=0.5] (2.8820, 0.3135, 2.7988) -- (2.8820, 0.3674, 2.8045) -- (2.8813, 0.3669, 2.8545) -- (2.8813, 0.3130, 2.8488) -- cycle;
\fill[blue!46.0, opacity=0.5] (2.8813, 0.3130, 2.8488) -- (2.8813, 0.3669, 2.8545) -- (2.8807, 0.3665, 2.9045) -- (2.8807, 0.3125, 2.8988) -- cycle;
\fill[blue!46.9, opacity=0.5] (2.8807, 0.3125, 2.8988) -- (2.8807, 0.3665, 2.9045) -- (2.8803, 0.3662, 2.9545) -- (2.8803, 0.3122, 2.9488) -- cycle;
\fill[blue!47.7, opacity=0.5] (2.8803, 0.3122, 2.9488) -- (2.8803, 0.3662, 2.9545) -- (2.8801, 0.3661, 3.0045) -- (2.8801, 0.3121, 2.9988) -- cycle;
\fill[blue!48.4, opacity=0.5] (2.8801, 0.3121, 2.9988) -- (2.8801, 0.3661, 3.0045) -- (2.8800, 0.3660, 3.0545) -- (2.8800, 0.3120, 3.0488) -- cycle;
\fill[blue!15.0, opacity=0.5] (3.0000, 0.4500, 0.0545) -- (3.0000, 0.5000, 0.0600) -- (2.9999, 0.4999, 0.1100) -- (2.9999, 0.4499, 0.1045) -- cycle;
\fill[blue!15.0, opacity=0.5] (2.9999, 0.4499, 0.1045) -- (2.9999, 0.4999, 0.1100) -- (2.9997, 0.4998, 0.1600) -- (2.9997, 0.4498, 0.1545) -- cycle;
\fill[blue!15.0, opacity=0.5] (2.9997, 0.4498, 0.1545) -- (2.9997, 0.4998, 0.1600) -- (2.9993, 0.4995, 0.2100) -- (2.9993, 0.4495, 0.2045) -- cycle;
\fill[blue!15.0, opacity=0.5] (2.9993, 0.4495, 0.2045) -- (2.9993, 0.4995, 0.2100) -- (2.9987, 0.4991, 0.2600) -- (2.9987, 0.4491, 0.2545) -- cycle;
\fill[blue!15.0, opacity=0.5] (2.9987, 0.4491, 0.2545) -- (2.9987, 0.4991, 0.2600) -- (2.9980, 0.4986, 0.3100) -- (2.9980, 0.4486, 0.3045) -- cycle;
\fill[blue!15.0, opacity=0.5] (2.9980, 0.4486, 0.3045) -- (2.9980, 0.4986, 0.3100) -- (2.9971, 0.4980, 0.3600) -- (2.9971, 0.4479, 0.3545) -- cycle;
\fill[blue!15.0, opacity=0.5] (2.9971, 0.4479, 0.3545) -- (2.9971, 0.4980, 0.3600) -- (2.9960, 0.4973, 0.4100) -- (2.9960, 0.4472, 0.4045) -- cycle;
\fill[blue!15.0, opacity=0.5] (2.9960, 0.4472, 0.4045) -- (2.9960, 0.4973, 0.4100) -- (2.9948, 0.4965, 0.4600) -- (2.9948, 0.4464, 0.4545) -- cycle;
\fill[blue!15.0, opacity=0.5] (2.9948, 0.4464, 0.4545) -- (2.9948, 0.4965, 0.4600) -- (2.9935, 0.4956, 0.5100) -- (2.9935, 0.4454, 0.5045) -- cycle;
\fill[blue!15.0, opacity=0.5] (2.9935, 0.4454, 0.5045) -- (2.9935, 0.4956, 0.5100) -- (2.9920, 0.4946, 0.5600) -- (2.9920, 0.4444, 0.5545) -- cycle;
\fill[blue!15.0, opacity=0.5] (2.9920, 0.4444, 0.5545) -- (2.9920, 0.4946, 0.5600) -- (2.9903, 0.4935, 0.6100) -- (2.9903, 0.4432, 0.6045) -- cycle;
\fill[blue!15.0, opacity=0.5] (2.9903, 0.4432, 0.6045) -- (2.9903, 0.4935, 0.6100) -- (2.9885, 0.4924, 0.6600) -- (2.9885, 0.4420, 0.6545) -- cycle;
\fill[blue!15.0, opacity=0.5] (2.9885, 0.4420, 0.6545) -- (2.9885, 0.4924, 0.6600) -- (2.9866, 0.4911, 0.7100) -- (2.9866, 0.4406, 0.7045) -- cycle;
\fill[blue!15.0, opacity=0.5] (2.9866, 0.4406, 0.7045) -- (2.9866, 0.4911, 0.7100) -- (2.9846, 0.4897, 0.7600) -- (2.9846, 0.4392, 0.7545) -- cycle;
\fill[blue!15.0, opacity=0.5] (2.9846, 0.4392, 0.7545) -- (2.9846, 0.4897, 0.7600) -- (2.9824, 0.4883, 0.8100) -- (2.9824, 0.4377, 0.8045) -- cycle;
\fill[blue!15.0, opacity=0.5] (2.9824, 0.4377, 0.8045) -- (2.9824, 0.4883, 0.8100) -- (2.9801, 0.4868, 0.8600) -- (2.9801, 0.4361, 0.8545) -- cycle;
\fill[blue!15.0, opacity=0.5] (2.9801, 0.4361, 0.8545) -- (2.9801, 0.4868, 0.8600) -- (2.9778, 0.4852, 0.9100) -- (2.9778, 0.4344, 0.9045) -- cycle;
\fill[blue!15.0, opacity=0.5] (2.9778, 0.4344, 0.9045) -- (2.9778, 0.4852, 0.9100) -- (2.9753, 0.4835, 0.9600) -- (2.9753, 0.4327, 0.9545) -- cycle;
\fill[blue!15.0, opacity=0.5] (2.9753, 0.4327, 0.9545) -- (2.9753, 0.4835, 0.9600) -- (2.9727, 0.4818, 1.0100) -- (2.9727, 0.4309, 1.0045) -- cycle;
\fill[blue!15.0, opacity=0.5] (2.9727, 0.4309, 1.0045) -- (2.9727, 0.4818, 1.0100) -- (2.9700, 0.4800, 1.0600) -- (2.9700, 0.4290, 1.0545) -- cycle;
\fill[blue!15.0, opacity=0.5] (2.9700, 0.4290, 1.0545) -- (2.9700, 0.4800, 1.0600) -- (2.9672, 0.4782, 1.1100) -- (2.9672, 0.4271, 1.1045) -- cycle;
\fill[blue!15.0, opacity=0.5] (2.9672, 0.4271, 1.1045) -- (2.9672, 0.4782, 1.1100) -- (2.9644, 0.4763, 1.1600) -- (2.9644, 0.4251, 1.1545) -- cycle;
\fill[blue!15.0, opacity=0.5] (2.9644, 0.4251, 1.1545) -- (2.9644, 0.4763, 1.1600) -- (2.9615, 0.4743, 1.2100) -- (2.9615, 0.4231, 1.2045) -- cycle;
\fill[blue!15.0, opacity=0.5] (2.9615, 0.4231, 1.2045) -- (2.9615, 0.4743, 1.2100) -- (2.9585, 0.4724, 1.2600) -- (2.9585, 0.4210, 1.2545) -- cycle;
\fill[blue!15.0, opacity=0.5] (2.9585, 0.4210, 1.2545) -- (2.9585, 0.4724, 1.2600) -- (2.9555, 0.4704, 1.3100) -- (2.9555, 0.4189, 1.3045) -- cycle;
\fill[blue!15.1, opacity=0.5] (2.9555, 0.4189, 1.3045) -- (2.9555, 0.4704, 1.3100) -- (2.9525, 0.4683, 1.3600) -- (2.9525, 0.4167, 1.3545) -- cycle;
\fill[blue!15.1, opacity=0.5] (2.9525, 0.4167, 1.3545) -- (2.9525, 0.4683, 1.3600) -- (2.9494, 0.4663, 1.4100) -- (2.9494, 0.4146, 1.4045) -- cycle;
\fill[blue!15.2, opacity=0.5] (2.9494, 0.4146, 1.4045) -- (2.9494, 0.4663, 1.4100) -- (2.9463, 0.4642, 1.4600) -- (2.9463, 0.4124, 1.4545) -- cycle;
\fill[blue!15.2, opacity=0.5] (2.9463, 0.4124, 1.4545) -- (2.9463, 0.4642, 1.4600) -- (2.9431, 0.4621, 1.5100) -- (2.9431, 0.4102, 1.5045) -- cycle;
\fill[blue!15.4, opacity=0.5] (2.9431, 0.4102, 1.5045) -- (2.9431, 0.4621, 1.5100) -- (2.9400, 0.4600, 1.5600) -- (2.9400, 0.4080, 1.5545) -- cycle;
\fill[blue!15.5, opacity=0.5] (2.9400, 0.4080, 1.5545) -- (2.9400, 0.4600, 1.5600) -- (2.9369, 0.4579, 1.6100) -- (2.9369, 0.4058, 1.6045) -- cycle;
\fill[blue!15.7, opacity=0.5] (2.9369, 0.4058, 1.6045) -- (2.9369, 0.4579, 1.6100) -- (2.9337, 0.4558, 1.6600) -- (2.9337, 0.4036, 1.6545) -- cycle;
\fill[blue!16.0, opacity=0.5] (2.9337, 0.4036, 1.6545) -- (2.9337, 0.4558, 1.6600) -- (2.9306, 0.4537, 1.7100) -- (2.9306, 0.4014, 1.7045) -- cycle;
\fill[blue!16.3, opacity=0.5] (2.9306, 0.4014, 1.7045) -- (2.9306, 0.4537, 1.7100) -- (2.9275, 0.4517, 1.7600) -- (2.9275, 0.3993, 1.7545) -- cycle;
\fill[blue!16.7, opacity=0.5] (2.9275, 0.3993, 1.7545) -- (2.9275, 0.4517, 1.7600) -- (2.9245, 0.4496, 1.8100) -- (2.9245, 0.3971, 1.8045) -- cycle;
\fill[blue!17.2, opacity=0.5] (2.9245, 0.3971, 1.8045) -- (2.9245, 0.4496, 1.8100) -- (2.9215, 0.4476, 1.8600) -- (2.9215, 0.3950, 1.8545) -- cycle;
\fill[blue!17.7, opacity=0.5] (2.9215, 0.3950, 1.8545) -- (2.9215, 0.4476, 1.8600) -- (2.9185, 0.4457, 1.9100) -- (2.9185, 0.3929, 1.9045) -- cycle;
\fill[blue!18.4, opacity=0.5] (2.9185, 0.3929, 1.9045) -- (2.9185, 0.4457, 1.9100) -- (2.9156, 0.4437, 1.9600) -- (2.9156, 0.3909, 1.9545) -- cycle;
\fill[blue!19.1, opacity=0.5] (2.9156, 0.3909, 1.9545) -- (2.9156, 0.4437, 1.9600) -- (2.9128, 0.4418, 2.0100) -- (2.9128, 0.3889, 2.0045) -- cycle;
\fill[blue!20.0, opacity=0.5] (2.9128, 0.3889, 2.0045) -- (2.9128, 0.4418, 2.0100) -- (2.9100, 0.4400, 2.0600) -- (2.9100, 0.3870, 2.0545) -- cycle;
\fill[blue!20.9, opacity=0.5] (2.9100, 0.3870, 2.0545) -- (2.9100, 0.4400, 2.0600) -- (2.9073, 0.4382, 2.1100) -- (2.9073, 0.3851, 2.1045) -- cycle;
\fill[blue!21.9, opacity=0.5] (2.9073, 0.3851, 2.1045) -- (2.9073, 0.4382, 2.1100) -- (2.9047, 0.4365, 2.1600) -- (2.9047, 0.3833, 2.1545) -- cycle;
\fill[blue!23.0, opacity=0.5] (2.9047, 0.3833, 2.1545) -- (2.9047, 0.4365, 2.1600) -- (2.9022, 0.4348, 2.2100) -- (2.9022, 0.3816, 2.2045) -- cycle;
\fill[blue!24.2, opacity=0.5] (2.9022, 0.3816, 2.2045) -- (2.9022, 0.4348, 2.2100) -- (2.8999, 0.4332, 2.2600) -- (2.8999, 0.3799, 2.2545) -- cycle;
\fill[blue!25.5, opacity=0.5] (2.8999, 0.3799, 2.2545) -- (2.8999, 0.4332, 2.2600) -- (2.8976, 0.4317, 2.3100) -- (2.8976, 0.3783, 2.3045) -- cycle;
\fill[blue!26.8, opacity=0.5] (2.8976, 0.3783, 2.3045) -- (2.8976, 0.4317, 2.3100) -- (2.8954, 0.4303, 2.3600) -- (2.8954, 0.3768, 2.3545) -- cycle;
\fill[blue!28.2, opacity=0.5] (2.8954, 0.3768, 2.3545) -- (2.8954, 0.4303, 2.3600) -- (2.8934, 0.4289, 2.4100) -- (2.8934, 0.3754, 2.4045) -- cycle;
\fill[blue!29.6, opacity=0.5] (2.8934, 0.3754, 2.4045) -- (2.8934, 0.4289, 2.4100) -- (2.8915, 0.4276, 2.4600) -- (2.8915, 0.3740, 2.4545) -- cycle;
\fill[blue!31.0, opacity=0.5] (2.8915, 0.3740, 2.4545) -- (2.8915, 0.4276, 2.4600) -- (2.8897, 0.4265, 2.5100) -- (2.8897, 0.3728, 2.5045) -- cycle;
\fill[blue!32.5, opacity=0.5] (2.8897, 0.3728, 2.5045) -- (2.8897, 0.4265, 2.5100) -- (2.8880, 0.4254, 2.5600) -- (2.8880, 0.3716, 2.5545) -- cycle;
\fill[blue!33.9, opacity=0.5] (2.8880, 0.3716, 2.5545) -- (2.8880, 0.4254, 2.5600) -- (2.8865, 0.4244, 2.6100) -- (2.8865, 0.3706, 2.6045) -- cycle;
\fill[blue!35.3, opacity=0.5] (2.8865, 0.3706, 2.6045) -- (2.8865, 0.4244, 2.6100) -- (2.8852, 0.4235, 2.6600) -- (2.8852, 0.3696, 2.6545) -- cycle;
\fill[blue!36.7, opacity=0.5] (2.8852, 0.3696, 2.6545) -- (2.8852, 0.4235, 2.6600) -- (2.8840, 0.4227, 2.7100) -- (2.8840, 0.3688, 2.7045) -- cycle;
\fill[blue!38.0, opacity=0.5] (2.8840, 0.3688, 2.7045) -- (2.8840, 0.4227, 2.7100) -- (2.8829, 0.4220, 2.7600) -- (2.8829, 0.3681, 2.7545) -- cycle;
\fill[blue!39.3, opacity=0.5] (2.8829, 0.3681, 2.7545) -- (2.8829, 0.4220, 2.7600) -- (2.8820, 0.4214, 2.8100) -- (2.8820, 0.3674, 2.8045) -- cycle;
\fill[blue!40.5, opacity=0.5] (2.8820, 0.3674, 2.8045) -- (2.8820, 0.4214, 2.8100) -- (2.8813, 0.4209, 2.8600) -- (2.8813, 0.3669, 2.8545) -- cycle;
\fill[blue!41.6, opacity=0.5] (2.8813, 0.3669, 2.8545) -- (2.8813, 0.4209, 2.8600) -- (2.8807, 0.4205, 2.9100) -- (2.8807, 0.3665, 2.9045) -- cycle;
\fill[blue!42.7, opacity=0.5] (2.8807, 0.3665, 2.9045) -- (2.8807, 0.4205, 2.9100) -- (2.8803, 0.4202, 2.9600) -- (2.8803, 0.3662, 2.9545) -- cycle;
\fill[blue!43.6, opacity=0.5] (2.8803, 0.3662, 2.9545) -- (2.8803, 0.4202, 2.9600) -- (2.8801, 0.4201, 3.0100) -- (2.8801, 0.3661, 3.0045) -- cycle;
\fill[blue!44.4, opacity=0.5] (2.8801, 0.3661, 3.0045) -- (2.8801, 0.4201, 3.0100) -- (2.8800, 0.4200, 3.0600) -- (2.8800, 0.3660, 3.0545) -- cycle;
\fill[blue!15.0, opacity=0.5] (3.0000, 0.5000, 0.0600) -- (3.0000, 0.5500, 0.0654) -- (2.9999, 0.5499, 0.1154) -- (2.9999, 0.4999, 0.1100) -- cycle;
\fill[blue!15.0, opacity=0.5] (2.9999, 0.4999, 0.1100) -- (2.9999, 0.5499, 0.1154) -- (2.9997, 0.5498, 0.1654) -- (2.9997, 0.4998, 0.1600) -- cycle;
\fill[blue!15.0, opacity=0.5] (2.9997, 0.4998, 0.1600) -- (2.9997, 0.5498, 0.1654) -- (2.9993, 0.5495, 0.2154) -- (2.9993, 0.4995, 0.2100) -- cycle;
\fill[blue!15.0, opacity=0.5] (2.9993, 0.4995, 0.2100) -- (2.9993, 0.5495, 0.2154) -- (2.9987, 0.5492, 0.2654) -- (2.9987, 0.4991, 0.2600) -- cycle;
\fill[blue!15.0, opacity=0.5] (2.9987, 0.4991, 0.2600) -- (2.9987, 0.5492, 0.2654) -- (2.9980, 0.5487, 0.3154) -- (2.9980, 0.4986, 0.3100) -- cycle;
\fill[blue!15.0, opacity=0.5] (2.9980, 0.4986, 0.3100) -- (2.9980, 0.5487, 0.3154) -- (2.9971, 0.5481, 0.3654) -- (2.9971, 0.4980, 0.3600) -- cycle;
\fill[blue!15.0, opacity=0.5] (2.9971, 0.4980, 0.3600) -- (2.9971, 0.5481, 0.3654) -- (2.9960, 0.5475, 0.4154) -- (2.9960, 0.4973, 0.4100) -- cycle;
\fill[blue!15.0, opacity=0.5] (2.9960, 0.4973, 0.4100) -- (2.9960, 0.5475, 0.4154) -- (2.9948, 0.5467, 0.4654) -- (2.9948, 0.4965, 0.4600) -- cycle;
\fill[blue!15.0, opacity=0.5] (2.9948, 0.4965, 0.4600) -- (2.9948, 0.5467, 0.4654) -- (2.9935, 0.5459, 0.5154) -- (2.9935, 0.4956, 0.5100) -- cycle;
\fill[blue!15.0, opacity=0.5] (2.9935, 0.4956, 0.5100) -- (2.9935, 0.5459, 0.5154) -- (2.9920, 0.5449, 0.5654) -- (2.9920, 0.4946, 0.5600) -- cycle;
\fill[blue!15.0, opacity=0.5] (2.9920, 0.4946, 0.5600) -- (2.9920, 0.5449, 0.5654) -- (2.9903, 0.5439, 0.6154) -- (2.9903, 0.4935, 0.6100) -- cycle;
\fill[blue!15.0, opacity=0.5] (2.9903, 0.4935, 0.6100) -- (2.9903, 0.5439, 0.6154) -- (2.9885, 0.5427, 0.6654) -- (2.9885, 0.4924, 0.6600) -- cycle;
\fill[blue!15.0, opacity=0.5] (2.9885, 0.4924, 0.6600) -- (2.9885, 0.5427, 0.6654) -- (2.9866, 0.5415, 0.7154) -- (2.9866, 0.4911, 0.7100) -- cycle;
\fill[blue!15.0, opacity=0.5] (2.9866, 0.4911, 0.7100) -- (2.9866, 0.5415, 0.7154) -- (2.9846, 0.5402, 0.7654) -- (2.9846, 0.4897, 0.7600) -- cycle;
\fill[blue!15.0, opacity=0.5] (2.9846, 0.4897, 0.7600) -- (2.9846, 0.5402, 0.7654) -- (2.9824, 0.5389, 0.8154) -- (2.9824, 0.4883, 0.8100) -- cycle;
\fill[blue!15.0, opacity=0.5] (2.9824, 0.4883, 0.8100) -- (2.9824, 0.5389, 0.8154) -- (2.9801, 0.5374, 0.8654) -- (2.9801, 0.4868, 0.8600) -- cycle;
\fill[blue!15.0, opacity=0.5] (2.9801, 0.4868, 0.8600) -- (2.9801, 0.5374, 0.8654) -- (2.9778, 0.5359, 0.9154) -- (2.9778, 0.4852, 0.9100) -- cycle;
\fill[blue!15.0, opacity=0.5] (2.9778, 0.4852, 0.9100) -- (2.9778, 0.5359, 0.9154) -- (2.9753, 0.5343, 0.9654) -- (2.9753, 0.4835, 0.9600) -- cycle;
\fill[blue!15.0, opacity=0.5] (2.9753, 0.4835, 0.9600) -- (2.9753, 0.5343, 0.9654) -- (2.9727, 0.5327, 1.0154) -- (2.9727, 0.4818, 1.0100) -- cycle;
\fill[blue!15.0, opacity=0.5] (2.9727, 0.4818, 1.0100) -- (2.9727, 0.5327, 1.0154) -- (2.9700, 0.5310, 1.0654) -- (2.9700, 0.4800, 1.0600) -- cycle;
\fill[blue!15.0, opacity=0.5] (2.9700, 0.4800, 1.0600) -- (2.9700, 0.5310, 1.0654) -- (2.9672, 0.5293, 1.1154) -- (2.9672, 0.4782, 1.1100) -- cycle;
\fill[blue!15.0, opacity=0.5] (2.9672, 0.4782, 1.1100) -- (2.9672, 0.5293, 1.1154) -- (2.9644, 0.5275, 1.1654) -- (2.9644, 0.4763, 1.1600) -- cycle;
\fill[blue!15.0, opacity=0.5] (2.9644, 0.4763, 1.1600) -- (2.9644, 0.5275, 1.1654) -- (2.9615, 0.5256, 1.2154) -- (2.9615, 0.4743, 1.2100) -- cycle;
\fill[blue!15.0, opacity=0.5] (2.9615, 0.4743, 1.2100) -- (2.9615, 0.5256, 1.2154) -- (2.9585, 0.5237, 1.2654) -- (2.9585, 0.4724, 1.2600) -- cycle;
\fill[blue!15.0, opacity=0.5] (2.9585, 0.4724, 1.2600) -- (2.9585, 0.5237, 1.2654) -- (2.9555, 0.5218, 1.3154) -- (2.9555, 0.4704, 1.3100) -- cycle;
\fill[blue!15.0, opacity=0.5] (2.9555, 0.4704, 1.3100) -- (2.9555, 0.5218, 1.3154) -- (2.9525, 0.5199, 1.3654) -- (2.9525, 0.4683, 1.3600) -- cycle;
\fill[blue!15.0, opacity=0.5] (2.9525, 0.4683, 1.3600) -- (2.9525, 0.5199, 1.3654) -- (2.9494, 0.5179, 1.4154) -- (2.9494, 0.4663, 1.4100) -- cycle;
\fill[blue!15.0, opacity=0.5] (2.9494, 0.4663, 1.4100) -- (2.9494, 0.5179, 1.4154) -- (2.9463, 0.5160, 1.4654) -- (2.9463, 0.4642, 1.4600) -- cycle;
\fill[blue!15.1, opacity=0.5] (2.9463, 0.4642, 1.4600) -- (2.9463, 0.5160, 1.4654) -- (2.9431, 0.5140, 1.5154) -- (2.9431, 0.4621, 1.5100) -- cycle;
\fill[blue!15.1, opacity=0.5] (2.9431, 0.4621, 1.5100) -- (2.9431, 0.5140, 1.5154) -- (2.9400, 0.5120, 1.5654) -- (2.9400, 0.4600, 1.5600) -- cycle;
\fill[blue!15.1, opacity=0.5] (2.9400, 0.4600, 1.5600) -- (2.9400, 0.5120, 1.5654) -- (2.9369, 0.5100, 1.6154) -- (2.9369, 0.4579, 1.6100) -- cycle;
\fill[blue!15.2, opacity=0.5] (2.9369, 0.4579, 1.6100) -- (2.9369, 0.5100, 1.6154) -- (2.9337, 0.5080, 1.6654) -- (2.9337, 0.4558, 1.6600) -- cycle;
\fill[blue!15.3, opacity=0.5] (2.9337, 0.4558, 1.6600) -- (2.9337, 0.5080, 1.6654) -- (2.9306, 0.5061, 1.7154) -- (2.9306, 0.4537, 1.7100) -- cycle;
\fill[blue!15.4, opacity=0.5] (2.9306, 0.4537, 1.7100) -- (2.9306, 0.5061, 1.7154) -- (2.9275, 0.5041, 1.7654) -- (2.9275, 0.4517, 1.7600) -- cycle;
\fill[blue!15.6, opacity=0.5] (2.9275, 0.4517, 1.7600) -- (2.9275, 0.5041, 1.7654) -- (2.9245, 0.5022, 1.8154) -- (2.9245, 0.4496, 1.8100) -- cycle;
\fill[blue!15.8, opacity=0.5] (2.9245, 0.4496, 1.8100) -- (2.9245, 0.5022, 1.8154) -- (2.9215, 0.5003, 1.8654) -- (2.9215, 0.4476, 1.8600) -- cycle;
\fill[blue!16.0, opacity=0.5] (2.9215, 0.4476, 1.8600) -- (2.9215, 0.5003, 1.8654) -- (2.9185, 0.4984, 1.9154) -- (2.9185, 0.4457, 1.9100) -- cycle;
\fill[blue!16.4, opacity=0.5] (2.9185, 0.4457, 1.9100) -- (2.9185, 0.4984, 1.9154) -- (2.9156, 0.4965, 1.9654) -- (2.9156, 0.4437, 1.9600) -- cycle;
\fill[blue!16.7, opacity=0.5] (2.9156, 0.4437, 1.9600) -- (2.9156, 0.4965, 1.9654) -- (2.9128, 0.4947, 2.0154) -- (2.9128, 0.4418, 2.0100) -- cycle;
\fill[blue!17.2, opacity=0.5] (2.9128, 0.4418, 2.0100) -- (2.9128, 0.4947, 2.0154) -- (2.9100, 0.4930, 2.0654) -- (2.9100, 0.4400, 2.0600) -- cycle;
\fill[blue!17.7, opacity=0.5] (2.9100, 0.4400, 2.0600) -- (2.9100, 0.4930, 2.0654) -- (2.9073, 0.4913, 2.1154) -- (2.9073, 0.4382, 2.1100) -- cycle;
\fill[blue!18.3, opacity=0.5] (2.9073, 0.4382, 2.1100) -- (2.9073, 0.4913, 2.1154) -- (2.9047, 0.4897, 2.1654) -- (2.9047, 0.4365, 2.1600) -- cycle;
\fill[blue!18.9, opacity=0.5] (2.9047, 0.4365, 2.1600) -- (2.9047, 0.4897, 2.1654) -- (2.9022, 0.4881, 2.2154) -- (2.9022, 0.4348, 2.2100) -- cycle;
\fill[blue!19.7, opacity=0.5] (2.9022, 0.4348, 2.2100) -- (2.9022, 0.4881, 2.2154) -- (2.8999, 0.4866, 2.2654) -- (2.8999, 0.4332, 2.2600) -- cycle;
\fill[blue!20.5, opacity=0.5] (2.8999, 0.4332, 2.2600) -- (2.8999, 0.4866, 2.2654) -- (2.8976, 0.4851, 2.3154) -- (2.8976, 0.4317, 2.3100) -- cycle;
\fill[blue!21.4, opacity=0.5] (2.8976, 0.4317, 2.3100) -- (2.8976, 0.4851, 2.3154) -- (2.8954, 0.4838, 2.3654) -- (2.8954, 0.4303, 2.3600) -- cycle;
\fill[blue!22.4, opacity=0.5] (2.8954, 0.4303, 2.3600) -- (2.8954, 0.4838, 2.3654) -- (2.8934, 0.4825, 2.4154) -- (2.8934, 0.4289, 2.4100) -- cycle;
\fill[blue!23.4, opacity=0.5] (2.8934, 0.4289, 2.4100) -- (2.8934, 0.4825, 2.4154) -- (2.8915, 0.4813, 2.4654) -- (2.8915, 0.4276, 2.4600) -- cycle;
\fill[blue!24.5, opacity=0.5] (2.8915, 0.4276, 2.4600) -- (2.8915, 0.4813, 2.4654) -- (2.8897, 0.4801, 2.5154) -- (2.8897, 0.4265, 2.5100) -- cycle;
\fill[blue!25.6, opacity=0.5] (2.8897, 0.4265, 2.5100) -- (2.8897, 0.4801, 2.5154) -- (2.8880, 0.4791, 2.5654) -- (2.8880, 0.4254, 2.5600) -- cycle;
\fill[blue!26.7, opacity=0.5] (2.8880, 0.4254, 2.5600) -- (2.8880, 0.4791, 2.5654) -- (2.8865, 0.4781, 2.6154) -- (2.8865, 0.4244, 2.6100) -- cycle;
\fill[blue!27.9, opacity=0.5] (2.8865, 0.4244, 2.6100) -- (2.8865, 0.4781, 2.6154) -- (2.8852, 0.4773, 2.6654) -- (2.8852, 0.4235, 2.6600) -- cycle;
\fill[blue!29.1, opacity=0.5] (2.8852, 0.4235, 2.6600) -- (2.8852, 0.4773, 2.6654) -- (2.8840, 0.4765, 2.7154) -- (2.8840, 0.4227, 2.7100) -- cycle;
\fill[blue!30.3, opacity=0.5] (2.8840, 0.4227, 2.7100) -- (2.8840, 0.4765, 2.7154) -- (2.8829, 0.4759, 2.7654) -- (2.8829, 0.4220, 2.7600) -- cycle;
\fill[blue!31.5, opacity=0.5] (2.8829, 0.4220, 2.7600) -- (2.8829, 0.4759, 2.7654) -- (2.8820, 0.4753, 2.8154) -- (2.8820, 0.4214, 2.8100) -- cycle;
\fill[blue!32.7, opacity=0.5] (2.8820, 0.4214, 2.8100) -- (2.8820, 0.4753, 2.8154) -- (2.8813, 0.4748, 2.8654) -- (2.8813, 0.4209, 2.8600) -- cycle;
\fill[blue!33.8, opacity=0.5] (2.8813, 0.4209, 2.8600) -- (2.8813, 0.4748, 2.8654) -- (2.8807, 0.4745, 2.9154) -- (2.8807, 0.4205, 2.9100) -- cycle;
\fill[blue!34.9, opacity=0.5] (2.8807, 0.4205, 2.9100) -- (2.8807, 0.4745, 2.9154) -- (2.8803, 0.4742, 2.9654) -- (2.8803, 0.4202, 2.9600) -- cycle;
\fill[blue!36.0, opacity=0.5] (2.8803, 0.4202, 2.9600) -- (2.8803, 0.4742, 2.9654) -- (2.8801, 0.4741, 3.0154) -- (2.8801, 0.4201, 3.0100) -- cycle;
\fill[blue!36.9, opacity=0.5] (2.8801, 0.4201, 3.0100) -- (2.8801, 0.4741, 3.0154) -- (2.8800, 0.4740, 3.0654) -- (2.8800, 0.4200, 3.0600) -- cycle;
\fill[blue!15.0, opacity=0.5] (3.0000, 0.5500, 0.0654) -- (3.0000, 0.6000, 0.0705) -- (2.9999, 0.6000, 0.1205) -- (2.9999, 0.5499, 0.1154) -- cycle;
\fill[blue!15.0, opacity=0.5] (2.9999, 0.5499, 0.1154) -- (2.9999, 0.6000, 0.1205) -- (2.9997, 0.5998, 0.1705) -- (2.9997, 0.5498, 0.1654) -- cycle;
\fill[blue!15.0, opacity=0.5] (2.9997, 0.5498, 0.1654) -- (2.9997, 0.5998, 0.1705) -- (2.9993, 0.5996, 0.2205) -- (2.9993, 0.5495, 0.2154) -- cycle;
\fill[blue!15.0, opacity=0.5] (2.9993, 0.5495, 0.2154) -- (2.9993, 0.5996, 0.2205) -- (2.9987, 0.5992, 0.2705) -- (2.9987, 0.5492, 0.2654) -- cycle;
\fill[blue!15.0, opacity=0.5] (2.9987, 0.5492, 0.2654) -- (2.9987, 0.5992, 0.2705) -- (2.9980, 0.5988, 0.3205) -- (2.9980, 0.5487, 0.3154) -- cycle;
\fill[blue!15.0, opacity=0.5] (2.9980, 0.5487, 0.3154) -- (2.9980, 0.5988, 0.3205) -- (2.9971, 0.5982, 0.3705) -- (2.9971, 0.5481, 0.3654) -- cycle;
\fill[blue!15.0, opacity=0.5] (2.9971, 0.5481, 0.3654) -- (2.9971, 0.5982, 0.3705) -- (2.9960, 0.5976, 0.4205) -- (2.9960, 0.5475, 0.4154) -- cycle;
\fill[blue!15.0, opacity=0.5] (2.9960, 0.5475, 0.4154) -- (2.9960, 0.5976, 0.4205) -- (2.9948, 0.5969, 0.4705) -- (2.9948, 0.5467, 0.4654) -- cycle;
\fill[blue!15.0, opacity=0.5] (2.9948, 0.5467, 0.4654) -- (2.9948, 0.5969, 0.4705) -- (2.9935, 0.5961, 0.5205) -- (2.9935, 0.5459, 0.5154) -- cycle;
\fill[blue!15.0, opacity=0.5] (2.9935, 0.5459, 0.5154) -- (2.9935, 0.5961, 0.5205) -- (2.9920, 0.5952, 0.5705) -- (2.9920, 0.5449, 0.5654) -- cycle;
\fill[blue!15.0, opacity=0.5] (2.9920, 0.5449, 0.5654) -- (2.9920, 0.5952, 0.5705) -- (2.9903, 0.5942, 0.6205) -- (2.9903, 0.5439, 0.6154) -- cycle;
\fill[blue!15.0, opacity=0.5] (2.9903, 0.5439, 0.6154) -- (2.9903, 0.5942, 0.6205) -- (2.9885, 0.5931, 0.6705) -- (2.9885, 0.5427, 0.6654) -- cycle;
\fill[blue!15.0, opacity=0.5] (2.9885, 0.5427, 0.6654) -- (2.9885, 0.5931, 0.6705) -- (2.9866, 0.5920, 0.7205) -- (2.9866, 0.5415, 0.7154) -- cycle;
\fill[blue!15.0, opacity=0.5] (2.9866, 0.5415, 0.7154) -- (2.9866, 0.5920, 0.7205) -- (2.9846, 0.5908, 0.7705) -- (2.9846, 0.5402, 0.7654) -- cycle;
\fill[blue!15.0, opacity=0.5] (2.9846, 0.5402, 0.7654) -- (2.9846, 0.5908, 0.7705) -- (2.9824, 0.5895, 0.8205) -- (2.9824, 0.5389, 0.8154) -- cycle;
\fill[blue!15.0, opacity=0.5] (2.9824, 0.5389, 0.8154) -- (2.9824, 0.5895, 0.8205) -- (2.9801, 0.5881, 0.8705) -- (2.9801, 0.5374, 0.8654) -- cycle;
\fill[blue!15.0, opacity=0.5] (2.9801, 0.5374, 0.8654) -- (2.9801, 0.5881, 0.8705) -- (2.9778, 0.5867, 0.9205) -- (2.9778, 0.5359, 0.9154) -- cycle;
\fill[blue!15.0, opacity=0.5] (2.9778, 0.5359, 0.9154) -- (2.9778, 0.5867, 0.9205) -- (2.9753, 0.5852, 0.9705) -- (2.9753, 0.5343, 0.9654) -- cycle;
\fill[blue!15.0, opacity=0.5] (2.9753, 0.5343, 0.9654) -- (2.9753, 0.5852, 0.9705) -- (2.9727, 0.5836, 1.0205) -- (2.9727, 0.5327, 1.0154) -- cycle;
\fill[blue!15.0, opacity=0.5] (2.9727, 0.5327, 1.0154) -- (2.9727, 0.5836, 1.0205) -- (2.9700, 0.5820, 1.0705) -- (2.9700, 0.5310, 1.0654) -- cycle;
\fill[blue!15.0, opacity=0.5] (2.9700, 0.5310, 1.0654) -- (2.9700, 0.5820, 1.0705) -- (2.9672, 0.5803, 1.1205) -- (2.9672, 0.5293, 1.1154) -- cycle;
\fill[blue!15.0, opacity=0.5] (2.9672, 0.5293, 1.1154) -- (2.9672, 0.5803, 1.1205) -- (2.9644, 0.5786, 1.1705) -- (2.9644, 0.5275, 1.1654) -- cycle;
\fill[blue!15.0, opacity=0.5] (2.9644, 0.5275, 1.1654) -- (2.9644, 0.5786, 1.1705) -- (2.9615, 0.5769, 1.2205) -- (2.9615, 0.5256, 1.2154) -- cycle;
\fill[blue!15.0, opacity=0.5] (2.9615, 0.5256, 1.2154) -- (2.9615, 0.5769, 1.2205) -- (2.9585, 0.5751, 1.2705) -- (2.9585, 0.5237, 1.2654) -- cycle;
\fill[blue!15.0, opacity=0.5] (2.9585, 0.5237, 1.2654) -- (2.9585, 0.5751, 1.2705) -- (2.9555, 0.5733, 1.3205) -- (2.9555, 0.5218, 1.3154) -- cycle;
\fill[blue!15.0, opacity=0.5] (2.9555, 0.5218, 1.3154) -- (2.9555, 0.5733, 1.3205) -- (2.9525, 0.5715, 1.3705) -- (2.9525, 0.5199, 1.3654) -- cycle;
\fill[blue!15.0, opacity=0.5] (2.9525, 0.5199, 1.3654) -- (2.9525, 0.5715, 1.3705) -- (2.9494, 0.5696, 1.4205) -- (2.9494, 0.5179, 1.4154) -- cycle;
\fill[blue!15.0, opacity=0.5] (2.9494, 0.5179, 1.4154) -- (2.9494, 0.5696, 1.4205) -- (2.9463, 0.5678, 1.4705) -- (2.9463, 0.5160, 1.4654) -- cycle;
\fill[blue!15.0, opacity=0.5] (2.9463, 0.5160, 1.4654) -- (2.9463, 0.5678, 1.4705) -- (2.9431, 0.5659, 1.5205) -- (2.9431, 0.5140, 1.5154) -- cycle;
\fill[blue!15.0, opacity=0.5] (2.9431, 0.5140, 1.5154) -- (2.9431, 0.5659, 1.5205) -- (2.9400, 0.5640, 1.5705) -- (2.9400, 0.5120, 1.5654) -- cycle;
\fill[blue!15.0, opacity=0.5] (2.9400, 0.5120, 1.5654) -- (2.9400, 0.5640, 1.5705) -- (2.9369, 0.5621, 1.6205) -- (2.9369, 0.5100, 1.6154) -- cycle;
\fill[blue!15.0, opacity=0.5] (2.9369, 0.5100, 1.6154) -- (2.9369, 0.5621, 1.6205) -- (2.9337, 0.5602, 1.6705) -- (2.9337, 0.5080, 1.6654) -- cycle;
\fill[blue!15.1, opacity=0.5] (2.9337, 0.5080, 1.6654) -- (2.9337, 0.5602, 1.6705) -- (2.9306, 0.5584, 1.7205) -- (2.9306, 0.5061, 1.7154) -- cycle;
\fill[blue!15.1, opacity=0.5] (2.9306, 0.5061, 1.7154) -- (2.9306, 0.5584, 1.7205) -- (2.9275, 0.5565, 1.7705) -- (2.9275, 0.5041, 1.7654) -- cycle;
\fill[blue!15.1, opacity=0.5] (2.9275, 0.5041, 1.7654) -- (2.9275, 0.5565, 1.7705) -- (2.9245, 0.5547, 1.8205) -- (2.9245, 0.5022, 1.8154) -- cycle;
\fill[blue!15.2, opacity=0.5] (2.9245, 0.5022, 1.8154) -- (2.9245, 0.5547, 1.8205) -- (2.9215, 0.5529, 1.8705) -- (2.9215, 0.5003, 1.8654) -- cycle;
\fill[blue!15.3, opacity=0.5] (2.9215, 0.5003, 1.8654) -- (2.9215, 0.5529, 1.8705) -- (2.9185, 0.5511, 1.9205) -- (2.9185, 0.4984, 1.9154) -- cycle;
\fill[blue!15.4, opacity=0.5] (2.9185, 0.4984, 1.9154) -- (2.9185, 0.5511, 1.9205) -- (2.9156, 0.5494, 1.9705) -- (2.9156, 0.4965, 1.9654) -- cycle;
\fill[blue!15.5, opacity=0.5] (2.9156, 0.4965, 1.9654) -- (2.9156, 0.5494, 1.9705) -- (2.9128, 0.5477, 2.0205) -- (2.9128, 0.4947, 2.0154) -- cycle;
\fill[blue!15.7, opacity=0.5] (2.9128, 0.4947, 2.0154) -- (2.9128, 0.5477, 2.0205) -- (2.9100, 0.5460, 2.0705) -- (2.9100, 0.4930, 2.0654) -- cycle;
\fill[blue!15.9, opacity=0.5] (2.9100, 0.4930, 2.0654) -- (2.9100, 0.5460, 2.0705) -- (2.9073, 0.5444, 2.1205) -- (2.9073, 0.4913, 2.1154) -- cycle;
\fill[blue!16.1, opacity=0.5] (2.9073, 0.4913, 2.1154) -- (2.9073, 0.5444, 2.1205) -- (2.9047, 0.5428, 2.1705) -- (2.9047, 0.4897, 2.1654) -- cycle;
\fill[blue!16.4, opacity=0.5] (2.9047, 0.4897, 2.1654) -- (2.9047, 0.5428, 2.1705) -- (2.9022, 0.5413, 2.2205) -- (2.9022, 0.4881, 2.2154) -- cycle;
\fill[blue!16.7, opacity=0.5] (2.9022, 0.4881, 2.2154) -- (2.9022, 0.5413, 2.2205) -- (2.8999, 0.5399, 2.2705) -- (2.8999, 0.4866, 2.2654) -- cycle;
\fill[blue!17.1, opacity=0.5] (2.8999, 0.4866, 2.2654) -- (2.8999, 0.5399, 2.2705) -- (2.8976, 0.5385, 2.3205) -- (2.8976, 0.4851, 2.3154) -- cycle;
\fill[blue!17.6, opacity=0.5] (2.8976, 0.4851, 2.3154) -- (2.8976, 0.5385, 2.3205) -- (2.8954, 0.5372, 2.3705) -- (2.8954, 0.4838, 2.3654) -- cycle;
\fill[blue!18.1, opacity=0.5] (2.8954, 0.4838, 2.3654) -- (2.8954, 0.5372, 2.3705) -- (2.8934, 0.5360, 2.4205) -- (2.8934, 0.4825, 2.4154) -- cycle;
\fill[blue!18.7, opacity=0.5] (2.8934, 0.4825, 2.4154) -- (2.8934, 0.5360, 2.4205) -- (2.8915, 0.5349, 2.4705) -- (2.8915, 0.4813, 2.4654) -- cycle;
\fill[blue!19.3, opacity=0.5] (2.8915, 0.4813, 2.4654) -- (2.8915, 0.5349, 2.4705) -- (2.8897, 0.5338, 2.5205) -- (2.8897, 0.4801, 2.5154) -- cycle;
\fill[blue!20.0, opacity=0.5] (2.8897, 0.4801, 2.5154) -- (2.8897, 0.5338, 2.5205) -- (2.8880, 0.5328, 2.5705) -- (2.8880, 0.4791, 2.5654) -- cycle;
\fill[blue!20.7, opacity=0.5] (2.8880, 0.4791, 2.5654) -- (2.8880, 0.5328, 2.5705) -- (2.8865, 0.5319, 2.6205) -- (2.8865, 0.4781, 2.6154) -- cycle;
\fill[blue!21.5, opacity=0.5] (2.8865, 0.4781, 2.6154) -- (2.8865, 0.5319, 2.6205) -- (2.8852, 0.5311, 2.6705) -- (2.8852, 0.4773, 2.6654) -- cycle;
\fill[blue!22.3, opacity=0.5] (2.8852, 0.4773, 2.6654) -- (2.8852, 0.5311, 2.6705) -- (2.8840, 0.5304, 2.7205) -- (2.8840, 0.4765, 2.7154) -- cycle;
\fill[blue!23.2, opacity=0.5] (2.8840, 0.4765, 2.7154) -- (2.8840, 0.5304, 2.7205) -- (2.8829, 0.5298, 2.7705) -- (2.8829, 0.4759, 2.7654) -- cycle;
\fill[blue!24.1, opacity=0.5] (2.8829, 0.4759, 2.7654) -- (2.8829, 0.5298, 2.7705) -- (2.8820, 0.5292, 2.8205) -- (2.8820, 0.4753, 2.8154) -- cycle;
\fill[blue!25.0, opacity=0.5] (2.8820, 0.4753, 2.8154) -- (2.8820, 0.5292, 2.8205) -- (2.8813, 0.5288, 2.8705) -- (2.8813, 0.4748, 2.8654) -- cycle;
\fill[blue!26.0, opacity=0.5] (2.8813, 0.4748, 2.8654) -- (2.8813, 0.5288, 2.8705) -- (2.8807, 0.5284, 2.9205) -- (2.8807, 0.4745, 2.9154) -- cycle;
\fill[blue!26.9, opacity=0.5] (2.8807, 0.4745, 2.9154) -- (2.8807, 0.5284, 2.9205) -- (2.8803, 0.5282, 2.9705) -- (2.8803, 0.4742, 2.9654) -- cycle;
\fill[blue!27.8, opacity=0.5] (2.8803, 0.4742, 2.9654) -- (2.8803, 0.5282, 2.9705) -- (2.8801, 0.5280, 3.0205) -- (2.8801, 0.4741, 3.0154) -- cycle;
\fill[blue!28.8, opacity=0.5] (2.8801, 0.4741, 3.0154) -- (2.8801, 0.5280, 3.0205) -- (2.8800, 0.5280, 3.0705) -- (2.8800, 0.4740, 3.0654) -- cycle;
\fill[blue!15.0, opacity=0.5] (3.0000, 0.6000, 0.0705) -- (3.0000, 0.6500, 0.0755) -- (2.9999, 0.6500, 0.1255) -- (2.9999, 0.6000, 0.1205) -- cycle;
\fill[blue!15.0, opacity=0.5] (2.9999, 0.6000, 0.1205) -- (2.9999, 0.6500, 0.1255) -- (2.9997, 0.6498, 0.1755) -- (2.9997, 0.5998, 0.1705) -- cycle;
\fill[blue!15.0, opacity=0.5] (2.9997, 0.5998, 0.1705) -- (2.9997, 0.6498, 0.1755) -- (2.9993, 0.6496, 0.2255) -- (2.9993, 0.5996, 0.2205) -- cycle;
\fill[blue!15.0, opacity=0.5] (2.9993, 0.5996, 0.2205) -- (2.9993, 0.6496, 0.2255) -- (2.9987, 0.6493, 0.2755) -- (2.9987, 0.5992, 0.2705) -- cycle;
\fill[blue!15.0, opacity=0.5] (2.9987, 0.5992, 0.2705) -- (2.9987, 0.6493, 0.2755) -- (2.9980, 0.6488, 0.3255) -- (2.9980, 0.5988, 0.3205) -- cycle;
\fill[blue!15.0, opacity=0.5] (2.9980, 0.5988, 0.3205) -- (2.9980, 0.6488, 0.3255) -- (2.9971, 0.6483, 0.3755) -- (2.9971, 0.5982, 0.3705) -- cycle;
\fill[blue!15.0, opacity=0.5] (2.9971, 0.5982, 0.3705) -- (2.9971, 0.6483, 0.3755) -- (2.9960, 0.6477, 0.4255) -- (2.9960, 0.5976, 0.4205) -- cycle;
\fill[blue!15.0, opacity=0.5] (2.9960, 0.5976, 0.4205) -- (2.9960, 0.6477, 0.4255) -- (2.9948, 0.6471, 0.4755) -- (2.9948, 0.5969, 0.4705) -- cycle;
\fill[blue!15.0, opacity=0.5] (2.9948, 0.5969, 0.4705) -- (2.9948, 0.6471, 0.4755) -- (2.9935, 0.6463, 0.5255) -- (2.9935, 0.5961, 0.5205) -- cycle;
\fill[blue!15.0, opacity=0.5] (2.9935, 0.5961, 0.5205) -- (2.9935, 0.6463, 0.5255) -- (2.9920, 0.6454, 0.5755) -- (2.9920, 0.5952, 0.5705) -- cycle;
\fill[blue!15.0, opacity=0.5] (2.9920, 0.5952, 0.5705) -- (2.9920, 0.6454, 0.5755) -- (2.9903, 0.6445, 0.6255) -- (2.9903, 0.5942, 0.6205) -- cycle;
\fill[blue!15.0, opacity=0.5] (2.9903, 0.5942, 0.6205) -- (2.9903, 0.6445, 0.6255) -- (2.9885, 0.6435, 0.6755) -- (2.9885, 0.5931, 0.6705) -- cycle;
\fill[blue!15.0, opacity=0.5] (2.9885, 0.5931, 0.6705) -- (2.9885, 0.6435, 0.6755) -- (2.9866, 0.6424, 0.7255) -- (2.9866, 0.5920, 0.7205) -- cycle;
\fill[blue!15.0, opacity=0.5] (2.9866, 0.5920, 0.7205) -- (2.9866, 0.6424, 0.7255) -- (2.9846, 0.6413, 0.7755) -- (2.9846, 0.5908, 0.7705) -- cycle;
\fill[blue!15.0, opacity=0.5] (2.9846, 0.5908, 0.7705) -- (2.9846, 0.6413, 0.7755) -- (2.9824, 0.6400, 0.8255) -- (2.9824, 0.5895, 0.8205) -- cycle;
\fill[blue!15.0, opacity=0.5] (2.9824, 0.5895, 0.8205) -- (2.9824, 0.6400, 0.8255) -- (2.9801, 0.6388, 0.8755) -- (2.9801, 0.5881, 0.8705) -- cycle;
\fill[blue!15.0, opacity=0.5] (2.9801, 0.5881, 0.8705) -- (2.9801, 0.6388, 0.8755) -- (2.9778, 0.6374, 0.9255) -- (2.9778, 0.5867, 0.9205) -- cycle;
\fill[blue!15.0, opacity=0.5] (2.9778, 0.5867, 0.9205) -- (2.9778, 0.6374, 0.9255) -- (2.9753, 0.6360, 0.9755) -- (2.9753, 0.5852, 0.9705) -- cycle;
\fill[blue!15.0, opacity=0.5] (2.9753, 0.5852, 0.9705) -- (2.9753, 0.6360, 0.9755) -- (2.9727, 0.6345, 1.0255) -- (2.9727, 0.5836, 1.0205) -- cycle;
\fill[blue!15.0, opacity=0.5] (2.9727, 0.5836, 1.0205) -- (2.9727, 0.6345, 1.0255) -- (2.9700, 0.6330, 1.0755) -- (2.9700, 0.5820, 1.0705) -- cycle;
\fill[blue!15.0, opacity=0.5] (2.9700, 0.5820, 1.0705) -- (2.9700, 0.6330, 1.0755) -- (2.9672, 0.6314, 1.1255) -- (2.9672, 0.5803, 1.1205) -- cycle;
\fill[blue!15.0, opacity=0.5] (2.9672, 0.5803, 1.1205) -- (2.9672, 0.6314, 1.1255) -- (2.9644, 0.6298, 1.1755) -- (2.9644, 0.5786, 1.1705) -- cycle;
\fill[blue!15.0, opacity=0.5] (2.9644, 0.5786, 1.1705) -- (2.9644, 0.6298, 1.1755) -- (2.9615, 0.6282, 1.2255) -- (2.9615, 0.5769, 1.2205) -- cycle;
\fill[blue!15.0, opacity=0.5] (2.9615, 0.5769, 1.2205) -- (2.9615, 0.6282, 1.2255) -- (2.9585, 0.6265, 1.2755) -- (2.9585, 0.5751, 1.2705) -- cycle;
\fill[blue!15.0, opacity=0.5] (2.9585, 0.5751, 1.2705) -- (2.9585, 0.6265, 1.2755) -- (2.9555, 0.6248, 1.3255) -- (2.9555, 0.5733, 1.3205) -- cycle;
\fill[blue!15.0, opacity=0.5] (2.9555, 0.5733, 1.3205) -- (2.9555, 0.6248, 1.3255) -- (2.9525, 0.6231, 1.3755) -- (2.9525, 0.5715, 1.3705) -- cycle;
\fill[blue!15.0, opacity=0.5] (2.9525, 0.5715, 1.3705) -- (2.9525, 0.6231, 1.3755) -- (2.9494, 0.6213, 1.4255) -- (2.9494, 0.5696, 1.4205) -- cycle;
\fill[blue!15.0, opacity=0.5] (2.9494, 0.5696, 1.4205) -- (2.9494, 0.6213, 1.4255) -- (2.9463, 0.6196, 1.4755) -- (2.9463, 0.5678, 1.4705) -- cycle;
\fill[blue!15.0, opacity=0.5] (2.9463, 0.5678, 1.4705) -- (2.9463, 0.6196, 1.4755) -- (2.9431, 0.6178, 1.5255) -- (2.9431, 0.5659, 1.5205) -- cycle;
\fill[blue!15.0, opacity=0.5] (2.9431, 0.5659, 1.5205) -- (2.9431, 0.6178, 1.5255) -- (2.9400, 0.6160, 1.5755) -- (2.9400, 0.5640, 1.5705) -- cycle;
\fill[blue!15.0, opacity=0.5] (2.9400, 0.5640, 1.5705) -- (2.9400, 0.6160, 1.5755) -- (2.9369, 0.6142, 1.6255) -- (2.9369, 0.5621, 1.6205) -- cycle;
\fill[blue!15.0, opacity=0.5] (2.9369, 0.5621, 1.6205) -- (2.9369, 0.6142, 1.6255) -- (2.9337, 0.6124, 1.6755) -- (2.9337, 0.5602, 1.6705) -- cycle;
\fill[blue!15.0, opacity=0.5] (2.9337, 0.5602, 1.6705) -- (2.9337, 0.6124, 1.6755) -- (2.9306, 0.6107, 1.7255) -- (2.9306, 0.5584, 1.7205) -- cycle;
\fill[blue!15.0, opacity=0.5] (2.9306, 0.5584, 1.7205) -- (2.9306, 0.6107, 1.7255) -- (2.9275, 0.6089, 1.7755) -- (2.9275, 0.5565, 1.7705) -- cycle;
\fill[blue!15.0, opacity=0.5] (2.9275, 0.5565, 1.7705) -- (2.9275, 0.6089, 1.7755) -- (2.9245, 0.6072, 1.8255) -- (2.9245, 0.5547, 1.8205) -- cycle;
\fill[blue!15.0, opacity=0.5] (2.9245, 0.5547, 1.8205) -- (2.9245, 0.6072, 1.8255) -- (2.9215, 0.6055, 1.8755) -- (2.9215, 0.5529, 1.8705) -- cycle;
\fill[blue!15.1, opacity=0.5] (2.9215, 0.5529, 1.8705) -- (2.9215, 0.6055, 1.8755) -- (2.9185, 0.6038, 1.9255) -- (2.9185, 0.5511, 1.9205) -- cycle;
\fill[blue!15.1, opacity=0.5] (2.9185, 0.5511, 1.9205) -- (2.9185, 0.6038, 1.9255) -- (2.9156, 0.6022, 1.9755) -- (2.9156, 0.5494, 1.9705) -- cycle;
\fill[blue!15.1, opacity=0.5] (2.9156, 0.5494, 1.9705) -- (2.9156, 0.6022, 1.9755) -- (2.9128, 0.6006, 2.0255) -- (2.9128, 0.5477, 2.0205) -- cycle;
\fill[blue!15.2, opacity=0.5] (2.9128, 0.5477, 2.0205) -- (2.9128, 0.6006, 2.0255) -- (2.9100, 0.5990, 2.0755) -- (2.9100, 0.5460, 2.0705) -- cycle;
\fill[blue!15.2, opacity=0.5] (2.9100, 0.5460, 2.0705) -- (2.9100, 0.5990, 2.0755) -- (2.9073, 0.5975, 2.1255) -- (2.9073, 0.5444, 2.1205) -- cycle;
\fill[blue!15.3, opacity=0.5] (2.9073, 0.5444, 2.1205) -- (2.9073, 0.5975, 2.1255) -- (2.9047, 0.5960, 2.1755) -- (2.9047, 0.5428, 2.1705) -- cycle;
\fill[blue!15.4, opacity=0.5] (2.9047, 0.5428, 2.1705) -- (2.9047, 0.5960, 2.1755) -- (2.9022, 0.5946, 2.2255) -- (2.9022, 0.5413, 2.2205) -- cycle;
\fill[blue!15.5, opacity=0.5] (2.9022, 0.5413, 2.2205) -- (2.9022, 0.5946, 2.2255) -- (2.8999, 0.5932, 2.2755) -- (2.8999, 0.5399, 2.2705) -- cycle;
\fill[blue!15.7, opacity=0.5] (2.8999, 0.5399, 2.2705) -- (2.8999, 0.5932, 2.2755) -- (2.8976, 0.5920, 2.3255) -- (2.8976, 0.5385, 2.3205) -- cycle;
\fill[blue!15.9, opacity=0.5] (2.8976, 0.5385, 2.3205) -- (2.8976, 0.5920, 2.3255) -- (2.8954, 0.5907, 2.3755) -- (2.8954, 0.5372, 2.3705) -- cycle;
\fill[blue!16.1, opacity=0.5] (2.8954, 0.5372, 2.3705) -- (2.8954, 0.5907, 2.3755) -- (2.8934, 0.5896, 2.4255) -- (2.8934, 0.5360, 2.4205) -- cycle;
\fill[blue!16.3, opacity=0.5] (2.8934, 0.5360, 2.4205) -- (2.8934, 0.5896, 2.4255) -- (2.8915, 0.5885, 2.4755) -- (2.8915, 0.5349, 2.4705) -- cycle;
\fill[blue!16.6, opacity=0.5] (2.8915, 0.5349, 2.4705) -- (2.8915, 0.5885, 2.4755) -- (2.8897, 0.5875, 2.5255) -- (2.8897, 0.5338, 2.5205) -- cycle;
\fill[blue!17.0, opacity=0.5] (2.8897, 0.5338, 2.5205) -- (2.8897, 0.5875, 2.5255) -- (2.8880, 0.5866, 2.5755) -- (2.8880, 0.5328, 2.5705) -- cycle;
\fill[blue!17.4, opacity=0.5] (2.8880, 0.5328, 2.5705) -- (2.8880, 0.5866, 2.5755) -- (2.8865, 0.5857, 2.6255) -- (2.8865, 0.5319, 2.6205) -- cycle;
\fill[blue!17.8, opacity=0.5] (2.8865, 0.5319, 2.6205) -- (2.8865, 0.5857, 2.6255) -- (2.8852, 0.5849, 2.6755) -- (2.8852, 0.5311, 2.6705) -- cycle;
\fill[blue!18.3, opacity=0.5] (2.8852, 0.5311, 2.6705) -- (2.8852, 0.5849, 2.6755) -- (2.8840, 0.5843, 2.7255) -- (2.8840, 0.5304, 2.7205) -- cycle;
\fill[blue!18.8, opacity=0.5] (2.8840, 0.5304, 2.7205) -- (2.8840, 0.5843, 2.7255) -- (2.8829, 0.5837, 2.7755) -- (2.8829, 0.5298, 2.7705) -- cycle;
\fill[blue!19.3, opacity=0.5] (2.8829, 0.5298, 2.7705) -- (2.8829, 0.5837, 2.7755) -- (2.8820, 0.5832, 2.8255) -- (2.8820, 0.5292, 2.8205) -- cycle;
\fill[blue!19.9, opacity=0.5] (2.8820, 0.5292, 2.8205) -- (2.8820, 0.5832, 2.8255) -- (2.8813, 0.5827, 2.8755) -- (2.8813, 0.5288, 2.8705) -- cycle;
\fill[blue!20.6, opacity=0.5] (2.8813, 0.5288, 2.8705) -- (2.8813, 0.5827, 2.8755) -- (2.8807, 0.5824, 2.9255) -- (2.8807, 0.5284, 2.9205) -- cycle;
\fill[blue!21.2, opacity=0.5] (2.8807, 0.5284, 2.9205) -- (2.8807, 0.5824, 2.9255) -- (2.8803, 0.5822, 2.9755) -- (2.8803, 0.5282, 2.9705) -- cycle;
\fill[blue!21.9, opacity=0.5] (2.8803, 0.5282, 2.9705) -- (2.8803, 0.5822, 2.9755) -- (2.8801, 0.5820, 3.0255) -- (2.8801, 0.5280, 3.0205) -- cycle;
\fill[blue!22.6, opacity=0.5] (2.8801, 0.5280, 3.0205) -- (2.8801, 0.5820, 3.0255) -- (2.8800, 0.5820, 3.0755) -- (2.8800, 0.5280, 3.0705) -- cycle;
\fill[blue!15.0, opacity=0.5] (3.0000, 0.6500, 0.0755) -- (3.0000, 0.7000, 0.0803) -- (2.9999, 0.7000, 0.1303) -- (2.9999, 0.6500, 0.1255) -- cycle;
\fill[blue!15.0, opacity=0.5] (2.9999, 0.6500, 0.1255) -- (2.9999, 0.7000, 0.1303) -- (2.9997, 0.6998, 0.1803) -- (2.9997, 0.6498, 0.1755) -- cycle;
\fill[blue!15.0, opacity=0.5] (2.9997, 0.6498, 0.1755) -- (2.9997, 0.6998, 0.1803) -- (2.9993, 0.6996, 0.2303) -- (2.9993, 0.6496, 0.2255) -- cycle;
\fill[blue!15.0, opacity=0.5] (2.9993, 0.6496, 0.2255) -- (2.9993, 0.6996, 0.2303) -- (2.9987, 0.6993, 0.2803) -- (2.9987, 0.6493, 0.2755) -- cycle;
\fill[blue!15.0, opacity=0.5] (2.9987, 0.6493, 0.2755) -- (2.9987, 0.6993, 0.2803) -- (2.9980, 0.6989, 0.3303) -- (2.9980, 0.6488, 0.3255) -- cycle;
\fill[blue!15.0, opacity=0.5] (2.9980, 0.6488, 0.3255) -- (2.9980, 0.6989, 0.3303) -- (2.9971, 0.6984, 0.3803) -- (2.9971, 0.6483, 0.3755) -- cycle;
\fill[blue!15.0, opacity=0.5] (2.9971, 0.6483, 0.3755) -- (2.9971, 0.6984, 0.3803) -- (2.9960, 0.6979, 0.4303) -- (2.9960, 0.6477, 0.4255) -- cycle;
\fill[blue!15.0, opacity=0.5] (2.9960, 0.6477, 0.4255) -- (2.9960, 0.6979, 0.4303) -- (2.9948, 0.6972, 0.4803) -- (2.9948, 0.6471, 0.4755) -- cycle;
\fill[blue!15.0, opacity=0.5] (2.9948, 0.6471, 0.4755) -- (2.9948, 0.6972, 0.4803) -- (2.9935, 0.6965, 0.5303) -- (2.9935, 0.6463, 0.5255) -- cycle;
\fill[blue!15.0, opacity=0.5] (2.9935, 0.6463, 0.5255) -- (2.9935, 0.6965, 0.5303) -- (2.9920, 0.6957, 0.5803) -- (2.9920, 0.6454, 0.5755) -- cycle;
\fill[blue!15.0, opacity=0.5] (2.9920, 0.6454, 0.5755) -- (2.9920, 0.6957, 0.5803) -- (2.9903, 0.6948, 0.6303) -- (2.9903, 0.6445, 0.6255) -- cycle;
\fill[blue!15.0, opacity=0.5] (2.9903, 0.6445, 0.6255) -- (2.9903, 0.6948, 0.6303) -- (2.9885, 0.6939, 0.6803) -- (2.9885, 0.6435, 0.6755) -- cycle;
\fill[blue!15.0, opacity=0.5] (2.9885, 0.6435, 0.6755) -- (2.9885, 0.6939, 0.6803) -- (2.9866, 0.6929, 0.7303) -- (2.9866, 0.6424, 0.7255) -- cycle;
\fill[blue!15.0, opacity=0.5] (2.9866, 0.6424, 0.7255) -- (2.9866, 0.6929, 0.7303) -- (2.9846, 0.6918, 0.7803) -- (2.9846, 0.6413, 0.7755) -- cycle;
\fill[blue!15.0, opacity=0.5] (2.9846, 0.6413, 0.7755) -- (2.9846, 0.6918, 0.7803) -- (2.9824, 0.6906, 0.8303) -- (2.9824, 0.6400, 0.8255) -- cycle;
\fill[blue!15.0, opacity=0.5] (2.9824, 0.6400, 0.8255) -- (2.9824, 0.6906, 0.8303) -- (2.9801, 0.6894, 0.8803) -- (2.9801, 0.6388, 0.8755) -- cycle;
\fill[blue!15.0, opacity=0.5] (2.9801, 0.6388, 0.8755) -- (2.9801, 0.6894, 0.8803) -- (2.9778, 0.6881, 0.9303) -- (2.9778, 0.6374, 0.9255) -- cycle;
\fill[blue!15.0, opacity=0.5] (2.9778, 0.6374, 0.9255) -- (2.9778, 0.6881, 0.9303) -- (2.9753, 0.6868, 0.9803) -- (2.9753, 0.6360, 0.9755) -- cycle;
\fill[blue!15.0, opacity=0.5] (2.9753, 0.6360, 0.9755) -- (2.9753, 0.6868, 0.9803) -- (2.9727, 0.6854, 1.0303) -- (2.9727, 0.6345, 1.0255) -- cycle;
\fill[blue!15.0, opacity=0.5] (2.9727, 0.6345, 1.0255) -- (2.9727, 0.6854, 1.0303) -- (2.9700, 0.6840, 1.0803) -- (2.9700, 0.6330, 1.0755) -- cycle;
\fill[blue!15.0, opacity=0.5] (2.9700, 0.6330, 1.0755) -- (2.9700, 0.6840, 1.0803) -- (2.9672, 0.6825, 1.1303) -- (2.9672, 0.6314, 1.1255) -- cycle;
\fill[blue!15.0, opacity=0.5] (2.9672, 0.6314, 1.1255) -- (2.9672, 0.6825, 1.1303) -- (2.9644, 0.6810, 1.1803) -- (2.9644, 0.6298, 1.1755) -- cycle;
\fill[blue!15.0, opacity=0.5] (2.9644, 0.6298, 1.1755) -- (2.9644, 0.6810, 1.1803) -- (2.9615, 0.6795, 1.2303) -- (2.9615, 0.6282, 1.2255) -- cycle;
\fill[blue!15.0, opacity=0.5] (2.9615, 0.6282, 1.2255) -- (2.9615, 0.6795, 1.2303) -- (2.9585, 0.6779, 1.2803) -- (2.9585, 0.6265, 1.2755) -- cycle;
\fill[blue!15.0, opacity=0.5] (2.9585, 0.6265, 1.2755) -- (2.9585, 0.6779, 1.2803) -- (2.9555, 0.6763, 1.3303) -- (2.9555, 0.6248, 1.3255) -- cycle;
\fill[blue!15.0, opacity=0.5] (2.9555, 0.6248, 1.3255) -- (2.9555, 0.6763, 1.3303) -- (2.9525, 0.6747, 1.3803) -- (2.9525, 0.6231, 1.3755) -- cycle;
\fill[blue!15.0, opacity=0.5] (2.9525, 0.6231, 1.3755) -- (2.9525, 0.6747, 1.3803) -- (2.9494, 0.6730, 1.4303) -- (2.9494, 0.6213, 1.4255) -- cycle;
\fill[blue!15.0, opacity=0.5] (2.9494, 0.6213, 1.4255) -- (2.9494, 0.6730, 1.4303) -- (2.9463, 0.6713, 1.4803) -- (2.9463, 0.6196, 1.4755) -- cycle;
\fill[blue!15.0, opacity=0.5] (2.9463, 0.6196, 1.4755) -- (2.9463, 0.6713, 1.4803) -- (2.9431, 0.6697, 1.5303) -- (2.9431, 0.6178, 1.5255) -- cycle;
\fill[blue!15.0, opacity=0.5] (2.9431, 0.6178, 1.5255) -- (2.9431, 0.6697, 1.5303) -- (2.9400, 0.6680, 1.5803) -- (2.9400, 0.6160, 1.5755) -- cycle;
\fill[blue!15.0, opacity=0.5] (2.9400, 0.6160, 1.5755) -- (2.9400, 0.6680, 1.5803) -- (2.9369, 0.6663, 1.6303) -- (2.9369, 0.6142, 1.6255) -- cycle;
\fill[blue!15.0, opacity=0.5] (2.9369, 0.6142, 1.6255) -- (2.9369, 0.6663, 1.6303) -- (2.9337, 0.6647, 1.6803) -- (2.9337, 0.6124, 1.6755) -- cycle;
\fill[blue!15.0, opacity=0.5] (2.9337, 0.6124, 1.6755) -- (2.9337, 0.6647, 1.6803) -- (2.9306, 0.6630, 1.7303) -- (2.9306, 0.6107, 1.7255) -- cycle;
\fill[blue!15.0, opacity=0.5] (2.9306, 0.6107, 1.7255) -- (2.9306, 0.6630, 1.7303) -- (2.9275, 0.6613, 1.7803) -- (2.9275, 0.6089, 1.7755) -- cycle;
\fill[blue!15.0, opacity=0.5] (2.9275, 0.6089, 1.7755) -- (2.9275, 0.6613, 1.7803) -- (2.9245, 0.6597, 1.8303) -- (2.9245, 0.6072, 1.8255) -- cycle;
\fill[blue!15.0, opacity=0.5] (2.9245, 0.6072, 1.8255) -- (2.9245, 0.6597, 1.8303) -- (2.9215, 0.6581, 1.8803) -- (2.9215, 0.6055, 1.8755) -- cycle;
\fill[blue!15.0, opacity=0.5] (2.9215, 0.6055, 1.8755) -- (2.9215, 0.6581, 1.8803) -- (2.9185, 0.6565, 1.9303) -- (2.9185, 0.6038, 1.9255) -- cycle;
\fill[blue!15.0, opacity=0.5] (2.9185, 0.6038, 1.9255) -- (2.9185, 0.6565, 1.9303) -- (2.9156, 0.6550, 1.9803) -- (2.9156, 0.6022, 1.9755) -- cycle;
\fill[blue!15.0, opacity=0.5] (2.9156, 0.6022, 1.9755) -- (2.9156, 0.6550, 1.9803) -- (2.9128, 0.6535, 2.0303) -- (2.9128, 0.6006, 2.0255) -- cycle;
\fill[blue!15.0, opacity=0.5] (2.9128, 0.6006, 2.0255) -- (2.9128, 0.6535, 2.0303) -- (2.9100, 0.6520, 2.0803) -- (2.9100, 0.5990, 2.0755) -- cycle;
\fill[blue!15.1, opacity=0.5] (2.9100, 0.5990, 2.0755) -- (2.9100, 0.6520, 2.0803) -- (2.9073, 0.6506, 2.1303) -- (2.9073, 0.5975, 2.1255) -- cycle;
\fill[blue!15.1, opacity=0.5] (2.9073, 0.5975, 2.1255) -- (2.9073, 0.6506, 2.1303) -- (2.9047, 0.6492, 2.1803) -- (2.9047, 0.5960, 2.1755) -- cycle;
\fill[blue!15.1, opacity=0.5] (2.9047, 0.5960, 2.1755) -- (2.9047, 0.6492, 2.1803) -- (2.9022, 0.6479, 2.2303) -- (2.9022, 0.5946, 2.2255) -- cycle;
\fill[blue!15.2, opacity=0.5] (2.9022, 0.5946, 2.2255) -- (2.9022, 0.6479, 2.2303) -- (2.8999, 0.6466, 2.2803) -- (2.8999, 0.5932, 2.2755) -- cycle;
\fill[blue!15.2, opacity=0.5] (2.8999, 0.5932, 2.2755) -- (2.8999, 0.6466, 2.2803) -- (2.8976, 0.6454, 2.3303) -- (2.8976, 0.5920, 2.3255) -- cycle;
\fill[blue!15.3, opacity=0.5] (2.8976, 0.5920, 2.3255) -- (2.8976, 0.6454, 2.3303) -- (2.8954, 0.6442, 2.3803) -- (2.8954, 0.5907, 2.3755) -- cycle;
\fill[blue!15.4, opacity=0.5] (2.8954, 0.5907, 2.3755) -- (2.8954, 0.6442, 2.3803) -- (2.8934, 0.6431, 2.4303) -- (2.8934, 0.5896, 2.4255) -- cycle;
\fill[blue!15.5, opacity=0.5] (2.8934, 0.5896, 2.4255) -- (2.8934, 0.6431, 2.4303) -- (2.8915, 0.6421, 2.4803) -- (2.8915, 0.5885, 2.4755) -- cycle;
\fill[blue!15.6, opacity=0.5] (2.8915, 0.5885, 2.4755) -- (2.8915, 0.6421, 2.4803) -- (2.8897, 0.6412, 2.5303) -- (2.8897, 0.5875, 2.5255) -- cycle;
\fill[blue!15.8, opacity=0.5] (2.8897, 0.5875, 2.5255) -- (2.8897, 0.6412, 2.5303) -- (2.8880, 0.6403, 2.5803) -- (2.8880, 0.5866, 2.5755) -- cycle;
\fill[blue!16.0, opacity=0.5] (2.8880, 0.5866, 2.5755) -- (2.8880, 0.6403, 2.5803) -- (2.8865, 0.6395, 2.6303) -- (2.8865, 0.5857, 2.6255) -- cycle;
\fill[blue!16.2, opacity=0.5] (2.8865, 0.5857, 2.6255) -- (2.8865, 0.6395, 2.6303) -- (2.8852, 0.6388, 2.6803) -- (2.8852, 0.5849, 2.6755) -- cycle;
\fill[blue!16.4, opacity=0.5] (2.8852, 0.5849, 2.6755) -- (2.8852, 0.6388, 2.6803) -- (2.8840, 0.6381, 2.7303) -- (2.8840, 0.5843, 2.7255) -- cycle;
\fill[blue!16.7, opacity=0.5] (2.8840, 0.5843, 2.7255) -- (2.8840, 0.6381, 2.7303) -- (2.8829, 0.6376, 2.7803) -- (2.8829, 0.5837, 2.7755) -- cycle;
\fill[blue!17.0, opacity=0.5] (2.8829, 0.5837, 2.7755) -- (2.8829, 0.6376, 2.7803) -- (2.8820, 0.6371, 2.8303) -- (2.8820, 0.5832, 2.8255) -- cycle;
\fill[blue!17.4, opacity=0.5] (2.8820, 0.5832, 2.8255) -- (2.8820, 0.6371, 2.8303) -- (2.8813, 0.6367, 2.8803) -- (2.8813, 0.5827, 2.8755) -- cycle;
\fill[blue!17.7, opacity=0.5] (2.8813, 0.5827, 2.8755) -- (2.8813, 0.6367, 2.8803) -- (2.8807, 0.6364, 2.9303) -- (2.8807, 0.5824, 2.9255) -- cycle;
\fill[blue!18.1, opacity=0.5] (2.8807, 0.5824, 2.9255) -- (2.8807, 0.6364, 2.9303) -- (2.8803, 0.6362, 2.9803) -- (2.8803, 0.5822, 2.9755) -- cycle;
\fill[blue!18.6, opacity=0.5] (2.8803, 0.5822, 2.9755) -- (2.8803, 0.6362, 2.9803) -- (2.8801, 0.6360, 3.0303) -- (2.8801, 0.5820, 3.0255) -- cycle;
\fill[blue!19.1, opacity=0.5] (2.8801, 0.5820, 3.0255) -- (2.8801, 0.6360, 3.0303) -- (2.8800, 0.6360, 3.0803) -- (2.8800, 0.5820, 3.0755) -- cycle;
\fill[blue!15.0, opacity=0.5] (3.0000, 0.7000, 0.0803) -- (3.0000, 0.7500, 0.0849) -- (2.9999, 0.7500, 0.1349) -- (2.9999, 0.7000, 0.1303) -- cycle;
\fill[blue!15.0, opacity=0.5] (2.9999, 0.7000, 0.1303) -- (2.9999, 0.7500, 0.1349) -- (2.9997, 0.7498, 0.1849) -- (2.9997, 0.6998, 0.1803) -- cycle;
\fill[blue!15.0, opacity=0.5] (2.9997, 0.6998, 0.1803) -- (2.9997, 0.7498, 0.1849) -- (2.9993, 0.7496, 0.2349) -- (2.9993, 0.6996, 0.2303) -- cycle;
\fill[blue!15.0, opacity=0.5] (2.9993, 0.6996, 0.2303) -- (2.9993, 0.7496, 0.2349) -- (2.9987, 0.7493, 0.2849) -- (2.9987, 0.6993, 0.2803) -- cycle;
\fill[blue!15.0, opacity=0.5] (2.9987, 0.6993, 0.2803) -- (2.9987, 0.7493, 0.2849) -- (2.9980, 0.7490, 0.3349) -- (2.9980, 0.6989, 0.3303) -- cycle;
\fill[blue!15.0, opacity=0.5] (2.9980, 0.6989, 0.3303) -- (2.9980, 0.7490, 0.3349) -- (2.9971, 0.7485, 0.3849) -- (2.9971, 0.6984, 0.3803) -- cycle;
\fill[blue!15.0, opacity=0.5] (2.9971, 0.6984, 0.3803) -- (2.9971, 0.7485, 0.3849) -- (2.9960, 0.7480, 0.4349) -- (2.9960, 0.6979, 0.4303) -- cycle;
\fill[blue!15.0, opacity=0.5] (2.9960, 0.6979, 0.4303) -- (2.9960, 0.7480, 0.4349) -- (2.9948, 0.7474, 0.4849) -- (2.9948, 0.6972, 0.4803) -- cycle;
\fill[blue!15.0, opacity=0.5] (2.9948, 0.6972, 0.4803) -- (2.9948, 0.7474, 0.4849) -- (2.9935, 0.7467, 0.5349) -- (2.9935, 0.6965, 0.5303) -- cycle;
\fill[blue!15.0, opacity=0.5] (2.9935, 0.6965, 0.5303) -- (2.9935, 0.7467, 0.5349) -- (2.9920, 0.7460, 0.5849) -- (2.9920, 0.6957, 0.5803) -- cycle;
\fill[blue!15.0, opacity=0.5] (2.9920, 0.6957, 0.5803) -- (2.9920, 0.7460, 0.5849) -- (2.9903, 0.7452, 0.6349) -- (2.9903, 0.6948, 0.6303) -- cycle;
\fill[blue!15.0, opacity=0.5] (2.9903, 0.6948, 0.6303) -- (2.9903, 0.7452, 0.6349) -- (2.9885, 0.7443, 0.6849) -- (2.9885, 0.6939, 0.6803) -- cycle;
\fill[blue!15.0, opacity=0.5] (2.9885, 0.6939, 0.6803) -- (2.9885, 0.7443, 0.6849) -- (2.9866, 0.7433, 0.7349) -- (2.9866, 0.6929, 0.7303) -- cycle;
\fill[blue!15.0, opacity=0.5] (2.9866, 0.6929, 0.7303) -- (2.9866, 0.7433, 0.7349) -- (2.9846, 0.7423, 0.7849) -- (2.9846, 0.6918, 0.7803) -- cycle;
\fill[blue!15.0, opacity=0.5] (2.9846, 0.6918, 0.7803) -- (2.9846, 0.7423, 0.7849) -- (2.9824, 0.7412, 0.8349) -- (2.9824, 0.6906, 0.8303) -- cycle;
\fill[blue!15.0, opacity=0.5] (2.9824, 0.6906, 0.8303) -- (2.9824, 0.7412, 0.8349) -- (2.9801, 0.7401, 0.8849) -- (2.9801, 0.6894, 0.8803) -- cycle;
\fill[blue!15.0, opacity=0.5] (2.9801, 0.6894, 0.8803) -- (2.9801, 0.7401, 0.8849) -- (2.9778, 0.7389, 0.9349) -- (2.9778, 0.6881, 0.9303) -- cycle;
\fill[blue!15.0, opacity=0.5] (2.9778, 0.6881, 0.9303) -- (2.9778, 0.7389, 0.9349) -- (2.9753, 0.7376, 0.9849) -- (2.9753, 0.6868, 0.9803) -- cycle;
\fill[blue!15.0, opacity=0.5] (2.9753, 0.6868, 0.9803) -- (2.9753, 0.7376, 0.9849) -- (2.9727, 0.7363, 1.0349) -- (2.9727, 0.6854, 1.0303) -- cycle;
\fill[blue!15.0, opacity=0.5] (2.9727, 0.6854, 1.0303) -- (2.9727, 0.7363, 1.0349) -- (2.9700, 0.7350, 1.0849) -- (2.9700, 0.6840, 1.0803) -- cycle;
\fill[blue!15.0, opacity=0.5] (2.9700, 0.6840, 1.0803) -- (2.9700, 0.7350, 1.0849) -- (2.9672, 0.7336, 1.1349) -- (2.9672, 0.6825, 1.1303) -- cycle;
\fill[blue!15.0, opacity=0.5] (2.9672, 0.6825, 1.1303) -- (2.9672, 0.7336, 1.1349) -- (2.9644, 0.7322, 1.1849) -- (2.9644, 0.6810, 1.1803) -- cycle;
\fill[blue!15.0, opacity=0.5] (2.9644, 0.6810, 1.1803) -- (2.9644, 0.7322, 1.1849) -- (2.9615, 0.7308, 1.2349) -- (2.9615, 0.6795, 1.2303) -- cycle;
\fill[blue!15.0, opacity=0.5] (2.9615, 0.6795, 1.2303) -- (2.9615, 0.7308, 1.2349) -- (2.9585, 0.7293, 1.2849) -- (2.9585, 0.6779, 1.2803) -- cycle;
\fill[blue!15.0, opacity=0.5] (2.9585, 0.6779, 1.2803) -- (2.9585, 0.7293, 1.2849) -- (2.9555, 0.7278, 1.3349) -- (2.9555, 0.6763, 1.3303) -- cycle;
\fill[blue!15.0, opacity=0.5] (2.9555, 0.6763, 1.3303) -- (2.9555, 0.7278, 1.3349) -- (2.9525, 0.7262, 1.3849) -- (2.9525, 0.6747, 1.3803) -- cycle;
\fill[blue!15.0, opacity=0.5] (2.9525, 0.6747, 1.3803) -- (2.9525, 0.7262, 1.3849) -- (2.9494, 0.7247, 1.4349) -- (2.9494, 0.6730, 1.4303) -- cycle;
\fill[blue!15.0, opacity=0.5] (2.9494, 0.6730, 1.4303) -- (2.9494, 0.7247, 1.4349) -- (2.9463, 0.7231, 1.4849) -- (2.9463, 0.6713, 1.4803) -- cycle;
\fill[blue!15.0, opacity=0.5] (2.9463, 0.6713, 1.4803) -- (2.9463, 0.7231, 1.4849) -- (2.9431, 0.7216, 1.5349) -- (2.9431, 0.6697, 1.5303) -- cycle;
\fill[blue!15.0, opacity=0.5] (2.9431, 0.6697, 1.5303) -- (2.9431, 0.7216, 1.5349) -- (2.9400, 0.7200, 1.5849) -- (2.9400, 0.6680, 1.5803) -- cycle;
\fill[blue!15.0, opacity=0.5] (2.9400, 0.6680, 1.5803) -- (2.9400, 0.7200, 1.5849) -- (2.9369, 0.7184, 1.6349) -- (2.9369, 0.6663, 1.6303) -- cycle;
\fill[blue!15.0, opacity=0.5] (2.9369, 0.6663, 1.6303) -- (2.9369, 0.7184, 1.6349) -- (2.9337, 0.7169, 1.6849) -- (2.9337, 0.6647, 1.6803) -- cycle;
\fill[blue!15.0, opacity=0.5] (2.9337, 0.6647, 1.6803) -- (2.9337, 0.7169, 1.6849) -- (2.9306, 0.7153, 1.7349) -- (2.9306, 0.6630, 1.7303) -- cycle;
\fill[blue!15.0, opacity=0.5] (2.9306, 0.6630, 1.7303) -- (2.9306, 0.7153, 1.7349) -- (2.9275, 0.7138, 1.7849) -- (2.9275, 0.6613, 1.7803) -- cycle;
\fill[blue!15.0, opacity=0.5] (2.9275, 0.6613, 1.7803) -- (2.9275, 0.7138, 1.7849) -- (2.9245, 0.7122, 1.8349) -- (2.9245, 0.6597, 1.8303) -- cycle;
\fill[blue!15.0, opacity=0.5] (2.9245, 0.6597, 1.8303) -- (2.9245, 0.7122, 1.8349) -- (2.9215, 0.7107, 1.8849) -- (2.9215, 0.6581, 1.8803) -- cycle;
\fill[blue!15.0, opacity=0.5] (2.9215, 0.6581, 1.8803) -- (2.9215, 0.7107, 1.8849) -- (2.9185, 0.7092, 1.9349) -- (2.9185, 0.6565, 1.9303) -- cycle;
\fill[blue!15.0, opacity=0.5] (2.9185, 0.6565, 1.9303) -- (2.9185, 0.7092, 1.9349) -- (2.9156, 0.7078, 1.9849) -- (2.9156, 0.6550, 1.9803) -- cycle;
\fill[blue!15.0, opacity=0.5] (2.9156, 0.6550, 1.9803) -- (2.9156, 0.7078, 1.9849) -- (2.9128, 0.7064, 2.0349) -- (2.9128, 0.6535, 2.0303) -- cycle;
\fill[blue!15.0, opacity=0.5] (2.9128, 0.6535, 2.0303) -- (2.9128, 0.7064, 2.0349) -- (2.9100, 0.7050, 2.0849) -- (2.9100, 0.6520, 2.0803) -- cycle;
\fill[blue!15.0, opacity=0.5] (2.9100, 0.6520, 2.0803) -- (2.9100, 0.7050, 2.0849) -- (2.9073, 0.7037, 2.1349) -- (2.9073, 0.6506, 2.1303) -- cycle;
\fill[blue!15.0, opacity=0.5] (2.9073, 0.6506, 2.1303) -- (2.9073, 0.7037, 2.1349) -- (2.9047, 0.7024, 2.1849) -- (2.9047, 0.6492, 2.1803) -- cycle;
\fill[blue!15.0, opacity=0.5] (2.9047, 0.6492, 2.1803) -- (2.9047, 0.7024, 2.1849) -- (2.9022, 0.7011, 2.2349) -- (2.9022, 0.6479, 2.2303) -- cycle;
\fill[blue!15.1, opacity=0.5] (2.9022, 0.6479, 2.2303) -- (2.9022, 0.7011, 2.2349) -- (2.8999, 0.6999, 2.2849) -- (2.8999, 0.6466, 2.2803) -- cycle;
\fill[blue!15.1, opacity=0.5] (2.8999, 0.6466, 2.2803) -- (2.8999, 0.6999, 2.2849) -- (2.8976, 0.6988, 2.3349) -- (2.8976, 0.6454, 2.3303) -- cycle;
\fill[blue!15.1, opacity=0.5] (2.8976, 0.6454, 2.3303) -- (2.8976, 0.6988, 2.3349) -- (2.8954, 0.6977, 2.3849) -- (2.8954, 0.6442, 2.3803) -- cycle;
\fill[blue!15.1, opacity=0.5] (2.8954, 0.6442, 2.3803) -- (2.8954, 0.6977, 2.3849) -- (2.8934, 0.6967, 2.4349) -- (2.8934, 0.6431, 2.4303) -- cycle;
\fill[blue!15.2, opacity=0.5] (2.8934, 0.6431, 2.4303) -- (2.8934, 0.6967, 2.4349) -- (2.8915, 0.6957, 2.4849) -- (2.8915, 0.6421, 2.4803) -- cycle;
\fill[blue!15.3, opacity=0.5] (2.8915, 0.6421, 2.4803) -- (2.8915, 0.6957, 2.4849) -- (2.8897, 0.6948, 2.5349) -- (2.8897, 0.6412, 2.5303) -- cycle;
\fill[blue!15.3, opacity=0.5] (2.8897, 0.6412, 2.5303) -- (2.8897, 0.6948, 2.5349) -- (2.8880, 0.6940, 2.5849) -- (2.8880, 0.6403, 2.5803) -- cycle;
\fill[blue!15.4, opacity=0.5] (2.8880, 0.6403, 2.5803) -- (2.8880, 0.6940, 2.5849) -- (2.8865, 0.6933, 2.6349) -- (2.8865, 0.6395, 2.6303) -- cycle;
\fill[blue!15.5, opacity=0.5] (2.8865, 0.6395, 2.6303) -- (2.8865, 0.6933, 2.6349) -- (2.8852, 0.6926, 2.6849) -- (2.8852, 0.6388, 2.6803) -- cycle;
\fill[blue!15.7, opacity=0.5] (2.8852, 0.6388, 2.6803) -- (2.8852, 0.6926, 2.6849) -- (2.8840, 0.6920, 2.7349) -- (2.8840, 0.6381, 2.7303) -- cycle;
\fill[blue!15.8, opacity=0.5] (2.8840, 0.6381, 2.7303) -- (2.8840, 0.6920, 2.7349) -- (2.8829, 0.6915, 2.7849) -- (2.8829, 0.6376, 2.7803) -- cycle;
\fill[blue!16.0, opacity=0.5] (2.8829, 0.6376, 2.7803) -- (2.8829, 0.6915, 2.7849) -- (2.8820, 0.6910, 2.8349) -- (2.8820, 0.6371, 2.8303) -- cycle;
\fill[blue!16.2, opacity=0.5] (2.8820, 0.6371, 2.8303) -- (2.8820, 0.6910, 2.8349) -- (2.8813, 0.6907, 2.8849) -- (2.8813, 0.6367, 2.8803) -- cycle;
\fill[blue!16.5, opacity=0.5] (2.8813, 0.6367, 2.8803) -- (2.8813, 0.6907, 2.8849) -- (2.8807, 0.6904, 2.9349) -- (2.8807, 0.6364, 2.9303) -- cycle;
\fill[blue!16.7, opacity=0.5] (2.8807, 0.6364, 2.9303) -- (2.8807, 0.6904, 2.9349) -- (2.8803, 0.6902, 2.9849) -- (2.8803, 0.6362, 2.9803) -- cycle;
\fill[blue!17.0, opacity=0.5] (2.8803, 0.6362, 2.9803) -- (2.8803, 0.6902, 2.9849) -- (2.8801, 0.6900, 3.0349) -- (2.8801, 0.6360, 3.0303) -- cycle;
\fill[blue!17.3, opacity=0.5] (2.8801, 0.6360, 3.0303) -- (2.8801, 0.6900, 3.0349) -- (2.8800, 0.6900, 3.0849) -- (2.8800, 0.6360, 3.0803) -- cycle;
\fill[blue!15.0, opacity=0.5] (3.0000, 0.7500, 0.0849) -- (3.0000, 0.8000, 0.0892) -- (2.9999, 0.8000, 0.1392) -- (2.9999, 0.7500, 0.1349) -- cycle;
\fill[blue!15.0, opacity=0.5] (2.9999, 0.7500, 0.1349) -- (2.9999, 0.8000, 0.1392) -- (2.9997, 0.7998, 0.1892) -- (2.9997, 0.7498, 0.1849) -- cycle;
\fill[blue!15.0, opacity=0.5] (2.9997, 0.7498, 0.1849) -- (2.9997, 0.7998, 0.1892) -- (2.9993, 0.7997, 0.2392) -- (2.9993, 0.7496, 0.2349) -- cycle;
\fill[blue!15.0, opacity=0.5] (2.9993, 0.7496, 0.2349) -- (2.9993, 0.7997, 0.2392) -- (2.9987, 0.7994, 0.2892) -- (2.9987, 0.7493, 0.2849) -- cycle;
\fill[blue!15.0, opacity=0.5] (2.9987, 0.7493, 0.2849) -- (2.9987, 0.7994, 0.2892) -- (2.9980, 0.7990, 0.3392) -- (2.9980, 0.7490, 0.3349) -- cycle;
\fill[blue!15.0, opacity=0.5] (2.9980, 0.7490, 0.3349) -- (2.9980, 0.7990, 0.3392) -- (2.9971, 0.7986, 0.3892) -- (2.9971, 0.7485, 0.3849) -- cycle;
\fill[blue!15.0, opacity=0.5] (2.9971, 0.7485, 0.3849) -- (2.9971, 0.7986, 0.3892) -- (2.9960, 0.7981, 0.4392) -- (2.9960, 0.7480, 0.4349) -- cycle;
\fill[blue!15.0, opacity=0.5] (2.9960, 0.7480, 0.4349) -- (2.9960, 0.7981, 0.4392) -- (2.9948, 0.7976, 0.4892) -- (2.9948, 0.7474, 0.4849) -- cycle;
\fill[blue!15.0, opacity=0.5] (2.9948, 0.7474, 0.4849) -- (2.9948, 0.7976, 0.4892) -- (2.9935, 0.7969, 0.5392) -- (2.9935, 0.7467, 0.5349) -- cycle;
\fill[blue!15.0, opacity=0.5] (2.9935, 0.7467, 0.5349) -- (2.9935, 0.7969, 0.5392) -- (2.9920, 0.7962, 0.5892) -- (2.9920, 0.7460, 0.5849) -- cycle;
\fill[blue!15.0, opacity=0.5] (2.9920, 0.7460, 0.5849) -- (2.9920, 0.7962, 0.5892) -- (2.9903, 0.7955, 0.6392) -- (2.9903, 0.7452, 0.6349) -- cycle;
\fill[blue!15.0, opacity=0.5] (2.9903, 0.7452, 0.6349) -- (2.9903, 0.7955, 0.6392) -- (2.9885, 0.7947, 0.6892) -- (2.9885, 0.7443, 0.6849) -- cycle;
\fill[blue!15.0, opacity=0.5] (2.9885, 0.7443, 0.6849) -- (2.9885, 0.7947, 0.6892) -- (2.9866, 0.7938, 0.7392) -- (2.9866, 0.7433, 0.7349) -- cycle;
\fill[blue!15.0, opacity=0.5] (2.9866, 0.7433, 0.7349) -- (2.9866, 0.7938, 0.7392) -- (2.9846, 0.7928, 0.7892) -- (2.9846, 0.7423, 0.7849) -- cycle;
\fill[blue!15.0, opacity=0.5] (2.9846, 0.7423, 0.7849) -- (2.9846, 0.7928, 0.7892) -- (2.9824, 0.7918, 0.8392) -- (2.9824, 0.7412, 0.8349) -- cycle;
\fill[blue!15.0, opacity=0.5] (2.9824, 0.7412, 0.8349) -- (2.9824, 0.7918, 0.8392) -- (2.9801, 0.7907, 0.8892) -- (2.9801, 0.7401, 0.8849) -- cycle;
\fill[blue!15.0, opacity=0.5] (2.9801, 0.7401, 0.8849) -- (2.9801, 0.7907, 0.8892) -- (2.9778, 0.7896, 0.9392) -- (2.9778, 0.7389, 0.9349) -- cycle;
\fill[blue!15.0, opacity=0.5] (2.9778, 0.7389, 0.9349) -- (2.9778, 0.7896, 0.9392) -- (2.9753, 0.7885, 0.9892) -- (2.9753, 0.7376, 0.9849) -- cycle;
\fill[blue!15.0, opacity=0.5] (2.9753, 0.7376, 0.9849) -- (2.9753, 0.7885, 0.9892) -- (2.9727, 0.7872, 1.0392) -- (2.9727, 0.7363, 1.0349) -- cycle;
\fill[blue!15.0, opacity=0.5] (2.9727, 0.7363, 1.0349) -- (2.9727, 0.7872, 1.0392) -- (2.9700, 0.7860, 1.0892) -- (2.9700, 0.7350, 1.0849) -- cycle;
\fill[blue!15.0, opacity=0.5] (2.9700, 0.7350, 1.0849) -- (2.9700, 0.7860, 1.0892) -- (2.9672, 0.7847, 1.1392) -- (2.9672, 0.7336, 1.1349) -- cycle;
\fill[blue!15.0, opacity=0.5] (2.9672, 0.7336, 1.1349) -- (2.9672, 0.7847, 1.1392) -- (2.9644, 0.7834, 1.1892) -- (2.9644, 0.7322, 1.1849) -- cycle;
\fill[blue!15.0, opacity=0.5] (2.9644, 0.7322, 1.1849) -- (2.9644, 0.7834, 1.1892) -- (2.9615, 0.7820, 1.2392) -- (2.9615, 0.7308, 1.2349) -- cycle;
\fill[blue!15.0, opacity=0.5] (2.9615, 0.7308, 1.2349) -- (2.9615, 0.7820, 1.2392) -- (2.9585, 0.7807, 1.2892) -- (2.9585, 0.7293, 1.2849) -- cycle;
\fill[blue!15.0, opacity=0.5] (2.9585, 0.7293, 1.2849) -- (2.9585, 0.7807, 1.2892) -- (2.9555, 0.7792, 1.3392) -- (2.9555, 0.7278, 1.3349) -- cycle;
\fill[blue!15.0, opacity=0.5] (2.9555, 0.7278, 1.3349) -- (2.9555, 0.7792, 1.3392) -- (2.9525, 0.7778, 1.3892) -- (2.9525, 0.7262, 1.3849) -- cycle;
\fill[blue!15.0, opacity=0.5] (2.9525, 0.7262, 1.3849) -- (2.9525, 0.7778, 1.3892) -- (2.9494, 0.7764, 1.4392) -- (2.9494, 0.7247, 1.4349) -- cycle;
\fill[blue!15.0, opacity=0.5] (2.9494, 0.7247, 1.4349) -- (2.9494, 0.7764, 1.4392) -- (2.9463, 0.7749, 1.4892) -- (2.9463, 0.7231, 1.4849) -- cycle;
\fill[blue!15.0, opacity=0.5] (2.9463, 0.7231, 1.4849) -- (2.9463, 0.7749, 1.4892) -- (2.9431, 0.7735, 1.5392) -- (2.9431, 0.7216, 1.5349) -- cycle;
\fill[blue!15.0, opacity=0.5] (2.9431, 0.7216, 1.5349) -- (2.9431, 0.7735, 1.5392) -- (2.9400, 0.7720, 1.5892) -- (2.9400, 0.7200, 1.5849) -- cycle;
\fill[blue!15.0, opacity=0.5] (2.9400, 0.7200, 1.5849) -- (2.9400, 0.7720, 1.5892) -- (2.9369, 0.7705, 1.6392) -- (2.9369, 0.7184, 1.6349) -- cycle;
\fill[blue!15.0, opacity=0.5] (2.9369, 0.7184, 1.6349) -- (2.9369, 0.7705, 1.6392) -- (2.9337, 0.7691, 1.6892) -- (2.9337, 0.7169, 1.6849) -- cycle;
\fill[blue!15.0, opacity=0.5] (2.9337, 0.7169, 1.6849) -- (2.9337, 0.7691, 1.6892) -- (2.9306, 0.7676, 1.7392) -- (2.9306, 0.7153, 1.7349) -- cycle;
\fill[blue!15.0, opacity=0.5] (2.9306, 0.7153, 1.7349) -- (2.9306, 0.7676, 1.7392) -- (2.9275, 0.7662, 1.7892) -- (2.9275, 0.7138, 1.7849) -- cycle;
\fill[blue!15.0, opacity=0.5] (2.9275, 0.7138, 1.7849) -- (2.9275, 0.7662, 1.7892) -- (2.9245, 0.7648, 1.8392) -- (2.9245, 0.7122, 1.8349) -- cycle;
\fill[blue!15.0, opacity=0.5] (2.9245, 0.7122, 1.8349) -- (2.9245, 0.7648, 1.8392) -- (2.9215, 0.7633, 1.8892) -- (2.9215, 0.7107, 1.8849) -- cycle;
\fill[blue!15.0, opacity=0.5] (2.9215, 0.7107, 1.8849) -- (2.9215, 0.7633, 1.8892) -- (2.9185, 0.7620, 1.9392) -- (2.9185, 0.7092, 1.9349) -- cycle;
\fill[blue!15.0, opacity=0.5] (2.9185, 0.7092, 1.9349) -- (2.9185, 0.7620, 1.9392) -- (2.9156, 0.7606, 1.9892) -- (2.9156, 0.7078, 1.9849) -- cycle;
\fill[blue!15.0, opacity=0.5] (2.9156, 0.7078, 1.9849) -- (2.9156, 0.7606, 1.9892) -- (2.9128, 0.7593, 2.0392) -- (2.9128, 0.7064, 2.0349) -- cycle;
\fill[blue!15.0, opacity=0.5] (2.9128, 0.7064, 2.0349) -- (2.9128, 0.7593, 2.0392) -- (2.9100, 0.7580, 2.0892) -- (2.9100, 0.7050, 2.0849) -- cycle;
\fill[blue!15.0, opacity=0.5] (2.9100, 0.7050, 2.0849) -- (2.9100, 0.7580, 2.0892) -- (2.9073, 0.7568, 2.1392) -- (2.9073, 0.7037, 2.1349) -- cycle;
\fill[blue!15.0, opacity=0.5] (2.9073, 0.7037, 2.1349) -- (2.9073, 0.7568, 2.1392) -- (2.9047, 0.7555, 2.1892) -- (2.9047, 0.7024, 2.1849) -- cycle;
\fill[blue!15.0, opacity=0.5] (2.9047, 0.7024, 2.1849) -- (2.9047, 0.7555, 2.1892) -- (2.9022, 0.7544, 2.2392) -- (2.9022, 0.7011, 2.2349) -- cycle;
\fill[blue!15.0, opacity=0.5] (2.9022, 0.7011, 2.2349) -- (2.9022, 0.7544, 2.2392) -- (2.8999, 0.7533, 2.2892) -- (2.8999, 0.6999, 2.2849) -- cycle;
\fill[blue!15.0, opacity=0.5] (2.8999, 0.6999, 2.2849) -- (2.8999, 0.7533, 2.2892) -- (2.8976, 0.7522, 2.3392) -- (2.8976, 0.6988, 2.3349) -- cycle;
\fill[blue!15.1, opacity=0.5] (2.8976, 0.6988, 2.3349) -- (2.8976, 0.7522, 2.3392) -- (2.8954, 0.7512, 2.3892) -- (2.8954, 0.6977, 2.3849) -- cycle;
\fill[blue!15.1, opacity=0.5] (2.8954, 0.6977, 2.3849) -- (2.8954, 0.7512, 2.3892) -- (2.8934, 0.7502, 2.4392) -- (2.8934, 0.6967, 2.4349) -- cycle;
\fill[blue!15.1, opacity=0.5] (2.8934, 0.6967, 2.4349) -- (2.8934, 0.7502, 2.4392) -- (2.8915, 0.7493, 2.4892) -- (2.8915, 0.6957, 2.4849) -- cycle;
\fill[blue!15.1, opacity=0.5] (2.8915, 0.6957, 2.4849) -- (2.8915, 0.7493, 2.4892) -- (2.8897, 0.7485, 2.5392) -- (2.8897, 0.6948, 2.5349) -- cycle;
\fill[blue!15.2, opacity=0.5] (2.8897, 0.6948, 2.5349) -- (2.8897, 0.7485, 2.5392) -- (2.8880, 0.7478, 2.5892) -- (2.8880, 0.6940, 2.5849) -- cycle;
\fill[blue!15.3, opacity=0.5] (2.8880, 0.6940, 2.5849) -- (2.8880, 0.7478, 2.5892) -- (2.8865, 0.7471, 2.6392) -- (2.8865, 0.6933, 2.6349) -- cycle;
\fill[blue!15.3, opacity=0.5] (2.8865, 0.6933, 2.6349) -- (2.8865, 0.7471, 2.6392) -- (2.8852, 0.7464, 2.6892) -- (2.8852, 0.6926, 2.6849) -- cycle;
\fill[blue!15.4, opacity=0.5] (2.8852, 0.6926, 2.6849) -- (2.8852, 0.7464, 2.6892) -- (2.8840, 0.7459, 2.7392) -- (2.8840, 0.6920, 2.7349) -- cycle;
\fill[blue!15.5, opacity=0.5] (2.8840, 0.6920, 2.7349) -- (2.8840, 0.7459, 2.7392) -- (2.8829, 0.7454, 2.7892) -- (2.8829, 0.6915, 2.7849) -- cycle;
\fill[blue!15.6, opacity=0.5] (2.8829, 0.6915, 2.7849) -- (2.8829, 0.7454, 2.7892) -- (2.8820, 0.7450, 2.8392) -- (2.8820, 0.6910, 2.8349) -- cycle;
\fill[blue!15.8, opacity=0.5] (2.8820, 0.6910, 2.8349) -- (2.8820, 0.7450, 2.8392) -- (2.8813, 0.7446, 2.8892) -- (2.8813, 0.6907, 2.8849) -- cycle;
\fill[blue!15.9, opacity=0.5] (2.8813, 0.6907, 2.8849) -- (2.8813, 0.7446, 2.8892) -- (2.8807, 0.7443, 2.9392) -- (2.8807, 0.6904, 2.9349) -- cycle;
\fill[blue!16.1, opacity=0.5] (2.8807, 0.6904, 2.9349) -- (2.8807, 0.7443, 2.9392) -- (2.8803, 0.7442, 2.9892) -- (2.8803, 0.6902, 2.9849) -- cycle;
\fill[blue!16.3, opacity=0.5] (2.8803, 0.6902, 2.9849) -- (2.8803, 0.7442, 2.9892) -- (2.8801, 0.7440, 3.0392) -- (2.8801, 0.6900, 3.0349) -- cycle;
\fill[blue!16.6, opacity=0.5] (2.8801, 0.6900, 3.0349) -- (2.8801, 0.7440, 3.0392) -- (2.8800, 0.7440, 3.0892) -- (2.8800, 0.6900, 3.0849) -- cycle;
\fill[blue!15.0, opacity=0.5] (3.0000, 0.8000, 0.0892) -- (3.0000, 0.8500, 0.0933) -- (2.9999, 0.8500, 0.1433) -- (2.9999, 0.8000, 0.1392) -- cycle;
\fill[blue!15.0, opacity=0.5] (2.9999, 0.8000, 0.1392) -- (2.9999, 0.8500, 0.1433) -- (2.9997, 0.8499, 0.1933) -- (2.9997, 0.7998, 0.1892) -- cycle;
\fill[blue!15.0, opacity=0.5] (2.9997, 0.7998, 0.1892) -- (2.9997, 0.8499, 0.1933) -- (2.9993, 0.8497, 0.2433) -- (2.9993, 0.7997, 0.2392) -- cycle;
\fill[blue!15.0, opacity=0.5] (2.9993, 0.7997, 0.2392) -- (2.9993, 0.8497, 0.2433) -- (2.9987, 0.8494, 0.2933) -- (2.9987, 0.7994, 0.2892) -- cycle;
\fill[blue!15.0, opacity=0.5] (2.9987, 0.7994, 0.2892) -- (2.9987, 0.8494, 0.2933) -- (2.9980, 0.8491, 0.3433) -- (2.9980, 0.7990, 0.3392) -- cycle;
\fill[blue!15.0, opacity=0.5] (2.9980, 0.7990, 0.3392) -- (2.9980, 0.8491, 0.3433) -- (2.9971, 0.8487, 0.3933) -- (2.9971, 0.7986, 0.3892) -- cycle;
\fill[blue!15.0, opacity=0.5] (2.9971, 0.7986, 0.3892) -- (2.9971, 0.8487, 0.3933) -- (2.9960, 0.8483, 0.4433) -- (2.9960, 0.7981, 0.4392) -- cycle;
\fill[blue!15.0, opacity=0.5] (2.9960, 0.7981, 0.4392) -- (2.9960, 0.8483, 0.4433) -- (2.9948, 0.8478, 0.4933) -- (2.9948, 0.7976, 0.4892) -- cycle;
\fill[blue!15.0, opacity=0.5] (2.9948, 0.7976, 0.4892) -- (2.9948, 0.8478, 0.4933) -- (2.9935, 0.8472, 0.5433) -- (2.9935, 0.7969, 0.5392) -- cycle;
\fill[blue!15.0, opacity=0.5] (2.9935, 0.7969, 0.5392) -- (2.9935, 0.8472, 0.5433) -- (2.9920, 0.8465, 0.5933) -- (2.9920, 0.7962, 0.5892) -- cycle;
\fill[blue!15.0, opacity=0.5] (2.9920, 0.7962, 0.5892) -- (2.9920, 0.8465, 0.5933) -- (2.9903, 0.8458, 0.6433) -- (2.9903, 0.7955, 0.6392) -- cycle;
\fill[blue!15.0, opacity=0.5] (2.9903, 0.7955, 0.6392) -- (2.9903, 0.8458, 0.6433) -- (2.9885, 0.8450, 0.6933) -- (2.9885, 0.7947, 0.6892) -- cycle;
\fill[blue!15.0, opacity=0.5] (2.9885, 0.7947, 0.6892) -- (2.9885, 0.8450, 0.6933) -- (2.9866, 0.8442, 0.7433) -- (2.9866, 0.7938, 0.7392) -- cycle;
\fill[blue!15.0, opacity=0.5] (2.9866, 0.7938, 0.7392) -- (2.9866, 0.8442, 0.7433) -- (2.9846, 0.8433, 0.7933) -- (2.9846, 0.7928, 0.7892) -- cycle;
\fill[blue!15.0, opacity=0.5] (2.9846, 0.7928, 0.7892) -- (2.9846, 0.8433, 0.7933) -- (2.9824, 0.8424, 0.8433) -- (2.9824, 0.7918, 0.8392) -- cycle;
\fill[blue!15.0, opacity=0.5] (2.9824, 0.7918, 0.8392) -- (2.9824, 0.8424, 0.8433) -- (2.9801, 0.8414, 0.8933) -- (2.9801, 0.7907, 0.8892) -- cycle;
\fill[blue!15.0, opacity=0.5] (2.9801, 0.7907, 0.8892) -- (2.9801, 0.8414, 0.8933) -- (2.9778, 0.8404, 0.9433) -- (2.9778, 0.7896, 0.9392) -- cycle;
\fill[blue!15.0, opacity=0.5] (2.9778, 0.7896, 0.9392) -- (2.9778, 0.8404, 0.9433) -- (2.9753, 0.8393, 0.9933) -- (2.9753, 0.7885, 0.9892) -- cycle;
\fill[blue!15.0, opacity=0.5] (2.9753, 0.7885, 0.9892) -- (2.9753, 0.8393, 0.9933) -- (2.9727, 0.8382, 1.0433) -- (2.9727, 0.7872, 1.0392) -- cycle;
\fill[blue!15.0, opacity=0.5] (2.9727, 0.7872, 1.0392) -- (2.9727, 0.8382, 1.0433) -- (2.9700, 0.8370, 1.0933) -- (2.9700, 0.7860, 1.0892) -- cycle;
\fill[blue!15.0, opacity=0.5] (2.9700, 0.7860, 1.0892) -- (2.9700, 0.8370, 1.0933) -- (2.9672, 0.8358, 1.1433) -- (2.9672, 0.7847, 1.1392) -- cycle;
\fill[blue!15.0, opacity=0.5] (2.9672, 0.7847, 1.1392) -- (2.9672, 0.8358, 1.1433) -- (2.9644, 0.8346, 1.1933) -- (2.9644, 0.7834, 1.1892) -- cycle;
\fill[blue!15.0, opacity=0.5] (2.9644, 0.7834, 1.1892) -- (2.9644, 0.8346, 1.1933) -- (2.9615, 0.8333, 1.2433) -- (2.9615, 0.7820, 1.2392) -- cycle;
\fill[blue!15.0, opacity=0.5] (2.9615, 0.7820, 1.2392) -- (2.9615, 0.8333, 1.2433) -- (2.9585, 0.8320, 1.2933) -- (2.9585, 0.7807, 1.2892) -- cycle;
\fill[blue!15.0, opacity=0.5] (2.9585, 0.7807, 1.2892) -- (2.9585, 0.8320, 1.2933) -- (2.9555, 0.8307, 1.3433) -- (2.9555, 0.7792, 1.3392) -- cycle;
\fill[blue!15.0, opacity=0.5] (2.9555, 0.7792, 1.3392) -- (2.9555, 0.8307, 1.3433) -- (2.9525, 0.8294, 1.3933) -- (2.9525, 0.7778, 1.3892) -- cycle;
\fill[blue!15.0, opacity=0.5] (2.9525, 0.7778, 1.3892) -- (2.9525, 0.8294, 1.3933) -- (2.9494, 0.8281, 1.4433) -- (2.9494, 0.7764, 1.4392) -- cycle;
\fill[blue!15.0, opacity=0.5] (2.9494, 0.7764, 1.4392) -- (2.9494, 0.8281, 1.4433) -- (2.9463, 0.8267, 1.4933) -- (2.9463, 0.7749, 1.4892) -- cycle;
\fill[blue!15.0, opacity=0.5] (2.9463, 0.7749, 1.4892) -- (2.9463, 0.8267, 1.4933) -- (2.9431, 0.8254, 1.5433) -- (2.9431, 0.7735, 1.5392) -- cycle;
\fill[blue!15.0, opacity=0.5] (2.9431, 0.7735, 1.5392) -- (2.9431, 0.8254, 1.5433) -- (2.9400, 0.8240, 1.5933) -- (2.9400, 0.7720, 1.5892) -- cycle;
\fill[blue!15.0, opacity=0.5] (2.9400, 0.7720, 1.5892) -- (2.9400, 0.8240, 1.5933) -- (2.9369, 0.8226, 1.6433) -- (2.9369, 0.7705, 1.6392) -- cycle;
\fill[blue!15.0, opacity=0.5] (2.9369, 0.7705, 1.6392) -- (2.9369, 0.8226, 1.6433) -- (2.9337, 0.8213, 1.6933) -- (2.9337, 0.7691, 1.6892) -- cycle;
\fill[blue!15.0, opacity=0.5] (2.9337, 0.7691, 1.6892) -- (2.9337, 0.8213, 1.6933) -- (2.9306, 0.8199, 1.7433) -- (2.9306, 0.7676, 1.7392) -- cycle;
\fill[blue!15.0, opacity=0.5] (2.9306, 0.7676, 1.7392) -- (2.9306, 0.8199, 1.7433) -- (2.9275, 0.8186, 1.7933) -- (2.9275, 0.7662, 1.7892) -- cycle;
\fill[blue!15.0, opacity=0.5] (2.9275, 0.7662, 1.7892) -- (2.9275, 0.8186, 1.7933) -- (2.9245, 0.8173, 1.8433) -- (2.9245, 0.7648, 1.8392) -- cycle;
\fill[blue!15.0, opacity=0.5] (2.9245, 0.7648, 1.8392) -- (2.9245, 0.8173, 1.8433) -- (2.9215, 0.8160, 1.8933) -- (2.9215, 0.7633, 1.8892) -- cycle;
\fill[blue!15.0, opacity=0.5] (2.9215, 0.7633, 1.8892) -- (2.9215, 0.8160, 1.8933) -- (2.9185, 0.8147, 1.9433) -- (2.9185, 0.7620, 1.9392) -- cycle;
\fill[blue!15.0, opacity=0.5] (2.9185, 0.7620, 1.9392) -- (2.9185, 0.8147, 1.9433) -- (2.9156, 0.8134, 1.9933) -- (2.9156, 0.7606, 1.9892) -- cycle;
\fill[blue!15.0, opacity=0.5] (2.9156, 0.7606, 1.9892) -- (2.9156, 0.8134, 1.9933) -- (2.9128, 0.8122, 2.0433) -- (2.9128, 0.7593, 2.0392) -- cycle;
\fill[blue!15.0, opacity=0.5] (2.9128, 0.7593, 2.0392) -- (2.9128, 0.8122, 2.0433) -- (2.9100, 0.8110, 2.0933) -- (2.9100, 0.7580, 2.0892) -- cycle;
\fill[blue!15.0, opacity=0.5] (2.9100, 0.7580, 2.0892) -- (2.9100, 0.8110, 2.0933) -- (2.9073, 0.8098, 2.1433) -- (2.9073, 0.7568, 2.1392) -- cycle;
\fill[blue!15.0, opacity=0.5] (2.9073, 0.7568, 2.1392) -- (2.9073, 0.8098, 2.1433) -- (2.9047, 0.8087, 2.1933) -- (2.9047, 0.7555, 2.1892) -- cycle;
\fill[blue!15.0, opacity=0.5] (2.9047, 0.7555, 2.1892) -- (2.9047, 0.8087, 2.1933) -- (2.9022, 0.8076, 2.2433) -- (2.9022, 0.7544, 2.2392) -- cycle;
\fill[blue!15.0, opacity=0.5] (2.9022, 0.7544, 2.2392) -- (2.9022, 0.8076, 2.2433) -- (2.8999, 0.8066, 2.2933) -- (2.8999, 0.7533, 2.2892) -- cycle;
\fill[blue!15.0, opacity=0.5] (2.8999, 0.7533, 2.2892) -- (2.8999, 0.8066, 2.2933) -- (2.8976, 0.8056, 2.3433) -- (2.8976, 0.7522, 2.3392) -- cycle;
\fill[blue!15.0, opacity=0.5] (2.8976, 0.7522, 2.3392) -- (2.8976, 0.8056, 2.3433) -- (2.8954, 0.8047, 2.3933) -- (2.8954, 0.7512, 2.3892) -- cycle;
\fill[blue!15.1, opacity=0.5] (2.8954, 0.7512, 2.3892) -- (2.8954, 0.8047, 2.3933) -- (2.8934, 0.8038, 2.4433) -- (2.8934, 0.7502, 2.4392) -- cycle;
\fill[blue!15.1, opacity=0.5] (2.8934, 0.7502, 2.4392) -- (2.8934, 0.8038, 2.4433) -- (2.8915, 0.8030, 2.4933) -- (2.8915, 0.7493, 2.4892) -- cycle;
\fill[blue!15.1, opacity=0.5] (2.8915, 0.7493, 2.4892) -- (2.8915, 0.8030, 2.4933) -- (2.8897, 0.8022, 2.5433) -- (2.8897, 0.7485, 2.5392) -- cycle;
\fill[blue!15.1, opacity=0.5] (2.8897, 0.7485, 2.5392) -- (2.8897, 0.8022, 2.5433) -- (2.8880, 0.8015, 2.5933) -- (2.8880, 0.7478, 2.5892) -- cycle;
\fill[blue!15.2, opacity=0.5] (2.8880, 0.7478, 2.5892) -- (2.8880, 0.8015, 2.5933) -- (2.8865, 0.8008, 2.6433) -- (2.8865, 0.7471, 2.6392) -- cycle;
\fill[blue!15.3, opacity=0.5] (2.8865, 0.7471, 2.6392) -- (2.8865, 0.8008, 2.6433) -- (2.8852, 0.8002, 2.6933) -- (2.8852, 0.7464, 2.6892) -- cycle;
\fill[blue!15.3, opacity=0.5] (2.8852, 0.7464, 2.6892) -- (2.8852, 0.8002, 2.6933) -- (2.8840, 0.7997, 2.7433) -- (2.8840, 0.7459, 2.7392) -- cycle;
\fill[blue!15.4, opacity=0.5] (2.8840, 0.7459, 2.7392) -- (2.8840, 0.7997, 2.7433) -- (2.8829, 0.7993, 2.7933) -- (2.8829, 0.7454, 2.7892) -- cycle;
\fill[blue!15.5, opacity=0.5] (2.8829, 0.7454, 2.7892) -- (2.8829, 0.7993, 2.7933) -- (2.8820, 0.7989, 2.8433) -- (2.8820, 0.7450, 2.8392) -- cycle;
\fill[blue!15.6, opacity=0.5] (2.8820, 0.7450, 2.8392) -- (2.8820, 0.7989, 2.8433) -- (2.8813, 0.7986, 2.8933) -- (2.8813, 0.7446, 2.8892) -- cycle;
\fill[blue!15.8, opacity=0.5] (2.8813, 0.7446, 2.8892) -- (2.8813, 0.7986, 2.8933) -- (2.8807, 0.7983, 2.9433) -- (2.8807, 0.7443, 2.9392) -- cycle;
\fill[blue!15.9, opacity=0.5] (2.8807, 0.7443, 2.9392) -- (2.8807, 0.7983, 2.9433) -- (2.8803, 0.7981, 2.9933) -- (2.8803, 0.7442, 2.9892) -- cycle;
\fill[blue!16.1, opacity=0.5] (2.8803, 0.7442, 2.9892) -- (2.8803, 0.7981, 2.9933) -- (2.8801, 0.7980, 3.0433) -- (2.8801, 0.7440, 3.0392) -- cycle;
\fill[blue!16.3, opacity=0.5] (2.8801, 0.7440, 3.0392) -- (2.8801, 0.7980, 3.0433) -- (2.8800, 0.7980, 3.0933) -- (2.8800, 0.7440, 3.0892) -- cycle;
\fill[blue!15.0, opacity=0.5] (3.0000, 0.8500, 0.0933) -- (3.0000, 0.9000, 0.0971) -- (2.9999, 0.9000, 0.1471) -- (2.9999, 0.8500, 0.1433) -- cycle;
\fill[blue!15.0, opacity=0.5] (2.9999, 0.8500, 0.1433) -- (2.9999, 0.9000, 0.1471) -- (2.9997, 0.8999, 0.1971) -- (2.9997, 0.8499, 0.1933) -- cycle;
\fill[blue!15.0, opacity=0.5] (2.9997, 0.8499, 0.1933) -- (2.9997, 0.8999, 0.1971) -- (2.9993, 0.8997, 0.2471) -- (2.9993, 0.8497, 0.2433) -- cycle;
\fill[blue!15.0, opacity=0.5] (2.9993, 0.8497, 0.2433) -- (2.9993, 0.8997, 0.2471) -- (2.9987, 0.8995, 0.2971) -- (2.9987, 0.8494, 0.2933) -- cycle;
\fill[blue!15.0, opacity=0.5] (2.9987, 0.8494, 0.2933) -- (2.9987, 0.8995, 0.2971) -- (2.9980, 0.8992, 0.3471) -- (2.9980, 0.8491, 0.3433) -- cycle;
\fill[blue!15.0, opacity=0.5] (2.9980, 0.8491, 0.3433) -- (2.9980, 0.8992, 0.3471) -- (2.9971, 0.8988, 0.3971) -- (2.9971, 0.8487, 0.3933) -- cycle;
\fill[blue!15.0, opacity=0.5] (2.9971, 0.8487, 0.3933) -- (2.9971, 0.8988, 0.3971) -- (2.9960, 0.8984, 0.4471) -- (2.9960, 0.8483, 0.4433) -- cycle;
\fill[blue!15.0, opacity=0.5] (2.9960, 0.8483, 0.4433) -- (2.9960, 0.8984, 0.4471) -- (2.9948, 0.8979, 0.4971) -- (2.9948, 0.8478, 0.4933) -- cycle;
\fill[blue!15.0, opacity=0.5] (2.9948, 0.8478, 0.4933) -- (2.9948, 0.8979, 0.4971) -- (2.9935, 0.8974, 0.5471) -- (2.9935, 0.8472, 0.5433) -- cycle;
\fill[blue!15.0, opacity=0.5] (2.9935, 0.8472, 0.5433) -- (2.9935, 0.8974, 0.5471) -- (2.9920, 0.8968, 0.5971) -- (2.9920, 0.8465, 0.5933) -- cycle;
\fill[blue!15.0, opacity=0.5] (2.9920, 0.8465, 0.5933) -- (2.9920, 0.8968, 0.5971) -- (2.9903, 0.8961, 0.6471) -- (2.9903, 0.8458, 0.6433) -- cycle;
\fill[blue!15.0, opacity=0.5] (2.9903, 0.8458, 0.6433) -- (2.9903, 0.8961, 0.6471) -- (2.9885, 0.8954, 0.6971) -- (2.9885, 0.8450, 0.6933) -- cycle;
\fill[blue!15.0, opacity=0.5] (2.9885, 0.8450, 0.6933) -- (2.9885, 0.8954, 0.6971) -- (2.9866, 0.8947, 0.7471) -- (2.9866, 0.8442, 0.7433) -- cycle;
\fill[blue!15.0, opacity=0.5] (2.9866, 0.8442, 0.7433) -- (2.9866, 0.8947, 0.7471) -- (2.9846, 0.8938, 0.7971) -- (2.9846, 0.8433, 0.7933) -- cycle;
\fill[blue!15.0, opacity=0.5] (2.9846, 0.8433, 0.7933) -- (2.9846, 0.8938, 0.7971) -- (2.9824, 0.8930, 0.8471) -- (2.9824, 0.8424, 0.8433) -- cycle;
\fill[blue!15.0, opacity=0.5] (2.9824, 0.8424, 0.8433) -- (2.9824, 0.8930, 0.8471) -- (2.9801, 0.8921, 0.8971) -- (2.9801, 0.8414, 0.8933) -- cycle;
\fill[blue!15.0, opacity=0.5] (2.9801, 0.8414, 0.8933) -- (2.9801, 0.8921, 0.8971) -- (2.9778, 0.8911, 0.9471) -- (2.9778, 0.8404, 0.9433) -- cycle;
\fill[blue!15.0, opacity=0.5] (2.9778, 0.8404, 0.9433) -- (2.9778, 0.8911, 0.9471) -- (2.9753, 0.8901, 0.9971) -- (2.9753, 0.8393, 0.9933) -- cycle;
\fill[blue!15.0, opacity=0.5] (2.9753, 0.8393, 0.9933) -- (2.9753, 0.8901, 0.9971) -- (2.9727, 0.8891, 1.0471) -- (2.9727, 0.8382, 1.0433) -- cycle;
\fill[blue!15.0, opacity=0.5] (2.9727, 0.8382, 1.0433) -- (2.9727, 0.8891, 1.0471) -- (2.9700, 0.8880, 1.0971) -- (2.9700, 0.8370, 1.0933) -- cycle;
\fill[blue!15.0, opacity=0.5] (2.9700, 0.8370, 1.0933) -- (2.9700, 0.8880, 1.0971) -- (2.9672, 0.8869, 1.1471) -- (2.9672, 0.8358, 1.1433) -- cycle;
\fill[blue!15.0, opacity=0.5] (2.9672, 0.8358, 1.1433) -- (2.9672, 0.8869, 1.1471) -- (2.9644, 0.8858, 1.1971) -- (2.9644, 0.8346, 1.1933) -- cycle;
\fill[blue!15.0, opacity=0.5] (2.9644, 0.8346, 1.1933) -- (2.9644, 0.8858, 1.1971) -- (2.9615, 0.8846, 1.2471) -- (2.9615, 0.8333, 1.2433) -- cycle;
\fill[blue!15.0, opacity=0.5] (2.9615, 0.8333, 1.2433) -- (2.9615, 0.8846, 1.2471) -- (2.9585, 0.8834, 1.2971) -- (2.9585, 0.8320, 1.2933) -- cycle;
\fill[blue!15.0, opacity=0.5] (2.9585, 0.8320, 1.2933) -- (2.9585, 0.8834, 1.2971) -- (2.9555, 0.8822, 1.3471) -- (2.9555, 0.8307, 1.3433) -- cycle;
\fill[blue!15.0, opacity=0.5] (2.9555, 0.8307, 1.3433) -- (2.9555, 0.8822, 1.3471) -- (2.9525, 0.8810, 1.3971) -- (2.9525, 0.8294, 1.3933) -- cycle;
\fill[blue!15.0, opacity=0.5] (2.9525, 0.8294, 1.3933) -- (2.9525, 0.8810, 1.3971) -- (2.9494, 0.8798, 1.4471) -- (2.9494, 0.8281, 1.4433) -- cycle;
\fill[blue!15.0, opacity=0.5] (2.9494, 0.8281, 1.4433) -- (2.9494, 0.8798, 1.4471) -- (2.9463, 0.8785, 1.4971) -- (2.9463, 0.8267, 1.4933) -- cycle;
\fill[blue!15.0, opacity=0.5] (2.9463, 0.8267, 1.4933) -- (2.9463, 0.8785, 1.4971) -- (2.9431, 0.8773, 1.5471) -- (2.9431, 0.8254, 1.5433) -- cycle;
\fill[blue!15.0, opacity=0.5] (2.9431, 0.8254, 1.5433) -- (2.9431, 0.8773, 1.5471) -- (2.9400, 0.8760, 1.5971) -- (2.9400, 0.8240, 1.5933) -- cycle;
\fill[blue!15.0, opacity=0.5] (2.9400, 0.8240, 1.5933) -- (2.9400, 0.8760, 1.5971) -- (2.9369, 0.8747, 1.6471) -- (2.9369, 0.8226, 1.6433) -- cycle;
\fill[blue!15.0, opacity=0.5] (2.9369, 0.8226, 1.6433) -- (2.9369, 0.8747, 1.6471) -- (2.9337, 0.8735, 1.6971) -- (2.9337, 0.8213, 1.6933) -- cycle;
\fill[blue!15.0, opacity=0.5] (2.9337, 0.8213, 1.6933) -- (2.9337, 0.8735, 1.6971) -- (2.9306, 0.8722, 1.7471) -- (2.9306, 0.8199, 1.7433) -- cycle;
\fill[blue!15.0, opacity=0.5] (2.9306, 0.8199, 1.7433) -- (2.9306, 0.8722, 1.7471) -- (2.9275, 0.8710, 1.7971) -- (2.9275, 0.8186, 1.7933) -- cycle;
\fill[blue!15.0, opacity=0.5] (2.9275, 0.8186, 1.7933) -- (2.9275, 0.8710, 1.7971) -- (2.9245, 0.8698, 1.8471) -- (2.9245, 0.8173, 1.8433) -- cycle;
\fill[blue!15.0, opacity=0.5] (2.9245, 0.8173, 1.8433) -- (2.9245, 0.8698, 1.8471) -- (2.9215, 0.8686, 1.8971) -- (2.9215, 0.8160, 1.8933) -- cycle;
\fill[blue!15.0, opacity=0.5] (2.9215, 0.8160, 1.8933) -- (2.9215, 0.8686, 1.8971) -- (2.9185, 0.8674, 1.9471) -- (2.9185, 0.8147, 1.9433) -- cycle;
\fill[blue!15.0, opacity=0.5] (2.9185, 0.8147, 1.9433) -- (2.9185, 0.8674, 1.9471) -- (2.9156, 0.8662, 1.9971) -- (2.9156, 0.8134, 1.9933) -- cycle;
\fill[blue!15.0, opacity=0.5] (2.9156, 0.8134, 1.9933) -- (2.9156, 0.8662, 1.9971) -- (2.9128, 0.8651, 2.0471) -- (2.9128, 0.8122, 2.0433) -- cycle;
\fill[blue!15.0, opacity=0.5] (2.9128, 0.8122, 2.0433) -- (2.9128, 0.8651, 2.0471) -- (2.9100, 0.8640, 2.0971) -- (2.9100, 0.8110, 2.0933) -- cycle;
\fill[blue!15.0, opacity=0.5] (2.9100, 0.8110, 2.0933) -- (2.9100, 0.8640, 2.0971) -- (2.9073, 0.8629, 2.1471) -- (2.9073, 0.8098, 2.1433) -- cycle;
\fill[blue!15.0, opacity=0.5] (2.9073, 0.8098, 2.1433) -- (2.9073, 0.8629, 2.1471) -- (2.9047, 0.8619, 2.1971) -- (2.9047, 0.8087, 2.1933) -- cycle;
\fill[blue!15.0, opacity=0.5] (2.9047, 0.8087, 2.1933) -- (2.9047, 0.8619, 2.1971) -- (2.9022, 0.8609, 2.2471) -- (2.9022, 0.8076, 2.2433) -- cycle;
\fill[blue!15.0, opacity=0.5] (2.9022, 0.8076, 2.2433) -- (2.9022, 0.8609, 2.2471) -- (2.8999, 0.8599, 2.2971) -- (2.8999, 0.8066, 2.2933) -- cycle;
\fill[blue!15.0, opacity=0.5] (2.8999, 0.8066, 2.2933) -- (2.8999, 0.8599, 2.2971) -- (2.8976, 0.8590, 2.3471) -- (2.8976, 0.8056, 2.3433) -- cycle;
\fill[blue!15.0, opacity=0.5] (2.8976, 0.8056, 2.3433) -- (2.8976, 0.8590, 2.3471) -- (2.8954, 0.8582, 2.3971) -- (2.8954, 0.8047, 2.3933) -- cycle;
\fill[blue!15.1, opacity=0.5] (2.8954, 0.8047, 2.3933) -- (2.8954, 0.8582, 2.3971) -- (2.8934, 0.8573, 2.4471) -- (2.8934, 0.8038, 2.4433) -- cycle;
\fill[blue!15.1, opacity=0.5] (2.8934, 0.8038, 2.4433) -- (2.8934, 0.8573, 2.4471) -- (2.8915, 0.8566, 2.4971) -- (2.8915, 0.8030, 2.4933) -- cycle;
\fill[blue!15.1, opacity=0.5] (2.8915, 0.8030, 2.4933) -- (2.8915, 0.8566, 2.4971) -- (2.8897, 0.8559, 2.5471) -- (2.8897, 0.8022, 2.5433) -- cycle;
\fill[blue!15.1, opacity=0.5] (2.8897, 0.8022, 2.5433) -- (2.8897, 0.8559, 2.5471) -- (2.8880, 0.8552, 2.5971) -- (2.8880, 0.8015, 2.5933) -- cycle;
\fill[blue!15.2, opacity=0.5] (2.8880, 0.8015, 2.5933) -- (2.8880, 0.8552, 2.5971) -- (2.8865, 0.8546, 2.6471) -- (2.8865, 0.8008, 2.6433) -- cycle;
\fill[blue!15.3, opacity=0.5] (2.8865, 0.8008, 2.6433) -- (2.8865, 0.8546, 2.6471) -- (2.8852, 0.8541, 2.6971) -- (2.8852, 0.8002, 2.6933) -- cycle;
\fill[blue!15.3, opacity=0.5] (2.8852, 0.8002, 2.6933) -- (2.8852, 0.8541, 2.6971) -- (2.8840, 0.8536, 2.7471) -- (2.8840, 0.7997, 2.7433) -- cycle;
\fill[blue!15.4, opacity=0.5] (2.8840, 0.7997, 2.7433) -- (2.8840, 0.8536, 2.7471) -- (2.8829, 0.8532, 2.7971) -- (2.8829, 0.7993, 2.7933) -- cycle;
\fill[blue!15.5, opacity=0.5] (2.8829, 0.7993, 2.7933) -- (2.8829, 0.8532, 2.7971) -- (2.8820, 0.8528, 2.8471) -- (2.8820, 0.7989, 2.8433) -- cycle;
\fill[blue!15.6, opacity=0.5] (2.8820, 0.7989, 2.8433) -- (2.8820, 0.8528, 2.8471) -- (2.8813, 0.8525, 2.8971) -- (2.8813, 0.7986, 2.8933) -- cycle;
\fill[blue!15.8, opacity=0.5] (2.8813, 0.7986, 2.8933) -- (2.8813, 0.8525, 2.8971) -- (2.8807, 0.8523, 2.9471) -- (2.8807, 0.7983, 2.9433) -- cycle;
\fill[blue!15.9, opacity=0.5] (2.8807, 0.7983, 2.9433) -- (2.8807, 0.8523, 2.9471) -- (2.8803, 0.8521, 2.9971) -- (2.8803, 0.7981, 2.9933) -- cycle;
\fill[blue!16.1, opacity=0.5] (2.8803, 0.7981, 2.9933) -- (2.8803, 0.8521, 2.9971) -- (2.8801, 0.8520, 3.0471) -- (2.8801, 0.7980, 3.0433) -- cycle;
\fill[blue!16.3, opacity=0.5] (2.8801, 0.7980, 3.0433) -- (2.8801, 0.8520, 3.0471) -- (2.8800, 0.8520, 3.0971) -- (2.8800, 0.7980, 3.0933) -- cycle;
\fill[blue!15.0, opacity=0.5] (3.0000, 0.9000, 0.0971) -- (3.0000, 0.9500, 0.1006) -- (2.9999, 0.9500, 0.1506) -- (2.9999, 0.9000, 0.1471) -- cycle;
\fill[blue!15.0, opacity=0.5] (2.9999, 0.9000, 0.1471) -- (2.9999, 0.9500, 0.1506) -- (2.9997, 0.9499, 0.2006) -- (2.9997, 0.8999, 0.1971) -- cycle;
\fill[blue!15.0, opacity=0.5] (2.9997, 0.8999, 0.1971) -- (2.9997, 0.9499, 0.2006) -- (2.9993, 0.9497, 0.2506) -- (2.9993, 0.8997, 0.2471) -- cycle;
\fill[blue!15.0, opacity=0.5] (2.9993, 0.8997, 0.2471) -- (2.9993, 0.9497, 0.2506) -- (2.9987, 0.9495, 0.3006) -- (2.9987, 0.8995, 0.2971) -- cycle;
\fill[blue!15.0, opacity=0.5] (2.9987, 0.8995, 0.2971) -- (2.9987, 0.9495, 0.3006) -- (2.9980, 0.9493, 0.3506) -- (2.9980, 0.8992, 0.3471) -- cycle;
\fill[blue!15.0, opacity=0.5] (2.9980, 0.8992, 0.3471) -- (2.9980, 0.9493, 0.3506) -- (2.9971, 0.9489, 0.4006) -- (2.9971, 0.8988, 0.3971) -- cycle;
\fill[blue!15.0, opacity=0.5] (2.9971, 0.8988, 0.3971) -- (2.9971, 0.9489, 0.4006) -- (2.9960, 0.9485, 0.4506) -- (2.9960, 0.8984, 0.4471) -- cycle;
\fill[blue!15.0, opacity=0.5] (2.9960, 0.8984, 0.4471) -- (2.9960, 0.9485, 0.4506) -- (2.9948, 0.9481, 0.5006) -- (2.9948, 0.8979, 0.4971) -- cycle;
\fill[blue!15.0, opacity=0.5] (2.9948, 0.8979, 0.4971) -- (2.9948, 0.9481, 0.5006) -- (2.9935, 0.9476, 0.5506) -- (2.9935, 0.8974, 0.5471) -- cycle;
\fill[blue!15.0, opacity=0.5] (2.9935, 0.8974, 0.5471) -- (2.9935, 0.9476, 0.5506) -- (2.9920, 0.9471, 0.6006) -- (2.9920, 0.8968, 0.5971) -- cycle;
\fill[blue!15.0, opacity=0.5] (2.9920, 0.8968, 0.5971) -- (2.9920, 0.9471, 0.6006) -- (2.9903, 0.9465, 0.6506) -- (2.9903, 0.8961, 0.6471) -- cycle;
\fill[blue!15.0, opacity=0.5] (2.9903, 0.8961, 0.6471) -- (2.9903, 0.9465, 0.6506) -- (2.9885, 0.9458, 0.7006) -- (2.9885, 0.8954, 0.6971) -- cycle;
\fill[blue!15.0, opacity=0.5] (2.9885, 0.8954, 0.6971) -- (2.9885, 0.9458, 0.7006) -- (2.9866, 0.9451, 0.7506) -- (2.9866, 0.8947, 0.7471) -- cycle;
\fill[blue!15.0, opacity=0.5] (2.9866, 0.8947, 0.7471) -- (2.9866, 0.9451, 0.7506) -- (2.9846, 0.9443, 0.8006) -- (2.9846, 0.8938, 0.7971) -- cycle;
\fill[blue!15.0, opacity=0.5] (2.9846, 0.8938, 0.7971) -- (2.9846, 0.9443, 0.8006) -- (2.9824, 0.9436, 0.8506) -- (2.9824, 0.8930, 0.8471) -- cycle;
\fill[blue!15.0, opacity=0.5] (2.9824, 0.8930, 0.8471) -- (2.9824, 0.9436, 0.8506) -- (2.9801, 0.9427, 0.9006) -- (2.9801, 0.8921, 0.8971) -- cycle;
\fill[blue!15.0, opacity=0.5] (2.9801, 0.8921, 0.8971) -- (2.9801, 0.9427, 0.9006) -- (2.9778, 0.9418, 0.9506) -- (2.9778, 0.8911, 0.9471) -- cycle;
\fill[blue!15.0, opacity=0.5] (2.9778, 0.8911, 0.9471) -- (2.9778, 0.9418, 0.9506) -- (2.9753, 0.9409, 1.0006) -- (2.9753, 0.8901, 0.9971) -- cycle;
\fill[blue!15.0, opacity=0.5] (2.9753, 0.8901, 0.9971) -- (2.9753, 0.9409, 1.0006) -- (2.9727, 0.9400, 1.0506) -- (2.9727, 0.8891, 1.0471) -- cycle;
\fill[blue!15.0, opacity=0.5] (2.9727, 0.8891, 1.0471) -- (2.9727, 0.9400, 1.0506) -- (2.9700, 0.9390, 1.1006) -- (2.9700, 0.8880, 1.0971) -- cycle;
\fill[blue!15.0, opacity=0.5] (2.9700, 0.8880, 1.0971) -- (2.9700, 0.9390, 1.1006) -- (2.9672, 0.9380, 1.1506) -- (2.9672, 0.8869, 1.1471) -- cycle;
\fill[blue!15.0, opacity=0.5] (2.9672, 0.8869, 1.1471) -- (2.9672, 0.9380, 1.1506) -- (2.9644, 0.9369, 1.2006) -- (2.9644, 0.8858, 1.1971) -- cycle;
\fill[blue!15.0, opacity=0.5] (2.9644, 0.8858, 1.1971) -- (2.9644, 0.9369, 1.2006) -- (2.9615, 0.9359, 1.2506) -- (2.9615, 0.8846, 1.2471) -- cycle;
\fill[blue!15.0, opacity=0.5] (2.9615, 0.8846, 1.2471) -- (2.9615, 0.9359, 1.2506) -- (2.9585, 0.9348, 1.3006) -- (2.9585, 0.8834, 1.2971) -- cycle;
\fill[blue!15.0, opacity=0.5] (2.9585, 0.8834, 1.2971) -- (2.9585, 0.9348, 1.3006) -- (2.9555, 0.9337, 1.3506) -- (2.9555, 0.8822, 1.3471) -- cycle;
\fill[blue!15.0, opacity=0.5] (2.9555, 0.8822, 1.3471) -- (2.9555, 0.9337, 1.3506) -- (2.9525, 0.9326, 1.4006) -- (2.9525, 0.8810, 1.3971) -- cycle;
\fill[blue!15.0, opacity=0.5] (2.9525, 0.8810, 1.3971) -- (2.9525, 0.9326, 1.4006) -- (2.9494, 0.9314, 1.4506) -- (2.9494, 0.8798, 1.4471) -- cycle;
\fill[blue!15.0, opacity=0.5] (2.9494, 0.8798, 1.4471) -- (2.9494, 0.9314, 1.4506) -- (2.9463, 0.9303, 1.5006) -- (2.9463, 0.8785, 1.4971) -- cycle;
\fill[blue!15.0, opacity=0.5] (2.9463, 0.8785, 1.4971) -- (2.9463, 0.9303, 1.5006) -- (2.9431, 0.9292, 1.5506) -- (2.9431, 0.8773, 1.5471) -- cycle;
\fill[blue!15.0, opacity=0.5] (2.9431, 0.8773, 1.5471) -- (2.9431, 0.9292, 1.5506) -- (2.9400, 0.9280, 1.6006) -- (2.9400, 0.8760, 1.5971) -- cycle;
\fill[blue!15.0, opacity=0.5] (2.9400, 0.8760, 1.5971) -- (2.9400, 0.9280, 1.6006) -- (2.9369, 0.9268, 1.6506) -- (2.9369, 0.8747, 1.6471) -- cycle;
\fill[blue!15.0, opacity=0.5] (2.9369, 0.8747, 1.6471) -- (2.9369, 0.9268, 1.6506) -- (2.9337, 0.9257, 1.7006) -- (2.9337, 0.8735, 1.6971) -- cycle;
\fill[blue!15.0, opacity=0.5] (2.9337, 0.8735, 1.6971) -- (2.9337, 0.9257, 1.7006) -- (2.9306, 0.9246, 1.7506) -- (2.9306, 0.8722, 1.7471) -- cycle;
\fill[blue!15.0, opacity=0.5] (2.9306, 0.8722, 1.7471) -- (2.9306, 0.9246, 1.7506) -- (2.9275, 0.9234, 1.8006) -- (2.9275, 0.8710, 1.7971) -- cycle;
\fill[blue!15.0, opacity=0.5] (2.9275, 0.8710, 1.7971) -- (2.9275, 0.9234, 1.8006) -- (2.9245, 0.9223, 1.8506) -- (2.9245, 0.8698, 1.8471) -- cycle;
\fill[blue!15.0, opacity=0.5] (2.9245, 0.8698, 1.8471) -- (2.9245, 0.9223, 1.8506) -- (2.9215, 0.9212, 1.9006) -- (2.9215, 0.8686, 1.8971) -- cycle;
\fill[blue!15.0, opacity=0.5] (2.9215, 0.8686, 1.8971) -- (2.9215, 0.9212, 1.9006) -- (2.9185, 0.9201, 1.9506) -- (2.9185, 0.8674, 1.9471) -- cycle;
\fill[blue!15.0, opacity=0.5] (2.9185, 0.8674, 1.9471) -- (2.9185, 0.9201, 1.9506) -- (2.9156, 0.9191, 2.0006) -- (2.9156, 0.8662, 1.9971) -- cycle;
\fill[blue!15.0, opacity=0.5] (2.9156, 0.8662, 1.9971) -- (2.9156, 0.9191, 2.0006) -- (2.9128, 0.9180, 2.0506) -- (2.9128, 0.8651, 2.0471) -- cycle;
\fill[blue!15.0, opacity=0.5] (2.9128, 0.8651, 2.0471) -- (2.9128, 0.9180, 2.0506) -- (2.9100, 0.9170, 2.1006) -- (2.9100, 0.8640, 2.0971) -- cycle;
\fill[blue!15.0, opacity=0.5] (2.9100, 0.8640, 2.0971) -- (2.9100, 0.9170, 2.1006) -- (2.9073, 0.9160, 2.1506) -- (2.9073, 0.8629, 2.1471) -- cycle;
\fill[blue!15.0, opacity=0.5] (2.9073, 0.8629, 2.1471) -- (2.9073, 0.9160, 2.1506) -- (2.9047, 0.9151, 2.2006) -- (2.9047, 0.8619, 2.1971) -- cycle;
\fill[blue!15.0, opacity=0.5] (2.9047, 0.8619, 2.1971) -- (2.9047, 0.9151, 2.2006) -- (2.9022, 0.9142, 2.2506) -- (2.9022, 0.8609, 2.2471) -- cycle;
\fill[blue!15.0, opacity=0.5] (2.9022, 0.8609, 2.2471) -- (2.9022, 0.9142, 2.2506) -- (2.8999, 0.9133, 2.3006) -- (2.8999, 0.8599, 2.2971) -- cycle;
\fill[blue!15.0, opacity=0.5] (2.8999, 0.8599, 2.2971) -- (2.8999, 0.9133, 2.3006) -- (2.8976, 0.9124, 2.3506) -- (2.8976, 0.8590, 2.3471) -- cycle;
\fill[blue!15.1, opacity=0.5] (2.8976, 0.8590, 2.3471) -- (2.8976, 0.9124, 2.3506) -- (2.8954, 0.9117, 2.4006) -- (2.8954, 0.8582, 2.3971) -- cycle;
\fill[blue!15.1, opacity=0.5] (2.8954, 0.8582, 2.3971) -- (2.8954, 0.9117, 2.4006) -- (2.8934, 0.9109, 2.4506) -- (2.8934, 0.8573, 2.4471) -- cycle;
\fill[blue!15.1, opacity=0.5] (2.8934, 0.8573, 2.4471) -- (2.8934, 0.9109, 2.4506) -- (2.8915, 0.9102, 2.5006) -- (2.8915, 0.8566, 2.4971) -- cycle;
\fill[blue!15.1, opacity=0.5] (2.8915, 0.8566, 2.4971) -- (2.8915, 0.9102, 2.5006) -- (2.8897, 0.9095, 2.5506) -- (2.8897, 0.8559, 2.5471) -- cycle;
\fill[blue!15.2, opacity=0.5] (2.8897, 0.8559, 2.5471) -- (2.8897, 0.9095, 2.5506) -- (2.8880, 0.9089, 2.6006) -- (2.8880, 0.8552, 2.5971) -- cycle;
\fill[blue!15.3, opacity=0.5] (2.8880, 0.8552, 2.5971) -- (2.8880, 0.9089, 2.6006) -- (2.8865, 0.9084, 2.6506) -- (2.8865, 0.8546, 2.6471) -- cycle;
\fill[blue!15.3, opacity=0.5] (2.8865, 0.8546, 2.6471) -- (2.8865, 0.9084, 2.6506) -- (2.8852, 0.9079, 2.7006) -- (2.8852, 0.8541, 2.6971) -- cycle;
\fill[blue!15.4, opacity=0.5] (2.8852, 0.8541, 2.6971) -- (2.8852, 0.9079, 2.7006) -- (2.8840, 0.9075, 2.7506) -- (2.8840, 0.8536, 2.7471) -- cycle;
\fill[blue!15.5, opacity=0.5] (2.8840, 0.8536, 2.7471) -- (2.8840, 0.9075, 2.7506) -- (2.8829, 0.9071, 2.8006) -- (2.8829, 0.8532, 2.7971) -- cycle;
\fill[blue!15.6, opacity=0.5] (2.8829, 0.8532, 2.7971) -- (2.8829, 0.9071, 2.8006) -- (2.8820, 0.9067, 2.8506) -- (2.8820, 0.8528, 2.8471) -- cycle;
\fill[blue!15.8, opacity=0.5] (2.8820, 0.8528, 2.8471) -- (2.8820, 0.9067, 2.8506) -- (2.8813, 0.9065, 2.9006) -- (2.8813, 0.8525, 2.8971) -- cycle;
\fill[blue!16.0, opacity=0.5] (2.8813, 0.8525, 2.8971) -- (2.8813, 0.9065, 2.9006) -- (2.8807, 0.9063, 2.9506) -- (2.8807, 0.8523, 2.9471) -- cycle;
\fill[blue!16.1, opacity=0.5] (2.8807, 0.8523, 2.9471) -- (2.8807, 0.9063, 2.9506) -- (2.8803, 0.9061, 3.0006) -- (2.8803, 0.8521, 2.9971) -- cycle;
\fill[blue!16.4, opacity=0.5] (2.8803, 0.8521, 2.9971) -- (2.8803, 0.9061, 3.0006) -- (2.8801, 0.9060, 3.0506) -- (2.8801, 0.8520, 3.0471) -- cycle;
\fill[blue!16.6, opacity=0.5] (2.8801, 0.8520, 3.0471) -- (2.8801, 0.9060, 3.0506) -- (2.8800, 0.9060, 3.1006) -- (2.8800, 0.8520, 3.0971) -- cycle;
\fill[blue!15.0, opacity=0.5] (3.0000, 0.9500, 0.1006) -- (3.0000, 1.0000, 0.1039) -- (2.9999, 1.0000, 0.1539) -- (2.9999, 0.9500, 0.1506) -- cycle;
\fill[blue!15.0, opacity=0.5] (2.9999, 0.9500, 0.1506) -- (2.9999, 1.0000, 0.1539) -- (2.9997, 0.9999, 0.2039) -- (2.9997, 0.9499, 0.2006) -- cycle;
\fill[blue!15.0, opacity=0.5] (2.9997, 0.9499, 0.2006) -- (2.9997, 0.9999, 0.2039) -- (2.9993, 0.9998, 0.2539) -- (2.9993, 0.9497, 0.2506) -- cycle;
\fill[blue!15.0, opacity=0.5] (2.9993, 0.9497, 0.2506) -- (2.9993, 0.9998, 0.2539) -- (2.9987, 0.9996, 0.3039) -- (2.9987, 0.9495, 0.3006) -- cycle;
\fill[blue!15.0, opacity=0.5] (2.9987, 0.9495, 0.3006) -- (2.9987, 0.9996, 0.3039) -- (2.9980, 0.9993, 0.3539) -- (2.9980, 0.9493, 0.3506) -- cycle;
\fill[blue!15.0, opacity=0.5] (2.9980, 0.9493, 0.3506) -- (2.9980, 0.9993, 0.3539) -- (2.9971, 0.9990, 0.4039) -- (2.9971, 0.9489, 0.4006) -- cycle;
\fill[blue!15.0, opacity=0.5] (2.9971, 0.9489, 0.4006) -- (2.9971, 0.9990, 0.4039) -- (2.9960, 0.9987, 0.4539) -- (2.9960, 0.9485, 0.4506) -- cycle;
\fill[blue!15.0, opacity=0.5] (2.9960, 0.9485, 0.4506) -- (2.9960, 0.9987, 0.4539) -- (2.9948, 0.9983, 0.5039) -- (2.9948, 0.9481, 0.5006) -- cycle;
\fill[blue!15.0, opacity=0.5] (2.9948, 0.9481, 0.5006) -- (2.9948, 0.9983, 0.5039) -- (2.9935, 0.9978, 0.5539) -- (2.9935, 0.9476, 0.5506) -- cycle;
\fill[blue!15.0, opacity=0.5] (2.9935, 0.9476, 0.5506) -- (2.9935, 0.9978, 0.5539) -- (2.9920, 0.9973, 0.6039) -- (2.9920, 0.9471, 0.6006) -- cycle;
\fill[blue!15.0, opacity=0.5] (2.9920, 0.9471, 0.6006) -- (2.9920, 0.9973, 0.6039) -- (2.9903, 0.9968, 0.6539) -- (2.9903, 0.9465, 0.6506) -- cycle;
\fill[blue!15.0, opacity=0.5] (2.9903, 0.9465, 0.6506) -- (2.9903, 0.9968, 0.6539) -- (2.9885, 0.9962, 0.7039) -- (2.9885, 0.9458, 0.7006) -- cycle;
\fill[blue!15.0, opacity=0.5] (2.9885, 0.9458, 0.7006) -- (2.9885, 0.9962, 0.7039) -- (2.9866, 0.9955, 0.7539) -- (2.9866, 0.9451, 0.7506) -- cycle;
\fill[blue!15.0, opacity=0.5] (2.9866, 0.9451, 0.7506) -- (2.9866, 0.9955, 0.7539) -- (2.9846, 0.9949, 0.8039) -- (2.9846, 0.9443, 0.8006) -- cycle;
\fill[blue!15.0, opacity=0.5] (2.9846, 0.9443, 0.8006) -- (2.9846, 0.9949, 0.8039) -- (2.9824, 0.9941, 0.8539) -- (2.9824, 0.9436, 0.8506) -- cycle;
\fill[blue!15.0, opacity=0.5] (2.9824, 0.9436, 0.8506) -- (2.9824, 0.9941, 0.8539) -- (2.9801, 0.9934, 0.9039) -- (2.9801, 0.9427, 0.9006) -- cycle;
\fill[blue!15.0, opacity=0.5] (2.9801, 0.9427, 0.9006) -- (2.9801, 0.9934, 0.9039) -- (2.9778, 0.9926, 0.9539) -- (2.9778, 0.9418, 0.9506) -- cycle;
\fill[blue!15.0, opacity=0.5] (2.9778, 0.9418, 0.9506) -- (2.9778, 0.9926, 0.9539) -- (2.9753, 0.9918, 1.0039) -- (2.9753, 0.9409, 1.0006) -- cycle;
\fill[blue!15.0, opacity=0.5] (2.9753, 0.9409, 1.0006) -- (2.9753, 0.9918, 1.0039) -- (2.9727, 0.9909, 1.0539) -- (2.9727, 0.9400, 1.0506) -- cycle;
\fill[blue!15.0, opacity=0.5] (2.9727, 0.9400, 1.0506) -- (2.9727, 0.9909, 1.0539) -- (2.9700, 0.9900, 1.1039) -- (2.9700, 0.9390, 1.1006) -- cycle;
\fill[blue!15.0, opacity=0.5] (2.9700, 0.9390, 1.1006) -- (2.9700, 0.9900, 1.1039) -- (2.9672, 0.9891, 1.1539) -- (2.9672, 0.9380, 1.1506) -- cycle;
\fill[blue!15.0, opacity=0.5] (2.9672, 0.9380, 1.1506) -- (2.9672, 0.9891, 1.1539) -- (2.9644, 0.9881, 1.2039) -- (2.9644, 0.9369, 1.2006) -- cycle;
\fill[blue!15.0, opacity=0.5] (2.9644, 0.9369, 1.2006) -- (2.9644, 0.9881, 1.2039) -- (2.9615, 0.9872, 1.2539) -- (2.9615, 0.9359, 1.2506) -- cycle;
\fill[blue!15.0, opacity=0.5] (2.9615, 0.9359, 1.2506) -- (2.9615, 0.9872, 1.2539) -- (2.9585, 0.9862, 1.3039) -- (2.9585, 0.9348, 1.3006) -- cycle;
\fill[blue!15.0, opacity=0.5] (2.9585, 0.9348, 1.3006) -- (2.9585, 0.9862, 1.3039) -- (2.9555, 0.9852, 1.3539) -- (2.9555, 0.9337, 1.3506) -- cycle;
\fill[blue!15.0, opacity=0.5] (2.9555, 0.9337, 1.3506) -- (2.9555, 0.9852, 1.3539) -- (2.9525, 0.9842, 1.4039) -- (2.9525, 0.9326, 1.4006) -- cycle;
\fill[blue!15.0, opacity=0.5] (2.9525, 0.9326, 1.4006) -- (2.9525, 0.9842, 1.4039) -- (2.9494, 0.9831, 1.4539) -- (2.9494, 0.9314, 1.4506) -- cycle;
\fill[blue!15.0, opacity=0.5] (2.9494, 0.9314, 1.4506) -- (2.9494, 0.9831, 1.4539) -- (2.9463, 0.9821, 1.5039) -- (2.9463, 0.9303, 1.5006) -- cycle;
\fill[blue!15.0, opacity=0.5] (2.9463, 0.9303, 1.5006) -- (2.9463, 0.9821, 1.5039) -- (2.9431, 0.9810, 1.5539) -- (2.9431, 0.9292, 1.5506) -- cycle;
\fill[blue!15.0, opacity=0.5] (2.9431, 0.9292, 1.5506) -- (2.9431, 0.9810, 1.5539) -- (2.9400, 0.9800, 1.6039) -- (2.9400, 0.9280, 1.6006) -- cycle;
\fill[blue!15.0, opacity=0.5] (2.9400, 0.9280, 1.6006) -- (2.9400, 0.9800, 1.6039) -- (2.9369, 0.9790, 1.6539) -- (2.9369, 0.9268, 1.6506) -- cycle;
\fill[blue!15.0, opacity=0.5] (2.9369, 0.9268, 1.6506) -- (2.9369, 0.9790, 1.6539) -- (2.9337, 0.9779, 1.7039) -- (2.9337, 0.9257, 1.7006) -- cycle;
\fill[blue!15.0, opacity=0.5] (2.9337, 0.9257, 1.7006) -- (2.9337, 0.9779, 1.7039) -- (2.9306, 0.9769, 1.7539) -- (2.9306, 0.9246, 1.7506) -- cycle;
\fill[blue!15.0, opacity=0.5] (2.9306, 0.9246, 1.7506) -- (2.9306, 0.9769, 1.7539) -- (2.9275, 0.9758, 1.8039) -- (2.9275, 0.9234, 1.8006) -- cycle;
\fill[blue!15.0, opacity=0.5] (2.9275, 0.9234, 1.8006) -- (2.9275, 0.9758, 1.8039) -- (2.9245, 0.9748, 1.8539) -- (2.9245, 0.9223, 1.8506) -- cycle;
\fill[blue!15.0, opacity=0.5] (2.9245, 0.9223, 1.8506) -- (2.9245, 0.9748, 1.8539) -- (2.9215, 0.9738, 1.9039) -- (2.9215, 0.9212, 1.9006) -- cycle;
\fill[blue!15.0, opacity=0.5] (2.9215, 0.9212, 1.9006) -- (2.9215, 0.9738, 1.9039) -- (2.9185, 0.9728, 1.9539) -- (2.9185, 0.9201, 1.9506) -- cycle;
\fill[blue!15.0, opacity=0.5] (2.9185, 0.9201, 1.9506) -- (2.9185, 0.9728, 1.9539) -- (2.9156, 0.9719, 2.0039) -- (2.9156, 0.9191, 2.0006) -- cycle;
\fill[blue!15.0, opacity=0.5] (2.9156, 0.9191, 2.0006) -- (2.9156, 0.9719, 2.0039) -- (2.9128, 0.9709, 2.0539) -- (2.9128, 0.9180, 2.0506) -- cycle;
\fill[blue!15.0, opacity=0.5] (2.9128, 0.9180, 2.0506) -- (2.9128, 0.9709, 2.0539) -- (2.9100, 0.9700, 2.1039) -- (2.9100, 0.9170, 2.1006) -- cycle;
\fill[blue!15.0, opacity=0.5] (2.9100, 0.9170, 2.1006) -- (2.9100, 0.9700, 2.1039) -- (2.9073, 0.9691, 2.1539) -- (2.9073, 0.9160, 2.1506) -- cycle;
\fill[blue!15.0, opacity=0.5] (2.9073, 0.9160, 2.1506) -- (2.9073, 0.9691, 2.1539) -- (2.9047, 0.9682, 2.2039) -- (2.9047, 0.9151, 2.2006) -- cycle;
\fill[blue!15.0, opacity=0.5] (2.9047, 0.9151, 2.2006) -- (2.9047, 0.9682, 2.2039) -- (2.9022, 0.9674, 2.2539) -- (2.9022, 0.9142, 2.2506) -- cycle;
\fill[blue!15.0, opacity=0.5] (2.9022, 0.9142, 2.2506) -- (2.9022, 0.9674, 2.2539) -- (2.8999, 0.9666, 2.3039) -- (2.8999, 0.9133, 2.3006) -- cycle;
\fill[blue!15.1, opacity=0.5] (2.8999, 0.9133, 2.3006) -- (2.8999, 0.9666, 2.3039) -- (2.8976, 0.9659, 2.3539) -- (2.8976, 0.9124, 2.3506) -- cycle;
\fill[blue!15.1, opacity=0.5] (2.8976, 0.9124, 2.3506) -- (2.8976, 0.9659, 2.3539) -- (2.8954, 0.9651, 2.4039) -- (2.8954, 0.9117, 2.4006) -- cycle;
\fill[blue!15.1, opacity=0.5] (2.8954, 0.9117, 2.4006) -- (2.8954, 0.9651, 2.4039) -- (2.8934, 0.9645, 2.4539) -- (2.8934, 0.9109, 2.4506) -- cycle;
\fill[blue!15.2, opacity=0.5] (2.8934, 0.9109, 2.4506) -- (2.8934, 0.9645, 2.4539) -- (2.8915, 0.9638, 2.5039) -- (2.8915, 0.9102, 2.5006) -- cycle;
\fill[blue!15.2, opacity=0.5] (2.8915, 0.9102, 2.5006) -- (2.8915, 0.9638, 2.5039) -- (2.8897, 0.9632, 2.5539) -- (2.8897, 0.9095, 2.5506) -- cycle;
\fill[blue!15.3, opacity=0.5] (2.8897, 0.9095, 2.5506) -- (2.8897, 0.9632, 2.5539) -- (2.8880, 0.9627, 2.6039) -- (2.8880, 0.9089, 2.6006) -- cycle;
\fill[blue!15.4, opacity=0.5] (2.8880, 0.9089, 2.6006) -- (2.8880, 0.9627, 2.6039) -- (2.8865, 0.9622, 2.6539) -- (2.8865, 0.9084, 2.6506) -- cycle;
\fill[blue!15.5, opacity=0.5] (2.8865, 0.9084, 2.6506) -- (2.8865, 0.9622, 2.6539) -- (2.8852, 0.9617, 2.7039) -- (2.8852, 0.9079, 2.7006) -- cycle;
\fill[blue!15.6, opacity=0.5] (2.8852, 0.9079, 2.7006) -- (2.8852, 0.9617, 2.7039) -- (2.8840, 0.9613, 2.7539) -- (2.8840, 0.9075, 2.7506) -- cycle;
\fill[blue!15.7, opacity=0.5] (2.8840, 0.9075, 2.7506) -- (2.8840, 0.9613, 2.7539) -- (2.8829, 0.9610, 2.8039) -- (2.8829, 0.9071, 2.8006) -- cycle;
\fill[blue!15.9, opacity=0.5] (2.8829, 0.9071, 2.8006) -- (2.8829, 0.9610, 2.8039) -- (2.8820, 0.9607, 2.8539) -- (2.8820, 0.9067, 2.8506) -- cycle;
\fill[blue!16.1, opacity=0.5] (2.8820, 0.9067, 2.8506) -- (2.8820, 0.9607, 2.8539) -- (2.8813, 0.9604, 2.9039) -- (2.8813, 0.9065, 2.9006) -- cycle;
\fill[blue!16.3, opacity=0.5] (2.8813, 0.9065, 2.9006) -- (2.8813, 0.9604, 2.9039) -- (2.8807, 0.9602, 2.9539) -- (2.8807, 0.9063, 2.9506) -- cycle;
\fill[blue!16.6, opacity=0.5] (2.8807, 0.9063, 2.9506) -- (2.8807, 0.9602, 2.9539) -- (2.8803, 0.9601, 3.0039) -- (2.8803, 0.9061, 3.0006) -- cycle;
\fill[blue!16.8, opacity=0.5] (2.8803, 0.9061, 3.0006) -- (2.8803, 0.9601, 3.0039) -- (2.8801, 0.9600, 3.0539) -- (2.8801, 0.9060, 3.0506) -- cycle;
\fill[blue!17.1, opacity=0.5] (2.8801, 0.9060, 3.0506) -- (2.8801, 0.9600, 3.0539) -- (2.8800, 0.9600, 3.1039) -- (2.8800, 0.9060, 3.1006) -- cycle;
\fill[blue!15.0, opacity=0.5] (3.0000, 1.0000, 0.1039) -- (3.0000, 1.0500, 0.1069) -- (2.9999, 1.0500, 0.1569) -- (2.9999, 1.0000, 0.1539) -- cycle;
\fill[blue!15.0, opacity=0.5] (2.9999, 1.0000, 0.1539) -- (2.9999, 1.0500, 0.1569) -- (2.9997, 1.0499, 0.2069) -- (2.9997, 0.9999, 0.2039) -- cycle;
\fill[blue!15.0, opacity=0.5] (2.9997, 0.9999, 0.2039) -- (2.9997, 1.0499, 0.2069) -- (2.9993, 1.0498, 0.2569) -- (2.9993, 0.9998, 0.2539) -- cycle;
\fill[blue!15.0, opacity=0.5] (2.9993, 0.9998, 0.2539) -- (2.9993, 1.0498, 0.2569) -- (2.9987, 1.0496, 0.3069) -- (2.9987, 0.9996, 0.3039) -- cycle;
\fill[blue!15.0, opacity=0.5] (2.9987, 0.9996, 0.3039) -- (2.9987, 1.0496, 0.3069) -- (2.9980, 1.0494, 0.3569) -- (2.9980, 0.9993, 0.3539) -- cycle;
\fill[blue!15.0, opacity=0.5] (2.9980, 0.9993, 0.3539) -- (2.9980, 1.0494, 0.3569) -- (2.9971, 1.0491, 0.4069) -- (2.9971, 0.9990, 0.4039) -- cycle;
\fill[blue!15.0, opacity=0.5] (2.9971, 0.9990, 0.4039) -- (2.9971, 1.0491, 0.4069) -- (2.9960, 1.0488, 0.4569) -- (2.9960, 0.9987, 0.4539) -- cycle;
\fill[blue!15.0, opacity=0.5] (2.9960, 0.9987, 0.4539) -- (2.9960, 1.0488, 0.4569) -- (2.9948, 1.0484, 0.5069) -- (2.9948, 0.9983, 0.5039) -- cycle;
\fill[blue!15.0, opacity=0.5] (2.9948, 0.9983, 0.5039) -- (2.9948, 1.0484, 0.5069) -- (2.9935, 1.0480, 0.5569) -- (2.9935, 0.9978, 0.5539) -- cycle;
\fill[blue!15.0, opacity=0.5] (2.9935, 0.9978, 0.5539) -- (2.9935, 1.0480, 0.5569) -- (2.9920, 1.0476, 0.6069) -- (2.9920, 0.9973, 0.6039) -- cycle;
\fill[blue!15.0, opacity=0.5] (2.9920, 0.9973, 0.6039) -- (2.9920, 1.0476, 0.6069) -- (2.9903, 1.0471, 0.6569) -- (2.9903, 0.9968, 0.6539) -- cycle;
\fill[blue!15.0, opacity=0.5] (2.9903, 0.9968, 0.6539) -- (2.9903, 1.0471, 0.6569) -- (2.9885, 1.0466, 0.7069) -- (2.9885, 0.9962, 0.7039) -- cycle;
\fill[blue!15.0, opacity=0.5] (2.9885, 0.9962, 0.7039) -- (2.9885, 1.0466, 0.7069) -- (2.9866, 1.0460, 0.7569) -- (2.9866, 0.9955, 0.7539) -- cycle;
\fill[blue!15.0, opacity=0.5] (2.9866, 0.9955, 0.7539) -- (2.9866, 1.0460, 0.7569) -- (2.9846, 1.0454, 0.8069) -- (2.9846, 0.9949, 0.8039) -- cycle;
\fill[blue!15.0, opacity=0.5] (2.9846, 0.9949, 0.8039) -- (2.9846, 1.0454, 0.8069) -- (2.9824, 1.0447, 0.8569) -- (2.9824, 0.9941, 0.8539) -- cycle;
\fill[blue!15.0, opacity=0.5] (2.9824, 0.9941, 0.8539) -- (2.9824, 1.0447, 0.8569) -- (2.9801, 1.0440, 0.9069) -- (2.9801, 0.9934, 0.9039) -- cycle;
\fill[blue!15.0, opacity=0.5] (2.9801, 0.9934, 0.9039) -- (2.9801, 1.0440, 0.9069) -- (2.9778, 1.0433, 0.9569) -- (2.9778, 0.9926, 0.9539) -- cycle;
\fill[blue!15.0, opacity=0.5] (2.9778, 0.9926, 0.9539) -- (2.9778, 1.0433, 0.9569) -- (2.9753, 1.0426, 1.0069) -- (2.9753, 0.9918, 1.0039) -- cycle;
\fill[blue!15.0, opacity=0.5] (2.9753, 0.9918, 1.0039) -- (2.9753, 1.0426, 1.0069) -- (2.9727, 1.0418, 1.0569) -- (2.9727, 0.9909, 1.0539) -- cycle;
\fill[blue!15.0, opacity=0.5] (2.9727, 0.9909, 1.0539) -- (2.9727, 1.0418, 1.0569) -- (2.9700, 1.0410, 1.1069) -- (2.9700, 0.9900, 1.1039) -- cycle;
\fill[blue!15.0, opacity=0.5] (2.9700, 0.9900, 1.1039) -- (2.9700, 1.0410, 1.1069) -- (2.9672, 1.0402, 1.1569) -- (2.9672, 0.9891, 1.1539) -- cycle;
\fill[blue!15.0, opacity=0.5] (2.9672, 0.9891, 1.1539) -- (2.9672, 1.0402, 1.1569) -- (2.9644, 1.0393, 1.2069) -- (2.9644, 0.9881, 1.2039) -- cycle;
\fill[blue!15.0, opacity=0.5] (2.9644, 0.9881, 1.2039) -- (2.9644, 1.0393, 1.2069) -- (2.9615, 1.0385, 1.2569) -- (2.9615, 0.9872, 1.2539) -- cycle;
\fill[blue!15.0, opacity=0.5] (2.9615, 0.9872, 1.2539) -- (2.9615, 1.0385, 1.2569) -- (2.9585, 1.0376, 1.3069) -- (2.9585, 0.9862, 1.3039) -- cycle;
\fill[blue!15.0, opacity=0.5] (2.9585, 0.9862, 1.3039) -- (2.9585, 1.0376, 1.3069) -- (2.9555, 1.0367, 1.3569) -- (2.9555, 0.9852, 1.3539) -- cycle;
\fill[blue!15.0, opacity=0.5] (2.9555, 0.9852, 1.3539) -- (2.9555, 1.0367, 1.3569) -- (2.9525, 1.0357, 1.4069) -- (2.9525, 0.9842, 1.4039) -- cycle;
\fill[blue!15.0, opacity=0.5] (2.9525, 0.9842, 1.4039) -- (2.9525, 1.0357, 1.4069) -- (2.9494, 1.0348, 1.4569) -- (2.9494, 0.9831, 1.4539) -- cycle;
\fill[blue!15.0, opacity=0.5] (2.9494, 0.9831, 1.4539) -- (2.9494, 1.0348, 1.4569) -- (2.9463, 1.0339, 1.5069) -- (2.9463, 0.9821, 1.5039) -- cycle;
\fill[blue!15.0, opacity=0.5] (2.9463, 0.9821, 1.5039) -- (2.9463, 1.0339, 1.5069) -- (2.9431, 1.0329, 1.5569) -- (2.9431, 0.9810, 1.5539) -- cycle;
\fill[blue!15.0, opacity=0.5] (2.9431, 0.9810, 1.5539) -- (2.9431, 1.0329, 1.5569) -- (2.9400, 1.0320, 1.6069) -- (2.9400, 0.9800, 1.6039) -- cycle;
\fill[blue!15.0, opacity=0.5] (2.9400, 0.9800, 1.6039) -- (2.9400, 1.0320, 1.6069) -- (2.9369, 1.0311, 1.6569) -- (2.9369, 0.9790, 1.6539) -- cycle;
\fill[blue!15.0, opacity=0.5] (2.9369, 0.9790, 1.6539) -- (2.9369, 1.0311, 1.6569) -- (2.9337, 1.0301, 1.7069) -- (2.9337, 0.9779, 1.7039) -- cycle;
\fill[blue!15.0, opacity=0.5] (2.9337, 0.9779, 1.7039) -- (2.9337, 1.0301, 1.7069) -- (2.9306, 1.0292, 1.7569) -- (2.9306, 0.9769, 1.7539) -- cycle;
\fill[blue!15.0, opacity=0.5] (2.9306, 0.9769, 1.7539) -- (2.9306, 1.0292, 1.7569) -- (2.9275, 1.0283, 1.8069) -- (2.9275, 0.9758, 1.8039) -- cycle;
\fill[blue!15.0, opacity=0.5] (2.9275, 0.9758, 1.8039) -- (2.9275, 1.0283, 1.8069) -- (2.9245, 1.0273, 1.8569) -- (2.9245, 0.9748, 1.8539) -- cycle;
\fill[blue!15.0, opacity=0.5] (2.9245, 0.9748, 1.8539) -- (2.9245, 1.0273, 1.8569) -- (2.9215, 1.0264, 1.9069) -- (2.9215, 0.9738, 1.9039) -- cycle;
\fill[blue!15.0, opacity=0.5] (2.9215, 0.9738, 1.9039) -- (2.9215, 1.0264, 1.9069) -- (2.9185, 1.0255, 1.9569) -- (2.9185, 0.9728, 1.9539) -- cycle;
\fill[blue!15.0, opacity=0.5] (2.9185, 0.9728, 1.9539) -- (2.9185, 1.0255, 1.9569) -- (2.9156, 1.0247, 2.0069) -- (2.9156, 0.9719, 2.0039) -- cycle;
\fill[blue!15.0, opacity=0.5] (2.9156, 0.9719, 2.0039) -- (2.9156, 1.0247, 2.0069) -- (2.9128, 1.0238, 2.0569) -- (2.9128, 0.9709, 2.0539) -- cycle;
\fill[blue!15.0, opacity=0.5] (2.9128, 0.9709, 2.0539) -- (2.9128, 1.0238, 2.0569) -- (2.9100, 1.0230, 2.1069) -- (2.9100, 0.9700, 2.1039) -- cycle;
\fill[blue!15.0, opacity=0.5] (2.9100, 0.9700, 2.1039) -- (2.9100, 1.0230, 2.1069) -- (2.9073, 1.0222, 2.1569) -- (2.9073, 0.9691, 2.1539) -- cycle;
\fill[blue!15.0, opacity=0.5] (2.9073, 0.9691, 2.1539) -- (2.9073, 1.0222, 2.1569) -- (2.9047, 1.0214, 2.2069) -- (2.9047, 0.9682, 2.2039) -- cycle;
\fill[blue!15.1, opacity=0.5] (2.9047, 0.9682, 2.2039) -- (2.9047, 1.0214, 2.2069) -- (2.9022, 1.0207, 2.2569) -- (2.9022, 0.9674, 2.2539) -- cycle;
\fill[blue!15.1, opacity=0.5] (2.9022, 0.9674, 2.2539) -- (2.9022, 1.0207, 2.2569) -- (2.8999, 1.0200, 2.3069) -- (2.8999, 0.9666, 2.3039) -- cycle;
\fill[blue!15.1, opacity=0.5] (2.8999, 0.9666, 2.3039) -- (2.8999, 1.0200, 2.3069) -- (2.8976, 1.0193, 2.3569) -- (2.8976, 0.9659, 2.3539) -- cycle;
\fill[blue!15.2, opacity=0.5] (2.8976, 0.9659, 2.3539) -- (2.8976, 1.0193, 2.3569) -- (2.8954, 1.0186, 2.4069) -- (2.8954, 0.9651, 2.4039) -- cycle;
\fill[blue!15.2, opacity=0.5] (2.8954, 0.9651, 2.4039) -- (2.8954, 1.0186, 2.4069) -- (2.8934, 1.0180, 2.4569) -- (2.8934, 0.9645, 2.4539) -- cycle;
\fill[blue!15.3, opacity=0.5] (2.8934, 0.9645, 2.4539) -- (2.8934, 1.0180, 2.4569) -- (2.8915, 1.0174, 2.5069) -- (2.8915, 0.9638, 2.5039) -- cycle;
\fill[blue!15.4, opacity=0.5] (2.8915, 0.9638, 2.5039) -- (2.8915, 1.0174, 2.5069) -- (2.8897, 1.0169, 2.5569) -- (2.8897, 0.9632, 2.5539) -- cycle;
\fill[blue!15.5, opacity=0.5] (2.8897, 0.9632, 2.5539) -- (2.8897, 1.0169, 2.5569) -- (2.8880, 1.0164, 2.6069) -- (2.8880, 0.9627, 2.6039) -- cycle;
\fill[blue!15.6, opacity=0.5] (2.8880, 0.9627, 2.6039) -- (2.8880, 1.0164, 2.6069) -- (2.8865, 1.0160, 2.6569) -- (2.8865, 0.9622, 2.6539) -- cycle;
\fill[blue!15.8, opacity=0.5] (2.8865, 0.9622, 2.6539) -- (2.8865, 1.0160, 2.6569) -- (2.8852, 1.0156, 2.7069) -- (2.8852, 0.9617, 2.7039) -- cycle;
\fill[blue!15.9, opacity=0.5] (2.8852, 0.9617, 2.7039) -- (2.8852, 1.0156, 2.7069) -- (2.8840, 1.0152, 2.7569) -- (2.8840, 0.9613, 2.7539) -- cycle;
\fill[blue!16.1, opacity=0.5] (2.8840, 0.9613, 2.7539) -- (2.8840, 1.0152, 2.7569) -- (2.8829, 1.0149, 2.8069) -- (2.8829, 0.9610, 2.8039) -- cycle;
\fill[blue!16.4, opacity=0.5] (2.8829, 0.9610, 2.8039) -- (2.8829, 1.0149, 2.8069) -- (2.8820, 1.0146, 2.8569) -- (2.8820, 0.9607, 2.8539) -- cycle;
\fill[blue!16.6, opacity=0.5] (2.8820, 0.9607, 2.8539) -- (2.8820, 1.0146, 2.8569) -- (2.8813, 1.0144, 2.9069) -- (2.8813, 0.9604, 2.9039) -- cycle;
\fill[blue!16.9, opacity=0.5] (2.8813, 0.9604, 2.9039) -- (2.8813, 1.0144, 2.9069) -- (2.8807, 1.0142, 2.9569) -- (2.8807, 0.9602, 2.9539) -- cycle;
\fill[blue!17.2, opacity=0.5] (2.8807, 0.9602, 2.9539) -- (2.8807, 1.0142, 2.9569) -- (2.8803, 1.0141, 3.0069) -- (2.8803, 0.9601, 3.0039) -- cycle;
\fill[blue!17.6, opacity=0.5] (2.8803, 0.9601, 3.0039) -- (2.8803, 1.0141, 3.0069) -- (2.8801, 1.0140, 3.0569) -- (2.8801, 0.9600, 3.0539) -- cycle;
\fill[blue!17.9, opacity=0.5] (2.8801, 0.9600, 3.0539) -- (2.8801, 1.0140, 3.0569) -- (2.8800, 1.0140, 3.1069) -- (2.8800, 0.9600, 3.1039) -- cycle;
\fill[blue!15.0, opacity=0.5] (3.0000, 1.0500, 0.1069) -- (3.0000, 1.1000, 0.1096) -- (2.9999, 1.1000, 0.1596) -- (2.9999, 1.0500, 0.1569) -- cycle;
\fill[blue!15.0, opacity=0.5] (2.9999, 1.0500, 0.1569) -- (2.9999, 1.1000, 0.1596) -- (2.9997, 1.0999, 0.2096) -- (2.9997, 1.0499, 0.2069) -- cycle;
\fill[blue!15.0, opacity=0.5] (2.9997, 1.0499, 0.2069) -- (2.9997, 1.0999, 0.2096) -- (2.9993, 1.0998, 0.2596) -- (2.9993, 1.0498, 0.2569) -- cycle;
\fill[blue!15.0, opacity=0.5] (2.9993, 1.0498, 0.2569) -- (2.9993, 1.0998, 0.2596) -- (2.9987, 1.0997, 0.3096) -- (2.9987, 1.0496, 0.3069) -- cycle;
\fill[blue!15.0, opacity=0.5] (2.9987, 1.0496, 0.3069) -- (2.9987, 1.0997, 0.3096) -- (2.9980, 1.0995, 0.3596) -- (2.9980, 1.0494, 0.3569) -- cycle;
\fill[blue!15.0, opacity=0.5] (2.9980, 1.0494, 0.3569) -- (2.9980, 1.0995, 0.3596) -- (2.9971, 1.0992, 0.4096) -- (2.9971, 1.0491, 0.4069) -- cycle;
\fill[blue!15.0, opacity=0.5] (2.9971, 1.0491, 0.4069) -- (2.9971, 1.0992, 0.4096) -- (2.9960, 1.0989, 0.4596) -- (2.9960, 1.0488, 0.4569) -- cycle;
\fill[blue!15.0, opacity=0.5] (2.9960, 1.0488, 0.4569) -- (2.9960, 1.0989, 0.4596) -- (2.9948, 1.0986, 0.5096) -- (2.9948, 1.0484, 0.5069) -- cycle;
\fill[blue!15.0, opacity=0.5] (2.9948, 1.0484, 0.5069) -- (2.9948, 1.0986, 0.5096) -- (2.9935, 1.0983, 0.5596) -- (2.9935, 1.0480, 0.5569) -- cycle;
\fill[blue!15.0, opacity=0.5] (2.9935, 1.0480, 0.5569) -- (2.9935, 1.0983, 0.5596) -- (2.9920, 1.0979, 0.6096) -- (2.9920, 1.0476, 0.6069) -- cycle;
\fill[blue!15.0, opacity=0.5] (2.9920, 1.0476, 0.6069) -- (2.9920, 1.0979, 0.6096) -- (2.9903, 1.0974, 0.6596) -- (2.9903, 1.0471, 0.6569) -- cycle;
\fill[blue!15.0, opacity=0.5] (2.9903, 1.0471, 0.6569) -- (2.9903, 1.0974, 0.6596) -- (2.9885, 1.0969, 0.7096) -- (2.9885, 1.0466, 0.7069) -- cycle;
\fill[blue!15.0, opacity=0.5] (2.9885, 1.0466, 0.7069) -- (2.9885, 1.0969, 0.7096) -- (2.9866, 1.0964, 0.7596) -- (2.9866, 1.0460, 0.7569) -- cycle;
\fill[blue!15.0, opacity=0.5] (2.9866, 1.0460, 0.7569) -- (2.9866, 1.0964, 0.7596) -- (2.9846, 1.0959, 0.8096) -- (2.9846, 1.0454, 0.8069) -- cycle;
\fill[blue!15.0, opacity=0.5] (2.9846, 1.0454, 0.8069) -- (2.9846, 1.0959, 0.8096) -- (2.9824, 1.0953, 0.8596) -- (2.9824, 1.0447, 0.8569) -- cycle;
\fill[blue!15.0, opacity=0.5] (2.9824, 1.0447, 0.8569) -- (2.9824, 1.0953, 0.8596) -- (2.9801, 1.0947, 0.9096) -- (2.9801, 1.0440, 0.9069) -- cycle;
\fill[blue!15.0, opacity=0.5] (2.9801, 1.0440, 0.9069) -- (2.9801, 1.0947, 0.9096) -- (2.9778, 1.0941, 0.9596) -- (2.9778, 1.0433, 0.9569) -- cycle;
\fill[blue!15.0, opacity=0.5] (2.9778, 1.0433, 0.9569) -- (2.9778, 1.0941, 0.9596) -- (2.9753, 1.0934, 1.0096) -- (2.9753, 1.0426, 1.0069) -- cycle;
\fill[blue!15.0, opacity=0.5] (2.9753, 1.0426, 1.0069) -- (2.9753, 1.0934, 1.0096) -- (2.9727, 1.0927, 1.0596) -- (2.9727, 1.0418, 1.0569) -- cycle;
\fill[blue!15.0, opacity=0.5] (2.9727, 1.0418, 1.0569) -- (2.9727, 1.0927, 1.0596) -- (2.9700, 1.0920, 1.1096) -- (2.9700, 1.0410, 1.1069) -- cycle;
\fill[blue!15.0, opacity=0.5] (2.9700, 1.0410, 1.1069) -- (2.9700, 1.0920, 1.1096) -- (2.9672, 1.0913, 1.1596) -- (2.9672, 1.0402, 1.1569) -- cycle;
\fill[blue!15.0, opacity=0.5] (2.9672, 1.0402, 1.1569) -- (2.9672, 1.0913, 1.1596) -- (2.9644, 1.0905, 1.2096) -- (2.9644, 1.0393, 1.2069) -- cycle;
\fill[blue!15.0, opacity=0.5] (2.9644, 1.0393, 1.2069) -- (2.9644, 1.0905, 1.2096) -- (2.9615, 1.0897, 1.2596) -- (2.9615, 1.0385, 1.2569) -- cycle;
\fill[blue!15.0, opacity=0.5] (2.9615, 1.0385, 1.2569) -- (2.9615, 1.0897, 1.2596) -- (2.9585, 1.0889, 1.3096) -- (2.9585, 1.0376, 1.3069) -- cycle;
\fill[blue!15.0, opacity=0.5] (2.9585, 1.0376, 1.3069) -- (2.9585, 1.0889, 1.3096) -- (2.9555, 1.0881, 1.3596) -- (2.9555, 1.0367, 1.3569) -- cycle;
\fill[blue!15.0, opacity=0.5] (2.9555, 1.0367, 1.3569) -- (2.9555, 1.0881, 1.3596) -- (2.9525, 1.0873, 1.4096) -- (2.9525, 1.0357, 1.4069) -- cycle;
\fill[blue!15.0, opacity=0.5] (2.9525, 1.0357, 1.4069) -- (2.9525, 1.0873, 1.4096) -- (2.9494, 1.0865, 1.4596) -- (2.9494, 1.0348, 1.4569) -- cycle;
\fill[blue!15.0, opacity=0.5] (2.9494, 1.0348, 1.4569) -- (2.9494, 1.0865, 1.4596) -- (2.9463, 1.0857, 1.5096) -- (2.9463, 1.0339, 1.5069) -- cycle;
\fill[blue!15.0, opacity=0.5] (2.9463, 1.0339, 1.5069) -- (2.9463, 1.0857, 1.5096) -- (2.9431, 1.0848, 1.5596) -- (2.9431, 1.0329, 1.5569) -- cycle;
\fill[blue!15.0, opacity=0.5] (2.9431, 1.0329, 1.5569) -- (2.9431, 1.0848, 1.5596) -- (2.9400, 1.0840, 1.6096) -- (2.9400, 1.0320, 1.6069) -- cycle;
\fill[blue!15.0, opacity=0.5] (2.9400, 1.0320, 1.6069) -- (2.9400, 1.0840, 1.6096) -- (2.9369, 1.0832, 1.6596) -- (2.9369, 1.0311, 1.6569) -- cycle;
\fill[blue!15.0, opacity=0.5] (2.9369, 1.0311, 1.6569) -- (2.9369, 1.0832, 1.6596) -- (2.9337, 1.0823, 1.7096) -- (2.9337, 1.0301, 1.7069) -- cycle;
\fill[blue!15.0, opacity=0.5] (2.9337, 1.0301, 1.7069) -- (2.9337, 1.0823, 1.7096) -- (2.9306, 1.0815, 1.7596) -- (2.9306, 1.0292, 1.7569) -- cycle;
\fill[blue!15.0, opacity=0.5] (2.9306, 1.0292, 1.7569) -- (2.9306, 1.0815, 1.7596) -- (2.9275, 1.0807, 1.8096) -- (2.9275, 1.0283, 1.8069) -- cycle;
\fill[blue!15.0, opacity=0.5] (2.9275, 1.0283, 1.8069) -- (2.9275, 1.0807, 1.8096) -- (2.9245, 1.0799, 1.8596) -- (2.9245, 1.0273, 1.8569) -- cycle;
\fill[blue!15.0, opacity=0.5] (2.9245, 1.0273, 1.8569) -- (2.9245, 1.0799, 1.8596) -- (2.9215, 1.0791, 1.9096) -- (2.9215, 1.0264, 1.9069) -- cycle;
\fill[blue!15.0, opacity=0.5] (2.9215, 1.0264, 1.9069) -- (2.9215, 1.0791, 1.9096) -- (2.9185, 1.0783, 1.9596) -- (2.9185, 1.0255, 1.9569) -- cycle;
\fill[blue!15.0, opacity=0.5] (2.9185, 1.0255, 1.9569) -- (2.9185, 1.0783, 1.9596) -- (2.9156, 1.0775, 2.0096) -- (2.9156, 1.0247, 2.0069) -- cycle;
\fill[blue!15.0, opacity=0.5] (2.9156, 1.0247, 2.0069) -- (2.9156, 1.0775, 2.0096) -- (2.9128, 1.0767, 2.0596) -- (2.9128, 1.0238, 2.0569) -- cycle;
\fill[blue!15.0, opacity=0.5] (2.9128, 1.0238, 2.0569) -- (2.9128, 1.0767, 2.0596) -- (2.9100, 1.0760, 2.1096) -- (2.9100, 1.0230, 2.1069) -- cycle;
\fill[blue!15.1, opacity=0.5] (2.9100, 1.0230, 2.1069) -- (2.9100, 1.0760, 2.1096) -- (2.9073, 1.0753, 2.1596) -- (2.9073, 1.0222, 2.1569) -- cycle;
\fill[blue!15.1, opacity=0.5] (2.9073, 1.0222, 2.1569) -- (2.9073, 1.0753, 2.1596) -- (2.9047, 1.0746, 2.2096) -- (2.9047, 1.0214, 2.2069) -- cycle;
\fill[blue!15.1, opacity=0.5] (2.9047, 1.0214, 2.2069) -- (2.9047, 1.0746, 2.2096) -- (2.9022, 1.0739, 2.2596) -- (2.9022, 1.0207, 2.2569) -- cycle;
\fill[blue!15.2, opacity=0.5] (2.9022, 1.0207, 2.2569) -- (2.9022, 1.0739, 2.2596) -- (2.8999, 1.0733, 2.3096) -- (2.8999, 1.0200, 2.3069) -- cycle;
\fill[blue!15.2, opacity=0.5] (2.8999, 1.0200, 2.3069) -- (2.8999, 1.0733, 2.3096) -- (2.8976, 1.0727, 2.3596) -- (2.8976, 1.0193, 2.3569) -- cycle;
\fill[blue!15.3, opacity=0.5] (2.8976, 1.0193, 2.3569) -- (2.8976, 1.0727, 2.3596) -- (2.8954, 1.0721, 2.4096) -- (2.8954, 1.0186, 2.4069) -- cycle;
\fill[blue!15.4, opacity=0.5] (2.8954, 1.0186, 2.4069) -- (2.8954, 1.0721, 2.4096) -- (2.8934, 1.0716, 2.4596) -- (2.8934, 1.0180, 2.4569) -- cycle;
\fill[blue!15.5, opacity=0.5] (2.8934, 1.0180, 2.4569) -- (2.8934, 1.0716, 2.4596) -- (2.8915, 1.0711, 2.5096) -- (2.8915, 1.0174, 2.5069) -- cycle;
\fill[blue!15.6, opacity=0.5] (2.8915, 1.0174, 2.5069) -- (2.8915, 1.0711, 2.5096) -- (2.8897, 1.0706, 2.5596) -- (2.8897, 1.0169, 2.5569) -- cycle;
\fill[blue!15.8, opacity=0.5] (2.8897, 1.0169, 2.5569) -- (2.8897, 1.0706, 2.5596) -- (2.8880, 1.0701, 2.6096) -- (2.8880, 1.0164, 2.6069) -- cycle;
\fill[blue!16.0, opacity=0.5] (2.8880, 1.0164, 2.6069) -- (2.8880, 1.0701, 2.6096) -- (2.8865, 1.0697, 2.6596) -- (2.8865, 1.0160, 2.6569) -- cycle;
\fill[blue!16.2, opacity=0.5] (2.8865, 1.0160, 2.6569) -- (2.8865, 1.0697, 2.6596) -- (2.8852, 1.0694, 2.7096) -- (2.8852, 1.0156, 2.7069) -- cycle;
\fill[blue!16.4, opacity=0.5] (2.8852, 1.0156, 2.7069) -- (2.8852, 1.0694, 2.7096) -- (2.8840, 1.0691, 2.7596) -- (2.8840, 1.0152, 2.7569) -- cycle;
\fill[blue!16.7, opacity=0.5] (2.8840, 1.0152, 2.7569) -- (2.8840, 1.0691, 2.7596) -- (2.8829, 1.0688, 2.8096) -- (2.8829, 1.0149, 2.8069) -- cycle;
\fill[blue!17.0, opacity=0.5] (2.8829, 1.0149, 2.8069) -- (2.8829, 1.0688, 2.8096) -- (2.8820, 1.0685, 2.8596) -- (2.8820, 1.0146, 2.8569) -- cycle;
\fill[blue!17.4, opacity=0.5] (2.8820, 1.0146, 2.8569) -- (2.8820, 1.0685, 2.8596) -- (2.8813, 1.0683, 2.9096) -- (2.8813, 1.0144, 2.9069) -- cycle;
\fill[blue!17.8, opacity=0.5] (2.8813, 1.0144, 2.9069) -- (2.8813, 1.0683, 2.9096) -- (2.8807, 1.0682, 2.9596) -- (2.8807, 1.0142, 2.9569) -- cycle;
\fill[blue!18.2, opacity=0.5] (2.8807, 1.0142, 2.9569) -- (2.8807, 1.0682, 2.9596) -- (2.8803, 1.0681, 3.0096) -- (2.8803, 1.0141, 3.0069) -- cycle;
\fill[blue!18.6, opacity=0.5] (2.8803, 1.0141, 3.0069) -- (2.8803, 1.0681, 3.0096) -- (2.8801, 1.0680, 3.0596) -- (2.8801, 1.0140, 3.0569) -- cycle;
\fill[blue!19.1, opacity=0.5] (2.8801, 1.0140, 3.0569) -- (2.8801, 1.0680, 3.0596) -- (2.8800, 1.0680, 3.1096) -- (2.8800, 1.0140, 3.1069) -- cycle;
\fill[blue!15.0, opacity=0.5] (3.0000, 1.1000, 0.1096) -- (3.0000, 1.1500, 0.1120) -- (2.9999, 1.1500, 0.1620) -- (2.9999, 1.1000, 0.1596) -- cycle;
\fill[blue!15.0, opacity=0.5] (2.9999, 1.1000, 0.1596) -- (2.9999, 1.1500, 0.1620) -- (2.9997, 1.1499, 0.2120) -- (2.9997, 1.0999, 0.2096) -- cycle;
\fill[blue!15.0, opacity=0.5] (2.9997, 1.0999, 0.2096) -- (2.9997, 1.1499, 0.2120) -- (2.9993, 1.1498, 0.2620) -- (2.9993, 1.0998, 0.2596) -- cycle;
\fill[blue!15.0, opacity=0.5] (2.9993, 1.0998, 0.2596) -- (2.9993, 1.1498, 0.2620) -- (2.9987, 1.1497, 0.3120) -- (2.9987, 1.0997, 0.3096) -- cycle;
\fill[blue!15.0, opacity=0.5] (2.9987, 1.0997, 0.3096) -- (2.9987, 1.1497, 0.3120) -- (2.9980, 1.1495, 0.3620) -- (2.9980, 1.0995, 0.3596) -- cycle;
\fill[blue!15.0, opacity=0.5] (2.9980, 1.0995, 0.3596) -- (2.9980, 1.1495, 0.3620) -- (2.9971, 1.1493, 0.4120) -- (2.9971, 1.0992, 0.4096) -- cycle;
\fill[blue!15.0, opacity=0.5] (2.9971, 1.0992, 0.4096) -- (2.9971, 1.1493, 0.4120) -- (2.9960, 1.1491, 0.4620) -- (2.9960, 1.0989, 0.4596) -- cycle;
\fill[blue!15.0, opacity=0.5] (2.9960, 1.0989, 0.4596) -- (2.9960, 1.1491, 0.4620) -- (2.9948, 1.1488, 0.5120) -- (2.9948, 1.0986, 0.5096) -- cycle;
\fill[blue!15.0, opacity=0.5] (2.9948, 1.0986, 0.5096) -- (2.9948, 1.1488, 0.5120) -- (2.9935, 1.1485, 0.5620) -- (2.9935, 1.0983, 0.5596) -- cycle;
\fill[blue!15.0, opacity=0.5] (2.9935, 1.0983, 0.5596) -- (2.9935, 1.1485, 0.5620) -- (2.9920, 1.1481, 0.6120) -- (2.9920, 1.0979, 0.6096) -- cycle;
\fill[blue!15.0, opacity=0.5] (2.9920, 1.0979, 0.6096) -- (2.9920, 1.1481, 0.6120) -- (2.9903, 1.1477, 0.6620) -- (2.9903, 1.0974, 0.6596) -- cycle;
\fill[blue!15.0, opacity=0.5] (2.9903, 1.0974, 0.6596) -- (2.9903, 1.1477, 0.6620) -- (2.9885, 1.1473, 0.7120) -- (2.9885, 1.0969, 0.7096) -- cycle;
\fill[blue!15.0, opacity=0.5] (2.9885, 1.0969, 0.7096) -- (2.9885, 1.1473, 0.7120) -- (2.9866, 1.1469, 0.7620) -- (2.9866, 1.0964, 0.7596) -- cycle;
\fill[blue!15.0, opacity=0.5] (2.9866, 1.0964, 0.7596) -- (2.9866, 1.1469, 0.7620) -- (2.9846, 1.1464, 0.8120) -- (2.9846, 1.0959, 0.8096) -- cycle;
\fill[blue!15.0, opacity=0.5] (2.9846, 1.0959, 0.8096) -- (2.9846, 1.1464, 0.8120) -- (2.9824, 1.1459, 0.8620) -- (2.9824, 1.0953, 0.8596) -- cycle;
\fill[blue!15.0, opacity=0.5] (2.9824, 1.0953, 0.8596) -- (2.9824, 1.1459, 0.8620) -- (2.9801, 1.1454, 0.9120) -- (2.9801, 1.0947, 0.9096) -- cycle;
\fill[blue!15.0, opacity=0.5] (2.9801, 1.0947, 0.9096) -- (2.9801, 1.1454, 0.9120) -- (2.9778, 1.1448, 0.9620) -- (2.9778, 1.0941, 0.9596) -- cycle;
\fill[blue!15.0, opacity=0.5] (2.9778, 1.0941, 0.9596) -- (2.9778, 1.1448, 0.9620) -- (2.9753, 1.1442, 1.0120) -- (2.9753, 1.0934, 1.0096) -- cycle;
\fill[blue!15.0, opacity=0.5] (2.9753, 1.0934, 1.0096) -- (2.9753, 1.1442, 1.0120) -- (2.9727, 1.1436, 1.0620) -- (2.9727, 1.0927, 1.0596) -- cycle;
\fill[blue!15.0, opacity=0.5] (2.9727, 1.0927, 1.0596) -- (2.9727, 1.1436, 1.0620) -- (2.9700, 1.1430, 1.1120) -- (2.9700, 1.0920, 1.1096) -- cycle;
\fill[blue!15.0, opacity=0.5] (2.9700, 1.0920, 1.1096) -- (2.9700, 1.1430, 1.1120) -- (2.9672, 1.1424, 1.1620) -- (2.9672, 1.0913, 1.1596) -- cycle;
\fill[blue!15.0, opacity=0.5] (2.9672, 1.0913, 1.1596) -- (2.9672, 1.1424, 1.1620) -- (2.9644, 1.1417, 1.2120) -- (2.9644, 1.0905, 1.2096) -- cycle;
\fill[blue!15.0, opacity=0.5] (2.9644, 1.0905, 1.2096) -- (2.9644, 1.1417, 1.2120) -- (2.9615, 1.1410, 1.2620) -- (2.9615, 1.0897, 1.2596) -- cycle;
\fill[blue!15.0, opacity=0.5] (2.9615, 1.0897, 1.2596) -- (2.9615, 1.1410, 1.2620) -- (2.9585, 1.1403, 1.3120) -- (2.9585, 1.0889, 1.3096) -- cycle;
\fill[blue!15.0, opacity=0.5] (2.9585, 1.0889, 1.3096) -- (2.9585, 1.1403, 1.3120) -- (2.9555, 1.1396, 1.3620) -- (2.9555, 1.0881, 1.3596) -- cycle;
\fill[blue!15.0, opacity=0.5] (2.9555, 1.0881, 1.3596) -- (2.9555, 1.1396, 1.3620) -- (2.9525, 1.1389, 1.4120) -- (2.9525, 1.0873, 1.4096) -- cycle;
\fill[blue!15.0, opacity=0.5] (2.9525, 1.0873, 1.4096) -- (2.9525, 1.1389, 1.4120) -- (2.9494, 1.1382, 1.4620) -- (2.9494, 1.0865, 1.4596) -- cycle;
\fill[blue!15.0, opacity=0.5] (2.9494, 1.0865, 1.4596) -- (2.9494, 1.1382, 1.4620) -- (2.9463, 1.1375, 1.5120) -- (2.9463, 1.0857, 1.5096) -- cycle;
\fill[blue!15.0, opacity=0.5] (2.9463, 1.0857, 1.5096) -- (2.9463, 1.1375, 1.5120) -- (2.9431, 1.1367, 1.5620) -- (2.9431, 1.0848, 1.5596) -- cycle;
\fill[blue!15.0, opacity=0.5] (2.9431, 1.0848, 1.5596) -- (2.9431, 1.1367, 1.5620) -- (2.9400, 1.1360, 1.6120) -- (2.9400, 1.0840, 1.6096) -- cycle;
\fill[blue!15.0, opacity=0.5] (2.9400, 1.0840, 1.6096) -- (2.9400, 1.1360, 1.6120) -- (2.9369, 1.1353, 1.6620) -- (2.9369, 1.0832, 1.6596) -- cycle;
\fill[blue!15.0, opacity=0.5] (2.9369, 1.0832, 1.6596) -- (2.9369, 1.1353, 1.6620) -- (2.9337, 1.1345, 1.7120) -- (2.9337, 1.0823, 1.7096) -- cycle;
\fill[blue!15.0, opacity=0.5] (2.9337, 1.0823, 1.7096) -- (2.9337, 1.1345, 1.7120) -- (2.9306, 1.1338, 1.7620) -- (2.9306, 1.0815, 1.7596) -- cycle;
\fill[blue!15.0, opacity=0.5] (2.9306, 1.0815, 1.7596) -- (2.9306, 1.1338, 1.7620) -- (2.9275, 1.1331, 1.8120) -- (2.9275, 1.0807, 1.8096) -- cycle;
\fill[blue!15.0, opacity=0.5] (2.9275, 1.0807, 1.8096) -- (2.9275, 1.1331, 1.8120) -- (2.9245, 1.1324, 1.8620) -- (2.9245, 1.0799, 1.8596) -- cycle;
\fill[blue!15.0, opacity=0.5] (2.9245, 1.0799, 1.8596) -- (2.9245, 1.1324, 1.8620) -- (2.9215, 1.1317, 1.9120) -- (2.9215, 1.0791, 1.9096) -- cycle;
\fill[blue!15.0, opacity=0.5] (2.9215, 1.0791, 1.9096) -- (2.9215, 1.1317, 1.9120) -- (2.9185, 1.1310, 1.9620) -- (2.9185, 1.0783, 1.9596) -- cycle;
\fill[blue!15.0, opacity=0.5] (2.9185, 1.0783, 1.9596) -- (2.9185, 1.1310, 1.9620) -- (2.9156, 1.1303, 2.0120) -- (2.9156, 1.0775, 2.0096) -- cycle;
\fill[blue!15.1, opacity=0.5] (2.9156, 1.0775, 2.0096) -- (2.9156, 1.1303, 2.0120) -- (2.9128, 1.1296, 2.0620) -- (2.9128, 1.0767, 2.0596) -- cycle;
\fill[blue!15.1, opacity=0.5] (2.9128, 1.0767, 2.0596) -- (2.9128, 1.1296, 2.0620) -- (2.9100, 1.1290, 2.1120) -- (2.9100, 1.0760, 2.1096) -- cycle;
\fill[blue!15.1, opacity=0.5] (2.9100, 1.0760, 2.1096) -- (2.9100, 1.1290, 2.1120) -- (2.9073, 1.1284, 2.1620) -- (2.9073, 1.0753, 2.1596) -- cycle;
\fill[blue!15.2, opacity=0.5] (2.9073, 1.0753, 2.1596) -- (2.9073, 1.1284, 2.1620) -- (2.9047, 1.1278, 2.2120) -- (2.9047, 1.0746, 2.2096) -- cycle;
\fill[blue!15.2, opacity=0.5] (2.9047, 1.0746, 2.2096) -- (2.9047, 1.1278, 2.2120) -- (2.9022, 1.1272, 2.2620) -- (2.9022, 1.0739, 2.2596) -- cycle;
\fill[blue!15.3, opacity=0.5] (2.9022, 1.0739, 2.2596) -- (2.9022, 1.1272, 2.2620) -- (2.8999, 1.1266, 2.3120) -- (2.8999, 1.0733, 2.3096) -- cycle;
\fill[blue!15.4, opacity=0.5] (2.8999, 1.0733, 2.3096) -- (2.8999, 1.1266, 2.3120) -- (2.8976, 1.1261, 2.3620) -- (2.8976, 1.0727, 2.3596) -- cycle;
\fill[blue!15.5, opacity=0.5] (2.8976, 1.0727, 2.3596) -- (2.8976, 1.1261, 2.3620) -- (2.8954, 1.1256, 2.4120) -- (2.8954, 1.0721, 2.4096) -- cycle;
\fill[blue!15.6, opacity=0.5] (2.8954, 1.0721, 2.4096) -- (2.8954, 1.1256, 2.4120) -- (2.8934, 1.1251, 2.4620) -- (2.8934, 1.0716, 2.4596) -- cycle;
\fill[blue!15.8, opacity=0.5] (2.8934, 1.0716, 2.4596) -- (2.8934, 1.1251, 2.4620) -- (2.8915, 1.1247, 2.5120) -- (2.8915, 1.0711, 2.5096) -- cycle;
\fill[blue!16.0, opacity=0.5] (2.8915, 1.0711, 2.5096) -- (2.8915, 1.1247, 2.5120) -- (2.8897, 1.1243, 2.5620) -- (2.8897, 1.0706, 2.5596) -- cycle;
\fill[blue!16.2, opacity=0.5] (2.8897, 1.0706, 2.5596) -- (2.8897, 1.1243, 2.5620) -- (2.8880, 1.1239, 2.6120) -- (2.8880, 1.0701, 2.6096) -- cycle;
\fill[blue!16.5, opacity=0.5] (2.8880, 1.0701, 2.6096) -- (2.8880, 1.1239, 2.6120) -- (2.8865, 1.1235, 2.6620) -- (2.8865, 1.0697, 2.6596) -- cycle;
\fill[blue!16.8, opacity=0.5] (2.8865, 1.0697, 2.6596) -- (2.8865, 1.1235, 2.6620) -- (2.8852, 1.1232, 2.7120) -- (2.8852, 1.0694, 2.7096) -- cycle;
\fill[blue!17.1, opacity=0.5] (2.8852, 1.0694, 2.7096) -- (2.8852, 1.1232, 2.7120) -- (2.8840, 1.1229, 2.7620) -- (2.8840, 1.0691, 2.7596) -- cycle;
\fill[blue!17.5, opacity=0.5] (2.8840, 1.0691, 2.7596) -- (2.8840, 1.1229, 2.7620) -- (2.8829, 1.1227, 2.8120) -- (2.8829, 1.0688, 2.8096) -- cycle;
\fill[blue!18.0, opacity=0.5] (2.8829, 1.0688, 2.8096) -- (2.8829, 1.1227, 2.8120) -- (2.8820, 1.1225, 2.8620) -- (2.8820, 1.0685, 2.8596) -- cycle;
\fill[blue!18.4, opacity=0.5] (2.8820, 1.0685, 2.8596) -- (2.8820, 1.1225, 2.8620) -- (2.8813, 1.1223, 2.9120) -- (2.8813, 1.0683, 2.9096) -- cycle;
\fill[blue!18.9, opacity=0.5] (2.8813, 1.0683, 2.9096) -- (2.8813, 1.1223, 2.9120) -- (2.8807, 1.1222, 2.9620) -- (2.8807, 1.0682, 2.9596) -- cycle;
\fill[blue!19.4, opacity=0.5] (2.8807, 1.0682, 2.9596) -- (2.8807, 1.1222, 2.9620) -- (2.8803, 1.1221, 3.0120) -- (2.8803, 1.0681, 3.0096) -- cycle;
\fill[blue!20.0, opacity=0.5] (2.8803, 1.0681, 3.0096) -- (2.8803, 1.1221, 3.0120) -- (2.8801, 1.1220, 3.0620) -- (2.8801, 1.0680, 3.0596) -- cycle;
\fill[blue!20.6, opacity=0.5] (2.8801, 1.0680, 3.0596) -- (2.8801, 1.1220, 3.0620) -- (2.8800, 1.1220, 3.1120) -- (2.8800, 1.0680, 3.1096) -- cycle;
\fill[blue!15.0, opacity=0.5] (3.0000, 1.1500, 0.1120) -- (3.0000, 1.2000, 0.1141) -- (2.9999, 1.2000, 0.1641) -- (2.9999, 1.1500, 0.1620) -- cycle;
\fill[blue!15.0, opacity=0.5] (2.9999, 1.1500, 0.1620) -- (2.9999, 1.2000, 0.1641) -- (2.9997, 1.1999, 0.2141) -- (2.9997, 1.1499, 0.2120) -- cycle;
\fill[blue!15.0, opacity=0.5] (2.9997, 1.1499, 0.2120) -- (2.9997, 1.1999, 0.2141) -- (2.9993, 1.1999, 0.2641) -- (2.9993, 1.1498, 0.2620) -- cycle;
\fill[blue!15.0, opacity=0.5] (2.9993, 1.1498, 0.2620) -- (2.9993, 1.1999, 0.2641) -- (2.9987, 1.1997, 0.3141) -- (2.9987, 1.1497, 0.3120) -- cycle;
\fill[blue!15.0, opacity=0.5] (2.9987, 1.1497, 0.3120) -- (2.9987, 1.1997, 0.3141) -- (2.9980, 1.1996, 0.3641) -- (2.9980, 1.1495, 0.3620) -- cycle;
\fill[blue!15.0, opacity=0.5] (2.9980, 1.1495, 0.3620) -- (2.9980, 1.1996, 0.3641) -- (2.9971, 1.1994, 0.4141) -- (2.9971, 1.1493, 0.4120) -- cycle;
\fill[blue!15.0, opacity=0.5] (2.9971, 1.1493, 0.4120) -- (2.9971, 1.1994, 0.4141) -- (2.9960, 1.1992, 0.4641) -- (2.9960, 1.1491, 0.4620) -- cycle;
\fill[blue!15.0, opacity=0.5] (2.9960, 1.1491, 0.4620) -- (2.9960, 1.1992, 0.4641) -- (2.9948, 1.1990, 0.5141) -- (2.9948, 1.1488, 0.5120) -- cycle;
\fill[blue!15.0, opacity=0.5] (2.9948, 1.1488, 0.5120) -- (2.9948, 1.1990, 0.5141) -- (2.9935, 1.1987, 0.5641) -- (2.9935, 1.1485, 0.5620) -- cycle;
\fill[blue!15.0, opacity=0.5] (2.9935, 1.1485, 0.5620) -- (2.9935, 1.1987, 0.5641) -- (2.9920, 1.1984, 0.6141) -- (2.9920, 1.1481, 0.6120) -- cycle;
\fill[blue!15.0, opacity=0.5] (2.9920, 1.1481, 0.6120) -- (2.9920, 1.1984, 0.6141) -- (2.9903, 1.1981, 0.6641) -- (2.9903, 1.1477, 0.6620) -- cycle;
\fill[blue!15.0, opacity=0.5] (2.9903, 1.1477, 0.6620) -- (2.9903, 1.1981, 0.6641) -- (2.9885, 1.1977, 0.7141) -- (2.9885, 1.1473, 0.7120) -- cycle;
\fill[blue!15.0, opacity=0.5] (2.9885, 1.1473, 0.7120) -- (2.9885, 1.1977, 0.7141) -- (2.9866, 1.1973, 0.7641) -- (2.9866, 1.1469, 0.7620) -- cycle;
\fill[blue!15.0, opacity=0.5] (2.9866, 1.1469, 0.7620) -- (2.9866, 1.1973, 0.7641) -- (2.9846, 1.1969, 0.8141) -- (2.9846, 1.1464, 0.8120) -- cycle;
\fill[blue!15.0, opacity=0.5] (2.9846, 1.1464, 0.8120) -- (2.9846, 1.1969, 0.8141) -- (2.9824, 1.1965, 0.8641) -- (2.9824, 1.1459, 0.8620) -- cycle;
\fill[blue!15.0, opacity=0.5] (2.9824, 1.1459, 0.8620) -- (2.9824, 1.1965, 0.8641) -- (2.9801, 1.1960, 0.9141) -- (2.9801, 1.1454, 0.9120) -- cycle;
\fill[blue!15.0, opacity=0.5] (2.9801, 1.1454, 0.9120) -- (2.9801, 1.1960, 0.9141) -- (2.9778, 1.1956, 0.9641) -- (2.9778, 1.1448, 0.9620) -- cycle;
\fill[blue!15.0, opacity=0.5] (2.9778, 1.1448, 0.9620) -- (2.9778, 1.1956, 0.9641) -- (2.9753, 1.1951, 1.0141) -- (2.9753, 1.1442, 1.0120) -- cycle;
\fill[blue!15.0, opacity=0.5] (2.9753, 1.1442, 1.0120) -- (2.9753, 1.1951, 1.0141) -- (2.9727, 1.1945, 1.0641) -- (2.9727, 1.1436, 1.0620) -- cycle;
\fill[blue!15.0, opacity=0.5] (2.9727, 1.1436, 1.0620) -- (2.9727, 1.1945, 1.0641) -- (2.9700, 1.1940, 1.1141) -- (2.9700, 1.1430, 1.1120) -- cycle;
\fill[blue!15.0, opacity=0.5] (2.9700, 1.1430, 1.1120) -- (2.9700, 1.1940, 1.1141) -- (2.9672, 1.1934, 1.1641) -- (2.9672, 1.1424, 1.1620) -- cycle;
\fill[blue!15.0, opacity=0.5] (2.9672, 1.1424, 1.1620) -- (2.9672, 1.1934, 1.1641) -- (2.9644, 1.1929, 1.2141) -- (2.9644, 1.1417, 1.2120) -- cycle;
\fill[blue!15.0, opacity=0.5] (2.9644, 1.1417, 1.2120) -- (2.9644, 1.1929, 1.2141) -- (2.9615, 1.1923, 1.2641) -- (2.9615, 1.1410, 1.2620) -- cycle;
\fill[blue!15.0, opacity=0.5] (2.9615, 1.1410, 1.2620) -- (2.9615, 1.1923, 1.2641) -- (2.9585, 1.1917, 1.3141) -- (2.9585, 1.1403, 1.3120) -- cycle;
\fill[blue!15.0, opacity=0.5] (2.9585, 1.1403, 1.3120) -- (2.9585, 1.1917, 1.3141) -- (2.9555, 1.1911, 1.3641) -- (2.9555, 1.1396, 1.3620) -- cycle;
\fill[blue!15.0, opacity=0.5] (2.9555, 1.1396, 1.3620) -- (2.9555, 1.1911, 1.3641) -- (2.9525, 1.1905, 1.4141) -- (2.9525, 1.1389, 1.4120) -- cycle;
\fill[blue!15.0, opacity=0.5] (2.9525, 1.1389, 1.4120) -- (2.9525, 1.1905, 1.4141) -- (2.9494, 1.1899, 1.4641) -- (2.9494, 1.1382, 1.4620) -- cycle;
\fill[blue!15.0, opacity=0.5] (2.9494, 1.1382, 1.4620) -- (2.9494, 1.1899, 1.4641) -- (2.9463, 1.1893, 1.5141) -- (2.9463, 1.1375, 1.5120) -- cycle;
\fill[blue!15.0, opacity=0.5] (2.9463, 1.1375, 1.5120) -- (2.9463, 1.1893, 1.5141) -- (2.9431, 1.1886, 1.5641) -- (2.9431, 1.1367, 1.5620) -- cycle;
\fill[blue!15.0, opacity=0.5] (2.9431, 1.1367, 1.5620) -- (2.9431, 1.1886, 1.5641) -- (2.9400, 1.1880, 1.6141) -- (2.9400, 1.1360, 1.6120) -- cycle;
\fill[blue!15.0, opacity=0.5] (2.9400, 1.1360, 1.6120) -- (2.9400, 1.1880, 1.6141) -- (2.9369, 1.1874, 1.6641) -- (2.9369, 1.1353, 1.6620) -- cycle;
\fill[blue!15.0, opacity=0.5] (2.9369, 1.1353, 1.6620) -- (2.9369, 1.1874, 1.6641) -- (2.9337, 1.1867, 1.7141) -- (2.9337, 1.1345, 1.7120) -- cycle;
\fill[blue!15.0, opacity=0.5] (2.9337, 1.1345, 1.7120) -- (2.9337, 1.1867, 1.7141) -- (2.9306, 1.1861, 1.7641) -- (2.9306, 1.1338, 1.7620) -- cycle;
\fill[blue!15.0, opacity=0.5] (2.9306, 1.1338, 1.7620) -- (2.9306, 1.1861, 1.7641) -- (2.9275, 1.1855, 1.8141) -- (2.9275, 1.1331, 1.8120) -- cycle;
\fill[blue!15.0, opacity=0.5] (2.9275, 1.1331, 1.8120) -- (2.9275, 1.1855, 1.8141) -- (2.9245, 1.1849, 1.8641) -- (2.9245, 1.1324, 1.8620) -- cycle;
\fill[blue!15.0, opacity=0.5] (2.9245, 1.1324, 1.8620) -- (2.9245, 1.1849, 1.8641) -- (2.9215, 1.1843, 1.9141) -- (2.9215, 1.1317, 1.9120) -- cycle;
\fill[blue!15.0, opacity=0.5] (2.9215, 1.1317, 1.9120) -- (2.9215, 1.1843, 1.9141) -- (2.9185, 1.1837, 1.9641) -- (2.9185, 1.1310, 1.9620) -- cycle;
\fill[blue!15.1, opacity=0.5] (2.9185, 1.1310, 1.9620) -- (2.9185, 1.1837, 1.9641) -- (2.9156, 1.1831, 2.0141) -- (2.9156, 1.1303, 2.0120) -- cycle;
\fill[blue!15.1, opacity=0.5] (2.9156, 1.1303, 2.0120) -- (2.9156, 1.1831, 2.0141) -- (2.9128, 1.1826, 2.0641) -- (2.9128, 1.1296, 2.0620) -- cycle;
\fill[blue!15.1, opacity=0.5] (2.9128, 1.1296, 2.0620) -- (2.9128, 1.1826, 2.0641) -- (2.9100, 1.1820, 2.1141) -- (2.9100, 1.1290, 2.1120) -- cycle;
\fill[blue!15.2, opacity=0.5] (2.9100, 1.1290, 2.1120) -- (2.9100, 1.1820, 2.1141) -- (2.9073, 1.1815, 2.1641) -- (2.9073, 1.1284, 2.1620) -- cycle;
\fill[blue!15.3, opacity=0.5] (2.9073, 1.1284, 2.1620) -- (2.9073, 1.1815, 2.1641) -- (2.9047, 1.1809, 2.2141) -- (2.9047, 1.1278, 2.2120) -- cycle;
\fill[blue!15.4, opacity=0.5] (2.9047, 1.1278, 2.2120) -- (2.9047, 1.1809, 2.2141) -- (2.9022, 1.1804, 2.2641) -- (2.9022, 1.1272, 2.2620) -- cycle;
\fill[blue!15.5, opacity=0.5] (2.9022, 1.1272, 2.2620) -- (2.9022, 1.1804, 2.2641) -- (2.8999, 1.1800, 2.3141) -- (2.8999, 1.1266, 2.3120) -- cycle;
\fill[blue!15.6, opacity=0.5] (2.8999, 1.1266, 2.3120) -- (2.8999, 1.1800, 2.3141) -- (2.8976, 1.1795, 2.3641) -- (2.8976, 1.1261, 2.3620) -- cycle;
\fill[blue!15.8, opacity=0.5] (2.8976, 1.1261, 2.3620) -- (2.8976, 1.1795, 2.3641) -- (2.8954, 1.1791, 2.4141) -- (2.8954, 1.1256, 2.4120) -- cycle;
\fill[blue!16.0, opacity=0.5] (2.8954, 1.1256, 2.4120) -- (2.8954, 1.1791, 2.4141) -- (2.8934, 1.1787, 2.4641) -- (2.8934, 1.1251, 2.4620) -- cycle;
\fill[blue!16.2, opacity=0.5] (2.8934, 1.1251, 2.4620) -- (2.8934, 1.1787, 2.4641) -- (2.8915, 1.1783, 2.5141) -- (2.8915, 1.1247, 2.5120) -- cycle;
\fill[blue!16.5, opacity=0.5] (2.8915, 1.1247, 2.5120) -- (2.8915, 1.1783, 2.5141) -- (2.8897, 1.1779, 2.5641) -- (2.8897, 1.1243, 2.5620) -- cycle;
\fill[blue!16.8, opacity=0.5] (2.8897, 1.1243, 2.5620) -- (2.8897, 1.1779, 2.5641) -- (2.8880, 1.1776, 2.6141) -- (2.8880, 1.1239, 2.6120) -- cycle;
\fill[blue!17.2, opacity=0.5] (2.8880, 1.1239, 2.6120) -- (2.8880, 1.1776, 2.6141) -- (2.8865, 1.1773, 2.6641) -- (2.8865, 1.1235, 2.6620) -- cycle;
\fill[blue!17.6, opacity=0.5] (2.8865, 1.1235, 2.6620) -- (2.8865, 1.1773, 2.6641) -- (2.8852, 1.1770, 2.7141) -- (2.8852, 1.1232, 2.7120) -- cycle;
\fill[blue!18.0, opacity=0.5] (2.8852, 1.1232, 2.7120) -- (2.8852, 1.1770, 2.7141) -- (2.8840, 1.1768, 2.7641) -- (2.8840, 1.1229, 2.7620) -- cycle;
\fill[blue!18.5, opacity=0.5] (2.8840, 1.1229, 2.7620) -- (2.8840, 1.1768, 2.7641) -- (2.8829, 1.1766, 2.8141) -- (2.8829, 1.1227, 2.8120) -- cycle;
\fill[blue!19.1, opacity=0.5] (2.8829, 1.1227, 2.8120) -- (2.8829, 1.1766, 2.8141) -- (2.8820, 1.1764, 2.8641) -- (2.8820, 1.1225, 2.8620) -- cycle;
\fill[blue!19.6, opacity=0.5] (2.8820, 1.1225, 2.8620) -- (2.8820, 1.1764, 2.8641) -- (2.8813, 1.1763, 2.9141) -- (2.8813, 1.1223, 2.9120) -- cycle;
\fill[blue!20.2, opacity=0.5] (2.8813, 1.1223, 2.9120) -- (2.8813, 1.1763, 2.9141) -- (2.8807, 1.1761, 2.9641) -- (2.8807, 1.1222, 2.9620) -- cycle;
\fill[blue!20.9, opacity=0.5] (2.8807, 1.1222, 2.9620) -- (2.8807, 1.1761, 2.9641) -- (2.8803, 1.1761, 3.0141) -- (2.8803, 1.1221, 3.0120) -- cycle;
\fill[blue!21.5, opacity=0.5] (2.8803, 1.1221, 3.0120) -- (2.8803, 1.1761, 3.0141) -- (2.8801, 1.1760, 3.0641) -- (2.8801, 1.1220, 3.0620) -- cycle;
\fill[blue!22.2, opacity=0.5] (2.8801, 1.1220, 3.0620) -- (2.8801, 1.1760, 3.0641) -- (2.8800, 1.1760, 3.1141) -- (2.8800, 1.1220, 3.1120) -- cycle;
\fill[blue!15.0, opacity=0.5] (3.0000, 1.2000, 0.1141) -- (3.0000, 1.2500, 0.1159) -- (2.9999, 1.2500, 0.1659) -- (2.9999, 1.2000, 0.1641) -- cycle;
\fill[blue!15.0, opacity=0.5] (2.9999, 1.2000, 0.1641) -- (2.9999, 1.2500, 0.1659) -- (2.9997, 1.2499, 0.2159) -- (2.9997, 1.1999, 0.2141) -- cycle;
\fill[blue!15.0, opacity=0.5] (2.9997, 1.1999, 0.2141) -- (2.9997, 1.2499, 0.2159) -- (2.9993, 1.2499, 0.2659) -- (2.9993, 1.1999, 0.2641) -- cycle;
\fill[blue!15.0, opacity=0.5] (2.9993, 1.1999, 0.2641) -- (2.9993, 1.2499, 0.2659) -- (2.9987, 1.2498, 0.3159) -- (2.9987, 1.1997, 0.3141) -- cycle;
\fill[blue!15.0, opacity=0.5] (2.9987, 1.1997, 0.3141) -- (2.9987, 1.2498, 0.3159) -- (2.9980, 1.2497, 0.3659) -- (2.9980, 1.1996, 0.3641) -- cycle;
\fill[blue!15.0, opacity=0.5] (2.9980, 1.1996, 0.3641) -- (2.9980, 1.2497, 0.3659) -- (2.9971, 1.2495, 0.4159) -- (2.9971, 1.1994, 0.4141) -- cycle;
\fill[blue!15.0, opacity=0.5] (2.9971, 1.1994, 0.4141) -- (2.9971, 1.2495, 0.4159) -- (2.9960, 1.2493, 0.4659) -- (2.9960, 1.1992, 0.4641) -- cycle;
\fill[blue!15.0, opacity=0.5] (2.9960, 1.1992, 0.4641) -- (2.9960, 1.2493, 0.4659) -- (2.9948, 1.2491, 0.5159) -- (2.9948, 1.1990, 0.5141) -- cycle;
\fill[blue!15.0, opacity=0.5] (2.9948, 1.1990, 0.5141) -- (2.9948, 1.2491, 0.5159) -- (2.9935, 1.2489, 0.5659) -- (2.9935, 1.1987, 0.5641) -- cycle;
\fill[blue!15.0, opacity=0.5] (2.9935, 1.1987, 0.5641) -- (2.9935, 1.2489, 0.5659) -- (2.9920, 1.2487, 0.6159) -- (2.9920, 1.1984, 0.6141) -- cycle;
\fill[blue!15.0, opacity=0.5] (2.9920, 1.1984, 0.6141) -- (2.9920, 1.2487, 0.6159) -- (2.9903, 1.2484, 0.6659) -- (2.9903, 1.1981, 0.6641) -- cycle;
\fill[blue!15.0, opacity=0.5] (2.9903, 1.1981, 0.6641) -- (2.9903, 1.2484, 0.6659) -- (2.9885, 1.2481, 0.7159) -- (2.9885, 1.1977, 0.7141) -- cycle;
\fill[blue!15.0, opacity=0.5] (2.9885, 1.1977, 0.7141) -- (2.9885, 1.2481, 0.7159) -- (2.9866, 1.2478, 0.7659) -- (2.9866, 1.1973, 0.7641) -- cycle;
\fill[blue!15.0, opacity=0.5] (2.9866, 1.1973, 0.7641) -- (2.9866, 1.2478, 0.7659) -- (2.9846, 1.2474, 0.8159) -- (2.9846, 1.1969, 0.8141) -- cycle;
\fill[blue!15.0, opacity=0.5] (2.9846, 1.1969, 0.8141) -- (2.9846, 1.2474, 0.8159) -- (2.9824, 1.2471, 0.8659) -- (2.9824, 1.1965, 0.8641) -- cycle;
\fill[blue!15.0, opacity=0.5] (2.9824, 1.1965, 0.8641) -- (2.9824, 1.2471, 0.8659) -- (2.9801, 1.2467, 0.9159) -- (2.9801, 1.1960, 0.9141) -- cycle;
\fill[blue!15.0, opacity=0.5] (2.9801, 1.1960, 0.9141) -- (2.9801, 1.2467, 0.9159) -- (2.9778, 1.2463, 0.9659) -- (2.9778, 1.1956, 0.9641) -- cycle;
\fill[blue!15.0, opacity=0.5] (2.9778, 1.1956, 0.9641) -- (2.9778, 1.2463, 0.9659) -- (2.9753, 1.2459, 1.0159) -- (2.9753, 1.1951, 1.0141) -- cycle;
\fill[blue!15.0, opacity=0.5] (2.9753, 1.1951, 1.0141) -- (2.9753, 1.2459, 1.0159) -- (2.9727, 1.2454, 1.0659) -- (2.9727, 1.1945, 1.0641) -- cycle;
\fill[blue!15.0, opacity=0.5] (2.9727, 1.1945, 1.0641) -- (2.9727, 1.2454, 1.0659) -- (2.9700, 1.2450, 1.1159) -- (2.9700, 1.1940, 1.1141) -- cycle;
\fill[blue!15.0, opacity=0.5] (2.9700, 1.1940, 1.1141) -- (2.9700, 1.2450, 1.1159) -- (2.9672, 1.2445, 1.1659) -- (2.9672, 1.1934, 1.1641) -- cycle;
\fill[blue!15.0, opacity=0.5] (2.9672, 1.1934, 1.1641) -- (2.9672, 1.2445, 1.1659) -- (2.9644, 1.2441, 1.2159) -- (2.9644, 1.1929, 1.2141) -- cycle;
\fill[blue!15.0, opacity=0.5] (2.9644, 1.1929, 1.2141) -- (2.9644, 1.2441, 1.2159) -- (2.9615, 1.2436, 1.2659) -- (2.9615, 1.1923, 1.2641) -- cycle;
\fill[blue!15.0, opacity=0.5] (2.9615, 1.1923, 1.2641) -- (2.9615, 1.2436, 1.2659) -- (2.9585, 1.2431, 1.3159) -- (2.9585, 1.1917, 1.3141) -- cycle;
\fill[blue!15.0, opacity=0.5] (2.9585, 1.1917, 1.3141) -- (2.9585, 1.2431, 1.3159) -- (2.9555, 1.2426, 1.3659) -- (2.9555, 1.1911, 1.3641) -- cycle;
\fill[blue!15.0, opacity=0.5] (2.9555, 1.1911, 1.3641) -- (2.9555, 1.2426, 1.3659) -- (2.9525, 1.2421, 1.4159) -- (2.9525, 1.1905, 1.4141) -- cycle;
\fill[blue!15.0, opacity=0.5] (2.9525, 1.1905, 1.4141) -- (2.9525, 1.2421, 1.4159) -- (2.9494, 1.2416, 1.4659) -- (2.9494, 1.1899, 1.4641) -- cycle;
\fill[blue!15.0, opacity=0.5] (2.9494, 1.1899, 1.4641) -- (2.9494, 1.2416, 1.4659) -- (2.9463, 1.2410, 1.5159) -- (2.9463, 1.1893, 1.5141) -- cycle;
\fill[blue!15.0, opacity=0.5] (2.9463, 1.1893, 1.5141) -- (2.9463, 1.2410, 1.5159) -- (2.9431, 1.2405, 1.5659) -- (2.9431, 1.1886, 1.5641) -- cycle;
\fill[blue!15.0, opacity=0.5] (2.9431, 1.1886, 1.5641) -- (2.9431, 1.2405, 1.5659) -- (2.9400, 1.2400, 1.6159) -- (2.9400, 1.1880, 1.6141) -- cycle;
\fill[blue!15.0, opacity=0.5] (2.9400, 1.1880, 1.6141) -- (2.9400, 1.2400, 1.6159) -- (2.9369, 1.2395, 1.6659) -- (2.9369, 1.1874, 1.6641) -- cycle;
\fill[blue!15.0, opacity=0.5] (2.9369, 1.1874, 1.6641) -- (2.9369, 1.2395, 1.6659) -- (2.9337, 1.2390, 1.7159) -- (2.9337, 1.1867, 1.7141) -- cycle;
\fill[blue!15.0, opacity=0.5] (2.9337, 1.1867, 1.7141) -- (2.9337, 1.2390, 1.7159) -- (2.9306, 1.2384, 1.7659) -- (2.9306, 1.1861, 1.7641) -- cycle;
\fill[blue!15.0, opacity=0.5] (2.9306, 1.1861, 1.7641) -- (2.9306, 1.2384, 1.7659) -- (2.9275, 1.2379, 1.8159) -- (2.9275, 1.1855, 1.8141) -- cycle;
\fill[blue!15.0, opacity=0.5] (2.9275, 1.1855, 1.8141) -- (2.9275, 1.2379, 1.8159) -- (2.9245, 1.2374, 1.8659) -- (2.9245, 1.1849, 1.8641) -- cycle;
\fill[blue!15.1, opacity=0.5] (2.9245, 1.1849, 1.8641) -- (2.9245, 1.2374, 1.8659) -- (2.9215, 1.2369, 1.9159) -- (2.9215, 1.1843, 1.9141) -- cycle;
\fill[blue!15.1, opacity=0.5] (2.9215, 1.1843, 1.9141) -- (2.9215, 1.2369, 1.9159) -- (2.9185, 1.2364, 1.9659) -- (2.9185, 1.1837, 1.9641) -- cycle;
\fill[blue!15.1, opacity=0.5] (2.9185, 1.1837, 1.9641) -- (2.9185, 1.2364, 1.9659) -- (2.9156, 1.2359, 2.0159) -- (2.9156, 1.1831, 2.0141) -- cycle;
\fill[blue!15.2, opacity=0.5] (2.9156, 1.1831, 2.0141) -- (2.9156, 1.2359, 2.0159) -- (2.9128, 1.2355, 2.0659) -- (2.9128, 1.1826, 2.0641) -- cycle;
\fill[blue!15.2, opacity=0.5] (2.9128, 1.1826, 2.0641) -- (2.9128, 1.2355, 2.0659) -- (2.9100, 1.2350, 2.1159) -- (2.9100, 1.1820, 2.1141) -- cycle;
\fill[blue!15.3, opacity=0.5] (2.9100, 1.1820, 2.1141) -- (2.9100, 1.2350, 2.1159) -- (2.9073, 1.2346, 2.1659) -- (2.9073, 1.1815, 2.1641) -- cycle;
\fill[blue!15.4, opacity=0.5] (2.9073, 1.1815, 2.1641) -- (2.9073, 1.2346, 2.1659) -- (2.9047, 1.2341, 2.2159) -- (2.9047, 1.1809, 2.2141) -- cycle;
\fill[blue!15.5, opacity=0.5] (2.9047, 1.1809, 2.2141) -- (2.9047, 1.2341, 2.2159) -- (2.9022, 1.2337, 2.2659) -- (2.9022, 1.1804, 2.2641) -- cycle;
\fill[blue!15.7, opacity=0.5] (2.9022, 1.1804, 2.2641) -- (2.9022, 1.2337, 2.2659) -- (2.8999, 1.2333, 2.3159) -- (2.8999, 1.1800, 2.3141) -- cycle;
\fill[blue!15.9, opacity=0.5] (2.8999, 1.1800, 2.3141) -- (2.8999, 1.2333, 2.3159) -- (2.8976, 1.2329, 2.3659) -- (2.8976, 1.1795, 2.3641) -- cycle;
\fill[blue!16.1, opacity=0.5] (2.8976, 1.1795, 2.3641) -- (2.8976, 1.2329, 2.3659) -- (2.8954, 1.2326, 2.4159) -- (2.8954, 1.1791, 2.4141) -- cycle;
\fill[blue!16.4, opacity=0.5] (2.8954, 1.1791, 2.4141) -- (2.8954, 1.2326, 2.4159) -- (2.8934, 1.2322, 2.4659) -- (2.8934, 1.1787, 2.4641) -- cycle;
\fill[blue!16.7, opacity=0.5] (2.8934, 1.1787, 2.4641) -- (2.8934, 1.2322, 2.4659) -- (2.8915, 1.2319, 2.5159) -- (2.8915, 1.1783, 2.5141) -- cycle;
\fill[blue!17.1, opacity=0.5] (2.8915, 1.1783, 2.5141) -- (2.8915, 1.2319, 2.5159) -- (2.8897, 1.2316, 2.5659) -- (2.8897, 1.1779, 2.5641) -- cycle;
\fill[blue!17.5, opacity=0.5] (2.8897, 1.1779, 2.5641) -- (2.8897, 1.2316, 2.5659) -- (2.8880, 1.2313, 2.6159) -- (2.8880, 1.1776, 2.6141) -- cycle;
\fill[blue!17.9, opacity=0.5] (2.8880, 1.1776, 2.6141) -- (2.8880, 1.2313, 2.6159) -- (2.8865, 1.2311, 2.6659) -- (2.8865, 1.1773, 2.6641) -- cycle;
\fill[blue!18.4, opacity=0.5] (2.8865, 1.1773, 2.6641) -- (2.8865, 1.2311, 2.6659) -- (2.8852, 1.2309, 2.7159) -- (2.8852, 1.1770, 2.7141) -- cycle;
\fill[blue!19.0, opacity=0.5] (2.8852, 1.1770, 2.7141) -- (2.8852, 1.2309, 2.7159) -- (2.8840, 1.2307, 2.7659) -- (2.8840, 1.1768, 2.7641) -- cycle;
\fill[blue!19.6, opacity=0.5] (2.8840, 1.1768, 2.7641) -- (2.8840, 1.2307, 2.7659) -- (2.8829, 1.2305, 2.8159) -- (2.8829, 1.1766, 2.8141) -- cycle;
\fill[blue!20.2, opacity=0.5] (2.8829, 1.1766, 2.8141) -- (2.8829, 1.2305, 2.8159) -- (2.8820, 1.2303, 2.8659) -- (2.8820, 1.1764, 2.8641) -- cycle;
\fill[blue!20.9, opacity=0.5] (2.8820, 1.1764, 2.8641) -- (2.8820, 1.2303, 2.8659) -- (2.8813, 1.2302, 2.9159) -- (2.8813, 1.1763, 2.9141) -- cycle;
\fill[blue!21.6, opacity=0.5] (2.8813, 1.1763, 2.9141) -- (2.8813, 1.2302, 2.9159) -- (2.8807, 1.2301, 2.9659) -- (2.8807, 1.1761, 2.9641) -- cycle;
\fill[blue!22.3, opacity=0.5] (2.8807, 1.1761, 2.9641) -- (2.8807, 1.2301, 2.9659) -- (2.8803, 1.2301, 3.0159) -- (2.8803, 1.1761, 3.0141) -- cycle;
\fill[blue!23.1, opacity=0.5] (2.8803, 1.1761, 3.0141) -- (2.8803, 1.2301, 3.0159) -- (2.8801, 1.2300, 3.0659) -- (2.8801, 1.1760, 3.0641) -- cycle;
\fill[blue!23.8, opacity=0.5] (2.8801, 1.1760, 3.0641) -- (2.8801, 1.2300, 3.0659) -- (2.8800, 1.2300, 3.1159) -- (2.8800, 1.1760, 3.1141) -- cycle;
\fill[blue!15.0, opacity=0.5] (3.0000, 1.2500, 0.1159) -- (3.0000, 1.3000, 0.1174) -- (2.9999, 1.3000, 0.1674) -- (2.9999, 1.2500, 0.1659) -- cycle;
\fill[blue!15.0, opacity=0.5] (2.9999, 1.2500, 0.1659) -- (2.9999, 1.3000, 0.1674) -- (2.9997, 1.3000, 0.2174) -- (2.9997, 1.2499, 0.2159) -- cycle;
\fill[blue!15.0, opacity=0.5] (2.9997, 1.2499, 0.2159) -- (2.9997, 1.3000, 0.2174) -- (2.9993, 1.2999, 0.2674) -- (2.9993, 1.2499, 0.2659) -- cycle;
\fill[blue!15.0, opacity=0.5] (2.9993, 1.2499, 0.2659) -- (2.9993, 1.2999, 0.2674) -- (2.9987, 1.2998, 0.3174) -- (2.9987, 1.2498, 0.3159) -- cycle;
\fill[blue!15.0, opacity=0.5] (2.9987, 1.2498, 0.3159) -- (2.9987, 1.2998, 0.3174) -- (2.9980, 1.2997, 0.3674) -- (2.9980, 1.2497, 0.3659) -- cycle;
\fill[blue!15.0, opacity=0.5] (2.9980, 1.2497, 0.3659) -- (2.9980, 1.2997, 0.3674) -- (2.9971, 1.2996, 0.4174) -- (2.9971, 1.2495, 0.4159) -- cycle;
\fill[blue!15.0, opacity=0.5] (2.9971, 1.2495, 0.4159) -- (2.9971, 1.2996, 0.4174) -- (2.9960, 1.2995, 0.4674) -- (2.9960, 1.2493, 0.4659) -- cycle;
\fill[blue!15.0, opacity=0.5] (2.9960, 1.2493, 0.4659) -- (2.9960, 1.2995, 0.4674) -- (2.9948, 1.2993, 0.5174) -- (2.9948, 1.2491, 0.5159) -- cycle;
\fill[blue!15.0, opacity=0.5] (2.9948, 1.2491, 0.5159) -- (2.9948, 1.2993, 0.5174) -- (2.9935, 1.2991, 0.5674) -- (2.9935, 1.2489, 0.5659) -- cycle;
\fill[blue!15.0, opacity=0.5] (2.9935, 1.2489, 0.5659) -- (2.9935, 1.2991, 0.5674) -- (2.9920, 1.2989, 0.6174) -- (2.9920, 1.2487, 0.6159) -- cycle;
\fill[blue!15.0, opacity=0.5] (2.9920, 1.2487, 0.6159) -- (2.9920, 1.2989, 0.6174) -- (2.9903, 1.2987, 0.6674) -- (2.9903, 1.2484, 0.6659) -- cycle;
\fill[blue!15.0, opacity=0.5] (2.9903, 1.2484, 0.6659) -- (2.9903, 1.2987, 0.6674) -- (2.9885, 1.2985, 0.7174) -- (2.9885, 1.2481, 0.7159) -- cycle;
\fill[blue!15.0, opacity=0.5] (2.9885, 1.2481, 0.7159) -- (2.9885, 1.2985, 0.7174) -- (2.9866, 1.2982, 0.7674) -- (2.9866, 1.2478, 0.7659) -- cycle;
\fill[blue!15.0, opacity=0.5] (2.9866, 1.2478, 0.7659) -- (2.9866, 1.2982, 0.7674) -- (2.9846, 1.2979, 0.8174) -- (2.9846, 1.2474, 0.8159) -- cycle;
\fill[blue!15.0, opacity=0.5] (2.9846, 1.2474, 0.8159) -- (2.9846, 1.2979, 0.8174) -- (2.9824, 1.2977, 0.8674) -- (2.9824, 1.2471, 0.8659) -- cycle;
\fill[blue!15.0, opacity=0.5] (2.9824, 1.2471, 0.8659) -- (2.9824, 1.2977, 0.8674) -- (2.9801, 1.2974, 0.9174) -- (2.9801, 1.2467, 0.9159) -- cycle;
\fill[blue!15.0, opacity=0.5] (2.9801, 1.2467, 0.9159) -- (2.9801, 1.2974, 0.9174) -- (2.9778, 1.2970, 0.9674) -- (2.9778, 1.2463, 0.9659) -- cycle;
\fill[blue!15.0, opacity=0.5] (2.9778, 1.2463, 0.9659) -- (2.9778, 1.2970, 0.9674) -- (2.9753, 1.2967, 1.0174) -- (2.9753, 1.2459, 1.0159) -- cycle;
\fill[blue!15.0, opacity=0.5] (2.9753, 1.2459, 1.0159) -- (2.9753, 1.2967, 1.0174) -- (2.9727, 1.2964, 1.0674) -- (2.9727, 1.2454, 1.0659) -- cycle;
\fill[blue!15.0, opacity=0.5] (2.9727, 1.2454, 1.0659) -- (2.9727, 1.2964, 1.0674) -- (2.9700, 1.2960, 1.1174) -- (2.9700, 1.2450, 1.1159) -- cycle;
\fill[blue!15.0, opacity=0.5] (2.9700, 1.2450, 1.1159) -- (2.9700, 1.2960, 1.1174) -- (2.9672, 1.2956, 1.1674) -- (2.9672, 1.2445, 1.1659) -- cycle;
\fill[blue!15.0, opacity=0.5] (2.9672, 1.2445, 1.1659) -- (2.9672, 1.2956, 1.1674) -- (2.9644, 1.2953, 1.2174) -- (2.9644, 1.2441, 1.2159) -- cycle;
\fill[blue!15.0, opacity=0.5] (2.9644, 1.2441, 1.2159) -- (2.9644, 1.2953, 1.2174) -- (2.9615, 1.2949, 1.2674) -- (2.9615, 1.2436, 1.2659) -- cycle;
\fill[blue!15.0, opacity=0.5] (2.9615, 1.2436, 1.2659) -- (2.9615, 1.2949, 1.2674) -- (2.9585, 1.2945, 1.3174) -- (2.9585, 1.2431, 1.3159) -- cycle;
\fill[blue!15.0, opacity=0.5] (2.9585, 1.2431, 1.3159) -- (2.9585, 1.2945, 1.3174) -- (2.9555, 1.2941, 1.3674) -- (2.9555, 1.2426, 1.3659) -- cycle;
\fill[blue!15.0, opacity=0.5] (2.9555, 1.2426, 1.3659) -- (2.9555, 1.2941, 1.3674) -- (2.9525, 1.2937, 1.4174) -- (2.9525, 1.2421, 1.4159) -- cycle;
\fill[blue!15.0, opacity=0.5] (2.9525, 1.2421, 1.4159) -- (2.9525, 1.2937, 1.4174) -- (2.9494, 1.2933, 1.4674) -- (2.9494, 1.2416, 1.4659) -- cycle;
\fill[blue!15.0, opacity=0.5] (2.9494, 1.2416, 1.4659) -- (2.9494, 1.2933, 1.4674) -- (2.9463, 1.2928, 1.5174) -- (2.9463, 1.2410, 1.5159) -- cycle;
\fill[blue!15.0, opacity=0.5] (2.9463, 1.2410, 1.5159) -- (2.9463, 1.2928, 1.5174) -- (2.9431, 1.2924, 1.5674) -- (2.9431, 1.2405, 1.5659) -- cycle;
\fill[blue!15.0, opacity=0.5] (2.9431, 1.2405, 1.5659) -- (2.9431, 1.2924, 1.5674) -- (2.9400, 1.2920, 1.6174) -- (2.9400, 1.2400, 1.6159) -- cycle;
\fill[blue!15.0, opacity=0.5] (2.9400, 1.2400, 1.6159) -- (2.9400, 1.2920, 1.6174) -- (2.9369, 1.2916, 1.6674) -- (2.9369, 1.2395, 1.6659) -- cycle;
\fill[blue!15.0, opacity=0.5] (2.9369, 1.2395, 1.6659) -- (2.9369, 1.2916, 1.6674) -- (2.9337, 1.2912, 1.7174) -- (2.9337, 1.2390, 1.7159) -- cycle;
\fill[blue!15.0, opacity=0.5] (2.9337, 1.2390, 1.7159) -- (2.9337, 1.2912, 1.7174) -- (2.9306, 1.2907, 1.7674) -- (2.9306, 1.2384, 1.7659) -- cycle;
\fill[blue!15.0, opacity=0.5] (2.9306, 1.2384, 1.7659) -- (2.9306, 1.2907, 1.7674) -- (2.9275, 1.2903, 1.8174) -- (2.9275, 1.2379, 1.8159) -- cycle;
\fill[blue!15.1, opacity=0.5] (2.9275, 1.2379, 1.8159) -- (2.9275, 1.2903, 1.8174) -- (2.9245, 1.2899, 1.8674) -- (2.9245, 1.2374, 1.8659) -- cycle;
\fill[blue!15.1, opacity=0.5] (2.9245, 1.2374, 1.8659) -- (2.9245, 1.2899, 1.8674) -- (2.9215, 1.2895, 1.9174) -- (2.9215, 1.2369, 1.9159) -- cycle;
\fill[blue!15.1, opacity=0.5] (2.9215, 1.2369, 1.9159) -- (2.9215, 1.2895, 1.9174) -- (2.9185, 1.2891, 1.9674) -- (2.9185, 1.2364, 1.9659) -- cycle;
\fill[blue!15.2, opacity=0.5] (2.9185, 1.2364, 1.9659) -- (2.9185, 1.2891, 1.9674) -- (2.9156, 1.2887, 2.0174) -- (2.9156, 1.2359, 2.0159) -- cycle;
\fill[blue!15.2, opacity=0.5] (2.9156, 1.2359, 2.0159) -- (2.9156, 1.2887, 2.0174) -- (2.9128, 1.2884, 2.0674) -- (2.9128, 1.2355, 2.0659) -- cycle;
\fill[blue!15.3, opacity=0.5] (2.9128, 1.2355, 2.0659) -- (2.9128, 1.2884, 2.0674) -- (2.9100, 1.2880, 2.1174) -- (2.9100, 1.2350, 2.1159) -- cycle;
\fill[blue!15.4, opacity=0.5] (2.9100, 1.2350, 2.1159) -- (2.9100, 1.2880, 2.1174) -- (2.9073, 1.2876, 2.1674) -- (2.9073, 1.2346, 2.1659) -- cycle;
\fill[blue!15.6, opacity=0.5] (2.9073, 1.2346, 2.1659) -- (2.9073, 1.2876, 2.1674) -- (2.9047, 1.2873, 2.2174) -- (2.9047, 1.2341, 2.2159) -- cycle;
\fill[blue!15.7, opacity=0.5] (2.9047, 1.2341, 2.2159) -- (2.9047, 1.2873, 2.2174) -- (2.9022, 1.2870, 2.2674) -- (2.9022, 1.2337, 2.2659) -- cycle;
\fill[blue!15.9, opacity=0.5] (2.9022, 1.2337, 2.2659) -- (2.9022, 1.2870, 2.2674) -- (2.8999, 1.2866, 2.3174) -- (2.8999, 1.2333, 2.3159) -- cycle;
\fill[blue!16.2, opacity=0.5] (2.8999, 1.2333, 2.3159) -- (2.8999, 1.2866, 2.3174) -- (2.8976, 1.2863, 2.3674) -- (2.8976, 1.2329, 2.3659) -- cycle;
\fill[blue!16.5, opacity=0.5] (2.8976, 1.2329, 2.3659) -- (2.8976, 1.2863, 2.3674) -- (2.8954, 1.2861, 2.4174) -- (2.8954, 1.2326, 2.4159) -- cycle;
\fill[blue!16.8, opacity=0.5] (2.8954, 1.2326, 2.4159) -- (2.8954, 1.2861, 2.4174) -- (2.8934, 1.2858, 2.4674) -- (2.8934, 1.2322, 2.4659) -- cycle;
\fill[blue!17.2, opacity=0.5] (2.8934, 1.2322, 2.4659) -- (2.8934, 1.2858, 2.4674) -- (2.8915, 1.2855, 2.5174) -- (2.8915, 1.2319, 2.5159) -- cycle;
\fill[blue!17.6, opacity=0.5] (2.8915, 1.2319, 2.5159) -- (2.8915, 1.2855, 2.5174) -- (2.8897, 1.2853, 2.5674) -- (2.8897, 1.2316, 2.5659) -- cycle;
\fill[blue!18.1, opacity=0.5] (2.8897, 1.2316, 2.5659) -- (2.8897, 1.2853, 2.5674) -- (2.8880, 1.2851, 2.6174) -- (2.8880, 1.2313, 2.6159) -- cycle;
\fill[blue!18.6, opacity=0.5] (2.8880, 1.2313, 2.6159) -- (2.8880, 1.2851, 2.6174) -- (2.8865, 1.2849, 2.6674) -- (2.8865, 1.2311, 2.6659) -- cycle;
\fill[blue!19.2, opacity=0.5] (2.8865, 1.2311, 2.6659) -- (2.8865, 1.2849, 2.6674) -- (2.8852, 1.2847, 2.7174) -- (2.8852, 1.2309, 2.7159) -- cycle;
\fill[blue!19.9, opacity=0.5] (2.8852, 1.2309, 2.7159) -- (2.8852, 1.2847, 2.7174) -- (2.8840, 1.2845, 2.7674) -- (2.8840, 1.2307, 2.7659) -- cycle;
\fill[blue!20.5, opacity=0.5] (2.8840, 1.2307, 2.7659) -- (2.8840, 1.2845, 2.7674) -- (2.8829, 1.2844, 2.8174) -- (2.8829, 1.2305, 2.8159) -- cycle;
\fill[blue!21.2, opacity=0.5] (2.8829, 1.2305, 2.8159) -- (2.8829, 1.2844, 2.8174) -- (2.8820, 1.2843, 2.8674) -- (2.8820, 1.2303, 2.8659) -- cycle;
\fill[blue!22.0, opacity=0.5] (2.8820, 1.2303, 2.8659) -- (2.8820, 1.2843, 2.8674) -- (2.8813, 1.2842, 2.9174) -- (2.8813, 1.2302, 2.9159) -- cycle;
\fill[blue!22.8, opacity=0.5] (2.8813, 1.2302, 2.9159) -- (2.8813, 1.2842, 2.9174) -- (2.8807, 1.2841, 2.9674) -- (2.8807, 1.2301, 2.9659) -- cycle;
\fill[blue!23.6, opacity=0.5] (2.8807, 1.2301, 2.9659) -- (2.8807, 1.2841, 2.9674) -- (2.8803, 1.2840, 3.0174) -- (2.8803, 1.2301, 3.0159) -- cycle;
\fill[blue!24.4, opacity=0.5] (2.8803, 1.2301, 3.0159) -- (2.8803, 1.2840, 3.0174) -- (2.8801, 1.2840, 3.0674) -- (2.8801, 1.2300, 3.0659) -- cycle;
\fill[blue!25.2, opacity=0.5] (2.8801, 1.2300, 3.0659) -- (2.8801, 1.2840, 3.0674) -- (2.8800, 1.2840, 3.1174) -- (2.8800, 1.2300, 3.1159) -- cycle;
\fill[blue!15.0, opacity=0.5] (3.0000, 1.3000, 0.1174) -- (3.0000, 1.3500, 0.1185) -- (2.9999, 1.3500, 0.1685) -- (2.9999, 1.3000, 0.1674) -- cycle;
\fill[blue!15.0, opacity=0.5] (2.9999, 1.3000, 0.1674) -- (2.9999, 1.3500, 0.1685) -- (2.9997, 1.3500, 0.2185) -- (2.9997, 1.3000, 0.2174) -- cycle;
\fill[blue!15.0, opacity=0.5] (2.9997, 1.3000, 0.2174) -- (2.9997, 1.3500, 0.2185) -- (2.9993, 1.3499, 0.2685) -- (2.9993, 1.2999, 0.2674) -- cycle;
\fill[blue!15.0, opacity=0.5] (2.9993, 1.2999, 0.2674) -- (2.9993, 1.3499, 0.2685) -- (2.9987, 1.3499, 0.3185) -- (2.9987, 1.2998, 0.3174) -- cycle;
\fill[blue!15.0, opacity=0.5] (2.9987, 1.2998, 0.3174) -- (2.9987, 1.3499, 0.3185) -- (2.9980, 1.3498, 0.3685) -- (2.9980, 1.2997, 0.3674) -- cycle;
\fill[blue!15.0, opacity=0.5] (2.9980, 1.2997, 0.3674) -- (2.9980, 1.3498, 0.3685) -- (2.9971, 1.3497, 0.4185) -- (2.9971, 1.2996, 0.4174) -- cycle;
\fill[blue!15.0, opacity=0.5] (2.9971, 1.2996, 0.4174) -- (2.9971, 1.3497, 0.4185) -- (2.9960, 1.3496, 0.4685) -- (2.9960, 1.2995, 0.4674) -- cycle;
\fill[blue!15.0, opacity=0.5] (2.9960, 1.2995, 0.4674) -- (2.9960, 1.3496, 0.4685) -- (2.9948, 1.3495, 0.5185) -- (2.9948, 1.2993, 0.5174) -- cycle;
\fill[blue!15.0, opacity=0.5] (2.9948, 1.2993, 0.5174) -- (2.9948, 1.3495, 0.5185) -- (2.9935, 1.3493, 0.5685) -- (2.9935, 1.2991, 0.5674) -- cycle;
\fill[blue!15.0, opacity=0.5] (2.9935, 1.2991, 0.5674) -- (2.9935, 1.3493, 0.5685) -- (2.9920, 1.3492, 0.6185) -- (2.9920, 1.2989, 0.6174) -- cycle;
\fill[blue!15.0, opacity=0.5] (2.9920, 1.2989, 0.6174) -- (2.9920, 1.3492, 0.6185) -- (2.9903, 1.3490, 0.6685) -- (2.9903, 1.2987, 0.6674) -- cycle;
\fill[blue!15.0, opacity=0.5] (2.9903, 1.2987, 0.6674) -- (2.9903, 1.3490, 0.6685) -- (2.9885, 1.3489, 0.7185) -- (2.9885, 1.2985, 0.7174) -- cycle;
\fill[blue!15.0, opacity=0.5] (2.9885, 1.2985, 0.7174) -- (2.9885, 1.3489, 0.7185) -- (2.9866, 1.3487, 0.7685) -- (2.9866, 1.2982, 0.7674) -- cycle;
\fill[blue!15.0, opacity=0.5] (2.9866, 1.2982, 0.7674) -- (2.9866, 1.3487, 0.7685) -- (2.9846, 1.3485, 0.8185) -- (2.9846, 1.2979, 0.8174) -- cycle;
\fill[blue!15.0, opacity=0.5] (2.9846, 1.2979, 0.8174) -- (2.9846, 1.3485, 0.8185) -- (2.9824, 1.3482, 0.8685) -- (2.9824, 1.2977, 0.8674) -- cycle;
\fill[blue!15.0, opacity=0.5] (2.9824, 1.2977, 0.8674) -- (2.9824, 1.3482, 0.8685) -- (2.9801, 1.3480, 0.9185) -- (2.9801, 1.2974, 0.9174) -- cycle;
\fill[blue!15.0, opacity=0.5] (2.9801, 1.2974, 0.9174) -- (2.9801, 1.3480, 0.9185) -- (2.9778, 1.3478, 0.9685) -- (2.9778, 1.2970, 0.9674) -- cycle;
\fill[blue!15.0, opacity=0.5] (2.9778, 1.2970, 0.9674) -- (2.9778, 1.3478, 0.9685) -- (2.9753, 1.3475, 1.0185) -- (2.9753, 1.2967, 1.0174) -- cycle;
\fill[blue!15.0, opacity=0.5] (2.9753, 1.2967, 1.0174) -- (2.9753, 1.3475, 1.0185) -- (2.9727, 1.3473, 1.0685) -- (2.9727, 1.2964, 1.0674) -- cycle;
\fill[blue!15.0, opacity=0.5] (2.9727, 1.2964, 1.0674) -- (2.9727, 1.3473, 1.0685) -- (2.9700, 1.3470, 1.1185) -- (2.9700, 1.2960, 1.1174) -- cycle;
\fill[blue!15.0, opacity=0.5] (2.9700, 1.2960, 1.1174) -- (2.9700, 1.3470, 1.1185) -- (2.9672, 1.3467, 1.1685) -- (2.9672, 1.2956, 1.1674) -- cycle;
\fill[blue!15.0, opacity=0.5] (2.9672, 1.2956, 1.1674) -- (2.9672, 1.3467, 1.1685) -- (2.9644, 1.3464, 1.2185) -- (2.9644, 1.2953, 1.2174) -- cycle;
\fill[blue!15.0, opacity=0.5] (2.9644, 1.2953, 1.2174) -- (2.9644, 1.3464, 1.2185) -- (2.9615, 1.3462, 1.2685) -- (2.9615, 1.2949, 1.2674) -- cycle;
\fill[blue!15.0, opacity=0.5] (2.9615, 1.2949, 1.2674) -- (2.9615, 1.3462, 1.2685) -- (2.9585, 1.3459, 1.3185) -- (2.9585, 1.2945, 1.3174) -- cycle;
\fill[blue!15.0, opacity=0.5] (2.9585, 1.2945, 1.3174) -- (2.9585, 1.3459, 1.3185) -- (2.9555, 1.3456, 1.3685) -- (2.9555, 1.2941, 1.3674) -- cycle;
\fill[blue!15.0, opacity=0.5] (2.9555, 1.2941, 1.3674) -- (2.9555, 1.3456, 1.3685) -- (2.9525, 1.3452, 1.4185) -- (2.9525, 1.2937, 1.4174) -- cycle;
\fill[blue!15.0, opacity=0.5] (2.9525, 1.2937, 1.4174) -- (2.9525, 1.3452, 1.4185) -- (2.9494, 1.3449, 1.4685) -- (2.9494, 1.2933, 1.4674) -- cycle;
\fill[blue!15.0, opacity=0.5] (2.9494, 1.2933, 1.4674) -- (2.9494, 1.3449, 1.4685) -- (2.9463, 1.3446, 1.5185) -- (2.9463, 1.2928, 1.5174) -- cycle;
\fill[blue!15.0, opacity=0.5] (2.9463, 1.2928, 1.5174) -- (2.9463, 1.3446, 1.5185) -- (2.9431, 1.3443, 1.5685) -- (2.9431, 1.2924, 1.5674) -- cycle;
\fill[blue!15.0, opacity=0.5] (2.9431, 1.2924, 1.5674) -- (2.9431, 1.3443, 1.5685) -- (2.9400, 1.3440, 1.6185) -- (2.9400, 1.2920, 1.6174) -- cycle;
\fill[blue!15.0, opacity=0.5] (2.9400, 1.2920, 1.6174) -- (2.9400, 1.3440, 1.6185) -- (2.9369, 1.3437, 1.6685) -- (2.9369, 1.2916, 1.6674) -- cycle;
\fill[blue!15.0, opacity=0.5] (2.9369, 1.2916, 1.6674) -- (2.9369, 1.3437, 1.6685) -- (2.9337, 1.3434, 1.7185) -- (2.9337, 1.2912, 1.7174) -- cycle;
\fill[blue!15.0, opacity=0.5] (2.9337, 1.2912, 1.7174) -- (2.9337, 1.3434, 1.7185) -- (2.9306, 1.3431, 1.7685) -- (2.9306, 1.2907, 1.7674) -- cycle;
\fill[blue!15.0, opacity=0.5] (2.9306, 1.2907, 1.7674) -- (2.9306, 1.3431, 1.7685) -- (2.9275, 1.3428, 1.8185) -- (2.9275, 1.2903, 1.8174) -- cycle;
\fill[blue!15.1, opacity=0.5] (2.9275, 1.2903, 1.8174) -- (2.9275, 1.3428, 1.8185) -- (2.9245, 1.3424, 1.8685) -- (2.9245, 1.2899, 1.8674) -- cycle;
\fill[blue!15.1, opacity=0.5] (2.9245, 1.2899, 1.8674) -- (2.9245, 1.3424, 1.8685) -- (2.9215, 1.3421, 1.9185) -- (2.9215, 1.2895, 1.9174) -- cycle;
\fill[blue!15.1, opacity=0.5] (2.9215, 1.2895, 1.9174) -- (2.9215, 1.3421, 1.9185) -- (2.9185, 1.3418, 1.9685) -- (2.9185, 1.2891, 1.9674) -- cycle;
\fill[blue!15.2, opacity=0.5] (2.9185, 1.2891, 1.9674) -- (2.9185, 1.3418, 1.9685) -- (2.9156, 1.3416, 2.0185) -- (2.9156, 1.2887, 2.0174) -- cycle;
\fill[blue!15.3, opacity=0.5] (2.9156, 1.2887, 2.0174) -- (2.9156, 1.3416, 2.0185) -- (2.9128, 1.3413, 2.0685) -- (2.9128, 1.2884, 2.0674) -- cycle;
\fill[blue!15.4, opacity=0.5] (2.9128, 1.2884, 2.0674) -- (2.9128, 1.3413, 2.0685) -- (2.9100, 1.3410, 2.1185) -- (2.9100, 1.2880, 2.1174) -- cycle;
\fill[blue!15.5, opacity=0.5] (2.9100, 1.2880, 2.1174) -- (2.9100, 1.3410, 2.1185) -- (2.9073, 1.3407, 2.1685) -- (2.9073, 1.2876, 2.1674) -- cycle;
\fill[blue!15.7, opacity=0.5] (2.9073, 1.2876, 2.1674) -- (2.9073, 1.3407, 2.1685) -- (2.9047, 1.3405, 2.2185) -- (2.9047, 1.2873, 2.2174) -- cycle;
\fill[blue!15.9, opacity=0.5] (2.9047, 1.2873, 2.2174) -- (2.9047, 1.3405, 2.2185) -- (2.9022, 1.3402, 2.2685) -- (2.9022, 1.2870, 2.2674) -- cycle;
\fill[blue!16.1, opacity=0.5] (2.9022, 1.2870, 2.2674) -- (2.9022, 1.3402, 2.2685) -- (2.8999, 1.3400, 2.3185) -- (2.8999, 1.2866, 2.3174) -- cycle;
\fill[blue!16.4, opacity=0.5] (2.8999, 1.2866, 2.3174) -- (2.8999, 1.3400, 2.3185) -- (2.8976, 1.3398, 2.3685) -- (2.8976, 1.2863, 2.3674) -- cycle;
\fill[blue!16.7, opacity=0.5] (2.8976, 1.2863, 2.3674) -- (2.8976, 1.3398, 2.3685) -- (2.8954, 1.3395, 2.4185) -- (2.8954, 1.2861, 2.4174) -- cycle;
\fill[blue!17.1, opacity=0.5] (2.8954, 1.2861, 2.4174) -- (2.8954, 1.3395, 2.4185) -- (2.8934, 1.3393, 2.4685) -- (2.8934, 1.2858, 2.4674) -- cycle;
\fill[blue!17.6, opacity=0.5] (2.8934, 1.2858, 2.4674) -- (2.8934, 1.3393, 2.4685) -- (2.8915, 1.3391, 2.5185) -- (2.8915, 1.2855, 2.5174) -- cycle;
\fill[blue!18.0, opacity=0.5] (2.8915, 1.2855, 2.5174) -- (2.8915, 1.3391, 2.5185) -- (2.8897, 1.3390, 2.5685) -- (2.8897, 1.2853, 2.5674) -- cycle;
\fill[blue!18.6, opacity=0.5] (2.8897, 1.2853, 2.5674) -- (2.8897, 1.3390, 2.5685) -- (2.8880, 1.3388, 2.6185) -- (2.8880, 1.2851, 2.6174) -- cycle;
\fill[blue!19.2, opacity=0.5] (2.8880, 1.2851, 2.6174) -- (2.8880, 1.3388, 2.6185) -- (2.8865, 1.3387, 2.6685) -- (2.8865, 1.2849, 2.6674) -- cycle;
\fill[blue!19.8, opacity=0.5] (2.8865, 1.2849, 2.6674) -- (2.8865, 1.3387, 2.6685) -- (2.8852, 1.3385, 2.7185) -- (2.8852, 1.2847, 2.7174) -- cycle;
\fill[blue!20.5, opacity=0.5] (2.8852, 1.2847, 2.7174) -- (2.8852, 1.3385, 2.7185) -- (2.8840, 1.3384, 2.7685) -- (2.8840, 1.2845, 2.7674) -- cycle;
\fill[blue!21.2, opacity=0.5] (2.8840, 1.2845, 2.7674) -- (2.8840, 1.3384, 2.7685) -- (2.8829, 1.3383, 2.8185) -- (2.8829, 1.2844, 2.8174) -- cycle;
\fill[blue!22.0, opacity=0.5] (2.8829, 1.2844, 2.8174) -- (2.8829, 1.3383, 2.8185) -- (2.8820, 1.3382, 2.8685) -- (2.8820, 1.2843, 2.8674) -- cycle;
\fill[blue!22.8, opacity=0.5] (2.8820, 1.2843, 2.8674) -- (2.8820, 1.3382, 2.8685) -- (2.8813, 1.3381, 2.9185) -- (2.8813, 1.2842, 2.9174) -- cycle;
\fill[blue!23.6, opacity=0.5] (2.8813, 1.2842, 2.9174) -- (2.8813, 1.3381, 2.9185) -- (2.8807, 1.3381, 2.9685) -- (2.8807, 1.2841, 2.9674) -- cycle;
\fill[blue!24.5, opacity=0.5] (2.8807, 1.2841, 2.9674) -- (2.8807, 1.3381, 2.9685) -- (2.8803, 1.3380, 3.0185) -- (2.8803, 1.2840, 3.0174) -- cycle;
\fill[blue!25.3, opacity=0.5] (2.8803, 1.2840, 3.0174) -- (2.8803, 1.3380, 3.0185) -- (2.8801, 1.3380, 3.0685) -- (2.8801, 1.2840, 3.0674) -- cycle;
\fill[blue!26.2, opacity=0.5] (2.8801, 1.2840, 3.0674) -- (2.8801, 1.3380, 3.0685) -- (2.8800, 1.3380, 3.1185) -- (2.8800, 1.2840, 3.1174) -- cycle;
\fill[blue!15.0, opacity=0.5] (3.0000, 1.3500, 0.1185) -- (3.0000, 1.4000, 0.1193) -- (2.9999, 1.4000, 0.1693) -- (2.9999, 1.3500, 0.1685) -- cycle;
\fill[blue!15.0, opacity=0.5] (2.9999, 1.3500, 0.1685) -- (2.9999, 1.4000, 0.1693) -- (2.9997, 1.4000, 0.2193) -- (2.9997, 1.3500, 0.2185) -- cycle;
\fill[blue!15.0, opacity=0.5] (2.9997, 1.3500, 0.2185) -- (2.9997, 1.4000, 0.2193) -- (2.9993, 1.4000, 0.2693) -- (2.9993, 1.3499, 0.2685) -- cycle;
\fill[blue!15.0, opacity=0.5] (2.9993, 1.3499, 0.2685) -- (2.9993, 1.4000, 0.2693) -- (2.9987, 1.3999, 0.3193) -- (2.9987, 1.3499, 0.3185) -- cycle;
\fill[blue!15.0, opacity=0.5] (2.9987, 1.3499, 0.3185) -- (2.9987, 1.3999, 0.3193) -- (2.9980, 1.3999, 0.3693) -- (2.9980, 1.3498, 0.3685) -- cycle;
\fill[blue!15.0, opacity=0.5] (2.9980, 1.3498, 0.3685) -- (2.9980, 1.3999, 0.3693) -- (2.9971, 1.3998, 0.4193) -- (2.9971, 1.3497, 0.4185) -- cycle;
\fill[blue!15.0, opacity=0.5] (2.9971, 1.3497, 0.4185) -- (2.9971, 1.3998, 0.4193) -- (2.9960, 1.3997, 0.4693) -- (2.9960, 1.3496, 0.4685) -- cycle;
\fill[blue!15.0, opacity=0.5] (2.9960, 1.3496, 0.4685) -- (2.9960, 1.3997, 0.4693) -- (2.9948, 1.3997, 0.5193) -- (2.9948, 1.3495, 0.5185) -- cycle;
\fill[blue!15.0, opacity=0.5] (2.9948, 1.3495, 0.5185) -- (2.9948, 1.3997, 0.5193) -- (2.9935, 1.3996, 0.5693) -- (2.9935, 1.3493, 0.5685) -- cycle;
\fill[blue!15.0, opacity=0.5] (2.9935, 1.3493, 0.5685) -- (2.9935, 1.3996, 0.5693) -- (2.9920, 1.3995, 0.6193) -- (2.9920, 1.3492, 0.6185) -- cycle;
\fill[blue!15.0, opacity=0.5] (2.9920, 1.3492, 0.6185) -- (2.9920, 1.3995, 0.6193) -- (2.9903, 1.3994, 0.6693) -- (2.9903, 1.3490, 0.6685) -- cycle;
\fill[blue!15.0, opacity=0.5] (2.9903, 1.3490, 0.6685) -- (2.9903, 1.3994, 0.6693) -- (2.9885, 1.3992, 0.7193) -- (2.9885, 1.3489, 0.7185) -- cycle;
\fill[blue!15.0, opacity=0.5] (2.9885, 1.3489, 0.7185) -- (2.9885, 1.3992, 0.7193) -- (2.9866, 1.3991, 0.7693) -- (2.9866, 1.3487, 0.7685) -- cycle;
\fill[blue!15.0, opacity=0.5] (2.9866, 1.3487, 0.7685) -- (2.9866, 1.3991, 0.7693) -- (2.9846, 1.3990, 0.8193) -- (2.9846, 1.3485, 0.8185) -- cycle;
\fill[blue!15.0, opacity=0.5] (2.9846, 1.3485, 0.8185) -- (2.9846, 1.3990, 0.8193) -- (2.9824, 1.3988, 0.8693) -- (2.9824, 1.3482, 0.8685) -- cycle;
\fill[blue!15.0, opacity=0.5] (2.9824, 1.3482, 0.8685) -- (2.9824, 1.3988, 0.8693) -- (2.9801, 1.3987, 0.9193) -- (2.9801, 1.3480, 0.9185) -- cycle;
\fill[blue!15.0, opacity=0.5] (2.9801, 1.3480, 0.9185) -- (2.9801, 1.3987, 0.9193) -- (2.9778, 1.3985, 0.9693) -- (2.9778, 1.3478, 0.9685) -- cycle;
\fill[blue!15.0, opacity=0.5] (2.9778, 1.3478, 0.9685) -- (2.9778, 1.3985, 0.9693) -- (2.9753, 1.3984, 1.0193) -- (2.9753, 1.3475, 1.0185) -- cycle;
\fill[blue!15.0, opacity=0.5] (2.9753, 1.3475, 1.0185) -- (2.9753, 1.3984, 1.0193) -- (2.9727, 1.3982, 1.0693) -- (2.9727, 1.3473, 1.0685) -- cycle;
\fill[blue!15.0, opacity=0.5] (2.9727, 1.3473, 1.0685) -- (2.9727, 1.3982, 1.0693) -- (2.9700, 1.3980, 1.1193) -- (2.9700, 1.3470, 1.1185) -- cycle;
\fill[blue!15.0, opacity=0.5] (2.9700, 1.3470, 1.1185) -- (2.9700, 1.3980, 1.1193) -- (2.9672, 1.3978, 1.1693) -- (2.9672, 1.3467, 1.1685) -- cycle;
\fill[blue!15.0, opacity=0.5] (2.9672, 1.3467, 1.1685) -- (2.9672, 1.3978, 1.1693) -- (2.9644, 1.3976, 1.2193) -- (2.9644, 1.3464, 1.2185) -- cycle;
\fill[blue!15.0, opacity=0.5] (2.9644, 1.3464, 1.2185) -- (2.9644, 1.3976, 1.2193) -- (2.9615, 1.3974, 1.2693) -- (2.9615, 1.3462, 1.2685) -- cycle;
\fill[blue!15.0, opacity=0.5] (2.9615, 1.3462, 1.2685) -- (2.9615, 1.3974, 1.2693) -- (2.9585, 1.3972, 1.3193) -- (2.9585, 1.3459, 1.3185) -- cycle;
\fill[blue!15.0, opacity=0.5] (2.9585, 1.3459, 1.3185) -- (2.9585, 1.3972, 1.3193) -- (2.9555, 1.3970, 1.3693) -- (2.9555, 1.3456, 1.3685) -- cycle;
\fill[blue!15.0, opacity=0.5] (2.9555, 1.3456, 1.3685) -- (2.9555, 1.3970, 1.3693) -- (2.9525, 1.3968, 1.4193) -- (2.9525, 1.3452, 1.4185) -- cycle;
\fill[blue!15.0, opacity=0.5] (2.9525, 1.3452, 1.4185) -- (2.9525, 1.3968, 1.4193) -- (2.9494, 1.3966, 1.4693) -- (2.9494, 1.3449, 1.4685) -- cycle;
\fill[blue!15.0, opacity=0.5] (2.9494, 1.3449, 1.4685) -- (2.9494, 1.3966, 1.4693) -- (2.9463, 1.3964, 1.5193) -- (2.9463, 1.3446, 1.5185) -- cycle;
\fill[blue!15.0, opacity=0.5] (2.9463, 1.3446, 1.5185) -- (2.9463, 1.3964, 1.5193) -- (2.9431, 1.3962, 1.5693) -- (2.9431, 1.3443, 1.5685) -- cycle;
\fill[blue!15.0, opacity=0.5] (2.9431, 1.3443, 1.5685) -- (2.9431, 1.3962, 1.5693) -- (2.9400, 1.3960, 1.6193) -- (2.9400, 1.3440, 1.6185) -- cycle;
\fill[blue!15.0, opacity=0.5] (2.9400, 1.3440, 1.6185) -- (2.9400, 1.3960, 1.6193) -- (2.9369, 1.3958, 1.6693) -- (2.9369, 1.3437, 1.6685) -- cycle;
\fill[blue!15.0, opacity=0.5] (2.9369, 1.3437, 1.6685) -- (2.9369, 1.3958, 1.6693) -- (2.9337, 1.3956, 1.7193) -- (2.9337, 1.3434, 1.7185) -- cycle;
\fill[blue!15.0, opacity=0.5] (2.9337, 1.3434, 1.7185) -- (2.9337, 1.3956, 1.7193) -- (2.9306, 1.3954, 1.7693) -- (2.9306, 1.3431, 1.7685) -- cycle;
\fill[blue!15.0, opacity=0.5] (2.9306, 1.3431, 1.7685) -- (2.9306, 1.3954, 1.7693) -- (2.9275, 1.3952, 1.8193) -- (2.9275, 1.3428, 1.8185) -- cycle;
\fill[blue!15.1, opacity=0.5] (2.9275, 1.3428, 1.8185) -- (2.9275, 1.3952, 1.8193) -- (2.9245, 1.3950, 1.8693) -- (2.9245, 1.3424, 1.8685) -- cycle;
\fill[blue!15.1, opacity=0.5] (2.9245, 1.3424, 1.8685) -- (2.9245, 1.3950, 1.8693) -- (2.9215, 1.3948, 1.9193) -- (2.9215, 1.3421, 1.9185) -- cycle;
\fill[blue!15.2, opacity=0.5] (2.9215, 1.3421, 1.9185) -- (2.9215, 1.3948, 1.9193) -- (2.9185, 1.3946, 1.9693) -- (2.9185, 1.3418, 1.9685) -- cycle;
\fill[blue!15.2, opacity=0.5] (2.9185, 1.3418, 1.9685) -- (2.9185, 1.3946, 1.9693) -- (2.9156, 1.3944, 2.0193) -- (2.9156, 1.3416, 2.0185) -- cycle;
\fill[blue!15.3, opacity=0.5] (2.9156, 1.3416, 2.0185) -- (2.9156, 1.3944, 2.0193) -- (2.9128, 1.3942, 2.0693) -- (2.9128, 1.3413, 2.0685) -- cycle;
\fill[blue!15.4, opacity=0.5] (2.9128, 1.3413, 2.0685) -- (2.9128, 1.3942, 2.0693) -- (2.9100, 1.3940, 2.1193) -- (2.9100, 1.3410, 2.1185) -- cycle;
\fill[blue!15.6, opacity=0.5] (2.9100, 1.3410, 2.1185) -- (2.9100, 1.3940, 2.1193) -- (2.9073, 1.3938, 2.1693) -- (2.9073, 1.3407, 2.1685) -- cycle;
\fill[blue!15.7, opacity=0.5] (2.9073, 1.3407, 2.1685) -- (2.9073, 1.3938, 2.1693) -- (2.9047, 1.3936, 2.2193) -- (2.9047, 1.3405, 2.2185) -- cycle;
\fill[blue!16.0, opacity=0.5] (2.9047, 1.3405, 2.2185) -- (2.9047, 1.3936, 2.2193) -- (2.9022, 1.3935, 2.2693) -- (2.9022, 1.3402, 2.2685) -- cycle;
\fill[blue!16.2, opacity=0.5] (2.9022, 1.3402, 2.2685) -- (2.9022, 1.3935, 2.2693) -- (2.8999, 1.3933, 2.3193) -- (2.8999, 1.3400, 2.3185) -- cycle;
\fill[blue!16.5, opacity=0.5] (2.8999, 1.3400, 2.3185) -- (2.8999, 1.3933, 2.3193) -- (2.8976, 1.3932, 2.3693) -- (2.8976, 1.3398, 2.3685) -- cycle;
\fill[blue!16.9, opacity=0.5] (2.8976, 1.3398, 2.3685) -- (2.8976, 1.3932, 2.3693) -- (2.8954, 1.3930, 2.4193) -- (2.8954, 1.3395, 2.4185) -- cycle;
\fill[blue!17.3, opacity=0.5] (2.8954, 1.3395, 2.4185) -- (2.8954, 1.3930, 2.4193) -- (2.8934, 1.3929, 2.4693) -- (2.8934, 1.3393, 2.4685) -- cycle;
\fill[blue!17.7, opacity=0.5] (2.8934, 1.3393, 2.4685) -- (2.8934, 1.3929, 2.4693) -- (2.8915, 1.3928, 2.5193) -- (2.8915, 1.3391, 2.5185) -- cycle;
\fill[blue!18.2, opacity=0.5] (2.8915, 1.3391, 2.5185) -- (2.8915, 1.3928, 2.5193) -- (2.8897, 1.3926, 2.5693) -- (2.8897, 1.3390, 2.5685) -- cycle;
\fill[blue!18.8, opacity=0.5] (2.8897, 1.3390, 2.5685) -- (2.8897, 1.3926, 2.5693) -- (2.8880, 1.3925, 2.6193) -- (2.8880, 1.3388, 2.6185) -- cycle;
\fill[blue!19.4, opacity=0.5] (2.8880, 1.3388, 2.6185) -- (2.8880, 1.3925, 2.6193) -- (2.8865, 1.3924, 2.6693) -- (2.8865, 1.3387, 2.6685) -- cycle;
\fill[blue!20.1, opacity=0.5] (2.8865, 1.3387, 2.6685) -- (2.8865, 1.3924, 2.6693) -- (2.8852, 1.3923, 2.7193) -- (2.8852, 1.3385, 2.7185) -- cycle;
\fill[blue!20.8, opacity=0.5] (2.8852, 1.3385, 2.7185) -- (2.8852, 1.3923, 2.7193) -- (2.8840, 1.3923, 2.7693) -- (2.8840, 1.3384, 2.7685) -- cycle;
\fill[blue!21.5, opacity=0.5] (2.8840, 1.3384, 2.7685) -- (2.8840, 1.3923, 2.7693) -- (2.8829, 1.3922, 2.8193) -- (2.8829, 1.3383, 2.8185) -- cycle;
\fill[blue!22.3, opacity=0.5] (2.8829, 1.3383, 2.8185) -- (2.8829, 1.3922, 2.8193) -- (2.8820, 1.3921, 2.8693) -- (2.8820, 1.3382, 2.8685) -- cycle;
\fill[blue!23.1, opacity=0.5] (2.8820, 1.3382, 2.8685) -- (2.8820, 1.3921, 2.8693) -- (2.8813, 1.3921, 2.9193) -- (2.8813, 1.3381, 2.9185) -- cycle;
\fill[blue!24.0, opacity=0.5] (2.8813, 1.3381, 2.9185) -- (2.8813, 1.3921, 2.9193) -- (2.8807, 1.3920, 2.9693) -- (2.8807, 1.3381, 2.9685) -- cycle;
\fill[blue!24.8, opacity=0.5] (2.8807, 1.3381, 2.9685) -- (2.8807, 1.3920, 2.9693) -- (2.8803, 1.3920, 3.0193) -- (2.8803, 1.3380, 3.0185) -- cycle;
\fill[blue!25.7, opacity=0.5] (2.8803, 1.3380, 3.0185) -- (2.8803, 1.3920, 3.0193) -- (2.8801, 1.3920, 3.0693) -- (2.8801, 1.3380, 3.0685) -- cycle;
\fill[blue!26.6, opacity=0.5] (2.8801, 1.3380, 3.0685) -- (2.8801, 1.3920, 3.0693) -- (2.8800, 1.3920, 3.1193) -- (2.8800, 1.3380, 3.1185) -- cycle;
\fill[blue!15.0, opacity=0.5] (3.0000, 1.4000, 0.1193) -- (3.0000, 1.4500, 0.1198) -- (2.9999, 1.4500, 0.1698) -- (2.9999, 1.4000, 0.1693) -- cycle;
\fill[blue!15.0, opacity=0.5] (2.9999, 1.4000, 0.1693) -- (2.9999, 1.4500, 0.1698) -- (2.9997, 1.4500, 0.2198) -- (2.9997, 1.4000, 0.2193) -- cycle;
\fill[blue!15.0, opacity=0.5] (2.9997, 1.4000, 0.2193) -- (2.9997, 1.4500, 0.2198) -- (2.9993, 1.4500, 0.2698) -- (2.9993, 1.4000, 0.2693) -- cycle;
\fill[blue!15.0, opacity=0.5] (2.9993, 1.4000, 0.2693) -- (2.9993, 1.4500, 0.2698) -- (2.9987, 1.4500, 0.3198) -- (2.9987, 1.3999, 0.3193) -- cycle;
\fill[blue!15.0, opacity=0.5] (2.9987, 1.3999, 0.3193) -- (2.9987, 1.4500, 0.3198) -- (2.9980, 1.4499, 0.3698) -- (2.9980, 1.3999, 0.3693) -- cycle;
\fill[blue!15.0, opacity=0.5] (2.9980, 1.3999, 0.3693) -- (2.9980, 1.4499, 0.3698) -- (2.9971, 1.4499, 0.4198) -- (2.9971, 1.3998, 0.4193) -- cycle;
\fill[blue!15.0, opacity=0.5] (2.9971, 1.3998, 0.4193) -- (2.9971, 1.4499, 0.4198) -- (2.9960, 1.4499, 0.4698) -- (2.9960, 1.3997, 0.4693) -- cycle;
\fill[blue!15.0, opacity=0.5] (2.9960, 1.3997, 0.4693) -- (2.9960, 1.4499, 0.4698) -- (2.9948, 1.4498, 0.5198) -- (2.9948, 1.3997, 0.5193) -- cycle;
\fill[blue!15.0, opacity=0.5] (2.9948, 1.3997, 0.5193) -- (2.9948, 1.4498, 0.5198) -- (2.9935, 1.4498, 0.5698) -- (2.9935, 1.3996, 0.5693) -- cycle;
\fill[blue!15.0, opacity=0.5] (2.9935, 1.3996, 0.5693) -- (2.9935, 1.4498, 0.5698) -- (2.9920, 1.4497, 0.6198) -- (2.9920, 1.3995, 0.6193) -- cycle;
\fill[blue!15.0, opacity=0.5] (2.9920, 1.3995, 0.6193) -- (2.9920, 1.4497, 0.6198) -- (2.9903, 1.4497, 0.6698) -- (2.9903, 1.3994, 0.6693) -- cycle;
\fill[blue!15.0, opacity=0.5] (2.9903, 1.3994, 0.6693) -- (2.9903, 1.4497, 0.6698) -- (2.9885, 1.4496, 0.7198) -- (2.9885, 1.3992, 0.7193) -- cycle;
\fill[blue!15.0, opacity=0.5] (2.9885, 1.3992, 0.7193) -- (2.9885, 1.4496, 0.7198) -- (2.9866, 1.4496, 0.7698) -- (2.9866, 1.3991, 0.7693) -- cycle;
\fill[blue!15.0, opacity=0.5] (2.9866, 1.3991, 0.7693) -- (2.9866, 1.4496, 0.7698) -- (2.9846, 1.4495, 0.8198) -- (2.9846, 1.3990, 0.8193) -- cycle;
\fill[blue!15.0, opacity=0.5] (2.9846, 1.3990, 0.8193) -- (2.9846, 1.4495, 0.8198) -- (2.9824, 1.4494, 0.8698) -- (2.9824, 1.3988, 0.8693) -- cycle;
\fill[blue!15.0, opacity=0.5] (2.9824, 1.3988, 0.8693) -- (2.9824, 1.4494, 0.8698) -- (2.9801, 1.4493, 0.9198) -- (2.9801, 1.3987, 0.9193) -- cycle;
\fill[blue!15.0, opacity=0.5] (2.9801, 1.3987, 0.9193) -- (2.9801, 1.4493, 0.9198) -- (2.9778, 1.4493, 0.9698) -- (2.9778, 1.3985, 0.9693) -- cycle;
\fill[blue!15.0, opacity=0.5] (2.9778, 1.3985, 0.9693) -- (2.9778, 1.4493, 0.9698) -- (2.9753, 1.4492, 1.0198) -- (2.9753, 1.3984, 1.0193) -- cycle;
\fill[blue!15.0, opacity=0.5] (2.9753, 1.3984, 1.0193) -- (2.9753, 1.4492, 1.0198) -- (2.9727, 1.4491, 1.0698) -- (2.9727, 1.3982, 1.0693) -- cycle;
\fill[blue!15.0, opacity=0.5] (2.9727, 1.3982, 1.0693) -- (2.9727, 1.4491, 1.0698) -- (2.9700, 1.4490, 1.1198) -- (2.9700, 1.3980, 1.1193) -- cycle;
\fill[blue!15.0, opacity=0.5] (2.9700, 1.3980, 1.1193) -- (2.9700, 1.4490, 1.1198) -- (2.9672, 1.4489, 1.1698) -- (2.9672, 1.3978, 1.1693) -- cycle;
\fill[blue!15.0, opacity=0.5] (2.9672, 1.3978, 1.1693) -- (2.9672, 1.4489, 1.1698) -- (2.9644, 1.4488, 1.2198) -- (2.9644, 1.3976, 1.2193) -- cycle;
\fill[blue!15.0, opacity=0.5] (2.9644, 1.3976, 1.2193) -- (2.9644, 1.4488, 1.2198) -- (2.9615, 1.4487, 1.2698) -- (2.9615, 1.3974, 1.2693) -- cycle;
\fill[blue!15.0, opacity=0.5] (2.9615, 1.3974, 1.2693) -- (2.9615, 1.4487, 1.2698) -- (2.9585, 1.4486, 1.3198) -- (2.9585, 1.3972, 1.3193) -- cycle;
\fill[blue!15.0, opacity=0.5] (2.9585, 1.3972, 1.3193) -- (2.9585, 1.4486, 1.3198) -- (2.9555, 1.4485, 1.3698) -- (2.9555, 1.3970, 1.3693) -- cycle;
\fill[blue!15.0, opacity=0.5] (2.9555, 1.3970, 1.3693) -- (2.9555, 1.4485, 1.3698) -- (2.9525, 1.4484, 1.4198) -- (2.9525, 1.3968, 1.4193) -- cycle;
\fill[blue!15.0, opacity=0.5] (2.9525, 1.3968, 1.4193) -- (2.9525, 1.4484, 1.4198) -- (2.9494, 1.4483, 1.4698) -- (2.9494, 1.3966, 1.4693) -- cycle;
\fill[blue!15.0, opacity=0.5] (2.9494, 1.3966, 1.4693) -- (2.9494, 1.4483, 1.4698) -- (2.9463, 1.4482, 1.5198) -- (2.9463, 1.3964, 1.5193) -- cycle;
\fill[blue!15.0, opacity=0.5] (2.9463, 1.3964, 1.5193) -- (2.9463, 1.4482, 1.5198) -- (2.9431, 1.4481, 1.5698) -- (2.9431, 1.3962, 1.5693) -- cycle;
\fill[blue!15.0, opacity=0.5] (2.9431, 1.3962, 1.5693) -- (2.9431, 1.4481, 1.5698) -- (2.9400, 1.4480, 1.6198) -- (2.9400, 1.3960, 1.6193) -- cycle;
\fill[blue!15.0, opacity=0.5] (2.9400, 1.3960, 1.6193) -- (2.9400, 1.4480, 1.6198) -- (2.9369, 1.4479, 1.6698) -- (2.9369, 1.3958, 1.6693) -- cycle;
\fill[blue!15.0, opacity=0.5] (2.9369, 1.3958, 1.6693) -- (2.9369, 1.4479, 1.6698) -- (2.9337, 1.4478, 1.7198) -- (2.9337, 1.3956, 1.7193) -- cycle;
\fill[blue!15.0, opacity=0.5] (2.9337, 1.3956, 1.7193) -- (2.9337, 1.4478, 1.7198) -- (2.9306, 1.4477, 1.7698) -- (2.9306, 1.3954, 1.7693) -- cycle;
\fill[blue!15.0, opacity=0.5] (2.9306, 1.3954, 1.7693) -- (2.9306, 1.4477, 1.7698) -- (2.9275, 1.4476, 1.8198) -- (2.9275, 1.3952, 1.8193) -- cycle;
\fill[blue!15.1, opacity=0.5] (2.9275, 1.3952, 1.8193) -- (2.9275, 1.4476, 1.8198) -- (2.9245, 1.4475, 1.8698) -- (2.9245, 1.3950, 1.8693) -- cycle;
\fill[blue!15.1, opacity=0.5] (2.9245, 1.3950, 1.8693) -- (2.9245, 1.4475, 1.8698) -- (2.9215, 1.4474, 1.9198) -- (2.9215, 1.3948, 1.9193) -- cycle;
\fill[blue!15.2, opacity=0.5] (2.9215, 1.3948, 1.9193) -- (2.9215, 1.4474, 1.9198) -- (2.9185, 1.4473, 1.9698) -- (2.9185, 1.3946, 1.9693) -- cycle;
\fill[blue!15.2, opacity=0.5] (2.9185, 1.3946, 1.9693) -- (2.9185, 1.4473, 1.9698) -- (2.9156, 1.4472, 2.0198) -- (2.9156, 1.3944, 2.0193) -- cycle;
\fill[blue!15.3, opacity=0.5] (2.9156, 1.3944, 2.0193) -- (2.9156, 1.4472, 2.0198) -- (2.9128, 1.4471, 2.0698) -- (2.9128, 1.3942, 2.0693) -- cycle;
\fill[blue!15.4, opacity=0.5] (2.9128, 1.3942, 2.0693) -- (2.9128, 1.4471, 2.0698) -- (2.9100, 1.4470, 2.1198) -- (2.9100, 1.3940, 2.1193) -- cycle;
\fill[blue!15.5, opacity=0.5] (2.9100, 1.3940, 2.1193) -- (2.9100, 1.4470, 2.1198) -- (2.9073, 1.4469, 2.1698) -- (2.9073, 1.3938, 2.1693) -- cycle;
\fill[blue!15.7, opacity=0.5] (2.9073, 1.3938, 2.1693) -- (2.9073, 1.4469, 2.1698) -- (2.9047, 1.4468, 2.2198) -- (2.9047, 1.3936, 2.2193) -- cycle;
\fill[blue!15.9, opacity=0.5] (2.9047, 1.3936, 2.2193) -- (2.9047, 1.4468, 2.2198) -- (2.9022, 1.4467, 2.2698) -- (2.9022, 1.3935, 2.2693) -- cycle;
\fill[blue!16.2, opacity=0.5] (2.9022, 1.3935, 2.2693) -- (2.9022, 1.4467, 2.2698) -- (2.8999, 1.4467, 2.3198) -- (2.8999, 1.3933, 2.3193) -- cycle;
\fill[blue!16.5, opacity=0.5] (2.8999, 1.3933, 2.3193) -- (2.8999, 1.4467, 2.3198) -- (2.8976, 1.4466, 2.3698) -- (2.8976, 1.3932, 2.3693) -- cycle;
\fill[blue!16.8, opacity=0.5] (2.8976, 1.3932, 2.3693) -- (2.8976, 1.4466, 2.3698) -- (2.8954, 1.4465, 2.4198) -- (2.8954, 1.3930, 2.4193) -- cycle;
\fill[blue!17.2, opacity=0.5] (2.8954, 1.3930, 2.4193) -- (2.8954, 1.4465, 2.4198) -- (2.8934, 1.4464, 2.4698) -- (2.8934, 1.3929, 2.4693) -- cycle;
\fill[blue!17.6, opacity=0.5] (2.8934, 1.3929, 2.4693) -- (2.8934, 1.4464, 2.4698) -- (2.8915, 1.4464, 2.5198) -- (2.8915, 1.3928, 2.5193) -- cycle;
\fill[blue!18.1, opacity=0.5] (2.8915, 1.3928, 2.5193) -- (2.8915, 1.4464, 2.5198) -- (2.8897, 1.4463, 2.5698) -- (2.8897, 1.3926, 2.5693) -- cycle;
\fill[blue!18.7, opacity=0.5] (2.8897, 1.3926, 2.5693) -- (2.8897, 1.4463, 2.5698) -- (2.8880, 1.4463, 2.6198) -- (2.8880, 1.3925, 2.6193) -- cycle;
\fill[blue!19.3, opacity=0.5] (2.8880, 1.3925, 2.6193) -- (2.8880, 1.4463, 2.6198) -- (2.8865, 1.4462, 2.6698) -- (2.8865, 1.3924, 2.6693) -- cycle;
\fill[blue!19.9, opacity=0.5] (2.8865, 1.3924, 2.6693) -- (2.8865, 1.4462, 2.6698) -- (2.8852, 1.4462, 2.7198) -- (2.8852, 1.3923, 2.7193) -- cycle;
\fill[blue!20.6, opacity=0.5] (2.8852, 1.3923, 2.7193) -- (2.8852, 1.4462, 2.7198) -- (2.8840, 1.4461, 2.7698) -- (2.8840, 1.3923, 2.7693) -- cycle;
\fill[blue!21.4, opacity=0.5] (2.8840, 1.3923, 2.7693) -- (2.8840, 1.4461, 2.7698) -- (2.8829, 1.4461, 2.8198) -- (2.8829, 1.3922, 2.8193) -- cycle;
\fill[blue!22.2, opacity=0.5] (2.8829, 1.3922, 2.8193) -- (2.8829, 1.4461, 2.8198) -- (2.8820, 1.4461, 2.8698) -- (2.8820, 1.3921, 2.8693) -- cycle;
\fill[blue!23.0, opacity=0.5] (2.8820, 1.3921, 2.8693) -- (2.8820, 1.4461, 2.8698) -- (2.8813, 1.4460, 2.9198) -- (2.8813, 1.3921, 2.9193) -- cycle;
\fill[blue!23.8, opacity=0.5] (2.8813, 1.3921, 2.9193) -- (2.8813, 1.4460, 2.9198) -- (2.8807, 1.4460, 2.9698) -- (2.8807, 1.3920, 2.9693) -- cycle;
\fill[blue!24.7, opacity=0.5] (2.8807, 1.3920, 2.9693) -- (2.8807, 1.4460, 2.9698) -- (2.8803, 1.4460, 3.0198) -- (2.8803, 1.3920, 3.0193) -- cycle;
\fill[blue!25.5, opacity=0.5] (2.8803, 1.3920, 3.0193) -- (2.8803, 1.4460, 3.0198) -- (2.8801, 1.4460, 3.0698) -- (2.8801, 1.3920, 3.0693) -- cycle;
\fill[blue!26.4, opacity=0.5] (2.8801, 1.3920, 3.0693) -- (2.8801, 1.4460, 3.0698) -- (2.8800, 1.4460, 3.1198) -- (2.8800, 1.3920, 3.1193) -- cycle;
\fill[blue!15.0, opacity=0.5] (3.0000, 1.4500, 0.1198) -- (3.0000, 1.5000, 0.1200) -- (2.9999, 1.5000, 0.1700) -- (2.9999, 1.4500, 0.1698) -- cycle;
\fill[blue!15.0, opacity=0.5] (2.9999, 1.4500, 0.1698) -- (2.9999, 1.5000, 0.1700) -- (2.9997, 1.5000, 0.2200) -- (2.9997, 1.4500, 0.2198) -- cycle;
\fill[blue!15.0, opacity=0.5] (2.9997, 1.4500, 0.2198) -- (2.9997, 1.5000, 0.2200) -- (2.9993, 1.5000, 0.2700) -- (2.9993, 1.4500, 0.2698) -- cycle;
\fill[blue!15.0, opacity=0.5] (2.9993, 1.4500, 0.2698) -- (2.9993, 1.5000, 0.2700) -- (2.9987, 1.5000, 0.3200) -- (2.9987, 1.4500, 0.3198) -- cycle;
\fill[blue!15.0, opacity=0.5] (2.9987, 1.4500, 0.3198) -- (2.9987, 1.5000, 0.3200) -- (2.9980, 1.5000, 0.3700) -- (2.9980, 1.4499, 0.3698) -- cycle;
\fill[blue!15.0, opacity=0.5] (2.9980, 1.4499, 0.3698) -- (2.9980, 1.5000, 0.3700) -- (2.9971, 1.5000, 0.4200) -- (2.9971, 1.4499, 0.4198) -- cycle;
\fill[blue!15.0, opacity=0.5] (2.9971, 1.4499, 0.4198) -- (2.9971, 1.5000, 0.4200) -- (2.9960, 1.5000, 0.4700) -- (2.9960, 1.4499, 0.4698) -- cycle;
\fill[blue!15.0, opacity=0.5] (2.9960, 1.4499, 0.4698) -- (2.9960, 1.5000, 0.4700) -- (2.9948, 1.5000, 0.5200) -- (2.9948, 1.4498, 0.5198) -- cycle;
\fill[blue!15.0, opacity=0.5] (2.9948, 1.4498, 0.5198) -- (2.9948, 1.5000, 0.5200) -- (2.9935, 1.5000, 0.5700) -- (2.9935, 1.4498, 0.5698) -- cycle;
\fill[blue!15.0, opacity=0.5] (2.9935, 1.4498, 0.5698) -- (2.9935, 1.5000, 0.5700) -- (2.9920, 1.5000, 0.6200) -- (2.9920, 1.4497, 0.6198) -- cycle;
\fill[blue!15.0, opacity=0.5] (2.9920, 1.4497, 0.6198) -- (2.9920, 1.5000, 0.6200) -- (2.9903, 1.5000, 0.6700) -- (2.9903, 1.4497, 0.6698) -- cycle;
\fill[blue!15.0, opacity=0.5] (2.9903, 1.4497, 0.6698) -- (2.9903, 1.5000, 0.6700) -- (2.9885, 1.5000, 0.7200) -- (2.9885, 1.4496, 0.7198) -- cycle;
\fill[blue!15.0, opacity=0.5] (2.9885, 1.4496, 0.7198) -- (2.9885, 1.5000, 0.7200) -- (2.9866, 1.5000, 0.7700) -- (2.9866, 1.4496, 0.7698) -- cycle;
\fill[blue!15.0, opacity=0.5] (2.9866, 1.4496, 0.7698) -- (2.9866, 1.5000, 0.7700) -- (2.9846, 1.5000, 0.8200) -- (2.9846, 1.4495, 0.8198) -- cycle;
\fill[blue!15.0, opacity=0.5] (2.9846, 1.4495, 0.8198) -- (2.9846, 1.5000, 0.8200) -- (2.9824, 1.5000, 0.8700) -- (2.9824, 1.4494, 0.8698) -- cycle;
\fill[blue!15.0, opacity=0.5] (2.9824, 1.4494, 0.8698) -- (2.9824, 1.5000, 0.8700) -- (2.9801, 1.5000, 0.9200) -- (2.9801, 1.4493, 0.9198) -- cycle;
\fill[blue!15.0, opacity=0.5] (2.9801, 1.4493, 0.9198) -- (2.9801, 1.5000, 0.9200) -- (2.9778, 1.5000, 0.9700) -- (2.9778, 1.4493, 0.9698) -- cycle;
\fill[blue!15.0, opacity=0.5] (2.9778, 1.4493, 0.9698) -- (2.9778, 1.5000, 0.9700) -- (2.9753, 1.5000, 1.0200) -- (2.9753, 1.4492, 1.0198) -- cycle;
\fill[blue!15.0, opacity=0.5] (2.9753, 1.4492, 1.0198) -- (2.9753, 1.5000, 1.0200) -- (2.9727, 1.5000, 1.0700) -- (2.9727, 1.4491, 1.0698) -- cycle;
\fill[blue!15.0, opacity=0.5] (2.9727, 1.4491, 1.0698) -- (2.9727, 1.5000, 1.0700) -- (2.9700, 1.5000, 1.1200) -- (2.9700, 1.4490, 1.1198) -- cycle;
\fill[blue!15.0, opacity=0.5] (2.9700, 1.4490, 1.1198) -- (2.9700, 1.5000, 1.1200) -- (2.9672, 1.5000, 1.1700) -- (2.9672, 1.4489, 1.1698) -- cycle;
\fill[blue!15.0, opacity=0.5] (2.9672, 1.4489, 1.1698) -- (2.9672, 1.5000, 1.1700) -- (2.9644, 1.5000, 1.2200) -- (2.9644, 1.4488, 1.2198) -- cycle;
\fill[blue!15.0, opacity=0.5] (2.9644, 1.4488, 1.2198) -- (2.9644, 1.5000, 1.2200) -- (2.9615, 1.5000, 1.2700) -- (2.9615, 1.4487, 1.2698) -- cycle;
\fill[blue!15.0, opacity=0.5] (2.9615, 1.4487, 1.2698) -- (2.9615, 1.5000, 1.2700) -- (2.9585, 1.5000, 1.3200) -- (2.9585, 1.4486, 1.3198) -- cycle;
\fill[blue!15.0, opacity=0.5] (2.9585, 1.4486, 1.3198) -- (2.9585, 1.5000, 1.3200) -- (2.9555, 1.5000, 1.3700) -- (2.9555, 1.4485, 1.3698) -- cycle;
\fill[blue!15.0, opacity=0.5] (2.9555, 1.4485, 1.3698) -- (2.9555, 1.5000, 1.3700) -- (2.9525, 1.5000, 1.4200) -- (2.9525, 1.4484, 1.4198) -- cycle;
\fill[blue!15.0, opacity=0.5] (2.9525, 1.4484, 1.4198) -- (2.9525, 1.5000, 1.4200) -- (2.9494, 1.5000, 1.4700) -- (2.9494, 1.4483, 1.4698) -- cycle;
\fill[blue!15.0, opacity=0.5] (2.9494, 1.4483, 1.4698) -- (2.9494, 1.5000, 1.4700) -- (2.9463, 1.5000, 1.5200) -- (2.9463, 1.4482, 1.5198) -- cycle;
\fill[blue!15.0, opacity=0.5] (2.9463, 1.4482, 1.5198) -- (2.9463, 1.5000, 1.5200) -- (2.9431, 1.5000, 1.5700) -- (2.9431, 1.4481, 1.5698) -- cycle;
\fill[blue!15.0, opacity=0.5] (2.9431, 1.4481, 1.5698) -- (2.9431, 1.5000, 1.5700) -- (2.9400, 1.5000, 1.6200) -- (2.9400, 1.4480, 1.6198) -- cycle;
\fill[blue!15.0, opacity=0.5] (2.9400, 1.4480, 1.6198) -- (2.9400, 1.5000, 1.6200) -- (2.9369, 1.5000, 1.6700) -- (2.9369, 1.4479, 1.6698) -- cycle;
\fill[blue!15.0, opacity=0.5] (2.9369, 1.4479, 1.6698) -- (2.9369, 1.5000, 1.6700) -- (2.9337, 1.5000, 1.7200) -- (2.9337, 1.4478, 1.7198) -- cycle;
\fill[blue!15.0, opacity=0.5] (2.9337, 1.4478, 1.7198) -- (2.9337, 1.5000, 1.7200) -- (2.9306, 1.5000, 1.7700) -- (2.9306, 1.4477, 1.7698) -- cycle;
\fill[blue!15.0, opacity=0.5] (2.9306, 1.4477, 1.7698) -- (2.9306, 1.5000, 1.7700) -- (2.9275, 1.5000, 1.8200) -- (2.9275, 1.4476, 1.8198) -- cycle;
\fill[blue!15.1, opacity=0.5] (2.9275, 1.4476, 1.8198) -- (2.9275, 1.5000, 1.8200) -- (2.9245, 1.5000, 1.8700) -- (2.9245, 1.4475, 1.8698) -- cycle;
\fill[blue!15.1, opacity=0.5] (2.9245, 1.4475, 1.8698) -- (2.9245, 1.5000, 1.8700) -- (2.9215, 1.5000, 1.9200) -- (2.9215, 1.4474, 1.9198) -- cycle;
\fill[blue!15.1, opacity=0.5] (2.9215, 1.4474, 1.9198) -- (2.9215, 1.5000, 1.9200) -- (2.9185, 1.5000, 1.9700) -- (2.9185, 1.4473, 1.9698) -- cycle;
\fill[blue!15.2, opacity=0.5] (2.9185, 1.4473, 1.9698) -- (2.9185, 1.5000, 1.9700) -- (2.9156, 1.5000, 2.0200) -- (2.9156, 1.4472, 2.0198) -- cycle;
\fill[blue!15.3, opacity=0.5] (2.9156, 1.4472, 2.0198) -- (2.9156, 1.5000, 2.0200) -- (2.9128, 1.5000, 2.0700) -- (2.9128, 1.4471, 2.0698) -- cycle;
\fill[blue!15.3, opacity=0.5] (2.9128, 1.4471, 2.0698) -- (2.9128, 1.5000, 2.0700) -- (2.9100, 1.5000, 2.1200) -- (2.9100, 1.4470, 2.1198) -- cycle;
\fill[blue!15.5, opacity=0.5] (2.9100, 1.4470, 2.1198) -- (2.9100, 1.5000, 2.1200) -- (2.9073, 1.5000, 2.1700) -- (2.9073, 1.4469, 2.1698) -- cycle;
\fill[blue!15.6, opacity=0.5] (2.9073, 1.4469, 2.1698) -- (2.9073, 1.5000, 2.1700) -- (2.9047, 1.5000, 2.2200) -- (2.9047, 1.4468, 2.2198) -- cycle;
\fill[blue!15.8, opacity=0.5] (2.9047, 1.4468, 2.2198) -- (2.9047, 1.5000, 2.2200) -- (2.9022, 1.5000, 2.2700) -- (2.9022, 1.4467, 2.2698) -- cycle;
\fill[blue!16.0, opacity=0.5] (2.9022, 1.4467, 2.2698) -- (2.9022, 1.5000, 2.2700) -- (2.8999, 1.5000, 2.3200) -- (2.8999, 1.4467, 2.3198) -- cycle;
\fill[blue!16.3, opacity=0.5] (2.8999, 1.4467, 2.3198) -- (2.8999, 1.5000, 2.3200) -- (2.8976, 1.5000, 2.3700) -- (2.8976, 1.4466, 2.3698) -- cycle;
\fill[blue!16.6, opacity=0.5] (2.8976, 1.4466, 2.3698) -- (2.8976, 1.5000, 2.3700) -- (2.8954, 1.5000, 2.4200) -- (2.8954, 1.4465, 2.4198) -- cycle;
\fill[blue!16.9, opacity=0.5] (2.8954, 1.4465, 2.4198) -- (2.8954, 1.5000, 2.4200) -- (2.8934, 1.5000, 2.4700) -- (2.8934, 1.4464, 2.4698) -- cycle;
\fill[blue!17.3, opacity=0.5] (2.8934, 1.4464, 2.4698) -- (2.8934, 1.5000, 2.4700) -- (2.8915, 1.5000, 2.5200) -- (2.8915, 1.4464, 2.5198) -- cycle;
\fill[blue!17.8, opacity=0.5] (2.8915, 1.4464, 2.5198) -- (2.8915, 1.5000, 2.5200) -- (2.8897, 1.5000, 2.5700) -- (2.8897, 1.4463, 2.5698) -- cycle;
\fill[blue!18.3, opacity=0.5] (2.8897, 1.4463, 2.5698) -- (2.8897, 1.5000, 2.5700) -- (2.8880, 1.5000, 2.6200) -- (2.8880, 1.4463, 2.6198) -- cycle;
\fill[blue!18.9, opacity=0.5] (2.8880, 1.4463, 2.6198) -- (2.8880, 1.5000, 2.6200) -- (2.8865, 1.5000, 2.6700) -- (2.8865, 1.4462, 2.6698) -- cycle;
\fill[blue!19.5, opacity=0.5] (2.8865, 1.4462, 2.6698) -- (2.8865, 1.5000, 2.6700) -- (2.8852, 1.5000, 2.7200) -- (2.8852, 1.4462, 2.7198) -- cycle;
\fill[blue!20.1, opacity=0.5] (2.8852, 1.4462, 2.7198) -- (2.8852, 1.5000, 2.7200) -- (2.8840, 1.5000, 2.7700) -- (2.8840, 1.4461, 2.7698) -- cycle;
\fill[blue!20.8, opacity=0.5] (2.8840, 1.4461, 2.7698) -- (2.8840, 1.5000, 2.7700) -- (2.8829, 1.5000, 2.8200) -- (2.8829, 1.4461, 2.8198) -- cycle;
\fill[blue!21.6, opacity=0.5] (2.8829, 1.4461, 2.8198) -- (2.8829, 1.5000, 2.8200) -- (2.8820, 1.5000, 2.8700) -- (2.8820, 1.4461, 2.8698) -- cycle;
\fill[blue!22.3, opacity=0.5] (2.8820, 1.4461, 2.8698) -- (2.8820, 1.5000, 2.8700) -- (2.8813, 1.5000, 2.9200) -- (2.8813, 1.4460, 2.9198) -- cycle;
\fill[blue!23.1, opacity=0.5] (2.8813, 1.4460, 2.9198) -- (2.8813, 1.5000, 2.9200) -- (2.8807, 1.5000, 2.9700) -- (2.8807, 1.4460, 2.9698) -- cycle;
\fill[blue!23.9, opacity=0.5] (2.8807, 1.4460, 2.9698) -- (2.8807, 1.5000, 2.9700) -- (2.8803, 1.5000, 3.0200) -- (2.8803, 1.4460, 3.0198) -- cycle;
\fill[blue!24.8, opacity=0.5] (2.8803, 1.4460, 3.0198) -- (2.8803, 1.5000, 3.0200) -- (2.8801, 1.5000, 3.0700) -- (2.8801, 1.4460, 3.0698) -- cycle;
\fill[blue!25.6, opacity=0.5] (2.8801, 1.4460, 3.0698) -- (2.8801, 1.5000, 3.0700) -- (2.8800, 1.5000, 3.1200) -- (2.8800, 1.4460, 3.1198) -- cycle;
\fill[blue!15.0, opacity=0.5] (3.0000, 1.5000, 0.1200) -- (3.0000, 1.5500, 0.1198) -- (2.9999, 1.5500, 0.1698) -- (2.9999, 1.5000, 0.1700) -- cycle;
\fill[blue!15.0, opacity=0.5] (2.9999, 1.5000, 0.1700) -- (2.9999, 1.5500, 0.1698) -- (2.9997, 1.5500, 0.2198) -- (2.9997, 1.5000, 0.2200) -- cycle;
\fill[blue!15.0, opacity=0.5] (2.9997, 1.5000, 0.2200) -- (2.9997, 1.5500, 0.2198) -- (2.9993, 1.5500, 0.2698) -- (2.9993, 1.5000, 0.2700) -- cycle;
\fill[blue!15.0, opacity=0.5] (2.9993, 1.5000, 0.2700) -- (2.9993, 1.5500, 0.2698) -- (2.9987, 1.5500, 0.3198) -- (2.9987, 1.5000, 0.3200) -- cycle;
\fill[blue!15.0, opacity=0.5] (2.9987, 1.5000, 0.3200) -- (2.9987, 1.5500, 0.3198) -- (2.9980, 1.5501, 0.3698) -- (2.9980, 1.5000, 0.3700) -- cycle;
\fill[blue!15.0, opacity=0.5] (2.9980, 1.5000, 0.3700) -- (2.9980, 1.5501, 0.3698) -- (2.9971, 1.5501, 0.4198) -- (2.9971, 1.5000, 0.4200) -- cycle;
\fill[blue!15.0, opacity=0.5] (2.9971, 1.5000, 0.4200) -- (2.9971, 1.5501, 0.4198) -- (2.9960, 1.5501, 0.4698) -- (2.9960, 1.5000, 0.4700) -- cycle;
\fill[blue!15.0, opacity=0.5] (2.9960, 1.5000, 0.4700) -- (2.9960, 1.5501, 0.4698) -- (2.9948, 1.5502, 0.5198) -- (2.9948, 1.5000, 0.5200) -- cycle;
\fill[blue!15.0, opacity=0.5] (2.9948, 1.5000, 0.5200) -- (2.9948, 1.5502, 0.5198) -- (2.9935, 1.5502, 0.5698) -- (2.9935, 1.5000, 0.5700) -- cycle;
\fill[blue!15.0, opacity=0.5] (2.9935, 1.5000, 0.5700) -- (2.9935, 1.5502, 0.5698) -- (2.9920, 1.5503, 0.6198) -- (2.9920, 1.5000, 0.6200) -- cycle;
\fill[blue!15.0, opacity=0.5] (2.9920, 1.5000, 0.6200) -- (2.9920, 1.5503, 0.6198) -- (2.9903, 1.5503, 0.6698) -- (2.9903, 1.5000, 0.6700) -- cycle;
\fill[blue!15.0, opacity=0.5] (2.9903, 1.5000, 0.6700) -- (2.9903, 1.5503, 0.6698) -- (2.9885, 1.5504, 0.7198) -- (2.9885, 1.5000, 0.7200) -- cycle;
\fill[blue!15.0, opacity=0.5] (2.9885, 1.5000, 0.7200) -- (2.9885, 1.5504, 0.7198) -- (2.9866, 1.5504, 0.7698) -- (2.9866, 1.5000, 0.7700) -- cycle;
\fill[blue!15.0, opacity=0.5] (2.9866, 1.5000, 0.7700) -- (2.9866, 1.5504, 0.7698) -- (2.9846, 1.5505, 0.8198) -- (2.9846, 1.5000, 0.8200) -- cycle;
\fill[blue!15.0, opacity=0.5] (2.9846, 1.5000, 0.8200) -- (2.9846, 1.5505, 0.8198) -- (2.9824, 1.5506, 0.8698) -- (2.9824, 1.5000, 0.8700) -- cycle;
\fill[blue!15.0, opacity=0.5] (2.9824, 1.5000, 0.8700) -- (2.9824, 1.5506, 0.8698) -- (2.9801, 1.5507, 0.9198) -- (2.9801, 1.5000, 0.9200) -- cycle;
\fill[blue!15.0, opacity=0.5] (2.9801, 1.5000, 0.9200) -- (2.9801, 1.5507, 0.9198) -- (2.9778, 1.5507, 0.9698) -- (2.9778, 1.5000, 0.9700) -- cycle;
\fill[blue!15.0, opacity=0.5] (2.9778, 1.5000, 0.9700) -- (2.9778, 1.5507, 0.9698) -- (2.9753, 1.5508, 1.0198) -- (2.9753, 1.5000, 1.0200) -- cycle;
\fill[blue!15.0, opacity=0.5] (2.9753, 1.5000, 1.0200) -- (2.9753, 1.5508, 1.0198) -- (2.9727, 1.5509, 1.0698) -- (2.9727, 1.5000, 1.0700) -- cycle;
\fill[blue!15.0, opacity=0.5] (2.9727, 1.5000, 1.0700) -- (2.9727, 1.5509, 1.0698) -- (2.9700, 1.5510, 1.1198) -- (2.9700, 1.5000, 1.1200) -- cycle;
\fill[blue!15.0, opacity=0.5] (2.9700, 1.5000, 1.1200) -- (2.9700, 1.5510, 1.1198) -- (2.9672, 1.5511, 1.1698) -- (2.9672, 1.5000, 1.1700) -- cycle;
\fill[blue!15.0, opacity=0.5] (2.9672, 1.5000, 1.1700) -- (2.9672, 1.5511, 1.1698) -- (2.9644, 1.5512, 1.2198) -- (2.9644, 1.5000, 1.2200) -- cycle;
\fill[blue!15.0, opacity=0.5] (2.9644, 1.5000, 1.2200) -- (2.9644, 1.5512, 1.2198) -- (2.9615, 1.5513, 1.2698) -- (2.9615, 1.5000, 1.2700) -- cycle;
\fill[blue!15.0, opacity=0.5] (2.9615, 1.5000, 1.2700) -- (2.9615, 1.5513, 1.2698) -- (2.9585, 1.5514, 1.3198) -- (2.9585, 1.5000, 1.3200) -- cycle;
\fill[blue!15.0, opacity=0.5] (2.9585, 1.5000, 1.3200) -- (2.9585, 1.5514, 1.3198) -- (2.9555, 1.5515, 1.3698) -- (2.9555, 1.5000, 1.3700) -- cycle;
\fill[blue!15.0, opacity=0.5] (2.9555, 1.5000, 1.3700) -- (2.9555, 1.5515, 1.3698) -- (2.9525, 1.5516, 1.4198) -- (2.9525, 1.5000, 1.4200) -- cycle;
\fill[blue!15.0, opacity=0.5] (2.9525, 1.5000, 1.4200) -- (2.9525, 1.5516, 1.4198) -- (2.9494, 1.5517, 1.4698) -- (2.9494, 1.5000, 1.4700) -- cycle;
\fill[blue!15.0, opacity=0.5] (2.9494, 1.5000, 1.4700) -- (2.9494, 1.5517, 1.4698) -- (2.9463, 1.5518, 1.5198) -- (2.9463, 1.5000, 1.5200) -- cycle;
\fill[blue!15.0, opacity=0.5] (2.9463, 1.5000, 1.5200) -- (2.9463, 1.5518, 1.5198) -- (2.9431, 1.5519, 1.5698) -- (2.9431, 1.5000, 1.5700) -- cycle;
\fill[blue!15.0, opacity=0.5] (2.9431, 1.5000, 1.5700) -- (2.9431, 1.5519, 1.5698) -- (2.9400, 1.5520, 1.6198) -- (2.9400, 1.5000, 1.6200) -- cycle;
\fill[blue!15.0, opacity=0.5] (2.9400, 1.5000, 1.6200) -- (2.9400, 1.5520, 1.6198) -- (2.9369, 1.5521, 1.6698) -- (2.9369, 1.5000, 1.6700) -- cycle;
\fill[blue!15.0, opacity=0.5] (2.9369, 1.5000, 1.6700) -- (2.9369, 1.5521, 1.6698) -- (2.9337, 1.5522, 1.7198) -- (2.9337, 1.5000, 1.7200) -- cycle;
\fill[blue!15.0, opacity=0.5] (2.9337, 1.5000, 1.7200) -- (2.9337, 1.5522, 1.7198) -- (2.9306, 1.5523, 1.7698) -- (2.9306, 1.5000, 1.7700) -- cycle;
\fill[blue!15.0, opacity=0.5] (2.9306, 1.5000, 1.7700) -- (2.9306, 1.5523, 1.7698) -- (2.9275, 1.5524, 1.8198) -- (2.9275, 1.5000, 1.8200) -- cycle;
\fill[blue!15.0, opacity=0.5] (2.9275, 1.5000, 1.8200) -- (2.9275, 1.5524, 1.8198) -- (2.9245, 1.5525, 1.8698) -- (2.9245, 1.5000, 1.8700) -- cycle;
\fill[blue!15.1, opacity=0.5] (2.9245, 1.5000, 1.8700) -- (2.9245, 1.5525, 1.8698) -- (2.9215, 1.5526, 1.9198) -- (2.9215, 1.5000, 1.9200) -- cycle;
\fill[blue!15.1, opacity=0.5] (2.9215, 1.5000, 1.9200) -- (2.9215, 1.5526, 1.9198) -- (2.9185, 1.5527, 1.9698) -- (2.9185, 1.5000, 1.9700) -- cycle;
\fill[blue!15.1, opacity=0.5] (2.9185, 1.5000, 1.9700) -- (2.9185, 1.5527, 1.9698) -- (2.9156, 1.5528, 2.0198) -- (2.9156, 1.5000, 2.0200) -- cycle;
\fill[blue!15.2, opacity=0.5] (2.9156, 1.5000, 2.0200) -- (2.9156, 1.5528, 2.0198) -- (2.9128, 1.5529, 2.0698) -- (2.9128, 1.5000, 2.0700) -- cycle;
\fill[blue!15.3, opacity=0.5] (2.9128, 1.5000, 2.0700) -- (2.9128, 1.5529, 2.0698) -- (2.9100, 1.5530, 2.1198) -- (2.9100, 1.5000, 2.1200) -- cycle;
\fill[blue!15.4, opacity=0.5] (2.9100, 1.5000, 2.1200) -- (2.9100, 1.5530, 2.1198) -- (2.9073, 1.5531, 2.1698) -- (2.9073, 1.5000, 2.1700) -- cycle;
\fill[blue!15.5, opacity=0.5] (2.9073, 1.5000, 2.1700) -- (2.9073, 1.5531, 2.1698) -- (2.9047, 1.5532, 2.2198) -- (2.9047, 1.5000, 2.2200) -- cycle;
\fill[blue!15.6, opacity=0.5] (2.9047, 1.5000, 2.2200) -- (2.9047, 1.5532, 2.2198) -- (2.9022, 1.5533, 2.2698) -- (2.9022, 1.5000, 2.2700) -- cycle;
\fill[blue!15.8, opacity=0.5] (2.9022, 1.5000, 2.2700) -- (2.9022, 1.5533, 2.2698) -- (2.8999, 1.5533, 2.3198) -- (2.8999, 1.5000, 2.3200) -- cycle;
\fill[blue!16.0, opacity=0.5] (2.8999, 1.5000, 2.3200) -- (2.8999, 1.5533, 2.3198) -- (2.8976, 1.5534, 2.3698) -- (2.8976, 1.5000, 2.3700) -- cycle;
\fill[blue!16.3, opacity=0.5] (2.8976, 1.5000, 2.3700) -- (2.8976, 1.5534, 2.3698) -- (2.8954, 1.5535, 2.4198) -- (2.8954, 1.5000, 2.4200) -- cycle;
\fill[blue!16.6, opacity=0.5] (2.8954, 1.5000, 2.4200) -- (2.8954, 1.5535, 2.4198) -- (2.8934, 1.5536, 2.4698) -- (2.8934, 1.5000, 2.4700) -- cycle;
\fill[blue!16.9, opacity=0.5] (2.8934, 1.5000, 2.4700) -- (2.8934, 1.5536, 2.4698) -- (2.8915, 1.5536, 2.5198) -- (2.8915, 1.5000, 2.5200) -- cycle;
\fill[blue!17.3, opacity=0.5] (2.8915, 1.5000, 2.5200) -- (2.8915, 1.5536, 2.5198) -- (2.8897, 1.5537, 2.5698) -- (2.8897, 1.5000, 2.5700) -- cycle;
\fill[blue!17.7, opacity=0.5] (2.8897, 1.5000, 2.5700) -- (2.8897, 1.5537, 2.5698) -- (2.8880, 1.5537, 2.6198) -- (2.8880, 1.5000, 2.6200) -- cycle;
\fill[blue!18.2, opacity=0.5] (2.8880, 1.5000, 2.6200) -- (2.8880, 1.5537, 2.6198) -- (2.8865, 1.5538, 2.6698) -- (2.8865, 1.5000, 2.6700) -- cycle;
\fill[blue!18.8, opacity=0.5] (2.8865, 1.5000, 2.6700) -- (2.8865, 1.5538, 2.6698) -- (2.8852, 1.5538, 2.7198) -- (2.8852, 1.5000, 2.7200) -- cycle;
\fill[blue!19.3, opacity=0.5] (2.8852, 1.5000, 2.7200) -- (2.8852, 1.5538, 2.7198) -- (2.8840, 1.5539, 2.7698) -- (2.8840, 1.5000, 2.7700) -- cycle;
\fill[blue!20.0, opacity=0.5] (2.8840, 1.5000, 2.7700) -- (2.8840, 1.5539, 2.7698) -- (2.8829, 1.5539, 2.8198) -- (2.8829, 1.5000, 2.8200) -- cycle;
\fill[blue!20.6, opacity=0.5] (2.8829, 1.5000, 2.8200) -- (2.8829, 1.5539, 2.8198) -- (2.8820, 1.5539, 2.8698) -- (2.8820, 1.5000, 2.8700) -- cycle;
\fill[blue!21.3, opacity=0.5] (2.8820, 1.5000, 2.8700) -- (2.8820, 1.5539, 2.8698) -- (2.8813, 1.5540, 2.9198) -- (2.8813, 1.5000, 2.9200) -- cycle;
\fill[blue!22.1, opacity=0.5] (2.8813, 1.5000, 2.9200) -- (2.8813, 1.5540, 2.9198) -- (2.8807, 1.5540, 2.9698) -- (2.8807, 1.5000, 2.9700) -- cycle;
\fill[blue!22.8, opacity=0.5] (2.8807, 1.5000, 2.9700) -- (2.8807, 1.5540, 2.9698) -- (2.8803, 1.5540, 3.0198) -- (2.8803, 1.5000, 3.0200) -- cycle;
\fill[blue!23.6, opacity=0.5] (2.8803, 1.5000, 3.0200) -- (2.8803, 1.5540, 3.0198) -- (2.8801, 1.5540, 3.0698) -- (2.8801, 1.5000, 3.0700) -- cycle;
\fill[blue!24.4, opacity=0.5] (2.8801, 1.5000, 3.0700) -- (2.8801, 1.5540, 3.0698) -- (2.8800, 1.5540, 3.1198) -- (2.8800, 1.5000, 3.1200) -- cycle;
\fill[blue!15.0, opacity=0.5] (3.0000, 1.5500, 0.1198) -- (3.0000, 1.6000, 0.1193) -- (2.9999, 1.6000, 0.1693) -- (2.9999, 1.5500, 0.1698) -- cycle;
\fill[blue!15.0, opacity=0.5] (2.9999, 1.5500, 0.1698) -- (2.9999, 1.6000, 0.1693) -- (2.9997, 1.6000, 0.2193) -- (2.9997, 1.5500, 0.2198) -- cycle;
\fill[blue!15.0, opacity=0.5] (2.9997, 1.5500, 0.2198) -- (2.9997, 1.6000, 0.2193) -- (2.9993, 1.6000, 0.2693) -- (2.9993, 1.5500, 0.2698) -- cycle;
\fill[blue!15.0, opacity=0.5] (2.9993, 1.5500, 0.2698) -- (2.9993, 1.6000, 0.2693) -- (2.9987, 1.6001, 0.3193) -- (2.9987, 1.5500, 0.3198) -- cycle;
\fill[blue!15.0, opacity=0.5] (2.9987, 1.5500, 0.3198) -- (2.9987, 1.6001, 0.3193) -- (2.9980, 1.6001, 0.3693) -- (2.9980, 1.5501, 0.3698) -- cycle;
\fill[blue!15.0, opacity=0.5] (2.9980, 1.5501, 0.3698) -- (2.9980, 1.6001, 0.3693) -- (2.9971, 1.6002, 0.4193) -- (2.9971, 1.5501, 0.4198) -- cycle;
\fill[blue!15.0, opacity=0.5] (2.9971, 1.5501, 0.4198) -- (2.9971, 1.6002, 0.4193) -- (2.9960, 1.6003, 0.4693) -- (2.9960, 1.5501, 0.4698) -- cycle;
\fill[blue!15.0, opacity=0.5] (2.9960, 1.5501, 0.4698) -- (2.9960, 1.6003, 0.4693) -- (2.9948, 1.6003, 0.5193) -- (2.9948, 1.5502, 0.5198) -- cycle;
\fill[blue!15.0, opacity=0.5] (2.9948, 1.5502, 0.5198) -- (2.9948, 1.6003, 0.5193) -- (2.9935, 1.6004, 0.5693) -- (2.9935, 1.5502, 0.5698) -- cycle;
\fill[blue!15.0, opacity=0.5] (2.9935, 1.5502, 0.5698) -- (2.9935, 1.6004, 0.5693) -- (2.9920, 1.6005, 0.6193) -- (2.9920, 1.5503, 0.6198) -- cycle;
\fill[blue!15.0, opacity=0.5] (2.9920, 1.5503, 0.6198) -- (2.9920, 1.6005, 0.6193) -- (2.9903, 1.6006, 0.6693) -- (2.9903, 1.5503, 0.6698) -- cycle;
\fill[blue!15.0, opacity=0.5] (2.9903, 1.5503, 0.6698) -- (2.9903, 1.6006, 0.6693) -- (2.9885, 1.6008, 0.7193) -- (2.9885, 1.5504, 0.7198) -- cycle;
\fill[blue!15.0, opacity=0.5] (2.9885, 1.5504, 0.7198) -- (2.9885, 1.6008, 0.7193) -- (2.9866, 1.6009, 0.7693) -- (2.9866, 1.5504, 0.7698) -- cycle;
\fill[blue!15.0, opacity=0.5] (2.9866, 1.5504, 0.7698) -- (2.9866, 1.6009, 0.7693) -- (2.9846, 1.6010, 0.8193) -- (2.9846, 1.5505, 0.8198) -- cycle;
\fill[blue!15.0, opacity=0.5] (2.9846, 1.5505, 0.8198) -- (2.9846, 1.6010, 0.8193) -- (2.9824, 1.6012, 0.8693) -- (2.9824, 1.5506, 0.8698) -- cycle;
\fill[blue!15.0, opacity=0.5] (2.9824, 1.5506, 0.8698) -- (2.9824, 1.6012, 0.8693) -- (2.9801, 1.6013, 0.9193) -- (2.9801, 1.5507, 0.9198) -- cycle;
\fill[blue!15.0, opacity=0.5] (2.9801, 1.5507, 0.9198) -- (2.9801, 1.6013, 0.9193) -- (2.9778, 1.6015, 0.9693) -- (2.9778, 1.5507, 0.9698) -- cycle;
\fill[blue!15.0, opacity=0.5] (2.9778, 1.5507, 0.9698) -- (2.9778, 1.6015, 0.9693) -- (2.9753, 1.6016, 1.0193) -- (2.9753, 1.5508, 1.0198) -- cycle;
\fill[blue!15.0, opacity=0.5] (2.9753, 1.5508, 1.0198) -- (2.9753, 1.6016, 1.0193) -- (2.9727, 1.6018, 1.0693) -- (2.9727, 1.5509, 1.0698) -- cycle;
\fill[blue!15.0, opacity=0.5] (2.9727, 1.5509, 1.0698) -- (2.9727, 1.6018, 1.0693) -- (2.9700, 1.6020, 1.1193) -- (2.9700, 1.5510, 1.1198) -- cycle;
\fill[blue!15.0, opacity=0.5] (2.9700, 1.5510, 1.1198) -- (2.9700, 1.6020, 1.1193) -- (2.9672, 1.6022, 1.1693) -- (2.9672, 1.5511, 1.1698) -- cycle;
\fill[blue!15.0, opacity=0.5] (2.9672, 1.5511, 1.1698) -- (2.9672, 1.6022, 1.1693) -- (2.9644, 1.6024, 1.2193) -- (2.9644, 1.5512, 1.2198) -- cycle;
\fill[blue!15.0, opacity=0.5] (2.9644, 1.5512, 1.2198) -- (2.9644, 1.6024, 1.2193) -- (2.9615, 1.6026, 1.2693) -- (2.9615, 1.5513, 1.2698) -- cycle;
\fill[blue!15.0, opacity=0.5] (2.9615, 1.5513, 1.2698) -- (2.9615, 1.6026, 1.2693) -- (2.9585, 1.6028, 1.3193) -- (2.9585, 1.5514, 1.3198) -- cycle;
\fill[blue!15.0, opacity=0.5] (2.9585, 1.5514, 1.3198) -- (2.9585, 1.6028, 1.3193) -- (2.9555, 1.6030, 1.3693) -- (2.9555, 1.5515, 1.3698) -- cycle;
\fill[blue!15.0, opacity=0.5] (2.9555, 1.5515, 1.3698) -- (2.9555, 1.6030, 1.3693) -- (2.9525, 1.6032, 1.4193) -- (2.9525, 1.5516, 1.4198) -- cycle;
\fill[blue!15.0, opacity=0.5] (2.9525, 1.5516, 1.4198) -- (2.9525, 1.6032, 1.4193) -- (2.9494, 1.6034, 1.4693) -- (2.9494, 1.5517, 1.4698) -- cycle;
\fill[blue!15.0, opacity=0.5] (2.9494, 1.5517, 1.4698) -- (2.9494, 1.6034, 1.4693) -- (2.9463, 1.6036, 1.5193) -- (2.9463, 1.5518, 1.5198) -- cycle;
\fill[blue!15.0, opacity=0.5] (2.9463, 1.5518, 1.5198) -- (2.9463, 1.6036, 1.5193) -- (2.9431, 1.6038, 1.5693) -- (2.9431, 1.5519, 1.5698) -- cycle;
\fill[blue!15.0, opacity=0.5] (2.9431, 1.5519, 1.5698) -- (2.9431, 1.6038, 1.5693) -- (2.9400, 1.6040, 1.6193) -- (2.9400, 1.5520, 1.6198) -- cycle;
\fill[blue!15.0, opacity=0.5] (2.9400, 1.5520, 1.6198) -- (2.9400, 1.6040, 1.6193) -- (2.9369, 1.6042, 1.6693) -- (2.9369, 1.5521, 1.6698) -- cycle;
\fill[blue!15.0, opacity=0.5] (2.9369, 1.5521, 1.6698) -- (2.9369, 1.6042, 1.6693) -- (2.9337, 1.6044, 1.7193) -- (2.9337, 1.5522, 1.7198) -- cycle;
\fill[blue!15.0, opacity=0.5] (2.9337, 1.5522, 1.7198) -- (2.9337, 1.6044, 1.7193) -- (2.9306, 1.6046, 1.7693) -- (2.9306, 1.5523, 1.7698) -- cycle;
\fill[blue!15.0, opacity=0.5] (2.9306, 1.5523, 1.7698) -- (2.9306, 1.6046, 1.7693) -- (2.9275, 1.6048, 1.8193) -- (2.9275, 1.5524, 1.8198) -- cycle;
\fill[blue!15.0, opacity=0.5] (2.9275, 1.5524, 1.8198) -- (2.9275, 1.6048, 1.8193) -- (2.9245, 1.6050, 1.8693) -- (2.9245, 1.5525, 1.8698) -- cycle;
\fill[blue!15.0, opacity=0.5] (2.9245, 1.5525, 1.8698) -- (2.9245, 1.6050, 1.8693) -- (2.9215, 1.6052, 1.9193) -- (2.9215, 1.5526, 1.9198) -- cycle;
\fill[blue!15.1, opacity=0.5] (2.9215, 1.5526, 1.9198) -- (2.9215, 1.6052, 1.9193) -- (2.9185, 1.6054, 1.9693) -- (2.9185, 1.5527, 1.9698) -- cycle;
\fill[blue!15.1, opacity=0.5] (2.9185, 1.5527, 1.9698) -- (2.9185, 1.6054, 1.9693) -- (2.9156, 1.6056, 2.0193) -- (2.9156, 1.5528, 2.0198) -- cycle;
\fill[blue!15.1, opacity=0.5] (2.9156, 1.5528, 2.0198) -- (2.9156, 1.6056, 2.0193) -- (2.9128, 1.6058, 2.0693) -- (2.9128, 1.5529, 2.0698) -- cycle;
\fill[blue!15.2, opacity=0.5] (2.9128, 1.5529, 2.0698) -- (2.9128, 1.6058, 2.0693) -- (2.9100, 1.6060, 2.1193) -- (2.9100, 1.5530, 2.1198) -- cycle;
\fill[blue!15.2, opacity=0.5] (2.9100, 1.5530, 2.1198) -- (2.9100, 1.6060, 2.1193) -- (2.9073, 1.6062, 2.1693) -- (2.9073, 1.5531, 2.1698) -- cycle;
\fill[blue!15.3, opacity=0.5] (2.9073, 1.5531, 2.1698) -- (2.9073, 1.6062, 2.1693) -- (2.9047, 1.6064, 2.2193) -- (2.9047, 1.5532, 2.2198) -- cycle;
\fill[blue!15.4, opacity=0.5] (2.9047, 1.5532, 2.2198) -- (2.9047, 1.6064, 2.2193) -- (2.9022, 1.6065, 2.2693) -- (2.9022, 1.5533, 2.2698) -- cycle;
\fill[blue!15.6, opacity=0.5] (2.9022, 1.5533, 2.2698) -- (2.9022, 1.6065, 2.2693) -- (2.8999, 1.6067, 2.3193) -- (2.8999, 1.5533, 2.3198) -- cycle;
\fill[blue!15.7, opacity=0.5] (2.8999, 1.5533, 2.3198) -- (2.8999, 1.6067, 2.3193) -- (2.8976, 1.6068, 2.3693) -- (2.8976, 1.5534, 2.3698) -- cycle;
\fill[blue!15.9, opacity=0.5] (2.8976, 1.5534, 2.3698) -- (2.8976, 1.6068, 2.3693) -- (2.8954, 1.6070, 2.4193) -- (2.8954, 1.5535, 2.4198) -- cycle;
\fill[blue!16.2, opacity=0.5] (2.8954, 1.5535, 2.4198) -- (2.8954, 1.6070, 2.4193) -- (2.8934, 1.6071, 2.4693) -- (2.8934, 1.5536, 2.4698) -- cycle;
\fill[blue!16.4, opacity=0.5] (2.8934, 1.5536, 2.4698) -- (2.8934, 1.6071, 2.4693) -- (2.8915, 1.6072, 2.5193) -- (2.8915, 1.5536, 2.5198) -- cycle;
\fill[blue!16.7, opacity=0.5] (2.8915, 1.5536, 2.5198) -- (2.8915, 1.6072, 2.5193) -- (2.8897, 1.6074, 2.5693) -- (2.8897, 1.5537, 2.5698) -- cycle;
\fill[blue!17.1, opacity=0.5] (2.8897, 1.5537, 2.5698) -- (2.8897, 1.6074, 2.5693) -- (2.8880, 1.6075, 2.6193) -- (2.8880, 1.5537, 2.6198) -- cycle;
\fill[blue!17.5, opacity=0.5] (2.8880, 1.5537, 2.6198) -- (2.8880, 1.6075, 2.6193) -- (2.8865, 1.6076, 2.6693) -- (2.8865, 1.5538, 2.6698) -- cycle;
\fill[blue!17.9, opacity=0.5] (2.8865, 1.5538, 2.6698) -- (2.8865, 1.6076, 2.6693) -- (2.8852, 1.6077, 2.7193) -- (2.8852, 1.5538, 2.7198) -- cycle;
\fill[blue!18.4, opacity=0.5] (2.8852, 1.5538, 2.7198) -- (2.8852, 1.6077, 2.7193) -- (2.8840, 1.6077, 2.7693) -- (2.8840, 1.5539, 2.7698) -- cycle;
\fill[blue!19.0, opacity=0.5] (2.8840, 1.5539, 2.7698) -- (2.8840, 1.6077, 2.7693) -- (2.8829, 1.6078, 2.8193) -- (2.8829, 1.5539, 2.8198) -- cycle;
\fill[blue!19.5, opacity=0.5] (2.8829, 1.5539, 2.8198) -- (2.8829, 1.6078, 2.8193) -- (2.8820, 1.6079, 2.8693) -- (2.8820, 1.5539, 2.8698) -- cycle;
\fill[blue!20.1, opacity=0.5] (2.8820, 1.5539, 2.8698) -- (2.8820, 1.6079, 2.8693) -- (2.8813, 1.6079, 2.9193) -- (2.8813, 1.5540, 2.9198) -- cycle;
\fill[blue!20.8, opacity=0.5] (2.8813, 1.5540, 2.9198) -- (2.8813, 1.6079, 2.9193) -- (2.8807, 1.6080, 2.9693) -- (2.8807, 1.5540, 2.9698) -- cycle;
\fill[blue!21.5, opacity=0.5] (2.8807, 1.5540, 2.9698) -- (2.8807, 1.6080, 2.9693) -- (2.8803, 1.6080, 3.0193) -- (2.8803, 1.5540, 3.0198) -- cycle;
\fill[blue!22.2, opacity=0.5] (2.8803, 1.5540, 3.0198) -- (2.8803, 1.6080, 3.0193) -- (2.8801, 1.6080, 3.0693) -- (2.8801, 1.5540, 3.0698) -- cycle;
\fill[blue!22.9, opacity=0.5] (2.8801, 1.5540, 3.0698) -- (2.8801, 1.6080, 3.0693) -- (2.8800, 1.6080, 3.1193) -- (2.8800, 1.5540, 3.1198) -- cycle;
\fill[blue!15.0, opacity=0.5] (3.0000, 1.6000, 0.1193) -- (3.0000, 1.6500, 0.1185) -- (2.9999, 1.6500, 0.1685) -- (2.9999, 1.6000, 0.1693) -- cycle;
\fill[blue!15.0, opacity=0.5] (2.9999, 1.6000, 0.1693) -- (2.9999, 1.6500, 0.1685) -- (2.9997, 1.6500, 0.2185) -- (2.9997, 1.6000, 0.2193) -- cycle;
\fill[blue!15.0, opacity=0.5] (2.9997, 1.6000, 0.2193) -- (2.9997, 1.6500, 0.2185) -- (2.9993, 1.6501, 0.2685) -- (2.9993, 1.6000, 0.2693) -- cycle;
\fill[blue!15.0, opacity=0.5] (2.9993, 1.6000, 0.2693) -- (2.9993, 1.6501, 0.2685) -- (2.9987, 1.6501, 0.3185) -- (2.9987, 1.6001, 0.3193) -- cycle;
\fill[blue!15.0, opacity=0.5] (2.9987, 1.6001, 0.3193) -- (2.9987, 1.6501, 0.3185) -- (2.9980, 1.6502, 0.3685) -- (2.9980, 1.6001, 0.3693) -- cycle;
\fill[blue!15.0, opacity=0.5] (2.9980, 1.6001, 0.3693) -- (2.9980, 1.6502, 0.3685) -- (2.9971, 1.6503, 0.4185) -- (2.9971, 1.6002, 0.4193) -- cycle;
\fill[blue!15.0, opacity=0.5] (2.9971, 1.6002, 0.4193) -- (2.9971, 1.6503, 0.4185) -- (2.9960, 1.6504, 0.4685) -- (2.9960, 1.6003, 0.4693) -- cycle;
\fill[blue!15.0, opacity=0.5] (2.9960, 1.6003, 0.4693) -- (2.9960, 1.6504, 0.4685) -- (2.9948, 1.6505, 0.5185) -- (2.9948, 1.6003, 0.5193) -- cycle;
\fill[blue!15.0, opacity=0.5] (2.9948, 1.6003, 0.5193) -- (2.9948, 1.6505, 0.5185) -- (2.9935, 1.6507, 0.5685) -- (2.9935, 1.6004, 0.5693) -- cycle;
\fill[blue!15.0, opacity=0.5] (2.9935, 1.6004, 0.5693) -- (2.9935, 1.6507, 0.5685) -- (2.9920, 1.6508, 0.6185) -- (2.9920, 1.6005, 0.6193) -- cycle;
\fill[blue!15.0, opacity=0.5] (2.9920, 1.6005, 0.6193) -- (2.9920, 1.6508, 0.6185) -- (2.9903, 1.6510, 0.6685) -- (2.9903, 1.6006, 0.6693) -- cycle;
\fill[blue!15.0, opacity=0.5] (2.9903, 1.6006, 0.6693) -- (2.9903, 1.6510, 0.6685) -- (2.9885, 1.6511, 0.7185) -- (2.9885, 1.6008, 0.7193) -- cycle;
\fill[blue!15.0, opacity=0.5] (2.9885, 1.6008, 0.7193) -- (2.9885, 1.6511, 0.7185) -- (2.9866, 1.6513, 0.7685) -- (2.9866, 1.6009, 0.7693) -- cycle;
\fill[blue!15.0, opacity=0.5] (2.9866, 1.6009, 0.7693) -- (2.9866, 1.6513, 0.7685) -- (2.9846, 1.6515, 0.8185) -- (2.9846, 1.6010, 0.8193) -- cycle;
\fill[blue!15.0, opacity=0.5] (2.9846, 1.6010, 0.8193) -- (2.9846, 1.6515, 0.8185) -- (2.9824, 1.6518, 0.8685) -- (2.9824, 1.6012, 0.8693) -- cycle;
\fill[blue!15.0, opacity=0.5] (2.9824, 1.6012, 0.8693) -- (2.9824, 1.6518, 0.8685) -- (2.9801, 1.6520, 0.9185) -- (2.9801, 1.6013, 0.9193) -- cycle;
\fill[blue!15.0, opacity=0.5] (2.9801, 1.6013, 0.9193) -- (2.9801, 1.6520, 0.9185) -- (2.9778, 1.6522, 0.9685) -- (2.9778, 1.6015, 0.9693) -- cycle;
\fill[blue!15.0, opacity=0.5] (2.9778, 1.6015, 0.9693) -- (2.9778, 1.6522, 0.9685) -- (2.9753, 1.6525, 1.0185) -- (2.9753, 1.6016, 1.0193) -- cycle;
\fill[blue!15.0, opacity=0.5] (2.9753, 1.6016, 1.0193) -- (2.9753, 1.6525, 1.0185) -- (2.9727, 1.6527, 1.0685) -- (2.9727, 1.6018, 1.0693) -- cycle;
\fill[blue!15.0, opacity=0.5] (2.9727, 1.6018, 1.0693) -- (2.9727, 1.6527, 1.0685) -- (2.9700, 1.6530, 1.1185) -- (2.9700, 1.6020, 1.1193) -- cycle;
\fill[blue!15.0, opacity=0.5] (2.9700, 1.6020, 1.1193) -- (2.9700, 1.6530, 1.1185) -- (2.9672, 1.6533, 1.1685) -- (2.9672, 1.6022, 1.1693) -- cycle;
\fill[blue!15.0, opacity=0.5] (2.9672, 1.6022, 1.1693) -- (2.9672, 1.6533, 1.1685) -- (2.9644, 1.6536, 1.2185) -- (2.9644, 1.6024, 1.2193) -- cycle;
\fill[blue!15.0, opacity=0.5] (2.9644, 1.6024, 1.2193) -- (2.9644, 1.6536, 1.2185) -- (2.9615, 1.6538, 1.2685) -- (2.9615, 1.6026, 1.2693) -- cycle;
\fill[blue!15.0, opacity=0.5] (2.9615, 1.6026, 1.2693) -- (2.9615, 1.6538, 1.2685) -- (2.9585, 1.6541, 1.3185) -- (2.9585, 1.6028, 1.3193) -- cycle;
\fill[blue!15.0, opacity=0.5] (2.9585, 1.6028, 1.3193) -- (2.9585, 1.6541, 1.3185) -- (2.9555, 1.6544, 1.3685) -- (2.9555, 1.6030, 1.3693) -- cycle;
\fill[blue!15.0, opacity=0.5] (2.9555, 1.6030, 1.3693) -- (2.9555, 1.6544, 1.3685) -- (2.9525, 1.6548, 1.4185) -- (2.9525, 1.6032, 1.4193) -- cycle;
\fill[blue!15.0, opacity=0.5] (2.9525, 1.6032, 1.4193) -- (2.9525, 1.6548, 1.4185) -- (2.9494, 1.6551, 1.4685) -- (2.9494, 1.6034, 1.4693) -- cycle;
\fill[blue!15.0, opacity=0.5] (2.9494, 1.6034, 1.4693) -- (2.9494, 1.6551, 1.4685) -- (2.9463, 1.6554, 1.5185) -- (2.9463, 1.6036, 1.5193) -- cycle;
\fill[blue!15.0, opacity=0.5] (2.9463, 1.6036, 1.5193) -- (2.9463, 1.6554, 1.5185) -- (2.9431, 1.6557, 1.5685) -- (2.9431, 1.6038, 1.5693) -- cycle;
\fill[blue!15.0, opacity=0.5] (2.9431, 1.6038, 1.5693) -- (2.9431, 1.6557, 1.5685) -- (2.9400, 1.6560, 1.6185) -- (2.9400, 1.6040, 1.6193) -- cycle;
\fill[blue!15.0, opacity=0.5] (2.9400, 1.6040, 1.6193) -- (2.9400, 1.6560, 1.6185) -- (2.9369, 1.6563, 1.6685) -- (2.9369, 1.6042, 1.6693) -- cycle;
\fill[blue!15.0, opacity=0.5] (2.9369, 1.6042, 1.6693) -- (2.9369, 1.6563, 1.6685) -- (2.9337, 1.6566, 1.7185) -- (2.9337, 1.6044, 1.7193) -- cycle;
\fill[blue!15.0, opacity=0.5] (2.9337, 1.6044, 1.7193) -- (2.9337, 1.6566, 1.7185) -- (2.9306, 1.6569, 1.7685) -- (2.9306, 1.6046, 1.7693) -- cycle;
\fill[blue!15.0, opacity=0.5] (2.9306, 1.6046, 1.7693) -- (2.9306, 1.6569, 1.7685) -- (2.9275, 1.6572, 1.8185) -- (2.9275, 1.6048, 1.8193) -- cycle;
\fill[blue!15.0, opacity=0.5] (2.9275, 1.6048, 1.8193) -- (2.9275, 1.6572, 1.8185) -- (2.9245, 1.6576, 1.8685) -- (2.9245, 1.6050, 1.8693) -- cycle;
\fill[blue!15.0, opacity=0.5] (2.9245, 1.6050, 1.8693) -- (2.9245, 1.6576, 1.8685) -- (2.9215, 1.6579, 1.9185) -- (2.9215, 1.6052, 1.9193) -- cycle;
\fill[blue!15.0, opacity=0.5] (2.9215, 1.6052, 1.9193) -- (2.9215, 1.6579, 1.9185) -- (2.9185, 1.6582, 1.9685) -- (2.9185, 1.6054, 1.9693) -- cycle;
\fill[blue!15.0, opacity=0.5] (2.9185, 1.6054, 1.9693) -- (2.9185, 1.6582, 1.9685) -- (2.9156, 1.6584, 2.0185) -- (2.9156, 1.6056, 2.0193) -- cycle;
\fill[blue!15.1, opacity=0.5] (2.9156, 1.6056, 2.0193) -- (2.9156, 1.6584, 2.0185) -- (2.9128, 1.6587, 2.0685) -- (2.9128, 1.6058, 2.0693) -- cycle;
\fill[blue!15.1, opacity=0.5] (2.9128, 1.6058, 2.0693) -- (2.9128, 1.6587, 2.0685) -- (2.9100, 1.6590, 2.1185) -- (2.9100, 1.6060, 2.1193) -- cycle;
\fill[blue!15.1, opacity=0.5] (2.9100, 1.6060, 2.1193) -- (2.9100, 1.6590, 2.1185) -- (2.9073, 1.6593, 2.1685) -- (2.9073, 1.6062, 2.1693) -- cycle;
\fill[blue!15.2, opacity=0.5] (2.9073, 1.6062, 2.1693) -- (2.9073, 1.6593, 2.1685) -- (2.9047, 1.6595, 2.2185) -- (2.9047, 1.6064, 2.2193) -- cycle;
\fill[blue!15.3, opacity=0.5] (2.9047, 1.6064, 2.2193) -- (2.9047, 1.6595, 2.2185) -- (2.9022, 1.6598, 2.2685) -- (2.9022, 1.6065, 2.2693) -- cycle;
\fill[blue!15.4, opacity=0.5] (2.9022, 1.6065, 2.2693) -- (2.9022, 1.6598, 2.2685) -- (2.8999, 1.6600, 2.3185) -- (2.8999, 1.6067, 2.3193) -- cycle;
\fill[blue!15.5, opacity=0.5] (2.8999, 1.6067, 2.3193) -- (2.8999, 1.6600, 2.3185) -- (2.8976, 1.6602, 2.3685) -- (2.8976, 1.6068, 2.3693) -- cycle;
\fill[blue!15.6, opacity=0.5] (2.8976, 1.6068, 2.3693) -- (2.8976, 1.6602, 2.3685) -- (2.8954, 1.6605, 2.4185) -- (2.8954, 1.6070, 2.4193) -- cycle;
\fill[blue!15.8, opacity=0.5] (2.8954, 1.6070, 2.4193) -- (2.8954, 1.6605, 2.4185) -- (2.8934, 1.6607, 2.4685) -- (2.8934, 1.6071, 2.4693) -- cycle;
\fill[blue!16.0, opacity=0.5] (2.8934, 1.6071, 2.4693) -- (2.8934, 1.6607, 2.4685) -- (2.8915, 1.6609, 2.5185) -- (2.8915, 1.6072, 2.5193) -- cycle;
\fill[blue!16.2, opacity=0.5] (2.8915, 1.6072, 2.5193) -- (2.8915, 1.6609, 2.5185) -- (2.8897, 1.6610, 2.5685) -- (2.8897, 1.6074, 2.5693) -- cycle;
\fill[blue!16.5, opacity=0.5] (2.8897, 1.6074, 2.5693) -- (2.8897, 1.6610, 2.5685) -- (2.8880, 1.6612, 2.6185) -- (2.8880, 1.6075, 2.6193) -- cycle;
\fill[blue!16.8, opacity=0.5] (2.8880, 1.6075, 2.6193) -- (2.8880, 1.6612, 2.6185) -- (2.8865, 1.6613, 2.6685) -- (2.8865, 1.6076, 2.6693) -- cycle;
\fill[blue!17.1, opacity=0.5] (2.8865, 1.6076, 2.6693) -- (2.8865, 1.6613, 2.6685) -- (2.8852, 1.6615, 2.7185) -- (2.8852, 1.6077, 2.7193) -- cycle;
\fill[blue!17.5, opacity=0.5] (2.8852, 1.6077, 2.7193) -- (2.8852, 1.6615, 2.7185) -- (2.8840, 1.6616, 2.7685) -- (2.8840, 1.6077, 2.7693) -- cycle;
\fill[blue!18.0, opacity=0.5] (2.8840, 1.6077, 2.7693) -- (2.8840, 1.6616, 2.7685) -- (2.8829, 1.6617, 2.8185) -- (2.8829, 1.6078, 2.8193) -- cycle;
\fill[blue!18.4, opacity=0.5] (2.8829, 1.6078, 2.8193) -- (2.8829, 1.6617, 2.8185) -- (2.8820, 1.6618, 2.8685) -- (2.8820, 1.6079, 2.8693) -- cycle;
\fill[blue!18.9, opacity=0.5] (2.8820, 1.6079, 2.8693) -- (2.8820, 1.6618, 2.8685) -- (2.8813, 1.6619, 2.9185) -- (2.8813, 1.6079, 2.9193) -- cycle;
\fill[blue!19.5, opacity=0.5] (2.8813, 1.6079, 2.9193) -- (2.8813, 1.6619, 2.9185) -- (2.8807, 1.6619, 2.9685) -- (2.8807, 1.6080, 2.9693) -- cycle;
\fill[blue!20.1, opacity=0.5] (2.8807, 1.6080, 2.9693) -- (2.8807, 1.6619, 2.9685) -- (2.8803, 1.6620, 3.0185) -- (2.8803, 1.6080, 3.0193) -- cycle;
\fill[blue!20.7, opacity=0.5] (2.8803, 1.6080, 3.0193) -- (2.8803, 1.6620, 3.0185) -- (2.8801, 1.6620, 3.0685) -- (2.8801, 1.6080, 3.0693) -- cycle;
\fill[blue!21.3, opacity=0.5] (2.8801, 1.6080, 3.0693) -- (2.8801, 1.6620, 3.0685) -- (2.8800, 1.6620, 3.1185) -- (2.8800, 1.6080, 3.1193) -- cycle;
\fill[blue!15.0, opacity=0.5] (3.0000, 1.6500, 0.1185) -- (3.0000, 1.7000, 0.1174) -- (2.9999, 1.7000, 0.1674) -- (2.9999, 1.6500, 0.1685) -- cycle;
\fill[blue!15.0, opacity=0.5] (2.9999, 1.6500, 0.1685) -- (2.9999, 1.7000, 0.1674) -- (2.9997, 1.7000, 0.2174) -- (2.9997, 1.6500, 0.2185) -- cycle;
\fill[blue!15.0, opacity=0.5] (2.9997, 1.6500, 0.2185) -- (2.9997, 1.7000, 0.2174) -- (2.9993, 1.7001, 0.2674) -- (2.9993, 1.6501, 0.2685) -- cycle;
\fill[blue!15.0, opacity=0.5] (2.9993, 1.6501, 0.2685) -- (2.9993, 1.7001, 0.2674) -- (2.9987, 1.7002, 0.3174) -- (2.9987, 1.6501, 0.3185) -- cycle;
\fill[blue!15.0, opacity=0.5] (2.9987, 1.6501, 0.3185) -- (2.9987, 1.7002, 0.3174) -- (2.9980, 1.7003, 0.3674) -- (2.9980, 1.6502, 0.3685) -- cycle;
\fill[blue!15.0, opacity=0.5] (2.9980, 1.6502, 0.3685) -- (2.9980, 1.7003, 0.3674) -- (2.9971, 1.7004, 0.4174) -- (2.9971, 1.6503, 0.4185) -- cycle;
\fill[blue!15.0, opacity=0.5] (2.9971, 1.6503, 0.4185) -- (2.9971, 1.7004, 0.4174) -- (2.9960, 1.7005, 0.4674) -- (2.9960, 1.6504, 0.4685) -- cycle;
\fill[blue!15.0, opacity=0.5] (2.9960, 1.6504, 0.4685) -- (2.9960, 1.7005, 0.4674) -- (2.9948, 1.7007, 0.5174) -- (2.9948, 1.6505, 0.5185) -- cycle;
\fill[blue!15.0, opacity=0.5] (2.9948, 1.6505, 0.5185) -- (2.9948, 1.7007, 0.5174) -- (2.9935, 1.7009, 0.5674) -- (2.9935, 1.6507, 0.5685) -- cycle;
\fill[blue!15.0, opacity=0.5] (2.9935, 1.6507, 0.5685) -- (2.9935, 1.7009, 0.5674) -- (2.9920, 1.7011, 0.6174) -- (2.9920, 1.6508, 0.6185) -- cycle;
\fill[blue!15.0, opacity=0.5] (2.9920, 1.6508, 0.6185) -- (2.9920, 1.7011, 0.6174) -- (2.9903, 1.7013, 0.6674) -- (2.9903, 1.6510, 0.6685) -- cycle;
\fill[blue!15.0, opacity=0.5] (2.9903, 1.6510, 0.6685) -- (2.9903, 1.7013, 0.6674) -- (2.9885, 1.7015, 0.7174) -- (2.9885, 1.6511, 0.7185) -- cycle;
\fill[blue!15.0, opacity=0.5] (2.9885, 1.6511, 0.7185) -- (2.9885, 1.7015, 0.7174) -- (2.9866, 1.7018, 0.7674) -- (2.9866, 1.6513, 0.7685) -- cycle;
\fill[blue!15.0, opacity=0.5] (2.9866, 1.6513, 0.7685) -- (2.9866, 1.7018, 0.7674) -- (2.9846, 1.7021, 0.8174) -- (2.9846, 1.6515, 0.8185) -- cycle;
\fill[blue!15.0, opacity=0.5] (2.9846, 1.6515, 0.8185) -- (2.9846, 1.7021, 0.8174) -- (2.9824, 1.7023, 0.8674) -- (2.9824, 1.6518, 0.8685) -- cycle;
\fill[blue!15.0, opacity=0.5] (2.9824, 1.6518, 0.8685) -- (2.9824, 1.7023, 0.8674) -- (2.9801, 1.7026, 0.9174) -- (2.9801, 1.6520, 0.9185) -- cycle;
\fill[blue!15.0, opacity=0.5] (2.9801, 1.6520, 0.9185) -- (2.9801, 1.7026, 0.9174) -- (2.9778, 1.7030, 0.9674) -- (2.9778, 1.6522, 0.9685) -- cycle;
\fill[blue!15.0, opacity=0.5] (2.9778, 1.6522, 0.9685) -- (2.9778, 1.7030, 0.9674) -- (2.9753, 1.7033, 1.0174) -- (2.9753, 1.6525, 1.0185) -- cycle;
\fill[blue!15.0, opacity=0.5] (2.9753, 1.6525, 1.0185) -- (2.9753, 1.7033, 1.0174) -- (2.9727, 1.7036, 1.0674) -- (2.9727, 1.6527, 1.0685) -- cycle;
\fill[blue!15.0, opacity=0.5] (2.9727, 1.6527, 1.0685) -- (2.9727, 1.7036, 1.0674) -- (2.9700, 1.7040, 1.1174) -- (2.9700, 1.6530, 1.1185) -- cycle;
\fill[blue!15.0, opacity=0.5] (2.9700, 1.6530, 1.1185) -- (2.9700, 1.7040, 1.1174) -- (2.9672, 1.7044, 1.1674) -- (2.9672, 1.6533, 1.1685) -- cycle;
\fill[blue!15.0, opacity=0.5] (2.9672, 1.6533, 1.1685) -- (2.9672, 1.7044, 1.1674) -- (2.9644, 1.7047, 1.2174) -- (2.9644, 1.6536, 1.2185) -- cycle;
\fill[blue!15.0, opacity=0.5] (2.9644, 1.6536, 1.2185) -- (2.9644, 1.7047, 1.2174) -- (2.9615, 1.7051, 1.2674) -- (2.9615, 1.6538, 1.2685) -- cycle;
\fill[blue!15.0, opacity=0.5] (2.9615, 1.6538, 1.2685) -- (2.9615, 1.7051, 1.2674) -- (2.9585, 1.7055, 1.3174) -- (2.9585, 1.6541, 1.3185) -- cycle;
\fill[blue!15.0, opacity=0.5] (2.9585, 1.6541, 1.3185) -- (2.9585, 1.7055, 1.3174) -- (2.9555, 1.7059, 1.3674) -- (2.9555, 1.6544, 1.3685) -- cycle;
\fill[blue!15.0, opacity=0.5] (2.9555, 1.6544, 1.3685) -- (2.9555, 1.7059, 1.3674) -- (2.9525, 1.7063, 1.4174) -- (2.9525, 1.6548, 1.4185) -- cycle;
\fill[blue!15.0, opacity=0.5] (2.9525, 1.6548, 1.4185) -- (2.9525, 1.7063, 1.4174) -- (2.9494, 1.7067, 1.4674) -- (2.9494, 1.6551, 1.4685) -- cycle;
\fill[blue!15.0, opacity=0.5] (2.9494, 1.6551, 1.4685) -- (2.9494, 1.7067, 1.4674) -- (2.9463, 1.7072, 1.5174) -- (2.9463, 1.6554, 1.5185) -- cycle;
\fill[blue!15.0, opacity=0.5] (2.9463, 1.6554, 1.5185) -- (2.9463, 1.7072, 1.5174) -- (2.9431, 1.7076, 1.5674) -- (2.9431, 1.6557, 1.5685) -- cycle;
\fill[blue!15.0, opacity=0.5] (2.9431, 1.6557, 1.5685) -- (2.9431, 1.7076, 1.5674) -- (2.9400, 1.7080, 1.6174) -- (2.9400, 1.6560, 1.6185) -- cycle;
\fill[blue!15.0, opacity=0.5] (2.9400, 1.6560, 1.6185) -- (2.9400, 1.7080, 1.6174) -- (2.9369, 1.7084, 1.6674) -- (2.9369, 1.6563, 1.6685) -- cycle;
\fill[blue!15.0, opacity=0.5] (2.9369, 1.6563, 1.6685) -- (2.9369, 1.7084, 1.6674) -- (2.9337, 1.7088, 1.7174) -- (2.9337, 1.6566, 1.7185) -- cycle;
\fill[blue!15.0, opacity=0.5] (2.9337, 1.6566, 1.7185) -- (2.9337, 1.7088, 1.7174) -- (2.9306, 1.7093, 1.7674) -- (2.9306, 1.6569, 1.7685) -- cycle;
\fill[blue!15.0, opacity=0.5] (2.9306, 1.6569, 1.7685) -- (2.9306, 1.7093, 1.7674) -- (2.9275, 1.7097, 1.8174) -- (2.9275, 1.6572, 1.8185) -- cycle;
\fill[blue!15.0, opacity=0.5] (2.9275, 1.6572, 1.8185) -- (2.9275, 1.7097, 1.8174) -- (2.9245, 1.7101, 1.8674) -- (2.9245, 1.6576, 1.8685) -- cycle;
\fill[blue!15.0, opacity=0.5] (2.9245, 1.6576, 1.8685) -- (2.9245, 1.7101, 1.8674) -- (2.9215, 1.7105, 1.9174) -- (2.9215, 1.6579, 1.9185) -- cycle;
\fill[blue!15.0, opacity=0.5] (2.9215, 1.6579, 1.9185) -- (2.9215, 1.7105, 1.9174) -- (2.9185, 1.7109, 1.9674) -- (2.9185, 1.6582, 1.9685) -- cycle;
\fill[blue!15.0, opacity=0.5] (2.9185, 1.6582, 1.9685) -- (2.9185, 1.7109, 1.9674) -- (2.9156, 1.7113, 2.0174) -- (2.9156, 1.6584, 2.0185) -- cycle;
\fill[blue!15.0, opacity=0.5] (2.9156, 1.6584, 2.0185) -- (2.9156, 1.7113, 2.0174) -- (2.9128, 1.7116, 2.0674) -- (2.9128, 1.6587, 2.0685) -- cycle;
\fill[blue!15.1, opacity=0.5] (2.9128, 1.6587, 2.0685) -- (2.9128, 1.7116, 2.0674) -- (2.9100, 1.7120, 2.1174) -- (2.9100, 1.6590, 2.1185) -- cycle;
\fill[blue!15.1, opacity=0.5] (2.9100, 1.6590, 2.1185) -- (2.9100, 1.7120, 2.1174) -- (2.9073, 1.7124, 2.1674) -- (2.9073, 1.6593, 2.1685) -- cycle;
\fill[blue!15.1, opacity=0.5] (2.9073, 1.6593, 2.1685) -- (2.9073, 1.7124, 2.1674) -- (2.9047, 1.7127, 2.2174) -- (2.9047, 1.6595, 2.2185) -- cycle;
\fill[blue!15.2, opacity=0.5] (2.9047, 1.6595, 2.2185) -- (2.9047, 1.7127, 2.2174) -- (2.9022, 1.7130, 2.2674) -- (2.9022, 1.6598, 2.2685) -- cycle;
\fill[blue!15.2, opacity=0.5] (2.9022, 1.6598, 2.2685) -- (2.9022, 1.7130, 2.2674) -- (2.8999, 1.7134, 2.3174) -- (2.8999, 1.6600, 2.3185) -- cycle;
\fill[blue!15.3, opacity=0.5] (2.8999, 1.6600, 2.3185) -- (2.8999, 1.7134, 2.3174) -- (2.8976, 1.7137, 2.3674) -- (2.8976, 1.6602, 2.3685) -- cycle;
\fill[blue!15.4, opacity=0.5] (2.8976, 1.6602, 2.3685) -- (2.8976, 1.7137, 2.3674) -- (2.8954, 1.7139, 2.4174) -- (2.8954, 1.6605, 2.4185) -- cycle;
\fill[blue!15.5, opacity=0.5] (2.8954, 1.6605, 2.4185) -- (2.8954, 1.7139, 2.4174) -- (2.8934, 1.7142, 2.4674) -- (2.8934, 1.6607, 2.4685) -- cycle;
\fill[blue!15.6, opacity=0.5] (2.8934, 1.6607, 2.4685) -- (2.8934, 1.7142, 2.4674) -- (2.8915, 1.7145, 2.5174) -- (2.8915, 1.6609, 2.5185) -- cycle;
\fill[blue!15.8, opacity=0.5] (2.8915, 1.6609, 2.5185) -- (2.8915, 1.7145, 2.5174) -- (2.8897, 1.7147, 2.5674) -- (2.8897, 1.6610, 2.5685) -- cycle;
\fill[blue!16.0, opacity=0.5] (2.8897, 1.6610, 2.5685) -- (2.8897, 1.7147, 2.5674) -- (2.8880, 1.7149, 2.6174) -- (2.8880, 1.6612, 2.6185) -- cycle;
\fill[blue!16.2, opacity=0.5] (2.8880, 1.6612, 2.6185) -- (2.8880, 1.7149, 2.6174) -- (2.8865, 1.7151, 2.6674) -- (2.8865, 1.6613, 2.6685) -- cycle;
\fill[blue!16.5, opacity=0.5] (2.8865, 1.6613, 2.6685) -- (2.8865, 1.7151, 2.6674) -- (2.8852, 1.7153, 2.7174) -- (2.8852, 1.6615, 2.7185) -- cycle;
\fill[blue!16.8, opacity=0.5] (2.8852, 1.6615, 2.7185) -- (2.8852, 1.7153, 2.7174) -- (2.8840, 1.7155, 2.7674) -- (2.8840, 1.6616, 2.7685) -- cycle;
\fill[blue!17.1, opacity=0.5] (2.8840, 1.6616, 2.7685) -- (2.8840, 1.7155, 2.7674) -- (2.8829, 1.7156, 2.8174) -- (2.8829, 1.6617, 2.8185) -- cycle;
\fill[blue!17.5, opacity=0.5] (2.8829, 1.6617, 2.8185) -- (2.8829, 1.7156, 2.8174) -- (2.8820, 1.7157, 2.8674) -- (2.8820, 1.6618, 2.8685) -- cycle;
\fill[blue!17.9, opacity=0.5] (2.8820, 1.6618, 2.8685) -- (2.8820, 1.7157, 2.8674) -- (2.8813, 1.7158, 2.9174) -- (2.8813, 1.6619, 2.9185) -- cycle;
\fill[blue!18.3, opacity=0.5] (2.8813, 1.6619, 2.9185) -- (2.8813, 1.7158, 2.9174) -- (2.8807, 1.7159, 2.9674) -- (2.8807, 1.6619, 2.9685) -- cycle;
\fill[blue!18.8, opacity=0.5] (2.8807, 1.6619, 2.9685) -- (2.8807, 1.7159, 2.9674) -- (2.8803, 1.7160, 3.0174) -- (2.8803, 1.6620, 3.0185) -- cycle;
\fill[blue!19.3, opacity=0.5] (2.8803, 1.6620, 3.0185) -- (2.8803, 1.7160, 3.0174) -- (2.8801, 1.7160, 3.0674) -- (2.8801, 1.6620, 3.0685) -- cycle;
\fill[blue!19.8, opacity=0.5] (2.8801, 1.6620, 3.0685) -- (2.8801, 1.7160, 3.0674) -- (2.8800, 1.7160, 3.1174) -- (2.8800, 1.6620, 3.1185) -- cycle;
\fill[blue!15.0, opacity=0.5] (3.0000, 1.7000, 0.1174) -- (3.0000, 1.7500, 0.1159) -- (2.9999, 1.7500, 0.1659) -- (2.9999, 1.7000, 0.1674) -- cycle;
\fill[blue!15.0, opacity=0.5] (2.9999, 1.7000, 0.1674) -- (2.9999, 1.7500, 0.1659) -- (2.9997, 1.7501, 0.2159) -- (2.9997, 1.7000, 0.2174) -- cycle;
\fill[blue!15.0, opacity=0.5] (2.9997, 1.7000, 0.2174) -- (2.9997, 1.7501, 0.2159) -- (2.9993, 1.7501, 0.2659) -- (2.9993, 1.7001, 0.2674) -- cycle;
\fill[blue!15.0, opacity=0.5] (2.9993, 1.7001, 0.2674) -- (2.9993, 1.7501, 0.2659) -- (2.9987, 1.7502, 0.3159) -- (2.9987, 1.7002, 0.3174) -- cycle;
\fill[blue!15.0, opacity=0.5] (2.9987, 1.7002, 0.3174) -- (2.9987, 1.7502, 0.3159) -- (2.9980, 1.7503, 0.3659) -- (2.9980, 1.7003, 0.3674) -- cycle;
\fill[blue!15.0, opacity=0.5] (2.9980, 1.7003, 0.3674) -- (2.9980, 1.7503, 0.3659) -- (2.9971, 1.7505, 0.4159) -- (2.9971, 1.7004, 0.4174) -- cycle;
\fill[blue!15.0, opacity=0.5] (2.9971, 1.7004, 0.4174) -- (2.9971, 1.7505, 0.4159) -- (2.9960, 1.7507, 0.4659) -- (2.9960, 1.7005, 0.4674) -- cycle;
\fill[blue!15.0, opacity=0.5] (2.9960, 1.7005, 0.4674) -- (2.9960, 1.7507, 0.4659) -- (2.9948, 1.7509, 0.5159) -- (2.9948, 1.7007, 0.5174) -- cycle;
\fill[blue!15.0, opacity=0.5] (2.9948, 1.7007, 0.5174) -- (2.9948, 1.7509, 0.5159) -- (2.9935, 1.7511, 0.5659) -- (2.9935, 1.7009, 0.5674) -- cycle;
\fill[blue!15.0, opacity=0.5] (2.9935, 1.7009, 0.5674) -- (2.9935, 1.7511, 0.5659) -- (2.9920, 1.7513, 0.6159) -- (2.9920, 1.7011, 0.6174) -- cycle;
\fill[blue!15.0, opacity=0.5] (2.9920, 1.7011, 0.6174) -- (2.9920, 1.7513, 0.6159) -- (2.9903, 1.7516, 0.6659) -- (2.9903, 1.7013, 0.6674) -- cycle;
\fill[blue!15.0, opacity=0.5] (2.9903, 1.7013, 0.6674) -- (2.9903, 1.7516, 0.6659) -- (2.9885, 1.7519, 0.7159) -- (2.9885, 1.7015, 0.7174) -- cycle;
\fill[blue!15.0, opacity=0.5] (2.9885, 1.7015, 0.7174) -- (2.9885, 1.7519, 0.7159) -- (2.9866, 1.7522, 0.7659) -- (2.9866, 1.7018, 0.7674) -- cycle;
\fill[blue!15.0, opacity=0.5] (2.9866, 1.7018, 0.7674) -- (2.9866, 1.7522, 0.7659) -- (2.9846, 1.7526, 0.8159) -- (2.9846, 1.7021, 0.8174) -- cycle;
\fill[blue!15.0, opacity=0.5] (2.9846, 1.7021, 0.8174) -- (2.9846, 1.7526, 0.8159) -- (2.9824, 1.7529, 0.8659) -- (2.9824, 1.7023, 0.8674) -- cycle;
\fill[blue!15.0, opacity=0.5] (2.9824, 1.7023, 0.8674) -- (2.9824, 1.7529, 0.8659) -- (2.9801, 1.7533, 0.9159) -- (2.9801, 1.7026, 0.9174) -- cycle;
\fill[blue!15.0, opacity=0.5] (2.9801, 1.7026, 0.9174) -- (2.9801, 1.7533, 0.9159) -- (2.9778, 1.7537, 0.9659) -- (2.9778, 1.7030, 0.9674) -- cycle;
\fill[blue!15.0, opacity=0.5] (2.9778, 1.7030, 0.9674) -- (2.9778, 1.7537, 0.9659) -- (2.9753, 1.7541, 1.0159) -- (2.9753, 1.7033, 1.0174) -- cycle;
\fill[blue!15.0, opacity=0.5] (2.9753, 1.7033, 1.0174) -- (2.9753, 1.7541, 1.0159) -- (2.9727, 1.7546, 1.0659) -- (2.9727, 1.7036, 1.0674) -- cycle;
\fill[blue!15.0, opacity=0.5] (2.9727, 1.7036, 1.0674) -- (2.9727, 1.7546, 1.0659) -- (2.9700, 1.7550, 1.1159) -- (2.9700, 1.7040, 1.1174) -- cycle;
\fill[blue!15.0, opacity=0.5] (2.9700, 1.7040, 1.1174) -- (2.9700, 1.7550, 1.1159) -- (2.9672, 1.7555, 1.1659) -- (2.9672, 1.7044, 1.1674) -- cycle;
\fill[blue!15.0, opacity=0.5] (2.9672, 1.7044, 1.1674) -- (2.9672, 1.7555, 1.1659) -- (2.9644, 1.7559, 1.2159) -- (2.9644, 1.7047, 1.2174) -- cycle;
\fill[blue!15.0, opacity=0.5] (2.9644, 1.7047, 1.2174) -- (2.9644, 1.7559, 1.2159) -- (2.9615, 1.7564, 1.2659) -- (2.9615, 1.7051, 1.2674) -- cycle;
\fill[blue!15.0, opacity=0.5] (2.9615, 1.7051, 1.2674) -- (2.9615, 1.7564, 1.2659) -- (2.9585, 1.7569, 1.3159) -- (2.9585, 1.7055, 1.3174) -- cycle;
\fill[blue!15.0, opacity=0.5] (2.9585, 1.7055, 1.3174) -- (2.9585, 1.7569, 1.3159) -- (2.9555, 1.7574, 1.3659) -- (2.9555, 1.7059, 1.3674) -- cycle;
\fill[blue!15.0, opacity=0.5] (2.9555, 1.7059, 1.3674) -- (2.9555, 1.7574, 1.3659) -- (2.9525, 1.7579, 1.4159) -- (2.9525, 1.7063, 1.4174) -- cycle;
\fill[blue!15.0, opacity=0.5] (2.9525, 1.7063, 1.4174) -- (2.9525, 1.7579, 1.4159) -- (2.9494, 1.7584, 1.4659) -- (2.9494, 1.7067, 1.4674) -- cycle;
\fill[blue!15.0, opacity=0.5] (2.9494, 1.7067, 1.4674) -- (2.9494, 1.7584, 1.4659) -- (2.9463, 1.7590, 1.5159) -- (2.9463, 1.7072, 1.5174) -- cycle;
\fill[blue!15.0, opacity=0.5] (2.9463, 1.7072, 1.5174) -- (2.9463, 1.7590, 1.5159) -- (2.9431, 1.7595, 1.5659) -- (2.9431, 1.7076, 1.5674) -- cycle;
\fill[blue!15.0, opacity=0.5] (2.9431, 1.7076, 1.5674) -- (2.9431, 1.7595, 1.5659) -- (2.9400, 1.7600, 1.6159) -- (2.9400, 1.7080, 1.6174) -- cycle;
\fill[blue!15.0, opacity=0.5] (2.9400, 1.7080, 1.6174) -- (2.9400, 1.7600, 1.6159) -- (2.9369, 1.7605, 1.6659) -- (2.9369, 1.7084, 1.6674) -- cycle;
\fill[blue!15.0, opacity=0.5] (2.9369, 1.7084, 1.6674) -- (2.9369, 1.7605, 1.6659) -- (2.9337, 1.7610, 1.7159) -- (2.9337, 1.7088, 1.7174) -- cycle;
\fill[blue!15.0, opacity=0.5] (2.9337, 1.7088, 1.7174) -- (2.9337, 1.7610, 1.7159) -- (2.9306, 1.7616, 1.7659) -- (2.9306, 1.7093, 1.7674) -- cycle;
\fill[blue!15.0, opacity=0.5] (2.9306, 1.7093, 1.7674) -- (2.9306, 1.7616, 1.7659) -- (2.9275, 1.7621, 1.8159) -- (2.9275, 1.7097, 1.8174) -- cycle;
\fill[blue!15.0, opacity=0.5] (2.9275, 1.7097, 1.8174) -- (2.9275, 1.7621, 1.8159) -- (2.9245, 1.7626, 1.8659) -- (2.9245, 1.7101, 1.8674) -- cycle;
\fill[blue!15.0, opacity=0.5] (2.9245, 1.7101, 1.8674) -- (2.9245, 1.7626, 1.8659) -- (2.9215, 1.7631, 1.9159) -- (2.9215, 1.7105, 1.9174) -- cycle;
\fill[blue!15.0, opacity=0.5] (2.9215, 1.7105, 1.9174) -- (2.9215, 1.7631, 1.9159) -- (2.9185, 1.7636, 1.9659) -- (2.9185, 1.7109, 1.9674) -- cycle;
\fill[blue!15.0, opacity=0.5] (2.9185, 1.7109, 1.9674) -- (2.9185, 1.7636, 1.9659) -- (2.9156, 1.7641, 2.0159) -- (2.9156, 1.7113, 2.0174) -- cycle;
\fill[blue!15.0, opacity=0.5] (2.9156, 1.7113, 2.0174) -- (2.9156, 1.7641, 2.0159) -- (2.9128, 1.7645, 2.0659) -- (2.9128, 1.7116, 2.0674) -- cycle;
\fill[blue!15.0, opacity=0.5] (2.9128, 1.7116, 2.0674) -- (2.9128, 1.7645, 2.0659) -- (2.9100, 1.7650, 2.1159) -- (2.9100, 1.7120, 2.1174) -- cycle;
\fill[blue!15.0, opacity=0.5] (2.9100, 1.7120, 2.1174) -- (2.9100, 1.7650, 2.1159) -- (2.9073, 1.7654, 2.1659) -- (2.9073, 1.7124, 2.1674) -- cycle;
\fill[blue!15.1, opacity=0.5] (2.9073, 1.7124, 2.1674) -- (2.9073, 1.7654, 2.1659) -- (2.9047, 1.7659, 2.2159) -- (2.9047, 1.7127, 2.2174) -- cycle;
\fill[blue!15.1, opacity=0.5] (2.9047, 1.7127, 2.2174) -- (2.9047, 1.7659, 2.2159) -- (2.9022, 1.7663, 2.2659) -- (2.9022, 1.7130, 2.2674) -- cycle;
\fill[blue!15.1, opacity=0.5] (2.9022, 1.7130, 2.2674) -- (2.9022, 1.7663, 2.2659) -- (2.8999, 1.7667, 2.3159) -- (2.8999, 1.7134, 2.3174) -- cycle;
\fill[blue!15.2, opacity=0.5] (2.8999, 1.7134, 2.3174) -- (2.8999, 1.7667, 2.3159) -- (2.8976, 1.7671, 2.3659) -- (2.8976, 1.7137, 2.3674) -- cycle;
\fill[blue!15.2, opacity=0.5] (2.8976, 1.7137, 2.3674) -- (2.8976, 1.7671, 2.3659) -- (2.8954, 1.7674, 2.4159) -- (2.8954, 1.7139, 2.4174) -- cycle;
\fill[blue!15.3, opacity=0.5] (2.8954, 1.7139, 2.4174) -- (2.8954, 1.7674, 2.4159) -- (2.8934, 1.7678, 2.4659) -- (2.8934, 1.7142, 2.4674) -- cycle;
\fill[blue!15.4, opacity=0.5] (2.8934, 1.7142, 2.4674) -- (2.8934, 1.7678, 2.4659) -- (2.8915, 1.7681, 2.5159) -- (2.8915, 1.7145, 2.5174) -- cycle;
\fill[blue!15.5, opacity=0.5] (2.8915, 1.7145, 2.5174) -- (2.8915, 1.7681, 2.5159) -- (2.8897, 1.7684, 2.5659) -- (2.8897, 1.7147, 2.5674) -- cycle;
\fill[blue!15.7, opacity=0.5] (2.8897, 1.7147, 2.5674) -- (2.8897, 1.7684, 2.5659) -- (2.8880, 1.7687, 2.6159) -- (2.8880, 1.7149, 2.6174) -- cycle;
\fill[blue!15.8, opacity=0.5] (2.8880, 1.7149, 2.6174) -- (2.8880, 1.7687, 2.6159) -- (2.8865, 1.7689, 2.6659) -- (2.8865, 1.7151, 2.6674) -- cycle;
\fill[blue!16.0, opacity=0.5] (2.8865, 1.7151, 2.6674) -- (2.8865, 1.7689, 2.6659) -- (2.8852, 1.7691, 2.7159) -- (2.8852, 1.7153, 2.7174) -- cycle;
\fill[blue!16.2, opacity=0.5] (2.8852, 1.7153, 2.7174) -- (2.8852, 1.7691, 2.7159) -- (2.8840, 1.7693, 2.7659) -- (2.8840, 1.7155, 2.7674) -- cycle;
\fill[blue!16.5, opacity=0.5] (2.8840, 1.7155, 2.7674) -- (2.8840, 1.7693, 2.7659) -- (2.8829, 1.7695, 2.8159) -- (2.8829, 1.7156, 2.8174) -- cycle;
\fill[blue!16.8, opacity=0.5] (2.8829, 1.7156, 2.8174) -- (2.8829, 1.7695, 2.8159) -- (2.8820, 1.7697, 2.8659) -- (2.8820, 1.7157, 2.8674) -- cycle;
\fill[blue!17.1, opacity=0.5] (2.8820, 1.7157, 2.8674) -- (2.8820, 1.7697, 2.8659) -- (2.8813, 1.7698, 2.9159) -- (2.8813, 1.7158, 2.9174) -- cycle;
\fill[blue!17.4, opacity=0.5] (2.8813, 1.7158, 2.9174) -- (2.8813, 1.7698, 2.9159) -- (2.8807, 1.7699, 2.9659) -- (2.8807, 1.7159, 2.9674) -- cycle;
\fill[blue!17.8, opacity=0.5] (2.8807, 1.7159, 2.9674) -- (2.8807, 1.7699, 2.9659) -- (2.8803, 1.7699, 3.0159) -- (2.8803, 1.7160, 3.0174) -- cycle;
\fill[blue!18.2, opacity=0.5] (2.8803, 1.7160, 3.0174) -- (2.8803, 1.7699, 3.0159) -- (2.8801, 1.7700, 3.0659) -- (2.8801, 1.7160, 3.0674) -- cycle;
\fill[blue!18.7, opacity=0.5] (2.8801, 1.7160, 3.0674) -- (2.8801, 1.7700, 3.0659) -- (2.8800, 1.7700, 3.1159) -- (2.8800, 1.7160, 3.1174) -- cycle;
\fill[blue!15.0, opacity=0.5] (3.0000, 1.7500, 0.1159) -- (3.0000, 1.8000, 0.1141) -- (2.9999, 1.8000, 0.1641) -- (2.9999, 1.7500, 0.1659) -- cycle;
\fill[blue!15.0, opacity=0.5] (2.9999, 1.7500, 0.1659) -- (2.9999, 1.8000, 0.1641) -- (2.9997, 1.8001, 0.2141) -- (2.9997, 1.7501, 0.2159) -- cycle;
\fill[blue!15.0, opacity=0.5] (2.9997, 1.7501, 0.2159) -- (2.9997, 1.8001, 0.2141) -- (2.9993, 1.8001, 0.2641) -- (2.9993, 1.7501, 0.2659) -- cycle;
\fill[blue!15.0, opacity=0.5] (2.9993, 1.7501, 0.2659) -- (2.9993, 1.8001, 0.2641) -- (2.9987, 1.8003, 0.3141) -- (2.9987, 1.7502, 0.3159) -- cycle;
\fill[blue!15.0, opacity=0.5] (2.9987, 1.7502, 0.3159) -- (2.9987, 1.8003, 0.3141) -- (2.9980, 1.8004, 0.3641) -- (2.9980, 1.7503, 0.3659) -- cycle;
\fill[blue!15.0, opacity=0.5] (2.9980, 1.7503, 0.3659) -- (2.9980, 1.8004, 0.3641) -- (2.9971, 1.8006, 0.4141) -- (2.9971, 1.7505, 0.4159) -- cycle;
\fill[blue!15.0, opacity=0.5] (2.9971, 1.7505, 0.4159) -- (2.9971, 1.8006, 0.4141) -- (2.9960, 1.8008, 0.4641) -- (2.9960, 1.7507, 0.4659) -- cycle;
\fill[blue!15.0, opacity=0.5] (2.9960, 1.7507, 0.4659) -- (2.9960, 1.8008, 0.4641) -- (2.9948, 1.8010, 0.5141) -- (2.9948, 1.7509, 0.5159) -- cycle;
\fill[blue!15.0, opacity=0.5] (2.9948, 1.7509, 0.5159) -- (2.9948, 1.8010, 0.5141) -- (2.9935, 1.8013, 0.5641) -- (2.9935, 1.7511, 0.5659) -- cycle;
\fill[blue!15.0, opacity=0.5] (2.9935, 1.7511, 0.5659) -- (2.9935, 1.8013, 0.5641) -- (2.9920, 1.8016, 0.6141) -- (2.9920, 1.7513, 0.6159) -- cycle;
\fill[blue!15.0, opacity=0.5] (2.9920, 1.7513, 0.6159) -- (2.9920, 1.8016, 0.6141) -- (2.9903, 1.8019, 0.6641) -- (2.9903, 1.7516, 0.6659) -- cycle;
\fill[blue!15.0, opacity=0.5] (2.9903, 1.7516, 0.6659) -- (2.9903, 1.8019, 0.6641) -- (2.9885, 1.8023, 0.7141) -- (2.9885, 1.7519, 0.7159) -- cycle;
\fill[blue!15.0, opacity=0.5] (2.9885, 1.7519, 0.7159) -- (2.9885, 1.8023, 0.7141) -- (2.9866, 1.8027, 0.7641) -- (2.9866, 1.7522, 0.7659) -- cycle;
\fill[blue!15.0, opacity=0.5] (2.9866, 1.7522, 0.7659) -- (2.9866, 1.8027, 0.7641) -- (2.9846, 1.8031, 0.8141) -- (2.9846, 1.7526, 0.8159) -- cycle;
\fill[blue!15.0, opacity=0.5] (2.9846, 1.7526, 0.8159) -- (2.9846, 1.8031, 0.8141) -- (2.9824, 1.8035, 0.8641) -- (2.9824, 1.7529, 0.8659) -- cycle;
\fill[blue!15.0, opacity=0.5] (2.9824, 1.7529, 0.8659) -- (2.9824, 1.8035, 0.8641) -- (2.9801, 1.8040, 0.9141) -- (2.9801, 1.7533, 0.9159) -- cycle;
\fill[blue!15.0, opacity=0.5] (2.9801, 1.7533, 0.9159) -- (2.9801, 1.8040, 0.9141) -- (2.9778, 1.8044, 0.9641) -- (2.9778, 1.7537, 0.9659) -- cycle;
\fill[blue!15.0, opacity=0.5] (2.9778, 1.7537, 0.9659) -- (2.9778, 1.8044, 0.9641) -- (2.9753, 1.8049, 1.0141) -- (2.9753, 1.7541, 1.0159) -- cycle;
\fill[blue!15.0, opacity=0.5] (2.9753, 1.7541, 1.0159) -- (2.9753, 1.8049, 1.0141) -- (2.9727, 1.8055, 1.0641) -- (2.9727, 1.7546, 1.0659) -- cycle;
\fill[blue!15.0, opacity=0.5] (2.9727, 1.7546, 1.0659) -- (2.9727, 1.8055, 1.0641) -- (2.9700, 1.8060, 1.1141) -- (2.9700, 1.7550, 1.1159) -- cycle;
\fill[blue!15.0, opacity=0.5] (2.9700, 1.7550, 1.1159) -- (2.9700, 1.8060, 1.1141) -- (2.9672, 1.8066, 1.1641) -- (2.9672, 1.7555, 1.1659) -- cycle;
\fill[blue!15.0, opacity=0.5] (2.9672, 1.7555, 1.1659) -- (2.9672, 1.8066, 1.1641) -- (2.9644, 1.8071, 1.2141) -- (2.9644, 1.7559, 1.2159) -- cycle;
\fill[blue!15.0, opacity=0.5] (2.9644, 1.7559, 1.2159) -- (2.9644, 1.8071, 1.2141) -- (2.9615, 1.8077, 1.2641) -- (2.9615, 1.7564, 1.2659) -- cycle;
\fill[blue!15.0, opacity=0.5] (2.9615, 1.7564, 1.2659) -- (2.9615, 1.8077, 1.2641) -- (2.9585, 1.8083, 1.3141) -- (2.9585, 1.7569, 1.3159) -- cycle;
\fill[blue!15.0, opacity=0.5] (2.9585, 1.7569, 1.3159) -- (2.9585, 1.8083, 1.3141) -- (2.9555, 1.8089, 1.3641) -- (2.9555, 1.7574, 1.3659) -- cycle;
\fill[blue!15.0, opacity=0.5] (2.9555, 1.7574, 1.3659) -- (2.9555, 1.8089, 1.3641) -- (2.9525, 1.8095, 1.4141) -- (2.9525, 1.7579, 1.4159) -- cycle;
\fill[blue!15.0, opacity=0.5] (2.9525, 1.7579, 1.4159) -- (2.9525, 1.8095, 1.4141) -- (2.9494, 1.8101, 1.4641) -- (2.9494, 1.7584, 1.4659) -- cycle;
\fill[blue!15.0, opacity=0.5] (2.9494, 1.7584, 1.4659) -- (2.9494, 1.8101, 1.4641) -- (2.9463, 1.8107, 1.5141) -- (2.9463, 1.7590, 1.5159) -- cycle;
\fill[blue!15.0, opacity=0.5] (2.9463, 1.7590, 1.5159) -- (2.9463, 1.8107, 1.5141) -- (2.9431, 1.8114, 1.5641) -- (2.9431, 1.7595, 1.5659) -- cycle;
\fill[blue!15.0, opacity=0.5] (2.9431, 1.7595, 1.5659) -- (2.9431, 1.8114, 1.5641) -- (2.9400, 1.8120, 1.6141) -- (2.9400, 1.7600, 1.6159) -- cycle;
\fill[blue!15.0, opacity=0.5] (2.9400, 1.7600, 1.6159) -- (2.9400, 1.8120, 1.6141) -- (2.9369, 1.8126, 1.6641) -- (2.9369, 1.7605, 1.6659) -- cycle;
\fill[blue!15.0, opacity=0.5] (2.9369, 1.7605, 1.6659) -- (2.9369, 1.8126, 1.6641) -- (2.9337, 1.8133, 1.7141) -- (2.9337, 1.7610, 1.7159) -- cycle;
\fill[blue!15.0, opacity=0.5] (2.9337, 1.7610, 1.7159) -- (2.9337, 1.8133, 1.7141) -- (2.9306, 1.8139, 1.7641) -- (2.9306, 1.7616, 1.7659) -- cycle;
\fill[blue!15.0, opacity=0.5] (2.9306, 1.7616, 1.7659) -- (2.9306, 1.8139, 1.7641) -- (2.9275, 1.8145, 1.8141) -- (2.9275, 1.7621, 1.8159) -- cycle;
\fill[blue!15.0, opacity=0.5] (2.9275, 1.7621, 1.8159) -- (2.9275, 1.8145, 1.8141) -- (2.9245, 1.8151, 1.8641) -- (2.9245, 1.7626, 1.8659) -- cycle;
\fill[blue!15.0, opacity=0.5] (2.9245, 1.7626, 1.8659) -- (2.9245, 1.8151, 1.8641) -- (2.9215, 1.8157, 1.9141) -- (2.9215, 1.7631, 1.9159) -- cycle;
\fill[blue!15.0, opacity=0.5] (2.9215, 1.7631, 1.9159) -- (2.9215, 1.8157, 1.9141) -- (2.9185, 1.8163, 1.9641) -- (2.9185, 1.7636, 1.9659) -- cycle;
\fill[blue!15.0, opacity=0.5] (2.9185, 1.7636, 1.9659) -- (2.9185, 1.8163, 1.9641) -- (2.9156, 1.8169, 2.0141) -- (2.9156, 1.7641, 2.0159) -- cycle;
\fill[blue!15.0, opacity=0.5] (2.9156, 1.7641, 2.0159) -- (2.9156, 1.8169, 2.0141) -- (2.9128, 1.8174, 2.0641) -- (2.9128, 1.7645, 2.0659) -- cycle;
\fill[blue!15.0, opacity=0.5] (2.9128, 1.7645, 2.0659) -- (2.9128, 1.8174, 2.0641) -- (2.9100, 1.8180, 2.1141) -- (2.9100, 1.7650, 2.1159) -- cycle;
\fill[blue!15.0, opacity=0.5] (2.9100, 1.7650, 2.1159) -- (2.9100, 1.8180, 2.1141) -- (2.9073, 1.8185, 2.1641) -- (2.9073, 1.7654, 2.1659) -- cycle;
\fill[blue!15.0, opacity=0.5] (2.9073, 1.7654, 2.1659) -- (2.9073, 1.8185, 2.1641) -- (2.9047, 1.8191, 2.2141) -- (2.9047, 1.7659, 2.2159) -- cycle;
\fill[blue!15.1, opacity=0.5] (2.9047, 1.7659, 2.2159) -- (2.9047, 1.8191, 2.2141) -- (2.9022, 1.8196, 2.2641) -- (2.9022, 1.7663, 2.2659) -- cycle;
\fill[blue!15.1, opacity=0.5] (2.9022, 1.7663, 2.2659) -- (2.9022, 1.8196, 2.2641) -- (2.8999, 1.8200, 2.3141) -- (2.8999, 1.7667, 2.3159) -- cycle;
\fill[blue!15.1, opacity=0.5] (2.8999, 1.7667, 2.3159) -- (2.8999, 1.8200, 2.3141) -- (2.8976, 1.8205, 2.3641) -- (2.8976, 1.7671, 2.3659) -- cycle;
\fill[blue!15.2, opacity=0.5] (2.8976, 1.7671, 2.3659) -- (2.8976, 1.8205, 2.3641) -- (2.8954, 1.8209, 2.4141) -- (2.8954, 1.7674, 2.4159) -- cycle;
\fill[blue!15.2, opacity=0.5] (2.8954, 1.7674, 2.4159) -- (2.8954, 1.8209, 2.4141) -- (2.8934, 1.8213, 2.4641) -- (2.8934, 1.7678, 2.4659) -- cycle;
\fill[blue!15.3, opacity=0.5] (2.8934, 1.7678, 2.4659) -- (2.8934, 1.8213, 2.4641) -- (2.8915, 1.8217, 2.5141) -- (2.8915, 1.7681, 2.5159) -- cycle;
\fill[blue!15.4, opacity=0.5] (2.8915, 1.7681, 2.5159) -- (2.8915, 1.8217, 2.5141) -- (2.8897, 1.8221, 2.5641) -- (2.8897, 1.7684, 2.5659) -- cycle;
\fill[blue!15.5, opacity=0.5] (2.8897, 1.7684, 2.5659) -- (2.8897, 1.8221, 2.5641) -- (2.8880, 1.8224, 2.6141) -- (2.8880, 1.7687, 2.6159) -- cycle;
\fill[blue!15.6, opacity=0.5] (2.8880, 1.7687, 2.6159) -- (2.8880, 1.8224, 2.6141) -- (2.8865, 1.8227, 2.6641) -- (2.8865, 1.7689, 2.6659) -- cycle;
\fill[blue!15.7, opacity=0.5] (2.8865, 1.7689, 2.6659) -- (2.8865, 1.8227, 2.6641) -- (2.8852, 1.8230, 2.7141) -- (2.8852, 1.7691, 2.7159) -- cycle;
\fill[blue!15.9, opacity=0.5] (2.8852, 1.7691, 2.7159) -- (2.8852, 1.8230, 2.7141) -- (2.8840, 1.8232, 2.7641) -- (2.8840, 1.7693, 2.7659) -- cycle;
\fill[blue!16.1, opacity=0.5] (2.8840, 1.7693, 2.7659) -- (2.8840, 1.8232, 2.7641) -- (2.8829, 1.8234, 2.8141) -- (2.8829, 1.7695, 2.8159) -- cycle;
\fill[blue!16.3, opacity=0.5] (2.8829, 1.7695, 2.8159) -- (2.8829, 1.8234, 2.8141) -- (2.8820, 1.8236, 2.8641) -- (2.8820, 1.7697, 2.8659) -- cycle;
\fill[blue!16.6, opacity=0.5] (2.8820, 1.7697, 2.8659) -- (2.8820, 1.8236, 2.8641) -- (2.8813, 1.8237, 2.9141) -- (2.8813, 1.7698, 2.9159) -- cycle;
\fill[blue!16.8, opacity=0.5] (2.8813, 1.7698, 2.9159) -- (2.8813, 1.8237, 2.9141) -- (2.8807, 1.8239, 2.9641) -- (2.8807, 1.7699, 2.9659) -- cycle;
\fill[blue!17.1, opacity=0.5] (2.8807, 1.7699, 2.9659) -- (2.8807, 1.8239, 2.9641) -- (2.8803, 1.8239, 3.0141) -- (2.8803, 1.7699, 3.0159) -- cycle;
\fill[blue!17.5, opacity=0.5] (2.8803, 1.7699, 3.0159) -- (2.8803, 1.8239, 3.0141) -- (2.8801, 1.8240, 3.0641) -- (2.8801, 1.7700, 3.0659) -- cycle;
\fill[blue!17.8, opacity=0.5] (2.8801, 1.7700, 3.0659) -- (2.8801, 1.8240, 3.0641) -- (2.8800, 1.8240, 3.1141) -- (2.8800, 1.7700, 3.1159) -- cycle;
\fill[blue!15.0, opacity=0.5] (3.0000, 1.8000, 0.1141) -- (3.0000, 1.8500, 0.1120) -- (2.9999, 1.8500, 0.1620) -- (2.9999, 1.8000, 0.1641) -- cycle;
\fill[blue!15.0, opacity=0.5] (2.9999, 1.8000, 0.1641) -- (2.9999, 1.8500, 0.1620) -- (2.9997, 1.8501, 0.2120) -- (2.9997, 1.8001, 0.2141) -- cycle;
\fill[blue!15.0, opacity=0.5] (2.9997, 1.8001, 0.2141) -- (2.9997, 1.8501, 0.2120) -- (2.9993, 1.8502, 0.2620) -- (2.9993, 1.8001, 0.2641) -- cycle;
\fill[blue!15.0, opacity=0.5] (2.9993, 1.8001, 0.2641) -- (2.9993, 1.8502, 0.2620) -- (2.9987, 1.8503, 0.3120) -- (2.9987, 1.8003, 0.3141) -- cycle;
\fill[blue!15.0, opacity=0.5] (2.9987, 1.8003, 0.3141) -- (2.9987, 1.8503, 0.3120) -- (2.9980, 1.8505, 0.3620) -- (2.9980, 1.8004, 0.3641) -- cycle;
\fill[blue!15.0, opacity=0.5] (2.9980, 1.8004, 0.3641) -- (2.9980, 1.8505, 0.3620) -- (2.9971, 1.8507, 0.4120) -- (2.9971, 1.8006, 0.4141) -- cycle;
\fill[blue!15.0, opacity=0.5] (2.9971, 1.8006, 0.4141) -- (2.9971, 1.8507, 0.4120) -- (2.9960, 1.8509, 0.4620) -- (2.9960, 1.8008, 0.4641) -- cycle;
\fill[blue!15.0, opacity=0.5] (2.9960, 1.8008, 0.4641) -- (2.9960, 1.8509, 0.4620) -- (2.9948, 1.8512, 0.5120) -- (2.9948, 1.8010, 0.5141) -- cycle;
\fill[blue!15.0, opacity=0.5] (2.9948, 1.8010, 0.5141) -- (2.9948, 1.8512, 0.5120) -- (2.9935, 1.8515, 0.5620) -- (2.9935, 1.8013, 0.5641) -- cycle;
\fill[blue!15.0, opacity=0.5] (2.9935, 1.8013, 0.5641) -- (2.9935, 1.8515, 0.5620) -- (2.9920, 1.8519, 0.6120) -- (2.9920, 1.8016, 0.6141) -- cycle;
\fill[blue!15.0, opacity=0.5] (2.9920, 1.8016, 0.6141) -- (2.9920, 1.8519, 0.6120) -- (2.9903, 1.8523, 0.6620) -- (2.9903, 1.8019, 0.6641) -- cycle;
\fill[blue!15.0, opacity=0.5] (2.9903, 1.8019, 0.6641) -- (2.9903, 1.8523, 0.6620) -- (2.9885, 1.8527, 0.7120) -- (2.9885, 1.8023, 0.7141) -- cycle;
\fill[blue!15.0, opacity=0.5] (2.9885, 1.8023, 0.7141) -- (2.9885, 1.8527, 0.7120) -- (2.9866, 1.8531, 0.7620) -- (2.9866, 1.8027, 0.7641) -- cycle;
\fill[blue!15.0, opacity=0.5] (2.9866, 1.8027, 0.7641) -- (2.9866, 1.8531, 0.7620) -- (2.9846, 1.8536, 0.8120) -- (2.9846, 1.8031, 0.8141) -- cycle;
\fill[blue!15.0, opacity=0.5] (2.9846, 1.8031, 0.8141) -- (2.9846, 1.8536, 0.8120) -- (2.9824, 1.8541, 0.8620) -- (2.9824, 1.8035, 0.8641) -- cycle;
\fill[blue!15.0, opacity=0.5] (2.9824, 1.8035, 0.8641) -- (2.9824, 1.8541, 0.8620) -- (2.9801, 1.8546, 0.9120) -- (2.9801, 1.8040, 0.9141) -- cycle;
\fill[blue!15.0, opacity=0.5] (2.9801, 1.8040, 0.9141) -- (2.9801, 1.8546, 0.9120) -- (2.9778, 1.8552, 0.9620) -- (2.9778, 1.8044, 0.9641) -- cycle;
\fill[blue!15.0, opacity=0.5] (2.9778, 1.8044, 0.9641) -- (2.9778, 1.8552, 0.9620) -- (2.9753, 1.8558, 1.0120) -- (2.9753, 1.8049, 1.0141) -- cycle;
\fill[blue!15.0, opacity=0.5] (2.9753, 1.8049, 1.0141) -- (2.9753, 1.8558, 1.0120) -- (2.9727, 1.8564, 1.0620) -- (2.9727, 1.8055, 1.0641) -- cycle;
\fill[blue!15.0, opacity=0.5] (2.9727, 1.8055, 1.0641) -- (2.9727, 1.8564, 1.0620) -- (2.9700, 1.8570, 1.1120) -- (2.9700, 1.8060, 1.1141) -- cycle;
\fill[blue!15.0, opacity=0.5] (2.9700, 1.8060, 1.1141) -- (2.9700, 1.8570, 1.1120) -- (2.9672, 1.8576, 1.1620) -- (2.9672, 1.8066, 1.1641) -- cycle;
\fill[blue!15.0, opacity=0.5] (2.9672, 1.8066, 1.1641) -- (2.9672, 1.8576, 1.1620) -- (2.9644, 1.8583, 1.2120) -- (2.9644, 1.8071, 1.2141) -- cycle;
\fill[blue!15.0, opacity=0.5] (2.9644, 1.8071, 1.2141) -- (2.9644, 1.8583, 1.2120) -- (2.9615, 1.8590, 1.2620) -- (2.9615, 1.8077, 1.2641) -- cycle;
\fill[blue!15.0, opacity=0.5] (2.9615, 1.8077, 1.2641) -- (2.9615, 1.8590, 1.2620) -- (2.9585, 1.8597, 1.3120) -- (2.9585, 1.8083, 1.3141) -- cycle;
\fill[blue!15.0, opacity=0.5] (2.9585, 1.8083, 1.3141) -- (2.9585, 1.8597, 1.3120) -- (2.9555, 1.8604, 1.3620) -- (2.9555, 1.8089, 1.3641) -- cycle;
\fill[blue!15.0, opacity=0.5] (2.9555, 1.8089, 1.3641) -- (2.9555, 1.8604, 1.3620) -- (2.9525, 1.8611, 1.4120) -- (2.9525, 1.8095, 1.4141) -- cycle;
\fill[blue!15.0, opacity=0.5] (2.9525, 1.8095, 1.4141) -- (2.9525, 1.8611, 1.4120) -- (2.9494, 1.8618, 1.4620) -- (2.9494, 1.8101, 1.4641) -- cycle;
\fill[blue!15.0, opacity=0.5] (2.9494, 1.8101, 1.4641) -- (2.9494, 1.8618, 1.4620) -- (2.9463, 1.8625, 1.5120) -- (2.9463, 1.8107, 1.5141) -- cycle;
\fill[blue!15.0, opacity=0.5] (2.9463, 1.8107, 1.5141) -- (2.9463, 1.8625, 1.5120) -- (2.9431, 1.8633, 1.5620) -- (2.9431, 1.8114, 1.5641) -- cycle;
\fill[blue!15.0, opacity=0.5] (2.9431, 1.8114, 1.5641) -- (2.9431, 1.8633, 1.5620) -- (2.9400, 1.8640, 1.6120) -- (2.9400, 1.8120, 1.6141) -- cycle;
\fill[blue!15.0, opacity=0.5] (2.9400, 1.8120, 1.6141) -- (2.9400, 1.8640, 1.6120) -- (2.9369, 1.8647, 1.6620) -- (2.9369, 1.8126, 1.6641) -- cycle;
\fill[blue!15.0, opacity=0.5] (2.9369, 1.8126, 1.6641) -- (2.9369, 1.8647, 1.6620) -- (2.9337, 1.8655, 1.7120) -- (2.9337, 1.8133, 1.7141) -- cycle;
\fill[blue!15.0, opacity=0.5] (2.9337, 1.8133, 1.7141) -- (2.9337, 1.8655, 1.7120) -- (2.9306, 1.8662, 1.7620) -- (2.9306, 1.8139, 1.7641) -- cycle;
\fill[blue!15.0, opacity=0.5] (2.9306, 1.8139, 1.7641) -- (2.9306, 1.8662, 1.7620) -- (2.9275, 1.8669, 1.8120) -- (2.9275, 1.8145, 1.8141) -- cycle;
\fill[blue!15.0, opacity=0.5] (2.9275, 1.8145, 1.8141) -- (2.9275, 1.8669, 1.8120) -- (2.9245, 1.8676, 1.8620) -- (2.9245, 1.8151, 1.8641) -- cycle;
\fill[blue!15.0, opacity=0.5] (2.9245, 1.8151, 1.8641) -- (2.9245, 1.8676, 1.8620) -- (2.9215, 1.8683, 1.9120) -- (2.9215, 1.8157, 1.9141) -- cycle;
\fill[blue!15.0, opacity=0.5] (2.9215, 1.8157, 1.9141) -- (2.9215, 1.8683, 1.9120) -- (2.9185, 1.8690, 1.9620) -- (2.9185, 1.8163, 1.9641) -- cycle;
\fill[blue!15.0, opacity=0.5] (2.9185, 1.8163, 1.9641) -- (2.9185, 1.8690, 1.9620) -- (2.9156, 1.8697, 2.0120) -- (2.9156, 1.8169, 2.0141) -- cycle;
\fill[blue!15.0, opacity=0.5] (2.9156, 1.8169, 2.0141) -- (2.9156, 1.8697, 2.0120) -- (2.9128, 1.8704, 2.0620) -- (2.9128, 1.8174, 2.0641) -- cycle;
\fill[blue!15.0, opacity=0.5] (2.9128, 1.8174, 2.0641) -- (2.9128, 1.8704, 2.0620) -- (2.9100, 1.8710, 2.1120) -- (2.9100, 1.8180, 2.1141) -- cycle;
\fill[blue!15.0, opacity=0.5] (2.9100, 1.8180, 2.1141) -- (2.9100, 1.8710, 2.1120) -- (2.9073, 1.8716, 2.1620) -- (2.9073, 1.8185, 2.1641) -- cycle;
\fill[blue!15.0, opacity=0.5] (2.9073, 1.8185, 2.1641) -- (2.9073, 1.8716, 2.1620) -- (2.9047, 1.8722, 2.2120) -- (2.9047, 1.8191, 2.2141) -- cycle;
\fill[blue!15.0, opacity=0.5] (2.9047, 1.8191, 2.2141) -- (2.9047, 1.8722, 2.2120) -- (2.9022, 1.8728, 2.2620) -- (2.9022, 1.8196, 2.2641) -- cycle;
\fill[blue!15.1, opacity=0.5] (2.9022, 1.8196, 2.2641) -- (2.9022, 1.8728, 2.2620) -- (2.8999, 1.8734, 2.3120) -- (2.8999, 1.8200, 2.3141) -- cycle;
\fill[blue!15.1, opacity=0.5] (2.8999, 1.8200, 2.3141) -- (2.8999, 1.8734, 2.3120) -- (2.8976, 1.8739, 2.3620) -- (2.8976, 1.8205, 2.3641) -- cycle;
\fill[blue!15.1, opacity=0.5] (2.8976, 1.8205, 2.3641) -- (2.8976, 1.8739, 2.3620) -- (2.8954, 1.8744, 2.4120) -- (2.8954, 1.8209, 2.4141) -- cycle;
\fill[blue!15.1, opacity=0.5] (2.8954, 1.8209, 2.4141) -- (2.8954, 1.8744, 2.4120) -- (2.8934, 1.8749, 2.4620) -- (2.8934, 1.8213, 2.4641) -- cycle;
\fill[blue!15.2, opacity=0.5] (2.8934, 1.8213, 2.4641) -- (2.8934, 1.8749, 2.4620) -- (2.8915, 1.8753, 2.5120) -- (2.8915, 1.8217, 2.5141) -- cycle;
\fill[blue!15.3, opacity=0.5] (2.8915, 1.8217, 2.5141) -- (2.8915, 1.8753, 2.5120) -- (2.8897, 1.8757, 2.5620) -- (2.8897, 1.8221, 2.5641) -- cycle;
\fill[blue!15.3, opacity=0.5] (2.8897, 1.8221, 2.5641) -- (2.8897, 1.8757, 2.5620) -- (2.8880, 1.8761, 2.6120) -- (2.8880, 1.8224, 2.6141) -- cycle;
\fill[blue!15.4, opacity=0.5] (2.8880, 1.8224, 2.6141) -- (2.8880, 1.8761, 2.6120) -- (2.8865, 1.8765, 2.6620) -- (2.8865, 1.8227, 2.6641) -- cycle;
\fill[blue!15.6, opacity=0.5] (2.8865, 1.8227, 2.6641) -- (2.8865, 1.8765, 2.6620) -- (2.8852, 1.8768, 2.7120) -- (2.8852, 1.8230, 2.7141) -- cycle;
\fill[blue!15.7, opacity=0.5] (2.8852, 1.8230, 2.7141) -- (2.8852, 1.8768, 2.7120) -- (2.8840, 1.8771, 2.7620) -- (2.8840, 1.8232, 2.7641) -- cycle;
\fill[blue!15.9, opacity=0.5] (2.8840, 1.8232, 2.7641) -- (2.8840, 1.8771, 2.7620) -- (2.8829, 1.8773, 2.8120) -- (2.8829, 1.8234, 2.8141) -- cycle;
\fill[blue!16.0, opacity=0.5] (2.8829, 1.8234, 2.8141) -- (2.8829, 1.8773, 2.8120) -- (2.8820, 1.8775, 2.8620) -- (2.8820, 1.8236, 2.8641) -- cycle;
\fill[blue!16.2, opacity=0.5] (2.8820, 1.8236, 2.8641) -- (2.8820, 1.8775, 2.8620) -- (2.8813, 1.8777, 2.9120) -- (2.8813, 1.8237, 2.9141) -- cycle;
\fill[blue!16.5, opacity=0.5] (2.8813, 1.8237, 2.9141) -- (2.8813, 1.8777, 2.9120) -- (2.8807, 1.8778, 2.9620) -- (2.8807, 1.8239, 2.9641) -- cycle;
\fill[blue!16.7, opacity=0.5] (2.8807, 1.8239, 2.9641) -- (2.8807, 1.8778, 2.9620) -- (2.8803, 1.8779, 3.0120) -- (2.8803, 1.8239, 3.0141) -- cycle;
\fill[blue!17.0, opacity=0.5] (2.8803, 1.8239, 3.0141) -- (2.8803, 1.8779, 3.0120) -- (2.8801, 1.8780, 3.0620) -- (2.8801, 1.8240, 3.0641) -- cycle;
\fill[blue!17.4, opacity=0.5] (2.8801, 1.8240, 3.0641) -- (2.8801, 1.8780, 3.0620) -- (2.8800, 1.8780, 3.1120) -- (2.8800, 1.8240, 3.1141) -- cycle;
\fill[blue!15.0, opacity=0.5] (3.0000, 1.8500, 0.1120) -- (3.0000, 1.9000, 0.1096) -- (2.9999, 1.9000, 0.1596) -- (2.9999, 1.8500, 0.1620) -- cycle;
\fill[blue!15.0, opacity=0.5] (2.9999, 1.8500, 0.1620) -- (2.9999, 1.9000, 0.1596) -- (2.9997, 1.9001, 0.2096) -- (2.9997, 1.8501, 0.2120) -- cycle;
\fill[blue!15.0, opacity=0.5] (2.9997, 1.8501, 0.2120) -- (2.9997, 1.9001, 0.2096) -- (2.9993, 1.9002, 0.2596) -- (2.9993, 1.8502, 0.2620) -- cycle;
\fill[blue!15.0, opacity=0.5] (2.9993, 1.8502, 0.2620) -- (2.9993, 1.9002, 0.2596) -- (2.9987, 1.9003, 0.3096) -- (2.9987, 1.8503, 0.3120) -- cycle;
\fill[blue!15.0, opacity=0.5] (2.9987, 1.8503, 0.3120) -- (2.9987, 1.9003, 0.3096) -- (2.9980, 1.9005, 0.3596) -- (2.9980, 1.8505, 0.3620) -- cycle;
\fill[blue!15.0, opacity=0.5] (2.9980, 1.8505, 0.3620) -- (2.9980, 1.9005, 0.3596) -- (2.9971, 1.9008, 0.4096) -- (2.9971, 1.8507, 0.4120) -- cycle;
\fill[blue!15.0, opacity=0.5] (2.9971, 1.8507, 0.4120) -- (2.9971, 1.9008, 0.4096) -- (2.9960, 1.9011, 0.4596) -- (2.9960, 1.8509, 0.4620) -- cycle;
\fill[blue!15.0, opacity=0.5] (2.9960, 1.8509, 0.4620) -- (2.9960, 1.9011, 0.4596) -- (2.9948, 1.9014, 0.5096) -- (2.9948, 1.8512, 0.5120) -- cycle;
\fill[blue!15.0, opacity=0.5] (2.9948, 1.8512, 0.5120) -- (2.9948, 1.9014, 0.5096) -- (2.9935, 1.9017, 0.5596) -- (2.9935, 1.8515, 0.5620) -- cycle;
\fill[blue!15.0, opacity=0.5] (2.9935, 1.8515, 0.5620) -- (2.9935, 1.9017, 0.5596) -- (2.9920, 1.9021, 0.6096) -- (2.9920, 1.8519, 0.6120) -- cycle;
\fill[blue!15.0, opacity=0.5] (2.9920, 1.8519, 0.6120) -- (2.9920, 1.9021, 0.6096) -- (2.9903, 1.9026, 0.6596) -- (2.9903, 1.8523, 0.6620) -- cycle;
\fill[blue!15.0, opacity=0.5] (2.9903, 1.8523, 0.6620) -- (2.9903, 1.9026, 0.6596) -- (2.9885, 1.9031, 0.7096) -- (2.9885, 1.8527, 0.7120) -- cycle;
\fill[blue!15.0, opacity=0.5] (2.9885, 1.8527, 0.7120) -- (2.9885, 1.9031, 0.7096) -- (2.9866, 1.9036, 0.7596) -- (2.9866, 1.8531, 0.7620) -- cycle;
\fill[blue!15.0, opacity=0.5] (2.9866, 1.8531, 0.7620) -- (2.9866, 1.9036, 0.7596) -- (2.9846, 1.9041, 0.8096) -- (2.9846, 1.8536, 0.8120) -- cycle;
\fill[blue!15.0, opacity=0.5] (2.9846, 1.8536, 0.8120) -- (2.9846, 1.9041, 0.8096) -- (2.9824, 1.9047, 0.8596) -- (2.9824, 1.8541, 0.8620) -- cycle;
\fill[blue!15.0, opacity=0.5] (2.9824, 1.8541, 0.8620) -- (2.9824, 1.9047, 0.8596) -- (2.9801, 1.9053, 0.9096) -- (2.9801, 1.8546, 0.9120) -- cycle;
\fill[blue!15.0, opacity=0.5] (2.9801, 1.8546, 0.9120) -- (2.9801, 1.9053, 0.9096) -- (2.9778, 1.9059, 0.9596) -- (2.9778, 1.8552, 0.9620) -- cycle;
\fill[blue!15.0, opacity=0.5] (2.9778, 1.8552, 0.9620) -- (2.9778, 1.9059, 0.9596) -- (2.9753, 1.9066, 1.0096) -- (2.9753, 1.8558, 1.0120) -- cycle;
\fill[blue!15.0, opacity=0.5] (2.9753, 1.8558, 1.0120) -- (2.9753, 1.9066, 1.0096) -- (2.9727, 1.9073, 1.0596) -- (2.9727, 1.8564, 1.0620) -- cycle;
\fill[blue!15.0, opacity=0.5] (2.9727, 1.8564, 1.0620) -- (2.9727, 1.9073, 1.0596) -- (2.9700, 1.9080, 1.1096) -- (2.9700, 1.8570, 1.1120) -- cycle;
\fill[blue!15.0, opacity=0.5] (2.9700, 1.8570, 1.1120) -- (2.9700, 1.9080, 1.1096) -- (2.9672, 1.9087, 1.1596) -- (2.9672, 1.8576, 1.1620) -- cycle;
\fill[blue!15.0, opacity=0.5] (2.9672, 1.8576, 1.1620) -- (2.9672, 1.9087, 1.1596) -- (2.9644, 1.9095, 1.2096) -- (2.9644, 1.8583, 1.2120) -- cycle;
\fill[blue!15.0, opacity=0.5] (2.9644, 1.8583, 1.2120) -- (2.9644, 1.9095, 1.2096) -- (2.9615, 1.9103, 1.2596) -- (2.9615, 1.8590, 1.2620) -- cycle;
\fill[blue!15.0, opacity=0.5] (2.9615, 1.8590, 1.2620) -- (2.9615, 1.9103, 1.2596) -- (2.9585, 1.9111, 1.3096) -- (2.9585, 1.8597, 1.3120) -- cycle;
\fill[blue!15.0, opacity=0.5] (2.9585, 1.8597, 1.3120) -- (2.9585, 1.9111, 1.3096) -- (2.9555, 1.9119, 1.3596) -- (2.9555, 1.8604, 1.3620) -- cycle;
\fill[blue!15.0, opacity=0.5] (2.9555, 1.8604, 1.3620) -- (2.9555, 1.9119, 1.3596) -- (2.9525, 1.9127, 1.4096) -- (2.9525, 1.8611, 1.4120) -- cycle;
\fill[blue!15.0, opacity=0.5] (2.9525, 1.8611, 1.4120) -- (2.9525, 1.9127, 1.4096) -- (2.9494, 1.9135, 1.4596) -- (2.9494, 1.8618, 1.4620) -- cycle;
\fill[blue!15.0, opacity=0.5] (2.9494, 1.8618, 1.4620) -- (2.9494, 1.9135, 1.4596) -- (2.9463, 1.9143, 1.5096) -- (2.9463, 1.8625, 1.5120) -- cycle;
\fill[blue!15.0, opacity=0.5] (2.9463, 1.8625, 1.5120) -- (2.9463, 1.9143, 1.5096) -- (2.9431, 1.9152, 1.5596) -- (2.9431, 1.8633, 1.5620) -- cycle;
\fill[blue!15.0, opacity=0.5] (2.9431, 1.8633, 1.5620) -- (2.9431, 1.9152, 1.5596) -- (2.9400, 1.9160, 1.6096) -- (2.9400, 1.8640, 1.6120) -- cycle;
\fill[blue!15.0, opacity=0.5] (2.9400, 1.8640, 1.6120) -- (2.9400, 1.9160, 1.6096) -- (2.9369, 1.9168, 1.6596) -- (2.9369, 1.8647, 1.6620) -- cycle;
\fill[blue!15.0, opacity=0.5] (2.9369, 1.8647, 1.6620) -- (2.9369, 1.9168, 1.6596) -- (2.9337, 1.9177, 1.7096) -- (2.9337, 1.8655, 1.7120) -- cycle;
\fill[blue!15.0, opacity=0.5] (2.9337, 1.8655, 1.7120) -- (2.9337, 1.9177, 1.7096) -- (2.9306, 1.9185, 1.7596) -- (2.9306, 1.8662, 1.7620) -- cycle;
\fill[blue!15.0, opacity=0.5] (2.9306, 1.8662, 1.7620) -- (2.9306, 1.9185, 1.7596) -- (2.9275, 1.9193, 1.8096) -- (2.9275, 1.8669, 1.8120) -- cycle;
\fill[blue!15.0, opacity=0.5] (2.9275, 1.8669, 1.8120) -- (2.9275, 1.9193, 1.8096) -- (2.9245, 1.9201, 1.8596) -- (2.9245, 1.8676, 1.8620) -- cycle;
\fill[blue!15.0, opacity=0.5] (2.9245, 1.8676, 1.8620) -- (2.9245, 1.9201, 1.8596) -- (2.9215, 1.9209, 1.9096) -- (2.9215, 1.8683, 1.9120) -- cycle;
\fill[blue!15.0, opacity=0.5] (2.9215, 1.8683, 1.9120) -- (2.9215, 1.9209, 1.9096) -- (2.9185, 1.9217, 1.9596) -- (2.9185, 1.8690, 1.9620) -- cycle;
\fill[blue!15.0, opacity=0.5] (2.9185, 1.8690, 1.9620) -- (2.9185, 1.9217, 1.9596) -- (2.9156, 1.9225, 2.0096) -- (2.9156, 1.8697, 2.0120) -- cycle;
\fill[blue!15.0, opacity=0.5] (2.9156, 1.8697, 2.0120) -- (2.9156, 1.9225, 2.0096) -- (2.9128, 1.9233, 2.0596) -- (2.9128, 1.8704, 2.0620) -- cycle;
\fill[blue!15.0, opacity=0.5] (2.9128, 1.8704, 2.0620) -- (2.9128, 1.9233, 2.0596) -- (2.9100, 1.9240, 2.1096) -- (2.9100, 1.8710, 2.1120) -- cycle;
\fill[blue!15.0, opacity=0.5] (2.9100, 1.8710, 2.1120) -- (2.9100, 1.9240, 2.1096) -- (2.9073, 1.9247, 2.1596) -- (2.9073, 1.8716, 2.1620) -- cycle;
\fill[blue!15.0, opacity=0.5] (2.9073, 1.8716, 2.1620) -- (2.9073, 1.9247, 2.1596) -- (2.9047, 1.9254, 2.2096) -- (2.9047, 1.8722, 2.2120) -- cycle;
\fill[blue!15.0, opacity=0.5] (2.9047, 1.8722, 2.2120) -- (2.9047, 1.9254, 2.2096) -- (2.9022, 1.9261, 2.2596) -- (2.9022, 1.8728, 2.2620) -- cycle;
\fill[blue!15.0, opacity=0.5] (2.9022, 1.8728, 2.2620) -- (2.9022, 1.9261, 2.2596) -- (2.8999, 1.9267, 2.3096) -- (2.8999, 1.8734, 2.3120) -- cycle;
\fill[blue!15.1, opacity=0.5] (2.8999, 1.8734, 2.3120) -- (2.8999, 1.9267, 2.3096) -- (2.8976, 1.9273, 2.3596) -- (2.8976, 1.8739, 2.3620) -- cycle;
\fill[blue!15.1, opacity=0.5] (2.8976, 1.8739, 2.3620) -- (2.8976, 1.9273, 2.3596) -- (2.8954, 1.9279, 2.4096) -- (2.8954, 1.8744, 2.4120) -- cycle;
\fill[blue!15.1, opacity=0.5] (2.8954, 1.8744, 2.4120) -- (2.8954, 1.9279, 2.4096) -- (2.8934, 1.9284, 2.4596) -- (2.8934, 1.8749, 2.4620) -- cycle;
\fill[blue!15.2, opacity=0.5] (2.8934, 1.8749, 2.4620) -- (2.8934, 1.9284, 2.4596) -- (2.8915, 1.9289, 2.5096) -- (2.8915, 1.8753, 2.5120) -- cycle;
\fill[blue!15.2, opacity=0.5] (2.8915, 1.8753, 2.5120) -- (2.8915, 1.9289, 2.5096) -- (2.8897, 1.9294, 2.5596) -- (2.8897, 1.8757, 2.5620) -- cycle;
\fill[blue!15.3, opacity=0.5] (2.8897, 1.8757, 2.5620) -- (2.8897, 1.9294, 2.5596) -- (2.8880, 1.9299, 2.6096) -- (2.8880, 1.8761, 2.6120) -- cycle;
\fill[blue!15.4, opacity=0.5] (2.8880, 1.8761, 2.6120) -- (2.8880, 1.9299, 2.6096) -- (2.8865, 1.9303, 2.6596) -- (2.8865, 1.8765, 2.6620) -- cycle;
\fill[blue!15.5, opacity=0.5] (2.8865, 1.8765, 2.6620) -- (2.8865, 1.9303, 2.6596) -- (2.8852, 1.9306, 2.7096) -- (2.8852, 1.8768, 2.7120) -- cycle;
\fill[blue!15.6, opacity=0.5] (2.8852, 1.8768, 2.7120) -- (2.8852, 1.9306, 2.7096) -- (2.8840, 1.9309, 2.7596) -- (2.8840, 1.8771, 2.7620) -- cycle;
\fill[blue!15.8, opacity=0.5] (2.8840, 1.8771, 2.7620) -- (2.8840, 1.9309, 2.7596) -- (2.8829, 1.9312, 2.8096) -- (2.8829, 1.8773, 2.8120) -- cycle;
\fill[blue!16.0, opacity=0.5] (2.8829, 1.8773, 2.8120) -- (2.8829, 1.9312, 2.8096) -- (2.8820, 1.9315, 2.8596) -- (2.8820, 1.8775, 2.8620) -- cycle;
\fill[blue!16.2, opacity=0.5] (2.8820, 1.8775, 2.8620) -- (2.8820, 1.9315, 2.8596) -- (2.8813, 1.9317, 2.9096) -- (2.8813, 1.8777, 2.9120) -- cycle;
\fill[blue!16.4, opacity=0.5] (2.8813, 1.8777, 2.9120) -- (2.8813, 1.9317, 2.9096) -- (2.8807, 1.9318, 2.9596) -- (2.8807, 1.8778, 2.9620) -- cycle;
\fill[blue!16.6, opacity=0.5] (2.8807, 1.8778, 2.9620) -- (2.8807, 1.9318, 2.9596) -- (2.8803, 1.9319, 3.0096) -- (2.8803, 1.8779, 3.0120) -- cycle;
\fill[blue!16.9, opacity=0.5] (2.8803, 1.8779, 3.0120) -- (2.8803, 1.9319, 3.0096) -- (2.8801, 1.9320, 3.0596) -- (2.8801, 1.8780, 3.0620) -- cycle;
\fill[blue!17.2, opacity=0.5] (2.8801, 1.8780, 3.0620) -- (2.8801, 1.9320, 3.0596) -- (2.8800, 1.9320, 3.1096) -- (2.8800, 1.8780, 3.1120) -- cycle;
\fill[blue!15.0, opacity=0.5] (3.0000, 1.9000, 0.1096) -- (3.0000, 1.9500, 0.1069) -- (2.9999, 1.9500, 0.1569) -- (2.9999, 1.9000, 0.1596) -- cycle;
\fill[blue!15.0, opacity=0.5] (2.9999, 1.9000, 0.1596) -- (2.9999, 1.9500, 0.1569) -- (2.9997, 1.9501, 0.2069) -- (2.9997, 1.9001, 0.2096) -- cycle;
\fill[blue!15.0, opacity=0.5] (2.9997, 1.9001, 0.2096) -- (2.9997, 1.9501, 0.2069) -- (2.9993, 1.9502, 0.2569) -- (2.9993, 1.9002, 0.2596) -- cycle;
\fill[blue!15.0, opacity=0.5] (2.9993, 1.9002, 0.2596) -- (2.9993, 1.9502, 0.2569) -- (2.9987, 1.9504, 0.3069) -- (2.9987, 1.9003, 0.3096) -- cycle;
\fill[blue!15.0, opacity=0.5] (2.9987, 1.9003, 0.3096) -- (2.9987, 1.9504, 0.3069) -- (2.9980, 1.9506, 0.3569) -- (2.9980, 1.9005, 0.3596) -- cycle;
\fill[blue!15.0, opacity=0.5] (2.9980, 1.9005, 0.3596) -- (2.9980, 1.9506, 0.3569) -- (2.9971, 1.9509, 0.4069) -- (2.9971, 1.9008, 0.4096) -- cycle;
\fill[blue!15.0, opacity=0.5] (2.9971, 1.9008, 0.4096) -- (2.9971, 1.9509, 0.4069) -- (2.9960, 1.9512, 0.4569) -- (2.9960, 1.9011, 0.4596) -- cycle;
\fill[blue!15.0, opacity=0.5] (2.9960, 1.9011, 0.4596) -- (2.9960, 1.9512, 0.4569) -- (2.9948, 1.9516, 0.5069) -- (2.9948, 1.9014, 0.5096) -- cycle;
\fill[blue!15.0, opacity=0.5] (2.9948, 1.9014, 0.5096) -- (2.9948, 1.9516, 0.5069) -- (2.9935, 1.9520, 0.5569) -- (2.9935, 1.9017, 0.5596) -- cycle;
\fill[blue!15.0, opacity=0.5] (2.9935, 1.9017, 0.5596) -- (2.9935, 1.9520, 0.5569) -- (2.9920, 1.9524, 0.6069) -- (2.9920, 1.9021, 0.6096) -- cycle;
\fill[blue!15.0, opacity=0.5] (2.9920, 1.9021, 0.6096) -- (2.9920, 1.9524, 0.6069) -- (2.9903, 1.9529, 0.6569) -- (2.9903, 1.9026, 0.6596) -- cycle;
\fill[blue!15.0, opacity=0.5] (2.9903, 1.9026, 0.6596) -- (2.9903, 1.9529, 0.6569) -- (2.9885, 1.9534, 0.7069) -- (2.9885, 1.9031, 0.7096) -- cycle;
\fill[blue!15.0, opacity=0.5] (2.9885, 1.9031, 0.7096) -- (2.9885, 1.9534, 0.7069) -- (2.9866, 1.9540, 0.7569) -- (2.9866, 1.9036, 0.7596) -- cycle;
\fill[blue!15.0, opacity=0.5] (2.9866, 1.9036, 0.7596) -- (2.9866, 1.9540, 0.7569) -- (2.9846, 1.9546, 0.8069) -- (2.9846, 1.9041, 0.8096) -- cycle;
\fill[blue!15.0, opacity=0.5] (2.9846, 1.9041, 0.8096) -- (2.9846, 1.9546, 0.8069) -- (2.9824, 1.9553, 0.8569) -- (2.9824, 1.9047, 0.8596) -- cycle;
\fill[blue!15.0, opacity=0.5] (2.9824, 1.9047, 0.8596) -- (2.9824, 1.9553, 0.8569) -- (2.9801, 1.9560, 0.9069) -- (2.9801, 1.9053, 0.9096) -- cycle;
\fill[blue!15.0, opacity=0.5] (2.9801, 1.9053, 0.9096) -- (2.9801, 1.9560, 0.9069) -- (2.9778, 1.9567, 0.9569) -- (2.9778, 1.9059, 0.9596) -- cycle;
\fill[blue!15.0, opacity=0.5] (2.9778, 1.9059, 0.9596) -- (2.9778, 1.9567, 0.9569) -- (2.9753, 1.9574, 1.0069) -- (2.9753, 1.9066, 1.0096) -- cycle;
\fill[blue!15.0, opacity=0.5] (2.9753, 1.9066, 1.0096) -- (2.9753, 1.9574, 1.0069) -- (2.9727, 1.9582, 1.0569) -- (2.9727, 1.9073, 1.0596) -- cycle;
\fill[blue!15.0, opacity=0.5] (2.9727, 1.9073, 1.0596) -- (2.9727, 1.9582, 1.0569) -- (2.9700, 1.9590, 1.1069) -- (2.9700, 1.9080, 1.1096) -- cycle;
\fill[blue!15.0, opacity=0.5] (2.9700, 1.9080, 1.1096) -- (2.9700, 1.9590, 1.1069) -- (2.9672, 1.9598, 1.1569) -- (2.9672, 1.9087, 1.1596) -- cycle;
\fill[blue!15.0, opacity=0.5] (2.9672, 1.9087, 1.1596) -- (2.9672, 1.9598, 1.1569) -- (2.9644, 1.9607, 1.2069) -- (2.9644, 1.9095, 1.2096) -- cycle;
\fill[blue!15.0, opacity=0.5] (2.9644, 1.9095, 1.2096) -- (2.9644, 1.9607, 1.2069) -- (2.9615, 1.9615, 1.2569) -- (2.9615, 1.9103, 1.2596) -- cycle;
\fill[blue!15.0, opacity=0.5] (2.9615, 1.9103, 1.2596) -- (2.9615, 1.9615, 1.2569) -- (2.9585, 1.9624, 1.3069) -- (2.9585, 1.9111, 1.3096) -- cycle;
\fill[blue!15.0, opacity=0.5] (2.9585, 1.9111, 1.3096) -- (2.9585, 1.9624, 1.3069) -- (2.9555, 1.9633, 1.3569) -- (2.9555, 1.9119, 1.3596) -- cycle;
\fill[blue!15.0, opacity=0.5] (2.9555, 1.9119, 1.3596) -- (2.9555, 1.9633, 1.3569) -- (2.9525, 1.9643, 1.4069) -- (2.9525, 1.9127, 1.4096) -- cycle;
\fill[blue!15.0, opacity=0.5] (2.9525, 1.9127, 1.4096) -- (2.9525, 1.9643, 1.4069) -- (2.9494, 1.9652, 1.4569) -- (2.9494, 1.9135, 1.4596) -- cycle;
\fill[blue!15.0, opacity=0.5] (2.9494, 1.9135, 1.4596) -- (2.9494, 1.9652, 1.4569) -- (2.9463, 1.9661, 1.5069) -- (2.9463, 1.9143, 1.5096) -- cycle;
\fill[blue!15.0, opacity=0.5] (2.9463, 1.9143, 1.5096) -- (2.9463, 1.9661, 1.5069) -- (2.9431, 1.9671, 1.5569) -- (2.9431, 1.9152, 1.5596) -- cycle;
\fill[blue!15.0, opacity=0.5] (2.9431, 1.9152, 1.5596) -- (2.9431, 1.9671, 1.5569) -- (2.9400, 1.9680, 1.6069) -- (2.9400, 1.9160, 1.6096) -- cycle;
\fill[blue!15.0, opacity=0.5] (2.9400, 1.9160, 1.6096) -- (2.9400, 1.9680, 1.6069) -- (2.9369, 1.9689, 1.6569) -- (2.9369, 1.9168, 1.6596) -- cycle;
\fill[blue!15.0, opacity=0.5] (2.9369, 1.9168, 1.6596) -- (2.9369, 1.9689, 1.6569) -- (2.9337, 1.9699, 1.7069) -- (2.9337, 1.9177, 1.7096) -- cycle;
\fill[blue!15.0, opacity=0.5] (2.9337, 1.9177, 1.7096) -- (2.9337, 1.9699, 1.7069) -- (2.9306, 1.9708, 1.7569) -- (2.9306, 1.9185, 1.7596) -- cycle;
\fill[blue!15.0, opacity=0.5] (2.9306, 1.9185, 1.7596) -- (2.9306, 1.9708, 1.7569) -- (2.9275, 1.9717, 1.8069) -- (2.9275, 1.9193, 1.8096) -- cycle;
\fill[blue!15.0, opacity=0.5] (2.9275, 1.9193, 1.8096) -- (2.9275, 1.9717, 1.8069) -- (2.9245, 1.9727, 1.8569) -- (2.9245, 1.9201, 1.8596) -- cycle;
\fill[blue!15.0, opacity=0.5] (2.9245, 1.9201, 1.8596) -- (2.9245, 1.9727, 1.8569) -- (2.9215, 1.9736, 1.9069) -- (2.9215, 1.9209, 1.9096) -- cycle;
\fill[blue!15.0, opacity=0.5] (2.9215, 1.9209, 1.9096) -- (2.9215, 1.9736, 1.9069) -- (2.9185, 1.9745, 1.9569) -- (2.9185, 1.9217, 1.9596) -- cycle;
\fill[blue!15.0, opacity=0.5] (2.9185, 1.9217, 1.9596) -- (2.9185, 1.9745, 1.9569) -- (2.9156, 1.9753, 2.0069) -- (2.9156, 1.9225, 2.0096) -- cycle;
\fill[blue!15.0, opacity=0.5] (2.9156, 1.9225, 2.0096) -- (2.9156, 1.9753, 2.0069) -- (2.9128, 1.9762, 2.0569) -- (2.9128, 1.9233, 2.0596) -- cycle;
\fill[blue!15.0, opacity=0.5] (2.9128, 1.9233, 2.0596) -- (2.9128, 1.9762, 2.0569) -- (2.9100, 1.9770, 2.1069) -- (2.9100, 1.9240, 2.1096) -- cycle;
\fill[blue!15.0, opacity=0.5] (2.9100, 1.9240, 2.1096) -- (2.9100, 1.9770, 2.1069) -- (2.9073, 1.9778, 2.1569) -- (2.9073, 1.9247, 2.1596) -- cycle;
\fill[blue!15.0, opacity=0.5] (2.9073, 1.9247, 2.1596) -- (2.9073, 1.9778, 2.1569) -- (2.9047, 1.9786, 2.2069) -- (2.9047, 1.9254, 2.2096) -- cycle;
\fill[blue!15.0, opacity=0.5] (2.9047, 1.9254, 2.2096) -- (2.9047, 1.9786, 2.2069) -- (2.9022, 1.9793, 2.2569) -- (2.9022, 1.9261, 2.2596) -- cycle;
\fill[blue!15.1, opacity=0.5] (2.9022, 1.9261, 2.2596) -- (2.9022, 1.9793, 2.2569) -- (2.8999, 1.9800, 2.3069) -- (2.8999, 1.9267, 2.3096) -- cycle;
\fill[blue!15.1, opacity=0.5] (2.8999, 1.9267, 2.3096) -- (2.8999, 1.9800, 2.3069) -- (2.8976, 1.9807, 2.3569) -- (2.8976, 1.9273, 2.3596) -- cycle;
\fill[blue!15.1, opacity=0.5] (2.8976, 1.9273, 2.3596) -- (2.8976, 1.9807, 2.3569) -- (2.8954, 1.9814, 2.4069) -- (2.8954, 1.9279, 2.4096) -- cycle;
\fill[blue!15.2, opacity=0.5] (2.8954, 1.9279, 2.4096) -- (2.8954, 1.9814, 2.4069) -- (2.8934, 1.9820, 2.4569) -- (2.8934, 1.9284, 2.4596) -- cycle;
\fill[blue!15.2, opacity=0.5] (2.8934, 1.9284, 2.4596) -- (2.8934, 1.9820, 2.4569) -- (2.8915, 1.9826, 2.5069) -- (2.8915, 1.9289, 2.5096) -- cycle;
\fill[blue!15.3, opacity=0.5] (2.8915, 1.9289, 2.5096) -- (2.8915, 1.9826, 2.5069) -- (2.8897, 1.9831, 2.5569) -- (2.8897, 1.9294, 2.5596) -- cycle;
\fill[blue!15.4, opacity=0.5] (2.8897, 1.9294, 2.5596) -- (2.8897, 1.9831, 2.5569) -- (2.8880, 1.9836, 2.6069) -- (2.8880, 1.9299, 2.6096) -- cycle;
\fill[blue!15.5, opacity=0.5] (2.8880, 1.9299, 2.6096) -- (2.8880, 1.9836, 2.6069) -- (2.8865, 1.9840, 2.6569) -- (2.8865, 1.9303, 2.6596) -- cycle;
\fill[blue!15.6, opacity=0.5] (2.8865, 1.9303, 2.6596) -- (2.8865, 1.9840, 2.6569) -- (2.8852, 1.9844, 2.7069) -- (2.8852, 1.9306, 2.7096) -- cycle;
\fill[blue!15.7, opacity=0.5] (2.8852, 1.9306, 2.7096) -- (2.8852, 1.9844, 2.7069) -- (2.8840, 1.9848, 2.7569) -- (2.8840, 1.9309, 2.7596) -- cycle;
\fill[blue!15.9, opacity=0.5] (2.8840, 1.9309, 2.7596) -- (2.8840, 1.9848, 2.7569) -- (2.8829, 1.9851, 2.8069) -- (2.8829, 1.9312, 2.8096) -- cycle;
\fill[blue!16.1, opacity=0.5] (2.8829, 1.9312, 2.8096) -- (2.8829, 1.9851, 2.8069) -- (2.8820, 1.9854, 2.8569) -- (2.8820, 1.9315, 2.8596) -- cycle;
\fill[blue!16.3, opacity=0.5] (2.8820, 1.9315, 2.8596) -- (2.8820, 1.9854, 2.8569) -- (2.8813, 1.9856, 2.9069) -- (2.8813, 1.9317, 2.9096) -- cycle;
\fill[blue!16.6, opacity=0.5] (2.8813, 1.9317, 2.9096) -- (2.8813, 1.9856, 2.9069) -- (2.8807, 1.9858, 2.9569) -- (2.8807, 1.9318, 2.9596) -- cycle;
\fill[blue!16.8, opacity=0.5] (2.8807, 1.9318, 2.9596) -- (2.8807, 1.9858, 2.9569) -- (2.8803, 1.9859, 3.0069) -- (2.8803, 1.9319, 3.0096) -- cycle;
\fill[blue!17.1, opacity=0.5] (2.8803, 1.9319, 3.0096) -- (2.8803, 1.9859, 3.0069) -- (2.8801, 1.9860, 3.0569) -- (2.8801, 1.9320, 3.0596) -- cycle;
\fill[blue!17.5, opacity=0.5] (2.8801, 1.9320, 3.0596) -- (2.8801, 1.9860, 3.0569) -- (2.8800, 1.9860, 3.1069) -- (2.8800, 1.9320, 3.1096) -- cycle;
\fill[blue!15.0, opacity=0.5] (3.0000, 1.9500, 0.1069) -- (3.0000, 2.0000, 0.1039) -- (2.9999, 2.0000, 0.1539) -- (2.9999, 1.9500, 0.1569) -- cycle;
\fill[blue!15.0, opacity=0.5] (2.9999, 1.9500, 0.1569) -- (2.9999, 2.0000, 0.1539) -- (2.9997, 2.0001, 0.2039) -- (2.9997, 1.9501, 0.2069) -- cycle;
\fill[blue!15.0, opacity=0.5] (2.9997, 1.9501, 0.2069) -- (2.9997, 2.0001, 0.2039) -- (2.9993, 2.0002, 0.2539) -- (2.9993, 1.9502, 0.2569) -- cycle;
\fill[blue!15.0, opacity=0.5] (2.9993, 1.9502, 0.2569) -- (2.9993, 2.0002, 0.2539) -- (2.9987, 2.0004, 0.3039) -- (2.9987, 1.9504, 0.3069) -- cycle;
\fill[blue!15.0, opacity=0.5] (2.9987, 1.9504, 0.3069) -- (2.9987, 2.0004, 0.3039) -- (2.9980, 2.0007, 0.3539) -- (2.9980, 1.9506, 0.3569) -- cycle;
\fill[blue!15.0, opacity=0.5] (2.9980, 1.9506, 0.3569) -- (2.9980, 2.0007, 0.3539) -- (2.9971, 2.0010, 0.4039) -- (2.9971, 1.9509, 0.4069) -- cycle;
\fill[blue!15.0, opacity=0.5] (2.9971, 1.9509, 0.4069) -- (2.9971, 2.0010, 0.4039) -- (2.9960, 2.0013, 0.4539) -- (2.9960, 1.9512, 0.4569) -- cycle;
\fill[blue!15.0, opacity=0.5] (2.9960, 1.9512, 0.4569) -- (2.9960, 2.0013, 0.4539) -- (2.9948, 2.0017, 0.5039) -- (2.9948, 1.9516, 0.5069) -- cycle;
\fill[blue!15.0, opacity=0.5] (2.9948, 1.9516, 0.5069) -- (2.9948, 2.0017, 0.5039) -- (2.9935, 2.0022, 0.5539) -- (2.9935, 1.9520, 0.5569) -- cycle;
\fill[blue!15.0, opacity=0.5] (2.9935, 1.9520, 0.5569) -- (2.9935, 2.0022, 0.5539) -- (2.9920, 2.0027, 0.6039) -- (2.9920, 1.9524, 0.6069) -- cycle;
\fill[blue!15.0, opacity=0.5] (2.9920, 1.9524, 0.6069) -- (2.9920, 2.0027, 0.6039) -- (2.9903, 2.0032, 0.6539) -- (2.9903, 1.9529, 0.6569) -- cycle;
\fill[blue!15.0, opacity=0.5] (2.9903, 1.9529, 0.6569) -- (2.9903, 2.0032, 0.6539) -- (2.9885, 2.0038, 0.7039) -- (2.9885, 1.9534, 0.7069) -- cycle;
\fill[blue!15.0, opacity=0.5] (2.9885, 1.9534, 0.7069) -- (2.9885, 2.0038, 0.7039) -- (2.9866, 2.0045, 0.7539) -- (2.9866, 1.9540, 0.7569) -- cycle;
\fill[blue!15.0, opacity=0.5] (2.9866, 1.9540, 0.7569) -- (2.9866, 2.0045, 0.7539) -- (2.9846, 2.0051, 0.8039) -- (2.9846, 1.9546, 0.8069) -- cycle;
\fill[blue!15.0, opacity=0.5] (2.9846, 1.9546, 0.8069) -- (2.9846, 2.0051, 0.8039) -- (2.9824, 2.0059, 0.8539) -- (2.9824, 1.9553, 0.8569) -- cycle;
\fill[blue!15.0, opacity=0.5] (2.9824, 1.9553, 0.8569) -- (2.9824, 2.0059, 0.8539) -- (2.9801, 2.0066, 0.9039) -- (2.9801, 1.9560, 0.9069) -- cycle;
\fill[blue!15.0, opacity=0.5] (2.9801, 1.9560, 0.9069) -- (2.9801, 2.0066, 0.9039) -- (2.9778, 2.0074, 0.9539) -- (2.9778, 1.9567, 0.9569) -- cycle;
\fill[blue!15.0, opacity=0.5] (2.9778, 1.9567, 0.9569) -- (2.9778, 2.0074, 0.9539) -- (2.9753, 2.0082, 1.0039) -- (2.9753, 1.9574, 1.0069) -- cycle;
\fill[blue!15.0, opacity=0.5] (2.9753, 1.9574, 1.0069) -- (2.9753, 2.0082, 1.0039) -- (2.9727, 2.0091, 1.0539) -- (2.9727, 1.9582, 1.0569) -- cycle;
\fill[blue!15.0, opacity=0.5] (2.9727, 1.9582, 1.0569) -- (2.9727, 2.0091, 1.0539) -- (2.9700, 2.0100, 1.1039) -- (2.9700, 1.9590, 1.1069) -- cycle;
\fill[blue!15.0, opacity=0.5] (2.9700, 1.9590, 1.1069) -- (2.9700, 2.0100, 1.1039) -- (2.9672, 2.0109, 1.1539) -- (2.9672, 1.9598, 1.1569) -- cycle;
\fill[blue!15.0, opacity=0.5] (2.9672, 1.9598, 1.1569) -- (2.9672, 2.0109, 1.1539) -- (2.9644, 2.0119, 1.2039) -- (2.9644, 1.9607, 1.2069) -- cycle;
\fill[blue!15.0, opacity=0.5] (2.9644, 1.9607, 1.2069) -- (2.9644, 2.0119, 1.2039) -- (2.9615, 2.0128, 1.2539) -- (2.9615, 1.9615, 1.2569) -- cycle;
\fill[blue!15.0, opacity=0.5] (2.9615, 1.9615, 1.2569) -- (2.9615, 2.0128, 1.2539) -- (2.9585, 2.0138, 1.3039) -- (2.9585, 1.9624, 1.3069) -- cycle;
\fill[blue!15.0, opacity=0.5] (2.9585, 1.9624, 1.3069) -- (2.9585, 2.0138, 1.3039) -- (2.9555, 2.0148, 1.3539) -- (2.9555, 1.9633, 1.3569) -- cycle;
\fill[blue!15.0, opacity=0.5] (2.9555, 1.9633, 1.3569) -- (2.9555, 2.0148, 1.3539) -- (2.9525, 2.0158, 1.4039) -- (2.9525, 1.9643, 1.4069) -- cycle;
\fill[blue!15.0, opacity=0.5] (2.9525, 1.9643, 1.4069) -- (2.9525, 2.0158, 1.4039) -- (2.9494, 2.0169, 1.4539) -- (2.9494, 1.9652, 1.4569) -- cycle;
\fill[blue!15.0, opacity=0.5] (2.9494, 1.9652, 1.4569) -- (2.9494, 2.0169, 1.4539) -- (2.9463, 2.0179, 1.5039) -- (2.9463, 1.9661, 1.5069) -- cycle;
\fill[blue!15.0, opacity=0.5] (2.9463, 1.9661, 1.5069) -- (2.9463, 2.0179, 1.5039) -- (2.9431, 2.0190, 1.5539) -- (2.9431, 1.9671, 1.5569) -- cycle;
\fill[blue!15.0, opacity=0.5] (2.9431, 1.9671, 1.5569) -- (2.9431, 2.0190, 1.5539) -- (2.9400, 2.0200, 1.6039) -- (2.9400, 1.9680, 1.6069) -- cycle;
\fill[blue!15.0, opacity=0.5] (2.9400, 1.9680, 1.6069) -- (2.9400, 2.0200, 1.6039) -- (2.9369, 2.0210, 1.6539) -- (2.9369, 1.9689, 1.6569) -- cycle;
\fill[blue!15.0, opacity=0.5] (2.9369, 1.9689, 1.6569) -- (2.9369, 2.0210, 1.6539) -- (2.9337, 2.0221, 1.7039) -- (2.9337, 1.9699, 1.7069) -- cycle;
\fill[blue!15.0, opacity=0.5] (2.9337, 1.9699, 1.7069) -- (2.9337, 2.0221, 1.7039) -- (2.9306, 2.0231, 1.7539) -- (2.9306, 1.9708, 1.7569) -- cycle;
\fill[blue!15.0, opacity=0.5] (2.9306, 1.9708, 1.7569) -- (2.9306, 2.0231, 1.7539) -- (2.9275, 2.0242, 1.8039) -- (2.9275, 1.9717, 1.8069) -- cycle;
\fill[blue!15.0, opacity=0.5] (2.9275, 1.9717, 1.8069) -- (2.9275, 2.0242, 1.8039) -- (2.9245, 2.0252, 1.8539) -- (2.9245, 1.9727, 1.8569) -- cycle;
\fill[blue!15.0, opacity=0.5] (2.9245, 1.9727, 1.8569) -- (2.9245, 2.0252, 1.8539) -- (2.9215, 2.0262, 1.9039) -- (2.9215, 1.9736, 1.9069) -- cycle;
\fill[blue!15.0, opacity=0.5] (2.9215, 1.9736, 1.9069) -- (2.9215, 2.0262, 1.9039) -- (2.9185, 2.0272, 1.9539) -- (2.9185, 1.9745, 1.9569) -- cycle;
\fill[blue!15.0, opacity=0.5] (2.9185, 1.9745, 1.9569) -- (2.9185, 2.0272, 1.9539) -- (2.9156, 2.0281, 2.0039) -- (2.9156, 1.9753, 2.0069) -- cycle;
\fill[blue!15.0, opacity=0.5] (2.9156, 1.9753, 2.0069) -- (2.9156, 2.0281, 2.0039) -- (2.9128, 2.0291, 2.0539) -- (2.9128, 1.9762, 2.0569) -- cycle;
\fill[blue!15.0, opacity=0.5] (2.9128, 1.9762, 2.0569) -- (2.9128, 2.0291, 2.0539) -- (2.9100, 2.0300, 2.1039) -- (2.9100, 1.9770, 2.1069) -- cycle;
\fill[blue!15.0, opacity=0.5] (2.9100, 1.9770, 2.1069) -- (2.9100, 2.0300, 2.1039) -- (2.9073, 2.0309, 2.1539) -- (2.9073, 1.9778, 2.1569) -- cycle;
\fill[blue!15.1, opacity=0.5] (2.9073, 1.9778, 2.1569) -- (2.9073, 2.0309, 2.1539) -- (2.9047, 2.0318, 2.2039) -- (2.9047, 1.9786, 2.2069) -- cycle;
\fill[blue!15.1, opacity=0.5] (2.9047, 1.9786, 2.2069) -- (2.9047, 2.0318, 2.2039) -- (2.9022, 2.0326, 2.2539) -- (2.9022, 1.9793, 2.2569) -- cycle;
\fill[blue!15.1, opacity=0.5] (2.9022, 1.9793, 2.2569) -- (2.9022, 2.0326, 2.2539) -- (2.8999, 2.0334, 2.3039) -- (2.8999, 1.9800, 2.3069) -- cycle;
\fill[blue!15.1, opacity=0.5] (2.8999, 1.9800, 2.3069) -- (2.8999, 2.0334, 2.3039) -- (2.8976, 2.0341, 2.3539) -- (2.8976, 1.9807, 2.3569) -- cycle;
\fill[blue!15.2, opacity=0.5] (2.8976, 1.9807, 2.3569) -- (2.8976, 2.0341, 2.3539) -- (2.8954, 2.0349, 2.4039) -- (2.8954, 1.9814, 2.4069) -- cycle;
\fill[blue!15.3, opacity=0.5] (2.8954, 1.9814, 2.4069) -- (2.8954, 2.0349, 2.4039) -- (2.8934, 2.0355, 2.4539) -- (2.8934, 1.9820, 2.4569) -- cycle;
\fill[blue!15.3, opacity=0.5] (2.8934, 1.9820, 2.4569) -- (2.8934, 2.0355, 2.4539) -- (2.8915, 2.0362, 2.5039) -- (2.8915, 1.9826, 2.5069) -- cycle;
\fill[blue!15.5, opacity=0.5] (2.8915, 1.9826, 2.5069) -- (2.8915, 2.0362, 2.5039) -- (2.8897, 2.0368, 2.5539) -- (2.8897, 1.9831, 2.5569) -- cycle;
\fill[blue!15.6, opacity=0.5] (2.8897, 1.9831, 2.5569) -- (2.8897, 2.0368, 2.5539) -- (2.8880, 2.0373, 2.6039) -- (2.8880, 1.9836, 2.6069) -- cycle;
\fill[blue!15.7, opacity=0.5] (2.8880, 1.9836, 2.6069) -- (2.8880, 2.0373, 2.6039) -- (2.8865, 2.0378, 2.6539) -- (2.8865, 1.9840, 2.6569) -- cycle;
\fill[blue!15.9, opacity=0.5] (2.8865, 1.9840, 2.6569) -- (2.8865, 2.0378, 2.6539) -- (2.8852, 2.0383, 2.7039) -- (2.8852, 1.9844, 2.7069) -- cycle;
\fill[blue!16.1, opacity=0.5] (2.8852, 1.9844, 2.7069) -- (2.8852, 2.0383, 2.7039) -- (2.8840, 2.0387, 2.7539) -- (2.8840, 1.9848, 2.7569) -- cycle;
\fill[blue!16.3, opacity=0.5] (2.8840, 1.9848, 2.7569) -- (2.8840, 2.0387, 2.7539) -- (2.8829, 2.0390, 2.8039) -- (2.8829, 1.9851, 2.8069) -- cycle;
\fill[blue!16.6, opacity=0.5] (2.8829, 1.9851, 2.8069) -- (2.8829, 2.0390, 2.8039) -- (2.8820, 2.0393, 2.8539) -- (2.8820, 1.9854, 2.8569) -- cycle;
\fill[blue!16.9, opacity=0.5] (2.8820, 1.9854, 2.8569) -- (2.8820, 2.0393, 2.8539) -- (2.8813, 2.0396, 2.9039) -- (2.8813, 1.9856, 2.9069) -- cycle;
\fill[blue!17.2, opacity=0.5] (2.8813, 1.9856, 2.9069) -- (2.8813, 2.0396, 2.9039) -- (2.8807, 2.0398, 2.9539) -- (2.8807, 1.9858, 2.9569) -- cycle;
\fill[blue!17.5, opacity=0.5] (2.8807, 1.9858, 2.9569) -- (2.8807, 2.0398, 2.9539) -- (2.8803, 2.0399, 3.0039) -- (2.8803, 1.9859, 3.0069) -- cycle;
\fill[blue!17.9, opacity=0.5] (2.8803, 1.9859, 3.0069) -- (2.8803, 2.0399, 3.0039) -- (2.8801, 2.0400, 3.0539) -- (2.8801, 1.9860, 3.0569) -- cycle;
\fill[blue!18.3, opacity=0.5] (2.8801, 1.9860, 3.0569) -- (2.8801, 2.0400, 3.0539) -- (2.8800, 2.0400, 3.1039) -- (2.8800, 1.9860, 3.1069) -- cycle;
\fill[blue!15.0, opacity=0.5] (3.0000, 2.0000, 0.1039) -- (3.0000, 2.0500, 0.1006) -- (2.9999, 2.0500, 0.1506) -- (2.9999, 2.0000, 0.1539) -- cycle;
\fill[blue!15.0, opacity=0.5] (2.9999, 2.0000, 0.1539) -- (2.9999, 2.0500, 0.1506) -- (2.9997, 2.0501, 0.2006) -- (2.9997, 2.0001, 0.2039) -- cycle;
\fill[blue!15.0, opacity=0.5] (2.9997, 2.0001, 0.2039) -- (2.9997, 2.0501, 0.2006) -- (2.9993, 2.0503, 0.2506) -- (2.9993, 2.0002, 0.2539) -- cycle;
\fill[blue!15.0, opacity=0.5] (2.9993, 2.0002, 0.2539) -- (2.9993, 2.0503, 0.2506) -- (2.9987, 2.0505, 0.3006) -- (2.9987, 2.0004, 0.3039) -- cycle;
\fill[blue!15.0, opacity=0.5] (2.9987, 2.0004, 0.3039) -- (2.9987, 2.0505, 0.3006) -- (2.9980, 2.0507, 0.3506) -- (2.9980, 2.0007, 0.3539) -- cycle;
\fill[blue!15.0, opacity=0.5] (2.9980, 2.0007, 0.3539) -- (2.9980, 2.0507, 0.3506) -- (2.9971, 2.0511, 0.4006) -- (2.9971, 2.0010, 0.4039) -- cycle;
\fill[blue!15.0, opacity=0.5] (2.9971, 2.0010, 0.4039) -- (2.9971, 2.0511, 0.4006) -- (2.9960, 2.0515, 0.4506) -- (2.9960, 2.0013, 0.4539) -- cycle;
\fill[blue!15.0, opacity=0.5] (2.9960, 2.0013, 0.4539) -- (2.9960, 2.0515, 0.4506) -- (2.9948, 2.0519, 0.5006) -- (2.9948, 2.0017, 0.5039) -- cycle;
\fill[blue!15.0, opacity=0.5] (2.9948, 2.0017, 0.5039) -- (2.9948, 2.0519, 0.5006) -- (2.9935, 2.0524, 0.5506) -- (2.9935, 2.0022, 0.5539) -- cycle;
\fill[blue!15.0, opacity=0.5] (2.9935, 2.0022, 0.5539) -- (2.9935, 2.0524, 0.5506) -- (2.9920, 2.0529, 0.6006) -- (2.9920, 2.0027, 0.6039) -- cycle;
\fill[blue!15.0, opacity=0.5] (2.9920, 2.0027, 0.6039) -- (2.9920, 2.0529, 0.6006) -- (2.9903, 2.0535, 0.6506) -- (2.9903, 2.0032, 0.6539) -- cycle;
\fill[blue!15.0, opacity=0.5] (2.9903, 2.0032, 0.6539) -- (2.9903, 2.0535, 0.6506) -- (2.9885, 2.0542, 0.7006) -- (2.9885, 2.0038, 0.7039) -- cycle;
\fill[blue!15.0, opacity=0.5] (2.9885, 2.0038, 0.7039) -- (2.9885, 2.0542, 0.7006) -- (2.9866, 2.0549, 0.7506) -- (2.9866, 2.0045, 0.7539) -- cycle;
\fill[blue!15.0, opacity=0.5] (2.9866, 2.0045, 0.7539) -- (2.9866, 2.0549, 0.7506) -- (2.9846, 2.0557, 0.8006) -- (2.9846, 2.0051, 0.8039) -- cycle;
\fill[blue!15.0, opacity=0.5] (2.9846, 2.0051, 0.8039) -- (2.9846, 2.0557, 0.8006) -- (2.9824, 2.0564, 0.8506) -- (2.9824, 2.0059, 0.8539) -- cycle;
\fill[blue!15.0, opacity=0.5] (2.9824, 2.0059, 0.8539) -- (2.9824, 2.0564, 0.8506) -- (2.9801, 2.0573, 0.9006) -- (2.9801, 2.0066, 0.9039) -- cycle;
\fill[blue!15.0, opacity=0.5] (2.9801, 2.0066, 0.9039) -- (2.9801, 2.0573, 0.9006) -- (2.9778, 2.0582, 0.9506) -- (2.9778, 2.0074, 0.9539) -- cycle;
\fill[blue!15.0, opacity=0.5] (2.9778, 2.0074, 0.9539) -- (2.9778, 2.0582, 0.9506) -- (2.9753, 2.0591, 1.0006) -- (2.9753, 2.0082, 1.0039) -- cycle;
\fill[blue!15.0, opacity=0.5] (2.9753, 2.0082, 1.0039) -- (2.9753, 2.0591, 1.0006) -- (2.9727, 2.0600, 1.0506) -- (2.9727, 2.0091, 1.0539) -- cycle;
\fill[blue!15.0, opacity=0.5] (2.9727, 2.0091, 1.0539) -- (2.9727, 2.0600, 1.0506) -- (2.9700, 2.0610, 1.1006) -- (2.9700, 2.0100, 1.1039) -- cycle;
\fill[blue!15.0, opacity=0.5] (2.9700, 2.0100, 1.1039) -- (2.9700, 2.0610, 1.1006) -- (2.9672, 2.0620, 1.1506) -- (2.9672, 2.0109, 1.1539) -- cycle;
\fill[blue!15.0, opacity=0.5] (2.9672, 2.0109, 1.1539) -- (2.9672, 2.0620, 1.1506) -- (2.9644, 2.0631, 1.2006) -- (2.9644, 2.0119, 1.2039) -- cycle;
\fill[blue!15.0, opacity=0.5] (2.9644, 2.0119, 1.2039) -- (2.9644, 2.0631, 1.2006) -- (2.9615, 2.0641, 1.2506) -- (2.9615, 2.0128, 1.2539) -- cycle;
\fill[blue!15.0, opacity=0.5] (2.9615, 2.0128, 1.2539) -- (2.9615, 2.0641, 1.2506) -- (2.9585, 2.0652, 1.3006) -- (2.9585, 2.0138, 1.3039) -- cycle;
\fill[blue!15.0, opacity=0.5] (2.9585, 2.0138, 1.3039) -- (2.9585, 2.0652, 1.3006) -- (2.9555, 2.0663, 1.3506) -- (2.9555, 2.0148, 1.3539) -- cycle;
\fill[blue!15.0, opacity=0.5] (2.9555, 2.0148, 1.3539) -- (2.9555, 2.0663, 1.3506) -- (2.9525, 2.0674, 1.4006) -- (2.9525, 2.0158, 1.4039) -- cycle;
\fill[blue!15.0, opacity=0.5] (2.9525, 2.0158, 1.4039) -- (2.9525, 2.0674, 1.4006) -- (2.9494, 2.0686, 1.4506) -- (2.9494, 2.0169, 1.4539) -- cycle;
\fill[blue!15.0, opacity=0.5] (2.9494, 2.0169, 1.4539) -- (2.9494, 2.0686, 1.4506) -- (2.9463, 2.0697, 1.5006) -- (2.9463, 2.0179, 1.5039) -- cycle;
\fill[blue!15.0, opacity=0.5] (2.9463, 2.0179, 1.5039) -- (2.9463, 2.0697, 1.5006) -- (2.9431, 2.0708, 1.5506) -- (2.9431, 2.0190, 1.5539) -- cycle;
\fill[blue!15.0, opacity=0.5] (2.9431, 2.0190, 1.5539) -- (2.9431, 2.0708, 1.5506) -- (2.9400, 2.0720, 1.6006) -- (2.9400, 2.0200, 1.6039) -- cycle;
\fill[blue!15.0, opacity=0.5] (2.9400, 2.0200, 1.6039) -- (2.9400, 2.0720, 1.6006) -- (2.9369, 2.0732, 1.6506) -- (2.9369, 2.0210, 1.6539) -- cycle;
\fill[blue!15.0, opacity=0.5] (2.9369, 2.0210, 1.6539) -- (2.9369, 2.0732, 1.6506) -- (2.9337, 2.0743, 1.7006) -- (2.9337, 2.0221, 1.7039) -- cycle;
\fill[blue!15.0, opacity=0.5] (2.9337, 2.0221, 1.7039) -- (2.9337, 2.0743, 1.7006) -- (2.9306, 2.0754, 1.7506) -- (2.9306, 2.0231, 1.7539) -- cycle;
\fill[blue!15.0, opacity=0.5] (2.9306, 2.0231, 1.7539) -- (2.9306, 2.0754, 1.7506) -- (2.9275, 2.0766, 1.8006) -- (2.9275, 2.0242, 1.8039) -- cycle;
\fill[blue!15.0, opacity=0.5] (2.9275, 2.0242, 1.8039) -- (2.9275, 2.0766, 1.8006) -- (2.9245, 2.0777, 1.8506) -- (2.9245, 2.0252, 1.8539) -- cycle;
\fill[blue!15.0, opacity=0.5] (2.9245, 2.0252, 1.8539) -- (2.9245, 2.0777, 1.8506) -- (2.9215, 2.0788, 1.9006) -- (2.9215, 2.0262, 1.9039) -- cycle;
\fill[blue!15.0, opacity=0.5] (2.9215, 2.0262, 1.9039) -- (2.9215, 2.0788, 1.9006) -- (2.9185, 2.0799, 1.9506) -- (2.9185, 2.0272, 1.9539) -- cycle;
\fill[blue!15.0, opacity=0.5] (2.9185, 2.0272, 1.9539) -- (2.9185, 2.0799, 1.9506) -- (2.9156, 2.0809, 2.0006) -- (2.9156, 2.0281, 2.0039) -- cycle;
\fill[blue!15.0, opacity=0.5] (2.9156, 2.0281, 2.0039) -- (2.9156, 2.0809, 2.0006) -- (2.9128, 2.0820, 2.0506) -- (2.9128, 2.0291, 2.0539) -- cycle;
\fill[blue!15.1, opacity=0.5] (2.9128, 2.0291, 2.0539) -- (2.9128, 2.0820, 2.0506) -- (2.9100, 2.0830, 2.1006) -- (2.9100, 2.0300, 2.1039) -- cycle;
\fill[blue!15.1, opacity=0.5] (2.9100, 2.0300, 2.1039) -- (2.9100, 2.0830, 2.1006) -- (2.9073, 2.0840, 2.1506) -- (2.9073, 2.0309, 2.1539) -- cycle;
\fill[blue!15.1, opacity=0.5] (2.9073, 2.0309, 2.1539) -- (2.9073, 2.0840, 2.1506) -- (2.9047, 2.0849, 2.2006) -- (2.9047, 2.0318, 2.2039) -- cycle;
\fill[blue!15.2, opacity=0.5] (2.9047, 2.0318, 2.2039) -- (2.9047, 2.0849, 2.2006) -- (2.9022, 2.0858, 2.2506) -- (2.9022, 2.0326, 2.2539) -- cycle;
\fill[blue!15.3, opacity=0.5] (2.9022, 2.0326, 2.2539) -- (2.9022, 2.0858, 2.2506) -- (2.8999, 2.0867, 2.3006) -- (2.8999, 2.0334, 2.3039) -- cycle;
\fill[blue!15.4, opacity=0.5] (2.8999, 2.0334, 2.3039) -- (2.8999, 2.0867, 2.3006) -- (2.8976, 2.0876, 2.3506) -- (2.8976, 2.0341, 2.3539) -- cycle;
\fill[blue!15.5, opacity=0.5] (2.8976, 2.0341, 2.3539) -- (2.8976, 2.0876, 2.3506) -- (2.8954, 2.0883, 2.4006) -- (2.8954, 2.0349, 2.4039) -- cycle;
\fill[blue!15.6, opacity=0.5] (2.8954, 2.0349, 2.4039) -- (2.8954, 2.0883, 2.4006) -- (2.8934, 2.0891, 2.4506) -- (2.8934, 2.0355, 2.4539) -- cycle;
\fill[blue!15.8, opacity=0.5] (2.8934, 2.0355, 2.4539) -- (2.8934, 2.0891, 2.4506) -- (2.8915, 2.0898, 2.5006) -- (2.8915, 2.0362, 2.5039) -- cycle;
\fill[blue!15.9, opacity=0.5] (2.8915, 2.0362, 2.5039) -- (2.8915, 2.0898, 2.5006) -- (2.8897, 2.0905, 2.5506) -- (2.8897, 2.0368, 2.5539) -- cycle;
\fill[blue!16.2, opacity=0.5] (2.8897, 2.0368, 2.5539) -- (2.8897, 2.0905, 2.5506) -- (2.8880, 2.0911, 2.6006) -- (2.8880, 2.0373, 2.6039) -- cycle;
\fill[blue!16.4, opacity=0.5] (2.8880, 2.0373, 2.6039) -- (2.8880, 2.0911, 2.6006) -- (2.8865, 2.0916, 2.6506) -- (2.8865, 2.0378, 2.6539) -- cycle;
\fill[blue!16.7, opacity=0.5] (2.8865, 2.0378, 2.6539) -- (2.8865, 2.0916, 2.6506) -- (2.8852, 2.0921, 2.7006) -- (2.8852, 2.0383, 2.7039) -- cycle;
\fill[blue!17.0, opacity=0.5] (2.8852, 2.0383, 2.7039) -- (2.8852, 2.0921, 2.7006) -- (2.8840, 2.0925, 2.7506) -- (2.8840, 2.0387, 2.7539) -- cycle;
\fill[blue!17.4, opacity=0.5] (2.8840, 2.0387, 2.7539) -- (2.8840, 2.0925, 2.7506) -- (2.8829, 2.0929, 2.8006) -- (2.8829, 2.0390, 2.8039) -- cycle;
\fill[blue!17.8, opacity=0.5] (2.8829, 2.0390, 2.8039) -- (2.8829, 2.0929, 2.8006) -- (2.8820, 2.0933, 2.8506) -- (2.8820, 2.0393, 2.8539) -- cycle;
\fill[blue!18.3, opacity=0.5] (2.8820, 2.0393, 2.8539) -- (2.8820, 2.0933, 2.8506) -- (2.8813, 2.0935, 2.9006) -- (2.8813, 2.0396, 2.9039) -- cycle;
\fill[blue!18.7, opacity=0.5] (2.8813, 2.0396, 2.9039) -- (2.8813, 2.0935, 2.9006) -- (2.8807, 2.0937, 2.9506) -- (2.8807, 2.0398, 2.9539) -- cycle;
\fill[blue!19.2, opacity=0.5] (2.8807, 2.0398, 2.9539) -- (2.8807, 2.0937, 2.9506) -- (2.8803, 2.0939, 3.0006) -- (2.8803, 2.0399, 3.0039) -- cycle;
\fill[blue!19.8, opacity=0.5] (2.8803, 2.0399, 3.0039) -- (2.8803, 2.0939, 3.0006) -- (2.8801, 2.0940, 3.0506) -- (2.8801, 2.0400, 3.0539) -- cycle;
\fill[blue!20.4, opacity=0.5] (2.8801, 2.0400, 3.0539) -- (2.8801, 2.0940, 3.0506) -- (2.8800, 2.0940, 3.1006) -- (2.8800, 2.0400, 3.1039) -- cycle;
\fill[blue!15.0, opacity=0.5] (3.0000, 2.0500, 0.1006) -- (3.0000, 2.1000, 0.0971) -- (2.9999, 2.1000, 0.1471) -- (2.9999, 2.0500, 0.1506) -- cycle;
\fill[blue!15.0, opacity=0.5] (2.9999, 2.0500, 0.1506) -- (2.9999, 2.1000, 0.1471) -- (2.9997, 2.1001, 0.1971) -- (2.9997, 2.0501, 0.2006) -- cycle;
\fill[blue!15.0, opacity=0.5] (2.9997, 2.0501, 0.2006) -- (2.9997, 2.1001, 0.1971) -- (2.9993, 2.1003, 0.2471) -- (2.9993, 2.0503, 0.2506) -- cycle;
\fill[blue!15.0, opacity=0.5] (2.9993, 2.0503, 0.2506) -- (2.9993, 2.1003, 0.2471) -- (2.9987, 2.1005, 0.2971) -- (2.9987, 2.0505, 0.3006) -- cycle;
\fill[blue!15.0, opacity=0.5] (2.9987, 2.0505, 0.3006) -- (2.9987, 2.1005, 0.2971) -- (2.9980, 2.1008, 0.3471) -- (2.9980, 2.0507, 0.3506) -- cycle;
\fill[blue!15.0, opacity=0.5] (2.9980, 2.0507, 0.3506) -- (2.9980, 2.1008, 0.3471) -- (2.9971, 2.1012, 0.3971) -- (2.9971, 2.0511, 0.4006) -- cycle;
\fill[blue!15.0, opacity=0.5] (2.9971, 2.0511, 0.4006) -- (2.9971, 2.1012, 0.3971) -- (2.9960, 2.1016, 0.4471) -- (2.9960, 2.0515, 0.4506) -- cycle;
\fill[blue!15.0, opacity=0.5] (2.9960, 2.0515, 0.4506) -- (2.9960, 2.1016, 0.4471) -- (2.9948, 2.1021, 0.4971) -- (2.9948, 2.0519, 0.5006) -- cycle;
\fill[blue!15.0, opacity=0.5] (2.9948, 2.0519, 0.5006) -- (2.9948, 2.1021, 0.4971) -- (2.9935, 2.1026, 0.5471) -- (2.9935, 2.0524, 0.5506) -- cycle;
\fill[blue!15.0, opacity=0.5] (2.9935, 2.0524, 0.5506) -- (2.9935, 2.1026, 0.5471) -- (2.9920, 2.1032, 0.5971) -- (2.9920, 2.0529, 0.6006) -- cycle;
\fill[blue!15.0, opacity=0.5] (2.9920, 2.0529, 0.6006) -- (2.9920, 2.1032, 0.5971) -- (2.9903, 2.1039, 0.6471) -- (2.9903, 2.0535, 0.6506) -- cycle;
\fill[blue!15.0, opacity=0.5] (2.9903, 2.0535, 0.6506) -- (2.9903, 2.1039, 0.6471) -- (2.9885, 2.1046, 0.6971) -- (2.9885, 2.0542, 0.7006) -- cycle;
\fill[blue!15.0, opacity=0.5] (2.9885, 2.0542, 0.7006) -- (2.9885, 2.1046, 0.6971) -- (2.9866, 2.1053, 0.7471) -- (2.9866, 2.0549, 0.7506) -- cycle;
\fill[blue!15.0, opacity=0.5] (2.9866, 2.0549, 0.7506) -- (2.9866, 2.1053, 0.7471) -- (2.9846, 2.1062, 0.7971) -- (2.9846, 2.0557, 0.8006) -- cycle;
\fill[blue!15.0, opacity=0.5] (2.9846, 2.0557, 0.8006) -- (2.9846, 2.1062, 0.7971) -- (2.9824, 2.1070, 0.8471) -- (2.9824, 2.0564, 0.8506) -- cycle;
\fill[blue!15.0, opacity=0.5] (2.9824, 2.0564, 0.8506) -- (2.9824, 2.1070, 0.8471) -- (2.9801, 2.1079, 0.8971) -- (2.9801, 2.0573, 0.9006) -- cycle;
\fill[blue!15.0, opacity=0.5] (2.9801, 2.0573, 0.9006) -- (2.9801, 2.1079, 0.8971) -- (2.9778, 2.1089, 0.9471) -- (2.9778, 2.0582, 0.9506) -- cycle;
\fill[blue!15.0, opacity=0.5] (2.9778, 2.0582, 0.9506) -- (2.9778, 2.1089, 0.9471) -- (2.9753, 2.1099, 0.9971) -- (2.9753, 2.0591, 1.0006) -- cycle;
\fill[blue!15.0, opacity=0.5] (2.9753, 2.0591, 1.0006) -- (2.9753, 2.1099, 0.9971) -- (2.9727, 2.1109, 1.0471) -- (2.9727, 2.0600, 1.0506) -- cycle;
\fill[blue!15.0, opacity=0.5] (2.9727, 2.0600, 1.0506) -- (2.9727, 2.1109, 1.0471) -- (2.9700, 2.1120, 1.0971) -- (2.9700, 2.0610, 1.1006) -- cycle;
\fill[blue!15.0, opacity=0.5] (2.9700, 2.0610, 1.1006) -- (2.9700, 2.1120, 1.0971) -- (2.9672, 2.1131, 1.1471) -- (2.9672, 2.0620, 1.1506) -- cycle;
\fill[blue!15.0, opacity=0.5] (2.9672, 2.0620, 1.1506) -- (2.9672, 2.1131, 1.1471) -- (2.9644, 2.1142, 1.1971) -- (2.9644, 2.0631, 1.2006) -- cycle;
\fill[blue!15.0, opacity=0.5] (2.9644, 2.0631, 1.2006) -- (2.9644, 2.1142, 1.1971) -- (2.9615, 2.1154, 1.2471) -- (2.9615, 2.0641, 1.2506) -- cycle;
\fill[blue!15.0, opacity=0.5] (2.9615, 2.0641, 1.2506) -- (2.9615, 2.1154, 1.2471) -- (2.9585, 2.1166, 1.2971) -- (2.9585, 2.0652, 1.3006) -- cycle;
\fill[blue!15.0, opacity=0.5] (2.9585, 2.0652, 1.3006) -- (2.9585, 2.1166, 1.2971) -- (2.9555, 2.1178, 1.3471) -- (2.9555, 2.0663, 1.3506) -- cycle;
\fill[blue!15.0, opacity=0.5] (2.9555, 2.0663, 1.3506) -- (2.9555, 2.1178, 1.3471) -- (2.9525, 2.1190, 1.3971) -- (2.9525, 2.0674, 1.4006) -- cycle;
\fill[blue!15.0, opacity=0.5] (2.9525, 2.0674, 1.4006) -- (2.9525, 2.1190, 1.3971) -- (2.9494, 2.1202, 1.4471) -- (2.9494, 2.0686, 1.4506) -- cycle;
\fill[blue!15.0, opacity=0.5] (2.9494, 2.0686, 1.4506) -- (2.9494, 2.1202, 1.4471) -- (2.9463, 2.1215, 1.4971) -- (2.9463, 2.0697, 1.5006) -- cycle;
\fill[blue!15.0, opacity=0.5] (2.9463, 2.0697, 1.5006) -- (2.9463, 2.1215, 1.4971) -- (2.9431, 2.1227, 1.5471) -- (2.9431, 2.0708, 1.5506) -- cycle;
\fill[blue!15.0, opacity=0.5] (2.9431, 2.0708, 1.5506) -- (2.9431, 2.1227, 1.5471) -- (2.9400, 2.1240, 1.5971) -- (2.9400, 2.0720, 1.6006) -- cycle;
\fill[blue!15.0, opacity=0.5] (2.9400, 2.0720, 1.6006) -- (2.9400, 2.1240, 1.5971) -- (2.9369, 2.1253, 1.6471) -- (2.9369, 2.0732, 1.6506) -- cycle;
\fill[blue!15.0, opacity=0.5] (2.9369, 2.0732, 1.6506) -- (2.9369, 2.1253, 1.6471) -- (2.9337, 2.1265, 1.6971) -- (2.9337, 2.0743, 1.7006) -- cycle;
\fill[blue!15.0, opacity=0.5] (2.9337, 2.0743, 1.7006) -- (2.9337, 2.1265, 1.6971) -- (2.9306, 2.1278, 1.7471) -- (2.9306, 2.0754, 1.7506) -- cycle;
\fill[blue!15.0, opacity=0.5] (2.9306, 2.0754, 1.7506) -- (2.9306, 2.1278, 1.7471) -- (2.9275, 2.1290, 1.7971) -- (2.9275, 2.0766, 1.8006) -- cycle;
\fill[blue!15.0, opacity=0.5] (2.9275, 2.0766, 1.8006) -- (2.9275, 2.1290, 1.7971) -- (2.9245, 2.1302, 1.8471) -- (2.9245, 2.0777, 1.8506) -- cycle;
\fill[blue!15.1, opacity=0.5] (2.9245, 2.0777, 1.8506) -- (2.9245, 2.1302, 1.8471) -- (2.9215, 2.1314, 1.8971) -- (2.9215, 2.0788, 1.9006) -- cycle;
\fill[blue!15.1, opacity=0.5] (2.9215, 2.0788, 1.9006) -- (2.9215, 2.1314, 1.8971) -- (2.9185, 2.1326, 1.9471) -- (2.9185, 2.0799, 1.9506) -- cycle;
\fill[blue!15.1, opacity=0.5] (2.9185, 2.0799, 1.9506) -- (2.9185, 2.1326, 1.9471) -- (2.9156, 2.1338, 1.9971) -- (2.9156, 2.0809, 2.0006) -- cycle;
\fill[blue!15.2, opacity=0.5] (2.9156, 2.0809, 2.0006) -- (2.9156, 2.1338, 1.9971) -- (2.9128, 2.1349, 2.0471) -- (2.9128, 2.0820, 2.0506) -- cycle;
\fill[blue!15.3, opacity=0.5] (2.9128, 2.0820, 2.0506) -- (2.9128, 2.1349, 2.0471) -- (2.9100, 2.1360, 2.0971) -- (2.9100, 2.0830, 2.1006) -- cycle;
\fill[blue!15.4, opacity=0.5] (2.9100, 2.0830, 2.1006) -- (2.9100, 2.1360, 2.0971) -- (2.9073, 2.1371, 2.1471) -- (2.9073, 2.0840, 2.1506) -- cycle;
\fill[blue!15.5, opacity=0.5] (2.9073, 2.0840, 2.1506) -- (2.9073, 2.1371, 2.1471) -- (2.9047, 2.1381, 2.1971) -- (2.9047, 2.0849, 2.2006) -- cycle;
\fill[blue!15.6, opacity=0.5] (2.9047, 2.0849, 2.2006) -- (2.9047, 2.1381, 2.1971) -- (2.9022, 2.1391, 2.2471) -- (2.9022, 2.0858, 2.2506) -- cycle;
\fill[blue!15.8, opacity=0.5] (2.9022, 2.0858, 2.2506) -- (2.9022, 2.1391, 2.2471) -- (2.8999, 2.1401, 2.2971) -- (2.8999, 2.0867, 2.3006) -- cycle;
\fill[blue!16.0, opacity=0.5] (2.8999, 2.0867, 2.3006) -- (2.8999, 2.1401, 2.2971) -- (2.8976, 2.1410, 2.3471) -- (2.8976, 2.0876, 2.3506) -- cycle;
\fill[blue!16.3, opacity=0.5] (2.8976, 2.0876, 2.3506) -- (2.8976, 2.1410, 2.3471) -- (2.8954, 2.1418, 2.3971) -- (2.8954, 2.0883, 2.4006) -- cycle;
\fill[blue!16.6, opacity=0.5] (2.8954, 2.0883, 2.4006) -- (2.8954, 2.1418, 2.3971) -- (2.8934, 2.1427, 2.4471) -- (2.8934, 2.0891, 2.4506) -- cycle;
\fill[blue!17.0, opacity=0.5] (2.8934, 2.0891, 2.4506) -- (2.8934, 2.1427, 2.4471) -- (2.8915, 2.1434, 2.4971) -- (2.8915, 2.0898, 2.5006) -- cycle;
\fill[blue!17.4, opacity=0.5] (2.8915, 2.0898, 2.5006) -- (2.8915, 2.1434, 2.4971) -- (2.8897, 2.1441, 2.5471) -- (2.8897, 2.0905, 2.5506) -- cycle;
\fill[blue!17.8, opacity=0.5] (2.8897, 2.0905, 2.5506) -- (2.8897, 2.1441, 2.5471) -- (2.8880, 2.1448, 2.5971) -- (2.8880, 2.0911, 2.6006) -- cycle;
\fill[blue!18.3, opacity=0.5] (2.8880, 2.0911, 2.6006) -- (2.8880, 2.1448, 2.5971) -- (2.8865, 2.1454, 2.6471) -- (2.8865, 2.0916, 2.6506) -- cycle;
\fill[blue!18.8, opacity=0.5] (2.8865, 2.0916, 2.6506) -- (2.8865, 2.1454, 2.6471) -- (2.8852, 2.1459, 2.6971) -- (2.8852, 2.0921, 2.7006) -- cycle;
\fill[blue!19.4, opacity=0.5] (2.8852, 2.0921, 2.7006) -- (2.8852, 2.1459, 2.6971) -- (2.8840, 2.1464, 2.7471) -- (2.8840, 2.0925, 2.7506) -- cycle;
\fill[blue!20.1, opacity=0.5] (2.8840, 2.0925, 2.7506) -- (2.8840, 2.1464, 2.7471) -- (2.8829, 2.1468, 2.7971) -- (2.8829, 2.0929, 2.8006) -- cycle;
\fill[blue!20.7, opacity=0.5] (2.8829, 2.0929, 2.8006) -- (2.8829, 2.1468, 2.7971) -- (2.8820, 2.1472, 2.8471) -- (2.8820, 2.0933, 2.8506) -- cycle;
\fill[blue!21.4, opacity=0.5] (2.8820, 2.0933, 2.8506) -- (2.8820, 2.1472, 2.8471) -- (2.8813, 2.1475, 2.8971) -- (2.8813, 2.0935, 2.9006) -- cycle;
\fill[blue!22.2, opacity=0.5] (2.8813, 2.0935, 2.9006) -- (2.8813, 2.1475, 2.8971) -- (2.8807, 2.1477, 2.9471) -- (2.8807, 2.0937, 2.9506) -- cycle;
\fill[blue!22.9, opacity=0.5] (2.8807, 2.0937, 2.9506) -- (2.8807, 2.1477, 2.9471) -- (2.8803, 2.1479, 2.9971) -- (2.8803, 2.0939, 3.0006) -- cycle;
\fill[blue!23.7, opacity=0.5] (2.8803, 2.0939, 3.0006) -- (2.8803, 2.1479, 2.9971) -- (2.8801, 2.1480, 3.0471) -- (2.8801, 2.0940, 3.0506) -- cycle;
\fill[blue!24.5, opacity=0.5] (2.8801, 2.0940, 3.0506) -- (2.8801, 2.1480, 3.0471) -- (2.8800, 2.1480, 3.0971) -- (2.8800, 2.0940, 3.1006) -- cycle;
\fill[blue!15.0, opacity=0.5] (3.0000, 2.1000, 0.0971) -- (3.0000, 2.1500, 0.0933) -- (2.9999, 2.1500, 0.1433) -- (2.9999, 2.1000, 0.1471) -- cycle;
\fill[blue!15.0, opacity=0.5] (2.9999, 2.1000, 0.1471) -- (2.9999, 2.1500, 0.1433) -- (2.9997, 2.1501, 0.1933) -- (2.9997, 2.1001, 0.1971) -- cycle;
\fill[blue!15.0, opacity=0.5] (2.9997, 2.1001, 0.1971) -- (2.9997, 2.1501, 0.1933) -- (2.9993, 2.1503, 0.2433) -- (2.9993, 2.1003, 0.2471) -- cycle;
\fill[blue!15.0, opacity=0.5] (2.9993, 2.1003, 0.2471) -- (2.9993, 2.1503, 0.2433) -- (2.9987, 2.1506, 0.2933) -- (2.9987, 2.1005, 0.2971) -- cycle;
\fill[blue!15.0, opacity=0.5] (2.9987, 2.1005, 0.2971) -- (2.9987, 2.1506, 0.2933) -- (2.9980, 2.1509, 0.3433) -- (2.9980, 2.1008, 0.3471) -- cycle;
\fill[blue!15.0, opacity=0.5] (2.9980, 2.1008, 0.3471) -- (2.9980, 2.1509, 0.3433) -- (2.9971, 2.1513, 0.3933) -- (2.9971, 2.1012, 0.3971) -- cycle;
\fill[blue!15.0, opacity=0.5] (2.9971, 2.1012, 0.3971) -- (2.9971, 2.1513, 0.3933) -- (2.9960, 2.1517, 0.4433) -- (2.9960, 2.1016, 0.4471) -- cycle;
\fill[blue!15.0, opacity=0.5] (2.9960, 2.1016, 0.4471) -- (2.9960, 2.1517, 0.4433) -- (2.9948, 2.1522, 0.4933) -- (2.9948, 2.1021, 0.4971) -- cycle;
\fill[blue!15.0, opacity=0.5] (2.9948, 2.1021, 0.4971) -- (2.9948, 2.1522, 0.4933) -- (2.9935, 2.1528, 0.5433) -- (2.9935, 2.1026, 0.5471) -- cycle;
\fill[blue!15.0, opacity=0.5] (2.9935, 2.1026, 0.5471) -- (2.9935, 2.1528, 0.5433) -- (2.9920, 2.1535, 0.5933) -- (2.9920, 2.1032, 0.5971) -- cycle;
\fill[blue!15.0, opacity=0.5] (2.9920, 2.1032, 0.5971) -- (2.9920, 2.1535, 0.5933) -- (2.9903, 2.1542, 0.6433) -- (2.9903, 2.1039, 0.6471) -- cycle;
\fill[blue!15.0, opacity=0.5] (2.9903, 2.1039, 0.6471) -- (2.9903, 2.1542, 0.6433) -- (2.9885, 2.1550, 0.6933) -- (2.9885, 2.1046, 0.6971) -- cycle;
\fill[blue!15.0, opacity=0.5] (2.9885, 2.1046, 0.6971) -- (2.9885, 2.1550, 0.6933) -- (2.9866, 2.1558, 0.7433) -- (2.9866, 2.1053, 0.7471) -- cycle;
\fill[blue!15.0, opacity=0.5] (2.9866, 2.1053, 0.7471) -- (2.9866, 2.1558, 0.7433) -- (2.9846, 2.1567, 0.7933) -- (2.9846, 2.1062, 0.7971) -- cycle;
\fill[blue!15.0, opacity=0.5] (2.9846, 2.1062, 0.7971) -- (2.9846, 2.1567, 0.7933) -- (2.9824, 2.1576, 0.8433) -- (2.9824, 2.1070, 0.8471) -- cycle;
\fill[blue!15.0, opacity=0.5] (2.9824, 2.1070, 0.8471) -- (2.9824, 2.1576, 0.8433) -- (2.9801, 2.1586, 0.8933) -- (2.9801, 2.1079, 0.8971) -- cycle;
\fill[blue!15.0, opacity=0.5] (2.9801, 2.1079, 0.8971) -- (2.9801, 2.1586, 0.8933) -- (2.9778, 2.1596, 0.9433) -- (2.9778, 2.1089, 0.9471) -- cycle;
\fill[blue!15.0, opacity=0.5] (2.9778, 2.1089, 0.9471) -- (2.9778, 2.1596, 0.9433) -- (2.9753, 2.1607, 0.9933) -- (2.9753, 2.1099, 0.9971) -- cycle;
\fill[blue!15.0, opacity=0.5] (2.9753, 2.1099, 0.9971) -- (2.9753, 2.1607, 0.9933) -- (2.9727, 2.1618, 1.0433) -- (2.9727, 2.1109, 1.0471) -- cycle;
\fill[blue!15.0, opacity=0.5] (2.9727, 2.1109, 1.0471) -- (2.9727, 2.1618, 1.0433) -- (2.9700, 2.1630, 1.0933) -- (2.9700, 2.1120, 1.0971) -- cycle;
\fill[blue!15.0, opacity=0.5] (2.9700, 2.1120, 1.0971) -- (2.9700, 2.1630, 1.0933) -- (2.9672, 2.1642, 1.1433) -- (2.9672, 2.1131, 1.1471) -- cycle;
\fill[blue!15.0, opacity=0.5] (2.9672, 2.1131, 1.1471) -- (2.9672, 2.1642, 1.1433) -- (2.9644, 2.1654, 1.1933) -- (2.9644, 2.1142, 1.1971) -- cycle;
\fill[blue!15.0, opacity=0.5] (2.9644, 2.1142, 1.1971) -- (2.9644, 2.1654, 1.1933) -- (2.9615, 2.1667, 1.2433) -- (2.9615, 2.1154, 1.2471) -- cycle;
\fill[blue!15.0, opacity=0.5] (2.9615, 2.1154, 1.2471) -- (2.9615, 2.1667, 1.2433) -- (2.9585, 2.1680, 1.2933) -- (2.9585, 2.1166, 1.2971) -- cycle;
\fill[blue!15.0, opacity=0.5] (2.9585, 2.1166, 1.2971) -- (2.9585, 2.1680, 1.2933) -- (2.9555, 2.1693, 1.3433) -- (2.9555, 2.1178, 1.3471) -- cycle;
\fill[blue!15.0, opacity=0.5] (2.9555, 2.1178, 1.3471) -- (2.9555, 2.1693, 1.3433) -- (2.9525, 2.1706, 1.3933) -- (2.9525, 2.1190, 1.3971) -- cycle;
\fill[blue!15.0, opacity=0.5] (2.9525, 2.1190, 1.3971) -- (2.9525, 2.1706, 1.3933) -- (2.9494, 2.1719, 1.4433) -- (2.9494, 2.1202, 1.4471) -- cycle;
\fill[blue!15.0, opacity=0.5] (2.9494, 2.1202, 1.4471) -- (2.9494, 2.1719, 1.4433) -- (2.9463, 2.1733, 1.4933) -- (2.9463, 2.1215, 1.4971) -- cycle;
\fill[blue!15.0, opacity=0.5] (2.9463, 2.1215, 1.4971) -- (2.9463, 2.1733, 1.4933) -- (2.9431, 2.1746, 1.5433) -- (2.9431, 2.1227, 1.5471) -- cycle;
\fill[blue!15.0, opacity=0.5] (2.9431, 2.1227, 1.5471) -- (2.9431, 2.1746, 1.5433) -- (2.9400, 2.1760, 1.5933) -- (2.9400, 2.1240, 1.5971) -- cycle;
\fill[blue!15.0, opacity=0.5] (2.9400, 2.1240, 1.5971) -- (2.9400, 2.1760, 1.5933) -- (2.9369, 2.1774, 1.6433) -- (2.9369, 2.1253, 1.6471) -- cycle;
\fill[blue!15.1, opacity=0.5] (2.9369, 2.1253, 1.6471) -- (2.9369, 2.1774, 1.6433) -- (2.9337, 2.1787, 1.6933) -- (2.9337, 2.1265, 1.6971) -- cycle;
\fill[blue!15.1, opacity=0.5] (2.9337, 2.1265, 1.6971) -- (2.9337, 2.1787, 1.6933) -- (2.9306, 2.1801, 1.7433) -- (2.9306, 2.1278, 1.7471) -- cycle;
\fill[blue!15.2, opacity=0.5] (2.9306, 2.1278, 1.7471) -- (2.9306, 2.1801, 1.7433) -- (2.9275, 2.1814, 1.7933) -- (2.9275, 2.1290, 1.7971) -- cycle;
\fill[blue!15.2, opacity=0.5] (2.9275, 2.1290, 1.7971) -- (2.9275, 2.1814, 1.7933) -- (2.9245, 2.1827, 1.8433) -- (2.9245, 2.1302, 1.8471) -- cycle;
\fill[blue!15.3, opacity=0.5] (2.9245, 2.1302, 1.8471) -- (2.9245, 2.1827, 1.8433) -- (2.9215, 2.1840, 1.8933) -- (2.9215, 2.1314, 1.8971) -- cycle;
\fill[blue!15.5, opacity=0.5] (2.9215, 2.1314, 1.8971) -- (2.9215, 2.1840, 1.8933) -- (2.9185, 2.1853, 1.9433) -- (2.9185, 2.1326, 1.9471) -- cycle;
\fill[blue!15.6, opacity=0.5] (2.9185, 2.1326, 1.9471) -- (2.9185, 2.1853, 1.9433) -- (2.9156, 2.1866, 1.9933) -- (2.9156, 2.1338, 1.9971) -- cycle;
\fill[blue!15.8, opacity=0.5] (2.9156, 2.1338, 1.9971) -- (2.9156, 2.1866, 1.9933) -- (2.9128, 2.1878, 2.0433) -- (2.9128, 2.1349, 2.0471) -- cycle;
\fill[blue!16.1, opacity=0.5] (2.9128, 2.1349, 2.0471) -- (2.9128, 2.1878, 2.0433) -- (2.9100, 2.1890, 2.0933) -- (2.9100, 2.1360, 2.0971) -- cycle;
\fill[blue!16.4, opacity=0.5] (2.9100, 2.1360, 2.0971) -- (2.9100, 2.1890, 2.0933) -- (2.9073, 2.1902, 2.1433) -- (2.9073, 2.1371, 2.1471) -- cycle;
\fill[blue!16.7, opacity=0.5] (2.9073, 2.1371, 2.1471) -- (2.9073, 2.1902, 2.1433) -- (2.9047, 2.1913, 2.1933) -- (2.9047, 2.1381, 2.1971) -- cycle;
\fill[blue!17.2, opacity=0.5] (2.9047, 2.1381, 2.1971) -- (2.9047, 2.1913, 2.1933) -- (2.9022, 2.1924, 2.2433) -- (2.9022, 2.1391, 2.2471) -- cycle;
\fill[blue!17.6, opacity=0.5] (2.9022, 2.1391, 2.2471) -- (2.9022, 2.1924, 2.2433) -- (2.8999, 2.1934, 2.2933) -- (2.8999, 2.1401, 2.2971) -- cycle;
\fill[blue!18.2, opacity=0.5] (2.8999, 2.1401, 2.2971) -- (2.8999, 2.1934, 2.2933) -- (2.8976, 2.1944, 2.3433) -- (2.8976, 2.1410, 2.3471) -- cycle;
\fill[blue!18.8, opacity=0.5] (2.8976, 2.1410, 2.3471) -- (2.8976, 2.1944, 2.3433) -- (2.8954, 2.1953, 2.3933) -- (2.8954, 2.1418, 2.3971) -- cycle;
\fill[blue!19.5, opacity=0.5] (2.8954, 2.1418, 2.3971) -- (2.8954, 2.1953, 2.3933) -- (2.8934, 2.1962, 2.4433) -- (2.8934, 2.1427, 2.4471) -- cycle;
\fill[blue!20.2, opacity=0.5] (2.8934, 2.1427, 2.4471) -- (2.8934, 2.1962, 2.4433) -- (2.8915, 2.1970, 2.4933) -- (2.8915, 2.1434, 2.4971) -- cycle;
\fill[blue!21.0, opacity=0.5] (2.8915, 2.1434, 2.4971) -- (2.8915, 2.1970, 2.4933) -- (2.8897, 2.1978, 2.5433) -- (2.8897, 2.1441, 2.5471) -- cycle;
\fill[blue!21.9, opacity=0.5] (2.8897, 2.1441, 2.5471) -- (2.8897, 2.1978, 2.5433) -- (2.8880, 2.1985, 2.5933) -- (2.8880, 2.1448, 2.5971) -- cycle;
\fill[blue!22.8, opacity=0.5] (2.8880, 2.1448, 2.5971) -- (2.8880, 2.1985, 2.5933) -- (2.8865, 2.1992, 2.6433) -- (2.8865, 2.1454, 2.6471) -- cycle;
\fill[blue!23.7, opacity=0.5] (2.8865, 2.1454, 2.6471) -- (2.8865, 2.1992, 2.6433) -- (2.8852, 2.1998, 2.6933) -- (2.8852, 2.1459, 2.6971) -- cycle;
\fill[blue!24.7, opacity=0.5] (2.8852, 2.1459, 2.6971) -- (2.8852, 2.1998, 2.6933) -- (2.8840, 2.2003, 2.7433) -- (2.8840, 2.1464, 2.7471) -- cycle;
\fill[blue!25.7, opacity=0.5] (2.8840, 2.1464, 2.7471) -- (2.8840, 2.2003, 2.7433) -- (2.8829, 2.2007, 2.7933) -- (2.8829, 2.1468, 2.7971) -- cycle;
\fill[blue!26.8, opacity=0.5] (2.8829, 2.1468, 2.7971) -- (2.8829, 2.2007, 2.7933) -- (2.8820, 2.2011, 2.8433) -- (2.8820, 2.1472, 2.8471) -- cycle;
\fill[blue!27.8, opacity=0.5] (2.8820, 2.1472, 2.8471) -- (2.8820, 2.2011, 2.8433) -- (2.8813, 2.2014, 2.8933) -- (2.8813, 2.1475, 2.8971) -- cycle;
\fill[blue!28.8, opacity=0.5] (2.8813, 2.1475, 2.8971) -- (2.8813, 2.2014, 2.8933) -- (2.8807, 2.2017, 2.9433) -- (2.8807, 2.1477, 2.9471) -- cycle;
\fill[blue!29.9, opacity=0.5] (2.8807, 2.1477, 2.9471) -- (2.8807, 2.2017, 2.9433) -- (2.8803, 2.2019, 2.9933) -- (2.8803, 2.1479, 2.9971) -- cycle;
\fill[blue!30.9, opacity=0.5] (2.8803, 2.1479, 2.9971) -- (2.8803, 2.2019, 2.9933) -- (2.8801, 2.2020, 3.0433) -- (2.8801, 2.1480, 3.0471) -- cycle;
\fill[blue!31.9, opacity=0.5] (2.8801, 2.1480, 3.0471) -- (2.8801, 2.2020, 3.0433) -- (2.8800, 2.2020, 3.0933) -- (2.8800, 2.1480, 3.0971) -- cycle;
\fill[blue!15.0, opacity=0.5] (3.0000, 2.1500, 0.0933) -- (3.0000, 2.2000, 0.0892) -- (2.9999, 2.2000, 0.1392) -- (2.9999, 2.1500, 0.1433) -- cycle;
\fill[blue!15.0, opacity=0.5] (2.9999, 2.1500, 0.1433) -- (2.9999, 2.2000, 0.1392) -- (2.9997, 2.2002, 0.1892) -- (2.9997, 2.1501, 0.1933) -- cycle;
\fill[blue!15.0, opacity=0.5] (2.9997, 2.1501, 0.1933) -- (2.9997, 2.2002, 0.1892) -- (2.9993, 2.2003, 0.2392) -- (2.9993, 2.1503, 0.2433) -- cycle;
\fill[blue!15.0, opacity=0.5] (2.9993, 2.1503, 0.2433) -- (2.9993, 2.2003, 0.2392) -- (2.9987, 2.2006, 0.2892) -- (2.9987, 2.1506, 0.2933) -- cycle;
\fill[blue!15.0, opacity=0.5] (2.9987, 2.1506, 0.2933) -- (2.9987, 2.2006, 0.2892) -- (2.9980, 2.2010, 0.3392) -- (2.9980, 2.1509, 0.3433) -- cycle;
\fill[blue!15.0, opacity=0.5] (2.9980, 2.1509, 0.3433) -- (2.9980, 2.2010, 0.3392) -- (2.9971, 2.2014, 0.3892) -- (2.9971, 2.1513, 0.3933) -- cycle;
\fill[blue!15.0, opacity=0.5] (2.9971, 2.1513, 0.3933) -- (2.9971, 2.2014, 0.3892) -- (2.9960, 2.2019, 0.4392) -- (2.9960, 2.1517, 0.4433) -- cycle;
\fill[blue!15.0, opacity=0.5] (2.9960, 2.1517, 0.4433) -- (2.9960, 2.2019, 0.4392) -- (2.9948, 2.2024, 0.4892) -- (2.9948, 2.1522, 0.4933) -- cycle;
\fill[blue!15.0, opacity=0.5] (2.9948, 2.1522, 0.4933) -- (2.9948, 2.2024, 0.4892) -- (2.9935, 2.2031, 0.5392) -- (2.9935, 2.1528, 0.5433) -- cycle;
\fill[blue!15.0, opacity=0.5] (2.9935, 2.1528, 0.5433) -- (2.9935, 2.2031, 0.5392) -- (2.9920, 2.2038, 0.5892) -- (2.9920, 2.1535, 0.5933) -- cycle;
\fill[blue!15.0, opacity=0.5] (2.9920, 2.1535, 0.5933) -- (2.9920, 2.2038, 0.5892) -- (2.9903, 2.2045, 0.6392) -- (2.9903, 2.1542, 0.6433) -- cycle;
\fill[blue!15.0, opacity=0.5] (2.9903, 2.1542, 0.6433) -- (2.9903, 2.2045, 0.6392) -- (2.9885, 2.2053, 0.6892) -- (2.9885, 2.1550, 0.6933) -- cycle;
\fill[blue!15.0, opacity=0.5] (2.9885, 2.1550, 0.6933) -- (2.9885, 2.2053, 0.6892) -- (2.9866, 2.2062, 0.7392) -- (2.9866, 2.1558, 0.7433) -- cycle;
\fill[blue!15.0, opacity=0.5] (2.9866, 2.1558, 0.7433) -- (2.9866, 2.2062, 0.7392) -- (2.9846, 2.2072, 0.7892) -- (2.9846, 2.1567, 0.7933) -- cycle;
\fill[blue!15.0, opacity=0.5] (2.9846, 2.1567, 0.7933) -- (2.9846, 2.2072, 0.7892) -- (2.9824, 2.2082, 0.8392) -- (2.9824, 2.1576, 0.8433) -- cycle;
\fill[blue!15.0, opacity=0.5] (2.9824, 2.1576, 0.8433) -- (2.9824, 2.2082, 0.8392) -- (2.9801, 2.2093, 0.8892) -- (2.9801, 2.1586, 0.8933) -- cycle;
\fill[blue!15.0, opacity=0.5] (2.9801, 2.1586, 0.8933) -- (2.9801, 2.2093, 0.8892) -- (2.9778, 2.2104, 0.9392) -- (2.9778, 2.1596, 0.9433) -- cycle;
\fill[blue!15.0, opacity=0.5] (2.9778, 2.1596, 0.9433) -- (2.9778, 2.2104, 0.9392) -- (2.9753, 2.2115, 0.9892) -- (2.9753, 2.1607, 0.9933) -- cycle;
\fill[blue!15.0, opacity=0.5] (2.9753, 2.1607, 0.9933) -- (2.9753, 2.2115, 0.9892) -- (2.9727, 2.2128, 1.0392) -- (2.9727, 2.1618, 1.0433) -- cycle;
\fill[blue!15.0, opacity=0.5] (2.9727, 2.1618, 1.0433) -- (2.9727, 2.2128, 1.0392) -- (2.9700, 2.2140, 1.0892) -- (2.9700, 2.1630, 1.0933) -- cycle;
\fill[blue!15.0, opacity=0.5] (2.9700, 2.1630, 1.0933) -- (2.9700, 2.2140, 1.0892) -- (2.9672, 2.2153, 1.1392) -- (2.9672, 2.1642, 1.1433) -- cycle;
\fill[blue!15.0, opacity=0.5] (2.9672, 2.1642, 1.1433) -- (2.9672, 2.2153, 1.1392) -- (2.9644, 2.2166, 1.1892) -- (2.9644, 2.1654, 1.1933) -- cycle;
\fill[blue!15.0, opacity=0.5] (2.9644, 2.1654, 1.1933) -- (2.9644, 2.2166, 1.1892) -- (2.9615, 2.2180, 1.2392) -- (2.9615, 2.1667, 1.2433) -- cycle;
\fill[blue!15.0, opacity=0.5] (2.9615, 2.1667, 1.2433) -- (2.9615, 2.2180, 1.2392) -- (2.9585, 2.2193, 1.2892) -- (2.9585, 2.1680, 1.2933) -- cycle;
\fill[blue!15.0, opacity=0.5] (2.9585, 2.1680, 1.2933) -- (2.9585, 2.2193, 1.2892) -- (2.9555, 2.2208, 1.3392) -- (2.9555, 2.1693, 1.3433) -- cycle;
\fill[blue!15.0, opacity=0.5] (2.9555, 2.1693, 1.3433) -- (2.9555, 2.2208, 1.3392) -- (2.9525, 2.2222, 1.3892) -- (2.9525, 2.1706, 1.3933) -- cycle;
\fill[blue!15.1, opacity=0.5] (2.9525, 2.1706, 1.3933) -- (2.9525, 2.2222, 1.3892) -- (2.9494, 2.2236, 1.4392) -- (2.9494, 2.1719, 1.4433) -- cycle;
\fill[blue!15.1, opacity=0.5] (2.9494, 2.1719, 1.4433) -- (2.9494, 2.2236, 1.4392) -- (2.9463, 2.2251, 1.4892) -- (2.9463, 2.1733, 1.4933) -- cycle;
\fill[blue!15.2, opacity=0.5] (2.9463, 2.1733, 1.4933) -- (2.9463, 2.2251, 1.4892) -- (2.9431, 2.2265, 1.5392) -- (2.9431, 2.1746, 1.5433) -- cycle;
\fill[blue!15.2, opacity=0.5] (2.9431, 2.1746, 1.5433) -- (2.9431, 2.2265, 1.5392) -- (2.9400, 2.2280, 1.5892) -- (2.9400, 2.1760, 1.5933) -- cycle;
\fill[blue!15.3, opacity=0.5] (2.9400, 2.1760, 1.5933) -- (2.9400, 2.2280, 1.5892) -- (2.9369, 2.2295, 1.6392) -- (2.9369, 2.1774, 1.6433) -- cycle;
\fill[blue!15.5, opacity=0.5] (2.9369, 2.1774, 1.6433) -- (2.9369, 2.2295, 1.6392) -- (2.9337, 2.2309, 1.6892) -- (2.9337, 2.1787, 1.6933) -- cycle;
\fill[blue!15.7, opacity=0.5] (2.9337, 2.1787, 1.6933) -- (2.9337, 2.2309, 1.6892) -- (2.9306, 2.2324, 1.7392) -- (2.9306, 2.1801, 1.7433) -- cycle;
\fill[blue!15.9, opacity=0.5] (2.9306, 2.1801, 1.7433) -- (2.9306, 2.2324, 1.7392) -- (2.9275, 2.2338, 1.7892) -- (2.9275, 2.1814, 1.7933) -- cycle;
\fill[blue!16.2, opacity=0.5] (2.9275, 2.1814, 1.7933) -- (2.9275, 2.2338, 1.7892) -- (2.9245, 2.2352, 1.8392) -- (2.9245, 2.1827, 1.8433) -- cycle;
\fill[blue!16.5, opacity=0.5] (2.9245, 2.1827, 1.8433) -- (2.9245, 2.2352, 1.8392) -- (2.9215, 2.2367, 1.8892) -- (2.9215, 2.1840, 1.8933) -- cycle;
\fill[blue!17.0, opacity=0.5] (2.9215, 2.1840, 1.8933) -- (2.9215, 2.2367, 1.8892) -- (2.9185, 2.2380, 1.9392) -- (2.9185, 2.1853, 1.9433) -- cycle;
\fill[blue!17.5, opacity=0.5] (2.9185, 2.1853, 1.9433) -- (2.9185, 2.2380, 1.9392) -- (2.9156, 2.2394, 1.9892) -- (2.9156, 2.1866, 1.9933) -- cycle;
\fill[blue!18.1, opacity=0.5] (2.9156, 2.1866, 1.9933) -- (2.9156, 2.2394, 1.9892) -- (2.9128, 2.2407, 2.0392) -- (2.9128, 2.1878, 2.0433) -- cycle;
\fill[blue!18.8, opacity=0.5] (2.9128, 2.1878, 2.0433) -- (2.9128, 2.2407, 2.0392) -- (2.9100, 2.2420, 2.0892) -- (2.9100, 2.1890, 2.0933) -- cycle;
\fill[blue!19.5, opacity=0.5] (2.9100, 2.1890, 2.0933) -- (2.9100, 2.2420, 2.0892) -- (2.9073, 2.2432, 2.1392) -- (2.9073, 2.1902, 2.1433) -- cycle;
\fill[blue!20.4, opacity=0.5] (2.9073, 2.1902, 2.1433) -- (2.9073, 2.2432, 2.1392) -- (2.9047, 2.2445, 2.1892) -- (2.9047, 2.1913, 2.1933) -- cycle;
\fill[blue!21.4, opacity=0.5] (2.9047, 2.1913, 2.1933) -- (2.9047, 2.2445, 2.1892) -- (2.9022, 2.2456, 2.2392) -- (2.9022, 2.1924, 2.2433) -- cycle;
\fill[blue!22.4, opacity=0.5] (2.9022, 2.1924, 2.2433) -- (2.9022, 2.2456, 2.2392) -- (2.8999, 2.2467, 2.2892) -- (2.8999, 2.1934, 2.2933) -- cycle;
\fill[blue!23.5, opacity=0.5] (2.8999, 2.1934, 2.2933) -- (2.8999, 2.2467, 2.2892) -- (2.8976, 2.2478, 2.3392) -- (2.8976, 2.1944, 2.3433) -- cycle;
\fill[blue!24.7, opacity=0.5] (2.8976, 2.1944, 2.3433) -- (2.8976, 2.2478, 2.3392) -- (2.8954, 2.2488, 2.3892) -- (2.8954, 2.1953, 2.3933) -- cycle;
\fill[blue!25.9, opacity=0.5] (2.8954, 2.1953, 2.3933) -- (2.8954, 2.2488, 2.3892) -- (2.8934, 2.2498, 2.4392) -- (2.8934, 2.1962, 2.4433) -- cycle;
\fill[blue!27.2, opacity=0.5] (2.8934, 2.1962, 2.4433) -- (2.8934, 2.2498, 2.4392) -- (2.8915, 2.2507, 2.4892) -- (2.8915, 2.1970, 2.4933) -- cycle;
\fill[blue!28.5, opacity=0.5] (2.8915, 2.1970, 2.4933) -- (2.8915, 2.2507, 2.4892) -- (2.8897, 2.2515, 2.5392) -- (2.8897, 2.1978, 2.5433) -- cycle;
\fill[blue!29.8, opacity=0.5] (2.8897, 2.1978, 2.5433) -- (2.8897, 2.2515, 2.5392) -- (2.8880, 2.2522, 2.5892) -- (2.8880, 2.1985, 2.5933) -- cycle;
\fill[blue!31.2, opacity=0.5] (2.8880, 2.1985, 2.5933) -- (2.8880, 2.2522, 2.5892) -- (2.8865, 2.2529, 2.6392) -- (2.8865, 2.1992, 2.6433) -- cycle;
\fill[blue!32.5, opacity=0.5] (2.8865, 2.1992, 2.6433) -- (2.8865, 2.2529, 2.6392) -- (2.8852, 2.2536, 2.6892) -- (2.8852, 2.1998, 2.6933) -- cycle;
\fill[blue!33.9, opacity=0.5] (2.8852, 2.1998, 2.6933) -- (2.8852, 2.2536, 2.6892) -- (2.8840, 2.2541, 2.7392) -- (2.8840, 2.2003, 2.7433) -- cycle;
\fill[blue!35.2, opacity=0.5] (2.8840, 2.2003, 2.7433) -- (2.8840, 2.2541, 2.7392) -- (2.8829, 2.2546, 2.7892) -- (2.8829, 2.2007, 2.7933) -- cycle;
\fill[blue!36.4, opacity=0.5] (2.8829, 2.2007, 2.7933) -- (2.8829, 2.2546, 2.7892) -- (2.8820, 2.2550, 2.8392) -- (2.8820, 2.2011, 2.8433) -- cycle;
\fill[blue!37.6, opacity=0.5] (2.8820, 2.2011, 2.8433) -- (2.8820, 2.2550, 2.8392) -- (2.8813, 2.2554, 2.8892) -- (2.8813, 2.2014, 2.8933) -- cycle;
\fill[blue!38.8, opacity=0.5] (2.8813, 2.2014, 2.8933) -- (2.8813, 2.2554, 2.8892) -- (2.8807, 2.2557, 2.9392) -- (2.8807, 2.2017, 2.9433) -- cycle;
\fill[blue!39.9, opacity=0.5] (2.8807, 2.2017, 2.9433) -- (2.8807, 2.2557, 2.9392) -- (2.8803, 2.2558, 2.9892) -- (2.8803, 2.2019, 2.9933) -- cycle;
\fill[blue!40.9, opacity=0.5] (2.8803, 2.2019, 2.9933) -- (2.8803, 2.2558, 2.9892) -- (2.8801, 2.2560, 3.0392) -- (2.8801, 2.2020, 3.0433) -- cycle;
\fill[blue!41.8, opacity=0.5] (2.8801, 2.2020, 3.0433) -- (2.8801, 2.2560, 3.0392) -- (2.8800, 2.2560, 3.0892) -- (2.8800, 2.2020, 3.0933) -- cycle;
\fill[blue!15.0, opacity=0.5] (3.0000, 2.2000, 0.0892) -- (3.0000, 2.2500, 0.0849) -- (2.9999, 2.2500, 0.1349) -- (2.9999, 2.2000, 0.1392) -- cycle;
\fill[blue!15.0, opacity=0.5] (2.9999, 2.2000, 0.1392) -- (2.9999, 2.2500, 0.1349) -- (2.9997, 2.2502, 0.1849) -- (2.9997, 2.2002, 0.1892) -- cycle;
\fill[blue!15.0, opacity=0.5] (2.9997, 2.2002, 0.1892) -- (2.9997, 2.2502, 0.1849) -- (2.9993, 2.2504, 0.2349) -- (2.9993, 2.2003, 0.2392) -- cycle;
\fill[blue!15.0, opacity=0.5] (2.9993, 2.2003, 0.2392) -- (2.9993, 2.2504, 0.2349) -- (2.9987, 2.2507, 0.2849) -- (2.9987, 2.2006, 0.2892) -- cycle;
\fill[blue!15.0, opacity=0.5] (2.9987, 2.2006, 0.2892) -- (2.9987, 2.2507, 0.2849) -- (2.9980, 2.2510, 0.3349) -- (2.9980, 2.2010, 0.3392) -- cycle;
\fill[blue!15.0, opacity=0.5] (2.9980, 2.2010, 0.3392) -- (2.9980, 2.2510, 0.3349) -- (2.9971, 2.2515, 0.3849) -- (2.9971, 2.2014, 0.3892) -- cycle;
\fill[blue!15.0, opacity=0.5] (2.9971, 2.2014, 0.3892) -- (2.9971, 2.2515, 0.3849) -- (2.9960, 2.2520, 0.4349) -- (2.9960, 2.2019, 0.4392) -- cycle;
\fill[blue!15.0, opacity=0.5] (2.9960, 2.2019, 0.4392) -- (2.9960, 2.2520, 0.4349) -- (2.9948, 2.2526, 0.4849) -- (2.9948, 2.2024, 0.4892) -- cycle;
\fill[blue!15.0, opacity=0.5] (2.9948, 2.2024, 0.4892) -- (2.9948, 2.2526, 0.4849) -- (2.9935, 2.2533, 0.5349) -- (2.9935, 2.2031, 0.5392) -- cycle;
\fill[blue!15.0, opacity=0.5] (2.9935, 2.2031, 0.5392) -- (2.9935, 2.2533, 0.5349) -- (2.9920, 2.2540, 0.5849) -- (2.9920, 2.2038, 0.5892) -- cycle;
\fill[blue!15.0, opacity=0.5] (2.9920, 2.2038, 0.5892) -- (2.9920, 2.2540, 0.5849) -- (2.9903, 2.2548, 0.6349) -- (2.9903, 2.2045, 0.6392) -- cycle;
\fill[blue!15.0, opacity=0.5] (2.9903, 2.2045, 0.6392) -- (2.9903, 2.2548, 0.6349) -- (2.9885, 2.2557, 0.6849) -- (2.9885, 2.2053, 0.6892) -- cycle;
\fill[blue!15.0, opacity=0.5] (2.9885, 2.2053, 0.6892) -- (2.9885, 2.2557, 0.6849) -- (2.9866, 2.2567, 0.7349) -- (2.9866, 2.2062, 0.7392) -- cycle;
\fill[blue!15.0, opacity=0.5] (2.9866, 2.2062, 0.7392) -- (2.9866, 2.2567, 0.7349) -- (2.9846, 2.2577, 0.7849) -- (2.9846, 2.2072, 0.7892) -- cycle;
\fill[blue!15.0, opacity=0.5] (2.9846, 2.2072, 0.7892) -- (2.9846, 2.2577, 0.7849) -- (2.9824, 2.2588, 0.8349) -- (2.9824, 2.2082, 0.8392) -- cycle;
\fill[blue!15.0, opacity=0.5] (2.9824, 2.2082, 0.8392) -- (2.9824, 2.2588, 0.8349) -- (2.9801, 2.2599, 0.8849) -- (2.9801, 2.2093, 0.8892) -- cycle;
\fill[blue!15.0, opacity=0.5] (2.9801, 2.2093, 0.8892) -- (2.9801, 2.2599, 0.8849) -- (2.9778, 2.2611, 0.9349) -- (2.9778, 2.2104, 0.9392) -- cycle;
\fill[blue!15.0, opacity=0.5] (2.9778, 2.2104, 0.9392) -- (2.9778, 2.2611, 0.9349) -- (2.9753, 2.2624, 0.9849) -- (2.9753, 2.2115, 0.9892) -- cycle;
\fill[blue!15.0, opacity=0.5] (2.9753, 2.2115, 0.9892) -- (2.9753, 2.2624, 0.9849) -- (2.9727, 2.2637, 1.0349) -- (2.9727, 2.2128, 1.0392) -- cycle;
\fill[blue!15.0, opacity=0.5] (2.9727, 2.2128, 1.0392) -- (2.9727, 2.2637, 1.0349) -- (2.9700, 2.2650, 1.0849) -- (2.9700, 2.2140, 1.0892) -- cycle;
\fill[blue!15.0, opacity=0.5] (2.9700, 2.2140, 1.0892) -- (2.9700, 2.2650, 1.0849) -- (2.9672, 2.2664, 1.1349) -- (2.9672, 2.2153, 1.1392) -- cycle;
\fill[blue!15.0, opacity=0.5] (2.9672, 2.2153, 1.1392) -- (2.9672, 2.2664, 1.1349) -- (2.9644, 2.2678, 1.1849) -- (2.9644, 2.2166, 1.1892) -- cycle;
\fill[blue!15.1, opacity=0.5] (2.9644, 2.2166, 1.1892) -- (2.9644, 2.2678, 1.1849) -- (2.9615, 2.2692, 1.2349) -- (2.9615, 2.2180, 1.2392) -- cycle;
\fill[blue!15.1, opacity=0.5] (2.9615, 2.2180, 1.2392) -- (2.9615, 2.2692, 1.2349) -- (2.9585, 2.2707, 1.2849) -- (2.9585, 2.2193, 1.2892) -- cycle;
\fill[blue!15.2, opacity=0.5] (2.9585, 2.2193, 1.2892) -- (2.9585, 2.2707, 1.2849) -- (2.9555, 2.2722, 1.3349) -- (2.9555, 2.2208, 1.3392) -- cycle;
\fill[blue!15.3, opacity=0.5] (2.9555, 2.2208, 1.3392) -- (2.9555, 2.2722, 1.3349) -- (2.9525, 2.2738, 1.3849) -- (2.9525, 2.2222, 1.3892) -- cycle;
\fill[blue!15.4, opacity=0.5] (2.9525, 2.2222, 1.3892) -- (2.9525, 2.2738, 1.3849) -- (2.9494, 2.2753, 1.4349) -- (2.9494, 2.2236, 1.4392) -- cycle;
\fill[blue!15.5, opacity=0.5] (2.9494, 2.2236, 1.4392) -- (2.9494, 2.2753, 1.4349) -- (2.9463, 2.2769, 1.4849) -- (2.9463, 2.2251, 1.4892) -- cycle;
\fill[blue!15.8, opacity=0.5] (2.9463, 2.2251, 1.4892) -- (2.9463, 2.2769, 1.4849) -- (2.9431, 2.2784, 1.5349) -- (2.9431, 2.2265, 1.5392) -- cycle;
\fill[blue!16.1, opacity=0.5] (2.9431, 2.2265, 1.5392) -- (2.9431, 2.2784, 1.5349) -- (2.9400, 2.2800, 1.5849) -- (2.9400, 2.2280, 1.5892) -- cycle;
\fill[blue!16.4, opacity=0.5] (2.9400, 2.2280, 1.5892) -- (2.9400, 2.2800, 1.5849) -- (2.9369, 2.2816, 1.6349) -- (2.9369, 2.2295, 1.6392) -- cycle;
\fill[blue!16.9, opacity=0.5] (2.9369, 2.2295, 1.6392) -- (2.9369, 2.2816, 1.6349) -- (2.9337, 2.2831, 1.6849) -- (2.9337, 2.2309, 1.6892) -- cycle;
\fill[blue!17.5, opacity=0.5] (2.9337, 2.2309, 1.6892) -- (2.9337, 2.2831, 1.6849) -- (2.9306, 2.2847, 1.7349) -- (2.9306, 2.2324, 1.7392) -- cycle;
\fill[blue!18.1, opacity=0.5] (2.9306, 2.2324, 1.7392) -- (2.9306, 2.2847, 1.7349) -- (2.9275, 2.2862, 1.7849) -- (2.9275, 2.2338, 1.7892) -- cycle;
\fill[blue!18.9, opacity=0.5] (2.9275, 2.2338, 1.7892) -- (2.9275, 2.2862, 1.7849) -- (2.9245, 2.2878, 1.8349) -- (2.9245, 2.2352, 1.8392) -- cycle;
\fill[blue!19.8, opacity=0.5] (2.9245, 2.2352, 1.8392) -- (2.9245, 2.2878, 1.8349) -- (2.9215, 2.2893, 1.8849) -- (2.9215, 2.2367, 1.8892) -- cycle;
\fill[blue!20.8, opacity=0.5] (2.9215, 2.2367, 1.8892) -- (2.9215, 2.2893, 1.8849) -- (2.9185, 2.2908, 1.9349) -- (2.9185, 2.2380, 1.9392) -- cycle;
\fill[blue!21.9, opacity=0.5] (2.9185, 2.2380, 1.9392) -- (2.9185, 2.2908, 1.9349) -- (2.9156, 2.2922, 1.9849) -- (2.9156, 2.2394, 1.9892) -- cycle;
\fill[blue!23.1, opacity=0.5] (2.9156, 2.2394, 1.9892) -- (2.9156, 2.2922, 1.9849) -- (2.9128, 2.2936, 2.0349) -- (2.9128, 2.2407, 2.0392) -- cycle;
\fill[blue!24.4, opacity=0.5] (2.9128, 2.2407, 2.0392) -- (2.9128, 2.2936, 2.0349) -- (2.9100, 2.2950, 2.0849) -- (2.9100, 2.2420, 2.0892) -- cycle;
\fill[blue!25.8, opacity=0.5] (2.9100, 2.2420, 2.0892) -- (2.9100, 2.2950, 2.0849) -- (2.9073, 2.2963, 2.1349) -- (2.9073, 2.2432, 2.1392) -- cycle;
\fill[blue!27.3, opacity=0.5] (2.9073, 2.2432, 2.1392) -- (2.9073, 2.2963, 2.1349) -- (2.9047, 2.2976, 2.1849) -- (2.9047, 2.2445, 2.1892) -- cycle;
\fill[blue!28.9, opacity=0.5] (2.9047, 2.2445, 2.1892) -- (2.9047, 2.2976, 2.1849) -- (2.9022, 2.2989, 2.2349) -- (2.9022, 2.2456, 2.2392) -- cycle;
\fill[blue!30.5, opacity=0.5] (2.9022, 2.2456, 2.2392) -- (2.9022, 2.2989, 2.2349) -- (2.8999, 2.3001, 2.2849) -- (2.8999, 2.2467, 2.2892) -- cycle;
\fill[blue!32.2, opacity=0.5] (2.8999, 2.2467, 2.2892) -- (2.8999, 2.3001, 2.2849) -- (2.8976, 2.3012, 2.3349) -- (2.8976, 2.2478, 2.3392) -- cycle;
\fill[blue!33.9, opacity=0.5] (2.8976, 2.2478, 2.3392) -- (2.8976, 2.3012, 2.3349) -- (2.8954, 2.3023, 2.3849) -- (2.8954, 2.2488, 2.3892) -- cycle;
\fill[blue!35.5, opacity=0.5] (2.8954, 2.2488, 2.3892) -- (2.8954, 2.3023, 2.3849) -- (2.8934, 2.3033, 2.4349) -- (2.8934, 2.2498, 2.4392) -- cycle;
\fill[blue!37.2, opacity=0.5] (2.8934, 2.2498, 2.4392) -- (2.8934, 2.3033, 2.4349) -- (2.8915, 2.3043, 2.4849) -- (2.8915, 2.2507, 2.4892) -- cycle;
\fill[blue!38.8, opacity=0.5] (2.8915, 2.2507, 2.4892) -- (2.8915, 2.3043, 2.4849) -- (2.8897, 2.3052, 2.5349) -- (2.8897, 2.2515, 2.5392) -- cycle;
\fill[blue!40.4, opacity=0.5] (2.8897, 2.2515, 2.5392) -- (2.8897, 2.3052, 2.5349) -- (2.8880, 2.3060, 2.5849) -- (2.8880, 2.2522, 2.5892) -- cycle;
\fill[blue!42.0, opacity=0.5] (2.8880, 2.2522, 2.5892) -- (2.8880, 2.3060, 2.5849) -- (2.8865, 2.3067, 2.6349) -- (2.8865, 2.2529, 2.6392) -- cycle;
\fill[blue!43.4, opacity=0.5] (2.8865, 2.2529, 2.6392) -- (2.8865, 2.3067, 2.6349) -- (2.8852, 2.3074, 2.6849) -- (2.8852, 2.2536, 2.6892) -- cycle;
\fill[blue!44.8, opacity=0.5] (2.8852, 2.2536, 2.6892) -- (2.8852, 2.3074, 2.6849) -- (2.8840, 2.3080, 2.7349) -- (2.8840, 2.2541, 2.7392) -- cycle;
\fill[blue!46.0, opacity=0.5] (2.8840, 2.2541, 2.7392) -- (2.8840, 2.3080, 2.7349) -- (2.8829, 2.3085, 2.7849) -- (2.8829, 2.2546, 2.7892) -- cycle;
\fill[blue!47.2, opacity=0.5] (2.8829, 2.2546, 2.7892) -- (2.8829, 2.3085, 2.7849) -- (2.8820, 2.3090, 2.8349) -- (2.8820, 2.2550, 2.8392) -- cycle;
\fill[blue!48.2, opacity=0.5] (2.8820, 2.2550, 2.8392) -- (2.8820, 2.3090, 2.8349) -- (2.8813, 2.3093, 2.8849) -- (2.8813, 2.2554, 2.8892) -- cycle;
\fill[blue!49.2, opacity=0.5] (2.8813, 2.2554, 2.8892) -- (2.8813, 2.3093, 2.8849) -- (2.8807, 2.3096, 2.9349) -- (2.8807, 2.2557, 2.9392) -- cycle;
\fill[blue!50.0, opacity=0.5] (2.8807, 2.2557, 2.9392) -- (2.8807, 2.3096, 2.9349) -- (2.8803, 2.3098, 2.9849) -- (2.8803, 2.2558, 2.9892) -- cycle;
\fill[blue!50.6, opacity=0.5] (2.8803, 2.2558, 2.9892) -- (2.8803, 2.3098, 2.9849) -- (2.8801, 2.3100, 3.0349) -- (2.8801, 2.2560, 3.0392) -- cycle;
\fill[blue!51.2, opacity=0.5] (2.8801, 2.2560, 3.0392) -- (2.8801, 2.3100, 3.0349) -- (2.8800, 2.3100, 3.0849) -- (2.8800, 2.2560, 3.0892) -- cycle;
\fill[blue!15.0, opacity=0.5] (3.0000, 2.2500, 0.0849) -- (3.0000, 2.3000, 0.0803) -- (2.9999, 2.3000, 0.1303) -- (2.9999, 2.2500, 0.1349) -- cycle;
\fill[blue!15.0, opacity=0.5] (2.9999, 2.2500, 0.1349) -- (2.9999, 2.3000, 0.1303) -- (2.9997, 2.3002, 0.1803) -- (2.9997, 2.2502, 0.1849) -- cycle;
\fill[blue!15.0, opacity=0.5] (2.9997, 2.2502, 0.1849) -- (2.9997, 2.3002, 0.1803) -- (2.9993, 2.3004, 0.2303) -- (2.9993, 2.2504, 0.2349) -- cycle;
\fill[blue!15.0, opacity=0.5] (2.9993, 2.2504, 0.2349) -- (2.9993, 2.3004, 0.2303) -- (2.9987, 2.3007, 0.2803) -- (2.9987, 2.2507, 0.2849) -- cycle;
\fill[blue!15.0, opacity=0.5] (2.9987, 2.2507, 0.2849) -- (2.9987, 2.3007, 0.2803) -- (2.9980, 2.3011, 0.3303) -- (2.9980, 2.2510, 0.3349) -- cycle;
\fill[blue!15.0, opacity=0.5] (2.9980, 2.2510, 0.3349) -- (2.9980, 2.3011, 0.3303) -- (2.9971, 2.3016, 0.3803) -- (2.9971, 2.2515, 0.3849) -- cycle;
\fill[blue!15.0, opacity=0.5] (2.9971, 2.2515, 0.3849) -- (2.9971, 2.3016, 0.3803) -- (2.9960, 2.3021, 0.4303) -- (2.9960, 2.2520, 0.4349) -- cycle;
\fill[blue!15.0, opacity=0.5] (2.9960, 2.2520, 0.4349) -- (2.9960, 2.3021, 0.4303) -- (2.9948, 2.3028, 0.4803) -- (2.9948, 2.2526, 0.4849) -- cycle;
\fill[blue!15.0, opacity=0.5] (2.9948, 2.2526, 0.4849) -- (2.9948, 2.3028, 0.4803) -- (2.9935, 2.3035, 0.5303) -- (2.9935, 2.2533, 0.5349) -- cycle;
\fill[blue!15.0, opacity=0.5] (2.9935, 2.2533, 0.5349) -- (2.9935, 2.3035, 0.5303) -- (2.9920, 2.3043, 0.5803) -- (2.9920, 2.2540, 0.5849) -- cycle;
\fill[blue!15.0, opacity=0.5] (2.9920, 2.2540, 0.5849) -- (2.9920, 2.3043, 0.5803) -- (2.9903, 2.3052, 0.6303) -- (2.9903, 2.2548, 0.6349) -- cycle;
\fill[blue!15.0, opacity=0.5] (2.9903, 2.2548, 0.6349) -- (2.9903, 2.3052, 0.6303) -- (2.9885, 2.3061, 0.6803) -- (2.9885, 2.2557, 0.6849) -- cycle;
\fill[blue!15.0, opacity=0.5] (2.9885, 2.2557, 0.6849) -- (2.9885, 2.3061, 0.6803) -- (2.9866, 2.3071, 0.7303) -- (2.9866, 2.2567, 0.7349) -- cycle;
\fill[blue!15.0, opacity=0.5] (2.9866, 2.2567, 0.7349) -- (2.9866, 2.3071, 0.7303) -- (2.9846, 2.3082, 0.7803) -- (2.9846, 2.2577, 0.7849) -- cycle;
\fill[blue!15.0, opacity=0.5] (2.9846, 2.2577, 0.7849) -- (2.9846, 2.3082, 0.7803) -- (2.9824, 2.3094, 0.8303) -- (2.9824, 2.2588, 0.8349) -- cycle;
\fill[blue!15.0, opacity=0.5] (2.9824, 2.2588, 0.8349) -- (2.9824, 2.3094, 0.8303) -- (2.9801, 2.3106, 0.8803) -- (2.9801, 2.2599, 0.8849) -- cycle;
\fill[blue!15.0, opacity=0.5] (2.9801, 2.2599, 0.8849) -- (2.9801, 2.3106, 0.8803) -- (2.9778, 2.3119, 0.9303) -- (2.9778, 2.2611, 0.9349) -- cycle;
\fill[blue!15.0, opacity=0.5] (2.9778, 2.2611, 0.9349) -- (2.9778, 2.3119, 0.9303) -- (2.9753, 2.3132, 0.9803) -- (2.9753, 2.2624, 0.9849) -- cycle;
\fill[blue!15.0, opacity=0.5] (2.9753, 2.2624, 0.9849) -- (2.9753, 2.3132, 0.9803) -- (2.9727, 2.3146, 1.0303) -- (2.9727, 2.2637, 1.0349) -- cycle;
\fill[blue!15.1, opacity=0.5] (2.9727, 2.2637, 1.0349) -- (2.9727, 2.3146, 1.0303) -- (2.9700, 2.3160, 1.0803) -- (2.9700, 2.2650, 1.0849) -- cycle;
\fill[blue!15.1, opacity=0.5] (2.9700, 2.2650, 1.0849) -- (2.9700, 2.3160, 1.0803) -- (2.9672, 2.3175, 1.1303) -- (2.9672, 2.2664, 1.1349) -- cycle;
\fill[blue!15.1, opacity=0.5] (2.9672, 2.2664, 1.1349) -- (2.9672, 2.3175, 1.1303) -- (2.9644, 2.3190, 1.1803) -- (2.9644, 2.2678, 1.1849) -- cycle;
\fill[blue!15.2, opacity=0.5] (2.9644, 2.2678, 1.1849) -- (2.9644, 2.3190, 1.1803) -- (2.9615, 2.3205, 1.2303) -- (2.9615, 2.2692, 1.2349) -- cycle;
\fill[blue!15.4, opacity=0.5] (2.9615, 2.2692, 1.2349) -- (2.9615, 2.3205, 1.2303) -- (2.9585, 2.3221, 1.2803) -- (2.9585, 2.2707, 1.2849) -- cycle;
\fill[blue!15.5, opacity=0.5] (2.9585, 2.2707, 1.2849) -- (2.9585, 2.3221, 1.2803) -- (2.9555, 2.3237, 1.3303) -- (2.9555, 2.2722, 1.3349) -- cycle;
\fill[blue!15.8, opacity=0.5] (2.9555, 2.2722, 1.3349) -- (2.9555, 2.3237, 1.3303) -- (2.9525, 2.3253, 1.3803) -- (2.9525, 2.2738, 1.3849) -- cycle;
\fill[blue!16.1, opacity=0.5] (2.9525, 2.2738, 1.3849) -- (2.9525, 2.3253, 1.3803) -- (2.9494, 2.3270, 1.4303) -- (2.9494, 2.2753, 1.4349) -- cycle;
\fill[blue!16.5, opacity=0.5] (2.9494, 2.2753, 1.4349) -- (2.9494, 2.3270, 1.4303) -- (2.9463, 2.3287, 1.4803) -- (2.9463, 2.2769, 1.4849) -- cycle;
\fill[blue!17.0, opacity=0.5] (2.9463, 2.2769, 1.4849) -- (2.9463, 2.3287, 1.4803) -- (2.9431, 2.3303, 1.5303) -- (2.9431, 2.2784, 1.5349) -- cycle;
\fill[blue!17.6, opacity=0.5] (2.9431, 2.2784, 1.5349) -- (2.9431, 2.3303, 1.5303) -- (2.9400, 2.3320, 1.5803) -- (2.9400, 2.2800, 1.5849) -- cycle;
\fill[blue!18.4, opacity=0.5] (2.9400, 2.2800, 1.5849) -- (2.9400, 2.3320, 1.5803) -- (2.9369, 2.3337, 1.6303) -- (2.9369, 2.2816, 1.6349) -- cycle;
\fill[blue!19.2, opacity=0.5] (2.9369, 2.2816, 1.6349) -- (2.9369, 2.3337, 1.6303) -- (2.9337, 2.3353, 1.6803) -- (2.9337, 2.2831, 1.6849) -- cycle;
\fill[blue!20.2, opacity=0.5] (2.9337, 2.2831, 1.6849) -- (2.9337, 2.3353, 1.6803) -- (2.9306, 2.3370, 1.7303) -- (2.9306, 2.2847, 1.7349) -- cycle;
\fill[blue!21.4, opacity=0.5] (2.9306, 2.2847, 1.7349) -- (2.9306, 2.3370, 1.7303) -- (2.9275, 2.3387, 1.7803) -- (2.9275, 2.2862, 1.7849) -- cycle;
\fill[blue!22.7, opacity=0.5] (2.9275, 2.2862, 1.7849) -- (2.9275, 2.3387, 1.7803) -- (2.9245, 2.3403, 1.8303) -- (2.9245, 2.2878, 1.8349) -- cycle;
\fill[blue!24.1, opacity=0.5] (2.9245, 2.2878, 1.8349) -- (2.9245, 2.3403, 1.8303) -- (2.9215, 2.3419, 1.8803) -- (2.9215, 2.2893, 1.8849) -- cycle;
\fill[blue!25.6, opacity=0.5] (2.9215, 2.2893, 1.8849) -- (2.9215, 2.3419, 1.8803) -- (2.9185, 2.3435, 1.9303) -- (2.9185, 2.2908, 1.9349) -- cycle;
\fill[blue!27.3, opacity=0.5] (2.9185, 2.2908, 1.9349) -- (2.9185, 2.3435, 1.9303) -- (2.9156, 2.3450, 1.9803) -- (2.9156, 2.2922, 1.9849) -- cycle;
\fill[blue!29.0, opacity=0.5] (2.9156, 2.2922, 1.9849) -- (2.9156, 2.3450, 1.9803) -- (2.9128, 2.3465, 2.0303) -- (2.9128, 2.2936, 2.0349) -- cycle;
\fill[blue!30.8, opacity=0.5] (2.9128, 2.2936, 2.0349) -- (2.9128, 2.3465, 2.0303) -- (2.9100, 2.3480, 2.0803) -- (2.9100, 2.2950, 2.0849) -- cycle;
\fill[blue!32.7, opacity=0.5] (2.9100, 2.2950, 2.0849) -- (2.9100, 2.3480, 2.0803) -- (2.9073, 2.3494, 2.1303) -- (2.9073, 2.2963, 2.1349) -- cycle;
\fill[blue!34.6, opacity=0.5] (2.9073, 2.2963, 2.1349) -- (2.9073, 2.3494, 2.1303) -- (2.9047, 2.3508, 2.1803) -- (2.9047, 2.2976, 2.1849) -- cycle;
\fill[blue!36.5, opacity=0.5] (2.9047, 2.2976, 2.1849) -- (2.9047, 2.3508, 2.1803) -- (2.9022, 2.3521, 2.2303) -- (2.9022, 2.2989, 2.2349) -- cycle;
\fill[blue!38.5, opacity=0.5] (2.9022, 2.2989, 2.2349) -- (2.9022, 2.3521, 2.2303) -- (2.8999, 2.3534, 2.2803) -- (2.8999, 2.3001, 2.2849) -- cycle;
\fill[blue!40.4, opacity=0.5] (2.8999, 2.3001, 2.2849) -- (2.8999, 2.3534, 2.2803) -- (2.8976, 2.3546, 2.3303) -- (2.8976, 2.3012, 2.3349) -- cycle;
\fill[blue!42.2, opacity=0.5] (2.8976, 2.3012, 2.3349) -- (2.8976, 2.3546, 2.3303) -- (2.8954, 2.3558, 2.3803) -- (2.8954, 2.3023, 2.3849) -- cycle;
\fill[blue!44.0, opacity=0.5] (2.8954, 2.3023, 2.3849) -- (2.8954, 2.3558, 2.3803) -- (2.8934, 2.3569, 2.4303) -- (2.8934, 2.3033, 2.4349) -- cycle;
\fill[blue!45.8, opacity=0.5] (2.8934, 2.3033, 2.4349) -- (2.8934, 2.3569, 2.4303) -- (2.8915, 2.3579, 2.4803) -- (2.8915, 2.3043, 2.4849) -- cycle;
\fill[blue!47.4, opacity=0.5] (2.8915, 2.3043, 2.4849) -- (2.8915, 2.3579, 2.4803) -- (2.8897, 2.3588, 2.5303) -- (2.8897, 2.3052, 2.5349) -- cycle;
\fill[blue!48.9, opacity=0.5] (2.8897, 2.3052, 2.5349) -- (2.8897, 2.3588, 2.5303) -- (2.8880, 2.3597, 2.5803) -- (2.8880, 2.3060, 2.5849) -- cycle;
\fill[blue!50.4, opacity=0.5] (2.8880, 2.3060, 2.5849) -- (2.8880, 2.3597, 2.5803) -- (2.8865, 2.3605, 2.6303) -- (2.8865, 2.3067, 2.6349) -- cycle;
\fill[blue!51.6, opacity=0.5] (2.8865, 2.3067, 2.6349) -- (2.8865, 2.3605, 2.6303) -- (2.8852, 2.3612, 2.6803) -- (2.8852, 2.3074, 2.6849) -- cycle;
\fill[blue!52.8, opacity=0.5] (2.8852, 2.3074, 2.6849) -- (2.8852, 2.3612, 2.6803) -- (2.8840, 2.3619, 2.7303) -- (2.8840, 2.3080, 2.7349) -- cycle;
\fill[blue!53.8, opacity=0.5] (2.8840, 2.3080, 2.7349) -- (2.8840, 2.3619, 2.7303) -- (2.8829, 2.3624, 2.7803) -- (2.8829, 2.3085, 2.7849) -- cycle;
\fill[blue!54.7, opacity=0.5] (2.8829, 2.3085, 2.7849) -- (2.8829, 2.3624, 2.7803) -- (2.8820, 2.3629, 2.8303) -- (2.8820, 2.3090, 2.8349) -- cycle;
\fill[blue!55.4, opacity=0.5] (2.8820, 2.3090, 2.8349) -- (2.8820, 2.3629, 2.8303) -- (2.8813, 2.3633, 2.8803) -- (2.8813, 2.3093, 2.8849) -- cycle;
\fill[blue!55.9, opacity=0.5] (2.8813, 2.3093, 2.8849) -- (2.8813, 2.3633, 2.8803) -- (2.8807, 2.3636, 2.9303) -- (2.8807, 2.3096, 2.9349) -- cycle;
\fill[blue!56.4, opacity=0.5] (2.8807, 2.3096, 2.9349) -- (2.8807, 2.3636, 2.9303) -- (2.8803, 2.3638, 2.9803) -- (2.8803, 2.3098, 2.9849) -- cycle;
\fill[blue!56.6, opacity=0.5] (2.8803, 2.3098, 2.9849) -- (2.8803, 2.3638, 2.9803) -- (2.8801, 2.3640, 3.0303) -- (2.8801, 2.3100, 3.0349) -- cycle;
\fill[blue!56.8, opacity=0.5] (2.8801, 2.3100, 3.0349) -- (2.8801, 2.3640, 3.0303) -- (2.8800, 2.3640, 3.0803) -- (2.8800, 2.3100, 3.0849) -- cycle;
\fill[blue!15.0, opacity=0.5] (3.0000, 2.3000, 0.0803) -- (3.0000, 2.3500, 0.0755) -- (2.9999, 2.3500, 0.1255) -- (2.9999, 2.3000, 0.1303) -- cycle;
\fill[blue!15.0, opacity=0.5] (2.9999, 2.3000, 0.1303) -- (2.9999, 2.3500, 0.1255) -- (2.9997, 2.3502, 0.1755) -- (2.9997, 2.3002, 0.1803) -- cycle;
\fill[blue!15.0, opacity=0.5] (2.9997, 2.3002, 0.1803) -- (2.9997, 2.3502, 0.1755) -- (2.9993, 2.3504, 0.2255) -- (2.9993, 2.3004, 0.2303) -- cycle;
\fill[blue!15.0, opacity=0.5] (2.9993, 2.3004, 0.2303) -- (2.9993, 2.3504, 0.2255) -- (2.9987, 2.3507, 0.2755) -- (2.9987, 2.3007, 0.2803) -- cycle;
\fill[blue!15.0, opacity=0.5] (2.9987, 2.3007, 0.2803) -- (2.9987, 2.3507, 0.2755) -- (2.9980, 2.3512, 0.3255) -- (2.9980, 2.3011, 0.3303) -- cycle;
\fill[blue!15.0, opacity=0.5] (2.9980, 2.3011, 0.3303) -- (2.9980, 2.3512, 0.3255) -- (2.9971, 2.3517, 0.3755) -- (2.9971, 2.3016, 0.3803) -- cycle;
\fill[blue!15.0, opacity=0.5] (2.9971, 2.3016, 0.3803) -- (2.9971, 2.3517, 0.3755) -- (2.9960, 2.3523, 0.4255) -- (2.9960, 2.3021, 0.4303) -- cycle;
\fill[blue!15.0, opacity=0.5] (2.9960, 2.3021, 0.4303) -- (2.9960, 2.3523, 0.4255) -- (2.9948, 2.3529, 0.4755) -- (2.9948, 2.3028, 0.4803) -- cycle;
\fill[blue!15.0, opacity=0.5] (2.9948, 2.3028, 0.4803) -- (2.9948, 2.3529, 0.4755) -- (2.9935, 2.3537, 0.5255) -- (2.9935, 2.3035, 0.5303) -- cycle;
\fill[blue!15.0, opacity=0.5] (2.9935, 2.3035, 0.5303) -- (2.9935, 2.3537, 0.5255) -- (2.9920, 2.3546, 0.5755) -- (2.9920, 2.3043, 0.5803) -- cycle;
\fill[blue!15.0, opacity=0.5] (2.9920, 2.3043, 0.5803) -- (2.9920, 2.3546, 0.5755) -- (2.9903, 2.3555, 0.6255) -- (2.9903, 2.3052, 0.6303) -- cycle;
\fill[blue!15.0, opacity=0.5] (2.9903, 2.3052, 0.6303) -- (2.9903, 2.3555, 0.6255) -- (2.9885, 2.3565, 0.6755) -- (2.9885, 2.3061, 0.6803) -- cycle;
\fill[blue!15.0, opacity=0.5] (2.9885, 2.3061, 0.6803) -- (2.9885, 2.3565, 0.6755) -- (2.9866, 2.3576, 0.7255) -- (2.9866, 2.3071, 0.7303) -- cycle;
\fill[blue!15.0, opacity=0.5] (2.9866, 2.3071, 0.7303) -- (2.9866, 2.3576, 0.7255) -- (2.9846, 2.3587, 0.7755) -- (2.9846, 2.3082, 0.7803) -- cycle;
\fill[blue!15.0, opacity=0.5] (2.9846, 2.3082, 0.7803) -- (2.9846, 2.3587, 0.7755) -- (2.9824, 2.3600, 0.8255) -- (2.9824, 2.3094, 0.8303) -- cycle;
\fill[blue!15.0, opacity=0.5] (2.9824, 2.3094, 0.8303) -- (2.9824, 2.3600, 0.8255) -- (2.9801, 2.3612, 0.8755) -- (2.9801, 2.3106, 0.8803) -- cycle;
\fill[blue!15.0, opacity=0.5] (2.9801, 2.3106, 0.8803) -- (2.9801, 2.3612, 0.8755) -- (2.9778, 2.3626, 0.9255) -- (2.9778, 2.3119, 0.9303) -- cycle;
\fill[blue!15.0, opacity=0.5] (2.9778, 2.3119, 0.9303) -- (2.9778, 2.3626, 0.9255) -- (2.9753, 2.3640, 0.9755) -- (2.9753, 2.3132, 0.9803) -- cycle;
\fill[blue!15.0, opacity=0.5] (2.9753, 2.3132, 0.9803) -- (2.9753, 2.3640, 0.9755) -- (2.9727, 2.3655, 1.0255) -- (2.9727, 2.3146, 1.0303) -- cycle;
\fill[blue!15.1, opacity=0.5] (2.9727, 2.3146, 1.0303) -- (2.9727, 2.3655, 1.0255) -- (2.9700, 2.3670, 1.0755) -- (2.9700, 2.3160, 1.0803) -- cycle;
\fill[blue!15.1, opacity=0.5] (2.9700, 2.3160, 1.0803) -- (2.9700, 2.3670, 1.0755) -- (2.9672, 2.3686, 1.1255) -- (2.9672, 2.3175, 1.1303) -- cycle;
\fill[blue!15.2, opacity=0.5] (2.9672, 2.3175, 1.1303) -- (2.9672, 2.3686, 1.1255) -- (2.9644, 2.3702, 1.1755) -- (2.9644, 2.3190, 1.1803) -- cycle;
\fill[blue!15.3, opacity=0.5] (2.9644, 2.3190, 1.1803) -- (2.9644, 2.3702, 1.1755) -- (2.9615, 2.3718, 1.2255) -- (2.9615, 2.3205, 1.2303) -- cycle;
\fill[blue!15.5, opacity=0.5] (2.9615, 2.3205, 1.2303) -- (2.9615, 2.3718, 1.2255) -- (2.9585, 2.3735, 1.2755) -- (2.9585, 2.3221, 1.2803) -- cycle;
\fill[blue!15.7, opacity=0.5] (2.9585, 2.3221, 1.2803) -- (2.9585, 2.3735, 1.2755) -- (2.9555, 2.3752, 1.3255) -- (2.9555, 2.3237, 1.3303) -- cycle;
\fill[blue!16.0, opacity=0.5] (2.9555, 2.3237, 1.3303) -- (2.9555, 2.3752, 1.3255) -- (2.9525, 2.3769, 1.3755) -- (2.9525, 2.3253, 1.3803) -- cycle;
\fill[blue!16.4, opacity=0.5] (2.9525, 2.3253, 1.3803) -- (2.9525, 2.3769, 1.3755) -- (2.9494, 2.3787, 1.4255) -- (2.9494, 2.3270, 1.4303) -- cycle;
\fill[blue!16.8, opacity=0.5] (2.9494, 2.3270, 1.4303) -- (2.9494, 2.3787, 1.4255) -- (2.9463, 2.3804, 1.4755) -- (2.9463, 2.3287, 1.4803) -- cycle;
\fill[blue!17.4, opacity=0.5] (2.9463, 2.3287, 1.4803) -- (2.9463, 2.3804, 1.4755) -- (2.9431, 2.3822, 1.5255) -- (2.9431, 2.3303, 1.5303) -- cycle;
\fill[blue!18.2, opacity=0.5] (2.9431, 2.3303, 1.5303) -- (2.9431, 2.3822, 1.5255) -- (2.9400, 2.3840, 1.5755) -- (2.9400, 2.3320, 1.5803) -- cycle;
\fill[blue!19.0, opacity=0.5] (2.9400, 2.3320, 1.5803) -- (2.9400, 2.3840, 1.5755) -- (2.9369, 2.3858, 1.6255) -- (2.9369, 2.3337, 1.6303) -- cycle;
\fill[blue!20.0, opacity=0.5] (2.9369, 2.3337, 1.6303) -- (2.9369, 2.3858, 1.6255) -- (2.9337, 2.3876, 1.6755) -- (2.9337, 2.3353, 1.6803) -- cycle;
\fill[blue!21.2, opacity=0.5] (2.9337, 2.3353, 1.6803) -- (2.9337, 2.3876, 1.6755) -- (2.9306, 2.3893, 1.7255) -- (2.9306, 2.3370, 1.7303) -- cycle;
\fill[blue!22.4, opacity=0.5] (2.9306, 2.3370, 1.7303) -- (2.9306, 2.3893, 1.7255) -- (2.9275, 2.3911, 1.7755) -- (2.9275, 2.3387, 1.7803) -- cycle;
\fill[blue!23.9, opacity=0.5] (2.9275, 2.3387, 1.7803) -- (2.9275, 2.3911, 1.7755) -- (2.9245, 2.3928, 1.8255) -- (2.9245, 2.3403, 1.8303) -- cycle;
\fill[blue!25.4, opacity=0.5] (2.9245, 2.3403, 1.8303) -- (2.9245, 2.3928, 1.8255) -- (2.9215, 2.3945, 1.8755) -- (2.9215, 2.3419, 1.8803) -- cycle;
\fill[blue!27.1, opacity=0.5] (2.9215, 2.3419, 1.8803) -- (2.9215, 2.3945, 1.8755) -- (2.9185, 2.3962, 1.9255) -- (2.9185, 2.3435, 1.9303) -- cycle;
\fill[blue!28.8, opacity=0.5] (2.9185, 2.3435, 1.9303) -- (2.9185, 2.3962, 1.9255) -- (2.9156, 2.3978, 1.9755) -- (2.9156, 2.3450, 1.9803) -- cycle;
\fill[blue!30.7, opacity=0.5] (2.9156, 2.3450, 1.9803) -- (2.9156, 2.3978, 1.9755) -- (2.9128, 2.3994, 2.0255) -- (2.9128, 2.3465, 2.0303) -- cycle;
\fill[blue!32.6, opacity=0.5] (2.9128, 2.3465, 2.0303) -- (2.9128, 2.3994, 2.0255) -- (2.9100, 2.4010, 2.0755) -- (2.9100, 2.3480, 2.0803) -- cycle;
\fill[blue!34.6, opacity=0.5] (2.9100, 2.3480, 2.0803) -- (2.9100, 2.4010, 2.0755) -- (2.9073, 2.4025, 2.1255) -- (2.9073, 2.3494, 2.1303) -- cycle;
\fill[blue!36.6, opacity=0.5] (2.9073, 2.3494, 2.1303) -- (2.9073, 2.4025, 2.1255) -- (2.9047, 2.4040, 2.1755) -- (2.9047, 2.3508, 2.1803) -- cycle;
\fill[blue!38.6, opacity=0.5] (2.9047, 2.3508, 2.1803) -- (2.9047, 2.4040, 2.1755) -- (2.9022, 2.4054, 2.2255) -- (2.9022, 2.3521, 2.2303) -- cycle;
\fill[blue!40.5, opacity=0.5] (2.9022, 2.3521, 2.2303) -- (2.9022, 2.4054, 2.2255) -- (2.8999, 2.4068, 2.2755) -- (2.8999, 2.3534, 2.2803) -- cycle;
\fill[blue!42.5, opacity=0.5] (2.8999, 2.3534, 2.2803) -- (2.8999, 2.4068, 2.2755) -- (2.8976, 2.4080, 2.3255) -- (2.8976, 2.3546, 2.3303) -- cycle;
\fill[blue!44.4, opacity=0.5] (2.8976, 2.3546, 2.3303) -- (2.8976, 2.4080, 2.3255) -- (2.8954, 2.4093, 2.3755) -- (2.8954, 2.3558, 2.3803) -- cycle;
\fill[blue!46.2, opacity=0.5] (2.8954, 2.3558, 2.3803) -- (2.8954, 2.4093, 2.3755) -- (2.8934, 2.4104, 2.4255) -- (2.8934, 2.3569, 2.4303) -- cycle;
\fill[blue!47.9, opacity=0.5] (2.8934, 2.3569, 2.4303) -- (2.8934, 2.4104, 2.4255) -- (2.8915, 2.4115, 2.4755) -- (2.8915, 2.3579, 2.4803) -- cycle;
\fill[blue!49.5, opacity=0.5] (2.8915, 2.3579, 2.4803) -- (2.8915, 2.4115, 2.4755) -- (2.8897, 2.4125, 2.5255) -- (2.8897, 2.3588, 2.5303) -- cycle;
\fill[blue!51.0, opacity=0.5] (2.8897, 2.3588, 2.5303) -- (2.8897, 2.4125, 2.5255) -- (2.8880, 2.4134, 2.5755) -- (2.8880, 2.3597, 2.5803) -- cycle;
\fill[blue!52.3, opacity=0.5] (2.8880, 2.3597, 2.5803) -- (2.8880, 2.4134, 2.5755) -- (2.8865, 2.4143, 2.6255) -- (2.8865, 2.3605, 2.6303) -- cycle;
\fill[blue!53.6, opacity=0.5] (2.8865, 2.3605, 2.6303) -- (2.8865, 2.4143, 2.6255) -- (2.8852, 2.4151, 2.6755) -- (2.8852, 2.3612, 2.6803) -- cycle;
\fill[blue!54.6, opacity=0.5] (2.8852, 2.3612, 2.6803) -- (2.8852, 2.4151, 2.6755) -- (2.8840, 2.4157, 2.7255) -- (2.8840, 2.3619, 2.7303) -- cycle;
\fill[blue!55.5, opacity=0.5] (2.8840, 2.3619, 2.7303) -- (2.8840, 2.4157, 2.7255) -- (2.8829, 2.4163, 2.7755) -- (2.8829, 2.3624, 2.7803) -- cycle;
\fill[blue!56.3, opacity=0.5] (2.8829, 2.3624, 2.7803) -- (2.8829, 2.4163, 2.7755) -- (2.8820, 2.4168, 2.8255) -- (2.8820, 2.3629, 2.8303) -- cycle;
\fill[blue!56.9, opacity=0.5] (2.8820, 2.3629, 2.8303) -- (2.8820, 2.4168, 2.8255) -- (2.8813, 2.4173, 2.8755) -- (2.8813, 2.3633, 2.8803) -- cycle;
\fill[blue!57.4, opacity=0.5] (2.8813, 2.3633, 2.8803) -- (2.8813, 2.4173, 2.8755) -- (2.8807, 2.4176, 2.9255) -- (2.8807, 2.3636, 2.9303) -- cycle;
\fill[blue!57.7, opacity=0.5] (2.8807, 2.3636, 2.9303) -- (2.8807, 2.4176, 2.9255) -- (2.8803, 2.4178, 2.9755) -- (2.8803, 2.3638, 2.9803) -- cycle;
\fill[blue!57.9, opacity=0.5] (2.8803, 2.3638, 2.9803) -- (2.8803, 2.4178, 2.9755) -- (2.8801, 2.4180, 3.0255) -- (2.8801, 2.3640, 3.0303) -- cycle;
\fill[blue!57.9, opacity=0.5] (2.8801, 2.3640, 3.0303) -- (2.8801, 2.4180, 3.0255) -- (2.8800, 2.4180, 3.0755) -- (2.8800, 2.3640, 3.0803) -- cycle;
\fill[blue!15.0, opacity=0.5] (3.0000, 2.3500, 0.0755) -- (3.0000, 2.4000, 0.0705) -- (2.9999, 2.4000, 0.1205) -- (2.9999, 2.3500, 0.1255) -- cycle;
\fill[blue!15.0, opacity=0.5] (2.9999, 2.3500, 0.1255) -- (2.9999, 2.4000, 0.1205) -- (2.9997, 2.4002, 0.1705) -- (2.9997, 2.3502, 0.1755) -- cycle;
\fill[blue!15.0, opacity=0.5] (2.9997, 2.3502, 0.1755) -- (2.9997, 2.4002, 0.1705) -- (2.9993, 2.4004, 0.2205) -- (2.9993, 2.3504, 0.2255) -- cycle;
\fill[blue!15.0, opacity=0.5] (2.9993, 2.3504, 0.2255) -- (2.9993, 2.4004, 0.2205) -- (2.9987, 2.4008, 0.2705) -- (2.9987, 2.3507, 0.2755) -- cycle;
\fill[blue!15.0, opacity=0.5] (2.9987, 2.3507, 0.2755) -- (2.9987, 2.4008, 0.2705) -- (2.9980, 2.4012, 0.3205) -- (2.9980, 2.3512, 0.3255) -- cycle;
\fill[blue!15.0, opacity=0.5] (2.9980, 2.3512, 0.3255) -- (2.9980, 2.4012, 0.3205) -- (2.9971, 2.4018, 0.3705) -- (2.9971, 2.3517, 0.3755) -- cycle;
\fill[blue!15.0, opacity=0.5] (2.9971, 2.3517, 0.3755) -- (2.9971, 2.4018, 0.3705) -- (2.9960, 2.4024, 0.4205) -- (2.9960, 2.3523, 0.4255) -- cycle;
\fill[blue!15.0, opacity=0.5] (2.9960, 2.3523, 0.4255) -- (2.9960, 2.4024, 0.4205) -- (2.9948, 2.4031, 0.4705) -- (2.9948, 2.3529, 0.4755) -- cycle;
\fill[blue!15.0, opacity=0.5] (2.9948, 2.3529, 0.4755) -- (2.9948, 2.4031, 0.4705) -- (2.9935, 2.4039, 0.5205) -- (2.9935, 2.3537, 0.5255) -- cycle;
\fill[blue!15.0, opacity=0.5] (2.9935, 2.3537, 0.5255) -- (2.9935, 2.4039, 0.5205) -- (2.9920, 2.4048, 0.5705) -- (2.9920, 2.3546, 0.5755) -- cycle;
\fill[blue!15.0, opacity=0.5] (2.9920, 2.3546, 0.5755) -- (2.9920, 2.4048, 0.5705) -- (2.9903, 2.4058, 0.6205) -- (2.9903, 2.3555, 0.6255) -- cycle;
\fill[blue!15.0, opacity=0.5] (2.9903, 2.3555, 0.6255) -- (2.9903, 2.4058, 0.6205) -- (2.9885, 2.4069, 0.6705) -- (2.9885, 2.3565, 0.6755) -- cycle;
\fill[blue!15.0, opacity=0.5] (2.9885, 2.3565, 0.6755) -- (2.9885, 2.4069, 0.6705) -- (2.9866, 2.4080, 0.7205) -- (2.9866, 2.3576, 0.7255) -- cycle;
\fill[blue!15.0, opacity=0.5] (2.9866, 2.3576, 0.7255) -- (2.9866, 2.4080, 0.7205) -- (2.9846, 2.4092, 0.7705) -- (2.9846, 2.3587, 0.7755) -- cycle;
\fill[blue!15.0, opacity=0.5] (2.9846, 2.3587, 0.7755) -- (2.9846, 2.4092, 0.7705) -- (2.9824, 2.4105, 0.8205) -- (2.9824, 2.3600, 0.8255) -- cycle;
\fill[blue!15.0, opacity=0.5] (2.9824, 2.3600, 0.8255) -- (2.9824, 2.4105, 0.8205) -- (2.9801, 2.4119, 0.8705) -- (2.9801, 2.3612, 0.8755) -- cycle;
\fill[blue!15.0, opacity=0.5] (2.9801, 2.3612, 0.8755) -- (2.9801, 2.4119, 0.8705) -- (2.9778, 2.4133, 0.9205) -- (2.9778, 2.3626, 0.9255) -- cycle;
\fill[blue!15.0, opacity=0.5] (2.9778, 2.3626, 0.9255) -- (2.9778, 2.4133, 0.9205) -- (2.9753, 2.4148, 0.9705) -- (2.9753, 2.3640, 0.9755) -- cycle;
\fill[blue!15.0, opacity=0.5] (2.9753, 2.3640, 0.9755) -- (2.9753, 2.4148, 0.9705) -- (2.9727, 2.4164, 1.0205) -- (2.9727, 2.3655, 1.0255) -- cycle;
\fill[blue!15.0, opacity=0.5] (2.9727, 2.3655, 1.0255) -- (2.9727, 2.4164, 1.0205) -- (2.9700, 2.4180, 1.0705) -- (2.9700, 2.3670, 1.0755) -- cycle;
\fill[blue!15.0, opacity=0.5] (2.9700, 2.3670, 1.0755) -- (2.9700, 2.4180, 1.0705) -- (2.9672, 2.4197, 1.1205) -- (2.9672, 2.3686, 1.1255) -- cycle;
\fill[blue!15.1, opacity=0.5] (2.9672, 2.3686, 1.1255) -- (2.9672, 2.4197, 1.1205) -- (2.9644, 2.4214, 1.1705) -- (2.9644, 2.3702, 1.1755) -- cycle;
\fill[blue!15.1, opacity=0.5] (2.9644, 2.3702, 1.1755) -- (2.9644, 2.4214, 1.1705) -- (2.9615, 2.4231, 1.2205) -- (2.9615, 2.3718, 1.2255) -- cycle;
\fill[blue!15.2, opacity=0.5] (2.9615, 2.3718, 1.2255) -- (2.9615, 2.4231, 1.2205) -- (2.9585, 2.4249, 1.2705) -- (2.9585, 2.3735, 1.2755) -- cycle;
\fill[blue!15.3, opacity=0.5] (2.9585, 2.3735, 1.2755) -- (2.9585, 2.4249, 1.2705) -- (2.9555, 2.4267, 1.3205) -- (2.9555, 2.3752, 1.3255) -- cycle;
\fill[blue!15.5, opacity=0.5] (2.9555, 2.3752, 1.3255) -- (2.9555, 2.4267, 1.3205) -- (2.9525, 2.4285, 1.3705) -- (2.9525, 2.3769, 1.3755) -- cycle;
\fill[blue!15.7, opacity=0.5] (2.9525, 2.3769, 1.3755) -- (2.9525, 2.4285, 1.3705) -- (2.9494, 2.4304, 1.4205) -- (2.9494, 2.3787, 1.4255) -- cycle;
\fill[blue!15.9, opacity=0.5] (2.9494, 2.3787, 1.4255) -- (2.9494, 2.4304, 1.4205) -- (2.9463, 2.4322, 1.4705) -- (2.9463, 2.3804, 1.4755) -- cycle;
\fill[blue!16.3, opacity=0.5] (2.9463, 2.3804, 1.4755) -- (2.9463, 2.4322, 1.4705) -- (2.9431, 2.4341, 1.5205) -- (2.9431, 2.3822, 1.5255) -- cycle;
\fill[blue!16.7, opacity=0.5] (2.9431, 2.3822, 1.5255) -- (2.9431, 2.4341, 1.5205) -- (2.9400, 2.4360, 1.5705) -- (2.9400, 2.3840, 1.5755) -- cycle;
\fill[blue!17.3, opacity=0.5] (2.9400, 2.3840, 1.5755) -- (2.9400, 2.4360, 1.5705) -- (2.9369, 2.4379, 1.6205) -- (2.9369, 2.3858, 1.6255) -- cycle;
\fill[blue!17.9, opacity=0.5] (2.9369, 2.3858, 1.6255) -- (2.9369, 2.4379, 1.6205) -- (2.9337, 2.4398, 1.6705) -- (2.9337, 2.3876, 1.6755) -- cycle;
\fill[blue!18.7, opacity=0.5] (2.9337, 2.3876, 1.6755) -- (2.9337, 2.4398, 1.6705) -- (2.9306, 2.4416, 1.7205) -- (2.9306, 2.3893, 1.7255) -- cycle;
\fill[blue!19.6, opacity=0.5] (2.9306, 2.3893, 1.7255) -- (2.9306, 2.4416, 1.7205) -- (2.9275, 2.4435, 1.7705) -- (2.9275, 2.3911, 1.7755) -- cycle;
\fill[blue!20.6, opacity=0.5] (2.9275, 2.3911, 1.7755) -- (2.9275, 2.4435, 1.7705) -- (2.9245, 2.4453, 1.8205) -- (2.9245, 2.3928, 1.8255) -- cycle;
\fill[blue!21.7, opacity=0.5] (2.9245, 2.3928, 1.8255) -- (2.9245, 2.4453, 1.8205) -- (2.9215, 2.4471, 1.8705) -- (2.9215, 2.3945, 1.8755) -- cycle;
\fill[blue!23.0, opacity=0.5] (2.9215, 2.3945, 1.8755) -- (2.9215, 2.4471, 1.8705) -- (2.9185, 2.4489, 1.9205) -- (2.9185, 2.3962, 1.9255) -- cycle;
\fill[blue!24.3, opacity=0.5] (2.9185, 2.3962, 1.9255) -- (2.9185, 2.4489, 1.9205) -- (2.9156, 2.4506, 1.9705) -- (2.9156, 2.3978, 1.9755) -- cycle;
\fill[blue!25.8, opacity=0.5] (2.9156, 2.3978, 1.9755) -- (2.9156, 2.4506, 1.9705) -- (2.9128, 2.4523, 2.0205) -- (2.9128, 2.3994, 2.0255) -- cycle;
\fill[blue!27.4, opacity=0.5] (2.9128, 2.3994, 2.0255) -- (2.9128, 2.4523, 2.0205) -- (2.9100, 2.4540, 2.0705) -- (2.9100, 2.4010, 2.0755) -- cycle;
\fill[blue!29.1, opacity=0.5] (2.9100, 2.4010, 2.0755) -- (2.9100, 2.4540, 2.0705) -- (2.9073, 2.4556, 2.1205) -- (2.9073, 2.4025, 2.1255) -- cycle;
\fill[blue!30.8, opacity=0.5] (2.9073, 2.4025, 2.1255) -- (2.9073, 2.4556, 2.1205) -- (2.9047, 2.4572, 2.1705) -- (2.9047, 2.4040, 2.1755) -- cycle;
\fill[blue!32.6, opacity=0.5] (2.9047, 2.4040, 2.1755) -- (2.9047, 2.4572, 2.1705) -- (2.9022, 2.4587, 2.2205) -- (2.9022, 2.4054, 2.2255) -- cycle;
\fill[blue!34.4, opacity=0.5] (2.9022, 2.4054, 2.2255) -- (2.9022, 2.4587, 2.2205) -- (2.8999, 2.4601, 2.2705) -- (2.8999, 2.4068, 2.2755) -- cycle;
\fill[blue!36.2, opacity=0.5] (2.8999, 2.4068, 2.2755) -- (2.8999, 2.4601, 2.2705) -- (2.8976, 2.4615, 2.3205) -- (2.8976, 2.4080, 2.3255) -- cycle;
\fill[blue!38.0, opacity=0.5] (2.8976, 2.4080, 2.3255) -- (2.8976, 2.4615, 2.3205) -- (2.8954, 2.4628, 2.3705) -- (2.8954, 2.4093, 2.3755) -- cycle;
\fill[blue!39.7, opacity=0.5] (2.8954, 2.4093, 2.3755) -- (2.8954, 2.4628, 2.3705) -- (2.8934, 2.4640, 2.4205) -- (2.8934, 2.4104, 2.4255) -- cycle;
\fill[blue!41.5, opacity=0.5] (2.8934, 2.4104, 2.4255) -- (2.8934, 2.4640, 2.4205) -- (2.8915, 2.4651, 2.4705) -- (2.8915, 2.4115, 2.4755) -- cycle;
\fill[blue!43.1, opacity=0.5] (2.8915, 2.4115, 2.4755) -- (2.8915, 2.4651, 2.4705) -- (2.8897, 2.4662, 2.5205) -- (2.8897, 2.4125, 2.5255) -- cycle;
\fill[blue!44.7, opacity=0.5] (2.8897, 2.4125, 2.5255) -- (2.8897, 2.4662, 2.5205) -- (2.8880, 2.4672, 2.5705) -- (2.8880, 2.4134, 2.5755) -- cycle;
\fill[blue!46.2, opacity=0.5] (2.8880, 2.4134, 2.5755) -- (2.8880, 2.4672, 2.5705) -- (2.8865, 2.4681, 2.6205) -- (2.8865, 2.4143, 2.6255) -- cycle;
\fill[blue!47.6, opacity=0.5] (2.8865, 2.4143, 2.6255) -- (2.8865, 2.4681, 2.6205) -- (2.8852, 2.4689, 2.6705) -- (2.8852, 2.4151, 2.6755) -- cycle;
\fill[blue!48.9, opacity=0.5] (2.8852, 2.4151, 2.6755) -- (2.8852, 2.4689, 2.6705) -- (2.8840, 2.4696, 2.7205) -- (2.8840, 2.4157, 2.7255) -- cycle;
\fill[blue!50.0, opacity=0.5] (2.8840, 2.4157, 2.7255) -- (2.8840, 2.4696, 2.7205) -- (2.8829, 2.4702, 2.7705) -- (2.8829, 2.4163, 2.7755) -- cycle;
\fill[blue!51.1, opacity=0.5] (2.8829, 2.4163, 2.7755) -- (2.8829, 2.4702, 2.7705) -- (2.8820, 2.4708, 2.8205) -- (2.8820, 2.4168, 2.8255) -- cycle;
\fill[blue!52.0, opacity=0.5] (2.8820, 2.4168, 2.8255) -- (2.8820, 2.4708, 2.8205) -- (2.8813, 2.4712, 2.8705) -- (2.8813, 2.4173, 2.8755) -- cycle;
\fill[blue!52.7, opacity=0.5] (2.8813, 2.4173, 2.8755) -- (2.8813, 2.4712, 2.8705) -- (2.8807, 2.4716, 2.9205) -- (2.8807, 2.4176, 2.9255) -- cycle;
\fill[blue!53.3, opacity=0.5] (2.8807, 2.4176, 2.9255) -- (2.8807, 2.4716, 2.9205) -- (2.8803, 2.4718, 2.9705) -- (2.8803, 2.4178, 2.9755) -- cycle;
\fill[blue!53.8, opacity=0.5] (2.8803, 2.4178, 2.9755) -- (2.8803, 2.4718, 2.9705) -- (2.8801, 2.4720, 3.0205) -- (2.8801, 2.4180, 3.0255) -- cycle;
\fill[blue!54.2, opacity=0.5] (2.8801, 2.4180, 3.0255) -- (2.8801, 2.4720, 3.0205) -- (2.8800, 2.4720, 3.0705) -- (2.8800, 2.4180, 3.0755) -- cycle;
\fill[blue!15.0, opacity=0.5] (3.0000, 2.4000, 0.0705) -- (3.0000, 2.4500, 0.0654) -- (2.9999, 2.4501, 0.1154) -- (2.9999, 2.4000, 0.1205) -- cycle;
\fill[blue!15.0, opacity=0.5] (2.9999, 2.4000, 0.1205) -- (2.9999, 2.4501, 0.1154) -- (2.9997, 2.4502, 0.1654) -- (2.9997, 2.4002, 0.1705) -- cycle;
\fill[blue!15.0, opacity=0.5] (2.9997, 2.4002, 0.1705) -- (2.9997, 2.4502, 0.1654) -- (2.9993, 2.4505, 0.2154) -- (2.9993, 2.4004, 0.2205) -- cycle;
\fill[blue!15.0, opacity=0.5] (2.9993, 2.4004, 0.2205) -- (2.9993, 2.4505, 0.2154) -- (2.9987, 2.4508, 0.2654) -- (2.9987, 2.4008, 0.2705) -- cycle;
\fill[blue!15.0, opacity=0.5] (2.9987, 2.4008, 0.2705) -- (2.9987, 2.4508, 0.2654) -- (2.9980, 2.4513, 0.3154) -- (2.9980, 2.4012, 0.3205) -- cycle;
\fill[blue!15.0, opacity=0.5] (2.9980, 2.4012, 0.3205) -- (2.9980, 2.4513, 0.3154) -- (2.9971, 2.4519, 0.3654) -- (2.9971, 2.4018, 0.3705) -- cycle;
\fill[blue!15.0, opacity=0.5] (2.9971, 2.4018, 0.3705) -- (2.9971, 2.4519, 0.3654) -- (2.9960, 2.4525, 0.4154) -- (2.9960, 2.4024, 0.4205) -- cycle;
\fill[blue!15.0, opacity=0.5] (2.9960, 2.4024, 0.4205) -- (2.9960, 2.4525, 0.4154) -- (2.9948, 2.4533, 0.4654) -- (2.9948, 2.4031, 0.4705) -- cycle;
\fill[blue!15.0, opacity=0.5] (2.9948, 2.4031, 0.4705) -- (2.9948, 2.4533, 0.4654) -- (2.9935, 2.4541, 0.5154) -- (2.9935, 2.4039, 0.5205) -- cycle;
\fill[blue!15.0, opacity=0.5] (2.9935, 2.4039, 0.5205) -- (2.9935, 2.4541, 0.5154) -- (2.9920, 2.4551, 0.5654) -- (2.9920, 2.4048, 0.5705) -- cycle;
\fill[blue!15.0, opacity=0.5] (2.9920, 2.4048, 0.5705) -- (2.9920, 2.4551, 0.5654) -- (2.9903, 2.4561, 0.6154) -- (2.9903, 2.4058, 0.6205) -- cycle;
\fill[blue!15.0, opacity=0.5] (2.9903, 2.4058, 0.6205) -- (2.9903, 2.4561, 0.6154) -- (2.9885, 2.4573, 0.6654) -- (2.9885, 2.4069, 0.6705) -- cycle;
\fill[blue!15.0, opacity=0.5] (2.9885, 2.4069, 0.6705) -- (2.9885, 2.4573, 0.6654) -- (2.9866, 2.4585, 0.7154) -- (2.9866, 2.4080, 0.7205) -- cycle;
\fill[blue!15.0, opacity=0.5] (2.9866, 2.4080, 0.7205) -- (2.9866, 2.4585, 0.7154) -- (2.9846, 2.4598, 0.7654) -- (2.9846, 2.4092, 0.7705) -- cycle;
\fill[blue!15.0, opacity=0.5] (2.9846, 2.4092, 0.7705) -- (2.9846, 2.4598, 0.7654) -- (2.9824, 2.4611, 0.8154) -- (2.9824, 2.4105, 0.8205) -- cycle;
\fill[blue!15.0, opacity=0.5] (2.9824, 2.4105, 0.8205) -- (2.9824, 2.4611, 0.8154) -- (2.9801, 2.4626, 0.8654) -- (2.9801, 2.4119, 0.8705) -- cycle;
\fill[blue!15.0, opacity=0.5] (2.9801, 2.4119, 0.8705) -- (2.9801, 2.4626, 0.8654) -- (2.9778, 2.4641, 0.9154) -- (2.9778, 2.4133, 0.9205) -- cycle;
\fill[blue!15.0, opacity=0.5] (2.9778, 2.4133, 0.9205) -- (2.9778, 2.4641, 0.9154) -- (2.9753, 2.4657, 0.9654) -- (2.9753, 2.4148, 0.9705) -- cycle;
\fill[blue!15.0, opacity=0.5] (2.9753, 2.4148, 0.9705) -- (2.9753, 2.4657, 0.9654) -- (2.9727, 2.4673, 1.0154) -- (2.9727, 2.4164, 1.0205) -- cycle;
\fill[blue!15.0, opacity=0.5] (2.9727, 2.4164, 1.0205) -- (2.9727, 2.4673, 1.0154) -- (2.9700, 2.4690, 1.0654) -- (2.9700, 2.4180, 1.0705) -- cycle;
\fill[blue!15.0, opacity=0.5] (2.9700, 2.4180, 1.0705) -- (2.9700, 2.4690, 1.0654) -- (2.9672, 2.4707, 1.1154) -- (2.9672, 2.4197, 1.1205) -- cycle;
\fill[blue!15.0, opacity=0.5] (2.9672, 2.4197, 1.1205) -- (2.9672, 2.4707, 1.1154) -- (2.9644, 2.4725, 1.1654) -- (2.9644, 2.4214, 1.1705) -- cycle;
\fill[blue!15.0, opacity=0.5] (2.9644, 2.4214, 1.1705) -- (2.9644, 2.4725, 1.1654) -- (2.9615, 2.4744, 1.2154) -- (2.9615, 2.4231, 1.2205) -- cycle;
\fill[blue!15.0, opacity=0.5] (2.9615, 2.4231, 1.2205) -- (2.9615, 2.4744, 1.2154) -- (2.9585, 2.4763, 1.2654) -- (2.9585, 2.4249, 1.2705) -- cycle;
\fill[blue!15.0, opacity=0.5] (2.9585, 2.4249, 1.2705) -- (2.9585, 2.4763, 1.2654) -- (2.9555, 2.4782, 1.3154) -- (2.9555, 2.4267, 1.3205) -- cycle;
\fill[blue!15.1, opacity=0.5] (2.9555, 2.4267, 1.3205) -- (2.9555, 2.4782, 1.3154) -- (2.9525, 2.4801, 1.3654) -- (2.9525, 2.4285, 1.3705) -- cycle;
\fill[blue!15.1, opacity=0.5] (2.9525, 2.4285, 1.3705) -- (2.9525, 2.4801, 1.3654) -- (2.9494, 2.4821, 1.4154) -- (2.9494, 2.4304, 1.4205) -- cycle;
\fill[blue!15.1, opacity=0.5] (2.9494, 2.4304, 1.4205) -- (2.9494, 2.4821, 1.4154) -- (2.9463, 2.4840, 1.4654) -- (2.9463, 2.4322, 1.4705) -- cycle;
\fill[blue!15.2, opacity=0.5] (2.9463, 2.4322, 1.4705) -- (2.9463, 2.4840, 1.4654) -- (2.9431, 2.4860, 1.5154) -- (2.9431, 2.4341, 1.5205) -- cycle;
\fill[blue!15.3, opacity=0.5] (2.9431, 2.4341, 1.5205) -- (2.9431, 2.4860, 1.5154) -- (2.9400, 2.4880, 1.5654) -- (2.9400, 2.4360, 1.5705) -- cycle;
\fill[blue!15.5, opacity=0.5] (2.9400, 2.4360, 1.5705) -- (2.9400, 2.4880, 1.5654) -- (2.9369, 2.4900, 1.6154) -- (2.9369, 2.4379, 1.6205) -- cycle;
\fill[blue!15.7, opacity=0.5] (2.9369, 2.4379, 1.6205) -- (2.9369, 2.4900, 1.6154) -- (2.9337, 2.4920, 1.6654) -- (2.9337, 2.4398, 1.6705) -- cycle;
\fill[blue!15.9, opacity=0.5] (2.9337, 2.4398, 1.6705) -- (2.9337, 2.4920, 1.6654) -- (2.9306, 2.4939, 1.7154) -- (2.9306, 2.4416, 1.7205) -- cycle;
\fill[blue!16.2, opacity=0.5] (2.9306, 2.4416, 1.7205) -- (2.9306, 2.4939, 1.7154) -- (2.9275, 2.4959, 1.7654) -- (2.9275, 2.4435, 1.7705) -- cycle;
\fill[blue!16.6, opacity=0.5] (2.9275, 2.4435, 1.7705) -- (2.9275, 2.4959, 1.7654) -- (2.9245, 2.4978, 1.8154) -- (2.9245, 2.4453, 1.8205) -- cycle;
\fill[blue!17.0, opacity=0.5] (2.9245, 2.4453, 1.8205) -- (2.9245, 2.4978, 1.8154) -- (2.9215, 2.4997, 1.8654) -- (2.9215, 2.4471, 1.8705) -- cycle;
\fill[blue!17.5, opacity=0.5] (2.9215, 2.4471, 1.8705) -- (2.9215, 2.4997, 1.8654) -- (2.9185, 2.5016, 1.9154) -- (2.9185, 2.4489, 1.9205) -- cycle;
\fill[blue!18.2, opacity=0.5] (2.9185, 2.4489, 1.9205) -- (2.9185, 2.5016, 1.9154) -- (2.9156, 2.5035, 1.9654) -- (2.9156, 2.4506, 1.9705) -- cycle;
\fill[blue!18.9, opacity=0.5] (2.9156, 2.4506, 1.9705) -- (2.9156, 2.5035, 1.9654) -- (2.9128, 2.5053, 2.0154) -- (2.9128, 2.4523, 2.0205) -- cycle;
\fill[blue!19.7, opacity=0.5] (2.9128, 2.4523, 2.0205) -- (2.9128, 2.5053, 2.0154) -- (2.9100, 2.5070, 2.0654) -- (2.9100, 2.4540, 2.0705) -- cycle;
\fill[blue!20.6, opacity=0.5] (2.9100, 2.4540, 2.0705) -- (2.9100, 2.5070, 2.0654) -- (2.9073, 2.5087, 2.1154) -- (2.9073, 2.4556, 2.1205) -- cycle;
\fill[blue!21.6, opacity=0.5] (2.9073, 2.4556, 2.1205) -- (2.9073, 2.5087, 2.1154) -- (2.9047, 2.5103, 2.1654) -- (2.9047, 2.4572, 2.1705) -- cycle;
\fill[blue!22.7, opacity=0.5] (2.9047, 2.4572, 2.1705) -- (2.9047, 2.5103, 2.1654) -- (2.9022, 2.5119, 2.2154) -- (2.9022, 2.4587, 2.2205) -- cycle;
\fill[blue!23.8, opacity=0.5] (2.9022, 2.4587, 2.2205) -- (2.9022, 2.5119, 2.2154) -- (2.8999, 2.5134, 2.2654) -- (2.8999, 2.4601, 2.2705) -- cycle;
\fill[blue!25.0, opacity=0.5] (2.8999, 2.4601, 2.2705) -- (2.8999, 2.5134, 2.2654) -- (2.8976, 2.5149, 2.3154) -- (2.8976, 2.4615, 2.3205) -- cycle;
\fill[blue!26.3, opacity=0.5] (2.8976, 2.4615, 2.3205) -- (2.8976, 2.5149, 2.3154) -- (2.8954, 2.5162, 2.3654) -- (2.8954, 2.4628, 2.3705) -- cycle;
\fill[blue!27.7, opacity=0.5] (2.8954, 2.4628, 2.3705) -- (2.8954, 2.5162, 2.3654) -- (2.8934, 2.5175, 2.4154) -- (2.8934, 2.4640, 2.4205) -- cycle;
\fill[blue!29.1, opacity=0.5] (2.8934, 2.4640, 2.4205) -- (2.8934, 2.5175, 2.4154) -- (2.8915, 2.5187, 2.4654) -- (2.8915, 2.4651, 2.4705) -- cycle;
\fill[blue!30.5, opacity=0.5] (2.8915, 2.4651, 2.4705) -- (2.8915, 2.5187, 2.4654) -- (2.8897, 2.5199, 2.5154) -- (2.8897, 2.4662, 2.5205) -- cycle;
\fill[blue!31.9, opacity=0.5] (2.8897, 2.4662, 2.5205) -- (2.8897, 2.5199, 2.5154) -- (2.8880, 2.5209, 2.5654) -- (2.8880, 2.4672, 2.5705) -- cycle;
\fill[blue!33.3, opacity=0.5] (2.8880, 2.4672, 2.5705) -- (2.8880, 2.5209, 2.5654) -- (2.8865, 2.5219, 2.6154) -- (2.8865, 2.4681, 2.6205) -- cycle;
\fill[blue!34.7, opacity=0.5] (2.8865, 2.4681, 2.6205) -- (2.8865, 2.5219, 2.6154) -- (2.8852, 2.5227, 2.6654) -- (2.8852, 2.4689, 2.6705) -- cycle;
\fill[blue!36.1, opacity=0.5] (2.8852, 2.4689, 2.6705) -- (2.8852, 2.5227, 2.6654) -- (2.8840, 2.5235, 2.7154) -- (2.8840, 2.4696, 2.7205) -- cycle;
\fill[blue!37.4, opacity=0.5] (2.8840, 2.4696, 2.7205) -- (2.8840, 2.5235, 2.7154) -- (2.8829, 2.5241, 2.7654) -- (2.8829, 2.4702, 2.7705) -- cycle;
\fill[blue!38.7, opacity=0.5] (2.8829, 2.4702, 2.7705) -- (2.8829, 2.5241, 2.7654) -- (2.8820, 2.5247, 2.8154) -- (2.8820, 2.4708, 2.8205) -- cycle;
\fill[blue!39.9, opacity=0.5] (2.8820, 2.4708, 2.8205) -- (2.8820, 2.5247, 2.8154) -- (2.8813, 2.5252, 2.8654) -- (2.8813, 2.4712, 2.8705) -- cycle;
\fill[blue!41.0, opacity=0.5] (2.8813, 2.4712, 2.8705) -- (2.8813, 2.5252, 2.8654) -- (2.8807, 2.5255, 2.9154) -- (2.8807, 2.4716, 2.9205) -- cycle;
\fill[blue!42.1, opacity=0.5] (2.8807, 2.4716, 2.9205) -- (2.8807, 2.5255, 2.9154) -- (2.8803, 2.5258, 2.9654) -- (2.8803, 2.4718, 2.9705) -- cycle;
\fill[blue!43.0, opacity=0.5] (2.8803, 2.4718, 2.9705) -- (2.8803, 2.5258, 2.9654) -- (2.8801, 2.5259, 3.0154) -- (2.8801, 2.4720, 3.0205) -- cycle;
\fill[blue!43.9, opacity=0.5] (2.8801, 2.4720, 3.0205) -- (2.8801, 2.5259, 3.0154) -- (2.8800, 2.5260, 3.0654) -- (2.8800, 2.4720, 3.0705) -- cycle;
\fill[blue!15.0, opacity=0.5] (3.0000, 2.4500, 0.0654) -- (3.0000, 2.5000, 0.0600) -- (2.9999, 2.5001, 0.1100) -- (2.9999, 2.4501, 0.1154) -- cycle;
\fill[blue!15.0, opacity=0.5] (2.9999, 2.4501, 0.1154) -- (2.9999, 2.5001, 0.1100) -- (2.9997, 2.5002, 0.1600) -- (2.9997, 2.4502, 0.1654) -- cycle;
\fill[blue!15.0, opacity=0.5] (2.9997, 2.4502, 0.1654) -- (2.9997, 2.5002, 0.1600) -- (2.9993, 2.5005, 0.2100) -- (2.9993, 2.4505, 0.2154) -- cycle;
\fill[blue!15.0, opacity=0.5] (2.9993, 2.4505, 0.2154) -- (2.9993, 2.5005, 0.2100) -- (2.9987, 2.5009, 0.2600) -- (2.9987, 2.4508, 0.2654) -- cycle;
\fill[blue!15.0, opacity=0.5] (2.9987, 2.4508, 0.2654) -- (2.9987, 2.5009, 0.2600) -- (2.9980, 2.5014, 0.3100) -- (2.9980, 2.4513, 0.3154) -- cycle;
\fill[blue!15.0, opacity=0.5] (2.9980, 2.4513, 0.3154) -- (2.9980, 2.5014, 0.3100) -- (2.9971, 2.5020, 0.3600) -- (2.9971, 2.4519, 0.3654) -- cycle;
\fill[blue!15.0, opacity=0.5] (2.9971, 2.4519, 0.3654) -- (2.9971, 2.5020, 0.3600) -- (2.9960, 2.5027, 0.4100) -- (2.9960, 2.4525, 0.4154) -- cycle;
\fill[blue!15.0, opacity=0.5] (2.9960, 2.4525, 0.4154) -- (2.9960, 2.5027, 0.4100) -- (2.9948, 2.5035, 0.4600) -- (2.9948, 2.4533, 0.4654) -- cycle;
\fill[blue!15.0, opacity=0.5] (2.9948, 2.4533, 0.4654) -- (2.9948, 2.5035, 0.4600) -- (2.9935, 2.5044, 0.5100) -- (2.9935, 2.4541, 0.5154) -- cycle;
\fill[blue!15.0, opacity=0.5] (2.9935, 2.4541, 0.5154) -- (2.9935, 2.5044, 0.5100) -- (2.9920, 2.5054, 0.5600) -- (2.9920, 2.4551, 0.5654) -- cycle;
\fill[blue!15.0, opacity=0.5] (2.9920, 2.4551, 0.5654) -- (2.9920, 2.5054, 0.5600) -- (2.9903, 2.5065, 0.6100) -- (2.9903, 2.4561, 0.6154) -- cycle;
\fill[blue!15.0, opacity=0.5] (2.9903, 2.4561, 0.6154) -- (2.9903, 2.5065, 0.6100) -- (2.9885, 2.5076, 0.6600) -- (2.9885, 2.4573, 0.6654) -- cycle;
\fill[blue!15.0, opacity=0.5] (2.9885, 2.4573, 0.6654) -- (2.9885, 2.5076, 0.6600) -- (2.9866, 2.5089, 0.7100) -- (2.9866, 2.4585, 0.7154) -- cycle;
\fill[blue!15.0, opacity=0.5] (2.9866, 2.4585, 0.7154) -- (2.9866, 2.5089, 0.7100) -- (2.9846, 2.5103, 0.7600) -- (2.9846, 2.4598, 0.7654) -- cycle;
\fill[blue!15.0, opacity=0.5] (2.9846, 2.4598, 0.7654) -- (2.9846, 2.5103, 0.7600) -- (2.9824, 2.5117, 0.8100) -- (2.9824, 2.4611, 0.8154) -- cycle;
\fill[blue!15.0, opacity=0.5] (2.9824, 2.4611, 0.8154) -- (2.9824, 2.5117, 0.8100) -- (2.9801, 2.5132, 0.8600) -- (2.9801, 2.4626, 0.8654) -- cycle;
\fill[blue!15.0, opacity=0.5] (2.9801, 2.4626, 0.8654) -- (2.9801, 2.5132, 0.8600) -- (2.9778, 2.5148, 0.9100) -- (2.9778, 2.4641, 0.9154) -- cycle;
\fill[blue!15.0, opacity=0.5] (2.9778, 2.4641, 0.9154) -- (2.9778, 2.5148, 0.9100) -- (2.9753, 2.5165, 0.9600) -- (2.9753, 2.4657, 0.9654) -- cycle;
\fill[blue!15.0, opacity=0.5] (2.9753, 2.4657, 0.9654) -- (2.9753, 2.5165, 0.9600) -- (2.9727, 2.5182, 1.0100) -- (2.9727, 2.4673, 1.0154) -- cycle;
\fill[blue!15.0, opacity=0.5] (2.9727, 2.4673, 1.0154) -- (2.9727, 2.5182, 1.0100) -- (2.9700, 2.5200, 1.0600) -- (2.9700, 2.4690, 1.0654) -- cycle;
\fill[blue!15.0, opacity=0.5] (2.9700, 2.4690, 1.0654) -- (2.9700, 2.5200, 1.0600) -- (2.9672, 2.5218, 1.1100) -- (2.9672, 2.4707, 1.1154) -- cycle;
\fill[blue!15.0, opacity=0.5] (2.9672, 2.4707, 1.1154) -- (2.9672, 2.5218, 1.1100) -- (2.9644, 2.5237, 1.1600) -- (2.9644, 2.4725, 1.1654) -- cycle;
\fill[blue!15.0, opacity=0.5] (2.9644, 2.4725, 1.1654) -- (2.9644, 2.5237, 1.1600) -- (2.9615, 2.5257, 1.2100) -- (2.9615, 2.4744, 1.2154) -- cycle;
\fill[blue!15.0, opacity=0.5] (2.9615, 2.4744, 1.2154) -- (2.9615, 2.5257, 1.2100) -- (2.9585, 2.5276, 1.2600) -- (2.9585, 2.4763, 1.2654) -- cycle;
\fill[blue!15.0, opacity=0.5] (2.9585, 2.4763, 1.2654) -- (2.9585, 2.5276, 1.2600) -- (2.9555, 2.5296, 1.3100) -- (2.9555, 2.4782, 1.3154) -- cycle;
\fill[blue!15.0, opacity=0.5] (2.9555, 2.4782, 1.3154) -- (2.9555, 2.5296, 1.3100) -- (2.9525, 2.5317, 1.3600) -- (2.9525, 2.4801, 1.3654) -- cycle;
\fill[blue!15.0, opacity=0.5] (2.9525, 2.4801, 1.3654) -- (2.9525, 2.5317, 1.3600) -- (2.9494, 2.5337, 1.4100) -- (2.9494, 2.4821, 1.4154) -- cycle;
\fill[blue!15.0, opacity=0.5] (2.9494, 2.4821, 1.4154) -- (2.9494, 2.5337, 1.4100) -- (2.9463, 2.5358, 1.4600) -- (2.9463, 2.4840, 1.4654) -- cycle;
\fill[blue!15.0, opacity=0.5] (2.9463, 2.4840, 1.4654) -- (2.9463, 2.5358, 1.4600) -- (2.9431, 2.5379, 1.5100) -- (2.9431, 2.4860, 1.5154) -- cycle;
\fill[blue!15.0, opacity=0.5] (2.9431, 2.4860, 1.5154) -- (2.9431, 2.5379, 1.5100) -- (2.9400, 2.5400, 1.5600) -- (2.9400, 2.4880, 1.5654) -- cycle;
\fill[blue!15.0, opacity=0.5] (2.9400, 2.4880, 1.5654) -- (2.9400, 2.5400, 1.5600) -- (2.9369, 2.5421, 1.6100) -- (2.9369, 2.4900, 1.6154) -- cycle;
\fill[blue!15.0, opacity=0.5] (2.9369, 2.4900, 1.6154) -- (2.9369, 2.5421, 1.6100) -- (2.9337, 2.5442, 1.6600) -- (2.9337, 2.4920, 1.6654) -- cycle;
\fill[blue!15.1, opacity=0.5] (2.9337, 2.4920, 1.6654) -- (2.9337, 2.5442, 1.6600) -- (2.9306, 2.5463, 1.7100) -- (2.9306, 2.4939, 1.7154) -- cycle;
\fill[blue!15.1, opacity=0.5] (2.9306, 2.4939, 1.7154) -- (2.9306, 2.5463, 1.7100) -- (2.9275, 2.5483, 1.7600) -- (2.9275, 2.4959, 1.7654) -- cycle;
\fill[blue!15.2, opacity=0.5] (2.9275, 2.4959, 1.7654) -- (2.9275, 2.5483, 1.7600) -- (2.9245, 2.5504, 1.8100) -- (2.9245, 2.4978, 1.8154) -- cycle;
\fill[blue!15.2, opacity=0.5] (2.9245, 2.4978, 1.8154) -- (2.9245, 2.5504, 1.8100) -- (2.9215, 2.5524, 1.8600) -- (2.9215, 2.4997, 1.8654) -- cycle;
\fill[blue!15.3, opacity=0.5] (2.9215, 2.4997, 1.8654) -- (2.9215, 2.5524, 1.8600) -- (2.9185, 2.5543, 1.9100) -- (2.9185, 2.5016, 1.9154) -- cycle;
\fill[blue!15.4, opacity=0.5] (2.9185, 2.5016, 1.9154) -- (2.9185, 2.5543, 1.9100) -- (2.9156, 2.5563, 1.9600) -- (2.9156, 2.5035, 1.9654) -- cycle;
\fill[blue!15.6, opacity=0.5] (2.9156, 2.5035, 1.9654) -- (2.9156, 2.5563, 1.9600) -- (2.9128, 2.5582, 2.0100) -- (2.9128, 2.5053, 2.0154) -- cycle;
\fill[blue!15.8, opacity=0.5] (2.9128, 2.5053, 2.0154) -- (2.9128, 2.5582, 2.0100) -- (2.9100, 2.5600, 2.0600) -- (2.9100, 2.5070, 2.0654) -- cycle;
\fill[blue!16.0, opacity=0.5] (2.9100, 2.5070, 2.0654) -- (2.9100, 2.5600, 2.0600) -- (2.9073, 2.5618, 2.1100) -- (2.9073, 2.5087, 2.1154) -- cycle;
\fill[blue!16.3, opacity=0.5] (2.9073, 2.5087, 2.1154) -- (2.9073, 2.5618, 2.1100) -- (2.9047, 2.5635, 2.1600) -- (2.9047, 2.5103, 2.1654) -- cycle;
\fill[blue!16.6, opacity=0.5] (2.9047, 2.5103, 2.1654) -- (2.9047, 2.5635, 2.1600) -- (2.9022, 2.5652, 2.2100) -- (2.9022, 2.5119, 2.2154) -- cycle;
\fill[blue!17.0, opacity=0.5] (2.9022, 2.5119, 2.2154) -- (2.9022, 2.5652, 2.2100) -- (2.8999, 2.5668, 2.2600) -- (2.8999, 2.5134, 2.2654) -- cycle;
\fill[blue!17.4, opacity=0.5] (2.8999, 2.5134, 2.2654) -- (2.8999, 2.5668, 2.2600) -- (2.8976, 2.5683, 2.3100) -- (2.8976, 2.5149, 2.3154) -- cycle;
\fill[blue!17.9, opacity=0.5] (2.8976, 2.5149, 2.3154) -- (2.8976, 2.5683, 2.3100) -- (2.8954, 2.5697, 2.3600) -- (2.8954, 2.5162, 2.3654) -- cycle;
\fill[blue!18.5, opacity=0.5] (2.8954, 2.5162, 2.3654) -- (2.8954, 2.5697, 2.3600) -- (2.8934, 2.5711, 2.4100) -- (2.8934, 2.5175, 2.4154) -- cycle;
\fill[blue!19.1, opacity=0.5] (2.8934, 2.5175, 2.4154) -- (2.8934, 2.5711, 2.4100) -- (2.8915, 2.5724, 2.4600) -- (2.8915, 2.5187, 2.4654) -- cycle;
\fill[blue!19.8, opacity=0.5] (2.8915, 2.5187, 2.4654) -- (2.8915, 2.5724, 2.4600) -- (2.8897, 2.5735, 2.5100) -- (2.8897, 2.5199, 2.5154) -- cycle;
\fill[blue!20.5, opacity=0.5] (2.8897, 2.5199, 2.5154) -- (2.8897, 2.5735, 2.5100) -- (2.8880, 2.5746, 2.5600) -- (2.8880, 2.5209, 2.5654) -- cycle;
\fill[blue!21.3, opacity=0.5] (2.8880, 2.5209, 2.5654) -- (2.8880, 2.5746, 2.5600) -- (2.8865, 2.5756, 2.6100) -- (2.8865, 2.5219, 2.6154) -- cycle;
\fill[blue!22.1, opacity=0.5] (2.8865, 2.5219, 2.6154) -- (2.8865, 2.5756, 2.6100) -- (2.8852, 2.5765, 2.6600) -- (2.8852, 2.5227, 2.6654) -- cycle;
\fill[blue!23.0, opacity=0.5] (2.8852, 2.5227, 2.6654) -- (2.8852, 2.5765, 2.6600) -- (2.8840, 2.5773, 2.7100) -- (2.8840, 2.5235, 2.7154) -- cycle;
\fill[blue!23.9, opacity=0.5] (2.8840, 2.5235, 2.7154) -- (2.8840, 2.5773, 2.7100) -- (2.8829, 2.5780, 2.7600) -- (2.8829, 2.5241, 2.7654) -- cycle;
\fill[blue!24.9, opacity=0.5] (2.8829, 2.5241, 2.7654) -- (2.8829, 2.5780, 2.7600) -- (2.8820, 2.5786, 2.8100) -- (2.8820, 2.5247, 2.8154) -- cycle;
\fill[blue!25.8, opacity=0.5] (2.8820, 2.5247, 2.8154) -- (2.8820, 2.5786, 2.8100) -- (2.8813, 2.5791, 2.8600) -- (2.8813, 2.5252, 2.8654) -- cycle;
\fill[blue!26.8, opacity=0.5] (2.8813, 2.5252, 2.8654) -- (2.8813, 2.5791, 2.8600) -- (2.8807, 2.5795, 2.9100) -- (2.8807, 2.5255, 2.9154) -- cycle;
\fill[blue!27.8, opacity=0.5] (2.8807, 2.5255, 2.9154) -- (2.8807, 2.5795, 2.9100) -- (2.8803, 2.5798, 2.9600) -- (2.8803, 2.5258, 2.9654) -- cycle;
\fill[blue!28.7, opacity=0.5] (2.8803, 2.5258, 2.9654) -- (2.8803, 2.5798, 2.9600) -- (2.8801, 2.5799, 3.0100) -- (2.8801, 2.5259, 3.0154) -- cycle;
\fill[blue!29.7, opacity=0.5] (2.8801, 2.5259, 3.0154) -- (2.8801, 2.5799, 3.0100) -- (2.8800, 2.5800, 3.0600) -- (2.8800, 2.5260, 3.0654) -- cycle;
\fill[blue!15.0, opacity=0.5] (3.0000, 2.5000, 0.0600) -- (3.0000, 2.5500, 0.0545) -- (2.9999, 2.5501, 0.1045) -- (2.9999, 2.5001, 0.1100) -- cycle;
\fill[blue!15.0, opacity=0.5] (2.9999, 2.5001, 0.1100) -- (2.9999, 2.5501, 0.1045) -- (2.9997, 2.5502, 0.1545) -- (2.9997, 2.5002, 0.1600) -- cycle;
\fill[blue!15.0, opacity=0.5] (2.9997, 2.5002, 0.1600) -- (2.9997, 2.5502, 0.1545) -- (2.9993, 2.5505, 0.2045) -- (2.9993, 2.5005, 0.2100) -- cycle;
\fill[blue!15.0, opacity=0.5] (2.9993, 2.5005, 0.2100) -- (2.9993, 2.5505, 0.2045) -- (2.9987, 2.5509, 0.2545) -- (2.9987, 2.5009, 0.2600) -- cycle;
\fill[blue!15.0, opacity=0.5] (2.9987, 2.5009, 0.2600) -- (2.9987, 2.5509, 0.2545) -- (2.9980, 2.5514, 0.3045) -- (2.9980, 2.5014, 0.3100) -- cycle;
\fill[blue!15.0, opacity=0.5] (2.9980, 2.5014, 0.3100) -- (2.9980, 2.5514, 0.3045) -- (2.9971, 2.5521, 0.3545) -- (2.9971, 2.5020, 0.3600) -- cycle;
\fill[blue!15.0, opacity=0.5] (2.9971, 2.5020, 0.3600) -- (2.9971, 2.5521, 0.3545) -- (2.9960, 2.5528, 0.4045) -- (2.9960, 2.5027, 0.4100) -- cycle;
\fill[blue!15.0, opacity=0.5] (2.9960, 2.5027, 0.4100) -- (2.9960, 2.5528, 0.4045) -- (2.9948, 2.5536, 0.4545) -- (2.9948, 2.5035, 0.4600) -- cycle;
\fill[blue!15.0, opacity=0.5] (2.9948, 2.5035, 0.4600) -- (2.9948, 2.5536, 0.4545) -- (2.9935, 2.5546, 0.5045) -- (2.9935, 2.5044, 0.5100) -- cycle;
\fill[blue!15.0, opacity=0.5] (2.9935, 2.5044, 0.5100) -- (2.9935, 2.5546, 0.5045) -- (2.9920, 2.5556, 0.5545) -- (2.9920, 2.5054, 0.5600) -- cycle;
\fill[blue!15.0, opacity=0.5] (2.9920, 2.5054, 0.5600) -- (2.9920, 2.5556, 0.5545) -- (2.9903, 2.5568, 0.6045) -- (2.9903, 2.5065, 0.6100) -- cycle;
\fill[blue!15.0, opacity=0.5] (2.9903, 2.5065, 0.6100) -- (2.9903, 2.5568, 0.6045) -- (2.9885, 2.5580, 0.6545) -- (2.9885, 2.5076, 0.6600) -- cycle;
\fill[blue!15.0, opacity=0.5] (2.9885, 2.5076, 0.6600) -- (2.9885, 2.5580, 0.6545) -- (2.9866, 2.5594, 0.7045) -- (2.9866, 2.5089, 0.7100) -- cycle;
\fill[blue!15.0, opacity=0.5] (2.9866, 2.5089, 0.7100) -- (2.9866, 2.5594, 0.7045) -- (2.9846, 2.5608, 0.7545) -- (2.9846, 2.5103, 0.7600) -- cycle;
\fill[blue!15.0, opacity=0.5] (2.9846, 2.5103, 0.7600) -- (2.9846, 2.5608, 0.7545) -- (2.9824, 2.5623, 0.8045) -- (2.9824, 2.5117, 0.8100) -- cycle;
\fill[blue!15.0, opacity=0.5] (2.9824, 2.5117, 0.8100) -- (2.9824, 2.5623, 0.8045) -- (2.9801, 2.5639, 0.8545) -- (2.9801, 2.5132, 0.8600) -- cycle;
\fill[blue!15.0, opacity=0.5] (2.9801, 2.5132, 0.8600) -- (2.9801, 2.5639, 0.8545) -- (2.9778, 2.5656, 0.9045) -- (2.9778, 2.5148, 0.9100) -- cycle;
\fill[blue!15.0, opacity=0.5] (2.9778, 2.5148, 0.9100) -- (2.9778, 2.5656, 0.9045) -- (2.9753, 2.5673, 0.9545) -- (2.9753, 2.5165, 0.9600) -- cycle;
\fill[blue!15.0, opacity=0.5] (2.9753, 2.5165, 0.9600) -- (2.9753, 2.5673, 0.9545) -- (2.9727, 2.5691, 1.0045) -- (2.9727, 2.5182, 1.0100) -- cycle;
\fill[blue!15.0, opacity=0.5] (2.9727, 2.5182, 1.0100) -- (2.9727, 2.5691, 1.0045) -- (2.9700, 2.5710, 1.0545) -- (2.9700, 2.5200, 1.0600) -- cycle;
\fill[blue!15.0, opacity=0.5] (2.9700, 2.5200, 1.0600) -- (2.9700, 2.5710, 1.0545) -- (2.9672, 2.5729, 1.1045) -- (2.9672, 2.5218, 1.1100) -- cycle;
\fill[blue!15.0, opacity=0.5] (2.9672, 2.5218, 1.1100) -- (2.9672, 2.5729, 1.1045) -- (2.9644, 2.5749, 1.1545) -- (2.9644, 2.5237, 1.1600) -- cycle;
\fill[blue!15.0, opacity=0.5] (2.9644, 2.5237, 1.1600) -- (2.9644, 2.5749, 1.1545) -- (2.9615, 2.5769, 1.2045) -- (2.9615, 2.5257, 1.2100) -- cycle;
\fill[blue!15.0, opacity=0.5] (2.9615, 2.5257, 1.2100) -- (2.9615, 2.5769, 1.2045) -- (2.9585, 2.5790, 1.2545) -- (2.9585, 2.5276, 1.2600) -- cycle;
\fill[blue!15.0, opacity=0.5] (2.9585, 2.5276, 1.2600) -- (2.9585, 2.5790, 1.2545) -- (2.9555, 2.5811, 1.3045) -- (2.9555, 2.5296, 1.3100) -- cycle;
\fill[blue!15.0, opacity=0.5] (2.9555, 2.5296, 1.3100) -- (2.9555, 2.5811, 1.3045) -- (2.9525, 2.5833, 1.3545) -- (2.9525, 2.5317, 1.3600) -- cycle;
\fill[blue!15.0, opacity=0.5] (2.9525, 2.5317, 1.3600) -- (2.9525, 2.5833, 1.3545) -- (2.9494, 2.5854, 1.4045) -- (2.9494, 2.5337, 1.4100) -- cycle;
\fill[blue!15.0, opacity=0.5] (2.9494, 2.5337, 1.4100) -- (2.9494, 2.5854, 1.4045) -- (2.9463, 2.5876, 1.4545) -- (2.9463, 2.5358, 1.4600) -- cycle;
\fill[blue!15.0, opacity=0.5] (2.9463, 2.5358, 1.4600) -- (2.9463, 2.5876, 1.4545) -- (2.9431, 2.5898, 1.5045) -- (2.9431, 2.5379, 1.5100) -- cycle;
\fill[blue!15.0, opacity=0.5] (2.9431, 2.5379, 1.5100) -- (2.9431, 2.5898, 1.5045) -- (2.9400, 2.5920, 1.5545) -- (2.9400, 2.5400, 1.5600) -- cycle;
\fill[blue!15.0, opacity=0.5] (2.9400, 2.5400, 1.5600) -- (2.9400, 2.5920, 1.5545) -- (2.9369, 2.5942, 1.6045) -- (2.9369, 2.5421, 1.6100) -- cycle;
\fill[blue!15.0, opacity=0.5] (2.9369, 2.5421, 1.6100) -- (2.9369, 2.5942, 1.6045) -- (2.9337, 2.5964, 1.6545) -- (2.9337, 2.5442, 1.6600) -- cycle;
\fill[blue!15.0, opacity=0.5] (2.9337, 2.5442, 1.6600) -- (2.9337, 2.5964, 1.6545) -- (2.9306, 2.5986, 1.7045) -- (2.9306, 2.5463, 1.7100) -- cycle;
\fill[blue!15.0, opacity=0.5] (2.9306, 2.5463, 1.7100) -- (2.9306, 2.5986, 1.7045) -- (2.9275, 2.6007, 1.7545) -- (2.9275, 2.5483, 1.7600) -- cycle;
\fill[blue!15.0, opacity=0.5] (2.9275, 2.5483, 1.7600) -- (2.9275, 2.6007, 1.7545) -- (2.9245, 2.6029, 1.8045) -- (2.9245, 2.5504, 1.8100) -- cycle;
\fill[blue!15.0, opacity=0.5] (2.9245, 2.5504, 1.8100) -- (2.9245, 2.6029, 1.8045) -- (2.9215, 2.6050, 1.8545) -- (2.9215, 2.5524, 1.8600) -- cycle;
\fill[blue!15.0, opacity=0.5] (2.9215, 2.5524, 1.8600) -- (2.9215, 2.6050, 1.8545) -- (2.9185, 2.6071, 1.9045) -- (2.9185, 2.5543, 1.9100) -- cycle;
\fill[blue!15.0, opacity=0.5] (2.9185, 2.5543, 1.9100) -- (2.9185, 2.6071, 1.9045) -- (2.9156, 2.6091, 1.9545) -- (2.9156, 2.5563, 1.9600) -- cycle;
\fill[blue!15.0, opacity=0.5] (2.9156, 2.5563, 1.9600) -- (2.9156, 2.6091, 1.9545) -- (2.9128, 2.6111, 2.0045) -- (2.9128, 2.5582, 2.0100) -- cycle;
\fill[blue!15.1, opacity=0.5] (2.9128, 2.5582, 2.0100) -- (2.9128, 2.6111, 2.0045) -- (2.9100, 2.6130, 2.0545) -- (2.9100, 2.5600, 2.0600) -- cycle;
\fill[blue!15.1, opacity=0.5] (2.9100, 2.5600, 2.0600) -- (2.9100, 2.6130, 2.0545) -- (2.9073, 2.6149, 2.1045) -- (2.9073, 2.5618, 2.1100) -- cycle;
\fill[blue!15.1, opacity=0.5] (2.9073, 2.5618, 2.1100) -- (2.9073, 2.6149, 2.1045) -- (2.9047, 2.6167, 2.1545) -- (2.9047, 2.5635, 2.1600) -- cycle;
\fill[blue!15.2, opacity=0.5] (2.9047, 2.5635, 2.1600) -- (2.9047, 2.6167, 2.1545) -- (2.9022, 2.6184, 2.2045) -- (2.9022, 2.5652, 2.2100) -- cycle;
\fill[blue!15.2, opacity=0.5] (2.9022, 2.5652, 2.2100) -- (2.9022, 2.6184, 2.2045) -- (2.8999, 2.6201, 2.2545) -- (2.8999, 2.5668, 2.2600) -- cycle;
\fill[blue!15.3, opacity=0.5] (2.8999, 2.5668, 2.2600) -- (2.8999, 2.6201, 2.2545) -- (2.8976, 2.6217, 2.3045) -- (2.8976, 2.5683, 2.3100) -- cycle;
\fill[blue!15.4, opacity=0.5] (2.8976, 2.5683, 2.3100) -- (2.8976, 2.6217, 2.3045) -- (2.8954, 2.6232, 2.3545) -- (2.8954, 2.5697, 2.3600) -- cycle;
\fill[blue!15.5, opacity=0.5] (2.8954, 2.5697, 2.3600) -- (2.8954, 2.6232, 2.3545) -- (2.8934, 2.6246, 2.4045) -- (2.8934, 2.5711, 2.4100) -- cycle;
\fill[blue!15.6, opacity=0.5] (2.8934, 2.5711, 2.4100) -- (2.8934, 2.6246, 2.4045) -- (2.8915, 2.6260, 2.4545) -- (2.8915, 2.5724, 2.4600) -- cycle;
\fill[blue!15.8, opacity=0.5] (2.8915, 2.5724, 2.4600) -- (2.8915, 2.6260, 2.4545) -- (2.8897, 2.6272, 2.5045) -- (2.8897, 2.5735, 2.5100) -- cycle;
\fill[blue!16.0, opacity=0.5] (2.8897, 2.5735, 2.5100) -- (2.8897, 2.6272, 2.5045) -- (2.8880, 2.6284, 2.5545) -- (2.8880, 2.5746, 2.5600) -- cycle;
\fill[blue!16.2, opacity=0.5] (2.8880, 2.5746, 2.5600) -- (2.8880, 2.6284, 2.5545) -- (2.8865, 2.6294, 2.6045) -- (2.8865, 2.5756, 2.6100) -- cycle;
\fill[blue!16.4, opacity=0.5] (2.8865, 2.5756, 2.6100) -- (2.8865, 2.6294, 2.6045) -- (2.8852, 2.6304, 2.6545) -- (2.8852, 2.5765, 2.6600) -- cycle;
\fill[blue!16.7, opacity=0.5] (2.8852, 2.5765, 2.6600) -- (2.8852, 2.6304, 2.6545) -- (2.8840, 2.6312, 2.7045) -- (2.8840, 2.5773, 2.7100) -- cycle;
\fill[blue!17.1, opacity=0.5] (2.8840, 2.5773, 2.7100) -- (2.8840, 2.6312, 2.7045) -- (2.8829, 2.6319, 2.7545) -- (2.8829, 2.5780, 2.7600) -- cycle;
\fill[blue!17.4, opacity=0.5] (2.8829, 2.5780, 2.7600) -- (2.8829, 2.6319, 2.7545) -- (2.8820, 2.6326, 2.8045) -- (2.8820, 2.5786, 2.8100) -- cycle;
\fill[blue!17.8, opacity=0.5] (2.8820, 2.5786, 2.8100) -- (2.8820, 2.6326, 2.8045) -- (2.8813, 2.6331, 2.8545) -- (2.8813, 2.5791, 2.8600) -- cycle;
\fill[blue!18.3, opacity=0.5] (2.8813, 2.5791, 2.8600) -- (2.8813, 2.6331, 2.8545) -- (2.8807, 2.6335, 2.9045) -- (2.8807, 2.5795, 2.9100) -- cycle;
\fill[blue!18.7, opacity=0.5] (2.8807, 2.5795, 2.9100) -- (2.8807, 2.6335, 2.9045) -- (2.8803, 2.6338, 2.9545) -- (2.8803, 2.5798, 2.9600) -- cycle;
\fill[blue!19.2, opacity=0.5] (2.8803, 2.5798, 2.9600) -- (2.8803, 2.6338, 2.9545) -- (2.8801, 2.6339, 3.0045) -- (2.8801, 2.5799, 3.0100) -- cycle;
\fill[blue!19.7, opacity=0.5] (2.8801, 2.5799, 3.0100) -- (2.8801, 2.6339, 3.0045) -- (2.8800, 2.6340, 3.0545) -- (2.8800, 2.5800, 3.0600) -- cycle;
\fill[blue!15.0, opacity=0.5] (3.0000, 2.5500, 0.0545) -- (3.0000, 2.6000, 0.0488) -- (2.9999, 2.6001, 0.0988) -- (2.9999, 2.5501, 0.1045) -- cycle;
\fill[blue!15.0, opacity=0.5] (2.9999, 2.5501, 0.1045) -- (2.9999, 2.6001, 0.0988) -- (2.9997, 2.6002, 0.1488) -- (2.9997, 2.5502, 0.1545) -- cycle;
\fill[blue!15.0, opacity=0.5] (2.9997, 2.5502, 0.1545) -- (2.9997, 2.6002, 0.1488) -- (2.9993, 2.6005, 0.1988) -- (2.9993, 2.5505, 0.2045) -- cycle;
\fill[blue!15.0, opacity=0.5] (2.9993, 2.5505, 0.2045) -- (2.9993, 2.6005, 0.1988) -- (2.9987, 2.6010, 0.2488) -- (2.9987, 2.5509, 0.2545) -- cycle;
\fill[blue!15.0, opacity=0.5] (2.9987, 2.5509, 0.2545) -- (2.9987, 2.6010, 0.2488) -- (2.9980, 2.6015, 0.2988) -- (2.9980, 2.5514, 0.3045) -- cycle;
\fill[blue!15.0, opacity=0.5] (2.9980, 2.5514, 0.3045) -- (2.9980, 2.6015, 0.2988) -- (2.9971, 2.6022, 0.3488) -- (2.9971, 2.5521, 0.3545) -- cycle;
\fill[blue!15.0, opacity=0.5] (2.9971, 2.5521, 0.3545) -- (2.9971, 2.6022, 0.3488) -- (2.9960, 2.6029, 0.3988) -- (2.9960, 2.5528, 0.4045) -- cycle;
\fill[blue!15.0, opacity=0.5] (2.9960, 2.5528, 0.4045) -- (2.9960, 2.6029, 0.3988) -- (2.9948, 2.6038, 0.4488) -- (2.9948, 2.5536, 0.4545) -- cycle;
\fill[blue!15.0, opacity=0.5] (2.9948, 2.5536, 0.4545) -- (2.9948, 2.6038, 0.4488) -- (2.9935, 2.6048, 0.4988) -- (2.9935, 2.5546, 0.5045) -- cycle;
\fill[blue!15.0, opacity=0.5] (2.9935, 2.5546, 0.5045) -- (2.9935, 2.6048, 0.4988) -- (2.9920, 2.6059, 0.5488) -- (2.9920, 2.5556, 0.5545) -- cycle;
\fill[blue!15.0, opacity=0.5] (2.9920, 2.5556, 0.5545) -- (2.9920, 2.6059, 0.5488) -- (2.9903, 2.6071, 0.5988) -- (2.9903, 2.5568, 0.6045) -- cycle;
\fill[blue!15.0, opacity=0.5] (2.9903, 2.5568, 0.6045) -- (2.9903, 2.6071, 0.5988) -- (2.9885, 2.6084, 0.6488) -- (2.9885, 2.5580, 0.6545) -- cycle;
\fill[blue!15.0, opacity=0.5] (2.9885, 2.5580, 0.6545) -- (2.9885, 2.6084, 0.6488) -- (2.9866, 2.6098, 0.6988) -- (2.9866, 2.5594, 0.7045) -- cycle;
\fill[blue!15.0, opacity=0.5] (2.9866, 2.5594, 0.7045) -- (2.9866, 2.6098, 0.6988) -- (2.9846, 2.6113, 0.7488) -- (2.9846, 2.5608, 0.7545) -- cycle;
\fill[blue!15.0, opacity=0.5] (2.9846, 2.5608, 0.7545) -- (2.9846, 2.6113, 0.7488) -- (2.9824, 2.6129, 0.7988) -- (2.9824, 2.5623, 0.8045) -- cycle;
\fill[blue!15.0, opacity=0.5] (2.9824, 2.5623, 0.8045) -- (2.9824, 2.6129, 0.7988) -- (2.9801, 2.6146, 0.8488) -- (2.9801, 2.5639, 0.8545) -- cycle;
\fill[blue!15.0, opacity=0.5] (2.9801, 2.5639, 0.8545) -- (2.9801, 2.6146, 0.8488) -- (2.9778, 2.6163, 0.8988) -- (2.9778, 2.5656, 0.9045) -- cycle;
\fill[blue!15.0, opacity=0.5] (2.9778, 2.5656, 0.9045) -- (2.9778, 2.6163, 0.8988) -- (2.9753, 2.6181, 0.9488) -- (2.9753, 2.5673, 0.9545) -- cycle;
\fill[blue!15.0, opacity=0.5] (2.9753, 2.5673, 0.9545) -- (2.9753, 2.6181, 0.9488) -- (2.9727, 2.6200, 0.9988) -- (2.9727, 2.5691, 1.0045) -- cycle;
\fill[blue!15.0, opacity=0.5] (2.9727, 2.5691, 1.0045) -- (2.9727, 2.6200, 0.9988) -- (2.9700, 2.6220, 1.0488) -- (2.9700, 2.5710, 1.0545) -- cycle;
\fill[blue!15.0, opacity=0.5] (2.9700, 2.5710, 1.0545) -- (2.9700, 2.6220, 1.0488) -- (2.9672, 2.6240, 1.0988) -- (2.9672, 2.5729, 1.1045) -- cycle;
\fill[blue!15.0, opacity=0.5] (2.9672, 2.5729, 1.1045) -- (2.9672, 2.6240, 1.0988) -- (2.9644, 2.6261, 1.1488) -- (2.9644, 2.5749, 1.1545) -- cycle;
\fill[blue!15.0, opacity=0.5] (2.9644, 2.5749, 1.1545) -- (2.9644, 2.6261, 1.1488) -- (2.9615, 2.6282, 1.1988) -- (2.9615, 2.5769, 1.2045) -- cycle;
\fill[blue!15.0, opacity=0.5] (2.9615, 2.5769, 1.2045) -- (2.9615, 2.6282, 1.1988) -- (2.9585, 2.6304, 1.2488) -- (2.9585, 2.5790, 1.2545) -- cycle;
\fill[blue!15.0, opacity=0.5] (2.9585, 2.5790, 1.2545) -- (2.9585, 2.6304, 1.2488) -- (2.9555, 2.6326, 1.2988) -- (2.9555, 2.5811, 1.3045) -- cycle;
\fill[blue!15.0, opacity=0.5] (2.9555, 2.5811, 1.3045) -- (2.9555, 2.6326, 1.2988) -- (2.9525, 2.6349, 1.3488) -- (2.9525, 2.5833, 1.3545) -- cycle;
\fill[blue!15.0, opacity=0.5] (2.9525, 2.5833, 1.3545) -- (2.9525, 2.6349, 1.3488) -- (2.9494, 2.6371, 1.3988) -- (2.9494, 2.5854, 1.4045) -- cycle;
\fill[blue!15.0, opacity=0.5] (2.9494, 2.5854, 1.4045) -- (2.9494, 2.6371, 1.3988) -- (2.9463, 2.6394, 1.4488) -- (2.9463, 2.5876, 1.4545) -- cycle;
\fill[blue!15.0, opacity=0.5] (2.9463, 2.5876, 1.4545) -- (2.9463, 2.6394, 1.4488) -- (2.9431, 2.6417, 1.4988) -- (2.9431, 2.5898, 1.5045) -- cycle;
\fill[blue!15.0, opacity=0.5] (2.9431, 2.5898, 1.5045) -- (2.9431, 2.6417, 1.4988) -- (2.9400, 2.6440, 1.5488) -- (2.9400, 2.5920, 1.5545) -- cycle;
\fill[blue!15.0, opacity=0.5] (2.9400, 2.5920, 1.5545) -- (2.9400, 2.6440, 1.5488) -- (2.9369, 2.6463, 1.5988) -- (2.9369, 2.5942, 1.6045) -- cycle;
\fill[blue!15.0, opacity=0.5] (2.9369, 2.5942, 1.6045) -- (2.9369, 2.6463, 1.5988) -- (2.9337, 2.6486, 1.6488) -- (2.9337, 2.5964, 1.6545) -- cycle;
\fill[blue!15.0, opacity=0.5] (2.9337, 2.5964, 1.6545) -- (2.9337, 2.6486, 1.6488) -- (2.9306, 2.6509, 1.6988) -- (2.9306, 2.5986, 1.7045) -- cycle;
\fill[blue!15.0, opacity=0.5] (2.9306, 2.5986, 1.7045) -- (2.9306, 2.6509, 1.6988) -- (2.9275, 2.6531, 1.7488) -- (2.9275, 2.6007, 1.7545) -- cycle;
\fill[blue!15.0, opacity=0.5] (2.9275, 2.6007, 1.7545) -- (2.9275, 2.6531, 1.7488) -- (2.9245, 2.6554, 1.7988) -- (2.9245, 2.6029, 1.8045) -- cycle;
\fill[blue!15.0, opacity=0.5] (2.9245, 2.6029, 1.8045) -- (2.9245, 2.6554, 1.7988) -- (2.9215, 2.6576, 1.8488) -- (2.9215, 2.6050, 1.8545) -- cycle;
\fill[blue!15.0, opacity=0.5] (2.9215, 2.6050, 1.8545) -- (2.9215, 2.6576, 1.8488) -- (2.9185, 2.6598, 1.8988) -- (2.9185, 2.6071, 1.9045) -- cycle;
\fill[blue!15.0, opacity=0.5] (2.9185, 2.6071, 1.9045) -- (2.9185, 2.6598, 1.8988) -- (2.9156, 2.6619, 1.9488) -- (2.9156, 2.6091, 1.9545) -- cycle;
\fill[blue!15.0, opacity=0.5] (2.9156, 2.6091, 1.9545) -- (2.9156, 2.6619, 1.9488) -- (2.9128, 2.6640, 1.9988) -- (2.9128, 2.6111, 2.0045) -- cycle;
\fill[blue!15.0, opacity=0.5] (2.9128, 2.6111, 2.0045) -- (2.9128, 2.6640, 1.9988) -- (2.9100, 2.6660, 2.0488) -- (2.9100, 2.6130, 2.0545) -- cycle;
\fill[blue!15.0, opacity=0.5] (2.9100, 2.6130, 2.0545) -- (2.9100, 2.6660, 2.0488) -- (2.9073, 2.6680, 2.0988) -- (2.9073, 2.6149, 2.1045) -- cycle;
\fill[blue!15.0, opacity=0.5] (2.9073, 2.6149, 2.1045) -- (2.9073, 2.6680, 2.0988) -- (2.9047, 2.6699, 2.1488) -- (2.9047, 2.6167, 2.1545) -- cycle;
\fill[blue!15.0, opacity=0.5] (2.9047, 2.6167, 2.1545) -- (2.9047, 2.6699, 2.1488) -- (2.9022, 2.6717, 2.1988) -- (2.9022, 2.6184, 2.2045) -- cycle;
\fill[blue!15.0, opacity=0.5] (2.9022, 2.6184, 2.2045) -- (2.9022, 2.6717, 2.1988) -- (2.8999, 2.6734, 2.2488) -- (2.8999, 2.6201, 2.2545) -- cycle;
\fill[blue!15.0, opacity=0.5] (2.8999, 2.6201, 2.2545) -- (2.8999, 2.6734, 2.2488) -- (2.8976, 2.6751, 2.2988) -- (2.8976, 2.6217, 2.3045) -- cycle;
\fill[blue!15.0, opacity=0.5] (2.8976, 2.6217, 2.3045) -- (2.8976, 2.6751, 2.2988) -- (2.8954, 2.6767, 2.3488) -- (2.8954, 2.6232, 2.3545) -- cycle;
\fill[blue!15.0, opacity=0.5] (2.8954, 2.6232, 2.3545) -- (2.8954, 2.6767, 2.3488) -- (2.8934, 2.6782, 2.3988) -- (2.8934, 2.6246, 2.4045) -- cycle;
\fill[blue!15.1, opacity=0.5] (2.8934, 2.6246, 2.4045) -- (2.8934, 2.6782, 2.3988) -- (2.8915, 2.6796, 2.4488) -- (2.8915, 2.6260, 2.4545) -- cycle;
\fill[blue!15.1, opacity=0.5] (2.8915, 2.6260, 2.4545) -- (2.8915, 2.6796, 2.4488) -- (2.8897, 2.6809, 2.4988) -- (2.8897, 2.6272, 2.5045) -- cycle;
\fill[blue!15.1, opacity=0.5] (2.8897, 2.6272, 2.5045) -- (2.8897, 2.6809, 2.4988) -- (2.8880, 2.6821, 2.5488) -- (2.8880, 2.6284, 2.5545) -- cycle;
\fill[blue!15.2, opacity=0.5] (2.8880, 2.6284, 2.5545) -- (2.8880, 2.6821, 2.5488) -- (2.8865, 2.6832, 2.5988) -- (2.8865, 2.6294, 2.6045) -- cycle;
\fill[blue!15.2, opacity=0.5] (2.8865, 2.6294, 2.6045) -- (2.8865, 2.6832, 2.5988) -- (2.8852, 2.6842, 2.6488) -- (2.8852, 2.6304, 2.6545) -- cycle;
\fill[blue!15.3, opacity=0.5] (2.8852, 2.6304, 2.6545) -- (2.8852, 2.6842, 2.6488) -- (2.8840, 2.6851, 2.6988) -- (2.8840, 2.6312, 2.7045) -- cycle;
\fill[blue!15.3, opacity=0.5] (2.8840, 2.6312, 2.7045) -- (2.8840, 2.6851, 2.6988) -- (2.8829, 2.6858, 2.7488) -- (2.8829, 2.6319, 2.7545) -- cycle;
\fill[blue!15.4, opacity=0.5] (2.8829, 2.6319, 2.7545) -- (2.8829, 2.6858, 2.7488) -- (2.8820, 2.6865, 2.7988) -- (2.8820, 2.6326, 2.8045) -- cycle;
\fill[blue!15.5, opacity=0.5] (2.8820, 2.6326, 2.8045) -- (2.8820, 2.6865, 2.7988) -- (2.8813, 2.6870, 2.8488) -- (2.8813, 2.6331, 2.8545) -- cycle;
\fill[blue!15.7, opacity=0.5] (2.8813, 2.6331, 2.8545) -- (2.8813, 2.6870, 2.8488) -- (2.8807, 2.6875, 2.8988) -- (2.8807, 2.6335, 2.9045) -- cycle;
\fill[blue!15.8, opacity=0.5] (2.8807, 2.6335, 2.9045) -- (2.8807, 2.6875, 2.8988) -- (2.8803, 2.6878, 2.9488) -- (2.8803, 2.6338, 2.9545) -- cycle;
\fill[blue!16.0, opacity=0.5] (2.8803, 2.6338, 2.9545) -- (2.8803, 2.6878, 2.9488) -- (2.8801, 2.6879, 2.9988) -- (2.8801, 2.6339, 3.0045) -- cycle;
\fill[blue!16.1, opacity=0.5] (2.8801, 2.6339, 3.0045) -- (2.8801, 2.6879, 2.9988) -- (2.8800, 2.6880, 3.0488) -- (2.8800, 2.6340, 3.0545) -- cycle;
\fill[blue!15.0, opacity=0.5] (3.0000, 2.6000, 0.0488) -- (3.0000, 2.6500, 0.0430) -- (2.9999, 2.6501, 0.0930) -- (2.9999, 2.6001, 0.0988) -- cycle;
\fill[blue!15.0, opacity=0.5] (2.9999, 2.6001, 0.0988) -- (2.9999, 2.6501, 0.0930) -- (2.9997, 2.6503, 0.1430) -- (2.9997, 2.6002, 0.1488) -- cycle;
\fill[blue!15.0, opacity=0.5] (2.9997, 2.6002, 0.1488) -- (2.9997, 2.6503, 0.1430) -- (2.9993, 2.6506, 0.1930) -- (2.9993, 2.6005, 0.1988) -- cycle;
\fill[blue!15.0, opacity=0.5] (2.9993, 2.6005, 0.1988) -- (2.9993, 2.6506, 0.1930) -- (2.9987, 2.6510, 0.2430) -- (2.9987, 2.6010, 0.2488) -- cycle;
\fill[blue!15.0, opacity=0.5] (2.9987, 2.6010, 0.2488) -- (2.9987, 2.6510, 0.2430) -- (2.9980, 2.6516, 0.2930) -- (2.9980, 2.6015, 0.2988) -- cycle;
\fill[blue!15.0, opacity=0.5] (2.9980, 2.6015, 0.2988) -- (2.9980, 2.6516, 0.2930) -- (2.9971, 2.6523, 0.3430) -- (2.9971, 2.6022, 0.3488) -- cycle;
\fill[blue!15.0, opacity=0.5] (2.9971, 2.6022, 0.3488) -- (2.9971, 2.6523, 0.3430) -- (2.9960, 2.6531, 0.3930) -- (2.9960, 2.6029, 0.3988) -- cycle;
\fill[blue!15.0, opacity=0.5] (2.9960, 2.6029, 0.3988) -- (2.9960, 2.6531, 0.3930) -- (2.9948, 2.6540, 0.4430) -- (2.9948, 2.6038, 0.4488) -- cycle;
\fill[blue!15.0, opacity=0.5] (2.9948, 2.6038, 0.4488) -- (2.9948, 2.6540, 0.4430) -- (2.9935, 2.6550, 0.4930) -- (2.9935, 2.6048, 0.4988) -- cycle;
\fill[blue!15.0, opacity=0.5] (2.9935, 2.6048, 0.4988) -- (2.9935, 2.6550, 0.4930) -- (2.9920, 2.6562, 0.5430) -- (2.9920, 2.6059, 0.5488) -- cycle;
\fill[blue!15.0, opacity=0.5] (2.9920, 2.6059, 0.5488) -- (2.9920, 2.6562, 0.5430) -- (2.9903, 2.6574, 0.5930) -- (2.9903, 2.6071, 0.5988) -- cycle;
\fill[blue!15.0, opacity=0.5] (2.9903, 2.6071, 0.5988) -- (2.9903, 2.6574, 0.5930) -- (2.9885, 2.6588, 0.6430) -- (2.9885, 2.6084, 0.6488) -- cycle;
\fill[blue!15.0, opacity=0.5] (2.9885, 2.6084, 0.6488) -- (2.9885, 2.6588, 0.6430) -- (2.9866, 2.6603, 0.6930) -- (2.9866, 2.6098, 0.6988) -- cycle;
\fill[blue!15.0, opacity=0.5] (2.9866, 2.6098, 0.6988) -- (2.9866, 2.6603, 0.6930) -- (2.9846, 2.6618, 0.7430) -- (2.9846, 2.6113, 0.7488) -- cycle;
\fill[blue!15.0, opacity=0.5] (2.9846, 2.6113, 0.7488) -- (2.9846, 2.6618, 0.7430) -- (2.9824, 2.6635, 0.7930) -- (2.9824, 2.6129, 0.7988) -- cycle;
\fill[blue!15.0, opacity=0.5] (2.9824, 2.6129, 0.7988) -- (2.9824, 2.6635, 0.7930) -- (2.9801, 2.6652, 0.8430) -- (2.9801, 2.6146, 0.8488) -- cycle;
\fill[blue!15.0, opacity=0.5] (2.9801, 2.6146, 0.8488) -- (2.9801, 2.6652, 0.8430) -- (2.9778, 2.6671, 0.8930) -- (2.9778, 2.6163, 0.8988) -- cycle;
\fill[blue!15.0, opacity=0.5] (2.9778, 2.6163, 0.8988) -- (2.9778, 2.6671, 0.8930) -- (2.9753, 2.6690, 0.9430) -- (2.9753, 2.6181, 0.9488) -- cycle;
\fill[blue!15.0, opacity=0.5] (2.9753, 2.6181, 0.9488) -- (2.9753, 2.6690, 0.9430) -- (2.9727, 2.6709, 0.9930) -- (2.9727, 2.6200, 0.9988) -- cycle;
\fill[blue!15.0, opacity=0.5] (2.9727, 2.6200, 0.9988) -- (2.9727, 2.6709, 0.9930) -- (2.9700, 2.6730, 1.0430) -- (2.9700, 2.6220, 1.0488) -- cycle;
\fill[blue!15.0, opacity=0.5] (2.9700, 2.6220, 1.0488) -- (2.9700, 2.6730, 1.0430) -- (2.9672, 2.6751, 1.0930) -- (2.9672, 2.6240, 1.0988) -- cycle;
\fill[blue!15.0, opacity=0.5] (2.9672, 2.6240, 1.0988) -- (2.9672, 2.6751, 1.0930) -- (2.9644, 2.6773, 1.1430) -- (2.9644, 2.6261, 1.1488) -- cycle;
\fill[blue!15.0, opacity=0.5] (2.9644, 2.6261, 1.1488) -- (2.9644, 2.6773, 1.1430) -- (2.9615, 2.6795, 1.1930) -- (2.9615, 2.6282, 1.1988) -- cycle;
\fill[blue!15.0, opacity=0.5] (2.9615, 2.6282, 1.1988) -- (2.9615, 2.6795, 1.1930) -- (2.9585, 2.6818, 1.2430) -- (2.9585, 2.6304, 1.2488) -- cycle;
\fill[blue!15.0, opacity=0.5] (2.9585, 2.6304, 1.2488) -- (2.9585, 2.6818, 1.2430) -- (2.9555, 2.6841, 1.2930) -- (2.9555, 2.6326, 1.2988) -- cycle;
\fill[blue!15.0, opacity=0.5] (2.9555, 2.6326, 1.2988) -- (2.9555, 2.6841, 1.2930) -- (2.9525, 2.6864, 1.3430) -- (2.9525, 2.6349, 1.3488) -- cycle;
\fill[blue!15.0, opacity=0.5] (2.9525, 2.6349, 1.3488) -- (2.9525, 2.6864, 1.3430) -- (2.9494, 2.6888, 1.3930) -- (2.9494, 2.6371, 1.3988) -- cycle;
\fill[blue!15.0, opacity=0.5] (2.9494, 2.6371, 1.3988) -- (2.9494, 2.6888, 1.3930) -- (2.9463, 2.6912, 1.4430) -- (2.9463, 2.6394, 1.4488) -- cycle;
\fill[blue!15.0, opacity=0.5] (2.9463, 2.6394, 1.4488) -- (2.9463, 2.6912, 1.4430) -- (2.9431, 2.6936, 1.4930) -- (2.9431, 2.6417, 1.4988) -- cycle;
\fill[blue!15.0, opacity=0.5] (2.9431, 2.6417, 1.4988) -- (2.9431, 2.6936, 1.4930) -- (2.9400, 2.6960, 1.5430) -- (2.9400, 2.6440, 1.5488) -- cycle;
\fill[blue!15.0, opacity=0.5] (2.9400, 2.6440, 1.5488) -- (2.9400, 2.6960, 1.5430) -- (2.9369, 2.6984, 1.5930) -- (2.9369, 2.6463, 1.5988) -- cycle;
\fill[blue!15.0, opacity=0.5] (2.9369, 2.6463, 1.5988) -- (2.9369, 2.6984, 1.5930) -- (2.9337, 2.7008, 1.6430) -- (2.9337, 2.6486, 1.6488) -- cycle;
\fill[blue!15.0, opacity=0.5] (2.9337, 2.6486, 1.6488) -- (2.9337, 2.7008, 1.6430) -- (2.9306, 2.7032, 1.6930) -- (2.9306, 2.6509, 1.6988) -- cycle;
\fill[blue!15.0, opacity=0.5] (2.9306, 2.6509, 1.6988) -- (2.9306, 2.7032, 1.6930) -- (2.9275, 2.7056, 1.7430) -- (2.9275, 2.6531, 1.7488) -- cycle;
\fill[blue!15.0, opacity=0.5] (2.9275, 2.6531, 1.7488) -- (2.9275, 2.7056, 1.7430) -- (2.9245, 2.7079, 1.7930) -- (2.9245, 2.6554, 1.7988) -- cycle;
\fill[blue!15.0, opacity=0.5] (2.9245, 2.6554, 1.7988) -- (2.9245, 2.7079, 1.7930) -- (2.9215, 2.7102, 1.8430) -- (2.9215, 2.6576, 1.8488) -- cycle;
\fill[blue!15.0, opacity=0.5] (2.9215, 2.6576, 1.8488) -- (2.9215, 2.7102, 1.8430) -- (2.9185, 2.7125, 1.8930) -- (2.9185, 2.6598, 1.8988) -- cycle;
\fill[blue!15.0, opacity=0.5] (2.9185, 2.6598, 1.8988) -- (2.9185, 2.7125, 1.8930) -- (2.9156, 2.7147, 1.9430) -- (2.9156, 2.6619, 1.9488) -- cycle;
\fill[blue!15.0, opacity=0.5] (2.9156, 2.6619, 1.9488) -- (2.9156, 2.7147, 1.9430) -- (2.9128, 2.7169, 1.9930) -- (2.9128, 2.6640, 1.9988) -- cycle;
\fill[blue!15.0, opacity=0.5] (2.9128, 2.6640, 1.9988) -- (2.9128, 2.7169, 1.9930) -- (2.9100, 2.7190, 2.0430) -- (2.9100, 2.6660, 2.0488) -- cycle;
\fill[blue!15.0, opacity=0.5] (2.9100, 2.6660, 2.0488) -- (2.9100, 2.7190, 2.0430) -- (2.9073, 2.7211, 2.0930) -- (2.9073, 2.6680, 2.0988) -- cycle;
\fill[blue!15.0, opacity=0.5] (2.9073, 2.6680, 2.0988) -- (2.9073, 2.7211, 2.0930) -- (2.9047, 2.7230, 2.1430) -- (2.9047, 2.6699, 2.1488) -- cycle;
\fill[blue!15.0, opacity=0.5] (2.9047, 2.6699, 2.1488) -- (2.9047, 2.7230, 2.1430) -- (2.9022, 2.7249, 2.1930) -- (2.9022, 2.6717, 2.1988) -- cycle;
\fill[blue!15.0, opacity=0.5] (2.9022, 2.6717, 2.1988) -- (2.9022, 2.7249, 2.1930) -- (2.8999, 2.7268, 2.2430) -- (2.8999, 2.6734, 2.2488) -- cycle;
\fill[blue!15.0, opacity=0.5] (2.8999, 2.6734, 2.2488) -- (2.8999, 2.7268, 2.2430) -- (2.8976, 2.7285, 2.2930) -- (2.8976, 2.6751, 2.2988) -- cycle;
\fill[blue!15.0, opacity=0.5] (2.8976, 2.6751, 2.2988) -- (2.8976, 2.7285, 2.2930) -- (2.8954, 2.7302, 2.3430) -- (2.8954, 2.6767, 2.3488) -- cycle;
\fill[blue!15.0, opacity=0.5] (2.8954, 2.6767, 2.3488) -- (2.8954, 2.7302, 2.3430) -- (2.8934, 2.7317, 2.3930) -- (2.8934, 2.6782, 2.3988) -- cycle;
\fill[blue!15.0, opacity=0.5] (2.8934, 2.6782, 2.3988) -- (2.8934, 2.7317, 2.3930) -- (2.8915, 2.7332, 2.4430) -- (2.8915, 2.6796, 2.4488) -- cycle;
\fill[blue!15.0, opacity=0.5] (2.8915, 2.6796, 2.4488) -- (2.8915, 2.7332, 2.4430) -- (2.8897, 2.7346, 2.4930) -- (2.8897, 2.6809, 2.4988) -- cycle;
\fill[blue!15.0, opacity=0.5] (2.8897, 2.6809, 2.4988) -- (2.8897, 2.7346, 2.4930) -- (2.8880, 2.7358, 2.5430) -- (2.8880, 2.6821, 2.5488) -- cycle;
\fill[blue!15.0, opacity=0.5] (2.8880, 2.6821, 2.5488) -- (2.8880, 2.7358, 2.5430) -- (2.8865, 2.7370, 2.5930) -- (2.8865, 2.6832, 2.5988) -- cycle;
\fill[blue!15.0, opacity=0.5] (2.8865, 2.6832, 2.5988) -- (2.8865, 2.7370, 2.5930) -- (2.8852, 2.7380, 2.6430) -- (2.8852, 2.6842, 2.6488) -- cycle;
\fill[blue!15.1, opacity=0.5] (2.8852, 2.6842, 2.6488) -- (2.8852, 2.7380, 2.6430) -- (2.8840, 2.7389, 2.6930) -- (2.8840, 2.6851, 2.6988) -- cycle;
\fill[blue!15.1, opacity=0.5] (2.8840, 2.6851, 2.6988) -- (2.8840, 2.7389, 2.6930) -- (2.8829, 2.7397, 2.7430) -- (2.8829, 2.6858, 2.7488) -- cycle;
\fill[blue!15.1, opacity=0.5] (2.8829, 2.6858, 2.7488) -- (2.8829, 2.7397, 2.7430) -- (2.8820, 2.7404, 2.7930) -- (2.8820, 2.6865, 2.7988) -- cycle;
\fill[blue!15.1, opacity=0.5] (2.8820, 2.6865, 2.7988) -- (2.8820, 2.7404, 2.7930) -- (2.8813, 2.7410, 2.8430) -- (2.8813, 2.6870, 2.8488) -- cycle;
\fill[blue!15.2, opacity=0.5] (2.8813, 2.6870, 2.8488) -- (2.8813, 2.7410, 2.8430) -- (2.8807, 2.7414, 2.8930) -- (2.8807, 2.6875, 2.8988) -- cycle;
\fill[blue!15.2, opacity=0.5] (2.8807, 2.6875, 2.8988) -- (2.8807, 2.7414, 2.8930) -- (2.8803, 2.7417, 2.9430) -- (2.8803, 2.6878, 2.9488) -- cycle;
\fill[blue!15.3, opacity=0.5] (2.8803, 2.6878, 2.9488) -- (2.8803, 2.7417, 2.9430) -- (2.8801, 2.7419, 2.9930) -- (2.8801, 2.6879, 2.9988) -- cycle;
\fill[blue!15.3, opacity=0.5] (2.8801, 2.6879, 2.9988) -- (2.8801, 2.7419, 2.9930) -- (2.8800, 2.7420, 3.0430) -- (2.8800, 2.6880, 3.0488) -- cycle;
\fill[blue!15.0, opacity=0.5] (3.0000, 2.6500, 0.0430) -- (3.0000, 2.7000, 0.0371) -- (2.9999, 2.7001, 0.0871) -- (2.9999, 2.6501, 0.0930) -- cycle;
\fill[blue!15.0, opacity=0.5] (2.9999, 2.6501, 0.0930) -- (2.9999, 2.7001, 0.0871) -- (2.9997, 2.7003, 0.1371) -- (2.9997, 2.6503, 0.1430) -- cycle;
\fill[blue!15.0, opacity=0.5] (2.9997, 2.6503, 0.1430) -- (2.9997, 2.7003, 0.1371) -- (2.9993, 2.7006, 0.1871) -- (2.9993, 2.6506, 0.1930) -- cycle;
\fill[blue!15.0, opacity=0.5] (2.9993, 2.6506, 0.1930) -- (2.9993, 2.7006, 0.1871) -- (2.9987, 2.7010, 0.2371) -- (2.9987, 2.6510, 0.2430) -- cycle;
\fill[blue!15.0, opacity=0.5] (2.9987, 2.6510, 0.2430) -- (2.9987, 2.7010, 0.2371) -- (2.9980, 2.7016, 0.2871) -- (2.9980, 2.6516, 0.2930) -- cycle;
\fill[blue!15.0, opacity=0.5] (2.9980, 2.6516, 0.2930) -- (2.9980, 2.7016, 0.2871) -- (2.9971, 2.7023, 0.3371) -- (2.9971, 2.6523, 0.3430) -- cycle;
\fill[blue!15.0, opacity=0.5] (2.9971, 2.6523, 0.3430) -- (2.9971, 2.7023, 0.3371) -- (2.9960, 2.7032, 0.3871) -- (2.9960, 2.6531, 0.3930) -- cycle;
\fill[blue!15.0, opacity=0.5] (2.9960, 2.6531, 0.3930) -- (2.9960, 2.7032, 0.3871) -- (2.9948, 2.7041, 0.4371) -- (2.9948, 2.6540, 0.4430) -- cycle;
\fill[blue!15.0, opacity=0.5] (2.9948, 2.6540, 0.4430) -- (2.9948, 2.7041, 0.4371) -- (2.9935, 2.7052, 0.4871) -- (2.9935, 2.6550, 0.4930) -- cycle;
\fill[blue!15.0, opacity=0.5] (2.9935, 2.6550, 0.4930) -- (2.9935, 2.7052, 0.4871) -- (2.9920, 2.7064, 0.5371) -- (2.9920, 2.6562, 0.5430) -- cycle;
\fill[blue!15.0, opacity=0.5] (2.9920, 2.6562, 0.5430) -- (2.9920, 2.7064, 0.5371) -- (2.9903, 2.7077, 0.5871) -- (2.9903, 2.6574, 0.5930) -- cycle;
\fill[blue!15.0, opacity=0.5] (2.9903, 2.6574, 0.5930) -- (2.9903, 2.7077, 0.5871) -- (2.9885, 2.7092, 0.6371) -- (2.9885, 2.6588, 0.6430) -- cycle;
\fill[blue!15.0, opacity=0.5] (2.9885, 2.6588, 0.6430) -- (2.9885, 2.7092, 0.6371) -- (2.9866, 2.7107, 0.6871) -- (2.9866, 2.6603, 0.6930) -- cycle;
\fill[blue!15.0, opacity=0.5] (2.9866, 2.6603, 0.6930) -- (2.9866, 2.7107, 0.6871) -- (2.9846, 2.7123, 0.7371) -- (2.9846, 2.6618, 0.7430) -- cycle;
\fill[blue!15.0, opacity=0.5] (2.9846, 2.6618, 0.7430) -- (2.9846, 2.7123, 0.7371) -- (2.9824, 2.7141, 0.7871) -- (2.9824, 2.6635, 0.7930) -- cycle;
\fill[blue!15.0, opacity=0.5] (2.9824, 2.6635, 0.7930) -- (2.9824, 2.7141, 0.7871) -- (2.9801, 2.7159, 0.8371) -- (2.9801, 2.6652, 0.8430) -- cycle;
\fill[blue!15.0, opacity=0.5] (2.9801, 2.6652, 0.8430) -- (2.9801, 2.7159, 0.8371) -- (2.9778, 2.7178, 0.8871) -- (2.9778, 2.6671, 0.8930) -- cycle;
\fill[blue!15.0, opacity=0.5] (2.9778, 2.6671, 0.8930) -- (2.9778, 2.7178, 0.8871) -- (2.9753, 2.7198, 0.9371) -- (2.9753, 2.6690, 0.9430) -- cycle;
\fill[blue!15.0, opacity=0.5] (2.9753, 2.6690, 0.9430) -- (2.9753, 2.7198, 0.9371) -- (2.9727, 2.7219, 0.9871) -- (2.9727, 2.6709, 0.9930) -- cycle;
\fill[blue!15.0, opacity=0.5] (2.9727, 2.6709, 0.9930) -- (2.9727, 2.7219, 0.9871) -- (2.9700, 2.7240, 1.0371) -- (2.9700, 2.6730, 1.0430) -- cycle;
\fill[blue!15.0, opacity=0.5] (2.9700, 2.6730, 1.0430) -- (2.9700, 2.7240, 1.0371) -- (2.9672, 2.7262, 1.0871) -- (2.9672, 2.6751, 1.0930) -- cycle;
\fill[blue!15.0, opacity=0.5] (2.9672, 2.6751, 1.0930) -- (2.9672, 2.7262, 1.0871) -- (2.9644, 2.7285, 1.1371) -- (2.9644, 2.6773, 1.1430) -- cycle;
\fill[blue!15.0, opacity=0.5] (2.9644, 2.6773, 1.1430) -- (2.9644, 2.7285, 1.1371) -- (2.9615, 2.7308, 1.1871) -- (2.9615, 2.6795, 1.1930) -- cycle;
\fill[blue!15.0, opacity=0.5] (2.9615, 2.6795, 1.1930) -- (2.9615, 2.7308, 1.1871) -- (2.9585, 2.7332, 1.2371) -- (2.9585, 2.6818, 1.2430) -- cycle;
\fill[blue!15.0, opacity=0.5] (2.9585, 2.6818, 1.2430) -- (2.9585, 2.7332, 1.2371) -- (2.9555, 2.7356, 1.2871) -- (2.9555, 2.6841, 1.2930) -- cycle;
\fill[blue!15.0, opacity=0.5] (2.9555, 2.6841, 1.2930) -- (2.9555, 2.7356, 1.2871) -- (2.9525, 2.7380, 1.3371) -- (2.9525, 2.6864, 1.3430) -- cycle;
\fill[blue!15.0, opacity=0.5] (2.9525, 2.6864, 1.3430) -- (2.9525, 2.7380, 1.3371) -- (2.9494, 2.7405, 1.3871) -- (2.9494, 2.6888, 1.3930) -- cycle;
\fill[blue!15.0, opacity=0.5] (2.9494, 2.6888, 1.3930) -- (2.9494, 2.7405, 1.3871) -- (2.9463, 2.7430, 1.4371) -- (2.9463, 2.6912, 1.4430) -- cycle;
\fill[blue!15.0, opacity=0.5] (2.9463, 2.6912, 1.4430) -- (2.9463, 2.7430, 1.4371) -- (2.9431, 2.7455, 1.4871) -- (2.9431, 2.6936, 1.4930) -- cycle;
\fill[blue!15.0, opacity=0.5] (2.9431, 2.6936, 1.4930) -- (2.9431, 2.7455, 1.4871) -- (2.9400, 2.7480, 1.5371) -- (2.9400, 2.6960, 1.5430) -- cycle;
\fill[blue!15.0, opacity=0.5] (2.9400, 2.6960, 1.5430) -- (2.9400, 2.7480, 1.5371) -- (2.9369, 2.7505, 1.5871) -- (2.9369, 2.6984, 1.5930) -- cycle;
\fill[blue!15.0, opacity=0.5] (2.9369, 2.6984, 1.5930) -- (2.9369, 2.7505, 1.5871) -- (2.9337, 2.7530, 1.6371) -- (2.9337, 2.7008, 1.6430) -- cycle;
\fill[blue!15.0, opacity=0.5] (2.9337, 2.7008, 1.6430) -- (2.9337, 2.7530, 1.6371) -- (2.9306, 2.7555, 1.6871) -- (2.9306, 2.7032, 1.6930) -- cycle;
\fill[blue!15.0, opacity=0.5] (2.9306, 2.7032, 1.6930) -- (2.9306, 2.7555, 1.6871) -- (2.9275, 2.7580, 1.7371) -- (2.9275, 2.7056, 1.7430) -- cycle;
\fill[blue!15.0, opacity=0.5] (2.9275, 2.7056, 1.7430) -- (2.9275, 2.7580, 1.7371) -- (2.9245, 2.7604, 1.7871) -- (2.9245, 2.7079, 1.7930) -- cycle;
\fill[blue!15.0, opacity=0.5] (2.9245, 2.7079, 1.7930) -- (2.9245, 2.7604, 1.7871) -- (2.9215, 2.7628, 1.8371) -- (2.9215, 2.7102, 1.8430) -- cycle;
\fill[blue!15.0, opacity=0.5] (2.9215, 2.7102, 1.8430) -- (2.9215, 2.7628, 1.8371) -- (2.9185, 2.7652, 1.8871) -- (2.9185, 2.7125, 1.8930) -- cycle;
\fill[blue!15.0, opacity=0.5] (2.9185, 2.7125, 1.8930) -- (2.9185, 2.7652, 1.8871) -- (2.9156, 2.7675, 1.9371) -- (2.9156, 2.7147, 1.9430) -- cycle;
\fill[blue!15.0, opacity=0.5] (2.9156, 2.7147, 1.9430) -- (2.9156, 2.7675, 1.9371) -- (2.9128, 2.7698, 1.9871) -- (2.9128, 2.7169, 1.9930) -- cycle;
\fill[blue!15.0, opacity=0.5] (2.9128, 2.7169, 1.9930) -- (2.9128, 2.7698, 1.9871) -- (2.9100, 2.7720, 2.0371) -- (2.9100, 2.7190, 2.0430) -- cycle;
\fill[blue!15.0, opacity=0.5] (2.9100, 2.7190, 2.0430) -- (2.9100, 2.7720, 2.0371) -- (2.9073, 2.7741, 2.0871) -- (2.9073, 2.7211, 2.0930) -- cycle;
\fill[blue!15.0, opacity=0.5] (2.9073, 2.7211, 2.0930) -- (2.9073, 2.7741, 2.0871) -- (2.9047, 2.7762, 2.1371) -- (2.9047, 2.7230, 2.1430) -- cycle;
\fill[blue!15.0, opacity=0.5] (2.9047, 2.7230, 2.1430) -- (2.9047, 2.7762, 2.1371) -- (2.9022, 2.7782, 2.1871) -- (2.9022, 2.7249, 2.1930) -- cycle;
\fill[blue!15.0, opacity=0.5] (2.9022, 2.7249, 2.1930) -- (2.9022, 2.7782, 2.1871) -- (2.8999, 2.7801, 2.2371) -- (2.8999, 2.7268, 2.2430) -- cycle;
\fill[blue!15.0, opacity=0.5] (2.8999, 2.7268, 2.2430) -- (2.8999, 2.7801, 2.2371) -- (2.8976, 2.7819, 2.2871) -- (2.8976, 2.7285, 2.2930) -- cycle;
\fill[blue!15.0, opacity=0.5] (2.8976, 2.7285, 2.2930) -- (2.8976, 2.7819, 2.2871) -- (2.8954, 2.7837, 2.3371) -- (2.8954, 2.7302, 2.3430) -- cycle;
\fill[blue!15.0, opacity=0.5] (2.8954, 2.7302, 2.3430) -- (2.8954, 2.7837, 2.3371) -- (2.8934, 2.7853, 2.3871) -- (2.8934, 2.7317, 2.3930) -- cycle;
\fill[blue!15.0, opacity=0.5] (2.8934, 2.7317, 2.3930) -- (2.8934, 2.7853, 2.3871) -- (2.8915, 2.7868, 2.4371) -- (2.8915, 2.7332, 2.4430) -- cycle;
\fill[blue!15.0, opacity=0.5] (2.8915, 2.7332, 2.4430) -- (2.8915, 2.7868, 2.4371) -- (2.8897, 2.7883, 2.4871) -- (2.8897, 2.7346, 2.4930) -- cycle;
\fill[blue!15.0, opacity=0.5] (2.8897, 2.7346, 2.4930) -- (2.8897, 2.7883, 2.4871) -- (2.8880, 2.7896, 2.5371) -- (2.8880, 2.7358, 2.5430) -- cycle;
\fill[blue!15.0, opacity=0.5] (2.8880, 2.7358, 2.5430) -- (2.8880, 2.7896, 2.5371) -- (2.8865, 2.7908, 2.5871) -- (2.8865, 2.7370, 2.5930) -- cycle;
\fill[blue!15.0, opacity=0.5] (2.8865, 2.7370, 2.5930) -- (2.8865, 2.7908, 2.5871) -- (2.8852, 2.7919, 2.6371) -- (2.8852, 2.7380, 2.6430) -- cycle;
\fill[blue!15.0, opacity=0.5] (2.8852, 2.7380, 2.6430) -- (2.8852, 2.7919, 2.6371) -- (2.8840, 2.7928, 2.6871) -- (2.8840, 2.7389, 2.6930) -- cycle;
\fill[blue!15.0, opacity=0.5] (2.8840, 2.7389, 2.6930) -- (2.8840, 2.7928, 2.6871) -- (2.8829, 2.7937, 2.7371) -- (2.8829, 2.7397, 2.7430) -- cycle;
\fill[blue!15.0, opacity=0.5] (2.8829, 2.7397, 2.7430) -- (2.8829, 2.7937, 2.7371) -- (2.8820, 2.7944, 2.7871) -- (2.8820, 2.7404, 2.7930) -- cycle;
\fill[blue!15.1, opacity=0.5] (2.8820, 2.7404, 2.7930) -- (2.8820, 2.7944, 2.7871) -- (2.8813, 2.7950, 2.8371) -- (2.8813, 2.7410, 2.8430) -- cycle;
\fill[blue!15.1, opacity=0.5] (2.8813, 2.7410, 2.8430) -- (2.8813, 2.7950, 2.8371) -- (2.8807, 2.7954, 2.8871) -- (2.8807, 2.7414, 2.8930) -- cycle;
\fill[blue!15.1, opacity=0.5] (2.8807, 2.7414, 2.8930) -- (2.8807, 2.7954, 2.8871) -- (2.8803, 2.7957, 2.9371) -- (2.8803, 2.7417, 2.9430) -- cycle;
\fill[blue!15.1, opacity=0.5] (2.8803, 2.7417, 2.9430) -- (2.8803, 2.7957, 2.9371) -- (2.8801, 2.7959, 2.9871) -- (2.8801, 2.7419, 2.9930) -- cycle;
\fill[blue!15.2, opacity=0.5] (2.8801, 2.7419, 2.9930) -- (2.8801, 2.7959, 2.9871) -- (2.8800, 2.7960, 3.0371) -- (2.8800, 2.7420, 3.0430) -- cycle;
\fill[blue!15.0, opacity=0.5] (3.0000, 2.7000, 0.0371) -- (3.0000, 2.7500, 0.0311) -- (2.9999, 2.7501, 0.0811) -- (2.9999, 2.7001, 0.0871) -- cycle;
\fill[blue!15.0, opacity=0.5] (2.9999, 2.7001, 0.0871) -- (2.9999, 2.7501, 0.0811) -- (2.9997, 2.7503, 0.1311) -- (2.9997, 2.7003, 0.1371) -- cycle;
\fill[blue!15.0, opacity=0.5] (2.9997, 2.7003, 0.1371) -- (2.9997, 2.7503, 0.1311) -- (2.9993, 2.7506, 0.1811) -- (2.9993, 2.7006, 0.1871) -- cycle;
\fill[blue!15.0, opacity=0.5] (2.9993, 2.7006, 0.1871) -- (2.9993, 2.7506, 0.1811) -- (2.9987, 2.7511, 0.2311) -- (2.9987, 2.7010, 0.2371) -- cycle;
\fill[blue!15.0, opacity=0.5] (2.9987, 2.7010, 0.2371) -- (2.9987, 2.7511, 0.2311) -- (2.9980, 2.7517, 0.2811) -- (2.9980, 2.7016, 0.2871) -- cycle;
\fill[blue!15.0, opacity=0.5] (2.9980, 2.7016, 0.2871) -- (2.9980, 2.7517, 0.2811) -- (2.9971, 2.7524, 0.3311) -- (2.9971, 2.7023, 0.3371) -- cycle;
\fill[blue!15.0, opacity=0.5] (2.9971, 2.7023, 0.3371) -- (2.9971, 2.7524, 0.3311) -- (2.9960, 2.7533, 0.3811) -- (2.9960, 2.7032, 0.3871) -- cycle;
\fill[blue!15.0, opacity=0.5] (2.9960, 2.7032, 0.3871) -- (2.9960, 2.7533, 0.3811) -- (2.9948, 2.7543, 0.4311) -- (2.9948, 2.7041, 0.4371) -- cycle;
\fill[blue!15.0, opacity=0.5] (2.9948, 2.7041, 0.4371) -- (2.9948, 2.7543, 0.4311) -- (2.9935, 2.7554, 0.4811) -- (2.9935, 2.7052, 0.4871) -- cycle;
\fill[blue!15.0, opacity=0.5] (2.9935, 2.7052, 0.4871) -- (2.9935, 2.7554, 0.4811) -- (2.9920, 2.7567, 0.5311) -- (2.9920, 2.7064, 0.5371) -- cycle;
\fill[blue!15.0, opacity=0.5] (2.9920, 2.7064, 0.5371) -- (2.9920, 2.7567, 0.5311) -- (2.9903, 2.7581, 0.5811) -- (2.9903, 2.7077, 0.5871) -- cycle;
\fill[blue!15.0, opacity=0.5] (2.9903, 2.7077, 0.5871) -- (2.9903, 2.7581, 0.5811) -- (2.9885, 2.7595, 0.6311) -- (2.9885, 2.7092, 0.6371) -- cycle;
\fill[blue!15.0, opacity=0.5] (2.9885, 2.7092, 0.6371) -- (2.9885, 2.7595, 0.6311) -- (2.9866, 2.7611, 0.6811) -- (2.9866, 2.7107, 0.6871) -- cycle;
\fill[blue!15.0, opacity=0.5] (2.9866, 2.7107, 0.6871) -- (2.9866, 2.7611, 0.6811) -- (2.9846, 2.7628, 0.7311) -- (2.9846, 2.7123, 0.7371) -- cycle;
\fill[blue!15.0, opacity=0.5] (2.9846, 2.7123, 0.7371) -- (2.9846, 2.7628, 0.7311) -- (2.9824, 2.7646, 0.7811) -- (2.9824, 2.7141, 0.7871) -- cycle;
\fill[blue!15.0, opacity=0.5] (2.9824, 2.7141, 0.7871) -- (2.9824, 2.7646, 0.7811) -- (2.9801, 2.7665, 0.8311) -- (2.9801, 2.7159, 0.8371) -- cycle;
\fill[blue!15.0, opacity=0.5] (2.9801, 2.7159, 0.8371) -- (2.9801, 2.7665, 0.8311) -- (2.9778, 2.7685, 0.8811) -- (2.9778, 2.7178, 0.8871) -- cycle;
\fill[blue!15.0, opacity=0.5] (2.9778, 2.7178, 0.8871) -- (2.9778, 2.7685, 0.8811) -- (2.9753, 2.7706, 0.9311) -- (2.9753, 2.7198, 0.9371) -- cycle;
\fill[blue!15.0, opacity=0.5] (2.9753, 2.7198, 0.9371) -- (2.9753, 2.7706, 0.9311) -- (2.9727, 2.7728, 0.9811) -- (2.9727, 2.7219, 0.9871) -- cycle;
\fill[blue!15.0, opacity=0.5] (2.9727, 2.7219, 0.9871) -- (2.9727, 2.7728, 0.9811) -- (2.9700, 2.7750, 1.0311) -- (2.9700, 2.7240, 1.0371) -- cycle;
\fill[blue!15.0, opacity=0.5] (2.9700, 2.7240, 1.0371) -- (2.9700, 2.7750, 1.0311) -- (2.9672, 2.7773, 1.0811) -- (2.9672, 2.7262, 1.0871) -- cycle;
\fill[blue!15.0, opacity=0.5] (2.9672, 2.7262, 1.0871) -- (2.9672, 2.7773, 1.0811) -- (2.9644, 2.7797, 1.1311) -- (2.9644, 2.7285, 1.1371) -- cycle;
\fill[blue!15.0, opacity=0.5] (2.9644, 2.7285, 1.1371) -- (2.9644, 2.7797, 1.1311) -- (2.9615, 2.7821, 1.1811) -- (2.9615, 2.7308, 1.1871) -- cycle;
\fill[blue!15.0, opacity=0.5] (2.9615, 2.7308, 1.1871) -- (2.9615, 2.7821, 1.1811) -- (2.9585, 2.7845, 1.2311) -- (2.9585, 2.7332, 1.2371) -- cycle;
\fill[blue!15.0, opacity=0.5] (2.9585, 2.7332, 1.2371) -- (2.9585, 2.7845, 1.2311) -- (2.9555, 2.7871, 1.2811) -- (2.9555, 2.7356, 1.2871) -- cycle;
\fill[blue!15.0, opacity=0.5] (2.9555, 2.7356, 1.2871) -- (2.9555, 2.7871, 1.2811) -- (2.9525, 2.7896, 1.3311) -- (2.9525, 2.7380, 1.3371) -- cycle;
\fill[blue!15.0, opacity=0.5] (2.9525, 2.7380, 1.3371) -- (2.9525, 2.7896, 1.3311) -- (2.9494, 2.7922, 1.3811) -- (2.9494, 2.7405, 1.3871) -- cycle;
\fill[blue!15.0, opacity=0.5] (2.9494, 2.7405, 1.3871) -- (2.9494, 2.7922, 1.3811) -- (2.9463, 2.7948, 1.4311) -- (2.9463, 2.7430, 1.4371) -- cycle;
\fill[blue!15.0, opacity=0.5] (2.9463, 2.7430, 1.4371) -- (2.9463, 2.7948, 1.4311) -- (2.9431, 2.7974, 1.4811) -- (2.9431, 2.7455, 1.4871) -- cycle;
\fill[blue!15.0, opacity=0.5] (2.9431, 2.7455, 1.4871) -- (2.9431, 2.7974, 1.4811) -- (2.9400, 2.8000, 1.5311) -- (2.9400, 2.7480, 1.5371) -- cycle;
\fill[blue!15.0, opacity=0.5] (2.9400, 2.7480, 1.5371) -- (2.9400, 2.8000, 1.5311) -- (2.9369, 2.8026, 1.5811) -- (2.9369, 2.7505, 1.5871) -- cycle;
\fill[blue!15.0, opacity=0.5] (2.9369, 2.7505, 1.5871) -- (2.9369, 2.8026, 1.5811) -- (2.9337, 2.8052, 1.6311) -- (2.9337, 2.7530, 1.6371) -- cycle;
\fill[blue!15.0, opacity=0.5] (2.9337, 2.7530, 1.6371) -- (2.9337, 2.8052, 1.6311) -- (2.9306, 2.8078, 1.6811) -- (2.9306, 2.7555, 1.6871) -- cycle;
\fill[blue!15.0, opacity=0.5] (2.9306, 2.7555, 1.6871) -- (2.9306, 2.8078, 1.6811) -- (2.9275, 2.8104, 1.7311) -- (2.9275, 2.7580, 1.7371) -- cycle;
\fill[blue!15.0, opacity=0.5] (2.9275, 2.7580, 1.7371) -- (2.9275, 2.8104, 1.7311) -- (2.9245, 2.8129, 1.7811) -- (2.9245, 2.7604, 1.7871) -- cycle;
\fill[blue!15.0, opacity=0.5] (2.9245, 2.7604, 1.7871) -- (2.9245, 2.8129, 1.7811) -- (2.9215, 2.8155, 1.8311) -- (2.9215, 2.7628, 1.8371) -- cycle;
\fill[blue!15.0, opacity=0.5] (2.9215, 2.7628, 1.8371) -- (2.9215, 2.8155, 1.8311) -- (2.9185, 2.8179, 1.8811) -- (2.9185, 2.7652, 1.8871) -- cycle;
\fill[blue!15.0, opacity=0.5] (2.9185, 2.7652, 1.8871) -- (2.9185, 2.8179, 1.8811) -- (2.9156, 2.8203, 1.9311) -- (2.9156, 2.7675, 1.9371) -- cycle;
\fill[blue!15.0, opacity=0.5] (2.9156, 2.7675, 1.9371) -- (2.9156, 2.8203, 1.9311) -- (2.9128, 2.8227, 1.9811) -- (2.9128, 2.7698, 1.9871) -- cycle;
\fill[blue!15.0, opacity=0.5] (2.9128, 2.7698, 1.9871) -- (2.9128, 2.8227, 1.9811) -- (2.9100, 2.8250, 2.0311) -- (2.9100, 2.7720, 2.0371) -- cycle;
\fill[blue!15.0, opacity=0.5] (2.9100, 2.7720, 2.0371) -- (2.9100, 2.8250, 2.0311) -- (2.9073, 2.8272, 2.0811) -- (2.9073, 2.7741, 2.0871) -- cycle;
\fill[blue!15.0, opacity=0.5] (2.9073, 2.7741, 2.0871) -- (2.9073, 2.8272, 2.0811) -- (2.9047, 2.8294, 2.1311) -- (2.9047, 2.7762, 2.1371) -- cycle;
\fill[blue!15.0, opacity=0.5] (2.9047, 2.7762, 2.1371) -- (2.9047, 2.8294, 2.1311) -- (2.9022, 2.8315, 2.1811) -- (2.9022, 2.7782, 2.1871) -- cycle;
\fill[blue!15.0, opacity=0.5] (2.9022, 2.7782, 2.1871) -- (2.9022, 2.8315, 2.1811) -- (2.8999, 2.8335, 2.2311) -- (2.8999, 2.7801, 2.2371) -- cycle;
\fill[blue!15.0, opacity=0.5] (2.8999, 2.7801, 2.2371) -- (2.8999, 2.8335, 2.2311) -- (2.8976, 2.8354, 2.2811) -- (2.8976, 2.7819, 2.2871) -- cycle;
\fill[blue!15.0, opacity=0.5] (2.8976, 2.7819, 2.2871) -- (2.8976, 2.8354, 2.2811) -- (2.8954, 2.8372, 2.3311) -- (2.8954, 2.7837, 2.3371) -- cycle;
\fill[blue!15.0, opacity=0.5] (2.8954, 2.7837, 2.3371) -- (2.8954, 2.8372, 2.3311) -- (2.8934, 2.8389, 2.3811) -- (2.8934, 2.7853, 2.3871) -- cycle;
\fill[blue!15.0, opacity=0.5] (2.8934, 2.7853, 2.3871) -- (2.8934, 2.8389, 2.3811) -- (2.8915, 2.8405, 2.4311) -- (2.8915, 2.7868, 2.4371) -- cycle;
\fill[blue!15.0, opacity=0.5] (2.8915, 2.7868, 2.4371) -- (2.8915, 2.8405, 2.4311) -- (2.8897, 2.8419, 2.4811) -- (2.8897, 2.7883, 2.4871) -- cycle;
\fill[blue!15.0, opacity=0.5] (2.8897, 2.7883, 2.4871) -- (2.8897, 2.8419, 2.4811) -- (2.8880, 2.8433, 2.5311) -- (2.8880, 2.7896, 2.5371) -- cycle;
\fill[blue!15.0, opacity=0.5] (2.8880, 2.7896, 2.5371) -- (2.8880, 2.8433, 2.5311) -- (2.8865, 2.8446, 2.5811) -- (2.8865, 2.7908, 2.5871) -- cycle;
\fill[blue!15.0, opacity=0.5] (2.8865, 2.7908, 2.5871) -- (2.8865, 2.8446, 2.5811) -- (2.8852, 2.8457, 2.6311) -- (2.8852, 2.7919, 2.6371) -- cycle;
\fill[blue!15.0, opacity=0.5] (2.8852, 2.7919, 2.6371) -- (2.8852, 2.8457, 2.6311) -- (2.8840, 2.8467, 2.6811) -- (2.8840, 2.7928, 2.6871) -- cycle;
\fill[blue!15.1, opacity=0.5] (2.8840, 2.7928, 2.6871) -- (2.8840, 2.8467, 2.6811) -- (2.8829, 2.8476, 2.7311) -- (2.8829, 2.7937, 2.7371) -- cycle;
\fill[blue!15.1, opacity=0.5] (2.8829, 2.7937, 2.7371) -- (2.8829, 2.8476, 2.7311) -- (2.8820, 2.8483, 2.7811) -- (2.8820, 2.7944, 2.7871) -- cycle;
\fill[blue!15.1, opacity=0.5] (2.8820, 2.7944, 2.7871) -- (2.8820, 2.8483, 2.7811) -- (2.8813, 2.8489, 2.8311) -- (2.8813, 2.7950, 2.8371) -- cycle;
\fill[blue!15.1, opacity=0.5] (2.8813, 2.7950, 2.8371) -- (2.8813, 2.8489, 2.8311) -- (2.8807, 2.8494, 2.8811) -- (2.8807, 2.7954, 2.8871) -- cycle;
\fill[blue!15.2, opacity=0.5] (2.8807, 2.7954, 2.8871) -- (2.8807, 2.8494, 2.8811) -- (2.8803, 2.8497, 2.9311) -- (2.8803, 2.7957, 2.9371) -- cycle;
\fill[blue!15.2, opacity=0.5] (2.8803, 2.7957, 2.9371) -- (2.8803, 2.8497, 2.9311) -- (2.8801, 2.8499, 2.9811) -- (2.8801, 2.7959, 2.9871) -- cycle;
\fill[blue!15.3, opacity=0.5] (2.8801, 2.7959, 2.9871) -- (2.8801, 2.8499, 2.9811) -- (2.8800, 2.8500, 3.0311) -- (2.8800, 2.7960, 3.0371) -- cycle;
\fill[blue!15.0, opacity=0.5] (3.0000, 2.7500, 0.0311) -- (3.0000, 2.8000, 0.0249) -- (2.9999, 2.8001, 0.0749) -- (2.9999, 2.7501, 0.0811) -- cycle;
\fill[blue!15.0, opacity=0.5] (2.9999, 2.7501, 0.0811) -- (2.9999, 2.8001, 0.0749) -- (2.9997, 2.8003, 0.1249) -- (2.9997, 2.7503, 0.1311) -- cycle;
\fill[blue!15.0, opacity=0.5] (2.9997, 2.7503, 0.1311) -- (2.9997, 2.8003, 0.1249) -- (2.9993, 2.8006, 0.1749) -- (2.9993, 2.7506, 0.1811) -- cycle;
\fill[blue!15.0, opacity=0.5] (2.9993, 2.7506, 0.1811) -- (2.9993, 2.8006, 0.1749) -- (2.9987, 2.8011, 0.2249) -- (2.9987, 2.7511, 0.2311) -- cycle;
\fill[blue!15.0, opacity=0.5] (2.9987, 2.7511, 0.2311) -- (2.9987, 2.8011, 0.2249) -- (2.9980, 2.8018, 0.2749) -- (2.9980, 2.7517, 0.2811) -- cycle;
\fill[blue!15.0, opacity=0.5] (2.9980, 2.7517, 0.2811) -- (2.9980, 2.8018, 0.2749) -- (2.9971, 2.8025, 0.3249) -- (2.9971, 2.7524, 0.3311) -- cycle;
\fill[blue!15.0, opacity=0.5] (2.9971, 2.7524, 0.3311) -- (2.9971, 2.8025, 0.3249) -- (2.9960, 2.8035, 0.3749) -- (2.9960, 2.7533, 0.3811) -- cycle;
\fill[blue!15.0, opacity=0.5] (2.9960, 2.7533, 0.3811) -- (2.9960, 2.8035, 0.3749) -- (2.9948, 2.8045, 0.4249) -- (2.9948, 2.7543, 0.4311) -- cycle;
\fill[blue!15.0, opacity=0.5] (2.9948, 2.7543, 0.4311) -- (2.9948, 2.8045, 0.4249) -- (2.9935, 2.8057, 0.4749) -- (2.9935, 2.7554, 0.4811) -- cycle;
\fill[blue!15.0, opacity=0.5] (2.9935, 2.7554, 0.4811) -- (2.9935, 2.8057, 0.4749) -- (2.9920, 2.8070, 0.5249) -- (2.9920, 2.7567, 0.5311) -- cycle;
\fill[blue!15.0, opacity=0.5] (2.9920, 2.7567, 0.5311) -- (2.9920, 2.8070, 0.5249) -- (2.9903, 2.8084, 0.5749) -- (2.9903, 2.7581, 0.5811) -- cycle;
\fill[blue!15.0, opacity=0.5] (2.9903, 2.7581, 0.5811) -- (2.9903, 2.8084, 0.5749) -- (2.9885, 2.8099, 0.6249) -- (2.9885, 2.7595, 0.6311) -- cycle;
\fill[blue!15.0, opacity=0.5] (2.9885, 2.7595, 0.6311) -- (2.9885, 2.8099, 0.6249) -- (2.9866, 2.8116, 0.6749) -- (2.9866, 2.7611, 0.6811) -- cycle;
\fill[blue!15.0, opacity=0.5] (2.9866, 2.7611, 0.6811) -- (2.9866, 2.8116, 0.6749) -- (2.9846, 2.8134, 0.7249) -- (2.9846, 2.7628, 0.7311) -- cycle;
\fill[blue!15.0, opacity=0.5] (2.9846, 2.7628, 0.7311) -- (2.9846, 2.8134, 0.7249) -- (2.9824, 2.8152, 0.7749) -- (2.9824, 2.7646, 0.7811) -- cycle;
\fill[blue!15.0, opacity=0.5] (2.9824, 2.7646, 0.7811) -- (2.9824, 2.8152, 0.7749) -- (2.9801, 2.8172, 0.8249) -- (2.9801, 2.7665, 0.8311) -- cycle;
\fill[blue!15.0, opacity=0.5] (2.9801, 2.7665, 0.8311) -- (2.9801, 2.8172, 0.8249) -- (2.9778, 2.8193, 0.8749) -- (2.9778, 2.7685, 0.8811) -- cycle;
\fill[blue!15.0, opacity=0.5] (2.9778, 2.7685, 0.8811) -- (2.9778, 2.8193, 0.8749) -- (2.9753, 2.8214, 0.9249) -- (2.9753, 2.7706, 0.9311) -- cycle;
\fill[blue!15.0, opacity=0.5] (2.9753, 2.7706, 0.9311) -- (2.9753, 2.8214, 0.9249) -- (2.9727, 2.8237, 0.9749) -- (2.9727, 2.7728, 0.9811) -- cycle;
\fill[blue!15.0, opacity=0.5] (2.9727, 2.7728, 0.9811) -- (2.9727, 2.8237, 0.9749) -- (2.9700, 2.8260, 1.0249) -- (2.9700, 2.7750, 1.0311) -- cycle;
\fill[blue!15.0, opacity=0.5] (2.9700, 2.7750, 1.0311) -- (2.9700, 2.8260, 1.0249) -- (2.9672, 2.8284, 1.0749) -- (2.9672, 2.7773, 1.0811) -- cycle;
\fill[blue!15.0, opacity=0.5] (2.9672, 2.7773, 1.0811) -- (2.9672, 2.8284, 1.0749) -- (2.9644, 2.8308, 1.1249) -- (2.9644, 2.7797, 1.1311) -- cycle;
\fill[blue!15.0, opacity=0.5] (2.9644, 2.7797, 1.1311) -- (2.9644, 2.8308, 1.1249) -- (2.9615, 2.8334, 1.1749) -- (2.9615, 2.7821, 1.1811) -- cycle;
\fill[blue!15.0, opacity=0.5] (2.9615, 2.7821, 1.1811) -- (2.9615, 2.8334, 1.1749) -- (2.9585, 2.8359, 1.2249) -- (2.9585, 2.7845, 1.2311) -- cycle;
\fill[blue!15.0, opacity=0.5] (2.9585, 2.7845, 1.2311) -- (2.9585, 2.8359, 1.2249) -- (2.9555, 2.8385, 1.2749) -- (2.9555, 2.7871, 1.2811) -- cycle;
\fill[blue!15.0, opacity=0.5] (2.9555, 2.7871, 1.2811) -- (2.9555, 2.8385, 1.2749) -- (2.9525, 2.8412, 1.3249) -- (2.9525, 2.7896, 1.3311) -- cycle;
\fill[blue!15.0, opacity=0.5] (2.9525, 2.7896, 1.3311) -- (2.9525, 2.8412, 1.3249) -- (2.9494, 2.8439, 1.3749) -- (2.9494, 2.7922, 1.3811) -- cycle;
\fill[blue!15.0, opacity=0.5] (2.9494, 2.7922, 1.3811) -- (2.9494, 2.8439, 1.3749) -- (2.9463, 2.8466, 1.4249) -- (2.9463, 2.7948, 1.4311) -- cycle;
\fill[blue!15.0, opacity=0.5] (2.9463, 2.7948, 1.4311) -- (2.9463, 2.8466, 1.4249) -- (2.9431, 2.8493, 1.4749) -- (2.9431, 2.7974, 1.4811) -- cycle;
\fill[blue!15.0, opacity=0.5] (2.9431, 2.7974, 1.4811) -- (2.9431, 2.8493, 1.4749) -- (2.9400, 2.8520, 1.5249) -- (2.9400, 2.8000, 1.5311) -- cycle;
\fill[blue!15.0, opacity=0.5] (2.9400, 2.8000, 1.5311) -- (2.9400, 2.8520, 1.5249) -- (2.9369, 2.8547, 1.5749) -- (2.9369, 2.8026, 1.5811) -- cycle;
\fill[blue!15.0, opacity=0.5] (2.9369, 2.8026, 1.5811) -- (2.9369, 2.8547, 1.5749) -- (2.9337, 2.8574, 1.6249) -- (2.9337, 2.8052, 1.6311) -- cycle;
\fill[blue!15.0, opacity=0.5] (2.9337, 2.8052, 1.6311) -- (2.9337, 2.8574, 1.6249) -- (2.9306, 2.8601, 1.6749) -- (2.9306, 2.8078, 1.6811) -- cycle;
\fill[blue!15.0, opacity=0.5] (2.9306, 2.8078, 1.6811) -- (2.9306, 2.8601, 1.6749) -- (2.9275, 2.8628, 1.7249) -- (2.9275, 2.8104, 1.7311) -- cycle;
\fill[blue!15.0, opacity=0.5] (2.9275, 2.8104, 1.7311) -- (2.9275, 2.8628, 1.7249) -- (2.9245, 2.8655, 1.7749) -- (2.9245, 2.8129, 1.7811) -- cycle;
\fill[blue!15.0, opacity=0.5] (2.9245, 2.8129, 1.7811) -- (2.9245, 2.8655, 1.7749) -- (2.9215, 2.8681, 1.8249) -- (2.9215, 2.8155, 1.8311) -- cycle;
\fill[blue!15.0, opacity=0.5] (2.9215, 2.8155, 1.8311) -- (2.9215, 2.8681, 1.8249) -- (2.9185, 2.8706, 1.8749) -- (2.9185, 2.8179, 1.8811) -- cycle;
\fill[blue!15.0, opacity=0.5] (2.9185, 2.8179, 1.8811) -- (2.9185, 2.8706, 1.8749) -- (2.9156, 2.8732, 1.9249) -- (2.9156, 2.8203, 1.9311) -- cycle;
\fill[blue!15.0, opacity=0.5] (2.9156, 2.8203, 1.9311) -- (2.9156, 2.8732, 1.9249) -- (2.9128, 2.8756, 1.9749) -- (2.9128, 2.8227, 1.9811) -- cycle;
\fill[blue!15.0, opacity=0.5] (2.9128, 2.8227, 1.9811) -- (2.9128, 2.8756, 1.9749) -- (2.9100, 2.8780, 2.0249) -- (2.9100, 2.8250, 2.0311) -- cycle;
\fill[blue!15.0, opacity=0.5] (2.9100, 2.8250, 2.0311) -- (2.9100, 2.8780, 2.0249) -- (2.9073, 2.8803, 2.0749) -- (2.9073, 2.8272, 2.0811) -- cycle;
\fill[blue!15.0, opacity=0.5] (2.9073, 2.8272, 2.0811) -- (2.9073, 2.8803, 2.0749) -- (2.9047, 2.8826, 2.1249) -- (2.9047, 2.8294, 2.1311) -- cycle;
\fill[blue!15.0, opacity=0.5] (2.9047, 2.8294, 2.1311) -- (2.9047, 2.8826, 2.1249) -- (2.9022, 2.8847, 2.1749) -- (2.9022, 2.8315, 2.1811) -- cycle;
\fill[blue!15.0, opacity=0.5] (2.9022, 2.8315, 2.1811) -- (2.9022, 2.8847, 2.1749) -- (2.8999, 2.8868, 2.2249) -- (2.8999, 2.8335, 2.2311) -- cycle;
\fill[blue!15.0, opacity=0.5] (2.8999, 2.8335, 2.2311) -- (2.8999, 2.8868, 2.2249) -- (2.8976, 2.8888, 2.2749) -- (2.8976, 2.8354, 2.2811) -- cycle;
\fill[blue!15.0, opacity=0.5] (2.8976, 2.8354, 2.2811) -- (2.8976, 2.8888, 2.2749) -- (2.8954, 2.8906, 2.3249) -- (2.8954, 2.8372, 2.3311) -- cycle;
\fill[blue!15.0, opacity=0.5] (2.8954, 2.8372, 2.3311) -- (2.8954, 2.8906, 2.3249) -- (2.8934, 2.8924, 2.3749) -- (2.8934, 2.8389, 2.3811) -- cycle;
\fill[blue!15.0, opacity=0.5] (2.8934, 2.8389, 2.3811) -- (2.8934, 2.8924, 2.3749) -- (2.8915, 2.8941, 2.4249) -- (2.8915, 2.8405, 2.4311) -- cycle;
\fill[blue!15.1, opacity=0.5] (2.8915, 2.8405, 2.4311) -- (2.8915, 2.8941, 2.4249) -- (2.8897, 2.8956, 2.4749) -- (2.8897, 2.8419, 2.4811) -- cycle;
\fill[blue!15.1, opacity=0.5] (2.8897, 2.8419, 2.4811) -- (2.8897, 2.8956, 2.4749) -- (2.8880, 2.8970, 2.5249) -- (2.8880, 2.8433, 2.5311) -- cycle;
\fill[blue!15.1, opacity=0.5] (2.8880, 2.8433, 2.5311) -- (2.8880, 2.8970, 2.5249) -- (2.8865, 2.8983, 2.5749) -- (2.8865, 2.8446, 2.5811) -- cycle;
\fill[blue!15.1, opacity=0.5] (2.8865, 2.8446, 2.5811) -- (2.8865, 2.8983, 2.5749) -- (2.8852, 2.8995, 2.6249) -- (2.8852, 2.8457, 2.6311) -- cycle;
\fill[blue!15.2, opacity=0.5] (2.8852, 2.8457, 2.6311) -- (2.8852, 2.8995, 2.6249) -- (2.8840, 2.9005, 2.6749) -- (2.8840, 2.8467, 2.6811) -- cycle;
\fill[blue!15.2, opacity=0.5] (2.8840, 2.8467, 2.6811) -- (2.8840, 2.9005, 2.6749) -- (2.8829, 2.9015, 2.7249) -- (2.8829, 2.8476, 2.7311) -- cycle;
\fill[blue!15.3, opacity=0.5] (2.8829, 2.8476, 2.7311) -- (2.8829, 2.9015, 2.7249) -- (2.8820, 2.9022, 2.7749) -- (2.8820, 2.8483, 2.7811) -- cycle;
\fill[blue!15.4, opacity=0.5] (2.8820, 2.8483, 2.7811) -- (2.8820, 2.9022, 2.7749) -- (2.8813, 2.9029, 2.8249) -- (2.8813, 2.8489, 2.8311) -- cycle;
\fill[blue!15.5, opacity=0.5] (2.8813, 2.8489, 2.8311) -- (2.8813, 2.9029, 2.8249) -- (2.8807, 2.9034, 2.8749) -- (2.8807, 2.8494, 2.8811) -- cycle;
\fill[blue!15.6, opacity=0.5] (2.8807, 2.8494, 2.8811) -- (2.8807, 2.9034, 2.8749) -- (2.8803, 2.9037, 2.9249) -- (2.8803, 2.8497, 2.9311) -- cycle;
\fill[blue!15.7, opacity=0.5] (2.8803, 2.8497, 2.9311) -- (2.8803, 2.9037, 2.9249) -- (2.8801, 2.9039, 2.9749) -- (2.8801, 2.8499, 2.9811) -- cycle;
\fill[blue!15.8, opacity=0.5] (2.8801, 2.8499, 2.9811) -- (2.8801, 2.9039, 2.9749) -- (2.8800, 2.9040, 3.0249) -- (2.8800, 2.8500, 3.0311) -- cycle;
\fill[blue!15.0, opacity=0.5] (3.0000, 2.8000, 0.0249) -- (3.0000, 2.8500, 0.0188) -- (2.9999, 2.8501, 0.0688) -- (2.9999, 2.8001, 0.0749) -- cycle;
\fill[blue!15.0, opacity=0.5] (2.9999, 2.8001, 0.0749) -- (2.9999, 2.8501, 0.0688) -- (2.9997, 2.8503, 0.1188) -- (2.9997, 2.8003, 0.1249) -- cycle;
\fill[blue!15.0, opacity=0.5] (2.9997, 2.8003, 0.1249) -- (2.9997, 2.8503, 0.1188) -- (2.9993, 2.8507, 0.1688) -- (2.9993, 2.8006, 0.1749) -- cycle;
\fill[blue!15.0, opacity=0.5] (2.9993, 2.8006, 0.1749) -- (2.9993, 2.8507, 0.1688) -- (2.9987, 2.8512, 0.2188) -- (2.9987, 2.8011, 0.2249) -- cycle;
\fill[blue!15.0, opacity=0.5] (2.9987, 2.8011, 0.2249) -- (2.9987, 2.8512, 0.2188) -- (2.9980, 2.8518, 0.2688) -- (2.9980, 2.8018, 0.2749) -- cycle;
\fill[blue!15.0, opacity=0.5] (2.9980, 2.8018, 0.2749) -- (2.9980, 2.8518, 0.2688) -- (2.9971, 2.8526, 0.3188) -- (2.9971, 2.8025, 0.3249) -- cycle;
\fill[blue!15.0, opacity=0.5] (2.9971, 2.8025, 0.3249) -- (2.9971, 2.8526, 0.3188) -- (2.9960, 2.8536, 0.3688) -- (2.9960, 2.8035, 0.3749) -- cycle;
\fill[blue!15.0, opacity=0.5] (2.9960, 2.8035, 0.3749) -- (2.9960, 2.8536, 0.3688) -- (2.9948, 2.8547, 0.4188) -- (2.9948, 2.8045, 0.4249) -- cycle;
\fill[blue!15.0, opacity=0.5] (2.9948, 2.8045, 0.4249) -- (2.9948, 2.8547, 0.4188) -- (2.9935, 2.8559, 0.4688) -- (2.9935, 2.8057, 0.4749) -- cycle;
\fill[blue!15.0, opacity=0.5] (2.9935, 2.8057, 0.4749) -- (2.9935, 2.8559, 0.4688) -- (2.9920, 2.8572, 0.5188) -- (2.9920, 2.8070, 0.5249) -- cycle;
\fill[blue!15.0, opacity=0.5] (2.9920, 2.8070, 0.5249) -- (2.9920, 2.8572, 0.5188) -- (2.9903, 2.8587, 0.5688) -- (2.9903, 2.8084, 0.5749) -- cycle;
\fill[blue!15.0, opacity=0.5] (2.9903, 2.8084, 0.5749) -- (2.9903, 2.8587, 0.5688) -- (2.9885, 2.8603, 0.6188) -- (2.9885, 2.8099, 0.6249) -- cycle;
\fill[blue!15.0, opacity=0.5] (2.9885, 2.8099, 0.6249) -- (2.9885, 2.8603, 0.6188) -- (2.9866, 2.8620, 0.6688) -- (2.9866, 2.8116, 0.6749) -- cycle;
\fill[blue!15.0, opacity=0.5] (2.9866, 2.8116, 0.6749) -- (2.9866, 2.8620, 0.6688) -- (2.9846, 2.8639, 0.7188) -- (2.9846, 2.8134, 0.7249) -- cycle;
\fill[blue!15.0, opacity=0.5] (2.9846, 2.8134, 0.7249) -- (2.9846, 2.8639, 0.7188) -- (2.9824, 2.8658, 0.7688) -- (2.9824, 2.8152, 0.7749) -- cycle;
\fill[blue!15.0, opacity=0.5] (2.9824, 2.8152, 0.7749) -- (2.9824, 2.8658, 0.7688) -- (2.9801, 2.8679, 0.8188) -- (2.9801, 2.8172, 0.8249) -- cycle;
\fill[blue!15.0, opacity=0.5] (2.9801, 2.8172, 0.8249) -- (2.9801, 2.8679, 0.8188) -- (2.9778, 2.8700, 0.8688) -- (2.9778, 2.8193, 0.8749) -- cycle;
\fill[blue!15.0, opacity=0.5] (2.9778, 2.8193, 0.8749) -- (2.9778, 2.8700, 0.8688) -- (2.9753, 2.8723, 0.9188) -- (2.9753, 2.8214, 0.9249) -- cycle;
\fill[blue!15.0, opacity=0.5] (2.9753, 2.8214, 0.9249) -- (2.9753, 2.8723, 0.9188) -- (2.9727, 2.8746, 0.9688) -- (2.9727, 2.8237, 0.9749) -- cycle;
\fill[blue!15.0, opacity=0.5] (2.9727, 2.8237, 0.9749) -- (2.9727, 2.8746, 0.9688) -- (2.9700, 2.8770, 1.0188) -- (2.9700, 2.8260, 1.0249) -- cycle;
\fill[blue!15.0, opacity=0.5] (2.9700, 2.8260, 1.0249) -- (2.9700, 2.8770, 1.0188) -- (2.9672, 2.8795, 1.0688) -- (2.9672, 2.8284, 1.0749) -- cycle;
\fill[blue!15.0, opacity=0.5] (2.9672, 2.8284, 1.0749) -- (2.9672, 2.8795, 1.0688) -- (2.9644, 2.8820, 1.1188) -- (2.9644, 2.8308, 1.1249) -- cycle;
\fill[blue!15.0, opacity=0.5] (2.9644, 2.8308, 1.1249) -- (2.9644, 2.8820, 1.1188) -- (2.9615, 2.8846, 1.1688) -- (2.9615, 2.8334, 1.1749) -- cycle;
\fill[blue!15.0, opacity=0.5] (2.9615, 2.8334, 1.1749) -- (2.9615, 2.8846, 1.1688) -- (2.9585, 2.8873, 1.2188) -- (2.9585, 2.8359, 1.2249) -- cycle;
\fill[blue!15.0, opacity=0.5] (2.9585, 2.8359, 1.2249) -- (2.9585, 2.8873, 1.2188) -- (2.9555, 2.8900, 1.2688) -- (2.9555, 2.8385, 1.2749) -- cycle;
\fill[blue!15.0, opacity=0.5] (2.9555, 2.8385, 1.2749) -- (2.9555, 2.8900, 1.2688) -- (2.9525, 2.8928, 1.3188) -- (2.9525, 2.8412, 1.3249) -- cycle;
\fill[blue!15.0, opacity=0.5] (2.9525, 2.8412, 1.3249) -- (2.9525, 2.8928, 1.3188) -- (2.9494, 2.8956, 1.3688) -- (2.9494, 2.8439, 1.3749) -- cycle;
\fill[blue!15.0, opacity=0.5] (2.9494, 2.8439, 1.3749) -- (2.9494, 2.8956, 1.3688) -- (2.9463, 2.8984, 1.4188) -- (2.9463, 2.8466, 1.4249) -- cycle;
\fill[blue!15.0, opacity=0.5] (2.9463, 2.8466, 1.4249) -- (2.9463, 2.8984, 1.4188) -- (2.9431, 2.9012, 1.4688) -- (2.9431, 2.8493, 1.4749) -- cycle;
\fill[blue!15.0, opacity=0.5] (2.9431, 2.8493, 1.4749) -- (2.9431, 2.9012, 1.4688) -- (2.9400, 2.9040, 1.5188) -- (2.9400, 2.8520, 1.5249) -- cycle;
\fill[blue!15.0, opacity=0.5] (2.9400, 2.8520, 1.5249) -- (2.9400, 2.9040, 1.5188) -- (2.9369, 2.9068, 1.5688) -- (2.9369, 2.8547, 1.5749) -- cycle;
\fill[blue!15.0, opacity=0.5] (2.9369, 2.8547, 1.5749) -- (2.9369, 2.9068, 1.5688) -- (2.9337, 2.9096, 1.6188) -- (2.9337, 2.8574, 1.6249) -- cycle;
\fill[blue!15.0, opacity=0.5] (2.9337, 2.8574, 1.6249) -- (2.9337, 2.9096, 1.6188) -- (2.9306, 2.9124, 1.6688) -- (2.9306, 2.8601, 1.6749) -- cycle;
\fill[blue!15.0, opacity=0.5] (2.9306, 2.8601, 1.6749) -- (2.9306, 2.9124, 1.6688) -- (2.9275, 2.9152, 1.7188) -- (2.9275, 2.8628, 1.7249) -- cycle;
\fill[blue!15.0, opacity=0.5] (2.9275, 2.8628, 1.7249) -- (2.9275, 2.9152, 1.7188) -- (2.9245, 2.9180, 1.7688) -- (2.9245, 2.8655, 1.7749) -- cycle;
\fill[blue!15.0, opacity=0.5] (2.9245, 2.8655, 1.7749) -- (2.9245, 2.9180, 1.7688) -- (2.9215, 2.9207, 1.8188) -- (2.9215, 2.8681, 1.8249) -- cycle;
\fill[blue!15.0, opacity=0.5] (2.9215, 2.8681, 1.8249) -- (2.9215, 2.9207, 1.8188) -- (2.9185, 2.9234, 1.8688) -- (2.9185, 2.8706, 1.8749) -- cycle;
\fill[blue!15.0, opacity=0.5] (2.9185, 2.8706, 1.8749) -- (2.9185, 2.9234, 1.8688) -- (2.9156, 2.9260, 1.9188) -- (2.9156, 2.8732, 1.9249) -- cycle;
\fill[blue!15.0, opacity=0.5] (2.9156, 2.8732, 1.9249) -- (2.9156, 2.9260, 1.9188) -- (2.9128, 2.9285, 1.9688) -- (2.9128, 2.8756, 1.9749) -- cycle;
\fill[blue!15.0, opacity=0.5] (2.9128, 2.8756, 1.9749) -- (2.9128, 2.9285, 1.9688) -- (2.9100, 2.9310, 2.0188) -- (2.9100, 2.8780, 2.0249) -- cycle;
\fill[blue!15.0, opacity=0.5] (2.9100, 2.8780, 2.0249) -- (2.9100, 2.9310, 2.0188) -- (2.9073, 2.9334, 2.0688) -- (2.9073, 2.8803, 2.0749) -- cycle;
\fill[blue!15.1, opacity=0.5] (2.9073, 2.8803, 2.0749) -- (2.9073, 2.9334, 2.0688) -- (2.9047, 2.9357, 2.1188) -- (2.9047, 2.8826, 2.1249) -- cycle;
\fill[blue!15.1, opacity=0.5] (2.9047, 2.8826, 2.1249) -- (2.9047, 2.9357, 2.1188) -- (2.9022, 2.9380, 2.1688) -- (2.9022, 2.8847, 2.1749) -- cycle;
\fill[blue!15.1, opacity=0.5] (2.9022, 2.8847, 2.1749) -- (2.9022, 2.9380, 2.1688) -- (2.8999, 2.9401, 2.2188) -- (2.8999, 2.8868, 2.2249) -- cycle;
\fill[blue!15.1, opacity=0.5] (2.8999, 2.8868, 2.2249) -- (2.8999, 2.9401, 2.2188) -- (2.8976, 2.9422, 2.2688) -- (2.8976, 2.8888, 2.2749) -- cycle;
\fill[blue!15.2, opacity=0.5] (2.8976, 2.8888, 2.2749) -- (2.8976, 2.9422, 2.2688) -- (2.8954, 2.9441, 2.3188) -- (2.8954, 2.8906, 2.3249) -- cycle;
\fill[blue!15.3, opacity=0.5] (2.8954, 2.8906, 2.3249) -- (2.8954, 2.9441, 2.3188) -- (2.8934, 2.9460, 2.3688) -- (2.8934, 2.8924, 2.3749) -- cycle;
\fill[blue!15.3, opacity=0.5] (2.8934, 2.8924, 2.3749) -- (2.8934, 2.9460, 2.3688) -- (2.8915, 2.9477, 2.4188) -- (2.8915, 2.8941, 2.4249) -- cycle;
\fill[blue!15.4, opacity=0.5] (2.8915, 2.8941, 2.4249) -- (2.8915, 2.9477, 2.4188) -- (2.8897, 2.9493, 2.4688) -- (2.8897, 2.8956, 2.4749) -- cycle;
\fill[blue!15.6, opacity=0.5] (2.8897, 2.8956, 2.4749) -- (2.8897, 2.9493, 2.4688) -- (2.8880, 2.9508, 2.5188) -- (2.8880, 2.8970, 2.5249) -- cycle;
\fill[blue!15.7, opacity=0.5] (2.8880, 2.8970, 2.5249) -- (2.8880, 2.9508, 2.5188) -- (2.8865, 2.9521, 2.5688) -- (2.8865, 2.8983, 2.5749) -- cycle;
\fill[blue!15.9, opacity=0.5] (2.8865, 2.8983, 2.5749) -- (2.8865, 2.9521, 2.5688) -- (2.8852, 2.9533, 2.6188) -- (2.8852, 2.8995, 2.6249) -- cycle;
\fill[blue!16.1, opacity=0.5] (2.8852, 2.8995, 2.6249) -- (2.8852, 2.9533, 2.6188) -- (2.8840, 2.9544, 2.6688) -- (2.8840, 2.9005, 2.6749) -- cycle;
\fill[blue!16.3, opacity=0.5] (2.8840, 2.9005, 2.6749) -- (2.8840, 2.9544, 2.6688) -- (2.8829, 2.9554, 2.7188) -- (2.8829, 2.9015, 2.7249) -- cycle;
\fill[blue!16.5, opacity=0.5] (2.8829, 2.9015, 2.7249) -- (2.8829, 2.9554, 2.7188) -- (2.8820, 2.9562, 2.7688) -- (2.8820, 2.9022, 2.7749) -- cycle;
\fill[blue!16.8, opacity=0.5] (2.8820, 2.9022, 2.7749) -- (2.8820, 2.9562, 2.7688) -- (2.8813, 2.9568, 2.8188) -- (2.8813, 2.9029, 2.8249) -- cycle;
\fill[blue!17.1, opacity=0.5] (2.8813, 2.9029, 2.8249) -- (2.8813, 2.9568, 2.8188) -- (2.8807, 2.9573, 2.8688) -- (2.8807, 2.9034, 2.8749) -- cycle;
\fill[blue!17.5, opacity=0.5] (2.8807, 2.9034, 2.8749) -- (2.8807, 2.9573, 2.8688) -- (2.8803, 2.9577, 2.9188) -- (2.8803, 2.9037, 2.9249) -- cycle;
\fill[blue!17.8, opacity=0.5] (2.8803, 2.9037, 2.9249) -- (2.8803, 2.9577, 2.9188) -- (2.8801, 2.9579, 2.9688) -- (2.8801, 2.9039, 2.9749) -- cycle;
\fill[blue!18.2, opacity=0.5] (2.8801, 2.9039, 2.9749) -- (2.8801, 2.9579, 2.9688) -- (2.8800, 2.9580, 3.0188) -- (2.8800, 2.9040, 3.0249) -- cycle;
\fill[blue!15.0, opacity=0.5] (3.0000, 2.8500, 0.0188) -- (3.0000, 2.9000, 0.0125) -- (2.9999, 2.9001, 0.0625) -- (2.9999, 2.8501, 0.0688) -- cycle;
\fill[blue!15.0, opacity=0.5] (2.9999, 2.8501, 0.0688) -- (2.9999, 2.9001, 0.0625) -- (2.9997, 2.9003, 0.1125) -- (2.9997, 2.8503, 0.1188) -- cycle;
\fill[blue!15.0, opacity=0.5] (2.9997, 2.8503, 0.1188) -- (2.9997, 2.9003, 0.1125) -- (2.9993, 2.9007, 0.1625) -- (2.9993, 2.8507, 0.1688) -- cycle;
\fill[blue!15.0, opacity=0.5] (2.9993, 2.8507, 0.1688) -- (2.9993, 2.9007, 0.1625) -- (2.9987, 2.9012, 0.2125) -- (2.9987, 2.8512, 0.2188) -- cycle;
\fill[blue!15.0, opacity=0.5] (2.9987, 2.8512, 0.2188) -- (2.9987, 2.9012, 0.2125) -- (2.9980, 2.9019, 0.2625) -- (2.9980, 2.8518, 0.2688) -- cycle;
\fill[blue!15.0, opacity=0.5] (2.9980, 2.8518, 0.2688) -- (2.9980, 2.9019, 0.2625) -- (2.9971, 2.9027, 0.3125) -- (2.9971, 2.8526, 0.3188) -- cycle;
\fill[blue!15.0, opacity=0.5] (2.9971, 2.8526, 0.3188) -- (2.9971, 2.9027, 0.3125) -- (2.9960, 2.9037, 0.3625) -- (2.9960, 2.8536, 0.3688) -- cycle;
\fill[blue!15.0, opacity=0.5] (2.9960, 2.8536, 0.3688) -- (2.9960, 2.9037, 0.3625) -- (2.9948, 2.9048, 0.4125) -- (2.9948, 2.8547, 0.4188) -- cycle;
\fill[blue!15.0, opacity=0.5] (2.9948, 2.8547, 0.4188) -- (2.9948, 2.9048, 0.4125) -- (2.9935, 2.9061, 0.4625) -- (2.9935, 2.8559, 0.4688) -- cycle;
\fill[blue!15.0, opacity=0.5] (2.9935, 2.8559, 0.4688) -- (2.9935, 2.9061, 0.4625) -- (2.9920, 2.9075, 0.5125) -- (2.9920, 2.8572, 0.5188) -- cycle;
\fill[blue!15.0, opacity=0.5] (2.9920, 2.8572, 0.5188) -- (2.9920, 2.9075, 0.5125) -- (2.9903, 2.9090, 0.5625) -- (2.9903, 2.8587, 0.5688) -- cycle;
\fill[blue!15.0, opacity=0.5] (2.9903, 2.8587, 0.5688) -- (2.9903, 2.9090, 0.5625) -- (2.9885, 2.9107, 0.6125) -- (2.9885, 2.8603, 0.6188) -- cycle;
\fill[blue!15.0, opacity=0.5] (2.9885, 2.8603, 0.6188) -- (2.9885, 2.9107, 0.6125) -- (2.9866, 2.9125, 0.6625) -- (2.9866, 2.8620, 0.6688) -- cycle;
\fill[blue!15.0, opacity=0.5] (2.9866, 2.8620, 0.6688) -- (2.9866, 2.9125, 0.6625) -- (2.9846, 2.9144, 0.7125) -- (2.9846, 2.8639, 0.7188) -- cycle;
\fill[blue!15.0, opacity=0.5] (2.9846, 2.8639, 0.7188) -- (2.9846, 2.9144, 0.7125) -- (2.9824, 2.9164, 0.7625) -- (2.9824, 2.8658, 0.7688) -- cycle;
\fill[blue!15.0, opacity=0.5] (2.9824, 2.8658, 0.7688) -- (2.9824, 2.9164, 0.7625) -- (2.9801, 2.9185, 0.8125) -- (2.9801, 2.8679, 0.8188) -- cycle;
\fill[blue!15.0, opacity=0.5] (2.9801, 2.8679, 0.8188) -- (2.9801, 2.9185, 0.8125) -- (2.9778, 2.9208, 0.8625) -- (2.9778, 2.8700, 0.8688) -- cycle;
\fill[blue!15.0, opacity=0.5] (2.9778, 2.8700, 0.8688) -- (2.9778, 2.9208, 0.8625) -- (2.9753, 2.9231, 0.9125) -- (2.9753, 2.8723, 0.9188) -- cycle;
\fill[blue!15.0, opacity=0.5] (2.9753, 2.8723, 0.9188) -- (2.9753, 2.9231, 0.9125) -- (2.9727, 2.9255, 0.9625) -- (2.9727, 2.8746, 0.9688) -- cycle;
\fill[blue!15.0, opacity=0.5] (2.9727, 2.8746, 0.9688) -- (2.9727, 2.9255, 0.9625) -- (2.9700, 2.9280, 1.0125) -- (2.9700, 2.8770, 1.0188) -- cycle;
\fill[blue!15.0, opacity=0.5] (2.9700, 2.8770, 1.0188) -- (2.9700, 2.9280, 1.0125) -- (2.9672, 2.9306, 1.0625) -- (2.9672, 2.8795, 1.0688) -- cycle;
\fill[blue!15.0, opacity=0.5] (2.9672, 2.8795, 1.0688) -- (2.9672, 2.9306, 1.0625) -- (2.9644, 2.9332, 1.1125) -- (2.9644, 2.8820, 1.1188) -- cycle;
\fill[blue!15.0, opacity=0.5] (2.9644, 2.8820, 1.1188) -- (2.9644, 2.9332, 1.1125) -- (2.9615, 2.9359, 1.1625) -- (2.9615, 2.8846, 1.1688) -- cycle;
\fill[blue!15.0, opacity=0.5] (2.9615, 2.8846, 1.1688) -- (2.9615, 2.9359, 1.1625) -- (2.9585, 2.9387, 1.2125) -- (2.9585, 2.8873, 1.2188) -- cycle;
\fill[blue!15.0, opacity=0.5] (2.9585, 2.8873, 1.2188) -- (2.9585, 2.9387, 1.2125) -- (2.9555, 2.9415, 1.2625) -- (2.9555, 2.8900, 1.2688) -- cycle;
\fill[blue!15.0, opacity=0.5] (2.9555, 2.8900, 1.2688) -- (2.9555, 2.9415, 1.2625) -- (2.9525, 2.9444, 1.3125) -- (2.9525, 2.8928, 1.3188) -- cycle;
\fill[blue!15.0, opacity=0.5] (2.9525, 2.8928, 1.3188) -- (2.9525, 2.9444, 1.3125) -- (2.9494, 2.9472, 1.3625) -- (2.9494, 2.8956, 1.3688) -- cycle;
\fill[blue!15.0, opacity=0.5] (2.9494, 2.8956, 1.3688) -- (2.9494, 2.9472, 1.3625) -- (2.9463, 2.9501, 1.4125) -- (2.9463, 2.8984, 1.4188) -- cycle;
\fill[blue!15.0, opacity=0.5] (2.9463, 2.8984, 1.4188) -- (2.9463, 2.9501, 1.4125) -- (2.9431, 2.9531, 1.4625) -- (2.9431, 2.9012, 1.4688) -- cycle;
\fill[blue!15.0, opacity=0.5] (2.9431, 2.9012, 1.4688) -- (2.9431, 2.9531, 1.4625) -- (2.9400, 2.9560, 1.5125) -- (2.9400, 2.9040, 1.5188) -- cycle;
\fill[blue!15.0, opacity=0.5] (2.9400, 2.9040, 1.5188) -- (2.9400, 2.9560, 1.5125) -- (2.9369, 2.9589, 1.5625) -- (2.9369, 2.9068, 1.5688) -- cycle;
\fill[blue!15.0, opacity=0.5] (2.9369, 2.9068, 1.5688) -- (2.9369, 2.9589, 1.5625) -- (2.9337, 2.9619, 1.6125) -- (2.9337, 2.9096, 1.6188) -- cycle;
\fill[blue!15.0, opacity=0.5] (2.9337, 2.9096, 1.6188) -- (2.9337, 2.9619, 1.6125) -- (2.9306, 2.9648, 1.6625) -- (2.9306, 2.9124, 1.6688) -- cycle;
\fill[blue!15.0, opacity=0.5] (2.9306, 2.9124, 1.6688) -- (2.9306, 2.9648, 1.6625) -- (2.9275, 2.9676, 1.7125) -- (2.9275, 2.9152, 1.7188) -- cycle;
\fill[blue!15.0, opacity=0.5] (2.9275, 2.9152, 1.7188) -- (2.9275, 2.9676, 1.7125) -- (2.9245, 2.9705, 1.7625) -- (2.9245, 2.9180, 1.7688) -- cycle;
\fill[blue!15.1, opacity=0.5] (2.9245, 2.9180, 1.7688) -- (2.9245, 2.9705, 1.7625) -- (2.9215, 2.9733, 1.8125) -- (2.9215, 2.9207, 1.8188) -- cycle;
\fill[blue!15.1, opacity=0.5] (2.9215, 2.9207, 1.8188) -- (2.9215, 2.9733, 1.8125) -- (2.9185, 2.9761, 1.8625) -- (2.9185, 2.9234, 1.8688) -- cycle;
\fill[blue!15.1, opacity=0.5] (2.9185, 2.9234, 1.8688) -- (2.9185, 2.9761, 1.8625) -- (2.9156, 2.9788, 1.9125) -- (2.9156, 2.9260, 1.9188) -- cycle;
\fill[blue!15.2, opacity=0.5] (2.9156, 2.9260, 1.9188) -- (2.9156, 2.9788, 1.9125) -- (2.9128, 2.9814, 1.9625) -- (2.9128, 2.9285, 1.9688) -- cycle;
\fill[blue!15.3, opacity=0.5] (2.9128, 2.9285, 1.9688) -- (2.9128, 2.9814, 1.9625) -- (2.9100, 2.9840, 2.0125) -- (2.9100, 2.9310, 2.0188) -- cycle;
\fill[blue!15.4, opacity=0.5] (2.9100, 2.9310, 2.0188) -- (2.9100, 2.9840, 2.0125) -- (2.9073, 2.9865, 2.0625) -- (2.9073, 2.9334, 2.0688) -- cycle;
\fill[blue!15.5, opacity=0.5] (2.9073, 2.9334, 2.0688) -- (2.9073, 2.9865, 2.0625) -- (2.9047, 2.9889, 2.1125) -- (2.9047, 2.9357, 2.1188) -- cycle;
\fill[blue!15.6, opacity=0.5] (2.9047, 2.9357, 2.1188) -- (2.9047, 2.9889, 2.1125) -- (2.9022, 2.9912, 2.1625) -- (2.9022, 2.9380, 2.1688) -- cycle;
\fill[blue!15.8, opacity=0.5] (2.9022, 2.9380, 2.1688) -- (2.9022, 2.9912, 2.1625) -- (2.8999, 2.9935, 2.2125) -- (2.8999, 2.9401, 2.2188) -- cycle;
\fill[blue!16.0, opacity=0.5] (2.8999, 2.9401, 2.2188) -- (2.8999, 2.9935, 2.2125) -- (2.8976, 2.9956, 2.2625) -- (2.8976, 2.9422, 2.2688) -- cycle;
\fill[blue!16.3, opacity=0.5] (2.8976, 2.9422, 2.2688) -- (2.8976, 2.9956, 2.2625) -- (2.8954, 2.9976, 2.3125) -- (2.8954, 2.9441, 2.3188) -- cycle;
\fill[blue!16.6, opacity=0.5] (2.8954, 2.9441, 2.3188) -- (2.8954, 2.9976, 2.3125) -- (2.8934, 2.9995, 2.3625) -- (2.8934, 2.9460, 2.3688) -- cycle;
\fill[blue!17.0, opacity=0.5] (2.8934, 2.9460, 2.3688) -- (2.8934, 2.9995, 2.3625) -- (2.8915, 3.0013, 2.4125) -- (2.8915, 2.9477, 2.4188) -- cycle;
\fill[blue!17.3, opacity=0.5] (2.8915, 2.9477, 2.4188) -- (2.8915, 3.0013, 2.4125) -- (2.8897, 3.0030, 2.4625) -- (2.8897, 2.9493, 2.4688) -- cycle;
\fill[blue!17.8, opacity=0.5] (2.8897, 2.9493, 2.4688) -- (2.8897, 3.0030, 2.4625) -- (2.8880, 3.0045, 2.5125) -- (2.8880, 2.9508, 2.5188) -- cycle;
\fill[blue!18.3, opacity=0.5] (2.8880, 2.9508, 2.5188) -- (2.8880, 3.0045, 2.5125) -- (2.8865, 3.0059, 2.5625) -- (2.8865, 2.9521, 2.5688) -- cycle;
\fill[blue!18.8, opacity=0.5] (2.8865, 2.9521, 2.5688) -- (2.8865, 3.0059, 2.5625) -- (2.8852, 3.0072, 2.6125) -- (2.8852, 2.9533, 2.6188) -- cycle;
\fill[blue!19.4, opacity=0.5] (2.8852, 2.9533, 2.6188) -- (2.8852, 3.0072, 2.6125) -- (2.8840, 3.0083, 2.6625) -- (2.8840, 2.9544, 2.6688) -- cycle;
\fill[blue!20.1, opacity=0.5] (2.8840, 2.9544, 2.6688) -- (2.8840, 3.0083, 2.6625) -- (2.8829, 3.0093, 2.7125) -- (2.8829, 2.9554, 2.7188) -- cycle;
\fill[blue!20.7, opacity=0.5] (2.8829, 2.9554, 2.7188) -- (2.8829, 3.0093, 2.7125) -- (2.8820, 3.0101, 2.7625) -- (2.8820, 2.9562, 2.7688) -- cycle;
\fill[blue!21.4, opacity=0.5] (2.8820, 2.9562, 2.7688) -- (2.8820, 3.0101, 2.7625) -- (2.8813, 3.0108, 2.8125) -- (2.8813, 2.9568, 2.8188) -- cycle;
\fill[blue!22.2, opacity=0.5] (2.8813, 2.9568, 2.8188) -- (2.8813, 3.0108, 2.8125) -- (2.8807, 3.0113, 2.8625) -- (2.8807, 2.9573, 2.8688) -- cycle;
\fill[blue!22.9, opacity=0.5] (2.8807, 2.9573, 2.8688) -- (2.8807, 3.0113, 2.8625) -- (2.8803, 3.0117, 2.9125) -- (2.8803, 2.9577, 2.9188) -- cycle;
\fill[blue!23.7, opacity=0.5] (2.8803, 2.9577, 2.9188) -- (2.8803, 3.0117, 2.9125) -- (2.8801, 3.0119, 2.9625) -- (2.8801, 2.9579, 2.9688) -- cycle;
\fill[blue!24.5, opacity=0.5] (2.8801, 2.9579, 2.9688) -- (2.8801, 3.0119, 2.9625) -- (2.8800, 3.0120, 3.0125) -- (2.8800, 2.9580, 3.0188) -- cycle;
\fill[blue!15.0, opacity=0.5] (3.0000, 2.9000, 0.0125) -- (3.0000, 2.9500, 0.0063) -- (2.9999, 2.9501, 0.0563) -- (2.9999, 2.9001, 0.0625) -- cycle;
\fill[blue!15.0, opacity=0.5] (2.9999, 2.9001, 0.0625) -- (2.9999, 2.9501, 0.0563) -- (2.9997, 2.9503, 0.1063) -- (2.9997, 2.9003, 0.1125) -- cycle;
\fill[blue!15.0, opacity=0.5] (2.9997, 2.9003, 0.1125) -- (2.9997, 2.9503, 0.1063) -- (2.9993, 2.9507, 0.1563) -- (2.9993, 2.9007, 0.1625) -- cycle;
\fill[blue!15.0, opacity=0.5] (2.9993, 2.9007, 0.1625) -- (2.9993, 2.9507, 0.1563) -- (2.9987, 2.9513, 0.2063) -- (2.9987, 2.9012, 0.2125) -- cycle;
\fill[blue!15.0, opacity=0.5] (2.9987, 2.9012, 0.2125) -- (2.9987, 2.9513, 0.2063) -- (2.9980, 2.9520, 0.2563) -- (2.9980, 2.9019, 0.2625) -- cycle;
\fill[blue!15.0, opacity=0.5] (2.9980, 2.9019, 0.2625) -- (2.9980, 2.9520, 0.2563) -- (2.9971, 2.9528, 0.3063) -- (2.9971, 2.9027, 0.3125) -- cycle;
\fill[blue!15.0, opacity=0.5] (2.9971, 2.9027, 0.3125) -- (2.9971, 2.9528, 0.3063) -- (2.9960, 2.9539, 0.3563) -- (2.9960, 2.9037, 0.3625) -- cycle;
\fill[blue!15.0, opacity=0.5] (2.9960, 2.9037, 0.3625) -- (2.9960, 2.9539, 0.3563) -- (2.9948, 2.9550, 0.4063) -- (2.9948, 2.9048, 0.4125) -- cycle;
\fill[blue!15.0, opacity=0.5] (2.9948, 2.9048, 0.4125) -- (2.9948, 2.9550, 0.4063) -- (2.9935, 2.9563, 0.4563) -- (2.9935, 2.9061, 0.4625) -- cycle;
\fill[blue!15.0, opacity=0.5] (2.9935, 2.9061, 0.4625) -- (2.9935, 2.9563, 0.4563) -- (2.9920, 2.9578, 0.5063) -- (2.9920, 2.9075, 0.5125) -- cycle;
\fill[blue!15.0, opacity=0.5] (2.9920, 2.9075, 0.5125) -- (2.9920, 2.9578, 0.5063) -- (2.9903, 2.9594, 0.5563) -- (2.9903, 2.9090, 0.5625) -- cycle;
\fill[blue!15.0, opacity=0.5] (2.9903, 2.9090, 0.5625) -- (2.9903, 2.9594, 0.5563) -- (2.9885, 2.9611, 0.6063) -- (2.9885, 2.9107, 0.6125) -- cycle;
\fill[blue!15.0, opacity=0.5] (2.9885, 2.9107, 0.6125) -- (2.9885, 2.9611, 0.6063) -- (2.9866, 2.9629, 0.6563) -- (2.9866, 2.9125, 0.6625) -- cycle;
\fill[blue!15.0, opacity=0.5] (2.9866, 2.9125, 0.6625) -- (2.9866, 2.9629, 0.6563) -- (2.9846, 2.9649, 0.7063) -- (2.9846, 2.9144, 0.7125) -- cycle;
\fill[blue!15.0, opacity=0.5] (2.9846, 2.9144, 0.7125) -- (2.9846, 2.9649, 0.7063) -- (2.9824, 2.9670, 0.7563) -- (2.9824, 2.9164, 0.7625) -- cycle;
\fill[blue!15.0, opacity=0.5] (2.9824, 2.9164, 0.7625) -- (2.9824, 2.9670, 0.7563) -- (2.9801, 2.9692, 0.8063) -- (2.9801, 2.9185, 0.8125) -- cycle;
\fill[blue!15.0, opacity=0.5] (2.9801, 2.9185, 0.8125) -- (2.9801, 2.9692, 0.8063) -- (2.9778, 2.9715, 0.8563) -- (2.9778, 2.9208, 0.8625) -- cycle;
\fill[blue!15.0, opacity=0.5] (2.9778, 2.9208, 0.8625) -- (2.9778, 2.9715, 0.8563) -- (2.9753, 2.9739, 0.9063) -- (2.9753, 2.9231, 0.9125) -- cycle;
\fill[blue!15.0, opacity=0.5] (2.9753, 2.9231, 0.9125) -- (2.9753, 2.9739, 0.9063) -- (2.9727, 2.9764, 0.9563) -- (2.9727, 2.9255, 0.9625) -- cycle;
\fill[blue!15.0, opacity=0.5] (2.9727, 2.9255, 0.9625) -- (2.9727, 2.9764, 0.9563) -- (2.9700, 2.9790, 1.0063) -- (2.9700, 2.9280, 1.0125) -- cycle;
\fill[blue!15.0, opacity=0.5] (2.9700, 2.9280, 1.0125) -- (2.9700, 2.9790, 1.0063) -- (2.9672, 2.9817, 1.0563) -- (2.9672, 2.9306, 1.0625) -- cycle;
\fill[blue!15.0, opacity=0.5] (2.9672, 2.9306, 1.0625) -- (2.9672, 2.9817, 1.0563) -- (2.9644, 2.9844, 1.1063) -- (2.9644, 2.9332, 1.1125) -- cycle;
\fill[blue!15.0, opacity=0.5] (2.9644, 2.9332, 1.1125) -- (2.9644, 2.9844, 1.1063) -- (2.9615, 2.9872, 1.1563) -- (2.9615, 2.9359, 1.1625) -- cycle;
\fill[blue!15.0, opacity=0.5] (2.9615, 2.9359, 1.1625) -- (2.9615, 2.9872, 1.1563) -- (2.9585, 2.9901, 1.2063) -- (2.9585, 2.9387, 1.2125) -- cycle;
\fill[blue!15.0, opacity=0.5] (2.9585, 2.9387, 1.2125) -- (2.9585, 2.9901, 1.2063) -- (2.9555, 2.9930, 1.2563) -- (2.9555, 2.9415, 1.2625) -- cycle;
\fill[blue!15.0, opacity=0.5] (2.9555, 2.9415, 1.2625) -- (2.9555, 2.9930, 1.2563) -- (2.9525, 2.9959, 1.3063) -- (2.9525, 2.9444, 1.3125) -- cycle;
\fill[blue!15.0, opacity=0.5] (2.9525, 2.9444, 1.3125) -- (2.9525, 2.9959, 1.3063) -- (2.9494, 2.9989, 1.3563) -- (2.9494, 2.9472, 1.3625) -- cycle;
\fill[blue!15.0, opacity=0.5] (2.9494, 2.9472, 1.3625) -- (2.9494, 2.9989, 1.3563) -- (2.9463, 3.0019, 1.4063) -- (2.9463, 2.9501, 1.4125) -- cycle;
\fill[blue!15.0, opacity=0.5] (2.9463, 2.9501, 1.4125) -- (2.9463, 3.0019, 1.4063) -- (2.9431, 3.0050, 1.4563) -- (2.9431, 2.9531, 1.4625) -- cycle;
\fill[blue!15.0, opacity=0.5] (2.9431, 2.9531, 1.4625) -- (2.9431, 3.0050, 1.4563) -- (2.9400, 3.0080, 1.5063) -- (2.9400, 2.9560, 1.5125) -- cycle;
\fill[blue!15.0, opacity=0.5] (2.9400, 2.9560, 1.5125) -- (2.9400, 3.0080, 1.5063) -- (2.9369, 3.0110, 1.5563) -- (2.9369, 2.9589, 1.5625) -- cycle;
\fill[blue!15.1, opacity=0.5] (2.9369, 2.9589, 1.5625) -- (2.9369, 3.0110, 1.5563) -- (2.9337, 3.0141, 1.6063) -- (2.9337, 2.9619, 1.6125) -- cycle;
\fill[blue!15.1, opacity=0.5] (2.9337, 2.9619, 1.6125) -- (2.9337, 3.0141, 1.6063) -- (2.9306, 3.0171, 1.6563) -- (2.9306, 2.9648, 1.6625) -- cycle;
\fill[blue!15.2, opacity=0.5] (2.9306, 2.9648, 1.6625) -- (2.9306, 3.0171, 1.6563) -- (2.9275, 3.0201, 1.7063) -- (2.9275, 2.9676, 1.7125) -- cycle;
\fill[blue!15.2, opacity=0.5] (2.9275, 2.9676, 1.7125) -- (2.9275, 3.0201, 1.7063) -- (2.9245, 3.0230, 1.7563) -- (2.9245, 2.9705, 1.7625) -- cycle;
\fill[blue!15.3, opacity=0.5] (2.9245, 2.9705, 1.7625) -- (2.9245, 3.0230, 1.7563) -- (2.9215, 3.0259, 1.8063) -- (2.9215, 2.9733, 1.8125) -- cycle;
\fill[blue!15.4, opacity=0.5] (2.9215, 2.9733, 1.8125) -- (2.9215, 3.0259, 1.8063) -- (2.9185, 3.0288, 1.8563) -- (2.9185, 2.9761, 1.8625) -- cycle;
\fill[blue!15.6, opacity=0.5] (2.9185, 2.9761, 1.8625) -- (2.9185, 3.0288, 1.8563) -- (2.9156, 3.0316, 1.9063) -- (2.9156, 2.9788, 1.9125) -- cycle;
\fill[blue!15.8, opacity=0.5] (2.9156, 2.9788, 1.9125) -- (2.9156, 3.0316, 1.9063) -- (2.9128, 3.0343, 1.9563) -- (2.9128, 2.9814, 1.9625) -- cycle;
\fill[blue!16.0, opacity=0.5] (2.9128, 2.9814, 1.9625) -- (2.9128, 3.0343, 1.9563) -- (2.9100, 3.0370, 2.0063) -- (2.9100, 2.9840, 2.0125) -- cycle;
\fill[blue!16.3, opacity=0.5] (2.9100, 2.9840, 2.0125) -- (2.9100, 3.0370, 2.0063) -- (2.9073, 3.0396, 2.0563) -- (2.9073, 2.9865, 2.0625) -- cycle;
\fill[blue!16.7, opacity=0.5] (2.9073, 2.9865, 2.0625) -- (2.9073, 3.0396, 2.0563) -- (2.9047, 3.0421, 2.1063) -- (2.9047, 2.9889, 2.1125) -- cycle;
\fill[blue!17.1, opacity=0.5] (2.9047, 2.9889, 2.1125) -- (2.9047, 3.0421, 2.1063) -- (2.9022, 3.0445, 2.1563) -- (2.9022, 2.9912, 2.1625) -- cycle;
\fill[blue!17.5, opacity=0.5] (2.9022, 2.9912, 2.1625) -- (2.9022, 3.0445, 2.1563) -- (2.8999, 3.0468, 2.2063) -- (2.8999, 2.9935, 2.2125) -- cycle;
\fill[blue!18.1, opacity=0.5] (2.8999, 2.9935, 2.2125) -- (2.8999, 3.0468, 2.2063) -- (2.8976, 3.0490, 2.2563) -- (2.8976, 2.9956, 2.2625) -- cycle;
\fill[blue!18.7, opacity=0.5] (2.8976, 2.9956, 2.2625) -- (2.8976, 3.0490, 2.2563) -- (2.8954, 3.0511, 2.3063) -- (2.8954, 2.9976, 2.3125) -- cycle;
\fill[blue!19.3, opacity=0.5] (2.8954, 2.9976, 2.3125) -- (2.8954, 3.0511, 2.3063) -- (2.8934, 3.0531, 2.3563) -- (2.8934, 2.9995, 2.3625) -- cycle;
\fill[blue!20.0, opacity=0.5] (2.8934, 2.9995, 2.3625) -- (2.8934, 3.0531, 2.3563) -- (2.8915, 3.0549, 2.4063) -- (2.8915, 3.0013, 2.4125) -- cycle;
\fill[blue!20.8, opacity=0.5] (2.8915, 3.0013, 2.4125) -- (2.8915, 3.0549, 2.4063) -- (2.8897, 3.0566, 2.4563) -- (2.8897, 3.0030, 2.4625) -- cycle;
\fill[blue!21.6, opacity=0.5] (2.8897, 3.0030, 2.4625) -- (2.8897, 3.0566, 2.4563) -- (2.8880, 3.0582, 2.5063) -- (2.8880, 3.0045, 2.5125) -- cycle;
\fill[blue!22.5, opacity=0.5] (2.8880, 3.0045, 2.5125) -- (2.8880, 3.0582, 2.5063) -- (2.8865, 3.0597, 2.5563) -- (2.8865, 3.0059, 2.5625) -- cycle;
\fill[blue!23.5, opacity=0.5] (2.8865, 3.0059, 2.5625) -- (2.8865, 3.0597, 2.5563) -- (2.8852, 3.0610, 2.6063) -- (2.8852, 3.0072, 2.6125) -- cycle;
\fill[blue!24.4, opacity=0.5] (2.8852, 3.0072, 2.6125) -- (2.8852, 3.0610, 2.6063) -- (2.8840, 3.0621, 2.6563) -- (2.8840, 3.0083, 2.6625) -- cycle;
\fill[blue!25.4, opacity=0.5] (2.8840, 3.0083, 2.6625) -- (2.8840, 3.0621, 2.6563) -- (2.8829, 3.0632, 2.7063) -- (2.8829, 3.0093, 2.7125) -- cycle;
\fill[blue!26.5, opacity=0.5] (2.8829, 3.0093, 2.7125) -- (2.8829, 3.0632, 2.7063) -- (2.8820, 3.0640, 2.7563) -- (2.8820, 3.0101, 2.7625) -- cycle;
\fill[blue!27.5, opacity=0.5] (2.8820, 3.0101, 2.7625) -- (2.8820, 3.0640, 2.7563) -- (2.8813, 3.0647, 2.8063) -- (2.8813, 3.0108, 2.8125) -- cycle;
\fill[blue!28.5, opacity=0.5] (2.8813, 3.0108, 2.8125) -- (2.8813, 3.0647, 2.8063) -- (2.8807, 3.0653, 2.8563) -- (2.8807, 3.0113, 2.8625) -- cycle;
\fill[blue!29.5, opacity=0.5] (2.8807, 3.0113, 2.8625) -- (2.8807, 3.0653, 2.8563) -- (2.8803, 3.0657, 2.9063) -- (2.8803, 3.0117, 2.9125) -- cycle;
\fill[blue!30.5, opacity=0.5] (2.8803, 3.0117, 2.9125) -- (2.8803, 3.0657, 2.9063) -- (2.8801, 3.0659, 2.9563) -- (2.8801, 3.0119, 2.9625) -- cycle;
\fill[blue!31.5, opacity=0.5] (2.8801, 3.0119, 2.9625) -- (2.8801, 3.0659, 2.9563) -- (2.8800, 3.0660, 3.0063) -- (2.8800, 3.0120, 3.0125) -- cycle;
\fill[blue!15.0, opacity=0.5] (3.0000, 2.9500, 0.0063) -- (3.0000, 3.0000, 0.0000) -- (2.9999, 3.0001, 0.0500) -- (2.9999, 2.9501, 0.0563) -- cycle;
\fill[blue!15.0, opacity=0.5] (2.9999, 2.9501, 0.0563) -- (2.9999, 3.0001, 0.0500) -- (2.9997, 3.0003, 0.1000) -- (2.9997, 2.9503, 0.1063) -- cycle;
\fill[blue!15.0, opacity=0.5] (2.9997, 2.9503, 0.1063) -- (2.9997, 3.0003, 0.1000) -- (2.9993, 3.0007, 0.1500) -- (2.9993, 2.9507, 0.1563) -- cycle;
\fill[blue!15.0, opacity=0.5] (2.9993, 2.9507, 0.1563) -- (2.9993, 3.0007, 0.1500) -- (2.9987, 3.0013, 0.2000) -- (2.9987, 2.9513, 0.2063) -- cycle;
\fill[blue!15.0, opacity=0.5] (2.9987, 2.9513, 0.2063) -- (2.9987, 3.0013, 0.2000) -- (2.9980, 3.0020, 0.2500) -- (2.9980, 2.9520, 0.2563) -- cycle;
\fill[blue!15.0, opacity=0.5] (2.9980, 2.9520, 0.2563) -- (2.9980, 3.0020, 0.2500) -- (2.9971, 3.0029, 0.3000) -- (2.9971, 2.9528, 0.3063) -- cycle;
\fill[blue!15.0, opacity=0.5] (2.9971, 2.9528, 0.3063) -- (2.9971, 3.0029, 0.3000) -- (2.9960, 3.0040, 0.3500) -- (2.9960, 2.9539, 0.3563) -- cycle;
\fill[blue!15.0, opacity=0.5] (2.9960, 2.9539, 0.3563) -- (2.9960, 3.0040, 0.3500) -- (2.9948, 3.0052, 0.4000) -- (2.9948, 2.9550, 0.4063) -- cycle;
\fill[blue!15.0, opacity=0.5] (2.9948, 2.9550, 0.4063) -- (2.9948, 3.0052, 0.4000) -- (2.9935, 3.0065, 0.4500) -- (2.9935, 2.9563, 0.4563) -- cycle;
\fill[blue!15.0, opacity=0.5] (2.9935, 2.9563, 0.4563) -- (2.9935, 3.0065, 0.4500) -- (2.9920, 3.0080, 0.5000) -- (2.9920, 2.9578, 0.5063) -- cycle;
\fill[blue!15.0, opacity=0.5] (2.9920, 2.9578, 0.5063) -- (2.9920, 3.0080, 0.5000) -- (2.9903, 3.0097, 0.5500) -- (2.9903, 2.9594, 0.5563) -- cycle;
\fill[blue!15.0, opacity=0.5] (2.9903, 2.9594, 0.5563) -- (2.9903, 3.0097, 0.5500) -- (2.9885, 3.0115, 0.6000) -- (2.9885, 2.9611, 0.6063) -- cycle;
\fill[blue!15.0, opacity=0.5] (2.9885, 2.9611, 0.6063) -- (2.9885, 3.0115, 0.6000) -- (2.9866, 3.0134, 0.6500) -- (2.9866, 2.9629, 0.6563) -- cycle;
\fill[blue!15.0, opacity=0.5] (2.9866, 2.9629, 0.6563) -- (2.9866, 3.0134, 0.6500) -- (2.9846, 3.0154, 0.7000) -- (2.9846, 2.9649, 0.7063) -- cycle;
\fill[blue!15.0, opacity=0.5] (2.9846, 2.9649, 0.7063) -- (2.9846, 3.0154, 0.7000) -- (2.9824, 3.0176, 0.7500) -- (2.9824, 2.9670, 0.7563) -- cycle;
\fill[blue!15.0, opacity=0.5] (2.9824, 2.9670, 0.7563) -- (2.9824, 3.0176, 0.7500) -- (2.9801, 3.0199, 0.8000) -- (2.9801, 2.9692, 0.8063) -- cycle;
\fill[blue!15.0, opacity=0.5] (2.9801, 2.9692, 0.8063) -- (2.9801, 3.0199, 0.8000) -- (2.9778, 3.0222, 0.8500) -- (2.9778, 2.9715, 0.8563) -- cycle;
\fill[blue!15.0, opacity=0.5] (2.9778, 2.9715, 0.8563) -- (2.9778, 3.0222, 0.8500) -- (2.9753, 3.0247, 0.9000) -- (2.9753, 2.9739, 0.9063) -- cycle;
\fill[blue!15.0, opacity=0.5] (2.9753, 2.9739, 0.9063) -- (2.9753, 3.0247, 0.9000) -- (2.9727, 3.0273, 0.9500) -- (2.9727, 2.9764, 0.9563) -- cycle;
\fill[blue!15.0, opacity=0.5] (2.9727, 2.9764, 0.9563) -- (2.9727, 3.0273, 0.9500) -- (2.9700, 3.0300, 1.0000) -- (2.9700, 2.9790, 1.0063) -- cycle;
\fill[blue!15.0, opacity=0.5] (2.9700, 2.9790, 1.0063) -- (2.9700, 3.0300, 1.0000) -- (2.9672, 3.0328, 1.0500) -- (2.9672, 2.9817, 1.0563) -- cycle;
\fill[blue!15.0, opacity=0.5] (2.9672, 2.9817, 1.0563) -- (2.9672, 3.0328, 1.0500) -- (2.9644, 3.0356, 1.1000) -- (2.9644, 2.9844, 1.1063) -- cycle;
\fill[blue!15.0, opacity=0.5] (2.9644, 2.9844, 1.1063) -- (2.9644, 3.0356, 1.1000) -- (2.9615, 3.0385, 1.1500) -- (2.9615, 2.9872, 1.1563) -- cycle;
\fill[blue!15.0, opacity=0.5] (2.9615, 2.9872, 1.1563) -- (2.9615, 3.0385, 1.1500) -- (2.9585, 3.0415, 1.2000) -- (2.9585, 2.9901, 1.2063) -- cycle;
\fill[blue!15.0, opacity=0.5] (2.9585, 2.9901, 1.2063) -- (2.9585, 3.0415, 1.2000) -- (2.9555, 3.0445, 1.2500) -- (2.9555, 2.9930, 1.2563) -- cycle;
\fill[blue!15.0, opacity=0.5] (2.9555, 2.9930, 1.2563) -- (2.9555, 3.0445, 1.2500) -- (2.9525, 3.0475, 1.3000) -- (2.9525, 2.9959, 1.3063) -- cycle;
\fill[blue!15.0, opacity=0.5] (2.9525, 2.9959, 1.3063) -- (2.9525, 3.0475, 1.3000) -- (2.9494, 3.0506, 1.3500) -- (2.9494, 2.9989, 1.3563) -- cycle;
\fill[blue!15.0, opacity=0.5] (2.9494, 2.9989, 1.3563) -- (2.9494, 3.0506, 1.3500) -- (2.9463, 3.0537, 1.4000) -- (2.9463, 3.0019, 1.4063) -- cycle;
\fill[blue!15.0, opacity=0.5] (2.9463, 3.0019, 1.4063) -- (2.9463, 3.0537, 1.4000) -- (2.9431, 3.0569, 1.4500) -- (2.9431, 3.0050, 1.4563) -- cycle;
\fill[blue!15.0, opacity=0.5] (2.9431, 3.0050, 1.4563) -- (2.9431, 3.0569, 1.4500) -- (2.9400, 3.0600, 1.5000) -- (2.9400, 3.0080, 1.5063) -- cycle;
\fill[blue!15.1, opacity=0.5] (2.9400, 3.0080, 1.5063) -- (2.9400, 3.0600, 1.5000) -- (2.9369, 3.0631, 1.5500) -- (2.9369, 3.0110, 1.5563) -- cycle;
\fill[blue!15.1, opacity=0.5] (2.9369, 3.0110, 1.5563) -- (2.9369, 3.0631, 1.5500) -- (2.9337, 3.0663, 1.6000) -- (2.9337, 3.0141, 1.6063) -- cycle;
\fill[blue!15.1, opacity=0.5] (2.9337, 3.0141, 1.6063) -- (2.9337, 3.0663, 1.6000) -- (2.9306, 3.0694, 1.6500) -- (2.9306, 3.0171, 1.6563) -- cycle;
\fill[blue!15.2, opacity=0.5] (2.9306, 3.0171, 1.6563) -- (2.9306, 3.0694, 1.6500) -- (2.9275, 3.0725, 1.7000) -- (2.9275, 3.0201, 1.7063) -- cycle;
\fill[blue!15.2, opacity=0.5] (2.9275, 3.0201, 1.7063) -- (2.9275, 3.0725, 1.7000) -- (2.9245, 3.0755, 1.7500) -- (2.9245, 3.0230, 1.7563) -- cycle;
\fill[blue!15.3, opacity=0.5] (2.9245, 3.0230, 1.7563) -- (2.9245, 3.0755, 1.7500) -- (2.9215, 3.0785, 1.8000) -- (2.9215, 3.0259, 1.8063) -- cycle;
\fill[blue!15.5, opacity=0.5] (2.9215, 3.0259, 1.8063) -- (2.9215, 3.0785, 1.8000) -- (2.9185, 3.0815, 1.8500) -- (2.9185, 3.0288, 1.8563) -- cycle;
\fill[blue!15.6, opacity=0.5] (2.9185, 3.0288, 1.8563) -- (2.9185, 3.0815, 1.8500) -- (2.9156, 3.0844, 1.9000) -- (2.9156, 3.0316, 1.9063) -- cycle;
\fill[blue!15.9, opacity=0.5] (2.9156, 3.0316, 1.9063) -- (2.9156, 3.0844, 1.9000) -- (2.9128, 3.0872, 1.9500) -- (2.9128, 3.0343, 1.9563) -- cycle;
\fill[blue!16.1, opacity=0.5] (2.9128, 3.0343, 1.9563) -- (2.9128, 3.0872, 1.9500) -- (2.9100, 3.0900, 2.0000) -- (2.9100, 3.0370, 2.0063) -- cycle;
\fill[blue!16.4, opacity=0.5] (2.9100, 3.0370, 2.0063) -- (2.9100, 3.0900, 2.0000) -- (2.9073, 3.0927, 2.0500) -- (2.9073, 3.0396, 2.0563) -- cycle;
\fill[blue!16.8, opacity=0.5] (2.9073, 3.0396, 2.0563) -- (2.9073, 3.0927, 2.0500) -- (2.9047, 3.0953, 2.1000) -- (2.9047, 3.0421, 2.1063) -- cycle;
\fill[blue!17.2, opacity=0.5] (2.9047, 3.0421, 2.1063) -- (2.9047, 3.0953, 2.1000) -- (2.9022, 3.0978, 2.1500) -- (2.9022, 3.0445, 2.1563) -- cycle;
\fill[blue!17.7, opacity=0.5] (2.9022, 3.0445, 2.1563) -- (2.9022, 3.0978, 2.1500) -- (2.8999, 3.1001, 2.2000) -- (2.8999, 3.0468, 2.2063) -- cycle;
\fill[blue!18.2, opacity=0.5] (2.8999, 3.0468, 2.2063) -- (2.8999, 3.1001, 2.2000) -- (2.8976, 3.1024, 2.2500) -- (2.8976, 3.0490, 2.2563) -- cycle;
\fill[blue!18.8, opacity=0.5] (2.8976, 3.0490, 2.2563) -- (2.8976, 3.1024, 2.2500) -- (2.8954, 3.1046, 2.3000) -- (2.8954, 3.0511, 2.3063) -- cycle;
\fill[blue!19.5, opacity=0.5] (2.8954, 3.0511, 2.3063) -- (2.8954, 3.1046, 2.3000) -- (2.8934, 3.1066, 2.3500) -- (2.8934, 3.0531, 2.3563) -- cycle;
\fill[blue!20.3, opacity=0.5] (2.8934, 3.0531, 2.3563) -- (2.8934, 3.1066, 2.3500) -- (2.8915, 3.1085, 2.4000) -- (2.8915, 3.0549, 2.4063) -- cycle;
\fill[blue!21.1, opacity=0.5] (2.8915, 3.0549, 2.4063) -- (2.8915, 3.1085, 2.4000) -- (2.8897, 3.1103, 2.4500) -- (2.8897, 3.0566, 2.4563) -- cycle;
\fill[blue!21.9, opacity=0.5] (2.8897, 3.0566, 2.4563) -- (2.8897, 3.1103, 2.4500) -- (2.8880, 3.1120, 2.5000) -- (2.8880, 3.0582, 2.5063) -- cycle;
\fill[blue!22.8, opacity=0.5] (2.8880, 3.0582, 2.5063) -- (2.8880, 3.1120, 2.5000) -- (2.8865, 3.1135, 2.5500) -- (2.8865, 3.0597, 2.5563) -- cycle;
\fill[blue!23.8, opacity=0.5] (2.8865, 3.0597, 2.5563) -- (2.8865, 3.1135, 2.5500) -- (2.8852, 3.1148, 2.6000) -- (2.8852, 3.0610, 2.6063) -- cycle;
\fill[blue!24.8, opacity=0.5] (2.8852, 3.0610, 2.6063) -- (2.8852, 3.1148, 2.6000) -- (2.8840, 3.1160, 2.6500) -- (2.8840, 3.0621, 2.6563) -- cycle;
\fill[blue!25.8, opacity=0.5] (2.8840, 3.0621, 2.6563) -- (2.8840, 3.1160, 2.6500) -- (2.8829, 3.1171, 2.7000) -- (2.8829, 3.0632, 2.7063) -- cycle;
\fill[blue!26.8, opacity=0.5] (2.8829, 3.0632, 2.7063) -- (2.8829, 3.1171, 2.7000) -- (2.8820, 3.1180, 2.7500) -- (2.8820, 3.0640, 2.7563) -- cycle;
\fill[blue!27.9, opacity=0.5] (2.8820, 3.0640, 2.7563) -- (2.8820, 3.1180, 2.7500) -- (2.8813, 3.1187, 2.8000) -- (2.8813, 3.0647, 2.8063) -- cycle;
\fill[blue!28.9, opacity=0.5] (2.8813, 3.0647, 2.8063) -- (2.8813, 3.1187, 2.8000) -- (2.8807, 3.1193, 2.8500) -- (2.8807, 3.0653, 2.8563) -- cycle;
\fill[blue!30.0, opacity=0.5] (2.8807, 3.0653, 2.8563) -- (2.8807, 3.1193, 2.8500) -- (2.8803, 3.1197, 2.9000) -- (2.8803, 3.0657, 2.9063) -- cycle;
\fill[blue!31.0, opacity=0.5] (2.8803, 3.0657, 2.9063) -- (2.8803, 3.1197, 2.9000) -- (2.8801, 3.1199, 2.9500) -- (2.8801, 3.0659, 2.9563) -- cycle;
\fill[blue!31.9, opacity=0.5] (2.8801, 3.0659, 2.9563) -- (2.8801, 3.1199, 2.9500) -- (2.8800, 3.1200, 3.0000) -- (2.8800, 3.0660, 3.0063) -- cycle;
% Top slice horizontal patches
\fill[blue!33.5, opacity=0.7] (0.1200, -0.1200, 3.0000) -- (0.1660, -0.1200, 3.0063) -- (0.1660, -0.0660, 3.0126) -- (0.1200, -0.0660, 3.0063) -- cycle;
\fill[blue!29.3, opacity=0.7] (0.1200, -0.0660, 3.0063) -- (0.1660, -0.0660, 3.0126) -- (0.1660, -0.0120, 3.0188) -- (0.1200, -0.0120, 3.0125) -- cycle;
\fill[blue!21.4, opacity=0.7] (0.1200, -0.0120, 3.0125) -- (0.1660, -0.0120, 3.0188) -- (0.1660, 0.0420, 3.0251) -- (0.1200, 0.0420, 3.0188) -- cycle;
\fill[blue!16.8, opacity=0.7] (0.1200, 0.0420, 3.0188) -- (0.1660, 0.0420, 3.0251) -- (0.1660, 0.0960, 3.0312) -- (0.1200, 0.0960, 3.0249) -- cycle;
\fill[blue!15.5, opacity=0.7] (0.1200, 0.0960, 3.0249) -- (0.1660, 0.0960, 3.0312) -- (0.1660, 0.1500, 3.0373) -- (0.1200, 0.1500, 3.0311) -- cycle;
\fill[blue!15.2, opacity=0.7] (0.1200, 0.1500, 3.0311) -- (0.1660, 0.1500, 3.0373) -- (0.1660, 0.2040, 3.0434) -- (0.1200, 0.2040, 3.0371) -- cycle;
\fill[blue!15.3, opacity=0.7] (0.1200, 0.2040, 3.0371) -- (0.1660, 0.2040, 3.0434) -- (0.1660, 0.2580, 3.0493) -- (0.1200, 0.2580, 3.0430) -- cycle;
\fill[blue!15.7, opacity=0.7] (0.1200, 0.2580, 3.0430) -- (0.1660, 0.2580, 3.0493) -- (0.1660, 0.3120, 3.0551) -- (0.1200, 0.3120, 3.0488) -- cycle;
\fill[blue!18.0, opacity=0.7] (0.1200, 0.3120, 3.0488) -- (0.1660, 0.3120, 3.0551) -- (0.1660, 0.3660, 3.0608) -- (0.1200, 0.3660, 3.0545) -- cycle;
\fill[blue!25.7, opacity=0.7] (0.1200, 0.3660, 3.0545) -- (0.1660, 0.3660, 3.0608) -- (0.1660, 0.4200, 3.0663) -- (0.1200, 0.4200, 3.0600) -- cycle;
\fill[blue!39.8, opacity=0.7] (0.1200, 0.4200, 3.0600) -- (0.1660, 0.4200, 3.0663) -- (0.1660, 0.4740, 3.0716) -- (0.1200, 0.4740, 3.0654) -- cycle;
\fill[blue!52.2, opacity=0.7] (0.1200, 0.4740, 3.0654) -- (0.1660, 0.4740, 3.0716) -- (0.1660, 0.5280, 3.0768) -- (0.1200, 0.5280, 3.0705) -- cycle;
\fill[blue!57.6, opacity=0.7] (0.1200, 0.5280, 3.0705) -- (0.1660, 0.5280, 3.0768) -- (0.1660, 0.5820, 3.0818) -- (0.1200, 0.5820, 3.0755) -- cycle;
\fill[blue!57.6, opacity=0.7] (0.1200, 0.5820, 3.0755) -- (0.1660, 0.5820, 3.0818) -- (0.1660, 0.6360, 3.0866) -- (0.1200, 0.6360, 3.0803) -- cycle;
\fill[blue!53.0, opacity=0.7] (0.1200, 0.6360, 3.0803) -- (0.1660, 0.6360, 3.0866) -- (0.1660, 0.6900, 3.0911) -- (0.1200, 0.6900, 3.0849) -- cycle;
\fill[blue!43.8, opacity=0.7] (0.1200, 0.6900, 3.0849) -- (0.1660, 0.6900, 3.0911) -- (0.1660, 0.7440, 3.0955) -- (0.1200, 0.7440, 3.0892) -- cycle;
\fill[blue!33.3, opacity=0.7] (0.1200, 0.7440, 3.0892) -- (0.1660, 0.7440, 3.0955) -- (0.1660, 0.7980, 3.0995) -- (0.1200, 0.7980, 3.0933) -- cycle;
\fill[blue!25.2, opacity=0.7] (0.1200, 0.7980, 3.0933) -- (0.1660, 0.7980, 3.0995) -- (0.1660, 0.8520, 3.1034) -- (0.1200, 0.8520, 3.0971) -- cycle;
\fill[blue!20.6, opacity=0.7] (0.1200, 0.8520, 3.0971) -- (0.1660, 0.8520, 3.1034) -- (0.1660, 0.9060, 3.1069) -- (0.1200, 0.9060, 3.1006) -- cycle;
\fill[blue!18.4, opacity=0.7] (0.1200, 0.9060, 3.1006) -- (0.1660, 0.9060, 3.1069) -- (0.1660, 0.9600, 3.1102) -- (0.1200, 0.9600, 3.1039) -- cycle;
\fill[blue!17.6, opacity=0.7] (0.1200, 0.9600, 3.1039) -- (0.1660, 0.9600, 3.1102) -- (0.1660, 1.0140, 3.1132) -- (0.1200, 1.0140, 3.1069) -- cycle;
\fill[blue!17.5, opacity=0.7] (0.1200, 1.0140, 3.1069) -- (0.1660, 1.0140, 3.1132) -- (0.1660, 1.0680, 3.1159) -- (0.1200, 1.0680, 3.1096) -- cycle;
\fill[blue!17.9, opacity=0.7] (0.1200, 1.0680, 3.1096) -- (0.1660, 1.0680, 3.1159) -- (0.1660, 1.1220, 3.1183) -- (0.1200, 1.1220, 3.1120) -- cycle;
\fill[blue!18.8, opacity=0.7] (0.1200, 1.1220, 3.1120) -- (0.1660, 1.1220, 3.1183) -- (0.1660, 1.1760, 3.1204) -- (0.1200, 1.1760, 3.1141) -- cycle;
\fill[blue!20.5, opacity=0.7] (0.1200, 1.1760, 3.1141) -- (0.1660, 1.1760, 3.1204) -- (0.1660, 1.2300, 3.1222) -- (0.1200, 1.2300, 3.1159) -- cycle;
\fill[blue!22.8, opacity=0.7] (0.1200, 1.2300, 3.1159) -- (0.1660, 1.2300, 3.1222) -- (0.1660, 1.2840, 3.1237) -- (0.1200, 1.2840, 3.1174) -- cycle;
\fill[blue!25.8, opacity=0.7] (0.1200, 1.2840, 3.1174) -- (0.1660, 1.2840, 3.1237) -- (0.1660, 1.3380, 3.1248) -- (0.1200, 1.3380, 3.1185) -- cycle;
\fill[blue!29.1, opacity=0.7] (0.1200, 1.3380, 3.1185) -- (0.1660, 1.3380, 3.1248) -- (0.1660, 1.3920, 3.1256) -- (0.1200, 1.3920, 3.1193) -- cycle;
\fill[blue!32.4, opacity=0.7] (0.1200, 1.3920, 3.1193) -- (0.1660, 1.3920, 3.1256) -- (0.1660, 1.4460, 3.1261) -- (0.1200, 1.4460, 3.1198) -- cycle;
\fill[blue!35.3, opacity=0.7] (0.1200, 1.4460, 3.1198) -- (0.1660, 1.4460, 3.1261) -- (0.1660, 1.5000, 3.1263) -- (0.1200, 1.5000, 3.1200) -- cycle;
\fill[blue!37.5, opacity=0.7] (0.1200, 1.5000, 3.1200) -- (0.1660, 1.5000, 3.1263) -- (0.1660, 1.5540, 3.1261) -- (0.1200, 1.5540, 3.1198) -- cycle;
\fill[blue!38.8, opacity=0.7] (0.1200, 1.5540, 3.1198) -- (0.1660, 1.5540, 3.1261) -- (0.1660, 1.6080, 3.1256) -- (0.1200, 1.6080, 3.1193) -- cycle;
\fill[blue!39.1, opacity=0.7] (0.1200, 1.6080, 3.1193) -- (0.1660, 1.6080, 3.1256) -- (0.1660, 1.6620, 3.1248) -- (0.1200, 1.6620, 3.1185) -- cycle;
\fill[blue!38.5, opacity=0.7] (0.1200, 1.6620, 3.1185) -- (0.1660, 1.6620, 3.1248) -- (0.1660, 1.7160, 3.1237) -- (0.1200, 1.7160, 3.1174) -- cycle;
\fill[blue!36.9, opacity=0.7] (0.1200, 1.7160, 3.1174) -- (0.1660, 1.7160, 3.1237) -- (0.1660, 1.7700, 3.1222) -- (0.1200, 1.7700, 3.1159) -- cycle;
\fill[blue!34.5, opacity=0.7] (0.1200, 1.7700, 3.1159) -- (0.1660, 1.7700, 3.1222) -- (0.1660, 1.8240, 3.1204) -- (0.1200, 1.8240, 3.1141) -- cycle;
\fill[blue!31.4, opacity=0.7] (0.1200, 1.8240, 3.1141) -- (0.1660, 1.8240, 3.1204) -- (0.1660, 1.8780, 3.1183) -- (0.1200, 1.8780, 3.1120) -- cycle;
\fill[blue!28.0, opacity=0.7] (0.1200, 1.8780, 3.1120) -- (0.1660, 1.8780, 3.1183) -- (0.1660, 1.9320, 3.1159) -- (0.1200, 1.9320, 3.1096) -- cycle;
\fill[blue!24.7, opacity=0.7] (0.1200, 1.9320, 3.1096) -- (0.1660, 1.9320, 3.1159) -- (0.1660, 1.9860, 3.1132) -- (0.1200, 1.9860, 3.1069) -- cycle;
\fill[blue!21.7, opacity=0.7] (0.1200, 1.9860, 3.1069) -- (0.1660, 1.9860, 3.1132) -- (0.1660, 2.0400, 3.1102) -- (0.1200, 2.0400, 3.1039) -- cycle;
\fill[blue!19.4, opacity=0.7] (0.1200, 2.0400, 3.1039) -- (0.1660, 2.0400, 3.1102) -- (0.1660, 2.0940, 3.1069) -- (0.1200, 2.0940, 3.1006) -- cycle;
\fill[blue!17.9, opacity=0.7] (0.1200, 2.0940, 3.1006) -- (0.1660, 2.0940, 3.1069) -- (0.1660, 2.1480, 3.1034) -- (0.1200, 2.1480, 3.0971) -- cycle;
\fill[blue!17.0, opacity=0.7] (0.1200, 2.1480, 3.0971) -- (0.1660, 2.1480, 3.1034) -- (0.1660, 2.2020, 3.0995) -- (0.1200, 2.2020, 3.0933) -- cycle;
\fill[blue!16.5, opacity=0.7] (0.1200, 2.2020, 3.0933) -- (0.1660, 2.2020, 3.0995) -- (0.1660, 2.2560, 3.0955) -- (0.1200, 2.2560, 3.0892) -- cycle;
\fill[blue!16.4, opacity=0.7] (0.1200, 2.2560, 3.0892) -- (0.1660, 2.2560, 3.0955) -- (0.1660, 2.3100, 3.0911) -- (0.1200, 2.3100, 3.0849) -- cycle;
\fill[blue!16.6, opacity=0.7] (0.1200, 2.3100, 3.0849) -- (0.1660, 2.3100, 3.0911) -- (0.1660, 2.3640, 3.0866) -- (0.1200, 2.3640, 3.0803) -- cycle;
\fill[blue!17.3, opacity=0.7] (0.1200, 2.3640, 3.0803) -- (0.1660, 2.3640, 3.0866) -- (0.1660, 2.4180, 3.0818) -- (0.1200, 2.4180, 3.0755) -- cycle;
\fill[blue!19.1, opacity=0.7] (0.1200, 2.4180, 3.0755) -- (0.1660, 2.4180, 3.0818) -- (0.1660, 2.4720, 3.0768) -- (0.1200, 2.4720, 3.0705) -- cycle;
\fill[blue!22.9, opacity=0.7] (0.1200, 2.4720, 3.0705) -- (0.1660, 2.4720, 3.0768) -- (0.1660, 2.5260, 3.0716) -- (0.1200, 2.5260, 3.0654) -- cycle;
\fill[blue!29.6, opacity=0.7] (0.1200, 2.5260, 3.0654) -- (0.1660, 2.5260, 3.0716) -- (0.1660, 2.5800, 3.0663) -- (0.1200, 2.5800, 3.0600) -- cycle;
\fill[blue!38.2, opacity=0.7] (0.1200, 2.5800, 3.0600) -- (0.1660, 2.5800, 3.0663) -- (0.1660, 2.6340, 3.0608) -- (0.1200, 2.6340, 3.0545) -- cycle;
\fill[blue!45.6, opacity=0.7] (0.1200, 2.6340, 3.0545) -- (0.1660, 2.6340, 3.0608) -- (0.1660, 2.6880, 3.0551) -- (0.1200, 2.6880, 3.0488) -- cycle;
\fill[blue!48.7, opacity=0.7] (0.1200, 2.6880, 3.0488) -- (0.1660, 2.6880, 3.0551) -- (0.1660, 2.7420, 3.0493) -- (0.1200, 2.7420, 3.0430) -- cycle;
\fill[blue!46.1, opacity=0.7] (0.1200, 2.7420, 3.0430) -- (0.1660, 2.7420, 3.0493) -- (0.1660, 2.7960, 3.0434) -- (0.1200, 2.7960, 3.0371) -- cycle;
\fill[blue!37.5, opacity=0.7] (0.1200, 2.7960, 3.0371) -- (0.1660, 2.7960, 3.0434) -- (0.1660, 2.8500, 3.0373) -- (0.1200, 2.8500, 3.0311) -- cycle;
\fill[blue!26.1, opacity=0.7] (0.1200, 2.8500, 3.0311) -- (0.1660, 2.8500, 3.0373) -- (0.1660, 2.9040, 3.0312) -- (0.1200, 2.9040, 3.0249) -- cycle;
\fill[blue!18.3, opacity=0.7] (0.1200, 2.9040, 3.0249) -- (0.1660, 2.9040, 3.0312) -- (0.1660, 2.9580, 3.0251) -- (0.1200, 2.9580, 3.0188) -- cycle;
\fill[blue!15.7, opacity=0.7] (0.1200, 2.9580, 3.0188) -- (0.1660, 2.9580, 3.0251) -- (0.1660, 3.0120, 3.0188) -- (0.1200, 3.0120, 3.0125) -- cycle;
\fill[blue!15.1, opacity=0.7] (0.1200, 3.0120, 3.0125) -- (0.1660, 3.0120, 3.0188) -- (0.1660, 3.0660, 3.0126) -- (0.1200, 3.0660, 3.0063) -- cycle;
\fill[blue!15.0, opacity=0.7] (0.1200, 3.0660, 3.0063) -- (0.1660, 3.0660, 3.0126) -- (0.1660, 3.1200, 3.0063) -- (0.1200, 3.1200, 3.0000) -- cycle;
\fill[blue!30.4, opacity=0.7] (0.1660, -0.1200, 3.0063) -- (0.2120, -0.1200, 3.0125) -- (0.2120, -0.0660, 3.0188) -- (0.1660, -0.0660, 3.0126) -- cycle;
\fill[blue!22.2, opacity=0.7] (0.1660, -0.0660, 3.0126) -- (0.2120, -0.0660, 3.0188) -- (0.2120, -0.0120, 3.0251) -- (0.1660, -0.0120, 3.0188) -- cycle;
\fill[blue!17.0, opacity=0.7] (0.1660, -0.0120, 3.0188) -- (0.2120, -0.0120, 3.0251) -- (0.2120, 0.0420, 3.0313) -- (0.1660, 0.0420, 3.0251) -- cycle;
\fill[blue!15.5, opacity=0.7] (0.1660, 0.0420, 3.0251) -- (0.2120, 0.0420, 3.0313) -- (0.2120, 0.0960, 3.0375) -- (0.1660, 0.0960, 3.0312) -- cycle;
\fill[blue!15.2, opacity=0.7] (0.1660, 0.0960, 3.0312) -- (0.2120, 0.0960, 3.0375) -- (0.2120, 0.1500, 3.0436) -- (0.1660, 0.1500, 3.0373) -- cycle;
\fill[blue!15.3, opacity=0.7] (0.1660, 0.1500, 3.0373) -- (0.2120, 0.1500, 3.0436) -- (0.2120, 0.2040, 3.0496) -- (0.1660, 0.2040, 3.0434) -- cycle;
\fill[blue!15.8, opacity=0.7] (0.1660, 0.2040, 3.0434) -- (0.2120, 0.2040, 3.0496) -- (0.2120, 0.2580, 3.0555) -- (0.1660, 0.2580, 3.0493) -- cycle;
\fill[blue!18.7, opacity=0.7] (0.1660, 0.2580, 3.0493) -- (0.2120, 0.2580, 3.0555) -- (0.2120, 0.3120, 3.0614) -- (0.1660, 0.3120, 3.0551) -- cycle;
\fill[blue!28.3, opacity=0.7] (0.1660, 0.3120, 3.0551) -- (0.2120, 0.3120, 3.0614) -- (0.2120, 0.3660, 3.0670) -- (0.1660, 0.3660, 3.0608) -- cycle;
\fill[blue!43.8, opacity=0.7] (0.1660, 0.3660, 3.0608) -- (0.2120, 0.3660, 3.0670) -- (0.2120, 0.4200, 3.0725) -- (0.1660, 0.4200, 3.0663) -- cycle;
\fill[blue!54.9, opacity=0.7] (0.1660, 0.4200, 3.0663) -- (0.2120, 0.4200, 3.0725) -- (0.2120, 0.4740, 3.0779) -- (0.1660, 0.4740, 3.0716) -- cycle;
\fill[blue!58.5, opacity=0.7] (0.1660, 0.4740, 3.0716) -- (0.2120, 0.4740, 3.0779) -- (0.2120, 0.5280, 3.0831) -- (0.1660, 0.5280, 3.0768) -- cycle;
\fill[blue!56.5, opacity=0.7] (0.1660, 0.5280, 3.0768) -- (0.2120, 0.5280, 3.0831) -- (0.2120, 0.5820, 3.0881) -- (0.1660, 0.5820, 3.0818) -- cycle;
\fill[blue!49.0, opacity=0.7] (0.1660, 0.5820, 3.0818) -- (0.2120, 0.5820, 3.0881) -- (0.2120, 0.6360, 3.0928) -- (0.1660, 0.6360, 3.0866) -- cycle;
\fill[blue!37.6, opacity=0.7] (0.1660, 0.6360, 3.0866) -- (0.2120, 0.6360, 3.0928) -- (0.2120, 0.6900, 3.0974) -- (0.1660, 0.6900, 3.0911) -- cycle;
\fill[blue!27.3, opacity=0.7] (0.1660, 0.6900, 3.0911) -- (0.2120, 0.6900, 3.0974) -- (0.2120, 0.7440, 3.1017) -- (0.1660, 0.7440, 3.0955) -- cycle;
\fill[blue!21.3, opacity=0.7] (0.1660, 0.7440, 3.0955) -- (0.2120, 0.7440, 3.1017) -- (0.2120, 0.7980, 3.1058) -- (0.1660, 0.7980, 3.0995) -- cycle;
\fill[blue!18.6, opacity=0.7] (0.1660, 0.7980, 3.0995) -- (0.2120, 0.7980, 3.1058) -- (0.2120, 0.8520, 3.1096) -- (0.1660, 0.8520, 3.1034) -- cycle;
\fill[blue!17.7, opacity=0.7] (0.1660, 0.8520, 3.1034) -- (0.2120, 0.8520, 3.1096) -- (0.2120, 0.9060, 3.1132) -- (0.1660, 0.9060, 3.1069) -- cycle;
\fill[blue!17.7, opacity=0.7] (0.1660, 0.9060, 3.1069) -- (0.2120, 0.9060, 3.1132) -- (0.2120, 0.9600, 3.1165) -- (0.1660, 0.9600, 3.1102) -- cycle;
\fill[blue!18.7, opacity=0.7] (0.1660, 0.9600, 3.1102) -- (0.2120, 0.9600, 3.1165) -- (0.2120, 1.0140, 3.1195) -- (0.1660, 1.0140, 3.1132) -- cycle;
\fill[blue!20.7, opacity=0.7] (0.1660, 1.0140, 3.1132) -- (0.2120, 1.0140, 3.1195) -- (0.2120, 1.0680, 3.1222) -- (0.1660, 1.0680, 3.1159) -- cycle;
\fill[blue!24.5, opacity=0.7] (0.1660, 1.0680, 3.1159) -- (0.2120, 1.0680, 3.1222) -- (0.2120, 1.1220, 3.1246) -- (0.1660, 1.1220, 3.1183) -- cycle;
\fill[blue!30.2, opacity=0.7] (0.1660, 1.1220, 3.1183) -- (0.2120, 1.1220, 3.1246) -- (0.2120, 1.1760, 3.1267) -- (0.1660, 1.1760, 3.1204) -- cycle;
\fill[blue!37.2, opacity=0.7] (0.1660, 1.1760, 3.1204) -- (0.2120, 1.1760, 3.1267) -- (0.2120, 1.2300, 3.1285) -- (0.1660, 1.2300, 3.1222) -- cycle;
\fill[blue!44.5, opacity=0.7] (0.1660, 1.2300, 3.1222) -- (0.2120, 1.2300, 3.1285) -- (0.2120, 1.2840, 3.1299) -- (0.1660, 1.2840, 3.1237) -- cycle;
\fill[blue!50.8, opacity=0.7] (0.1660, 1.2840, 3.1237) -- (0.2120, 1.2840, 3.1299) -- (0.2120, 1.3380, 3.1311) -- (0.1660, 1.3380, 3.1248) -- cycle;
\fill[blue!55.6, opacity=0.7] (0.1660, 1.3380, 3.1248) -- (0.2120, 1.3380, 3.1311) -- (0.2120, 1.3920, 3.1319) -- (0.1660, 1.3920, 3.1256) -- cycle;
\fill[blue!58.7, opacity=0.7] (0.1660, 1.3920, 3.1256) -- (0.2120, 1.3920, 3.1319) -- (0.2120, 1.4460, 3.1324) -- (0.1660, 1.4460, 3.1261) -- cycle;
\fill[blue!60.5, opacity=0.7] (0.1660, 1.4460, 3.1261) -- (0.2120, 1.4460, 3.1324) -- (0.2120, 1.5000, 3.1325) -- (0.1660, 1.5000, 3.1263) -- cycle;
\fill[blue!61.6, opacity=0.7] (0.1660, 1.5000, 3.1263) -- (0.2120, 1.5000, 3.1325) -- (0.2120, 1.5540, 3.1324) -- (0.1660, 1.5540, 3.1261) -- cycle;
\fill[blue!62.0, opacity=0.7] (0.1660, 1.5540, 3.1261) -- (0.2120, 1.5540, 3.1324) -- (0.2120, 1.6080, 3.1319) -- (0.1660, 1.6080, 3.1256) -- cycle;
\fill[blue!62.1, opacity=0.7] (0.1660, 1.6080, 3.1256) -- (0.2120, 1.6080, 3.1319) -- (0.2120, 1.6620, 3.1311) -- (0.1660, 1.6620, 3.1248) -- cycle;
\fill[blue!61.9, opacity=0.7] (0.1660, 1.6620, 3.1248) -- (0.2120, 1.6620, 3.1311) -- (0.2120, 1.7160, 3.1299) -- (0.1660, 1.7160, 3.1237) -- cycle;
\fill[blue!61.2, opacity=0.7] (0.1660, 1.7160, 3.1237) -- (0.2120, 1.7160, 3.1299) -- (0.2120, 1.7700, 3.1285) -- (0.1660, 1.7700, 3.1222) -- cycle;
\fill[blue!59.9, opacity=0.7] (0.1660, 1.7700, 3.1222) -- (0.2120, 1.7700, 3.1285) -- (0.2120, 1.8240, 3.1267) -- (0.1660, 1.8240, 3.1204) -- cycle;
\fill[blue!57.7, opacity=0.7] (0.1660, 1.8240, 3.1204) -- (0.2120, 1.8240, 3.1267) -- (0.2120, 1.8780, 3.1246) -- (0.1660, 1.8780, 3.1183) -- cycle;
\fill[blue!54.2, opacity=0.7] (0.1660, 1.8780, 3.1183) -- (0.2120, 1.8780, 3.1246) -- (0.2120, 1.9320, 3.1222) -- (0.1660, 1.9320, 3.1159) -- cycle;
\fill[blue!49.2, opacity=0.7] (0.1660, 1.9320, 3.1159) -- (0.2120, 1.9320, 3.1222) -- (0.2120, 1.9860, 3.1195) -- (0.1660, 1.9860, 3.1132) -- cycle;
\fill[blue!42.6, opacity=0.7] (0.1660, 1.9860, 3.1132) -- (0.2120, 1.9860, 3.1195) -- (0.2120, 2.0400, 3.1165) -- (0.1660, 2.0400, 3.1102) -- cycle;
\fill[blue!35.3, opacity=0.7] (0.1660, 2.0400, 3.1102) -- (0.2120, 2.0400, 3.1165) -- (0.2120, 2.0940, 3.1132) -- (0.1660, 2.0940, 3.1069) -- cycle;
\fill[blue!28.4, opacity=0.7] (0.1660, 2.0940, 3.1069) -- (0.2120, 2.0940, 3.1132) -- (0.2120, 2.1480, 3.1096) -- (0.1660, 2.1480, 3.1034) -- cycle;
\fill[blue!23.0, opacity=0.7] (0.1660, 2.1480, 3.1034) -- (0.2120, 2.1480, 3.1096) -- (0.2120, 2.2020, 3.1058) -- (0.1660, 2.2020, 3.0995) -- cycle;
\fill[blue!19.5, opacity=0.7] (0.1660, 2.2020, 3.0995) -- (0.2120, 2.2020, 3.1058) -- (0.2120, 2.2560, 3.1017) -- (0.1660, 2.2560, 3.0955) -- cycle;
\fill[blue!17.5, opacity=0.7] (0.1660, 2.2560, 3.0955) -- (0.2120, 2.2560, 3.1017) -- (0.2120, 2.3100, 3.0974) -- (0.1660, 2.3100, 3.0911) -- cycle;
\fill[blue!16.6, opacity=0.7] (0.1660, 2.3100, 3.0911) -- (0.2120, 2.3100, 3.0974) -- (0.2120, 2.3640, 3.0928) -- (0.1660, 2.3640, 3.0866) -- cycle;
\fill[blue!16.3, opacity=0.7] (0.1660, 2.3640, 3.0866) -- (0.2120, 2.3640, 3.0928) -- (0.2120, 2.4180, 3.0881) -- (0.1660, 2.4180, 3.0818) -- cycle;
\fill[blue!16.4, opacity=0.7] (0.1660, 2.4180, 3.0818) -- (0.2120, 2.4180, 3.0881) -- (0.2120, 2.4720, 3.0831) -- (0.1660, 2.4720, 3.0768) -- cycle;
\fill[blue!17.2, opacity=0.7] (0.1660, 2.4720, 3.0768) -- (0.2120, 2.4720, 3.0831) -- (0.2120, 2.5260, 3.0779) -- (0.1660, 2.5260, 3.0716) -- cycle;
\fill[blue!19.2, opacity=0.7] (0.1660, 2.5260, 3.0716) -- (0.2120, 2.5260, 3.0779) -- (0.2120, 2.5800, 3.0725) -- (0.1660, 2.5800, 3.0663) -- cycle;
\fill[blue!24.0, opacity=0.7] (0.1660, 2.5800, 3.0663) -- (0.2120, 2.5800, 3.0725) -- (0.2120, 2.6340, 3.0670) -- (0.1660, 2.6340, 3.0608) -- cycle;
\fill[blue!32.1, opacity=0.7] (0.1660, 2.6340, 3.0608) -- (0.2120, 2.6340, 3.0670) -- (0.2120, 2.6880, 3.0614) -- (0.1660, 2.6880, 3.0551) -- cycle;
\fill[blue!41.3, opacity=0.7] (0.1660, 2.6880, 3.0551) -- (0.2120, 2.6880, 3.0614) -- (0.2120, 2.7420, 3.0555) -- (0.1660, 2.7420, 3.0493) -- cycle;
\fill[blue!47.4, opacity=0.7] (0.1660, 2.7420, 3.0493) -- (0.2120, 2.7420, 3.0555) -- (0.2120, 2.7960, 3.0496) -- (0.1660, 2.7960, 3.0434) -- cycle;
\fill[blue!47.5, opacity=0.7] (0.1660, 2.7960, 3.0434) -- (0.2120, 2.7960, 3.0496) -- (0.2120, 2.8500, 3.0436) -- (0.1660, 2.8500, 3.0373) -- cycle;
\fill[blue!40.8, opacity=0.7] (0.1660, 2.8500, 3.0373) -- (0.2120, 2.8500, 3.0436) -- (0.2120, 2.9040, 3.0375) -- (0.1660, 2.9040, 3.0312) -- cycle;
\fill[blue!29.0, opacity=0.7] (0.1660, 2.9040, 3.0312) -- (0.2120, 2.9040, 3.0375) -- (0.2120, 2.9580, 3.0313) -- (0.1660, 2.9580, 3.0251) -- cycle;
\fill[blue!19.5, opacity=0.7] (0.1660, 2.9580, 3.0251) -- (0.2120, 2.9580, 3.0313) -- (0.2120, 3.0120, 3.0251) -- (0.1660, 3.0120, 3.0188) -- cycle;
\fill[blue!15.9, opacity=0.7] (0.1660, 3.0120, 3.0188) -- (0.2120, 3.0120, 3.0251) -- (0.2120, 3.0660, 3.0188) -- (0.1660, 3.0660, 3.0126) -- cycle;
\fill[blue!15.2, opacity=0.7] (0.1660, 3.0660, 3.0126) -- (0.2120, 3.0660, 3.0188) -- (0.2120, 3.1200, 3.0125) -- (0.1660, 3.1200, 3.0063) -- cycle;
\fill[blue!23.6, opacity=0.7] (0.2120, -0.1200, 3.0125) -- (0.2580, -0.1200, 3.0188) -- (0.2580, -0.0660, 3.0251) -- (0.2120, -0.0660, 3.0188) -- cycle;
\fill[blue!17.5, opacity=0.7] (0.2120, -0.0660, 3.0188) -- (0.2580, -0.0660, 3.0251) -- (0.2580, -0.0120, 3.0313) -- (0.2120, -0.0120, 3.0251) -- cycle;
\fill[blue!15.6, opacity=0.7] (0.2120, -0.0120, 3.0251) -- (0.2580, -0.0120, 3.0313) -- (0.2580, 0.0420, 3.0375) -- (0.2120, 0.0420, 3.0313) -- cycle;
\fill[blue!15.2, opacity=0.7] (0.2120, 0.0420, 3.0313) -- (0.2580, 0.0420, 3.0375) -- (0.2580, 0.0960, 3.0437) -- (0.2120, 0.0960, 3.0375) -- cycle;
\fill[blue!15.3, opacity=0.7] (0.2120, 0.0960, 3.0375) -- (0.2580, 0.0960, 3.0437) -- (0.2580, 0.1500, 3.0498) -- (0.2120, 0.1500, 3.0436) -- cycle;
\fill[blue!15.9, opacity=0.7] (0.2120, 0.1500, 3.0436) -- (0.2580, 0.1500, 3.0498) -- (0.2580, 0.2040, 3.0559) -- (0.2120, 0.2040, 3.0496) -- cycle;
\fill[blue!19.1, opacity=0.7] (0.2120, 0.2040, 3.0496) -- (0.2580, 0.2040, 3.0559) -- (0.2580, 0.2580, 3.0618) -- (0.2120, 0.2580, 3.0555) -- cycle;
\fill[blue!30.0, opacity=0.7] (0.2120, 0.2580, 3.0555) -- (0.2580, 0.2580, 3.0618) -- (0.2580, 0.3120, 3.0676) -- (0.2120, 0.3120, 3.0614) -- cycle;
\fill[blue!46.3, opacity=0.7] (0.2120, 0.3120, 3.0614) -- (0.2580, 0.3120, 3.0676) -- (0.2580, 0.3660, 3.0733) -- (0.2120, 0.3660, 3.0670) -- cycle;
\fill[blue!56.5, opacity=0.7] (0.2120, 0.3660, 3.0670) -- (0.2580, 0.3660, 3.0733) -- (0.2580, 0.4200, 3.0788) -- (0.2120, 0.4200, 3.0725) -- cycle;
\fill[blue!58.7, opacity=0.7] (0.2120, 0.4200, 3.0725) -- (0.2580, 0.4200, 3.0788) -- (0.2580, 0.4740, 3.0841) -- (0.2120, 0.4740, 3.0779) -- cycle;
\fill[blue!55.1, opacity=0.7] (0.2120, 0.4740, 3.0779) -- (0.2580, 0.4740, 3.0841) -- (0.2580, 0.5280, 3.0893) -- (0.2120, 0.5280, 3.0831) -- cycle;
\fill[blue!45.2, opacity=0.7] (0.2120, 0.5280, 3.0831) -- (0.2580, 0.5280, 3.0893) -- (0.2580, 0.5820, 3.0943) -- (0.2120, 0.5820, 3.0881) -- cycle;
\fill[blue!32.7, opacity=0.7] (0.2120, 0.5820, 3.0881) -- (0.2580, 0.5820, 3.0943) -- (0.2580, 0.6360, 3.0991) -- (0.2120, 0.6360, 3.0928) -- cycle;
\fill[blue!23.7, opacity=0.7] (0.2120, 0.6360, 3.0928) -- (0.2580, 0.6360, 3.0991) -- (0.2580, 0.6900, 3.1036) -- (0.2120, 0.6900, 3.0974) -- cycle;
\fill[blue!19.4, opacity=0.7] (0.2120, 0.6900, 3.0974) -- (0.2580, 0.6900, 3.1036) -- (0.2580, 0.7440, 3.1079) -- (0.2120, 0.7440, 3.1017) -- cycle;
\fill[blue!17.9, opacity=0.7] (0.2120, 0.7440, 3.1017) -- (0.2580, 0.7440, 3.1079) -- (0.2580, 0.7980, 3.1120) -- (0.2120, 0.7980, 3.1058) -- cycle;
\fill[blue!17.9, opacity=0.7] (0.2120, 0.7980, 3.1058) -- (0.2580, 0.7980, 3.1120) -- (0.2580, 0.8520, 3.1159) -- (0.2120, 0.8520, 3.1096) -- cycle;
\fill[blue!19.0, opacity=0.7] (0.2120, 0.8520, 3.1096) -- (0.2580, 0.8520, 3.1159) -- (0.2580, 0.9060, 3.1194) -- (0.2120, 0.9060, 3.1132) -- cycle;
\fill[blue!22.0, opacity=0.7] (0.2120, 0.9060, 3.1132) -- (0.2580, 0.9060, 3.1194) -- (0.2580, 0.9600, 3.1227) -- (0.2120, 0.9600, 3.1165) -- cycle;
\fill[blue!27.8, opacity=0.7] (0.2120, 0.9600, 3.1165) -- (0.2580, 0.9600, 3.1227) -- (0.2580, 1.0140, 3.1257) -- (0.2120, 1.0140, 3.1195) -- cycle;
\fill[blue!36.8, opacity=0.7] (0.2120, 1.0140, 3.1195) -- (0.2580, 1.0140, 3.1257) -- (0.2580, 1.0680, 3.1284) -- (0.2120, 1.0680, 3.1222) -- cycle;
\fill[blue!47.2, opacity=0.7] (0.2120, 1.0680, 3.1222) -- (0.2580, 1.0680, 3.1284) -- (0.2580, 1.1220, 3.1308) -- (0.2120, 1.1220, 3.1246) -- cycle;
\fill[blue!55.9, opacity=0.7] (0.2120, 1.1220, 3.1246) -- (0.2580, 1.1220, 3.1308) -- (0.2580, 1.1760, 3.1329) -- (0.2120, 1.1760, 3.1267) -- cycle;
\fill[blue!61.2, opacity=0.7] (0.2120, 1.1760, 3.1267) -- (0.2580, 1.1760, 3.1329) -- (0.2580, 1.2300, 3.1347) -- (0.2120, 1.2300, 3.1285) -- cycle;
\fill[blue!63.3, opacity=0.7] (0.2120, 1.2300, 3.1285) -- (0.2580, 1.2300, 3.1347) -- (0.2580, 1.2840, 3.1361) -- (0.2120, 1.2840, 3.1299) -- cycle;
\fill[blue!63.5, opacity=0.7] (0.2120, 1.2840, 3.1299) -- (0.2580, 1.2840, 3.1361) -- (0.2580, 1.3380, 3.1373) -- (0.2120, 1.3380, 3.1311) -- cycle;
\fill[blue!63.0, opacity=0.7] (0.2120, 1.3380, 3.1311) -- (0.2580, 1.3380, 3.1373) -- (0.2580, 1.3920, 3.1381) -- (0.2120, 1.3920, 3.1319) -- cycle;
\fill[blue!62.4, opacity=0.7] (0.2120, 1.3920, 3.1319) -- (0.2580, 1.3920, 3.1381) -- (0.2580, 1.4460, 3.1386) -- (0.2120, 1.4460, 3.1324) -- cycle;
\fill[blue!62.0, opacity=0.7] (0.2120, 1.4460, 3.1324) -- (0.2580, 1.4460, 3.1386) -- (0.2580, 1.5000, 3.1388) -- (0.2120, 1.5000, 3.1325) -- cycle;
\fill[blue!61.7, opacity=0.7] (0.2120, 1.5000, 3.1325) -- (0.2580, 1.5000, 3.1388) -- (0.2580, 1.5540, 3.1386) -- (0.2120, 1.5540, 3.1324) -- cycle;
\fill[blue!61.7, opacity=0.7] (0.2120, 1.5540, 3.1324) -- (0.2580, 1.5540, 3.1386) -- (0.2580, 1.6080, 3.1381) -- (0.2120, 1.6080, 3.1319) -- cycle;
\fill[blue!61.8, opacity=0.7] (0.2120, 1.6080, 3.1319) -- (0.2580, 1.6080, 3.1381) -- (0.2580, 1.6620, 3.1373) -- (0.2120, 1.6620, 3.1311) -- cycle;
\fill[blue!62.0, opacity=0.7] (0.2120, 1.6620, 3.1311) -- (0.2580, 1.6620, 3.1373) -- (0.2580, 1.7160, 3.1361) -- (0.2120, 1.7160, 3.1299) -- cycle;
\fill[blue!62.2, opacity=0.7] (0.2120, 1.7160, 3.1299) -- (0.2580, 1.7160, 3.1361) -- (0.2580, 1.7700, 3.1347) -- (0.2120, 1.7700, 3.1285) -- cycle;
\fill[blue!62.6, opacity=0.7] (0.2120, 1.7700, 3.1285) -- (0.2580, 1.7700, 3.1347) -- (0.2580, 1.8240, 3.1329) -- (0.2120, 1.8240, 3.1267) -- cycle;
\fill[blue!63.0, opacity=0.7] (0.2120, 1.8240, 3.1267) -- (0.2580, 1.8240, 3.1329) -- (0.2580, 1.8780, 3.1308) -- (0.2120, 1.8780, 3.1246) -- cycle;
\fill[blue!63.4, opacity=0.7] (0.2120, 1.8780, 3.1246) -- (0.2580, 1.8780, 3.1308) -- (0.2580, 1.9320, 3.1284) -- (0.2120, 1.9320, 3.1222) -- cycle;
\fill[blue!63.5, opacity=0.7] (0.2120, 1.9320, 3.1222) -- (0.2580, 1.9320, 3.1284) -- (0.2580, 1.9860, 3.1257) -- (0.2120, 1.9860, 3.1195) -- cycle;
\fill[blue!62.6, opacity=0.7] (0.2120, 1.9860, 3.1195) -- (0.2580, 1.9860, 3.1257) -- (0.2580, 2.0400, 3.1227) -- (0.2120, 2.0400, 3.1165) -- cycle;
\fill[blue!59.7, opacity=0.7] (0.2120, 2.0400, 3.1165) -- (0.2580, 2.0400, 3.1227) -- (0.2580, 2.0940, 3.1194) -- (0.2120, 2.0940, 3.1132) -- cycle;
\fill[blue!53.9, opacity=0.7] (0.2120, 2.0940, 3.1132) -- (0.2580, 2.0940, 3.1194) -- (0.2580, 2.1480, 3.1159) -- (0.2120, 2.1480, 3.1096) -- cycle;
\fill[blue!45.0, opacity=0.7] (0.2120, 2.1480, 3.1096) -- (0.2580, 2.1480, 3.1159) -- (0.2580, 2.2020, 3.1120) -- (0.2120, 2.2020, 3.1058) -- cycle;
\fill[blue!34.8, opacity=0.7] (0.2120, 2.2020, 3.1058) -- (0.2580, 2.2020, 3.1120) -- (0.2580, 2.2560, 3.1079) -- (0.2120, 2.2560, 3.1017) -- cycle;
\fill[blue!26.1, opacity=0.7] (0.2120, 2.2560, 3.1017) -- (0.2580, 2.2560, 3.1079) -- (0.2580, 2.3100, 3.1036) -- (0.2120, 2.3100, 3.0974) -- cycle;
\fill[blue!20.5, opacity=0.7] (0.2120, 2.3100, 3.0974) -- (0.2580, 2.3100, 3.1036) -- (0.2580, 2.3640, 3.0991) -- (0.2120, 2.3640, 3.0928) -- cycle;
\fill[blue!17.7, opacity=0.7] (0.2120, 2.3640, 3.0928) -- (0.2580, 2.3640, 3.0991) -- (0.2580, 2.4180, 3.0943) -- (0.2120, 2.4180, 3.0881) -- cycle;
\fill[blue!16.5, opacity=0.7] (0.2120, 2.4180, 3.0881) -- (0.2580, 2.4180, 3.0943) -- (0.2580, 2.4720, 3.0893) -- (0.2120, 2.4720, 3.0831) -- cycle;
\fill[blue!16.2, opacity=0.7] (0.2120, 2.4720, 3.0831) -- (0.2580, 2.4720, 3.0893) -- (0.2580, 2.5260, 3.0841) -- (0.2120, 2.5260, 3.0779) -- cycle;
\fill[blue!16.4, opacity=0.7] (0.2120, 2.5260, 3.0779) -- (0.2580, 2.5260, 3.0841) -- (0.2580, 2.5800, 3.0788) -- (0.2120, 2.5800, 3.0725) -- cycle;
\fill[blue!17.5, opacity=0.7] (0.2120, 2.5800, 3.0725) -- (0.2580, 2.5800, 3.0788) -- (0.2580, 2.6340, 3.0733) -- (0.2120, 2.6340, 3.0670) -- cycle;
\fill[blue!20.6, opacity=0.7] (0.2120, 2.6340, 3.0670) -- (0.2580, 2.6340, 3.0733) -- (0.2580, 2.6880, 3.0676) -- (0.2120, 2.6880, 3.0614) -- cycle;
\fill[blue!27.2, opacity=0.7] (0.2120, 2.6880, 3.0614) -- (0.2580, 2.6880, 3.0676) -- (0.2580, 2.7420, 3.0618) -- (0.2120, 2.7420, 3.0555) -- cycle;
\fill[blue!37.0, opacity=0.7] (0.2120, 2.7420, 3.0555) -- (0.2580, 2.7420, 3.0618) -- (0.2580, 2.7960, 3.0559) -- (0.2120, 2.7960, 3.0496) -- cycle;
\fill[blue!45.3, opacity=0.7] (0.2120, 2.7960, 3.0496) -- (0.2580, 2.7960, 3.0559) -- (0.2580, 2.8500, 3.0498) -- (0.2120, 2.8500, 3.0436) -- cycle;
\fill[blue!47.8, opacity=0.7] (0.2120, 2.8500, 3.0436) -- (0.2580, 2.8500, 3.0498) -- (0.2580, 2.9040, 3.0437) -- (0.2120, 2.9040, 3.0375) -- cycle;
\fill[blue!42.7, opacity=0.7] (0.2120, 2.9040, 3.0375) -- (0.2580, 2.9040, 3.0437) -- (0.2580, 2.9580, 3.0375) -- (0.2120, 2.9580, 3.0313) -- cycle;
\fill[blue!31.1, opacity=0.7] (0.2120, 2.9580, 3.0313) -- (0.2580, 2.9580, 3.0375) -- (0.2580, 3.0120, 3.0313) -- (0.2120, 3.0120, 3.0251) -- cycle;
\fill[blue!20.3, opacity=0.7] (0.2120, 3.0120, 3.0251) -- (0.2580, 3.0120, 3.0313) -- (0.2580, 3.0660, 3.0251) -- (0.2120, 3.0660, 3.0188) -- cycle;
\fill[blue!16.0, opacity=0.7] (0.2120, 3.0660, 3.0188) -- (0.2580, 3.0660, 3.0251) -- (0.2580, 3.1200, 3.0188) -- (0.2120, 3.1200, 3.0125) -- cycle;
\fill[blue!18.3, opacity=0.7] (0.2580, -0.1200, 3.0188) -- (0.3040, -0.1200, 3.0249) -- (0.3040, -0.0660, 3.0312) -- (0.2580, -0.0660, 3.0251) -- cycle;
\fill[blue!15.8, opacity=0.7] (0.2580, -0.0660, 3.0251) -- (0.3040, -0.0660, 3.0312) -- (0.3040, -0.0120, 3.0375) -- (0.2580, -0.0120, 3.0313) -- cycle;
\fill[blue!15.3, opacity=0.7] (0.2580, -0.0120, 3.0313) -- (0.3040, -0.0120, 3.0375) -- (0.3040, 0.0420, 3.0437) -- (0.2580, 0.0420, 3.0375) -- cycle;
\fill[blue!15.3, opacity=0.7] (0.2580, 0.0420, 3.0375) -- (0.3040, 0.0420, 3.0437) -- (0.3040, 0.0960, 3.0499) -- (0.2580, 0.0960, 3.0437) -- cycle;
\fill[blue!15.8, opacity=0.7] (0.2580, 0.0960, 3.0437) -- (0.3040, 0.0960, 3.0499) -- (0.3040, 0.1500, 3.0560) -- (0.2580, 0.1500, 3.0498) -- cycle;
\fill[blue!19.1, opacity=0.7] (0.2580, 0.1500, 3.0498) -- (0.3040, 0.1500, 3.0560) -- (0.3040, 0.2040, 3.0620) -- (0.2580, 0.2040, 3.0559) -- cycle;
\fill[blue!30.4, opacity=0.7] (0.2580, 0.2040, 3.0559) -- (0.3040, 0.2040, 3.0620) -- (0.3040, 0.2580, 3.0680) -- (0.2580, 0.2580, 3.0618) -- cycle;
\fill[blue!47.4, opacity=0.7] (0.2580, 0.2580, 3.0618) -- (0.3040, 0.2580, 3.0680) -- (0.3040, 0.3120, 3.0738) -- (0.2580, 0.3120, 3.0676) -- cycle;
\fill[blue!57.3, opacity=0.7] (0.2580, 0.3120, 3.0676) -- (0.3040, 0.3120, 3.0738) -- (0.3040, 0.3660, 3.0794) -- (0.2580, 0.3660, 3.0733) -- cycle;
\fill[blue!58.9, opacity=0.7] (0.2580, 0.3660, 3.0733) -- (0.3040, 0.3660, 3.0794) -- (0.3040, 0.4200, 3.0849) -- (0.2580, 0.4200, 3.0788) -- cycle;
\fill[blue!53.9, opacity=0.7] (0.2580, 0.4200, 3.0788) -- (0.3040, 0.4200, 3.0849) -- (0.3040, 0.4740, 3.0903) -- (0.2580, 0.4740, 3.0841) -- cycle;
\fill[blue!42.2, opacity=0.7] (0.2580, 0.4740, 3.0841) -- (0.3040, 0.4740, 3.0903) -- (0.3040, 0.5280, 3.0955) -- (0.2580, 0.5280, 3.0893) -- cycle;
\fill[blue!29.4, opacity=0.7] (0.2580, 0.5280, 3.0893) -- (0.3040, 0.5280, 3.0955) -- (0.3040, 0.5820, 3.1005) -- (0.2580, 0.5820, 3.0943) -- cycle;
\fill[blue!21.7, opacity=0.7] (0.2580, 0.5820, 3.0943) -- (0.3040, 0.5820, 3.1005) -- (0.3040, 0.6360, 3.1052) -- (0.2580, 0.6360, 3.0991) -- cycle;
\fill[blue!18.6, opacity=0.7] (0.2580, 0.6360, 3.0991) -- (0.3040, 0.6360, 3.1052) -- (0.3040, 0.6900, 3.1098) -- (0.2580, 0.6900, 3.1036) -- cycle;
\fill[blue!17.9, opacity=0.7] (0.2580, 0.6900, 3.1036) -- (0.3040, 0.6900, 3.1098) -- (0.3040, 0.7440, 3.1141) -- (0.2580, 0.7440, 3.1079) -- cycle;
\fill[blue!18.7, opacity=0.7] (0.2580, 0.7440, 3.1079) -- (0.3040, 0.7440, 3.1141) -- (0.3040, 0.7980, 3.1182) -- (0.2580, 0.7980, 3.1120) -- cycle;
\fill[blue!21.7, opacity=0.7] (0.2580, 0.7980, 3.1120) -- (0.3040, 0.7980, 3.1182) -- (0.3040, 0.8520, 3.1220) -- (0.2580, 0.8520, 3.1159) -- cycle;
\fill[blue!28.4, opacity=0.7] (0.2580, 0.8520, 3.1159) -- (0.3040, 0.8520, 3.1220) -- (0.3040, 0.9060, 3.1256) -- (0.2580, 0.9060, 3.1194) -- cycle;
\fill[blue!39.5, opacity=0.7] (0.2580, 0.9060, 3.1194) -- (0.3040, 0.9060, 3.1256) -- (0.3040, 0.9600, 3.1289) -- (0.2580, 0.9600, 3.1227) -- cycle;
\fill[blue!52.0, opacity=0.7] (0.2580, 0.9600, 3.1227) -- (0.3040, 0.9600, 3.1289) -- (0.3040, 1.0140, 3.1319) -- (0.2580, 1.0140, 3.1257) -- cycle;
\fill[blue!60.5, opacity=0.7] (0.2580, 1.0140, 3.1257) -- (0.3040, 1.0140, 3.1319) -- (0.3040, 1.0680, 3.1346) -- (0.2580, 1.0680, 3.1284) -- cycle;
\fill[blue!63.5, opacity=0.7] (0.2580, 1.0680, 3.1284) -- (0.3040, 1.0680, 3.1346) -- (0.3040, 1.1220, 3.1370) -- (0.2580, 1.1220, 3.1308) -- cycle;
\fill[blue!62.9, opacity=0.7] (0.2580, 1.1220, 3.1308) -- (0.3040, 1.1220, 3.1370) -- (0.3040, 1.1760, 3.1391) -- (0.2580, 1.1760, 3.1329) -- cycle;
\fill[blue!61.5, opacity=0.7] (0.2580, 1.1760, 3.1329) -- (0.3040, 1.1760, 3.1391) -- (0.3040, 1.2300, 3.1409) -- (0.2580, 1.2300, 3.1347) -- cycle;
\fill[blue!60.5, opacity=0.7] (0.2580, 1.2300, 3.1347) -- (0.3040, 1.2300, 3.1409) -- (0.3040, 1.2840, 3.1423) -- (0.2580, 1.2840, 3.1361) -- cycle;
\fill[blue!60.2, opacity=0.7] (0.2580, 1.2840, 3.1361) -- (0.3040, 1.2840, 3.1423) -- (0.3040, 1.3380, 3.1435) -- (0.2580, 1.3380, 3.1373) -- cycle;
\fill[blue!60.6, opacity=0.7] (0.2580, 1.3380, 3.1373) -- (0.3040, 1.3380, 3.1435) -- (0.3040, 1.3920, 3.1443) -- (0.2580, 1.3920, 3.1381) -- cycle;
\fill[blue!61.2, opacity=0.7] (0.2580, 1.3920, 3.1381) -- (0.3040, 1.3920, 3.1443) -- (0.3040, 1.4460, 3.1448) -- (0.2580, 1.4460, 3.1386) -- cycle;
\fill[blue!61.9, opacity=0.7] (0.2580, 1.4460, 3.1386) -- (0.3040, 1.4460, 3.1448) -- (0.3040, 1.5000, 3.1449) -- (0.2580, 1.5000, 3.1388) -- cycle;
\fill[blue!62.4, opacity=0.7] (0.2580, 1.5000, 3.1388) -- (0.3040, 1.5000, 3.1449) -- (0.3040, 1.5540, 3.1448) -- (0.2580, 1.5540, 3.1386) -- cycle;
\fill[blue!62.8, opacity=0.7] (0.2580, 1.5540, 3.1386) -- (0.3040, 1.5540, 3.1448) -- (0.3040, 1.6080, 3.1443) -- (0.2580, 1.6080, 3.1381) -- cycle;
\fill[blue!62.9, opacity=0.7] (0.2580, 1.6080, 3.1381) -- (0.3040, 1.6080, 3.1443) -- (0.3040, 1.6620, 3.1435) -- (0.2580, 1.6620, 3.1373) -- cycle;
\fill[blue!63.0, opacity=0.7] (0.2580, 1.6620, 3.1373) -- (0.3040, 1.6620, 3.1435) -- (0.3040, 1.7160, 3.1423) -- (0.2580, 1.7160, 3.1361) -- cycle;
\fill[blue!62.9, opacity=0.7] (0.2580, 1.7160, 3.1361) -- (0.3040, 1.7160, 3.1423) -- (0.3040, 1.7700, 3.1409) -- (0.2580, 1.7700, 3.1347) -- cycle;
\fill[blue!62.8, opacity=0.7] (0.2580, 1.7700, 3.1347) -- (0.3040, 1.7700, 3.1409) -- (0.3040, 1.8240, 3.1391) -- (0.2580, 1.8240, 3.1329) -- cycle;
\fill[blue!62.5, opacity=0.7] (0.2580, 1.8240, 3.1329) -- (0.3040, 1.8240, 3.1391) -- (0.3040, 1.8780, 3.1370) -- (0.2580, 1.8780, 3.1308) -- cycle;
\fill[blue!62.3, opacity=0.7] (0.2580, 1.8780, 3.1308) -- (0.3040, 1.8780, 3.1370) -- (0.3040, 1.9320, 3.1346) -- (0.2580, 1.9320, 3.1284) -- cycle;
\fill[blue!62.2, opacity=0.7] (0.2580, 1.9320, 3.1284) -- (0.3040, 1.9320, 3.1346) -- (0.3040, 1.9860, 3.1319) -- (0.2580, 1.9860, 3.1257) -- cycle;
\fill[blue!62.4, opacity=0.7] (0.2580, 1.9860, 3.1257) -- (0.3040, 1.9860, 3.1319) -- (0.3040, 2.0400, 3.1289) -- (0.2580, 2.0400, 3.1227) -- cycle;
\fill[blue!63.0, opacity=0.7] (0.2580, 2.0400, 3.1227) -- (0.3040, 2.0400, 3.1289) -- (0.3040, 2.0940, 3.1256) -- (0.2580, 2.0940, 3.1194) -- cycle;
\fill[blue!63.6, opacity=0.7] (0.2580, 2.0940, 3.1194) -- (0.3040, 2.0940, 3.1256) -- (0.3040, 2.1480, 3.1220) -- (0.2580, 2.1480, 3.1159) -- cycle;
\fill[blue!62.8, opacity=0.7] (0.2580, 2.1480, 3.1159) -- (0.3040, 2.1480, 3.1220) -- (0.3040, 2.2020, 3.1182) -- (0.2580, 2.2020, 3.1120) -- cycle;
\fill[blue!58.8, opacity=0.7] (0.2580, 2.2020, 3.1120) -- (0.3040, 2.2020, 3.1182) -- (0.3040, 2.2560, 3.1141) -- (0.2580, 2.2560, 3.1079) -- cycle;
\fill[blue!49.9, opacity=0.7] (0.2580, 2.2560, 3.1079) -- (0.3040, 2.2560, 3.1141) -- (0.3040, 2.3100, 3.1098) -- (0.2580, 2.3100, 3.1036) -- cycle;
\fill[blue!37.7, opacity=0.7] (0.2580, 2.3100, 3.1036) -- (0.3040, 2.3100, 3.1098) -- (0.3040, 2.3640, 3.1052) -- (0.2580, 2.3640, 3.0991) -- cycle;
\fill[blue!26.9, opacity=0.7] (0.2580, 2.3640, 3.0991) -- (0.3040, 2.3640, 3.1052) -- (0.3040, 2.4180, 3.1005) -- (0.2580, 2.4180, 3.0943) -- cycle;
\fill[blue!20.3, opacity=0.7] (0.2580, 2.4180, 3.0943) -- (0.3040, 2.4180, 3.1005) -- (0.3040, 2.4720, 3.0955) -- (0.2580, 2.4720, 3.0893) -- cycle;
\fill[blue!17.3, opacity=0.7] (0.2580, 2.4720, 3.0893) -- (0.3040, 2.4720, 3.0955) -- (0.3040, 2.5260, 3.0903) -- (0.2580, 2.5260, 3.0841) -- cycle;
\fill[blue!16.3, opacity=0.7] (0.2580, 2.5260, 3.0841) -- (0.3040, 2.5260, 3.0903) -- (0.3040, 2.5800, 3.0849) -- (0.2580, 2.5800, 3.0788) -- cycle;
\fill[blue!16.2, opacity=0.7] (0.2580, 2.5800, 3.0788) -- (0.3040, 2.5800, 3.0849) -- (0.3040, 2.6340, 3.0794) -- (0.2580, 2.6340, 3.0733) -- cycle;
\fill[blue!16.7, opacity=0.7] (0.2580, 2.6340, 3.0733) -- (0.3040, 2.6340, 3.0794) -- (0.3040, 2.6880, 3.0738) -- (0.2580, 2.6880, 3.0676) -- cycle;
\fill[blue!18.7, opacity=0.7] (0.2580, 2.6880, 3.0676) -- (0.3040, 2.6880, 3.0738) -- (0.3040, 2.7420, 3.0680) -- (0.2580, 2.7420, 3.0618) -- cycle;
\fill[blue!23.9, opacity=0.7] (0.2580, 2.7420, 3.0618) -- (0.3040, 2.7420, 3.0680) -- (0.3040, 2.7960, 3.0620) -- (0.2580, 2.7960, 3.0559) -- cycle;
\fill[blue!33.4, opacity=0.7] (0.2580, 2.7960, 3.0559) -- (0.3040, 2.7960, 3.0620) -- (0.3040, 2.8500, 3.0560) -- (0.2580, 2.8500, 3.0498) -- cycle;
\fill[blue!43.2, opacity=0.7] (0.2580, 2.8500, 3.0498) -- (0.3040, 2.8500, 3.0560) -- (0.3040, 2.9040, 3.0499) -- (0.2580, 2.9040, 3.0437) -- cycle;
\fill[blue!47.5, opacity=0.7] (0.2580, 2.9040, 3.0437) -- (0.3040, 2.9040, 3.0499) -- (0.3040, 2.9580, 3.0437) -- (0.2580, 2.9580, 3.0375) -- cycle;
\fill[blue!43.6, opacity=0.7] (0.2580, 2.9580, 3.0375) -- (0.3040, 2.9580, 3.0437) -- (0.3040, 3.0120, 3.0375) -- (0.2580, 3.0120, 3.0313) -- cycle;
\fill[blue!32.1, opacity=0.7] (0.2580, 3.0120, 3.0313) -- (0.3040, 3.0120, 3.0375) -- (0.3040, 3.0660, 3.0312) -- (0.2580, 3.0660, 3.0251) -- cycle;
\fill[blue!20.7, opacity=0.7] (0.2580, 3.0660, 3.0251) -- (0.3040, 3.0660, 3.0312) -- (0.3040, 3.1200, 3.0249) -- (0.2580, 3.1200, 3.0188) -- cycle;
\fill[blue!16.1, opacity=0.7] (0.3040, -0.1200, 3.0249) -- (0.3500, -0.1200, 3.0311) -- (0.3500, -0.0660, 3.0373) -- (0.3040, -0.0660, 3.0312) -- cycle;
\fill[blue!15.3, opacity=0.7] (0.3040, -0.0660, 3.0312) -- (0.3500, -0.0660, 3.0373) -- (0.3500, -0.0120, 3.0436) -- (0.3040, -0.0120, 3.0375) -- cycle;
\fill[blue!15.3, opacity=0.7] (0.3040, -0.0120, 3.0375) -- (0.3500, -0.0120, 3.0436) -- (0.3500, 0.0420, 3.0498) -- (0.3040, 0.0420, 3.0437) -- cycle;
\fill[blue!15.7, opacity=0.7] (0.3040, 0.0420, 3.0437) -- (0.3500, 0.0420, 3.0498) -- (0.3500, 0.0960, 3.0560) -- (0.3040, 0.0960, 3.0499) -- cycle;
\fill[blue!18.5, opacity=0.7] (0.3040, 0.0960, 3.0499) -- (0.3500, 0.0960, 3.0560) -- (0.3500, 0.1500, 3.0621) -- (0.3040, 0.1500, 3.0560) -- cycle;
\fill[blue!29.6, opacity=0.7] (0.3040, 0.1500, 3.0560) -- (0.3500, 0.1500, 3.0621) -- (0.3500, 0.2040, 3.0681) -- (0.3040, 0.2040, 3.0620) -- cycle;
\fill[blue!47.3, opacity=0.7] (0.3040, 0.2040, 3.0620) -- (0.3500, 0.2040, 3.0681) -- (0.3500, 0.2580, 3.0741) -- (0.3040, 0.2580, 3.0680) -- cycle;
\fill[blue!57.7, opacity=0.7] (0.3040, 0.2580, 3.0680) -- (0.3500, 0.2580, 3.0741) -- (0.3500, 0.3120, 3.0799) -- (0.3040, 0.3120, 3.0738) -- cycle;
\fill[blue!59.0, opacity=0.7] (0.3040, 0.3120, 3.0738) -- (0.3500, 0.3120, 3.0799) -- (0.3500, 0.3660, 3.0855) -- (0.3040, 0.3660, 3.0794) -- cycle;
\fill[blue!53.4, opacity=0.7] (0.3040, 0.3660, 3.0794) -- (0.3500, 0.3660, 3.0855) -- (0.3500, 0.4200, 3.0911) -- (0.3040, 0.4200, 3.0849) -- cycle;
\fill[blue!40.5, opacity=0.7] (0.3040, 0.4200, 3.0849) -- (0.3500, 0.4200, 3.0911) -- (0.3500, 0.4740, 3.0964) -- (0.3040, 0.4740, 3.0903) -- cycle;
\fill[blue!27.5, opacity=0.7] (0.3040, 0.4740, 3.0903) -- (0.3500, 0.4740, 3.0964) -- (0.3500, 0.5280, 3.1016) -- (0.3040, 0.5280, 3.0955) -- cycle;
\fill[blue!20.6, opacity=0.7] (0.3040, 0.5280, 3.0955) -- (0.3500, 0.5280, 3.1016) -- (0.3500, 0.5820, 3.1066) -- (0.3040, 0.5820, 3.1005) -- cycle;
\fill[blue!18.3, opacity=0.7] (0.3040, 0.5820, 3.1005) -- (0.3500, 0.5820, 3.1066) -- (0.3500, 0.6360, 3.1114) -- (0.3040, 0.6360, 3.1052) -- cycle;
\fill[blue!18.2, opacity=0.7] (0.3040, 0.6360, 3.1052) -- (0.3500, 0.6360, 3.1114) -- (0.3500, 0.6900, 3.1159) -- (0.3040, 0.6900, 3.1098) -- cycle;
\fill[blue!20.1, opacity=0.7] (0.3040, 0.6900, 3.1098) -- (0.3500, 0.6900, 3.1159) -- (0.3500, 0.7440, 3.1202) -- (0.3040, 0.7440, 3.1141) -- cycle;
\fill[blue!25.9, opacity=0.7] (0.3040, 0.7440, 3.1141) -- (0.3500, 0.7440, 3.1202) -- (0.3500, 0.7980, 3.1243) -- (0.3040, 0.7980, 3.1182) -- cycle;
\fill[blue!37.3, opacity=0.7] (0.3040, 0.7980, 3.1182) -- (0.3500, 0.7980, 3.1243) -- (0.3500, 0.8520, 3.1281) -- (0.3040, 0.8520, 3.1220) -- cycle;
\fill[blue!51.7, opacity=0.7] (0.3040, 0.8520, 3.1220) -- (0.3500, 0.8520, 3.1281) -- (0.3500, 0.9060, 3.1317) -- (0.3040, 0.9060, 3.1256) -- cycle;
\fill[blue!61.4, opacity=0.7] (0.3040, 0.9060, 3.1256) -- (0.3500, 0.9060, 3.1317) -- (0.3500, 0.9600, 3.1350) -- (0.3040, 0.9600, 3.1289) -- cycle;
\fill[blue!63.5, opacity=0.7] (0.3040, 0.9600, 3.1289) -- (0.3500, 0.9600, 3.1350) -- (0.3500, 1.0140, 3.1380) -- (0.3040, 1.0140, 3.1319) -- cycle;
\fill[blue!61.7, opacity=0.7] (0.3040, 1.0140, 3.1319) -- (0.3500, 1.0140, 3.1380) -- (0.3500, 1.0680, 3.1407) -- (0.3040, 1.0680, 3.1346) -- cycle;
\fill[blue!59.9, opacity=0.7] (0.3040, 1.0680, 3.1346) -- (0.3500, 1.0680, 3.1407) -- (0.3500, 1.1220, 3.1431) -- (0.3040, 1.1220, 3.1370) -- cycle;
\fill[blue!59.6, opacity=0.7] (0.3040, 1.1220, 3.1370) -- (0.3500, 1.1220, 3.1431) -- (0.3500, 1.1760, 3.1452) -- (0.3040, 1.1760, 3.1391) -- cycle;
\fill[blue!60.6, opacity=0.7] (0.3040, 1.1760, 3.1391) -- (0.3500, 1.1760, 3.1452) -- (0.3500, 1.2300, 3.1470) -- (0.3040, 1.2300, 3.1409) -- cycle;
\fill[blue!62.2, opacity=0.7] (0.3040, 1.2300, 3.1409) -- (0.3500, 1.2300, 3.1470) -- (0.3500, 1.2840, 3.1484) -- (0.3040, 1.2840, 3.1423) -- cycle;
\fill[blue!63.3, opacity=0.7] (0.3040, 1.2840, 3.1423) -- (0.3500, 1.2840, 3.1484) -- (0.3500, 1.3380, 3.1496) -- (0.3040, 1.3380, 3.1435) -- cycle;
\fill[blue!63.5, opacity=0.7] (0.3040, 1.3380, 3.1435) -- (0.3500, 1.3380, 3.1496) -- (0.3500, 1.3920, 3.1504) -- (0.3040, 1.3920, 3.1443) -- cycle;
\fill[blue!62.9, opacity=0.7] (0.3040, 1.3920, 3.1443) -- (0.3500, 1.3920, 3.1504) -- (0.3500, 1.4460, 3.1509) -- (0.3040, 1.4460, 3.1448) -- cycle;
\fill[blue!61.7, opacity=0.7] (0.3040, 1.4460, 3.1448) -- (0.3500, 1.4460, 3.1509) -- (0.3500, 1.5000, 3.1511) -- (0.3040, 1.5000, 3.1449) -- cycle;
\fill[blue!60.4, opacity=0.7] (0.3040, 1.5000, 3.1449) -- (0.3500, 1.5000, 3.1511) -- (0.3500, 1.5540, 3.1509) -- (0.3040, 1.5540, 3.1448) -- cycle;
\fill[blue!59.3, opacity=0.7] (0.3040, 1.5540, 3.1448) -- (0.3500, 1.5540, 3.1509) -- (0.3500, 1.6080, 3.1504) -- (0.3040, 1.6080, 3.1443) -- cycle;
\fill[blue!58.7, opacity=0.7] (0.3040, 1.6080, 3.1443) -- (0.3500, 1.6080, 3.1504) -- (0.3500, 1.6620, 3.1496) -- (0.3040, 1.6620, 3.1435) -- cycle;
\fill[blue!58.6, opacity=0.7] (0.3040, 1.6620, 3.1435) -- (0.3500, 1.6620, 3.1496) -- (0.3500, 1.7160, 3.1484) -- (0.3040, 1.7160, 3.1423) -- cycle;
\fill[blue!59.2, opacity=0.7] (0.3040, 1.7160, 3.1423) -- (0.3500, 1.7160, 3.1484) -- (0.3500, 1.7700, 3.1470) -- (0.3040, 1.7700, 3.1409) -- cycle;
\fill[blue!60.2, opacity=0.7] (0.3040, 1.7700, 3.1409) -- (0.3500, 1.7700, 3.1470) -- (0.3500, 1.8240, 3.1452) -- (0.3040, 1.8240, 3.1391) -- cycle;
\fill[blue!61.5, opacity=0.7] (0.3040, 1.8240, 3.1391) -- (0.3500, 1.8240, 3.1452) -- (0.3500, 1.8780, 3.1431) -- (0.3040, 1.8780, 3.1370) -- cycle;
\fill[blue!62.6, opacity=0.7] (0.3040, 1.8780, 3.1370) -- (0.3500, 1.8780, 3.1431) -- (0.3500, 1.9320, 3.1407) -- (0.3040, 1.9320, 3.1346) -- cycle;
\fill[blue!63.4, opacity=0.7] (0.3040, 1.9320, 3.1346) -- (0.3500, 1.9320, 3.1407) -- (0.3500, 1.9860, 3.1380) -- (0.3040, 1.9860, 3.1319) -- cycle;
\fill[blue!63.5, opacity=0.7] (0.3040, 1.9860, 3.1319) -- (0.3500, 1.9860, 3.1380) -- (0.3500, 2.0400, 3.1350) -- (0.3040, 2.0400, 3.1289) -- cycle;
\fill[blue!63.2, opacity=0.7] (0.3040, 2.0400, 3.1289) -- (0.3500, 2.0400, 3.1350) -- (0.3500, 2.0940, 3.1317) -- (0.3040, 2.0940, 3.1256) -- cycle;
\fill[blue!62.7, opacity=0.7] (0.3040, 2.0940, 3.1256) -- (0.3500, 2.0940, 3.1317) -- (0.3500, 2.1480, 3.1281) -- (0.3040, 2.1480, 3.1220) -- cycle;
\fill[blue!62.7, opacity=0.7] (0.3040, 2.1480, 3.1220) -- (0.3500, 2.1480, 3.1281) -- (0.3500, 2.2020, 3.1243) -- (0.3040, 2.2020, 3.1182) -- cycle;
\fill[blue!63.3, opacity=0.7] (0.3040, 2.2020, 3.1182) -- (0.3500, 2.2020, 3.1243) -- (0.3500, 2.2560, 3.1202) -- (0.3040, 2.2560, 3.1141) -- cycle;
\fill[blue!63.4, opacity=0.7] (0.3040, 2.2560, 3.1141) -- (0.3500, 2.2560, 3.1202) -- (0.3500, 2.3100, 3.1159) -- (0.3040, 2.3100, 3.1098) -- cycle;
\fill[blue!59.9, opacity=0.7] (0.3040, 2.3100, 3.1098) -- (0.3500, 2.3100, 3.1159) -- (0.3500, 2.3640, 3.1114) -- (0.3040, 2.3640, 3.1052) -- cycle;
\fill[blue!50.1, opacity=0.7] (0.3040, 2.3640, 3.1052) -- (0.3500, 2.3640, 3.1114) -- (0.3500, 2.4180, 3.1066) -- (0.3040, 2.4180, 3.1005) -- cycle;
\fill[blue!36.1, opacity=0.7] (0.3040, 2.4180, 3.1005) -- (0.3500, 2.4180, 3.1066) -- (0.3500, 2.4720, 3.1016) -- (0.3040, 2.4720, 3.0955) -- cycle;
\fill[blue!24.7, opacity=0.7] (0.3040, 2.4720, 3.0955) -- (0.3500, 2.4720, 3.1016) -- (0.3500, 2.5260, 3.0964) -- (0.3040, 2.5260, 3.0903) -- cycle;
\fill[blue!18.8, opacity=0.7] (0.3040, 2.5260, 3.0903) -- (0.3500, 2.5260, 3.0964) -- (0.3500, 2.5800, 3.0911) -- (0.3040, 2.5800, 3.0849) -- cycle;
\fill[blue!16.7, opacity=0.7] (0.3040, 2.5800, 3.0849) -- (0.3500, 2.5800, 3.0911) -- (0.3500, 2.6340, 3.0855) -- (0.3040, 2.6340, 3.0794) -- cycle;
\fill[blue!16.1, opacity=0.7] (0.3040, 2.6340, 3.0794) -- (0.3500, 2.6340, 3.0855) -- (0.3500, 2.6880, 3.0799) -- (0.3040, 2.6880, 3.0738) -- cycle;
\fill[blue!16.3, opacity=0.7] (0.3040, 2.6880, 3.0738) -- (0.3500, 2.6880, 3.0799) -- (0.3500, 2.7420, 3.0741) -- (0.3040, 2.7420, 3.0680) -- cycle;
\fill[blue!17.7, opacity=0.7] (0.3040, 2.7420, 3.0680) -- (0.3500, 2.7420, 3.0741) -- (0.3500, 2.7960, 3.0681) -- (0.3040, 2.7960, 3.0620) -- cycle;
\fill[blue!21.9, opacity=0.7] (0.3040, 2.7960, 3.0620) -- (0.3500, 2.7960, 3.0681) -- (0.3500, 2.8500, 3.0621) -- (0.3040, 2.8500, 3.0560) -- cycle;
\fill[blue!30.9, opacity=0.7] (0.3040, 2.8500, 3.0560) -- (0.3500, 2.8500, 3.0621) -- (0.3500, 2.9040, 3.0560) -- (0.3040, 2.9040, 3.0499) -- cycle;
\fill[blue!41.5, opacity=0.7] (0.3040, 2.9040, 3.0499) -- (0.3500, 2.9040, 3.0560) -- (0.3500, 2.9580, 3.0498) -- (0.3040, 2.9580, 3.0437) -- cycle;
\fill[blue!47.0, opacity=0.7] (0.3040, 2.9580, 3.0437) -- (0.3500, 2.9580, 3.0498) -- (0.3500, 3.0120, 3.0436) -- (0.3040, 3.0120, 3.0375) -- cycle;
\fill[blue!43.8, opacity=0.7] (0.3040, 3.0120, 3.0375) -- (0.3500, 3.0120, 3.0436) -- (0.3500, 3.0660, 3.0373) -- (0.3040, 3.0660, 3.0312) -- cycle;
\fill[blue!32.2, opacity=0.7] (0.3040, 3.0660, 3.0312) -- (0.3500, 3.0660, 3.0373) -- (0.3500, 3.1200, 3.0311) -- (0.3040, 3.1200, 3.0249) -- cycle;
\fill[blue!15.4, opacity=0.7] (0.3500, -0.1200, 3.0311) -- (0.3960, -0.1200, 3.0371) -- (0.3960, -0.0660, 3.0434) -- (0.3500, -0.0660, 3.0373) -- cycle;
\fill[blue!15.2, opacity=0.7] (0.3500, -0.0660, 3.0373) -- (0.3960, -0.0660, 3.0434) -- (0.3960, -0.0120, 3.0496) -- (0.3500, -0.0120, 3.0436) -- cycle;
\fill[blue!15.5, opacity=0.7] (0.3500, -0.0120, 3.0436) -- (0.3960, -0.0120, 3.0496) -- (0.3960, 0.0420, 3.0559) -- (0.3500, 0.0420, 3.0498) -- cycle;
\fill[blue!17.7, opacity=0.7] (0.3500, 0.0420, 3.0498) -- (0.3960, 0.0420, 3.0559) -- (0.3960, 0.0960, 3.0620) -- (0.3500, 0.0960, 3.0560) -- cycle;
\fill[blue!27.6, opacity=0.7] (0.3500, 0.0960, 3.0560) -- (0.3960, 0.0960, 3.0620) -- (0.3960, 0.1500, 3.0681) -- (0.3500, 0.1500, 3.0621) -- cycle;
\fill[blue!45.9, opacity=0.7] (0.3500, 0.1500, 3.0621) -- (0.3960, 0.1500, 3.0681) -- (0.3960, 0.2040, 3.0742) -- (0.3500, 0.2040, 3.0681) -- cycle;
\fill[blue!57.6, opacity=0.7] (0.3500, 0.2040, 3.0681) -- (0.3960, 0.2040, 3.0742) -- (0.3960, 0.2580, 3.0801) -- (0.3500, 0.2580, 3.0741) -- cycle;
\fill[blue!59.3, opacity=0.7] (0.3500, 0.2580, 3.0741) -- (0.3960, 0.2580, 3.0801) -- (0.3960, 0.3120, 3.0859) -- (0.3500, 0.3120, 3.0799) -- cycle;
\fill[blue!53.6, opacity=0.7] (0.3500, 0.3120, 3.0799) -- (0.3960, 0.3120, 3.0859) -- (0.3960, 0.3660, 3.0916) -- (0.3500, 0.3660, 3.0855) -- cycle;
\fill[blue!40.0, opacity=0.7] (0.3500, 0.3660, 3.0855) -- (0.3960, 0.3660, 3.0916) -- (0.3960, 0.4200, 3.0971) -- (0.3500, 0.4200, 3.0911) -- cycle;
\fill[blue!26.7, opacity=0.7] (0.3500, 0.4200, 3.0911) -- (0.3960, 0.4200, 3.0971) -- (0.3960, 0.4740, 3.1024) -- (0.3500, 0.4740, 3.0964) -- cycle;
\fill[blue!20.2, opacity=0.7] (0.3500, 0.4740, 3.0964) -- (0.3960, 0.4740, 3.1024) -- (0.3960, 0.5280, 3.1076) -- (0.3500, 0.5280, 3.1016) -- cycle;
\fill[blue!18.2, opacity=0.7] (0.3500, 0.5280, 3.1016) -- (0.3960, 0.5280, 3.1076) -- (0.3960, 0.5820, 3.1126) -- (0.3500, 0.5820, 3.1066) -- cycle;
\fill[blue!18.7, opacity=0.7] (0.3500, 0.5820, 3.1066) -- (0.3960, 0.5820, 3.1126) -- (0.3960, 0.6360, 3.1174) -- (0.3500, 0.6360, 3.1114) -- cycle;
\fill[blue!22.0, opacity=0.7] (0.3500, 0.6360, 3.1114) -- (0.3960, 0.6360, 3.1174) -- (0.3960, 0.6900, 3.1219) -- (0.3500, 0.6900, 3.1159) -- cycle;
\fill[blue!31.0, opacity=0.7] (0.3500, 0.6900, 3.1159) -- (0.3960, 0.6900, 3.1219) -- (0.3960, 0.7440, 3.1263) -- (0.3500, 0.7440, 3.1202) -- cycle;
\fill[blue!46.4, opacity=0.7] (0.3500, 0.7440, 3.1202) -- (0.3960, 0.7440, 3.1263) -- (0.3960, 0.7980, 3.1303) -- (0.3500, 0.7980, 3.1243) -- cycle;
\fill[blue!59.8, opacity=0.7] (0.3500, 0.7980, 3.1243) -- (0.3960, 0.7980, 3.1303) -- (0.3960, 0.8520, 3.1342) -- (0.3500, 0.8520, 3.1281) -- cycle;
\fill[blue!63.6, opacity=0.7] (0.3500, 0.8520, 3.1281) -- (0.3960, 0.8520, 3.1342) -- (0.3960, 0.9060, 3.1377) -- (0.3500, 0.9060, 3.1317) -- cycle;
\fill[blue!61.3, opacity=0.7] (0.3500, 0.9060, 3.1317) -- (0.3960, 0.9060, 3.1377) -- (0.3960, 0.9600, 3.1410) -- (0.3500, 0.9600, 3.1350) -- cycle;
\fill[blue!59.1, opacity=0.7] (0.3500, 0.9600, 3.1350) -- (0.3960, 0.9600, 3.1410) -- (0.3960, 1.0140, 3.1440) -- (0.3500, 1.0140, 3.1380) -- cycle;
\fill[blue!59.4, opacity=0.7] (0.3500, 1.0140, 3.1380) -- (0.3960, 1.0140, 3.1440) -- (0.3960, 1.0680, 3.1467) -- (0.3500, 1.0680, 3.1407) -- cycle;
\fill[blue!61.4, opacity=0.7] (0.3500, 1.0680, 3.1407) -- (0.3960, 1.0680, 3.1467) -- (0.3960, 1.1220, 3.1491) -- (0.3500, 1.1220, 3.1431) -- cycle;
\fill[blue!63.3, opacity=0.7] (0.3500, 1.1220, 3.1431) -- (0.3960, 1.1220, 3.1491) -- (0.3960, 1.1760, 3.1512) -- (0.3500, 1.1760, 3.1452) -- cycle;
\fill[blue!63.2, opacity=0.7] (0.3500, 1.1760, 3.1452) -- (0.3960, 1.1760, 3.1512) -- (0.3960, 1.2300, 3.1530) -- (0.3500, 1.2300, 3.1470) -- cycle;
\fill[blue!60.7, opacity=0.7] (0.3500, 1.2300, 3.1470) -- (0.3960, 1.2300, 3.1530) -- (0.3960, 1.2840, 3.1545) -- (0.3500, 1.2840, 3.1484) -- cycle;
\fill[blue!56.4, opacity=0.7] (0.3500, 1.2840, 3.1484) -- (0.3960, 1.2840, 3.1545) -- (0.3960, 1.3380, 3.1556) -- (0.3500, 1.3380, 3.1496) -- cycle;
\fill[blue!51.6, opacity=0.7] (0.3500, 1.3380, 3.1496) -- (0.3960, 1.3380, 3.1556) -- (0.3960, 1.3920, 3.1564) -- (0.3500, 1.3920, 3.1504) -- cycle;
\fill[blue!47.3, opacity=0.7] (0.3500, 1.3920, 3.1504) -- (0.3960, 1.3920, 3.1564) -- (0.3960, 1.4460, 3.1569) -- (0.3500, 1.4460, 3.1509) -- cycle;
\fill[blue!43.8, opacity=0.7] (0.3500, 1.4460, 3.1509) -- (0.3960, 1.4460, 3.1569) -- (0.3960, 1.5000, 3.1571) -- (0.3500, 1.5000, 3.1511) -- cycle;
\fill[blue!41.3, opacity=0.7] (0.3500, 1.5000, 3.1511) -- (0.3960, 1.5000, 3.1571) -- (0.3960, 1.5540, 3.1569) -- (0.3500, 1.5540, 3.1509) -- cycle;
\fill[blue!39.7, opacity=0.7] (0.3500, 1.5540, 3.1509) -- (0.3960, 1.5540, 3.1569) -- (0.3960, 1.6080, 3.1564) -- (0.3500, 1.6080, 3.1504) -- cycle;
\fill[blue!38.8, opacity=0.7] (0.3500, 1.6080, 3.1504) -- (0.3960, 1.6080, 3.1564) -- (0.3960, 1.6620, 3.1556) -- (0.3500, 1.6620, 3.1496) -- cycle;
\fill[blue!38.7, opacity=0.7] (0.3500, 1.6620, 3.1496) -- (0.3960, 1.6620, 3.1556) -- (0.3960, 1.7160, 3.1545) -- (0.3500, 1.7160, 3.1484) -- cycle;
\fill[blue!39.4, opacity=0.7] (0.3500, 1.7160, 3.1484) -- (0.3960, 1.7160, 3.1545) -- (0.3960, 1.7700, 3.1530) -- (0.3500, 1.7700, 3.1470) -- cycle;
\fill[blue!40.8, opacity=0.7] (0.3500, 1.7700, 3.1470) -- (0.3960, 1.7700, 3.1530) -- (0.3960, 1.8240, 3.1512) -- (0.3500, 1.8240, 3.1452) -- cycle;
\fill[blue!43.2, opacity=0.7] (0.3500, 1.8240, 3.1452) -- (0.3960, 1.8240, 3.1512) -- (0.3960, 1.8780, 3.1491) -- (0.3500, 1.8780, 3.1431) -- cycle;
\fill[blue!46.6, opacity=0.7] (0.3500, 1.8780, 3.1431) -- (0.3960, 1.8780, 3.1491) -- (0.3960, 1.9320, 3.1467) -- (0.3500, 1.9320, 3.1407) -- cycle;
\fill[blue!50.9, opacity=0.7] (0.3500, 1.9320, 3.1407) -- (0.3960, 1.9320, 3.1467) -- (0.3960, 1.9860, 3.1440) -- (0.3500, 1.9860, 3.1380) -- cycle;
\fill[blue!55.6, opacity=0.7] (0.3500, 1.9860, 3.1380) -- (0.3960, 1.9860, 3.1440) -- (0.3960, 2.0400, 3.1410) -- (0.3500, 2.0400, 3.1350) -- cycle;
\fill[blue!59.8, opacity=0.7] (0.3500, 2.0400, 3.1350) -- (0.3960, 2.0400, 3.1410) -- (0.3960, 2.0940, 3.1377) -- (0.3500, 2.0940, 3.1317) -- cycle;
\fill[blue!62.7, opacity=0.7] (0.3500, 2.0940, 3.1317) -- (0.3960, 2.0940, 3.1377) -- (0.3960, 2.1480, 3.1342) -- (0.3500, 2.1480, 3.1281) -- cycle;
\fill[blue!63.6, opacity=0.7] (0.3500, 2.1480, 3.1281) -- (0.3960, 2.1480, 3.1342) -- (0.3960, 2.2020, 3.1303) -- (0.3500, 2.2020, 3.1243) -- cycle;
\fill[blue!63.2, opacity=0.7] (0.3500, 2.2020, 3.1243) -- (0.3960, 2.2020, 3.1303) -- (0.3960, 2.2560, 3.1263) -- (0.3500, 2.2560, 3.1202) -- cycle;
\fill[blue!62.9, opacity=0.7] (0.3500, 2.2560, 3.1202) -- (0.3960, 2.2560, 3.1263) -- (0.3960, 2.3100, 3.1219) -- (0.3500, 2.3100, 3.1159) -- cycle;
\fill[blue!63.3, opacity=0.7] (0.3500, 2.3100, 3.1159) -- (0.3960, 2.3100, 3.1219) -- (0.3960, 2.3640, 3.1174) -- (0.3500, 2.3640, 3.1114) -- cycle;
\fill[blue!63.2, opacity=0.7] (0.3500, 2.3640, 3.1114) -- (0.3960, 2.3640, 3.1174) -- (0.3960, 2.4180, 3.1126) -- (0.3500, 2.4180, 3.1066) -- cycle;
\fill[blue!58.3, opacity=0.7] (0.3500, 2.4180, 3.1066) -- (0.3960, 2.4180, 3.1126) -- (0.3960, 2.4720, 3.1076) -- (0.3500, 2.4720, 3.1016) -- cycle;
\fill[blue!45.7, opacity=0.7] (0.3500, 2.4720, 3.1016) -- (0.3960, 2.4720, 3.1076) -- (0.3960, 2.5260, 3.1024) -- (0.3500, 2.5260, 3.0964) -- cycle;
\fill[blue!30.6, opacity=0.7] (0.3500, 2.5260, 3.0964) -- (0.3960, 2.5260, 3.1024) -- (0.3960, 2.5800, 3.0971) -- (0.3500, 2.5800, 3.0911) -- cycle;
\fill[blue!21.0, opacity=0.7] (0.3500, 2.5800, 3.0911) -- (0.3960, 2.5800, 3.0971) -- (0.3960, 2.6340, 3.0916) -- (0.3500, 2.6340, 3.0855) -- cycle;
\fill[blue!17.2, opacity=0.7] (0.3500, 2.6340, 3.0855) -- (0.3960, 2.6340, 3.0916) -- (0.3960, 2.6880, 3.0859) -- (0.3500, 2.6880, 3.0799) -- cycle;
\fill[blue!16.2, opacity=0.7] (0.3500, 2.6880, 3.0799) -- (0.3960, 2.6880, 3.0859) -- (0.3960, 2.7420, 3.0801) -- (0.3500, 2.7420, 3.0741) -- cycle;
\fill[blue!16.1, opacity=0.7] (0.3500, 2.7420, 3.0741) -- (0.3960, 2.7420, 3.0801) -- (0.3960, 2.7960, 3.0742) -- (0.3500, 2.7960, 3.0681) -- cycle;
\fill[blue!17.1, opacity=0.7] (0.3500, 2.7960, 3.0681) -- (0.3960, 2.7960, 3.0742) -- (0.3960, 2.8500, 3.0681) -- (0.3500, 2.8500, 3.0621) -- cycle;
\fill[blue!20.7, opacity=0.7] (0.3500, 2.8500, 3.0621) -- (0.3960, 2.8500, 3.0681) -- (0.3960, 2.9040, 3.0620) -- (0.3500, 2.9040, 3.0560) -- cycle;
\fill[blue!29.3, opacity=0.7] (0.3500, 2.9040, 3.0560) -- (0.3960, 2.9040, 3.0620) -- (0.3960, 2.9580, 3.0559) -- (0.3500, 2.9580, 3.0498) -- cycle;
\fill[blue!40.5, opacity=0.7] (0.3500, 2.9580, 3.0498) -- (0.3960, 2.9580, 3.0559) -- (0.3960, 3.0120, 3.0496) -- (0.3500, 3.0120, 3.0436) -- cycle;
\fill[blue!46.6, opacity=0.7] (0.3500, 3.0120, 3.0436) -- (0.3960, 3.0120, 3.0496) -- (0.3960, 3.0660, 3.0434) -- (0.3500, 3.0660, 3.0373) -- cycle;
\fill[blue!43.4, opacity=0.7] (0.3500, 3.0660, 3.0373) -- (0.3960, 3.0660, 3.0434) -- (0.3960, 3.1200, 3.0371) -- (0.3500, 3.1200, 3.0311) -- cycle;
\fill[blue!15.3, opacity=0.7] (0.3960, -0.1200, 3.0371) -- (0.4420, -0.1200, 3.0430) -- (0.4420, -0.0660, 3.0493) -- (0.3960, -0.0660, 3.0434) -- cycle;
\fill[blue!15.4, opacity=0.7] (0.3960, -0.0660, 3.0434) -- (0.4420, -0.0660, 3.0493) -- (0.4420, -0.0120, 3.0555) -- (0.3960, -0.0120, 3.0496) -- cycle;
\fill[blue!16.9, opacity=0.7] (0.3960, -0.0120, 3.0496) -- (0.4420, -0.0120, 3.0555) -- (0.4420, 0.0420, 3.0618) -- (0.3960, 0.0420, 3.0559) -- cycle;
\fill[blue!24.7, opacity=0.7] (0.3960, 0.0420, 3.0559) -- (0.4420, 0.0420, 3.0618) -- (0.4420, 0.0960, 3.0680) -- (0.3960, 0.0960, 3.0620) -- cycle;
\fill[blue!43.0, opacity=0.7] (0.3960, 0.0960, 3.0620) -- (0.4420, 0.0960, 3.0680) -- (0.4420, 0.1500, 3.0741) -- (0.3960, 0.1500, 3.0681) -- cycle;
\fill[blue!56.9, opacity=0.7] (0.3960, 0.1500, 3.0681) -- (0.4420, 0.1500, 3.0741) -- (0.4420, 0.2040, 3.0801) -- (0.3960, 0.2040, 3.0742) -- cycle;
\fill[blue!59.7, opacity=0.7] (0.3960, 0.2040, 3.0742) -- (0.4420, 0.2040, 3.0801) -- (0.4420, 0.2580, 3.0860) -- (0.3960, 0.2580, 3.0801) -- cycle;
\fill[blue!54.7, opacity=0.7] (0.3960, 0.2580, 3.0801) -- (0.4420, 0.2580, 3.0860) -- (0.4420, 0.3120, 3.0918) -- (0.3960, 0.3120, 3.0859) -- cycle;
\fill[blue!41.0, opacity=0.7] (0.3960, 0.3120, 3.0859) -- (0.4420, 0.3120, 3.0918) -- (0.4420, 0.3660, 3.0975) -- (0.3960, 0.3660, 3.0916) -- cycle;
\fill[blue!26.9, opacity=0.7] (0.3960, 0.3660, 3.0916) -- (0.4420, 0.3660, 3.0975) -- (0.4420, 0.4200, 3.1030) -- (0.3960, 0.4200, 3.0971) -- cycle;
\fill[blue!20.1, opacity=0.7] (0.3960, 0.4200, 3.0971) -- (0.4420, 0.4200, 3.1030) -- (0.4420, 0.4740, 3.1084) -- (0.3960, 0.4740, 3.1024) -- cycle;
\fill[blue!18.3, opacity=0.7] (0.3960, 0.4740, 3.1024) -- (0.4420, 0.4740, 3.1084) -- (0.4420, 0.5280, 3.1135) -- (0.3960, 0.5280, 3.1076) -- cycle;
\fill[blue!19.1, opacity=0.7] (0.3960, 0.5280, 3.1076) -- (0.4420, 0.5280, 3.1135) -- (0.4420, 0.5820, 3.1185) -- (0.3960, 0.5820, 3.1126) -- cycle;
\fill[blue!23.8, opacity=0.7] (0.3960, 0.5820, 3.1126) -- (0.4420, 0.5820, 3.1185) -- (0.4420, 0.6360, 3.1233) -- (0.3960, 0.6360, 3.1174) -- cycle;
\fill[blue!36.0, opacity=0.7] (0.3960, 0.6360, 3.1174) -- (0.4420, 0.6360, 3.1233) -- (0.4420, 0.6900, 3.1279) -- (0.3960, 0.6900, 3.1219) -- cycle;
\fill[blue!53.2, opacity=0.7] (0.3960, 0.6900, 3.1219) -- (0.4420, 0.6900, 3.1279) -- (0.4420, 0.7440, 3.1322) -- (0.3960, 0.7440, 3.1263) -- cycle;
\fill[blue!63.0, opacity=0.7] (0.3960, 0.7440, 3.1263) -- (0.4420, 0.7440, 3.1322) -- (0.4420, 0.7980, 3.1363) -- (0.3960, 0.7980, 3.1303) -- cycle;
\fill[blue!62.2, opacity=0.7] (0.3960, 0.7980, 3.1303) -- (0.4420, 0.7980, 3.1363) -- (0.4420, 0.8520, 3.1401) -- (0.3960, 0.8520, 3.1342) -- cycle;
\fill[blue!59.0, opacity=0.7] (0.3960, 0.8520, 3.1342) -- (0.4420, 0.8520, 3.1401) -- (0.4420, 0.9060, 3.1436) -- (0.3960, 0.9060, 3.1377) -- cycle;
\fill[blue!58.7, opacity=0.7] (0.3960, 0.9060, 3.1377) -- (0.4420, 0.9060, 3.1436) -- (0.4420, 0.9600, 3.1469) -- (0.3960, 0.9600, 3.1410) -- cycle;
\fill[blue!61.2, opacity=0.7] (0.3960, 0.9600, 3.1410) -- (0.4420, 0.9600, 3.1469) -- (0.4420, 1.0140, 3.1499) -- (0.3960, 1.0140, 3.1440) -- cycle;
\fill[blue!63.5, opacity=0.7] (0.3960, 1.0140, 3.1440) -- (0.4420, 1.0140, 3.1499) -- (0.4420, 1.0680, 3.1526) -- (0.3960, 1.0680, 3.1467) -- cycle;
\fill[blue!62.2, opacity=0.7] (0.3960, 1.0680, 3.1467) -- (0.4420, 1.0680, 3.1526) -- (0.4420, 1.1220, 3.1550) -- (0.3960, 1.1220, 3.1491) -- cycle;
\fill[blue!56.9, opacity=0.7] (0.3960, 1.1220, 3.1491) -- (0.4420, 1.1220, 3.1550) -- (0.4420, 1.1760, 3.1571) -- (0.3960, 1.1760, 3.1512) -- cycle;
\fill[blue!49.6, opacity=0.7] (0.3960, 1.1760, 3.1512) -- (0.4420, 1.1760, 3.1571) -- (0.4420, 1.2300, 3.1589) -- (0.3960, 1.2300, 3.1530) -- cycle;
\fill[blue!42.9, opacity=0.7] (0.3960, 1.2300, 3.1530) -- (0.4420, 1.2300, 3.1589) -- (0.4420, 1.2840, 3.1604) -- (0.3960, 1.2840, 3.1545) -- cycle;
\fill[blue!37.7, opacity=0.7] (0.3960, 1.2840, 3.1545) -- (0.4420, 1.2840, 3.1604) -- (0.4420, 1.3380, 3.1615) -- (0.3960, 1.3380, 3.1556) -- cycle;
\fill[blue!34.2, opacity=0.7] (0.3960, 1.3380, 3.1556) -- (0.4420, 1.3380, 3.1615) -- (0.4420, 1.3920, 3.1623) -- (0.3960, 1.3920, 3.1564) -- cycle;
\fill[blue!32.1, opacity=0.7] (0.3960, 1.3920, 3.1564) -- (0.4420, 1.3920, 3.1623) -- (0.4420, 1.4460, 3.1628) -- (0.3960, 1.4460, 3.1569) -- cycle;
\fill[blue!30.7, opacity=0.7] (0.3960, 1.4460, 3.1569) -- (0.4420, 1.4460, 3.1628) -- (0.4420, 1.5000, 3.1630) -- (0.3960, 1.5000, 3.1571) -- cycle;
\fill[blue!29.9, opacity=0.7] (0.3960, 1.5000, 3.1571) -- (0.4420, 1.5000, 3.1630) -- (0.4420, 1.5540, 3.1628) -- (0.3960, 1.5540, 3.1569) -- cycle;
\fill[blue!29.4, opacity=0.7] (0.3960, 1.5540, 3.1569) -- (0.4420, 1.5540, 3.1628) -- (0.4420, 1.6080, 3.1623) -- (0.3960, 1.6080, 3.1564) -- cycle;
\fill[blue!29.1, opacity=0.7] (0.3960, 1.6080, 3.1564) -- (0.4420, 1.6080, 3.1623) -- (0.4420, 1.6620, 3.1615) -- (0.3960, 1.6620, 3.1556) -- cycle;
\fill[blue!28.8, opacity=0.7] (0.3960, 1.6620, 3.1556) -- (0.4420, 1.6620, 3.1615) -- (0.4420, 1.7160, 3.1604) -- (0.3960, 1.7160, 3.1545) -- cycle;
\fill[blue!28.7, opacity=0.7] (0.3960, 1.7160, 3.1545) -- (0.4420, 1.7160, 3.1604) -- (0.4420, 1.7700, 3.1589) -- (0.3960, 1.7700, 3.1530) -- cycle;
\fill[blue!28.7, opacity=0.7] (0.3960, 1.7700, 3.1530) -- (0.4420, 1.7700, 3.1589) -- (0.4420, 1.8240, 3.1571) -- (0.3960, 1.8240, 3.1512) -- cycle;
\fill[blue!29.2, opacity=0.7] (0.3960, 1.8240, 3.1512) -- (0.4420, 1.8240, 3.1571) -- (0.4420, 1.8780, 3.1550) -- (0.3960, 1.8780, 3.1491) -- cycle;
\fill[blue!30.3, opacity=0.7] (0.3960, 1.8780, 3.1491) -- (0.4420, 1.8780, 3.1550) -- (0.4420, 1.9320, 3.1526) -- (0.3960, 1.9320, 3.1467) -- cycle;
\fill[blue!32.4, opacity=0.7] (0.3960, 1.9320, 3.1467) -- (0.4420, 1.9320, 3.1526) -- (0.4420, 1.9860, 3.1499) -- (0.3960, 1.9860, 3.1440) -- cycle;
\fill[blue!35.8, opacity=0.7] (0.3960, 1.9860, 3.1440) -- (0.4420, 1.9860, 3.1499) -- (0.4420, 2.0400, 3.1469) -- (0.3960, 2.0400, 3.1410) -- cycle;
\fill[blue!41.0, opacity=0.7] (0.3960, 2.0400, 3.1410) -- (0.4420, 2.0400, 3.1469) -- (0.4420, 2.0940, 3.1436) -- (0.3960, 2.0940, 3.1377) -- cycle;
\fill[blue!47.7, opacity=0.7] (0.3960, 2.0940, 3.1377) -- (0.4420, 2.0940, 3.1436) -- (0.4420, 2.1480, 3.1401) -- (0.3960, 2.1480, 3.1342) -- cycle;
\fill[blue!55.1, opacity=0.7] (0.3960, 2.1480, 3.1342) -- (0.4420, 2.1480, 3.1401) -- (0.4420, 2.2020, 3.1363) -- (0.3960, 2.2020, 3.1303) -- cycle;
\fill[blue!60.8, opacity=0.7] (0.3960, 2.2020, 3.1303) -- (0.4420, 2.2020, 3.1363) -- (0.4420, 2.2560, 3.1322) -- (0.3960, 2.2560, 3.1263) -- cycle;
\fill[blue!63.4, opacity=0.7] (0.3960, 2.2560, 3.1263) -- (0.4420, 2.2560, 3.1322) -- (0.4420, 2.3100, 3.1279) -- (0.3960, 2.3100, 3.1219) -- cycle;
\fill[blue!63.4, opacity=0.7] (0.3960, 2.3100, 3.1219) -- (0.4420, 2.3100, 3.1279) -- (0.4420, 2.3640, 3.1233) -- (0.3960, 2.3640, 3.1174) -- cycle;
\fill[blue!63.1, opacity=0.7] (0.3960, 2.3640, 3.1174) -- (0.4420, 2.3640, 3.1233) -- (0.4420, 2.4180, 3.1185) -- (0.3960, 2.4180, 3.1126) -- cycle;
\fill[blue!63.5, opacity=0.7] (0.3960, 2.4180, 3.1126) -- (0.4420, 2.4180, 3.1185) -- (0.4420, 2.4720, 3.1135) -- (0.3960, 2.4720, 3.1076) -- cycle;
\fill[blue!62.1, opacity=0.7] (0.3960, 2.4720, 3.1076) -- (0.4420, 2.4720, 3.1135) -- (0.4420, 2.5260, 3.1084) -- (0.3960, 2.5260, 3.1024) -- cycle;
\fill[blue!52.9, opacity=0.7] (0.3960, 2.5260, 3.1024) -- (0.4420, 2.5260, 3.1084) -- (0.4420, 2.5800, 3.1030) -- (0.3960, 2.5800, 3.0971) -- cycle;
\fill[blue!36.6, opacity=0.7] (0.3960, 2.5800, 3.0971) -- (0.4420, 2.5800, 3.1030) -- (0.4420, 2.6340, 3.0975) -- (0.3960, 2.6340, 3.0916) -- cycle;
\fill[blue!23.4, opacity=0.7] (0.3960, 2.6340, 3.0916) -- (0.4420, 2.6340, 3.0975) -- (0.4420, 2.6880, 3.0918) -- (0.3960, 2.6880, 3.0859) -- cycle;
\fill[blue!17.8, opacity=0.7] (0.3960, 2.6880, 3.0859) -- (0.4420, 2.6880, 3.0918) -- (0.4420, 2.7420, 3.0860) -- (0.3960, 2.7420, 3.0801) -- cycle;
\fill[blue!16.2, opacity=0.7] (0.3960, 2.7420, 3.0801) -- (0.4420, 2.7420, 3.0860) -- (0.4420, 2.7960, 3.0801) -- (0.3960, 2.7960, 3.0742) -- cycle;
\fill[blue!16.0, opacity=0.7] (0.3960, 2.7960, 3.0742) -- (0.4420, 2.7960, 3.0801) -- (0.4420, 2.8500, 3.0741) -- (0.3960, 2.8500, 3.0681) -- cycle;
\fill[blue!16.9, opacity=0.7] (0.3960, 2.8500, 3.0681) -- (0.4420, 2.8500, 3.0741) -- (0.4420, 2.9040, 3.0680) -- (0.3960, 2.9040, 3.0620) -- cycle;
\fill[blue!20.1, opacity=0.7] (0.3960, 2.9040, 3.0620) -- (0.4420, 2.9040, 3.0680) -- (0.4420, 2.9580, 3.0618) -- (0.3960, 2.9580, 3.0559) -- cycle;
\fill[blue!28.7, opacity=0.7] (0.3960, 2.9580, 3.0559) -- (0.4420, 2.9580, 3.0618) -- (0.4420, 3.0120, 3.0555) -- (0.3960, 3.0120, 3.0496) -- cycle;
\fill[blue!40.3, opacity=0.7] (0.3960, 3.0120, 3.0496) -- (0.4420, 3.0120, 3.0555) -- (0.4420, 3.0660, 3.0493) -- (0.3960, 3.0660, 3.0434) -- cycle;
\fill[blue!46.3, opacity=0.7] (0.3960, 3.0660, 3.0434) -- (0.4420, 3.0660, 3.0493) -- (0.4420, 3.1200, 3.0430) -- (0.3960, 3.1200, 3.0371) -- cycle;
\fill[blue!15.3, opacity=0.7] (0.4420, -0.1200, 3.0430) -- (0.4880, -0.1200, 3.0488) -- (0.4880, -0.0660, 3.0551) -- (0.4420, -0.0660, 3.0493) -- cycle;
\fill[blue!16.1, opacity=0.7] (0.4420, -0.0660, 3.0493) -- (0.4880, -0.0660, 3.0551) -- (0.4880, -0.0120, 3.0614) -- (0.4420, -0.0120, 3.0555) -- cycle;
\fill[blue!21.6, opacity=0.7] (0.4420, -0.0120, 3.0555) -- (0.4880, -0.0120, 3.0614) -- (0.4880, 0.0420, 3.0676) -- (0.4420, 0.0420, 3.0618) -- cycle;
\fill[blue!38.4, opacity=0.7] (0.4420, 0.0420, 3.0618) -- (0.4880, 0.0420, 3.0676) -- (0.4880, 0.0960, 3.0738) -- (0.4420, 0.0960, 3.0680) -- cycle;
\fill[blue!55.2, opacity=0.7] (0.4420, 0.0960, 3.0680) -- (0.4880, 0.0960, 3.0738) -- (0.4880, 0.1500, 3.0799) -- (0.4420, 0.1500, 3.0741) -- cycle;
\fill[blue!60.0, opacity=0.7] (0.4420, 0.1500, 3.0741) -- (0.4880, 0.1500, 3.0799) -- (0.4880, 0.2040, 3.0859) -- (0.4420, 0.2040, 3.0801) -- cycle;
\fill[blue!56.4, opacity=0.7] (0.4420, 0.2040, 3.0801) -- (0.4880, 0.2040, 3.0859) -- (0.4880, 0.2580, 3.0918) -- (0.4420, 0.2580, 3.0860) -- cycle;
\fill[blue!43.2, opacity=0.7] (0.4420, 0.2580, 3.0860) -- (0.4880, 0.2580, 3.0918) -- (0.4880, 0.3120, 3.0976) -- (0.4420, 0.3120, 3.0918) -- cycle;
\fill[blue!28.0, opacity=0.7] (0.4420, 0.3120, 3.0918) -- (0.4880, 0.3120, 3.0976) -- (0.4880, 0.3660, 3.1033) -- (0.4420, 0.3660, 3.0975) -- cycle;
\fill[blue!20.4, opacity=0.7] (0.4420, 0.3660, 3.0975) -- (0.4880, 0.3660, 3.1033) -- (0.4880, 0.4200, 3.1088) -- (0.4420, 0.4200, 3.1030) -- cycle;
\fill[blue!18.4, opacity=0.7] (0.4420, 0.4200, 3.1030) -- (0.4880, 0.4200, 3.1088) -- (0.4880, 0.4740, 3.1142) -- (0.4420, 0.4740, 3.1084) -- cycle;
\fill[blue!19.4, opacity=0.7] (0.4420, 0.4740, 3.1084) -- (0.4880, 0.4740, 3.1142) -- (0.4880, 0.5280, 3.1193) -- (0.4420, 0.5280, 3.1135) -- cycle;
\fill[blue!25.1, opacity=0.7] (0.4420, 0.5280, 3.1135) -- (0.4880, 0.5280, 3.1193) -- (0.4880, 0.5820, 3.1243) -- (0.4420, 0.5820, 3.1185) -- cycle;
\fill[blue!39.6, opacity=0.7] (0.4420, 0.5820, 3.1185) -- (0.4880, 0.5820, 3.1243) -- (0.4880, 0.6360, 3.1291) -- (0.4420, 0.6360, 3.1233) -- cycle;
\fill[blue!57.4, opacity=0.7] (0.4420, 0.6360, 3.1233) -- (0.4880, 0.6360, 3.1291) -- (0.4880, 0.6900, 3.1337) -- (0.4420, 0.6900, 3.1279) -- cycle;
\fill[blue!63.6, opacity=0.7] (0.4420, 0.6900, 3.1279) -- (0.4880, 0.6900, 3.1337) -- (0.4880, 0.7440, 3.1380) -- (0.4420, 0.7440, 3.1322) -- cycle;
\fill[blue!60.3, opacity=0.7] (0.4420, 0.7440, 3.1322) -- (0.4880, 0.7440, 3.1380) -- (0.4880, 0.7980, 3.1421) -- (0.4420, 0.7980, 3.1363) -- cycle;
\fill[blue!57.9, opacity=0.7] (0.4420, 0.7980, 3.1363) -- (0.4880, 0.7980, 3.1421) -- (0.4880, 0.8520, 3.1459) -- (0.4420, 0.8520, 3.1401) -- cycle;
\fill[blue!59.8, opacity=0.7] (0.4420, 0.8520, 3.1401) -- (0.4880, 0.8520, 3.1459) -- (0.4880, 0.9060, 3.1494) -- (0.4420, 0.9060, 3.1436) -- cycle;
\fill[blue!63.1, opacity=0.7] (0.4420, 0.9060, 3.1436) -- (0.4880, 0.9060, 3.1494) -- (0.4880, 0.9600, 3.1527) -- (0.4420, 0.9600, 3.1469) -- cycle;
\fill[blue!62.5, opacity=0.7] (0.4420, 0.9600, 3.1469) -- (0.4880, 0.9600, 3.1527) -- (0.4880, 1.0140, 3.1557) -- (0.4420, 1.0140, 3.1499) -- cycle;
\fill[blue!56.0, opacity=0.7] (0.4420, 1.0140, 3.1499) -- (0.4880, 1.0140, 3.1557) -- (0.4880, 1.0680, 3.1584) -- (0.4420, 1.0680, 3.1526) -- cycle;
\fill[blue!46.9, opacity=0.7] (0.4420, 1.0680, 3.1526) -- (0.4880, 1.0680, 3.1584) -- (0.4880, 1.1220, 3.1608) -- (0.4420, 1.1220, 3.1550) -- cycle;
\fill[blue!39.2, opacity=0.7] (0.4420, 1.1220, 3.1550) -- (0.4880, 1.1220, 3.1608) -- (0.4880, 1.1760, 3.1629) -- (0.4420, 1.1760, 3.1571) -- cycle;
\fill[blue!34.2, opacity=0.7] (0.4420, 1.1760, 3.1571) -- (0.4880, 1.1760, 3.1629) -- (0.4880, 1.2300, 3.1647) -- (0.4420, 1.2300, 3.1589) -- cycle;
\fill[blue!31.6, opacity=0.7] (0.4420, 1.2300, 3.1589) -- (0.4880, 1.2300, 3.1647) -- (0.4880, 1.2840, 3.1662) -- (0.4420, 1.2840, 3.1604) -- cycle;
\fill[blue!30.8, opacity=0.7] (0.4420, 1.2840, 3.1604) -- (0.4880, 1.2840, 3.1662) -- (0.4880, 1.3380, 3.1673) -- (0.4420, 1.3380, 3.1615) -- cycle;
\fill[blue!31.1, opacity=0.7] (0.4420, 1.3380, 3.1615) -- (0.4880, 1.3380, 3.1673) -- (0.4880, 1.3920, 3.1682) -- (0.4420, 1.3920, 3.1623) -- cycle;
\fill[blue!32.0, opacity=0.7] (0.4420, 1.3920, 3.1623) -- (0.4880, 1.3920, 3.1682) -- (0.4880, 1.4460, 3.1686) -- (0.4420, 1.4460, 3.1628) -- cycle;
\fill[blue!33.2, opacity=0.7] (0.4420, 1.4460, 3.1628) -- (0.4880, 1.4460, 3.1686) -- (0.4880, 1.5000, 3.1688) -- (0.4420, 1.5000, 3.1630) -- cycle;
\fill[blue!34.3, opacity=0.7] (0.4420, 1.5000, 3.1630) -- (0.4880, 1.5000, 3.1688) -- (0.4880, 1.5540, 3.1686) -- (0.4420, 1.5540, 3.1628) -- cycle;
\fill[blue!34.9, opacity=0.7] (0.4420, 1.5540, 3.1628) -- (0.4880, 1.5540, 3.1686) -- (0.4880, 1.6080, 3.1682) -- (0.4420, 1.6080, 3.1623) -- cycle;
\fill[blue!35.0, opacity=0.7] (0.4420, 1.6080, 3.1623) -- (0.4880, 1.6080, 3.1682) -- (0.4880, 1.6620, 3.1673) -- (0.4420, 1.6620, 3.1615) -- cycle;
\fill[blue!34.4, opacity=0.7] (0.4420, 1.6620, 3.1615) -- (0.4880, 1.6620, 3.1673) -- (0.4880, 1.7160, 3.1662) -- (0.4420, 1.7160, 3.1604) -- cycle;
\fill[blue!33.2, opacity=0.7] (0.4420, 1.7160, 3.1604) -- (0.4880, 1.7160, 3.1662) -- (0.4880, 1.7700, 3.1647) -- (0.4420, 1.7700, 3.1589) -- cycle;
\fill[blue!31.6, opacity=0.7] (0.4420, 1.7700, 3.1589) -- (0.4880, 1.7700, 3.1647) -- (0.4880, 1.8240, 3.1629) -- (0.4420, 1.8240, 3.1571) -- cycle;
\fill[blue!29.9, opacity=0.7] (0.4420, 1.8240, 3.1571) -- (0.4880, 1.8240, 3.1629) -- (0.4880, 1.8780, 3.1608) -- (0.4420, 1.8780, 3.1550) -- cycle;
\fill[blue!28.4, opacity=0.7] (0.4420, 1.8780, 3.1550) -- (0.4880, 1.8780, 3.1608) -- (0.4880, 1.9320, 3.1584) -- (0.4420, 1.9320, 3.1526) -- cycle;
\fill[blue!27.3, opacity=0.7] (0.4420, 1.9320, 3.1526) -- (0.4880, 1.9320, 3.1584) -- (0.4880, 1.9860, 3.1557) -- (0.4420, 1.9860, 3.1499) -- cycle;
\fill[blue!27.1, opacity=0.7] (0.4420, 1.9860, 3.1499) -- (0.4880, 1.9860, 3.1557) -- (0.4880, 2.0400, 3.1527) -- (0.4420, 2.0400, 3.1469) -- cycle;
\fill[blue!28.1, opacity=0.7] (0.4420, 2.0400, 3.1469) -- (0.4880, 2.0400, 3.1527) -- (0.4880, 2.0940, 3.1494) -- (0.4420, 2.0940, 3.1436) -- cycle;
\fill[blue!30.7, opacity=0.7] (0.4420, 2.0940, 3.1436) -- (0.4880, 2.0940, 3.1494) -- (0.4880, 2.1480, 3.1459) -- (0.4420, 2.1480, 3.1401) -- cycle;
\fill[blue!35.8, opacity=0.7] (0.4420, 2.1480, 3.1401) -- (0.4880, 2.1480, 3.1459) -- (0.4880, 2.2020, 3.1421) -- (0.4420, 2.2020, 3.1363) -- cycle;
\fill[blue!43.6, opacity=0.7] (0.4420, 2.2020, 3.1363) -- (0.4880, 2.2020, 3.1421) -- (0.4880, 2.2560, 3.1380) -- (0.4420, 2.2560, 3.1322) -- cycle;
\fill[blue!53.0, opacity=0.7] (0.4420, 2.2560, 3.1322) -- (0.4880, 2.2560, 3.1380) -- (0.4880, 2.3100, 3.1337) -- (0.4420, 2.3100, 3.1279) -- cycle;
\fill[blue!60.4, opacity=0.7] (0.4420, 2.3100, 3.1279) -- (0.4880, 2.3100, 3.1337) -- (0.4880, 2.3640, 3.1291) -- (0.4420, 2.3640, 3.1233) -- cycle;
\fill[blue!63.4, opacity=0.7] (0.4420, 2.3640, 3.1233) -- (0.4880, 2.3640, 3.1291) -- (0.4880, 2.4180, 3.1243) -- (0.4420, 2.4180, 3.1185) -- cycle;
\fill[blue!63.3, opacity=0.7] (0.4420, 2.4180, 3.1185) -- (0.4880, 2.4180, 3.1243) -- (0.4880, 2.4720, 3.1193) -- (0.4420, 2.4720, 3.1135) -- cycle;
\fill[blue!63.3, opacity=0.7] (0.4420, 2.4720, 3.1135) -- (0.4880, 2.4720, 3.1193) -- (0.4880, 2.5260, 3.1142) -- (0.4420, 2.5260, 3.1084) -- cycle;
\fill[blue!63.3, opacity=0.7] (0.4420, 2.5260, 3.1084) -- (0.4880, 2.5260, 3.1142) -- (0.4880, 2.5800, 3.1088) -- (0.4420, 2.5800, 3.1030) -- cycle;
\fill[blue!57.3, opacity=0.7] (0.4420, 2.5800, 3.1030) -- (0.4880, 2.5800, 3.1088) -- (0.4880, 2.6340, 3.1033) -- (0.4420, 2.6340, 3.0975) -- cycle;
\fill[blue!41.4, opacity=0.7] (0.4420, 2.6340, 3.0975) -- (0.4880, 2.6340, 3.1033) -- (0.4880, 2.6880, 3.0976) -- (0.4420, 2.6880, 3.0918) -- cycle;
\fill[blue!25.6, opacity=0.7] (0.4420, 2.6880, 3.0918) -- (0.4880, 2.6880, 3.0976) -- (0.4880, 2.7420, 3.0918) -- (0.4420, 2.7420, 3.0860) -- cycle;
\fill[blue!18.3, opacity=0.7] (0.4420, 2.7420, 3.0860) -- (0.4880, 2.7420, 3.0918) -- (0.4880, 2.7960, 3.0859) -- (0.4420, 2.7960, 3.0801) -- cycle;
\fill[blue!16.3, opacity=0.7] (0.4420, 2.7960, 3.0801) -- (0.4880, 2.7960, 3.0859) -- (0.4880, 2.8500, 3.0799) -- (0.4420, 2.8500, 3.0741) -- cycle;
\fill[blue!16.0, opacity=0.7] (0.4420, 2.8500, 3.0741) -- (0.4880, 2.8500, 3.0799) -- (0.4880, 2.9040, 3.0738) -- (0.4420, 2.9040, 3.0680) -- cycle;
\fill[blue!16.7, opacity=0.7] (0.4420, 2.9040, 3.0680) -- (0.4880, 2.9040, 3.0738) -- (0.4880, 2.9580, 3.0676) -- (0.4420, 2.9580, 3.0618) -- cycle;
\fill[blue!20.1, opacity=0.7] (0.4420, 2.9580, 3.0618) -- (0.4880, 2.9580, 3.0676) -- (0.4880, 3.0120, 3.0614) -- (0.4420, 3.0120, 3.0555) -- cycle;
\fill[blue!29.0, opacity=0.7] (0.4420, 3.0120, 3.0555) -- (0.4880, 3.0120, 3.0614) -- (0.4880, 3.0660, 3.0551) -- (0.4420, 3.0660, 3.0493) -- cycle;
\fill[blue!40.8, opacity=0.7] (0.4420, 3.0660, 3.0493) -- (0.4880, 3.0660, 3.0551) -- (0.4880, 3.1200, 3.0488) -- (0.4420, 3.1200, 3.0430) -- cycle;
\fill[blue!15.7, opacity=0.7] (0.4880, -0.1200, 3.0488) -- (0.5340, -0.1200, 3.0545) -- (0.5340, -0.0660, 3.0608) -- (0.4880, -0.0660, 3.0551) -- cycle;
\fill[blue!18.8, opacity=0.7] (0.4880, -0.0660, 3.0551) -- (0.5340, -0.0660, 3.0608) -- (0.5340, -0.0120, 3.0670) -- (0.4880, -0.0120, 3.0614) -- cycle;
\fill[blue!32.4, opacity=0.7] (0.4880, -0.0120, 3.0614) -- (0.5340, -0.0120, 3.0670) -- (0.5340, 0.0420, 3.0733) -- (0.4880, 0.0420, 3.0676) -- cycle;
\fill[blue!51.9, opacity=0.7] (0.4880, 0.0420, 3.0676) -- (0.5340, 0.0420, 3.0733) -- (0.5340, 0.0960, 3.0794) -- (0.4880, 0.0960, 3.0738) -- cycle;
\fill[blue!59.9, opacity=0.7] (0.4880, 0.0960, 3.0738) -- (0.5340, 0.0960, 3.0794) -- (0.5340, 0.1500, 3.0855) -- (0.4880, 0.1500, 3.0799) -- cycle;
\fill[blue!58.2, opacity=0.7] (0.4880, 0.1500, 3.0799) -- (0.5340, 0.1500, 3.0855) -- (0.5340, 0.2040, 3.0916) -- (0.4880, 0.2040, 3.0859) -- cycle;
\fill[blue!46.8, opacity=0.7] (0.4880, 0.2040, 3.0859) -- (0.5340, 0.2040, 3.0916) -- (0.5340, 0.2580, 3.0975) -- (0.4880, 0.2580, 3.0918) -- cycle;
\fill[blue!30.4, opacity=0.7] (0.4880, 0.2580, 3.0918) -- (0.5340, 0.2580, 3.0975) -- (0.5340, 0.3120, 3.1033) -- (0.4880, 0.3120, 3.0976) -- cycle;
\fill[blue!21.1, opacity=0.7] (0.4880, 0.3120, 3.0976) -- (0.5340, 0.3120, 3.1033) -- (0.5340, 0.3660, 3.1090) -- (0.4880, 0.3660, 3.1033) -- cycle;
\fill[blue!18.5, opacity=0.7] (0.4880, 0.3660, 3.1033) -- (0.5340, 0.3660, 3.1090) -- (0.5340, 0.4200, 3.1145) -- (0.4880, 0.4200, 3.1088) -- cycle;
\fill[blue!19.5, opacity=0.7] (0.4880, 0.4200, 3.1088) -- (0.5340, 0.4200, 3.1145) -- (0.5340, 0.4740, 3.1198) -- (0.4880, 0.4740, 3.1142) -- cycle;
\fill[blue!25.6, opacity=0.7] (0.4880, 0.4740, 3.1142) -- (0.5340, 0.4740, 3.1198) -- (0.5340, 0.5280, 3.1250) -- (0.4880, 0.5280, 3.1193) -- cycle;
\fill[blue!41.4, opacity=0.7] (0.4880, 0.5280, 3.1193) -- (0.5340, 0.5280, 3.1250) -- (0.5340, 0.5820, 3.1300) -- (0.4880, 0.5820, 3.1243) -- cycle;
\fill[blue!59.3, opacity=0.7] (0.4880, 0.5820, 3.1243) -- (0.5340, 0.5820, 3.1300) -- (0.5340, 0.6360, 3.1348) -- (0.4880, 0.6360, 3.1291) -- cycle;
\fill[blue!63.2, opacity=0.7] (0.4880, 0.6360, 3.1291) -- (0.5340, 0.6360, 3.1348) -- (0.5340, 0.6900, 3.1393) -- (0.4880, 0.6900, 3.1337) -- cycle;
\fill[blue!58.9, opacity=0.7] (0.4880, 0.6900, 3.1337) -- (0.5340, 0.6900, 3.1393) -- (0.5340, 0.7440, 3.1437) -- (0.4880, 0.7440, 3.1380) -- cycle;
\fill[blue!57.8, opacity=0.7] (0.4880, 0.7440, 3.1380) -- (0.5340, 0.7440, 3.1437) -- (0.5340, 0.7980, 3.1477) -- (0.4880, 0.7980, 3.1421) -- cycle;
\fill[blue!61.2, opacity=0.7] (0.4880, 0.7980, 3.1421) -- (0.5340, 0.7980, 3.1477) -- (0.5340, 0.8520, 3.1516) -- (0.4880, 0.8520, 3.1459) -- cycle;
\fill[blue!63.5, opacity=0.7] (0.4880, 0.8520, 3.1459) -- (0.5340, 0.8520, 3.1516) -- (0.5340, 0.9060, 3.1551) -- (0.4880, 0.9060, 3.1494) -- cycle;
\fill[blue!58.7, opacity=0.7] (0.4880, 0.9060, 3.1494) -- (0.5340, 0.9060, 3.1551) -- (0.5340, 0.9600, 3.1584) -- (0.4880, 0.9600, 3.1527) -- cycle;
\fill[blue!48.6, opacity=0.7] (0.4880, 0.9600, 3.1527) -- (0.5340, 0.9600, 3.1584) -- (0.5340, 1.0140, 3.1614) -- (0.4880, 1.0140, 3.1557) -- cycle;
\fill[blue!39.3, opacity=0.7] (0.4880, 1.0140, 3.1557) -- (0.5340, 1.0140, 3.1614) -- (0.5340, 1.0680, 3.1641) -- (0.4880, 1.0680, 3.1584) -- cycle;
\fill[blue!33.8, opacity=0.7] (0.4880, 1.0680, 3.1584) -- (0.5340, 1.0680, 3.1641) -- (0.5340, 1.1220, 3.1665) -- (0.4880, 1.1220, 3.1608) -- cycle;
\fill[blue!31.9, opacity=0.7] (0.4880, 1.1220, 3.1608) -- (0.5340, 1.1220, 3.1665) -- (0.5340, 1.1760, 3.1686) -- (0.4880, 1.1760, 3.1629) -- cycle;
\fill[blue!32.6, opacity=0.7] (0.4880, 1.1760, 3.1629) -- (0.5340, 1.1760, 3.1686) -- (0.5340, 1.2300, 3.1704) -- (0.4880, 1.2300, 3.1647) -- cycle;
\fill[blue!35.3, opacity=0.7] (0.4880, 1.2300, 3.1647) -- (0.5340, 1.2300, 3.1704) -- (0.5340, 1.2840, 3.1719) -- (0.4880, 1.2840, 3.1662) -- cycle;
\fill[blue!39.5, opacity=0.7] (0.4880, 1.2840, 3.1662) -- (0.5340, 1.2840, 3.1719) -- (0.5340, 1.3380, 3.1730) -- (0.4880, 1.3380, 3.1673) -- cycle;
\fill[blue!44.4, opacity=0.7] (0.4880, 1.3380, 3.1673) -- (0.5340, 1.3380, 3.1730) -- (0.5340, 1.3920, 3.1738) -- (0.4880, 1.3920, 3.1682) -- cycle;
\fill[blue!49.0, opacity=0.7] (0.4880, 1.3920, 3.1682) -- (0.5340, 1.3920, 3.1738) -- (0.5340, 1.4460, 3.1743) -- (0.4880, 1.4460, 3.1686) -- cycle;
\fill[blue!52.8, opacity=0.7] (0.4880, 1.4460, 3.1686) -- (0.5340, 1.4460, 3.1743) -- (0.5340, 1.5000, 3.1745) -- (0.4880, 1.5000, 3.1688) -- cycle;
\fill[blue!55.4, opacity=0.7] (0.4880, 1.5000, 3.1688) -- (0.5340, 1.5000, 3.1745) -- (0.5340, 1.5540, 3.1743) -- (0.4880, 1.5540, 3.1686) -- cycle;
\fill[blue!56.7, opacity=0.7] (0.4880, 1.5540, 3.1686) -- (0.5340, 1.5540, 3.1743) -- (0.5340, 1.6080, 3.1738) -- (0.4880, 1.6080, 3.1682) -- cycle;
\fill[blue!57.1, opacity=0.7] (0.4880, 1.6080, 3.1682) -- (0.5340, 1.6080, 3.1738) -- (0.5340, 1.6620, 3.1730) -- (0.4880, 1.6620, 3.1673) -- cycle;
\fill[blue!56.4, opacity=0.7] (0.4880, 1.6620, 3.1673) -- (0.5340, 1.6620, 3.1730) -- (0.5340, 1.7160, 3.1719) -- (0.4880, 1.7160, 3.1662) -- cycle;
\fill[blue!54.6, opacity=0.7] (0.4880, 1.7160, 3.1662) -- (0.5340, 1.7160, 3.1719) -- (0.5340, 1.7700, 3.1704) -- (0.4880, 1.7700, 3.1647) -- cycle;
\fill[blue!51.5, opacity=0.7] (0.4880, 1.7700, 3.1647) -- (0.5340, 1.7700, 3.1704) -- (0.5340, 1.8240, 3.1686) -- (0.4880, 1.8240, 3.1629) -- cycle;
\fill[blue!47.2, opacity=0.7] (0.4880, 1.8240, 3.1629) -- (0.5340, 1.8240, 3.1686) -- (0.5340, 1.8780, 3.1665) -- (0.4880, 1.8780, 3.1608) -- cycle;
\fill[blue!42.0, opacity=0.7] (0.4880, 1.8780, 3.1608) -- (0.5340, 1.8780, 3.1665) -- (0.5340, 1.9320, 3.1641) -- (0.4880, 1.9320, 3.1584) -- cycle;
\fill[blue!36.6, opacity=0.7] (0.4880, 1.9320, 3.1584) -- (0.5340, 1.9320, 3.1641) -- (0.5340, 1.9860, 3.1614) -- (0.4880, 1.9860, 3.1557) -- cycle;
\fill[blue!31.8, opacity=0.7] (0.4880, 1.9860, 3.1557) -- (0.5340, 1.9860, 3.1614) -- (0.5340, 2.0400, 3.1584) -- (0.4880, 2.0400, 3.1527) -- cycle;
\fill[blue!28.4, opacity=0.7] (0.4880, 2.0400, 3.1527) -- (0.5340, 2.0400, 3.1584) -- (0.5340, 2.0940, 3.1551) -- (0.4880, 2.0940, 3.1494) -- cycle;
\fill[blue!26.6, opacity=0.7] (0.4880, 2.0940, 3.1494) -- (0.5340, 2.0940, 3.1551) -- (0.5340, 2.1480, 3.1516) -- (0.4880, 2.1480, 3.1459) -- cycle;
\fill[blue!26.6, opacity=0.7] (0.4880, 2.1480, 3.1459) -- (0.5340, 2.1480, 3.1516) -- (0.5340, 2.2020, 3.1477) -- (0.4880, 2.2020, 3.1421) -- cycle;
\fill[blue!28.9, opacity=0.7] (0.4880, 2.2020, 3.1421) -- (0.5340, 2.2020, 3.1477) -- (0.5340, 2.2560, 3.1437) -- (0.4880, 2.2560, 3.1380) -- cycle;
\fill[blue!34.5, opacity=0.7] (0.4880, 2.2560, 3.1380) -- (0.5340, 2.2560, 3.1437) -- (0.5340, 2.3100, 3.1393) -- (0.4880, 2.3100, 3.1337) -- cycle;
\fill[blue!43.8, opacity=0.7] (0.4880, 2.3100, 3.1337) -- (0.5340, 2.3100, 3.1393) -- (0.5340, 2.3640, 3.1348) -- (0.4880, 2.3640, 3.1291) -- cycle;
\fill[blue!54.6, opacity=0.7] (0.4880, 2.3640, 3.1291) -- (0.5340, 2.3640, 3.1348) -- (0.5340, 2.4180, 3.1300) -- (0.4880, 2.4180, 3.1243) -- cycle;
\fill[blue!61.8, opacity=0.7] (0.4880, 2.4180, 3.1243) -- (0.5340, 2.4180, 3.1300) -- (0.5340, 2.4720, 3.1250) -- (0.4880, 2.4720, 3.1193) -- cycle;
\fill[blue!63.6, opacity=0.7] (0.4880, 2.4720, 3.1193) -- (0.5340, 2.4720, 3.1250) -- (0.5340, 2.5260, 3.1198) -- (0.4880, 2.5260, 3.1142) -- cycle;
\fill[blue!63.3, opacity=0.7] (0.4880, 2.5260, 3.1142) -- (0.5340, 2.5260, 3.1198) -- (0.5340, 2.5800, 3.1145) -- (0.4880, 2.5800, 3.1088) -- cycle;
\fill[blue!63.5, opacity=0.7] (0.4880, 2.5800, 3.1088) -- (0.5340, 2.5800, 3.1145) -- (0.5340, 2.6340, 3.1090) -- (0.4880, 2.6340, 3.1033) -- cycle;
\fill[blue!59.5, opacity=0.7] (0.4880, 2.6340, 3.1033) -- (0.5340, 2.6340, 3.1090) -- (0.5340, 2.6880, 3.1033) -- (0.4880, 2.6880, 3.0976) -- cycle;
\fill[blue!44.5, opacity=0.7] (0.4880, 2.6880, 3.0976) -- (0.5340, 2.6880, 3.1033) -- (0.5340, 2.7420, 3.0975) -- (0.4880, 2.7420, 3.0918) -- cycle;
\fill[blue!27.0, opacity=0.7] (0.4880, 2.7420, 3.0918) -- (0.5340, 2.7420, 3.0975) -- (0.5340, 2.7960, 3.0916) -- (0.4880, 2.7960, 3.0859) -- cycle;
\fill[blue!18.5, opacity=0.7] (0.4880, 2.7960, 3.0859) -- (0.5340, 2.7960, 3.0916) -- (0.5340, 2.8500, 3.0855) -- (0.4880, 2.8500, 3.0799) -- cycle;
\fill[blue!16.3, opacity=0.7] (0.4880, 2.8500, 3.0799) -- (0.5340, 2.8500, 3.0855) -- (0.5340, 2.9040, 3.0794) -- (0.4880, 2.9040, 3.0738) -- cycle;
\fill[blue!15.9, opacity=0.7] (0.4880, 2.9040, 3.0738) -- (0.5340, 2.9040, 3.0794) -- (0.5340, 2.9580, 3.0733) -- (0.4880, 2.9580, 3.0676) -- cycle;
\fill[blue!16.8, opacity=0.7] (0.4880, 2.9580, 3.0676) -- (0.5340, 2.9580, 3.0733) -- (0.5340, 3.0120, 3.0670) -- (0.4880, 3.0120, 3.0614) -- cycle;
\fill[blue!20.4, opacity=0.7] (0.4880, 3.0120, 3.0614) -- (0.5340, 3.0120, 3.0670) -- (0.5340, 3.0660, 3.0608) -- (0.4880, 3.0660, 3.0551) -- cycle;
\fill[blue!30.1, opacity=0.7] (0.4880, 3.0660, 3.0551) -- (0.5340, 3.0660, 3.0608) -- (0.5340, 3.1200, 3.0545) -- (0.4880, 3.1200, 3.0488) -- cycle;
\fill[blue!16.9, opacity=0.7] (0.5340, -0.1200, 3.0545) -- (0.5800, -0.1200, 3.0600) -- (0.5800, -0.0660, 3.0663) -- (0.5340, -0.0660, 3.0608) -- cycle;
\fill[blue!26.0, opacity=0.7] (0.5340, -0.0660, 3.0608) -- (0.5800, -0.0660, 3.0663) -- (0.5800, -0.0120, 3.0725) -- (0.5340, -0.0120, 3.0670) -- cycle;
\fill[blue!46.3, opacity=0.7] (0.5340, -0.0120, 3.0670) -- (0.5800, -0.0120, 3.0725) -- (0.5800, 0.0420, 3.0788) -- (0.5340, 0.0420, 3.0733) -- cycle;
\fill[blue!58.9, opacity=0.7] (0.5340, 0.0420, 3.0733) -- (0.5800, 0.0420, 3.0788) -- (0.5800, 0.0960, 3.0849) -- (0.5340, 0.0960, 3.0794) -- cycle;
\fill[blue!59.8, opacity=0.7] (0.5340, 0.0960, 3.0794) -- (0.5800, 0.0960, 3.0849) -- (0.5800, 0.1500, 3.0911) -- (0.5340, 0.1500, 3.0855) -- cycle;
\fill[blue!51.2, opacity=0.7] (0.5340, 0.1500, 3.0855) -- (0.5800, 0.1500, 3.0911) -- (0.5800, 0.2040, 3.0971) -- (0.5340, 0.2040, 3.0916) -- cycle;
\fill[blue!34.2, opacity=0.7] (0.5340, 0.2040, 3.0916) -- (0.5800, 0.2040, 3.0971) -- (0.5800, 0.2580, 3.1030) -- (0.5340, 0.2580, 3.0975) -- cycle;
\fill[blue!22.5, opacity=0.7] (0.5340, 0.2580, 3.0975) -- (0.5800, 0.2580, 3.1030) -- (0.5800, 0.3120, 3.1088) -- (0.5340, 0.3120, 3.1033) -- cycle;
\fill[blue!18.8, opacity=0.7] (0.5340, 0.3120, 3.1033) -- (0.5800, 0.3120, 3.1088) -- (0.5800, 0.3660, 3.1145) -- (0.5340, 0.3660, 3.1090) -- cycle;
\fill[blue!19.3, opacity=0.7] (0.5340, 0.3660, 3.1090) -- (0.5800, 0.3660, 3.1145) -- (0.5800, 0.4200, 3.1200) -- (0.5340, 0.4200, 3.1145) -- cycle;
\fill[blue!25.0, opacity=0.7] (0.5340, 0.4200, 3.1145) -- (0.5800, 0.4200, 3.1200) -- (0.5800, 0.4740, 3.1254) -- (0.5340, 0.4740, 3.1198) -- cycle;
\fill[blue!41.0, opacity=0.7] (0.5340, 0.4740, 3.1198) -- (0.5800, 0.4740, 3.1254) -- (0.5800, 0.5280, 3.1305) -- (0.5340, 0.5280, 3.1250) -- cycle;
\fill[blue!59.8, opacity=0.7] (0.5340, 0.5280, 3.1250) -- (0.5800, 0.5280, 3.1305) -- (0.5800, 0.5820, 3.1355) -- (0.5340, 0.5820, 3.1300) -- cycle;
\fill[blue!62.8, opacity=0.7] (0.5340, 0.5820, 3.1300) -- (0.5800, 0.5820, 3.1355) -- (0.5800, 0.6360, 3.1403) -- (0.5340, 0.6360, 3.1348) -- cycle;
\fill[blue!58.0, opacity=0.7] (0.5340, 0.6360, 3.1348) -- (0.5800, 0.6360, 3.1403) -- (0.5800, 0.6900, 3.1449) -- (0.5340, 0.6900, 3.1393) -- cycle;
\fill[blue!57.8, opacity=0.7] (0.5340, 0.6900, 3.1393) -- (0.5800, 0.6900, 3.1449) -- (0.5800, 0.7440, 3.1492) -- (0.5340, 0.7440, 3.1437) -- cycle;
\fill[blue!62.1, opacity=0.7] (0.5340, 0.7440, 3.1437) -- (0.5800, 0.7440, 3.1492) -- (0.5800, 0.7980, 3.1533) -- (0.5340, 0.7980, 3.1477) -- cycle;
\fill[blue!62.8, opacity=0.7] (0.5340, 0.7980, 3.1477) -- (0.5800, 0.7980, 3.1533) -- (0.5800, 0.8520, 3.1571) -- (0.5340, 0.8520, 3.1516) -- cycle;
\fill[blue!54.4, opacity=0.7] (0.5340, 0.8520, 3.1516) -- (0.5800, 0.8520, 3.1571) -- (0.5800, 0.9060, 3.1606) -- (0.5340, 0.9060, 3.1551) -- cycle;
\fill[blue!43.0, opacity=0.7] (0.5340, 0.9060, 3.1551) -- (0.5800, 0.9060, 3.1606) -- (0.5800, 0.9600, 3.1639) -- (0.5340, 0.9600, 3.1584) -- cycle;
\fill[blue!35.3, opacity=0.7] (0.5340, 0.9600, 3.1584) -- (0.5800, 0.9600, 3.1639) -- (0.5800, 1.0140, 3.1669) -- (0.5340, 1.0140, 3.1614) -- cycle;
\fill[blue!32.6, opacity=0.7] (0.5340, 1.0140, 3.1614) -- (0.5800, 1.0140, 3.1669) -- (0.5800, 1.0680, 3.1696) -- (0.5340, 1.0680, 3.1641) -- cycle;
\fill[blue!34.0, opacity=0.7] (0.5340, 1.0680, 3.1641) -- (0.5800, 1.0680, 3.1696) -- (0.5800, 1.1220, 3.1720) -- (0.5340, 1.1220, 3.1665) -- cycle;
\fill[blue!38.9, opacity=0.7] (0.5340, 1.1220, 3.1665) -- (0.5800, 1.1220, 3.1720) -- (0.5800, 1.1760, 3.1741) -- (0.5340, 1.1760, 3.1686) -- cycle;
\fill[blue!46.4, opacity=0.7] (0.5340, 1.1760, 3.1686) -- (0.5800, 1.1760, 3.1741) -- (0.5800, 1.2300, 3.1759) -- (0.5340, 1.2300, 3.1704) -- cycle;
\fill[blue!54.4, opacity=0.7] (0.5340, 1.2300, 3.1704) -- (0.5800, 1.2300, 3.1759) -- (0.5800, 1.2840, 3.1774) -- (0.5340, 1.2840, 3.1719) -- cycle;
\fill[blue!60.5, opacity=0.7] (0.5340, 1.2840, 3.1719) -- (0.5800, 1.2840, 3.1774) -- (0.5800, 1.3380, 3.1785) -- (0.5340, 1.3380, 3.1730) -- cycle;
\fill[blue!63.3, opacity=0.7] (0.5340, 1.3380, 3.1730) -- (0.5800, 1.3380, 3.1785) -- (0.5800, 1.3920, 3.1793) -- (0.5340, 1.3920, 3.1738) -- cycle;
\fill[blue!63.3, opacity=0.7] (0.5340, 1.3920, 3.1738) -- (0.5800, 1.3920, 3.1793) -- (0.5800, 1.4460, 3.1798) -- (0.5340, 1.4460, 3.1743) -- cycle;
\fill[blue!61.9, opacity=0.7] (0.5340, 1.4460, 3.1743) -- (0.5800, 1.4460, 3.1798) -- (0.5800, 1.5000, 3.1800) -- (0.5340, 1.5000, 3.1745) -- cycle;
\fill[blue!60.3, opacity=0.7] (0.5340, 1.5000, 3.1745) -- (0.5800, 1.5000, 3.1800) -- (0.5800, 1.5540, 3.1798) -- (0.5340, 1.5540, 3.1743) -- cycle;
\fill[blue!59.2, opacity=0.7] (0.5340, 1.5540, 3.1743) -- (0.5800, 1.5540, 3.1798) -- (0.5800, 1.6080, 3.1793) -- (0.5340, 1.6080, 3.1738) -- cycle;
\fill[blue!58.9, opacity=0.7] (0.5340, 1.6080, 3.1738) -- (0.5800, 1.6080, 3.1793) -- (0.5800, 1.6620, 3.1785) -- (0.5340, 1.6620, 3.1730) -- cycle;
\fill[blue!59.5, opacity=0.7] (0.5340, 1.6620, 3.1730) -- (0.5800, 1.6620, 3.1785) -- (0.5800, 1.7160, 3.1774) -- (0.5340, 1.7160, 3.1719) -- cycle;
\fill[blue!60.8, opacity=0.7] (0.5340, 1.7160, 3.1719) -- (0.5800, 1.7160, 3.1774) -- (0.5800, 1.7700, 3.1759) -- (0.5340, 1.7700, 3.1704) -- cycle;
\fill[blue!62.4, opacity=0.7] (0.5340, 1.7700, 3.1704) -- (0.5800, 1.7700, 3.1759) -- (0.5800, 1.8240, 3.1741) -- (0.5340, 1.8240, 3.1686) -- cycle;
\fill[blue!63.5, opacity=0.7] (0.5340, 1.8240, 3.1686) -- (0.5800, 1.8240, 3.1741) -- (0.5800, 1.8780, 3.1720) -- (0.5340, 1.8780, 3.1665) -- cycle;
\fill[blue!62.9, opacity=0.7] (0.5340, 1.8780, 3.1665) -- (0.5800, 1.8780, 3.1720) -- (0.5800, 1.9320, 3.1696) -- (0.5340, 1.9320, 3.1641) -- cycle;
\fill[blue!59.2, opacity=0.7] (0.5340, 1.9320, 3.1641) -- (0.5800, 1.9320, 3.1696) -- (0.5800, 1.9860, 3.1669) -- (0.5340, 1.9860, 3.1614) -- cycle;
\fill[blue!52.2, opacity=0.7] (0.5340, 1.9860, 3.1614) -- (0.5800, 1.9860, 3.1669) -- (0.5800, 2.0400, 3.1639) -- (0.5340, 2.0400, 3.1584) -- cycle;
\fill[blue!43.2, opacity=0.7] (0.5340, 2.0400, 3.1584) -- (0.5800, 2.0400, 3.1639) -- (0.5800, 2.0940, 3.1606) -- (0.5340, 2.0940, 3.1551) -- cycle;
\fill[blue!34.9, opacity=0.7] (0.5340, 2.0940, 3.1551) -- (0.5800, 2.0940, 3.1606) -- (0.5800, 2.1480, 3.1571) -- (0.5340, 2.1480, 3.1516) -- cycle;
\fill[blue!29.1, opacity=0.7] (0.5340, 2.1480, 3.1516) -- (0.5800, 2.1480, 3.1571) -- (0.5800, 2.2020, 3.1533) -- (0.5340, 2.2020, 3.1477) -- cycle;
\fill[blue!26.2, opacity=0.7] (0.5340, 2.2020, 3.1477) -- (0.5800, 2.2020, 3.1533) -- (0.5800, 2.2560, 3.1492) -- (0.5340, 2.2560, 3.1437) -- cycle;
\fill[blue!26.1, opacity=0.7] (0.5340, 2.2560, 3.1437) -- (0.5800, 2.2560, 3.1492) -- (0.5800, 2.3100, 3.1449) -- (0.5340, 2.3100, 3.1393) -- cycle;
\fill[blue!29.1, opacity=0.7] (0.5340, 2.3100, 3.1393) -- (0.5800, 2.3100, 3.1449) -- (0.5800, 2.3640, 3.1403) -- (0.5340, 2.3640, 3.1348) -- cycle;
\fill[blue!36.5, opacity=0.7] (0.5340, 2.3640, 3.1348) -- (0.5800, 2.3640, 3.1403) -- (0.5800, 2.4180, 3.1355) -- (0.5340, 2.4180, 3.1300) -- cycle;
\fill[blue!48.1, opacity=0.7] (0.5340, 2.4180, 3.1300) -- (0.5800, 2.4180, 3.1355) -- (0.5800, 2.4720, 3.1305) -- (0.5340, 2.4720, 3.1250) -- cycle;
\fill[blue!59.0, opacity=0.7] (0.5340, 2.4720, 3.1250) -- (0.5800, 2.4720, 3.1305) -- (0.5800, 2.5260, 3.1254) -- (0.5340, 2.5260, 3.1198) -- cycle;
\fill[blue!63.4, opacity=0.7] (0.5340, 2.5260, 3.1198) -- (0.5800, 2.5260, 3.1254) -- (0.5800, 2.5800, 3.1200) -- (0.5340, 2.5800, 3.1145) -- cycle;
\fill[blue!63.4, opacity=0.7] (0.5340, 2.5800, 3.1145) -- (0.5800, 2.5800, 3.1200) -- (0.5800, 2.6340, 3.1145) -- (0.5340, 2.6340, 3.1090) -- cycle;
\fill[blue!63.6, opacity=0.7] (0.5340, 2.6340, 3.1090) -- (0.5800, 2.6340, 3.1145) -- (0.5800, 2.6880, 3.1088) -- (0.5340, 2.6880, 3.1033) -- cycle;
\fill[blue!60.5, opacity=0.7] (0.5340, 2.6880, 3.1033) -- (0.5800, 2.6880, 3.1088) -- (0.5800, 2.7420, 3.1030) -- (0.5340, 2.7420, 3.0975) -- cycle;
\fill[blue!45.7, opacity=0.7] (0.5340, 2.7420, 3.0975) -- (0.5800, 2.7420, 3.1030) -- (0.5800, 2.7960, 3.0971) -- (0.5340, 2.7960, 3.0916) -- cycle;
\fill[blue!27.2, opacity=0.7] (0.5340, 2.7960, 3.0916) -- (0.5800, 2.7960, 3.0971) -- (0.5800, 2.8500, 3.0911) -- (0.5340, 2.8500, 3.0855) -- cycle;
\fill[blue!18.4, opacity=0.7] (0.5340, 2.8500, 3.0855) -- (0.5800, 2.8500, 3.0911) -- (0.5800, 2.9040, 3.0849) -- (0.5340, 2.9040, 3.0794) -- cycle;
\fill[blue!16.2, opacity=0.7] (0.5340, 2.9040, 3.0794) -- (0.5800, 2.9040, 3.0849) -- (0.5800, 2.9580, 3.0788) -- (0.5340, 2.9580, 3.0733) -- cycle;
\fill[blue!15.9, opacity=0.7] (0.5340, 2.9580, 3.0733) -- (0.5800, 2.9580, 3.0788) -- (0.5800, 3.0120, 3.0725) -- (0.5340, 3.0120, 3.0670) -- cycle;
\fill[blue!16.9, opacity=0.7] (0.5340, 3.0120, 3.0670) -- (0.5800, 3.0120, 3.0725) -- (0.5800, 3.0660, 3.0663) -- (0.5340, 3.0660, 3.0608) -- cycle;
\fill[blue!21.4, opacity=0.7] (0.5340, 3.0660, 3.0608) -- (0.5800, 3.0660, 3.0663) -- (0.5800, 3.1200, 3.0600) -- (0.5340, 3.1200, 3.0545) -- cycle;
\fill[blue!20.7, opacity=0.7] (0.5800, -0.1200, 3.0600) -- (0.6260, -0.1200, 3.0654) -- (0.6260, -0.0660, 3.0716) -- (0.5800, -0.0660, 3.0663) -- cycle;
\fill[blue!38.1, opacity=0.7] (0.5800, -0.0660, 3.0663) -- (0.6260, -0.0660, 3.0716) -- (0.6260, -0.0120, 3.0779) -- (0.5800, -0.0120, 3.0725) -- cycle;
\fill[blue!56.2, opacity=0.7] (0.5800, -0.0120, 3.0725) -- (0.6260, -0.0120, 3.0779) -- (0.6260, 0.0420, 3.0841) -- (0.5800, 0.0420, 3.0788) -- cycle;
\fill[blue!60.6, opacity=0.7] (0.5800, 0.0420, 3.0788) -- (0.6260, 0.0420, 3.0841) -- (0.6260, 0.0960, 3.0903) -- (0.5800, 0.0960, 3.0849) -- cycle;
\fill[blue!55.6, opacity=0.7] (0.5800, 0.0960, 3.0849) -- (0.6260, 0.0960, 3.0903) -- (0.6260, 0.1500, 3.0964) -- (0.5800, 0.1500, 3.0911) -- cycle;
\fill[blue!39.8, opacity=0.7] (0.5800, 0.1500, 3.0911) -- (0.6260, 0.1500, 3.0964) -- (0.6260, 0.2040, 3.1024) -- (0.5800, 0.2040, 3.0971) -- cycle;
\fill[blue!25.0, opacity=0.7] (0.5800, 0.2040, 3.0971) -- (0.6260, 0.2040, 3.1024) -- (0.6260, 0.2580, 3.1084) -- (0.5800, 0.2580, 3.1030) -- cycle;
\fill[blue!19.4, opacity=0.7] (0.5800, 0.2580, 3.1030) -- (0.6260, 0.2580, 3.1084) -- (0.6260, 0.3120, 3.1142) -- (0.5800, 0.3120, 3.1088) -- cycle;
\fill[blue!19.0, opacity=0.7] (0.5800, 0.3120, 3.1088) -- (0.6260, 0.3120, 3.1142) -- (0.6260, 0.3660, 3.1198) -- (0.5800, 0.3660, 3.1145) -- cycle;
\fill[blue!23.6, opacity=0.7] (0.5800, 0.3660, 3.1145) -- (0.6260, 0.3660, 3.1198) -- (0.6260, 0.4200, 3.1254) -- (0.5800, 0.4200, 3.1200) -- cycle;
\fill[blue!38.7, opacity=0.7] (0.5800, 0.4200, 3.1200) -- (0.6260, 0.4200, 3.1254) -- (0.6260, 0.4740, 3.1307) -- (0.5800, 0.4740, 3.1254) -- cycle;
\fill[blue!58.9, opacity=0.7] (0.5800, 0.4740, 3.1254) -- (0.6260, 0.4740, 3.1307) -- (0.6260, 0.5280, 3.1359) -- (0.5800, 0.5280, 3.1305) -- cycle;
\fill[blue!62.8, opacity=0.7] (0.5800, 0.5280, 3.1305) -- (0.6260, 0.5280, 3.1359) -- (0.6260, 0.5820, 3.1409) -- (0.5800, 0.5820, 3.1355) -- cycle;
\fill[blue!57.6, opacity=0.7] (0.5800, 0.5820, 3.1355) -- (0.6260, 0.5820, 3.1409) -- (0.6260, 0.6360, 3.1457) -- (0.5800, 0.6360, 3.1403) -- cycle;
\fill[blue!57.7, opacity=0.7] (0.5800, 0.6360, 3.1403) -- (0.6260, 0.6360, 3.1457) -- (0.6260, 0.6900, 3.1502) -- (0.5800, 0.6900, 3.1449) -- cycle;
\fill[blue!62.5, opacity=0.7] (0.5800, 0.6900, 3.1449) -- (0.6260, 0.6900, 3.1502) -- (0.6260, 0.7440, 3.1545) -- (0.5800, 0.7440, 3.1492) -- cycle;
\fill[blue!61.9, opacity=0.7] (0.5800, 0.7440, 3.1492) -- (0.6260, 0.7440, 3.1545) -- (0.6260, 0.7980, 3.1586) -- (0.5800, 0.7980, 3.1533) -- cycle;
\fill[blue!51.3, opacity=0.7] (0.5800, 0.7980, 3.1533) -- (0.6260, 0.7980, 3.1586) -- (0.6260, 0.8520, 3.1624) -- (0.5800, 0.8520, 3.1571) -- cycle;
\fill[blue!39.8, opacity=0.7] (0.5800, 0.8520, 3.1571) -- (0.6260, 0.8520, 3.1624) -- (0.6260, 0.9060, 3.1660) -- (0.5800, 0.9060, 3.1606) -- cycle;
\fill[blue!34.0, opacity=0.7] (0.5800, 0.9060, 3.1606) -- (0.6260, 0.9060, 3.1660) -- (0.6260, 0.9600, 3.1693) -- (0.5800, 0.9600, 3.1639) -- cycle;
\fill[blue!34.0, opacity=0.7] (0.5800, 0.9600, 3.1639) -- (0.6260, 0.9600, 3.1693) -- (0.6260, 1.0140, 3.1723) -- (0.5800, 1.0140, 3.1669) -- cycle;
\fill[blue!39.1, opacity=0.7] (0.5800, 1.0140, 3.1669) -- (0.6260, 1.0140, 3.1723) -- (0.6260, 1.0680, 3.1750) -- (0.5800, 1.0680, 3.1696) -- cycle;
\fill[blue!48.5, opacity=0.7] (0.5800, 1.0680, 3.1696) -- (0.6260, 1.0680, 3.1750) -- (0.6260, 1.1220, 3.1774) -- (0.5800, 1.1220, 3.1720) -- cycle;
\fill[blue!58.5, opacity=0.7] (0.5800, 1.1220, 3.1720) -- (0.6260, 1.1220, 3.1774) -- (0.6260, 1.1760, 3.1795) -- (0.5800, 1.1760, 3.1741) -- cycle;
\fill[blue!63.4, opacity=0.7] (0.5800, 1.1760, 3.1741) -- (0.6260, 1.1760, 3.1795) -- (0.6260, 1.2300, 3.1813) -- (0.5800, 1.2300, 3.1759) -- cycle;
\fill[blue!61.6, opacity=0.7] (0.5800, 1.2300, 3.1759) -- (0.6260, 1.2300, 3.1813) -- (0.6260, 1.2840, 3.1827) -- (0.5800, 1.2840, 3.1774) -- cycle;
\fill[blue!56.0, opacity=0.7] (0.5800, 1.2840, 3.1774) -- (0.6260, 1.2840, 3.1827) -- (0.6260, 1.3380, 3.1839) -- (0.5800, 1.3380, 3.1785) -- cycle;
\fill[blue!50.1, opacity=0.7] (0.5800, 1.3380, 3.1785) -- (0.6260, 1.3380, 3.1839) -- (0.6260, 1.3920, 3.1847) -- (0.5800, 1.3920, 3.1793) -- cycle;
\fill[blue!45.5, opacity=0.7] (0.5800, 1.3920, 3.1793) -- (0.6260, 1.3920, 3.1847) -- (0.6260, 1.4460, 3.1852) -- (0.5800, 1.4460, 3.1798) -- cycle;
\fill[blue!42.5, opacity=0.7] (0.5800, 1.4460, 3.1798) -- (0.6260, 1.4460, 3.1852) -- (0.6260, 1.5000, 3.1854) -- (0.5800, 1.5000, 3.1800) -- cycle;
\fill[blue!40.8, opacity=0.7] (0.5800, 1.5000, 3.1800) -- (0.6260, 1.5000, 3.1854) -- (0.6260, 1.5540, 3.1852) -- (0.5800, 1.5540, 3.1798) -- cycle;
\fill[blue!40.1, opacity=0.7] (0.5800, 1.5540, 3.1798) -- (0.6260, 1.5540, 3.1852) -- (0.6260, 1.6080, 3.1847) -- (0.5800, 1.6080, 3.1793) -- cycle;
\fill[blue!40.2, opacity=0.7] (0.5800, 1.6080, 3.1793) -- (0.6260, 1.6080, 3.1847) -- (0.6260, 1.6620, 3.1839) -- (0.5800, 1.6620, 3.1785) -- cycle;
\fill[blue!40.9, opacity=0.7] (0.5800, 1.6620, 3.1785) -- (0.6260, 1.6620, 3.1839) -- (0.6260, 1.7160, 3.1827) -- (0.5800, 1.7160, 3.1774) -- cycle;
\fill[blue!42.2, opacity=0.7] (0.5800, 1.7160, 3.1774) -- (0.6260, 1.7160, 3.1827) -- (0.6260, 1.7700, 3.1813) -- (0.5800, 1.7700, 3.1759) -- cycle;
\fill[blue!44.5, opacity=0.7] (0.5800, 1.7700, 3.1759) -- (0.6260, 1.7700, 3.1813) -- (0.6260, 1.8240, 3.1795) -- (0.5800, 1.8240, 3.1741) -- cycle;
\fill[blue!47.9, opacity=0.7] (0.5800, 1.8240, 3.1741) -- (0.6260, 1.8240, 3.1795) -- (0.6260, 1.8780, 3.1774) -- (0.5800, 1.8780, 3.1720) -- cycle;
\fill[blue!52.5, opacity=0.7] (0.5800, 1.8780, 3.1720) -- (0.6260, 1.8780, 3.1774) -- (0.6260, 1.9320, 3.1750) -- (0.5800, 1.9320, 3.1696) -- cycle;
\fill[blue!58.0, opacity=0.7] (0.5800, 1.9320, 3.1696) -- (0.6260, 1.9320, 3.1750) -- (0.6260, 1.9860, 3.1723) -- (0.5800, 1.9860, 3.1669) -- cycle;
\fill[blue!62.6, opacity=0.7] (0.5800, 1.9860, 3.1669) -- (0.6260, 1.9860, 3.1723) -- (0.6260, 2.0400, 3.1693) -- (0.5800, 2.0400, 3.1639) -- cycle;
\fill[blue!63.0, opacity=0.7] (0.5800, 2.0400, 3.1639) -- (0.6260, 2.0400, 3.1693) -- (0.6260, 2.0940, 3.1660) -- (0.5800, 2.0940, 3.1606) -- cycle;
\fill[blue!56.7, opacity=0.7] (0.5800, 2.0940, 3.1606) -- (0.6260, 2.0940, 3.1660) -- (0.6260, 2.1480, 3.1624) -- (0.5800, 2.1480, 3.1571) -- cycle;
\fill[blue!45.6, opacity=0.7] (0.5800, 2.1480, 3.1571) -- (0.6260, 2.1480, 3.1624) -- (0.6260, 2.2020, 3.1586) -- (0.5800, 2.2020, 3.1533) -- cycle;
\fill[blue!34.8, opacity=0.7] (0.5800, 2.2020, 3.1533) -- (0.6260, 2.2020, 3.1586) -- (0.6260, 2.2560, 3.1545) -- (0.5800, 2.2560, 3.1492) -- cycle;
\fill[blue!28.1, opacity=0.7] (0.5800, 2.2560, 3.1492) -- (0.6260, 2.2560, 3.1545) -- (0.6260, 2.3100, 3.1502) -- (0.5800, 2.3100, 3.1449) -- cycle;
\fill[blue!25.4, opacity=0.7] (0.5800, 2.3100, 3.1449) -- (0.6260, 2.3100, 3.1502) -- (0.6260, 2.3640, 3.1457) -- (0.5800, 2.3640, 3.1403) -- cycle;
\fill[blue!26.4, opacity=0.7] (0.5800, 2.3640, 3.1403) -- (0.6260, 2.3640, 3.1457) -- (0.6260, 2.4180, 3.1409) -- (0.5800, 2.4180, 3.1355) -- cycle;
\fill[blue!31.9, opacity=0.7] (0.5800, 2.4180, 3.1355) -- (0.6260, 2.4180, 3.1409) -- (0.6260, 2.4720, 3.1359) -- (0.5800, 2.4720, 3.1305) -- cycle;
\fill[blue!42.9, opacity=0.7] (0.5800, 2.4720, 3.1305) -- (0.6260, 2.4720, 3.1359) -- (0.6260, 2.5260, 3.1307) -- (0.5800, 2.5260, 3.1254) -- cycle;
\fill[blue!55.9, opacity=0.7] (0.5800, 2.5260, 3.1254) -- (0.6260, 2.5260, 3.1307) -- (0.6260, 2.5800, 3.1254) -- (0.5800, 2.5800, 3.1200) -- cycle;
\fill[blue!62.8, opacity=0.7] (0.5800, 2.5800, 3.1200) -- (0.6260, 2.5800, 3.1254) -- (0.6260, 2.6340, 3.1198) -- (0.5800, 2.6340, 3.1145) -- cycle;
\fill[blue!63.5, opacity=0.7] (0.5800, 2.6340, 3.1145) -- (0.6260, 2.6340, 3.1198) -- (0.6260, 2.6880, 3.1142) -- (0.5800, 2.6880, 3.1088) -- cycle;
\fill[blue!63.6, opacity=0.7] (0.5800, 2.6880, 3.1088) -- (0.6260, 2.6880, 3.1142) -- (0.6260, 2.7420, 3.1084) -- (0.5800, 2.7420, 3.1030) -- cycle;
\fill[blue!60.5, opacity=0.7] (0.5800, 2.7420, 3.1030) -- (0.6260, 2.7420, 3.1084) -- (0.6260, 2.7960, 3.1024) -- (0.5800, 2.7960, 3.0971) -- cycle;
\fill[blue!45.0, opacity=0.7] (0.5800, 2.7960, 3.0971) -- (0.6260, 2.7960, 3.1024) -- (0.6260, 2.8500, 3.0964) -- (0.5800, 2.8500, 3.0911) -- cycle;
\fill[blue!26.1, opacity=0.7] (0.5800, 2.8500, 3.0911) -- (0.6260, 2.8500, 3.0964) -- (0.6260, 2.9040, 3.0903) -- (0.5800, 2.9040, 3.0849) -- cycle;
\fill[blue!17.9, opacity=0.7] (0.5800, 2.9040, 3.0849) -- (0.6260, 2.9040, 3.0903) -- (0.6260, 2.9580, 3.0841) -- (0.5800, 2.9580, 3.0788) -- cycle;
\fill[blue!16.0, opacity=0.7] (0.5800, 2.9580, 3.0788) -- (0.6260, 2.9580, 3.0841) -- (0.6260, 3.0120, 3.0779) -- (0.5800, 3.0120, 3.0725) -- cycle;
\fill[blue!16.0, opacity=0.7] (0.5800, 3.0120, 3.0725) -- (0.6260, 3.0120, 3.0779) -- (0.6260, 3.0660, 3.0716) -- (0.5800, 3.0660, 3.0663) -- cycle;
\fill[blue!17.4, opacity=0.7] (0.5800, 3.0660, 3.0663) -- (0.6260, 3.0660, 3.0716) -- (0.6260, 3.1200, 3.0654) -- (0.5800, 3.1200, 3.0600) -- cycle;
\fill[blue!28.8, opacity=0.7] (0.6260, -0.1200, 3.0654) -- (0.6720, -0.1200, 3.0705) -- (0.6720, -0.0660, 3.0768) -- (0.6260, -0.0660, 3.0716) -- cycle;
\fill[blue!50.2, opacity=0.7] (0.6260, -0.0660, 3.0716) -- (0.6720, -0.0660, 3.0768) -- (0.6720, -0.0120, 3.0831) -- (0.6260, -0.0120, 3.0779) -- cycle;
\fill[blue!60.2, opacity=0.7] (0.6260, -0.0120, 3.0779) -- (0.6720, -0.0120, 3.0831) -- (0.6720, 0.0420, 3.0893) -- (0.6260, 0.0420, 3.0841) -- cycle;
\fill[blue!59.1, opacity=0.7] (0.6260, 0.0420, 3.0841) -- (0.6720, 0.0420, 3.0893) -- (0.6720, 0.0960, 3.0955) -- (0.6260, 0.0960, 3.0903) -- cycle;
\fill[blue!46.9, opacity=0.7] (0.6260, 0.0960, 3.0903) -- (0.6720, 0.0960, 3.0955) -- (0.6720, 0.1500, 3.1016) -- (0.6260, 0.1500, 3.0964) -- cycle;
\fill[blue!29.3, opacity=0.7] (0.6260, 0.1500, 3.0964) -- (0.6720, 0.1500, 3.1016) -- (0.6720, 0.2040, 3.1076) -- (0.6260, 0.2040, 3.1024) -- cycle;
\fill[blue!20.5, opacity=0.7] (0.6260, 0.2040, 3.1024) -- (0.6720, 0.2040, 3.1076) -- (0.6720, 0.2580, 3.1135) -- (0.6260, 0.2580, 3.1084) -- cycle;
\fill[blue!18.8, opacity=0.7] (0.6260, 0.2580, 3.1084) -- (0.6720, 0.2580, 3.1135) -- (0.6720, 0.3120, 3.1193) -- (0.6260, 0.3120, 3.1142) -- cycle;
\fill[blue!21.9, opacity=0.7] (0.6260, 0.3120, 3.1142) -- (0.6720, 0.3120, 3.1193) -- (0.6720, 0.3660, 3.1250) -- (0.6260, 0.3660, 3.1198) -- cycle;
\fill[blue!34.6, opacity=0.7] (0.6260, 0.3660, 3.1198) -- (0.6720, 0.3660, 3.1250) -- (0.6720, 0.4200, 3.1305) -- (0.6260, 0.4200, 3.1254) -- cycle;
\fill[blue!56.4, opacity=0.7] (0.6260, 0.4200, 3.1254) -- (0.6720, 0.4200, 3.1305) -- (0.6720, 0.4740, 3.1359) -- (0.6260, 0.4740, 3.1307) -- cycle;
\fill[blue!63.2, opacity=0.7] (0.6260, 0.4740, 3.1307) -- (0.6720, 0.4740, 3.1359) -- (0.6720, 0.5280, 3.1411) -- (0.6260, 0.5280, 3.1359) -- cycle;
\fill[blue!57.6, opacity=0.7] (0.6260, 0.5280, 3.1359) -- (0.6720, 0.5280, 3.1411) -- (0.6720, 0.5820, 3.1461) -- (0.6260, 0.5820, 3.1409) -- cycle;
\fill[blue!57.1, opacity=0.7] (0.6260, 0.5820, 3.1409) -- (0.6720, 0.5820, 3.1461) -- (0.6720, 0.6360, 3.1508) -- (0.6260, 0.6360, 3.1457) -- cycle;
\fill[blue!62.4, opacity=0.7] (0.6260, 0.6360, 3.1457) -- (0.6720, 0.6360, 3.1508) -- (0.6720, 0.6900, 3.1554) -- (0.6260, 0.6900, 3.1502) -- cycle;
\fill[blue!61.6, opacity=0.7] (0.6260, 0.6900, 3.1502) -- (0.6720, 0.6900, 3.1554) -- (0.6720, 0.7440, 3.1597) -- (0.6260, 0.7440, 3.1545) -- cycle;
\fill[blue!49.9, opacity=0.7] (0.6260, 0.7440, 3.1545) -- (0.6720, 0.7440, 3.1597) -- (0.6720, 0.7980, 3.1638) -- (0.6260, 0.7980, 3.1586) -- cycle;
\fill[blue!38.4, opacity=0.7] (0.6260, 0.7980, 3.1586) -- (0.6720, 0.7980, 3.1638) -- (0.6720, 0.8520, 3.1676) -- (0.6260, 0.8520, 3.1624) -- cycle;
\fill[blue!34.0, opacity=0.7] (0.6260, 0.8520, 3.1624) -- (0.6720, 0.8520, 3.1676) -- (0.6720, 0.9060, 3.1712) -- (0.6260, 0.9060, 3.1660) -- cycle;
\fill[blue!36.5, opacity=0.7] (0.6260, 0.9060, 3.1660) -- (0.6720, 0.9060, 3.1712) -- (0.6720, 0.9600, 3.1745) -- (0.6260, 0.9600, 3.1693) -- cycle;
\fill[blue!45.3, opacity=0.7] (0.6260, 0.9600, 3.1693) -- (0.6720, 0.9600, 3.1745) -- (0.6720, 1.0140, 3.1775) -- (0.6260, 1.0140, 3.1723) -- cycle;
\fill[blue!57.3, opacity=0.7] (0.6260, 1.0140, 3.1723) -- (0.6720, 1.0140, 3.1775) -- (0.6720, 1.0680, 3.1802) -- (0.6260, 1.0680, 3.1750) -- cycle;
\fill[blue!63.5, opacity=0.7] (0.6260, 1.0680, 3.1750) -- (0.6720, 1.0680, 3.1802) -- (0.6720, 1.1220, 3.1826) -- (0.6260, 1.1220, 3.1774) -- cycle;
\fill[blue!59.4, opacity=0.7] (0.6260, 1.1220, 3.1774) -- (0.6720, 1.1220, 3.1826) -- (0.6720, 1.1760, 3.1847) -- (0.6260, 1.1760, 3.1795) -- cycle;
\fill[blue!50.3, opacity=0.7] (0.6260, 1.1760, 3.1795) -- (0.6720, 1.1760, 3.1847) -- (0.6720, 1.2300, 3.1864) -- (0.6260, 1.2300, 3.1813) -- cycle;
\fill[blue!42.3, opacity=0.7] (0.6260, 1.2300, 3.1813) -- (0.6720, 1.2300, 3.1864) -- (0.6720, 1.2840, 3.1879) -- (0.6260, 1.2840, 3.1827) -- cycle;
\fill[blue!37.5, opacity=0.7] (0.6260, 1.2840, 3.1827) -- (0.6720, 1.2840, 3.1879) -- (0.6720, 1.3380, 3.1891) -- (0.6260, 1.3380, 3.1839) -- cycle;
\fill[blue!35.2, opacity=0.7] (0.6260, 1.3380, 3.1839) -- (0.6720, 1.3380, 3.1891) -- (0.6720, 1.3920, 3.1899) -- (0.6260, 1.3920, 3.1847) -- cycle;
\fill[blue!34.6, opacity=0.7] (0.6260, 1.3920, 3.1847) -- (0.6720, 1.3920, 3.1899) -- (0.6720, 1.4460, 3.1904) -- (0.6260, 1.4460, 3.1852) -- cycle;
\fill[blue!35.0, opacity=0.7] (0.6260, 1.4460, 3.1852) -- (0.6720, 1.4460, 3.1904) -- (0.6720, 1.5000, 3.1905) -- (0.6260, 1.5000, 3.1854) -- cycle;
\fill[blue!35.9, opacity=0.7] (0.6260, 1.5000, 3.1854) -- (0.6720, 1.5000, 3.1905) -- (0.6720, 1.5540, 3.1904) -- (0.6260, 1.5540, 3.1852) -- cycle;
\fill[blue!36.8, opacity=0.7] (0.6260, 1.5540, 3.1852) -- (0.6720, 1.5540, 3.1904) -- (0.6720, 1.6080, 3.1899) -- (0.6260, 1.6080, 3.1847) -- cycle;
\fill[blue!37.5, opacity=0.7] (0.6260, 1.6080, 3.1847) -- (0.6720, 1.6080, 3.1899) -- (0.6720, 1.6620, 3.1891) -- (0.6260, 1.6620, 3.1839) -- cycle;
\fill[blue!37.8, opacity=0.7] (0.6260, 1.6620, 3.1839) -- (0.6720, 1.6620, 3.1891) -- (0.6720, 1.7160, 3.1879) -- (0.6260, 1.7160, 3.1827) -- cycle;
\fill[blue!37.9, opacity=0.7] (0.6260, 1.7160, 3.1827) -- (0.6720, 1.7160, 3.1879) -- (0.6720, 1.7700, 3.1864) -- (0.6260, 1.7700, 3.1813) -- cycle;
\fill[blue!37.9, opacity=0.7] (0.6260, 1.7700, 3.1813) -- (0.6720, 1.7700, 3.1864) -- (0.6720, 1.8240, 3.1847) -- (0.6260, 1.8240, 3.1795) -- cycle;
\fill[blue!38.1, opacity=0.7] (0.6260, 1.8240, 3.1795) -- (0.6720, 1.8240, 3.1847) -- (0.6720, 1.8780, 3.1826) -- (0.6260, 1.8780, 3.1774) -- cycle;
\fill[blue!39.2, opacity=0.7] (0.6260, 1.8780, 3.1774) -- (0.6720, 1.8780, 3.1826) -- (0.6720, 1.9320, 3.1802) -- (0.6260, 1.9320, 3.1750) -- cycle;
\fill[blue!41.7, opacity=0.7] (0.6260, 1.9320, 3.1750) -- (0.6720, 1.9320, 3.1802) -- (0.6720, 1.9860, 3.1775) -- (0.6260, 1.9860, 3.1723) -- cycle;
\fill[blue!46.4, opacity=0.7] (0.6260, 1.9860, 3.1723) -- (0.6720, 1.9860, 3.1775) -- (0.6720, 2.0400, 3.1745) -- (0.6260, 2.0400, 3.1693) -- cycle;
\fill[blue!53.4, opacity=0.7] (0.6260, 2.0400, 3.1693) -- (0.6720, 2.0400, 3.1745) -- (0.6720, 2.0940, 3.1712) -- (0.6260, 2.0940, 3.1660) -- cycle;
\fill[blue!60.9, opacity=0.7] (0.6260, 2.0940, 3.1660) -- (0.6720, 2.0940, 3.1712) -- (0.6720, 2.1480, 3.1676) -- (0.6260, 2.1480, 3.1624) -- cycle;
\fill[blue!63.3, opacity=0.7] (0.6260, 2.1480, 3.1624) -- (0.6720, 2.1480, 3.1676) -- (0.6720, 2.2020, 3.1638) -- (0.6260, 2.2020, 3.1586) -- cycle;
\fill[blue!55.9, opacity=0.7] (0.6260, 2.2020, 3.1586) -- (0.6720, 2.2020, 3.1638) -- (0.6720, 2.2560, 3.1597) -- (0.6260, 2.2560, 3.1545) -- cycle;
\fill[blue!42.4, opacity=0.7] (0.6260, 2.2560, 3.1545) -- (0.6720, 2.2560, 3.1597) -- (0.6720, 2.3100, 3.1554) -- (0.6260, 2.3100, 3.1502) -- cycle;
\fill[blue!31.3, opacity=0.7] (0.6260, 2.3100, 3.1502) -- (0.6720, 2.3100, 3.1554) -- (0.6720, 2.3640, 3.1508) -- (0.6260, 2.3640, 3.1457) -- cycle;
\fill[blue!25.9, opacity=0.7] (0.6260, 2.3640, 3.1457) -- (0.6720, 2.3640, 3.1508) -- (0.6720, 2.4180, 3.1461) -- (0.6260, 2.4180, 3.1409) -- cycle;
\fill[blue!25.2, opacity=0.7] (0.6260, 2.4180, 3.1409) -- (0.6720, 2.4180, 3.1461) -- (0.6720, 2.4720, 3.1411) -- (0.6260, 2.4720, 3.1359) -- cycle;
\fill[blue!29.2, opacity=0.7] (0.6260, 2.4720, 3.1359) -- (0.6720, 2.4720, 3.1411) -- (0.6720, 2.5260, 3.1359) -- (0.6260, 2.5260, 3.1307) -- cycle;
\fill[blue!39.4, opacity=0.7] (0.6260, 2.5260, 3.1307) -- (0.6720, 2.5260, 3.1359) -- (0.6720, 2.5800, 3.1305) -- (0.6260, 2.5800, 3.1254) -- cycle;
\fill[blue!53.6, opacity=0.7] (0.6260, 2.5800, 3.1254) -- (0.6720, 2.5800, 3.1305) -- (0.6720, 2.6340, 3.1250) -- (0.6260, 2.6340, 3.1198) -- cycle;
\fill[blue!62.4, opacity=0.7] (0.6260, 2.6340, 3.1198) -- (0.6720, 2.6340, 3.1250) -- (0.6720, 2.6880, 3.1193) -- (0.6260, 2.6880, 3.1142) -- cycle;
\fill[blue!63.5, opacity=0.7] (0.6260, 2.6880, 3.1142) -- (0.6720, 2.6880, 3.1193) -- (0.6720, 2.7420, 3.1135) -- (0.6260, 2.7420, 3.1084) -- cycle;
\fill[blue!63.6, opacity=0.7] (0.6260, 2.7420, 3.1084) -- (0.6720, 2.7420, 3.1135) -- (0.6720, 2.7960, 3.1076) -- (0.6260, 2.7960, 3.1024) -- cycle;
\fill[blue!59.8, opacity=0.7] (0.6260, 2.7960, 3.1024) -- (0.6720, 2.7960, 3.1076) -- (0.6720, 2.8500, 3.1016) -- (0.6260, 2.8500, 3.0964) -- cycle;
\fill[blue!42.6, opacity=0.7] (0.6260, 2.8500, 3.0964) -- (0.6720, 2.8500, 3.1016) -- (0.6720, 2.9040, 3.0955) -- (0.6260, 2.9040, 3.0903) -- cycle;
\fill[blue!24.1, opacity=0.7] (0.6260, 2.9040, 3.0903) -- (0.6720, 2.9040, 3.0955) -- (0.6720, 2.9580, 3.0893) -- (0.6260, 2.9580, 3.0841) -- cycle;
\fill[blue!17.2, opacity=0.7] (0.6260, 2.9580, 3.0841) -- (0.6720, 2.9580, 3.0893) -- (0.6720, 3.0120, 3.0831) -- (0.6260, 3.0120, 3.0779) -- cycle;
\fill[blue!15.9, opacity=0.7] (0.6260, 3.0120, 3.0779) -- (0.6720, 3.0120, 3.0831) -- (0.6720, 3.0660, 3.0768) -- (0.6260, 3.0660, 3.0716) -- cycle;
\fill[blue!16.1, opacity=0.7] (0.6260, 3.0660, 3.0716) -- (0.6720, 3.0660, 3.0768) -- (0.6720, 3.1200, 3.0705) -- (0.6260, 3.1200, 3.0654) -- cycle;
\fill[blue!40.3, opacity=0.7] (0.6720, -0.1200, 3.0705) -- (0.7180, -0.1200, 3.0755) -- (0.7180, -0.0660, 3.0818) -- (0.6720, -0.0660, 3.0768) -- cycle;
\fill[blue!57.7, opacity=0.7] (0.6720, -0.0660, 3.0768) -- (0.7180, -0.0660, 3.0818) -- (0.7180, -0.0120, 3.0881) -- (0.6720, -0.0120, 3.0831) -- cycle;
\fill[blue!60.8, opacity=0.7] (0.6720, -0.0120, 3.0831) -- (0.7180, -0.0120, 3.0881) -- (0.7180, 0.0420, 3.0943) -- (0.6720, 0.0420, 3.0893) -- cycle;
\fill[blue!54.0, opacity=0.7] (0.6720, 0.0420, 3.0893) -- (0.7180, 0.0420, 3.0943) -- (0.7180, 0.0960, 3.1005) -- (0.6720, 0.0960, 3.0955) -- cycle;
\fill[blue!36.2, opacity=0.7] (0.6720, 0.0960, 3.0955) -- (0.7180, 0.0960, 3.1005) -- (0.7180, 0.1500, 3.1066) -- (0.6720, 0.1500, 3.1016) -- cycle;
\fill[blue!22.9, opacity=0.7] (0.6720, 0.1500, 3.1016) -- (0.7180, 0.1500, 3.1066) -- (0.7180, 0.2040, 3.1126) -- (0.6720, 0.2040, 3.1076) -- cycle;
\fill[blue!19.0, opacity=0.7] (0.6720, 0.2040, 3.1076) -- (0.7180, 0.2040, 3.1126) -- (0.7180, 0.2580, 3.1185) -- (0.6720, 0.2580, 3.1135) -- cycle;
\fill[blue!20.3, opacity=0.7] (0.6720, 0.2580, 3.1135) -- (0.7180, 0.2580, 3.1185) -- (0.7180, 0.3120, 3.1243) -- (0.6720, 0.3120, 3.1193) -- cycle;
\fill[blue!29.6, opacity=0.7] (0.6720, 0.3120, 3.1193) -- (0.7180, 0.3120, 3.1243) -- (0.7180, 0.3660, 3.1300) -- (0.6720, 0.3660, 3.1250) -- cycle;
\fill[blue!51.3, opacity=0.7] (0.6720, 0.3660, 3.1250) -- (0.7180, 0.3660, 3.1300) -- (0.7180, 0.4200, 3.1355) -- (0.6720, 0.4200, 3.1305) -- cycle;
\fill[blue!63.6, opacity=0.7] (0.6720, 0.4200, 3.1305) -- (0.7180, 0.4200, 3.1355) -- (0.7180, 0.4740, 3.1409) -- (0.6720, 0.4740, 3.1359) -- cycle;
\fill[blue!58.3, opacity=0.7] (0.6720, 0.4740, 3.1359) -- (0.7180, 0.4740, 3.1409) -- (0.7180, 0.5280, 3.1461) -- (0.6720, 0.5280, 3.1411) -- cycle;
\fill[blue!56.3, opacity=0.7] (0.6720, 0.5280, 3.1411) -- (0.7180, 0.5280, 3.1461) -- (0.7180, 0.5820, 3.1510) -- (0.6720, 0.5820, 3.1461) -- cycle;
\fill[blue!61.7, opacity=0.7] (0.6720, 0.5820, 3.1461) -- (0.7180, 0.5820, 3.1510) -- (0.7180, 0.6360, 3.1558) -- (0.6720, 0.6360, 3.1508) -- cycle;
\fill[blue!62.2, opacity=0.7] (0.6720, 0.6360, 3.1508) -- (0.7180, 0.6360, 3.1558) -- (0.7180, 0.6900, 3.1604) -- (0.6720, 0.6900, 3.1554) -- cycle;
\fill[blue!50.2, opacity=0.7] (0.6720, 0.6900, 3.1554) -- (0.7180, 0.6900, 3.1604) -- (0.7180, 0.7440, 3.1647) -- (0.6720, 0.7440, 3.1597) -- cycle;
\fill[blue!38.3, opacity=0.7] (0.6720, 0.7440, 3.1597) -- (0.7180, 0.7440, 3.1647) -- (0.7180, 0.7980, 3.1688) -- (0.6720, 0.7980, 3.1638) -- cycle;
\fill[blue!34.5, opacity=0.7] (0.6720, 0.7980, 3.1638) -- (0.7180, 0.7980, 3.1688) -- (0.7180, 0.8520, 3.1726) -- (0.6720, 0.8520, 3.1676) -- cycle;
\fill[blue!38.9, opacity=0.7] (0.6720, 0.8520, 3.1676) -- (0.7180, 0.8520, 3.1726) -- (0.7180, 0.9060, 3.1762) -- (0.6720, 0.9060, 3.1712) -- cycle;
\fill[blue!50.6, opacity=0.7] (0.6720, 0.9060, 3.1712) -- (0.7180, 0.9060, 3.1762) -- (0.7180, 0.9600, 3.1794) -- (0.6720, 0.9600, 3.1745) -- cycle;
\fill[blue!62.1, opacity=0.7] (0.6720, 0.9600, 3.1745) -- (0.7180, 0.9600, 3.1794) -- (0.7180, 1.0140, 3.1824) -- (0.6720, 1.0140, 3.1775) -- cycle;
\fill[blue!61.3, opacity=0.7] (0.6720, 1.0140, 3.1775) -- (0.7180, 1.0140, 3.1824) -- (0.7180, 1.0680, 3.1851) -- (0.6720, 1.0680, 3.1802) -- cycle;
\fill[blue!50.4, opacity=0.7] (0.6720, 1.0680, 3.1802) -- (0.7180, 1.0680, 3.1851) -- (0.7180, 1.1220, 3.1875) -- (0.6720, 1.1220, 3.1826) -- cycle;
\fill[blue!40.2, opacity=0.7] (0.6720, 1.1220, 3.1826) -- (0.7180, 1.1220, 3.1875) -- (0.7180, 1.1760, 3.1896) -- (0.6720, 1.1760, 3.1847) -- cycle;
\fill[blue!34.7, opacity=0.7] (0.6720, 1.1760, 3.1847) -- (0.7180, 1.1760, 3.1896) -- (0.7180, 1.2300, 3.1914) -- (0.6720, 1.2300, 3.1864) -- cycle;
\fill[blue!33.2, opacity=0.7] (0.6720, 1.2300, 3.1864) -- (0.7180, 1.2300, 3.1914) -- (0.7180, 1.2840, 3.1929) -- (0.6720, 1.2840, 3.1879) -- cycle;
\fill[blue!34.3, opacity=0.7] (0.6720, 1.2840, 3.1879) -- (0.7180, 1.2840, 3.1929) -- (0.7180, 1.3380, 3.1940) -- (0.6720, 1.3380, 3.1891) -- cycle;
\fill[blue!37.2, opacity=0.7] (0.6720, 1.3380, 3.1891) -- (0.7180, 1.3380, 3.1940) -- (0.7180, 1.3920, 3.1949) -- (0.6720, 1.3920, 3.1899) -- cycle;
\fill[blue!41.1, opacity=0.7] (0.6720, 1.3920, 3.1899) -- (0.7180, 1.3920, 3.1949) -- (0.7180, 1.4460, 3.1954) -- (0.6720, 1.4460, 3.1904) -- cycle;
\fill[blue!45.1, opacity=0.7] (0.6720, 1.4460, 3.1904) -- (0.7180, 1.4460, 3.1954) -- (0.7180, 1.5000, 3.1955) -- (0.6720, 1.5000, 3.1905) -- cycle;
\fill[blue!48.4, opacity=0.7] (0.6720, 1.5000, 3.1905) -- (0.7180, 1.5000, 3.1955) -- (0.7180, 1.5540, 3.1954) -- (0.6720, 1.5540, 3.1904) -- cycle;
\fill[blue!50.7, opacity=0.7] (0.6720, 1.5540, 3.1904) -- (0.7180, 1.5540, 3.1954) -- (0.7180, 1.6080, 3.1949) -- (0.6720, 1.6080, 3.1899) -- cycle;
\fill[blue!51.9, opacity=0.7] (0.6720, 1.6080, 3.1899) -- (0.7180, 1.6080, 3.1949) -- (0.7180, 1.6620, 3.1940) -- (0.6720, 1.6620, 3.1891) -- cycle;
\fill[blue!52.0, opacity=0.7] (0.6720, 1.6620, 3.1891) -- (0.7180, 1.6620, 3.1940) -- (0.7180, 1.7160, 3.1929) -- (0.6720, 1.7160, 3.1879) -- cycle;
\fill[blue!51.0, opacity=0.7] (0.6720, 1.7160, 3.1879) -- (0.7180, 1.7160, 3.1929) -- (0.7180, 1.7700, 3.1914) -- (0.6720, 1.7700, 3.1864) -- cycle;
\fill[blue!49.0, opacity=0.7] (0.6720, 1.7700, 3.1864) -- (0.7180, 1.7700, 3.1914) -- (0.7180, 1.8240, 3.1896) -- (0.6720, 1.8240, 3.1847) -- cycle;
\fill[blue!46.3, opacity=0.7] (0.6720, 1.8240, 3.1847) -- (0.7180, 1.8240, 3.1896) -- (0.7180, 1.8780, 3.1875) -- (0.6720, 1.8780, 3.1826) -- cycle;
\fill[blue!43.3, opacity=0.7] (0.6720, 1.8780, 3.1826) -- (0.7180, 1.8780, 3.1875) -- (0.7180, 1.9320, 3.1851) -- (0.6720, 1.9320, 3.1802) -- cycle;
\fill[blue!40.8, opacity=0.7] (0.6720, 1.9320, 3.1802) -- (0.7180, 1.9320, 3.1851) -- (0.7180, 1.9860, 3.1824) -- (0.6720, 1.9860, 3.1775) -- cycle;
\fill[blue!39.6, opacity=0.7] (0.6720, 1.9860, 3.1775) -- (0.7180, 1.9860, 3.1824) -- (0.7180, 2.0400, 3.1794) -- (0.6720, 2.0400, 3.1745) -- cycle;
\fill[blue!40.8, opacity=0.7] (0.6720, 2.0400, 3.1745) -- (0.7180, 2.0400, 3.1794) -- (0.7180, 2.0940, 3.1762) -- (0.6720, 2.0940, 3.1712) -- cycle;
\fill[blue!45.3, opacity=0.7] (0.6720, 2.0940, 3.1712) -- (0.7180, 2.0940, 3.1762) -- (0.7180, 2.1480, 3.1726) -- (0.6720, 2.1480, 3.1676) -- cycle;
\fill[blue!53.5, opacity=0.7] (0.6720, 2.1480, 3.1676) -- (0.7180, 2.1480, 3.1726) -- (0.7180, 2.2020, 3.1688) -- (0.6720, 2.2020, 3.1638) -- cycle;
\fill[blue!62.0, opacity=0.7] (0.6720, 2.2020, 3.1638) -- (0.7180, 2.2020, 3.1688) -- (0.7180, 2.2560, 3.1647) -- (0.6720, 2.2560, 3.1597) -- cycle;
\fill[blue!61.9, opacity=0.7] (0.6720, 2.2560, 3.1597) -- (0.7180, 2.2560, 3.1647) -- (0.7180, 2.3100, 3.1604) -- (0.6720, 2.3100, 3.1554) -- cycle;
\fill[blue!49.4, opacity=0.7] (0.6720, 2.3100, 3.1554) -- (0.7180, 2.3100, 3.1604) -- (0.7180, 2.3640, 3.1558) -- (0.6720, 2.3640, 3.1508) -- cycle;
\fill[blue!34.8, opacity=0.7] (0.6720, 2.3640, 3.1508) -- (0.7180, 2.3640, 3.1558) -- (0.7180, 2.4180, 3.1510) -- (0.6720, 2.4180, 3.1461) -- cycle;
\fill[blue!26.7, opacity=0.7] (0.6720, 2.4180, 3.1461) -- (0.7180, 2.4180, 3.1510) -- (0.7180, 2.4720, 3.1461) -- (0.6720, 2.4720, 3.1411) -- cycle;
\fill[blue!24.6, opacity=0.7] (0.6720, 2.4720, 3.1411) -- (0.7180, 2.4720, 3.1461) -- (0.7180, 2.5260, 3.1409) -- (0.6720, 2.5260, 3.1359) -- cycle;
\fill[blue!27.7, opacity=0.7] (0.6720, 2.5260, 3.1359) -- (0.7180, 2.5260, 3.1409) -- (0.7180, 2.5800, 3.1355) -- (0.6720, 2.5800, 3.1305) -- cycle;
\fill[blue!37.6, opacity=0.7] (0.6720, 2.5800, 3.1305) -- (0.7180, 2.5800, 3.1355) -- (0.7180, 2.6340, 3.1300) -- (0.6720, 2.6340, 3.1250) -- cycle;
\fill[blue!52.6, opacity=0.7] (0.6720, 2.6340, 3.1250) -- (0.7180, 2.6340, 3.1300) -- (0.7180, 2.6880, 3.1243) -- (0.6720, 2.6880, 3.1193) -- cycle;
\fill[blue!62.3, opacity=0.7] (0.6720, 2.6880, 3.1193) -- (0.7180, 2.6880, 3.1243) -- (0.7180, 2.7420, 3.1185) -- (0.6720, 2.7420, 3.1135) -- cycle;
\fill[blue!63.6, opacity=0.7] (0.6720, 2.7420, 3.1135) -- (0.7180, 2.7420, 3.1185) -- (0.7180, 2.7960, 3.1126) -- (0.6720, 2.7960, 3.1076) -- cycle;
\fill[blue!63.5, opacity=0.7] (0.6720, 2.7960, 3.1076) -- (0.7180, 2.7960, 3.1126) -- (0.7180, 2.8500, 3.1066) -- (0.6720, 2.8500, 3.1016) -- cycle;
\fill[blue!57.9, opacity=0.7] (0.6720, 2.8500, 3.1016) -- (0.7180, 2.8500, 3.1066) -- (0.7180, 2.9040, 3.1005) -- (0.6720, 2.9040, 3.0955) -- cycle;
\fill[blue!38.3, opacity=0.7] (0.6720, 2.9040, 3.0955) -- (0.7180, 2.9040, 3.1005) -- (0.7180, 2.9580, 3.0943) -- (0.6720, 2.9580, 3.0893) -- cycle;
\fill[blue!21.6, opacity=0.7] (0.6720, 2.9580, 3.0893) -- (0.7180, 2.9580, 3.0943) -- (0.7180, 3.0120, 3.0881) -- (0.6720, 3.0120, 3.0831) -- cycle;
\fill[blue!16.6, opacity=0.7] (0.6720, 3.0120, 3.0831) -- (0.7180, 3.0120, 3.0881) -- (0.7180, 3.0660, 3.0818) -- (0.6720, 3.0660, 3.0768) -- cycle;
\fill[blue!15.8, opacity=0.7] (0.6720, 3.0660, 3.0768) -- (0.7180, 3.0660, 3.0818) -- (0.7180, 3.1200, 3.0755) -- (0.6720, 3.1200, 3.0705) -- cycle;
\fill[blue!50.8, opacity=0.7] (0.7180, -0.1200, 3.0755) -- (0.7640, -0.1200, 3.0803) -- (0.7640, -0.0660, 3.0866) -- (0.7180, -0.0660, 3.0818) -- cycle;
\fill[blue!60.7, opacity=0.7] (0.7180, -0.0660, 3.0818) -- (0.7640, -0.0660, 3.0866) -- (0.7640, -0.0120, 3.0928) -- (0.7180, -0.0120, 3.0881) -- cycle;
\fill[blue!59.0, opacity=0.7] (0.7180, -0.0120, 3.0881) -- (0.7640, -0.0120, 3.0928) -- (0.7640, 0.0420, 3.0991) -- (0.7180, 0.0420, 3.0943) -- cycle;
\fill[blue!45.5, opacity=0.7] (0.7180, 0.0420, 3.0943) -- (0.7640, 0.0420, 3.0991) -- (0.7640, 0.0960, 3.1052) -- (0.7180, 0.0960, 3.1005) -- cycle;
\fill[blue!27.7, opacity=0.7] (0.7180, 0.0960, 3.1005) -- (0.7640, 0.0960, 3.1052) -- (0.7640, 0.1500, 3.1114) -- (0.7180, 0.1500, 3.1066) -- cycle;
\fill[blue!20.0, opacity=0.7] (0.7180, 0.1500, 3.1066) -- (0.7640, 0.1500, 3.1114) -- (0.7640, 0.2040, 3.1174) -- (0.7180, 0.2040, 3.1126) -- cycle;
\fill[blue!19.3, opacity=0.7] (0.7180, 0.2040, 3.1126) -- (0.7640, 0.2040, 3.1174) -- (0.7640, 0.2580, 3.1233) -- (0.7180, 0.2580, 3.1185) -- cycle;
\fill[blue!24.9, opacity=0.7] (0.7180, 0.2580, 3.1185) -- (0.7640, 0.2580, 3.1233) -- (0.7640, 0.3120, 3.1291) -- (0.7180, 0.3120, 3.1243) -- cycle;
\fill[blue!43.5, opacity=0.7] (0.7180, 0.3120, 3.1243) -- (0.7640, 0.3120, 3.1291) -- (0.7640, 0.3660, 3.1348) -- (0.7180, 0.3660, 3.1300) -- cycle;
\fill[blue!62.5, opacity=0.7] (0.7180, 0.3660, 3.1300) -- (0.7640, 0.3660, 3.1348) -- (0.7640, 0.4200, 3.1403) -- (0.7180, 0.4200, 3.1355) -- cycle;
\fill[blue!59.9, opacity=0.7] (0.7180, 0.4200, 3.1355) -- (0.7640, 0.4200, 3.1403) -- (0.7640, 0.4740, 3.1457) -- (0.7180, 0.4740, 3.1409) -- cycle;
\fill[blue!55.5, opacity=0.7] (0.7180, 0.4740, 3.1409) -- (0.7640, 0.4740, 3.1457) -- (0.7640, 0.5280, 3.1508) -- (0.7180, 0.5280, 3.1461) -- cycle;
\fill[blue!60.1, opacity=0.7] (0.7180, 0.5280, 3.1461) -- (0.7640, 0.5280, 3.1508) -- (0.7640, 0.5820, 3.1558) -- (0.7180, 0.5820, 3.1510) -- cycle;
\fill[blue!63.1, opacity=0.7] (0.7180, 0.5820, 3.1510) -- (0.7640, 0.5820, 3.1558) -- (0.7640, 0.6360, 3.1606) -- (0.7180, 0.6360, 3.1558) -- cycle;
\fill[blue!52.2, opacity=0.7] (0.7180, 0.6360, 3.1558) -- (0.7640, 0.6360, 3.1606) -- (0.7640, 0.6900, 3.1651) -- (0.7180, 0.6900, 3.1604) -- cycle;
\fill[blue!39.3, opacity=0.7] (0.7180, 0.6900, 3.1604) -- (0.7640, 0.6900, 3.1651) -- (0.7640, 0.7440, 3.1695) -- (0.7180, 0.7440, 3.1647) -- cycle;
\fill[blue!35.1, opacity=0.7] (0.7180, 0.7440, 3.1647) -- (0.7640, 0.7440, 3.1695) -- (0.7640, 0.7980, 3.1736) -- (0.7180, 0.7980, 3.1688) -- cycle;
\fill[blue!40.4, opacity=0.7] (0.7180, 0.7980, 3.1688) -- (0.7640, 0.7980, 3.1736) -- (0.7640, 0.8520, 3.1774) -- (0.7180, 0.8520, 3.1726) -- cycle;
\fill[blue!53.7, opacity=0.7] (0.7180, 0.8520, 3.1726) -- (0.7640, 0.8520, 3.1774) -- (0.7640, 0.9060, 3.1809) -- (0.7180, 0.9060, 3.1762) -- cycle;
\fill[blue!63.5, opacity=0.7] (0.7180, 0.9060, 3.1762) -- (0.7640, 0.9060, 3.1809) -- (0.7640, 0.9600, 3.1842) -- (0.7180, 0.9600, 3.1794) -- cycle;
\fill[blue!56.7, opacity=0.7] (0.7180, 0.9600, 3.1794) -- (0.7640, 0.9600, 3.1842) -- (0.7640, 1.0140, 3.1872) -- (0.7180, 1.0140, 3.1824) -- cycle;
\fill[blue!43.2, opacity=0.7] (0.7180, 1.0140, 3.1824) -- (0.7640, 1.0140, 3.1872) -- (0.7640, 1.0680, 3.1899) -- (0.7180, 1.0680, 3.1851) -- cycle;
\fill[blue!34.7, opacity=0.7] (0.7180, 1.0680, 3.1851) -- (0.7640, 1.0680, 3.1899) -- (0.7640, 1.1220, 3.1923) -- (0.7180, 1.1220, 3.1875) -- cycle;
\fill[blue!32.2, opacity=0.7] (0.7180, 1.1220, 3.1875) -- (0.7640, 1.1220, 3.1923) -- (0.7640, 1.1760, 3.1944) -- (0.7180, 1.1760, 3.1896) -- cycle;
\fill[blue!34.0, opacity=0.7] (0.7180, 1.1760, 3.1896) -- (0.7640, 1.1760, 3.1944) -- (0.7640, 1.2300, 3.1962) -- (0.7180, 1.2300, 3.1914) -- cycle;
\fill[blue!39.2, opacity=0.7] (0.7180, 1.2300, 3.1914) -- (0.7640, 1.2300, 3.1962) -- (0.7640, 1.2840, 3.1977) -- (0.7180, 1.2840, 3.1929) -- cycle;
\fill[blue!46.3, opacity=0.7] (0.7180, 1.2840, 3.1929) -- (0.7640, 1.2840, 3.1977) -- (0.7640, 1.3380, 3.1988) -- (0.7180, 1.3380, 3.1940) -- cycle;
\fill[blue!53.5, opacity=0.7] (0.7180, 1.3380, 3.1940) -- (0.7640, 1.3380, 3.1988) -- (0.7640, 1.3920, 3.1996) -- (0.7180, 1.3920, 3.1949) -- cycle;
\fill[blue!58.8, opacity=0.7] (0.7180, 1.3920, 3.1949) -- (0.7640, 1.3920, 3.1996) -- (0.7640, 1.4460, 3.2001) -- (0.7180, 1.4460, 3.1954) -- cycle;
\fill[blue!61.9, opacity=0.7] (0.7180, 1.4460, 3.1954) -- (0.7640, 1.4460, 3.2001) -- (0.7640, 1.5000, 3.2003) -- (0.7180, 1.5000, 3.1955) -- cycle;
\fill[blue!63.2, opacity=0.7] (0.7180, 1.5000, 3.1955) -- (0.7640, 1.5000, 3.2003) -- (0.7640, 1.5540, 3.2001) -- (0.7180, 1.5540, 3.1954) -- cycle;
\fill[blue!63.5, opacity=0.7] (0.7180, 1.5540, 3.1954) -- (0.7640, 1.5540, 3.2001) -- (0.7640, 1.6080, 3.1996) -- (0.7180, 1.6080, 3.1949) -- cycle;
\fill[blue!63.6, opacity=0.7] (0.7180, 1.6080, 3.1949) -- (0.7640, 1.6080, 3.1996) -- (0.7640, 1.6620, 3.1988) -- (0.7180, 1.6620, 3.1940) -- cycle;
\fill[blue!63.6, opacity=0.7] (0.7180, 1.6620, 3.1940) -- (0.7640, 1.6620, 3.1988) -- (0.7640, 1.7160, 3.1977) -- (0.7180, 1.7160, 3.1929) -- cycle;
\fill[blue!63.6, opacity=0.7] (0.7180, 1.7160, 3.1929) -- (0.7640, 1.7160, 3.1977) -- (0.7640, 1.7700, 3.1962) -- (0.7180, 1.7700, 3.1914) -- cycle;
\fill[blue!63.3, opacity=0.7] (0.7180, 1.7700, 3.1914) -- (0.7640, 1.7700, 3.1962) -- (0.7640, 1.8240, 3.1944) -- (0.7180, 1.8240, 3.1896) -- cycle;
\fill[blue!62.1, opacity=0.7] (0.7180, 1.8240, 3.1896) -- (0.7640, 1.8240, 3.1944) -- (0.7640, 1.8780, 3.1923) -- (0.7180, 1.8780, 3.1875) -- cycle;
\fill[blue!59.4, opacity=0.7] (0.7180, 1.8780, 3.1875) -- (0.7640, 1.8780, 3.1923) -- (0.7640, 1.9320, 3.1899) -- (0.7180, 1.9320, 3.1851) -- cycle;
\fill[blue!54.8, opacity=0.7] (0.7180, 1.9320, 3.1851) -- (0.7640, 1.9320, 3.1899) -- (0.7640, 1.9860, 3.1872) -- (0.7180, 1.9860, 3.1824) -- cycle;
\fill[blue!49.0, opacity=0.7] (0.7180, 1.9860, 3.1824) -- (0.7640, 1.9860, 3.1872) -- (0.7640, 2.0400, 3.1842) -- (0.7180, 2.0400, 3.1794) -- cycle;
\fill[blue!43.8, opacity=0.7] (0.7180, 2.0400, 3.1794) -- (0.7640, 2.0400, 3.1842) -- (0.7640, 2.0940, 3.1809) -- (0.7180, 2.0940, 3.1762) -- cycle;
\fill[blue!40.9, opacity=0.7] (0.7180, 2.0940, 3.1762) -- (0.7640, 2.0940, 3.1809) -- (0.7640, 2.1480, 3.1774) -- (0.7180, 2.1480, 3.1726) -- cycle;
\fill[blue!41.8, opacity=0.7] (0.7180, 2.1480, 3.1726) -- (0.7640, 2.1480, 3.1774) -- (0.7640, 2.2020, 3.1736) -- (0.7180, 2.2020, 3.1688) -- cycle;
\fill[blue!47.7, opacity=0.7] (0.7180, 2.2020, 3.1688) -- (0.7640, 2.2020, 3.1736) -- (0.7640, 2.2560, 3.1695) -- (0.7180, 2.2560, 3.1647) -- cycle;
\fill[blue!57.8, opacity=0.7] (0.7180, 2.2560, 3.1647) -- (0.7640, 2.2560, 3.1695) -- (0.7640, 2.3100, 3.1651) -- (0.7180, 2.3100, 3.1604) -- cycle;
\fill[blue!63.5, opacity=0.7] (0.7180, 2.3100, 3.1604) -- (0.7640, 2.3100, 3.1651) -- (0.7640, 2.3640, 3.1606) -- (0.7180, 2.3640, 3.1558) -- cycle;
\fill[blue!54.3, opacity=0.7] (0.7180, 2.3640, 3.1558) -- (0.7640, 2.3640, 3.1606) -- (0.7640, 2.4180, 3.1558) -- (0.7180, 2.4180, 3.1510) -- cycle;
\fill[blue!37.6, opacity=0.7] (0.7180, 2.4180, 3.1510) -- (0.7640, 2.4180, 3.1558) -- (0.7640, 2.4720, 3.1508) -- (0.7180, 2.4720, 3.1461) -- cycle;
\fill[blue!27.2, opacity=0.7] (0.7180, 2.4720, 3.1461) -- (0.7640, 2.4720, 3.1508) -- (0.7640, 2.5260, 3.1457) -- (0.7180, 2.5260, 3.1409) -- cycle;
\fill[blue!24.3, opacity=0.7] (0.7180, 2.5260, 3.1409) -- (0.7640, 2.5260, 3.1457) -- (0.7640, 2.5800, 3.1403) -- (0.7180, 2.5800, 3.1355) -- cycle;
\fill[blue!27.1, opacity=0.7] (0.7180, 2.5800, 3.1355) -- (0.7640, 2.5800, 3.1403) -- (0.7640, 2.6340, 3.1348) -- (0.7180, 2.6340, 3.1300) -- cycle;
\fill[blue!37.3, opacity=0.7] (0.7180, 2.6340, 3.1300) -- (0.7640, 2.6340, 3.1348) -- (0.7640, 2.6880, 3.1291) -- (0.7180, 2.6880, 3.1243) -- cycle;
\fill[blue!53.1, opacity=0.7] (0.7180, 2.6880, 3.1243) -- (0.7640, 2.6880, 3.1291) -- (0.7640, 2.7420, 3.1233) -- (0.7180, 2.7420, 3.1185) -- cycle;
\fill[blue!62.5, opacity=0.7] (0.7180, 2.7420, 3.1185) -- (0.7640, 2.7420, 3.1233) -- (0.7640, 2.7960, 3.1174) -- (0.7180, 2.7960, 3.1126) -- cycle;
\fill[blue!63.6, opacity=0.7] (0.7180, 2.7960, 3.1126) -- (0.7640, 2.7960, 3.1174) -- (0.7640, 2.8500, 3.1114) -- (0.7180, 2.8500, 3.1066) -- cycle;
\fill[blue!63.2, opacity=0.7] (0.7180, 2.8500, 3.1066) -- (0.7640, 2.8500, 3.1114) -- (0.7640, 2.9040, 3.1052) -- (0.7180, 2.9040, 3.1005) -- cycle;
\fill[blue!54.3, opacity=0.7] (0.7180, 2.9040, 3.1005) -- (0.7640, 2.9040, 3.1052) -- (0.7640, 2.9580, 3.0991) -- (0.7180, 2.9580, 3.0943) -- cycle;
\fill[blue!32.6, opacity=0.7] (0.7180, 2.9580, 3.0943) -- (0.7640, 2.9580, 3.0991) -- (0.7640, 3.0120, 3.0928) -- (0.7180, 3.0120, 3.0881) -- cycle;
\fill[blue!19.3, opacity=0.7] (0.7180, 3.0120, 3.0881) -- (0.7640, 3.0120, 3.0928) -- (0.7640, 3.0660, 3.0866) -- (0.7180, 3.0660, 3.0818) -- cycle;
\fill[blue!16.1, opacity=0.7] (0.7180, 3.0660, 3.0818) -- (0.7640, 3.0660, 3.0866) -- (0.7640, 3.1200, 3.0803) -- (0.7180, 3.1200, 3.0755) -- cycle;
\fill[blue!57.5, opacity=0.7] (0.7640, -0.1200, 3.0803) -- (0.8100, -0.1200, 3.0849) -- (0.8100, -0.0660, 3.0911) -- (0.7640, -0.0660, 3.0866) -- cycle;
\fill[blue!61.2, opacity=0.7] (0.7640, -0.0660, 3.0866) -- (0.8100, -0.0660, 3.0911) -- (0.8100, -0.0120, 3.0974) -- (0.7640, -0.0120, 3.0928) -- cycle;
\fill[blue!54.5, opacity=0.7] (0.7640, -0.0120, 3.0928) -- (0.8100, -0.0120, 3.0974) -- (0.8100, 0.0420, 3.1036) -- (0.7640, 0.0420, 3.0991) -- cycle;
\fill[blue!36.1, opacity=0.7] (0.7640, 0.0420, 3.0991) -- (0.8100, 0.0420, 3.1036) -- (0.8100, 0.0960, 3.1098) -- (0.7640, 0.0960, 3.1052) -- cycle;
\fill[blue!22.7, opacity=0.7] (0.7640, 0.0960, 3.1052) -- (0.8100, 0.0960, 3.1098) -- (0.8100, 0.1500, 3.1159) -- (0.7640, 0.1500, 3.1114) -- cycle;
\fill[blue!19.2, opacity=0.7] (0.7640, 0.1500, 3.1114) -- (0.8100, 0.1500, 3.1159) -- (0.8100, 0.2040, 3.1219) -- (0.7640, 0.2040, 3.1174) -- cycle;
\fill[blue!21.4, opacity=0.7] (0.7640, 0.2040, 3.1174) -- (0.8100, 0.2040, 3.1219) -- (0.8100, 0.2580, 3.1279) -- (0.7640, 0.2580, 3.1233) -- cycle;
\fill[blue!34.3, opacity=0.7] (0.7640, 0.2580, 3.1233) -- (0.8100, 0.2580, 3.1279) -- (0.8100, 0.3120, 3.1337) -- (0.7640, 0.3120, 3.1291) -- cycle;
\fill[blue!57.7, opacity=0.7] (0.7640, 0.3120, 3.1291) -- (0.8100, 0.3120, 3.1337) -- (0.8100, 0.3660, 3.1393) -- (0.7640, 0.3660, 3.1348) -- cycle;
\fill[blue!62.3, opacity=0.7] (0.7640, 0.3660, 3.1348) -- (0.8100, 0.3660, 3.1393) -- (0.8100, 0.4200, 3.1449) -- (0.7640, 0.4200, 3.1403) -- cycle;
\fill[blue!55.6, opacity=0.7] (0.7640, 0.4200, 3.1403) -- (0.8100, 0.4200, 3.1449) -- (0.8100, 0.4740, 3.1502) -- (0.7640, 0.4740, 3.1457) -- cycle;
\fill[blue!57.8, opacity=0.7] (0.7640, 0.4740, 3.1457) -- (0.8100, 0.4740, 3.1502) -- (0.8100, 0.5280, 3.1554) -- (0.7640, 0.5280, 3.1508) -- cycle;
\fill[blue!63.5, opacity=0.7] (0.7640, 0.5280, 3.1508) -- (0.8100, 0.5280, 3.1554) -- (0.8100, 0.5820, 3.1604) -- (0.7640, 0.5820, 3.1558) -- cycle;
\fill[blue!55.9, opacity=0.7] (0.7640, 0.5820, 3.1558) -- (0.8100, 0.5820, 3.1604) -- (0.8100, 0.6360, 3.1651) -- (0.7640, 0.6360, 3.1606) -- cycle;
\fill[blue!41.4, opacity=0.7] (0.7640, 0.6360, 3.1606) -- (0.8100, 0.6360, 3.1651) -- (0.8100, 0.6900, 3.1697) -- (0.7640, 0.6900, 3.1651) -- cycle;
\fill[blue!35.6, opacity=0.7] (0.7640, 0.6900, 3.1651) -- (0.8100, 0.6900, 3.1697) -- (0.8100, 0.7440, 3.1740) -- (0.7640, 0.7440, 3.1695) -- cycle;
\fill[blue!40.5, opacity=0.7] (0.7640, 0.7440, 3.1695) -- (0.8100, 0.7440, 3.1740) -- (0.8100, 0.7980, 3.1781) -- (0.7640, 0.7980, 3.1736) -- cycle;
\fill[blue!54.6, opacity=0.7] (0.7640, 0.7980, 3.1736) -- (0.8100, 0.7980, 3.1781) -- (0.8100, 0.8520, 3.1819) -- (0.7640, 0.8520, 3.1774) -- cycle;
\fill[blue!63.5, opacity=0.7] (0.7640, 0.8520, 3.1774) -- (0.8100, 0.8520, 3.1819) -- (0.8100, 0.9060, 3.1855) -- (0.7640, 0.9060, 3.1809) -- cycle;
\fill[blue!53.3, opacity=0.7] (0.7640, 0.9060, 3.1809) -- (0.8100, 0.9060, 3.1855) -- (0.8100, 0.9600, 3.1888) -- (0.7640, 0.9600, 3.1842) -- cycle;
\fill[blue!39.0, opacity=0.7] (0.7640, 0.9600, 3.1842) -- (0.8100, 0.9600, 3.1888) -- (0.8100, 1.0140, 3.1918) -- (0.7640, 1.0140, 3.1872) -- cycle;
\fill[blue!32.1, opacity=0.7] (0.7640, 1.0140, 3.1872) -- (0.8100, 1.0140, 3.1918) -- (0.8100, 1.0680, 3.1945) -- (0.7640, 1.0680, 3.1899) -- cycle;
\fill[blue!32.1, opacity=0.7] (0.7640, 1.0680, 3.1899) -- (0.8100, 1.0680, 3.1945) -- (0.8100, 1.1220, 3.1969) -- (0.7640, 1.1220, 3.1923) -- cycle;
\fill[blue!37.3, opacity=0.7] (0.7640, 1.1220, 3.1923) -- (0.8100, 1.1220, 3.1969) -- (0.8100, 1.1760, 3.1990) -- (0.7640, 1.1760, 3.1944) -- cycle;
\fill[blue!46.5, opacity=0.7] (0.7640, 1.1760, 3.1944) -- (0.8100, 1.1760, 3.1990) -- (0.8100, 1.2300, 3.2008) -- (0.7640, 1.2300, 3.1962) -- cycle;
\fill[blue!56.1, opacity=0.7] (0.7640, 1.2300, 3.1962) -- (0.8100, 1.2300, 3.2008) -- (0.8100, 1.2840, 3.2022) -- (0.7640, 1.2840, 3.1977) -- cycle;
\fill[blue!62.1, opacity=0.7] (0.7640, 1.2840, 3.1977) -- (0.8100, 1.2840, 3.2022) -- (0.8100, 1.3380, 3.2034) -- (0.7640, 1.3380, 3.1988) -- cycle;
\fill[blue!63.6, opacity=0.7] (0.7640, 1.3380, 3.1988) -- (0.8100, 1.3380, 3.2034) -- (0.8100, 1.3920, 3.2042) -- (0.7640, 1.3920, 3.1996) -- cycle;
\fill[blue!62.3, opacity=0.7] (0.7640, 1.3920, 3.1996) -- (0.8100, 1.3920, 3.2042) -- (0.8100, 1.4460, 3.2047) -- (0.7640, 1.4460, 3.2001) -- cycle;
\fill[blue!60.3, opacity=0.7] (0.7640, 1.4460, 3.2001) -- (0.8100, 1.4460, 3.2047) -- (0.8100, 1.5000, 3.2049) -- (0.7640, 1.5000, 3.2003) -- cycle;
\fill[blue!58.7, opacity=0.7] (0.7640, 1.5000, 3.2003) -- (0.8100, 1.5000, 3.2049) -- (0.8100, 1.5540, 3.2047) -- (0.7640, 1.5540, 3.2001) -- cycle;
\fill[blue!57.5, opacity=0.7] (0.7640, 1.5540, 3.2001) -- (0.8100, 1.5540, 3.2047) -- (0.8100, 1.6080, 3.2042) -- (0.7640, 1.6080, 3.1996) -- cycle;
\fill[blue!56.8, opacity=0.7] (0.7640, 1.6080, 3.1996) -- (0.8100, 1.6080, 3.2042) -- (0.8100, 1.6620, 3.2034) -- (0.7640, 1.6620, 3.1988) -- cycle;
\fill[blue!56.5, opacity=0.7] (0.7640, 1.6620, 3.1988) -- (0.8100, 1.6620, 3.2034) -- (0.8100, 1.7160, 3.2022) -- (0.7640, 1.7160, 3.1977) -- cycle;
\fill[blue!56.7, opacity=0.7] (0.7640, 1.7160, 3.1977) -- (0.8100, 1.7160, 3.2022) -- (0.8100, 1.7700, 3.2008) -- (0.7640, 1.7700, 3.1962) -- cycle;
\fill[blue!57.6, opacity=0.7] (0.7640, 1.7700, 3.1962) -- (0.8100, 1.7700, 3.2008) -- (0.8100, 1.8240, 3.1990) -- (0.7640, 1.8240, 3.1944) -- cycle;
\fill[blue!59.3, opacity=0.7] (0.7640, 1.8240, 3.1944) -- (0.8100, 1.8240, 3.1990) -- (0.8100, 1.8780, 3.1969) -- (0.7640, 1.8780, 3.1923) -- cycle;
\fill[blue!61.5, opacity=0.7] (0.7640, 1.8780, 3.1923) -- (0.8100, 1.8780, 3.1969) -- (0.8100, 1.9320, 3.1945) -- (0.7640, 1.9320, 3.1899) -- cycle;
\fill[blue!63.4, opacity=0.7] (0.7640, 1.9320, 3.1899) -- (0.8100, 1.9320, 3.1945) -- (0.8100, 1.9860, 3.1918) -- (0.7640, 1.9860, 3.1872) -- cycle;
\fill[blue!62.8, opacity=0.7] (0.7640, 1.9860, 3.1872) -- (0.8100, 1.9860, 3.1918) -- (0.8100, 2.0400, 3.1888) -- (0.7640, 2.0400, 3.1842) -- cycle;
\fill[blue!58.3, opacity=0.7] (0.7640, 2.0400, 3.1842) -- (0.8100, 2.0400, 3.1888) -- (0.8100, 2.0940, 3.1855) -- (0.7640, 2.0940, 3.1809) -- cycle;
\fill[blue!50.9, opacity=0.7] (0.7640, 2.0940, 3.1809) -- (0.8100, 2.0940, 3.1855) -- (0.8100, 2.1480, 3.1819) -- (0.7640, 2.1480, 3.1774) -- cycle;
\fill[blue!44.2, opacity=0.7] (0.7640, 2.1480, 3.1774) -- (0.8100, 2.1480, 3.1819) -- (0.8100, 2.2020, 3.1781) -- (0.7640, 2.2020, 3.1736) -- cycle;
\fill[blue!41.5, opacity=0.7] (0.7640, 2.2020, 3.1736) -- (0.8100, 2.2020, 3.1781) -- (0.8100, 2.2560, 3.1740) -- (0.7640, 2.2560, 3.1695) -- cycle;
\fill[blue!44.8, opacity=0.7] (0.7640, 2.2560, 3.1695) -- (0.8100, 2.2560, 3.1740) -- (0.8100, 2.3100, 3.1697) -- (0.7640, 2.3100, 3.1651) -- cycle;
\fill[blue!54.3, opacity=0.7] (0.7640, 2.3100, 3.1651) -- (0.8100, 2.3100, 3.1697) -- (0.8100, 2.3640, 3.1651) -- (0.7640, 2.3640, 3.1606) -- cycle;
\fill[blue!63.3, opacity=0.7] (0.7640, 2.3640, 3.1606) -- (0.8100, 2.3640, 3.1651) -- (0.8100, 2.4180, 3.1604) -- (0.7640, 2.4180, 3.1558) -- cycle;
\fill[blue!56.6, opacity=0.7] (0.7640, 2.4180, 3.1558) -- (0.8100, 2.4180, 3.1604) -- (0.8100, 2.4720, 3.1554) -- (0.7640, 2.4720, 3.1508) -- cycle;
\fill[blue!38.7, opacity=0.7] (0.7640, 2.4720, 3.1508) -- (0.8100, 2.4720, 3.1554) -- (0.8100, 2.5260, 3.1502) -- (0.7640, 2.5260, 3.1457) -- cycle;
\fill[blue!27.1, opacity=0.7] (0.7640, 2.5260, 3.1457) -- (0.8100, 2.5260, 3.1502) -- (0.8100, 2.5800, 3.1449) -- (0.7640, 2.5800, 3.1403) -- cycle;
\fill[blue!24.0, opacity=0.7] (0.7640, 2.5800, 3.1403) -- (0.8100, 2.5800, 3.1449) -- (0.8100, 2.6340, 3.1393) -- (0.7640, 2.6340, 3.1348) -- cycle;
\fill[blue!27.2, opacity=0.7] (0.7640, 2.6340, 3.1348) -- (0.8100, 2.6340, 3.1393) -- (0.8100, 2.6880, 3.1337) -- (0.7640, 2.6880, 3.1291) -- cycle;
\fill[blue!38.5, opacity=0.7] (0.7640, 2.6880, 3.1291) -- (0.8100, 2.6880, 3.1337) -- (0.8100, 2.7420, 3.1279) -- (0.7640, 2.7420, 3.1233) -- cycle;
\fill[blue!54.9, opacity=0.7] (0.7640, 2.7420, 3.1233) -- (0.8100, 2.7420, 3.1279) -- (0.8100, 2.7960, 3.1219) -- (0.7640, 2.7960, 3.1174) -- cycle;
\fill[blue!63.0, opacity=0.7] (0.7640, 2.7960, 3.1174) -- (0.8100, 2.7960, 3.1219) -- (0.8100, 2.8500, 3.1159) -- (0.7640, 2.8500, 3.1114) -- cycle;
\fill[blue!63.6, opacity=0.7] (0.7640, 2.8500, 3.1114) -- (0.8100, 2.8500, 3.1159) -- (0.8100, 2.9040, 3.1098) -- (0.7640, 2.9040, 3.1052) -- cycle;
\fill[blue!62.2, opacity=0.7] (0.7640, 2.9040, 3.1052) -- (0.8100, 2.9040, 3.1098) -- (0.8100, 2.9580, 3.1036) -- (0.7640, 2.9580, 3.0991) -- cycle;
\fill[blue!48.2, opacity=0.7] (0.7640, 2.9580, 3.0991) -- (0.8100, 2.9580, 3.1036) -- (0.8100, 3.0120, 3.0974) -- (0.7640, 3.0120, 3.0928) -- cycle;
\fill[blue!26.6, opacity=0.7] (0.7640, 3.0120, 3.0928) -- (0.8100, 3.0120, 3.0974) -- (0.8100, 3.0660, 3.0911) -- (0.7640, 3.0660, 3.0866) -- cycle;
\fill[blue!17.5, opacity=0.7] (0.7640, 3.0660, 3.0866) -- (0.8100, 3.0660, 3.0911) -- (0.8100, 3.1200, 3.0849) -- (0.7640, 3.1200, 3.0803) -- cycle;
\fill[blue!60.6, opacity=0.7] (0.8100, -0.1200, 3.0849) -- (0.8560, -0.1200, 3.0892) -- (0.8560, -0.0660, 3.0955) -- (0.8100, -0.0660, 3.0911) -- cycle;
\fill[blue!60.0, opacity=0.7] (0.8100, -0.0660, 3.0911) -- (0.8560, -0.0660, 3.0955) -- (0.8560, -0.0120, 3.1017) -- (0.8100, -0.0120, 3.0974) -- cycle;
\fill[blue!47.5, opacity=0.7] (0.8100, -0.0120, 3.0974) -- (0.8560, -0.0120, 3.1017) -- (0.8560, 0.0420, 3.1079) -- (0.8100, 0.0420, 3.1036) -- cycle;
\fill[blue!28.7, opacity=0.7] (0.8100, 0.0420, 3.1036) -- (0.8560, 0.0420, 3.1079) -- (0.8560, 0.0960, 3.1141) -- (0.8100, 0.0960, 3.1098) -- cycle;
\fill[blue!20.3, opacity=0.7] (0.8100, 0.0960, 3.1098) -- (0.8560, 0.0960, 3.1141) -- (0.8560, 0.1500, 3.1202) -- (0.8100, 0.1500, 3.1159) -- cycle;
\fill[blue!19.7, opacity=0.7] (0.8100, 0.1500, 3.1159) -- (0.8560, 0.1500, 3.1202) -- (0.8560, 0.2040, 3.1263) -- (0.8100, 0.2040, 3.1219) -- cycle;
\fill[blue!26.4, opacity=0.7] (0.8100, 0.2040, 3.1219) -- (0.8560, 0.2040, 3.1263) -- (0.8560, 0.2580, 3.1322) -- (0.8100, 0.2580, 3.1279) -- cycle;
\fill[blue!47.6, opacity=0.7] (0.8100, 0.2580, 3.1279) -- (0.8560, 0.2580, 3.1322) -- (0.8560, 0.3120, 3.1380) -- (0.8100, 0.3120, 3.1337) -- cycle;
\fill[blue!63.5, opacity=0.7] (0.8100, 0.3120, 3.1337) -- (0.8560, 0.3120, 3.1380) -- (0.8560, 0.3660, 3.1437) -- (0.8100, 0.3660, 3.1393) -- cycle;
\fill[blue!57.5, opacity=0.7] (0.8100, 0.3660, 3.1393) -- (0.8560, 0.3660, 3.1437) -- (0.8560, 0.4200, 3.1492) -- (0.8100, 0.4200, 3.1449) -- cycle;
\fill[blue!55.4, opacity=0.7] (0.8100, 0.4200, 3.1449) -- (0.8560, 0.4200, 3.1492) -- (0.8560, 0.4740, 3.1545) -- (0.8100, 0.4740, 3.1502) -- cycle;
\fill[blue!62.1, opacity=0.7] (0.8100, 0.4740, 3.1502) -- (0.8560, 0.4740, 3.1545) -- (0.8560, 0.5280, 3.1597) -- (0.8100, 0.5280, 3.1554) -- cycle;
\fill[blue!60.2, opacity=0.7] (0.8100, 0.5280, 3.1554) -- (0.8560, 0.5280, 3.1597) -- (0.8560, 0.5820, 3.1647) -- (0.8100, 0.5820, 3.1604) -- cycle;
\fill[blue!45.4, opacity=0.7] (0.8100, 0.5820, 3.1604) -- (0.8560, 0.5820, 3.1647) -- (0.8560, 0.6360, 3.1695) -- (0.8100, 0.6360, 3.1651) -- cycle;
\fill[blue!36.5, opacity=0.7] (0.8100, 0.6360, 3.1651) -- (0.8560, 0.6360, 3.1695) -- (0.8560, 0.6900, 3.1740) -- (0.8100, 0.6900, 3.1697) -- cycle;
\fill[blue!39.3, opacity=0.7] (0.8100, 0.6900, 3.1697) -- (0.8560, 0.6900, 3.1740) -- (0.8560, 0.7440, 3.1784) -- (0.8100, 0.7440, 3.1740) -- cycle;
\fill[blue!53.3, opacity=0.7] (0.8100, 0.7440, 3.1740) -- (0.8560, 0.7440, 3.1784) -- (0.8560, 0.7980, 3.1824) -- (0.8100, 0.7980, 3.1781) -- cycle;
\fill[blue!63.6, opacity=0.7] (0.8100, 0.7980, 3.1781) -- (0.8560, 0.7980, 3.1824) -- (0.8560, 0.8520, 3.1863) -- (0.8100, 0.8520, 3.1819) -- cycle;
\fill[blue!52.3, opacity=0.7] (0.8100, 0.8520, 3.1819) -- (0.8560, 0.8520, 3.1863) -- (0.8560, 0.9060, 3.1898) -- (0.8100, 0.9060, 3.1855) -- cycle;
\fill[blue!37.1, opacity=0.7] (0.8100, 0.9060, 3.1855) -- (0.8560, 0.9060, 3.1898) -- (0.8560, 0.9600, 3.1931) -- (0.8100, 0.9600, 3.1888) -- cycle;
\fill[blue!31.0, opacity=0.7] (0.8100, 0.9600, 3.1888) -- (0.8560, 0.9600, 3.1931) -- (0.8560, 1.0140, 3.1961) -- (0.8100, 1.0140, 3.1918) -- cycle;
\fill[blue!32.6, opacity=0.7] (0.8100, 1.0140, 3.1918) -- (0.8560, 1.0140, 3.1961) -- (0.8560, 1.0680, 3.1988) -- (0.8100, 1.0680, 3.1945) -- cycle;
\fill[blue!40.7, opacity=0.7] (0.8100, 1.0680, 3.1945) -- (0.8560, 1.0680, 3.1988) -- (0.8560, 1.1220, 3.2012) -- (0.8100, 1.1220, 3.1969) -- cycle;
\fill[blue!52.7, opacity=0.7] (0.8100, 1.1220, 3.1969) -- (0.8560, 1.1220, 3.2012) -- (0.8560, 1.1760, 3.2033) -- (0.8100, 1.1760, 3.1990) -- cycle;
\fill[blue!61.7, opacity=0.7] (0.8100, 1.1760, 3.1990) -- (0.8560, 1.1760, 3.2033) -- (0.8560, 1.2300, 3.2051) -- (0.8100, 1.2300, 3.2008) -- cycle;
\fill[blue!63.4, opacity=0.7] (0.8100, 1.2300, 3.2008) -- (0.8560, 1.2300, 3.2051) -- (0.8560, 1.2840, 3.2066) -- (0.8100, 1.2840, 3.2022) -- cycle;
\fill[blue!61.0, opacity=0.7] (0.8100, 1.2840, 3.2022) -- (0.8560, 1.2840, 3.2066) -- (0.8560, 1.3380, 3.2077) -- (0.8100, 1.3380, 3.2034) -- cycle;
\fill[blue!58.5, opacity=0.7] (0.8100, 1.3380, 3.2034) -- (0.8560, 1.3380, 3.2077) -- (0.8560, 1.3920, 3.2085) -- (0.8100, 1.3920, 3.2042) -- cycle;
\fill[blue!57.3, opacity=0.7] (0.8100, 1.3920, 3.2042) -- (0.8560, 1.3920, 3.2085) -- (0.8560, 1.4460, 3.2090) -- (0.8100, 1.4460, 3.2047) -- cycle;
\fill[blue!57.2, opacity=0.7] (0.8100, 1.4460, 3.2047) -- (0.8560, 1.4460, 3.2090) -- (0.8560, 1.5000, 3.2092) -- (0.8100, 1.5000, 3.2049) -- cycle;
\fill[blue!57.6, opacity=0.7] (0.8100, 1.5000, 3.2049) -- (0.8560, 1.5000, 3.2092) -- (0.8560, 1.5540, 3.2090) -- (0.8100, 1.5540, 3.2047) -- cycle;
\fill[blue!57.9, opacity=0.7] (0.8100, 1.5540, 3.2047) -- (0.8560, 1.5540, 3.2090) -- (0.8560, 1.6080, 3.2085) -- (0.8100, 1.6080, 3.2042) -- cycle;
\fill[blue!57.8, opacity=0.7] (0.8100, 1.6080, 3.2042) -- (0.8560, 1.6080, 3.2085) -- (0.8560, 1.6620, 3.2077) -- (0.8100, 1.6620, 3.2034) -- cycle;
\fill[blue!57.1, opacity=0.7] (0.8100, 1.6620, 3.2034) -- (0.8560, 1.6620, 3.2077) -- (0.8560, 1.7160, 3.2066) -- (0.8100, 1.7160, 3.2022) -- cycle;
\fill[blue!55.9, opacity=0.7] (0.8100, 1.7160, 3.2022) -- (0.8560, 1.7160, 3.2066) -- (0.8560, 1.7700, 3.2051) -- (0.8100, 1.7700, 3.2008) -- cycle;
\fill[blue!54.5, opacity=0.7] (0.8100, 1.7700, 3.2008) -- (0.8560, 1.7700, 3.2051) -- (0.8560, 1.8240, 3.2033) -- (0.8100, 1.8240, 3.1990) -- cycle;
\fill[blue!53.3, opacity=0.7] (0.8100, 1.8240, 3.1990) -- (0.8560, 1.8240, 3.2033) -- (0.8560, 1.8780, 3.2012) -- (0.8100, 1.8780, 3.1969) -- cycle;
\fill[blue!53.1, opacity=0.7] (0.8100, 1.8780, 3.1969) -- (0.8560, 1.8780, 3.2012) -- (0.8560, 1.9320, 3.1988) -- (0.8100, 1.9320, 3.1945) -- cycle;
\fill[blue!54.9, opacity=0.7] (0.8100, 1.9320, 3.1945) -- (0.8560, 1.9320, 3.1988) -- (0.8560, 1.9860, 3.1961) -- (0.8100, 1.9860, 3.1918) -- cycle;
\fill[blue!58.5, opacity=0.7] (0.8100, 1.9860, 3.1918) -- (0.8560, 1.9860, 3.1961) -- (0.8560, 2.0400, 3.1931) -- (0.8100, 2.0400, 3.1888) -- cycle;
\fill[blue!62.6, opacity=0.7] (0.8100, 2.0400, 3.1888) -- (0.8560, 2.0400, 3.1931) -- (0.8560, 2.0940, 3.1898) -- (0.8100, 2.0940, 3.1855) -- cycle;
\fill[blue!63.1, opacity=0.7] (0.8100, 2.0940, 3.1855) -- (0.8560, 2.0940, 3.1898) -- (0.8560, 2.1480, 3.1863) -- (0.8100, 2.1480, 3.1819) -- cycle;
\fill[blue!57.3, opacity=0.7] (0.8100, 2.1480, 3.1819) -- (0.8560, 2.1480, 3.1863) -- (0.8560, 2.2020, 3.1824) -- (0.8100, 2.2020, 3.1781) -- cycle;
\fill[blue!48.5, opacity=0.7] (0.8100, 2.2020, 3.1781) -- (0.8560, 2.2020, 3.1824) -- (0.8560, 2.2560, 3.1784) -- (0.8100, 2.2560, 3.1740) -- cycle;
\fill[blue!42.8, opacity=0.7] (0.8100, 2.2560, 3.1740) -- (0.8560, 2.2560, 3.1784) -- (0.8560, 2.3100, 3.1740) -- (0.8100, 2.3100, 3.1697) -- cycle;
\fill[blue!43.8, opacity=0.7] (0.8100, 2.3100, 3.1697) -- (0.8560, 2.3100, 3.1740) -- (0.8560, 2.3640, 3.1695) -- (0.8100, 2.3640, 3.1651) -- cycle;
\fill[blue!52.6, opacity=0.7] (0.8100, 2.3640, 3.1651) -- (0.8560, 2.3640, 3.1695) -- (0.8560, 2.4180, 3.1647) -- (0.8100, 2.4180, 3.1604) -- cycle;
\fill[blue!63.0, opacity=0.7] (0.8100, 2.4180, 3.1604) -- (0.8560, 2.4180, 3.1647) -- (0.8560, 2.4720, 3.1597) -- (0.8100, 2.4720, 3.1554) -- cycle;
\fill[blue!57.0, opacity=0.7] (0.8100, 2.4720, 3.1554) -- (0.8560, 2.4720, 3.1597) -- (0.8560, 2.5260, 3.1545) -- (0.8100, 2.5260, 3.1502) -- cycle;
\fill[blue!38.0, opacity=0.7] (0.8100, 2.5260, 3.1502) -- (0.8560, 2.5260, 3.1545) -- (0.8560, 2.5800, 3.1492) -- (0.8100, 2.5800, 3.1449) -- cycle;
\fill[blue!26.3, opacity=0.7] (0.8100, 2.5800, 3.1449) -- (0.8560, 2.5800, 3.1492) -- (0.8560, 2.6340, 3.1437) -- (0.8100, 2.6340, 3.1393) -- cycle;
\fill[blue!23.8, opacity=0.7] (0.8100, 2.6340, 3.1393) -- (0.8560, 2.6340, 3.1437) -- (0.8560, 2.6880, 3.1380) -- (0.8100, 2.6880, 3.1337) -- cycle;
\fill[blue!28.1, opacity=0.7] (0.8100, 2.6880, 3.1337) -- (0.8560, 2.6880, 3.1380) -- (0.8560, 2.7420, 3.1322) -- (0.8100, 2.7420, 3.1279) -- cycle;
\fill[blue!41.4, opacity=0.7] (0.8100, 2.7420, 3.1279) -- (0.8560, 2.7420, 3.1322) -- (0.8560, 2.7960, 3.1263) -- (0.8100, 2.7960, 3.1219) -- cycle;
\fill[blue!57.7, opacity=0.7] (0.8100, 2.7960, 3.1219) -- (0.8560, 2.7960, 3.1263) -- (0.8560, 2.8500, 3.1202) -- (0.8100, 2.8500, 3.1159) -- cycle;
\fill[blue!63.4, opacity=0.7] (0.8100, 2.8500, 3.1159) -- (0.8560, 2.8500, 3.1202) -- (0.8560, 2.9040, 3.1141) -- (0.8100, 2.9040, 3.1098) -- cycle;
\fill[blue!63.5, opacity=0.7] (0.8100, 2.9040, 3.1098) -- (0.8560, 2.9040, 3.1141) -- (0.8560, 2.9580, 3.1079) -- (0.8100, 2.9580, 3.1036) -- cycle;
\fill[blue!59.4, opacity=0.7] (0.8100, 2.9580, 3.1036) -- (0.8560, 2.9580, 3.1079) -- (0.8560, 3.0120, 3.1017) -- (0.8100, 3.0120, 3.0974) -- cycle;
\fill[blue!39.6, opacity=0.7] (0.8100, 3.0120, 3.0974) -- (0.8560, 3.0120, 3.1017) -- (0.8560, 3.0660, 3.0955) -- (0.8100, 3.0660, 3.0911) -- cycle;
\fill[blue!21.4, opacity=0.7] (0.8100, 3.0660, 3.0911) -- (0.8560, 3.0660, 3.0955) -- (0.8560, 3.1200, 3.0892) -- (0.8100, 3.1200, 3.0849) -- cycle;
\fill[blue!61.5, opacity=0.7] (0.8560, -0.1200, 3.0892) -- (0.9020, -0.1200, 3.0933) -- (0.9020, -0.0660, 3.0995) -- (0.8560, -0.0660, 3.0955) -- cycle;
\fill[blue!57.1, opacity=0.7] (0.8560, -0.0660, 3.0955) -- (0.9020, -0.0660, 3.0995) -- (0.9020, -0.0120, 3.1058) -- (0.8560, -0.0120, 3.1017) -- cycle;
\fill[blue!39.6, opacity=0.7] (0.8560, -0.0120, 3.1017) -- (0.9020, -0.0120, 3.1058) -- (0.9020, 0.0420, 3.1120) -- (0.8560, 0.0420, 3.1079) -- cycle;
\fill[blue!23.9, opacity=0.7] (0.8560, 0.0420, 3.1079) -- (0.9020, 0.0420, 3.1120) -- (0.9020, 0.0960, 3.1182) -- (0.8560, 0.0960, 3.1141) -- cycle;
\fill[blue!19.4, opacity=0.7] (0.8560, 0.0960, 3.1141) -- (0.9020, 0.0960, 3.1182) -- (0.9020, 0.1500, 3.1243) -- (0.8560, 0.1500, 3.1202) -- cycle;
\fill[blue!21.5, opacity=0.7] (0.8560, 0.1500, 3.1202) -- (0.9020, 0.1500, 3.1243) -- (0.9020, 0.2040, 3.1303) -- (0.8560, 0.2040, 3.1263) -- cycle;
\fill[blue!35.0, opacity=0.7] (0.8560, 0.2040, 3.1263) -- (0.9020, 0.2040, 3.1303) -- (0.9020, 0.2580, 3.1363) -- (0.8560, 0.2580, 3.1322) -- cycle;
\fill[blue!59.0, opacity=0.7] (0.8560, 0.2580, 3.1322) -- (0.9020, 0.2580, 3.1363) -- (0.9020, 0.3120, 3.1421) -- (0.8560, 0.3120, 3.1380) -- cycle;
\fill[blue!61.3, opacity=0.7] (0.8560, 0.3120, 3.1380) -- (0.9020, 0.3120, 3.1421) -- (0.9020, 0.3660, 3.1477) -- (0.8560, 0.3660, 3.1437) -- cycle;
\fill[blue!54.5, opacity=0.7] (0.8560, 0.3660, 3.1437) -- (0.9020, 0.3660, 3.1477) -- (0.9020, 0.4200, 3.1533) -- (0.8560, 0.4200, 3.1492) -- cycle;
\fill[blue!58.7, opacity=0.7] (0.8560, 0.4200, 3.1492) -- (0.9020, 0.4200, 3.1533) -- (0.9020, 0.4740, 3.1586) -- (0.8560, 0.4740, 3.1545) -- cycle;
\fill[blue!63.3, opacity=0.7] (0.8560, 0.4740, 3.1545) -- (0.9020, 0.4740, 3.1586) -- (0.9020, 0.5280, 3.1638) -- (0.8560, 0.5280, 3.1597) -- cycle;
\fill[blue!51.5, opacity=0.7] (0.8560, 0.5280, 3.1597) -- (0.9020, 0.5280, 3.1638) -- (0.9020, 0.5820, 3.1688) -- (0.8560, 0.5820, 3.1647) -- cycle;
\fill[blue!38.6, opacity=0.7] (0.8560, 0.5820, 3.1647) -- (0.9020, 0.5820, 3.1688) -- (0.9020, 0.6360, 3.1736) -- (0.8560, 0.6360, 3.1695) -- cycle;
\fill[blue!37.8, opacity=0.7] (0.8560, 0.6360, 3.1695) -- (0.9020, 0.6360, 3.1736) -- (0.9020, 0.6900, 3.1781) -- (0.8560, 0.6900, 3.1740) -- cycle;
\fill[blue!49.8, opacity=0.7] (0.8560, 0.6900, 3.1740) -- (0.9020, 0.6900, 3.1781) -- (0.9020, 0.7440, 3.1824) -- (0.8560, 0.7440, 3.1784) -- cycle;
\fill[blue!63.3, opacity=0.7] (0.8560, 0.7440, 3.1784) -- (0.9020, 0.7440, 3.1824) -- (0.9020, 0.7980, 3.1865) -- (0.8560, 0.7980, 3.1824) -- cycle;
\fill[blue!54.0, opacity=0.7] (0.8560, 0.7980, 3.1824) -- (0.9020, 0.7980, 3.1865) -- (0.9020, 0.8520, 3.1903) -- (0.8560, 0.8520, 3.1863) -- cycle;
\fill[blue!37.1, opacity=0.7] (0.8560, 0.8520, 3.1863) -- (0.9020, 0.8520, 3.1903) -- (0.9020, 0.9060, 3.1939) -- (0.8560, 0.9060, 3.1898) -- cycle;
\fill[blue!30.3, opacity=0.7] (0.8560, 0.9060, 3.1898) -- (0.9020, 0.9060, 3.1939) -- (0.9020, 0.9600, 3.1972) -- (0.8560, 0.9600, 3.1931) -- cycle;
\fill[blue!32.5, opacity=0.7] (0.8560, 0.9600, 3.1931) -- (0.9020, 0.9600, 3.1972) -- (0.9020, 1.0140, 3.2002) -- (0.8560, 1.0140, 3.1961) -- cycle;
\fill[blue!42.6, opacity=0.7] (0.8560, 1.0140, 3.1961) -- (0.9020, 1.0140, 3.2002) -- (0.9020, 1.0680, 3.2029) -- (0.8560, 1.0680, 3.1988) -- cycle;
\fill[blue!56.2, opacity=0.7] (0.8560, 1.0680, 3.1988) -- (0.9020, 1.0680, 3.2029) -- (0.9020, 1.1220, 3.2053) -- (0.8560, 1.1220, 3.2012) -- cycle;
\fill[blue!63.3, opacity=0.7] (0.8560, 1.1220, 3.2012) -- (0.9020, 1.1220, 3.2053) -- (0.9020, 1.1760, 3.2074) -- (0.8560, 1.1760, 3.2033) -- cycle;
\fill[blue!62.0, opacity=0.7] (0.8560, 1.1760, 3.2033) -- (0.9020, 1.1760, 3.2074) -- (0.9020, 1.2300, 3.2092) -- (0.8560, 1.2300, 3.2051) -- cycle;
\fill[blue!59.1, opacity=0.7] (0.8560, 1.2300, 3.2051) -- (0.9020, 1.2300, 3.2092) -- (0.9020, 1.2840, 3.2106) -- (0.8560, 1.2840, 3.2066) -- cycle;
\fill[blue!58.6, opacity=0.7] (0.8560, 1.2840, 3.2066) -- (0.9020, 1.2840, 3.2106) -- (0.9020, 1.3380, 3.2118) -- (0.8560, 1.3380, 3.2077) -- cycle;
\fill[blue!60.3, opacity=0.7] (0.8560, 1.3380, 3.2077) -- (0.9020, 1.3380, 3.2118) -- (0.9020, 1.3920, 3.2126) -- (0.8560, 1.3920, 3.2085) -- cycle;
\fill[blue!62.4, opacity=0.7] (0.8560, 1.3920, 3.2085) -- (0.9020, 1.3920, 3.2126) -- (0.9020, 1.4460, 3.2131) -- (0.8560, 1.4460, 3.2090) -- cycle;
\fill[blue!63.5, opacity=0.7] (0.8560, 1.4460, 3.2090) -- (0.9020, 1.4460, 3.2131) -- (0.9020, 1.5000, 3.2133) -- (0.8560, 1.5000, 3.2092) -- cycle;
\fill[blue!63.5, opacity=0.7] (0.8560, 1.5000, 3.2092) -- (0.9020, 1.5000, 3.2133) -- (0.9020, 1.5540, 3.2131) -- (0.8560, 1.5540, 3.2090) -- cycle;
\fill[blue!63.0, opacity=0.7] (0.8560, 1.5540, 3.2090) -- (0.9020, 1.5540, 3.2131) -- (0.9020, 1.6080, 3.2126) -- (0.8560, 1.6080, 3.2085) -- cycle;
\fill[blue!62.9, opacity=0.7] (0.8560, 1.6080, 3.2085) -- (0.9020, 1.6080, 3.2126) -- (0.9020, 1.6620, 3.2118) -- (0.8560, 1.6620, 3.2077) -- cycle;
\fill[blue!63.2, opacity=0.7] (0.8560, 1.6620, 3.2077) -- (0.9020, 1.6620, 3.2118) -- (0.9020, 1.7160, 3.2106) -- (0.8560, 1.7160, 3.2066) -- cycle;
\fill[blue!63.6, opacity=0.7] (0.8560, 1.7160, 3.2066) -- (0.9020, 1.7160, 3.2106) -- (0.9020, 1.7700, 3.2092) -- (0.8560, 1.7700, 3.2051) -- cycle;
\fill[blue!63.0, opacity=0.7] (0.8560, 1.7700, 3.2051) -- (0.9020, 1.7700, 3.2092) -- (0.9020, 1.8240, 3.2074) -- (0.8560, 1.8240, 3.2033) -- cycle;
\fill[blue!60.8, opacity=0.7] (0.8560, 1.8240, 3.2033) -- (0.9020, 1.8240, 3.2074) -- (0.9020, 1.8780, 3.2053) -- (0.8560, 1.8780, 3.2012) -- cycle;
\fill[blue!57.1, opacity=0.7] (0.8560, 1.8780, 3.2012) -- (0.9020, 1.8780, 3.2053) -- (0.9020, 1.9320, 3.2029) -- (0.8560, 1.9320, 3.1988) -- cycle;
\fill[blue!53.3, opacity=0.7] (0.8560, 1.9320, 3.1988) -- (0.9020, 1.9320, 3.2029) -- (0.9020, 1.9860, 3.2002) -- (0.8560, 1.9860, 3.1961) -- cycle;
\fill[blue!51.6, opacity=0.7] (0.8560, 1.9860, 3.1961) -- (0.9020, 1.9860, 3.2002) -- (0.9020, 2.0400, 3.1972) -- (0.8560, 2.0400, 3.1931) -- cycle;
\fill[blue!53.3, opacity=0.7] (0.8560, 2.0400, 3.1931) -- (0.9020, 2.0400, 3.1972) -- (0.9020, 2.0940, 3.1939) -- (0.8560, 2.0940, 3.1898) -- cycle;
\fill[blue!58.5, opacity=0.7] (0.8560, 2.0940, 3.1898) -- (0.9020, 2.0940, 3.1939) -- (0.9020, 2.1480, 3.1903) -- (0.8560, 2.1480, 3.1863) -- cycle;
\fill[blue!63.2, opacity=0.7] (0.8560, 2.1480, 3.1863) -- (0.9020, 2.1480, 3.1903) -- (0.9020, 2.2020, 3.1865) -- (0.8560, 2.2020, 3.1824) -- cycle;
\fill[blue!61.1, opacity=0.7] (0.8560, 2.2020, 3.1824) -- (0.9020, 2.2020, 3.1865) -- (0.9020, 2.2560, 3.1824) -- (0.8560, 2.2560, 3.1784) -- cycle;
\fill[blue!52.1, opacity=0.7] (0.8560, 2.2560, 3.1784) -- (0.9020, 2.2560, 3.1824) -- (0.9020, 2.3100, 3.1781) -- (0.8560, 2.3100, 3.1740) -- cycle;
\fill[blue!44.3, opacity=0.7] (0.8560, 2.3100, 3.1740) -- (0.9020, 2.3100, 3.1781) -- (0.9020, 2.3640, 3.1736) -- (0.8560, 2.3640, 3.1695) -- cycle;
\fill[blue!43.9, opacity=0.7] (0.8560, 2.3640, 3.1695) -- (0.9020, 2.3640, 3.1736) -- (0.9020, 2.4180, 3.1688) -- (0.8560, 2.4180, 3.1647) -- cycle;
\fill[blue!52.6, opacity=0.7] (0.8560, 2.4180, 3.1647) -- (0.9020, 2.4180, 3.1688) -- (0.9020, 2.4720, 3.1638) -- (0.8560, 2.4720, 3.1597) -- cycle;
\fill[blue!63.2, opacity=0.7] (0.8560, 2.4720, 3.1597) -- (0.9020, 2.4720, 3.1638) -- (0.9020, 2.5260, 3.1586) -- (0.8560, 2.5260, 3.1545) -- cycle;
\fill[blue!55.2, opacity=0.7] (0.8560, 2.5260, 3.1545) -- (0.9020, 2.5260, 3.1586) -- (0.9020, 2.5800, 3.1533) -- (0.8560, 2.5800, 3.1492) -- cycle;
\fill[blue!35.4, opacity=0.7] (0.8560, 2.5800, 3.1492) -- (0.9020, 2.5800, 3.1533) -- (0.9020, 2.6340, 3.1477) -- (0.8560, 2.6340, 3.1437) -- cycle;
\fill[blue!25.1, opacity=0.7] (0.8560, 2.6340, 3.1437) -- (0.9020, 2.6340, 3.1477) -- (0.9020, 2.6880, 3.1421) -- (0.8560, 2.6880, 3.1380) -- cycle;
\fill[blue!23.8, opacity=0.7] (0.8560, 2.6880, 3.1380) -- (0.9020, 2.6880, 3.1421) -- (0.9020, 2.7420, 3.1363) -- (0.8560, 2.7420, 3.1322) -- cycle;
\fill[blue!30.2, opacity=0.7] (0.8560, 2.7420, 3.1322) -- (0.9020, 2.7420, 3.1363) -- (0.9020, 2.7960, 3.1303) -- (0.8560, 2.7960, 3.1263) -- cycle;
\fill[blue!46.1, opacity=0.7] (0.8560, 2.7960, 3.1263) -- (0.9020, 2.7960, 3.1303) -- (0.9020, 2.8500, 3.1243) -- (0.8560, 2.8500, 3.1202) -- cycle;
\fill[blue!60.6, opacity=0.7] (0.8560, 2.8500, 3.1202) -- (0.9020, 2.8500, 3.1243) -- (0.9020, 2.9040, 3.1182) -- (0.8560, 2.9040, 3.1141) -- cycle;
\fill[blue!63.6, opacity=0.7] (0.8560, 2.9040, 3.1141) -- (0.9020, 2.9040, 3.1182) -- (0.9020, 2.9580, 3.1120) -- (0.8560, 2.9580, 3.1079) -- cycle;
\fill[blue!63.1, opacity=0.7] (0.8560, 2.9580, 3.1079) -- (0.9020, 2.9580, 3.1120) -- (0.9020, 3.0120, 3.1058) -- (0.8560, 3.0120, 3.1017) -- cycle;
\fill[blue!53.1, opacity=0.7] (0.8560, 3.0120, 3.1017) -- (0.9020, 3.0120, 3.1058) -- (0.9020, 3.0660, 3.0995) -- (0.8560, 3.0660, 3.0955) -- cycle;
\fill[blue!30.0, opacity=0.7] (0.8560, 3.0660, 3.0955) -- (0.9020, 3.0660, 3.0995) -- (0.9020, 3.1200, 3.0933) -- (0.8560, 3.1200, 3.0892) -- cycle;
\fill[blue!61.4, opacity=0.7] (0.9020, -0.1200, 3.0933) -- (0.9480, -0.1200, 3.0971) -- (0.9480, -0.0660, 3.1034) -- (0.9020, -0.0660, 3.0995) -- cycle;
\fill[blue!52.4, opacity=0.7] (0.9020, -0.0660, 3.0995) -- (0.9480, -0.0660, 3.1034) -- (0.9480, -0.0120, 3.1096) -- (0.9020, -0.0120, 3.1058) -- cycle;
\fill[blue!32.6, opacity=0.7] (0.9020, -0.0120, 3.1058) -- (0.9480, -0.0120, 3.1096) -- (0.9480, 0.0420, 3.1159) -- (0.9020, 0.0420, 3.1120) -- cycle;
\fill[blue!21.3, opacity=0.7] (0.9020, 0.0420, 3.1120) -- (0.9480, 0.0420, 3.1159) -- (0.9480, 0.0960, 3.1220) -- (0.9020, 0.0960, 3.1182) -- cycle;
\fill[blue!19.6, opacity=0.7] (0.9020, 0.0960, 3.1182) -- (0.9480, 0.0960, 3.1220) -- (0.9480, 0.1500, 3.1281) -- (0.9020, 0.1500, 3.1243) -- cycle;
\fill[blue!25.4, opacity=0.7] (0.9020, 0.1500, 3.1243) -- (0.9480, 0.1500, 3.1281) -- (0.9480, 0.2040, 3.1342) -- (0.9020, 0.2040, 3.1303) -- cycle;
\fill[blue!46.2, opacity=0.7] (0.9020, 0.2040, 3.1303) -- (0.9480, 0.2040, 3.1342) -- (0.9480, 0.2580, 3.1401) -- (0.9020, 0.2580, 3.1363) -- cycle;
\fill[blue!63.5, opacity=0.7] (0.9020, 0.2580, 3.1363) -- (0.9480, 0.2580, 3.1401) -- (0.9480, 0.3120, 3.1459) -- (0.9020, 0.3120, 3.1421) -- cycle;
\fill[blue!56.9, opacity=0.7] (0.9020, 0.3120, 3.1421) -- (0.9480, 0.3120, 3.1459) -- (0.9480, 0.3660, 3.1516) -- (0.9020, 0.3660, 3.1477) -- cycle;
\fill[blue!54.9, opacity=0.7] (0.9020, 0.3660, 3.1477) -- (0.9480, 0.3660, 3.1516) -- (0.9480, 0.4200, 3.1571) -- (0.9020, 0.4200, 3.1533) -- cycle;
\fill[blue!62.4, opacity=0.7] (0.9020, 0.4200, 3.1533) -- (0.9480, 0.4200, 3.1571) -- (0.9480, 0.4740, 3.1624) -- (0.9020, 0.4740, 3.1586) -- cycle;
\fill[blue!58.9, opacity=0.7] (0.9020, 0.4740, 3.1586) -- (0.9480, 0.4740, 3.1624) -- (0.9480, 0.5280, 3.1676) -- (0.9020, 0.5280, 3.1638) -- cycle;
\fill[blue!43.3, opacity=0.7] (0.9020, 0.5280, 3.1638) -- (0.9480, 0.5280, 3.1676) -- (0.9480, 0.5820, 3.1726) -- (0.9020, 0.5820, 3.1688) -- cycle;
\fill[blue!37.1, opacity=0.7] (0.9020, 0.5820, 3.1688) -- (0.9480, 0.5820, 3.1726) -- (0.9480, 0.6360, 3.1774) -- (0.9020, 0.6360, 3.1736) -- cycle;
\fill[blue!44.8, opacity=0.7] (0.9020, 0.6360, 3.1736) -- (0.9480, 0.6360, 3.1774) -- (0.9480, 0.6900, 3.1819) -- (0.9020, 0.6900, 3.1781) -- cycle;
\fill[blue!61.1, opacity=0.7] (0.9020, 0.6900, 3.1781) -- (0.9480, 0.6900, 3.1819) -- (0.9480, 0.7440, 3.1863) -- (0.9020, 0.7440, 3.1824) -- cycle;
\fill[blue!58.0, opacity=0.7] (0.9020, 0.7440, 3.1824) -- (0.9480, 0.7440, 3.1863) -- (0.9480, 0.7980, 3.1903) -- (0.9020, 0.7980, 3.1865) -- cycle;
\fill[blue!39.2, opacity=0.7] (0.9020, 0.7980, 3.1865) -- (0.9480, 0.7980, 3.1903) -- (0.9480, 0.8520, 3.1942) -- (0.9020, 0.8520, 3.1903) -- cycle;
\fill[blue!30.1, opacity=0.7] (0.9020, 0.8520, 3.1903) -- (0.9480, 0.8520, 3.1942) -- (0.9480, 0.9060, 3.1977) -- (0.9020, 0.9060, 3.1939) -- cycle;
\fill[blue!31.5, opacity=0.7] (0.9020, 0.9060, 3.1939) -- (0.9480, 0.9060, 3.1977) -- (0.9480, 0.9600, 3.2010) -- (0.9020, 0.9600, 3.1972) -- cycle;
\fill[blue!42.1, opacity=0.7] (0.9020, 0.9600, 3.1972) -- (0.9480, 0.9600, 3.2010) -- (0.9480, 1.0140, 3.2040) -- (0.9020, 1.0140, 3.2002) -- cycle;
\fill[blue!57.1, opacity=0.7] (0.9020, 1.0140, 3.2002) -- (0.9480, 1.0140, 3.2040) -- (0.9480, 1.0680, 3.2067) -- (0.9020, 1.0680, 3.2029) -- cycle;
\fill[blue!63.6, opacity=0.7] (0.9020, 1.0680, 3.2029) -- (0.9480, 1.0680, 3.2067) -- (0.9480, 1.1220, 3.2091) -- (0.9020, 1.1220, 3.2053) -- cycle;
\fill[blue!61.1, opacity=0.7] (0.9020, 1.1220, 3.2053) -- (0.9480, 1.1220, 3.2091) -- (0.9480, 1.1760, 3.2112) -- (0.9020, 1.1760, 3.2074) -- cycle;
\fill[blue!59.3, opacity=0.7] (0.9020, 1.1760, 3.2074) -- (0.9480, 1.1760, 3.2112) -- (0.9480, 1.2300, 3.2130) -- (0.9020, 1.2300, 3.2092) -- cycle;
\fill[blue!61.2, opacity=0.7] (0.9020, 1.2300, 3.2092) -- (0.9480, 1.2300, 3.2130) -- (0.9480, 1.2840, 3.2145) -- (0.9020, 1.2840, 3.2106) -- cycle;
\fill[blue!63.5, opacity=0.7] (0.9020, 1.2840, 3.2106) -- (0.9480, 1.2840, 3.2145) -- (0.9480, 1.3380, 3.2156) -- (0.9020, 1.3380, 3.2118) -- cycle;
\fill[blue!62.0, opacity=0.7] (0.9020, 1.3380, 3.2118) -- (0.9480, 1.3380, 3.2156) -- (0.9480, 1.3920, 3.2164) -- (0.9020, 1.3920, 3.2126) -- cycle;
\fill[blue!56.2, opacity=0.7] (0.9020, 1.3920, 3.2126) -- (0.9480, 1.3920, 3.2164) -- (0.9480, 1.4460, 3.2169) -- (0.9020, 1.4460, 3.2131) -- cycle;
\fill[blue!49.2, opacity=0.7] (0.9020, 1.4460, 3.2131) -- (0.9480, 1.4460, 3.2169) -- (0.9480, 1.5000, 3.2171) -- (0.9020, 1.5000, 3.2133) -- cycle;
\fill[blue!43.7, opacity=0.7] (0.9020, 1.5000, 3.2133) -- (0.9480, 1.5000, 3.2171) -- (0.9480, 1.5540, 3.2169) -- (0.9020, 1.5540, 3.2131) -- cycle;
\fill[blue!40.6, opacity=0.7] (0.9020, 1.5540, 3.2131) -- (0.9480, 1.5540, 3.2169) -- (0.9480, 1.6080, 3.2164) -- (0.9020, 1.6080, 3.2126) -- cycle;
\fill[blue!39.9, opacity=0.7] (0.9020, 1.6080, 3.2126) -- (0.9480, 1.6080, 3.2164) -- (0.9480, 1.6620, 3.2156) -- (0.9020, 1.6620, 3.2118) -- cycle;
\fill[blue!41.5, opacity=0.7] (0.9020, 1.6620, 3.2118) -- (0.9480, 1.6620, 3.2156) -- (0.9480, 1.7160, 3.2145) -- (0.9020, 1.7160, 3.2106) -- cycle;
\fill[blue!45.4, opacity=0.7] (0.9020, 1.7160, 3.2106) -- (0.9480, 1.7160, 3.2145) -- (0.9480, 1.7700, 3.2130) -- (0.9020, 1.7700, 3.2092) -- cycle;
\fill[blue!51.4, opacity=0.7] (0.9020, 1.7700, 3.2092) -- (0.9480, 1.7700, 3.2130) -- (0.9480, 1.8240, 3.2112) -- (0.9020, 1.8240, 3.2074) -- cycle;
\fill[blue!58.2, opacity=0.7] (0.9020, 1.8240, 3.2074) -- (0.9480, 1.8240, 3.2112) -- (0.9480, 1.8780, 3.2091) -- (0.9020, 1.8780, 3.2053) -- cycle;
\fill[blue!63.0, opacity=0.7] (0.9020, 1.8780, 3.2053) -- (0.9480, 1.8780, 3.2091) -- (0.9480, 1.9320, 3.2067) -- (0.9020, 1.9320, 3.2029) -- cycle;
\fill[blue!62.6, opacity=0.7] (0.9020, 1.9320, 3.2029) -- (0.9480, 1.9320, 3.2067) -- (0.9480, 1.9860, 3.2040) -- (0.9020, 1.9860, 3.2002) -- cycle;
\fill[blue!57.5, opacity=0.7] (0.9020, 1.9860, 3.2002) -- (0.9480, 1.9860, 3.2040) -- (0.9480, 2.0400, 3.2010) -- (0.9020, 2.0400, 3.1972) -- cycle;
\fill[blue!52.1, opacity=0.7] (0.9020, 2.0400, 3.1972) -- (0.9480, 2.0400, 3.2010) -- (0.9480, 2.0940, 3.1977) -- (0.9020, 2.0940, 3.1939) -- cycle;
\fill[blue!50.7, opacity=0.7] (0.9020, 2.0940, 3.1939) -- (0.9480, 2.0940, 3.1977) -- (0.9480, 2.1480, 3.1942) -- (0.9020, 2.1480, 3.1903) -- cycle;
\fill[blue!54.7, opacity=0.7] (0.9020, 2.1480, 3.1903) -- (0.9480, 2.1480, 3.1942) -- (0.9480, 2.2020, 3.1903) -- (0.9020, 2.2020, 3.1865) -- cycle;
\fill[blue!61.6, opacity=0.7] (0.9020, 2.2020, 3.1865) -- (0.9480, 2.2020, 3.1903) -- (0.9480, 2.2560, 3.1863) -- (0.9020, 2.2560, 3.1824) -- cycle;
\fill[blue!62.7, opacity=0.7] (0.9020, 2.2560, 3.1824) -- (0.9480, 2.2560, 3.1863) -- (0.9480, 2.3100, 3.1819) -- (0.9020, 2.3100, 3.1781) -- cycle;
\fill[blue!54.1, opacity=0.7] (0.9020, 2.3100, 3.1781) -- (0.9480, 2.3100, 3.1819) -- (0.9480, 2.3640, 3.1774) -- (0.9020, 2.3640, 3.1736) -- cycle;
\fill[blue!45.2, opacity=0.7] (0.9020, 2.3640, 3.1736) -- (0.9480, 2.3640, 3.1774) -- (0.9480, 2.4180, 3.1726) -- (0.9020, 2.4180, 3.1688) -- cycle;
\fill[blue!44.6, opacity=0.7] (0.9020, 2.4180, 3.1688) -- (0.9480, 2.4180, 3.1726) -- (0.9480, 2.4720, 3.1676) -- (0.9020, 2.4720, 3.1638) -- cycle;
\fill[blue!54.1, opacity=0.7] (0.9020, 2.4720, 3.1638) -- (0.9480, 2.4720, 3.1676) -- (0.9480, 2.5260, 3.1624) -- (0.9020, 2.5260, 3.1586) -- cycle;
\fill[blue!63.6, opacity=0.7] (0.9020, 2.5260, 3.1586) -- (0.9480, 2.5260, 3.1624) -- (0.9480, 2.5800, 3.1571) -- (0.9020, 2.5800, 3.1533) -- cycle;
\fill[blue!51.2, opacity=0.7] (0.9020, 2.5800, 3.1533) -- (0.9480, 2.5800, 3.1571) -- (0.9480, 2.6340, 3.1516) -- (0.9020, 2.6340, 3.1477) -- cycle;
\fill[blue!31.6, opacity=0.7] (0.9020, 2.6340, 3.1477) -- (0.9480, 2.6340, 3.1516) -- (0.9480, 2.6880, 3.1459) -- (0.9020, 2.6880, 3.1421) -- cycle;
\fill[blue!23.8, opacity=0.7] (0.9020, 2.6880, 3.1421) -- (0.9480, 2.6880, 3.1459) -- (0.9480, 2.7420, 3.1401) -- (0.9020, 2.7420, 3.1363) -- cycle;
\fill[blue!24.6, opacity=0.7] (0.9020, 2.7420, 3.1363) -- (0.9480, 2.7420, 3.1401) -- (0.9480, 2.7960, 3.1342) -- (0.9020, 2.7960, 3.1303) -- cycle;
\fill[blue!34.2, opacity=0.7] (0.9020, 2.7960, 3.1303) -- (0.9480, 2.7960, 3.1342) -- (0.9480, 2.8500, 3.1281) -- (0.9020, 2.8500, 3.1243) -- cycle;
\fill[blue!52.1, opacity=0.7] (0.9020, 2.8500, 3.1243) -- (0.9480, 2.8500, 3.1281) -- (0.9480, 2.9040, 3.1220) -- (0.9020, 2.9040, 3.1182) -- cycle;
\fill[blue!62.6, opacity=0.7] (0.9020, 2.9040, 3.1182) -- (0.9480, 2.9040, 3.1220) -- (0.9480, 2.9580, 3.1159) -- (0.9020, 2.9580, 3.1120) -- cycle;
\fill[blue!63.6, opacity=0.7] (0.9020, 2.9580, 3.1120) -- (0.9480, 2.9580, 3.1159) -- (0.9480, 3.0120, 3.1096) -- (0.9020, 3.0120, 3.1058) -- cycle;
\fill[blue!60.8, opacity=0.7] (0.9020, 3.0120, 3.1058) -- (0.9480, 3.0120, 3.1096) -- (0.9480, 3.0660, 3.1034) -- (0.9020, 3.0660, 3.0995) -- cycle;
\fill[blue!42.4, opacity=0.7] (0.9020, 3.0660, 3.0995) -- (0.9480, 3.0660, 3.1034) -- (0.9480, 3.1200, 3.0971) -- (0.9020, 3.1200, 3.0933) -- cycle;
\fill[blue!60.3, opacity=0.7] (0.9480, -0.1200, 3.0971) -- (0.9940, -0.1200, 3.1006) -- (0.9940, -0.0660, 3.1069) -- (0.9480, -0.0660, 3.1034) -- cycle;
\fill[blue!46.6, opacity=0.7] (0.9480, -0.0660, 3.1034) -- (0.9940, -0.0660, 3.1069) -- (0.9940, -0.0120, 3.1132) -- (0.9480, -0.0120, 3.1096) -- cycle;
\fill[blue!27.6, opacity=0.7] (0.9480, -0.0120, 3.1096) -- (0.9940, -0.0120, 3.1132) -- (0.9940, 0.0420, 3.1194) -- (0.9480, 0.0420, 3.1159) -- cycle;
\fill[blue!20.1, opacity=0.7] (0.9480, 0.0420, 3.1159) -- (0.9940, 0.0420, 3.1194) -- (0.9940, 0.0960, 3.1256) -- (0.9480, 0.0960, 3.1220) -- cycle;
\fill[blue!20.7, opacity=0.7] (0.9480, 0.0960, 3.1220) -- (0.9940, 0.0960, 3.1256) -- (0.9940, 0.1500, 3.1317) -- (0.9480, 0.1500, 3.1281) -- cycle;
\fill[blue!31.5, opacity=0.7] (0.9480, 0.1500, 3.1281) -- (0.9940, 0.1500, 3.1317) -- (0.9940, 0.2040, 3.1377) -- (0.9480, 0.2040, 3.1342) -- cycle;
\fill[blue!56.4, opacity=0.7] (0.9480, 0.2040, 3.1342) -- (0.9940, 0.2040, 3.1377) -- (0.9940, 0.2580, 3.1436) -- (0.9480, 0.2580, 3.1401) -- cycle;
\fill[blue!62.0, opacity=0.7] (0.9480, 0.2580, 3.1401) -- (0.9940, 0.2580, 3.1436) -- (0.9940, 0.3120, 3.1494) -- (0.9480, 0.3120, 3.1459) -- cycle;
\fill[blue!54.0, opacity=0.7] (0.9480, 0.3120, 3.1459) -- (0.9940, 0.3120, 3.1494) -- (0.9940, 0.3660, 3.1551) -- (0.9480, 0.3660, 3.1516) -- cycle;
\fill[blue!57.7, opacity=0.7] (0.9480, 0.3660, 3.1516) -- (0.9940, 0.3660, 3.1551) -- (0.9940, 0.4200, 3.1606) -- (0.9480, 0.4200, 3.1571) -- cycle;
\fill[blue!63.4, opacity=0.7] (0.9480, 0.4200, 3.1571) -- (0.9940, 0.4200, 3.1606) -- (0.9940, 0.4740, 3.1660) -- (0.9480, 0.4740, 3.1624) -- cycle;
\fill[blue!51.5, opacity=0.7] (0.9480, 0.4740, 3.1624) -- (0.9940, 0.4740, 3.1660) -- (0.9940, 0.5280, 3.1712) -- (0.9480, 0.5280, 3.1676) -- cycle;
\fill[blue!38.8, opacity=0.7] (0.9480, 0.5280, 3.1676) -- (0.9940, 0.5280, 3.1712) -- (0.9940, 0.5820, 3.1762) -- (0.9480, 0.5820, 3.1726) -- cycle;
\fill[blue!40.1, opacity=0.7] (0.9480, 0.5820, 3.1726) -- (0.9940, 0.5820, 3.1762) -- (0.9940, 0.6360, 3.1809) -- (0.9480, 0.6360, 3.1774) -- cycle;
\fill[blue!55.2, opacity=0.7] (0.9480, 0.6360, 3.1774) -- (0.9940, 0.6360, 3.1809) -- (0.9940, 0.6900, 3.1855) -- (0.9480, 0.6900, 3.1819) -- cycle;
\fill[blue!62.5, opacity=0.7] (0.9480, 0.6900, 3.1819) -- (0.9940, 0.6900, 3.1855) -- (0.9940, 0.7440, 3.1898) -- (0.9480, 0.7440, 3.1863) -- cycle;
\fill[blue!44.3, opacity=0.7] (0.9480, 0.7440, 3.1863) -- (0.9940, 0.7440, 3.1898) -- (0.9940, 0.7980, 3.1939) -- (0.9480, 0.7980, 3.1903) -- cycle;
\fill[blue!30.9, opacity=0.7] (0.9480, 0.7980, 3.1903) -- (0.9940, 0.7980, 3.1939) -- (0.9940, 0.8520, 3.1977) -- (0.9480, 0.8520, 3.1942) -- cycle;
\fill[blue!29.8, opacity=0.7] (0.9480, 0.8520, 3.1942) -- (0.9940, 0.8520, 3.1977) -- (0.9940, 0.9060, 3.2013) -- (0.9480, 0.9060, 3.1977) -- cycle;
\fill[blue!39.2, opacity=0.7] (0.9480, 0.9060, 3.1977) -- (0.9940, 0.9060, 3.2013) -- (0.9940, 0.9600, 3.2046) -- (0.9480, 0.9600, 3.2010) -- cycle;
\fill[blue!55.5, opacity=0.7] (0.9480, 0.9600, 3.2010) -- (0.9940, 0.9600, 3.2046) -- (0.9940, 1.0140, 3.2076) -- (0.9480, 1.0140, 3.2040) -- cycle;
\fill[blue!63.5, opacity=0.7] (0.9480, 1.0140, 3.2040) -- (0.9940, 1.0140, 3.2076) -- (0.9940, 1.0680, 3.2103) -- (0.9480, 1.0680, 3.2067) -- cycle;
\fill[blue!61.2, opacity=0.7] (0.9480, 1.0680, 3.2067) -- (0.9940, 1.0680, 3.2103) -- (0.9940, 1.1220, 3.2127) -- (0.9480, 1.1220, 3.2091) -- cycle;
\fill[blue!60.3, opacity=0.7] (0.9480, 1.1220, 3.2091) -- (0.9940, 1.1220, 3.2127) -- (0.9940, 1.1760, 3.2148) -- (0.9480, 1.1760, 3.2112) -- cycle;
\fill[blue!63.0, opacity=0.7] (0.9480, 1.1760, 3.2112) -- (0.9940, 1.1760, 3.2148) -- (0.9940, 1.2300, 3.2166) -- (0.9480, 1.2300, 3.2130) -- cycle;
\fill[blue!62.0, opacity=0.7] (0.9480, 1.2300, 3.2130) -- (0.9940, 1.2300, 3.2166) -- (0.9940, 1.2840, 3.2180) -- (0.9480, 1.2840, 3.2145) -- cycle;
\fill[blue!52.3, opacity=0.7] (0.9480, 1.2840, 3.2145) -- (0.9940, 1.2840, 3.2180) -- (0.9940, 1.3380, 3.2192) -- (0.9480, 1.3380, 3.2156) -- cycle;
\fill[blue!39.5, opacity=0.7] (0.9480, 1.3380, 3.2156) -- (0.9940, 1.3380, 3.2192) -- (0.9940, 1.3920, 3.2200) -- (0.9480, 1.3920, 3.2164) -- cycle;
\fill[blue!30.3, opacity=0.7] (0.9480, 1.3920, 3.2164) -- (0.9940, 1.3920, 3.2200) -- (0.9940, 1.4460, 3.2205) -- (0.9480, 1.4460, 3.2169) -- cycle;
\fill[blue!25.3, opacity=0.7] (0.9480, 1.4460, 3.2169) -- (0.9940, 1.4460, 3.2205) -- (0.9940, 1.5000, 3.2206) -- (0.9480, 1.5000, 3.2171) -- cycle;
\fill[blue!22.9, opacity=0.7] (0.9480, 1.5000, 3.2171) -- (0.9940, 1.5000, 3.2206) -- (0.9940, 1.5540, 3.2205) -- (0.9480, 1.5540, 3.2169) -- cycle;
\fill[blue!21.9, opacity=0.7] (0.9480, 1.5540, 3.2169) -- (0.9940, 1.5540, 3.2205) -- (0.9940, 1.6080, 3.2200) -- (0.9480, 1.6080, 3.2164) -- cycle;
\fill[blue!21.9, opacity=0.7] (0.9480, 1.6080, 3.2164) -- (0.9940, 1.6080, 3.2200) -- (0.9940, 1.6620, 3.2192) -- (0.9480, 1.6620, 3.2156) -- cycle;
\fill[blue!22.5, opacity=0.7] (0.9480, 1.6620, 3.2156) -- (0.9940, 1.6620, 3.2192) -- (0.9940, 1.7160, 3.2180) -- (0.9480, 1.7160, 3.2145) -- cycle;
\fill[blue!24.1, opacity=0.7] (0.9480, 1.7160, 3.2145) -- (0.9940, 1.7160, 3.2180) -- (0.9940, 1.7700, 3.2166) -- (0.9480, 1.7700, 3.2130) -- cycle;
\fill[blue!27.2, opacity=0.7] (0.9480, 1.7700, 3.2130) -- (0.9940, 1.7700, 3.2166) -- (0.9940, 1.8240, 3.2148) -- (0.9480, 1.8240, 3.2112) -- cycle;
\fill[blue!32.9, opacity=0.7] (0.9480, 1.8240, 3.2112) -- (0.9940, 1.8240, 3.2148) -- (0.9940, 1.8780, 3.2127) -- (0.9480, 1.8780, 3.2091) -- cycle;
\fill[blue!42.4, opacity=0.7] (0.9480, 1.8780, 3.2091) -- (0.9940, 1.8780, 3.2127) -- (0.9940, 1.9320, 3.2103) -- (0.9480, 1.9320, 3.2067) -- cycle;
\fill[blue!54.6, opacity=0.7] (0.9480, 1.9320, 3.2067) -- (0.9940, 1.9320, 3.2103) -- (0.9940, 1.9860, 3.2076) -- (0.9480, 1.9860, 3.2040) -- cycle;
\fill[blue!63.0, opacity=0.7] (0.9480, 1.9860, 3.2040) -- (0.9940, 1.9860, 3.2076) -- (0.9940, 2.0400, 3.2046) -- (0.9480, 2.0400, 3.2010) -- cycle;
\fill[blue!61.4, opacity=0.7] (0.9480, 2.0400, 3.2010) -- (0.9940, 2.0400, 3.2046) -- (0.9940, 2.0940, 3.2013) -- (0.9480, 2.0940, 3.1977) -- cycle;
\fill[blue!54.1, opacity=0.7] (0.9480, 2.0940, 3.1977) -- (0.9940, 2.0940, 3.2013) -- (0.9940, 2.1480, 3.1977) -- (0.9480, 2.1480, 3.1942) -- cycle;
\fill[blue!49.7, opacity=0.7] (0.9480, 2.1480, 3.1942) -- (0.9940, 2.1480, 3.1977) -- (0.9940, 2.2020, 3.1939) -- (0.9480, 2.2020, 3.1903) -- cycle;
\fill[blue!52.4, opacity=0.7] (0.9480, 2.2020, 3.1903) -- (0.9940, 2.2020, 3.1939) -- (0.9940, 2.2560, 3.1898) -- (0.9480, 2.2560, 3.1863) -- cycle;
\fill[blue!60.3, opacity=0.7] (0.9480, 2.2560, 3.1863) -- (0.9940, 2.2560, 3.1898) -- (0.9940, 2.3100, 3.1855) -- (0.9480, 2.3100, 3.1819) -- cycle;
\fill[blue!63.1, opacity=0.7] (0.9480, 2.3100, 3.1819) -- (0.9940, 2.3100, 3.1855) -- (0.9940, 2.3640, 3.1809) -- (0.9480, 2.3640, 3.1774) -- cycle;
\fill[blue!54.4, opacity=0.7] (0.9480, 2.3640, 3.1774) -- (0.9940, 2.3640, 3.1809) -- (0.9940, 2.4180, 3.1762) -- (0.9480, 2.4180, 3.1726) -- cycle;
\fill[blue!45.4, opacity=0.7] (0.9480, 2.4180, 3.1726) -- (0.9940, 2.4180, 3.1762) -- (0.9940, 2.4720, 3.1712) -- (0.9480, 2.4720, 3.1676) -- cycle;
\fill[blue!45.9, opacity=0.7] (0.9480, 2.4720, 3.1676) -- (0.9940, 2.4720, 3.1712) -- (0.9940, 2.5260, 3.1660) -- (0.9480, 2.5260, 3.1624) -- cycle;
\fill[blue!57.0, opacity=0.7] (0.9480, 2.5260, 3.1624) -- (0.9940, 2.5260, 3.1660) -- (0.9940, 2.5800, 3.1606) -- (0.9480, 2.5800, 3.1571) -- cycle;
\fill[blue!62.8, opacity=0.7] (0.9480, 2.5800, 3.1571) -- (0.9940, 2.5800, 3.1606) -- (0.9940, 2.6340, 3.1551) -- (0.9480, 2.6340, 3.1516) -- cycle;
\fill[blue!44.6, opacity=0.7] (0.9480, 2.6340, 3.1516) -- (0.9940, 2.6340, 3.1551) -- (0.9940, 2.6880, 3.1494) -- (0.9480, 2.6880, 3.1459) -- cycle;
\fill[blue!27.6, opacity=0.7] (0.9480, 2.6880, 3.1459) -- (0.9940, 2.6880, 3.1494) -- (0.9940, 2.7420, 3.1436) -- (0.9480, 2.7420, 3.1401) -- cycle;
\fill[blue!23.0, opacity=0.7] (0.9480, 2.7420, 3.1401) -- (0.9940, 2.7420, 3.1436) -- (0.9940, 2.7960, 3.1377) -- (0.9480, 2.7960, 3.1342) -- cycle;
\fill[blue!26.6, opacity=0.7] (0.9480, 2.7960, 3.1342) -- (0.9940, 2.7960, 3.1377) -- (0.9940, 2.8500, 3.1317) -- (0.9480, 2.8500, 3.1281) -- cycle;
\fill[blue!40.6, opacity=0.7] (0.9480, 2.8500, 3.1281) -- (0.9940, 2.8500, 3.1317) -- (0.9940, 2.9040, 3.1256) -- (0.9480, 2.9040, 3.1220) -- cycle;
\fill[blue!58.1, opacity=0.7] (0.9480, 2.9040, 3.1220) -- (0.9940, 2.9040, 3.1256) -- (0.9940, 2.9580, 3.1194) -- (0.9480, 2.9580, 3.1159) -- cycle;
\fill[blue!63.4, opacity=0.7] (0.9480, 2.9580, 3.1159) -- (0.9940, 2.9580, 3.1194) -- (0.9940, 3.0120, 3.1132) -- (0.9480, 3.0120, 3.1096) -- cycle;
\fill[blue!63.2, opacity=0.7] (0.9480, 3.0120, 3.1096) -- (0.9940, 3.0120, 3.1132) -- (0.9940, 3.0660, 3.1069) -- (0.9480, 3.0660, 3.1034) -- cycle;
\fill[blue!53.9, opacity=0.7] (0.9480, 3.0660, 3.1034) -- (0.9940, 3.0660, 3.1069) -- (0.9940, 3.1200, 3.1006) -- (0.9480, 3.1200, 3.0971) -- cycle;
\fill[blue!58.3, opacity=0.7] (0.9940, -0.1200, 3.1006) -- (1.0400, -0.1200, 3.1039) -- (1.0400, -0.0660, 3.1102) -- (0.9940, -0.0660, 3.1069) -- cycle;
\fill[blue!40.8, opacity=0.7] (0.9940, -0.0660, 3.1069) -- (1.0400, -0.0660, 3.1102) -- (1.0400, -0.0120, 3.1165) -- (0.9940, -0.0120, 3.1132) -- cycle;
\fill[blue!24.3, opacity=0.7] (0.9940, -0.0120, 3.1132) -- (1.0400, -0.0120, 3.1165) -- (1.0400, 0.0420, 3.1227) -- (0.9940, 0.0420, 3.1194) -- cycle;
\fill[blue!19.7, opacity=0.7] (0.9940, 0.0420, 3.1194) -- (1.0400, 0.0420, 3.1227) -- (1.0400, 0.0960, 3.1289) -- (0.9940, 0.0960, 3.1256) -- cycle;
\fill[blue!22.7, opacity=0.7] (0.9940, 0.0960, 3.1256) -- (1.0400, 0.0960, 3.1289) -- (1.0400, 0.1500, 3.1350) -- (0.9940, 0.1500, 3.1317) -- cycle;
\fill[blue!39.5, opacity=0.7] (0.9940, 0.1500, 3.1317) -- (1.0400, 0.1500, 3.1350) -- (1.0400, 0.2040, 3.1410) -- (0.9940, 0.2040, 3.1377) -- cycle;
\fill[blue!62.3, opacity=0.7] (0.9940, 0.2040, 3.1377) -- (1.0400, 0.2040, 3.1410) -- (1.0400, 0.2580, 3.1469) -- (0.9940, 0.2580, 3.1436) -- cycle;
\fill[blue!58.3, opacity=0.7] (0.9940, 0.2580, 3.1436) -- (1.0400, 0.2580, 3.1469) -- (1.0400, 0.3120, 3.1527) -- (0.9940, 0.3120, 3.1494) -- cycle;
\fill[blue!53.5, opacity=0.7] (0.9940, 0.3120, 3.1494) -- (1.0400, 0.3120, 3.1527) -- (1.0400, 0.3660, 3.1584) -- (0.9940, 0.3660, 3.1551) -- cycle;
\fill[blue!61.0, opacity=0.7] (0.9940, 0.3660, 3.1551) -- (1.0400, 0.3660, 3.1584) -- (1.0400, 0.4200, 3.1639) -- (0.9940, 0.4200, 3.1606) -- cycle;
\fill[blue!60.6, opacity=0.7] (0.9940, 0.4200, 3.1606) -- (1.0400, 0.4200, 3.1639) -- (1.0400, 0.4740, 3.1693) -- (0.9940, 0.4740, 3.1660) -- cycle;
\fill[blue!44.9, opacity=0.7] (0.9940, 0.4740, 3.1660) -- (1.0400, 0.4740, 3.1693) -- (1.0400, 0.5280, 3.1745) -- (0.9940, 0.5280, 3.1712) -- cycle;
\fill[blue!38.0, opacity=0.7] (0.9940, 0.5280, 3.1712) -- (1.0400, 0.5280, 3.1745) -- (1.0400, 0.5820, 3.1794) -- (0.9940, 0.5820, 3.1762) -- cycle;
\fill[blue!46.8, opacity=0.7] (0.9940, 0.5820, 3.1762) -- (1.0400, 0.5820, 3.1794) -- (1.0400, 0.6360, 3.1842) -- (0.9940, 0.6360, 3.1809) -- cycle;
\fill[blue!62.8, opacity=0.7] (0.9940, 0.6360, 3.1809) -- (1.0400, 0.6360, 3.1842) -- (1.0400, 0.6900, 3.1888) -- (0.9940, 0.6900, 3.1855) -- cycle;
\fill[blue!53.1, opacity=0.7] (0.9940, 0.6900, 3.1855) -- (1.0400, 0.6900, 3.1888) -- (1.0400, 0.7440, 3.1931) -- (0.9940, 0.7440, 3.1898) -- cycle;
\fill[blue!34.0, opacity=0.7] (0.9940, 0.7440, 3.1898) -- (1.0400, 0.7440, 3.1931) -- (1.0400, 0.7980, 3.1972) -- (0.9940, 0.7980, 3.1939) -- cycle;
\fill[blue!28.5, opacity=0.7] (0.9940, 0.7980, 3.1939) -- (1.0400, 0.7980, 3.1972) -- (1.0400, 0.8520, 3.2010) -- (0.9940, 0.8520, 3.1977) -- cycle;
\fill[blue!34.6, opacity=0.7] (0.9940, 0.8520, 3.1977) -- (1.0400, 0.8520, 3.2010) -- (1.0400, 0.9060, 3.2046) -- (0.9940, 0.9060, 3.2013) -- cycle;
\fill[blue!51.0, opacity=0.7] (0.9940, 0.9060, 3.2013) -- (1.0400, 0.9060, 3.2046) -- (1.0400, 0.9600, 3.2078) -- (0.9940, 0.9600, 3.2046) -- cycle;
\fill[blue!63.0, opacity=0.7] (0.9940, 0.9600, 3.2046) -- (1.0400, 0.9600, 3.2078) -- (1.0400, 1.0140, 3.2108) -- (0.9940, 1.0140, 3.2076) -- cycle;
\fill[blue!61.8, opacity=0.7] (0.9940, 1.0140, 3.2076) -- (1.0400, 1.0140, 3.2108) -- (1.0400, 1.0680, 3.2135) -- (0.9940, 1.0680, 3.2103) -- cycle;
\fill[blue!60.9, opacity=0.7] (0.9940, 1.0680, 3.2103) -- (1.0400, 1.0680, 3.2135) -- (1.0400, 1.1220, 3.2160) -- (0.9940, 1.1220, 3.2127) -- cycle;
\fill[blue!63.5, opacity=0.7] (0.9940, 1.1220, 3.2127) -- (1.0400, 1.1220, 3.2160) -- (1.0400, 1.1760, 3.2180) -- (0.9940, 1.1760, 3.2148) -- cycle;
\fill[blue!58.5, opacity=0.7] (0.9940, 1.1760, 3.2148) -- (1.0400, 1.1760, 3.2180) -- (1.0400, 1.2300, 3.2198) -- (0.9940, 1.2300, 3.2166) -- cycle;
\fill[blue!42.5, opacity=0.7] (0.9940, 1.2300, 3.2166) -- (1.0400, 1.2300, 3.2198) -- (1.0400, 1.2840, 3.2213) -- (0.9940, 1.2840, 3.2180) -- cycle;
\fill[blue!28.6, opacity=0.7] (0.9940, 1.2840, 3.2180) -- (1.0400, 1.2840, 3.2213) -- (1.0400, 1.3380, 3.2224) -- (0.9940, 1.3380, 3.2192) -- cycle;
\fill[blue!21.8, opacity=0.7] (0.9940, 1.3380, 3.2192) -- (1.0400, 1.3380, 3.2224) -- (1.0400, 1.3920, 3.2233) -- (0.9940, 1.3920, 3.2200) -- cycle;
\fill[blue!19.3, opacity=0.7] (0.9940, 1.3920, 3.2200) -- (1.0400, 1.3920, 3.2233) -- (1.0400, 1.4460, 3.2238) -- (0.9940, 1.4460, 3.2205) -- cycle;
\fill[blue!18.5, opacity=0.7] (0.9940, 1.4460, 3.2205) -- (1.0400, 1.4460, 3.2238) -- (1.0400, 1.5000, 3.2239) -- (0.9940, 1.5000, 3.2206) -- cycle;
\fill[blue!18.5, opacity=0.7] (0.9940, 1.5000, 3.2206) -- (1.0400, 1.5000, 3.2239) -- (1.0400, 1.5540, 3.2238) -- (0.9940, 1.5540, 3.2205) -- cycle;
\fill[blue!18.7, opacity=0.7] (0.9940, 1.5540, 3.2205) -- (1.0400, 1.5540, 3.2238) -- (1.0400, 1.6080, 3.2233) -- (0.9940, 1.6080, 3.2200) -- cycle;
\fill[blue!19.0, opacity=0.7] (0.9940, 1.6080, 3.2200) -- (1.0400, 1.6080, 3.2233) -- (1.0400, 1.6620, 3.2224) -- (0.9940, 1.6620, 3.2192) -- cycle;
\fill[blue!19.2, opacity=0.7] (0.9940, 1.6620, 3.2192) -- (1.0400, 1.6620, 3.2224) -- (1.0400, 1.7160, 3.2213) -- (0.9940, 1.7160, 3.2180) -- cycle;
\fill[blue!19.5, opacity=0.7] (0.9940, 1.7160, 3.2180) -- (1.0400, 1.7160, 3.2213) -- (1.0400, 1.7700, 3.2198) -- (0.9940, 1.7700, 3.2166) -- cycle;
\fill[blue!20.0, opacity=0.7] (0.9940, 1.7700, 3.2166) -- (1.0400, 1.7700, 3.2198) -- (1.0400, 1.8240, 3.2180) -- (0.9940, 1.8240, 3.2148) -- cycle;
\fill[blue!21.2, opacity=0.7] (0.9940, 1.8240, 3.2148) -- (1.0400, 1.8240, 3.2180) -- (1.0400, 1.8780, 3.2160) -- (0.9940, 1.8780, 3.2127) -- cycle;
\fill[blue!24.3, opacity=0.7] (0.9940, 1.8780, 3.2127) -- (1.0400, 1.8780, 3.2160) -- (1.0400, 1.9320, 3.2135) -- (0.9940, 1.9320, 3.2103) -- cycle;
\fill[blue!31.2, opacity=0.7] (0.9940, 1.9320, 3.2103) -- (1.0400, 1.9320, 3.2135) -- (1.0400, 1.9860, 3.2108) -- (0.9940, 1.9860, 3.2076) -- cycle;
\fill[blue!44.2, opacity=0.7] (0.9940, 1.9860, 3.2076) -- (1.0400, 1.9860, 3.2108) -- (1.0400, 2.0400, 3.2078) -- (0.9940, 2.0400, 3.2046) -- cycle;
\fill[blue!59.1, opacity=0.7] (0.9940, 2.0400, 3.2046) -- (1.0400, 2.0400, 3.2078) -- (1.0400, 2.0940, 3.2046) -- (0.9940, 2.0940, 3.2013) -- cycle;
\fill[blue!63.1, opacity=0.7] (0.9940, 2.0940, 3.2013) -- (1.0400, 2.0940, 3.2046) -- (1.0400, 2.1480, 3.2010) -- (0.9940, 2.1480, 3.1977) -- cycle;
\fill[blue!55.5, opacity=0.7] (0.9940, 2.1480, 3.1977) -- (1.0400, 2.1480, 3.2010) -- (1.0400, 2.2020, 3.1972) -- (0.9940, 2.2020, 3.1939) -- cycle;
\fill[blue!49.2, opacity=0.7] (0.9940, 2.2020, 3.1939) -- (1.0400, 2.2020, 3.1972) -- (1.0400, 2.2560, 3.1931) -- (0.9940, 2.2560, 3.1898) -- cycle;
\fill[blue!51.3, opacity=0.7] (0.9940, 2.2560, 3.1898) -- (1.0400, 2.2560, 3.1931) -- (1.0400, 2.3100, 3.1888) -- (0.9940, 2.3100, 3.1855) -- cycle;
\fill[blue!60.0, opacity=0.7] (0.9940, 2.3100, 3.1855) -- (1.0400, 2.3100, 3.1888) -- (1.0400, 2.3640, 3.1842) -- (0.9940, 2.3640, 3.1809) -- cycle;
\fill[blue!62.9, opacity=0.7] (0.9940, 2.3640, 3.1809) -- (1.0400, 2.3640, 3.1842) -- (1.0400, 2.4180, 3.1794) -- (0.9940, 2.4180, 3.1762) -- cycle;
\fill[blue!53.2, opacity=0.7] (0.9940, 2.4180, 3.1762) -- (1.0400, 2.4180, 3.1794) -- (1.0400, 2.4720, 3.1745) -- (0.9940, 2.4720, 3.1712) -- cycle;
\fill[blue!45.1, opacity=0.7] (0.9940, 2.4720, 3.1712) -- (1.0400, 2.4720, 3.1745) -- (1.0400, 2.5260, 3.1693) -- (0.9940, 2.5260, 3.1660) -- cycle;
\fill[blue!48.4, opacity=0.7] (0.9940, 2.5260, 3.1660) -- (1.0400, 2.5260, 3.1693) -- (1.0400, 2.5800, 3.1639) -- (0.9940, 2.5800, 3.1606) -- cycle;
\fill[blue!60.9, opacity=0.7] (0.9940, 2.5800, 3.1606) -- (1.0400, 2.5800, 3.1639) -- (1.0400, 2.6340, 3.1584) -- (0.9940, 2.6340, 3.1551) -- cycle;
\fill[blue!58.5, opacity=0.7] (0.9940, 2.6340, 3.1551) -- (1.0400, 2.6340, 3.1584) -- (1.0400, 2.6880, 3.1527) -- (0.9940, 2.6880, 3.1494) -- cycle;
\fill[blue!36.5, opacity=0.7] (0.9940, 2.6880, 3.1494) -- (1.0400, 2.6880, 3.1527) -- (1.0400, 2.7420, 3.1469) -- (0.9940, 2.7420, 3.1436) -- cycle;
\fill[blue!24.4, opacity=0.7] (0.9940, 2.7420, 3.1436) -- (1.0400, 2.7420, 3.1469) -- (1.0400, 2.7960, 3.1410) -- (0.9940, 2.7960, 3.1377) -- cycle;
\fill[blue!23.3, opacity=0.7] (0.9940, 2.7960, 3.1377) -- (1.0400, 2.7960, 3.1410) -- (1.0400, 2.8500, 3.1350) -- (0.9940, 2.8500, 3.1317) -- cycle;
\fill[blue!31.1, opacity=0.7] (0.9940, 2.8500, 3.1317) -- (1.0400, 2.8500, 3.1350) -- (1.0400, 2.9040, 3.1289) -- (0.9940, 2.9040, 3.1256) -- cycle;
\fill[blue!49.2, opacity=0.7] (0.9940, 2.9040, 3.1256) -- (1.0400, 2.9040, 3.1289) -- (1.0400, 2.9580, 3.1227) -- (0.9940, 2.9580, 3.1194) -- cycle;
\fill[blue!62.0, opacity=0.7] (0.9940, 2.9580, 3.1194) -- (1.0400, 2.9580, 3.1227) -- (1.0400, 3.0120, 3.1165) -- (0.9940, 3.0120, 3.1132) -- cycle;
\fill[blue!63.5, opacity=0.7] (0.9940, 3.0120, 3.1132) -- (1.0400, 3.0120, 3.1165) -- (1.0400, 3.0660, 3.1102) -- (0.9940, 3.0660, 3.1069) -- cycle;
\fill[blue!60.4, opacity=0.7] (0.9940, 3.0660, 3.1069) -- (1.0400, 3.0660, 3.1102) -- (1.0400, 3.1200, 3.1039) -- (0.9940, 3.1200, 3.1006) -- cycle;
\fill[blue!55.5, opacity=0.7] (1.0400, -0.1200, 3.1039) -- (1.0860, -0.1200, 3.1069) -- (1.0860, -0.0660, 3.1132) -- (1.0400, -0.0660, 3.1102) -- cycle;
\fill[blue!35.7, opacity=0.7] (1.0400, -0.0660, 3.1102) -- (1.0860, -0.0660, 3.1132) -- (1.0860, -0.0120, 3.1195) -- (1.0400, -0.0120, 3.1165) -- cycle;
\fill[blue!22.3, opacity=0.7] (1.0400, -0.0120, 3.1165) -- (1.0860, -0.0120, 3.1195) -- (1.0860, 0.0420, 3.1257) -- (1.0400, 0.0420, 3.1227) -- cycle;
\fill[blue!19.9, opacity=0.7] (1.0400, 0.0420, 3.1227) -- (1.0860, 0.0420, 3.1257) -- (1.0860, 0.0960, 3.1319) -- (1.0400, 0.0960, 3.1289) -- cycle;
\fill[blue!25.8, opacity=0.7] (1.0400, 0.0960, 3.1289) -- (1.0860, 0.0960, 3.1319) -- (1.0860, 0.1500, 3.1380) -- (1.0400, 0.1500, 3.1350) -- cycle;
\fill[blue!48.0, opacity=0.7] (1.0400, 0.1500, 3.1350) -- (1.0860, 0.1500, 3.1380) -- (1.0860, 0.2040, 3.1440) -- (1.0400, 0.2040, 3.1410) -- cycle;
\fill[blue!63.5, opacity=0.7] (1.0400, 0.2040, 3.1410) -- (1.0860, 0.2040, 3.1440) -- (1.0860, 0.2580, 3.1499) -- (1.0400, 0.2580, 3.1469) -- cycle;
\fill[blue!55.0, opacity=0.7] (1.0400, 0.2580, 3.1469) -- (1.0860, 0.2580, 3.1499) -- (1.0860, 0.3120, 3.1557) -- (1.0400, 0.3120, 3.1527) -- cycle;
\fill[blue!54.8, opacity=0.7] (1.0400, 0.3120, 3.1527) -- (1.0860, 0.3120, 3.1557) -- (1.0860, 0.3660, 3.1614) -- (1.0400, 0.3660, 3.1584) -- cycle;
\fill[blue!63.2, opacity=0.7] (1.0400, 0.3660, 3.1584) -- (1.0860, 0.3660, 3.1614) -- (1.0860, 0.4200, 3.1669) -- (1.0400, 0.4200, 3.1639) -- cycle;
\fill[blue!55.6, opacity=0.7] (1.0400, 0.4200, 3.1639) -- (1.0860, 0.4200, 3.1669) -- (1.0860, 0.4740, 3.1723) -- (1.0400, 0.4740, 3.1693) -- cycle;
\fill[blue!40.9, opacity=0.7] (1.0400, 0.4740, 3.1693) -- (1.0860, 0.4740, 3.1723) -- (1.0860, 0.5280, 3.1775) -- (1.0400, 0.5280, 3.1745) -- cycle;
\fill[blue!40.2, opacity=0.7] (1.0400, 0.5280, 3.1745) -- (1.0860, 0.5280, 3.1775) -- (1.0860, 0.5820, 3.1824) -- (1.0400, 0.5820, 3.1794) -- cycle;
\fill[blue!55.1, opacity=0.7] (1.0400, 0.5820, 3.1794) -- (1.0860, 0.5820, 3.1824) -- (1.0860, 0.6360, 3.1872) -- (1.0400, 0.6360, 3.1842) -- cycle;
\fill[blue!62.1, opacity=0.7] (1.0400, 0.6360, 3.1842) -- (1.0860, 0.6360, 3.1872) -- (1.0860, 0.6900, 3.1918) -- (1.0400, 0.6900, 3.1888) -- cycle;
\fill[blue!41.6, opacity=0.7] (1.0400, 0.6900, 3.1888) -- (1.0860, 0.6900, 3.1918) -- (1.0860, 0.7440, 3.1961) -- (1.0400, 0.7440, 3.1931) -- cycle;
\fill[blue!29.0, opacity=0.7] (1.0400, 0.7440, 3.1931) -- (1.0860, 0.7440, 3.1961) -- (1.0860, 0.7980, 3.2002) -- (1.0400, 0.7980, 3.1972) -- cycle;
\fill[blue!30.1, opacity=0.7] (1.0400, 0.7980, 3.1972) -- (1.0860, 0.7980, 3.2002) -- (1.0860, 0.8520, 3.2040) -- (1.0400, 0.8520, 3.2010) -- cycle;
\fill[blue!43.3, opacity=0.7] (1.0400, 0.8520, 3.2010) -- (1.0860, 0.8520, 3.2040) -- (1.0860, 0.9060, 3.2076) -- (1.0400, 0.9060, 3.2046) -- cycle;
\fill[blue!60.4, opacity=0.7] (1.0400, 0.9060, 3.2046) -- (1.0860, 0.9060, 3.2076) -- (1.0860, 0.9600, 3.2108) -- (1.0400, 0.9600, 3.2078) -- cycle;
\fill[blue!62.9, opacity=0.7] (1.0400, 0.9600, 3.2078) -- (1.0860, 0.9600, 3.2108) -- (1.0860, 1.0140, 3.2138) -- (1.0400, 1.0140, 3.2108) -- cycle;
\fill[blue!61.1, opacity=0.7] (1.0400, 1.0140, 3.2108) -- (1.0860, 1.0140, 3.2138) -- (1.0860, 1.0680, 3.2165) -- (1.0400, 1.0680, 3.2135) -- cycle;
\fill[blue!63.5, opacity=0.7] (1.0400, 1.0680, 3.2135) -- (1.0860, 1.0680, 3.2165) -- (1.0860, 1.1220, 3.2190) -- (1.0400, 1.1220, 3.2160) -- cycle;
\fill[blue!57.2, opacity=0.7] (1.0400, 1.1220, 3.2160) -- (1.0860, 1.1220, 3.2190) -- (1.0860, 1.1760, 3.2210) -- (1.0400, 1.1760, 3.2180) -- cycle;
\fill[blue!37.8, opacity=0.7] (1.0400, 1.1760, 3.2180) -- (1.0860, 1.1760, 3.2210) -- (1.0860, 1.2300, 3.2228) -- (1.0400, 1.2300, 3.2198) -- cycle;
\fill[blue!24.0, opacity=0.7] (1.0400, 1.2300, 3.2198) -- (1.0860, 1.2300, 3.2228) -- (1.0860, 1.2840, 3.2243) -- (1.0400, 1.2840, 3.2213) -- cycle;
\fill[blue!19.1, opacity=0.7] (1.0400, 1.2840, 3.2213) -- (1.0860, 1.2840, 3.2243) -- (1.0860, 1.3380, 3.2254) -- (1.0400, 1.3380, 3.2224) -- cycle;
\fill[blue!17.9, opacity=0.7] (1.0400, 1.3380, 3.2224) -- (1.0860, 1.3380, 3.2254) -- (1.0860, 1.3920, 3.2263) -- (1.0400, 1.3920, 3.2233) -- cycle;
\fill[blue!18.1, opacity=0.7] (1.0400, 1.3920, 3.2233) -- (1.0860, 1.3920, 3.2263) -- (1.0860, 1.4460, 3.2268) -- (1.0400, 1.4460, 3.2238) -- cycle;
\fill[blue!19.3, opacity=0.7] (1.0400, 1.4460, 3.2238) -- (1.0860, 1.4460, 3.2268) -- (1.0860, 1.5000, 3.2269) -- (1.0400, 1.5000, 3.2239) -- cycle;
\fill[blue!21.1, opacity=0.7] (1.0400, 1.5000, 3.2239) -- (1.0860, 1.5000, 3.2269) -- (1.0860, 1.5540, 3.2268) -- (1.0400, 1.5540, 3.2238) -- cycle;
\fill[blue!22.9, opacity=0.7] (1.0400, 1.5540, 3.2238) -- (1.0860, 1.5540, 3.2268) -- (1.0860, 1.6080, 3.2263) -- (1.0400, 1.6080, 3.2233) -- cycle;
\fill[blue!24.0, opacity=0.7] (1.0400, 1.6080, 3.2233) -- (1.0860, 1.6080, 3.2263) -- (1.0860, 1.6620, 3.2254) -- (1.0400, 1.6620, 3.2224) -- cycle;
\fill[blue!24.0, opacity=0.7] (1.0400, 1.6620, 3.2224) -- (1.0860, 1.6620, 3.2254) -- (1.0860, 1.7160, 3.2243) -- (1.0400, 1.7160, 3.2213) -- cycle;
\fill[blue!23.1, opacity=0.7] (1.0400, 1.7160, 3.2213) -- (1.0860, 1.7160, 3.2243) -- (1.0860, 1.7700, 3.2228) -- (1.0400, 1.7700, 3.2198) -- cycle;
\fill[blue!21.8, opacity=0.7] (1.0400, 1.7700, 3.2198) -- (1.0860, 1.7700, 3.2228) -- (1.0860, 1.8240, 3.2210) -- (1.0400, 1.8240, 3.2180) -- cycle;
\fill[blue!20.6, opacity=0.7] (1.0400, 1.8240, 3.2180) -- (1.0860, 1.8240, 3.2210) -- (1.0860, 1.8780, 3.2190) -- (1.0400, 1.8780, 3.2160) -- cycle;
\fill[blue!20.3, opacity=0.7] (1.0400, 1.8780, 3.2160) -- (1.0860, 1.8780, 3.2190) -- (1.0860, 1.9320, 3.2165) -- (1.0400, 1.9320, 3.2135) -- cycle;
\fill[blue!21.6, opacity=0.7] (1.0400, 1.9320, 3.2135) -- (1.0860, 1.9320, 3.2165) -- (1.0860, 1.9860, 3.2138) -- (1.0400, 1.9860, 3.2108) -- cycle;
\fill[blue!26.2, opacity=0.7] (1.0400, 1.9860, 3.2108) -- (1.0860, 1.9860, 3.2138) -- (1.0860, 2.0400, 3.2108) -- (1.0400, 2.0400, 3.2078) -- cycle;
\fill[blue!37.9, opacity=0.7] (1.0400, 2.0400, 3.2078) -- (1.0860, 2.0400, 3.2108) -- (1.0860, 2.0940, 3.2076) -- (1.0400, 2.0940, 3.2046) -- cycle;
\fill[blue!55.6, opacity=0.7] (1.0400, 2.0940, 3.2046) -- (1.0860, 2.0940, 3.2076) -- (1.0860, 2.1480, 3.2040) -- (1.0400, 2.1480, 3.2010) -- cycle;
\fill[blue!63.5, opacity=0.7] (1.0400, 2.1480, 3.2010) -- (1.0860, 2.1480, 3.2040) -- (1.0860, 2.2020, 3.2002) -- (1.0400, 2.2020, 3.1972) -- cycle;
\fill[blue!55.4, opacity=0.7] (1.0400, 2.2020, 3.1972) -- (1.0860, 2.2020, 3.2002) -- (1.0860, 2.2560, 3.1961) -- (1.0400, 2.2560, 3.1931) -- cycle;
\fill[blue!48.4, opacity=0.7] (1.0400, 2.2560, 3.1931) -- (1.0860, 2.2560, 3.1961) -- (1.0860, 2.3100, 3.1918) -- (1.0400, 2.3100, 3.1888) -- cycle;
\fill[blue!51.4, opacity=0.7] (1.0400, 2.3100, 3.1888) -- (1.0860, 2.3100, 3.1918) -- (1.0860, 2.3640, 3.1872) -- (1.0400, 2.3640, 3.1842) -- cycle;
\fill[blue!61.0, opacity=0.7] (1.0400, 2.3640, 3.1842) -- (1.0860, 2.3640, 3.1872) -- (1.0860, 2.4180, 3.1824) -- (1.0400, 2.4180, 3.1794) -- cycle;
\fill[blue!61.9, opacity=0.7] (1.0400, 2.4180, 3.1794) -- (1.0860, 2.4180, 3.1824) -- (1.0860, 2.4720, 3.1775) -- (1.0400, 2.4720, 3.1745) -- cycle;
\fill[blue!50.8, opacity=0.7] (1.0400, 2.4720, 3.1745) -- (1.0860, 2.4720, 3.1775) -- (1.0860, 2.5260, 3.1723) -- (1.0400, 2.5260, 3.1693) -- cycle;
\fill[blue!45.3, opacity=0.7] (1.0400, 2.5260, 3.1693) -- (1.0860, 2.5260, 3.1723) -- (1.0860, 2.5800, 3.1669) -- (1.0400, 2.5800, 3.1639) -- cycle;
\fill[blue!52.8, opacity=0.7] (1.0400, 2.5800, 3.1639) -- (1.0860, 2.5800, 3.1669) -- (1.0860, 2.6340, 3.1614) -- (1.0400, 2.6340, 3.1584) -- cycle;
\fill[blue!63.5, opacity=0.7] (1.0400, 2.6340, 3.1584) -- (1.0860, 2.6340, 3.1614) -- (1.0860, 2.6880, 3.1557) -- (1.0400, 2.6880, 3.1527) -- cycle;
\fill[blue!49.4, opacity=0.7] (1.0400, 2.6880, 3.1527) -- (1.0860, 2.6880, 3.1557) -- (1.0860, 2.7420, 3.1499) -- (1.0400, 2.7420, 3.1469) -- cycle;
\fill[blue!29.1, opacity=0.7] (1.0400, 2.7420, 3.1469) -- (1.0860, 2.7420, 3.1499) -- (1.0860, 2.7960, 3.1440) -- (1.0400, 2.7960, 3.1410) -- cycle;
\fill[blue!22.7, opacity=0.7] (1.0400, 2.7960, 3.1410) -- (1.0860, 2.7960, 3.1440) -- (1.0860, 2.8500, 3.1380) -- (1.0400, 2.8500, 3.1350) -- cycle;
\fill[blue!25.5, opacity=0.7] (1.0400, 2.8500, 3.1350) -- (1.0860, 2.8500, 3.1380) -- (1.0860, 2.9040, 3.1319) -- (1.0400, 2.9040, 3.1289) -- cycle;
\fill[blue!39.1, opacity=0.7] (1.0400, 2.9040, 3.1289) -- (1.0860, 2.9040, 3.1319) -- (1.0860, 2.9580, 3.1257) -- (1.0400, 2.9580, 3.1227) -- cycle;
\fill[blue!57.6, opacity=0.7] (1.0400, 2.9580, 3.1227) -- (1.0860, 2.9580, 3.1257) -- (1.0860, 3.0120, 3.1195) -- (1.0400, 3.0120, 3.1165) -- cycle;
\fill[blue!63.4, opacity=0.7] (1.0400, 3.0120, 3.1165) -- (1.0860, 3.0120, 3.1195) -- (1.0860, 3.0660, 3.1132) -- (1.0400, 3.0660, 3.1102) -- cycle;
\fill[blue!62.9, opacity=0.7] (1.0400, 3.0660, 3.1102) -- (1.0860, 3.0660, 3.1132) -- (1.0860, 3.1200, 3.1069) -- (1.0400, 3.1200, 3.1039) -- cycle;
\fill[blue!52.4, opacity=0.7] (1.0860, -0.1200, 3.1069) -- (1.1320, -0.1200, 3.1096) -- (1.1320, -0.0660, 3.1159) -- (1.0860, -0.0660, 3.1132) -- cycle;
\fill[blue!31.8, opacity=0.7] (1.0860, -0.0660, 3.1132) -- (1.1320, -0.0660, 3.1159) -- (1.1320, -0.0120, 3.1222) -- (1.0860, -0.0120, 3.1195) -- cycle;
\fill[blue!21.2, opacity=0.7] (1.0860, -0.0120, 3.1195) -- (1.1320, -0.0120, 3.1222) -- (1.1320, 0.0420, 3.1284) -- (1.0860, 0.0420, 3.1257) -- cycle;
\fill[blue!20.5, opacity=0.7] (1.0860, 0.0420, 3.1257) -- (1.1320, 0.0420, 3.1284) -- (1.1320, 0.0960, 3.1346) -- (1.0860, 0.0960, 3.1319) -- cycle;
\fill[blue!29.9, opacity=0.7] (1.0860, 0.0960, 3.1319) -- (1.1320, 0.0960, 3.1346) -- (1.1320, 0.1500, 3.1407) -- (1.0860, 0.1500, 3.1380) -- cycle;
\fill[blue!55.1, opacity=0.7] (1.0860, 0.1500, 3.1380) -- (1.1320, 0.1500, 3.1407) -- (1.1320, 0.2040, 3.1467) -- (1.0860, 0.2040, 3.1440) -- cycle;
\fill[blue!62.0, opacity=0.7] (1.0860, 0.2040, 3.1440) -- (1.1320, 0.2040, 3.1467) -- (1.1320, 0.2580, 3.1526) -- (1.0860, 0.2580, 3.1499) -- cycle;
\fill[blue!53.0, opacity=0.7] (1.0860, 0.2580, 3.1499) -- (1.1320, 0.2580, 3.1526) -- (1.1320, 0.3120, 3.1584) -- (1.0860, 0.3120, 3.1557) -- cycle;
\fill[blue!57.0, opacity=0.7] (1.0860, 0.3120, 3.1557) -- (1.1320, 0.3120, 3.1584) -- (1.1320, 0.3660, 3.1641) -- (1.0860, 0.3660, 3.1614) -- cycle;
\fill[blue!63.4, opacity=0.7] (1.0860, 0.3660, 3.1614) -- (1.1320, 0.3660, 3.1641) -- (1.1320, 0.4200, 3.1696) -- (1.0860, 0.4200, 3.1669) -- cycle;
\fill[blue!50.5, opacity=0.7] (1.0860, 0.4200, 3.1669) -- (1.1320, 0.4200, 3.1696) -- (1.1320, 0.4740, 3.1750) -- (1.0860, 0.4740, 3.1723) -- cycle;
\fill[blue!39.3, opacity=0.7] (1.0860, 0.4740, 3.1723) -- (1.1320, 0.4740, 3.1750) -- (1.1320, 0.5280, 3.1802) -- (1.0860, 0.5280, 3.1775) -- cycle;
\fill[blue!44.6, opacity=0.7] (1.0860, 0.5280, 3.1775) -- (1.1320, 0.5280, 3.1802) -- (1.1320, 0.5820, 3.1851) -- (1.0860, 0.5820, 3.1824) -- cycle;
\fill[blue!61.5, opacity=0.7] (1.0860, 0.5820, 3.1824) -- (1.1320, 0.5820, 3.1851) -- (1.1320, 0.6360, 3.1899) -- (1.0860, 0.6360, 3.1872) -- cycle;
\fill[blue!54.7, opacity=0.7] (1.0860, 0.6360, 3.1872) -- (1.1320, 0.6360, 3.1899) -- (1.1320, 0.6900, 3.1945) -- (1.0860, 0.6900, 3.1918) -- cycle;
\fill[blue!33.7, opacity=0.7] (1.0860, 0.6900, 3.1918) -- (1.1320, 0.6900, 3.1945) -- (1.1320, 0.7440, 3.1988) -- (1.0860, 0.7440, 3.1961) -- cycle;
\fill[blue!27.6, opacity=0.7] (1.0860, 0.7440, 3.1961) -- (1.1320, 0.7440, 3.1988) -- (1.1320, 0.7980, 3.2029) -- (1.0860, 0.7980, 3.2002) -- cycle;
\fill[blue!34.5, opacity=0.7] (1.0860, 0.7980, 3.2002) -- (1.1320, 0.7980, 3.2029) -- (1.1320, 0.8520, 3.2067) -- (1.0860, 0.8520, 3.2040) -- cycle;
\fill[blue!52.9, opacity=0.7] (1.0860, 0.8520, 3.2040) -- (1.1320, 0.8520, 3.2067) -- (1.1320, 0.9060, 3.2103) -- (1.0860, 0.9060, 3.2076) -- cycle;
\fill[blue!63.5, opacity=0.7] (1.0860, 0.9060, 3.2076) -- (1.1320, 0.9060, 3.2103) -- (1.1320, 0.9600, 3.2135) -- (1.0860, 0.9600, 3.2108) -- cycle;
\fill[blue!61.6, opacity=0.7] (1.0860, 0.9600, 3.2108) -- (1.1320, 0.9600, 3.2135) -- (1.1320, 1.0140, 3.2165) -- (1.0860, 1.0140, 3.2138) -- cycle;
\fill[blue!63.1, opacity=0.7] (1.0860, 1.0140, 3.2138) -- (1.1320, 1.0140, 3.2165) -- (1.1320, 1.0680, 3.2193) -- (1.0860, 1.0680, 3.2165) -- cycle;
\fill[blue!59.2, opacity=0.7] (1.0860, 1.0680, 3.2165) -- (1.1320, 1.0680, 3.2193) -- (1.1320, 1.1220, 3.2217) -- (1.0860, 1.1220, 3.2190) -- cycle;
\fill[blue!38.5, opacity=0.7] (1.0860, 1.1220, 3.2190) -- (1.1320, 1.1220, 3.2217) -- (1.1320, 1.1760, 3.2238) -- (1.0860, 1.1760, 3.2210) -- cycle;
\fill[blue!22.9, opacity=0.7] (1.0860, 1.1760, 3.2210) -- (1.1320, 1.1760, 3.2238) -- (1.1320, 1.2300, 3.2255) -- (1.0860, 1.2300, 3.2228) -- cycle;
\fill[blue!18.2, opacity=0.7] (1.0860, 1.2300, 3.2228) -- (1.1320, 1.2300, 3.2255) -- (1.1320, 1.2840, 3.2270) -- (1.0860, 1.2840, 3.2243) -- cycle;
\fill[blue!17.5, opacity=0.7] (1.0860, 1.2840, 3.2243) -- (1.1320, 1.2840, 3.2270) -- (1.1320, 1.3380, 3.2281) -- (1.0860, 1.3380, 3.2254) -- cycle;
\fill[blue!18.8, opacity=0.7] (1.0860, 1.3380, 3.2254) -- (1.1320, 1.3380, 3.2281) -- (1.1320, 1.3920, 3.2290) -- (1.0860, 1.3920, 3.2263) -- cycle;
\fill[blue!22.3, opacity=0.7] (1.0860, 1.3920, 3.2263) -- (1.1320, 1.3920, 3.2290) -- (1.1320, 1.4460, 3.2295) -- (1.0860, 1.4460, 3.2268) -- cycle;
\fill[blue!28.3, opacity=0.7] (1.0860, 1.4460, 3.2268) -- (1.1320, 1.4460, 3.2295) -- (1.1320, 1.5000, 3.2296) -- (1.0860, 1.5000, 3.2269) -- cycle;
\fill[blue!34.9, opacity=0.7] (1.0860, 1.5000, 3.2269) -- (1.1320, 1.5000, 3.2296) -- (1.1320, 1.5540, 3.2295) -- (1.0860, 1.5540, 3.2268) -- cycle;
\fill[blue!40.0, opacity=0.7] (1.0860, 1.5540, 3.2268) -- (1.1320, 1.5540, 3.2295) -- (1.1320, 1.6080, 3.2290) -- (1.0860, 1.6080, 3.2263) -- cycle;
\fill[blue!42.5, opacity=0.7] (1.0860, 1.6080, 3.2263) -- (1.1320, 1.6080, 3.2290) -- (1.1320, 1.6620, 3.2281) -- (1.0860, 1.6620, 3.2254) -- cycle;
\fill[blue!42.3, opacity=0.7] (1.0860, 1.6620, 3.2254) -- (1.1320, 1.6620, 3.2281) -- (1.1320, 1.7160, 3.2270) -- (1.0860, 1.7160, 3.2243) -- cycle;
\fill[blue!39.4, opacity=0.7] (1.0860, 1.7160, 3.2243) -- (1.1320, 1.7160, 3.2270) -- (1.1320, 1.7700, 3.2255) -- (1.0860, 1.7700, 3.2228) -- cycle;
\fill[blue!34.4, opacity=0.7] (1.0860, 1.7700, 3.2228) -- (1.1320, 1.7700, 3.2255) -- (1.1320, 1.8240, 3.2238) -- (1.0860, 1.8240, 3.2210) -- cycle;
\fill[blue!28.6, opacity=0.7] (1.0860, 1.8240, 3.2210) -- (1.1320, 1.8240, 3.2238) -- (1.1320, 1.8780, 3.2217) -- (1.0860, 1.8780, 3.2190) -- cycle;
\fill[blue!23.9, opacity=0.7] (1.0860, 1.8780, 3.2190) -- (1.1320, 1.8780, 3.2217) -- (1.1320, 1.9320, 3.2193) -- (1.0860, 1.9320, 3.2165) -- cycle;
\fill[blue!21.3, opacity=0.7] (1.0860, 1.9320, 3.2165) -- (1.1320, 1.9320, 3.2193) -- (1.1320, 1.9860, 3.2165) -- (1.0860, 1.9860, 3.2138) -- cycle;
\fill[blue!21.2, opacity=0.7] (1.0860, 1.9860, 3.2138) -- (1.1320, 1.9860, 3.2165) -- (1.1320, 2.0400, 3.2135) -- (1.0860, 2.0400, 3.2108) -- cycle;
\fill[blue!24.6, opacity=0.7] (1.0860, 2.0400, 3.2108) -- (1.1320, 2.0400, 3.2135) -- (1.1320, 2.0940, 3.2103) -- (1.0860, 2.0940, 3.2076) -- cycle;
\fill[blue!35.7, opacity=0.7] (1.0860, 2.0940, 3.2076) -- (1.1320, 2.0940, 3.2103) -- (1.1320, 2.1480, 3.2067) -- (1.0860, 2.1480, 3.2040) -- cycle;
\fill[blue!55.2, opacity=0.7] (1.0860, 2.1480, 3.2040) -- (1.1320, 2.1480, 3.2067) -- (1.1320, 2.2020, 3.2029) -- (1.0860, 2.2020, 3.2002) -- cycle;
\fill[blue!63.3, opacity=0.7] (1.0860, 2.2020, 3.2002) -- (1.1320, 2.2020, 3.2029) -- (1.1320, 2.2560, 3.1988) -- (1.0860, 2.2560, 3.1961) -- cycle;
\fill[blue!53.7, opacity=0.7] (1.0860, 2.2560, 3.1961) -- (1.1320, 2.2560, 3.1988) -- (1.1320, 2.3100, 3.1945) -- (1.0860, 2.3100, 3.1918) -- cycle;
\fill[blue!47.5, opacity=0.7] (1.0860, 2.3100, 3.1918) -- (1.1320, 2.3100, 3.1945) -- (1.1320, 2.3640, 3.1899) -- (1.0860, 2.3640, 3.1872) -- cycle;
\fill[blue!52.9, opacity=0.7] (1.0860, 2.3640, 3.1872) -- (1.1320, 2.3640, 3.1899) -- (1.1320, 2.4180, 3.1851) -- (1.0860, 2.4180, 3.1824) -- cycle;
\fill[blue!62.8, opacity=0.7] (1.0860, 2.4180, 3.1824) -- (1.1320, 2.4180, 3.1851) -- (1.1320, 2.4720, 3.1802) -- (1.0860, 2.4720, 3.1775) -- cycle;
\fill[blue!59.1, opacity=0.7] (1.0860, 2.4720, 3.1775) -- (1.1320, 2.4720, 3.1802) -- (1.1320, 2.5260, 3.1750) -- (1.0860, 2.5260, 3.1723) -- cycle;
\fill[blue!47.9, opacity=0.7] (1.0860, 2.5260, 3.1723) -- (1.1320, 2.5260, 3.1750) -- (1.1320, 2.5800, 3.1696) -- (1.0860, 2.5800, 3.1669) -- cycle;
\fill[blue!47.2, opacity=0.7] (1.0860, 2.5800, 3.1669) -- (1.1320, 2.5800, 3.1696) -- (1.1320, 2.6340, 3.1641) -- (1.0860, 2.6340, 3.1614) -- cycle;
\fill[blue!59.1, opacity=0.7] (1.0860, 2.6340, 3.1614) -- (1.1320, 2.6340, 3.1641) -- (1.1320, 2.6880, 3.1584) -- (1.0860, 2.6880, 3.1557) -- cycle;
\fill[blue!60.3, opacity=0.7] (1.0860, 2.6880, 3.1557) -- (1.1320, 2.6880, 3.1584) -- (1.1320, 2.7420, 3.1526) -- (1.0860, 2.7420, 3.1499) -- cycle;
\fill[blue!37.5, opacity=0.7] (1.0860, 2.7420, 3.1499) -- (1.1320, 2.7420, 3.1526) -- (1.1320, 2.7960, 3.1467) -- (1.0860, 2.7960, 3.1440) -- cycle;
\fill[blue!24.2, opacity=0.7] (1.0860, 2.7960, 3.1440) -- (1.1320, 2.7960, 3.1467) -- (1.1320, 2.8500, 3.1407) -- (1.0860, 2.8500, 3.1380) -- cycle;
\fill[blue!22.9, opacity=0.7] (1.0860, 2.8500, 3.1380) -- (1.1320, 2.8500, 3.1407) -- (1.1320, 2.9040, 3.1346) -- (1.0860, 2.9040, 3.1319) -- cycle;
\fill[blue!31.1, opacity=0.7] (1.0860, 2.9040, 3.1319) -- (1.1320, 2.9040, 3.1346) -- (1.1320, 2.9580, 3.1284) -- (1.0860, 2.9580, 3.1257) -- cycle;
\fill[blue!50.1, opacity=0.7] (1.0860, 2.9580, 3.1257) -- (1.1320, 2.9580, 3.1284) -- (1.1320, 3.0120, 3.1222) -- (1.0860, 3.0120, 3.1195) -- cycle;
\fill[blue!62.3, opacity=0.7] (1.0860, 3.0120, 3.1195) -- (1.1320, 3.0120, 3.1222) -- (1.1320, 3.0660, 3.1159) -- (1.0860, 3.0660, 3.1132) -- cycle;
\fill[blue!63.4, opacity=0.7] (1.0860, 3.0660, 3.1132) -- (1.1320, 3.0660, 3.1159) -- (1.1320, 3.1200, 3.1096) -- (1.0860, 3.1200, 3.1069) -- cycle;
\fill[blue!49.2, opacity=0.7] (1.1320, -0.1200, 3.1096) -- (1.1780, -0.1200, 3.1120) -- (1.1780, -0.0660, 3.1183) -- (1.1320, -0.0660, 3.1159) -- cycle;
\fill[blue!29.0, opacity=0.7] (1.1320, -0.0660, 3.1159) -- (1.1780, -0.0660, 3.1183) -- (1.1780, -0.0120, 3.1246) -- (1.1320, -0.0120, 3.1222) -- cycle;
\fill[blue!20.6, opacity=0.7] (1.1320, -0.0120, 3.1222) -- (1.1780, -0.0120, 3.1246) -- (1.1780, 0.0420, 3.1308) -- (1.1320, 0.0420, 3.1284) -- cycle;
\fill[blue!21.5, opacity=0.7] (1.1320, 0.0420, 3.1284) -- (1.1780, 0.0420, 3.1308) -- (1.1780, 0.0960, 3.1370) -- (1.1320, 0.0960, 3.1346) -- cycle;
\fill[blue!34.4, opacity=0.7] (1.1320, 0.0960, 3.1346) -- (1.1780, 0.0960, 3.1370) -- (1.1780, 0.1500, 3.1431) -- (1.1320, 0.1500, 3.1407) -- cycle;
\fill[blue!59.9, opacity=0.7] (1.1320, 0.1500, 3.1407) -- (1.1780, 0.1500, 3.1431) -- (1.1780, 0.2040, 3.1491) -- (1.1320, 0.2040, 3.1467) -- cycle;
\fill[blue!59.5, opacity=0.7] (1.1320, 0.2040, 3.1467) -- (1.1780, 0.2040, 3.1491) -- (1.1780, 0.2580, 3.1550) -- (1.1320, 0.2580, 3.1526) -- cycle;
\fill[blue!52.2, opacity=0.7] (1.1320, 0.2580, 3.1526) -- (1.1780, 0.2580, 3.1550) -- (1.1780, 0.3120, 3.1608) -- (1.1320, 0.3120, 3.1584) -- cycle;
\fill[blue!59.3, opacity=0.7] (1.1320, 0.3120, 3.1584) -- (1.1780, 0.3120, 3.1608) -- (1.1780, 0.3660, 3.1665) -- (1.1320, 0.3660, 3.1641) -- cycle;
\fill[blue!62.0, opacity=0.7] (1.1320, 0.3660, 3.1641) -- (1.1780, 0.3660, 3.1665) -- (1.1780, 0.4200, 3.1720) -- (1.1320, 0.4200, 3.1696) -- cycle;
\fill[blue!46.6, opacity=0.7] (1.1320, 0.4200, 3.1696) -- (1.1780, 0.4200, 3.1720) -- (1.1780, 0.4740, 3.1774) -- (1.1320, 0.4740, 3.1750) -- cycle;
\fill[blue!39.5, opacity=0.7] (1.1320, 0.4740, 3.1750) -- (1.1780, 0.4740, 3.1774) -- (1.1780, 0.5280, 3.1826) -- (1.1320, 0.5280, 3.1802) -- cycle;
\fill[blue!49.9, opacity=0.7] (1.1320, 0.5280, 3.1802) -- (1.1780, 0.5280, 3.1826) -- (1.1780, 0.5820, 3.1875) -- (1.1320, 0.5820, 3.1851) -- cycle;
\fill[blue!63.6, opacity=0.7] (1.1320, 0.5820, 3.1851) -- (1.1780, 0.5820, 3.1875) -- (1.1780, 0.6360, 3.1923) -- (1.1320, 0.6360, 3.1899) -- cycle;
\fill[blue!46.0, opacity=0.7] (1.1320, 0.6360, 3.1899) -- (1.1780, 0.6360, 3.1923) -- (1.1780, 0.6900, 3.1969) -- (1.1320, 0.6900, 3.1945) -- cycle;
\fill[blue!29.4, opacity=0.7] (1.1320, 0.6900, 3.1945) -- (1.1780, 0.6900, 3.1969) -- (1.1780, 0.7440, 3.2012) -- (1.1320, 0.7440, 3.1988) -- cycle;
\fill[blue!28.3, opacity=0.7] (1.1320, 0.7440, 3.1988) -- (1.1780, 0.7440, 3.2012) -- (1.1780, 0.7980, 3.2053) -- (1.1320, 0.7980, 3.2029) -- cycle;
\fill[blue!40.7, opacity=0.7] (1.1320, 0.7980, 3.2029) -- (1.1780, 0.7980, 3.2053) -- (1.1780, 0.8520, 3.2091) -- (1.1320, 0.8520, 3.2067) -- cycle;
\fill[blue!59.7, opacity=0.7] (1.1320, 0.8520, 3.2067) -- (1.1780, 0.8520, 3.2091) -- (1.1780, 0.9060, 3.2127) -- (1.1320, 0.9060, 3.2103) -- cycle;
\fill[blue!63.0, opacity=0.7] (1.1320, 0.9060, 3.2103) -- (1.1780, 0.9060, 3.2127) -- (1.1780, 0.9600, 3.2160) -- (1.1320, 0.9600, 3.2135) -- cycle;
\fill[blue!62.2, opacity=0.7] (1.1320, 0.9600, 3.2135) -- (1.1780, 0.9600, 3.2160) -- (1.1780, 1.0140, 3.2190) -- (1.1320, 1.0140, 3.2165) -- cycle;
\fill[blue!62.6, opacity=0.7] (1.1320, 1.0140, 3.2165) -- (1.1780, 1.0140, 3.2190) -- (1.1780, 1.0680, 3.2217) -- (1.1320, 1.0680, 3.2193) -- cycle;
\fill[blue!44.8, opacity=0.7] (1.1320, 1.0680, 3.2193) -- (1.1780, 1.0680, 3.2217) -- (1.1780, 1.1220, 3.2241) -- (1.1320, 1.1220, 3.2217) -- cycle;
\fill[blue!24.5, opacity=0.7] (1.1320, 1.1220, 3.2217) -- (1.1780, 1.1220, 3.2241) -- (1.1780, 1.1760, 3.2262) -- (1.1320, 1.1760, 3.2238) -- cycle;
\fill[blue!18.1, opacity=0.7] (1.1320, 1.1760, 3.2238) -- (1.1780, 1.1760, 3.2262) -- (1.1780, 1.2300, 3.2279) -- (1.1320, 1.2300, 3.2255) -- cycle;
\fill[blue!17.3, opacity=0.7] (1.1320, 1.2300, 3.2255) -- (1.1780, 1.2300, 3.2279) -- (1.1780, 1.2840, 3.2294) -- (1.1320, 1.2840, 3.2270) -- cycle;
\fill[blue!19.1, opacity=0.7] (1.1320, 1.2840, 3.2270) -- (1.1780, 1.2840, 3.2294) -- (1.1780, 1.3380, 3.2306) -- (1.1320, 1.3380, 3.2281) -- cycle;
\fill[blue!25.3, opacity=0.7] (1.1320, 1.3380, 3.2281) -- (1.1780, 1.3380, 3.2306) -- (1.1780, 1.3920, 3.2314) -- (1.1320, 1.3920, 3.2290) -- cycle;
\fill[blue!36.5, opacity=0.7] (1.1320, 1.3920, 3.2290) -- (1.1780, 1.3920, 3.2314) -- (1.1780, 1.4460, 3.2319) -- (1.1320, 1.4460, 3.2295) -- cycle;
\fill[blue!47.6, opacity=0.7] (1.1320, 1.4460, 3.2295) -- (1.1780, 1.4460, 3.2319) -- (1.1780, 1.5000, 3.2320) -- (1.1320, 1.5000, 3.2296) -- cycle;
\fill[blue!54.5, opacity=0.7] (1.1320, 1.5000, 3.2296) -- (1.1780, 1.5000, 3.2320) -- (1.1780, 1.5540, 3.2319) -- (1.1320, 1.5540, 3.2295) -- cycle;
\fill[blue!57.8, opacity=0.7] (1.1320, 1.5540, 3.2295) -- (1.1780, 1.5540, 3.2319) -- (1.1780, 1.6080, 3.2314) -- (1.1320, 1.6080, 3.2290) -- cycle;
\fill[blue!59.2, opacity=0.7] (1.1320, 1.6080, 3.2290) -- (1.1780, 1.6080, 3.2314) -- (1.1780, 1.6620, 3.2306) -- (1.1320, 1.6620, 3.2281) -- cycle;
\fill[blue!59.4, opacity=0.7] (1.1320, 1.6620, 3.2281) -- (1.1780, 1.6620, 3.2306) -- (1.1780, 1.7160, 3.2294) -- (1.1320, 1.7160, 3.2270) -- cycle;
\fill[blue!58.3, opacity=0.7] (1.1320, 1.7160, 3.2270) -- (1.1780, 1.7160, 3.2294) -- (1.1780, 1.7700, 3.2279) -- (1.1320, 1.7700, 3.2255) -- cycle;
\fill[blue!55.0, opacity=0.7] (1.1320, 1.7700, 3.2255) -- (1.1780, 1.7700, 3.2279) -- (1.1780, 1.8240, 3.2262) -- (1.1320, 1.8240, 3.2238) -- cycle;
\fill[blue!48.1, opacity=0.7] (1.1320, 1.8240, 3.2238) -- (1.1780, 1.8240, 3.2262) -- (1.1780, 1.8780, 3.2241) -- (1.1320, 1.8780, 3.2217) -- cycle;
\fill[blue!37.9, opacity=0.7] (1.1320, 1.8780, 3.2217) -- (1.1780, 1.8780, 3.2241) -- (1.1780, 1.9320, 3.2217) -- (1.1320, 1.9320, 3.2193) -- cycle;
\fill[blue!28.3, opacity=0.7] (1.1320, 1.9320, 3.2193) -- (1.1780, 1.9320, 3.2217) -- (1.1780, 1.9860, 3.2190) -- (1.1320, 1.9860, 3.2165) -- cycle;
\fill[blue!22.8, opacity=0.7] (1.1320, 1.9860, 3.2165) -- (1.1780, 1.9860, 3.2190) -- (1.1780, 2.0400, 3.2160) -- (1.1320, 2.0400, 3.2135) -- cycle;
\fill[blue!21.4, opacity=0.7] (1.1320, 2.0400, 3.2135) -- (1.1780, 2.0400, 3.2160) -- (1.1780, 2.0940, 3.2127) -- (1.1320, 2.0940, 3.2103) -- cycle;
\fill[blue!24.7, opacity=0.7] (1.1320, 2.0940, 3.2103) -- (1.1780, 2.0940, 3.2127) -- (1.1780, 2.1480, 3.2091) -- (1.1320, 2.1480, 3.2067) -- cycle;
\fill[blue!37.0, opacity=0.7] (1.1320, 2.1480, 3.2067) -- (1.1780, 2.1480, 3.2091) -- (1.1780, 2.2020, 3.2053) -- (1.1320, 2.2020, 3.2029) -- cycle;
\fill[blue!57.8, opacity=0.7] (1.1320, 2.2020, 3.2029) -- (1.1780, 2.2020, 3.2053) -- (1.1780, 2.2560, 3.2012) -- (1.1320, 2.2560, 3.1988) -- cycle;
\fill[blue!61.9, opacity=0.7] (1.1320, 2.2560, 3.1988) -- (1.1780, 2.2560, 3.2012) -- (1.1780, 2.3100, 3.1969) -- (1.1320, 2.3100, 3.1945) -- cycle;
\fill[blue!50.5, opacity=0.7] (1.1320, 2.3100, 3.1945) -- (1.1780, 2.3100, 3.1969) -- (1.1780, 2.3640, 3.1923) -- (1.1320, 2.3640, 3.1899) -- cycle;
\fill[blue!47.4, opacity=0.7] (1.1320, 2.3640, 3.1899) -- (1.1780, 2.3640, 3.1923) -- (1.1780, 2.4180, 3.1875) -- (1.1320, 2.4180, 3.1851) -- cycle;
\fill[blue!56.3, opacity=0.7] (1.1320, 2.4180, 3.1851) -- (1.1780, 2.4180, 3.1875) -- (1.1780, 2.4720, 3.1826) -- (1.1320, 2.4720, 3.1802) -- cycle;
\fill[blue!63.5, opacity=0.7] (1.1320, 2.4720, 3.1802) -- (1.1780, 2.4720, 3.1826) -- (1.1780, 2.5260, 3.1774) -- (1.1320, 2.5260, 3.1750) -- cycle;
\fill[blue!54.3, opacity=0.7] (1.1320, 2.5260, 3.1750) -- (1.1780, 2.5260, 3.1774) -- (1.1780, 2.5800, 3.1720) -- (1.1320, 2.5800, 3.1696) -- cycle;
\fill[blue!46.3, opacity=0.7] (1.1320, 2.5800, 3.1696) -- (1.1780, 2.5800, 3.1720) -- (1.1780, 2.6340, 3.1665) -- (1.1320, 2.6340, 3.1641) -- cycle;
\fill[blue!52.4, opacity=0.7] (1.1320, 2.6340, 3.1641) -- (1.1780, 2.6340, 3.1665) -- (1.1780, 2.6880, 3.1608) -- (1.1320, 2.6880, 3.1584) -- cycle;
\fill[blue!63.5, opacity=0.7] (1.1320, 2.6880, 3.1584) -- (1.1780, 2.6880, 3.1608) -- (1.1780, 2.7420, 3.1550) -- (1.1320, 2.7420, 3.1526) -- cycle;
\fill[blue!48.5, opacity=0.7] (1.1320, 2.7420, 3.1526) -- (1.1780, 2.7420, 3.1550) -- (1.1780, 2.7960, 3.1491) -- (1.1320, 2.7960, 3.1467) -- cycle;
\fill[blue!27.9, opacity=0.7] (1.1320, 2.7960, 3.1467) -- (1.1780, 2.7960, 3.1491) -- (1.1780, 2.8500, 3.1431) -- (1.1320, 2.8500, 3.1407) -- cycle;
\fill[blue!22.3, opacity=0.7] (1.1320, 2.8500, 3.1407) -- (1.1780, 2.8500, 3.1431) -- (1.1780, 2.9040, 3.1370) -- (1.1320, 2.9040, 3.1346) -- cycle;
\fill[blue!26.1, opacity=0.7] (1.1320, 2.9040, 3.1346) -- (1.1780, 2.9040, 3.1370) -- (1.1780, 2.9580, 3.1308) -- (1.1320, 2.9580, 3.1284) -- cycle;
\fill[blue!41.6, opacity=0.7] (1.1320, 2.9580, 3.1284) -- (1.1780, 2.9580, 3.1308) -- (1.1780, 3.0120, 3.1246) -- (1.1320, 3.0120, 3.1222) -- cycle;
\fill[blue!59.4, opacity=0.7] (1.1320, 3.0120, 3.1222) -- (1.1780, 3.0120, 3.1246) -- (1.1780, 3.0660, 3.1183) -- (1.1320, 3.0660, 3.1159) -- cycle;
\fill[blue!63.4, opacity=0.7] (1.1320, 3.0660, 3.1159) -- (1.1780, 3.0660, 3.1183) -- (1.1780, 3.1200, 3.1120) -- (1.1320, 3.1200, 3.1096) -- cycle;
\fill[blue!46.4, opacity=0.7] (1.1780, -0.1200, 3.1120) -- (1.2240, -0.1200, 3.1141) -- (1.2240, -0.0660, 3.1204) -- (1.1780, -0.0660, 3.1183) -- cycle;
\fill[blue!27.1, opacity=0.7] (1.1780, -0.0660, 3.1183) -- (1.2240, -0.0660, 3.1204) -- (1.2240, -0.0120, 3.1267) -- (1.1780, -0.0120, 3.1246) -- cycle;
\fill[blue!20.4, opacity=0.7] (1.1780, -0.0120, 3.1246) -- (1.2240, -0.0120, 3.1267) -- (1.2240, 0.0420, 3.1329) -- (1.1780, 0.0420, 3.1308) -- cycle;
\fill[blue!22.6, opacity=0.7] (1.1780, 0.0420, 3.1308) -- (1.2240, 0.0420, 3.1329) -- (1.2240, 0.0960, 3.1391) -- (1.1780, 0.0960, 3.1370) -- cycle;
\fill[blue!38.9, opacity=0.7] (1.1780, 0.0960, 3.1370) -- (1.2240, 0.0960, 3.1391) -- (1.2240, 0.1500, 3.1452) -- (1.1780, 0.1500, 3.1431) -- cycle;
\fill[blue!62.4, opacity=0.7] (1.1780, 0.1500, 3.1431) -- (1.2240, 0.1500, 3.1452) -- (1.2240, 0.2040, 3.1512) -- (1.1780, 0.2040, 3.1491) -- cycle;
\fill[blue!57.1, opacity=0.7] (1.1780, 0.2040, 3.1491) -- (1.2240, 0.2040, 3.1512) -- (1.2240, 0.2580, 3.1571) -- (1.1780, 0.2580, 3.1550) -- cycle;
\fill[blue!52.2, opacity=0.7] (1.1780, 0.2580, 3.1550) -- (1.2240, 0.2580, 3.1571) -- (1.2240, 0.3120, 3.1629) -- (1.1780, 0.3120, 3.1608) -- cycle;
\fill[blue!61.0, opacity=0.7] (1.1780, 0.3120, 3.1608) -- (1.2240, 0.3120, 3.1629) -- (1.2240, 0.3660, 3.1686) -- (1.1780, 0.3660, 3.1665) -- cycle;
\fill[blue!59.8, opacity=0.7] (1.1780, 0.3660, 3.1665) -- (1.2240, 0.3660, 3.1686) -- (1.2240, 0.4200, 3.1741) -- (1.1780, 0.4200, 3.1720) -- cycle;
\fill[blue!44.1, opacity=0.7] (1.1780, 0.4200, 3.1720) -- (1.2240, 0.4200, 3.1741) -- (1.2240, 0.4740, 3.1795) -- (1.1780, 0.4740, 3.1774) -- cycle;
\fill[blue!40.8, opacity=0.7] (1.1780, 0.4740, 3.1774) -- (1.2240, 0.4740, 3.1795) -- (1.2240, 0.5280, 3.1847) -- (1.1780, 0.5280, 3.1826) -- cycle;
\fill[blue!55.0, opacity=0.7] (1.1780, 0.5280, 3.1826) -- (1.2240, 0.5280, 3.1847) -- (1.2240, 0.5820, 3.1896) -- (1.1780, 0.5820, 3.1875) -- cycle;
\fill[blue!61.6, opacity=0.7] (1.1780, 0.5820, 3.1875) -- (1.2240, 0.5820, 3.1896) -- (1.2240, 0.6360, 3.1944) -- (1.1780, 0.6360, 3.1923) -- cycle;
\fill[blue!39.1, opacity=0.7] (1.1780, 0.6360, 3.1923) -- (1.2240, 0.6360, 3.1944) -- (1.2240, 0.6900, 3.1990) -- (1.1780, 0.6900, 3.1969) -- cycle;
\fill[blue!27.3, opacity=0.7] (1.1780, 0.6900, 3.1969) -- (1.2240, 0.6900, 3.1990) -- (1.2240, 0.7440, 3.2033) -- (1.1780, 0.7440, 3.2012) -- cycle;
\fill[blue!30.2, opacity=0.7] (1.1780, 0.7440, 3.2012) -- (1.2240, 0.7440, 3.2033) -- (1.2240, 0.7980, 3.2074) -- (1.1780, 0.7980, 3.2053) -- cycle;
\fill[blue!47.1, opacity=0.7] (1.1780, 0.7980, 3.2053) -- (1.2240, 0.7980, 3.2074) -- (1.2240, 0.8520, 3.2112) -- (1.1780, 0.8520, 3.2091) -- cycle;
\fill[blue!62.7, opacity=0.7] (1.1780, 0.8520, 3.2091) -- (1.2240, 0.8520, 3.2112) -- (1.2240, 0.9060, 3.2148) -- (1.1780, 0.9060, 3.2127) -- cycle;
\fill[blue!62.4, opacity=0.7] (1.1780, 0.9060, 3.2127) -- (1.2240, 0.9060, 3.2148) -- (1.2240, 0.9600, 3.2180) -- (1.1780, 0.9600, 3.2160) -- cycle;
\fill[blue!63.4, opacity=0.7] (1.1780, 0.9600, 3.2160) -- (1.2240, 0.9600, 3.2180) -- (1.2240, 1.0140, 3.2210) -- (1.1780, 1.0140, 3.2190) -- cycle;
\fill[blue!55.6, opacity=0.7] (1.1780, 1.0140, 3.2190) -- (1.2240, 1.0140, 3.2210) -- (1.2240, 1.0680, 3.2238) -- (1.1780, 1.0680, 3.2217) -- cycle;
\fill[blue!30.8, opacity=0.7] (1.1780, 1.0680, 3.2217) -- (1.2240, 1.0680, 3.2238) -- (1.2240, 1.1220, 3.2262) -- (1.1780, 1.1220, 3.2241) -- cycle;
\fill[blue!19.0, opacity=0.7] (1.1780, 1.1220, 3.2241) -- (1.2240, 1.1220, 3.2262) -- (1.2240, 1.1760, 3.2283) -- (1.1780, 1.1760, 3.2262) -- cycle;
\fill[blue!17.0, opacity=0.7] (1.1780, 1.1760, 3.2262) -- (1.2240, 1.1760, 3.2283) -- (1.2240, 1.2300, 3.2300) -- (1.1780, 1.2300, 3.2279) -- cycle;
\fill[blue!18.3, opacity=0.7] (1.1780, 1.2300, 3.2279) -- (1.2240, 1.2300, 3.2300) -- (1.2240, 1.2840, 3.2315) -- (1.1780, 1.2840, 3.2294) -- cycle;
\fill[blue!25.2, opacity=0.7] (1.1780, 1.2840, 3.2294) -- (1.2240, 1.2840, 3.2315) -- (1.2240, 1.3380, 3.2326) -- (1.1780, 1.3380, 3.2306) -- cycle;
\fill[blue!39.8, opacity=0.7] (1.1780, 1.3380, 3.2306) -- (1.2240, 1.3380, 3.2326) -- (1.2240, 1.3920, 3.2335) -- (1.1780, 1.3920, 3.2314) -- cycle;
\fill[blue!52.9, opacity=0.7] (1.1780, 1.3920, 3.2314) -- (1.2240, 1.3920, 3.2335) -- (1.2240, 1.4460, 3.2340) -- (1.1780, 1.4460, 3.2319) -- cycle;
\fill[blue!58.4, opacity=0.7] (1.1780, 1.4460, 3.2319) -- (1.2240, 1.4460, 3.2340) -- (1.2240, 1.5000, 3.2341) -- (1.1780, 1.5000, 3.2320) -- cycle;
\fill[blue!59.6, opacity=0.7] (1.1780, 1.5000, 3.2320) -- (1.2240, 1.5000, 3.2341) -- (1.2240, 1.5540, 3.2340) -- (1.1780, 1.5540, 3.2319) -- cycle;
\fill[blue!59.7, opacity=0.7] (1.1780, 1.5540, 3.2319) -- (1.2240, 1.5540, 3.2340) -- (1.2240, 1.6080, 3.2335) -- (1.1780, 1.6080, 3.2314) -- cycle;
\fill[blue!60.1, opacity=0.7] (1.1780, 1.6080, 3.2314) -- (1.2240, 1.6080, 3.2335) -- (1.2240, 1.6620, 3.2326) -- (1.1780, 1.6620, 3.2306) -- cycle;
\fill[blue!61.1, opacity=0.7] (1.1780, 1.6620, 3.2306) -- (1.2240, 1.6620, 3.2326) -- (1.2240, 1.7160, 3.2315) -- (1.1780, 1.7160, 3.2294) -- cycle;
\fill[blue!62.1, opacity=0.7] (1.1780, 1.7160, 3.2294) -- (1.2240, 1.7160, 3.2315) -- (1.2240, 1.7700, 3.2300) -- (1.1780, 1.7700, 3.2279) -- cycle;
\fill[blue!62.4, opacity=0.7] (1.1780, 1.7700, 3.2279) -- (1.2240, 1.7700, 3.2300) -- (1.2240, 1.8240, 3.2283) -- (1.1780, 1.8240, 3.2262) -- cycle;
\fill[blue!61.3, opacity=0.7] (1.1780, 1.8240, 3.2262) -- (1.2240, 1.8240, 3.2283) -- (1.2240, 1.8780, 3.2262) -- (1.1780, 1.8780, 3.2241) -- cycle;
\fill[blue!56.3, opacity=0.7] (1.1780, 1.8780, 3.2241) -- (1.2240, 1.8780, 3.2262) -- (1.2240, 1.9320, 3.2238) -- (1.1780, 1.9320, 3.2217) -- cycle;
\fill[blue!44.9, opacity=0.7] (1.1780, 1.9320, 3.2217) -- (1.2240, 1.9320, 3.2238) -- (1.2240, 1.9860, 3.2210) -- (1.1780, 1.9860, 3.2190) -- cycle;
\fill[blue!31.5, opacity=0.7] (1.1780, 1.9860, 3.2190) -- (1.2240, 1.9860, 3.2210) -- (1.2240, 2.0400, 3.2180) -- (1.1780, 2.0400, 3.2160) -- cycle;
\fill[blue!23.7, opacity=0.7] (1.1780, 2.0400, 3.2160) -- (1.2240, 2.0400, 3.2180) -- (1.2240, 2.0940, 3.2148) -- (1.1780, 2.0940, 3.2127) -- cycle;
\fill[blue!21.9, opacity=0.7] (1.1780, 2.0940, 3.2127) -- (1.2240, 2.0940, 3.2148) -- (1.2240, 2.1480, 3.2112) -- (1.1780, 2.1480, 3.2091) -- cycle;
\fill[blue!26.2, opacity=0.7] (1.1780, 2.1480, 3.2091) -- (1.2240, 2.1480, 3.2112) -- (1.2240, 2.2020, 3.2074) -- (1.1780, 2.2020, 3.2053) -- cycle;
\fill[blue!42.0, opacity=0.7] (1.1780, 2.2020, 3.2053) -- (1.2240, 2.2020, 3.2074) -- (1.2240, 2.2560, 3.2033) -- (1.1780, 2.2560, 3.2012) -- cycle;
\fill[blue!62.0, opacity=0.7] (1.1780, 2.2560, 3.2012) -- (1.2240, 2.2560, 3.2033) -- (1.2240, 2.3100, 3.1990) -- (1.1780, 2.3100, 3.1969) -- cycle;
\fill[blue!57.9, opacity=0.7] (1.1780, 2.3100, 3.1969) -- (1.2240, 2.3100, 3.1990) -- (1.2240, 2.3640, 3.1944) -- (1.1780, 2.3640, 3.1923) -- cycle;
\fill[blue!47.2, opacity=0.7] (1.1780, 2.3640, 3.1923) -- (1.2240, 2.3640, 3.1944) -- (1.2240, 2.4180, 3.1896) -- (1.1780, 2.4180, 3.1875) -- cycle;
\fill[blue!49.4, opacity=0.7] (1.1780, 2.4180, 3.1875) -- (1.2240, 2.4180, 3.1896) -- (1.2240, 2.4720, 3.1847) -- (1.1780, 2.4720, 3.1826) -- cycle;
\fill[blue!61.0, opacity=0.7] (1.1780, 2.4720, 3.1826) -- (1.2240, 2.4720, 3.1847) -- (1.2240, 2.5260, 3.1795) -- (1.1780, 2.5260, 3.1774) -- cycle;
\fill[blue!60.8, opacity=0.7] (1.1780, 2.5260, 3.1774) -- (1.2240, 2.5260, 3.1795) -- (1.2240, 2.5800, 3.1741) -- (1.1780, 2.5800, 3.1720) -- cycle;
\fill[blue!49.2, opacity=0.7] (1.1780, 2.5800, 3.1720) -- (1.2240, 2.5800, 3.1741) -- (1.2240, 2.6340, 3.1686) -- (1.1780, 2.6340, 3.1665) -- cycle;
\fill[blue!48.1, opacity=0.7] (1.1780, 2.6340, 3.1665) -- (1.2240, 2.6340, 3.1686) -- (1.2240, 2.6880, 3.1629) -- (1.1780, 2.6880, 3.1608) -- cycle;
\fill[blue!60.3, opacity=0.7] (1.1780, 2.6880, 3.1608) -- (1.2240, 2.6880, 3.1629) -- (1.2240, 2.7420, 3.1571) -- (1.1780, 2.7420, 3.1550) -- cycle;
\fill[blue!58.1, opacity=0.7] (1.1780, 2.7420, 3.1550) -- (1.2240, 2.7420, 3.1571) -- (1.2240, 2.7960, 3.1512) -- (1.1780, 2.7960, 3.1491) -- cycle;
\fill[blue!34.0, opacity=0.7] (1.1780, 2.7960, 3.1491) -- (1.2240, 2.7960, 3.1512) -- (1.2240, 2.8500, 3.1452) -- (1.1780, 2.8500, 3.1431) -- cycle;
\fill[blue!22.9, opacity=0.7] (1.1780, 2.8500, 3.1431) -- (1.2240, 2.8500, 3.1452) -- (1.2240, 2.9040, 3.1391) -- (1.1780, 2.9040, 3.1370) -- cycle;
\fill[blue!23.3, opacity=0.7] (1.1780, 2.9040, 3.1370) -- (1.2240, 2.9040, 3.1391) -- (1.2240, 2.9580, 3.1329) -- (1.1780, 2.9580, 3.1308) -- cycle;
\fill[blue!34.3, opacity=0.7] (1.1780, 2.9580, 3.1308) -- (1.2240, 2.9580, 3.1329) -- (1.2240, 3.0120, 3.1267) -- (1.1780, 3.0120, 3.1246) -- cycle;
\fill[blue!54.3, opacity=0.7] (1.1780, 3.0120, 3.1246) -- (1.2240, 3.0120, 3.1267) -- (1.2240, 3.0660, 3.1204) -- (1.1780, 3.0660, 3.1183) -- cycle;
\fill[blue!63.0, opacity=0.7] (1.1780, 3.0660, 3.1183) -- (1.2240, 3.0660, 3.1204) -- (1.2240, 3.1200, 3.1141) -- (1.1780, 3.1200, 3.1120) -- cycle;
\fill[blue!44.2, opacity=0.7] (1.2240, -0.1200, 3.1141) -- (1.2700, -0.1200, 3.1159) -- (1.2700, -0.0660, 3.1222) -- (1.2240, -0.0660, 3.1204) -- cycle;
\fill[blue!25.8, opacity=0.7] (1.2240, -0.0660, 3.1204) -- (1.2700, -0.0660, 3.1222) -- (1.2700, -0.0120, 3.1285) -- (1.2240, -0.0120, 3.1267) -- cycle;
\fill[blue!20.4, opacity=0.7] (1.2240, -0.0120, 3.1267) -- (1.2700, -0.0120, 3.1285) -- (1.2700, 0.0420, 3.1347) -- (1.2240, 0.0420, 3.1329) -- cycle;
\fill[blue!23.8, opacity=0.7] (1.2240, 0.0420, 3.1329) -- (1.2700, 0.0420, 3.1347) -- (1.2700, 0.0960, 3.1409) -- (1.2240, 0.0960, 3.1391) -- cycle;
\fill[blue!42.8, opacity=0.7] (1.2240, 0.0960, 3.1391) -- (1.2700, 0.0960, 3.1409) -- (1.2700, 0.1500, 3.1470) -- (1.2240, 0.1500, 3.1452) -- cycle;
\fill[blue!63.4, opacity=0.7] (1.2240, 0.1500, 3.1452) -- (1.2700, 0.1500, 3.1470) -- (1.2700, 0.2040, 3.1530) -- (1.2240, 0.2040, 3.1512) -- cycle;
\fill[blue!55.2, opacity=0.7] (1.2240, 0.2040, 3.1512) -- (1.2700, 0.2040, 3.1530) -- (1.2700, 0.2580, 3.1589) -- (1.2240, 0.2580, 3.1571) -- cycle;
\fill[blue!52.5, opacity=0.7] (1.2240, 0.2580, 3.1571) -- (1.2700, 0.2580, 3.1589) -- (1.2700, 0.3120, 3.1647) -- (1.2240, 0.3120, 3.1629) -- cycle;
\fill[blue!62.2, opacity=0.7] (1.2240, 0.3120, 3.1629) -- (1.2700, 0.3120, 3.1647) -- (1.2700, 0.3660, 3.1704) -- (1.2240, 0.3660, 3.1686) -- cycle;
\fill[blue!57.7, opacity=0.7] (1.2240, 0.3660, 3.1686) -- (1.2700, 0.3660, 3.1704) -- (1.2700, 0.4200, 3.1759) -- (1.2240, 0.4200, 3.1741) -- cycle;
\fill[blue!42.7, opacity=0.7] (1.2240, 0.4200, 3.1741) -- (1.2700, 0.4200, 3.1759) -- (1.2700, 0.4740, 3.1813) -- (1.2240, 0.4740, 3.1795) -- cycle;
\fill[blue!42.6, opacity=0.7] (1.2240, 0.4740, 3.1795) -- (1.2700, 0.4740, 3.1813) -- (1.2700, 0.5280, 3.1864) -- (1.2240, 0.5280, 3.1847) -- cycle;
\fill[blue!58.8, opacity=0.7] (1.2240, 0.5280, 3.1847) -- (1.2700, 0.5280, 3.1864) -- (1.2700, 0.5820, 3.1914) -- (1.2240, 0.5820, 3.1896) -- cycle;
\fill[blue!57.8, opacity=0.7] (1.2240, 0.5820, 3.1896) -- (1.2700, 0.5820, 3.1914) -- (1.2700, 0.6360, 3.1962) -- (1.2240, 0.6360, 3.1944) -- cycle;
\fill[blue!34.5, opacity=0.7] (1.2240, 0.6360, 3.1944) -- (1.2700, 0.6360, 3.1962) -- (1.2700, 0.6900, 3.2008) -- (1.2240, 0.6900, 3.1990) -- cycle;
\fill[blue!26.4, opacity=0.7] (1.2240, 0.6900, 3.1990) -- (1.2700, 0.6900, 3.2008) -- (1.2700, 0.7440, 3.2051) -- (1.2240, 0.7440, 3.2033) -- cycle;
\fill[blue!32.5, opacity=0.7] (1.2240, 0.7440, 3.2033) -- (1.2700, 0.7440, 3.2051) -- (1.2700, 0.7980, 3.2092) -- (1.2240, 0.7980, 3.2074) -- cycle;
\fill[blue!52.1, opacity=0.7] (1.2240, 0.7980, 3.2074) -- (1.2700, 0.7980, 3.2092) -- (1.2700, 0.8520, 3.2130) -- (1.2240, 0.8520, 3.2112) -- cycle;
\fill[blue!63.5, opacity=0.7] (1.2240, 0.8520, 3.2112) -- (1.2700, 0.8520, 3.2130) -- (1.2700, 0.9060, 3.2166) -- (1.2240, 0.9060, 3.2148) -- cycle;
\fill[blue!62.5, opacity=0.7] (1.2240, 0.9060, 3.2148) -- (1.2700, 0.9060, 3.2166) -- (1.2700, 0.9600, 3.2198) -- (1.2240, 0.9600, 3.2180) -- cycle;
\fill[blue!63.1, opacity=0.7] (1.2240, 0.9600, 3.2180) -- (1.2700, 0.9600, 3.2198) -- (1.2700, 1.0140, 3.2228) -- (1.2240, 1.0140, 3.2210) -- cycle;
\fill[blue!45.2, opacity=0.7] (1.2240, 1.0140, 3.2210) -- (1.2700, 1.0140, 3.2228) -- (1.2700, 1.0680, 3.2255) -- (1.2240, 1.0680, 3.2238) -- cycle;
\fill[blue!23.0, opacity=0.7] (1.2240, 1.0680, 3.2238) -- (1.2700, 1.0680, 3.2255) -- (1.2700, 1.1220, 3.2279) -- (1.2240, 1.1220, 3.2262) -- cycle;
\fill[blue!17.2, opacity=0.7] (1.2240, 1.1220, 3.2262) -- (1.2700, 1.1220, 3.2279) -- (1.2700, 1.1760, 3.2300) -- (1.2240, 1.1760, 3.2283) -- cycle;
\fill[blue!17.1, opacity=0.7] (1.2240, 1.1760, 3.2283) -- (1.2700, 1.1760, 3.2300) -- (1.2700, 1.2300, 3.2318) -- (1.2240, 1.2300, 3.2300) -- cycle;
\fill[blue!21.6, opacity=0.7] (1.2240, 1.2300, 3.2300) -- (1.2700, 1.2300, 3.2318) -- (1.2700, 1.2840, 3.2333) -- (1.2240, 1.2840, 3.2315) -- cycle;
\fill[blue!36.3, opacity=0.7] (1.2240, 1.2840, 3.2315) -- (1.2700, 1.2840, 3.2333) -- (1.2700, 1.3380, 3.2344) -- (1.2240, 1.3380, 3.2326) -- cycle;
\fill[blue!52.4, opacity=0.7] (1.2240, 1.3380, 3.2326) -- (1.2700, 1.3380, 3.2344) -- (1.2700, 1.3920, 3.2353) -- (1.2240, 1.3920, 3.2335) -- cycle;
\fill[blue!57.6, opacity=0.7] (1.2240, 1.3920, 3.2335) -- (1.2700, 1.3920, 3.2353) -- (1.2700, 1.4460, 3.2357) -- (1.2240, 1.4460, 3.2340) -- cycle;
\fill[blue!55.2, opacity=0.7] (1.2240, 1.4460, 3.2340) -- (1.2700, 1.4460, 3.2357) -- (1.2700, 1.5000, 3.2359) -- (1.2240, 1.5000, 3.2341) -- cycle;
\fill[blue!49.0, opacity=0.7] (1.2240, 1.5000, 3.2341) -- (1.2700, 1.5000, 3.2359) -- (1.2700, 1.5540, 3.2357) -- (1.2240, 1.5540, 3.2340) -- cycle;
\fill[blue!43.9, opacity=0.7] (1.2240, 1.5540, 3.2340) -- (1.2700, 1.5540, 3.2357) -- (1.2700, 1.6080, 3.2353) -- (1.2240, 1.6080, 3.2335) -- cycle;
\fill[blue!43.2, opacity=0.7] (1.2240, 1.6080, 3.2335) -- (1.2700, 1.6080, 3.2353) -- (1.2700, 1.6620, 3.2344) -- (1.2240, 1.6620, 3.2326) -- cycle;
\fill[blue!47.1, opacity=0.7] (1.2240, 1.6620, 3.2326) -- (1.2700, 1.6620, 3.2344) -- (1.2700, 1.7160, 3.2333) -- (1.2240, 1.7160, 3.2315) -- cycle;
\fill[blue!54.1, opacity=0.7] (1.2240, 1.7160, 3.2315) -- (1.2700, 1.7160, 3.2333) -- (1.2700, 1.7700, 3.2318) -- (1.2240, 1.7700, 3.2300) -- cycle;
\fill[blue!60.2, opacity=0.7] (1.2240, 1.7700, 3.2300) -- (1.2700, 1.7700, 3.2318) -- (1.2700, 1.8240, 3.2300) -- (1.2240, 1.8240, 3.2283) -- cycle;
\fill[blue!62.8, opacity=0.7] (1.2240, 1.8240, 3.2283) -- (1.2700, 1.8240, 3.2300) -- (1.2700, 1.8780, 3.2279) -- (1.2240, 1.8780, 3.2262) -- cycle;
\fill[blue!62.9, opacity=0.7] (1.2240, 1.8780, 3.2262) -- (1.2700, 1.8780, 3.2279) -- (1.2700, 1.9320, 3.2255) -- (1.2240, 1.9320, 3.2238) -- cycle;
\fill[blue!59.5, opacity=0.7] (1.2240, 1.9320, 3.2238) -- (1.2700, 1.9320, 3.2255) -- (1.2700, 1.9860, 3.2228) -- (1.2240, 1.9860, 3.2210) -- cycle;
\fill[blue!47.5, opacity=0.7] (1.2240, 1.9860, 3.2210) -- (1.2700, 1.9860, 3.2228) -- (1.2700, 2.0400, 3.2198) -- (1.2240, 2.0400, 3.2180) -- cycle;
\fill[blue!31.9, opacity=0.7] (1.2240, 2.0400, 3.2180) -- (1.2700, 2.0400, 3.2198) -- (1.2700, 2.0940, 3.2166) -- (1.2240, 2.0940, 3.2148) -- cycle;
\fill[blue!23.6, opacity=0.7] (1.2240, 2.0940, 3.2148) -- (1.2700, 2.0940, 3.2166) -- (1.2700, 2.1480, 3.2130) -- (1.2240, 2.1480, 3.2112) -- cycle;
\fill[blue!22.6, opacity=0.7] (1.2240, 2.1480, 3.2112) -- (1.2700, 2.1480, 3.2130) -- (1.2700, 2.2020, 3.2092) -- (1.2240, 2.2020, 3.2074) -- cycle;
\fill[blue!30.0, opacity=0.7] (1.2240, 2.2020, 3.2074) -- (1.2700, 2.2020, 3.2092) -- (1.2700, 2.2560, 3.2051) -- (1.2240, 2.2560, 3.2033) -- cycle;
\fill[blue!51.0, opacity=0.7] (1.2240, 2.2560, 3.2033) -- (1.2700, 2.2560, 3.2051) -- (1.2700, 2.3100, 3.2008) -- (1.2240, 2.3100, 3.1990) -- cycle;
\fill[blue!63.4, opacity=0.7] (1.2240, 2.3100, 3.1990) -- (1.2700, 2.3100, 3.2008) -- (1.2700, 2.3640, 3.1962) -- (1.2240, 2.3640, 3.1944) -- cycle;
\fill[blue!51.5, opacity=0.7] (1.2240, 2.3640, 3.1944) -- (1.2700, 2.3640, 3.1962) -- (1.2700, 2.4180, 3.1914) -- (1.2240, 2.4180, 3.1896) -- cycle;
\fill[blue!45.9, opacity=0.7] (1.2240, 2.4180, 3.1896) -- (1.2700, 2.4180, 3.1914) -- (1.2700, 2.4720, 3.1864) -- (1.2240, 2.4720, 3.1847) -- cycle;
\fill[blue!54.7, opacity=0.7] (1.2240, 2.4720, 3.1847) -- (1.2700, 2.4720, 3.1864) -- (1.2700, 2.5260, 3.1813) -- (1.2240, 2.5260, 3.1795) -- cycle;
\fill[blue!63.6, opacity=0.7] (1.2240, 2.5260, 3.1795) -- (1.2700, 2.5260, 3.1813) -- (1.2700, 2.5800, 3.1759) -- (1.2240, 2.5800, 3.1741) -- cycle;
\fill[blue!54.4, opacity=0.7] (1.2240, 2.5800, 3.1741) -- (1.2700, 2.5800, 3.1759) -- (1.2700, 2.6340, 3.1704) -- (1.2240, 2.6340, 3.1686) -- cycle;
\fill[blue!47.1, opacity=0.7] (1.2240, 2.6340, 3.1686) -- (1.2700, 2.6340, 3.1704) -- (1.2700, 2.6880, 3.1647) -- (1.2240, 2.6880, 3.1629) -- cycle;
\fill[blue!55.1, opacity=0.7] (1.2240, 2.6880, 3.1629) -- (1.2700, 2.6880, 3.1647) -- (1.2700, 2.7420, 3.1589) -- (1.2240, 2.7420, 3.1571) -- cycle;
\fill[blue!63.0, opacity=0.7] (1.2240, 2.7420, 3.1571) -- (1.2700, 2.7420, 3.1589) -- (1.2700, 2.7960, 3.1530) -- (1.2240, 2.7960, 3.1512) -- cycle;
\fill[blue!41.9, opacity=0.7] (1.2240, 2.7960, 3.1512) -- (1.2700, 2.7960, 3.1530) -- (1.2700, 2.8500, 3.1470) -- (1.2240, 2.8500, 3.1452) -- cycle;
\fill[blue!24.8, opacity=0.7] (1.2240, 2.8500, 3.1452) -- (1.2700, 2.8500, 3.1470) -- (1.2700, 2.9040, 3.1409) -- (1.2240, 2.9040, 3.1391) -- cycle;
\fill[blue!22.1, opacity=0.7] (1.2240, 2.9040, 3.1391) -- (1.2700, 2.9040, 3.1409) -- (1.2700, 2.9580, 3.1347) -- (1.2240, 2.9580, 3.1329) -- cycle;
\fill[blue!29.0, opacity=0.7] (1.2240, 2.9580, 3.1329) -- (1.2700, 2.9580, 3.1347) -- (1.2700, 3.0120, 3.1285) -- (1.2240, 3.0120, 3.1267) -- cycle;
\fill[blue!48.1, opacity=0.7] (1.2240, 3.0120, 3.1267) -- (1.2700, 3.0120, 3.1285) -- (1.2700, 3.0660, 3.1222) -- (1.2240, 3.0660, 3.1204) -- cycle;
\fill[blue!61.9, opacity=0.7] (1.2240, 3.0660, 3.1204) -- (1.2700, 3.0660, 3.1222) -- (1.2700, 3.1200, 3.1159) -- (1.2240, 3.1200, 3.1141) -- cycle;
\fill[blue!42.8, opacity=0.7] (1.2700, -0.1200, 3.1159) -- (1.3160, -0.1200, 3.1174) -- (1.3160, -0.0660, 3.1237) -- (1.2700, -0.0660, 3.1222) -- cycle;
\fill[blue!25.1, opacity=0.7] (1.2700, -0.0660, 3.1222) -- (1.3160, -0.0660, 3.1237) -- (1.3160, -0.0120, 3.1299) -- (1.2700, -0.0120, 3.1285) -- cycle;
\fill[blue!20.5, opacity=0.7] (1.2700, -0.0120, 3.1285) -- (1.3160, -0.0120, 3.1299) -- (1.3160, 0.0420, 3.1361) -- (1.2700, 0.0420, 3.1347) -- cycle;
\fill[blue!24.9, opacity=0.7] (1.2700, 0.0420, 3.1347) -- (1.3160, 0.0420, 3.1361) -- (1.3160, 0.0960, 3.1423) -- (1.2700, 0.0960, 3.1409) -- cycle;
\fill[blue!45.8, opacity=0.7] (1.2700, 0.0960, 3.1409) -- (1.3160, 0.0960, 3.1423) -- (1.3160, 0.1500, 3.1484) -- (1.2700, 0.1500, 3.1470) -- cycle;
\fill[blue!63.6, opacity=0.7] (1.2700, 0.1500, 3.1470) -- (1.3160, 0.1500, 3.1484) -- (1.3160, 0.2040, 3.1545) -- (1.2700, 0.2040, 3.1530) -- cycle;
\fill[blue!53.9, opacity=0.7] (1.2700, 0.2040, 3.1530) -- (1.3160, 0.2040, 3.1545) -- (1.3160, 0.2580, 3.1604) -- (1.2700, 0.2580, 3.1589) -- cycle;
\fill[blue!52.8, opacity=0.7] (1.2700, 0.2580, 3.1589) -- (1.3160, 0.2580, 3.1604) -- (1.3160, 0.3120, 3.1662) -- (1.2700, 0.3120, 3.1647) -- cycle;
\fill[blue!62.8, opacity=0.7] (1.2700, 0.3120, 3.1647) -- (1.3160, 0.3120, 3.1662) -- (1.3160, 0.3660, 3.1719) -- (1.2700, 0.3660, 3.1704) -- cycle;
\fill[blue!56.2, opacity=0.7] (1.2700, 0.3660, 3.1704) -- (1.3160, 0.3660, 3.1719) -- (1.3160, 0.4200, 3.1774) -- (1.2700, 0.4200, 3.1759) -- cycle;
\fill[blue!42.2, opacity=0.7] (1.2700, 0.4200, 3.1759) -- (1.3160, 0.4200, 3.1774) -- (1.3160, 0.4740, 3.1827) -- (1.2700, 0.4740, 3.1813) -- cycle;
\fill[blue!44.5, opacity=0.7] (1.2700, 0.4740, 3.1813) -- (1.3160, 0.4740, 3.1827) -- (1.3160, 0.5280, 3.1879) -- (1.2700, 0.5280, 3.1864) -- cycle;
\fill[blue!61.2, opacity=0.7] (1.2700, 0.5280, 3.1864) -- (1.3160, 0.5280, 3.1879) -- (1.3160, 0.5820, 3.1929) -- (1.2700, 0.5820, 3.1914) -- cycle;
\fill[blue!54.1, opacity=0.7] (1.2700, 0.5820, 3.1914) -- (1.3160, 0.5820, 3.1929) -- (1.3160, 0.6360, 3.1977) -- (1.2700, 0.6360, 3.1962) -- cycle;
\fill[blue!31.7, opacity=0.7] (1.2700, 0.6360, 3.1962) -- (1.3160, 0.6360, 3.1977) -- (1.3160, 0.6900, 3.2022) -- (1.2700, 0.6900, 3.2008) -- cycle;
\fill[blue!26.1, opacity=0.7] (1.2700, 0.6900, 3.2008) -- (1.3160, 0.6900, 3.2022) -- (1.3160, 0.7440, 3.2066) -- (1.2700, 0.7440, 3.2051) -- cycle;
\fill[blue!34.5, opacity=0.7] (1.2700, 0.7440, 3.2051) -- (1.3160, 0.7440, 3.2066) -- (1.3160, 0.7980, 3.2106) -- (1.2700, 0.7980, 3.2092) -- cycle;
\fill[blue!55.3, opacity=0.7] (1.2700, 0.7980, 3.2092) -- (1.3160, 0.7980, 3.2106) -- (1.3160, 0.8520, 3.2145) -- (1.2700, 0.8520, 3.2130) -- cycle;
\fill[blue!63.6, opacity=0.7] (1.2700, 0.8520, 3.2130) -- (1.3160, 0.8520, 3.2145) -- (1.3160, 0.9060, 3.2180) -- (1.2700, 0.9060, 3.2166) -- cycle;
\fill[blue!62.9, opacity=0.7] (1.2700, 0.9060, 3.2166) -- (1.3160, 0.9060, 3.2180) -- (1.3160, 0.9600, 3.2213) -- (1.2700, 0.9600, 3.2198) -- cycle;
\fill[blue!60.8, opacity=0.7] (1.2700, 0.9600, 3.2198) -- (1.3160, 0.9600, 3.2213) -- (1.3160, 1.0140, 3.2243) -- (1.2700, 1.0140, 3.2228) -- cycle;
\fill[blue!36.5, opacity=0.7] (1.2700, 1.0140, 3.2228) -- (1.3160, 1.0140, 3.2243) -- (1.3160, 1.0680, 3.2270) -- (1.2700, 1.0680, 3.2255) -- cycle;
\fill[blue!19.6, opacity=0.7] (1.2700, 1.0680, 3.2255) -- (1.3160, 1.0680, 3.2270) -- (1.3160, 1.1220, 3.2294) -- (1.2700, 1.1220, 3.2279) -- cycle;
\fill[blue!16.6, opacity=0.7] (1.2700, 1.1220, 3.2279) -- (1.3160, 1.1220, 3.2294) -- (1.3160, 1.1760, 3.2315) -- (1.2700, 1.1760, 3.2300) -- cycle;
\fill[blue!17.9, opacity=0.7] (1.2700, 1.1760, 3.2300) -- (1.3160, 1.1760, 3.2315) -- (1.3160, 1.2300, 3.2333) -- (1.2700, 1.2300, 3.2318) -- cycle;
\fill[blue!26.9, opacity=0.7] (1.2700, 1.2300, 3.2318) -- (1.3160, 1.2300, 3.2333) -- (1.3160, 1.2840, 3.2348) -- (1.2700, 1.2840, 3.2333) -- cycle;
\fill[blue!46.3, opacity=0.7] (1.2700, 1.2840, 3.2333) -- (1.3160, 1.2840, 3.2348) -- (1.3160, 1.3380, 3.2359) -- (1.2700, 1.3380, 3.2344) -- cycle;
\fill[blue!55.9, opacity=0.7] (1.2700, 1.3380, 3.2344) -- (1.3160, 1.3380, 3.2359) -- (1.3160, 1.3920, 3.2367) -- (1.2700, 1.3920, 3.2353) -- cycle;
\fill[blue!51.1, opacity=0.7] (1.2700, 1.3920, 3.2353) -- (1.3160, 1.3920, 3.2367) -- (1.3160, 1.4460, 3.2372) -- (1.2700, 1.4460, 3.2357) -- cycle;
\fill[blue!36.3, opacity=0.7] (1.2700, 1.4460, 3.2357) -- (1.3160, 1.4460, 3.2372) -- (1.3160, 1.5000, 3.2374) -- (1.2700, 1.5000, 3.2359) -- cycle;
\fill[blue!24.9, opacity=0.7] (1.2700, 1.5000, 3.2359) -- (1.3160, 1.5000, 3.2374) -- (1.3160, 1.5540, 3.2372) -- (1.2700, 1.5540, 3.2357) -- cycle;
\fill[blue!20.8, opacity=0.7] (1.2700, 1.5540, 3.2357) -- (1.3160, 1.5540, 3.2372) -- (1.3160, 1.6080, 3.2367) -- (1.2700, 1.6080, 3.2353) -- cycle;
\fill[blue!20.4, opacity=0.7] (1.2700, 1.6080, 3.2353) -- (1.3160, 1.6080, 3.2367) -- (1.3160, 1.6620, 3.2359) -- (1.2700, 1.6620, 3.2344) -- cycle;
\fill[blue!22.9, opacity=0.7] (1.2700, 1.6620, 3.2344) -- (1.3160, 1.6620, 3.2359) -- (1.3160, 1.7160, 3.2348) -- (1.2700, 1.7160, 3.2333) -- cycle;
\fill[blue!29.9, opacity=0.7] (1.2700, 1.7160, 3.2333) -- (1.3160, 1.7160, 3.2348) -- (1.3160, 1.7700, 3.2333) -- (1.2700, 1.7700, 3.2318) -- cycle;
\fill[blue!43.2, opacity=0.7] (1.2700, 1.7700, 3.2318) -- (1.3160, 1.7700, 3.2333) -- (1.3160, 1.8240, 3.2315) -- (1.2700, 1.8240, 3.2300) -- cycle;
\fill[blue!57.1, opacity=0.7] (1.2700, 1.8240, 3.2300) -- (1.3160, 1.8240, 3.2315) -- (1.3160, 1.8780, 3.2294) -- (1.2700, 1.8780, 3.2279) -- cycle;
\fill[blue!62.8, opacity=0.7] (1.2700, 1.8780, 3.2279) -- (1.3160, 1.8780, 3.2294) -- (1.3160, 1.9320, 3.2270) -- (1.2700, 1.9320, 3.2255) -- cycle;
\fill[blue!63.3, opacity=0.7] (1.2700, 1.9320, 3.2255) -- (1.3160, 1.9320, 3.2270) -- (1.3160, 1.9860, 3.2243) -- (1.2700, 1.9860, 3.2228) -- cycle;
\fill[blue!59.9, opacity=0.7] (1.2700, 1.9860, 3.2228) -- (1.3160, 1.9860, 3.2243) -- (1.3160, 2.0400, 3.2213) -- (1.2700, 2.0400, 3.2198) -- cycle;
\fill[blue!45.8, opacity=0.7] (1.2700, 2.0400, 3.2198) -- (1.3160, 2.0400, 3.2213) -- (1.3160, 2.0940, 3.2180) -- (1.2700, 2.0940, 3.2166) -- cycle;
\fill[blue!29.6, opacity=0.7] (1.2700, 2.0940, 3.2166) -- (1.3160, 2.0940, 3.2180) -- (1.3160, 2.1480, 3.2145) -- (1.2700, 2.1480, 3.2130) -- cycle;
\fill[blue!23.0, opacity=0.7] (1.2700, 2.1480, 3.2130) -- (1.3160, 2.1480, 3.2145) -- (1.3160, 2.2020, 3.2106) -- (1.2700, 2.2020, 3.2092) -- cycle;
\fill[blue!24.6, opacity=0.7] (1.2700, 2.2020, 3.2092) -- (1.3160, 2.2020, 3.2106) -- (1.3160, 2.2560, 3.2066) -- (1.2700, 2.2560, 3.2051) -- cycle;
\fill[blue!38.3, opacity=0.7] (1.2700, 2.2560, 3.2051) -- (1.3160, 2.2560, 3.2066) -- (1.3160, 2.3100, 3.2022) -- (1.2700, 2.3100, 3.2008) -- cycle;
\fill[blue!61.0, opacity=0.7] (1.2700, 2.3100, 3.2008) -- (1.3160, 2.3100, 3.2022) -- (1.3160, 2.3640, 3.1977) -- (1.2700, 2.3640, 3.1962) -- cycle;
\fill[blue!57.9, opacity=0.7] (1.2700, 2.3640, 3.1962) -- (1.3160, 2.3640, 3.1977) -- (1.3160, 2.4180, 3.1929) -- (1.2700, 2.4180, 3.1914) -- cycle;
\fill[blue!46.0, opacity=0.7] (1.2700, 2.4180, 3.1914) -- (1.3160, 2.4180, 3.1929) -- (1.3160, 2.4720, 3.1879) -- (1.2700, 2.4720, 3.1864) -- cycle;
\fill[blue!49.0, opacity=0.7] (1.2700, 2.4720, 3.1864) -- (1.3160, 2.4720, 3.1879) -- (1.3160, 2.5260, 3.1827) -- (1.2700, 2.5260, 3.1813) -- cycle;
\fill[blue!61.6, opacity=0.7] (1.2700, 2.5260, 3.1813) -- (1.3160, 2.5260, 3.1827) -- (1.3160, 2.5800, 3.1774) -- (1.2700, 2.5800, 3.1759) -- cycle;
\fill[blue!59.6, opacity=0.7] (1.2700, 2.5800, 3.1759) -- (1.3160, 2.5800, 3.1774) -- (1.3160, 2.6340, 3.1719) -- (1.2700, 2.6340, 3.1704) -- cycle;
\fill[blue!48.6, opacity=0.7] (1.2700, 2.6340, 3.1704) -- (1.3160, 2.6340, 3.1719) -- (1.3160, 2.6880, 3.1662) -- (1.2700, 2.6880, 3.1647) -- cycle;
\fill[blue!51.1, opacity=0.7] (1.2700, 2.6880, 3.1647) -- (1.3160, 2.6880, 3.1662) -- (1.3160, 2.7420, 3.1604) -- (1.2700, 2.7420, 3.1589) -- cycle;
\fill[blue!63.2, opacity=0.7] (1.2700, 2.7420, 3.1589) -- (1.3160, 2.7420, 3.1604) -- (1.3160, 2.7960, 3.1545) -- (1.2700, 2.7960, 3.1530) -- cycle;
\fill[blue!50.0, opacity=0.7] (1.2700, 2.7960, 3.1530) -- (1.3160, 2.7960, 3.1545) -- (1.3160, 2.8500, 3.1484) -- (1.2700, 2.8500, 3.1470) -- cycle;
\fill[blue!27.8, opacity=0.7] (1.2700, 2.8500, 3.1470) -- (1.3160, 2.8500, 3.1484) -- (1.3160, 2.9040, 3.1423) -- (1.2700, 2.9040, 3.1409) -- cycle;
\fill[blue!21.8, opacity=0.7] (1.2700, 2.9040, 3.1409) -- (1.3160, 2.9040, 3.1423) -- (1.3160, 2.9580, 3.1361) -- (1.2700, 2.9580, 3.1347) -- cycle;
\fill[blue!25.6, opacity=0.7] (1.2700, 2.9580, 3.1347) -- (1.3160, 2.9580, 3.1361) -- (1.3160, 3.0120, 3.1299) -- (1.2700, 3.0120, 3.1285) -- cycle;
\fill[blue!41.9, opacity=0.7] (1.2700, 3.0120, 3.1285) -- (1.3160, 3.0120, 3.1299) -- (1.3160, 3.0660, 3.1237) -- (1.2700, 3.0660, 3.1222) -- cycle;
\fill[blue!59.7, opacity=0.7] (1.2700, 3.0660, 3.1222) -- (1.3160, 3.0660, 3.1237) -- (1.3160, 3.1200, 3.1174) -- (1.2700, 3.1200, 3.1159) -- cycle;
\fill[blue!42.1, opacity=0.7] (1.3160, -0.1200, 3.1174) -- (1.3620, -0.1200, 3.1185) -- (1.3620, -0.0660, 3.1248) -- (1.3160, -0.0660, 3.1237) -- cycle;
\fill[blue!24.9, opacity=0.7] (1.3160, -0.0660, 3.1237) -- (1.3620, -0.0660, 3.1248) -- (1.3620, -0.0120, 3.1311) -- (1.3160, -0.0120, 3.1299) -- cycle;
\fill[blue!20.6, opacity=0.7] (1.3160, -0.0120, 3.1299) -- (1.3620, -0.0120, 3.1311) -- (1.3620, 0.0420, 3.1373) -- (1.3160, 0.0420, 3.1361) -- cycle;
\fill[blue!25.7, opacity=0.7] (1.3160, 0.0420, 3.1361) -- (1.3620, 0.0420, 3.1373) -- (1.3620, 0.0960, 3.1435) -- (1.3160, 0.0960, 3.1423) -- cycle;
\fill[blue!47.7, opacity=0.7] (1.3160, 0.0960, 3.1423) -- (1.3620, 0.0960, 3.1435) -- (1.3620, 0.1500, 3.1496) -- (1.3160, 0.1500, 3.1484) -- cycle;
\fill[blue!63.4, opacity=0.7] (1.3160, 0.1500, 3.1484) -- (1.3620, 0.1500, 3.1496) -- (1.3620, 0.2040, 3.1556) -- (1.3160, 0.2040, 3.1545) -- cycle;
\fill[blue!53.0, opacity=0.7] (1.3160, 0.2040, 3.1545) -- (1.3620, 0.2040, 3.1556) -- (1.3620, 0.2580, 3.1615) -- (1.3160, 0.2580, 3.1604) -- cycle;
\fill[blue!53.0, opacity=0.7] (1.3160, 0.2580, 3.1604) -- (1.3620, 0.2580, 3.1615) -- (1.3620, 0.3120, 3.1673) -- (1.3160, 0.3120, 3.1662) -- cycle;
\fill[blue!63.1, opacity=0.7] (1.3160, 0.3120, 3.1662) -- (1.3620, 0.3120, 3.1673) -- (1.3620, 0.3660, 3.1730) -- (1.3160, 0.3660, 3.1719) -- cycle;
\fill[blue!55.4, opacity=0.7] (1.3160, 0.3660, 3.1719) -- (1.3620, 0.3660, 3.1730) -- (1.3620, 0.4200, 3.1785) -- (1.3160, 0.4200, 3.1774) -- cycle;
\fill[blue!42.2, opacity=0.7] (1.3160, 0.4200, 3.1774) -- (1.3620, 0.4200, 3.1785) -- (1.3620, 0.4740, 3.1839) -- (1.3160, 0.4740, 3.1827) -- cycle;
\fill[blue!45.9, opacity=0.7] (1.3160, 0.4740, 3.1827) -- (1.3620, 0.4740, 3.1839) -- (1.3620, 0.5280, 3.1891) -- (1.3160, 0.5280, 3.1879) -- cycle;
\fill[blue!62.3, opacity=0.7] (1.3160, 0.5280, 3.1879) -- (1.3620, 0.5280, 3.1891) -- (1.3620, 0.5820, 3.1940) -- (1.3160, 0.5820, 3.1929) -- cycle;
\fill[blue!51.5, opacity=0.7] (1.3160, 0.5820, 3.1929) -- (1.3620, 0.5820, 3.1940) -- (1.3620, 0.6360, 3.1988) -- (1.3160, 0.6360, 3.1977) -- cycle;
\fill[blue!30.1, opacity=0.7] (1.3160, 0.6360, 3.1977) -- (1.3620, 0.6360, 3.1988) -- (1.3620, 0.6900, 3.2034) -- (1.3160, 0.6900, 3.2022) -- cycle;
\fill[blue!25.9, opacity=0.7] (1.3160, 0.6900, 3.2022) -- (1.3620, 0.6900, 3.2034) -- (1.3620, 0.7440, 3.2077) -- (1.3160, 0.7440, 3.2066) -- cycle;
\fill[blue!35.7, opacity=0.7] (1.3160, 0.7440, 3.2066) -- (1.3620, 0.7440, 3.2077) -- (1.3620, 0.7980, 3.2118) -- (1.3160, 0.7980, 3.2106) -- cycle;
\fill[blue!56.7, opacity=0.7] (1.3160, 0.7980, 3.2106) -- (1.3620, 0.7980, 3.2118) -- (1.3620, 0.8520, 3.2156) -- (1.3160, 0.8520, 3.2145) -- cycle;
\fill[blue!63.5, opacity=0.7] (1.3160, 0.8520, 3.2145) -- (1.3620, 0.8520, 3.2156) -- (1.3620, 0.9060, 3.2192) -- (1.3160, 0.9060, 3.2180) -- cycle;
\fill[blue!63.3, opacity=0.7] (1.3160, 0.9060, 3.2180) -- (1.3620, 0.9060, 3.2192) -- (1.3620, 0.9600, 3.2224) -- (1.3160, 0.9600, 3.2213) -- cycle;
\fill[blue!58.0, opacity=0.7] (1.3160, 0.9600, 3.2213) -- (1.3620, 0.9600, 3.2224) -- (1.3620, 1.0140, 3.2254) -- (1.3160, 1.0140, 3.2243) -- cycle;
\fill[blue!31.4, opacity=0.7] (1.3160, 1.0140, 3.2243) -- (1.3620, 1.0140, 3.2254) -- (1.3620, 1.0680, 3.2281) -- (1.3160, 1.0680, 3.2270) -- cycle;
\fill[blue!18.1, opacity=0.7] (1.3160, 1.0680, 3.2270) -- (1.3620, 1.0680, 3.2281) -- (1.3620, 1.1220, 3.2306) -- (1.3160, 1.1220, 3.2294) -- cycle;
\fill[blue!16.5, opacity=0.7] (1.3160, 1.1220, 3.2294) -- (1.3620, 1.1220, 3.2306) -- (1.3620, 1.1760, 3.2326) -- (1.3160, 1.1760, 3.2315) -- cycle;
\fill[blue!18.7, opacity=0.7] (1.3160, 1.1760, 3.2315) -- (1.3620, 1.1760, 3.2326) -- (1.3620, 1.2300, 3.2344) -- (1.3160, 1.2300, 3.2333) -- cycle;
\fill[blue!31.6, opacity=0.7] (1.3160, 1.2300, 3.2333) -- (1.3620, 1.2300, 3.2344) -- (1.3620, 1.2840, 3.2359) -- (1.3160, 1.2840, 3.2348) -- cycle;
\fill[blue!50.8, opacity=0.7] (1.3160, 1.2840, 3.2348) -- (1.3620, 1.2840, 3.2359) -- (1.3620, 1.3380, 3.2370) -- (1.3160, 1.3380, 3.2359) -- cycle;
\fill[blue!52.8, opacity=0.7] (1.3160, 1.3380, 3.2359) -- (1.3620, 1.3380, 3.2370) -- (1.3620, 1.3920, 3.2379) -- (1.3160, 1.3920, 3.2367) -- cycle;
\fill[blue!35.4, opacity=0.7] (1.3160, 1.3920, 3.2367) -- (1.3620, 1.3920, 3.2379) -- (1.3620, 1.4460, 3.2384) -- (1.3160, 1.4460, 3.2372) -- cycle;
\fill[blue!19.5, opacity=0.7] (1.3160, 1.4460, 3.2372) -- (1.3620, 1.4460, 3.2384) -- (1.3620, 1.5000, 3.2385) -- (1.3160, 1.5000, 3.2374) -- cycle;
\fill[blue!16.1, opacity=0.7] (1.3160, 1.5000, 3.2374) -- (1.3620, 1.5000, 3.2385) -- (1.3620, 1.5540, 3.2384) -- (1.3160, 1.5540, 3.2372) -- cycle;
\fill[blue!15.6, opacity=0.7] (1.3160, 1.5540, 3.2372) -- (1.3620, 1.5540, 3.2384) -- (1.3620, 1.6080, 3.2379) -- (1.3160, 1.6080, 3.2367) -- cycle;
\fill[blue!15.7, opacity=0.7] (1.3160, 1.6080, 3.2367) -- (1.3620, 1.6080, 3.2379) -- (1.3620, 1.6620, 3.2370) -- (1.3160, 1.6620, 3.2359) -- cycle;
\fill[blue!16.1, opacity=0.7] (1.3160, 1.6620, 3.2359) -- (1.3620, 1.6620, 3.2370) -- (1.3620, 1.7160, 3.2359) -- (1.3160, 1.7160, 3.2348) -- cycle;
\fill[blue!17.5, opacity=0.7] (1.3160, 1.7160, 3.2348) -- (1.3620, 1.7160, 3.2359) -- (1.3620, 1.7700, 3.2344) -- (1.3160, 1.7700, 3.2333) -- cycle;
\fill[blue!23.1, opacity=0.7] (1.3160, 1.7700, 3.2333) -- (1.3620, 1.7700, 3.2344) -- (1.3620, 1.8240, 3.2326) -- (1.3160, 1.8240, 3.2315) -- cycle;
\fill[blue!38.5, opacity=0.7] (1.3160, 1.8240, 3.2315) -- (1.3620, 1.8240, 3.2326) -- (1.3620, 1.8780, 3.2306) -- (1.3160, 1.8780, 3.2294) -- cycle;
\fill[blue!57.0, opacity=0.7] (1.3160, 1.8780, 3.2294) -- (1.3620, 1.8780, 3.2306) -- (1.3620, 1.9320, 3.2281) -- (1.3160, 1.9320, 3.2270) -- cycle;
\fill[blue!63.3, opacity=0.7] (1.3160, 1.9320, 3.2270) -- (1.3620, 1.9320, 3.2281) -- (1.3620, 1.9860, 3.2254) -- (1.3160, 1.9860, 3.2243) -- cycle;
\fill[blue!63.3, opacity=0.7] (1.3160, 1.9860, 3.2243) -- (1.3620, 1.9860, 3.2254) -- (1.3620, 2.0400, 3.2224) -- (1.3160, 2.0400, 3.2213) -- cycle;
\fill[blue!57.7, opacity=0.7] (1.3160, 2.0400, 3.2213) -- (1.3620, 2.0400, 3.2224) -- (1.3620, 2.0940, 3.2192) -- (1.3160, 2.0940, 3.2180) -- cycle;
\fill[blue!40.0, opacity=0.7] (1.3160, 2.0940, 3.2180) -- (1.3620, 2.0940, 3.2192) -- (1.3620, 2.1480, 3.2156) -- (1.3160, 2.1480, 3.2145) -- cycle;
\fill[blue!26.1, opacity=0.7] (1.3160, 2.1480, 3.2145) -- (1.3620, 2.1480, 3.2156) -- (1.3620, 2.2020, 3.2118) -- (1.3160, 2.2020, 3.2106) -- cycle;
\fill[blue!23.1, opacity=0.7] (1.3160, 2.2020, 3.2106) -- (1.3620, 2.2020, 3.2118) -- (1.3620, 2.2560, 3.2077) -- (1.3160, 2.2560, 3.2066) -- cycle;
\fill[blue!30.2, opacity=0.7] (1.3160, 2.2560, 3.2066) -- (1.3620, 2.2560, 3.2077) -- (1.3620, 2.3100, 3.2034) -- (1.3160, 2.3100, 3.2022) -- cycle;
\fill[blue!52.4, opacity=0.7] (1.3160, 2.3100, 3.2022) -- (1.3620, 2.3100, 3.2034) -- (1.3620, 2.3640, 3.1988) -- (1.3160, 2.3640, 3.1977) -- cycle;
\fill[blue!62.6, opacity=0.7] (1.3160, 2.3640, 3.1977) -- (1.3620, 2.3640, 3.1988) -- (1.3620, 2.4180, 3.1940) -- (1.3160, 2.4180, 3.1929) -- cycle;
\fill[blue!48.8, opacity=0.7] (1.3160, 2.4180, 3.1929) -- (1.3620, 2.4180, 3.1940) -- (1.3620, 2.4720, 3.1891) -- (1.3160, 2.4720, 3.1879) -- cycle;
\fill[blue!45.5, opacity=0.7] (1.3160, 2.4720, 3.1879) -- (1.3620, 2.4720, 3.1891) -- (1.3620, 2.5260, 3.1839) -- (1.3160, 2.5260, 3.1827) -- cycle;
\fill[blue!57.1, opacity=0.7] (1.3160, 2.5260, 3.1827) -- (1.3620, 2.5260, 3.1839) -- (1.3620, 2.5800, 3.1785) -- (1.3160, 2.5800, 3.1774) -- cycle;
\fill[blue!62.9, opacity=0.7] (1.3160, 2.5800, 3.1774) -- (1.3620, 2.5800, 3.1785) -- (1.3620, 2.6340, 3.1730) -- (1.3160, 2.6340, 3.1719) -- cycle;
\fill[blue!51.7, opacity=0.7] (1.3160, 2.6340, 3.1719) -- (1.3620, 2.6340, 3.1730) -- (1.3620, 2.6880, 3.1673) -- (1.3160, 2.6880, 3.1662) -- cycle;
\fill[blue!49.0, opacity=0.7] (1.3160, 2.6880, 3.1662) -- (1.3620, 2.6880, 3.1673) -- (1.3620, 2.7420, 3.1615) -- (1.3160, 2.7420, 3.1604) -- cycle;
\fill[blue!60.7, opacity=0.7] (1.3160, 2.7420, 3.1604) -- (1.3620, 2.7420, 3.1615) -- (1.3620, 2.7960, 3.1556) -- (1.3160, 2.7960, 3.1545) -- cycle;
\fill[blue!56.6, opacity=0.7] (1.3160, 2.7960, 3.1545) -- (1.3620, 2.7960, 3.1556) -- (1.3620, 2.8500, 3.1496) -- (1.3160, 2.8500, 3.1484) -- cycle;
\fill[blue!31.7, opacity=0.7] (1.3160, 2.8500, 3.1484) -- (1.3620, 2.8500, 3.1496) -- (1.3620, 2.9040, 3.1435) -- (1.3160, 2.9040, 3.1423) -- cycle;
\fill[blue!22.1, opacity=0.7] (1.3160, 2.9040, 3.1423) -- (1.3620, 2.9040, 3.1435) -- (1.3620, 2.9580, 3.1373) -- (1.3160, 2.9580, 3.1361) -- cycle;
\fill[blue!23.5, opacity=0.7] (1.3160, 2.9580, 3.1361) -- (1.3620, 2.9580, 3.1373) -- (1.3620, 3.0120, 3.1311) -- (1.3160, 3.0120, 3.1299) -- cycle;
\fill[blue!36.6, opacity=0.7] (1.3160, 3.0120, 3.1299) -- (1.3620, 3.0120, 3.1311) -- (1.3620, 3.0660, 3.1248) -- (1.3160, 3.0660, 3.1237) -- cycle;
\fill[blue!56.7, opacity=0.7] (1.3160, 3.0660, 3.1237) -- (1.3620, 3.0660, 3.1248) -- (1.3620, 3.1200, 3.1185) -- (1.3160, 3.1200, 3.1174) -- cycle;
\fill[blue!42.1, opacity=0.7] (1.3620, -0.1200, 3.1185) -- (1.4080, -0.1200, 3.1193) -- (1.4080, -0.0660, 3.1256) -- (1.3620, -0.0660, 3.1248) -- cycle;
\fill[blue!24.9, opacity=0.7] (1.3620, -0.0660, 3.1248) -- (1.4080, -0.0660, 3.1256) -- (1.4080, -0.0120, 3.1319) -- (1.3620, -0.0120, 3.1311) -- cycle;
\fill[blue!20.7, opacity=0.7] (1.3620, -0.0120, 3.1311) -- (1.4080, -0.0120, 3.1319) -- (1.4080, 0.0420, 3.1381) -- (1.3620, 0.0420, 3.1373) -- cycle;
\fill[blue!26.0, opacity=0.7] (1.3620, 0.0420, 3.1373) -- (1.4080, 0.0420, 3.1381) -- (1.4080, 0.0960, 3.1443) -- (1.3620, 0.0960, 3.1435) -- cycle;
\fill[blue!48.4, opacity=0.7] (1.3620, 0.0960, 3.1435) -- (1.4080, 0.0960, 3.1443) -- (1.4080, 0.1500, 3.1504) -- (1.3620, 0.1500, 3.1496) -- cycle;
\fill[blue!63.3, opacity=0.7] (1.3620, 0.1500, 3.1496) -- (1.4080, 0.1500, 3.1504) -- (1.4080, 0.2040, 3.1564) -- (1.3620, 0.2040, 3.1556) -- cycle;
\fill[blue!52.5, opacity=0.7] (1.3620, 0.2040, 3.1556) -- (1.4080, 0.2040, 3.1564) -- (1.4080, 0.2580, 3.1623) -- (1.3620, 0.2580, 3.1615) -- cycle;
\fill[blue!52.7, opacity=0.7] (1.3620, 0.2580, 3.1615) -- (1.4080, 0.2580, 3.1623) -- (1.4080, 0.3120, 3.1682) -- (1.3620, 0.3120, 3.1673) -- cycle;
\fill[blue!63.0, opacity=0.7] (1.3620, 0.3120, 3.1673) -- (1.4080, 0.3120, 3.1682) -- (1.4080, 0.3660, 3.1738) -- (1.3620, 0.3660, 3.1730) -- cycle;
\fill[blue!55.5, opacity=0.7] (1.3620, 0.3660, 3.1730) -- (1.4080, 0.3660, 3.1738) -- (1.4080, 0.4200, 3.1793) -- (1.3620, 0.4200, 3.1785) -- cycle;
\fill[blue!42.5, opacity=0.7] (1.3620, 0.4200, 3.1785) -- (1.4080, 0.4200, 3.1793) -- (1.4080, 0.4740, 3.1847) -- (1.3620, 0.4740, 3.1839) -- cycle;
\fill[blue!46.6, opacity=0.7] (1.3620, 0.4740, 3.1839) -- (1.4080, 0.4740, 3.1847) -- (1.4080, 0.5280, 3.1899) -- (1.3620, 0.5280, 3.1891) -- cycle;
\fill[blue!62.7, opacity=0.7] (1.3620, 0.5280, 3.1891) -- (1.4080, 0.5280, 3.1899) -- (1.4080, 0.5820, 3.1949) -- (1.3620, 0.5820, 3.1940) -- cycle;
\fill[blue!50.4, opacity=0.7] (1.3620, 0.5820, 3.1940) -- (1.4080, 0.5820, 3.1949) -- (1.4080, 0.6360, 3.1996) -- (1.3620, 0.6360, 3.1988) -- cycle;
\fill[blue!29.3, opacity=0.7] (1.3620, 0.6360, 3.1988) -- (1.4080, 0.6360, 3.1996) -- (1.4080, 0.6900, 3.2042) -- (1.3620, 0.6900, 3.2034) -- cycle;
\fill[blue!25.5, opacity=0.7] (1.3620, 0.6900, 3.2034) -- (1.4080, 0.6900, 3.2042) -- (1.4080, 0.7440, 3.2085) -- (1.3620, 0.7440, 3.2077) -- cycle;
\fill[blue!35.6, opacity=0.7] (1.3620, 0.7440, 3.2077) -- (1.4080, 0.7440, 3.2085) -- (1.4080, 0.7980, 3.2126) -- (1.3620, 0.7980, 3.2118) -- cycle;
\fill[blue!56.7, opacity=0.7] (1.3620, 0.7980, 3.2118) -- (1.4080, 0.7980, 3.2126) -- (1.4080, 0.8520, 3.2164) -- (1.3620, 0.8520, 3.2156) -- cycle;
\fill[blue!63.5, opacity=0.7] (1.3620, 0.8520, 3.2156) -- (1.4080, 0.8520, 3.2164) -- (1.4080, 0.9060, 3.2200) -- (1.3620, 0.9060, 3.2192) -- cycle;
\fill[blue!63.5, opacity=0.7] (1.3620, 0.9060, 3.2192) -- (1.4080, 0.9060, 3.2200) -- (1.4080, 0.9600, 3.2233) -- (1.3620, 0.9600, 3.2224) -- cycle;
\fill[blue!56.4, opacity=0.7] (1.3620, 0.9600, 3.2224) -- (1.4080, 0.9600, 3.2233) -- (1.4080, 1.0140, 3.2263) -- (1.3620, 1.0140, 3.2254) -- cycle;
\fill[blue!29.4, opacity=0.7] (1.3620, 1.0140, 3.2254) -- (1.4080, 1.0140, 3.2263) -- (1.4080, 1.0680, 3.2290) -- (1.3620, 1.0680, 3.2281) -- cycle;
\fill[blue!17.6, opacity=0.7] (1.3620, 1.0680, 3.2281) -- (1.4080, 1.0680, 3.2290) -- (1.4080, 1.1220, 3.2314) -- (1.3620, 1.1220, 3.2306) -- cycle;
\fill[blue!16.3, opacity=0.7] (1.3620, 1.1220, 3.2306) -- (1.4080, 1.1220, 3.2314) -- (1.4080, 1.1760, 3.2335) -- (1.3620, 1.1760, 3.2326) -- cycle;
\fill[blue!18.9, opacity=0.7] (1.3620, 1.1760, 3.2326) -- (1.4080, 1.1760, 3.2335) -- (1.4080, 1.2300, 3.2353) -- (1.3620, 1.2300, 3.2344) -- cycle;
\fill[blue!33.0, opacity=0.7] (1.3620, 1.2300, 3.2344) -- (1.4080, 1.2300, 3.2353) -- (1.4080, 1.2840, 3.2367) -- (1.3620, 1.2840, 3.2359) -- cycle;
\fill[blue!51.1, opacity=0.7] (1.3620, 1.2840, 3.2359) -- (1.4080, 1.2840, 3.2367) -- (1.4080, 1.3380, 3.2379) -- (1.3620, 1.3380, 3.2370) -- cycle;
\fill[blue!47.4, opacity=0.7] (1.3620, 1.3380, 3.2370) -- (1.4080, 1.3380, 3.2379) -- (1.4080, 1.3920, 3.2387) -- (1.3620, 1.3920, 3.2379) -- cycle;
\fill[blue!23.5, opacity=0.7] (1.3620, 1.3920, 3.2379) -- (1.4080, 1.3920, 3.2387) -- (1.4080, 1.4460, 3.2392) -- (1.3620, 1.4460, 3.2384) -- cycle;
\fill[blue!15.7, opacity=0.7] (1.3620, 1.4460, 3.2384) -- (1.4080, 1.4460, 3.2392) -- (1.4080, 1.5000, 3.2393) -- (1.3620, 1.5000, 3.2385) -- cycle;
\fill[blue!15.3, opacity=0.7] (1.3620, 1.5000, 3.2385) -- (1.4080, 1.5000, 3.2393) -- (1.4080, 1.5540, 3.2392) -- (1.3620, 1.5540, 3.2384) -- cycle;
\fill[blue!15.4, opacity=0.7] (1.3620, 1.5540, 3.2384) -- (1.4080, 1.5540, 3.2392) -- (1.4080, 1.6080, 3.2387) -- (1.3620, 1.6080, 3.2379) -- cycle;
\fill[blue!15.6, opacity=0.7] (1.3620, 1.6080, 3.2379) -- (1.4080, 1.6080, 3.2387) -- (1.4080, 1.6620, 3.2379) -- (1.3620, 1.6620, 3.2370) -- cycle;
\fill[blue!15.6, opacity=0.7] (1.3620, 1.6620, 3.2370) -- (1.4080, 1.6620, 3.2379) -- (1.4080, 1.7160, 3.2367) -- (1.3620, 1.7160, 3.2359) -- cycle;
\fill[blue!15.7, opacity=0.7] (1.3620, 1.7160, 3.2359) -- (1.4080, 1.7160, 3.2367) -- (1.4080, 1.7700, 3.2353) -- (1.3620, 1.7700, 3.2344) -- cycle;
\fill[blue!16.8, opacity=0.7] (1.3620, 1.7700, 3.2344) -- (1.4080, 1.7700, 3.2353) -- (1.4080, 1.8240, 3.2335) -- (1.3620, 1.8240, 3.2326) -- cycle;
\fill[blue!22.8, opacity=0.7] (1.3620, 1.8240, 3.2326) -- (1.4080, 1.8240, 3.2335) -- (1.4080, 1.8780, 3.2314) -- (1.3620, 1.8780, 3.2306) -- cycle;
\fill[blue!41.9, opacity=0.7] (1.3620, 1.8780, 3.2306) -- (1.4080, 1.8780, 3.2314) -- (1.4080, 1.9320, 3.2290) -- (1.3620, 1.9320, 3.2281) -- cycle;
\fill[blue!60.4, opacity=0.7] (1.3620, 1.9320, 3.2281) -- (1.4080, 1.9320, 3.2290) -- (1.4080, 1.9860, 3.2263) -- (1.3620, 1.9860, 3.2254) -- cycle;
\fill[blue!63.6, opacity=0.7] (1.3620, 1.9860, 3.2254) -- (1.4080, 1.9860, 3.2263) -- (1.4080, 2.0400, 3.2233) -- (1.3620, 2.0400, 3.2224) -- cycle;
\fill[blue!62.7, opacity=0.7] (1.3620, 2.0400, 3.2224) -- (1.4080, 2.0400, 3.2233) -- (1.4080, 2.0940, 3.2200) -- (1.3620, 2.0940, 3.2192) -- cycle;
\fill[blue!51.1, opacity=0.7] (1.3620, 2.0940, 3.2192) -- (1.4080, 2.0940, 3.2200) -- (1.4080, 2.1480, 3.2164) -- (1.3620, 2.1480, 3.2156) -- cycle;
\fill[blue!31.9, opacity=0.7] (1.3620, 2.1480, 3.2156) -- (1.4080, 2.1480, 3.2164) -- (1.4080, 2.2020, 3.2126) -- (1.3620, 2.2020, 3.2118) -- cycle;
\fill[blue!23.7, opacity=0.7] (1.3620, 2.2020, 3.2118) -- (1.4080, 2.2020, 3.2126) -- (1.4080, 2.2560, 3.2085) -- (1.3620, 2.2560, 3.2077) -- cycle;
\fill[blue!26.1, opacity=0.7] (1.3620, 2.2560, 3.2077) -- (1.4080, 2.2560, 3.2085) -- (1.4080, 2.3100, 3.2042) -- (1.3620, 2.3100, 3.2034) -- cycle;
\fill[blue!43.2, opacity=0.7] (1.3620, 2.3100, 3.2034) -- (1.4080, 2.3100, 3.2042) -- (1.4080, 2.3640, 3.1996) -- (1.3620, 2.3640, 3.1988) -- cycle;
\fill[blue!63.3, opacity=0.7] (1.3620, 2.3640, 3.1988) -- (1.4080, 2.3640, 3.1996) -- (1.4080, 2.4180, 3.1949) -- (1.3620, 2.4180, 3.1940) -- cycle;
\fill[blue!53.0, opacity=0.7] (1.3620, 2.4180, 3.1940) -- (1.4080, 2.4180, 3.1949) -- (1.4080, 2.4720, 3.1899) -- (1.3620, 2.4720, 3.1891) -- cycle;
\fill[blue!44.1, opacity=0.7] (1.3620, 2.4720, 3.1891) -- (1.4080, 2.4720, 3.1899) -- (1.4080, 2.5260, 3.1847) -- (1.3620, 2.5260, 3.1839) -- cycle;
\fill[blue!52.3, opacity=0.7] (1.3620, 2.5260, 3.1839) -- (1.4080, 2.5260, 3.1847) -- (1.4080, 2.5800, 3.1793) -- (1.3620, 2.5800, 3.1785) -- cycle;
\fill[blue!63.5, opacity=0.7] (1.3620, 2.5800, 3.1785) -- (1.4080, 2.5800, 3.1793) -- (1.4080, 2.6340, 3.1738) -- (1.3620, 2.6340, 3.1730) -- cycle;
\fill[blue!55.2, opacity=0.7] (1.3620, 2.6340, 3.1730) -- (1.4080, 2.6340, 3.1738) -- (1.4080, 2.6880, 3.1682) -- (1.3620, 2.6880, 3.1673) -- cycle;
\fill[blue!48.4, opacity=0.7] (1.3620, 2.6880, 3.1673) -- (1.4080, 2.6880, 3.1682) -- (1.4080, 2.7420, 3.1623) -- (1.3620, 2.7420, 3.1615) -- cycle;
\fill[blue!57.8, opacity=0.7] (1.3620, 2.7420, 3.1615) -- (1.4080, 2.7420, 3.1623) -- (1.4080, 2.7960, 3.1564) -- (1.3620, 2.7960, 3.1556) -- cycle;
\fill[blue!60.9, opacity=0.7] (1.3620, 2.7960, 3.1556) -- (1.4080, 2.7960, 3.1564) -- (1.4080, 2.8500, 3.1504) -- (1.3620, 2.8500, 3.1496) -- cycle;
\fill[blue!36.1, opacity=0.7] (1.3620, 2.8500, 3.1496) -- (1.4080, 2.8500, 3.1504) -- (1.4080, 2.9040, 3.1443) -- (1.3620, 2.9040, 3.1435) -- cycle;
\fill[blue!22.7, opacity=0.7] (1.3620, 2.9040, 3.1435) -- (1.4080, 2.9040, 3.1443) -- (1.4080, 2.9580, 3.1381) -- (1.3620, 2.9580, 3.1373) -- cycle;
\fill[blue!22.3, opacity=0.7] (1.3620, 2.9580, 3.1373) -- (1.4080, 2.9580, 3.1381) -- (1.4080, 3.0120, 3.1319) -- (1.3620, 3.0120, 3.1311) -- cycle;
\fill[blue!32.5, opacity=0.7] (1.3620, 3.0120, 3.1311) -- (1.4080, 3.0120, 3.1319) -- (1.4080, 3.0660, 3.1256) -- (1.3620, 3.0660, 3.1248) -- cycle;
\fill[blue!53.2, opacity=0.7] (1.3620, 3.0660, 3.1248) -- (1.4080, 3.0660, 3.1256) -- (1.4080, 3.1200, 3.1193) -- (1.3620, 3.1200, 3.1185) -- cycle;
\fill[blue!42.9, opacity=0.7] (1.4080, -0.1200, 3.1193) -- (1.4540, -0.1200, 3.1198) -- (1.4540, -0.0660, 3.1261) -- (1.4080, -0.0660, 3.1256) -- cycle;
\fill[blue!25.4, opacity=0.7] (1.4080, -0.0660, 3.1256) -- (1.4540, -0.0660, 3.1261) -- (1.4540, -0.0120, 3.1324) -- (1.4080, -0.0120, 3.1319) -- cycle;
\fill[blue!20.8, opacity=0.7] (1.4080, -0.0120, 3.1319) -- (1.4540, -0.0120, 3.1324) -- (1.4540, 0.0420, 3.1386) -- (1.4080, 0.0420, 3.1381) -- cycle;
\fill[blue!26.0, opacity=0.7] (1.4080, 0.0420, 3.1381) -- (1.4540, 0.0420, 3.1386) -- (1.4540, 0.0960, 3.1448) -- (1.4080, 0.0960, 3.1443) -- cycle;
\fill[blue!48.1, opacity=0.7] (1.4080, 0.0960, 3.1443) -- (1.4540, 0.0960, 3.1448) -- (1.4540, 0.1500, 3.1509) -- (1.4080, 0.1500, 3.1504) -- cycle;
\fill[blue!63.4, opacity=0.7] (1.4080, 0.1500, 3.1504) -- (1.4540, 0.1500, 3.1509) -- (1.4540, 0.2040, 3.1569) -- (1.4080, 0.2040, 3.1564) -- cycle;
\fill[blue!52.4, opacity=0.7] (1.4080, 0.2040, 3.1564) -- (1.4540, 0.2040, 3.1569) -- (1.4540, 0.2580, 3.1628) -- (1.4080, 0.2580, 3.1623) -- cycle;
\fill[blue!52.1, opacity=0.7] (1.4080, 0.2580, 3.1623) -- (1.4540, 0.2580, 3.1628) -- (1.4540, 0.3120, 3.1686) -- (1.4080, 0.3120, 3.1682) -- cycle;
\fill[blue!62.8, opacity=0.7] (1.4080, 0.3120, 3.1682) -- (1.4540, 0.3120, 3.1686) -- (1.4540, 0.3660, 3.1743) -- (1.4080, 0.3660, 3.1738) -- cycle;
\fill[blue!56.4, opacity=0.7] (1.4080, 0.3660, 3.1738) -- (1.4540, 0.3660, 3.1743) -- (1.4540, 0.4200, 3.1798) -- (1.4080, 0.4200, 3.1793) -- cycle;
\fill[blue!43.2, opacity=0.7] (1.4080, 0.4200, 3.1793) -- (1.4540, 0.4200, 3.1798) -- (1.4540, 0.4740, 3.1852) -- (1.4080, 0.4740, 3.1847) -- cycle;
\fill[blue!46.7, opacity=0.7] (1.4080, 0.4740, 3.1847) -- (1.4540, 0.4740, 3.1852) -- (1.4540, 0.5280, 3.1904) -- (1.4080, 0.5280, 3.1899) -- cycle;
\fill[blue!62.5, opacity=0.7] (1.4080, 0.5280, 3.1899) -- (1.4540, 0.5280, 3.1904) -- (1.4540, 0.5820, 3.1954) -- (1.4080, 0.5820, 3.1949) -- cycle;
\fill[blue!50.9, opacity=0.7] (1.4080, 0.5820, 3.1949) -- (1.4540, 0.5820, 3.1954) -- (1.4540, 0.6360, 3.2001) -- (1.4080, 0.6360, 3.1996) -- cycle;
\fill[blue!29.4, opacity=0.7] (1.4080, 0.6360, 3.1996) -- (1.4540, 0.6360, 3.2001) -- (1.4540, 0.6900, 3.2047) -- (1.4080, 0.6900, 3.2042) -- cycle;
\fill[blue!25.0, opacity=0.7] (1.4080, 0.6900, 3.2042) -- (1.4540, 0.6900, 3.2047) -- (1.4540, 0.7440, 3.2090) -- (1.4080, 0.7440, 3.2085) -- cycle;
\fill[blue!34.2, opacity=0.7] (1.4080, 0.7440, 3.2085) -- (1.4540, 0.7440, 3.2090) -- (1.4540, 0.7980, 3.2131) -- (1.4080, 0.7980, 3.2126) -- cycle;
\fill[blue!55.2, opacity=0.7] (1.4080, 0.7980, 3.2126) -- (1.4540, 0.7980, 3.2131) -- (1.4540, 0.8520, 3.2169) -- (1.4080, 0.8520, 3.2164) -- cycle;
\fill[blue!63.6, opacity=0.7] (1.4080, 0.8520, 3.2164) -- (1.4540, 0.8520, 3.2169) -- (1.4540, 0.9060, 3.2205) -- (1.4080, 0.9060, 3.2200) -- cycle;
\fill[blue!63.5, opacity=0.7] (1.4080, 0.9060, 3.2200) -- (1.4540, 0.9060, 3.2205) -- (1.4540, 0.9600, 3.2238) -- (1.4080, 0.9600, 3.2233) -- cycle;
\fill[blue!56.9, opacity=0.7] (1.4080, 0.9600, 3.2233) -- (1.4540, 0.9600, 3.2238) -- (1.4540, 1.0140, 3.2268) -- (1.4080, 1.0140, 3.2263) -- cycle;
\fill[blue!30.2, opacity=0.7] (1.4080, 1.0140, 3.2263) -- (1.4540, 1.0140, 3.2268) -- (1.4540, 1.0680, 3.2295) -- (1.4080, 1.0680, 3.2290) -- cycle;
\fill[blue!17.7, opacity=0.7] (1.4080, 1.0680, 3.2290) -- (1.4540, 1.0680, 3.2295) -- (1.4540, 1.1220, 3.2319) -- (1.4080, 1.1220, 3.2314) -- cycle;
\fill[blue!16.2, opacity=0.7] (1.4080, 1.1220, 3.2314) -- (1.4540, 1.1220, 3.2319) -- (1.4540, 1.1760, 3.2340) -- (1.4080, 1.1760, 3.2335) -- cycle;
\fill[blue!18.1, opacity=0.7] (1.4080, 1.1760, 3.2335) -- (1.4540, 1.1760, 3.2340) -- (1.4540, 1.2300, 3.2357) -- (1.4080, 1.2300, 3.2353) -- cycle;
\fill[blue!30.0, opacity=0.7] (1.4080, 1.2300, 3.2353) -- (1.4540, 1.2300, 3.2357) -- (1.4540, 1.2840, 3.2372) -- (1.4080, 1.2840, 3.2367) -- cycle;
\fill[blue!48.5, opacity=0.7] (1.4080, 1.2840, 3.2367) -- (1.4540, 1.2840, 3.2372) -- (1.4540, 1.3380, 3.2384) -- (1.4080, 1.3380, 3.2379) -- cycle;
\fill[blue!45.6, opacity=0.7] (1.4080, 1.3380, 3.2379) -- (1.4540, 1.3380, 3.2384) -- (1.4540, 1.3920, 3.2392) -- (1.4080, 1.3920, 3.2387) -- cycle;
\fill[blue!20.5, opacity=0.7] (1.4080, 1.3920, 3.2387) -- (1.4540, 1.3920, 3.2392) -- (1.4540, 1.4460, 3.2397) -- (1.4080, 1.4460, 3.2392) -- cycle;
\fill[blue!15.2, opacity=0.7] (1.4080, 1.4460, 3.2392) -- (1.4540, 1.4460, 3.2397) -- (1.4540, 1.5000, 3.2398) -- (1.4080, 1.5000, 3.2393) -- cycle;
\fill[blue!16.0, opacity=0.7] (1.4080, 1.5000, 3.2393) -- (1.4540, 1.5000, 3.2398) -- (1.4540, 1.5540, 3.2397) -- (1.4080, 1.5540, 3.2392) -- cycle;
\fill[blue!20.0, opacity=0.7] (1.4080, 1.5540, 3.2392) -- (1.4540, 1.5540, 3.2397) -- (1.4540, 1.6080, 3.2392) -- (1.4080, 1.6080, 3.2387) -- cycle;
\fill[blue!19.7, opacity=0.7] (1.4080, 1.6080, 3.2387) -- (1.4540, 1.6080, 3.2392) -- (1.4540, 1.6620, 3.2384) -- (1.4080, 1.6620, 3.2379) -- cycle;
\fill[blue!17.1, opacity=0.7] (1.4080, 1.6620, 3.2379) -- (1.4540, 1.6620, 3.2384) -- (1.4540, 1.7160, 3.2372) -- (1.4080, 1.7160, 3.2367) -- cycle;
\fill[blue!15.9, opacity=0.7] (1.4080, 1.7160, 3.2367) -- (1.4540, 1.7160, 3.2372) -- (1.4540, 1.7700, 3.2357) -- (1.4080, 1.7700, 3.2353) -- cycle;
\fill[blue!15.8, opacity=0.7] (1.4080, 1.7700, 3.2353) -- (1.4540, 1.7700, 3.2357) -- (1.4540, 1.8240, 3.2340) -- (1.4080, 1.8240, 3.2335) -- cycle;
\fill[blue!17.5, opacity=0.7] (1.4080, 1.8240, 3.2335) -- (1.4540, 1.8240, 3.2340) -- (1.4540, 1.8780, 3.2319) -- (1.4080, 1.8780, 3.2314) -- cycle;
\fill[blue!27.8, opacity=0.7] (1.4080, 1.8780, 3.2314) -- (1.4540, 1.8780, 3.2319) -- (1.4540, 1.9320, 3.2295) -- (1.4080, 1.9320, 3.2290) -- cycle;
\fill[blue!52.1, opacity=0.7] (1.4080, 1.9320, 3.2290) -- (1.4540, 1.9320, 3.2295) -- (1.4540, 1.9860, 3.2268) -- (1.4080, 1.9860, 3.2263) -- cycle;
\fill[blue!63.3, opacity=0.7] (1.4080, 1.9860, 3.2263) -- (1.4540, 1.9860, 3.2268) -- (1.4540, 2.0400, 3.2238) -- (1.4080, 2.0400, 3.2233) -- cycle;
\fill[blue!63.6, opacity=0.7] (1.4080, 2.0400, 3.2233) -- (1.4540, 2.0400, 3.2238) -- (1.4540, 2.0940, 3.2205) -- (1.4080, 2.0940, 3.2200) -- cycle;
\fill[blue!58.7, opacity=0.7] (1.4080, 2.0940, 3.2200) -- (1.4540, 2.0940, 3.2205) -- (1.4540, 2.1480, 3.2169) -- (1.4080, 2.1480, 3.2164) -- cycle;
\fill[blue!39.4, opacity=0.7] (1.4080, 2.1480, 3.2164) -- (1.4540, 2.1480, 3.2169) -- (1.4540, 2.2020, 3.2131) -- (1.4080, 2.2020, 3.2126) -- cycle;
\fill[blue!25.7, opacity=0.7] (1.4080, 2.2020, 3.2126) -- (1.4540, 2.2020, 3.2131) -- (1.4540, 2.2560, 3.2090) -- (1.4080, 2.2560, 3.2085) -- cycle;
\fill[blue!24.5, opacity=0.7] (1.4080, 2.2560, 3.2085) -- (1.4540, 2.2560, 3.2090) -- (1.4540, 2.3100, 3.2047) -- (1.4080, 2.3100, 3.2042) -- cycle;
\fill[blue!36.5, opacity=0.7] (1.4080, 2.3100, 3.2042) -- (1.4540, 2.3100, 3.2047) -- (1.4540, 2.3640, 3.2001) -- (1.4080, 2.3640, 3.1996) -- cycle;
\fill[blue!60.6, opacity=0.7] (1.4080, 2.3640, 3.1996) -- (1.4540, 2.3640, 3.2001) -- (1.4540, 2.4180, 3.1954) -- (1.4080, 2.4180, 3.1949) -- cycle;
\fill[blue!57.0, opacity=0.7] (1.4080, 2.4180, 3.1949) -- (1.4540, 2.4180, 3.1954) -- (1.4540, 2.4720, 3.1904) -- (1.4080, 2.4720, 3.1899) -- cycle;
\fill[blue!44.3, opacity=0.7] (1.4080, 2.4720, 3.1899) -- (1.4540, 2.4720, 3.1904) -- (1.4540, 2.5260, 3.1852) -- (1.4080, 2.5260, 3.1847) -- cycle;
\fill[blue!48.5, opacity=0.7] (1.4080, 2.5260, 3.1847) -- (1.4540, 2.5260, 3.1852) -- (1.4540, 2.5800, 3.1798) -- (1.4080, 2.5800, 3.1793) -- cycle;
\fill[blue!62.2, opacity=0.7] (1.4080, 2.5800, 3.1793) -- (1.4540, 2.5800, 3.1798) -- (1.4540, 2.6340, 3.1743) -- (1.4080, 2.6340, 3.1738) -- cycle;
\fill[blue!58.3, opacity=0.7] (1.4080, 2.6340, 3.1738) -- (1.4540, 2.6340, 3.1743) -- (1.4540, 2.6880, 3.1686) -- (1.4080, 2.6880, 3.1682) -- cycle;
\fill[blue!49.0, opacity=0.7] (1.4080, 2.6880, 3.1682) -- (1.4540, 2.6880, 3.1686) -- (1.4540, 2.7420, 3.1628) -- (1.4080, 2.7420, 3.1623) -- cycle;
\fill[blue!55.3, opacity=0.7] (1.4080, 2.7420, 3.1623) -- (1.4540, 2.7420, 3.1628) -- (1.4540, 2.7960, 3.1569) -- (1.4080, 2.7960, 3.1564) -- cycle;
\fill[blue!62.9, opacity=0.7] (1.4080, 2.7960, 3.1564) -- (1.4540, 2.7960, 3.1569) -- (1.4540, 2.8500, 3.1509) -- (1.4080, 2.8500, 3.1504) -- cycle;
\fill[blue!40.4, opacity=0.7] (1.4080, 2.8500, 3.1504) -- (1.4540, 2.8500, 3.1509) -- (1.4540, 2.9040, 3.1448) -- (1.4080, 2.9040, 3.1443) -- cycle;
\fill[blue!23.7, opacity=0.7] (1.4080, 2.9040, 3.1443) -- (1.4540, 2.9040, 3.1448) -- (1.4540, 2.9580, 3.1386) -- (1.4080, 2.9580, 3.1381) -- cycle;
\fill[blue!21.6, opacity=0.7] (1.4080, 2.9580, 3.1381) -- (1.4540, 2.9580, 3.1386) -- (1.4540, 3.0120, 3.1324) -- (1.4080, 3.0120, 3.1319) -- cycle;
\fill[blue!29.6, opacity=0.7] (1.4080, 3.0120, 3.1319) -- (1.4540, 3.0120, 3.1324) -- (1.4540, 3.0660, 3.1261) -- (1.4080, 3.0660, 3.1256) -- cycle;
\fill[blue!49.8, opacity=0.7] (1.4080, 3.0660, 3.1256) -- (1.4540, 3.0660, 3.1261) -- (1.4540, 3.1200, 3.1198) -- (1.4080, 3.1200, 3.1193) -- cycle;
\fill[blue!44.5, opacity=0.7] (1.4540, -0.1200, 3.1198) -- (1.5000, -0.1200, 3.1200) -- (1.5000, -0.0660, 3.1263) -- (1.4540, -0.0660, 3.1261) -- cycle;
\fill[blue!26.2, opacity=0.7] (1.4540, -0.0660, 3.1261) -- (1.5000, -0.0660, 3.1263) -- (1.5000, -0.0120, 3.1325) -- (1.4540, -0.0120, 3.1324) -- cycle;
\fill[blue!21.0, opacity=0.7] (1.4540, -0.0120, 3.1324) -- (1.5000, -0.0120, 3.1325) -- (1.5000, 0.0420, 3.1388) -- (1.4540, 0.0420, 3.1386) -- cycle;
\fill[blue!25.5, opacity=0.7] (1.4540, 0.0420, 3.1386) -- (1.5000, 0.0420, 3.1388) -- (1.5000, 0.0960, 3.1449) -- (1.4540, 0.0960, 3.1448) -- cycle;
\fill[blue!46.6, opacity=0.7] (1.4540, 0.0960, 3.1448) -- (1.5000, 0.0960, 3.1449) -- (1.5000, 0.1500, 3.1511) -- (1.4540, 0.1500, 3.1509) -- cycle;
\fill[blue!63.5, opacity=0.7] (1.4540, 0.1500, 3.1509) -- (1.5000, 0.1500, 3.1511) -- (1.5000, 0.2040, 3.1571) -- (1.4540, 0.2040, 3.1569) -- cycle;
\fill[blue!52.7, opacity=0.7] (1.4540, 0.2040, 3.1569) -- (1.5000, 0.2040, 3.1571) -- (1.5000, 0.2580, 3.1630) -- (1.4540, 0.2580, 3.1628) -- cycle;
\fill[blue!51.1, opacity=0.7] (1.4540, 0.2580, 3.1628) -- (1.5000, 0.2580, 3.1630) -- (1.5000, 0.3120, 3.1688) -- (1.4540, 0.3120, 3.1686) -- cycle;
\fill[blue!62.0, opacity=0.7] (1.4540, 0.3120, 3.1686) -- (1.5000, 0.3120, 3.1688) -- (1.5000, 0.3660, 3.1745) -- (1.4540, 0.3660, 3.1743) -- cycle;
\fill[blue!58.0, opacity=0.7] (1.4540, 0.3660, 3.1743) -- (1.5000, 0.3660, 3.1745) -- (1.5000, 0.4200, 3.1800) -- (1.4540, 0.4200, 3.1798) -- cycle;
\fill[blue!44.2, opacity=0.7] (1.4540, 0.4200, 3.1798) -- (1.5000, 0.4200, 3.1800) -- (1.5000, 0.4740, 3.1854) -- (1.4540, 0.4740, 3.1852) -- cycle;
\fill[blue!46.1, opacity=0.7] (1.4540, 0.4740, 3.1852) -- (1.5000, 0.4740, 3.1854) -- (1.5000, 0.5280, 3.1905) -- (1.4540, 0.5280, 3.1904) -- cycle;
\fill[blue!61.7, opacity=0.7] (1.4540, 0.5280, 3.1904) -- (1.5000, 0.5280, 3.1905) -- (1.5000, 0.5820, 3.1955) -- (1.4540, 0.5820, 3.1954) -- cycle;
\fill[blue!53.0, opacity=0.7] (1.4540, 0.5820, 3.1954) -- (1.5000, 0.5820, 3.1955) -- (1.5000, 0.6360, 3.2003) -- (1.4540, 0.6360, 3.2001) -- cycle;
\fill[blue!30.3, opacity=0.7] (1.4540, 0.6360, 3.2001) -- (1.5000, 0.6360, 3.2003) -- (1.5000, 0.6900, 3.2049) -- (1.4540, 0.6900, 3.2047) -- cycle;
\fill[blue!24.5, opacity=0.7] (1.4540, 0.6900, 3.2047) -- (1.5000, 0.6900, 3.2049) -- (1.5000, 0.7440, 3.2092) -- (1.4540, 0.7440, 3.2090) -- cycle;
\fill[blue!31.7, opacity=0.7] (1.4540, 0.7440, 3.2090) -- (1.5000, 0.7440, 3.2092) -- (1.5000, 0.7980, 3.2133) -- (1.4540, 0.7980, 3.2131) -- cycle;
\fill[blue!52.0, opacity=0.7] (1.4540, 0.7980, 3.2131) -- (1.5000, 0.7980, 3.2133) -- (1.5000, 0.8520, 3.2171) -- (1.4540, 0.8520, 3.2169) -- cycle;
\fill[blue!63.4, opacity=0.7] (1.4540, 0.8520, 3.2169) -- (1.5000, 0.8520, 3.2171) -- (1.5000, 0.9060, 3.2206) -- (1.4540, 0.9060, 3.2205) -- cycle;
\fill[blue!63.5, opacity=0.7] (1.4540, 0.9060, 3.2205) -- (1.5000, 0.9060, 3.2206) -- (1.5000, 0.9600, 3.2239) -- (1.4540, 0.9600, 3.2238) -- cycle;
\fill[blue!59.2, opacity=0.7] (1.4540, 0.9600, 3.2238) -- (1.5000, 0.9600, 3.2239) -- (1.5000, 1.0140, 3.2269) -- (1.4540, 1.0140, 3.2268) -- cycle;
\fill[blue!34.0, opacity=0.7] (1.4540, 1.0140, 3.2268) -- (1.5000, 1.0140, 3.2269) -- (1.5000, 1.0680, 3.2296) -- (1.4540, 1.0680, 3.2295) -- cycle;
\fill[blue!18.5, opacity=0.7] (1.4540, 1.0680, 3.2295) -- (1.5000, 1.0680, 3.2296) -- (1.5000, 1.1220, 3.2320) -- (1.4540, 1.1220, 3.2319) -- cycle;
\fill[blue!16.1, opacity=0.7] (1.4540, 1.1220, 3.2319) -- (1.5000, 1.1220, 3.2320) -- (1.5000, 1.1760, 3.2341) -- (1.4540, 1.1760, 3.2340) -- cycle;
\fill[blue!16.8, opacity=0.7] (1.4540, 1.1760, 3.2340) -- (1.5000, 1.1760, 3.2341) -- (1.5000, 1.2300, 3.2359) -- (1.4540, 1.2300, 3.2357) -- cycle;
\fill[blue!23.6, opacity=0.7] (1.4540, 1.2300, 3.2357) -- (1.5000, 1.2300, 3.2359) -- (1.5000, 1.2840, 3.2374) -- (1.4540, 1.2840, 3.2372) -- cycle;
\fill[blue!41.0, opacity=0.7] (1.4540, 1.2840, 3.2372) -- (1.5000, 1.2840, 3.2374) -- (1.5000, 1.3380, 3.2385) -- (1.4540, 1.3380, 3.2384) -- cycle;
\fill[blue!48.0, opacity=0.7] (1.4540, 1.3380, 3.2384) -- (1.5000, 1.3380, 3.2385) -- (1.5000, 1.3920, 3.2393) -- (1.4540, 1.3920, 3.2392) -- cycle;
\fill[blue!29.3, opacity=0.7] (1.4540, 1.3920, 3.2392) -- (1.5000, 1.3920, 3.2393) -- (1.5000, 1.4460, 3.2398) -- (1.4540, 1.4460, 3.2397) -- cycle;
\fill[blue!15.2, opacity=0.7] (1.4540, 1.4460, 3.2397) -- (1.5000, 1.4460, 3.2398) -- (1.5000, 1.5000, 3.2400) -- (1.4540, 1.5000, 3.2398) -- cycle;
\fill[blue!37.0, opacity=0.7] (1.4540, 1.5000, 3.2398) -- (1.5000, 1.5000, 3.2400) -- (1.5000, 1.5540, 3.2398) -- (1.4540, 1.5540, 3.2397) -- cycle;
\fill[blue!43.5, opacity=0.7] (1.4540, 1.5540, 3.2397) -- (1.5000, 1.5540, 3.2398) -- (1.5000, 1.6080, 3.2393) -- (1.4540, 1.6080, 3.2392) -- cycle;
\fill[blue!39.1, opacity=0.7] (1.4540, 1.6080, 3.2392) -- (1.5000, 1.6080, 3.2393) -- (1.5000, 1.6620, 3.2385) -- (1.4540, 1.6620, 3.2384) -- cycle;
\fill[blue!26.5, opacity=0.7] (1.4540, 1.6620, 3.2384) -- (1.5000, 1.6620, 3.2385) -- (1.5000, 1.7160, 3.2374) -- (1.4540, 1.7160, 3.2372) -- cycle;
\fill[blue!18.0, opacity=0.7] (1.4540, 1.7160, 3.2372) -- (1.5000, 1.7160, 3.2374) -- (1.5000, 1.7700, 3.2359) -- (1.4540, 1.7700, 3.2357) -- cycle;
\fill[blue!16.0, opacity=0.7] (1.4540, 1.7700, 3.2357) -- (1.5000, 1.7700, 3.2359) -- (1.5000, 1.8240, 3.2341) -- (1.4540, 1.8240, 3.2340) -- cycle;
\fill[blue!16.3, opacity=0.7] (1.4540, 1.8240, 3.2340) -- (1.5000, 1.8240, 3.2341) -- (1.5000, 1.8780, 3.2320) -- (1.4540, 1.8780, 3.2319) -- cycle;
\fill[blue!21.0, opacity=0.7] (1.4540, 1.8780, 3.2319) -- (1.5000, 1.8780, 3.2320) -- (1.5000, 1.9320, 3.2296) -- (1.4540, 1.9320, 3.2295) -- cycle;
\fill[blue!41.6, opacity=0.7] (1.4540, 1.9320, 3.2295) -- (1.5000, 1.9320, 3.2296) -- (1.5000, 1.9860, 3.2269) -- (1.4540, 1.9860, 3.2268) -- cycle;
\fill[blue!61.8, opacity=0.7] (1.4540, 1.9860, 3.2268) -- (1.5000, 1.9860, 3.2269) -- (1.5000, 2.0400, 3.2239) -- (1.4540, 2.0400, 3.2238) -- cycle;
\fill[blue!63.5, opacity=0.7] (1.4540, 2.0400, 3.2238) -- (1.5000, 2.0400, 3.2239) -- (1.5000, 2.0940, 3.2206) -- (1.4540, 2.0940, 3.2205) -- cycle;
\fill[blue!62.2, opacity=0.7] (1.4540, 2.0940, 3.2205) -- (1.5000, 2.0940, 3.2206) -- (1.5000, 2.1480, 3.2171) -- (1.4540, 2.1480, 3.2169) -- cycle;
\fill[blue!46.6, opacity=0.7] (1.4540, 2.1480, 3.2169) -- (1.5000, 2.1480, 3.2171) -- (1.5000, 2.2020, 3.2133) -- (1.4540, 2.2020, 3.2131) -- cycle;
\fill[blue!28.6, opacity=0.7] (1.4540, 2.2020, 3.2131) -- (1.5000, 2.2020, 3.2133) -- (1.5000, 2.2560, 3.2092) -- (1.4540, 2.2560, 3.2090) -- cycle;
\fill[blue!24.2, opacity=0.7] (1.4540, 2.2560, 3.2090) -- (1.5000, 2.2560, 3.2092) -- (1.5000, 2.3100, 3.2049) -- (1.4540, 2.3100, 3.2047) -- cycle;
\fill[blue!32.5, opacity=0.7] (1.4540, 2.3100, 3.2047) -- (1.5000, 2.3100, 3.2049) -- (1.5000, 2.3640, 3.2003) -- (1.4540, 2.3640, 3.2001) -- cycle;
\fill[blue!56.6, opacity=0.7] (1.4540, 2.3640, 3.2001) -- (1.5000, 2.3640, 3.2003) -- (1.5000, 2.4180, 3.1955) -- (1.4540, 2.4180, 3.1954) -- cycle;
\fill[blue!60.0, opacity=0.7] (1.4540, 2.4180, 3.1954) -- (1.5000, 2.4180, 3.1955) -- (1.5000, 2.4720, 3.1905) -- (1.4540, 2.4720, 3.1904) -- cycle;
\fill[blue!45.1, opacity=0.7] (1.4540, 2.4720, 3.1904) -- (1.5000, 2.4720, 3.1905) -- (1.5000, 2.5260, 3.1854) -- (1.4540, 2.5260, 3.1852) -- cycle;
\fill[blue!45.9, opacity=0.7] (1.4540, 2.5260, 3.1852) -- (1.5000, 2.5260, 3.1854) -- (1.5000, 2.5800, 3.1800) -- (1.4540, 2.5800, 3.1798) -- cycle;
\fill[blue!60.1, opacity=0.7] (1.4540, 2.5800, 3.1798) -- (1.5000, 2.5800, 3.1800) -- (1.5000, 2.6340, 3.1745) -- (1.4540, 2.6340, 3.1743) -- cycle;
\fill[blue!60.6, opacity=0.7] (1.4540, 2.6340, 3.1743) -- (1.5000, 2.6340, 3.1745) -- (1.5000, 2.6880, 3.1688) -- (1.4540, 2.6880, 3.1686) -- cycle;
\fill[blue!50.0, opacity=0.7] (1.4540, 2.6880, 3.1686) -- (1.5000, 2.6880, 3.1688) -- (1.5000, 2.7420, 3.1630) -- (1.4540, 2.7420, 3.1628) -- cycle;
\fill[blue!53.7, opacity=0.7] (1.4540, 2.7420, 3.1628) -- (1.5000, 2.7420, 3.1630) -- (1.5000, 2.7960, 3.1571) -- (1.4540, 2.7960, 3.1569) -- cycle;
\fill[blue!63.5, opacity=0.7] (1.4540, 2.7960, 3.1569) -- (1.5000, 2.7960, 3.1571) -- (1.5000, 2.8500, 3.1511) -- (1.4540, 2.8500, 3.1509) -- cycle;
\fill[blue!43.9, opacity=0.7] (1.4540, 2.8500, 3.1509) -- (1.5000, 2.8500, 3.1511) -- (1.5000, 2.9040, 3.1449) -- (1.4540, 2.9040, 3.1448) -- cycle;
\fill[blue!24.6, opacity=0.7] (1.4540, 2.9040, 3.1448) -- (1.5000, 2.9040, 3.1449) -- (1.5000, 2.9580, 3.1388) -- (1.4540, 2.9580, 3.1386) -- cycle;
\fill[blue!21.2, opacity=0.7] (1.4540, 2.9580, 3.1386) -- (1.5000, 2.9580, 3.1388) -- (1.5000, 3.0120, 3.1325) -- (1.4540, 3.0120, 3.1324) -- cycle;
\fill[blue!27.5, opacity=0.7] (1.4540, 3.0120, 3.1324) -- (1.5000, 3.0120, 3.1325) -- (1.5000, 3.0660, 3.1263) -- (1.4540, 3.0660, 3.1261) -- cycle;
\fill[blue!46.8, opacity=0.7] (1.4540, 3.0660, 3.1261) -- (1.5000, 3.0660, 3.1263) -- (1.5000, 3.1200, 3.1200) -- (1.4540, 3.1200, 3.1198) -- cycle;
\fill[blue!46.8, opacity=0.7] (1.5000, -0.1200, 3.1200) -- (1.5460, -0.1200, 3.1198) -- (1.5460, -0.0660, 3.1261) -- (1.5000, -0.0660, 3.1263) -- cycle;
\fill[blue!27.5, opacity=0.7] (1.5000, -0.0660, 3.1263) -- (1.5460, -0.0660, 3.1261) -- (1.5460, -0.0120, 3.1324) -- (1.5000, -0.0120, 3.1325) -- cycle;
\fill[blue!21.2, opacity=0.7] (1.5000, -0.0120, 3.1325) -- (1.5460, -0.0120, 3.1324) -- (1.5460, 0.0420, 3.1386) -- (1.5000, 0.0420, 3.1388) -- cycle;
\fill[blue!24.6, opacity=0.7] (1.5000, 0.0420, 3.1388) -- (1.5460, 0.0420, 3.1386) -- (1.5460, 0.0960, 3.1448) -- (1.5000, 0.0960, 3.1449) -- cycle;
\fill[blue!43.9, opacity=0.7] (1.5000, 0.0960, 3.1449) -- (1.5460, 0.0960, 3.1448) -- (1.5460, 0.1500, 3.1509) -- (1.5000, 0.1500, 3.1511) -- cycle;
\fill[blue!63.5, opacity=0.7] (1.5000, 0.1500, 3.1511) -- (1.5460, 0.1500, 3.1509) -- (1.5460, 0.2040, 3.1569) -- (1.5000, 0.2040, 3.1571) -- cycle;
\fill[blue!53.7, opacity=0.7] (1.5000, 0.2040, 3.1571) -- (1.5460, 0.2040, 3.1569) -- (1.5460, 0.2580, 3.1628) -- (1.5000, 0.2580, 3.1630) -- cycle;
\fill[blue!50.0, opacity=0.7] (1.5000, 0.2580, 3.1630) -- (1.5460, 0.2580, 3.1628) -- (1.5460, 0.3120, 3.1686) -- (1.5000, 0.3120, 3.1688) -- cycle;
\fill[blue!60.6, opacity=0.7] (1.5000, 0.3120, 3.1688) -- (1.5460, 0.3120, 3.1686) -- (1.5460, 0.3660, 3.1743) -- (1.5000, 0.3660, 3.1745) -- cycle;
\fill[blue!60.1, opacity=0.7] (1.5000, 0.3660, 3.1745) -- (1.5460, 0.3660, 3.1743) -- (1.5460, 0.4200, 3.1798) -- (1.5000, 0.4200, 3.1800) -- cycle;
\fill[blue!45.9, opacity=0.7] (1.5000, 0.4200, 3.1800) -- (1.5460, 0.4200, 3.1798) -- (1.5460, 0.4740, 3.1852) -- (1.5000, 0.4740, 3.1854) -- cycle;
\fill[blue!45.1, opacity=0.7] (1.5000, 0.4740, 3.1854) -- (1.5460, 0.4740, 3.1852) -- (1.5460, 0.5280, 3.1904) -- (1.5000, 0.5280, 3.1905) -- cycle;
\fill[blue!60.0, opacity=0.7] (1.5000, 0.5280, 3.1905) -- (1.5460, 0.5280, 3.1904) -- (1.5460, 0.5820, 3.1954) -- (1.5000, 0.5820, 3.1955) -- cycle;
\fill[blue!56.6, opacity=0.7] (1.5000, 0.5820, 3.1955) -- (1.5460, 0.5820, 3.1954) -- (1.5460, 0.6360, 3.2001) -- (1.5000, 0.6360, 3.2003) -- cycle;
\fill[blue!32.5, opacity=0.7] (1.5000, 0.6360, 3.2003) -- (1.5460, 0.6360, 3.2001) -- (1.5460, 0.6900, 3.2047) -- (1.5000, 0.6900, 3.2049) -- cycle;
\fill[blue!24.2, opacity=0.7] (1.5000, 0.6900, 3.2049) -- (1.5460, 0.6900, 3.2047) -- (1.5460, 0.7440, 3.2090) -- (1.5000, 0.7440, 3.2092) -- cycle;
\fill[blue!28.6, opacity=0.7] (1.5000, 0.7440, 3.2092) -- (1.5460, 0.7440, 3.2090) -- (1.5460, 0.7980, 3.2131) -- (1.5000, 0.7980, 3.2133) -- cycle;
\fill[blue!46.6, opacity=0.7] (1.5000, 0.7980, 3.2133) -- (1.5460, 0.7980, 3.2131) -- (1.5460, 0.8520, 3.2169) -- (1.5000, 0.8520, 3.2171) -- cycle;
\fill[blue!62.2, opacity=0.7] (1.5000, 0.8520, 3.2171) -- (1.5460, 0.8520, 3.2169) -- (1.5460, 0.9060, 3.2205) -- (1.5000, 0.9060, 3.2206) -- cycle;
\fill[blue!63.5, opacity=0.7] (1.5000, 0.9060, 3.2206) -- (1.5460, 0.9060, 3.2205) -- (1.5460, 0.9600, 3.2238) -- (1.5000, 0.9600, 3.2239) -- cycle;
\fill[blue!61.8, opacity=0.7] (1.5000, 0.9600, 3.2239) -- (1.5460, 0.9600, 3.2238) -- (1.5460, 1.0140, 3.2268) -- (1.5000, 1.0140, 3.2269) -- cycle;
\fill[blue!41.6, opacity=0.7] (1.5000, 1.0140, 3.2269) -- (1.5460, 1.0140, 3.2268) -- (1.5460, 1.0680, 3.2295) -- (1.5000, 1.0680, 3.2296) -- cycle;
\fill[blue!21.0, opacity=0.7] (1.5000, 1.0680, 3.2296) -- (1.5460, 1.0680, 3.2295) -- (1.5460, 1.1220, 3.2319) -- (1.5000, 1.1220, 3.2320) -- cycle;
\fill[blue!16.3, opacity=0.7] (1.5000, 1.1220, 3.2320) -- (1.5460, 1.1220, 3.2319) -- (1.5460, 1.1760, 3.2340) -- (1.5000, 1.1760, 3.2341) -- cycle;
\fill[blue!16.0, opacity=0.7] (1.5000, 1.1760, 3.2341) -- (1.5460, 1.1760, 3.2340) -- (1.5460, 1.2300, 3.2357) -- (1.5000, 1.2300, 3.2359) -- cycle;
\fill[blue!18.0, opacity=0.7] (1.5000, 1.2300, 3.2359) -- (1.5460, 1.2300, 3.2357) -- (1.5460, 1.2840, 3.2372) -- (1.5000, 1.2840, 3.2374) -- cycle;
\fill[blue!26.5, opacity=0.7] (1.5000, 1.2840, 3.2374) -- (1.5460, 1.2840, 3.2372) -- (1.5460, 1.3380, 3.2384) -- (1.5000, 1.3380, 3.2385) -- cycle;
\fill[blue!39.1, opacity=0.7] (1.5000, 1.3380, 3.2385) -- (1.5460, 1.3380, 3.2384) -- (1.5460, 1.3920, 3.2392) -- (1.5000, 1.3920, 3.2393) -- cycle;
\fill[blue!43.5, opacity=0.7] (1.5000, 1.3920, 3.2393) -- (1.5460, 1.3920, 3.2392) -- (1.5460, 1.4460, 3.2397) -- (1.5000, 1.4460, 3.2398) -- cycle;
\fill[blue!37.0, opacity=0.7] (1.5000, 1.4460, 3.2398) -- (1.5460, 1.4460, 3.2397) -- (1.5460, 1.5000, 3.2398) -- (1.5000, 1.5000, 3.2400) -- cycle;
\fill[blue!15.2, opacity=0.7] (1.5000, 1.5000, 3.2400) -- (1.5460, 1.5000, 3.2398) -- (1.5460, 1.5540, 3.2397) -- (1.5000, 1.5540, 3.2398) -- cycle;
\fill[blue!29.3, opacity=0.7] (1.5000, 1.5540, 3.2398) -- (1.5460, 1.5540, 3.2397) -- (1.5460, 1.6080, 3.2392) -- (1.5000, 1.6080, 3.2393) -- cycle;
\fill[blue!48.0, opacity=0.7] (1.5000, 1.6080, 3.2393) -- (1.5460, 1.6080, 3.2392) -- (1.5460, 1.6620, 3.2384) -- (1.5000, 1.6620, 3.2385) -- cycle;
\fill[blue!41.0, opacity=0.7] (1.5000, 1.6620, 3.2385) -- (1.5460, 1.6620, 3.2384) -- (1.5460, 1.7160, 3.2372) -- (1.5000, 1.7160, 3.2374) -- cycle;
\fill[blue!23.6, opacity=0.7] (1.5000, 1.7160, 3.2374) -- (1.5460, 1.7160, 3.2372) -- (1.5460, 1.7700, 3.2357) -- (1.5000, 1.7700, 3.2359) -- cycle;
\fill[blue!16.8, opacity=0.7] (1.5000, 1.7700, 3.2359) -- (1.5460, 1.7700, 3.2357) -- (1.5460, 1.8240, 3.2340) -- (1.5000, 1.8240, 3.2341) -- cycle;
\fill[blue!16.1, opacity=0.7] (1.5000, 1.8240, 3.2341) -- (1.5460, 1.8240, 3.2340) -- (1.5460, 1.8780, 3.2319) -- (1.5000, 1.8780, 3.2320) -- cycle;
\fill[blue!18.5, opacity=0.7] (1.5000, 1.8780, 3.2320) -- (1.5460, 1.8780, 3.2319) -- (1.5460, 1.9320, 3.2295) -- (1.5000, 1.9320, 3.2296) -- cycle;
\fill[blue!34.0, opacity=0.7] (1.5000, 1.9320, 3.2296) -- (1.5460, 1.9320, 3.2295) -- (1.5460, 1.9860, 3.2268) -- (1.5000, 1.9860, 3.2269) -- cycle;
\fill[blue!59.2, opacity=0.7] (1.5000, 1.9860, 3.2269) -- (1.5460, 1.9860, 3.2268) -- (1.5460, 2.0400, 3.2238) -- (1.5000, 2.0400, 3.2239) -- cycle;
\fill[blue!63.5, opacity=0.7] (1.5000, 2.0400, 3.2239) -- (1.5460, 2.0400, 3.2238) -- (1.5460, 2.0940, 3.2205) -- (1.5000, 2.0940, 3.2206) -- cycle;
\fill[blue!63.4, opacity=0.7] (1.5000, 2.0940, 3.2206) -- (1.5460, 2.0940, 3.2205) -- (1.5460, 2.1480, 3.2169) -- (1.5000, 2.1480, 3.2171) -- cycle;
\fill[blue!52.0, opacity=0.7] (1.5000, 2.1480, 3.2171) -- (1.5460, 2.1480, 3.2169) -- (1.5460, 2.2020, 3.2131) -- (1.5000, 2.2020, 3.2133) -- cycle;
\fill[blue!31.7, opacity=0.7] (1.5000, 2.2020, 3.2133) -- (1.5460, 2.2020, 3.2131) -- (1.5460, 2.2560, 3.2090) -- (1.5000, 2.2560, 3.2092) -- cycle;
\fill[blue!24.5, opacity=0.7] (1.5000, 2.2560, 3.2092) -- (1.5460, 2.2560, 3.2090) -- (1.5460, 2.3100, 3.2047) -- (1.5000, 2.3100, 3.2049) -- cycle;
\fill[blue!30.3, opacity=0.7] (1.5000, 2.3100, 3.2049) -- (1.5460, 2.3100, 3.2047) -- (1.5460, 2.3640, 3.2001) -- (1.5000, 2.3640, 3.2003) -- cycle;
\fill[blue!53.0, opacity=0.7] (1.5000, 2.3640, 3.2003) -- (1.5460, 2.3640, 3.2001) -- (1.5460, 2.4180, 3.1954) -- (1.5000, 2.4180, 3.1955) -- cycle;
\fill[blue!61.7, opacity=0.7] (1.5000, 2.4180, 3.1955) -- (1.5460, 2.4180, 3.1954) -- (1.5460, 2.4720, 3.1904) -- (1.5000, 2.4720, 3.1905) -- cycle;
\fill[blue!46.1, opacity=0.7] (1.5000, 2.4720, 3.1905) -- (1.5460, 2.4720, 3.1904) -- (1.5460, 2.5260, 3.1852) -- (1.5000, 2.5260, 3.1854) -- cycle;
\fill[blue!44.2, opacity=0.7] (1.5000, 2.5260, 3.1854) -- (1.5460, 2.5260, 3.1852) -- (1.5460, 2.5800, 3.1798) -- (1.5000, 2.5800, 3.1800) -- cycle;
\fill[blue!58.0, opacity=0.7] (1.5000, 2.5800, 3.1800) -- (1.5460, 2.5800, 3.1798) -- (1.5460, 2.6340, 3.1743) -- (1.5000, 2.6340, 3.1745) -- cycle;
\fill[blue!62.0, opacity=0.7] (1.5000, 2.6340, 3.1745) -- (1.5460, 2.6340, 3.1743) -- (1.5460, 2.6880, 3.1686) -- (1.5000, 2.6880, 3.1688) -- cycle;
\fill[blue!51.1, opacity=0.7] (1.5000, 2.6880, 3.1688) -- (1.5460, 2.6880, 3.1686) -- (1.5460, 2.7420, 3.1628) -- (1.5000, 2.7420, 3.1630) -- cycle;
\fill[blue!52.7, opacity=0.7] (1.5000, 2.7420, 3.1630) -- (1.5460, 2.7420, 3.1628) -- (1.5460, 2.7960, 3.1569) -- (1.5000, 2.7960, 3.1571) -- cycle;
\fill[blue!63.5, opacity=0.7] (1.5000, 2.7960, 3.1571) -- (1.5460, 2.7960, 3.1569) -- (1.5460, 2.8500, 3.1509) -- (1.5000, 2.8500, 3.1511) -- cycle;
\fill[blue!46.6, opacity=0.7] (1.5000, 2.8500, 3.1511) -- (1.5460, 2.8500, 3.1509) -- (1.5460, 2.9040, 3.1448) -- (1.5000, 2.9040, 3.1449) -- cycle;
\fill[blue!25.5, opacity=0.7] (1.5000, 2.9040, 3.1449) -- (1.5460, 2.9040, 3.1448) -- (1.5460, 2.9580, 3.1386) -- (1.5000, 2.9580, 3.1388) -- cycle;
\fill[blue!21.0, opacity=0.7] (1.5000, 2.9580, 3.1388) -- (1.5460, 2.9580, 3.1386) -- (1.5460, 3.0120, 3.1324) -- (1.5000, 3.0120, 3.1325) -- cycle;
\fill[blue!26.2, opacity=0.7] (1.5000, 3.0120, 3.1325) -- (1.5460, 3.0120, 3.1324) -- (1.5460, 3.0660, 3.1261) -- (1.5000, 3.0660, 3.1263) -- cycle;
\fill[blue!44.5, opacity=0.7] (1.5000, 3.0660, 3.1263) -- (1.5460, 3.0660, 3.1261) -- (1.5460, 3.1200, 3.1198) -- (1.5000, 3.1200, 3.1200) -- cycle;
\fill[blue!49.8, opacity=0.7] (1.5460, -0.1200, 3.1198) -- (1.5920, -0.1200, 3.1193) -- (1.5920, -0.0660, 3.1256) -- (1.5460, -0.0660, 3.1261) -- cycle;
\fill[blue!29.6, opacity=0.7] (1.5460, -0.0660, 3.1261) -- (1.5920, -0.0660, 3.1256) -- (1.5920, -0.0120, 3.1319) -- (1.5460, -0.0120, 3.1324) -- cycle;
\fill[blue!21.6, opacity=0.7] (1.5460, -0.0120, 3.1324) -- (1.5920, -0.0120, 3.1319) -- (1.5920, 0.0420, 3.1381) -- (1.5460, 0.0420, 3.1386) -- cycle;
\fill[blue!23.7, opacity=0.7] (1.5460, 0.0420, 3.1386) -- (1.5920, 0.0420, 3.1381) -- (1.5920, 0.0960, 3.1443) -- (1.5460, 0.0960, 3.1448) -- cycle;
\fill[blue!40.4, opacity=0.7] (1.5460, 0.0960, 3.1448) -- (1.5920, 0.0960, 3.1443) -- (1.5920, 0.1500, 3.1504) -- (1.5460, 0.1500, 3.1509) -- cycle;
\fill[blue!62.9, opacity=0.7] (1.5460, 0.1500, 3.1509) -- (1.5920, 0.1500, 3.1504) -- (1.5920, 0.2040, 3.1564) -- (1.5460, 0.2040, 3.1569) -- cycle;
\fill[blue!55.3, opacity=0.7] (1.5460, 0.2040, 3.1569) -- (1.5920, 0.2040, 3.1564) -- (1.5920, 0.2580, 3.1623) -- (1.5460, 0.2580, 3.1628) -- cycle;
\fill[blue!49.0, opacity=0.7] (1.5460, 0.2580, 3.1628) -- (1.5920, 0.2580, 3.1623) -- (1.5920, 0.3120, 3.1682) -- (1.5460, 0.3120, 3.1686) -- cycle;
\fill[blue!58.3, opacity=0.7] (1.5460, 0.3120, 3.1686) -- (1.5920, 0.3120, 3.1682) -- (1.5920, 0.3660, 3.1738) -- (1.5460, 0.3660, 3.1743) -- cycle;
\fill[blue!62.2, opacity=0.7] (1.5460, 0.3660, 3.1743) -- (1.5920, 0.3660, 3.1738) -- (1.5920, 0.4200, 3.1793) -- (1.5460, 0.4200, 3.1798) -- cycle;
\fill[blue!48.5, opacity=0.7] (1.5460, 0.4200, 3.1798) -- (1.5920, 0.4200, 3.1793) -- (1.5920, 0.4740, 3.1847) -- (1.5460, 0.4740, 3.1852) -- cycle;
\fill[blue!44.3, opacity=0.7] (1.5460, 0.4740, 3.1852) -- (1.5920, 0.4740, 3.1847) -- (1.5920, 0.5280, 3.1899) -- (1.5460, 0.5280, 3.1904) -- cycle;
\fill[blue!57.0, opacity=0.7] (1.5460, 0.5280, 3.1904) -- (1.5920, 0.5280, 3.1899) -- (1.5920, 0.5820, 3.1949) -- (1.5460, 0.5820, 3.1954) -- cycle;
\fill[blue!60.6, opacity=0.7] (1.5460, 0.5820, 3.1954) -- (1.5920, 0.5820, 3.1949) -- (1.5920, 0.6360, 3.1996) -- (1.5460, 0.6360, 3.2001) -- cycle;
\fill[blue!36.5, opacity=0.7] (1.5460, 0.6360, 3.2001) -- (1.5920, 0.6360, 3.1996) -- (1.5920, 0.6900, 3.2042) -- (1.5460, 0.6900, 3.2047) -- cycle;
\fill[blue!24.5, opacity=0.7] (1.5460, 0.6900, 3.2047) -- (1.5920, 0.6900, 3.2042) -- (1.5920, 0.7440, 3.2085) -- (1.5460, 0.7440, 3.2090) -- cycle;
\fill[blue!25.7, opacity=0.7] (1.5460, 0.7440, 3.2090) -- (1.5920, 0.7440, 3.2085) -- (1.5920, 0.7980, 3.2126) -- (1.5460, 0.7980, 3.2131) -- cycle;
\fill[blue!39.4, opacity=0.7] (1.5460, 0.7980, 3.2131) -- (1.5920, 0.7980, 3.2126) -- (1.5920, 0.8520, 3.2164) -- (1.5460, 0.8520, 3.2169) -- cycle;
\fill[blue!58.7, opacity=0.7] (1.5460, 0.8520, 3.2169) -- (1.5920, 0.8520, 3.2164) -- (1.5920, 0.9060, 3.2200) -- (1.5460, 0.9060, 3.2205) -- cycle;
\fill[blue!63.6, opacity=0.7] (1.5460, 0.9060, 3.2205) -- (1.5920, 0.9060, 3.2200) -- (1.5920, 0.9600, 3.2233) -- (1.5460, 0.9600, 3.2238) -- cycle;
\fill[blue!63.3, opacity=0.7] (1.5460, 0.9600, 3.2238) -- (1.5920, 0.9600, 3.2233) -- (1.5920, 1.0140, 3.2263) -- (1.5460, 1.0140, 3.2268) -- cycle;
\fill[blue!52.1, opacity=0.7] (1.5460, 1.0140, 3.2268) -- (1.5920, 1.0140, 3.2263) -- (1.5920, 1.0680, 3.2290) -- (1.5460, 1.0680, 3.2295) -- cycle;
\fill[blue!27.8, opacity=0.7] (1.5460, 1.0680, 3.2295) -- (1.5920, 1.0680, 3.2290) -- (1.5920, 1.1220, 3.2314) -- (1.5460, 1.1220, 3.2319) -- cycle;
\fill[blue!17.5, opacity=0.7] (1.5460, 1.1220, 3.2319) -- (1.5920, 1.1220, 3.2314) -- (1.5920, 1.1760, 3.2335) -- (1.5460, 1.1760, 3.2340) -- cycle;
\fill[blue!15.8, opacity=0.7] (1.5460, 1.1760, 3.2340) -- (1.5920, 1.1760, 3.2335) -- (1.5920, 1.2300, 3.2353) -- (1.5460, 1.2300, 3.2357) -- cycle;
\fill[blue!15.9, opacity=0.7] (1.5460, 1.2300, 3.2357) -- (1.5920, 1.2300, 3.2353) -- (1.5920, 1.2840, 3.2367) -- (1.5460, 1.2840, 3.2372) -- cycle;
\fill[blue!17.1, opacity=0.7] (1.5460, 1.2840, 3.2372) -- (1.5920, 1.2840, 3.2367) -- (1.5920, 1.3380, 3.2379) -- (1.5460, 1.3380, 3.2384) -- cycle;
\fill[blue!19.7, opacity=0.7] (1.5460, 1.3380, 3.2384) -- (1.5920, 1.3380, 3.2379) -- (1.5920, 1.3920, 3.2387) -- (1.5460, 1.3920, 3.2392) -- cycle;
\fill[blue!20.0, opacity=0.7] (1.5460, 1.3920, 3.2392) -- (1.5920, 1.3920, 3.2387) -- (1.5920, 1.4460, 3.2392) -- (1.5460, 1.4460, 3.2397) -- cycle;
\fill[blue!16.0, opacity=0.7] (1.5460, 1.4460, 3.2397) -- (1.5920, 1.4460, 3.2392) -- (1.5920, 1.5000, 3.2393) -- (1.5460, 1.5000, 3.2398) -- cycle;
\fill[blue!15.2, opacity=0.7] (1.5460, 1.5000, 3.2398) -- (1.5920, 1.5000, 3.2393) -- (1.5920, 1.5540, 3.2392) -- (1.5460, 1.5540, 3.2397) -- cycle;
\fill[blue!20.5, opacity=0.7] (1.5460, 1.5540, 3.2397) -- (1.5920, 1.5540, 3.2392) -- (1.5920, 1.6080, 3.2387) -- (1.5460, 1.6080, 3.2392) -- cycle;
\fill[blue!45.6, opacity=0.7] (1.5460, 1.6080, 3.2392) -- (1.5920, 1.6080, 3.2387) -- (1.5920, 1.6620, 3.2379) -- (1.5460, 1.6620, 3.2384) -- cycle;
\fill[blue!48.5, opacity=0.7] (1.5460, 1.6620, 3.2384) -- (1.5920, 1.6620, 3.2379) -- (1.5920, 1.7160, 3.2367) -- (1.5460, 1.7160, 3.2372) -- cycle;
\fill[blue!30.0, opacity=0.7] (1.5460, 1.7160, 3.2372) -- (1.5920, 1.7160, 3.2367) -- (1.5920, 1.7700, 3.2353) -- (1.5460, 1.7700, 3.2357) -- cycle;
\fill[blue!18.1, opacity=0.7] (1.5460, 1.7700, 3.2357) -- (1.5920, 1.7700, 3.2353) -- (1.5920, 1.8240, 3.2335) -- (1.5460, 1.8240, 3.2340) -- cycle;
\fill[blue!16.2, opacity=0.7] (1.5460, 1.8240, 3.2340) -- (1.5920, 1.8240, 3.2335) -- (1.5920, 1.8780, 3.2314) -- (1.5460, 1.8780, 3.2319) -- cycle;
\fill[blue!17.7, opacity=0.7] (1.5460, 1.8780, 3.2319) -- (1.5920, 1.8780, 3.2314) -- (1.5920, 1.9320, 3.2290) -- (1.5460, 1.9320, 3.2295) -- cycle;
\fill[blue!30.2, opacity=0.7] (1.5460, 1.9320, 3.2295) -- (1.5920, 1.9320, 3.2290) -- (1.5920, 1.9860, 3.2263) -- (1.5460, 1.9860, 3.2268) -- cycle;
\fill[blue!56.9, opacity=0.7] (1.5460, 1.9860, 3.2268) -- (1.5920, 1.9860, 3.2263) -- (1.5920, 2.0400, 3.2233) -- (1.5460, 2.0400, 3.2238) -- cycle;
\fill[blue!63.5, opacity=0.7] (1.5460, 2.0400, 3.2238) -- (1.5920, 2.0400, 3.2233) -- (1.5920, 2.0940, 3.2200) -- (1.5460, 2.0940, 3.2205) -- cycle;
\fill[blue!63.6, opacity=0.7] (1.5460, 2.0940, 3.2205) -- (1.5920, 2.0940, 3.2200) -- (1.5920, 2.1480, 3.2164) -- (1.5460, 2.1480, 3.2169) -- cycle;
\fill[blue!55.2, opacity=0.7] (1.5460, 2.1480, 3.2169) -- (1.5920, 2.1480, 3.2164) -- (1.5920, 2.2020, 3.2126) -- (1.5460, 2.2020, 3.2131) -- cycle;
\fill[blue!34.2, opacity=0.7] (1.5460, 2.2020, 3.2131) -- (1.5920, 2.2020, 3.2126) -- (1.5920, 2.2560, 3.2085) -- (1.5460, 2.2560, 3.2090) -- cycle;
\fill[blue!25.0, opacity=0.7] (1.5460, 2.2560, 3.2090) -- (1.5920, 2.2560, 3.2085) -- (1.5920, 2.3100, 3.2042) -- (1.5460, 2.3100, 3.2047) -- cycle;
\fill[blue!29.4, opacity=0.7] (1.5460, 2.3100, 3.2047) -- (1.5920, 2.3100, 3.2042) -- (1.5920, 2.3640, 3.1996) -- (1.5460, 2.3640, 3.2001) -- cycle;
\fill[blue!50.9, opacity=0.7] (1.5460, 2.3640, 3.2001) -- (1.5920, 2.3640, 3.1996) -- (1.5920, 2.4180, 3.1949) -- (1.5460, 2.4180, 3.1954) -- cycle;
\fill[blue!62.5, opacity=0.7] (1.5460, 2.4180, 3.1954) -- (1.5920, 2.4180, 3.1949) -- (1.5920, 2.4720, 3.1899) -- (1.5460, 2.4720, 3.1904) -- cycle;
\fill[blue!46.7, opacity=0.7] (1.5460, 2.4720, 3.1904) -- (1.5920, 2.4720, 3.1899) -- (1.5920, 2.5260, 3.1847) -- (1.5460, 2.5260, 3.1852) -- cycle;
\fill[blue!43.2, opacity=0.7] (1.5460, 2.5260, 3.1852) -- (1.5920, 2.5260, 3.1847) -- (1.5920, 2.5800, 3.1793) -- (1.5460, 2.5800, 3.1798) -- cycle;
\fill[blue!56.4, opacity=0.7] (1.5460, 2.5800, 3.1798) -- (1.5920, 2.5800, 3.1793) -- (1.5920, 2.6340, 3.1738) -- (1.5460, 2.6340, 3.1743) -- cycle;
\fill[blue!62.8, opacity=0.7] (1.5460, 2.6340, 3.1743) -- (1.5920, 2.6340, 3.1738) -- (1.5920, 2.6880, 3.1682) -- (1.5460, 2.6880, 3.1686) -- cycle;
\fill[blue!52.1, opacity=0.7] (1.5460, 2.6880, 3.1686) -- (1.5920, 2.6880, 3.1682) -- (1.5920, 2.7420, 3.1623) -- (1.5460, 2.7420, 3.1628) -- cycle;
\fill[blue!52.4, opacity=0.7] (1.5460, 2.7420, 3.1628) -- (1.5920, 2.7420, 3.1623) -- (1.5920, 2.7960, 3.1564) -- (1.5460, 2.7960, 3.1569) -- cycle;
\fill[blue!63.4, opacity=0.7] (1.5460, 2.7960, 3.1569) -- (1.5920, 2.7960, 3.1564) -- (1.5920, 2.8500, 3.1504) -- (1.5460, 2.8500, 3.1509) -- cycle;
\fill[blue!48.1, opacity=0.7] (1.5460, 2.8500, 3.1509) -- (1.5920, 2.8500, 3.1504) -- (1.5920, 2.9040, 3.1443) -- (1.5460, 2.9040, 3.1448) -- cycle;
\fill[blue!26.0, opacity=0.7] (1.5460, 2.9040, 3.1448) -- (1.5920, 2.9040, 3.1443) -- (1.5920, 2.9580, 3.1381) -- (1.5460, 2.9580, 3.1386) -- cycle;
\fill[blue!20.8, opacity=0.7] (1.5460, 2.9580, 3.1386) -- (1.5920, 2.9580, 3.1381) -- (1.5920, 3.0120, 3.1319) -- (1.5460, 3.0120, 3.1324) -- cycle;
\fill[blue!25.4, opacity=0.7] (1.5460, 3.0120, 3.1324) -- (1.5920, 3.0120, 3.1319) -- (1.5920, 3.0660, 3.1256) -- (1.5460, 3.0660, 3.1261) -- cycle;
\fill[blue!42.9, opacity=0.7] (1.5460, 3.0660, 3.1261) -- (1.5920, 3.0660, 3.1256) -- (1.5920, 3.1200, 3.1193) -- (1.5460, 3.1200, 3.1198) -- cycle;
\fill[blue!53.2, opacity=0.7] (1.5920, -0.1200, 3.1193) -- (1.6380, -0.1200, 3.1185) -- (1.6380, -0.0660, 3.1248) -- (1.5920, -0.0660, 3.1256) -- cycle;
\fill[blue!32.5, opacity=0.7] (1.5920, -0.0660, 3.1256) -- (1.6380, -0.0660, 3.1248) -- (1.6380, -0.0120, 3.1311) -- (1.5920, -0.0120, 3.1319) -- cycle;
\fill[blue!22.3, opacity=0.7] (1.5920, -0.0120, 3.1319) -- (1.6380, -0.0120, 3.1311) -- (1.6380, 0.0420, 3.1373) -- (1.5920, 0.0420, 3.1381) -- cycle;
\fill[blue!22.7, opacity=0.7] (1.5920, 0.0420, 3.1381) -- (1.6380, 0.0420, 3.1373) -- (1.6380, 0.0960, 3.1435) -- (1.5920, 0.0960, 3.1443) -- cycle;
\fill[blue!36.1, opacity=0.7] (1.5920, 0.0960, 3.1443) -- (1.6380, 0.0960, 3.1435) -- (1.6380, 0.1500, 3.1496) -- (1.5920, 0.1500, 3.1504) -- cycle;
\fill[blue!60.9, opacity=0.7] (1.5920, 0.1500, 3.1504) -- (1.6380, 0.1500, 3.1496) -- (1.6380, 0.2040, 3.1556) -- (1.5920, 0.2040, 3.1564) -- cycle;
\fill[blue!57.8, opacity=0.7] (1.5920, 0.2040, 3.1564) -- (1.6380, 0.2040, 3.1556) -- (1.6380, 0.2580, 3.1615) -- (1.5920, 0.2580, 3.1623) -- cycle;
\fill[blue!48.4, opacity=0.7] (1.5920, 0.2580, 3.1623) -- (1.6380, 0.2580, 3.1615) -- (1.6380, 0.3120, 3.1673) -- (1.5920, 0.3120, 3.1682) -- cycle;
\fill[blue!55.2, opacity=0.7] (1.5920, 0.3120, 3.1682) -- (1.6380, 0.3120, 3.1673) -- (1.6380, 0.3660, 3.1730) -- (1.5920, 0.3660, 3.1738) -- cycle;
\fill[blue!63.5, opacity=0.7] (1.5920, 0.3660, 3.1738) -- (1.6380, 0.3660, 3.1730) -- (1.6380, 0.4200, 3.1785) -- (1.5920, 0.4200, 3.1793) -- cycle;
\fill[blue!52.3, opacity=0.7] (1.5920, 0.4200, 3.1793) -- (1.6380, 0.4200, 3.1785) -- (1.6380, 0.4740, 3.1839) -- (1.5920, 0.4740, 3.1847) -- cycle;
\fill[blue!44.1, opacity=0.7] (1.5920, 0.4740, 3.1847) -- (1.6380, 0.4740, 3.1839) -- (1.6380, 0.5280, 3.1891) -- (1.5920, 0.5280, 3.1899) -- cycle;
\fill[blue!53.0, opacity=0.7] (1.5920, 0.5280, 3.1899) -- (1.6380, 0.5280, 3.1891) -- (1.6380, 0.5820, 3.1940) -- (1.5920, 0.5820, 3.1949) -- cycle;
\fill[blue!63.3, opacity=0.7] (1.5920, 0.5820, 3.1949) -- (1.6380, 0.5820, 3.1940) -- (1.6380, 0.6360, 3.1988) -- (1.5920, 0.6360, 3.1996) -- cycle;
\fill[blue!43.2, opacity=0.7] (1.5920, 0.6360, 3.1996) -- (1.6380, 0.6360, 3.1988) -- (1.6380, 0.6900, 3.2034) -- (1.5920, 0.6900, 3.2042) -- cycle;
\fill[blue!26.1, opacity=0.7] (1.5920, 0.6900, 3.2042) -- (1.6380, 0.6900, 3.2034) -- (1.6380, 0.7440, 3.2077) -- (1.5920, 0.7440, 3.2085) -- cycle;
\fill[blue!23.7, opacity=0.7] (1.5920, 0.7440, 3.2085) -- (1.6380, 0.7440, 3.2077) -- (1.6380, 0.7980, 3.2118) -- (1.5920, 0.7980, 3.2126) -- cycle;
\fill[blue!31.9, opacity=0.7] (1.5920, 0.7980, 3.2126) -- (1.6380, 0.7980, 3.2118) -- (1.6380, 0.8520, 3.2156) -- (1.5920, 0.8520, 3.2164) -- cycle;
\fill[blue!51.1, opacity=0.7] (1.5920, 0.8520, 3.2164) -- (1.6380, 0.8520, 3.2156) -- (1.6380, 0.9060, 3.2192) -- (1.5920, 0.9060, 3.2200) -- cycle;
\fill[blue!62.7, opacity=0.7] (1.5920, 0.9060, 3.2200) -- (1.6380, 0.9060, 3.2192) -- (1.6380, 0.9600, 3.2224) -- (1.5920, 0.9600, 3.2233) -- cycle;
\fill[blue!63.6, opacity=0.7] (1.5920, 0.9600, 3.2233) -- (1.6380, 0.9600, 3.2224) -- (1.6380, 1.0140, 3.2254) -- (1.5920, 1.0140, 3.2263) -- cycle;
\fill[blue!60.4, opacity=0.7] (1.5920, 1.0140, 3.2263) -- (1.6380, 1.0140, 3.2254) -- (1.6380, 1.0680, 3.2281) -- (1.5920, 1.0680, 3.2290) -- cycle;
\fill[blue!41.9, opacity=0.7] (1.5920, 1.0680, 3.2290) -- (1.6380, 1.0680, 3.2281) -- (1.6380, 1.1220, 3.2306) -- (1.5920, 1.1220, 3.2314) -- cycle;
\fill[blue!22.8, opacity=0.7] (1.5920, 1.1220, 3.2314) -- (1.6380, 1.1220, 3.2306) -- (1.6380, 1.1760, 3.2326) -- (1.5920, 1.1760, 3.2335) -- cycle;
\fill[blue!16.8, opacity=0.7] (1.5920, 1.1760, 3.2335) -- (1.6380, 1.1760, 3.2326) -- (1.6380, 1.2300, 3.2344) -- (1.5920, 1.2300, 3.2353) -- cycle;
\fill[blue!15.7, opacity=0.7] (1.5920, 1.2300, 3.2353) -- (1.6380, 1.2300, 3.2344) -- (1.6380, 1.2840, 3.2359) -- (1.5920, 1.2840, 3.2367) -- cycle;
\fill[blue!15.6, opacity=0.7] (1.5920, 1.2840, 3.2367) -- (1.6380, 1.2840, 3.2359) -- (1.6380, 1.3380, 3.2370) -- (1.5920, 1.3380, 3.2379) -- cycle;
\fill[blue!15.6, opacity=0.7] (1.5920, 1.3380, 3.2379) -- (1.6380, 1.3380, 3.2370) -- (1.6380, 1.3920, 3.2379) -- (1.5920, 1.3920, 3.2387) -- cycle;
\fill[blue!15.4, opacity=0.7] (1.5920, 1.3920, 3.2387) -- (1.6380, 1.3920, 3.2379) -- (1.6380, 1.4460, 3.2384) -- (1.5920, 1.4460, 3.2392) -- cycle;
\fill[blue!15.3, opacity=0.7] (1.5920, 1.4460, 3.2392) -- (1.6380, 1.4460, 3.2384) -- (1.6380, 1.5000, 3.2385) -- (1.5920, 1.5000, 3.2393) -- cycle;
\fill[blue!15.7, opacity=0.7] (1.5920, 1.5000, 3.2393) -- (1.6380, 1.5000, 3.2385) -- (1.6380, 1.5540, 3.2384) -- (1.5920, 1.5540, 3.2392) -- cycle;
\fill[blue!23.5, opacity=0.7] (1.5920, 1.5540, 3.2392) -- (1.6380, 1.5540, 3.2384) -- (1.6380, 1.6080, 3.2379) -- (1.5920, 1.6080, 3.2387) -- cycle;
\fill[blue!47.4, opacity=0.7] (1.5920, 1.6080, 3.2387) -- (1.6380, 1.6080, 3.2379) -- (1.6380, 1.6620, 3.2370) -- (1.5920, 1.6620, 3.2379) -- cycle;
\fill[blue!51.1, opacity=0.7] (1.5920, 1.6620, 3.2379) -- (1.6380, 1.6620, 3.2370) -- (1.6380, 1.7160, 3.2359) -- (1.5920, 1.7160, 3.2367) -- cycle;
\fill[blue!33.0, opacity=0.7] (1.5920, 1.7160, 3.2367) -- (1.6380, 1.7160, 3.2359) -- (1.6380, 1.7700, 3.2344) -- (1.5920, 1.7700, 3.2353) -- cycle;
\fill[blue!18.9, opacity=0.7] (1.5920, 1.7700, 3.2353) -- (1.6380, 1.7700, 3.2344) -- (1.6380, 1.8240, 3.2326) -- (1.5920, 1.8240, 3.2335) -- cycle;
\fill[blue!16.3, opacity=0.7] (1.5920, 1.8240, 3.2335) -- (1.6380, 1.8240, 3.2326) -- (1.6380, 1.8780, 3.2306) -- (1.5920, 1.8780, 3.2314) -- cycle;
\fill[blue!17.6, opacity=0.7] (1.5920, 1.8780, 3.2314) -- (1.6380, 1.8780, 3.2306) -- (1.6380, 1.9320, 3.2281) -- (1.5920, 1.9320, 3.2290) -- cycle;
\fill[blue!29.4, opacity=0.7] (1.5920, 1.9320, 3.2290) -- (1.6380, 1.9320, 3.2281) -- (1.6380, 1.9860, 3.2254) -- (1.5920, 1.9860, 3.2263) -- cycle;
\fill[blue!56.4, opacity=0.7] (1.5920, 1.9860, 3.2263) -- (1.6380, 1.9860, 3.2254) -- (1.6380, 2.0400, 3.2224) -- (1.5920, 2.0400, 3.2233) -- cycle;
\fill[blue!63.5, opacity=0.7] (1.5920, 2.0400, 3.2233) -- (1.6380, 2.0400, 3.2224) -- (1.6380, 2.0940, 3.2192) -- (1.5920, 2.0940, 3.2200) -- cycle;
\fill[blue!63.5, opacity=0.7] (1.5920, 2.0940, 3.2200) -- (1.6380, 2.0940, 3.2192) -- (1.6380, 2.1480, 3.2156) -- (1.5920, 2.1480, 3.2164) -- cycle;
\fill[blue!56.7, opacity=0.7] (1.5920, 2.1480, 3.2164) -- (1.6380, 2.1480, 3.2156) -- (1.6380, 2.2020, 3.2118) -- (1.5920, 2.2020, 3.2126) -- cycle;
\fill[blue!35.6, opacity=0.7] (1.5920, 2.2020, 3.2126) -- (1.6380, 2.2020, 3.2118) -- (1.6380, 2.2560, 3.2077) -- (1.5920, 2.2560, 3.2085) -- cycle;
\fill[blue!25.5, opacity=0.7] (1.5920, 2.2560, 3.2085) -- (1.6380, 2.2560, 3.2077) -- (1.6380, 2.3100, 3.2034) -- (1.5920, 2.3100, 3.2042) -- cycle;
\fill[blue!29.3, opacity=0.7] (1.5920, 2.3100, 3.2042) -- (1.6380, 2.3100, 3.2034) -- (1.6380, 2.3640, 3.1988) -- (1.5920, 2.3640, 3.1996) -- cycle;
\fill[blue!50.4, opacity=0.7] (1.5920, 2.3640, 3.1996) -- (1.6380, 2.3640, 3.1988) -- (1.6380, 2.4180, 3.1940) -- (1.5920, 2.4180, 3.1949) -- cycle;
\fill[blue!62.7, opacity=0.7] (1.5920, 2.4180, 3.1949) -- (1.6380, 2.4180, 3.1940) -- (1.6380, 2.4720, 3.1891) -- (1.5920, 2.4720, 3.1899) -- cycle;
\fill[blue!46.6, opacity=0.7] (1.5920, 2.4720, 3.1899) -- (1.6380, 2.4720, 3.1891) -- (1.6380, 2.5260, 3.1839) -- (1.5920, 2.5260, 3.1847) -- cycle;
\fill[blue!42.5, opacity=0.7] (1.5920, 2.5260, 3.1847) -- (1.6380, 2.5260, 3.1839) -- (1.6380, 2.5800, 3.1785) -- (1.5920, 2.5800, 3.1793) -- cycle;
\fill[blue!55.5, opacity=0.7] (1.5920, 2.5800, 3.1793) -- (1.6380, 2.5800, 3.1785) -- (1.6380, 2.6340, 3.1730) -- (1.5920, 2.6340, 3.1738) -- cycle;
\fill[blue!63.0, opacity=0.7] (1.5920, 2.6340, 3.1738) -- (1.6380, 2.6340, 3.1730) -- (1.6380, 2.6880, 3.1673) -- (1.5920, 2.6880, 3.1682) -- cycle;
\fill[blue!52.7, opacity=0.7] (1.5920, 2.6880, 3.1682) -- (1.6380, 2.6880, 3.1673) -- (1.6380, 2.7420, 3.1615) -- (1.5920, 2.7420, 3.1623) -- cycle;
\fill[blue!52.5, opacity=0.7] (1.5920, 2.7420, 3.1623) -- (1.6380, 2.7420, 3.1615) -- (1.6380, 2.7960, 3.1556) -- (1.5920, 2.7960, 3.1564) -- cycle;
\fill[blue!63.3, opacity=0.7] (1.5920, 2.7960, 3.1564) -- (1.6380, 2.7960, 3.1556) -- (1.6380, 2.8500, 3.1496) -- (1.5920, 2.8500, 3.1504) -- cycle;
\fill[blue!48.4, opacity=0.7] (1.5920, 2.8500, 3.1504) -- (1.6380, 2.8500, 3.1496) -- (1.6380, 2.9040, 3.1435) -- (1.5920, 2.9040, 3.1443) -- cycle;
\fill[blue!26.0, opacity=0.7] (1.5920, 2.9040, 3.1443) -- (1.6380, 2.9040, 3.1435) -- (1.6380, 2.9580, 3.1373) -- (1.5920, 2.9580, 3.1381) -- cycle;
\fill[blue!20.7, opacity=0.7] (1.5920, 2.9580, 3.1381) -- (1.6380, 2.9580, 3.1373) -- (1.6380, 3.0120, 3.1311) -- (1.5920, 3.0120, 3.1319) -- cycle;
\fill[blue!24.9, opacity=0.7] (1.5920, 3.0120, 3.1319) -- (1.6380, 3.0120, 3.1311) -- (1.6380, 3.0660, 3.1248) -- (1.5920, 3.0660, 3.1256) -- cycle;
\fill[blue!42.1, opacity=0.7] (1.5920, 3.0660, 3.1256) -- (1.6380, 3.0660, 3.1248) -- (1.6380, 3.1200, 3.1185) -- (1.5920, 3.1200, 3.1193) -- cycle;
\fill[blue!56.7, opacity=0.7] (1.6380, -0.1200, 3.1185) -- (1.6840, -0.1200, 3.1174) -- (1.6840, -0.0660, 3.1237) -- (1.6380, -0.0660, 3.1248) -- cycle;
\fill[blue!36.6, opacity=0.7] (1.6380, -0.0660, 3.1248) -- (1.6840, -0.0660, 3.1237) -- (1.6840, -0.0120, 3.1299) -- (1.6380, -0.0120, 3.1311) -- cycle;
\fill[blue!23.5, opacity=0.7] (1.6380, -0.0120, 3.1311) -- (1.6840, -0.0120, 3.1299) -- (1.6840, 0.0420, 3.1361) -- (1.6380, 0.0420, 3.1373) -- cycle;
\fill[blue!22.1, opacity=0.7] (1.6380, 0.0420, 3.1373) -- (1.6840, 0.0420, 3.1361) -- (1.6840, 0.0960, 3.1423) -- (1.6380, 0.0960, 3.1435) -- cycle;
\fill[blue!31.7, opacity=0.7] (1.6380, 0.0960, 3.1435) -- (1.6840, 0.0960, 3.1423) -- (1.6840, 0.1500, 3.1484) -- (1.6380, 0.1500, 3.1496) -- cycle;
\fill[blue!56.6, opacity=0.7] (1.6380, 0.1500, 3.1496) -- (1.6840, 0.1500, 3.1484) -- (1.6840, 0.2040, 3.1545) -- (1.6380, 0.2040, 3.1556) -- cycle;
\fill[blue!60.7, opacity=0.7] (1.6380, 0.2040, 3.1556) -- (1.6840, 0.2040, 3.1545) -- (1.6840, 0.2580, 3.1604) -- (1.6380, 0.2580, 3.1615) -- cycle;
\fill[blue!49.0, opacity=0.7] (1.6380, 0.2580, 3.1615) -- (1.6840, 0.2580, 3.1604) -- (1.6840, 0.3120, 3.1662) -- (1.6380, 0.3120, 3.1673) -- cycle;
\fill[blue!51.7, opacity=0.7] (1.6380, 0.3120, 3.1673) -- (1.6840, 0.3120, 3.1662) -- (1.6840, 0.3660, 3.1719) -- (1.6380, 0.3660, 3.1730) -- cycle;
\fill[blue!62.9, opacity=0.7] (1.6380, 0.3660, 3.1730) -- (1.6840, 0.3660, 3.1719) -- (1.6840, 0.4200, 3.1774) -- (1.6380, 0.4200, 3.1785) -- cycle;
\fill[blue!57.1, opacity=0.7] (1.6380, 0.4200, 3.1785) -- (1.6840, 0.4200, 3.1774) -- (1.6840, 0.4740, 3.1827) -- (1.6380, 0.4740, 3.1839) -- cycle;
\fill[blue!45.5, opacity=0.7] (1.6380, 0.4740, 3.1839) -- (1.6840, 0.4740, 3.1827) -- (1.6840, 0.5280, 3.1879) -- (1.6380, 0.5280, 3.1891) -- cycle;
\fill[blue!48.8, opacity=0.7] (1.6380, 0.5280, 3.1891) -- (1.6840, 0.5280, 3.1879) -- (1.6840, 0.5820, 3.1929) -- (1.6380, 0.5820, 3.1940) -- cycle;
\fill[blue!62.6, opacity=0.7] (1.6380, 0.5820, 3.1940) -- (1.6840, 0.5820, 3.1929) -- (1.6840, 0.6360, 3.1977) -- (1.6380, 0.6360, 3.1988) -- cycle;
\fill[blue!52.4, opacity=0.7] (1.6380, 0.6360, 3.1988) -- (1.6840, 0.6360, 3.1977) -- (1.6840, 0.6900, 3.2022) -- (1.6380, 0.6900, 3.2034) -- cycle;
\fill[blue!30.2, opacity=0.7] (1.6380, 0.6900, 3.2034) -- (1.6840, 0.6900, 3.2022) -- (1.6840, 0.7440, 3.2066) -- (1.6380, 0.7440, 3.2077) -- cycle;
\fill[blue!23.1, opacity=0.7] (1.6380, 0.7440, 3.2077) -- (1.6840, 0.7440, 3.2066) -- (1.6840, 0.7980, 3.2106) -- (1.6380, 0.7980, 3.2118) -- cycle;
\fill[blue!26.1, opacity=0.7] (1.6380, 0.7980, 3.2118) -- (1.6840, 0.7980, 3.2106) -- (1.6840, 0.8520, 3.2145) -- (1.6380, 0.8520, 3.2156) -- cycle;
\fill[blue!40.0, opacity=0.7] (1.6380, 0.8520, 3.2156) -- (1.6840, 0.8520, 3.2145) -- (1.6840, 0.9060, 3.2180) -- (1.6380, 0.9060, 3.2192) -- cycle;
\fill[blue!57.7, opacity=0.7] (1.6380, 0.9060, 3.2192) -- (1.6840, 0.9060, 3.2180) -- (1.6840, 0.9600, 3.2213) -- (1.6380, 0.9600, 3.2224) -- cycle;
\fill[blue!63.3, opacity=0.7] (1.6380, 0.9600, 3.2224) -- (1.6840, 0.9600, 3.2213) -- (1.6840, 1.0140, 3.2243) -- (1.6380, 1.0140, 3.2254) -- cycle;
\fill[blue!63.3, opacity=0.7] (1.6380, 1.0140, 3.2254) -- (1.6840, 1.0140, 3.2243) -- (1.6840, 1.0680, 3.2270) -- (1.6380, 1.0680, 3.2281) -- cycle;
\fill[blue!57.0, opacity=0.7] (1.6380, 1.0680, 3.2281) -- (1.6840, 1.0680, 3.2270) -- (1.6840, 1.1220, 3.2294) -- (1.6380, 1.1220, 3.2306) -- cycle;
\fill[blue!38.5, opacity=0.7] (1.6380, 1.1220, 3.2306) -- (1.6840, 1.1220, 3.2294) -- (1.6840, 1.1760, 3.2315) -- (1.6380, 1.1760, 3.2326) -- cycle;
\fill[blue!23.1, opacity=0.7] (1.6380, 1.1760, 3.2326) -- (1.6840, 1.1760, 3.2315) -- (1.6840, 1.2300, 3.2333) -- (1.6380, 1.2300, 3.2344) -- cycle;
\fill[blue!17.5, opacity=0.7] (1.6380, 1.2300, 3.2344) -- (1.6840, 1.2300, 3.2333) -- (1.6840, 1.2840, 3.2348) -- (1.6380, 1.2840, 3.2359) -- cycle;
\fill[blue!16.1, opacity=0.7] (1.6380, 1.2840, 3.2359) -- (1.6840, 1.2840, 3.2348) -- (1.6840, 1.3380, 3.2359) -- (1.6380, 1.3380, 3.2370) -- cycle;
\fill[blue!15.7, opacity=0.7] (1.6380, 1.3380, 3.2370) -- (1.6840, 1.3380, 3.2359) -- (1.6840, 1.3920, 3.2367) -- (1.6380, 1.3920, 3.2379) -- cycle;
\fill[blue!15.6, opacity=0.7] (1.6380, 1.3920, 3.2379) -- (1.6840, 1.3920, 3.2367) -- (1.6840, 1.4460, 3.2372) -- (1.6380, 1.4460, 3.2384) -- cycle;
\fill[blue!16.1, opacity=0.7] (1.6380, 1.4460, 3.2384) -- (1.6840, 1.4460, 3.2372) -- (1.6840, 1.5000, 3.2374) -- (1.6380, 1.5000, 3.2385) -- cycle;
\fill[blue!19.5, opacity=0.7] (1.6380, 1.5000, 3.2385) -- (1.6840, 1.5000, 3.2374) -- (1.6840, 1.5540, 3.2372) -- (1.6380, 1.5540, 3.2384) -- cycle;
\fill[blue!35.4, opacity=0.7] (1.6380, 1.5540, 3.2384) -- (1.6840, 1.5540, 3.2372) -- (1.6840, 1.6080, 3.2367) -- (1.6380, 1.6080, 3.2379) -- cycle;
\fill[blue!52.8, opacity=0.7] (1.6380, 1.6080, 3.2379) -- (1.6840, 1.6080, 3.2367) -- (1.6840, 1.6620, 3.2359) -- (1.6380, 1.6620, 3.2370) -- cycle;
\fill[blue!50.8, opacity=0.7] (1.6380, 1.6620, 3.2370) -- (1.6840, 1.6620, 3.2359) -- (1.6840, 1.7160, 3.2348) -- (1.6380, 1.7160, 3.2359) -- cycle;
\fill[blue!31.6, opacity=0.7] (1.6380, 1.7160, 3.2359) -- (1.6840, 1.7160, 3.2348) -- (1.6840, 1.7700, 3.2333) -- (1.6380, 1.7700, 3.2344) -- cycle;
\fill[blue!18.7, opacity=0.7] (1.6380, 1.7700, 3.2344) -- (1.6840, 1.7700, 3.2333) -- (1.6840, 1.8240, 3.2315) -- (1.6380, 1.8240, 3.2326) -- cycle;
\fill[blue!16.5, opacity=0.7] (1.6380, 1.8240, 3.2326) -- (1.6840, 1.8240, 3.2315) -- (1.6840, 1.8780, 3.2294) -- (1.6380, 1.8780, 3.2306) -- cycle;
\fill[blue!18.1, opacity=0.7] (1.6380, 1.8780, 3.2306) -- (1.6840, 1.8780, 3.2294) -- (1.6840, 1.9320, 3.2270) -- (1.6380, 1.9320, 3.2281) -- cycle;
\fill[blue!31.4, opacity=0.7] (1.6380, 1.9320, 3.2281) -- (1.6840, 1.9320, 3.2270) -- (1.6840, 1.9860, 3.2243) -- (1.6380, 1.9860, 3.2254) -- cycle;
\fill[blue!58.0, opacity=0.7] (1.6380, 1.9860, 3.2254) -- (1.6840, 1.9860, 3.2243) -- (1.6840, 2.0400, 3.2213) -- (1.6380, 2.0400, 3.2224) -- cycle;
\fill[blue!63.3, opacity=0.7] (1.6380, 2.0400, 3.2224) -- (1.6840, 2.0400, 3.2213) -- (1.6840, 2.0940, 3.2180) -- (1.6380, 2.0940, 3.2192) -- cycle;
\fill[blue!63.5, opacity=0.7] (1.6380, 2.0940, 3.2192) -- (1.6840, 2.0940, 3.2180) -- (1.6840, 2.1480, 3.2145) -- (1.6380, 2.1480, 3.2156) -- cycle;
\fill[blue!56.7, opacity=0.7] (1.6380, 2.1480, 3.2156) -- (1.6840, 2.1480, 3.2145) -- (1.6840, 2.2020, 3.2106) -- (1.6380, 2.2020, 3.2118) -- cycle;
\fill[blue!35.7, opacity=0.7] (1.6380, 2.2020, 3.2118) -- (1.6840, 2.2020, 3.2106) -- (1.6840, 2.2560, 3.2066) -- (1.6380, 2.2560, 3.2077) -- cycle;
\fill[blue!25.9, opacity=0.7] (1.6380, 2.2560, 3.2077) -- (1.6840, 2.2560, 3.2066) -- (1.6840, 2.3100, 3.2022) -- (1.6380, 2.3100, 3.2034) -- cycle;
\fill[blue!30.1, opacity=0.7] (1.6380, 2.3100, 3.2034) -- (1.6840, 2.3100, 3.2022) -- (1.6840, 2.3640, 3.1977) -- (1.6380, 2.3640, 3.1988) -- cycle;
\fill[blue!51.5, opacity=0.7] (1.6380, 2.3640, 3.1988) -- (1.6840, 2.3640, 3.1977) -- (1.6840, 2.4180, 3.1929) -- (1.6380, 2.4180, 3.1940) -- cycle;
\fill[blue!62.3, opacity=0.7] (1.6380, 2.4180, 3.1940) -- (1.6840, 2.4180, 3.1929) -- (1.6840, 2.4720, 3.1879) -- (1.6380, 2.4720, 3.1891) -- cycle;
\fill[blue!45.9, opacity=0.7] (1.6380, 2.4720, 3.1891) -- (1.6840, 2.4720, 3.1879) -- (1.6840, 2.5260, 3.1827) -- (1.6380, 2.5260, 3.1839) -- cycle;
\fill[blue!42.2, opacity=0.7] (1.6380, 2.5260, 3.1839) -- (1.6840, 2.5260, 3.1827) -- (1.6840, 2.5800, 3.1774) -- (1.6380, 2.5800, 3.1785) -- cycle;
\fill[blue!55.4, opacity=0.7] (1.6380, 2.5800, 3.1785) -- (1.6840, 2.5800, 3.1774) -- (1.6840, 2.6340, 3.1719) -- (1.6380, 2.6340, 3.1730) -- cycle;
\fill[blue!63.1, opacity=0.7] (1.6380, 2.6340, 3.1730) -- (1.6840, 2.6340, 3.1719) -- (1.6840, 2.6880, 3.1662) -- (1.6380, 2.6880, 3.1673) -- cycle;
\fill[blue!53.0, opacity=0.7] (1.6380, 2.6880, 3.1673) -- (1.6840, 2.6880, 3.1662) -- (1.6840, 2.7420, 3.1604) -- (1.6380, 2.7420, 3.1615) -- cycle;
\fill[blue!53.0, opacity=0.7] (1.6380, 2.7420, 3.1615) -- (1.6840, 2.7420, 3.1604) -- (1.6840, 2.7960, 3.1545) -- (1.6380, 2.7960, 3.1556) -- cycle;
\fill[blue!63.4, opacity=0.7] (1.6380, 2.7960, 3.1556) -- (1.6840, 2.7960, 3.1545) -- (1.6840, 2.8500, 3.1484) -- (1.6380, 2.8500, 3.1496) -- cycle;
\fill[blue!47.7, opacity=0.7] (1.6380, 2.8500, 3.1496) -- (1.6840, 2.8500, 3.1484) -- (1.6840, 2.9040, 3.1423) -- (1.6380, 2.9040, 3.1435) -- cycle;
\fill[blue!25.7, opacity=0.7] (1.6380, 2.9040, 3.1435) -- (1.6840, 2.9040, 3.1423) -- (1.6840, 2.9580, 3.1361) -- (1.6380, 2.9580, 3.1373) -- cycle;
\fill[blue!20.6, opacity=0.7] (1.6380, 2.9580, 3.1373) -- (1.6840, 2.9580, 3.1361) -- (1.6840, 3.0120, 3.1299) -- (1.6380, 3.0120, 3.1311) -- cycle;
\fill[blue!24.9, opacity=0.7] (1.6380, 3.0120, 3.1311) -- (1.6840, 3.0120, 3.1299) -- (1.6840, 3.0660, 3.1237) -- (1.6380, 3.0660, 3.1248) -- cycle;
\fill[blue!42.1, opacity=0.7] (1.6380, 3.0660, 3.1248) -- (1.6840, 3.0660, 3.1237) -- (1.6840, 3.1200, 3.1174) -- (1.6380, 3.1200, 3.1185) -- cycle;
\fill[blue!59.7, opacity=0.7] (1.6840, -0.1200, 3.1174) -- (1.7300, -0.1200, 3.1159) -- (1.7300, -0.0660, 3.1222) -- (1.6840, -0.0660, 3.1237) -- cycle;
\fill[blue!41.9, opacity=0.7] (1.6840, -0.0660, 3.1237) -- (1.7300, -0.0660, 3.1222) -- (1.7300, -0.0120, 3.1285) -- (1.6840, -0.0120, 3.1299) -- cycle;
\fill[blue!25.6, opacity=0.7] (1.6840, -0.0120, 3.1299) -- (1.7300, -0.0120, 3.1285) -- (1.7300, 0.0420, 3.1347) -- (1.6840, 0.0420, 3.1361) -- cycle;
\fill[blue!21.8, opacity=0.7] (1.6840, 0.0420, 3.1361) -- (1.7300, 0.0420, 3.1347) -- (1.7300, 0.0960, 3.1409) -- (1.6840, 0.0960, 3.1423) -- cycle;
\fill[blue!27.8, opacity=0.7] (1.6840, 0.0960, 3.1423) -- (1.7300, 0.0960, 3.1409) -- (1.7300, 0.1500, 3.1470) -- (1.6840, 0.1500, 3.1484) -- cycle;
\fill[blue!50.0, opacity=0.7] (1.6840, 0.1500, 3.1484) -- (1.7300, 0.1500, 3.1470) -- (1.7300, 0.2040, 3.1530) -- (1.6840, 0.2040, 3.1545) -- cycle;
\fill[blue!63.2, opacity=0.7] (1.6840, 0.2040, 3.1545) -- (1.7300, 0.2040, 3.1530) -- (1.7300, 0.2580, 3.1589) -- (1.6840, 0.2580, 3.1604) -- cycle;
\fill[blue!51.1, opacity=0.7] (1.6840, 0.2580, 3.1604) -- (1.7300, 0.2580, 3.1589) -- (1.7300, 0.3120, 3.1647) -- (1.6840, 0.3120, 3.1662) -- cycle;
\fill[blue!48.6, opacity=0.7] (1.6840, 0.3120, 3.1662) -- (1.7300, 0.3120, 3.1647) -- (1.7300, 0.3660, 3.1704) -- (1.6840, 0.3660, 3.1719) -- cycle;
\fill[blue!59.6, opacity=0.7] (1.6840, 0.3660, 3.1719) -- (1.7300, 0.3660, 3.1704) -- (1.7300, 0.4200, 3.1759) -- (1.6840, 0.4200, 3.1774) -- cycle;
\fill[blue!61.6, opacity=0.7] (1.6840, 0.4200, 3.1774) -- (1.7300, 0.4200, 3.1759) -- (1.7300, 0.4740, 3.1813) -- (1.6840, 0.4740, 3.1827) -- cycle;
\fill[blue!49.0, opacity=0.7] (1.6840, 0.4740, 3.1827) -- (1.7300, 0.4740, 3.1813) -- (1.7300, 0.5280, 3.1864) -- (1.6840, 0.5280, 3.1879) -- cycle;
\fill[blue!46.0, opacity=0.7] (1.6840, 0.5280, 3.1879) -- (1.7300, 0.5280, 3.1864) -- (1.7300, 0.5820, 3.1914) -- (1.6840, 0.5820, 3.1929) -- cycle;
\fill[blue!57.9, opacity=0.7] (1.6840, 0.5820, 3.1929) -- (1.7300, 0.5820, 3.1914) -- (1.7300, 0.6360, 3.1962) -- (1.6840, 0.6360, 3.1977) -- cycle;
\fill[blue!61.0, opacity=0.7] (1.6840, 0.6360, 3.1977) -- (1.7300, 0.6360, 3.1962) -- (1.7300, 0.6900, 3.2008) -- (1.6840, 0.6900, 3.2022) -- cycle;
\fill[blue!38.3, opacity=0.7] (1.6840, 0.6900, 3.2022) -- (1.7300, 0.6900, 3.2008) -- (1.7300, 0.7440, 3.2051) -- (1.6840, 0.7440, 3.2066) -- cycle;
\fill[blue!24.6, opacity=0.7] (1.6840, 0.7440, 3.2066) -- (1.7300, 0.7440, 3.2051) -- (1.7300, 0.7980, 3.2092) -- (1.6840, 0.7980, 3.2106) -- cycle;
\fill[blue!23.0, opacity=0.7] (1.6840, 0.7980, 3.2106) -- (1.7300, 0.7980, 3.2092) -- (1.7300, 0.8520, 3.2130) -- (1.6840, 0.8520, 3.2145) -- cycle;
\fill[blue!29.6, opacity=0.7] (1.6840, 0.8520, 3.2145) -- (1.7300, 0.8520, 3.2130) -- (1.7300, 0.9060, 3.2166) -- (1.6840, 0.9060, 3.2180) -- cycle;
\fill[blue!45.8, opacity=0.7] (1.6840, 0.9060, 3.2180) -- (1.7300, 0.9060, 3.2166) -- (1.7300, 0.9600, 3.2198) -- (1.6840, 0.9600, 3.2213) -- cycle;
\fill[blue!59.9, opacity=0.7] (1.6840, 0.9600, 3.2213) -- (1.7300, 0.9600, 3.2198) -- (1.7300, 1.0140, 3.2228) -- (1.6840, 1.0140, 3.2243) -- cycle;
\fill[blue!63.3, opacity=0.7] (1.6840, 1.0140, 3.2243) -- (1.7300, 1.0140, 3.2228) -- (1.7300, 1.0680, 3.2255) -- (1.6840, 1.0680, 3.2270) -- cycle;
\fill[blue!62.8, opacity=0.7] (1.6840, 1.0680, 3.2270) -- (1.7300, 1.0680, 3.2255) -- (1.7300, 1.1220, 3.2279) -- (1.6840, 1.1220, 3.2294) -- cycle;
\fill[blue!57.1, opacity=0.7] (1.6840, 1.1220, 3.2294) -- (1.7300, 1.1220, 3.2279) -- (1.7300, 1.1760, 3.2300) -- (1.6840, 1.1760, 3.2315) -- cycle;
\fill[blue!43.2, opacity=0.7] (1.6840, 1.1760, 3.2315) -- (1.7300, 1.1760, 3.2300) -- (1.7300, 1.2300, 3.2318) -- (1.6840, 1.2300, 3.2333) -- cycle;
\fill[blue!29.9, opacity=0.7] (1.6840, 1.2300, 3.2333) -- (1.7300, 1.2300, 3.2318) -- (1.7300, 1.2840, 3.2333) -- (1.6840, 1.2840, 3.2348) -- cycle;
\fill[blue!22.9, opacity=0.7] (1.6840, 1.2840, 3.2348) -- (1.7300, 1.2840, 3.2333) -- (1.7300, 1.3380, 3.2344) -- (1.6840, 1.3380, 3.2359) -- cycle;
\fill[blue!20.4, opacity=0.7] (1.6840, 1.3380, 3.2359) -- (1.7300, 1.3380, 3.2344) -- (1.7300, 1.3920, 3.2353) -- (1.6840, 1.3920, 3.2367) -- cycle;
\fill[blue!20.8, opacity=0.7] (1.6840, 1.3920, 3.2367) -- (1.7300, 1.3920, 3.2353) -- (1.7300, 1.4460, 3.2357) -- (1.6840, 1.4460, 3.2372) -- cycle;
\fill[blue!24.9, opacity=0.7] (1.6840, 1.4460, 3.2372) -- (1.7300, 1.4460, 3.2357) -- (1.7300, 1.5000, 3.2359) -- (1.6840, 1.5000, 3.2374) -- cycle;
\fill[blue!36.3, opacity=0.7] (1.6840, 1.5000, 3.2374) -- (1.7300, 1.5000, 3.2359) -- (1.7300, 1.5540, 3.2357) -- (1.6840, 1.5540, 3.2372) -- cycle;
\fill[blue!51.1, opacity=0.7] (1.6840, 1.5540, 3.2372) -- (1.7300, 1.5540, 3.2357) -- (1.7300, 1.6080, 3.2353) -- (1.6840, 1.6080, 3.2367) -- cycle;
\fill[blue!55.9, opacity=0.7] (1.6840, 1.6080, 3.2367) -- (1.7300, 1.6080, 3.2353) -- (1.7300, 1.6620, 3.2344) -- (1.6840, 1.6620, 3.2359) -- cycle;
\fill[blue!46.3, opacity=0.7] (1.6840, 1.6620, 3.2359) -- (1.7300, 1.6620, 3.2344) -- (1.7300, 1.7160, 3.2333) -- (1.6840, 1.7160, 3.2348) -- cycle;
\fill[blue!26.9, opacity=0.7] (1.6840, 1.7160, 3.2348) -- (1.7300, 1.7160, 3.2333) -- (1.7300, 1.7700, 3.2318) -- (1.6840, 1.7700, 3.2333) -- cycle;
\fill[blue!17.9, opacity=0.7] (1.6840, 1.7700, 3.2333) -- (1.7300, 1.7700, 3.2318) -- (1.7300, 1.8240, 3.2300) -- (1.6840, 1.8240, 3.2315) -- cycle;
\fill[blue!16.6, opacity=0.7] (1.6840, 1.8240, 3.2315) -- (1.7300, 1.8240, 3.2300) -- (1.7300, 1.8780, 3.2279) -- (1.6840, 1.8780, 3.2294) -- cycle;
\fill[blue!19.6, opacity=0.7] (1.6840, 1.8780, 3.2294) -- (1.7300, 1.8780, 3.2279) -- (1.7300, 1.9320, 3.2255) -- (1.6840, 1.9320, 3.2270) -- cycle;
\fill[blue!36.5, opacity=0.7] (1.6840, 1.9320, 3.2270) -- (1.7300, 1.9320, 3.2255) -- (1.7300, 1.9860, 3.2228) -- (1.6840, 1.9860, 3.2243) -- cycle;
\fill[blue!60.8, opacity=0.7] (1.6840, 1.9860, 3.2243) -- (1.7300, 1.9860, 3.2228) -- (1.7300, 2.0400, 3.2198) -- (1.6840, 2.0400, 3.2213) -- cycle;
\fill[blue!62.9, opacity=0.7] (1.6840, 2.0400, 3.2213) -- (1.7300, 2.0400, 3.2198) -- (1.7300, 2.0940, 3.2166) -- (1.6840, 2.0940, 3.2180) -- cycle;
\fill[blue!63.6, opacity=0.7] (1.6840, 2.0940, 3.2180) -- (1.7300, 2.0940, 3.2166) -- (1.7300, 2.1480, 3.2130) -- (1.6840, 2.1480, 3.2145) -- cycle;
\fill[blue!55.3, opacity=0.7] (1.6840, 2.1480, 3.2145) -- (1.7300, 2.1480, 3.2130) -- (1.7300, 2.2020, 3.2092) -- (1.6840, 2.2020, 3.2106) -- cycle;
\fill[blue!34.5, opacity=0.7] (1.6840, 2.2020, 3.2106) -- (1.7300, 2.2020, 3.2092) -- (1.7300, 2.2560, 3.2051) -- (1.6840, 2.2560, 3.2066) -- cycle;
\fill[blue!26.1, opacity=0.7] (1.6840, 2.2560, 3.2066) -- (1.7300, 2.2560, 3.2051) -- (1.7300, 2.3100, 3.2008) -- (1.6840, 2.3100, 3.2022) -- cycle;
\fill[blue!31.7, opacity=0.7] (1.6840, 2.3100, 3.2022) -- (1.7300, 2.3100, 3.2008) -- (1.7300, 2.3640, 3.1962) -- (1.6840, 2.3640, 3.1977) -- cycle;
\fill[blue!54.1, opacity=0.7] (1.6840, 2.3640, 3.1977) -- (1.7300, 2.3640, 3.1962) -- (1.7300, 2.4180, 3.1914) -- (1.6840, 2.4180, 3.1929) -- cycle;
\fill[blue!61.2, opacity=0.7] (1.6840, 2.4180, 3.1929) -- (1.7300, 2.4180, 3.1914) -- (1.7300, 2.4720, 3.1864) -- (1.6840, 2.4720, 3.1879) -- cycle;
\fill[blue!44.5, opacity=0.7] (1.6840, 2.4720, 3.1879) -- (1.7300, 2.4720, 3.1864) -- (1.7300, 2.5260, 3.1813) -- (1.6840, 2.5260, 3.1827) -- cycle;
\fill[blue!42.2, opacity=0.7] (1.6840, 2.5260, 3.1827) -- (1.7300, 2.5260, 3.1813) -- (1.7300, 2.5800, 3.1759) -- (1.6840, 2.5800, 3.1774) -- cycle;
\fill[blue!56.2, opacity=0.7] (1.6840, 2.5800, 3.1774) -- (1.7300, 2.5800, 3.1759) -- (1.7300, 2.6340, 3.1704) -- (1.6840, 2.6340, 3.1719) -- cycle;
\fill[blue!62.8, opacity=0.7] (1.6840, 2.6340, 3.1719) -- (1.7300, 2.6340, 3.1704) -- (1.7300, 2.6880, 3.1647) -- (1.6840, 2.6880, 3.1662) -- cycle;
\fill[blue!52.8, opacity=0.7] (1.6840, 2.6880, 3.1662) -- (1.7300, 2.6880, 3.1647) -- (1.7300, 2.7420, 3.1589) -- (1.6840, 2.7420, 3.1604) -- cycle;
\fill[blue!53.9, opacity=0.7] (1.6840, 2.7420, 3.1604) -- (1.7300, 2.7420, 3.1589) -- (1.7300, 2.7960, 3.1530) -- (1.6840, 2.7960, 3.1545) -- cycle;
\fill[blue!63.6, opacity=0.7] (1.6840, 2.7960, 3.1545) -- (1.7300, 2.7960, 3.1530) -- (1.7300, 2.8500, 3.1470) -- (1.6840, 2.8500, 3.1484) -- cycle;
\fill[blue!45.8, opacity=0.7] (1.6840, 2.8500, 3.1484) -- (1.7300, 2.8500, 3.1470) -- (1.7300, 2.9040, 3.1409) -- (1.6840, 2.9040, 3.1423) -- cycle;
\fill[blue!24.9, opacity=0.7] (1.6840, 2.9040, 3.1423) -- (1.7300, 2.9040, 3.1409) -- (1.7300, 2.9580, 3.1347) -- (1.6840, 2.9580, 3.1361) -- cycle;
\fill[blue!20.5, opacity=0.7] (1.6840, 2.9580, 3.1361) -- (1.7300, 2.9580, 3.1347) -- (1.7300, 3.0120, 3.1285) -- (1.6840, 3.0120, 3.1299) -- cycle;
\fill[blue!25.1, opacity=0.7] (1.6840, 3.0120, 3.1299) -- (1.7300, 3.0120, 3.1285) -- (1.7300, 3.0660, 3.1222) -- (1.6840, 3.0660, 3.1237) -- cycle;
\fill[blue!42.8, opacity=0.7] (1.6840, 3.0660, 3.1237) -- (1.7300, 3.0660, 3.1222) -- (1.7300, 3.1200, 3.1159) -- (1.6840, 3.1200, 3.1174) -- cycle;
\fill[blue!61.9, opacity=0.7] (1.7300, -0.1200, 3.1159) -- (1.7760, -0.1200, 3.1141) -- (1.7760, -0.0660, 3.1204) -- (1.7300, -0.0660, 3.1222) -- cycle;
\fill[blue!48.1, opacity=0.7] (1.7300, -0.0660, 3.1222) -- (1.7760, -0.0660, 3.1204) -- (1.7760, -0.0120, 3.1267) -- (1.7300, -0.0120, 3.1285) -- cycle;
\fill[blue!29.0, opacity=0.7] (1.7300, -0.0120, 3.1285) -- (1.7760, -0.0120, 3.1267) -- (1.7760, 0.0420, 3.1329) -- (1.7300, 0.0420, 3.1347) -- cycle;
\fill[blue!22.1, opacity=0.7] (1.7300, 0.0420, 3.1347) -- (1.7760, 0.0420, 3.1329) -- (1.7760, 0.0960, 3.1391) -- (1.7300, 0.0960, 3.1409) -- cycle;
\fill[blue!24.8, opacity=0.7] (1.7300, 0.0960, 3.1409) -- (1.7760, 0.0960, 3.1391) -- (1.7760, 0.1500, 3.1452) -- (1.7300, 0.1500, 3.1470) -- cycle;
\fill[blue!41.9, opacity=0.7] (1.7300, 0.1500, 3.1470) -- (1.7760, 0.1500, 3.1452) -- (1.7760, 0.2040, 3.1512) -- (1.7300, 0.2040, 3.1530) -- cycle;
\fill[blue!63.0, opacity=0.7] (1.7300, 0.2040, 3.1530) -- (1.7760, 0.2040, 3.1512) -- (1.7760, 0.2580, 3.1571) -- (1.7300, 0.2580, 3.1589) -- cycle;
\fill[blue!55.1, opacity=0.7] (1.7300, 0.2580, 3.1589) -- (1.7760, 0.2580, 3.1571) -- (1.7760, 0.3120, 3.1629) -- (1.7300, 0.3120, 3.1647) -- cycle;
\fill[blue!47.1, opacity=0.7] (1.7300, 0.3120, 3.1647) -- (1.7760, 0.3120, 3.1629) -- (1.7760, 0.3660, 3.1686) -- (1.7300, 0.3660, 3.1704) -- cycle;
\fill[blue!54.4, opacity=0.7] (1.7300, 0.3660, 3.1704) -- (1.7760, 0.3660, 3.1686) -- (1.7760, 0.4200, 3.1741) -- (1.7300, 0.4200, 3.1759) -- cycle;
\fill[blue!63.6, opacity=0.7] (1.7300, 0.4200, 3.1759) -- (1.7760, 0.4200, 3.1741) -- (1.7760, 0.4740, 3.1795) -- (1.7300, 0.4740, 3.1813) -- cycle;
\fill[blue!54.7, opacity=0.7] (1.7300, 0.4740, 3.1813) -- (1.7760, 0.4740, 3.1795) -- (1.7760, 0.5280, 3.1847) -- (1.7300, 0.5280, 3.1864) -- cycle;
\fill[blue!45.9, opacity=0.7] (1.7300, 0.5280, 3.1864) -- (1.7760, 0.5280, 3.1847) -- (1.7760, 0.5820, 3.1896) -- (1.7300, 0.5820, 3.1914) -- cycle;
\fill[blue!51.5, opacity=0.7] (1.7300, 0.5820, 3.1914) -- (1.7760, 0.5820, 3.1896) -- (1.7760, 0.6360, 3.1944) -- (1.7300, 0.6360, 3.1962) -- cycle;
\fill[blue!63.4, opacity=0.7] (1.7300, 0.6360, 3.1962) -- (1.7760, 0.6360, 3.1944) -- (1.7760, 0.6900, 3.1990) -- (1.7300, 0.6900, 3.2008) -- cycle;
\fill[blue!51.0, opacity=0.7] (1.7300, 0.6900, 3.2008) -- (1.7760, 0.6900, 3.1990) -- (1.7760, 0.7440, 3.2033) -- (1.7300, 0.7440, 3.2051) -- cycle;
\fill[blue!30.0, opacity=0.7] (1.7300, 0.7440, 3.2051) -- (1.7760, 0.7440, 3.2033) -- (1.7760, 0.7980, 3.2074) -- (1.7300, 0.7980, 3.2092) -- cycle;
\fill[blue!22.6, opacity=0.7] (1.7300, 0.7980, 3.2092) -- (1.7760, 0.7980, 3.2074) -- (1.7760, 0.8520, 3.2112) -- (1.7300, 0.8520, 3.2130) -- cycle;
\fill[blue!23.6, opacity=0.7] (1.7300, 0.8520, 3.2130) -- (1.7760, 0.8520, 3.2112) -- (1.7760, 0.9060, 3.2148) -- (1.7300, 0.9060, 3.2166) -- cycle;
\fill[blue!31.9, opacity=0.7] (1.7300, 0.9060, 3.2166) -- (1.7760, 0.9060, 3.2148) -- (1.7760, 0.9600, 3.2180) -- (1.7300, 0.9600, 3.2198) -- cycle;
\fill[blue!47.5, opacity=0.7] (1.7300, 0.9600, 3.2198) -- (1.7760, 0.9600, 3.2180) -- (1.7760, 1.0140, 3.2210) -- (1.7300, 1.0140, 3.2228) -- cycle;
\fill[blue!59.5, opacity=0.7] (1.7300, 1.0140, 3.2228) -- (1.7760, 1.0140, 3.2210) -- (1.7760, 1.0680, 3.2238) -- (1.7300, 1.0680, 3.2255) -- cycle;
\fill[blue!62.9, opacity=0.7] (1.7300, 1.0680, 3.2255) -- (1.7760, 1.0680, 3.2238) -- (1.7760, 1.1220, 3.2262) -- (1.7300, 1.1220, 3.2279) -- cycle;
\fill[blue!62.8, opacity=0.7] (1.7300, 1.1220, 3.2279) -- (1.7760, 1.1220, 3.2262) -- (1.7760, 1.1760, 3.2283) -- (1.7300, 1.1760, 3.2300) -- cycle;
\fill[blue!60.2, opacity=0.7] (1.7300, 1.1760, 3.2300) -- (1.7760, 1.1760, 3.2283) -- (1.7760, 1.2300, 3.2300) -- (1.7300, 1.2300, 3.2318) -- cycle;
\fill[blue!54.1, opacity=0.7] (1.7300, 1.2300, 3.2318) -- (1.7760, 1.2300, 3.2300) -- (1.7760, 1.2840, 3.2315) -- (1.7300, 1.2840, 3.2333) -- cycle;
\fill[blue!47.1, opacity=0.7] (1.7300, 1.2840, 3.2333) -- (1.7760, 1.2840, 3.2315) -- (1.7760, 1.3380, 3.2326) -- (1.7300, 1.3380, 3.2344) -- cycle;
\fill[blue!43.2, opacity=0.7] (1.7300, 1.3380, 3.2344) -- (1.7760, 1.3380, 3.2326) -- (1.7760, 1.3920, 3.2335) -- (1.7300, 1.3920, 3.2353) -- cycle;
\fill[blue!43.9, opacity=0.7] (1.7300, 1.3920, 3.2353) -- (1.7760, 1.3920, 3.2335) -- (1.7760, 1.4460, 3.2340) -- (1.7300, 1.4460, 3.2357) -- cycle;
\fill[blue!49.0, opacity=0.7] (1.7300, 1.4460, 3.2357) -- (1.7760, 1.4460, 3.2340) -- (1.7760, 1.5000, 3.2341) -- (1.7300, 1.5000, 3.2359) -- cycle;
\fill[blue!55.2, opacity=0.7] (1.7300, 1.5000, 3.2359) -- (1.7760, 1.5000, 3.2341) -- (1.7760, 1.5540, 3.2340) -- (1.7300, 1.5540, 3.2357) -- cycle;
\fill[blue!57.6, opacity=0.7] (1.7300, 1.5540, 3.2357) -- (1.7760, 1.5540, 3.2340) -- (1.7760, 1.6080, 3.2335) -- (1.7300, 1.6080, 3.2353) -- cycle;
\fill[blue!52.4, opacity=0.7] (1.7300, 1.6080, 3.2353) -- (1.7760, 1.6080, 3.2335) -- (1.7760, 1.6620, 3.2326) -- (1.7300, 1.6620, 3.2344) -- cycle;
\fill[blue!36.3, opacity=0.7] (1.7300, 1.6620, 3.2344) -- (1.7760, 1.6620, 3.2326) -- (1.7760, 1.7160, 3.2315) -- (1.7300, 1.7160, 3.2333) -- cycle;
\fill[blue!21.6, opacity=0.7] (1.7300, 1.7160, 3.2333) -- (1.7760, 1.7160, 3.2315) -- (1.7760, 1.7700, 3.2300) -- (1.7300, 1.7700, 3.2318) -- cycle;
\fill[blue!17.1, opacity=0.7] (1.7300, 1.7700, 3.2318) -- (1.7760, 1.7700, 3.2300) -- (1.7760, 1.8240, 3.2283) -- (1.7300, 1.8240, 3.2300) -- cycle;
\fill[blue!17.2, opacity=0.7] (1.7300, 1.8240, 3.2300) -- (1.7760, 1.8240, 3.2283) -- (1.7760, 1.8780, 3.2262) -- (1.7300, 1.8780, 3.2279) -- cycle;
\fill[blue!23.0, opacity=0.7] (1.7300, 1.8780, 3.2279) -- (1.7760, 1.8780, 3.2262) -- (1.7760, 1.9320, 3.2238) -- (1.7300, 1.9320, 3.2255) -- cycle;
\fill[blue!45.2, opacity=0.7] (1.7300, 1.9320, 3.2255) -- (1.7760, 1.9320, 3.2238) -- (1.7760, 1.9860, 3.2210) -- (1.7300, 1.9860, 3.2228) -- cycle;
\fill[blue!63.1, opacity=0.7] (1.7300, 1.9860, 3.2228) -- (1.7760, 1.9860, 3.2210) -- (1.7760, 2.0400, 3.2180) -- (1.7300, 2.0400, 3.2198) -- cycle;
\fill[blue!62.5, opacity=0.7] (1.7300, 2.0400, 3.2198) -- (1.7760, 2.0400, 3.2180) -- (1.7760, 2.0940, 3.2148) -- (1.7300, 2.0940, 3.2166) -- cycle;
\fill[blue!63.5, opacity=0.7] (1.7300, 2.0940, 3.2166) -- (1.7760, 2.0940, 3.2148) -- (1.7760, 2.1480, 3.2112) -- (1.7300, 2.1480, 3.2130) -- cycle;
\fill[blue!52.1, opacity=0.7] (1.7300, 2.1480, 3.2130) -- (1.7760, 2.1480, 3.2112) -- (1.7760, 2.2020, 3.2074) -- (1.7300, 2.2020, 3.2092) -- cycle;
\fill[blue!32.5, opacity=0.7] (1.7300, 2.2020, 3.2092) -- (1.7760, 2.2020, 3.2074) -- (1.7760, 2.2560, 3.2033) -- (1.7300, 2.2560, 3.2051) -- cycle;
\fill[blue!26.4, opacity=0.7] (1.7300, 2.2560, 3.2051) -- (1.7760, 2.2560, 3.2033) -- (1.7760, 2.3100, 3.1990) -- (1.7300, 2.3100, 3.2008) -- cycle;
\fill[blue!34.5, opacity=0.7] (1.7300, 2.3100, 3.2008) -- (1.7760, 2.3100, 3.1990) -- (1.7760, 2.3640, 3.1944) -- (1.7300, 2.3640, 3.1962) -- cycle;
\fill[blue!57.8, opacity=0.7] (1.7300, 2.3640, 3.1962) -- (1.7760, 2.3640, 3.1944) -- (1.7760, 2.4180, 3.1896) -- (1.7300, 2.4180, 3.1914) -- cycle;
\fill[blue!58.8, opacity=0.7] (1.7300, 2.4180, 3.1914) -- (1.7760, 2.4180, 3.1896) -- (1.7760, 2.4720, 3.1847) -- (1.7300, 2.4720, 3.1864) -- cycle;
\fill[blue!42.6, opacity=0.7] (1.7300, 2.4720, 3.1864) -- (1.7760, 2.4720, 3.1847) -- (1.7760, 2.5260, 3.1795) -- (1.7300, 2.5260, 3.1813) -- cycle;
\fill[blue!42.7, opacity=0.7] (1.7300, 2.5260, 3.1813) -- (1.7760, 2.5260, 3.1795) -- (1.7760, 2.5800, 3.1741) -- (1.7300, 2.5800, 3.1759) -- cycle;
\fill[blue!57.7, opacity=0.7] (1.7300, 2.5800, 3.1759) -- (1.7760, 2.5800, 3.1741) -- (1.7760, 2.6340, 3.1686) -- (1.7300, 2.6340, 3.1704) -- cycle;
\fill[blue!62.2, opacity=0.7] (1.7300, 2.6340, 3.1704) -- (1.7760, 2.6340, 3.1686) -- (1.7760, 2.6880, 3.1629) -- (1.7300, 2.6880, 3.1647) -- cycle;
\fill[blue!52.5, opacity=0.7] (1.7300, 2.6880, 3.1647) -- (1.7760, 2.6880, 3.1629) -- (1.7760, 2.7420, 3.1571) -- (1.7300, 2.7420, 3.1589) -- cycle;
\fill[blue!55.2, opacity=0.7] (1.7300, 2.7420, 3.1589) -- (1.7760, 2.7420, 3.1571) -- (1.7760, 2.7960, 3.1512) -- (1.7300, 2.7960, 3.1530) -- cycle;
\fill[blue!63.4, opacity=0.7] (1.7300, 2.7960, 3.1530) -- (1.7760, 2.7960, 3.1512) -- (1.7760, 2.8500, 3.1452) -- (1.7300, 2.8500, 3.1470) -- cycle;
\fill[blue!42.8, opacity=0.7] (1.7300, 2.8500, 3.1470) -- (1.7760, 2.8500, 3.1452) -- (1.7760, 2.9040, 3.1391) -- (1.7300, 2.9040, 3.1409) -- cycle;
\fill[blue!23.8, opacity=0.7] (1.7300, 2.9040, 3.1409) -- (1.7760, 2.9040, 3.1391) -- (1.7760, 2.9580, 3.1329) -- (1.7300, 2.9580, 3.1347) -- cycle;
\fill[blue!20.4, opacity=0.7] (1.7300, 2.9580, 3.1347) -- (1.7760, 2.9580, 3.1329) -- (1.7760, 3.0120, 3.1267) -- (1.7300, 3.0120, 3.1285) -- cycle;
\fill[blue!25.8, opacity=0.7] (1.7300, 3.0120, 3.1285) -- (1.7760, 3.0120, 3.1267) -- (1.7760, 3.0660, 3.1204) -- (1.7300, 3.0660, 3.1222) -- cycle;
\fill[blue!44.2, opacity=0.7] (1.7300, 3.0660, 3.1222) -- (1.7760, 3.0660, 3.1204) -- (1.7760, 3.1200, 3.1141) -- (1.7300, 3.1200, 3.1159) -- cycle;
\fill[blue!63.0, opacity=0.7] (1.7760, -0.1200, 3.1141) -- (1.8220, -0.1200, 3.1120) -- (1.8220, -0.0660, 3.1183) -- (1.7760, -0.0660, 3.1204) -- cycle;
\fill[blue!54.3, opacity=0.7] (1.7760, -0.0660, 3.1204) -- (1.8220, -0.0660, 3.1183) -- (1.8220, -0.0120, 3.1246) -- (1.7760, -0.0120, 3.1267) -- cycle;
\fill[blue!34.3, opacity=0.7] (1.7760, -0.0120, 3.1267) -- (1.8220, -0.0120, 3.1246) -- (1.8220, 0.0420, 3.1308) -- (1.7760, 0.0420, 3.1329) -- cycle;
\fill[blue!23.3, opacity=0.7] (1.7760, 0.0420, 3.1329) -- (1.8220, 0.0420, 3.1308) -- (1.8220, 0.0960, 3.1370) -- (1.7760, 0.0960, 3.1391) -- cycle;
\fill[blue!22.9, opacity=0.7] (1.7760, 0.0960, 3.1391) -- (1.8220, 0.0960, 3.1370) -- (1.8220, 0.1500, 3.1431) -- (1.7760, 0.1500, 3.1452) -- cycle;
\fill[blue!34.0, opacity=0.7] (1.7760, 0.1500, 3.1452) -- (1.8220, 0.1500, 3.1431) -- (1.8220, 0.2040, 3.1491) -- (1.7760, 0.2040, 3.1512) -- cycle;
\fill[blue!58.1, opacity=0.7] (1.7760, 0.2040, 3.1512) -- (1.8220, 0.2040, 3.1491) -- (1.8220, 0.2580, 3.1550) -- (1.7760, 0.2580, 3.1571) -- cycle;
\fill[blue!60.3, opacity=0.7] (1.7760, 0.2580, 3.1571) -- (1.8220, 0.2580, 3.1550) -- (1.8220, 0.3120, 3.1608) -- (1.7760, 0.3120, 3.1629) -- cycle;
\fill[blue!48.1, opacity=0.7] (1.7760, 0.3120, 3.1629) -- (1.8220, 0.3120, 3.1608) -- (1.8220, 0.3660, 3.1665) -- (1.7760, 0.3660, 3.1686) -- cycle;
\fill[blue!49.2, opacity=0.7] (1.7760, 0.3660, 3.1686) -- (1.8220, 0.3660, 3.1665) -- (1.8220, 0.4200, 3.1720) -- (1.7760, 0.4200, 3.1741) -- cycle;
\fill[blue!60.8, opacity=0.7] (1.7760, 0.4200, 3.1741) -- (1.8220, 0.4200, 3.1720) -- (1.8220, 0.4740, 3.1774) -- (1.7760, 0.4740, 3.1795) -- cycle;
\fill[blue!61.0, opacity=0.7] (1.7760, 0.4740, 3.1795) -- (1.8220, 0.4740, 3.1774) -- (1.8220, 0.5280, 3.1826) -- (1.7760, 0.5280, 3.1847) -- cycle;
\fill[blue!49.4, opacity=0.7] (1.7760, 0.5280, 3.1847) -- (1.8220, 0.5280, 3.1826) -- (1.8220, 0.5820, 3.1875) -- (1.7760, 0.5820, 3.1896) -- cycle;
\fill[blue!47.2, opacity=0.7] (1.7760, 0.5820, 3.1896) -- (1.8220, 0.5820, 3.1875) -- (1.8220, 0.6360, 3.1923) -- (1.7760, 0.6360, 3.1944) -- cycle;
\fill[blue!57.9, opacity=0.7] (1.7760, 0.6360, 3.1944) -- (1.8220, 0.6360, 3.1923) -- (1.8220, 0.6900, 3.1969) -- (1.7760, 0.6900, 3.1990) -- cycle;
\fill[blue!62.0, opacity=0.7] (1.7760, 0.6900, 3.1990) -- (1.8220, 0.6900, 3.1969) -- (1.8220, 0.7440, 3.2012) -- (1.7760, 0.7440, 3.2033) -- cycle;
\fill[blue!42.0, opacity=0.7] (1.7760, 0.7440, 3.2033) -- (1.8220, 0.7440, 3.2012) -- (1.8220, 0.7980, 3.2053) -- (1.7760, 0.7980, 3.2074) -- cycle;
\fill[blue!26.2, opacity=0.7] (1.7760, 0.7980, 3.2074) -- (1.8220, 0.7980, 3.2053) -- (1.8220, 0.8520, 3.2091) -- (1.7760, 0.8520, 3.2112) -- cycle;
\fill[blue!21.9, opacity=0.7] (1.7760, 0.8520, 3.2112) -- (1.8220, 0.8520, 3.2091) -- (1.8220, 0.9060, 3.2127) -- (1.7760, 0.9060, 3.2148) -- cycle;
\fill[blue!23.7, opacity=0.7] (1.7760, 0.9060, 3.2148) -- (1.8220, 0.9060, 3.2127) -- (1.8220, 0.9600, 3.2160) -- (1.7760, 0.9600, 3.2180) -- cycle;
\fill[blue!31.5, opacity=0.7] (1.7760, 0.9600, 3.2180) -- (1.8220, 0.9600, 3.2160) -- (1.8220, 1.0140, 3.2190) -- (1.7760, 1.0140, 3.2210) -- cycle;
\fill[blue!44.9, opacity=0.7] (1.7760, 1.0140, 3.2210) -- (1.8220, 1.0140, 3.2190) -- (1.8220, 1.0680, 3.2217) -- (1.7760, 1.0680, 3.2238) -- cycle;
\fill[blue!56.3, opacity=0.7] (1.7760, 1.0680, 3.2238) -- (1.8220, 1.0680, 3.2217) -- (1.8220, 1.1220, 3.2241) -- (1.7760, 1.1220, 3.2262) -- cycle;
\fill[blue!61.3, opacity=0.7] (1.7760, 1.1220, 3.2262) -- (1.8220, 1.1220, 3.2241) -- (1.8220, 1.1760, 3.2262) -- (1.7760, 1.1760, 3.2283) -- cycle;
\fill[blue!62.4, opacity=0.7] (1.7760, 1.1760, 3.2283) -- (1.8220, 1.1760, 3.2262) -- (1.8220, 1.2300, 3.2279) -- (1.7760, 1.2300, 3.2300) -- cycle;
\fill[blue!62.1, opacity=0.7] (1.7760, 1.2300, 3.2300) -- (1.8220, 1.2300, 3.2279) -- (1.8220, 1.2840, 3.2294) -- (1.7760, 1.2840, 3.2315) -- cycle;
\fill[blue!61.1, opacity=0.7] (1.7760, 1.2840, 3.2315) -- (1.8220, 1.2840, 3.2294) -- (1.8220, 1.3380, 3.2306) -- (1.7760, 1.3380, 3.2326) -- cycle;
\fill[blue!60.1, opacity=0.7] (1.7760, 1.3380, 3.2326) -- (1.8220, 1.3380, 3.2306) -- (1.8220, 1.3920, 3.2314) -- (1.7760, 1.3920, 3.2335) -- cycle;
\fill[blue!59.7, opacity=0.7] (1.7760, 1.3920, 3.2335) -- (1.8220, 1.3920, 3.2314) -- (1.8220, 1.4460, 3.2319) -- (1.7760, 1.4460, 3.2340) -- cycle;
\fill[blue!59.6, opacity=0.7] (1.7760, 1.4460, 3.2340) -- (1.8220, 1.4460, 3.2319) -- (1.8220, 1.5000, 3.2320) -- (1.7760, 1.5000, 3.2341) -- cycle;
\fill[blue!58.4, opacity=0.7] (1.7760, 1.5000, 3.2341) -- (1.8220, 1.5000, 3.2320) -- (1.8220, 1.5540, 3.2319) -- (1.7760, 1.5540, 3.2340) -- cycle;
\fill[blue!52.9, opacity=0.7] (1.7760, 1.5540, 3.2340) -- (1.8220, 1.5540, 3.2319) -- (1.8220, 1.6080, 3.2314) -- (1.7760, 1.6080, 3.2335) -- cycle;
\fill[blue!39.8, opacity=0.7] (1.7760, 1.6080, 3.2335) -- (1.8220, 1.6080, 3.2314) -- (1.8220, 1.6620, 3.2306) -- (1.7760, 1.6620, 3.2326) -- cycle;
\fill[blue!25.2, opacity=0.7] (1.7760, 1.6620, 3.2326) -- (1.8220, 1.6620, 3.2306) -- (1.8220, 1.7160, 3.2294) -- (1.7760, 1.7160, 3.2315) -- cycle;
\fill[blue!18.3, opacity=0.7] (1.7760, 1.7160, 3.2315) -- (1.8220, 1.7160, 3.2294) -- (1.8220, 1.7700, 3.2279) -- (1.7760, 1.7700, 3.2300) -- cycle;
\fill[blue!17.0, opacity=0.7] (1.7760, 1.7700, 3.2300) -- (1.8220, 1.7700, 3.2279) -- (1.8220, 1.8240, 3.2262) -- (1.7760, 1.8240, 3.2283) -- cycle;
\fill[blue!19.0, opacity=0.7] (1.7760, 1.8240, 3.2283) -- (1.8220, 1.8240, 3.2262) -- (1.8220, 1.8780, 3.2241) -- (1.7760, 1.8780, 3.2262) -- cycle;
\fill[blue!30.8, opacity=0.7] (1.7760, 1.8780, 3.2262) -- (1.8220, 1.8780, 3.2241) -- (1.8220, 1.9320, 3.2217) -- (1.7760, 1.9320, 3.2238) -- cycle;
\fill[blue!55.6, opacity=0.7] (1.7760, 1.9320, 3.2238) -- (1.8220, 1.9320, 3.2217) -- (1.8220, 1.9860, 3.2190) -- (1.7760, 1.9860, 3.2210) -- cycle;
\fill[blue!63.4, opacity=0.7] (1.7760, 1.9860, 3.2210) -- (1.8220, 1.9860, 3.2190) -- (1.8220, 2.0400, 3.2160) -- (1.7760, 2.0400, 3.2180) -- cycle;
\fill[blue!62.4, opacity=0.7] (1.7760, 2.0400, 3.2180) -- (1.8220, 2.0400, 3.2160) -- (1.8220, 2.0940, 3.2127) -- (1.7760, 2.0940, 3.2148) -- cycle;
\fill[blue!62.7, opacity=0.7] (1.7760, 2.0940, 3.2148) -- (1.8220, 2.0940, 3.2127) -- (1.8220, 2.1480, 3.2091) -- (1.7760, 2.1480, 3.2112) -- cycle;
\fill[blue!47.1, opacity=0.7] (1.7760, 2.1480, 3.2112) -- (1.8220, 2.1480, 3.2091) -- (1.8220, 2.2020, 3.2053) -- (1.7760, 2.2020, 3.2074) -- cycle;
\fill[blue!30.2, opacity=0.7] (1.7760, 2.2020, 3.2074) -- (1.8220, 2.2020, 3.2053) -- (1.8220, 2.2560, 3.2012) -- (1.7760, 2.2560, 3.2033) -- cycle;
\fill[blue!27.3, opacity=0.7] (1.7760, 2.2560, 3.2033) -- (1.8220, 2.2560, 3.2012) -- (1.8220, 2.3100, 3.1969) -- (1.7760, 2.3100, 3.1990) -- cycle;
\fill[blue!39.1, opacity=0.7] (1.7760, 2.3100, 3.1990) -- (1.8220, 2.3100, 3.1969) -- (1.8220, 2.3640, 3.1923) -- (1.7760, 2.3640, 3.1944) -- cycle;
\fill[blue!61.6, opacity=0.7] (1.7760, 2.3640, 3.1944) -- (1.8220, 2.3640, 3.1923) -- (1.8220, 2.4180, 3.1875) -- (1.7760, 2.4180, 3.1896) -- cycle;
\fill[blue!55.0, opacity=0.7] (1.7760, 2.4180, 3.1896) -- (1.8220, 2.4180, 3.1875) -- (1.8220, 2.4720, 3.1826) -- (1.7760, 2.4720, 3.1847) -- cycle;
\fill[blue!40.8, opacity=0.7] (1.7760, 2.4720, 3.1847) -- (1.8220, 2.4720, 3.1826) -- (1.8220, 2.5260, 3.1774) -- (1.7760, 2.5260, 3.1795) -- cycle;
\fill[blue!44.1, opacity=0.7] (1.7760, 2.5260, 3.1795) -- (1.8220, 2.5260, 3.1774) -- (1.8220, 2.5800, 3.1720) -- (1.7760, 2.5800, 3.1741) -- cycle;
\fill[blue!59.8, opacity=0.7] (1.7760, 2.5800, 3.1741) -- (1.8220, 2.5800, 3.1720) -- (1.8220, 2.6340, 3.1665) -- (1.7760, 2.6340, 3.1686) -- cycle;
\fill[blue!61.0, opacity=0.7] (1.7760, 2.6340, 3.1686) -- (1.8220, 2.6340, 3.1665) -- (1.8220, 2.6880, 3.1608) -- (1.7760, 2.6880, 3.1629) -- cycle;
\fill[blue!52.2, opacity=0.7] (1.7760, 2.6880, 3.1629) -- (1.8220, 2.6880, 3.1608) -- (1.8220, 2.7420, 3.1550) -- (1.7760, 2.7420, 3.1571) -- cycle;
\fill[blue!57.1, opacity=0.7] (1.7760, 2.7420, 3.1571) -- (1.8220, 2.7420, 3.1550) -- (1.8220, 2.7960, 3.1491) -- (1.7760, 2.7960, 3.1512) -- cycle;
\fill[blue!62.4, opacity=0.7] (1.7760, 2.7960, 3.1512) -- (1.8220, 2.7960, 3.1491) -- (1.8220, 2.8500, 3.1431) -- (1.7760, 2.8500, 3.1452) -- cycle;
\fill[blue!38.9, opacity=0.7] (1.7760, 2.8500, 3.1452) -- (1.8220, 2.8500, 3.1431) -- (1.8220, 2.9040, 3.1370) -- (1.7760, 2.9040, 3.1391) -- cycle;
\fill[blue!22.6, opacity=0.7] (1.7760, 2.9040, 3.1391) -- (1.8220, 2.9040, 3.1370) -- (1.8220, 2.9580, 3.1308) -- (1.7760, 2.9580, 3.1329) -- cycle;
\fill[blue!20.4, opacity=0.7] (1.7760, 2.9580, 3.1329) -- (1.8220, 2.9580, 3.1308) -- (1.8220, 3.0120, 3.1246) -- (1.7760, 3.0120, 3.1267) -- cycle;
\fill[blue!27.1, opacity=0.7] (1.7760, 3.0120, 3.1267) -- (1.8220, 3.0120, 3.1246) -- (1.8220, 3.0660, 3.1183) -- (1.7760, 3.0660, 3.1204) -- cycle;
\fill[blue!46.4, opacity=0.7] (1.7760, 3.0660, 3.1204) -- (1.8220, 3.0660, 3.1183) -- (1.8220, 3.1200, 3.1120) -- (1.7760, 3.1200, 3.1141) -- cycle;
\fill[blue!63.4, opacity=0.7] (1.8220, -0.1200, 3.1120) -- (1.8680, -0.1200, 3.1096) -- (1.8680, -0.0660, 3.1159) -- (1.8220, -0.0660, 3.1183) -- cycle;
\fill[blue!59.4, opacity=0.7] (1.8220, -0.0660, 3.1183) -- (1.8680, -0.0660, 3.1159) -- (1.8680, -0.0120, 3.1222) -- (1.8220, -0.0120, 3.1246) -- cycle;
\fill[blue!41.6, opacity=0.7] (1.8220, -0.0120, 3.1246) -- (1.8680, -0.0120, 3.1222) -- (1.8680, 0.0420, 3.1284) -- (1.8220, 0.0420, 3.1308) -- cycle;
\fill[blue!26.1, opacity=0.7] (1.8220, 0.0420, 3.1308) -- (1.8680, 0.0420, 3.1284) -- (1.8680, 0.0960, 3.1346) -- (1.8220, 0.0960, 3.1370) -- cycle;
\fill[blue!22.3, opacity=0.7] (1.8220, 0.0960, 3.1370) -- (1.8680, 0.0960, 3.1346) -- (1.8680, 0.1500, 3.1407) -- (1.8220, 0.1500, 3.1431) -- cycle;
\fill[blue!27.9, opacity=0.7] (1.8220, 0.1500, 3.1431) -- (1.8680, 0.1500, 3.1407) -- (1.8680, 0.2040, 3.1467) -- (1.8220, 0.2040, 3.1491) -- cycle;
\fill[blue!48.5, opacity=0.7] (1.8220, 0.2040, 3.1491) -- (1.8680, 0.2040, 3.1467) -- (1.8680, 0.2580, 3.1526) -- (1.8220, 0.2580, 3.1550) -- cycle;
\fill[blue!63.5, opacity=0.7] (1.8220, 0.2580, 3.1550) -- (1.8680, 0.2580, 3.1526) -- (1.8680, 0.3120, 3.1584) -- (1.8220, 0.3120, 3.1608) -- cycle;
\fill[blue!52.4, opacity=0.7] (1.8220, 0.3120, 3.1608) -- (1.8680, 0.3120, 3.1584) -- (1.8680, 0.3660, 3.1641) -- (1.8220, 0.3660, 3.1665) -- cycle;
\fill[blue!46.3, opacity=0.7] (1.8220, 0.3660, 3.1665) -- (1.8680, 0.3660, 3.1641) -- (1.8680, 0.4200, 3.1696) -- (1.8220, 0.4200, 3.1720) -- cycle;
\fill[blue!54.3, opacity=0.7] (1.8220, 0.4200, 3.1720) -- (1.8680, 0.4200, 3.1696) -- (1.8680, 0.4740, 3.1750) -- (1.8220, 0.4740, 3.1774) -- cycle;
\fill[blue!63.5, opacity=0.7] (1.8220, 0.4740, 3.1774) -- (1.8680, 0.4740, 3.1750) -- (1.8680, 0.5280, 3.1802) -- (1.8220, 0.5280, 3.1826) -- cycle;
\fill[blue!56.3, opacity=0.7] (1.8220, 0.5280, 3.1826) -- (1.8680, 0.5280, 3.1802) -- (1.8680, 0.5820, 3.1851) -- (1.8220, 0.5820, 3.1875) -- cycle;
\fill[blue!47.4, opacity=0.7] (1.8220, 0.5820, 3.1875) -- (1.8680, 0.5820, 3.1851) -- (1.8680, 0.6360, 3.1899) -- (1.8220, 0.6360, 3.1923) -- cycle;
\fill[blue!50.5, opacity=0.7] (1.8220, 0.6360, 3.1923) -- (1.8680, 0.6360, 3.1899) -- (1.8680, 0.6900, 3.1945) -- (1.8220, 0.6900, 3.1969) -- cycle;
\fill[blue!61.9, opacity=0.7] (1.8220, 0.6900, 3.1969) -- (1.8680, 0.6900, 3.1945) -- (1.8680, 0.7440, 3.1988) -- (1.8220, 0.7440, 3.2012) -- cycle;
\fill[blue!57.8, opacity=0.7] (1.8220, 0.7440, 3.2012) -- (1.8680, 0.7440, 3.1988) -- (1.8680, 0.7980, 3.2029) -- (1.8220, 0.7980, 3.2053) -- cycle;
\fill[blue!37.0, opacity=0.7] (1.8220, 0.7980, 3.2053) -- (1.8680, 0.7980, 3.2029) -- (1.8680, 0.8520, 3.2067) -- (1.8220, 0.8520, 3.2091) -- cycle;
\fill[blue!24.7, opacity=0.7] (1.8220, 0.8520, 3.2091) -- (1.8680, 0.8520, 3.2067) -- (1.8680, 0.9060, 3.2103) -- (1.8220, 0.9060, 3.2127) -- cycle;
\fill[blue!21.4, opacity=0.7] (1.8220, 0.9060, 3.2127) -- (1.8680, 0.9060, 3.2103) -- (1.8680, 0.9600, 3.2135) -- (1.8220, 0.9600, 3.2160) -- cycle;
\fill[blue!22.8, opacity=0.7] (1.8220, 0.9600, 3.2160) -- (1.8680, 0.9600, 3.2135) -- (1.8680, 1.0140, 3.2165) -- (1.8220, 1.0140, 3.2190) -- cycle;
\fill[blue!28.3, opacity=0.7] (1.8220, 1.0140, 3.2190) -- (1.8680, 1.0140, 3.2165) -- (1.8680, 1.0680, 3.2193) -- (1.8220, 1.0680, 3.2217) -- cycle;
\fill[blue!37.9, opacity=0.7] (1.8220, 1.0680, 3.2217) -- (1.8680, 1.0680, 3.2193) -- (1.8680, 1.1220, 3.2217) -- (1.8220, 1.1220, 3.2241) -- cycle;
\fill[blue!48.1, opacity=0.7] (1.8220, 1.1220, 3.2241) -- (1.8680, 1.1220, 3.2217) -- (1.8680, 1.1760, 3.2238) -- (1.8220, 1.1760, 3.2262) -- cycle;
\fill[blue!55.0, opacity=0.7] (1.8220, 1.1760, 3.2262) -- (1.8680, 1.1760, 3.2238) -- (1.8680, 1.2300, 3.2255) -- (1.8220, 1.2300, 3.2279) -- cycle;
\fill[blue!58.3, opacity=0.7] (1.8220, 1.2300, 3.2279) -- (1.8680, 1.2300, 3.2255) -- (1.8680, 1.2840, 3.2270) -- (1.8220, 1.2840, 3.2294) -- cycle;
\fill[blue!59.4, opacity=0.7] (1.8220, 1.2840, 3.2294) -- (1.8680, 1.2840, 3.2270) -- (1.8680, 1.3380, 3.2281) -- (1.8220, 1.3380, 3.2306) -- cycle;
\fill[blue!59.2, opacity=0.7] (1.8220, 1.3380, 3.2306) -- (1.8680, 1.3380, 3.2281) -- (1.8680, 1.3920, 3.2290) -- (1.8220, 1.3920, 3.2314) -- cycle;
\fill[blue!57.8, opacity=0.7] (1.8220, 1.3920, 3.2314) -- (1.8680, 1.3920, 3.2290) -- (1.8680, 1.4460, 3.2295) -- (1.8220, 1.4460, 3.2319) -- cycle;
\fill[blue!54.5, opacity=0.7] (1.8220, 1.4460, 3.2319) -- (1.8680, 1.4460, 3.2295) -- (1.8680, 1.5000, 3.2296) -- (1.8220, 1.5000, 3.2320) -- cycle;
\fill[blue!47.6, opacity=0.7] (1.8220, 1.5000, 3.2320) -- (1.8680, 1.5000, 3.2296) -- (1.8680, 1.5540, 3.2295) -- (1.8220, 1.5540, 3.2319) -- cycle;
\fill[blue!36.5, opacity=0.7] (1.8220, 1.5540, 3.2319) -- (1.8680, 1.5540, 3.2295) -- (1.8680, 1.6080, 3.2290) -- (1.8220, 1.6080, 3.2314) -- cycle;
\fill[blue!25.3, opacity=0.7] (1.8220, 1.6080, 3.2314) -- (1.8680, 1.6080, 3.2290) -- (1.8680, 1.6620, 3.2281) -- (1.8220, 1.6620, 3.2306) -- cycle;
\fill[blue!19.1, opacity=0.7] (1.8220, 1.6620, 3.2306) -- (1.8680, 1.6620, 3.2281) -- (1.8680, 1.7160, 3.2270) -- (1.8220, 1.7160, 3.2294) -- cycle;
\fill[blue!17.3, opacity=0.7] (1.8220, 1.7160, 3.2294) -- (1.8680, 1.7160, 3.2270) -- (1.8680, 1.7700, 3.2255) -- (1.8220, 1.7700, 3.2279) -- cycle;
\fill[blue!18.1, opacity=0.7] (1.8220, 1.7700, 3.2279) -- (1.8680, 1.7700, 3.2255) -- (1.8680, 1.8240, 3.2238) -- (1.8220, 1.8240, 3.2262) -- cycle;
\fill[blue!24.5, opacity=0.7] (1.8220, 1.8240, 3.2262) -- (1.8680, 1.8240, 3.2238) -- (1.8680, 1.8780, 3.2217) -- (1.8220, 1.8780, 3.2241) -- cycle;
\fill[blue!44.8, opacity=0.7] (1.8220, 1.8780, 3.2241) -- (1.8680, 1.8780, 3.2217) -- (1.8680, 1.9320, 3.2193) -- (1.8220, 1.9320, 3.2217) -- cycle;
\fill[blue!62.6, opacity=0.7] (1.8220, 1.9320, 3.2217) -- (1.8680, 1.9320, 3.2193) -- (1.8680, 1.9860, 3.2165) -- (1.8220, 1.9860, 3.2190) -- cycle;
\fill[blue!62.2, opacity=0.7] (1.8220, 1.9860, 3.2190) -- (1.8680, 1.9860, 3.2165) -- (1.8680, 2.0400, 3.2135) -- (1.8220, 2.0400, 3.2160) -- cycle;
\fill[blue!63.0, opacity=0.7] (1.8220, 2.0400, 3.2160) -- (1.8680, 2.0400, 3.2135) -- (1.8680, 2.0940, 3.2103) -- (1.8220, 2.0940, 3.2127) -- cycle;
\fill[blue!59.7, opacity=0.7] (1.8220, 2.0940, 3.2127) -- (1.8680, 2.0940, 3.2103) -- (1.8680, 2.1480, 3.2067) -- (1.8220, 2.1480, 3.2091) -- cycle;
\fill[blue!40.7, opacity=0.7] (1.8220, 2.1480, 3.2091) -- (1.8680, 2.1480, 3.2067) -- (1.8680, 2.2020, 3.2029) -- (1.8220, 2.2020, 3.2053) -- cycle;
\fill[blue!28.3, opacity=0.7] (1.8220, 2.2020, 3.2053) -- (1.8680, 2.2020, 3.2029) -- (1.8680, 2.2560, 3.1988) -- (1.8220, 2.2560, 3.2012) -- cycle;
\fill[blue!29.4, opacity=0.7] (1.8220, 2.2560, 3.2012) -- (1.8680, 2.2560, 3.1988) -- (1.8680, 2.3100, 3.1945) -- (1.8220, 2.3100, 3.1969) -- cycle;
\fill[blue!46.0, opacity=0.7] (1.8220, 2.3100, 3.1969) -- (1.8680, 2.3100, 3.1945) -- (1.8680, 2.3640, 3.1899) -- (1.8220, 2.3640, 3.1923) -- cycle;
\fill[blue!63.6, opacity=0.7] (1.8220, 2.3640, 3.1923) -- (1.8680, 2.3640, 3.1899) -- (1.8680, 2.4180, 3.1851) -- (1.8220, 2.4180, 3.1875) -- cycle;
\fill[blue!49.9, opacity=0.7] (1.8220, 2.4180, 3.1875) -- (1.8680, 2.4180, 3.1851) -- (1.8680, 2.4720, 3.1802) -- (1.8220, 2.4720, 3.1826) -- cycle;
\fill[blue!39.5, opacity=0.7] (1.8220, 2.4720, 3.1826) -- (1.8680, 2.4720, 3.1802) -- (1.8680, 2.5260, 3.1750) -- (1.8220, 2.5260, 3.1774) -- cycle;
\fill[blue!46.6, opacity=0.7] (1.8220, 2.5260, 3.1774) -- (1.8680, 2.5260, 3.1750) -- (1.8680, 2.5800, 3.1696) -- (1.8220, 2.5800, 3.1720) -- cycle;
\fill[blue!62.0, opacity=0.7] (1.8220, 2.5800, 3.1720) -- (1.8680, 2.5800, 3.1696) -- (1.8680, 2.6340, 3.1641) -- (1.8220, 2.6340, 3.1665) -- cycle;
\fill[blue!59.3, opacity=0.7] (1.8220, 2.6340, 3.1665) -- (1.8680, 2.6340, 3.1641) -- (1.8680, 2.6880, 3.1584) -- (1.8220, 2.6880, 3.1608) -- cycle;
\fill[blue!52.2, opacity=0.7] (1.8220, 2.6880, 3.1608) -- (1.8680, 2.6880, 3.1584) -- (1.8680, 2.7420, 3.1526) -- (1.8220, 2.7420, 3.1550) -- cycle;
\fill[blue!59.5, opacity=0.7] (1.8220, 2.7420, 3.1550) -- (1.8680, 2.7420, 3.1526) -- (1.8680, 2.7960, 3.1467) -- (1.8220, 2.7960, 3.1491) -- cycle;
\fill[blue!59.9, opacity=0.7] (1.8220, 2.7960, 3.1491) -- (1.8680, 2.7960, 3.1467) -- (1.8680, 2.8500, 3.1407) -- (1.8220, 2.8500, 3.1431) -- cycle;
\fill[blue!34.4, opacity=0.7] (1.8220, 2.8500, 3.1431) -- (1.8680, 2.8500, 3.1407) -- (1.8680, 2.9040, 3.1346) -- (1.8220, 2.9040, 3.1370) -- cycle;
\fill[blue!21.5, opacity=0.7] (1.8220, 2.9040, 3.1370) -- (1.8680, 2.9040, 3.1346) -- (1.8680, 2.9580, 3.1284) -- (1.8220, 2.9580, 3.1308) -- cycle;
\fill[blue!20.6, opacity=0.7] (1.8220, 2.9580, 3.1308) -- (1.8680, 2.9580, 3.1284) -- (1.8680, 3.0120, 3.1222) -- (1.8220, 3.0120, 3.1246) -- cycle;
\fill[blue!29.0, opacity=0.7] (1.8220, 3.0120, 3.1246) -- (1.8680, 3.0120, 3.1222) -- (1.8680, 3.0660, 3.1159) -- (1.8220, 3.0660, 3.1183) -- cycle;
\fill[blue!49.2, opacity=0.7] (1.8220, 3.0660, 3.1183) -- (1.8680, 3.0660, 3.1159) -- (1.8680, 3.1200, 3.1096) -- (1.8220, 3.1200, 3.1120) -- cycle;
\fill[blue!63.4, opacity=0.7] (1.8680, -0.1200, 3.1096) -- (1.9140, -0.1200, 3.1069) -- (1.9140, -0.0660, 3.1132) -- (1.8680, -0.0660, 3.1159) -- cycle;
\fill[blue!62.3, opacity=0.7] (1.8680, -0.0660, 3.1159) -- (1.9140, -0.0660, 3.1132) -- (1.9140, -0.0120, 3.1195) -- (1.8680, -0.0120, 3.1222) -- cycle;
\fill[blue!50.1, opacity=0.7] (1.8680, -0.0120, 3.1222) -- (1.9140, -0.0120, 3.1195) -- (1.9140, 0.0420, 3.1257) -- (1.8680, 0.0420, 3.1284) -- cycle;
\fill[blue!31.1, opacity=0.7] (1.8680, 0.0420, 3.1284) -- (1.9140, 0.0420, 3.1257) -- (1.9140, 0.0960, 3.1319) -- (1.8680, 0.0960, 3.1346) -- cycle;
\fill[blue!22.9, opacity=0.7] (1.8680, 0.0960, 3.1346) -- (1.9140, 0.0960, 3.1319) -- (1.9140, 0.1500, 3.1380) -- (1.8680, 0.1500, 3.1407) -- cycle;
\fill[blue!24.2, opacity=0.7] (1.8680, 0.1500, 3.1407) -- (1.9140, 0.1500, 3.1380) -- (1.9140, 0.2040, 3.1440) -- (1.8680, 0.2040, 3.1467) -- cycle;
\fill[blue!37.5, opacity=0.7] (1.8680, 0.2040, 3.1467) -- (1.9140, 0.2040, 3.1440) -- (1.9140, 0.2580, 3.1499) -- (1.8680, 0.2580, 3.1526) -- cycle;
\fill[blue!60.3, opacity=0.7] (1.8680, 0.2580, 3.1526) -- (1.9140, 0.2580, 3.1499) -- (1.9140, 0.3120, 3.1557) -- (1.8680, 0.3120, 3.1584) -- cycle;
\fill[blue!59.1, opacity=0.7] (1.8680, 0.3120, 3.1584) -- (1.9140, 0.3120, 3.1557) -- (1.9140, 0.3660, 3.1614) -- (1.8680, 0.3660, 3.1641) -- cycle;
\fill[blue!47.2, opacity=0.7] (1.8680, 0.3660, 3.1641) -- (1.9140, 0.3660, 3.1614) -- (1.9140, 0.4200, 3.1669) -- (1.8680, 0.4200, 3.1696) -- cycle;
\fill[blue!47.9, opacity=0.7] (1.8680, 0.4200, 3.1696) -- (1.9140, 0.4200, 3.1669) -- (1.9140, 0.4740, 3.1723) -- (1.8680, 0.4740, 3.1750) -- cycle;
\fill[blue!59.1, opacity=0.7] (1.8680, 0.4740, 3.1750) -- (1.9140, 0.4740, 3.1723) -- (1.9140, 0.5280, 3.1775) -- (1.8680, 0.5280, 3.1802) -- cycle;
\fill[blue!62.8, opacity=0.7] (1.8680, 0.5280, 3.1802) -- (1.9140, 0.5280, 3.1775) -- (1.9140, 0.5820, 3.1824) -- (1.8680, 0.5820, 3.1851) -- cycle;
\fill[blue!52.9, opacity=0.7] (1.8680, 0.5820, 3.1851) -- (1.9140, 0.5820, 3.1824) -- (1.9140, 0.6360, 3.1872) -- (1.8680, 0.6360, 3.1899) -- cycle;
\fill[blue!47.5, opacity=0.7] (1.8680, 0.6360, 3.1899) -- (1.9140, 0.6360, 3.1872) -- (1.9140, 0.6900, 3.1918) -- (1.8680, 0.6900, 3.1945) -- cycle;
\fill[blue!53.7, opacity=0.7] (1.8680, 0.6900, 3.1945) -- (1.9140, 0.6900, 3.1918) -- (1.9140, 0.7440, 3.1961) -- (1.8680, 0.7440, 3.1988) -- cycle;
\fill[blue!63.3, opacity=0.7] (1.8680, 0.7440, 3.1988) -- (1.9140, 0.7440, 3.1961) -- (1.9140, 0.7980, 3.2002) -- (1.8680, 0.7980, 3.2029) -- cycle;
\fill[blue!55.2, opacity=0.7] (1.8680, 0.7980, 3.2029) -- (1.9140, 0.7980, 3.2002) -- (1.9140, 0.8520, 3.2040) -- (1.8680, 0.8520, 3.2067) -- cycle;
\fill[blue!35.7, opacity=0.7] (1.8680, 0.8520, 3.2067) -- (1.9140, 0.8520, 3.2040) -- (1.9140, 0.9060, 3.2076) -- (1.8680, 0.9060, 3.2103) -- cycle;
\fill[blue!24.6, opacity=0.7] (1.8680, 0.9060, 3.2103) -- (1.9140, 0.9060, 3.2076) -- (1.9140, 0.9600, 3.2108) -- (1.8680, 0.9600, 3.2135) -- cycle;
\fill[blue!21.2, opacity=0.7] (1.8680, 0.9600, 3.2135) -- (1.9140, 0.9600, 3.2108) -- (1.9140, 1.0140, 3.2138) -- (1.8680, 1.0140, 3.2165) -- cycle;
\fill[blue!21.3, opacity=0.7] (1.8680, 1.0140, 3.2165) -- (1.9140, 1.0140, 3.2138) -- (1.9140, 1.0680, 3.2165) -- (1.8680, 1.0680, 3.2193) -- cycle;
\fill[blue!23.9, opacity=0.7] (1.8680, 1.0680, 3.2193) -- (1.9140, 1.0680, 3.2165) -- (1.9140, 1.1220, 3.2190) -- (1.8680, 1.1220, 3.2217) -- cycle;
\fill[blue!28.6, opacity=0.7] (1.8680, 1.1220, 3.2217) -- (1.9140, 1.1220, 3.2190) -- (1.9140, 1.1760, 3.2210) -- (1.8680, 1.1760, 3.2238) -- cycle;
\fill[blue!34.4, opacity=0.7] (1.8680, 1.1760, 3.2238) -- (1.9140, 1.1760, 3.2210) -- (1.9140, 1.2300, 3.2228) -- (1.8680, 1.2300, 3.2255) -- cycle;
\fill[blue!39.4, opacity=0.7] (1.8680, 1.2300, 3.2255) -- (1.9140, 1.2300, 3.2228) -- (1.9140, 1.2840, 3.2243) -- (1.8680, 1.2840, 3.2270) -- cycle;
\fill[blue!42.3, opacity=0.7] (1.8680, 1.2840, 3.2270) -- (1.9140, 1.2840, 3.2243) -- (1.9140, 1.3380, 3.2254) -- (1.8680, 1.3380, 3.2281) -- cycle;
\fill[blue!42.5, opacity=0.7] (1.8680, 1.3380, 3.2281) -- (1.9140, 1.3380, 3.2254) -- (1.9140, 1.3920, 3.2263) -- (1.8680, 1.3920, 3.2290) -- cycle;
\fill[blue!40.0, opacity=0.7] (1.8680, 1.3920, 3.2290) -- (1.9140, 1.3920, 3.2263) -- (1.9140, 1.4460, 3.2268) -- (1.8680, 1.4460, 3.2295) -- cycle;
\fill[blue!34.9, opacity=0.7] (1.8680, 1.4460, 3.2295) -- (1.9140, 1.4460, 3.2268) -- (1.9140, 1.5000, 3.2269) -- (1.8680, 1.5000, 3.2296) -- cycle;
\fill[blue!28.3, opacity=0.7] (1.8680, 1.5000, 3.2296) -- (1.9140, 1.5000, 3.2269) -- (1.9140, 1.5540, 3.2268) -- (1.8680, 1.5540, 3.2295) -- cycle;
\fill[blue!22.3, opacity=0.7] (1.8680, 1.5540, 3.2295) -- (1.9140, 1.5540, 3.2268) -- (1.9140, 1.6080, 3.2263) -- (1.8680, 1.6080, 3.2290) -- cycle;
\fill[blue!18.8, opacity=0.7] (1.8680, 1.6080, 3.2290) -- (1.9140, 1.6080, 3.2263) -- (1.9140, 1.6620, 3.2254) -- (1.8680, 1.6620, 3.2281) -- cycle;
\fill[blue!17.5, opacity=0.7] (1.8680, 1.6620, 3.2281) -- (1.9140, 1.6620, 3.2254) -- (1.9140, 1.7160, 3.2243) -- (1.8680, 1.7160, 3.2270) -- cycle;
\fill[blue!18.2, opacity=0.7] (1.8680, 1.7160, 3.2270) -- (1.9140, 1.7160, 3.2243) -- (1.9140, 1.7700, 3.2228) -- (1.8680, 1.7700, 3.2255) -- cycle;
\fill[blue!22.9, opacity=0.7] (1.8680, 1.7700, 3.2255) -- (1.9140, 1.7700, 3.2228) -- (1.9140, 1.8240, 3.2210) -- (1.8680, 1.8240, 3.2238) -- cycle;
\fill[blue!38.5, opacity=0.7] (1.8680, 1.8240, 3.2238) -- (1.9140, 1.8240, 3.2210) -- (1.9140, 1.8780, 3.2190) -- (1.8680, 1.8780, 3.2217) -- cycle;
\fill[blue!59.2, opacity=0.7] (1.8680, 1.8780, 3.2217) -- (1.9140, 1.8780, 3.2190) -- (1.9140, 1.9320, 3.2165) -- (1.8680, 1.9320, 3.2193) -- cycle;
\fill[blue!63.1, opacity=0.7] (1.8680, 1.9320, 3.2193) -- (1.9140, 1.9320, 3.2165) -- (1.9140, 1.9860, 3.2138) -- (1.8680, 1.9860, 3.2165) -- cycle;
\fill[blue!61.6, opacity=0.7] (1.8680, 1.9860, 3.2165) -- (1.9140, 1.9860, 3.2138) -- (1.9140, 2.0400, 3.2108) -- (1.8680, 2.0400, 3.2135) -- cycle;
\fill[blue!63.5, opacity=0.7] (1.8680, 2.0400, 3.2135) -- (1.9140, 2.0400, 3.2108) -- (1.9140, 2.0940, 3.2076) -- (1.8680, 2.0940, 3.2103) -- cycle;
\fill[blue!52.9, opacity=0.7] (1.8680, 2.0940, 3.2103) -- (1.9140, 2.0940, 3.2076) -- (1.9140, 2.1480, 3.2040) -- (1.8680, 2.1480, 3.2067) -- cycle;
\fill[blue!34.5, opacity=0.7] (1.8680, 2.1480, 3.2067) -- (1.9140, 2.1480, 3.2040) -- (1.9140, 2.2020, 3.2002) -- (1.8680, 2.2020, 3.2029) -- cycle;
\fill[blue!27.6, opacity=0.7] (1.8680, 2.2020, 3.2029) -- (1.9140, 2.2020, 3.2002) -- (1.9140, 2.2560, 3.1961) -- (1.8680, 2.2560, 3.1988) -- cycle;
\fill[blue!33.7, opacity=0.7] (1.8680, 2.2560, 3.1988) -- (1.9140, 2.2560, 3.1961) -- (1.9140, 2.3100, 3.1918) -- (1.8680, 2.3100, 3.1945) -- cycle;
\fill[blue!54.7, opacity=0.7] (1.8680, 2.3100, 3.1945) -- (1.9140, 2.3100, 3.1918) -- (1.9140, 2.3640, 3.1872) -- (1.8680, 2.3640, 3.1899) -- cycle;
\fill[blue!61.5, opacity=0.7] (1.8680, 2.3640, 3.1899) -- (1.9140, 2.3640, 3.1872) -- (1.9140, 2.4180, 3.1824) -- (1.8680, 2.4180, 3.1851) -- cycle;
\fill[blue!44.6, opacity=0.7] (1.8680, 2.4180, 3.1851) -- (1.9140, 2.4180, 3.1824) -- (1.9140, 2.4720, 3.1775) -- (1.8680, 2.4720, 3.1802) -- cycle;
\fill[blue!39.3, opacity=0.7] (1.8680, 2.4720, 3.1802) -- (1.9140, 2.4720, 3.1775) -- (1.9140, 2.5260, 3.1723) -- (1.8680, 2.5260, 3.1750) -- cycle;
\fill[blue!50.5, opacity=0.7] (1.8680, 2.5260, 3.1750) -- (1.9140, 2.5260, 3.1723) -- (1.9140, 2.5800, 3.1669) -- (1.8680, 2.5800, 3.1696) -- cycle;
\fill[blue!63.4, opacity=0.7] (1.8680, 2.5800, 3.1696) -- (1.9140, 2.5800, 3.1669) -- (1.9140, 2.6340, 3.1614) -- (1.8680, 2.6340, 3.1641) -- cycle;
\fill[blue!57.0, opacity=0.7] (1.8680, 2.6340, 3.1641) -- (1.9140, 2.6340, 3.1614) -- (1.9140, 2.6880, 3.1557) -- (1.8680, 2.6880, 3.1584) -- cycle;
\fill[blue!53.0, opacity=0.7] (1.8680, 2.6880, 3.1584) -- (1.9140, 2.6880, 3.1557) -- (1.9140, 2.7420, 3.1499) -- (1.8680, 2.7420, 3.1526) -- cycle;
\fill[blue!62.0, opacity=0.7] (1.8680, 2.7420, 3.1526) -- (1.9140, 2.7420, 3.1499) -- (1.9140, 2.7960, 3.1440) -- (1.8680, 2.7960, 3.1467) -- cycle;
\fill[blue!55.1, opacity=0.7] (1.8680, 2.7960, 3.1467) -- (1.9140, 2.7960, 3.1440) -- (1.9140, 2.8500, 3.1380) -- (1.8680, 2.8500, 3.1407) -- cycle;
\fill[blue!29.9, opacity=0.7] (1.8680, 2.8500, 3.1407) -- (1.9140, 2.8500, 3.1380) -- (1.9140, 2.9040, 3.1319) -- (1.8680, 2.9040, 3.1346) -- cycle;
\fill[blue!20.5, opacity=0.7] (1.8680, 2.9040, 3.1346) -- (1.9140, 2.9040, 3.1319) -- (1.9140, 2.9580, 3.1257) -- (1.8680, 2.9580, 3.1284) -- cycle;
\fill[blue!21.2, opacity=0.7] (1.8680, 2.9580, 3.1284) -- (1.9140, 2.9580, 3.1257) -- (1.9140, 3.0120, 3.1195) -- (1.8680, 3.0120, 3.1222) -- cycle;
\fill[blue!31.8, opacity=0.7] (1.8680, 3.0120, 3.1222) -- (1.9140, 3.0120, 3.1195) -- (1.9140, 3.0660, 3.1132) -- (1.8680, 3.0660, 3.1159) -- cycle;
\fill[blue!52.4, opacity=0.7] (1.8680, 3.0660, 3.1159) -- (1.9140, 3.0660, 3.1132) -- (1.9140, 3.1200, 3.1069) -- (1.8680, 3.1200, 3.1096) -- cycle;
\fill[blue!62.9, opacity=0.7] (1.9140, -0.1200, 3.1069) -- (1.9600, -0.1200, 3.1039) -- (1.9600, -0.0660, 3.1102) -- (1.9140, -0.0660, 3.1132) -- cycle;
\fill[blue!63.4, opacity=0.7] (1.9140, -0.0660, 3.1132) -- (1.9600, -0.0660, 3.1102) -- (1.9600, -0.0120, 3.1165) -- (1.9140, -0.0120, 3.1195) -- cycle;
\fill[blue!57.6, opacity=0.7] (1.9140, -0.0120, 3.1195) -- (1.9600, -0.0120, 3.1165) -- (1.9600, 0.0420, 3.1227) -- (1.9140, 0.0420, 3.1257) -- cycle;
\fill[blue!39.1, opacity=0.7] (1.9140, 0.0420, 3.1257) -- (1.9600, 0.0420, 3.1227) -- (1.9600, 0.0960, 3.1289) -- (1.9140, 0.0960, 3.1319) -- cycle;
\fill[blue!25.5, opacity=0.7] (1.9140, 0.0960, 3.1319) -- (1.9600, 0.0960, 3.1289) -- (1.9600, 0.1500, 3.1350) -- (1.9140, 0.1500, 3.1380) -- cycle;
\fill[blue!22.7, opacity=0.7] (1.9140, 0.1500, 3.1380) -- (1.9600, 0.1500, 3.1350) -- (1.9600, 0.2040, 3.1410) -- (1.9140, 0.2040, 3.1440) -- cycle;
\fill[blue!29.1, opacity=0.7] (1.9140, 0.2040, 3.1440) -- (1.9600, 0.2040, 3.1410) -- (1.9600, 0.2580, 3.1469) -- (1.9140, 0.2580, 3.1499) -- cycle;
\fill[blue!49.4, opacity=0.7] (1.9140, 0.2580, 3.1499) -- (1.9600, 0.2580, 3.1469) -- (1.9600, 0.3120, 3.1527) -- (1.9140, 0.3120, 3.1557) -- cycle;
\fill[blue!63.5, opacity=0.7] (1.9140, 0.3120, 3.1557) -- (1.9600, 0.3120, 3.1527) -- (1.9600, 0.3660, 3.1584) -- (1.9140, 0.3660, 3.1614) -- cycle;
\fill[blue!52.8, opacity=0.7] (1.9140, 0.3660, 3.1614) -- (1.9600, 0.3660, 3.1584) -- (1.9600, 0.4200, 3.1639) -- (1.9140, 0.4200, 3.1669) -- cycle;
\fill[blue!45.3, opacity=0.7] (1.9140, 0.4200, 3.1669) -- (1.9600, 0.4200, 3.1639) -- (1.9600, 0.4740, 3.1693) -- (1.9140, 0.4740, 3.1723) -- cycle;
\fill[blue!50.8, opacity=0.7] (1.9140, 0.4740, 3.1723) -- (1.9600, 0.4740, 3.1693) -- (1.9600, 0.5280, 3.1745) -- (1.9140, 0.5280, 3.1775) -- cycle;
\fill[blue!61.9, opacity=0.7] (1.9140, 0.5280, 3.1775) -- (1.9600, 0.5280, 3.1745) -- (1.9600, 0.5820, 3.1794) -- (1.9140, 0.5820, 3.1824) -- cycle;
\fill[blue!61.0, opacity=0.7] (1.9140, 0.5820, 3.1824) -- (1.9600, 0.5820, 3.1794) -- (1.9600, 0.6360, 3.1842) -- (1.9140, 0.6360, 3.1872) -- cycle;
\fill[blue!51.4, opacity=0.7] (1.9140, 0.6360, 3.1872) -- (1.9600, 0.6360, 3.1842) -- (1.9600, 0.6900, 3.1888) -- (1.9140, 0.6900, 3.1918) -- cycle;
\fill[blue!48.4, opacity=0.7] (1.9140, 0.6900, 3.1918) -- (1.9600, 0.6900, 3.1888) -- (1.9600, 0.7440, 3.1931) -- (1.9140, 0.7440, 3.1961) -- cycle;
\fill[blue!55.4, opacity=0.7] (1.9140, 0.7440, 3.1961) -- (1.9600, 0.7440, 3.1931) -- (1.9600, 0.7980, 3.1972) -- (1.9140, 0.7980, 3.2002) -- cycle;
\fill[blue!63.5, opacity=0.7] (1.9140, 0.7980, 3.2002) -- (1.9600, 0.7980, 3.1972) -- (1.9600, 0.8520, 3.2010) -- (1.9140, 0.8520, 3.2040) -- cycle;
\fill[blue!55.6, opacity=0.7] (1.9140, 0.8520, 3.2040) -- (1.9600, 0.8520, 3.2010) -- (1.9600, 0.9060, 3.2046) -- (1.9140, 0.9060, 3.2076) -- cycle;
\fill[blue!37.9, opacity=0.7] (1.9140, 0.9060, 3.2076) -- (1.9600, 0.9060, 3.2046) -- (1.9600, 0.9600, 3.2078) -- (1.9140, 0.9600, 3.2108) -- cycle;
\fill[blue!26.2, opacity=0.7] (1.9140, 0.9600, 3.2108) -- (1.9600, 0.9600, 3.2078) -- (1.9600, 1.0140, 3.2108) -- (1.9140, 1.0140, 3.2138) -- cycle;
\fill[blue!21.6, opacity=0.7] (1.9140, 1.0140, 3.2138) -- (1.9600, 1.0140, 3.2108) -- (1.9600, 1.0680, 3.2135) -- (1.9140, 1.0680, 3.2165) -- cycle;
\fill[blue!20.3, opacity=0.7] (1.9140, 1.0680, 3.2165) -- (1.9600, 1.0680, 3.2135) -- (1.9600, 1.1220, 3.2160) -- (1.9140, 1.1220, 3.2190) -- cycle;
\fill[blue!20.6, opacity=0.7] (1.9140, 1.1220, 3.2190) -- (1.9600, 1.1220, 3.2160) -- (1.9600, 1.1760, 3.2180) -- (1.9140, 1.1760, 3.2210) -- cycle;
\fill[blue!21.8, opacity=0.7] (1.9140, 1.1760, 3.2210) -- (1.9600, 1.1760, 3.2180) -- (1.9600, 1.2300, 3.2198) -- (1.9140, 1.2300, 3.2228) -- cycle;
\fill[blue!23.1, opacity=0.7] (1.9140, 1.2300, 3.2228) -- (1.9600, 1.2300, 3.2198) -- (1.9600, 1.2840, 3.2213) -- (1.9140, 1.2840, 3.2243) -- cycle;
\fill[blue!24.0, opacity=0.7] (1.9140, 1.2840, 3.2243) -- (1.9600, 1.2840, 3.2213) -- (1.9600, 1.3380, 3.2224) -- (1.9140, 1.3380, 3.2254) -- cycle;
\fill[blue!24.0, opacity=0.7] (1.9140, 1.3380, 3.2254) -- (1.9600, 1.3380, 3.2224) -- (1.9600, 1.3920, 3.2233) -- (1.9140, 1.3920, 3.2263) -- cycle;
\fill[blue!22.9, opacity=0.7] (1.9140, 1.3920, 3.2263) -- (1.9600, 1.3920, 3.2233) -- (1.9600, 1.4460, 3.2238) -- (1.9140, 1.4460, 3.2268) -- cycle;
\fill[blue!21.1, opacity=0.7] (1.9140, 1.4460, 3.2268) -- (1.9600, 1.4460, 3.2238) -- (1.9600, 1.5000, 3.2239) -- (1.9140, 1.5000, 3.2269) -- cycle;
\fill[blue!19.3, opacity=0.7] (1.9140, 1.5000, 3.2269) -- (1.9600, 1.5000, 3.2239) -- (1.9600, 1.5540, 3.2238) -- (1.9140, 1.5540, 3.2268) -- cycle;
\fill[blue!18.1, opacity=0.7] (1.9140, 1.5540, 3.2268) -- (1.9600, 1.5540, 3.2238) -- (1.9600, 1.6080, 3.2233) -- (1.9140, 1.6080, 3.2263) -- cycle;
\fill[blue!17.9, opacity=0.7] (1.9140, 1.6080, 3.2263) -- (1.9600, 1.6080, 3.2233) -- (1.9600, 1.6620, 3.2224) -- (1.9140, 1.6620, 3.2254) -- cycle;
\fill[blue!19.1, opacity=0.7] (1.9140, 1.6620, 3.2254) -- (1.9600, 1.6620, 3.2224) -- (1.9600, 1.7160, 3.2213) -- (1.9140, 1.7160, 3.2243) -- cycle;
\fill[blue!24.0, opacity=0.7] (1.9140, 1.7160, 3.2243) -- (1.9600, 1.7160, 3.2213) -- (1.9600, 1.7700, 3.2198) -- (1.9140, 1.7700, 3.2228) -- cycle;
\fill[blue!37.8, opacity=0.7] (1.9140, 1.7700, 3.2228) -- (1.9600, 1.7700, 3.2198) -- (1.9600, 1.8240, 3.2180) -- (1.9140, 1.8240, 3.2210) -- cycle;
\fill[blue!57.2, opacity=0.7] (1.9140, 1.8240, 3.2210) -- (1.9600, 1.8240, 3.2180) -- (1.9600, 1.8780, 3.2160) -- (1.9140, 1.8780, 3.2190) -- cycle;
\fill[blue!63.5, opacity=0.7] (1.9140, 1.8780, 3.2190) -- (1.9600, 1.8780, 3.2160) -- (1.9600, 1.9320, 3.2135) -- (1.9140, 1.9320, 3.2165) -- cycle;
\fill[blue!61.1, opacity=0.7] (1.9140, 1.9320, 3.2165) -- (1.9600, 1.9320, 3.2135) -- (1.9600, 1.9860, 3.2108) -- (1.9140, 1.9860, 3.2138) -- cycle;
\fill[blue!62.9, opacity=0.7] (1.9140, 1.9860, 3.2138) -- (1.9600, 1.9860, 3.2108) -- (1.9600, 2.0400, 3.2078) -- (1.9140, 2.0400, 3.2108) -- cycle;
\fill[blue!60.4, opacity=0.7] (1.9140, 2.0400, 3.2108) -- (1.9600, 2.0400, 3.2078) -- (1.9600, 2.0940, 3.2046) -- (1.9140, 2.0940, 3.2076) -- cycle;
\fill[blue!43.3, opacity=0.7] (1.9140, 2.0940, 3.2076) -- (1.9600, 2.0940, 3.2046) -- (1.9600, 2.1480, 3.2010) -- (1.9140, 2.1480, 3.2040) -- cycle;
\fill[blue!30.1, opacity=0.7] (1.9140, 2.1480, 3.2040) -- (1.9600, 2.1480, 3.2010) -- (1.9600, 2.2020, 3.1972) -- (1.9140, 2.2020, 3.2002) -- cycle;
\fill[blue!29.0, opacity=0.7] (1.9140, 2.2020, 3.2002) -- (1.9600, 2.2020, 3.1972) -- (1.9600, 2.2560, 3.1931) -- (1.9140, 2.2560, 3.1961) -- cycle;
\fill[blue!41.6, opacity=0.7] (1.9140, 2.2560, 3.1961) -- (1.9600, 2.2560, 3.1931) -- (1.9600, 2.3100, 3.1888) -- (1.9140, 2.3100, 3.1918) -- cycle;
\fill[blue!62.1, opacity=0.7] (1.9140, 2.3100, 3.1918) -- (1.9600, 2.3100, 3.1888) -- (1.9600, 2.3640, 3.1842) -- (1.9140, 2.3640, 3.1872) -- cycle;
\fill[blue!55.1, opacity=0.7] (1.9140, 2.3640, 3.1872) -- (1.9600, 2.3640, 3.1842) -- (1.9600, 2.4180, 3.1794) -- (1.9140, 2.4180, 3.1824) -- cycle;
\fill[blue!40.2, opacity=0.7] (1.9140, 2.4180, 3.1824) -- (1.9600, 2.4180, 3.1794) -- (1.9600, 2.4720, 3.1745) -- (1.9140, 2.4720, 3.1775) -- cycle;
\fill[blue!40.9, opacity=0.7] (1.9140, 2.4720, 3.1775) -- (1.9600, 2.4720, 3.1745) -- (1.9600, 2.5260, 3.1693) -- (1.9140, 2.5260, 3.1723) -- cycle;
\fill[blue!55.6, opacity=0.7] (1.9140, 2.5260, 3.1723) -- (1.9600, 2.5260, 3.1693) -- (1.9600, 2.5800, 3.1639) -- (1.9140, 2.5800, 3.1669) -- cycle;
\fill[blue!63.2, opacity=0.7] (1.9140, 2.5800, 3.1669) -- (1.9600, 2.5800, 3.1639) -- (1.9600, 2.6340, 3.1584) -- (1.9140, 2.6340, 3.1614) -- cycle;
\fill[blue!54.8, opacity=0.7] (1.9140, 2.6340, 3.1614) -- (1.9600, 2.6340, 3.1584) -- (1.9600, 2.6880, 3.1527) -- (1.9140, 2.6880, 3.1557) -- cycle;
\fill[blue!55.0, opacity=0.7] (1.9140, 2.6880, 3.1557) -- (1.9600, 2.6880, 3.1527) -- (1.9600, 2.7420, 3.1469) -- (1.9140, 2.7420, 3.1499) -- cycle;
\fill[blue!63.5, opacity=0.7] (1.9140, 2.7420, 3.1499) -- (1.9600, 2.7420, 3.1469) -- (1.9600, 2.7960, 3.1410) -- (1.9140, 2.7960, 3.1440) -- cycle;
\fill[blue!48.0, opacity=0.7] (1.9140, 2.7960, 3.1440) -- (1.9600, 2.7960, 3.1410) -- (1.9600, 2.8500, 3.1350) -- (1.9140, 2.8500, 3.1380) -- cycle;
\fill[blue!25.8, opacity=0.7] (1.9140, 2.8500, 3.1380) -- (1.9600, 2.8500, 3.1350) -- (1.9600, 2.9040, 3.1289) -- (1.9140, 2.9040, 3.1319) -- cycle;
\fill[blue!19.9, opacity=0.7] (1.9140, 2.9040, 3.1319) -- (1.9600, 2.9040, 3.1289) -- (1.9600, 2.9580, 3.1227) -- (1.9140, 2.9580, 3.1257) -- cycle;
\fill[blue!22.3, opacity=0.7] (1.9140, 2.9580, 3.1257) -- (1.9600, 2.9580, 3.1227) -- (1.9600, 3.0120, 3.1165) -- (1.9140, 3.0120, 3.1195) -- cycle;
\fill[blue!35.7, opacity=0.7] (1.9140, 3.0120, 3.1195) -- (1.9600, 3.0120, 3.1165) -- (1.9600, 3.0660, 3.1102) -- (1.9140, 3.0660, 3.1132) -- cycle;
\fill[blue!55.5, opacity=0.7] (1.9140, 3.0660, 3.1132) -- (1.9600, 3.0660, 3.1102) -- (1.9600, 3.1200, 3.1039) -- (1.9140, 3.1200, 3.1069) -- cycle;
\fill[blue!60.4, opacity=0.7] (1.9600, -0.1200, 3.1039) -- (2.0060, -0.1200, 3.1006) -- (2.0060, -0.0660, 3.1069) -- (1.9600, -0.0660, 3.1102) -- cycle;
\fill[blue!63.5, opacity=0.7] (1.9600, -0.0660, 3.1102) -- (2.0060, -0.0660, 3.1069) -- (2.0060, -0.0120, 3.1132) -- (1.9600, -0.0120, 3.1165) -- cycle;
\fill[blue!62.0, opacity=0.7] (1.9600, -0.0120, 3.1165) -- (2.0060, -0.0120, 3.1132) -- (2.0060, 0.0420, 3.1194) -- (1.9600, 0.0420, 3.1227) -- cycle;
\fill[blue!49.2, opacity=0.7] (1.9600, 0.0420, 3.1227) -- (2.0060, 0.0420, 3.1194) -- (2.0060, 0.0960, 3.1256) -- (1.9600, 0.0960, 3.1289) -- cycle;
\fill[blue!31.1, opacity=0.7] (1.9600, 0.0960, 3.1289) -- (2.0060, 0.0960, 3.1256) -- (2.0060, 0.1500, 3.1317) -- (1.9600, 0.1500, 3.1350) -- cycle;
\fill[blue!23.3, opacity=0.7] (1.9600, 0.1500, 3.1350) -- (2.0060, 0.1500, 3.1317) -- (2.0060, 0.2040, 3.1377) -- (1.9600, 0.2040, 3.1410) -- cycle;
\fill[blue!24.4, opacity=0.7] (1.9600, 0.2040, 3.1410) -- (2.0060, 0.2040, 3.1377) -- (2.0060, 0.2580, 3.1436) -- (1.9600, 0.2580, 3.1469) -- cycle;
\fill[blue!36.5, opacity=0.7] (1.9600, 0.2580, 3.1469) -- (2.0060, 0.2580, 3.1436) -- (2.0060, 0.3120, 3.1494) -- (1.9600, 0.3120, 3.1527) -- cycle;
\fill[blue!58.5, opacity=0.7] (1.9600, 0.3120, 3.1527) -- (2.0060, 0.3120, 3.1494) -- (2.0060, 0.3660, 3.1551) -- (1.9600, 0.3660, 3.1584) -- cycle;
\fill[blue!60.9, opacity=0.7] (1.9600, 0.3660, 3.1584) -- (2.0060, 0.3660, 3.1551) -- (2.0060, 0.4200, 3.1606) -- (1.9600, 0.4200, 3.1639) -- cycle;
\fill[blue!48.4, opacity=0.7] (1.9600, 0.4200, 3.1639) -- (2.0060, 0.4200, 3.1606) -- (2.0060, 0.4740, 3.1660) -- (1.9600, 0.4740, 3.1693) -- cycle;
\fill[blue!45.1, opacity=0.7] (1.9600, 0.4740, 3.1693) -- (2.0060, 0.4740, 3.1660) -- (2.0060, 0.5280, 3.1712) -- (1.9600, 0.5280, 3.1745) -- cycle;
\fill[blue!53.2, opacity=0.7] (1.9600, 0.5280, 3.1745) -- (2.0060, 0.5280, 3.1712) -- (2.0060, 0.5820, 3.1762) -- (1.9600, 0.5820, 3.1794) -- cycle;
\fill[blue!62.9, opacity=0.7] (1.9600, 0.5820, 3.1794) -- (2.0060, 0.5820, 3.1762) -- (2.0060, 0.6360, 3.1809) -- (1.9600, 0.6360, 3.1842) -- cycle;
\fill[blue!60.0, opacity=0.7] (1.9600, 0.6360, 3.1842) -- (2.0060, 0.6360, 3.1809) -- (2.0060, 0.6900, 3.1855) -- (1.9600, 0.6900, 3.1888) -- cycle;
\fill[blue!51.3, opacity=0.7] (1.9600, 0.6900, 3.1888) -- (2.0060, 0.6900, 3.1855) -- (2.0060, 0.7440, 3.1898) -- (1.9600, 0.7440, 3.1931) -- cycle;
\fill[blue!49.2, opacity=0.7] (1.9600, 0.7440, 3.1931) -- (2.0060, 0.7440, 3.1898) -- (2.0060, 0.7980, 3.1939) -- (1.9600, 0.7980, 3.1972) -- cycle;
\fill[blue!55.5, opacity=0.7] (1.9600, 0.7980, 3.1972) -- (2.0060, 0.7980, 3.1939) -- (2.0060, 0.8520, 3.1977) -- (1.9600, 0.8520, 3.2010) -- cycle;
\fill[blue!63.1, opacity=0.7] (1.9600, 0.8520, 3.2010) -- (2.0060, 0.8520, 3.1977) -- (2.0060, 0.9060, 3.2013) -- (1.9600, 0.9060, 3.2046) -- cycle;
\fill[blue!59.1, opacity=0.7] (1.9600, 0.9060, 3.2046) -- (2.0060, 0.9060, 3.2013) -- (2.0060, 0.9600, 3.2046) -- (1.9600, 0.9600, 3.2078) -- cycle;
\fill[blue!44.2, opacity=0.7] (1.9600, 0.9600, 3.2078) -- (2.0060, 0.9600, 3.2046) -- (2.0060, 1.0140, 3.2076) -- (1.9600, 1.0140, 3.2108) -- cycle;
\fill[blue!31.2, opacity=0.7] (1.9600, 1.0140, 3.2108) -- (2.0060, 1.0140, 3.2076) -- (2.0060, 1.0680, 3.2103) -- (1.9600, 1.0680, 3.2135) -- cycle;
\fill[blue!24.3, opacity=0.7] (1.9600, 1.0680, 3.2135) -- (2.0060, 1.0680, 3.2103) -- (2.0060, 1.1220, 3.2127) -- (1.9600, 1.1220, 3.2160) -- cycle;
\fill[blue!21.2, opacity=0.7] (1.9600, 1.1220, 3.2160) -- (2.0060, 1.1220, 3.2127) -- (2.0060, 1.1760, 3.2148) -- (1.9600, 1.1760, 3.2180) -- cycle;
\fill[blue!20.0, opacity=0.7] (1.9600, 1.1760, 3.2180) -- (2.0060, 1.1760, 3.2148) -- (2.0060, 1.2300, 3.2166) -- (1.9600, 1.2300, 3.2198) -- cycle;
\fill[blue!19.5, opacity=0.7] (1.9600, 1.2300, 3.2198) -- (2.0060, 1.2300, 3.2166) -- (2.0060, 1.2840, 3.2180) -- (1.9600, 1.2840, 3.2213) -- cycle;
\fill[blue!19.2, opacity=0.7] (1.9600, 1.2840, 3.2213) -- (2.0060, 1.2840, 3.2180) -- (2.0060, 1.3380, 3.2192) -- (1.9600, 1.3380, 3.2224) -- cycle;
\fill[blue!19.0, opacity=0.7] (1.9600, 1.3380, 3.2224) -- (2.0060, 1.3380, 3.2192) -- (2.0060, 1.3920, 3.2200) -- (1.9600, 1.3920, 3.2233) -- cycle;
\fill[blue!18.7, opacity=0.7] (1.9600, 1.3920, 3.2233) -- (2.0060, 1.3920, 3.2200) -- (2.0060, 1.4460, 3.2205) -- (1.9600, 1.4460, 3.2238) -- cycle;
\fill[blue!18.5, opacity=0.7] (1.9600, 1.4460, 3.2238) -- (2.0060, 1.4460, 3.2205) -- (2.0060, 1.5000, 3.2206) -- (1.9600, 1.5000, 3.2239) -- cycle;
\fill[blue!18.5, opacity=0.7] (1.9600, 1.5000, 3.2239) -- (2.0060, 1.5000, 3.2206) -- (2.0060, 1.5540, 3.2205) -- (1.9600, 1.5540, 3.2238) -- cycle;
\fill[blue!19.3, opacity=0.7] (1.9600, 1.5540, 3.2238) -- (2.0060, 1.5540, 3.2205) -- (2.0060, 1.6080, 3.2200) -- (1.9600, 1.6080, 3.2233) -- cycle;
\fill[blue!21.8, opacity=0.7] (1.9600, 1.6080, 3.2233) -- (2.0060, 1.6080, 3.2200) -- (2.0060, 1.6620, 3.2192) -- (1.9600, 1.6620, 3.2224) -- cycle;
\fill[blue!28.6, opacity=0.7] (1.9600, 1.6620, 3.2224) -- (2.0060, 1.6620, 3.2192) -- (2.0060, 1.7160, 3.2180) -- (1.9600, 1.7160, 3.2213) -- cycle;
\fill[blue!42.5, opacity=0.7] (1.9600, 1.7160, 3.2213) -- (2.0060, 1.7160, 3.2180) -- (2.0060, 1.7700, 3.2166) -- (1.9600, 1.7700, 3.2198) -- cycle;
\fill[blue!58.5, opacity=0.7] (1.9600, 1.7700, 3.2198) -- (2.0060, 1.7700, 3.2166) -- (2.0060, 1.8240, 3.2148) -- (1.9600, 1.8240, 3.2180) -- cycle;
\fill[blue!63.5, opacity=0.7] (1.9600, 1.8240, 3.2180) -- (2.0060, 1.8240, 3.2148) -- (2.0060, 1.8780, 3.2127) -- (1.9600, 1.8780, 3.2160) -- cycle;
\fill[blue!60.9, opacity=0.7] (1.9600, 1.8780, 3.2160) -- (2.0060, 1.8780, 3.2127) -- (2.0060, 1.9320, 3.2103) -- (1.9600, 1.9320, 3.2135) -- cycle;
\fill[blue!61.8, opacity=0.7] (1.9600, 1.9320, 3.2135) -- (2.0060, 1.9320, 3.2103) -- (2.0060, 1.9860, 3.2076) -- (1.9600, 1.9860, 3.2108) -- cycle;
\fill[blue!63.0, opacity=0.7] (1.9600, 1.9860, 3.2108) -- (2.0060, 1.9860, 3.2076) -- (2.0060, 2.0400, 3.2046) -- (1.9600, 2.0400, 3.2078) -- cycle;
\fill[blue!51.0, opacity=0.7] (1.9600, 2.0400, 3.2078) -- (2.0060, 2.0400, 3.2046) -- (2.0060, 2.0940, 3.2013) -- (1.9600, 2.0940, 3.2046) -- cycle;
\fill[blue!34.6, opacity=0.7] (1.9600, 2.0940, 3.2046) -- (2.0060, 2.0940, 3.2013) -- (2.0060, 2.1480, 3.1977) -- (1.9600, 2.1480, 3.2010) -- cycle;
\fill[blue!28.5, opacity=0.7] (1.9600, 2.1480, 3.2010) -- (2.0060, 2.1480, 3.1977) -- (2.0060, 2.2020, 3.1939) -- (1.9600, 2.2020, 3.1972) -- cycle;
\fill[blue!34.0, opacity=0.7] (1.9600, 2.2020, 3.1972) -- (2.0060, 2.2020, 3.1939) -- (2.0060, 2.2560, 3.1898) -- (1.9600, 2.2560, 3.1931) -- cycle;
\fill[blue!53.1, opacity=0.7] (1.9600, 2.2560, 3.1931) -- (2.0060, 2.2560, 3.1898) -- (2.0060, 2.3100, 3.1855) -- (1.9600, 2.3100, 3.1888) -- cycle;
\fill[blue!62.8, opacity=0.7] (1.9600, 2.3100, 3.1888) -- (2.0060, 2.3100, 3.1855) -- (2.0060, 2.3640, 3.1809) -- (1.9600, 2.3640, 3.1842) -- cycle;
\fill[blue!46.8, opacity=0.7] (1.9600, 2.3640, 3.1842) -- (2.0060, 2.3640, 3.1809) -- (2.0060, 2.4180, 3.1762) -- (1.9600, 2.4180, 3.1794) -- cycle;
\fill[blue!38.0, opacity=0.7] (1.9600, 2.4180, 3.1794) -- (2.0060, 2.4180, 3.1762) -- (2.0060, 2.4720, 3.1712) -- (1.9600, 2.4720, 3.1745) -- cycle;
\fill[blue!44.9, opacity=0.7] (1.9600, 2.4720, 3.1745) -- (2.0060, 2.4720, 3.1712) -- (2.0060, 2.5260, 3.1660) -- (1.9600, 2.5260, 3.1693) -- cycle;
\fill[blue!60.6, opacity=0.7] (1.9600, 2.5260, 3.1693) -- (2.0060, 2.5260, 3.1660) -- (2.0060, 2.5800, 3.1606) -- (1.9600, 2.5800, 3.1639) -- cycle;
\fill[blue!61.0, opacity=0.7] (1.9600, 2.5800, 3.1639) -- (2.0060, 2.5800, 3.1606) -- (2.0060, 2.6340, 3.1551) -- (1.9600, 2.6340, 3.1584) -- cycle;
\fill[blue!53.5, opacity=0.7] (1.9600, 2.6340, 3.1584) -- (2.0060, 2.6340, 3.1551) -- (2.0060, 2.6880, 3.1494) -- (1.9600, 2.6880, 3.1527) -- cycle;
\fill[blue!58.3, opacity=0.7] (1.9600, 2.6880, 3.1527) -- (2.0060, 2.6880, 3.1494) -- (2.0060, 2.7420, 3.1436) -- (1.9600, 2.7420, 3.1469) -- cycle;
\fill[blue!62.3, opacity=0.7] (1.9600, 2.7420, 3.1469) -- (2.0060, 2.7420, 3.1436) -- (2.0060, 2.7960, 3.1377) -- (1.9600, 2.7960, 3.1410) -- cycle;
\fill[blue!39.5, opacity=0.7] (1.9600, 2.7960, 3.1410) -- (2.0060, 2.7960, 3.1377) -- (2.0060, 2.8500, 3.1317) -- (1.9600, 2.8500, 3.1350) -- cycle;
\fill[blue!22.7, opacity=0.7] (1.9600, 2.8500, 3.1350) -- (2.0060, 2.8500, 3.1317) -- (2.0060, 2.9040, 3.1256) -- (1.9600, 2.9040, 3.1289) -- cycle;
\fill[blue!19.7, opacity=0.7] (1.9600, 2.9040, 3.1289) -- (2.0060, 2.9040, 3.1256) -- (2.0060, 2.9580, 3.1194) -- (1.9600, 2.9580, 3.1227) -- cycle;
\fill[blue!24.3, opacity=0.7] (1.9600, 2.9580, 3.1227) -- (2.0060, 2.9580, 3.1194) -- (2.0060, 3.0120, 3.1132) -- (1.9600, 3.0120, 3.1165) -- cycle;
\fill[blue!40.8, opacity=0.7] (1.9600, 3.0120, 3.1165) -- (2.0060, 3.0120, 3.1132) -- (2.0060, 3.0660, 3.1069) -- (1.9600, 3.0660, 3.1102) -- cycle;
\fill[blue!58.3, opacity=0.7] (1.9600, 3.0660, 3.1102) -- (2.0060, 3.0660, 3.1069) -- (2.0060, 3.1200, 3.1006) -- (1.9600, 3.1200, 3.1039) -- cycle;
\fill[blue!53.9, opacity=0.7] (2.0060, -0.1200, 3.1006) -- (2.0520, -0.1200, 3.0971) -- (2.0520, -0.0660, 3.1034) -- (2.0060, -0.0660, 3.1069) -- cycle;
\fill[blue!63.2, opacity=0.7] (2.0060, -0.0660, 3.1069) -- (2.0520, -0.0660, 3.1034) -- (2.0520, -0.0120, 3.1096) -- (2.0060, -0.0120, 3.1132) -- cycle;
\fill[blue!63.4, opacity=0.7] (2.0060, -0.0120, 3.1132) -- (2.0520, -0.0120, 3.1096) -- (2.0520, 0.0420, 3.1159) -- (2.0060, 0.0420, 3.1194) -- cycle;
\fill[blue!58.1, opacity=0.7] (2.0060, 0.0420, 3.1194) -- (2.0520, 0.0420, 3.1159) -- (2.0520, 0.0960, 3.1220) -- (2.0060, 0.0960, 3.1256) -- cycle;
\fill[blue!40.6, opacity=0.7] (2.0060, 0.0960, 3.1256) -- (2.0520, 0.0960, 3.1220) -- (2.0520, 0.1500, 3.1281) -- (2.0060, 0.1500, 3.1317) -- cycle;
\fill[blue!26.6, opacity=0.7] (2.0060, 0.1500, 3.1317) -- (2.0520, 0.1500, 3.1281) -- (2.0520, 0.2040, 3.1342) -- (2.0060, 0.2040, 3.1377) -- cycle;
\fill[blue!23.0, opacity=0.7] (2.0060, 0.2040, 3.1377) -- (2.0520, 0.2040, 3.1342) -- (2.0520, 0.2580, 3.1401) -- (2.0060, 0.2580, 3.1436) -- cycle;
\fill[blue!27.6, opacity=0.7] (2.0060, 0.2580, 3.1436) -- (2.0520, 0.2580, 3.1401) -- (2.0520, 0.3120, 3.1459) -- (2.0060, 0.3120, 3.1494) -- cycle;
\fill[blue!44.6, opacity=0.7] (2.0060, 0.3120, 3.1494) -- (2.0520, 0.3120, 3.1459) -- (2.0520, 0.3660, 3.1516) -- (2.0060, 0.3660, 3.1551) -- cycle;
\fill[blue!62.8, opacity=0.7] (2.0060, 0.3660, 3.1551) -- (2.0520, 0.3660, 3.1516) -- (2.0520, 0.4200, 3.1571) -- (2.0060, 0.4200, 3.1606) -- cycle;
\fill[blue!57.0, opacity=0.7] (2.0060, 0.4200, 3.1606) -- (2.0520, 0.4200, 3.1571) -- (2.0520, 0.4740, 3.1624) -- (2.0060, 0.4740, 3.1660) -- cycle;
\fill[blue!45.9, opacity=0.7] (2.0060, 0.4740, 3.1660) -- (2.0520, 0.4740, 3.1624) -- (2.0520, 0.5280, 3.1676) -- (2.0060, 0.5280, 3.1712) -- cycle;
\fill[blue!45.4, opacity=0.7] (2.0060, 0.5280, 3.1712) -- (2.0520, 0.5280, 3.1676) -- (2.0520, 0.5820, 3.1726) -- (2.0060, 0.5820, 3.1762) -- cycle;
\fill[blue!54.4, opacity=0.7] (2.0060, 0.5820, 3.1762) -- (2.0520, 0.5820, 3.1726) -- (2.0520, 0.6360, 3.1774) -- (2.0060, 0.6360, 3.1809) -- cycle;
\fill[blue!63.1, opacity=0.7] (2.0060, 0.6360, 3.1809) -- (2.0520, 0.6360, 3.1774) -- (2.0520, 0.6900, 3.1819) -- (2.0060, 0.6900, 3.1855) -- cycle;
\fill[blue!60.3, opacity=0.7] (2.0060, 0.6900, 3.1855) -- (2.0520, 0.6900, 3.1819) -- (2.0520, 0.7440, 3.1863) -- (2.0060, 0.7440, 3.1898) -- cycle;
\fill[blue!52.4, opacity=0.7] (2.0060, 0.7440, 3.1898) -- (2.0520, 0.7440, 3.1863) -- (2.0520, 0.7980, 3.1903) -- (2.0060, 0.7980, 3.1939) -- cycle;
\fill[blue!49.7, opacity=0.7] (2.0060, 0.7980, 3.1939) -- (2.0520, 0.7980, 3.1903) -- (2.0520, 0.8520, 3.1942) -- (2.0060, 0.8520, 3.1977) -- cycle;
\fill[blue!54.1, opacity=0.7] (2.0060, 0.8520, 3.1977) -- (2.0520, 0.8520, 3.1942) -- (2.0520, 0.9060, 3.1977) -- (2.0060, 0.9060, 3.2013) -- cycle;
\fill[blue!61.4, opacity=0.7] (2.0060, 0.9060, 3.2013) -- (2.0520, 0.9060, 3.1977) -- (2.0520, 0.9600, 3.2010) -- (2.0060, 0.9600, 3.2046) -- cycle;
\fill[blue!63.0, opacity=0.7] (2.0060, 0.9600, 3.2046) -- (2.0520, 0.9600, 3.2010) -- (2.0520, 1.0140, 3.2040) -- (2.0060, 1.0140, 3.2076) -- cycle;
\fill[blue!54.6, opacity=0.7] (2.0060, 1.0140, 3.2076) -- (2.0520, 1.0140, 3.2040) -- (2.0520, 1.0680, 3.2067) -- (2.0060, 1.0680, 3.2103) -- cycle;
\fill[blue!42.4, opacity=0.7] (2.0060, 1.0680, 3.2103) -- (2.0520, 1.0680, 3.2067) -- (2.0520, 1.1220, 3.2091) -- (2.0060, 1.1220, 3.2127) -- cycle;
\fill[blue!32.9, opacity=0.7] (2.0060, 1.1220, 3.2127) -- (2.0520, 1.1220, 3.2091) -- (2.0520, 1.1760, 3.2112) -- (2.0060, 1.1760, 3.2148) -- cycle;
\fill[blue!27.2, opacity=0.7] (2.0060, 1.1760, 3.2148) -- (2.0520, 1.1760, 3.2112) -- (2.0520, 1.2300, 3.2130) -- (2.0060, 1.2300, 3.2166) -- cycle;
\fill[blue!24.1, opacity=0.7] (2.0060, 1.2300, 3.2166) -- (2.0520, 1.2300, 3.2130) -- (2.0520, 1.2840, 3.2145) -- (2.0060, 1.2840, 3.2180) -- cycle;
\fill[blue!22.5, opacity=0.7] (2.0060, 1.2840, 3.2180) -- (2.0520, 1.2840, 3.2145) -- (2.0520, 1.3380, 3.2156) -- (2.0060, 1.3380, 3.2192) -- cycle;
\fill[blue!21.9, opacity=0.7] (2.0060, 1.3380, 3.2192) -- (2.0520, 1.3380, 3.2156) -- (2.0520, 1.3920, 3.2164) -- (2.0060, 1.3920, 3.2200) -- cycle;
\fill[blue!21.9, opacity=0.7] (2.0060, 1.3920, 3.2200) -- (2.0520, 1.3920, 3.2164) -- (2.0520, 1.4460, 3.2169) -- (2.0060, 1.4460, 3.2205) -- cycle;
\fill[blue!22.9, opacity=0.7] (2.0060, 1.4460, 3.2205) -- (2.0520, 1.4460, 3.2169) -- (2.0520, 1.5000, 3.2171) -- (2.0060, 1.5000, 3.2206) -- cycle;
\fill[blue!25.3, opacity=0.7] (2.0060, 1.5000, 3.2206) -- (2.0520, 1.5000, 3.2171) -- (2.0520, 1.5540, 3.2169) -- (2.0060, 1.5540, 3.2205) -- cycle;
\fill[blue!30.3, opacity=0.7] (2.0060, 1.5540, 3.2205) -- (2.0520, 1.5540, 3.2169) -- (2.0520, 1.6080, 3.2164) -- (2.0060, 1.6080, 3.2200) -- cycle;
\fill[blue!39.5, opacity=0.7] (2.0060, 1.6080, 3.2200) -- (2.0520, 1.6080, 3.2164) -- (2.0520, 1.6620, 3.2156) -- (2.0060, 1.6620, 3.2192) -- cycle;
\fill[blue!52.3, opacity=0.7] (2.0060, 1.6620, 3.2192) -- (2.0520, 1.6620, 3.2156) -- (2.0520, 1.7160, 3.2145) -- (2.0060, 1.7160, 3.2180) -- cycle;
\fill[blue!62.0, opacity=0.7] (2.0060, 1.7160, 3.2180) -- (2.0520, 1.7160, 3.2145) -- (2.0520, 1.7700, 3.2130) -- (2.0060, 1.7700, 3.2166) -- cycle;
\fill[blue!63.0, opacity=0.7] (2.0060, 1.7700, 3.2166) -- (2.0520, 1.7700, 3.2130) -- (2.0520, 1.8240, 3.2112) -- (2.0060, 1.8240, 3.2148) -- cycle;
\fill[blue!60.3, opacity=0.7] (2.0060, 1.8240, 3.2148) -- (2.0520, 1.8240, 3.2112) -- (2.0520, 1.8780, 3.2091) -- (2.0060, 1.8780, 3.2127) -- cycle;
\fill[blue!61.2, opacity=0.7] (2.0060, 1.8780, 3.2127) -- (2.0520, 1.8780, 3.2091) -- (2.0520, 1.9320, 3.2067) -- (2.0060, 1.9320, 3.2103) -- cycle;
\fill[blue!63.5, opacity=0.7] (2.0060, 1.9320, 3.2103) -- (2.0520, 1.9320, 3.2067) -- (2.0520, 1.9860, 3.2040) -- (2.0060, 1.9860, 3.2076) -- cycle;
\fill[blue!55.5, opacity=0.7] (2.0060, 1.9860, 3.2076) -- (2.0520, 1.9860, 3.2040) -- (2.0520, 2.0400, 3.2010) -- (2.0060, 2.0400, 3.2046) -- cycle;
\fill[blue!39.2, opacity=0.7] (2.0060, 2.0400, 3.2046) -- (2.0520, 2.0400, 3.2010) -- (2.0520, 2.0940, 3.1977) -- (2.0060, 2.0940, 3.2013) -- cycle;
\fill[blue!29.8, opacity=0.7] (2.0060, 2.0940, 3.2013) -- (2.0520, 2.0940, 3.1977) -- (2.0520, 2.1480, 3.1942) -- (2.0060, 2.1480, 3.1977) -- cycle;
\fill[blue!30.9, opacity=0.7] (2.0060, 2.1480, 3.1977) -- (2.0520, 2.1480, 3.1942) -- (2.0520, 2.2020, 3.1903) -- (2.0060, 2.2020, 3.1939) -- cycle;
\fill[blue!44.3, opacity=0.7] (2.0060, 2.2020, 3.1939) -- (2.0520, 2.2020, 3.1903) -- (2.0520, 2.2560, 3.1863) -- (2.0060, 2.2560, 3.1898) -- cycle;
\fill[blue!62.5, opacity=0.7] (2.0060, 2.2560, 3.1898) -- (2.0520, 2.2560, 3.1863) -- (2.0520, 2.3100, 3.1819) -- (2.0060, 2.3100, 3.1855) -- cycle;
\fill[blue!55.2, opacity=0.7] (2.0060, 2.3100, 3.1855) -- (2.0520, 2.3100, 3.1819) -- (2.0520, 2.3640, 3.1774) -- (2.0060, 2.3640, 3.1809) -- cycle;
\fill[blue!40.1, opacity=0.7] (2.0060, 2.3640, 3.1809) -- (2.0520, 2.3640, 3.1774) -- (2.0520, 2.4180, 3.1726) -- (2.0060, 2.4180, 3.1762) -- cycle;
\fill[blue!38.8, opacity=0.7] (2.0060, 2.4180, 3.1762) -- (2.0520, 2.4180, 3.1726) -- (2.0520, 2.4720, 3.1676) -- (2.0060, 2.4720, 3.1712) -- cycle;
\fill[blue!51.5, opacity=0.7] (2.0060, 2.4720, 3.1712) -- (2.0520, 2.4720, 3.1676) -- (2.0520, 2.5260, 3.1624) -- (2.0060, 2.5260, 3.1660) -- cycle;
\fill[blue!63.4, opacity=0.7] (2.0060, 2.5260, 3.1660) -- (2.0520, 2.5260, 3.1624) -- (2.0520, 2.5800, 3.1571) -- (2.0060, 2.5800, 3.1606) -- cycle;
\fill[blue!57.7, opacity=0.7] (2.0060, 2.5800, 3.1606) -- (2.0520, 2.5800, 3.1571) -- (2.0520, 2.6340, 3.1516) -- (2.0060, 2.6340, 3.1551) -- cycle;
\fill[blue!54.0, opacity=0.7] (2.0060, 2.6340, 3.1551) -- (2.0520, 2.6340, 3.1516) -- (2.0520, 2.6880, 3.1459) -- (2.0060, 2.6880, 3.1494) -- cycle;
\fill[blue!62.0, opacity=0.7] (2.0060, 2.6880, 3.1494) -- (2.0520, 2.6880, 3.1459) -- (2.0520, 2.7420, 3.1401) -- (2.0060, 2.7420, 3.1436) -- cycle;
\fill[blue!56.4, opacity=0.7] (2.0060, 2.7420, 3.1436) -- (2.0520, 2.7420, 3.1401) -- (2.0520, 2.7960, 3.1342) -- (2.0060, 2.7960, 3.1377) -- cycle;
\fill[blue!31.5, opacity=0.7] (2.0060, 2.7960, 3.1377) -- (2.0520, 2.7960, 3.1342) -- (2.0520, 2.8500, 3.1281) -- (2.0060, 2.8500, 3.1317) -- cycle;
\fill[blue!20.7, opacity=0.7] (2.0060, 2.8500, 3.1317) -- (2.0520, 2.8500, 3.1281) -- (2.0520, 2.9040, 3.1220) -- (2.0060, 2.9040, 3.1256) -- cycle;
\fill[blue!20.1, opacity=0.7] (2.0060, 2.9040, 3.1256) -- (2.0520, 2.9040, 3.1220) -- (2.0520, 2.9580, 3.1159) -- (2.0060, 2.9580, 3.1194) -- cycle;
\fill[blue!27.6, opacity=0.7] (2.0060, 2.9580, 3.1194) -- (2.0520, 2.9580, 3.1159) -- (2.0520, 3.0120, 3.1096) -- (2.0060, 3.0120, 3.1132) -- cycle;
\fill[blue!46.6, opacity=0.7] (2.0060, 3.0120, 3.1132) -- (2.0520, 3.0120, 3.1096) -- (2.0520, 3.0660, 3.1034) -- (2.0060, 3.0660, 3.1069) -- cycle;
\fill[blue!60.3, opacity=0.7] (2.0060, 3.0660, 3.1069) -- (2.0520, 3.0660, 3.1034) -- (2.0520, 3.1200, 3.0971) -- (2.0060, 3.1200, 3.1006) -- cycle;
\fill[blue!42.4, opacity=0.7] (2.0520, -0.1200, 3.0971) -- (2.0980, -0.1200, 3.0933) -- (2.0980, -0.0660, 3.0995) -- (2.0520, -0.0660, 3.1034) -- cycle;
\fill[blue!60.8, opacity=0.7] (2.0520, -0.0660, 3.1034) -- (2.0980, -0.0660, 3.0995) -- (2.0980, -0.0120, 3.1058) -- (2.0520, -0.0120, 3.1096) -- cycle;
\fill[blue!63.6, opacity=0.7] (2.0520, -0.0120, 3.1096) -- (2.0980, -0.0120, 3.1058) -- (2.0980, 0.0420, 3.1120) -- (2.0520, 0.0420, 3.1159) -- cycle;
\fill[blue!62.6, opacity=0.7] (2.0520, 0.0420, 3.1159) -- (2.0980, 0.0420, 3.1120) -- (2.0980, 0.0960, 3.1182) -- (2.0520, 0.0960, 3.1220) -- cycle;
\fill[blue!52.1, opacity=0.7] (2.0520, 0.0960, 3.1220) -- (2.0980, 0.0960, 3.1182) -- (2.0980, 0.1500, 3.1243) -- (2.0520, 0.1500, 3.1281) -- cycle;
\fill[blue!34.2, opacity=0.7] (2.0520, 0.1500, 3.1281) -- (2.0980, 0.1500, 3.1243) -- (2.0980, 0.2040, 3.1303) -- (2.0520, 0.2040, 3.1342) -- cycle;
\fill[blue!24.6, opacity=0.7] (2.0520, 0.2040, 3.1342) -- (2.0980, 0.2040, 3.1303) -- (2.0980, 0.2580, 3.1363) -- (2.0520, 0.2580, 3.1401) -- cycle;
\fill[blue!23.8, opacity=0.7] (2.0520, 0.2580, 3.1401) -- (2.0980, 0.2580, 3.1363) -- (2.0980, 0.3120, 3.1421) -- (2.0520, 0.3120, 3.1459) -- cycle;
\fill[blue!31.6, opacity=0.7] (2.0520, 0.3120, 3.1459) -- (2.0980, 0.3120, 3.1421) -- (2.0980, 0.3660, 3.1477) -- (2.0520, 0.3660, 3.1516) -- cycle;
\fill[blue!51.2, opacity=0.7] (2.0520, 0.3660, 3.1516) -- (2.0980, 0.3660, 3.1477) -- (2.0980, 0.4200, 3.1533) -- (2.0520, 0.4200, 3.1571) -- cycle;
\fill[blue!63.6, opacity=0.7] (2.0520, 0.4200, 3.1571) -- (2.0980, 0.4200, 3.1533) -- (2.0980, 0.4740, 3.1586) -- (2.0520, 0.4740, 3.1624) -- cycle;
\fill[blue!54.1, opacity=0.7] (2.0520, 0.4740, 3.1624) -- (2.0980, 0.4740, 3.1586) -- (2.0980, 0.5280, 3.1638) -- (2.0520, 0.5280, 3.1676) -- cycle;
\fill[blue!44.6, opacity=0.7] (2.0520, 0.5280, 3.1676) -- (2.0980, 0.5280, 3.1638) -- (2.0980, 0.5820, 3.1688) -- (2.0520, 0.5820, 3.1726) -- cycle;
\fill[blue!45.2, opacity=0.7] (2.0520, 0.5820, 3.1726) -- (2.0980, 0.5820, 3.1688) -- (2.0980, 0.6360, 3.1736) -- (2.0520, 0.6360, 3.1774) -- cycle;
\fill[blue!54.1, opacity=0.7] (2.0520, 0.6360, 3.1774) -- (2.0980, 0.6360, 3.1736) -- (2.0980, 0.6900, 3.1781) -- (2.0520, 0.6900, 3.1819) -- cycle;
\fill[blue!62.7, opacity=0.7] (2.0520, 0.6900, 3.1819) -- (2.0980, 0.6900, 3.1781) -- (2.0980, 0.7440, 3.1824) -- (2.0520, 0.7440, 3.1863) -- cycle;
\fill[blue!61.6, opacity=0.7] (2.0520, 0.7440, 3.1863) -- (2.0980, 0.7440, 3.1824) -- (2.0980, 0.7980, 3.1865) -- (2.0520, 0.7980, 3.1903) -- cycle;
\fill[blue!54.7, opacity=0.7] (2.0520, 0.7980, 3.1903) -- (2.0980, 0.7980, 3.1865) -- (2.0980, 0.8520, 3.1903) -- (2.0520, 0.8520, 3.1942) -- cycle;
\fill[blue!50.7, opacity=0.7] (2.0520, 0.8520, 3.1942) -- (2.0980, 0.8520, 3.1903) -- (2.0980, 0.9060, 3.1939) -- (2.0520, 0.9060, 3.1977) -- cycle;
\fill[blue!52.1, opacity=0.7] (2.0520, 0.9060, 3.1977) -- (2.0980, 0.9060, 3.1939) -- (2.0980, 0.9600, 3.1972) -- (2.0520, 0.9600, 3.2010) -- cycle;
\fill[blue!57.5, opacity=0.7] (2.0520, 0.9600, 3.2010) -- (2.0980, 0.9600, 3.1972) -- (2.0980, 1.0140, 3.2002) -- (2.0520, 1.0140, 3.2040) -- cycle;
\fill[blue!62.6, opacity=0.7] (2.0520, 1.0140, 3.2040) -- (2.0980, 1.0140, 3.2002) -- (2.0980, 1.0680, 3.2029) -- (2.0520, 1.0680, 3.2067) -- cycle;
\fill[blue!63.0, opacity=0.7] (2.0520, 1.0680, 3.2067) -- (2.0980, 1.0680, 3.2029) -- (2.0980, 1.1220, 3.2053) -- (2.0520, 1.1220, 3.2091) -- cycle;
\fill[blue!58.2, opacity=0.7] (2.0520, 1.1220, 3.2091) -- (2.0980, 1.1220, 3.2053) -- (2.0980, 1.1760, 3.2074) -- (2.0520, 1.1760, 3.2112) -- cycle;
\fill[blue!51.4, opacity=0.7] (2.0520, 1.1760, 3.2112) -- (2.0980, 1.1760, 3.2074) -- (2.0980, 1.2300, 3.2092) -- (2.0520, 1.2300, 3.2130) -- cycle;
\fill[blue!45.4, opacity=0.7] (2.0520, 1.2300, 3.2130) -- (2.0980, 1.2300, 3.2092) -- (2.0980, 1.2840, 3.2106) -- (2.0520, 1.2840, 3.2145) -- cycle;
\fill[blue!41.5, opacity=0.7] (2.0520, 1.2840, 3.2145) -- (2.0980, 1.2840, 3.2106) -- (2.0980, 1.3380, 3.2118) -- (2.0520, 1.3380, 3.2156) -- cycle;
\fill[blue!39.9, opacity=0.7] (2.0520, 1.3380, 3.2156) -- (2.0980, 1.3380, 3.2118) -- (2.0980, 1.3920, 3.2126) -- (2.0520, 1.3920, 3.2164) -- cycle;
\fill[blue!40.6, opacity=0.7] (2.0520, 1.3920, 3.2164) -- (2.0980, 1.3920, 3.2126) -- (2.0980, 1.4460, 3.2131) -- (2.0520, 1.4460, 3.2169) -- cycle;
\fill[blue!43.7, opacity=0.7] (2.0520, 1.4460, 3.2169) -- (2.0980, 1.4460, 3.2131) -- (2.0980, 1.5000, 3.2133) -- (2.0520, 1.5000, 3.2171) -- cycle;
\fill[blue!49.2, opacity=0.7] (2.0520, 1.5000, 3.2171) -- (2.0980, 1.5000, 3.2133) -- (2.0980, 1.5540, 3.2131) -- (2.0520, 1.5540, 3.2169) -- cycle;
\fill[blue!56.2, opacity=0.7] (2.0520, 1.5540, 3.2169) -- (2.0980, 1.5540, 3.2131) -- (2.0980, 1.6080, 3.2126) -- (2.0520, 1.6080, 3.2164) -- cycle;
\fill[blue!62.0, opacity=0.7] (2.0520, 1.6080, 3.2164) -- (2.0980, 1.6080, 3.2126) -- (2.0980, 1.6620, 3.2118) -- (2.0520, 1.6620, 3.2156) -- cycle;
\fill[blue!63.5, opacity=0.7] (2.0520, 1.6620, 3.2156) -- (2.0980, 1.6620, 3.2118) -- (2.0980, 1.7160, 3.2106) -- (2.0520, 1.7160, 3.2145) -- cycle;
\fill[blue!61.2, opacity=0.7] (2.0520, 1.7160, 3.2145) -- (2.0980, 1.7160, 3.2106) -- (2.0980, 1.7700, 3.2092) -- (2.0520, 1.7700, 3.2130) -- cycle;
\fill[blue!59.3, opacity=0.7] (2.0520, 1.7700, 3.2130) -- (2.0980, 1.7700, 3.2092) -- (2.0980, 1.8240, 3.2074) -- (2.0520, 1.8240, 3.2112) -- cycle;
\fill[blue!61.1, opacity=0.7] (2.0520, 1.8240, 3.2112) -- (2.0980, 1.8240, 3.2074) -- (2.0980, 1.8780, 3.2053) -- (2.0520, 1.8780, 3.2091) -- cycle;
\fill[blue!63.6, opacity=0.7] (2.0520, 1.8780, 3.2091) -- (2.0980, 1.8780, 3.2053) -- (2.0980, 1.9320, 3.2029) -- (2.0520, 1.9320, 3.2067) -- cycle;
\fill[blue!57.1, opacity=0.7] (2.0520, 1.9320, 3.2067) -- (2.0980, 1.9320, 3.2029) -- (2.0980, 1.9860, 3.2002) -- (2.0520, 1.9860, 3.2040) -- cycle;
\fill[blue!42.1, opacity=0.7] (2.0520, 1.9860, 3.2040) -- (2.0980, 1.9860, 3.2002) -- (2.0980, 2.0400, 3.1972) -- (2.0520, 2.0400, 3.2010) -- cycle;
\fill[blue!31.5, opacity=0.7] (2.0520, 2.0400, 3.2010) -- (2.0980, 2.0400, 3.1972) -- (2.0980, 2.0940, 3.1939) -- (2.0520, 2.0940, 3.1977) -- cycle;
\fill[blue!30.1, opacity=0.7] (2.0520, 2.0940, 3.1977) -- (2.0980, 2.0940, 3.1939) -- (2.0980, 2.1480, 3.1903) -- (2.0520, 2.1480, 3.1942) -- cycle;
\fill[blue!39.2, opacity=0.7] (2.0520, 2.1480, 3.1942) -- (2.0980, 2.1480, 3.1903) -- (2.0980, 2.2020, 3.1865) -- (2.0520, 2.2020, 3.1903) -- cycle;
\fill[blue!58.0, opacity=0.7] (2.0520, 2.2020, 3.1903) -- (2.0980, 2.2020, 3.1865) -- (2.0980, 2.2560, 3.1824) -- (2.0520, 2.2560, 3.1863) -- cycle;
\fill[blue!61.1, opacity=0.7] (2.0520, 2.2560, 3.1863) -- (2.0980, 2.2560, 3.1824) -- (2.0980, 2.3100, 3.1781) -- (2.0520, 2.3100, 3.1819) -- cycle;
\fill[blue!44.8, opacity=0.7] (2.0520, 2.3100, 3.1819) -- (2.0980, 2.3100, 3.1781) -- (2.0980, 2.3640, 3.1736) -- (2.0520, 2.3640, 3.1774) -- cycle;
\fill[blue!37.1, opacity=0.7] (2.0520, 2.3640, 3.1774) -- (2.0980, 2.3640, 3.1736) -- (2.0980, 2.4180, 3.1688) -- (2.0520, 2.4180, 3.1726) -- cycle;
\fill[blue!43.3, opacity=0.7] (2.0520, 2.4180, 3.1726) -- (2.0980, 2.4180, 3.1688) -- (2.0980, 2.4720, 3.1638) -- (2.0520, 2.4720, 3.1676) -- cycle;
\fill[blue!58.9, opacity=0.7] (2.0520, 2.4720, 3.1676) -- (2.0980, 2.4720, 3.1638) -- (2.0980, 2.5260, 3.1586) -- (2.0520, 2.5260, 3.1624) -- cycle;
\fill[blue!62.4, opacity=0.7] (2.0520, 2.5260, 3.1624) -- (2.0980, 2.5260, 3.1586) -- (2.0980, 2.5800, 3.1533) -- (2.0520, 2.5800, 3.1571) -- cycle;
\fill[blue!54.9, opacity=0.7] (2.0520, 2.5800, 3.1571) -- (2.0980, 2.5800, 3.1533) -- (2.0980, 2.6340, 3.1477) -- (2.0520, 2.6340, 3.1516) -- cycle;
\fill[blue!56.9, opacity=0.7] (2.0520, 2.6340, 3.1516) -- (2.0980, 2.6340, 3.1477) -- (2.0980, 2.6880, 3.1421) -- (2.0520, 2.6880, 3.1459) -- cycle;
\fill[blue!63.5, opacity=0.7] (2.0520, 2.6880, 3.1459) -- (2.0980, 2.6880, 3.1421) -- (2.0980, 2.7420, 3.1363) -- (2.0520, 2.7420, 3.1401) -- cycle;
\fill[blue!46.2, opacity=0.7] (2.0520, 2.7420, 3.1401) -- (2.0980, 2.7420, 3.1363) -- (2.0980, 2.7960, 3.1303) -- (2.0520, 2.7960, 3.1342) -- cycle;
\fill[blue!25.4, opacity=0.7] (2.0520, 2.7960, 3.1342) -- (2.0980, 2.7960, 3.1303) -- (2.0980, 2.8500, 3.1243) -- (2.0520, 2.8500, 3.1281) -- cycle;
\fill[blue!19.6, opacity=0.7] (2.0520, 2.8500, 3.1281) -- (2.0980, 2.8500, 3.1243) -- (2.0980, 2.9040, 3.1182) -- (2.0520, 2.9040, 3.1220) -- cycle;
\fill[blue!21.3, opacity=0.7] (2.0520, 2.9040, 3.1220) -- (2.0980, 2.9040, 3.1182) -- (2.0980, 2.9580, 3.1120) -- (2.0520, 2.9580, 3.1159) -- cycle;
\fill[blue!32.6, opacity=0.7] (2.0520, 2.9580, 3.1159) -- (2.0980, 2.9580, 3.1120) -- (2.0980, 3.0120, 3.1058) -- (2.0520, 3.0120, 3.1096) -- cycle;
\fill[blue!52.4, opacity=0.7] (2.0520, 3.0120, 3.1096) -- (2.0980, 3.0120, 3.1058) -- (2.0980, 3.0660, 3.0995) -- (2.0520, 3.0660, 3.1034) -- cycle;
\fill[blue!61.4, opacity=0.7] (2.0520, 3.0660, 3.1034) -- (2.0980, 3.0660, 3.0995) -- (2.0980, 3.1200, 3.0933) -- (2.0520, 3.1200, 3.0971) -- cycle;
\fill[blue!30.0, opacity=0.7] (2.0980, -0.1200, 3.0933) -- (2.1440, -0.1200, 3.0892) -- (2.1440, -0.0660, 3.0955) -- (2.0980, -0.0660, 3.0995) -- cycle;
\fill[blue!53.1, opacity=0.7] (2.0980, -0.0660, 3.0995) -- (2.1440, -0.0660, 3.0955) -- (2.1440, -0.0120, 3.1017) -- (2.0980, -0.0120, 3.1058) -- cycle;
\fill[blue!63.1, opacity=0.7] (2.0980, -0.0120, 3.1058) -- (2.1440, -0.0120, 3.1017) -- (2.1440, 0.0420, 3.1079) -- (2.0980, 0.0420, 3.1120) -- cycle;
\fill[blue!63.6, opacity=0.7] (2.0980, 0.0420, 3.1120) -- (2.1440, 0.0420, 3.1079) -- (2.1440, 0.0960, 3.1141) -- (2.0980, 0.0960, 3.1182) -- cycle;
\fill[blue!60.6, opacity=0.7] (2.0980, 0.0960, 3.1182) -- (2.1440, 0.0960, 3.1141) -- (2.1440, 0.1500, 3.1202) -- (2.0980, 0.1500, 3.1243) -- cycle;
\fill[blue!46.1, opacity=0.7] (2.0980, 0.1500, 3.1243) -- (2.1440, 0.1500, 3.1202) -- (2.1440, 0.2040, 3.1263) -- (2.0980, 0.2040, 3.1303) -- cycle;
\fill[blue!30.2, opacity=0.7] (2.0980, 0.2040, 3.1303) -- (2.1440, 0.2040, 3.1263) -- (2.1440, 0.2580, 3.1322) -- (2.0980, 0.2580, 3.1363) -- cycle;
\fill[blue!23.8, opacity=0.7] (2.0980, 0.2580, 3.1363) -- (2.1440, 0.2580, 3.1322) -- (2.1440, 0.3120, 3.1380) -- (2.0980, 0.3120, 3.1421) -- cycle;
\fill[blue!25.1, opacity=0.7] (2.0980, 0.3120, 3.1421) -- (2.1440, 0.3120, 3.1380) -- (2.1440, 0.3660, 3.1437) -- (2.0980, 0.3660, 3.1477) -- cycle;
\fill[blue!35.4, opacity=0.7] (2.0980, 0.3660, 3.1477) -- (2.1440, 0.3660, 3.1437) -- (2.1440, 0.4200, 3.1492) -- (2.0980, 0.4200, 3.1533) -- cycle;
\fill[blue!55.2, opacity=0.7] (2.0980, 0.4200, 3.1533) -- (2.1440, 0.4200, 3.1492) -- (2.1440, 0.4740, 3.1545) -- (2.0980, 0.4740, 3.1586) -- cycle;
\fill[blue!63.2, opacity=0.7] (2.0980, 0.4740, 3.1586) -- (2.1440, 0.4740, 3.1545) -- (2.1440, 0.5280, 3.1597) -- (2.0980, 0.5280, 3.1638) -- cycle;
\fill[blue!52.6, opacity=0.7] (2.0980, 0.5280, 3.1638) -- (2.1440, 0.5280, 3.1597) -- (2.1440, 0.5820, 3.1647) -- (2.0980, 0.5820, 3.1688) -- cycle;
\fill[blue!43.9, opacity=0.7] (2.0980, 0.5820, 3.1688) -- (2.1440, 0.5820, 3.1647) -- (2.1440, 0.6360, 3.1695) -- (2.0980, 0.6360, 3.1736) -- cycle;
\fill[blue!44.3, opacity=0.7] (2.0980, 0.6360, 3.1736) -- (2.1440, 0.6360, 3.1695) -- (2.1440, 0.6900, 3.1740) -- (2.0980, 0.6900, 3.1781) -- cycle;
\fill[blue!52.1, opacity=0.7] (2.0980, 0.6900, 3.1781) -- (2.1440, 0.6900, 3.1740) -- (2.1440, 0.7440, 3.1784) -- (2.0980, 0.7440, 3.1824) -- cycle;
\fill[blue!61.1, opacity=0.7] (2.0980, 0.7440, 3.1824) -- (2.1440, 0.7440, 3.1784) -- (2.1440, 0.7980, 3.1824) -- (2.0980, 0.7980, 3.1865) -- cycle;
\fill[blue!63.2, opacity=0.7] (2.0980, 0.7980, 3.1865) -- (2.1440, 0.7980, 3.1824) -- (2.1440, 0.8520, 3.1863) -- (2.0980, 0.8520, 3.1903) -- cycle;
\fill[blue!58.5, opacity=0.7] (2.0980, 0.8520, 3.1903) -- (2.1440, 0.8520, 3.1863) -- (2.1440, 0.9060, 3.1898) -- (2.0980, 0.9060, 3.1939) -- cycle;
\fill[blue!53.3, opacity=0.7] (2.0980, 0.9060, 3.1939) -- (2.1440, 0.9060, 3.1898) -- (2.1440, 0.9600, 3.1931) -- (2.0980, 0.9600, 3.1972) -- cycle;
\fill[blue!51.6, opacity=0.7] (2.0980, 0.9600, 3.1972) -- (2.1440, 0.9600, 3.1931) -- (2.1440, 1.0140, 3.1961) -- (2.0980, 1.0140, 3.2002) -- cycle;
\fill[blue!53.3, opacity=0.7] (2.0980, 1.0140, 3.2002) -- (2.1440, 1.0140, 3.1961) -- (2.1440, 1.0680, 3.1988) -- (2.0980, 1.0680, 3.2029) -- cycle;
\fill[blue!57.1, opacity=0.7] (2.0980, 1.0680, 3.2029) -- (2.1440, 1.0680, 3.1988) -- (2.1440, 1.1220, 3.2012) -- (2.0980, 1.1220, 3.2053) -- cycle;
\fill[blue!60.8, opacity=0.7] (2.0980, 1.1220, 3.2053) -- (2.1440, 1.1220, 3.2012) -- (2.1440, 1.1760, 3.2033) -- (2.0980, 1.1760, 3.2074) -- cycle;
\fill[blue!63.0, opacity=0.7] (2.0980, 1.1760, 3.2074) -- (2.1440, 1.1760, 3.2033) -- (2.1440, 1.2300, 3.2051) -- (2.0980, 1.2300, 3.2092) -- cycle;
\fill[blue!63.6, opacity=0.7] (2.0980, 1.2300, 3.2092) -- (2.1440, 1.2300, 3.2051) -- (2.1440, 1.2840, 3.2066) -- (2.0980, 1.2840, 3.2106) -- cycle;
\fill[blue!63.2, opacity=0.7] (2.0980, 1.2840, 3.2106) -- (2.1440, 1.2840, 3.2066) -- (2.1440, 1.3380, 3.2077) -- (2.0980, 1.3380, 3.2118) -- cycle;
\fill[blue!62.9, opacity=0.7] (2.0980, 1.3380, 3.2118) -- (2.1440, 1.3380, 3.2077) -- (2.1440, 1.3920, 3.2085) -- (2.0980, 1.3920, 3.2126) -- cycle;
\fill[blue!63.0, opacity=0.7] (2.0980, 1.3920, 3.2126) -- (2.1440, 1.3920, 3.2085) -- (2.1440, 1.4460, 3.2090) -- (2.0980, 1.4460, 3.2131) -- cycle;
\fill[blue!63.5, opacity=0.7] (2.0980, 1.4460, 3.2131) -- (2.1440, 1.4460, 3.2090) -- (2.1440, 1.5000, 3.2092) -- (2.0980, 1.5000, 3.2133) -- cycle;
\fill[blue!63.5, opacity=0.7] (2.0980, 1.5000, 3.2133) -- (2.1440, 1.5000, 3.2092) -- (2.1440, 1.5540, 3.2090) -- (2.0980, 1.5540, 3.2131) -- cycle;
\fill[blue!62.4, opacity=0.7] (2.0980, 1.5540, 3.2131) -- (2.1440, 1.5540, 3.2090) -- (2.1440, 1.6080, 3.2085) -- (2.0980, 1.6080, 3.2126) -- cycle;
\fill[blue!60.3, opacity=0.7] (2.0980, 1.6080, 3.2126) -- (2.1440, 1.6080, 3.2085) -- (2.1440, 1.6620, 3.2077) -- (2.0980, 1.6620, 3.2118) -- cycle;
\fill[blue!58.6, opacity=0.7] (2.0980, 1.6620, 3.2118) -- (2.1440, 1.6620, 3.2077) -- (2.1440, 1.7160, 3.2066) -- (2.0980, 1.7160, 3.2106) -- cycle;
\fill[blue!59.1, opacity=0.7] (2.0980, 1.7160, 3.2106) -- (2.1440, 1.7160, 3.2066) -- (2.1440, 1.7700, 3.2051) -- (2.0980, 1.7700, 3.2092) -- cycle;
\fill[blue!62.0, opacity=0.7] (2.0980, 1.7700, 3.2092) -- (2.1440, 1.7700, 3.2051) -- (2.1440, 1.8240, 3.2033) -- (2.0980, 1.8240, 3.2074) -- cycle;
\fill[blue!63.3, opacity=0.7] (2.0980, 1.8240, 3.2074) -- (2.1440, 1.8240, 3.2033) -- (2.1440, 1.8780, 3.2012) -- (2.0980, 1.8780, 3.2053) -- cycle;
\fill[blue!56.2, opacity=0.7] (2.0980, 1.8780, 3.2053) -- (2.1440, 1.8780, 3.2012) -- (2.1440, 1.9320, 3.1988) -- (2.0980, 1.9320, 3.2029) -- cycle;
\fill[blue!42.6, opacity=0.7] (2.0980, 1.9320, 3.2029) -- (2.1440, 1.9320, 3.1988) -- (2.1440, 1.9860, 3.1961) -- (2.0980, 1.9860, 3.2002) -- cycle;
\fill[blue!32.5, opacity=0.7] (2.0980, 1.9860, 3.2002) -- (2.1440, 1.9860, 3.1961) -- (2.1440, 2.0400, 3.1931) -- (2.0980, 2.0400, 3.1972) -- cycle;
\fill[blue!30.3, opacity=0.7] (2.0980, 2.0400, 3.1972) -- (2.1440, 2.0400, 3.1931) -- (2.1440, 2.0940, 3.1898) -- (2.0980, 2.0940, 3.1939) -- cycle;
\fill[blue!37.1, opacity=0.7] (2.0980, 2.0940, 3.1939) -- (2.1440, 2.0940, 3.1898) -- (2.1440, 2.1480, 3.1863) -- (2.0980, 2.1480, 3.1903) -- cycle;
\fill[blue!54.0, opacity=0.7] (2.0980, 2.1480, 3.1903) -- (2.1440, 2.1480, 3.1863) -- (2.1440, 2.2020, 3.1824) -- (2.0980, 2.2020, 3.1865) -- cycle;
\fill[blue!63.3, opacity=0.7] (2.0980, 2.2020, 3.1865) -- (2.1440, 2.2020, 3.1824) -- (2.1440, 2.2560, 3.1784) -- (2.0980, 2.2560, 3.1824) -- cycle;
\fill[blue!49.8, opacity=0.7] (2.0980, 2.2560, 3.1824) -- (2.1440, 2.2560, 3.1784) -- (2.1440, 2.3100, 3.1740) -- (2.0980, 2.3100, 3.1781) -- cycle;
\fill[blue!37.8, opacity=0.7] (2.0980, 2.3100, 3.1781) -- (2.1440, 2.3100, 3.1740) -- (2.1440, 2.3640, 3.1695) -- (2.0980, 2.3640, 3.1736) -- cycle;
\fill[blue!38.6, opacity=0.7] (2.0980, 2.3640, 3.1736) -- (2.1440, 2.3640, 3.1695) -- (2.1440, 2.4180, 3.1647) -- (2.0980, 2.4180, 3.1688) -- cycle;
\fill[blue!51.5, opacity=0.7] (2.0980, 2.4180, 3.1688) -- (2.1440, 2.4180, 3.1647) -- (2.1440, 2.4720, 3.1597) -- (2.0980, 2.4720, 3.1638) -- cycle;
\fill[blue!63.3, opacity=0.7] (2.0980, 2.4720, 3.1638) -- (2.1440, 2.4720, 3.1597) -- (2.1440, 2.5260, 3.1545) -- (2.0980, 2.5260, 3.1586) -- cycle;
\fill[blue!58.7, opacity=0.7] (2.0980, 2.5260, 3.1586) -- (2.1440, 2.5260, 3.1545) -- (2.1440, 2.5800, 3.1492) -- (2.0980, 2.5800, 3.1533) -- cycle;
\fill[blue!54.5, opacity=0.7] (2.0980, 2.5800, 3.1533) -- (2.1440, 2.5800, 3.1492) -- (2.1440, 2.6340, 3.1437) -- (2.0980, 2.6340, 3.1477) -- cycle;
\fill[blue!61.3, opacity=0.7] (2.0980, 2.6340, 3.1477) -- (2.1440, 2.6340, 3.1437) -- (2.1440, 2.6880, 3.1380) -- (2.0980, 2.6880, 3.1421) -- cycle;
\fill[blue!59.0, opacity=0.7] (2.0980, 2.6880, 3.1421) -- (2.1440, 2.6880, 3.1380) -- (2.1440, 2.7420, 3.1322) -- (2.0980, 2.7420, 3.1363) -- cycle;
\fill[blue!35.0, opacity=0.7] (2.0980, 2.7420, 3.1363) -- (2.1440, 2.7420, 3.1322) -- (2.1440, 2.7960, 3.1263) -- (2.0980, 2.7960, 3.1303) -- cycle;
\fill[blue!21.5, opacity=0.7] (2.0980, 2.7960, 3.1303) -- (2.1440, 2.7960, 3.1263) -- (2.1440, 2.8500, 3.1202) -- (2.0980, 2.8500, 3.1243) -- cycle;
\fill[blue!19.4, opacity=0.7] (2.0980, 2.8500, 3.1243) -- (2.1440, 2.8500, 3.1202) -- (2.1440, 2.9040, 3.1141) -- (2.0980, 2.9040, 3.1182) -- cycle;
\fill[blue!23.9, opacity=0.7] (2.0980, 2.9040, 3.1182) -- (2.1440, 2.9040, 3.1141) -- (2.1440, 2.9580, 3.1079) -- (2.0980, 2.9580, 3.1120) -- cycle;
\fill[blue!39.6, opacity=0.7] (2.0980, 2.9580, 3.1120) -- (2.1440, 2.9580, 3.1079) -- (2.1440, 3.0120, 3.1017) -- (2.0980, 3.0120, 3.1058) -- cycle;
\fill[blue!57.1, opacity=0.7] (2.0980, 3.0120, 3.1058) -- (2.1440, 3.0120, 3.1017) -- (2.1440, 3.0660, 3.0955) -- (2.0980, 3.0660, 3.0995) -- cycle;
\fill[blue!61.5, opacity=0.7] (2.0980, 3.0660, 3.0995) -- (2.1440, 3.0660, 3.0955) -- (2.1440, 3.1200, 3.0892) -- (2.0980, 3.1200, 3.0933) -- cycle;
\fill[blue!21.4, opacity=0.7] (2.1440, -0.1200, 3.0892) -- (2.1900, -0.1200, 3.0849) -- (2.1900, -0.0660, 3.0911) -- (2.1440, -0.0660, 3.0955) -- cycle;
\fill[blue!39.6, opacity=0.7] (2.1440, -0.0660, 3.0955) -- (2.1900, -0.0660, 3.0911) -- (2.1900, -0.0120, 3.0974) -- (2.1440, -0.0120, 3.1017) -- cycle;
\fill[blue!59.4, opacity=0.7] (2.1440, -0.0120, 3.1017) -- (2.1900, -0.0120, 3.0974) -- (2.1900, 0.0420, 3.1036) -- (2.1440, 0.0420, 3.1079) -- cycle;
\fill[blue!63.5, opacity=0.7] (2.1440, 0.0420, 3.1079) -- (2.1900, 0.0420, 3.1036) -- (2.1900, 0.0960, 3.1098) -- (2.1440, 0.0960, 3.1141) -- cycle;
\fill[blue!63.4, opacity=0.7] (2.1440, 0.0960, 3.1141) -- (2.1900, 0.0960, 3.1098) -- (2.1900, 0.1500, 3.1159) -- (2.1440, 0.1500, 3.1202) -- cycle;
\fill[blue!57.7, opacity=0.7] (2.1440, 0.1500, 3.1202) -- (2.1900, 0.1500, 3.1159) -- (2.1900, 0.2040, 3.1219) -- (2.1440, 0.2040, 3.1263) -- cycle;
\fill[blue!41.4, opacity=0.7] (2.1440, 0.2040, 3.1263) -- (2.1900, 0.2040, 3.1219) -- (2.1900, 0.2580, 3.1279) -- (2.1440, 0.2580, 3.1322) -- cycle;
\fill[blue!28.1, opacity=0.7] (2.1440, 0.2580, 3.1322) -- (2.1900, 0.2580, 3.1279) -- (2.1900, 0.3120, 3.1337) -- (2.1440, 0.3120, 3.1380) -- cycle;
\fill[blue!23.8, opacity=0.7] (2.1440, 0.3120, 3.1380) -- (2.1900, 0.3120, 3.1337) -- (2.1900, 0.3660, 3.1393) -- (2.1440, 0.3660, 3.1437) -- cycle;
\fill[blue!26.3, opacity=0.7] (2.1440, 0.3660, 3.1437) -- (2.1900, 0.3660, 3.1393) -- (2.1900, 0.4200, 3.1449) -- (2.1440, 0.4200, 3.1492) -- cycle;
\fill[blue!38.0, opacity=0.7] (2.1440, 0.4200, 3.1492) -- (2.1900, 0.4200, 3.1449) -- (2.1900, 0.4740, 3.1502) -- (2.1440, 0.4740, 3.1545) -- cycle;
\fill[blue!57.0, opacity=0.7] (2.1440, 0.4740, 3.1545) -- (2.1900, 0.4740, 3.1502) -- (2.1900, 0.5280, 3.1554) -- (2.1440, 0.5280, 3.1597) -- cycle;
\fill[blue!63.0, opacity=0.7] (2.1440, 0.5280, 3.1597) -- (2.1900, 0.5280, 3.1554) -- (2.1900, 0.5820, 3.1604) -- (2.1440, 0.5820, 3.1647) -- cycle;
\fill[blue!52.6, opacity=0.7] (2.1440, 0.5820, 3.1647) -- (2.1900, 0.5820, 3.1604) -- (2.1900, 0.6360, 3.1651) -- (2.1440, 0.6360, 3.1695) -- cycle;
\fill[blue!43.8, opacity=0.7] (2.1440, 0.6360, 3.1695) -- (2.1900, 0.6360, 3.1651) -- (2.1900, 0.6900, 3.1697) -- (2.1440, 0.6900, 3.1740) -- cycle;
\fill[blue!42.8, opacity=0.7] (2.1440, 0.6900, 3.1740) -- (2.1900, 0.6900, 3.1697) -- (2.1900, 0.7440, 3.1740) -- (2.1440, 0.7440, 3.1784) -- cycle;
\fill[blue!48.5, opacity=0.7] (2.1440, 0.7440, 3.1784) -- (2.1900, 0.7440, 3.1740) -- (2.1900, 0.7980, 3.1781) -- (2.1440, 0.7980, 3.1824) -- cycle;
\fill[blue!57.3, opacity=0.7] (2.1440, 0.7980, 3.1824) -- (2.1900, 0.7980, 3.1781) -- (2.1900, 0.8520, 3.1819) -- (2.1440, 0.8520, 3.1863) -- cycle;
\fill[blue!63.1, opacity=0.7] (2.1440, 0.8520, 3.1863) -- (2.1900, 0.8520, 3.1819) -- (2.1900, 0.9060, 3.1855) -- (2.1440, 0.9060, 3.1898) -- cycle;
\fill[blue!62.6, opacity=0.7] (2.1440, 0.9060, 3.1898) -- (2.1900, 0.9060, 3.1855) -- (2.1900, 0.9600, 3.1888) -- (2.1440, 0.9600, 3.1931) -- cycle;
\fill[blue!58.5, opacity=0.7] (2.1440, 0.9600, 3.1931) -- (2.1900, 0.9600, 3.1888) -- (2.1900, 1.0140, 3.1918) -- (2.1440, 1.0140, 3.1961) -- cycle;
\fill[blue!54.9, opacity=0.7] (2.1440, 1.0140, 3.1961) -- (2.1900, 1.0140, 3.1918) -- (2.1900, 1.0680, 3.1945) -- (2.1440, 1.0680, 3.1988) -- cycle;
\fill[blue!53.1, opacity=0.7] (2.1440, 1.0680, 3.1988) -- (2.1900, 1.0680, 3.1945) -- (2.1900, 1.1220, 3.1969) -- (2.1440, 1.1220, 3.2012) -- cycle;
\fill[blue!53.3, opacity=0.7] (2.1440, 1.1220, 3.2012) -- (2.1900, 1.1220, 3.1969) -- (2.1900, 1.1760, 3.1990) -- (2.1440, 1.1760, 3.2033) -- cycle;
\fill[blue!54.5, opacity=0.7] (2.1440, 1.1760, 3.2033) -- (2.1900, 1.1760, 3.1990) -- (2.1900, 1.2300, 3.2008) -- (2.1440, 1.2300, 3.2051) -- cycle;
\fill[blue!55.9, opacity=0.7] (2.1440, 1.2300, 3.2051) -- (2.1900, 1.2300, 3.2008) -- (2.1900, 1.2840, 3.2022) -- (2.1440, 1.2840, 3.2066) -- cycle;
\fill[blue!57.1, opacity=0.7] (2.1440, 1.2840, 3.2066) -- (2.1900, 1.2840, 3.2022) -- (2.1900, 1.3380, 3.2034) -- (2.1440, 1.3380, 3.2077) -- cycle;
\fill[blue!57.8, opacity=0.7] (2.1440, 1.3380, 3.2077) -- (2.1900, 1.3380, 3.2034) -- (2.1900, 1.3920, 3.2042) -- (2.1440, 1.3920, 3.2085) -- cycle;
\fill[blue!57.9, opacity=0.7] (2.1440, 1.3920, 3.2085) -- (2.1900, 1.3920, 3.2042) -- (2.1900, 1.4460, 3.2047) -- (2.1440, 1.4460, 3.2090) -- cycle;
\fill[blue!57.6, opacity=0.7] (2.1440, 1.4460, 3.2090) -- (2.1900, 1.4460, 3.2047) -- (2.1900, 1.5000, 3.2049) -- (2.1440, 1.5000, 3.2092) -- cycle;
\fill[blue!57.2, opacity=0.7] (2.1440, 1.5000, 3.2092) -- (2.1900, 1.5000, 3.2049) -- (2.1900, 1.5540, 3.2047) -- (2.1440, 1.5540, 3.2090) -- cycle;
\fill[blue!57.3, opacity=0.7] (2.1440, 1.5540, 3.2090) -- (2.1900, 1.5540, 3.2047) -- (2.1900, 1.6080, 3.2042) -- (2.1440, 1.6080, 3.2085) -- cycle;
\fill[blue!58.5, opacity=0.7] (2.1440, 1.6080, 3.2085) -- (2.1900, 1.6080, 3.2042) -- (2.1900, 1.6620, 3.2034) -- (2.1440, 1.6620, 3.2077) -- cycle;
\fill[blue!61.0, opacity=0.7] (2.1440, 1.6620, 3.2077) -- (2.1900, 1.6620, 3.2034) -- (2.1900, 1.7160, 3.2022) -- (2.1440, 1.7160, 3.2066) -- cycle;
\fill[blue!63.4, opacity=0.7] (2.1440, 1.7160, 3.2066) -- (2.1900, 1.7160, 3.2022) -- (2.1900, 1.7700, 3.2008) -- (2.1440, 1.7700, 3.2051) -- cycle;
\fill[blue!61.7, opacity=0.7] (2.1440, 1.7700, 3.2051) -- (2.1900, 1.7700, 3.2008) -- (2.1900, 1.8240, 3.1990) -- (2.1440, 1.8240, 3.2033) -- cycle;
\fill[blue!52.7, opacity=0.7] (2.1440, 1.8240, 3.2033) -- (2.1900, 1.8240, 3.1990) -- (2.1900, 1.8780, 3.1969) -- (2.1440, 1.8780, 3.2012) -- cycle;
\fill[blue!40.7, opacity=0.7] (2.1440, 1.8780, 3.2012) -- (2.1900, 1.8780, 3.1969) -- (2.1900, 1.9320, 3.1945) -- (2.1440, 1.9320, 3.1988) -- cycle;
\fill[blue!32.6, opacity=0.7] (2.1440, 1.9320, 3.1988) -- (2.1900, 1.9320, 3.1945) -- (2.1900, 1.9860, 3.1918) -- (2.1440, 1.9860, 3.1961) -- cycle;
\fill[blue!31.0, opacity=0.7] (2.1440, 1.9860, 3.1961) -- (2.1900, 1.9860, 3.1918) -- (2.1900, 2.0400, 3.1888) -- (2.1440, 2.0400, 3.1931) -- cycle;
\fill[blue!37.1, opacity=0.7] (2.1440, 2.0400, 3.1931) -- (2.1900, 2.0400, 3.1888) -- (2.1900, 2.0940, 3.1855) -- (2.1440, 2.0940, 3.1898) -- cycle;
\fill[blue!52.3, opacity=0.7] (2.1440, 2.0940, 3.1898) -- (2.1900, 2.0940, 3.1855) -- (2.1900, 2.1480, 3.1819) -- (2.1440, 2.1480, 3.1863) -- cycle;
\fill[blue!63.6, opacity=0.7] (2.1440, 2.1480, 3.1863) -- (2.1900, 2.1480, 3.1819) -- (2.1900, 2.2020, 3.1781) -- (2.1440, 2.2020, 3.1824) -- cycle;
\fill[blue!53.3, opacity=0.7] (2.1440, 2.2020, 3.1824) -- (2.1900, 2.2020, 3.1781) -- (2.1900, 2.2560, 3.1740) -- (2.1440, 2.2560, 3.1784) -- cycle;
\fill[blue!39.3, opacity=0.7] (2.1440, 2.2560, 3.1784) -- (2.1900, 2.2560, 3.1740) -- (2.1900, 2.3100, 3.1697) -- (2.1440, 2.3100, 3.1740) -- cycle;
\fill[blue!36.5, opacity=0.7] (2.1440, 2.3100, 3.1740) -- (2.1900, 2.3100, 3.1697) -- (2.1900, 2.3640, 3.1651) -- (2.1440, 2.3640, 3.1695) -- cycle;
\fill[blue!45.4, opacity=0.7] (2.1440, 2.3640, 3.1695) -- (2.1900, 2.3640, 3.1651) -- (2.1900, 2.4180, 3.1604) -- (2.1440, 2.4180, 3.1647) -- cycle;
\fill[blue!60.2, opacity=0.7] (2.1440, 2.4180, 3.1647) -- (2.1900, 2.4180, 3.1604) -- (2.1900, 2.4720, 3.1554) -- (2.1440, 2.4720, 3.1597) -- cycle;
\fill[blue!62.1, opacity=0.7] (2.1440, 2.4720, 3.1597) -- (2.1900, 2.4720, 3.1554) -- (2.1900, 2.5260, 3.1502) -- (2.1440, 2.5260, 3.1545) -- cycle;
\fill[blue!55.4, opacity=0.7] (2.1440, 2.5260, 3.1545) -- (2.1900, 2.5260, 3.1502) -- (2.1900, 2.5800, 3.1449) -- (2.1440, 2.5800, 3.1492) -- cycle;
\fill[blue!57.5, opacity=0.7] (2.1440, 2.5800, 3.1492) -- (2.1900, 2.5800, 3.1449) -- (2.1900, 2.6340, 3.1393) -- (2.1440, 2.6340, 3.1437) -- cycle;
\fill[blue!63.5, opacity=0.7] (2.1440, 2.6340, 3.1437) -- (2.1900, 2.6340, 3.1393) -- (2.1900, 2.6880, 3.1337) -- (2.1440, 2.6880, 3.1380) -- cycle;
\fill[blue!47.6, opacity=0.7] (2.1440, 2.6880, 3.1380) -- (2.1900, 2.6880, 3.1337) -- (2.1900, 2.7420, 3.1279) -- (2.1440, 2.7420, 3.1322) -- cycle;
\fill[blue!26.4, opacity=0.7] (2.1440, 2.7420, 3.1322) -- (2.1900, 2.7420, 3.1279) -- (2.1900, 2.7960, 3.1219) -- (2.1440, 2.7960, 3.1263) -- cycle;
\fill[blue!19.7, opacity=0.7] (2.1440, 2.7960, 3.1263) -- (2.1900, 2.7960, 3.1219) -- (2.1900, 2.8500, 3.1159) -- (2.1440, 2.8500, 3.1202) -- cycle;
\fill[blue!20.3, opacity=0.7] (2.1440, 2.8500, 3.1202) -- (2.1900, 2.8500, 3.1159) -- (2.1900, 2.9040, 3.1098) -- (2.1440, 2.9040, 3.1141) -- cycle;
\fill[blue!28.7, opacity=0.7] (2.1440, 2.9040, 3.1141) -- (2.1900, 2.9040, 3.1098) -- (2.1900, 2.9580, 3.1036) -- (2.1440, 2.9580, 3.1079) -- cycle;
\fill[blue!47.5, opacity=0.7] (2.1440, 2.9580, 3.1079) -- (2.1900, 2.9580, 3.1036) -- (2.1900, 3.0120, 3.0974) -- (2.1440, 3.0120, 3.1017) -- cycle;
\fill[blue!60.0, opacity=0.7] (2.1440, 3.0120, 3.1017) -- (2.1900, 3.0120, 3.0974) -- (2.1900, 3.0660, 3.0911) -- (2.1440, 3.0660, 3.0955) -- cycle;
\fill[blue!60.6, opacity=0.7] (2.1440, 3.0660, 3.0955) -- (2.1900, 3.0660, 3.0911) -- (2.1900, 3.1200, 3.0849) -- (2.1440, 3.1200, 3.0892) -- cycle;
\fill[blue!17.5, opacity=0.7] (2.1900, -0.1200, 3.0849) -- (2.2360, -0.1200, 3.0803) -- (2.2360, -0.0660, 3.0866) -- (2.1900, -0.0660, 3.0911) -- cycle;
\fill[blue!26.6, opacity=0.7] (2.1900, -0.0660, 3.0911) -- (2.2360, -0.0660, 3.0866) -- (2.2360, -0.0120, 3.0928) -- (2.1900, -0.0120, 3.0974) -- cycle;
\fill[blue!48.2, opacity=0.7] (2.1900, -0.0120, 3.0974) -- (2.2360, -0.0120, 3.0928) -- (2.2360, 0.0420, 3.0991) -- (2.1900, 0.0420, 3.1036) -- cycle;
\fill[blue!62.2, opacity=0.7] (2.1900, 0.0420, 3.1036) -- (2.2360, 0.0420, 3.0991) -- (2.2360, 0.0960, 3.1052) -- (2.1900, 0.0960, 3.1098) -- cycle;
\fill[blue!63.6, opacity=0.7] (2.1900, 0.0960, 3.1098) -- (2.2360, 0.0960, 3.1052) -- (2.2360, 0.1500, 3.1114) -- (2.1900, 0.1500, 3.1159) -- cycle;
\fill[blue!63.0, opacity=0.7] (2.1900, 0.1500, 3.1159) -- (2.2360, 0.1500, 3.1114) -- (2.2360, 0.2040, 3.1174) -- (2.1900, 0.2040, 3.1219) -- cycle;
\fill[blue!54.9, opacity=0.7] (2.1900, 0.2040, 3.1219) -- (2.2360, 0.2040, 3.1174) -- (2.2360, 0.2580, 3.1233) -- (2.1900, 0.2580, 3.1279) -- cycle;
\fill[blue!38.5, opacity=0.7] (2.1900, 0.2580, 3.1279) -- (2.2360, 0.2580, 3.1233) -- (2.2360, 0.3120, 3.1291) -- (2.1900, 0.3120, 3.1337) -- cycle;
\fill[blue!27.2, opacity=0.7] (2.1900, 0.3120, 3.1337) -- (2.2360, 0.3120, 3.1291) -- (2.2360, 0.3660, 3.1348) -- (2.1900, 0.3660, 3.1393) -- cycle;
\fill[blue!24.0, opacity=0.7] (2.1900, 0.3660, 3.1393) -- (2.2360, 0.3660, 3.1348) -- (2.2360, 0.4200, 3.1403) -- (2.1900, 0.4200, 3.1449) -- cycle;
\fill[blue!27.1, opacity=0.7] (2.1900, 0.4200, 3.1449) -- (2.2360, 0.4200, 3.1403) -- (2.2360, 0.4740, 3.1457) -- (2.1900, 0.4740, 3.1502) -- cycle;
\fill[blue!38.7, opacity=0.7] (2.1900, 0.4740, 3.1502) -- (2.2360, 0.4740, 3.1457) -- (2.2360, 0.5280, 3.1508) -- (2.1900, 0.5280, 3.1554) -- cycle;
\fill[blue!56.6, opacity=0.7] (2.1900, 0.5280, 3.1554) -- (2.2360, 0.5280, 3.1508) -- (2.2360, 0.5820, 3.1558) -- (2.1900, 0.5820, 3.1604) -- cycle;
\fill[blue!63.3, opacity=0.7] (2.1900, 0.5820, 3.1604) -- (2.2360, 0.5820, 3.1558) -- (2.2360, 0.6360, 3.1606) -- (2.1900, 0.6360, 3.1651) -- cycle;
\fill[blue!54.3, opacity=0.7] (2.1900, 0.6360, 3.1651) -- (2.2360, 0.6360, 3.1606) -- (2.2360, 0.6900, 3.1651) -- (2.1900, 0.6900, 3.1697) -- cycle;
\fill[blue!44.8, opacity=0.7] (2.1900, 0.6900, 3.1697) -- (2.2360, 0.6900, 3.1651) -- (2.2360, 0.7440, 3.1695) -- (2.1900, 0.7440, 3.1740) -- cycle;
\fill[blue!41.5, opacity=0.7] (2.1900, 0.7440, 3.1740) -- (2.2360, 0.7440, 3.1695) -- (2.2360, 0.7980, 3.1736) -- (2.1900, 0.7980, 3.1781) -- cycle;
\fill[blue!44.2, opacity=0.7] (2.1900, 0.7980, 3.1781) -- (2.2360, 0.7980, 3.1736) -- (2.2360, 0.8520, 3.1774) -- (2.1900, 0.8520, 3.1819) -- cycle;
\fill[blue!50.9, opacity=0.7] (2.1900, 0.8520, 3.1819) -- (2.2360, 0.8520, 3.1774) -- (2.2360, 0.9060, 3.1809) -- (2.1900, 0.9060, 3.1855) -- cycle;
\fill[blue!58.3, opacity=0.7] (2.1900, 0.9060, 3.1855) -- (2.2360, 0.9060, 3.1809) -- (2.2360, 0.9600, 3.1842) -- (2.1900, 0.9600, 3.1888) -- cycle;
\fill[blue!62.8, opacity=0.7] (2.1900, 0.9600, 3.1888) -- (2.2360, 0.9600, 3.1842) -- (2.2360, 1.0140, 3.1872) -- (2.1900, 1.0140, 3.1918) -- cycle;
\fill[blue!63.4, opacity=0.7] (2.1900, 1.0140, 3.1918) -- (2.2360, 1.0140, 3.1872) -- (2.2360, 1.0680, 3.1899) -- (2.1900, 1.0680, 3.1945) -- cycle;
\fill[blue!61.5, opacity=0.7] (2.1900, 1.0680, 3.1945) -- (2.2360, 1.0680, 3.1899) -- (2.2360, 1.1220, 3.1923) -- (2.1900, 1.1220, 3.1969) -- cycle;
\fill[blue!59.3, opacity=0.7] (2.1900, 1.1220, 3.1969) -- (2.2360, 1.1220, 3.1923) -- (2.2360, 1.1760, 3.1944) -- (2.1900, 1.1760, 3.1990) -- cycle;
\fill[blue!57.6, opacity=0.7] (2.1900, 1.1760, 3.1990) -- (2.2360, 1.1760, 3.1944) -- (2.2360, 1.2300, 3.1962) -- (2.1900, 1.2300, 3.2008) -- cycle;
\fill[blue!56.7, opacity=0.7] (2.1900, 1.2300, 3.2008) -- (2.2360, 1.2300, 3.1962) -- (2.2360, 1.2840, 3.1977) -- (2.1900, 1.2840, 3.2022) -- cycle;
\fill[blue!56.5, opacity=0.7] (2.1900, 1.2840, 3.2022) -- (2.2360, 1.2840, 3.1977) -- (2.2360, 1.3380, 3.1988) -- (2.1900, 1.3380, 3.2034) -- cycle;
\fill[blue!56.8, opacity=0.7] (2.1900, 1.3380, 3.2034) -- (2.2360, 1.3380, 3.1988) -- (2.2360, 1.3920, 3.1996) -- (2.1900, 1.3920, 3.2042) -- cycle;
\fill[blue!57.5, opacity=0.7] (2.1900, 1.3920, 3.2042) -- (2.2360, 1.3920, 3.1996) -- (2.2360, 1.4460, 3.2001) -- (2.1900, 1.4460, 3.2047) -- cycle;
\fill[blue!58.7, opacity=0.7] (2.1900, 1.4460, 3.2047) -- (2.2360, 1.4460, 3.2001) -- (2.2360, 1.5000, 3.2003) -- (2.1900, 1.5000, 3.2049) -- cycle;
\fill[blue!60.3, opacity=0.7] (2.1900, 1.5000, 3.2049) -- (2.2360, 1.5000, 3.2003) -- (2.2360, 1.5540, 3.2001) -- (2.1900, 1.5540, 3.2047) -- cycle;
\fill[blue!62.3, opacity=0.7] (2.1900, 1.5540, 3.2047) -- (2.2360, 1.5540, 3.2001) -- (2.2360, 1.6080, 3.1996) -- (2.1900, 1.6080, 3.2042) -- cycle;
\fill[blue!63.6, opacity=0.7] (2.1900, 1.6080, 3.2042) -- (2.2360, 1.6080, 3.1996) -- (2.2360, 1.6620, 3.1988) -- (2.1900, 1.6620, 3.2034) -- cycle;
\fill[blue!62.1, opacity=0.7] (2.1900, 1.6620, 3.2034) -- (2.2360, 1.6620, 3.1988) -- (2.2360, 1.7160, 3.1977) -- (2.1900, 1.7160, 3.2022) -- cycle;
\fill[blue!56.1, opacity=0.7] (2.1900, 1.7160, 3.2022) -- (2.2360, 1.7160, 3.1977) -- (2.2360, 1.7700, 3.1962) -- (2.1900, 1.7700, 3.2008) -- cycle;
\fill[blue!46.5, opacity=0.7] (2.1900, 1.7700, 3.2008) -- (2.2360, 1.7700, 3.1962) -- (2.2360, 1.8240, 3.1944) -- (2.1900, 1.8240, 3.1990) -- cycle;
\fill[blue!37.3, opacity=0.7] (2.1900, 1.8240, 3.1990) -- (2.2360, 1.8240, 3.1944) -- (2.2360, 1.8780, 3.1923) -- (2.1900, 1.8780, 3.1969) -- cycle;
\fill[blue!32.1, opacity=0.7] (2.1900, 1.8780, 3.1969) -- (2.2360, 1.8780, 3.1923) -- (2.2360, 1.9320, 3.1899) -- (2.1900, 1.9320, 3.1945) -- cycle;
\fill[blue!32.1, opacity=0.7] (2.1900, 1.9320, 3.1945) -- (2.2360, 1.9320, 3.1899) -- (2.2360, 1.9860, 3.1872) -- (2.1900, 1.9860, 3.1918) -- cycle;
\fill[blue!39.0, opacity=0.7] (2.1900, 1.9860, 3.1918) -- (2.2360, 1.9860, 3.1872) -- (2.2360, 2.0400, 3.1842) -- (2.1900, 2.0400, 3.1888) -- cycle;
\fill[blue!53.3, opacity=0.7] (2.1900, 2.0400, 3.1888) -- (2.2360, 2.0400, 3.1842) -- (2.2360, 2.0940, 3.1809) -- (2.1900, 2.0940, 3.1855) -- cycle;
\fill[blue!63.5, opacity=0.7] (2.1900, 2.0940, 3.1855) -- (2.2360, 2.0940, 3.1809) -- (2.2360, 2.1480, 3.1774) -- (2.1900, 2.1480, 3.1819) -- cycle;
\fill[blue!54.6, opacity=0.7] (2.1900, 2.1480, 3.1819) -- (2.2360, 2.1480, 3.1774) -- (2.2360, 2.2020, 3.1736) -- (2.1900, 2.2020, 3.1781) -- cycle;
\fill[blue!40.5, opacity=0.7] (2.1900, 2.2020, 3.1781) -- (2.2360, 2.2020, 3.1736) -- (2.2360, 2.2560, 3.1695) -- (2.1900, 2.2560, 3.1740) -- cycle;
\fill[blue!35.6, opacity=0.7] (2.1900, 2.2560, 3.1740) -- (2.2360, 2.2560, 3.1695) -- (2.2360, 2.3100, 3.1651) -- (2.1900, 2.3100, 3.1697) -- cycle;
\fill[blue!41.4, opacity=0.7] (2.1900, 2.3100, 3.1697) -- (2.2360, 2.3100, 3.1651) -- (2.2360, 2.3640, 3.1606) -- (2.1900, 2.3640, 3.1651) -- cycle;
\fill[blue!55.9, opacity=0.7] (2.1900, 2.3640, 3.1651) -- (2.2360, 2.3640, 3.1606) -- (2.2360, 2.4180, 3.1558) -- (2.1900, 2.4180, 3.1604) -- cycle;
\fill[blue!63.5, opacity=0.7] (2.1900, 2.4180, 3.1604) -- (2.2360, 2.4180, 3.1558) -- (2.2360, 2.4720, 3.1508) -- (2.1900, 2.4720, 3.1554) -- cycle;
\fill[blue!57.8, opacity=0.7] (2.1900, 2.4720, 3.1554) -- (2.2360, 2.4720, 3.1508) -- (2.2360, 2.5260, 3.1457) -- (2.1900, 2.5260, 3.1502) -- cycle;
\fill[blue!55.6, opacity=0.7] (2.1900, 2.5260, 3.1502) -- (2.2360, 2.5260, 3.1457) -- (2.2360, 2.5800, 3.1403) -- (2.1900, 2.5800, 3.1449) -- cycle;
\fill[blue!62.3, opacity=0.7] (2.1900, 2.5800, 3.1449) -- (2.2360, 2.5800, 3.1403) -- (2.2360, 2.6340, 3.1348) -- (2.1900, 2.6340, 3.1393) -- cycle;
\fill[blue!57.7, opacity=0.7] (2.1900, 2.6340, 3.1393) -- (2.2360, 2.6340, 3.1348) -- (2.2360, 2.6880, 3.1291) -- (2.1900, 2.6880, 3.1337) -- cycle;
\fill[blue!34.3, opacity=0.7] (2.1900, 2.6880, 3.1337) -- (2.2360, 2.6880, 3.1291) -- (2.2360, 2.7420, 3.1233) -- (2.1900, 2.7420, 3.1279) -- cycle;
\fill[blue!21.4, opacity=0.7] (2.1900, 2.7420, 3.1279) -- (2.2360, 2.7420, 3.1233) -- (2.2360, 2.7960, 3.1174) -- (2.1900, 2.7960, 3.1219) -- cycle;
\fill[blue!19.2, opacity=0.7] (2.1900, 2.7960, 3.1219) -- (2.2360, 2.7960, 3.1174) -- (2.2360, 2.8500, 3.1114) -- (2.1900, 2.8500, 3.1159) -- cycle;
\fill[blue!22.7, opacity=0.7] (2.1900, 2.8500, 3.1159) -- (2.2360, 2.8500, 3.1114) -- (2.2360, 2.9040, 3.1052) -- (2.1900, 2.9040, 3.1098) -- cycle;
\fill[blue!36.1, opacity=0.7] (2.1900, 2.9040, 3.1098) -- (2.2360, 2.9040, 3.1052) -- (2.2360, 2.9580, 3.0991) -- (2.1900, 2.9580, 3.1036) -- cycle;
\fill[blue!54.5, opacity=0.7] (2.1900, 2.9580, 3.1036) -- (2.2360, 2.9580, 3.0991) -- (2.2360, 3.0120, 3.0928) -- (2.1900, 3.0120, 3.0974) -- cycle;
\fill[blue!61.2, opacity=0.7] (2.1900, 3.0120, 3.0974) -- (2.2360, 3.0120, 3.0928) -- (2.2360, 3.0660, 3.0866) -- (2.1900, 3.0660, 3.0911) -- cycle;
\fill[blue!57.5, opacity=0.7] (2.1900, 3.0660, 3.0911) -- (2.2360, 3.0660, 3.0866) -- (2.2360, 3.1200, 3.0803) -- (2.1900, 3.1200, 3.0849) -- cycle;
\fill[blue!16.1, opacity=0.7] (2.2360, -0.1200, 3.0803) -- (2.2820, -0.1200, 3.0755) -- (2.2820, -0.0660, 3.0818) -- (2.2360, -0.0660, 3.0866) -- cycle;
\fill[blue!19.3, opacity=0.7] (2.2360, -0.0660, 3.0866) -- (2.2820, -0.0660, 3.0818) -- (2.2820, -0.0120, 3.0881) -- (2.2360, -0.0120, 3.0928) -- cycle;
\fill[blue!32.6, opacity=0.7] (2.2360, -0.0120, 3.0928) -- (2.2820, -0.0120, 3.0881) -- (2.2820, 0.0420, 3.0943) -- (2.2360, 0.0420, 3.0991) -- cycle;
\fill[blue!54.3, opacity=0.7] (2.2360, 0.0420, 3.0991) -- (2.2820, 0.0420, 3.0943) -- (2.2820, 0.0960, 3.1005) -- (2.2360, 0.0960, 3.1052) -- cycle;
\fill[blue!63.2, opacity=0.7] (2.2360, 0.0960, 3.1052) -- (2.2820, 0.0960, 3.1005) -- (2.2820, 0.1500, 3.1066) -- (2.2360, 0.1500, 3.1114) -- cycle;
\fill[blue!63.6, opacity=0.7] (2.2360, 0.1500, 3.1114) -- (2.2820, 0.1500, 3.1066) -- (2.2820, 0.2040, 3.1126) -- (2.2360, 0.2040, 3.1174) -- cycle;
\fill[blue!62.5, opacity=0.7] (2.2360, 0.2040, 3.1174) -- (2.2820, 0.2040, 3.1126) -- (2.2820, 0.2580, 3.1185) -- (2.2360, 0.2580, 3.1233) -- cycle;
\fill[blue!53.1, opacity=0.7] (2.2360, 0.2580, 3.1233) -- (2.2820, 0.2580, 3.1185) -- (2.2820, 0.3120, 3.1243) -- (2.2360, 0.3120, 3.1291) -- cycle;
\fill[blue!37.3, opacity=0.7] (2.2360, 0.3120, 3.1291) -- (2.2820, 0.3120, 3.1243) -- (2.2820, 0.3660, 3.1300) -- (2.2360, 0.3660, 3.1348) -- cycle;
\fill[blue!27.1, opacity=0.7] (2.2360, 0.3660, 3.1348) -- (2.2820, 0.3660, 3.1300) -- (2.2820, 0.4200, 3.1355) -- (2.2360, 0.4200, 3.1403) -- cycle;
\fill[blue!24.3, opacity=0.7] (2.2360, 0.4200, 3.1403) -- (2.2820, 0.4200, 3.1355) -- (2.2820, 0.4740, 3.1409) -- (2.2360, 0.4740, 3.1457) -- cycle;
\fill[blue!27.2, opacity=0.7] (2.2360, 0.4740, 3.1457) -- (2.2820, 0.4740, 3.1409) -- (2.2820, 0.5280, 3.1461) -- (2.2360, 0.5280, 3.1508) -- cycle;
\fill[blue!37.6, opacity=0.7] (2.2360, 0.5280, 3.1508) -- (2.2820, 0.5280, 3.1461) -- (2.2820, 0.5820, 3.1510) -- (2.2360, 0.5820, 3.1558) -- cycle;
\fill[blue!54.3, opacity=0.7] (2.2360, 0.5820, 3.1558) -- (2.2820, 0.5820, 3.1510) -- (2.2820, 0.6360, 3.1558) -- (2.2360, 0.6360, 3.1606) -- cycle;
\fill[blue!63.5, opacity=0.7] (2.2360, 0.6360, 3.1606) -- (2.2820, 0.6360, 3.1558) -- (2.2820, 0.6900, 3.1604) -- (2.2360, 0.6900, 3.1651) -- cycle;
\fill[blue!57.8, opacity=0.7] (2.2360, 0.6900, 3.1651) -- (2.2820, 0.6900, 3.1604) -- (2.2820, 0.7440, 3.1647) -- (2.2360, 0.7440, 3.1695) -- cycle;
\fill[blue!47.7, opacity=0.7] (2.2360, 0.7440, 3.1695) -- (2.2820, 0.7440, 3.1647) -- (2.2820, 0.7980, 3.1688) -- (2.2360, 0.7980, 3.1736) -- cycle;
\fill[blue!41.8, opacity=0.7] (2.2360, 0.7980, 3.1736) -- (2.2820, 0.7980, 3.1688) -- (2.2820, 0.8520, 3.1726) -- (2.2360, 0.8520, 3.1774) -- cycle;
\fill[blue!40.9, opacity=0.7] (2.2360, 0.8520, 3.1774) -- (2.2820, 0.8520, 3.1726) -- (2.2820, 0.9060, 3.1762) -- (2.2360, 0.9060, 3.1809) -- cycle;
\fill[blue!43.8, opacity=0.7] (2.2360, 0.9060, 3.1809) -- (2.2820, 0.9060, 3.1762) -- (2.2820, 0.9600, 3.1794) -- (2.2360, 0.9600, 3.1842) -- cycle;
\fill[blue!49.0, opacity=0.7] (2.2360, 0.9600, 3.1842) -- (2.2820, 0.9600, 3.1794) -- (2.2820, 1.0140, 3.1824) -- (2.2360, 1.0140, 3.1872) -- cycle;
\fill[blue!54.8, opacity=0.7] (2.2360, 1.0140, 3.1872) -- (2.2820, 1.0140, 3.1824) -- (2.2820, 1.0680, 3.1851) -- (2.2360, 1.0680, 3.1899) -- cycle;
\fill[blue!59.4, opacity=0.7] (2.2360, 1.0680, 3.1899) -- (2.2820, 1.0680, 3.1851) -- (2.2820, 1.1220, 3.1875) -- (2.2360, 1.1220, 3.1923) -- cycle;
\fill[blue!62.1, opacity=0.7] (2.2360, 1.1220, 3.1923) -- (2.2820, 1.1220, 3.1875) -- (2.2820, 1.1760, 3.1896) -- (2.2360, 1.1760, 3.1944) -- cycle;
\fill[blue!63.3, opacity=0.7] (2.2360, 1.1760, 3.1944) -- (2.2820, 1.1760, 3.1896) -- (2.2820, 1.2300, 3.1914) -- (2.2360, 1.2300, 3.1962) -- cycle;
\fill[blue!63.6, opacity=0.7] (2.2360, 1.2300, 3.1962) -- (2.2820, 1.2300, 3.1914) -- (2.2820, 1.2840, 3.1929) -- (2.2360, 1.2840, 3.1977) -- cycle;
\fill[blue!63.6, opacity=0.7] (2.2360, 1.2840, 3.1977) -- (2.2820, 1.2840, 3.1929) -- (2.2820, 1.3380, 3.1940) -- (2.2360, 1.3380, 3.1988) -- cycle;
\fill[blue!63.6, opacity=0.7] (2.2360, 1.3380, 3.1988) -- (2.2820, 1.3380, 3.1940) -- (2.2820, 1.3920, 3.1949) -- (2.2360, 1.3920, 3.1996) -- cycle;
\fill[blue!63.5, opacity=0.7] (2.2360, 1.3920, 3.1996) -- (2.2820, 1.3920, 3.1949) -- (2.2820, 1.4460, 3.1954) -- (2.2360, 1.4460, 3.2001) -- cycle;
\fill[blue!63.2, opacity=0.7] (2.2360, 1.4460, 3.2001) -- (2.2820, 1.4460, 3.1954) -- (2.2820, 1.5000, 3.1955) -- (2.2360, 1.5000, 3.2003) -- cycle;
\fill[blue!61.9, opacity=0.7] (2.2360, 1.5000, 3.2003) -- (2.2820, 1.5000, 3.1955) -- (2.2820, 1.5540, 3.1954) -- (2.2360, 1.5540, 3.2001) -- cycle;
\fill[blue!58.8, opacity=0.7] (2.2360, 1.5540, 3.2001) -- (2.2820, 1.5540, 3.1954) -- (2.2820, 1.6080, 3.1949) -- (2.2360, 1.6080, 3.1996) -- cycle;
\fill[blue!53.5, opacity=0.7] (2.2360, 1.6080, 3.1996) -- (2.2820, 1.6080, 3.1949) -- (2.2820, 1.6620, 3.1940) -- (2.2360, 1.6620, 3.1988) -- cycle;
\fill[blue!46.3, opacity=0.7] (2.2360, 1.6620, 3.1988) -- (2.2820, 1.6620, 3.1940) -- (2.2820, 1.7160, 3.1929) -- (2.2360, 1.7160, 3.1977) -- cycle;
\fill[blue!39.2, opacity=0.7] (2.2360, 1.7160, 3.1977) -- (2.2820, 1.7160, 3.1929) -- (2.2820, 1.7700, 3.1914) -- (2.2360, 1.7700, 3.1962) -- cycle;
\fill[blue!34.0, opacity=0.7] (2.2360, 1.7700, 3.1962) -- (2.2820, 1.7700, 3.1914) -- (2.2820, 1.8240, 3.1896) -- (2.2360, 1.8240, 3.1944) -- cycle;
\fill[blue!32.2, opacity=0.7] (2.2360, 1.8240, 3.1944) -- (2.2820, 1.8240, 3.1896) -- (2.2820, 1.8780, 3.1875) -- (2.2360, 1.8780, 3.1923) -- cycle;
\fill[blue!34.7, opacity=0.7] (2.2360, 1.8780, 3.1923) -- (2.2820, 1.8780, 3.1875) -- (2.2820, 1.9320, 3.1851) -- (2.2360, 1.9320, 3.1899) -- cycle;
\fill[blue!43.2, opacity=0.7] (2.2360, 1.9320, 3.1899) -- (2.2820, 1.9320, 3.1851) -- (2.2820, 1.9860, 3.1824) -- (2.2360, 1.9860, 3.1872) -- cycle;
\fill[blue!56.7, opacity=0.7] (2.2360, 1.9860, 3.1872) -- (2.2820, 1.9860, 3.1824) -- (2.2820, 2.0400, 3.1794) -- (2.2360, 2.0400, 3.1842) -- cycle;
\fill[blue!63.5, opacity=0.7] (2.2360, 2.0400, 3.1842) -- (2.2820, 2.0400, 3.1794) -- (2.2820, 2.0940, 3.1762) -- (2.2360, 2.0940, 3.1809) -- cycle;
\fill[blue!53.7, opacity=0.7] (2.2360, 2.0940, 3.1809) -- (2.2820, 2.0940, 3.1762) -- (2.2820, 2.1480, 3.1726) -- (2.2360, 2.1480, 3.1774) -- cycle;
\fill[blue!40.4, opacity=0.7] (2.2360, 2.1480, 3.1774) -- (2.2820, 2.1480, 3.1726) -- (2.2820, 2.2020, 3.1688) -- (2.2360, 2.2020, 3.1736) -- cycle;
\fill[blue!35.1, opacity=0.7] (2.2360, 2.2020, 3.1736) -- (2.2820, 2.2020, 3.1688) -- (2.2820, 2.2560, 3.1647) -- (2.2360, 2.2560, 3.1695) -- cycle;
\fill[blue!39.3, opacity=0.7] (2.2360, 2.2560, 3.1695) -- (2.2820, 2.2560, 3.1647) -- (2.2820, 2.3100, 3.1604) -- (2.2360, 2.3100, 3.1651) -- cycle;
\fill[blue!52.2, opacity=0.7] (2.2360, 2.3100, 3.1651) -- (2.2820, 2.3100, 3.1604) -- (2.2820, 2.3640, 3.1558) -- (2.2360, 2.3640, 3.1606) -- cycle;
\fill[blue!63.1, opacity=0.7] (2.2360, 2.3640, 3.1606) -- (2.2820, 2.3640, 3.1558) -- (2.2820, 2.4180, 3.1510) -- (2.2360, 2.4180, 3.1558) -- cycle;
\fill[blue!60.1, opacity=0.7] (2.2360, 2.4180, 3.1558) -- (2.2820, 2.4180, 3.1510) -- (2.2820, 2.4720, 3.1461) -- (2.2360, 2.4720, 3.1508) -- cycle;
\fill[blue!55.5, opacity=0.7] (2.2360, 2.4720, 3.1508) -- (2.2820, 2.4720, 3.1461) -- (2.2820, 2.5260, 3.1409) -- (2.2360, 2.5260, 3.1457) -- cycle;
\fill[blue!59.9, opacity=0.7] (2.2360, 2.5260, 3.1457) -- (2.2820, 2.5260, 3.1409) -- (2.2820, 2.5800, 3.1355) -- (2.2360, 2.5800, 3.1403) -- cycle;
\fill[blue!62.5, opacity=0.7] (2.2360, 2.5800, 3.1403) -- (2.2820, 2.5800, 3.1355) -- (2.2820, 2.6340, 3.1300) -- (2.2360, 2.6340, 3.1348) -- cycle;
\fill[blue!43.5, opacity=0.7] (2.2360, 2.6340, 3.1348) -- (2.2820, 2.6340, 3.1300) -- (2.2820, 2.6880, 3.1243) -- (2.2360, 2.6880, 3.1291) -- cycle;
\fill[blue!24.9, opacity=0.7] (2.2360, 2.6880, 3.1291) -- (2.2820, 2.6880, 3.1243) -- (2.2820, 2.7420, 3.1185) -- (2.2360, 2.7420, 3.1233) -- cycle;
\fill[blue!19.3, opacity=0.7] (2.2360, 2.7420, 3.1233) -- (2.2820, 2.7420, 3.1185) -- (2.2820, 2.7960, 3.1126) -- (2.2360, 2.7960, 3.1174) -- cycle;
\fill[blue!20.0, opacity=0.7] (2.2360, 2.7960, 3.1174) -- (2.2820, 2.7960, 3.1126) -- (2.2820, 2.8500, 3.1066) -- (2.2360, 2.8500, 3.1114) -- cycle;
\fill[blue!27.7, opacity=0.7] (2.2360, 2.8500, 3.1114) -- (2.2820, 2.8500, 3.1066) -- (2.2820, 2.9040, 3.1005) -- (2.2360, 2.9040, 3.1052) -- cycle;
\fill[blue!45.5, opacity=0.7] (2.2360, 2.9040, 3.1052) -- (2.2820, 2.9040, 3.1005) -- (2.2820, 2.9580, 3.0943) -- (2.2360, 2.9580, 3.0991) -- cycle;
\fill[blue!59.0, opacity=0.7] (2.2360, 2.9580, 3.0991) -- (2.2820, 2.9580, 3.0943) -- (2.2820, 3.0120, 3.0881) -- (2.2360, 3.0120, 3.0928) -- cycle;
\fill[blue!60.7, opacity=0.7] (2.2360, 3.0120, 3.0928) -- (2.2820, 3.0120, 3.0881) -- (2.2820, 3.0660, 3.0818) -- (2.2360, 3.0660, 3.0866) -- cycle;
\fill[blue!50.8, opacity=0.7] (2.2360, 3.0660, 3.0866) -- (2.2820, 3.0660, 3.0818) -- (2.2820, 3.1200, 3.0755) -- (2.2360, 3.1200, 3.0803) -- cycle;
\fill[blue!15.8, opacity=0.7] (2.2820, -0.1200, 3.0755) -- (2.3280, -0.1200, 3.0705) -- (2.3280, -0.0660, 3.0768) -- (2.2820, -0.0660, 3.0818) -- cycle;
\fill[blue!16.6, opacity=0.7] (2.2820, -0.0660, 3.0818) -- (2.3280, -0.0660, 3.0768) -- (2.3280, -0.0120, 3.0831) -- (2.2820, -0.0120, 3.0881) -- cycle;
\fill[blue!21.6, opacity=0.7] (2.2820, -0.0120, 3.0881) -- (2.3280, -0.0120, 3.0831) -- (2.3280, 0.0420, 3.0893) -- (2.2820, 0.0420, 3.0943) -- cycle;
\fill[blue!38.3, opacity=0.7] (2.2820, 0.0420, 3.0943) -- (2.3280, 0.0420, 3.0893) -- (2.3280, 0.0960, 3.0955) -- (2.2820, 0.0960, 3.1005) -- cycle;
\fill[blue!57.9, opacity=0.7] (2.2820, 0.0960, 3.1005) -- (2.3280, 0.0960, 3.0955) -- (2.3280, 0.1500, 3.1016) -- (2.2820, 0.1500, 3.1066) -- cycle;
\fill[blue!63.5, opacity=0.7] (2.2820, 0.1500, 3.1066) -- (2.3280, 0.1500, 3.1016) -- (2.3280, 0.2040, 3.1076) -- (2.2820, 0.2040, 3.1126) -- cycle;
\fill[blue!63.6, opacity=0.7] (2.2820, 0.2040, 3.1126) -- (2.3280, 0.2040, 3.1076) -- (2.3280, 0.2580, 3.1135) -- (2.2820, 0.2580, 3.1185) -- cycle;
\fill[blue!62.3, opacity=0.7] (2.2820, 0.2580, 3.1185) -- (2.3280, 0.2580, 3.1135) -- (2.3280, 0.3120, 3.1193) -- (2.2820, 0.3120, 3.1243) -- cycle;
\fill[blue!52.6, opacity=0.7] (2.2820, 0.3120, 3.1243) -- (2.3280, 0.3120, 3.1193) -- (2.3280, 0.3660, 3.1250) -- (2.2820, 0.3660, 3.1300) -- cycle;
\fill[blue!37.6, opacity=0.7] (2.2820, 0.3660, 3.1300) -- (2.3280, 0.3660, 3.1250) -- (2.3280, 0.4200, 3.1305) -- (2.2820, 0.4200, 3.1355) -- cycle;
\fill[blue!27.7, opacity=0.7] (2.2820, 0.4200, 3.1355) -- (2.3280, 0.4200, 3.1305) -- (2.3280, 0.4740, 3.1359) -- (2.2820, 0.4740, 3.1409) -- cycle;
\fill[blue!24.6, opacity=0.7] (2.2820, 0.4740, 3.1409) -- (2.3280, 0.4740, 3.1359) -- (2.3280, 0.5280, 3.1411) -- (2.2820, 0.5280, 3.1461) -- cycle;
\fill[blue!26.7, opacity=0.7] (2.2820, 0.5280, 3.1461) -- (2.3280, 0.5280, 3.1411) -- (2.3280, 0.5820, 3.1461) -- (2.2820, 0.5820, 3.1510) -- cycle;
\fill[blue!34.8, opacity=0.7] (2.2820, 0.5820, 3.1510) -- (2.3280, 0.5820, 3.1461) -- (2.3280, 0.6360, 3.1508) -- (2.2820, 0.6360, 3.1558) -- cycle;
\fill[blue!49.4, opacity=0.7] (2.2820, 0.6360, 3.1558) -- (2.3280, 0.6360, 3.1508) -- (2.3280, 0.6900, 3.1554) -- (2.2820, 0.6900, 3.1604) -- cycle;
\fill[blue!61.9, opacity=0.7] (2.2820, 0.6900, 3.1604) -- (2.3280, 0.6900, 3.1554) -- (2.3280, 0.7440, 3.1597) -- (2.2820, 0.7440, 3.1647) -- cycle;
\fill[blue!62.0, opacity=0.7] (2.2820, 0.7440, 3.1647) -- (2.3280, 0.7440, 3.1597) -- (2.3280, 0.7980, 3.1638) -- (2.2820, 0.7980, 3.1688) -- cycle;
\fill[blue!53.5, opacity=0.7] (2.2820, 0.7980, 3.1688) -- (2.3280, 0.7980, 3.1638) -- (2.3280, 0.8520, 3.1676) -- (2.2820, 0.8520, 3.1726) -- cycle;
\fill[blue!45.3, opacity=0.7] (2.2820, 0.8520, 3.1726) -- (2.3280, 0.8520, 3.1676) -- (2.3280, 0.9060, 3.1712) -- (2.2820, 0.9060, 3.1762) -- cycle;
\fill[blue!40.8, opacity=0.7] (2.2820, 0.9060, 3.1762) -- (2.3280, 0.9060, 3.1712) -- (2.3280, 0.9600, 3.1745) -- (2.2820, 0.9600, 3.1794) -- cycle;
\fill[blue!39.6, opacity=0.7] (2.2820, 0.9600, 3.1794) -- (2.3280, 0.9600, 3.1745) -- (2.3280, 1.0140, 3.1775) -- (2.2820, 1.0140, 3.1824) -- cycle;
\fill[blue!40.8, opacity=0.7] (2.2820, 1.0140, 3.1824) -- (2.3280, 1.0140, 3.1775) -- (2.3280, 1.0680, 3.1802) -- (2.2820, 1.0680, 3.1851) -- cycle;
\fill[blue!43.3, opacity=0.7] (2.2820, 1.0680, 3.1851) -- (2.3280, 1.0680, 3.1802) -- (2.3280, 1.1220, 3.1826) -- (2.2820, 1.1220, 3.1875) -- cycle;
\fill[blue!46.3, opacity=0.7] (2.2820, 1.1220, 3.1875) -- (2.3280, 1.1220, 3.1826) -- (2.3280, 1.1760, 3.1847) -- (2.2820, 1.1760, 3.1896) -- cycle;
\fill[blue!49.0, opacity=0.7] (2.2820, 1.1760, 3.1896) -- (2.3280, 1.1760, 3.1847) -- (2.3280, 1.2300, 3.1864) -- (2.2820, 1.2300, 3.1914) -- cycle;
\fill[blue!51.0, opacity=0.7] (2.2820, 1.2300, 3.1914) -- (2.3280, 1.2300, 3.1864) -- (2.3280, 1.2840, 3.1879) -- (2.2820, 1.2840, 3.1929) -- cycle;
\fill[blue!52.0, opacity=0.7] (2.2820, 1.2840, 3.1929) -- (2.3280, 1.2840, 3.1879) -- (2.3280, 1.3380, 3.1891) -- (2.2820, 1.3380, 3.1940) -- cycle;
\fill[blue!51.9, opacity=0.7] (2.2820, 1.3380, 3.1940) -- (2.3280, 1.3380, 3.1891) -- (2.3280, 1.3920, 3.1899) -- (2.2820, 1.3920, 3.1949) -- cycle;
\fill[blue!50.7, opacity=0.7] (2.2820, 1.3920, 3.1949) -- (2.3280, 1.3920, 3.1899) -- (2.3280, 1.4460, 3.1904) -- (2.2820, 1.4460, 3.1954) -- cycle;
\fill[blue!48.4, opacity=0.7] (2.2820, 1.4460, 3.1954) -- (2.3280, 1.4460, 3.1904) -- (2.3280, 1.5000, 3.1905) -- (2.2820, 1.5000, 3.1955) -- cycle;
\fill[blue!45.1, opacity=0.7] (2.2820, 1.5000, 3.1955) -- (2.3280, 1.5000, 3.1905) -- (2.3280, 1.5540, 3.1904) -- (2.2820, 1.5540, 3.1954) -- cycle;
\fill[blue!41.1, opacity=0.7] (2.2820, 1.5540, 3.1954) -- (2.3280, 1.5540, 3.1904) -- (2.3280, 1.6080, 3.1899) -- (2.2820, 1.6080, 3.1949) -- cycle;
\fill[blue!37.2, opacity=0.7] (2.2820, 1.6080, 3.1949) -- (2.3280, 1.6080, 3.1899) -- (2.3280, 1.6620, 3.1891) -- (2.2820, 1.6620, 3.1940) -- cycle;
\fill[blue!34.3, opacity=0.7] (2.2820, 1.6620, 3.1940) -- (2.3280, 1.6620, 3.1891) -- (2.3280, 1.7160, 3.1879) -- (2.2820, 1.7160, 3.1929) -- cycle;
\fill[blue!33.2, opacity=0.7] (2.2820, 1.7160, 3.1929) -- (2.3280, 1.7160, 3.1879) -- (2.3280, 1.7700, 3.1864) -- (2.2820, 1.7700, 3.1914) -- cycle;
\fill[blue!34.7, opacity=0.7] (2.2820, 1.7700, 3.1914) -- (2.3280, 1.7700, 3.1864) -- (2.3280, 1.8240, 3.1847) -- (2.2820, 1.8240, 3.1896) -- cycle;
\fill[blue!40.2, opacity=0.7] (2.2820, 1.8240, 3.1896) -- (2.3280, 1.8240, 3.1847) -- (2.3280, 1.8780, 3.1826) -- (2.2820, 1.8780, 3.1875) -- cycle;
\fill[blue!50.4, opacity=0.7] (2.2820, 1.8780, 3.1875) -- (2.3280, 1.8780, 3.1826) -- (2.3280, 1.9320, 3.1802) -- (2.2820, 1.9320, 3.1851) -- cycle;
\fill[blue!61.3, opacity=0.7] (2.2820, 1.9320, 3.1851) -- (2.3280, 1.9320, 3.1802) -- (2.3280, 1.9860, 3.1775) -- (2.2820, 1.9860, 3.1824) -- cycle;
\fill[blue!62.1, opacity=0.7] (2.2820, 1.9860, 3.1824) -- (2.3280, 1.9860, 3.1775) -- (2.3280, 2.0400, 3.1745) -- (2.2820, 2.0400, 3.1794) -- cycle;
\fill[blue!50.6, opacity=0.7] (2.2820, 2.0400, 3.1794) -- (2.3280, 2.0400, 3.1745) -- (2.3280, 2.0940, 3.1712) -- (2.2820, 2.0940, 3.1762) -- cycle;
\fill[blue!38.9, opacity=0.7] (2.2820, 2.0940, 3.1762) -- (2.3280, 2.0940, 3.1712) -- (2.3280, 2.1480, 3.1676) -- (2.2820, 2.1480, 3.1726) -- cycle;
\fill[blue!34.5, opacity=0.7] (2.2820, 2.1480, 3.1726) -- (2.3280, 2.1480, 3.1676) -- (2.3280, 2.2020, 3.1638) -- (2.2820, 2.2020, 3.1688) -- cycle;
\fill[blue!38.3, opacity=0.7] (2.2820, 2.2020, 3.1688) -- (2.3280, 2.2020, 3.1638) -- (2.3280, 2.2560, 3.1597) -- (2.2820, 2.2560, 3.1647) -- cycle;
\fill[blue!50.2, opacity=0.7] (2.2820, 2.2560, 3.1647) -- (2.3280, 2.2560, 3.1597) -- (2.3280, 2.3100, 3.1554) -- (2.2820, 2.3100, 3.1604) -- cycle;
\fill[blue!62.2, opacity=0.7] (2.2820, 2.3100, 3.1604) -- (2.3280, 2.3100, 3.1554) -- (2.3280, 2.3640, 3.1508) -- (2.2820, 2.3640, 3.1558) -- cycle;
\fill[blue!61.7, opacity=0.7] (2.2820, 2.3640, 3.1558) -- (2.3280, 2.3640, 3.1508) -- (2.3280, 2.4180, 3.1461) -- (2.2820, 2.4180, 3.1510) -- cycle;
\fill[blue!56.3, opacity=0.7] (2.2820, 2.4180, 3.1510) -- (2.3280, 2.4180, 3.1461) -- (2.3280, 2.4720, 3.1411) -- (2.2820, 2.4720, 3.1461) -- cycle;
\fill[blue!58.3, opacity=0.7] (2.2820, 2.4720, 3.1461) -- (2.3280, 2.4720, 3.1411) -- (2.3280, 2.5260, 3.1359) -- (2.2820, 2.5260, 3.1409) -- cycle;
\fill[blue!63.6, opacity=0.7] (2.2820, 2.5260, 3.1409) -- (2.3280, 2.5260, 3.1359) -- (2.3280, 2.5800, 3.1305) -- (2.2820, 2.5800, 3.1355) -- cycle;
\fill[blue!51.3, opacity=0.7] (2.2820, 2.5800, 3.1355) -- (2.3280, 2.5800, 3.1305) -- (2.3280, 2.6340, 3.1250) -- (2.2820, 2.6340, 3.1300) -- cycle;
\fill[blue!29.6, opacity=0.7] (2.2820, 2.6340, 3.1300) -- (2.3280, 2.6340, 3.1250) -- (2.3280, 2.6880, 3.1193) -- (2.2820, 2.6880, 3.1243) -- cycle;
\fill[blue!20.3, opacity=0.7] (2.2820, 2.6880, 3.1243) -- (2.3280, 2.6880, 3.1193) -- (2.3280, 2.7420, 3.1135) -- (2.2820, 2.7420, 3.1185) -- cycle;
\fill[blue!19.0, opacity=0.7] (2.2820, 2.7420, 3.1185) -- (2.3280, 2.7420, 3.1135) -- (2.3280, 2.7960, 3.1076) -- (2.2820, 2.7960, 3.1126) -- cycle;
\fill[blue!22.9, opacity=0.7] (2.2820, 2.7960, 3.1126) -- (2.3280, 2.7960, 3.1076) -- (2.3280, 2.8500, 3.1016) -- (2.2820, 2.8500, 3.1066) -- cycle;
\fill[blue!36.2, opacity=0.7] (2.2820, 2.8500, 3.1066) -- (2.3280, 2.8500, 3.1016) -- (2.3280, 2.9040, 3.0955) -- (2.2820, 2.9040, 3.1005) -- cycle;
\fill[blue!54.0, opacity=0.7] (2.2820, 2.9040, 3.1005) -- (2.3280, 2.9040, 3.0955) -- (2.3280, 2.9580, 3.0893) -- (2.2820, 2.9580, 3.0943) -- cycle;
\fill[blue!60.8, opacity=0.7] (2.2820, 2.9580, 3.0943) -- (2.3280, 2.9580, 3.0893) -- (2.3280, 3.0120, 3.0831) -- (2.2820, 3.0120, 3.0881) -- cycle;
\fill[blue!57.7, opacity=0.7] (2.2820, 3.0120, 3.0881) -- (2.3280, 3.0120, 3.0831) -- (2.3280, 3.0660, 3.0768) -- (2.2820, 3.0660, 3.0818) -- cycle;
\fill[blue!40.3, opacity=0.7] (2.2820, 3.0660, 3.0818) -- (2.3280, 3.0660, 3.0768) -- (2.3280, 3.1200, 3.0705) -- (2.2820, 3.1200, 3.0755) -- cycle;
\fill[blue!16.1, opacity=0.7] (2.3280, -0.1200, 3.0705) -- (2.3740, -0.1200, 3.0654) -- (2.3740, -0.0660, 3.0716) -- (2.3280, -0.0660, 3.0768) -- cycle;
\fill[blue!15.9, opacity=0.7] (2.3280, -0.0660, 3.0768) -- (2.3740, -0.0660, 3.0716) -- (2.3740, -0.0120, 3.0779) -- (2.3280, -0.0120, 3.0831) -- cycle;
\fill[blue!17.2, opacity=0.7] (2.3280, -0.0120, 3.0831) -- (2.3740, -0.0120, 3.0779) -- (2.3740, 0.0420, 3.0841) -- (2.3280, 0.0420, 3.0893) -- cycle;
\fill[blue!24.1, opacity=0.7] (2.3280, 0.0420, 3.0893) -- (2.3740, 0.0420, 3.0841) -- (2.3740, 0.0960, 3.0903) -- (2.3280, 0.0960, 3.0955) -- cycle;
\fill[blue!42.6, opacity=0.7] (2.3280, 0.0960, 3.0955) -- (2.3740, 0.0960, 3.0903) -- (2.3740, 0.1500, 3.0964) -- (2.3280, 0.1500, 3.1016) -- cycle;
\fill[blue!59.8, opacity=0.7] (2.3280, 0.1500, 3.1016) -- (2.3740, 0.1500, 3.0964) -- (2.3740, 0.2040, 3.1024) -- (2.3280, 0.2040, 3.1076) -- cycle;
\fill[blue!63.6, opacity=0.7] (2.3280, 0.2040, 3.1076) -- (2.3740, 0.2040, 3.1024) -- (2.3740, 0.2580, 3.1084) -- (2.3280, 0.2580, 3.1135) -- cycle;
\fill[blue!63.5, opacity=0.7] (2.3280, 0.2580, 3.1135) -- (2.3740, 0.2580, 3.1084) -- (2.3740, 0.3120, 3.1142) -- (2.3280, 0.3120, 3.1193) -- cycle;
\fill[blue!62.4, opacity=0.7] (2.3280, 0.3120, 3.1193) -- (2.3740, 0.3120, 3.1142) -- (2.3740, 0.3660, 3.1198) -- (2.3280, 0.3660, 3.1250) -- cycle;
\fill[blue!53.6, opacity=0.7] (2.3280, 0.3660, 3.1250) -- (2.3740, 0.3660, 3.1198) -- (2.3740, 0.4200, 3.1254) -- (2.3280, 0.4200, 3.1305) -- cycle;
\fill[blue!39.4, opacity=0.7] (2.3280, 0.4200, 3.1305) -- (2.3740, 0.4200, 3.1254) -- (2.3740, 0.4740, 3.1307) -- (2.3280, 0.4740, 3.1359) -- cycle;
\fill[blue!29.2, opacity=0.7] (2.3280, 0.4740, 3.1359) -- (2.3740, 0.4740, 3.1307) -- (2.3740, 0.5280, 3.1359) -- (2.3280, 0.5280, 3.1411) -- cycle;
\fill[blue!25.2, opacity=0.7] (2.3280, 0.5280, 3.1411) -- (2.3740, 0.5280, 3.1359) -- (2.3740, 0.5820, 3.1409) -- (2.3280, 0.5820, 3.1461) -- cycle;
\fill[blue!25.9, opacity=0.7] (2.3280, 0.5820, 3.1461) -- (2.3740, 0.5820, 3.1409) -- (2.3740, 0.6360, 3.1457) -- (2.3280, 0.6360, 3.1508) -- cycle;
\fill[blue!31.3, opacity=0.7] (2.3280, 0.6360, 3.1508) -- (2.3740, 0.6360, 3.1457) -- (2.3740, 0.6900, 3.1502) -- (2.3280, 0.6900, 3.1554) -- cycle;
\fill[blue!42.4, opacity=0.7] (2.3280, 0.6900, 3.1554) -- (2.3740, 0.6900, 3.1502) -- (2.3740, 0.7440, 3.1545) -- (2.3280, 0.7440, 3.1597) -- cycle;
\fill[blue!55.9, opacity=0.7] (2.3280, 0.7440, 3.1597) -- (2.3740, 0.7440, 3.1545) -- (2.3740, 0.7980, 3.1586) -- (2.3280, 0.7980, 3.1638) -- cycle;
\fill[blue!63.3, opacity=0.7] (2.3280, 0.7980, 3.1638) -- (2.3740, 0.7980, 3.1586) -- (2.3740, 0.8520, 3.1624) -- (2.3280, 0.8520, 3.1676) -- cycle;
\fill[blue!60.9, opacity=0.7] (2.3280, 0.8520, 3.1676) -- (2.3740, 0.8520, 3.1624) -- (2.3740, 0.9060, 3.1660) -- (2.3280, 0.9060, 3.1712) -- cycle;
\fill[blue!53.4, opacity=0.7] (2.3280, 0.9060, 3.1712) -- (2.3740, 0.9060, 3.1660) -- (2.3740, 0.9600, 3.1693) -- (2.3280, 0.9600, 3.1745) -- cycle;
\fill[blue!46.4, opacity=0.7] (2.3280, 0.9600, 3.1745) -- (2.3740, 0.9600, 3.1693) -- (2.3740, 1.0140, 3.1723) -- (2.3280, 1.0140, 3.1775) -- cycle;
\fill[blue!41.7, opacity=0.7] (2.3280, 1.0140, 3.1775) -- (2.3740, 1.0140, 3.1723) -- (2.3740, 1.0680, 3.1750) -- (2.3280, 1.0680, 3.1802) -- cycle;
\fill[blue!39.2, opacity=0.7] (2.3280, 1.0680, 3.1802) -- (2.3740, 1.0680, 3.1750) -- (2.3740, 1.1220, 3.1774) -- (2.3280, 1.1220, 3.1826) -- cycle;
\fill[blue!38.1, opacity=0.7] (2.3280, 1.1220, 3.1826) -- (2.3740, 1.1220, 3.1774) -- (2.3740, 1.1760, 3.1795) -- (2.3280, 1.1760, 3.1847) -- cycle;
\fill[blue!37.9, opacity=0.7] (2.3280, 1.1760, 3.1847) -- (2.3740, 1.1760, 3.1795) -- (2.3740, 1.2300, 3.1813) -- (2.3280, 1.2300, 3.1864) -- cycle;
\fill[blue!37.9, opacity=0.7] (2.3280, 1.2300, 3.1864) -- (2.3740, 1.2300, 3.1813) -- (2.3740, 1.2840, 3.1827) -- (2.3280, 1.2840, 3.1879) -- cycle;
\fill[blue!37.8, opacity=0.7] (2.3280, 1.2840, 3.1879) -- (2.3740, 1.2840, 3.1827) -- (2.3740, 1.3380, 3.1839) -- (2.3280, 1.3380, 3.1891) -- cycle;
\fill[blue!37.5, opacity=0.7] (2.3280, 1.3380, 3.1891) -- (2.3740, 1.3380, 3.1839) -- (2.3740, 1.3920, 3.1847) -- (2.3280, 1.3920, 3.1899) -- cycle;
\fill[blue!36.8, opacity=0.7] (2.3280, 1.3920, 3.1899) -- (2.3740, 1.3920, 3.1847) -- (2.3740, 1.4460, 3.1852) -- (2.3280, 1.4460, 3.1904) -- cycle;
\fill[blue!35.9, opacity=0.7] (2.3280, 1.4460, 3.1904) -- (2.3740, 1.4460, 3.1852) -- (2.3740, 1.5000, 3.1854) -- (2.3280, 1.5000, 3.1905) -- cycle;
\fill[blue!35.0, opacity=0.7] (2.3280, 1.5000, 3.1905) -- (2.3740, 1.5000, 3.1854) -- (2.3740, 1.5540, 3.1852) -- (2.3280, 1.5540, 3.1904) -- cycle;
\fill[blue!34.6, opacity=0.7] (2.3280, 1.5540, 3.1904) -- (2.3740, 1.5540, 3.1852) -- (2.3740, 1.6080, 3.1847) -- (2.3280, 1.6080, 3.1899) -- cycle;
\fill[blue!35.2, opacity=0.7] (2.3280, 1.6080, 3.1899) -- (2.3740, 1.6080, 3.1847) -- (2.3740, 1.6620, 3.1839) -- (2.3280, 1.6620, 3.1891) -- cycle;
\fill[blue!37.5, opacity=0.7] (2.3280, 1.6620, 3.1891) -- (2.3740, 1.6620, 3.1839) -- (2.3740, 1.7160, 3.1827) -- (2.3280, 1.7160, 3.1879) -- cycle;
\fill[blue!42.3, opacity=0.7] (2.3280, 1.7160, 3.1879) -- (2.3740, 1.7160, 3.1827) -- (2.3740, 1.7700, 3.1813) -- (2.3280, 1.7700, 3.1864) -- cycle;
\fill[blue!50.3, opacity=0.7] (2.3280, 1.7700, 3.1864) -- (2.3740, 1.7700, 3.1813) -- (2.3740, 1.8240, 3.1795) -- (2.3280, 1.8240, 3.1847) -- cycle;
\fill[blue!59.4, opacity=0.7] (2.3280, 1.8240, 3.1847) -- (2.3740, 1.8240, 3.1795) -- (2.3740, 1.8780, 3.1774) -- (2.3280, 1.8780, 3.1826) -- cycle;
\fill[blue!63.5, opacity=0.7] (2.3280, 1.8780, 3.1826) -- (2.3740, 1.8780, 3.1774) -- (2.3740, 1.9320, 3.1750) -- (2.3280, 1.9320, 3.1802) -- cycle;
\fill[blue!57.3, opacity=0.7] (2.3280, 1.9320, 3.1802) -- (2.3740, 1.9320, 3.1750) -- (2.3740, 1.9860, 3.1723) -- (2.3280, 1.9860, 3.1775) -- cycle;
\fill[blue!45.3, opacity=0.7] (2.3280, 1.9860, 3.1775) -- (2.3740, 1.9860, 3.1723) -- (2.3740, 2.0400, 3.1693) -- (2.3280, 2.0400, 3.1745) -- cycle;
\fill[blue!36.5, opacity=0.7] (2.3280, 2.0400, 3.1745) -- (2.3740, 2.0400, 3.1693) -- (2.3740, 2.0940, 3.1660) -- (2.3280, 2.0940, 3.1712) -- cycle;
\fill[blue!34.0, opacity=0.7] (2.3280, 2.0940, 3.1712) -- (2.3740, 2.0940, 3.1660) -- (2.3740, 2.1480, 3.1624) -- (2.3280, 2.1480, 3.1676) -- cycle;
\fill[blue!38.4, opacity=0.7] (2.3280, 2.1480, 3.1676) -- (2.3740, 2.1480, 3.1624) -- (2.3740, 2.2020, 3.1586) -- (2.3280, 2.2020, 3.1638) -- cycle;
\fill[blue!49.9, opacity=0.7] (2.3280, 2.2020, 3.1638) -- (2.3740, 2.2020, 3.1586) -- (2.3740, 2.2560, 3.1545) -- (2.3280, 2.2560, 3.1597) -- cycle;
\fill[blue!61.6, opacity=0.7] (2.3280, 2.2560, 3.1597) -- (2.3740, 2.2560, 3.1545) -- (2.3740, 2.3100, 3.1502) -- (2.3280, 2.3100, 3.1554) -- cycle;
\fill[blue!62.4, opacity=0.7] (2.3280, 2.3100, 3.1554) -- (2.3740, 2.3100, 3.1502) -- (2.3740, 2.3640, 3.1457) -- (2.3280, 2.3640, 3.1508) -- cycle;
\fill[blue!57.1, opacity=0.7] (2.3280, 2.3640, 3.1508) -- (2.3740, 2.3640, 3.1457) -- (2.3740, 2.4180, 3.1409) -- (2.3280, 2.4180, 3.1461) -- cycle;
\fill[blue!57.6, opacity=0.7] (2.3280, 2.4180, 3.1461) -- (2.3740, 2.4180, 3.1409) -- (2.3740, 2.4720, 3.1359) -- (2.3280, 2.4720, 3.1411) -- cycle;
\fill[blue!63.2, opacity=0.7] (2.3280, 2.4720, 3.1411) -- (2.3740, 2.4720, 3.1359) -- (2.3740, 2.5260, 3.1307) -- (2.3280, 2.5260, 3.1359) -- cycle;
\fill[blue!56.4, opacity=0.7] (2.3280, 2.5260, 3.1359) -- (2.3740, 2.5260, 3.1307) -- (2.3740, 2.5800, 3.1254) -- (2.3280, 2.5800, 3.1305) -- cycle;
\fill[blue!34.6, opacity=0.7] (2.3280, 2.5800, 3.1305) -- (2.3740, 2.5800, 3.1254) -- (2.3740, 2.6340, 3.1198) -- (2.3280, 2.6340, 3.1250) -- cycle;
\fill[blue!21.9, opacity=0.7] (2.3280, 2.6340, 3.1250) -- (2.3740, 2.6340, 3.1198) -- (2.3740, 2.6880, 3.1142) -- (2.3280, 2.6880, 3.1193) -- cycle;
\fill[blue!18.8, opacity=0.7] (2.3280, 2.6880, 3.1193) -- (2.3740, 2.6880, 3.1142) -- (2.3740, 2.7420, 3.1084) -- (2.3280, 2.7420, 3.1135) -- cycle;
\fill[blue!20.5, opacity=0.7] (2.3280, 2.7420, 3.1135) -- (2.3740, 2.7420, 3.1084) -- (2.3740, 2.7960, 3.1024) -- (2.3280, 2.7960, 3.1076) -- cycle;
\fill[blue!29.3, opacity=0.7] (2.3280, 2.7960, 3.1076) -- (2.3740, 2.7960, 3.1024) -- (2.3740, 2.8500, 3.0964) -- (2.3280, 2.8500, 3.1016) -- cycle;
\fill[blue!46.9, opacity=0.7] (2.3280, 2.8500, 3.1016) -- (2.3740, 2.8500, 3.0964) -- (2.3740, 2.9040, 3.0903) -- (2.3280, 2.9040, 3.0955) -- cycle;
\fill[blue!59.1, opacity=0.7] (2.3280, 2.9040, 3.0955) -- (2.3740, 2.9040, 3.0903) -- (2.3740, 2.9580, 3.0841) -- (2.3280, 2.9580, 3.0893) -- cycle;
\fill[blue!60.2, opacity=0.7] (2.3280, 2.9580, 3.0893) -- (2.3740, 2.9580, 3.0841) -- (2.3740, 3.0120, 3.0779) -- (2.3280, 3.0120, 3.0831) -- cycle;
\fill[blue!50.2, opacity=0.7] (2.3280, 3.0120, 3.0831) -- (2.3740, 3.0120, 3.0779) -- (2.3740, 3.0660, 3.0716) -- (2.3280, 3.0660, 3.0768) -- cycle;
\fill[blue!28.8, opacity=0.7] (2.3280, 3.0660, 3.0768) -- (2.3740, 3.0660, 3.0716) -- (2.3740, 3.1200, 3.0654) -- (2.3280, 3.1200, 3.0705) -- cycle;
\fill[blue!17.4, opacity=0.7] (2.3740, -0.1200, 3.0654) -- (2.4200, -0.1200, 3.0600) -- (2.4200, -0.0660, 3.0663) -- (2.3740, -0.0660, 3.0716) -- cycle;
\fill[blue!16.0, opacity=0.7] (2.3740, -0.0660, 3.0716) -- (2.4200, -0.0660, 3.0663) -- (2.4200, -0.0120, 3.0725) -- (2.3740, -0.0120, 3.0779) -- cycle;
\fill[blue!16.0, opacity=0.7] (2.3740, -0.0120, 3.0779) -- (2.4200, -0.0120, 3.0725) -- (2.4200, 0.0420, 3.0788) -- (2.3740, 0.0420, 3.0841) -- cycle;
\fill[blue!17.9, opacity=0.7] (2.3740, 0.0420, 3.0841) -- (2.4200, 0.0420, 3.0788) -- (2.4200, 0.0960, 3.0849) -- (2.3740, 0.0960, 3.0903) -- cycle;
\fill[blue!26.1, opacity=0.7] (2.3740, 0.0960, 3.0903) -- (2.4200, 0.0960, 3.0849) -- (2.4200, 0.1500, 3.0911) -- (2.3740, 0.1500, 3.0964) -- cycle;
\fill[blue!45.0, opacity=0.7] (2.3740, 0.1500, 3.0964) -- (2.4200, 0.1500, 3.0911) -- (2.4200, 0.2040, 3.0971) -- (2.3740, 0.2040, 3.1024) -- cycle;
\fill[blue!60.5, opacity=0.7] (2.3740, 0.2040, 3.1024) -- (2.4200, 0.2040, 3.0971) -- (2.4200, 0.2580, 3.1030) -- (2.3740, 0.2580, 3.1084) -- cycle;
\fill[blue!63.6, opacity=0.7] (2.3740, 0.2580, 3.1084) -- (2.4200, 0.2580, 3.1030) -- (2.4200, 0.3120, 3.1088) -- (2.3740, 0.3120, 3.1142) -- cycle;
\fill[blue!63.5, opacity=0.7] (2.3740, 0.3120, 3.1142) -- (2.4200, 0.3120, 3.1088) -- (2.4200, 0.3660, 3.1145) -- (2.3740, 0.3660, 3.1198) -- cycle;
\fill[blue!62.8, opacity=0.7] (2.3740, 0.3660, 3.1198) -- (2.4200, 0.3660, 3.1145) -- (2.4200, 0.4200, 3.1200) -- (2.3740, 0.4200, 3.1254) -- cycle;
\fill[blue!55.9, opacity=0.7] (2.3740, 0.4200, 3.1254) -- (2.4200, 0.4200, 3.1200) -- (2.4200, 0.4740, 3.1254) -- (2.3740, 0.4740, 3.1307) -- cycle;
\fill[blue!42.9, opacity=0.7] (2.3740, 0.4740, 3.1307) -- (2.4200, 0.4740, 3.1254) -- (2.4200, 0.5280, 3.1305) -- (2.3740, 0.5280, 3.1359) -- cycle;
\fill[blue!31.9, opacity=0.7] (2.3740, 0.5280, 3.1359) -- (2.4200, 0.5280, 3.1305) -- (2.4200, 0.5820, 3.1355) -- (2.3740, 0.5820, 3.1409) -- cycle;
\fill[blue!26.4, opacity=0.7] (2.3740, 0.5820, 3.1409) -- (2.4200, 0.5820, 3.1355) -- (2.4200, 0.6360, 3.1403) -- (2.3740, 0.6360, 3.1457) -- cycle;
\fill[blue!25.4, opacity=0.7] (2.3740, 0.6360, 3.1457) -- (2.4200, 0.6360, 3.1403) -- (2.4200, 0.6900, 3.1449) -- (2.3740, 0.6900, 3.1502) -- cycle;
\fill[blue!28.1, opacity=0.7] (2.3740, 0.6900, 3.1502) -- (2.4200, 0.6900, 3.1449) -- (2.4200, 0.7440, 3.1492) -- (2.3740, 0.7440, 3.1545) -- cycle;
\fill[blue!34.8, opacity=0.7] (2.3740, 0.7440, 3.1545) -- (2.4200, 0.7440, 3.1492) -- (2.4200, 0.7980, 3.1533) -- (2.3740, 0.7980, 3.1586) -- cycle;
\fill[blue!45.6, opacity=0.7] (2.3740, 0.7980, 3.1586) -- (2.4200, 0.7980, 3.1533) -- (2.4200, 0.8520, 3.1571) -- (2.3740, 0.8520, 3.1624) -- cycle;
\fill[blue!56.7, opacity=0.7] (2.3740, 0.8520, 3.1624) -- (2.4200, 0.8520, 3.1571) -- (2.4200, 0.9060, 3.1606) -- (2.3740, 0.9060, 3.1660) -- cycle;
\fill[blue!63.0, opacity=0.7] (2.3740, 0.9060, 3.1660) -- (2.4200, 0.9060, 3.1606) -- (2.4200, 0.9600, 3.1639) -- (2.3740, 0.9600, 3.1693) -- cycle;
\fill[blue!62.6, opacity=0.7] (2.3740, 0.9600, 3.1693) -- (2.4200, 0.9600, 3.1639) -- (2.4200, 1.0140, 3.1669) -- (2.3740, 1.0140, 3.1723) -- cycle;
\fill[blue!58.0, opacity=0.7] (2.3740, 1.0140, 3.1723) -- (2.4200, 1.0140, 3.1669) -- (2.4200, 1.0680, 3.1696) -- (2.3740, 1.0680, 3.1750) -- cycle;
\fill[blue!52.5, opacity=0.7] (2.3740, 1.0680, 3.1750) -- (2.4200, 1.0680, 3.1696) -- (2.4200, 1.1220, 3.1720) -- (2.3740, 1.1220, 3.1774) -- cycle;
\fill[blue!47.9, opacity=0.7] (2.3740, 1.1220, 3.1774) -- (2.4200, 1.1220, 3.1720) -- (2.4200, 1.1760, 3.1741) -- (2.3740, 1.1760, 3.1795) -- cycle;
\fill[blue!44.5, opacity=0.7] (2.3740, 1.1760, 3.1795) -- (2.4200, 1.1760, 3.1741) -- (2.4200, 1.2300, 3.1759) -- (2.3740, 1.2300, 3.1813) -- cycle;
\fill[blue!42.2, opacity=0.7] (2.3740, 1.2300, 3.1813) -- (2.4200, 1.2300, 3.1759) -- (2.4200, 1.2840, 3.1774) -- (2.3740, 1.2840, 3.1827) -- cycle;
\fill[blue!40.9, opacity=0.7] (2.3740, 1.2840, 3.1827) -- (2.4200, 1.2840, 3.1774) -- (2.4200, 1.3380, 3.1785) -- (2.3740, 1.3380, 3.1839) -- cycle;
\fill[blue!40.2, opacity=0.7] (2.3740, 1.3380, 3.1839) -- (2.4200, 1.3380, 3.1785) -- (2.4200, 1.3920, 3.1793) -- (2.3740, 1.3920, 3.1847) -- cycle;
\fill[blue!40.1, opacity=0.7] (2.3740, 1.3920, 3.1847) -- (2.4200, 1.3920, 3.1793) -- (2.4200, 1.4460, 3.1798) -- (2.3740, 1.4460, 3.1852) -- cycle;
\fill[blue!40.8, opacity=0.7] (2.3740, 1.4460, 3.1852) -- (2.4200, 1.4460, 3.1798) -- (2.4200, 1.5000, 3.1800) -- (2.3740, 1.5000, 3.1854) -- cycle;
\fill[blue!42.5, opacity=0.7] (2.3740, 1.5000, 3.1854) -- (2.4200, 1.5000, 3.1800) -- (2.4200, 1.5540, 3.1798) -- (2.3740, 1.5540, 3.1852) -- cycle;
\fill[blue!45.5, opacity=0.7] (2.3740, 1.5540, 3.1852) -- (2.4200, 1.5540, 3.1798) -- (2.4200, 1.6080, 3.1793) -- (2.3740, 1.6080, 3.1847) -- cycle;
\fill[blue!50.1, opacity=0.7] (2.3740, 1.6080, 3.1847) -- (2.4200, 1.6080, 3.1793) -- (2.4200, 1.6620, 3.1785) -- (2.3740, 1.6620, 3.1839) -- cycle;
\fill[blue!56.0, opacity=0.7] (2.3740, 1.6620, 3.1839) -- (2.4200, 1.6620, 3.1785) -- (2.4200, 1.7160, 3.1774) -- (2.3740, 1.7160, 3.1827) -- cycle;
\fill[blue!61.6, opacity=0.7] (2.3740, 1.7160, 3.1827) -- (2.4200, 1.7160, 3.1774) -- (2.4200, 1.7700, 3.1759) -- (2.3740, 1.7700, 3.1813) -- cycle;
\fill[blue!63.4, opacity=0.7] (2.3740, 1.7700, 3.1813) -- (2.4200, 1.7700, 3.1759) -- (2.4200, 1.8240, 3.1741) -- (2.3740, 1.8240, 3.1795) -- cycle;
\fill[blue!58.5, opacity=0.7] (2.3740, 1.8240, 3.1795) -- (2.4200, 1.8240, 3.1741) -- (2.4200, 1.8780, 3.1720) -- (2.3740, 1.8780, 3.1774) -- cycle;
\fill[blue!48.5, opacity=0.7] (2.3740, 1.8780, 3.1774) -- (2.4200, 1.8780, 3.1720) -- (2.4200, 1.9320, 3.1696) -- (2.3740, 1.9320, 3.1750) -- cycle;
\fill[blue!39.1, opacity=0.7] (2.3740, 1.9320, 3.1750) -- (2.4200, 1.9320, 3.1696) -- (2.4200, 1.9860, 3.1669) -- (2.3740, 1.9860, 3.1723) -- cycle;
\fill[blue!34.0, opacity=0.7] (2.3740, 1.9860, 3.1723) -- (2.4200, 1.9860, 3.1669) -- (2.4200, 2.0400, 3.1639) -- (2.3740, 2.0400, 3.1693) -- cycle;
\fill[blue!34.0, opacity=0.7] (2.3740, 2.0400, 3.1693) -- (2.4200, 2.0400, 3.1639) -- (2.4200, 2.0940, 3.1606) -- (2.3740, 2.0940, 3.1660) -- cycle;
\fill[blue!39.8, opacity=0.7] (2.3740, 2.0940, 3.1660) -- (2.4200, 2.0940, 3.1606) -- (2.4200, 2.1480, 3.1571) -- (2.3740, 2.1480, 3.1624) -- cycle;
\fill[blue!51.3, opacity=0.7] (2.3740, 2.1480, 3.1624) -- (2.4200, 2.1480, 3.1571) -- (2.4200, 2.2020, 3.1533) -- (2.3740, 2.2020, 3.1586) -- cycle;
\fill[blue!61.9, opacity=0.7] (2.3740, 2.2020, 3.1586) -- (2.4200, 2.2020, 3.1533) -- (2.4200, 2.2560, 3.1492) -- (2.3740, 2.2560, 3.1545) -- cycle;
\fill[blue!62.5, opacity=0.7] (2.3740, 2.2560, 3.1545) -- (2.4200, 2.2560, 3.1492) -- (2.4200, 2.3100, 3.1449) -- (2.3740, 2.3100, 3.1502) -- cycle;
\fill[blue!57.7, opacity=0.7] (2.3740, 2.3100, 3.1502) -- (2.4200, 2.3100, 3.1449) -- (2.4200, 2.3640, 3.1403) -- (2.3740, 2.3640, 3.1457) -- cycle;
\fill[blue!57.6, opacity=0.7] (2.3740, 2.3640, 3.1457) -- (2.4200, 2.3640, 3.1403) -- (2.4200, 2.4180, 3.1355) -- (2.3740, 2.4180, 3.1409) -- cycle;
\fill[blue!62.8, opacity=0.7] (2.3740, 2.4180, 3.1409) -- (2.4200, 2.4180, 3.1355) -- (2.4200, 2.4720, 3.1305) -- (2.3740, 2.4720, 3.1359) -- cycle;
\fill[blue!58.9, opacity=0.7] (2.3740, 2.4720, 3.1359) -- (2.4200, 2.4720, 3.1305) -- (2.4200, 2.5260, 3.1254) -- (2.3740, 2.5260, 3.1307) -- cycle;
\fill[blue!38.7, opacity=0.7] (2.3740, 2.5260, 3.1307) -- (2.4200, 2.5260, 3.1254) -- (2.4200, 2.5800, 3.1200) -- (2.3740, 2.5800, 3.1254) -- cycle;
\fill[blue!23.6, opacity=0.7] (2.3740, 2.5800, 3.1254) -- (2.4200, 2.5800, 3.1200) -- (2.4200, 2.6340, 3.1145) -- (2.3740, 2.6340, 3.1198) -- cycle;
\fill[blue!19.0, opacity=0.7] (2.3740, 2.6340, 3.1198) -- (2.4200, 2.6340, 3.1145) -- (2.4200, 2.6880, 3.1088) -- (2.3740, 2.6880, 3.1142) -- cycle;
\fill[blue!19.4, opacity=0.7] (2.3740, 2.6880, 3.1142) -- (2.4200, 2.6880, 3.1088) -- (2.4200, 2.7420, 3.1030) -- (2.3740, 2.7420, 3.1084) -- cycle;
\fill[blue!25.0, opacity=0.7] (2.3740, 2.7420, 3.1084) -- (2.4200, 2.7420, 3.1030) -- (2.4200, 2.7960, 3.0971) -- (2.3740, 2.7960, 3.1024) -- cycle;
\fill[blue!39.8, opacity=0.7] (2.3740, 2.7960, 3.1024) -- (2.4200, 2.7960, 3.0971) -- (2.4200, 2.8500, 3.0911) -- (2.3740, 2.8500, 3.0964) -- cycle;
\fill[blue!55.6, opacity=0.7] (2.3740, 2.8500, 3.0964) -- (2.4200, 2.8500, 3.0911) -- (2.4200, 2.9040, 3.0849) -- (2.3740, 2.9040, 3.0903) -- cycle;
\fill[blue!60.6, opacity=0.7] (2.3740, 2.9040, 3.0903) -- (2.4200, 2.9040, 3.0849) -- (2.4200, 2.9580, 3.0788) -- (2.3740, 2.9580, 3.0841) -- cycle;
\fill[blue!56.2, opacity=0.7] (2.3740, 2.9580, 3.0841) -- (2.4200, 2.9580, 3.0788) -- (2.4200, 3.0120, 3.0725) -- (2.3740, 3.0120, 3.0779) -- cycle;
\fill[blue!38.1, opacity=0.7] (2.3740, 3.0120, 3.0779) -- (2.4200, 3.0120, 3.0725) -- (2.4200, 3.0660, 3.0663) -- (2.3740, 3.0660, 3.0716) -- cycle;
\fill[blue!20.7, opacity=0.7] (2.3740, 3.0660, 3.0716) -- (2.4200, 3.0660, 3.0663) -- (2.4200, 3.1200, 3.0600) -- (2.3740, 3.1200, 3.0654) -- cycle;
\fill[blue!21.4, opacity=0.7] (2.4200, -0.1200, 3.0600) -- (2.4660, -0.1200, 3.0545) -- (2.4660, -0.0660, 3.0608) -- (2.4200, -0.0660, 3.0663) -- cycle;
\fill[blue!16.9, opacity=0.7] (2.4200, -0.0660, 3.0663) -- (2.4660, -0.0660, 3.0608) -- (2.4660, -0.0120, 3.0670) -- (2.4200, -0.0120, 3.0725) -- cycle;
\fill[blue!15.9, opacity=0.7] (2.4200, -0.0120, 3.0725) -- (2.4660, -0.0120, 3.0670) -- (2.4660, 0.0420, 3.0733) -- (2.4200, 0.0420, 3.0788) -- cycle;
\fill[blue!16.2, opacity=0.7] (2.4200, 0.0420, 3.0788) -- (2.4660, 0.0420, 3.0733) -- (2.4660, 0.0960, 3.0794) -- (2.4200, 0.0960, 3.0849) -- cycle;
\fill[blue!18.4, opacity=0.7] (2.4200, 0.0960, 3.0849) -- (2.4660, 0.0960, 3.0794) -- (2.4660, 0.1500, 3.0855) -- (2.4200, 0.1500, 3.0911) -- cycle;
\fill[blue!27.2, opacity=0.7] (2.4200, 0.1500, 3.0911) -- (2.4660, 0.1500, 3.0855) -- (2.4660, 0.2040, 3.0916) -- (2.4200, 0.2040, 3.0971) -- cycle;
\fill[blue!45.7, opacity=0.7] (2.4200, 0.2040, 3.0971) -- (2.4660, 0.2040, 3.0916) -- (2.4660, 0.2580, 3.0975) -- (2.4200, 0.2580, 3.1030) -- cycle;
\fill[blue!60.5, opacity=0.7] (2.4200, 0.2580, 3.1030) -- (2.4660, 0.2580, 3.0975) -- (2.4660, 0.3120, 3.1033) -- (2.4200, 0.3120, 3.1088) -- cycle;
\fill[blue!63.6, opacity=0.7] (2.4200, 0.3120, 3.1088) -- (2.4660, 0.3120, 3.1033) -- (2.4660, 0.3660, 3.1090) -- (2.4200, 0.3660, 3.1145) -- cycle;
\fill[blue!63.4, opacity=0.7] (2.4200, 0.3660, 3.1145) -- (2.4660, 0.3660, 3.1090) -- (2.4660, 0.4200, 3.1145) -- (2.4200, 0.4200, 3.1200) -- cycle;
\fill[blue!63.4, opacity=0.7] (2.4200, 0.4200, 3.1200) -- (2.4660, 0.4200, 3.1145) -- (2.4660, 0.4740, 3.1198) -- (2.4200, 0.4740, 3.1254) -- cycle;
\fill[blue!59.0, opacity=0.7] (2.4200, 0.4740, 3.1254) -- (2.4660, 0.4740, 3.1198) -- (2.4660, 0.5280, 3.1250) -- (2.4200, 0.5280, 3.1305) -- cycle;
\fill[blue!48.1, opacity=0.7] (2.4200, 0.5280, 3.1305) -- (2.4660, 0.5280, 3.1250) -- (2.4660, 0.5820, 3.1300) -- (2.4200, 0.5820, 3.1355) -- cycle;
\fill[blue!36.5, opacity=0.7] (2.4200, 0.5820, 3.1355) -- (2.4660, 0.5820, 3.1300) -- (2.4660, 0.6360, 3.1348) -- (2.4200, 0.6360, 3.1403) -- cycle;
\fill[blue!29.1, opacity=0.7] (2.4200, 0.6360, 3.1403) -- (2.4660, 0.6360, 3.1348) -- (2.4660, 0.6900, 3.1393) -- (2.4200, 0.6900, 3.1449) -- cycle;
\fill[blue!26.1, opacity=0.7] (2.4200, 0.6900, 3.1449) -- (2.4660, 0.6900, 3.1393) -- (2.4660, 0.7440, 3.1437) -- (2.4200, 0.7440, 3.1492) -- cycle;
\fill[blue!26.2, opacity=0.7] (2.4200, 0.7440, 3.1492) -- (2.4660, 0.7440, 3.1437) -- (2.4660, 0.7980, 3.1477) -- (2.4200, 0.7980, 3.1533) -- cycle;
\fill[blue!29.1, opacity=0.7] (2.4200, 0.7980, 3.1533) -- (2.4660, 0.7980, 3.1477) -- (2.4660, 0.8520, 3.1516) -- (2.4200, 0.8520, 3.1571) -- cycle;
\fill[blue!34.9, opacity=0.7] (2.4200, 0.8520, 3.1571) -- (2.4660, 0.8520, 3.1516) -- (2.4660, 0.9060, 3.1551) -- (2.4200, 0.9060, 3.1606) -- cycle;
\fill[blue!43.2, opacity=0.7] (2.4200, 0.9060, 3.1606) -- (2.4660, 0.9060, 3.1551) -- (2.4660, 0.9600, 3.1584) -- (2.4200, 0.9600, 3.1639) -- cycle;
\fill[blue!52.2, opacity=0.7] (2.4200, 0.9600, 3.1639) -- (2.4660, 0.9600, 3.1584) -- (2.4660, 1.0140, 3.1614) -- (2.4200, 1.0140, 3.1669) -- cycle;
\fill[blue!59.2, opacity=0.7] (2.4200, 1.0140, 3.1669) -- (2.4660, 1.0140, 3.1614) -- (2.4660, 1.0680, 3.1641) -- (2.4200, 1.0680, 3.1696) -- cycle;
\fill[blue!62.9, opacity=0.7] (2.4200, 1.0680, 3.1696) -- (2.4660, 1.0680, 3.1641) -- (2.4660, 1.1220, 3.1665) -- (2.4200, 1.1220, 3.1720) -- cycle;
\fill[blue!63.5, opacity=0.7] (2.4200, 1.1220, 3.1720) -- (2.4660, 1.1220, 3.1665) -- (2.4660, 1.1760, 3.1686) -- (2.4200, 1.1760, 3.1741) -- cycle;
\fill[blue!62.4, opacity=0.7] (2.4200, 1.1760, 3.1741) -- (2.4660, 1.1760, 3.1686) -- (2.4660, 1.2300, 3.1704) -- (2.4200, 1.2300, 3.1759) -- cycle;
\fill[blue!60.8, opacity=0.7] (2.4200, 1.2300, 3.1759) -- (2.4660, 1.2300, 3.1704) -- (2.4660, 1.2840, 3.1719) -- (2.4200, 1.2840, 3.1774) -- cycle;
\fill[blue!59.5, opacity=0.7] (2.4200, 1.2840, 3.1774) -- (2.4660, 1.2840, 3.1719) -- (2.4660, 1.3380, 3.1730) -- (2.4200, 1.3380, 3.1785) -- cycle;
\fill[blue!58.9, opacity=0.7] (2.4200, 1.3380, 3.1785) -- (2.4660, 1.3380, 3.1730) -- (2.4660, 1.3920, 3.1738) -- (2.4200, 1.3920, 3.1793) -- cycle;
\fill[blue!59.2, opacity=0.7] (2.4200, 1.3920, 3.1793) -- (2.4660, 1.3920, 3.1738) -- (2.4660, 1.4460, 3.1743) -- (2.4200, 1.4460, 3.1798) -- cycle;
\fill[blue!60.3, opacity=0.7] (2.4200, 1.4460, 3.1798) -- (2.4660, 1.4460, 3.1743) -- (2.4660, 1.5000, 3.1745) -- (2.4200, 1.5000, 3.1800) -- cycle;
\fill[blue!61.9, opacity=0.7] (2.4200, 1.5000, 3.1800) -- (2.4660, 1.5000, 3.1745) -- (2.4660, 1.5540, 3.1743) -- (2.4200, 1.5540, 3.1798) -- cycle;
\fill[blue!63.3, opacity=0.7] (2.4200, 1.5540, 3.1798) -- (2.4660, 1.5540, 3.1743) -- (2.4660, 1.6080, 3.1738) -- (2.4200, 1.6080, 3.1793) -- cycle;
\fill[blue!63.3, opacity=0.7] (2.4200, 1.6080, 3.1793) -- (2.4660, 1.6080, 3.1738) -- (2.4660, 1.6620, 3.1730) -- (2.4200, 1.6620, 3.1785) -- cycle;
\fill[blue!60.5, opacity=0.7] (2.4200, 1.6620, 3.1785) -- (2.4660, 1.6620, 3.1730) -- (2.4660, 1.7160, 3.1719) -- (2.4200, 1.7160, 3.1774) -- cycle;
\fill[blue!54.4, opacity=0.7] (2.4200, 1.7160, 3.1774) -- (2.4660, 1.7160, 3.1719) -- (2.4660, 1.7700, 3.1704) -- (2.4200, 1.7700, 3.1759) -- cycle;
\fill[blue!46.4, opacity=0.7] (2.4200, 1.7700, 3.1759) -- (2.4660, 1.7700, 3.1704) -- (2.4660, 1.8240, 3.1686) -- (2.4200, 1.8240, 3.1741) -- cycle;
\fill[blue!38.9, opacity=0.7] (2.4200, 1.8240, 3.1741) -- (2.4660, 1.8240, 3.1686) -- (2.4660, 1.8780, 3.1665) -- (2.4200, 1.8780, 3.1720) -- cycle;
\fill[blue!34.0, opacity=0.7] (2.4200, 1.8780, 3.1720) -- (2.4660, 1.8780, 3.1665) -- (2.4660, 1.9320, 3.1641) -- (2.4200, 1.9320, 3.1696) -- cycle;
\fill[blue!32.6, opacity=0.7] (2.4200, 1.9320, 3.1696) -- (2.4660, 1.9320, 3.1641) -- (2.4660, 1.9860, 3.1614) -- (2.4200, 1.9860, 3.1669) -- cycle;
\fill[blue!35.3, opacity=0.7] (2.4200, 1.9860, 3.1669) -- (2.4660, 1.9860, 3.1614) -- (2.4660, 2.0400, 3.1584) -- (2.4200, 2.0400, 3.1639) -- cycle;
\fill[blue!43.0, opacity=0.7] (2.4200, 2.0400, 3.1639) -- (2.4660, 2.0400, 3.1584) -- (2.4660, 2.0940, 3.1551) -- (2.4200, 2.0940, 3.1606) -- cycle;
\fill[blue!54.4, opacity=0.7] (2.4200, 2.0940, 3.1606) -- (2.4660, 2.0940, 3.1551) -- (2.4660, 2.1480, 3.1516) -- (2.4200, 2.1480, 3.1571) -- cycle;
\fill[blue!62.8, opacity=0.7] (2.4200, 2.1480, 3.1571) -- (2.4660, 2.1480, 3.1516) -- (2.4660, 2.2020, 3.1477) -- (2.4200, 2.2020, 3.1533) -- cycle;
\fill[blue!62.1, opacity=0.7] (2.4200, 2.2020, 3.1533) -- (2.4660, 2.2020, 3.1477) -- (2.4660, 2.2560, 3.1437) -- (2.4200, 2.2560, 3.1492) -- cycle;
\fill[blue!57.8, opacity=0.7] (2.4200, 2.2560, 3.1492) -- (2.4660, 2.2560, 3.1437) -- (2.4660, 2.3100, 3.1393) -- (2.4200, 2.3100, 3.1449) -- cycle;
\fill[blue!58.0, opacity=0.7] (2.4200, 2.3100, 3.1449) -- (2.4660, 2.3100, 3.1393) -- (2.4660, 2.3640, 3.1348) -- (2.4200, 2.3640, 3.1403) -- cycle;
\fill[blue!62.8, opacity=0.7] (2.4200, 2.3640, 3.1403) -- (2.4660, 2.3640, 3.1348) -- (2.4660, 2.4180, 3.1300) -- (2.4200, 2.4180, 3.1355) -- cycle;
\fill[blue!59.8, opacity=0.7] (2.4200, 2.4180, 3.1355) -- (2.4660, 2.4180, 3.1300) -- (2.4660, 2.4720, 3.1250) -- (2.4200, 2.4720, 3.1305) -- cycle;
\fill[blue!41.0, opacity=0.7] (2.4200, 2.4720, 3.1305) -- (2.4660, 2.4720, 3.1250) -- (2.4660, 2.5260, 3.1198) -- (2.4200, 2.5260, 3.1254) -- cycle;
\fill[blue!25.0, opacity=0.7] (2.4200, 2.5260, 3.1254) -- (2.4660, 2.5260, 3.1198) -- (2.4660, 2.5800, 3.1145) -- (2.4200, 2.5800, 3.1200) -- cycle;
\fill[blue!19.3, opacity=0.7] (2.4200, 2.5800, 3.1200) -- (2.4660, 2.5800, 3.1145) -- (2.4660, 2.6340, 3.1090) -- (2.4200, 2.6340, 3.1145) -- cycle;
\fill[blue!18.8, opacity=0.7] (2.4200, 2.6340, 3.1145) -- (2.4660, 2.6340, 3.1090) -- (2.4660, 2.6880, 3.1033) -- (2.4200, 2.6880, 3.1088) -- cycle;
\fill[blue!22.5, opacity=0.7] (2.4200, 2.6880, 3.1088) -- (2.4660, 2.6880, 3.1033) -- (2.4660, 2.7420, 3.0975) -- (2.4200, 2.7420, 3.1030) -- cycle;
\fill[blue!34.2, opacity=0.7] (2.4200, 2.7420, 3.1030) -- (2.4660, 2.7420, 3.0975) -- (2.4660, 2.7960, 3.0916) -- (2.4200, 2.7960, 3.0971) -- cycle;
\fill[blue!51.2, opacity=0.7] (2.4200, 2.7960, 3.0971) -- (2.4660, 2.7960, 3.0916) -- (2.4660, 2.8500, 3.0855) -- (2.4200, 2.8500, 3.0911) -- cycle;
\fill[blue!59.8, opacity=0.7] (2.4200, 2.8500, 3.0911) -- (2.4660, 2.8500, 3.0855) -- (2.4660, 2.9040, 3.0794) -- (2.4200, 2.9040, 3.0849) -- cycle;
\fill[blue!58.9, opacity=0.7] (2.4200, 2.9040, 3.0849) -- (2.4660, 2.9040, 3.0794) -- (2.4660, 2.9580, 3.0733) -- (2.4200, 2.9580, 3.0788) -- cycle;
\fill[blue!46.3, opacity=0.7] (2.4200, 2.9580, 3.0788) -- (2.4660, 2.9580, 3.0733) -- (2.4660, 3.0120, 3.0670) -- (2.4200, 3.0120, 3.0725) -- cycle;
\fill[blue!26.0, opacity=0.7] (2.4200, 3.0120, 3.0725) -- (2.4660, 3.0120, 3.0670) -- (2.4660, 3.0660, 3.0608) -- (2.4200, 3.0660, 3.0663) -- cycle;
\fill[blue!16.9, opacity=0.7] (2.4200, 3.0660, 3.0663) -- (2.4660, 3.0660, 3.0608) -- (2.4660, 3.1200, 3.0545) -- (2.4200, 3.1200, 3.0600) -- cycle;
\fill[blue!30.1, opacity=0.7] (2.4660, -0.1200, 3.0545) -- (2.5120, -0.1200, 3.0488) -- (2.5120, -0.0660, 3.0551) -- (2.4660, -0.0660, 3.0608) -- cycle;
\fill[blue!20.4, opacity=0.7] (2.4660, -0.0660, 3.0608) -- (2.5120, -0.0660, 3.0551) -- (2.5120, -0.0120, 3.0614) -- (2.4660, -0.0120, 3.0670) -- cycle;
\fill[blue!16.8, opacity=0.7] (2.4660, -0.0120, 3.0670) -- (2.5120, -0.0120, 3.0614) -- (2.5120, 0.0420, 3.0676) -- (2.4660, 0.0420, 3.0733) -- cycle;
\fill[blue!15.9, opacity=0.7] (2.4660, 0.0420, 3.0733) -- (2.5120, 0.0420, 3.0676) -- (2.5120, 0.0960, 3.0738) -- (2.4660, 0.0960, 3.0794) -- cycle;
\fill[blue!16.3, opacity=0.7] (2.4660, 0.0960, 3.0794) -- (2.5120, 0.0960, 3.0738) -- (2.5120, 0.1500, 3.0799) -- (2.4660, 0.1500, 3.0855) -- cycle;
\fill[blue!18.5, opacity=0.7] (2.4660, 0.1500, 3.0855) -- (2.5120, 0.1500, 3.0799) -- (2.5120, 0.2040, 3.0859) -- (2.4660, 0.2040, 3.0916) -- cycle;
\fill[blue!27.0, opacity=0.7] (2.4660, 0.2040, 3.0916) -- (2.5120, 0.2040, 3.0859) -- (2.5120, 0.2580, 3.0918) -- (2.4660, 0.2580, 3.0975) -- cycle;
\fill[blue!44.5, opacity=0.7] (2.4660, 0.2580, 3.0975) -- (2.5120, 0.2580, 3.0918) -- (2.5120, 0.3120, 3.0976) -- (2.4660, 0.3120, 3.1033) -- cycle;
\fill[blue!59.5, opacity=0.7] (2.4660, 0.3120, 3.1033) -- (2.5120, 0.3120, 3.0976) -- (2.5120, 0.3660, 3.1033) -- (2.4660, 0.3660, 3.1090) -- cycle;
\fill[blue!63.5, opacity=0.7] (2.4660, 0.3660, 3.1090) -- (2.5120, 0.3660, 3.1033) -- (2.5120, 0.4200, 3.1088) -- (2.4660, 0.4200, 3.1145) -- cycle;
\fill[blue!63.3, opacity=0.7] (2.4660, 0.4200, 3.1145) -- (2.5120, 0.4200, 3.1088) -- (2.5120, 0.4740, 3.1142) -- (2.4660, 0.4740, 3.1198) -- cycle;
\fill[blue!63.6, opacity=0.7] (2.4660, 0.4740, 3.1198) -- (2.5120, 0.4740, 3.1142) -- (2.5120, 0.5280, 3.1193) -- (2.4660, 0.5280, 3.1250) -- cycle;
\fill[blue!61.8, opacity=0.7] (2.4660, 0.5280, 3.1250) -- (2.5120, 0.5280, 3.1193) -- (2.5120, 0.5820, 3.1243) -- (2.4660, 0.5820, 3.1300) -- cycle;
\fill[blue!54.6, opacity=0.7] (2.4660, 0.5820, 3.1300) -- (2.5120, 0.5820, 3.1243) -- (2.5120, 0.6360, 3.1291) -- (2.4660, 0.6360, 3.1348) -- cycle;
\fill[blue!43.8, opacity=0.7] (2.4660, 0.6360, 3.1348) -- (2.5120, 0.6360, 3.1291) -- (2.5120, 0.6900, 3.1337) -- (2.4660, 0.6900, 3.1393) -- cycle;
\fill[blue!34.5, opacity=0.7] (2.4660, 0.6900, 3.1393) -- (2.5120, 0.6900, 3.1337) -- (2.5120, 0.7440, 3.1380) -- (2.4660, 0.7440, 3.1437) -- cycle;
\fill[blue!28.9, opacity=0.7] (2.4660, 0.7440, 3.1437) -- (2.5120, 0.7440, 3.1380) -- (2.5120, 0.7980, 3.1421) -- (2.4660, 0.7980, 3.1477) -- cycle;
\fill[blue!26.6, opacity=0.7] (2.4660, 0.7980, 3.1477) -- (2.5120, 0.7980, 3.1421) -- (2.5120, 0.8520, 3.1459) -- (2.4660, 0.8520, 3.1516) -- cycle;
\fill[blue!26.6, opacity=0.7] (2.4660, 0.8520, 3.1516) -- (2.5120, 0.8520, 3.1459) -- (2.5120, 0.9060, 3.1494) -- (2.4660, 0.9060, 3.1551) -- cycle;
\fill[blue!28.4, opacity=0.7] (2.4660, 0.9060, 3.1551) -- (2.5120, 0.9060, 3.1494) -- (2.5120, 0.9600, 3.1527) -- (2.4660, 0.9600, 3.1584) -- cycle;
\fill[blue!31.8, opacity=0.7] (2.4660, 0.9600, 3.1584) -- (2.5120, 0.9600, 3.1527) -- (2.5120, 1.0140, 3.1557) -- (2.4660, 1.0140, 3.1614) -- cycle;
\fill[blue!36.6, opacity=0.7] (2.4660, 1.0140, 3.1614) -- (2.5120, 1.0140, 3.1557) -- (2.5120, 1.0680, 3.1584) -- (2.4660, 1.0680, 3.1641) -- cycle;
\fill[blue!42.0, opacity=0.7] (2.4660, 1.0680, 3.1641) -- (2.5120, 1.0680, 3.1584) -- (2.5120, 1.1220, 3.1608) -- (2.4660, 1.1220, 3.1665) -- cycle;
\fill[blue!47.2, opacity=0.7] (2.4660, 1.1220, 3.1665) -- (2.5120, 1.1220, 3.1608) -- (2.5120, 1.1760, 3.1629) -- (2.4660, 1.1760, 3.1686) -- cycle;
\fill[blue!51.5, opacity=0.7] (2.4660, 1.1760, 3.1686) -- (2.5120, 1.1760, 3.1629) -- (2.5120, 1.2300, 3.1647) -- (2.4660, 1.2300, 3.1704) -- cycle;
\fill[blue!54.6, opacity=0.7] (2.4660, 1.2300, 3.1704) -- (2.5120, 1.2300, 3.1647) -- (2.5120, 1.2840, 3.1662) -- (2.4660, 1.2840, 3.1719) -- cycle;
\fill[blue!56.4, opacity=0.7] (2.4660, 1.2840, 3.1719) -- (2.5120, 1.2840, 3.1662) -- (2.5120, 1.3380, 3.1673) -- (2.4660, 1.3380, 3.1730) -- cycle;
\fill[blue!57.1, opacity=0.7] (2.4660, 1.3380, 3.1730) -- (2.5120, 1.3380, 3.1673) -- (2.5120, 1.3920, 3.1682) -- (2.4660, 1.3920, 3.1738) -- cycle;
\fill[blue!56.7, opacity=0.7] (2.4660, 1.3920, 3.1738) -- (2.5120, 1.3920, 3.1682) -- (2.5120, 1.4460, 3.1686) -- (2.4660, 1.4460, 3.1743) -- cycle;
\fill[blue!55.4, opacity=0.7] (2.4660, 1.4460, 3.1743) -- (2.5120, 1.4460, 3.1686) -- (2.5120, 1.5000, 3.1688) -- (2.4660, 1.5000, 3.1745) -- cycle;
\fill[blue!52.8, opacity=0.7] (2.4660, 1.5000, 3.1745) -- (2.5120, 1.5000, 3.1688) -- (2.5120, 1.5540, 3.1686) -- (2.4660, 1.5540, 3.1743) -- cycle;
\fill[blue!49.0, opacity=0.7] (2.4660, 1.5540, 3.1743) -- (2.5120, 1.5540, 3.1686) -- (2.5120, 1.6080, 3.1682) -- (2.4660, 1.6080, 3.1738) -- cycle;
\fill[blue!44.4, opacity=0.7] (2.4660, 1.6080, 3.1738) -- (2.5120, 1.6080, 3.1682) -- (2.5120, 1.6620, 3.1673) -- (2.4660, 1.6620, 3.1730) -- cycle;
\fill[blue!39.5, opacity=0.7] (2.4660, 1.6620, 3.1730) -- (2.5120, 1.6620, 3.1673) -- (2.5120, 1.7160, 3.1662) -- (2.4660, 1.7160, 3.1719) -- cycle;
\fill[blue!35.3, opacity=0.7] (2.4660, 1.7160, 3.1719) -- (2.5120, 1.7160, 3.1662) -- (2.5120, 1.7700, 3.1647) -- (2.4660, 1.7700, 3.1704) -- cycle;
\fill[blue!32.6, opacity=0.7] (2.4660, 1.7700, 3.1704) -- (2.5120, 1.7700, 3.1647) -- (2.5120, 1.8240, 3.1629) -- (2.4660, 1.8240, 3.1686) -- cycle;
\fill[blue!31.9, opacity=0.7] (2.4660, 1.8240, 3.1686) -- (2.5120, 1.8240, 3.1629) -- (2.5120, 1.8780, 3.1608) -- (2.4660, 1.8780, 3.1665) -- cycle;
\fill[blue!33.8, opacity=0.7] (2.4660, 1.8780, 3.1665) -- (2.5120, 1.8780, 3.1608) -- (2.5120, 1.9320, 3.1584) -- (2.4660, 1.9320, 3.1641) -- cycle;
\fill[blue!39.3, opacity=0.7] (2.4660, 1.9320, 3.1641) -- (2.5120, 1.9320, 3.1584) -- (2.5120, 1.9860, 3.1557) -- (2.4660, 1.9860, 3.1614) -- cycle;
\fill[blue!48.6, opacity=0.7] (2.4660, 1.9860, 3.1614) -- (2.5120, 1.9860, 3.1557) -- (2.5120, 2.0400, 3.1527) -- (2.4660, 2.0400, 3.1584) -- cycle;
\fill[blue!58.7, opacity=0.7] (2.4660, 2.0400, 3.1584) -- (2.5120, 2.0400, 3.1527) -- (2.5120, 2.0940, 3.1494) -- (2.4660, 2.0940, 3.1551) -- cycle;
\fill[blue!63.5, opacity=0.7] (2.4660, 2.0940, 3.1551) -- (2.5120, 2.0940, 3.1494) -- (2.5120, 2.1480, 3.1459) -- (2.4660, 2.1480, 3.1516) -- cycle;
\fill[blue!61.2, opacity=0.7] (2.4660, 2.1480, 3.1516) -- (2.5120, 2.1480, 3.1459) -- (2.5120, 2.2020, 3.1421) -- (2.4660, 2.2020, 3.1477) -- cycle;
\fill[blue!57.8, opacity=0.7] (2.4660, 2.2020, 3.1477) -- (2.5120, 2.2020, 3.1421) -- (2.5120, 2.2560, 3.1380) -- (2.4660, 2.2560, 3.1437) -- cycle;
\fill[blue!58.9, opacity=0.7] (2.4660, 2.2560, 3.1437) -- (2.5120, 2.2560, 3.1380) -- (2.5120, 2.3100, 3.1337) -- (2.4660, 2.3100, 3.1393) -- cycle;
\fill[blue!63.2, opacity=0.7] (2.4660, 2.3100, 3.1393) -- (2.5120, 2.3100, 3.1337) -- (2.5120, 2.3640, 3.1291) -- (2.4660, 2.3640, 3.1348) -- cycle;
\fill[blue!59.3, opacity=0.7] (2.4660, 2.3640, 3.1348) -- (2.5120, 2.3640, 3.1291) -- (2.5120, 2.4180, 3.1243) -- (2.4660, 2.4180, 3.1300) -- cycle;
\fill[blue!41.4, opacity=0.7] (2.4660, 2.4180, 3.1300) -- (2.5120, 2.4180, 3.1243) -- (2.5120, 2.4720, 3.1193) -- (2.4660, 2.4720, 3.1250) -- cycle;
\fill[blue!25.6, opacity=0.7] (2.4660, 2.4720, 3.1250) -- (2.5120, 2.4720, 3.1193) -- (2.5120, 2.5260, 3.1142) -- (2.4660, 2.5260, 3.1198) -- cycle;
\fill[blue!19.5, opacity=0.7] (2.4660, 2.5260, 3.1198) -- (2.5120, 2.5260, 3.1142) -- (2.5120, 2.5800, 3.1088) -- (2.4660, 2.5800, 3.1145) -- cycle;
\fill[blue!18.5, opacity=0.7] (2.4660, 2.5800, 3.1145) -- (2.5120, 2.5800, 3.1088) -- (2.5120, 2.6340, 3.1033) -- (2.4660, 2.6340, 3.1090) -- cycle;
\fill[blue!21.1, opacity=0.7] (2.4660, 2.6340, 3.1090) -- (2.5120, 2.6340, 3.1033) -- (2.5120, 2.6880, 3.0976) -- (2.4660, 2.6880, 3.1033) -- cycle;
\fill[blue!30.4, opacity=0.7] (2.4660, 2.6880, 3.1033) -- (2.5120, 2.6880, 3.0976) -- (2.5120, 2.7420, 3.0918) -- (2.4660, 2.7420, 3.0975) -- cycle;
\fill[blue!46.8, opacity=0.7] (2.4660, 2.7420, 3.0975) -- (2.5120, 2.7420, 3.0918) -- (2.5120, 2.7960, 3.0859) -- (2.4660, 2.7960, 3.0916) -- cycle;
\fill[blue!58.2, opacity=0.7] (2.4660, 2.7960, 3.0916) -- (2.5120, 2.7960, 3.0859) -- (2.5120, 2.8500, 3.0799) -- (2.4660, 2.8500, 3.0855) -- cycle;
\fill[blue!59.9, opacity=0.7] (2.4660, 2.8500, 3.0855) -- (2.5120, 2.8500, 3.0799) -- (2.5120, 2.9040, 3.0738) -- (2.4660, 2.9040, 3.0794) -- cycle;
\fill[blue!51.9, opacity=0.7] (2.4660, 2.9040, 3.0794) -- (2.5120, 2.9040, 3.0738) -- (2.5120, 2.9580, 3.0676) -- (2.4660, 2.9580, 3.0733) -- cycle;
\fill[blue!32.4, opacity=0.7] (2.4660, 2.9580, 3.0733) -- (2.5120, 2.9580, 3.0676) -- (2.5120, 3.0120, 3.0614) -- (2.4660, 3.0120, 3.0670) -- cycle;
\fill[blue!18.8, opacity=0.7] (2.4660, 3.0120, 3.0670) -- (2.5120, 3.0120, 3.0614) -- (2.5120, 3.0660, 3.0551) -- (2.4660, 3.0660, 3.0608) -- cycle;
\fill[blue!15.7, opacity=0.7] (2.4660, 3.0660, 3.0608) -- (2.5120, 3.0660, 3.0551) -- (2.5120, 3.1200, 3.0488) -- (2.4660, 3.1200, 3.0545) -- cycle;
\fill[blue!40.8, opacity=0.7] (2.5120, -0.1200, 3.0488) -- (2.5580, -0.1200, 3.0430) -- (2.5580, -0.0660, 3.0493) -- (2.5120, -0.0660, 3.0551) -- cycle;
\fill[blue!29.0, opacity=0.7] (2.5120, -0.0660, 3.0551) -- (2.5580, -0.0660, 3.0493) -- (2.5580, -0.0120, 3.0555) -- (2.5120, -0.0120, 3.0614) -- cycle;
\fill[blue!20.1, opacity=0.7] (2.5120, -0.0120, 3.0614) -- (2.5580, -0.0120, 3.0555) -- (2.5580, 0.0420, 3.0618) -- (2.5120, 0.0420, 3.0676) -- cycle;
\fill[blue!16.7, opacity=0.7] (2.5120, 0.0420, 3.0676) -- (2.5580, 0.0420, 3.0618) -- (2.5580, 0.0960, 3.0680) -- (2.5120, 0.0960, 3.0738) -- cycle;
\fill[blue!16.0, opacity=0.7] (2.5120, 0.0960, 3.0738) -- (2.5580, 0.0960, 3.0680) -- (2.5580, 0.1500, 3.0741) -- (2.5120, 0.1500, 3.0799) -- cycle;
\fill[blue!16.3, opacity=0.7] (2.5120, 0.1500, 3.0799) -- (2.5580, 0.1500, 3.0741) -- (2.5580, 0.2040, 3.0801) -- (2.5120, 0.2040, 3.0859) -- cycle;
\fill[blue!18.3, opacity=0.7] (2.5120, 0.2040, 3.0859) -- (2.5580, 0.2040, 3.0801) -- (2.5580, 0.2580, 3.0860) -- (2.5120, 0.2580, 3.0918) -- cycle;
\fill[blue!25.6, opacity=0.7] (2.5120, 0.2580, 3.0918) -- (2.5580, 0.2580, 3.0860) -- (2.5580, 0.3120, 3.0918) -- (2.5120, 0.3120, 3.0976) -- cycle;
\fill[blue!41.4, opacity=0.7] (2.5120, 0.3120, 3.0976) -- (2.5580, 0.3120, 3.0918) -- (2.5580, 0.3660, 3.0975) -- (2.5120, 0.3660, 3.1033) -- cycle;
\fill[blue!57.3, opacity=0.7] (2.5120, 0.3660, 3.1033) -- (2.5580, 0.3660, 3.0975) -- (2.5580, 0.4200, 3.1030) -- (2.5120, 0.4200, 3.1088) -- cycle;
\fill[blue!63.3, opacity=0.7] (2.5120, 0.4200, 3.1088) -- (2.5580, 0.4200, 3.1030) -- (2.5580, 0.4740, 3.1084) -- (2.5120, 0.4740, 3.1142) -- cycle;
\fill[blue!63.3, opacity=0.7] (2.5120, 0.4740, 3.1142) -- (2.5580, 0.4740, 3.1084) -- (2.5580, 0.5280, 3.1135) -- (2.5120, 0.5280, 3.1193) -- cycle;
\fill[blue!63.3, opacity=0.7] (2.5120, 0.5280, 3.1193) -- (2.5580, 0.5280, 3.1135) -- (2.5580, 0.5820, 3.1185) -- (2.5120, 0.5820, 3.1243) -- cycle;
\fill[blue!63.4, opacity=0.7] (2.5120, 0.5820, 3.1243) -- (2.5580, 0.5820, 3.1185) -- (2.5580, 0.6360, 3.1233) -- (2.5120, 0.6360, 3.1291) -- cycle;
\fill[blue!60.4, opacity=0.7] (2.5120, 0.6360, 3.1291) -- (2.5580, 0.6360, 3.1233) -- (2.5580, 0.6900, 3.1279) -- (2.5120, 0.6900, 3.1337) -- cycle;
\fill[blue!53.0, opacity=0.7] (2.5120, 0.6900, 3.1337) -- (2.5580, 0.6900, 3.1279) -- (2.5580, 0.7440, 3.1322) -- (2.5120, 0.7440, 3.1380) -- cycle;
\fill[blue!43.6, opacity=0.7] (2.5120, 0.7440, 3.1380) -- (2.5580, 0.7440, 3.1322) -- (2.5580, 0.7980, 3.1363) -- (2.5120, 0.7980, 3.1421) -- cycle;
\fill[blue!35.8, opacity=0.7] (2.5120, 0.7980, 3.1421) -- (2.5580, 0.7980, 3.1363) -- (2.5580, 0.8520, 3.1401) -- (2.5120, 0.8520, 3.1459) -- cycle;
\fill[blue!30.7, opacity=0.7] (2.5120, 0.8520, 3.1459) -- (2.5580, 0.8520, 3.1401) -- (2.5580, 0.9060, 3.1436) -- (2.5120, 0.9060, 3.1494) -- cycle;
\fill[blue!28.1, opacity=0.7] (2.5120, 0.9060, 3.1494) -- (2.5580, 0.9060, 3.1436) -- (2.5580, 0.9600, 3.1469) -- (2.5120, 0.9600, 3.1527) -- cycle;
\fill[blue!27.1, opacity=0.7] (2.5120, 0.9600, 3.1527) -- (2.5580, 0.9600, 3.1469) -- (2.5580, 1.0140, 3.1499) -- (2.5120, 1.0140, 3.1557) -- cycle;
\fill[blue!27.3, opacity=0.7] (2.5120, 1.0140, 3.1557) -- (2.5580, 1.0140, 3.1499) -- (2.5580, 1.0680, 3.1526) -- (2.5120, 1.0680, 3.1584) -- cycle;
\fill[blue!28.4, opacity=0.7] (2.5120, 1.0680, 3.1584) -- (2.5580, 1.0680, 3.1526) -- (2.5580, 1.1220, 3.1550) -- (2.5120, 1.1220, 3.1608) -- cycle;
\fill[blue!29.9, opacity=0.7] (2.5120, 1.1220, 3.1608) -- (2.5580, 1.1220, 3.1550) -- (2.5580, 1.1760, 3.1571) -- (2.5120, 1.1760, 3.1629) -- cycle;
\fill[blue!31.6, opacity=0.7] (2.5120, 1.1760, 3.1629) -- (2.5580, 1.1760, 3.1571) -- (2.5580, 1.2300, 3.1589) -- (2.5120, 1.2300, 3.1647) -- cycle;
\fill[blue!33.2, opacity=0.7] (2.5120, 1.2300, 3.1647) -- (2.5580, 1.2300, 3.1589) -- (2.5580, 1.2840, 3.1604) -- (2.5120, 1.2840, 3.1662) -- cycle;
\fill[blue!34.4, opacity=0.7] (2.5120, 1.2840, 3.1662) -- (2.5580, 1.2840, 3.1604) -- (2.5580, 1.3380, 3.1615) -- (2.5120, 1.3380, 3.1673) -- cycle;
\fill[blue!35.0, opacity=0.7] (2.5120, 1.3380, 3.1673) -- (2.5580, 1.3380, 3.1615) -- (2.5580, 1.3920, 3.1623) -- (2.5120, 1.3920, 3.1682) -- cycle;
\fill[blue!34.9, opacity=0.7] (2.5120, 1.3920, 3.1682) -- (2.5580, 1.3920, 3.1623) -- (2.5580, 1.4460, 3.1628) -- (2.5120, 1.4460, 3.1686) -- cycle;
\fill[blue!34.3, opacity=0.7] (2.5120, 1.4460, 3.1686) -- (2.5580, 1.4460, 3.1628) -- (2.5580, 1.5000, 3.1630) -- (2.5120, 1.5000, 3.1688) -- cycle;
\fill[blue!33.2, opacity=0.7] (2.5120, 1.5000, 3.1688) -- (2.5580, 1.5000, 3.1630) -- (2.5580, 1.5540, 3.1628) -- (2.5120, 1.5540, 3.1686) -- cycle;
\fill[blue!32.0, opacity=0.7] (2.5120, 1.5540, 3.1686) -- (2.5580, 1.5540, 3.1628) -- (2.5580, 1.6080, 3.1623) -- (2.5120, 1.6080, 3.1682) -- cycle;
\fill[blue!31.1, opacity=0.7] (2.5120, 1.6080, 3.1682) -- (2.5580, 1.6080, 3.1623) -- (2.5580, 1.6620, 3.1615) -- (2.5120, 1.6620, 3.1673) -- cycle;
\fill[blue!30.8, opacity=0.7] (2.5120, 1.6620, 3.1673) -- (2.5580, 1.6620, 3.1615) -- (2.5580, 1.7160, 3.1604) -- (2.5120, 1.7160, 3.1662) -- cycle;
\fill[blue!31.6, opacity=0.7] (2.5120, 1.7160, 3.1662) -- (2.5580, 1.7160, 3.1604) -- (2.5580, 1.7700, 3.1589) -- (2.5120, 1.7700, 3.1647) -- cycle;
\fill[blue!34.2, opacity=0.7] (2.5120, 1.7700, 3.1647) -- (2.5580, 1.7700, 3.1589) -- (2.5580, 1.8240, 3.1571) -- (2.5120, 1.8240, 3.1629) -- cycle;
\fill[blue!39.2, opacity=0.7] (2.5120, 1.8240, 3.1629) -- (2.5580, 1.8240, 3.1571) -- (2.5580, 1.8780, 3.1550) -- (2.5120, 1.8780, 3.1608) -- cycle;
\fill[blue!46.9, opacity=0.7] (2.5120, 1.8780, 3.1608) -- (2.5580, 1.8780, 3.1550) -- (2.5580, 1.9320, 3.1526) -- (2.5120, 1.9320, 3.1584) -- cycle;
\fill[blue!56.0, opacity=0.7] (2.5120, 1.9320, 3.1584) -- (2.5580, 1.9320, 3.1526) -- (2.5580, 1.9860, 3.1499) -- (2.5120, 1.9860, 3.1557) -- cycle;
\fill[blue!62.5, opacity=0.7] (2.5120, 1.9860, 3.1557) -- (2.5580, 1.9860, 3.1499) -- (2.5580, 2.0400, 3.1469) -- (2.5120, 2.0400, 3.1527) -- cycle;
\fill[blue!63.1, opacity=0.7] (2.5120, 2.0400, 3.1527) -- (2.5580, 2.0400, 3.1469) -- (2.5580, 2.0940, 3.1436) -- (2.5120, 2.0940, 3.1494) -- cycle;
\fill[blue!59.8, opacity=0.7] (2.5120, 2.0940, 3.1494) -- (2.5580, 2.0940, 3.1436) -- (2.5580, 2.1480, 3.1401) -- (2.5120, 2.1480, 3.1459) -- cycle;
\fill[blue!57.9, opacity=0.7] (2.5120, 2.1480, 3.1459) -- (2.5580, 2.1480, 3.1401) -- (2.5580, 2.2020, 3.1363) -- (2.5120, 2.2020, 3.1421) -- cycle;
\fill[blue!60.3, opacity=0.7] (2.5120, 2.2020, 3.1421) -- (2.5580, 2.2020, 3.1363) -- (2.5580, 2.2560, 3.1322) -- (2.5120, 2.2560, 3.1380) -- cycle;
\fill[blue!63.6, opacity=0.7] (2.5120, 2.2560, 3.1380) -- (2.5580, 2.2560, 3.1322) -- (2.5580, 2.3100, 3.1279) -- (2.5120, 2.3100, 3.1337) -- cycle;
\fill[blue!57.4, opacity=0.7] (2.5120, 2.3100, 3.1337) -- (2.5580, 2.3100, 3.1279) -- (2.5580, 2.3640, 3.1233) -- (2.5120, 2.3640, 3.1291) -- cycle;
\fill[blue!39.6, opacity=0.7] (2.5120, 2.3640, 3.1291) -- (2.5580, 2.3640, 3.1233) -- (2.5580, 2.4180, 3.1185) -- (2.5120, 2.4180, 3.1243) -- cycle;
\fill[blue!25.1, opacity=0.7] (2.5120, 2.4180, 3.1243) -- (2.5580, 2.4180, 3.1185) -- (2.5580, 2.4720, 3.1135) -- (2.5120, 2.4720, 3.1193) -- cycle;
\fill[blue!19.4, opacity=0.7] (2.5120, 2.4720, 3.1193) -- (2.5580, 2.4720, 3.1135) -- (2.5580, 2.5260, 3.1084) -- (2.5120, 2.5260, 3.1142) -- cycle;
\fill[blue!18.4, opacity=0.7] (2.5120, 2.5260, 3.1142) -- (2.5580, 2.5260, 3.1084) -- (2.5580, 2.5800, 3.1030) -- (2.5120, 2.5800, 3.1088) -- cycle;
\fill[blue!20.4, opacity=0.7] (2.5120, 2.5800, 3.1088) -- (2.5580, 2.5800, 3.1030) -- (2.5580, 2.6340, 3.0975) -- (2.5120, 2.6340, 3.1033) -- cycle;
\fill[blue!28.0, opacity=0.7] (2.5120, 2.6340, 3.1033) -- (2.5580, 2.6340, 3.0975) -- (2.5580, 2.6880, 3.0918) -- (2.5120, 2.6880, 3.0976) -- cycle;
\fill[blue!43.2, opacity=0.7] (2.5120, 2.6880, 3.0976) -- (2.5580, 2.6880, 3.0918) -- (2.5580, 2.7420, 3.0860) -- (2.5120, 2.7420, 3.0918) -- cycle;
\fill[blue!56.4, opacity=0.7] (2.5120, 2.7420, 3.0918) -- (2.5580, 2.7420, 3.0860) -- (2.5580, 2.7960, 3.0801) -- (2.5120, 2.7960, 3.0859) -- cycle;
\fill[blue!60.0, opacity=0.7] (2.5120, 2.7960, 3.0859) -- (2.5580, 2.7960, 3.0801) -- (2.5580, 2.8500, 3.0741) -- (2.5120, 2.8500, 3.0799) -- cycle;
\fill[blue!55.2, opacity=0.7] (2.5120, 2.8500, 3.0799) -- (2.5580, 2.8500, 3.0741) -- (2.5580, 2.9040, 3.0680) -- (2.5120, 2.9040, 3.0738) -- cycle;
\fill[blue!38.4, opacity=0.7] (2.5120, 2.9040, 3.0738) -- (2.5580, 2.9040, 3.0680) -- (2.5580, 2.9580, 3.0618) -- (2.5120, 2.9580, 3.0676) -- cycle;
\fill[blue!21.6, opacity=0.7] (2.5120, 2.9580, 3.0676) -- (2.5580, 2.9580, 3.0618) -- (2.5580, 3.0120, 3.0555) -- (2.5120, 3.0120, 3.0614) -- cycle;
\fill[blue!16.1, opacity=0.7] (2.5120, 3.0120, 3.0614) -- (2.5580, 3.0120, 3.0555) -- (2.5580, 3.0660, 3.0493) -- (2.5120, 3.0660, 3.0551) -- cycle;
\fill[blue!15.3, opacity=0.7] (2.5120, 3.0660, 3.0551) -- (2.5580, 3.0660, 3.0493) -- (2.5580, 3.1200, 3.0430) -- (2.5120, 3.1200, 3.0488) -- cycle;
\fill[blue!46.3, opacity=0.7] (2.5580, -0.1200, 3.0430) -- (2.6040, -0.1200, 3.0371) -- (2.6040, -0.0660, 3.0434) -- (2.5580, -0.0660, 3.0493) -- cycle;
\fill[blue!40.3, opacity=0.7] (2.5580, -0.0660, 3.0493) -- (2.6040, -0.0660, 3.0434) -- (2.6040, -0.0120, 3.0496) -- (2.5580, -0.0120, 3.0555) -- cycle;
\fill[blue!28.7, opacity=0.7] (2.5580, -0.0120, 3.0555) -- (2.6040, -0.0120, 3.0496) -- (2.6040, 0.0420, 3.0559) -- (2.5580, 0.0420, 3.0618) -- cycle;
\fill[blue!20.1, opacity=0.7] (2.5580, 0.0420, 3.0618) -- (2.6040, 0.0420, 3.0559) -- (2.6040, 0.0960, 3.0620) -- (2.5580, 0.0960, 3.0680) -- cycle;
\fill[blue!16.9, opacity=0.7] (2.5580, 0.0960, 3.0680) -- (2.6040, 0.0960, 3.0620) -- (2.6040, 0.1500, 3.0681) -- (2.5580, 0.1500, 3.0741) -- cycle;
\fill[blue!16.0, opacity=0.7] (2.5580, 0.1500, 3.0741) -- (2.6040, 0.1500, 3.0681) -- (2.6040, 0.2040, 3.0742) -- (2.5580, 0.2040, 3.0801) -- cycle;
\fill[blue!16.2, opacity=0.7] (2.5580, 0.2040, 3.0801) -- (2.6040, 0.2040, 3.0742) -- (2.6040, 0.2580, 3.0801) -- (2.5580, 0.2580, 3.0860) -- cycle;
\fill[blue!17.8, opacity=0.7] (2.5580, 0.2580, 3.0860) -- (2.6040, 0.2580, 3.0801) -- (2.6040, 0.3120, 3.0859) -- (2.5580, 0.3120, 3.0918) -- cycle;
\fill[blue!23.4, opacity=0.7] (2.5580, 0.3120, 3.0918) -- (2.6040, 0.3120, 3.0859) -- (2.6040, 0.3660, 3.0916) -- (2.5580, 0.3660, 3.0975) -- cycle;
\fill[blue!36.6, opacity=0.7] (2.5580, 0.3660, 3.0975) -- (2.6040, 0.3660, 3.0916) -- (2.6040, 0.4200, 3.0971) -- (2.5580, 0.4200, 3.1030) -- cycle;
\fill[blue!52.9, opacity=0.7] (2.5580, 0.4200, 3.1030) -- (2.6040, 0.4200, 3.0971) -- (2.6040, 0.4740, 3.1024) -- (2.5580, 0.4740, 3.1084) -- cycle;
\fill[blue!62.1, opacity=0.7] (2.5580, 0.4740, 3.1084) -- (2.6040, 0.4740, 3.1024) -- (2.6040, 0.5280, 3.1076) -- (2.5580, 0.5280, 3.1135) -- cycle;
\fill[blue!63.5, opacity=0.7] (2.5580, 0.5280, 3.1135) -- (2.6040, 0.5280, 3.1076) -- (2.6040, 0.5820, 3.1126) -- (2.5580, 0.5820, 3.1185) -- cycle;
\fill[blue!63.1, opacity=0.7] (2.5580, 0.5820, 3.1185) -- (2.6040, 0.5820, 3.1126) -- (2.6040, 0.6360, 3.1174) -- (2.5580, 0.6360, 3.1233) -- cycle;
\fill[blue!63.4, opacity=0.7] (2.5580, 0.6360, 3.1233) -- (2.6040, 0.6360, 3.1174) -- (2.6040, 0.6900, 3.1219) -- (2.5580, 0.6900, 3.1279) -- cycle;
\fill[blue!63.4, opacity=0.7] (2.5580, 0.6900, 3.1279) -- (2.6040, 0.6900, 3.1219) -- (2.6040, 0.7440, 3.1263) -- (2.5580, 0.7440, 3.1322) -- cycle;
\fill[blue!60.8, opacity=0.7] (2.5580, 0.7440, 3.1322) -- (2.6040, 0.7440, 3.1263) -- (2.6040, 0.7980, 3.1303) -- (2.5580, 0.7980, 3.1363) -- cycle;
\fill[blue!55.1, opacity=0.7] (2.5580, 0.7980, 3.1363) -- (2.6040, 0.7980, 3.1303) -- (2.6040, 0.8520, 3.1342) -- (2.5580, 0.8520, 3.1401) -- cycle;
\fill[blue!47.7, opacity=0.7] (2.5580, 0.8520, 3.1401) -- (2.6040, 0.8520, 3.1342) -- (2.6040, 0.9060, 3.1377) -- (2.5580, 0.9060, 3.1436) -- cycle;
\fill[blue!41.0, opacity=0.7] (2.5580, 0.9060, 3.1436) -- (2.6040, 0.9060, 3.1377) -- (2.6040, 0.9600, 3.1410) -- (2.5580, 0.9600, 3.1469) -- cycle;
\fill[blue!35.8, opacity=0.7] (2.5580, 0.9600, 3.1469) -- (2.6040, 0.9600, 3.1410) -- (2.6040, 1.0140, 3.1440) -- (2.5580, 1.0140, 3.1499) -- cycle;
\fill[blue!32.4, opacity=0.7] (2.5580, 1.0140, 3.1499) -- (2.6040, 1.0140, 3.1440) -- (2.6040, 1.0680, 3.1467) -- (2.5580, 1.0680, 3.1526) -- cycle;
\fill[blue!30.3, opacity=0.7] (2.5580, 1.0680, 3.1526) -- (2.6040, 1.0680, 3.1467) -- (2.6040, 1.1220, 3.1491) -- (2.5580, 1.1220, 3.1550) -- cycle;
\fill[blue!29.2, opacity=0.7] (2.5580, 1.1220, 3.1550) -- (2.6040, 1.1220, 3.1491) -- (2.6040, 1.1760, 3.1512) -- (2.5580, 1.1760, 3.1571) -- cycle;
\fill[blue!28.7, opacity=0.7] (2.5580, 1.1760, 3.1571) -- (2.6040, 1.1760, 3.1512) -- (2.6040, 1.2300, 3.1530) -- (2.5580, 1.2300, 3.1589) -- cycle;
\fill[blue!28.7, opacity=0.7] (2.5580, 1.2300, 3.1589) -- (2.6040, 1.2300, 3.1530) -- (2.6040, 1.2840, 3.1545) -- (2.5580, 1.2840, 3.1604) -- cycle;
\fill[blue!28.8, opacity=0.7] (2.5580, 1.2840, 3.1604) -- (2.6040, 1.2840, 3.1545) -- (2.6040, 1.3380, 3.1556) -- (2.5580, 1.3380, 3.1615) -- cycle;
\fill[blue!29.1, opacity=0.7] (2.5580, 1.3380, 3.1615) -- (2.6040, 1.3380, 3.1556) -- (2.6040, 1.3920, 3.1564) -- (2.5580, 1.3920, 3.1623) -- cycle;
\fill[blue!29.4, opacity=0.7] (2.5580, 1.3920, 3.1623) -- (2.6040, 1.3920, 3.1564) -- (2.6040, 1.4460, 3.1569) -- (2.5580, 1.4460, 3.1628) -- cycle;
\fill[blue!29.9, opacity=0.7] (2.5580, 1.4460, 3.1628) -- (2.6040, 1.4460, 3.1569) -- (2.6040, 1.5000, 3.1571) -- (2.5580, 1.5000, 3.1630) -- cycle;
\fill[blue!30.7, opacity=0.7] (2.5580, 1.5000, 3.1630) -- (2.6040, 1.5000, 3.1571) -- (2.6040, 1.5540, 3.1569) -- (2.5580, 1.5540, 3.1628) -- cycle;
\fill[blue!32.1, opacity=0.7] (2.5580, 1.5540, 3.1628) -- (2.6040, 1.5540, 3.1569) -- (2.6040, 1.6080, 3.1564) -- (2.5580, 1.6080, 3.1623) -- cycle;
\fill[blue!34.2, opacity=0.7] (2.5580, 1.6080, 3.1623) -- (2.6040, 1.6080, 3.1564) -- (2.6040, 1.6620, 3.1556) -- (2.5580, 1.6620, 3.1615) -- cycle;
\fill[blue!37.7, opacity=0.7] (2.5580, 1.6620, 3.1615) -- (2.6040, 1.6620, 3.1556) -- (2.6040, 1.7160, 3.1545) -- (2.5580, 1.7160, 3.1604) -- cycle;
\fill[blue!42.9, opacity=0.7] (2.5580, 1.7160, 3.1604) -- (2.6040, 1.7160, 3.1545) -- (2.6040, 1.7700, 3.1530) -- (2.5580, 1.7700, 3.1589) -- cycle;
\fill[blue!49.6, opacity=0.7] (2.5580, 1.7700, 3.1589) -- (2.6040, 1.7700, 3.1530) -- (2.6040, 1.8240, 3.1512) -- (2.5580, 1.8240, 3.1571) -- cycle;
\fill[blue!56.9, opacity=0.7] (2.5580, 1.8240, 3.1571) -- (2.6040, 1.8240, 3.1512) -- (2.6040, 1.8780, 3.1491) -- (2.5580, 1.8780, 3.1550) -- cycle;
\fill[blue!62.2, opacity=0.7] (2.5580, 1.8780, 3.1550) -- (2.6040, 1.8780, 3.1491) -- (2.6040, 1.9320, 3.1467) -- (2.5580, 1.9320, 3.1526) -- cycle;
\fill[blue!63.5, opacity=0.7] (2.5580, 1.9320, 3.1526) -- (2.6040, 1.9320, 3.1467) -- (2.6040, 1.9860, 3.1440) -- (2.5580, 1.9860, 3.1499) -- cycle;
\fill[blue!61.2, opacity=0.7] (2.5580, 1.9860, 3.1499) -- (2.6040, 1.9860, 3.1440) -- (2.6040, 2.0400, 3.1410) -- (2.5580, 2.0400, 3.1469) -- cycle;
\fill[blue!58.7, opacity=0.7] (2.5580, 2.0400, 3.1469) -- (2.6040, 2.0400, 3.1410) -- (2.6040, 2.0940, 3.1377) -- (2.5580, 2.0940, 3.1436) -- cycle;
\fill[blue!59.0, opacity=0.7] (2.5580, 2.0940, 3.1436) -- (2.6040, 2.0940, 3.1377) -- (2.6040, 2.1480, 3.1342) -- (2.5580, 2.1480, 3.1401) -- cycle;
\fill[blue!62.2, opacity=0.7] (2.5580, 2.1480, 3.1401) -- (2.6040, 2.1480, 3.1342) -- (2.6040, 2.2020, 3.1303) -- (2.5580, 2.2020, 3.1363) -- cycle;
\fill[blue!63.0, opacity=0.7] (2.5580, 2.2020, 3.1363) -- (2.6040, 2.2020, 3.1303) -- (2.6040, 2.2560, 3.1263) -- (2.5580, 2.2560, 3.1322) -- cycle;
\fill[blue!53.2, opacity=0.7] (2.5580, 2.2560, 3.1322) -- (2.6040, 2.2560, 3.1263) -- (2.6040, 2.3100, 3.1219) -- (2.5580, 2.3100, 3.1279) -- cycle;
\fill[blue!36.0, opacity=0.7] (2.5580, 2.3100, 3.1279) -- (2.6040, 2.3100, 3.1219) -- (2.6040, 2.3640, 3.1174) -- (2.5580, 2.3640, 3.1233) -- cycle;
\fill[blue!23.8, opacity=0.7] (2.5580, 2.3640, 3.1233) -- (2.6040, 2.3640, 3.1174) -- (2.6040, 2.4180, 3.1126) -- (2.5580, 2.4180, 3.1185) -- cycle;
\fill[blue!19.1, opacity=0.7] (2.5580, 2.4180, 3.1185) -- (2.6040, 2.4180, 3.1126) -- (2.6040, 2.4720, 3.1076) -- (2.5580, 2.4720, 3.1135) -- cycle;
\fill[blue!18.3, opacity=0.7] (2.5580, 2.4720, 3.1135) -- (2.6040, 2.4720, 3.1076) -- (2.6040, 2.5260, 3.1024) -- (2.5580, 2.5260, 3.1084) -- cycle;
\fill[blue!20.1, opacity=0.7] (2.5580, 2.5260, 3.1084) -- (2.6040, 2.5260, 3.1024) -- (2.6040, 2.5800, 3.0971) -- (2.5580, 2.5800, 3.1030) -- cycle;
\fill[blue!26.9, opacity=0.7] (2.5580, 2.5800, 3.1030) -- (2.6040, 2.5800, 3.0971) -- (2.6040, 2.6340, 3.0916) -- (2.5580, 2.6340, 3.0975) -- cycle;
\fill[blue!41.0, opacity=0.7] (2.5580, 2.6340, 3.0975) -- (2.6040, 2.6340, 3.0916) -- (2.6040, 2.6880, 3.0859) -- (2.5580, 2.6880, 3.0918) -- cycle;
\fill[blue!54.7, opacity=0.7] (2.5580, 2.6880, 3.0918) -- (2.6040, 2.6880, 3.0859) -- (2.6040, 2.7420, 3.0801) -- (2.5580, 2.7420, 3.0860) -- cycle;
\fill[blue!59.7, opacity=0.7] (2.5580, 2.7420, 3.0860) -- (2.6040, 2.7420, 3.0801) -- (2.6040, 2.7960, 3.0742) -- (2.5580, 2.7960, 3.0801) -- cycle;
\fill[blue!56.9, opacity=0.7] (2.5580, 2.7960, 3.0801) -- (2.6040, 2.7960, 3.0742) -- (2.6040, 2.8500, 3.0681) -- (2.5580, 2.8500, 3.0741) -- cycle;
\fill[blue!43.0, opacity=0.7] (2.5580, 2.8500, 3.0741) -- (2.6040, 2.8500, 3.0681) -- (2.6040, 2.9040, 3.0620) -- (2.5580, 2.9040, 3.0680) -- cycle;
\fill[blue!24.7, opacity=0.7] (2.5580, 2.9040, 3.0680) -- (2.6040, 2.9040, 3.0620) -- (2.6040, 2.9580, 3.0559) -- (2.5580, 2.9580, 3.0618) -- cycle;
\fill[blue!16.9, opacity=0.7] (2.5580, 2.9580, 3.0618) -- (2.6040, 2.9580, 3.0559) -- (2.6040, 3.0120, 3.0496) -- (2.5580, 3.0120, 3.0555) -- cycle;
\fill[blue!15.4, opacity=0.7] (2.5580, 3.0120, 3.0555) -- (2.6040, 3.0120, 3.0496) -- (2.6040, 3.0660, 3.0434) -- (2.5580, 3.0660, 3.0493) -- cycle;
\fill[blue!15.3, opacity=0.7] (2.5580, 3.0660, 3.0493) -- (2.6040, 3.0660, 3.0434) -- (2.6040, 3.1200, 3.0371) -- (2.5580, 3.1200, 3.0430) -- cycle;
\fill[blue!43.4, opacity=0.7] (2.6040, -0.1200, 3.0371) -- (2.6500, -0.1200, 3.0311) -- (2.6500, -0.0660, 3.0373) -- (2.6040, -0.0660, 3.0434) -- cycle;
\fill[blue!46.6, opacity=0.7] (2.6040, -0.0660, 3.0434) -- (2.6500, -0.0660, 3.0373) -- (2.6500, -0.0120, 3.0436) -- (2.6040, -0.0120, 3.0496) -- cycle;
\fill[blue!40.5, opacity=0.7] (2.6040, -0.0120, 3.0496) -- (2.6500, -0.0120, 3.0436) -- (2.6500, 0.0420, 3.0498) -- (2.6040, 0.0420, 3.0559) -- cycle;
\fill[blue!29.3, opacity=0.7] (2.6040, 0.0420, 3.0559) -- (2.6500, 0.0420, 3.0498) -- (2.6500, 0.0960, 3.0560) -- (2.6040, 0.0960, 3.0620) -- cycle;
\fill[blue!20.7, opacity=0.7] (2.6040, 0.0960, 3.0620) -- (2.6500, 0.0960, 3.0560) -- (2.6500, 0.1500, 3.0621) -- (2.6040, 0.1500, 3.0681) -- cycle;
\fill[blue!17.1, opacity=0.7] (2.6040, 0.1500, 3.0681) -- (2.6500, 0.1500, 3.0621) -- (2.6500, 0.2040, 3.0681) -- (2.6040, 0.2040, 3.0742) -- cycle;
\fill[blue!16.1, opacity=0.7] (2.6040, 0.2040, 3.0742) -- (2.6500, 0.2040, 3.0681) -- (2.6500, 0.2580, 3.0741) -- (2.6040, 0.2580, 3.0801) -- cycle;
\fill[blue!16.2, opacity=0.7] (2.6040, 0.2580, 3.0801) -- (2.6500, 0.2580, 3.0741) -- (2.6500, 0.3120, 3.0799) -- (2.6040, 0.3120, 3.0859) -- cycle;
\fill[blue!17.2, opacity=0.7] (2.6040, 0.3120, 3.0859) -- (2.6500, 0.3120, 3.0799) -- (2.6500, 0.3660, 3.0855) -- (2.6040, 0.3660, 3.0916) -- cycle;
\fill[blue!21.0, opacity=0.7] (2.6040, 0.3660, 3.0916) -- (2.6500, 0.3660, 3.0855) -- (2.6500, 0.4200, 3.0911) -- (2.6040, 0.4200, 3.0971) -- cycle;
\fill[blue!30.6, opacity=0.7] (2.6040, 0.4200, 3.0971) -- (2.6500, 0.4200, 3.0911) -- (2.6500, 0.4740, 3.0964) -- (2.6040, 0.4740, 3.1024) -- cycle;
\fill[blue!45.7, opacity=0.7] (2.6040, 0.4740, 3.1024) -- (2.6500, 0.4740, 3.0964) -- (2.6500, 0.5280, 3.1016) -- (2.6040, 0.5280, 3.1076) -- cycle;
\fill[blue!58.3, opacity=0.7] (2.6040, 0.5280, 3.1076) -- (2.6500, 0.5280, 3.1016) -- (2.6500, 0.5820, 3.1066) -- (2.6040, 0.5820, 3.1126) -- cycle;
\fill[blue!63.2, opacity=0.7] (2.6040, 0.5820, 3.1126) -- (2.6500, 0.5820, 3.1066) -- (2.6500, 0.6360, 3.1114) -- (2.6040, 0.6360, 3.1174) -- cycle;
\fill[blue!63.3, opacity=0.7] (2.6040, 0.6360, 3.1174) -- (2.6500, 0.6360, 3.1114) -- (2.6500, 0.6900, 3.1159) -- (2.6040, 0.6900, 3.1219) -- cycle;
\fill[blue!62.9, opacity=0.7] (2.6040, 0.6900, 3.1219) -- (2.6500, 0.6900, 3.1159) -- (2.6500, 0.7440, 3.1202) -- (2.6040, 0.7440, 3.1263) -- cycle;
\fill[blue!63.2, opacity=0.7] (2.6040, 0.7440, 3.1263) -- (2.6500, 0.7440, 3.1202) -- (2.6500, 0.7980, 3.1243) -- (2.6040, 0.7980, 3.1303) -- cycle;
\fill[blue!63.6, opacity=0.7] (2.6040, 0.7980, 3.1303) -- (2.6500, 0.7980, 3.1243) -- (2.6500, 0.8520, 3.1281) -- (2.6040, 0.8520, 3.1342) -- cycle;
\fill[blue!62.7, opacity=0.7] (2.6040, 0.8520, 3.1342) -- (2.6500, 0.8520, 3.1281) -- (2.6500, 0.9060, 3.1317) -- (2.6040, 0.9060, 3.1377) -- cycle;
\fill[blue!59.8, opacity=0.7] (2.6040, 0.9060, 3.1377) -- (2.6500, 0.9060, 3.1317) -- (2.6500, 0.9600, 3.1350) -- (2.6040, 0.9600, 3.1410) -- cycle;
\fill[blue!55.6, opacity=0.7] (2.6040, 0.9600, 3.1410) -- (2.6500, 0.9600, 3.1350) -- (2.6500, 1.0140, 3.1380) -- (2.6040, 1.0140, 3.1440) -- cycle;
\fill[blue!50.9, opacity=0.7] (2.6040, 1.0140, 3.1440) -- (2.6500, 1.0140, 3.1380) -- (2.6500, 1.0680, 3.1407) -- (2.6040, 1.0680, 3.1467) -- cycle;
\fill[blue!46.6, opacity=0.7] (2.6040, 1.0680, 3.1467) -- (2.6500, 1.0680, 3.1407) -- (2.6500, 1.1220, 3.1431) -- (2.6040, 1.1220, 3.1491) -- cycle;
\fill[blue!43.2, opacity=0.7] (2.6040, 1.1220, 3.1491) -- (2.6500, 1.1220, 3.1431) -- (2.6500, 1.1760, 3.1452) -- (2.6040, 1.1760, 3.1512) -- cycle;
\fill[blue!40.8, opacity=0.7] (2.6040, 1.1760, 3.1512) -- (2.6500, 1.1760, 3.1452) -- (2.6500, 1.2300, 3.1470) -- (2.6040, 1.2300, 3.1530) -- cycle;
\fill[blue!39.4, opacity=0.7] (2.6040, 1.2300, 3.1530) -- (2.6500, 1.2300, 3.1470) -- (2.6500, 1.2840, 3.1484) -- (2.6040, 1.2840, 3.1545) -- cycle;
\fill[blue!38.7, opacity=0.7] (2.6040, 1.2840, 3.1545) -- (2.6500, 1.2840, 3.1484) -- (2.6500, 1.3380, 3.1496) -- (2.6040, 1.3380, 3.1556) -- cycle;
\fill[blue!38.8, opacity=0.7] (2.6040, 1.3380, 3.1556) -- (2.6500, 1.3380, 3.1496) -- (2.6500, 1.3920, 3.1504) -- (2.6040, 1.3920, 3.1564) -- cycle;
\fill[blue!39.7, opacity=0.7] (2.6040, 1.3920, 3.1564) -- (2.6500, 1.3920, 3.1504) -- (2.6500, 1.4460, 3.1509) -- (2.6040, 1.4460, 3.1569) -- cycle;
\fill[blue!41.3, opacity=0.7] (2.6040, 1.4460, 3.1569) -- (2.6500, 1.4460, 3.1509) -- (2.6500, 1.5000, 3.1511) -- (2.6040, 1.5000, 3.1571) -- cycle;
\fill[blue!43.8, opacity=0.7] (2.6040, 1.5000, 3.1571) -- (2.6500, 1.5000, 3.1511) -- (2.6500, 1.5540, 3.1509) -- (2.6040, 1.5540, 3.1569) -- cycle;
\fill[blue!47.3, opacity=0.7] (2.6040, 1.5540, 3.1569) -- (2.6500, 1.5540, 3.1509) -- (2.6500, 1.6080, 3.1504) -- (2.6040, 1.6080, 3.1564) -- cycle;
\fill[blue!51.6, opacity=0.7] (2.6040, 1.6080, 3.1564) -- (2.6500, 1.6080, 3.1504) -- (2.6500, 1.6620, 3.1496) -- (2.6040, 1.6620, 3.1556) -- cycle;
\fill[blue!56.4, opacity=0.7] (2.6040, 1.6620, 3.1556) -- (2.6500, 1.6620, 3.1496) -- (2.6500, 1.7160, 3.1484) -- (2.6040, 1.7160, 3.1545) -- cycle;
\fill[blue!60.7, opacity=0.7] (2.6040, 1.7160, 3.1545) -- (2.6500, 1.7160, 3.1484) -- (2.6500, 1.7700, 3.1470) -- (2.6040, 1.7700, 3.1530) -- cycle;
\fill[blue!63.2, opacity=0.7] (2.6040, 1.7700, 3.1530) -- (2.6500, 1.7700, 3.1470) -- (2.6500, 1.8240, 3.1452) -- (2.6040, 1.8240, 3.1512) -- cycle;
\fill[blue!63.3, opacity=0.7] (2.6040, 1.8240, 3.1512) -- (2.6500, 1.8240, 3.1452) -- (2.6500, 1.8780, 3.1431) -- (2.6040, 1.8780, 3.1491) -- cycle;
\fill[blue!61.4, opacity=0.7] (2.6040, 1.8780, 3.1491) -- (2.6500, 1.8780, 3.1431) -- (2.6500, 1.9320, 3.1407) -- (2.6040, 1.9320, 3.1467) -- cycle;
\fill[blue!59.4, opacity=0.7] (2.6040, 1.9320, 3.1467) -- (2.6500, 1.9320, 3.1407) -- (2.6500, 1.9860, 3.1380) -- (2.6040, 1.9860, 3.1440) -- cycle;
\fill[blue!59.1, opacity=0.7] (2.6040, 1.9860, 3.1440) -- (2.6500, 1.9860, 3.1380) -- (2.6500, 2.0400, 3.1350) -- (2.6040, 2.0400, 3.1410) -- cycle;
\fill[blue!61.3, opacity=0.7] (2.6040, 2.0400, 3.1410) -- (2.6500, 2.0400, 3.1350) -- (2.6500, 2.0940, 3.1317) -- (2.6040, 2.0940, 3.1377) -- cycle;
\fill[blue!63.6, opacity=0.7] (2.6040, 2.0940, 3.1377) -- (2.6500, 2.0940, 3.1317) -- (2.6500, 2.1480, 3.1281) -- (2.6040, 2.1480, 3.1342) -- cycle;
\fill[blue!59.8, opacity=0.7] (2.6040, 2.1480, 3.1342) -- (2.6500, 2.1480, 3.1281) -- (2.6500, 2.2020, 3.1243) -- (2.6040, 2.2020, 3.1303) -- cycle;
\fill[blue!46.4, opacity=0.7] (2.6040, 2.2020, 3.1303) -- (2.6500, 2.2020, 3.1243) -- (2.6500, 2.2560, 3.1202) -- (2.6040, 2.2560, 3.1263) -- cycle;
\fill[blue!31.0, opacity=0.7] (2.6040, 2.2560, 3.1263) -- (2.6500, 2.2560, 3.1202) -- (2.6500, 2.3100, 3.1159) -- (2.6040, 2.3100, 3.1219) -- cycle;
\fill[blue!22.0, opacity=0.7] (2.6040, 2.3100, 3.1219) -- (2.6500, 2.3100, 3.1159) -- (2.6500, 2.3640, 3.1114) -- (2.6040, 2.3640, 3.1174) -- cycle;
\fill[blue!18.7, opacity=0.7] (2.6040, 2.3640, 3.1174) -- (2.6500, 2.3640, 3.1114) -- (2.6500, 2.4180, 3.1066) -- (2.6040, 2.4180, 3.1126) -- cycle;
\fill[blue!18.2, opacity=0.7] (2.6040, 2.4180, 3.1126) -- (2.6500, 2.4180, 3.1066) -- (2.6500, 2.4720, 3.1016) -- (2.6040, 2.4720, 3.1076) -- cycle;
\fill[blue!20.2, opacity=0.7] (2.6040, 2.4720, 3.1076) -- (2.6500, 2.4720, 3.1016) -- (2.6500, 2.5260, 3.0964) -- (2.6040, 2.5260, 3.1024) -- cycle;
\fill[blue!26.7, opacity=0.7] (2.6040, 2.5260, 3.1024) -- (2.6500, 2.5260, 3.0964) -- (2.6500, 2.5800, 3.0911) -- (2.6040, 2.5800, 3.0971) -- cycle;
\fill[blue!40.0, opacity=0.7] (2.6040, 2.5800, 3.0971) -- (2.6500, 2.5800, 3.0911) -- (2.6500, 2.6340, 3.0855) -- (2.6040, 2.6340, 3.0916) -- cycle;
\fill[blue!53.6, opacity=0.7] (2.6040, 2.6340, 3.0916) -- (2.6500, 2.6340, 3.0855) -- (2.6500, 2.6880, 3.0799) -- (2.6040, 2.6880, 3.0859) -- cycle;
\fill[blue!59.3, opacity=0.7] (2.6040, 2.6880, 3.0859) -- (2.6500, 2.6880, 3.0799) -- (2.6500, 2.7420, 3.0741) -- (2.6040, 2.7420, 3.0801) -- cycle;
\fill[blue!57.6, opacity=0.7] (2.6040, 2.7420, 3.0801) -- (2.6500, 2.7420, 3.0741) -- (2.6500, 2.7960, 3.0681) -- (2.6040, 2.7960, 3.0742) -- cycle;
\fill[blue!45.9, opacity=0.7] (2.6040, 2.7960, 3.0742) -- (2.6500, 2.7960, 3.0681) -- (2.6500, 2.8500, 3.0621) -- (2.6040, 2.8500, 3.0681) -- cycle;
\fill[blue!27.6, opacity=0.7] (2.6040, 2.8500, 3.0681) -- (2.6500, 2.8500, 3.0621) -- (2.6500, 2.9040, 3.0560) -- (2.6040, 2.9040, 3.0620) -- cycle;
\fill[blue!17.7, opacity=0.7] (2.6040, 2.9040, 3.0620) -- (2.6500, 2.9040, 3.0560) -- (2.6500, 2.9580, 3.0498) -- (2.6040, 2.9580, 3.0559) -- cycle;
\fill[blue!15.5, opacity=0.7] (2.6040, 2.9580, 3.0559) -- (2.6500, 2.9580, 3.0498) -- (2.6500, 3.0120, 3.0436) -- (2.6040, 3.0120, 3.0496) -- cycle;
\fill[blue!15.2, opacity=0.7] (2.6040, 3.0120, 3.0496) -- (2.6500, 3.0120, 3.0436) -- (2.6500, 3.0660, 3.0373) -- (2.6040, 3.0660, 3.0434) -- cycle;
\fill[blue!15.4, opacity=0.7] (2.6040, 3.0660, 3.0434) -- (2.6500, 3.0660, 3.0373) -- (2.6500, 3.1200, 3.0311) -- (2.6040, 3.1200, 3.0371) -- cycle;
\fill[blue!32.2, opacity=0.7] (2.6500, -0.1200, 3.0311) -- (2.6960, -0.1200, 3.0249) -- (2.6960, -0.0660, 3.0312) -- (2.6500, -0.0660, 3.0373) -- cycle;
\fill[blue!43.8, opacity=0.7] (2.6500, -0.0660, 3.0373) -- (2.6960, -0.0660, 3.0312) -- (2.6960, -0.0120, 3.0375) -- (2.6500, -0.0120, 3.0436) -- cycle;
\fill[blue!47.0, opacity=0.7] (2.6500, -0.0120, 3.0436) -- (2.6960, -0.0120, 3.0375) -- (2.6960, 0.0420, 3.0437) -- (2.6500, 0.0420, 3.0498) -- cycle;
\fill[blue!41.5, opacity=0.7] (2.6500, 0.0420, 3.0498) -- (2.6960, 0.0420, 3.0437) -- (2.6960, 0.0960, 3.0499) -- (2.6500, 0.0960, 3.0560) -- cycle;
\fill[blue!30.9, opacity=0.7] (2.6500, 0.0960, 3.0560) -- (2.6960, 0.0960, 3.0499) -- (2.6960, 0.1500, 3.0560) -- (2.6500, 0.1500, 3.0621) -- cycle;
\fill[blue!21.9, opacity=0.7] (2.6500, 0.1500, 3.0621) -- (2.6960, 0.1500, 3.0560) -- (2.6960, 0.2040, 3.0620) -- (2.6500, 0.2040, 3.0681) -- cycle;
\fill[blue!17.7, opacity=0.7] (2.6500, 0.2040, 3.0681) -- (2.6960, 0.2040, 3.0620) -- (2.6960, 0.2580, 3.0680) -- (2.6500, 0.2580, 3.0741) -- cycle;
\fill[blue!16.3, opacity=0.7] (2.6500, 0.2580, 3.0741) -- (2.6960, 0.2580, 3.0680) -- (2.6960, 0.3120, 3.0738) -- (2.6500, 0.3120, 3.0799) -- cycle;
\fill[blue!16.1, opacity=0.7] (2.6500, 0.3120, 3.0799) -- (2.6960, 0.3120, 3.0738) -- (2.6960, 0.3660, 3.0794) -- (2.6500, 0.3660, 3.0855) -- cycle;
\fill[blue!16.7, opacity=0.7] (2.6500, 0.3660, 3.0855) -- (2.6960, 0.3660, 3.0794) -- (2.6960, 0.4200, 3.0849) -- (2.6500, 0.4200, 3.0911) -- cycle;
\fill[blue!18.8, opacity=0.7] (2.6500, 0.4200, 3.0911) -- (2.6960, 0.4200, 3.0849) -- (2.6960, 0.4740, 3.0903) -- (2.6500, 0.4740, 3.0964) -- cycle;
\fill[blue!24.7, opacity=0.7] (2.6500, 0.4740, 3.0964) -- (2.6960, 0.4740, 3.0903) -- (2.6960, 0.5280, 3.0955) -- (2.6500, 0.5280, 3.1016) -- cycle;
\fill[blue!36.1, opacity=0.7] (2.6500, 0.5280, 3.1016) -- (2.6960, 0.5280, 3.0955) -- (2.6960, 0.5820, 3.1005) -- (2.6500, 0.5820, 3.1066) -- cycle;
\fill[blue!50.1, opacity=0.7] (2.6500, 0.5820, 3.1066) -- (2.6960, 0.5820, 3.1005) -- (2.6960, 0.6360, 3.1052) -- (2.6500, 0.6360, 3.1114) -- cycle;
\fill[blue!59.9, opacity=0.7] (2.6500, 0.6360, 3.1114) -- (2.6960, 0.6360, 3.1052) -- (2.6960, 0.6900, 3.1098) -- (2.6500, 0.6900, 3.1159) -- cycle;
\fill[blue!63.4, opacity=0.7] (2.6500, 0.6900, 3.1159) -- (2.6960, 0.6900, 3.1098) -- (2.6960, 0.7440, 3.1141) -- (2.6500, 0.7440, 3.1202) -- cycle;
\fill[blue!63.3, opacity=0.7] (2.6500, 0.7440, 3.1202) -- (2.6960, 0.7440, 3.1141) -- (2.6960, 0.7980, 3.1182) -- (2.6500, 0.7980, 3.1243) -- cycle;
\fill[blue!62.7, opacity=0.7] (2.6500, 0.7980, 3.1243) -- (2.6960, 0.7980, 3.1182) -- (2.6960, 0.8520, 3.1220) -- (2.6500, 0.8520, 3.1281) -- cycle;
\fill[blue!62.7, opacity=0.7] (2.6500, 0.8520, 3.1281) -- (2.6960, 0.8520, 3.1220) -- (2.6960, 0.9060, 3.1256) -- (2.6500, 0.9060, 3.1317) -- cycle;
\fill[blue!63.2, opacity=0.7] (2.6500, 0.9060, 3.1317) -- (2.6960, 0.9060, 3.1256) -- (2.6960, 0.9600, 3.1289) -- (2.6500, 0.9600, 3.1350) -- cycle;
\fill[blue!63.5, opacity=0.7] (2.6500, 0.9600, 3.1350) -- (2.6960, 0.9600, 3.1289) -- (2.6960, 1.0140, 3.1319) -- (2.6500, 1.0140, 3.1380) -- cycle;
\fill[blue!63.4, opacity=0.7] (2.6500, 1.0140, 3.1380) -- (2.6960, 1.0140, 3.1319) -- (2.6960, 1.0680, 3.1346) -- (2.6500, 1.0680, 3.1407) -- cycle;
\fill[blue!62.6, opacity=0.7] (2.6500, 1.0680, 3.1407) -- (2.6960, 1.0680, 3.1346) -- (2.6960, 1.1220, 3.1370) -- (2.6500, 1.1220, 3.1431) -- cycle;
\fill[blue!61.5, opacity=0.7] (2.6500, 1.1220, 3.1431) -- (2.6960, 1.1220, 3.1370) -- (2.6960, 1.1760, 3.1391) -- (2.6500, 1.1760, 3.1452) -- cycle;
\fill[blue!60.2, opacity=0.7] (2.6500, 1.1760, 3.1452) -- (2.6960, 1.1760, 3.1391) -- (2.6960, 1.2300, 3.1409) -- (2.6500, 1.2300, 3.1470) -- cycle;
\fill[blue!59.2, opacity=0.7] (2.6500, 1.2300, 3.1470) -- (2.6960, 1.2300, 3.1409) -- (2.6960, 1.2840, 3.1423) -- (2.6500, 1.2840, 3.1484) -- cycle;
\fill[blue!58.6, opacity=0.7] (2.6500, 1.2840, 3.1484) -- (2.6960, 1.2840, 3.1423) -- (2.6960, 1.3380, 3.1435) -- (2.6500, 1.3380, 3.1496) -- cycle;
\fill[blue!58.7, opacity=0.7] (2.6500, 1.3380, 3.1496) -- (2.6960, 1.3380, 3.1435) -- (2.6960, 1.3920, 3.1443) -- (2.6500, 1.3920, 3.1504) -- cycle;
\fill[blue!59.3, opacity=0.7] (2.6500, 1.3920, 3.1504) -- (2.6960, 1.3920, 3.1443) -- (2.6960, 1.4460, 3.1448) -- (2.6500, 1.4460, 3.1509) -- cycle;
\fill[blue!60.4, opacity=0.7] (2.6500, 1.4460, 3.1509) -- (2.6960, 1.4460, 3.1448) -- (2.6960, 1.5000, 3.1449) -- (2.6500, 1.5000, 3.1511) -- cycle;
\fill[blue!61.7, opacity=0.7] (2.6500, 1.5000, 3.1511) -- (2.6960, 1.5000, 3.1449) -- (2.6960, 1.5540, 3.1448) -- (2.6500, 1.5540, 3.1509) -- cycle;
\fill[blue!62.9, opacity=0.7] (2.6500, 1.5540, 3.1509) -- (2.6960, 1.5540, 3.1448) -- (2.6960, 1.6080, 3.1443) -- (2.6500, 1.6080, 3.1504) -- cycle;
\fill[blue!63.5, opacity=0.7] (2.6500, 1.6080, 3.1504) -- (2.6960, 1.6080, 3.1443) -- (2.6960, 1.6620, 3.1435) -- (2.6500, 1.6620, 3.1496) -- cycle;
\fill[blue!63.3, opacity=0.7] (2.6500, 1.6620, 3.1496) -- (2.6960, 1.6620, 3.1435) -- (2.6960, 1.7160, 3.1423) -- (2.6500, 1.7160, 3.1484) -- cycle;
\fill[blue!62.2, opacity=0.7] (2.6500, 1.7160, 3.1484) -- (2.6960, 1.7160, 3.1423) -- (2.6960, 1.7700, 3.1409) -- (2.6500, 1.7700, 3.1470) -- cycle;
\fill[blue!60.6, opacity=0.7] (2.6500, 1.7700, 3.1470) -- (2.6960, 1.7700, 3.1409) -- (2.6960, 1.8240, 3.1391) -- (2.6500, 1.8240, 3.1452) -- cycle;
\fill[blue!59.6, opacity=0.7] (2.6500, 1.8240, 3.1452) -- (2.6960, 1.8240, 3.1391) -- (2.6960, 1.8780, 3.1370) -- (2.6500, 1.8780, 3.1431) -- cycle;
\fill[blue!59.9, opacity=0.7] (2.6500, 1.8780, 3.1431) -- (2.6960, 1.8780, 3.1370) -- (2.6960, 1.9320, 3.1346) -- (2.6500, 1.9320, 3.1407) -- cycle;
\fill[blue!61.7, opacity=0.7] (2.6500, 1.9320, 3.1407) -- (2.6960, 1.9320, 3.1346) -- (2.6960, 1.9860, 3.1319) -- (2.6500, 1.9860, 3.1380) -- cycle;
\fill[blue!63.5, opacity=0.7] (2.6500, 1.9860, 3.1380) -- (2.6960, 1.9860, 3.1319) -- (2.6960, 2.0400, 3.1289) -- (2.6500, 2.0400, 3.1350) -- cycle;
\fill[blue!61.4, opacity=0.7] (2.6500, 2.0400, 3.1350) -- (2.6960, 2.0400, 3.1289) -- (2.6960, 2.0940, 3.1256) -- (2.6500, 2.0940, 3.1317) -- cycle;
\fill[blue!51.7, opacity=0.7] (2.6500, 2.0940, 3.1317) -- (2.6960, 2.0940, 3.1256) -- (2.6960, 2.1480, 3.1220) -- (2.6500, 2.1480, 3.1281) -- cycle;
\fill[blue!37.3, opacity=0.7] (2.6500, 2.1480, 3.1281) -- (2.6960, 2.1480, 3.1220) -- (2.6960, 2.2020, 3.1182) -- (2.6500, 2.2020, 3.1243) -- cycle;
\fill[blue!25.9, opacity=0.7] (2.6500, 2.2020, 3.1243) -- (2.6960, 2.2020, 3.1182) -- (2.6960, 2.2560, 3.1141) -- (2.6500, 2.2560, 3.1202) -- cycle;
\fill[blue!20.1, opacity=0.7] (2.6500, 2.2560, 3.1202) -- (2.6960, 2.2560, 3.1141) -- (2.6960, 2.3100, 3.1098) -- (2.6500, 2.3100, 3.1159) -- cycle;
\fill[blue!18.2, opacity=0.7] (2.6500, 2.3100, 3.1159) -- (2.6960, 2.3100, 3.1098) -- (2.6960, 2.3640, 3.1052) -- (2.6500, 2.3640, 3.1114) -- cycle;
\fill[blue!18.3, opacity=0.7] (2.6500, 2.3640, 3.1114) -- (2.6960, 2.3640, 3.1052) -- (2.6960, 2.4180, 3.1005) -- (2.6500, 2.4180, 3.1066) -- cycle;
\fill[blue!20.6, opacity=0.7] (2.6500, 2.4180, 3.1066) -- (2.6960, 2.4180, 3.1005) -- (2.6960, 2.4720, 3.0955) -- (2.6500, 2.4720, 3.1016) -- cycle;
\fill[blue!27.5, opacity=0.7] (2.6500, 2.4720, 3.1016) -- (2.6960, 2.4720, 3.0955) -- (2.6960, 2.5260, 3.0903) -- (2.6500, 2.5260, 3.0964) -- cycle;
\fill[blue!40.5, opacity=0.7] (2.6500, 2.5260, 3.0964) -- (2.6960, 2.5260, 3.0903) -- (2.6960, 2.5800, 3.0849) -- (2.6500, 2.5800, 3.0911) -- cycle;
\fill[blue!53.4, opacity=0.7] (2.6500, 2.5800, 3.0911) -- (2.6960, 2.5800, 3.0849) -- (2.6960, 2.6340, 3.0794) -- (2.6500, 2.6340, 3.0855) -- cycle;
\fill[blue!59.0, opacity=0.7] (2.6500, 2.6340, 3.0855) -- (2.6960, 2.6340, 3.0794) -- (2.6960, 2.6880, 3.0738) -- (2.6500, 2.6880, 3.0799) -- cycle;
\fill[blue!57.7, opacity=0.7] (2.6500, 2.6880, 3.0799) -- (2.6960, 2.6880, 3.0738) -- (2.6960, 2.7420, 3.0680) -- (2.6500, 2.7420, 3.0741) -- cycle;
\fill[blue!47.3, opacity=0.7] (2.6500, 2.7420, 3.0741) -- (2.6960, 2.7420, 3.0680) -- (2.6960, 2.7960, 3.0620) -- (2.6500, 2.7960, 3.0681) -- cycle;
\fill[blue!29.6, opacity=0.7] (2.6500, 2.7960, 3.0681) -- (2.6960, 2.7960, 3.0620) -- (2.6960, 2.8500, 3.0560) -- (2.6500, 2.8500, 3.0621) -- cycle;
\fill[blue!18.5, opacity=0.7] (2.6500, 2.8500, 3.0621) -- (2.6960, 2.8500, 3.0560) -- (2.6960, 2.9040, 3.0499) -- (2.6500, 2.9040, 3.0560) -- cycle;
\fill[blue!15.7, opacity=0.7] (2.6500, 2.9040, 3.0560) -- (2.6960, 2.9040, 3.0499) -- (2.6960, 2.9580, 3.0437) -- (2.6500, 2.9580, 3.0498) -- cycle;
\fill[blue!15.3, opacity=0.7] (2.6500, 2.9580, 3.0498) -- (2.6960, 2.9580, 3.0437) -- (2.6960, 3.0120, 3.0375) -- (2.6500, 3.0120, 3.0436) -- cycle;
\fill[blue!15.3, opacity=0.7] (2.6500, 3.0120, 3.0436) -- (2.6960, 3.0120, 3.0375) -- (2.6960, 3.0660, 3.0312) -- (2.6500, 3.0660, 3.0373) -- cycle;
\fill[blue!16.1, opacity=0.7] (2.6500, 3.0660, 3.0373) -- (2.6960, 3.0660, 3.0312) -- (2.6960, 3.1200, 3.0249) -- (2.6500, 3.1200, 3.0311) -- cycle;
\fill[blue!20.7, opacity=0.7] (2.6960, -0.1200, 3.0249) -- (2.7420, -0.1200, 3.0188) -- (2.7420, -0.0660, 3.0251) -- (2.6960, -0.0660, 3.0312) -- cycle;
\fill[blue!32.1, opacity=0.7] (2.6960, -0.0660, 3.0312) -- (2.7420, -0.0660, 3.0251) -- (2.7420, -0.0120, 3.0313) -- (2.6960, -0.0120, 3.0375) -- cycle;
\fill[blue!43.6, opacity=0.7] (2.6960, -0.0120, 3.0375) -- (2.7420, -0.0120, 3.0313) -- (2.7420, 0.0420, 3.0375) -- (2.6960, 0.0420, 3.0437) -- cycle;
\fill[blue!47.5, opacity=0.7] (2.6960, 0.0420, 3.0437) -- (2.7420, 0.0420, 3.0375) -- (2.7420, 0.0960, 3.0437) -- (2.6960, 0.0960, 3.0499) -- cycle;
\fill[blue!43.2, opacity=0.7] (2.6960, 0.0960, 3.0499) -- (2.7420, 0.0960, 3.0437) -- (2.7420, 0.1500, 3.0498) -- (2.6960, 0.1500, 3.0560) -- cycle;
\fill[blue!33.4, opacity=0.7] (2.6960, 0.1500, 3.0560) -- (2.7420, 0.1500, 3.0498) -- (2.7420, 0.2040, 3.0559) -- (2.6960, 0.2040, 3.0620) -- cycle;
\fill[blue!23.9, opacity=0.7] (2.6960, 0.2040, 3.0620) -- (2.7420, 0.2040, 3.0559) -- (2.7420, 0.2580, 3.0618) -- (2.6960, 0.2580, 3.0680) -- cycle;
\fill[blue!18.7, opacity=0.7] (2.6960, 0.2580, 3.0680) -- (2.7420, 0.2580, 3.0618) -- (2.7420, 0.3120, 3.0676) -- (2.6960, 0.3120, 3.0738) -- cycle;
\fill[blue!16.7, opacity=0.7] (2.6960, 0.3120, 3.0738) -- (2.7420, 0.3120, 3.0676) -- (2.7420, 0.3660, 3.0733) -- (2.6960, 0.3660, 3.0794) -- cycle;
\fill[blue!16.2, opacity=0.7] (2.6960, 0.3660, 3.0794) -- (2.7420, 0.3660, 3.0733) -- (2.7420, 0.4200, 3.0788) -- (2.6960, 0.4200, 3.0849) -- cycle;
\fill[blue!16.3, opacity=0.7] (2.6960, 0.4200, 3.0849) -- (2.7420, 0.4200, 3.0788) -- (2.7420, 0.4740, 3.0841) -- (2.6960, 0.4740, 3.0903) -- cycle;
\fill[blue!17.3, opacity=0.7] (2.6960, 0.4740, 3.0903) -- (2.7420, 0.4740, 3.0841) -- (2.7420, 0.5280, 3.0893) -- (2.6960, 0.5280, 3.0955) -- cycle;
\fill[blue!20.3, opacity=0.7] (2.6960, 0.5280, 3.0955) -- (2.7420, 0.5280, 3.0893) -- (2.7420, 0.5820, 3.0943) -- (2.6960, 0.5820, 3.1005) -- cycle;
\fill[blue!26.9, opacity=0.7] (2.6960, 0.5820, 3.1005) -- (2.7420, 0.5820, 3.0943) -- (2.7420, 0.6360, 3.0991) -- (2.6960, 0.6360, 3.1052) -- cycle;
\fill[blue!37.7, opacity=0.7] (2.6960, 0.6360, 3.1052) -- (2.7420, 0.6360, 3.0991) -- (2.7420, 0.6900, 3.1036) -- (2.6960, 0.6900, 3.1098) -- cycle;
\fill[blue!49.9, opacity=0.7] (2.6960, 0.6900, 3.1098) -- (2.7420, 0.6900, 3.1036) -- (2.7420, 0.7440, 3.1079) -- (2.6960, 0.7440, 3.1141) -- cycle;
\fill[blue!58.8, opacity=0.7] (2.6960, 0.7440, 3.1141) -- (2.7420, 0.7440, 3.1079) -- (2.7420, 0.7980, 3.1120) -- (2.6960, 0.7980, 3.1182) -- cycle;
\fill[blue!62.8, opacity=0.7] (2.6960, 0.7980, 3.1182) -- (2.7420, 0.7980, 3.1120) -- (2.7420, 0.8520, 3.1159) -- (2.6960, 0.8520, 3.1220) -- cycle;
\fill[blue!63.6, opacity=0.7] (2.6960, 0.8520, 3.1220) -- (2.7420, 0.8520, 3.1159) -- (2.7420, 0.9060, 3.1194) -- (2.6960, 0.9060, 3.1256) -- cycle;
\fill[blue!63.0, opacity=0.7] (2.6960, 0.9060, 3.1256) -- (2.7420, 0.9060, 3.1194) -- (2.7420, 0.9600, 3.1227) -- (2.6960, 0.9600, 3.1289) -- cycle;
\fill[blue!62.4, opacity=0.7] (2.6960, 0.9600, 3.1289) -- (2.7420, 0.9600, 3.1227) -- (2.7420, 1.0140, 3.1257) -- (2.6960, 1.0140, 3.1319) -- cycle;
\fill[blue!62.2, opacity=0.7] (2.6960, 1.0140, 3.1319) -- (2.7420, 1.0140, 3.1257) -- (2.7420, 1.0680, 3.1284) -- (2.6960, 1.0680, 3.1346) -- cycle;
\fill[blue!62.3, opacity=0.7] (2.6960, 1.0680, 3.1346) -- (2.7420, 1.0680, 3.1284) -- (2.7420, 1.1220, 3.1308) -- (2.6960, 1.1220, 3.1370) -- cycle;
\fill[blue!62.5, opacity=0.7] (2.6960, 1.1220, 3.1370) -- (2.7420, 1.1220, 3.1308) -- (2.7420, 1.1760, 3.1329) -- (2.6960, 1.1760, 3.1391) -- cycle;
\fill[blue!62.8, opacity=0.7] (2.6960, 1.1760, 3.1391) -- (2.7420, 1.1760, 3.1329) -- (2.7420, 1.2300, 3.1347) -- (2.6960, 1.2300, 3.1409) -- cycle;
\fill[blue!62.9, opacity=0.7] (2.6960, 1.2300, 3.1409) -- (2.7420, 1.2300, 3.1347) -- (2.7420, 1.2840, 3.1361) -- (2.6960, 1.2840, 3.1423) -- cycle;
\fill[blue!63.0, opacity=0.7] (2.6960, 1.2840, 3.1423) -- (2.7420, 1.2840, 3.1361) -- (2.7420, 1.3380, 3.1373) -- (2.6960, 1.3380, 3.1435) -- cycle;
\fill[blue!62.9, opacity=0.7] (2.6960, 1.3380, 3.1435) -- (2.7420, 1.3380, 3.1373) -- (2.7420, 1.3920, 3.1381) -- (2.6960, 1.3920, 3.1443) -- cycle;
\fill[blue!62.8, opacity=0.7] (2.6960, 1.3920, 3.1443) -- (2.7420, 1.3920, 3.1381) -- (2.7420, 1.4460, 3.1386) -- (2.6960, 1.4460, 3.1448) -- cycle;
\fill[blue!62.4, opacity=0.7] (2.6960, 1.4460, 3.1448) -- (2.7420, 1.4460, 3.1386) -- (2.7420, 1.5000, 3.1388) -- (2.6960, 1.5000, 3.1449) -- cycle;
\fill[blue!61.9, opacity=0.7] (2.6960, 1.5000, 3.1449) -- (2.7420, 1.5000, 3.1388) -- (2.7420, 1.5540, 3.1386) -- (2.6960, 1.5540, 3.1448) -- cycle;
\fill[blue!61.2, opacity=0.7] (2.6960, 1.5540, 3.1448) -- (2.7420, 1.5540, 3.1386) -- (2.7420, 1.6080, 3.1381) -- (2.6960, 1.6080, 3.1443) -- cycle;
\fill[blue!60.6, opacity=0.7] (2.6960, 1.6080, 3.1443) -- (2.7420, 1.6080, 3.1381) -- (2.7420, 1.6620, 3.1373) -- (2.6960, 1.6620, 3.1435) -- cycle;
\fill[blue!60.2, opacity=0.7] (2.6960, 1.6620, 3.1435) -- (2.7420, 1.6620, 3.1373) -- (2.7420, 1.7160, 3.1361) -- (2.6960, 1.7160, 3.1423) -- cycle;
\fill[blue!60.5, opacity=0.7] (2.6960, 1.7160, 3.1423) -- (2.7420, 1.7160, 3.1361) -- (2.7420, 1.7700, 3.1347) -- (2.6960, 1.7700, 3.1409) -- cycle;
\fill[blue!61.5, opacity=0.7] (2.6960, 1.7700, 3.1409) -- (2.7420, 1.7700, 3.1347) -- (2.7420, 1.8240, 3.1329) -- (2.6960, 1.8240, 3.1391) -- cycle;
\fill[blue!62.9, opacity=0.7] (2.6960, 1.8240, 3.1391) -- (2.7420, 1.8240, 3.1329) -- (2.7420, 1.8780, 3.1308) -- (2.6960, 1.8780, 3.1370) -- cycle;
\fill[blue!63.5, opacity=0.7] (2.6960, 1.8780, 3.1370) -- (2.7420, 1.8780, 3.1308) -- (2.7420, 1.9320, 3.1284) -- (2.6960, 1.9320, 3.1346) -- cycle;
\fill[blue!60.5, opacity=0.7] (2.6960, 1.9320, 3.1346) -- (2.7420, 1.9320, 3.1284) -- (2.7420, 1.9860, 3.1257) -- (2.6960, 1.9860, 3.1319) -- cycle;
\fill[blue!52.0, opacity=0.7] (2.6960, 1.9860, 3.1319) -- (2.7420, 1.9860, 3.1257) -- (2.7420, 2.0400, 3.1227) -- (2.6960, 2.0400, 3.1289) -- cycle;
\fill[blue!39.5, opacity=0.7] (2.6960, 2.0400, 3.1289) -- (2.7420, 2.0400, 3.1227) -- (2.7420, 2.0940, 3.1194) -- (2.6960, 2.0940, 3.1256) -- cycle;
\fill[blue!28.4, opacity=0.7] (2.6960, 2.0940, 3.1256) -- (2.7420, 2.0940, 3.1194) -- (2.7420, 2.1480, 3.1159) -- (2.6960, 2.1480, 3.1220) -- cycle;
\fill[blue!21.7, opacity=0.7] (2.6960, 2.1480, 3.1220) -- (2.7420, 2.1480, 3.1159) -- (2.7420, 2.2020, 3.1120) -- (2.6960, 2.2020, 3.1182) -- cycle;
\fill[blue!18.7, opacity=0.7] (2.6960, 2.2020, 3.1182) -- (2.7420, 2.2020, 3.1120) -- (2.7420, 2.2560, 3.1079) -- (2.6960, 2.2560, 3.1141) -- cycle;
\fill[blue!17.9, opacity=0.7] (2.6960, 2.2560, 3.1141) -- (2.7420, 2.2560, 3.1079) -- (2.7420, 2.3100, 3.1036) -- (2.6960, 2.3100, 3.1098) -- cycle;
\fill[blue!18.6, opacity=0.7] (2.6960, 2.3100, 3.1098) -- (2.7420, 2.3100, 3.1036) -- (2.7420, 2.3640, 3.0991) -- (2.6960, 2.3640, 3.1052) -- cycle;
\fill[blue!21.7, opacity=0.7] (2.6960, 2.3640, 3.1052) -- (2.7420, 2.3640, 3.0991) -- (2.7420, 2.4180, 3.0943) -- (2.6960, 2.4180, 3.1005) -- cycle;
\fill[blue!29.4, opacity=0.7] (2.6960, 2.4180, 3.1005) -- (2.7420, 2.4180, 3.0943) -- (2.7420, 2.4720, 3.0893) -- (2.6960, 2.4720, 3.0955) -- cycle;
\fill[blue!42.2, opacity=0.7] (2.6960, 2.4720, 3.0955) -- (2.7420, 2.4720, 3.0893) -- (2.7420, 2.5260, 3.0841) -- (2.6960, 2.5260, 3.0903) -- cycle;
\fill[blue!53.9, opacity=0.7] (2.6960, 2.5260, 3.0903) -- (2.7420, 2.5260, 3.0841) -- (2.7420, 2.5800, 3.0788) -- (2.6960, 2.5800, 3.0849) -- cycle;
\fill[blue!58.9, opacity=0.7] (2.6960, 2.5800, 3.0849) -- (2.7420, 2.5800, 3.0788) -- (2.7420, 2.6340, 3.0733) -- (2.6960, 2.6340, 3.0794) -- cycle;
\fill[blue!57.3, opacity=0.7] (2.6960, 2.6340, 3.0794) -- (2.7420, 2.6340, 3.0733) -- (2.7420, 2.6880, 3.0676) -- (2.6960, 2.6880, 3.0738) -- cycle;
\fill[blue!47.4, opacity=0.7] (2.6960, 2.6880, 3.0738) -- (2.7420, 2.6880, 3.0676) -- (2.7420, 2.7420, 3.0618) -- (2.6960, 2.7420, 3.0680) -- cycle;
\fill[blue!30.4, opacity=0.7] (2.6960, 2.7420, 3.0680) -- (2.7420, 2.7420, 3.0618) -- (2.7420, 2.7960, 3.0559) -- (2.6960, 2.7960, 3.0620) -- cycle;
\fill[blue!19.1, opacity=0.7] (2.6960, 2.7960, 3.0620) -- (2.7420, 2.7960, 3.0559) -- (2.7420, 2.8500, 3.0498) -- (2.6960, 2.8500, 3.0560) -- cycle;
\fill[blue!15.8, opacity=0.7] (2.6960, 2.8500, 3.0560) -- (2.7420, 2.8500, 3.0498) -- (2.7420, 2.9040, 3.0437) -- (2.6960, 2.9040, 3.0499) -- cycle;
\fill[blue!15.3, opacity=0.7] (2.6960, 2.9040, 3.0499) -- (2.7420, 2.9040, 3.0437) -- (2.7420, 2.9580, 3.0375) -- (2.6960, 2.9580, 3.0437) -- cycle;
\fill[blue!15.3, opacity=0.7] (2.6960, 2.9580, 3.0437) -- (2.7420, 2.9580, 3.0375) -- (2.7420, 3.0120, 3.0313) -- (2.6960, 3.0120, 3.0375) -- cycle;
\fill[blue!15.8, opacity=0.7] (2.6960, 3.0120, 3.0375) -- (2.7420, 3.0120, 3.0313) -- (2.7420, 3.0660, 3.0251) -- (2.6960, 3.0660, 3.0312) -- cycle;
\fill[blue!18.3, opacity=0.7] (2.6960, 3.0660, 3.0312) -- (2.7420, 3.0660, 3.0251) -- (2.7420, 3.1200, 3.0188) -- (2.6960, 3.1200, 3.0249) -- cycle;
\fill[blue!16.0, opacity=0.7] (2.7420, -0.1200, 3.0188) -- (2.7880, -0.1200, 3.0125) -- (2.7880, -0.0660, 3.0188) -- (2.7420, -0.0660, 3.0251) -- cycle;
\fill[blue!20.3, opacity=0.7] (2.7420, -0.0660, 3.0251) -- (2.7880, -0.0660, 3.0188) -- (2.7880, -0.0120, 3.0251) -- (2.7420, -0.0120, 3.0313) -- cycle;
\fill[blue!31.1, opacity=0.7] (2.7420, -0.0120, 3.0313) -- (2.7880, -0.0120, 3.0251) -- (2.7880, 0.0420, 3.0313) -- (2.7420, 0.0420, 3.0375) -- cycle;
\fill[blue!42.7, opacity=0.7] (2.7420, 0.0420, 3.0375) -- (2.7880, 0.0420, 3.0313) -- (2.7880, 0.0960, 3.0375) -- (2.7420, 0.0960, 3.0437) -- cycle;
\fill[blue!47.8, opacity=0.7] (2.7420, 0.0960, 3.0437) -- (2.7880, 0.0960, 3.0375) -- (2.7880, 0.1500, 3.0436) -- (2.7420, 0.1500, 3.0498) -- cycle;
\fill[blue!45.3, opacity=0.7] (2.7420, 0.1500, 3.0498) -- (2.7880, 0.1500, 3.0436) -- (2.7880, 0.2040, 3.0496) -- (2.7420, 0.2040, 3.0559) -- cycle;
\fill[blue!37.0, opacity=0.7] (2.7420, 0.2040, 3.0559) -- (2.7880, 0.2040, 3.0496) -- (2.7880, 0.2580, 3.0555) -- (2.7420, 0.2580, 3.0618) -- cycle;
\fill[blue!27.2, opacity=0.7] (2.7420, 0.2580, 3.0618) -- (2.7880, 0.2580, 3.0555) -- (2.7880, 0.3120, 3.0614) -- (2.7420, 0.3120, 3.0676) -- cycle;
\fill[blue!20.6, opacity=0.7] (2.7420, 0.3120, 3.0676) -- (2.7880, 0.3120, 3.0614) -- (2.7880, 0.3660, 3.0670) -- (2.7420, 0.3660, 3.0733) -- cycle;
\fill[blue!17.5, opacity=0.7] (2.7420, 0.3660, 3.0733) -- (2.7880, 0.3660, 3.0670) -- (2.7880, 0.4200, 3.0725) -- (2.7420, 0.4200, 3.0788) -- cycle;
\fill[blue!16.4, opacity=0.7] (2.7420, 0.4200, 3.0788) -- (2.7880, 0.4200, 3.0725) -- (2.7880, 0.4740, 3.0779) -- (2.7420, 0.4740, 3.0841) -- cycle;
\fill[blue!16.2, opacity=0.7] (2.7420, 0.4740, 3.0841) -- (2.7880, 0.4740, 3.0779) -- (2.7880, 0.5280, 3.0831) -- (2.7420, 0.5280, 3.0893) -- cycle;
\fill[blue!16.5, opacity=0.7] (2.7420, 0.5280, 3.0893) -- (2.7880, 0.5280, 3.0831) -- (2.7880, 0.5820, 3.0881) -- (2.7420, 0.5820, 3.0943) -- cycle;
\fill[blue!17.7, opacity=0.7] (2.7420, 0.5820, 3.0943) -- (2.7880, 0.5820, 3.0881) -- (2.7880, 0.6360, 3.0928) -- (2.7420, 0.6360, 3.0991) -- cycle;
\fill[blue!20.5, opacity=0.7] (2.7420, 0.6360, 3.0991) -- (2.7880, 0.6360, 3.0928) -- (2.7880, 0.6900, 3.0974) -- (2.7420, 0.6900, 3.1036) -- cycle;
\fill[blue!26.1, opacity=0.7] (2.7420, 0.6900, 3.1036) -- (2.7880, 0.6900, 3.0974) -- (2.7880, 0.7440, 3.1017) -- (2.7420, 0.7440, 3.1079) -- cycle;
\fill[blue!34.8, opacity=0.7] (2.7420, 0.7440, 3.1079) -- (2.7880, 0.7440, 3.1017) -- (2.7880, 0.7980, 3.1058) -- (2.7420, 0.7980, 3.1120) -- cycle;
\fill[blue!45.0, opacity=0.7] (2.7420, 0.7980, 3.1120) -- (2.7880, 0.7980, 3.1058) -- (2.7880, 0.8520, 3.1096) -- (2.7420, 0.8520, 3.1159) -- cycle;
\fill[blue!53.9, opacity=0.7] (2.7420, 0.8520, 3.1159) -- (2.7880, 0.8520, 3.1096) -- (2.7880, 0.9060, 3.1132) -- (2.7420, 0.9060, 3.1194) -- cycle;
\fill[blue!59.7, opacity=0.7] (2.7420, 0.9060, 3.1194) -- (2.7880, 0.9060, 3.1132) -- (2.7880, 0.9600, 3.1165) -- (2.7420, 0.9600, 3.1227) -- cycle;
\fill[blue!62.6, opacity=0.7] (2.7420, 0.9600, 3.1227) -- (2.7880, 0.9600, 3.1165) -- (2.7880, 1.0140, 3.1195) -- (2.7420, 1.0140, 3.1257) -- cycle;
\fill[blue!63.5, opacity=0.7] (2.7420, 1.0140, 3.1257) -- (2.7880, 1.0140, 3.1195) -- (2.7880, 1.0680, 3.1222) -- (2.7420, 1.0680, 3.1284) -- cycle;
\fill[blue!63.4, opacity=0.7] (2.7420, 1.0680, 3.1284) -- (2.7880, 1.0680, 3.1222) -- (2.7880, 1.1220, 3.1246) -- (2.7420, 1.1220, 3.1308) -- cycle;
\fill[blue!63.0, opacity=0.7] (2.7420, 1.1220, 3.1308) -- (2.7880, 1.1220, 3.1246) -- (2.7880, 1.1760, 3.1267) -- (2.7420, 1.1760, 3.1329) -- cycle;
\fill[blue!62.6, opacity=0.7] (2.7420, 1.1760, 3.1329) -- (2.7880, 1.1760, 3.1267) -- (2.7880, 1.2300, 3.1285) -- (2.7420, 1.2300, 3.1347) -- cycle;
\fill[blue!62.2, opacity=0.7] (2.7420, 1.2300, 3.1347) -- (2.7880, 1.2300, 3.1285) -- (2.7880, 1.2840, 3.1299) -- (2.7420, 1.2840, 3.1361) -- cycle;
\fill[blue!62.0, opacity=0.7] (2.7420, 1.2840, 3.1361) -- (2.7880, 1.2840, 3.1299) -- (2.7880, 1.3380, 3.1311) -- (2.7420, 1.3380, 3.1373) -- cycle;
\fill[blue!61.8, opacity=0.7] (2.7420, 1.3380, 3.1373) -- (2.7880, 1.3380, 3.1311) -- (2.7880, 1.3920, 3.1319) -- (2.7420, 1.3920, 3.1381) -- cycle;
\fill[blue!61.7, opacity=0.7] (2.7420, 1.3920, 3.1381) -- (2.7880, 1.3920, 3.1319) -- (2.7880, 1.4460, 3.1324) -- (2.7420, 1.4460, 3.1386) -- cycle;
\fill[blue!61.7, opacity=0.7] (2.7420, 1.4460, 3.1386) -- (2.7880, 1.4460, 3.1324) -- (2.7880, 1.5000, 3.1325) -- (2.7420, 1.5000, 3.1388) -- cycle;
\fill[blue!62.0, opacity=0.7] (2.7420, 1.5000, 3.1388) -- (2.7880, 1.5000, 3.1325) -- (2.7880, 1.5540, 3.1324) -- (2.7420, 1.5540, 3.1386) -- cycle;
\fill[blue!62.4, opacity=0.7] (2.7420, 1.5540, 3.1386) -- (2.7880, 1.5540, 3.1324) -- (2.7880, 1.6080, 3.1319) -- (2.7420, 1.6080, 3.1381) -- cycle;
\fill[blue!63.0, opacity=0.7] (2.7420, 1.6080, 3.1381) -- (2.7880, 1.6080, 3.1319) -- (2.7880, 1.6620, 3.1311) -- (2.7420, 1.6620, 3.1373) -- cycle;
\fill[blue!63.5, opacity=0.7] (2.7420, 1.6620, 3.1373) -- (2.7880, 1.6620, 3.1311) -- (2.7880, 1.7160, 3.1299) -- (2.7420, 1.7160, 3.1361) -- cycle;
\fill[blue!63.3, opacity=0.7] (2.7420, 1.7160, 3.1361) -- (2.7880, 1.7160, 3.1299) -- (2.7880, 1.7700, 3.1285) -- (2.7420, 1.7700, 3.1347) -- cycle;
\fill[blue!61.2, opacity=0.7] (2.7420, 1.7700, 3.1347) -- (2.7880, 1.7700, 3.1285) -- (2.7880, 1.8240, 3.1267) -- (2.7420, 1.8240, 3.1329) -- cycle;
\fill[blue!55.9, opacity=0.7] (2.7420, 1.8240, 3.1329) -- (2.7880, 1.8240, 3.1267) -- (2.7880, 1.8780, 3.1246) -- (2.7420, 1.8780, 3.1308) -- cycle;
\fill[blue!47.2, opacity=0.7] (2.7420, 1.8780, 3.1308) -- (2.7880, 1.8780, 3.1246) -- (2.7880, 1.9320, 3.1222) -- (2.7420, 1.9320, 3.1284) -- cycle;
\fill[blue!36.8, opacity=0.7] (2.7420, 1.9320, 3.1284) -- (2.7880, 1.9320, 3.1222) -- (2.7880, 1.9860, 3.1195) -- (2.7420, 1.9860, 3.1257) -- cycle;
\fill[blue!27.8, opacity=0.7] (2.7420, 1.9860, 3.1257) -- (2.7880, 1.9860, 3.1195) -- (2.7880, 2.0400, 3.1165) -- (2.7420, 2.0400, 3.1227) -- cycle;
\fill[blue!22.0, opacity=0.7] (2.7420, 2.0400, 3.1227) -- (2.7880, 2.0400, 3.1165) -- (2.7880, 2.0940, 3.1132) -- (2.7420, 2.0940, 3.1194) -- cycle;
\fill[blue!19.0, opacity=0.7] (2.7420, 2.0940, 3.1194) -- (2.7880, 2.0940, 3.1132) -- (2.7880, 2.1480, 3.1096) -- (2.7420, 2.1480, 3.1159) -- cycle;
\fill[blue!17.9, opacity=0.7] (2.7420, 2.1480, 3.1159) -- (2.7880, 2.1480, 3.1096) -- (2.7880, 2.2020, 3.1058) -- (2.7420, 2.2020, 3.1120) -- cycle;
\fill[blue!17.9, opacity=0.7] (2.7420, 2.2020, 3.1120) -- (2.7880, 2.2020, 3.1058) -- (2.7880, 2.2560, 3.1017) -- (2.7420, 2.2560, 3.1079) -- cycle;
\fill[blue!19.4, opacity=0.7] (2.7420, 2.2560, 3.1079) -- (2.7880, 2.2560, 3.1017) -- (2.7880, 2.3100, 3.0974) -- (2.7420, 2.3100, 3.1036) -- cycle;
\fill[blue!23.7, opacity=0.7] (2.7420, 2.3100, 3.1036) -- (2.7880, 2.3100, 3.0974) -- (2.7880, 2.3640, 3.0928) -- (2.7420, 2.3640, 3.0991) -- cycle;
\fill[blue!32.7, opacity=0.7] (2.7420, 2.3640, 3.0991) -- (2.7880, 2.3640, 3.0928) -- (2.7880, 2.4180, 3.0881) -- (2.7420, 2.4180, 3.0943) -- cycle;
\fill[blue!45.2, opacity=0.7] (2.7420, 2.4180, 3.0943) -- (2.7880, 2.4180, 3.0881) -- (2.7880, 2.4720, 3.0831) -- (2.7420, 2.4720, 3.0893) -- cycle;
\fill[blue!55.1, opacity=0.7] (2.7420, 2.4720, 3.0893) -- (2.7880, 2.4720, 3.0831) -- (2.7880, 2.5260, 3.0779) -- (2.7420, 2.5260, 3.0841) -- cycle;
\fill[blue!58.7, opacity=0.7] (2.7420, 2.5260, 3.0841) -- (2.7880, 2.5260, 3.0779) -- (2.7880, 2.5800, 3.0725) -- (2.7420, 2.5800, 3.0788) -- cycle;
\fill[blue!56.5, opacity=0.7] (2.7420, 2.5800, 3.0788) -- (2.7880, 2.5800, 3.0725) -- (2.7880, 2.6340, 3.0670) -- (2.7420, 2.6340, 3.0733) -- cycle;
\fill[blue!46.3, opacity=0.7] (2.7420, 2.6340, 3.0733) -- (2.7880, 2.6340, 3.0670) -- (2.7880, 2.6880, 3.0614) -- (2.7420, 2.6880, 3.0676) -- cycle;
\fill[blue!30.0, opacity=0.7] (2.7420, 2.6880, 3.0676) -- (2.7880, 2.6880, 3.0614) -- (2.7880, 2.7420, 3.0555) -- (2.7420, 2.7420, 3.0618) -- cycle;
\fill[blue!19.1, opacity=0.7] (2.7420, 2.7420, 3.0618) -- (2.7880, 2.7420, 3.0555) -- (2.7880, 2.7960, 3.0496) -- (2.7420, 2.7960, 3.0559) -- cycle;
\fill[blue!15.9, opacity=0.7] (2.7420, 2.7960, 3.0559) -- (2.7880, 2.7960, 3.0496) -- (2.7880, 2.8500, 3.0436) -- (2.7420, 2.8500, 3.0498) -- cycle;
\fill[blue!15.3, opacity=0.7] (2.7420, 2.8500, 3.0498) -- (2.7880, 2.8500, 3.0436) -- (2.7880, 2.9040, 3.0375) -- (2.7420, 2.9040, 3.0437) -- cycle;
\fill[blue!15.2, opacity=0.7] (2.7420, 2.9040, 3.0437) -- (2.7880, 2.9040, 3.0375) -- (2.7880, 2.9580, 3.0313) -- (2.7420, 2.9580, 3.0375) -- cycle;
\fill[blue!15.6, opacity=0.7] (2.7420, 2.9580, 3.0375) -- (2.7880, 2.9580, 3.0313) -- (2.7880, 3.0120, 3.0251) -- (2.7420, 3.0120, 3.0313) -- cycle;
\fill[blue!17.5, opacity=0.7] (2.7420, 3.0120, 3.0313) -- (2.7880, 3.0120, 3.0251) -- (2.7880, 3.0660, 3.0188) -- (2.7420, 3.0660, 3.0251) -- cycle;
\fill[blue!23.6, opacity=0.7] (2.7420, 3.0660, 3.0251) -- (2.7880, 3.0660, 3.0188) -- (2.7880, 3.1200, 3.0125) -- (2.7420, 3.1200, 3.0188) -- cycle;
\fill[blue!15.2, opacity=0.7] (2.7880, -0.1200, 3.0125) -- (2.8340, -0.1200, 3.0063) -- (2.8340, -0.0660, 3.0126) -- (2.7880, -0.0660, 3.0188) -- cycle;
\fill[blue!15.9, opacity=0.7] (2.7880, -0.0660, 3.0188) -- (2.8340, -0.0660, 3.0126) -- (2.8340, -0.0120, 3.0188) -- (2.7880, -0.0120, 3.0251) -- cycle;
\fill[blue!19.5, opacity=0.7] (2.7880, -0.0120, 3.0251) -- (2.8340, -0.0120, 3.0188) -- (2.8340, 0.0420, 3.0251) -- (2.7880, 0.0420, 3.0313) -- cycle;
\fill[blue!29.0, opacity=0.7] (2.7880, 0.0420, 3.0313) -- (2.8340, 0.0420, 3.0251) -- (2.8340, 0.0960, 3.0312) -- (2.7880, 0.0960, 3.0375) -- cycle;
\fill[blue!40.8, opacity=0.7] (2.7880, 0.0960, 3.0375) -- (2.8340, 0.0960, 3.0312) -- (2.8340, 0.1500, 3.0373) -- (2.7880, 0.1500, 3.0436) -- cycle;
\fill[blue!47.5, opacity=0.7] (2.7880, 0.1500, 3.0436) -- (2.8340, 0.1500, 3.0373) -- (2.8340, 0.2040, 3.0434) -- (2.7880, 0.2040, 3.0496) -- cycle;
\fill[blue!47.4, opacity=0.7] (2.7880, 0.2040, 3.0496) -- (2.8340, 0.2040, 3.0434) -- (2.8340, 0.2580, 3.0493) -- (2.7880, 0.2580, 3.0555) -- cycle;
\fill[blue!41.3, opacity=0.7] (2.7880, 0.2580, 3.0555) -- (2.8340, 0.2580, 3.0493) -- (2.8340, 0.3120, 3.0551) -- (2.7880, 0.3120, 3.0614) -- cycle;
\fill[blue!32.1, opacity=0.7] (2.7880, 0.3120, 3.0614) -- (2.8340, 0.3120, 3.0551) -- (2.8340, 0.3660, 3.0608) -- (2.7880, 0.3660, 3.0670) -- cycle;
\fill[blue!24.0, opacity=0.7] (2.7880, 0.3660, 3.0670) -- (2.8340, 0.3660, 3.0608) -- (2.8340, 0.4200, 3.0663) -- (2.7880, 0.4200, 3.0725) -- cycle;
\fill[blue!19.2, opacity=0.7] (2.7880, 0.4200, 3.0725) -- (2.8340, 0.4200, 3.0663) -- (2.8340, 0.4740, 3.0716) -- (2.7880, 0.4740, 3.0779) -- cycle;
\fill[blue!17.2, opacity=0.7] (2.7880, 0.4740, 3.0779) -- (2.8340, 0.4740, 3.0716) -- (2.8340, 0.5280, 3.0768) -- (2.7880, 0.5280, 3.0831) -- cycle;
\fill[blue!16.4, opacity=0.7] (2.7880, 0.5280, 3.0831) -- (2.8340, 0.5280, 3.0768) -- (2.8340, 0.5820, 3.0818) -- (2.7880, 0.5820, 3.0881) -- cycle;
\fill[blue!16.3, opacity=0.7] (2.7880, 0.5820, 3.0881) -- (2.8340, 0.5820, 3.0818) -- (2.8340, 0.6360, 3.0866) -- (2.7880, 0.6360, 3.0928) -- cycle;
\fill[blue!16.6, opacity=0.7] (2.7880, 0.6360, 3.0928) -- (2.8340, 0.6360, 3.0866) -- (2.8340, 0.6900, 3.0911) -- (2.7880, 0.6900, 3.0974) -- cycle;
\fill[blue!17.5, opacity=0.7] (2.7880, 0.6900, 3.0974) -- (2.8340, 0.6900, 3.0911) -- (2.8340, 0.7440, 3.0955) -- (2.7880, 0.7440, 3.1017) -- cycle;
\fill[blue!19.5, opacity=0.7] (2.7880, 0.7440, 3.1017) -- (2.8340, 0.7440, 3.0955) -- (2.8340, 0.7980, 3.0995) -- (2.7880, 0.7980, 3.1058) -- cycle;
\fill[blue!23.0, opacity=0.7] (2.7880, 0.7980, 3.1058) -- (2.8340, 0.7980, 3.0995) -- (2.8340, 0.8520, 3.1034) -- (2.7880, 0.8520, 3.1096) -- cycle;
\fill[blue!28.4, opacity=0.7] (2.7880, 0.8520, 3.1096) -- (2.8340, 0.8520, 3.1034) -- (2.8340, 0.9060, 3.1069) -- (2.7880, 0.9060, 3.1132) -- cycle;
\fill[blue!35.3, opacity=0.7] (2.7880, 0.9060, 3.1132) -- (2.8340, 0.9060, 3.1069) -- (2.8340, 0.9600, 3.1102) -- (2.7880, 0.9600, 3.1165) -- cycle;
\fill[blue!42.6, opacity=0.7] (2.7880, 0.9600, 3.1165) -- (2.8340, 0.9600, 3.1102) -- (2.8340, 1.0140, 3.1132) -- (2.7880, 1.0140, 3.1195) -- cycle;
\fill[blue!49.2, opacity=0.7] (2.7880, 1.0140, 3.1195) -- (2.8340, 1.0140, 3.1132) -- (2.8340, 1.0680, 3.1159) -- (2.7880, 1.0680, 3.1222) -- cycle;
\fill[blue!54.2, opacity=0.7] (2.7880, 1.0680, 3.1222) -- (2.8340, 1.0680, 3.1159) -- (2.8340, 1.1220, 3.1183) -- (2.7880, 1.1220, 3.1246) -- cycle;
\fill[blue!57.7, opacity=0.7] (2.7880, 1.1220, 3.1246) -- (2.8340, 1.1220, 3.1183) -- (2.8340, 1.1760, 3.1204) -- (2.7880, 1.1760, 3.1267) -- cycle;
\fill[blue!59.9, opacity=0.7] (2.7880, 1.1760, 3.1267) -- (2.8340, 1.1760, 3.1204) -- (2.8340, 1.2300, 3.1222) -- (2.7880, 1.2300, 3.1285) -- cycle;
\fill[blue!61.2, opacity=0.7] (2.7880, 1.2300, 3.1285) -- (2.8340, 1.2300, 3.1222) -- (2.8340, 1.2840, 3.1237) -- (2.7880, 1.2840, 3.1299) -- cycle;
\fill[blue!61.9, opacity=0.7] (2.7880, 1.2840, 3.1299) -- (2.8340, 1.2840, 3.1237) -- (2.8340, 1.3380, 3.1248) -- (2.7880, 1.3380, 3.1311) -- cycle;
\fill[blue!62.1, opacity=0.7] (2.7880, 1.3380, 3.1311) -- (2.8340, 1.3380, 3.1248) -- (2.8340, 1.3920, 3.1256) -- (2.7880, 1.3920, 3.1319) -- cycle;
\fill[blue!62.0, opacity=0.7] (2.7880, 1.3920, 3.1319) -- (2.8340, 1.3920, 3.1256) -- (2.8340, 1.4460, 3.1261) -- (2.7880, 1.4460, 3.1324) -- cycle;
\fill[blue!61.6, opacity=0.7] (2.7880, 1.4460, 3.1324) -- (2.8340, 1.4460, 3.1261) -- (2.8340, 1.5000, 3.1263) -- (2.7880, 1.5000, 3.1325) -- cycle;
\fill[blue!60.5, opacity=0.7] (2.7880, 1.5000, 3.1325) -- (2.8340, 1.5000, 3.1263) -- (2.8340, 1.5540, 3.1261) -- (2.7880, 1.5540, 3.1324) -- cycle;
\fill[blue!58.7, opacity=0.7] (2.7880, 1.5540, 3.1324) -- (2.8340, 1.5540, 3.1261) -- (2.8340, 1.6080, 3.1256) -- (2.7880, 1.6080, 3.1319) -- cycle;
\fill[blue!55.6, opacity=0.7] (2.7880, 1.6080, 3.1319) -- (2.8340, 1.6080, 3.1256) -- (2.8340, 1.6620, 3.1248) -- (2.7880, 1.6620, 3.1311) -- cycle;
\fill[blue!50.8, opacity=0.7] (2.7880, 1.6620, 3.1311) -- (2.8340, 1.6620, 3.1248) -- (2.8340, 1.7160, 3.1237) -- (2.7880, 1.7160, 3.1299) -- cycle;
\fill[blue!44.5, opacity=0.7] (2.7880, 1.7160, 3.1299) -- (2.8340, 1.7160, 3.1237) -- (2.8340, 1.7700, 3.1222) -- (2.7880, 1.7700, 3.1285) -- cycle;
\fill[blue!37.2, opacity=0.7] (2.7880, 1.7700, 3.1285) -- (2.8340, 1.7700, 3.1222) -- (2.8340, 1.8240, 3.1204) -- (2.7880, 1.8240, 3.1267) -- cycle;
\fill[blue!30.2, opacity=0.7] (2.7880, 1.8240, 3.1267) -- (2.8340, 1.8240, 3.1204) -- (2.8340, 1.8780, 3.1183) -- (2.7880, 1.8780, 3.1246) -- cycle;
\fill[blue!24.5, opacity=0.7] (2.7880, 1.8780, 3.1246) -- (2.8340, 1.8780, 3.1183) -- (2.8340, 1.9320, 3.1159) -- (2.7880, 1.9320, 3.1222) -- cycle;
\fill[blue!20.7, opacity=0.7] (2.7880, 1.9320, 3.1222) -- (2.8340, 1.9320, 3.1159) -- (2.8340, 1.9860, 3.1132) -- (2.7880, 1.9860, 3.1195) -- cycle;
\fill[blue!18.7, opacity=0.7] (2.7880, 1.9860, 3.1195) -- (2.8340, 1.9860, 3.1132) -- (2.8340, 2.0400, 3.1102) -- (2.7880, 2.0400, 3.1165) -- cycle;
\fill[blue!17.7, opacity=0.7] (2.7880, 2.0400, 3.1165) -- (2.8340, 2.0400, 3.1102) -- (2.8340, 2.0940, 3.1069) -- (2.7880, 2.0940, 3.1132) -- cycle;
\fill[blue!17.7, opacity=0.7] (2.7880, 2.0940, 3.1132) -- (2.8340, 2.0940, 3.1069) -- (2.8340, 2.1480, 3.1034) -- (2.7880, 2.1480, 3.1096) -- cycle;
\fill[blue!18.6, opacity=0.7] (2.7880, 2.1480, 3.1096) -- (2.8340, 2.1480, 3.1034) -- (2.8340, 2.2020, 3.0995) -- (2.7880, 2.2020, 3.1058) -- cycle;
\fill[blue!21.3, opacity=0.7] (2.7880, 2.2020, 3.1058) -- (2.8340, 2.2020, 3.0995) -- (2.8340, 2.2560, 3.0955) -- (2.7880, 2.2560, 3.1017) -- cycle;
\fill[blue!27.3, opacity=0.7] (2.7880, 2.2560, 3.1017) -- (2.8340, 2.2560, 3.0955) -- (2.8340, 2.3100, 3.0911) -- (2.7880, 2.3100, 3.0974) -- cycle;
\fill[blue!37.6, opacity=0.7] (2.7880, 2.3100, 3.0974) -- (2.8340, 2.3100, 3.0911) -- (2.8340, 2.3640, 3.0866) -- (2.7880, 2.3640, 3.0928) -- cycle;
\fill[blue!49.0, opacity=0.7] (2.7880, 2.3640, 3.0928) -- (2.8340, 2.3640, 3.0866) -- (2.8340, 2.4180, 3.0818) -- (2.7880, 2.4180, 3.0881) -- cycle;
\fill[blue!56.5, opacity=0.7] (2.7880, 2.4180, 3.0881) -- (2.8340, 2.4180, 3.0818) -- (2.8340, 2.4720, 3.0768) -- (2.7880, 2.4720, 3.0831) -- cycle;
\fill[blue!58.5, opacity=0.7] (2.7880, 2.4720, 3.0831) -- (2.8340, 2.4720, 3.0768) -- (2.8340, 2.5260, 3.0716) -- (2.7880, 2.5260, 3.0779) -- cycle;
\fill[blue!54.9, opacity=0.7] (2.7880, 2.5260, 3.0779) -- (2.8340, 2.5260, 3.0716) -- (2.8340, 2.5800, 3.0663) -- (2.7880, 2.5800, 3.0725) -- cycle;
\fill[blue!43.8, opacity=0.7] (2.7880, 2.5800, 3.0725) -- (2.8340, 2.5800, 3.0663) -- (2.8340, 2.6340, 3.0608) -- (2.7880, 2.6340, 3.0670) -- cycle;
\fill[blue!28.3, opacity=0.7] (2.7880, 2.6340, 3.0670) -- (2.8340, 2.6340, 3.0608) -- (2.8340, 2.6880, 3.0551) -- (2.7880, 2.6880, 3.0614) -- cycle;
\fill[blue!18.7, opacity=0.7] (2.7880, 2.6880, 3.0614) -- (2.8340, 2.6880, 3.0551) -- (2.8340, 2.7420, 3.0493) -- (2.7880, 2.7420, 3.0555) -- cycle;
\fill[blue!15.8, opacity=0.7] (2.7880, 2.7420, 3.0555) -- (2.8340, 2.7420, 3.0493) -- (2.8340, 2.7960, 3.0434) -- (2.7880, 2.7960, 3.0496) -- cycle;
\fill[blue!15.3, opacity=0.7] (2.7880, 2.7960, 3.0496) -- (2.8340, 2.7960, 3.0434) -- (2.8340, 2.8500, 3.0373) -- (2.7880, 2.8500, 3.0436) -- cycle;
\fill[blue!15.2, opacity=0.7] (2.7880, 2.8500, 3.0436) -- (2.8340, 2.8500, 3.0373) -- (2.8340, 2.9040, 3.0312) -- (2.7880, 2.9040, 3.0375) -- cycle;
\fill[blue!15.5, opacity=0.7] (2.7880, 2.9040, 3.0375) -- (2.8340, 2.9040, 3.0312) -- (2.8340, 2.9580, 3.0251) -- (2.7880, 2.9580, 3.0313) -- cycle;
\fill[blue!17.0, opacity=0.7] (2.7880, 2.9580, 3.0313) -- (2.8340, 2.9580, 3.0251) -- (2.8340, 3.0120, 3.0188) -- (2.7880, 3.0120, 3.0251) -- cycle;
\fill[blue!22.2, opacity=0.7] (2.7880, 3.0120, 3.0251) -- (2.8340, 3.0120, 3.0188) -- (2.8340, 3.0660, 3.0126) -- (2.7880, 3.0660, 3.0188) -- cycle;
\fill[blue!30.4, opacity=0.7] (2.7880, 3.0660, 3.0188) -- (2.8340, 3.0660, 3.0126) -- (2.8340, 3.1200, 3.0063) -- (2.7880, 3.1200, 3.0125) -- cycle;
\fill[blue!15.0, opacity=0.7] (2.8340, -0.1200, 3.0063) -- (2.8800, -0.1200, 3.0000) -- (2.8800, -0.0660, 3.0063) -- (2.8340, -0.0660, 3.0126) -- cycle;
\fill[blue!15.1, opacity=0.7] (2.8340, -0.0660, 3.0126) -- (2.8800, -0.0660, 3.0063) -- (2.8800, -0.0120, 3.0125) -- (2.8340, -0.0120, 3.0188) -- cycle;
\fill[blue!15.7, opacity=0.7] (2.8340, -0.0120, 3.0188) -- (2.8800, -0.0120, 3.0125) -- (2.8800, 0.0420, 3.0188) -- (2.8340, 0.0420, 3.0251) -- cycle;
\fill[blue!18.3, opacity=0.7] (2.8340, 0.0420, 3.0251) -- (2.8800, 0.0420, 3.0188) -- (2.8800, 0.0960, 3.0249) -- (2.8340, 0.0960, 3.0312) -- cycle;
\fill[blue!26.1, opacity=0.7] (2.8340, 0.0960, 3.0312) -- (2.8800, 0.0960, 3.0249) -- (2.8800, 0.1500, 3.0311) -- (2.8340, 0.1500, 3.0373) -- cycle;
\fill[blue!37.5, opacity=0.7] (2.8340, 0.1500, 3.0373) -- (2.8800, 0.1500, 3.0311) -- (2.8800, 0.2040, 3.0371) -- (2.8340, 0.2040, 3.0434) -- cycle;
\fill[blue!46.1, opacity=0.7] (2.8340, 0.2040, 3.0434) -- (2.8800, 0.2040, 3.0371) -- (2.8800, 0.2580, 3.0430) -- (2.8340, 0.2580, 3.0493) -- cycle;
\fill[blue!48.7, opacity=0.7] (2.8340, 0.2580, 3.0493) -- (2.8800, 0.2580, 3.0430) -- (2.8800, 0.3120, 3.0488) -- (2.8340, 0.3120, 3.0551) -- cycle;
\fill[blue!45.6, opacity=0.7] (2.8340, 0.3120, 3.0551) -- (2.8800, 0.3120, 3.0488) -- (2.8800, 0.3660, 3.0545) -- (2.8340, 0.3660, 3.0608) -- cycle;
\fill[blue!38.2, opacity=0.7] (2.8340, 0.3660, 3.0608) -- (2.8800, 0.3660, 3.0545) -- (2.8800, 0.4200, 3.0600) -- (2.8340, 0.4200, 3.0663) -- cycle;
\fill[blue!29.6, opacity=0.7] (2.8340, 0.4200, 3.0663) -- (2.8800, 0.4200, 3.0600) -- (2.8800, 0.4740, 3.0654) -- (2.8340, 0.4740, 3.0716) -- cycle;
\fill[blue!22.9, opacity=0.7] (2.8340, 0.4740, 3.0716) -- (2.8800, 0.4740, 3.0654) -- (2.8800, 0.5280, 3.0705) -- (2.8340, 0.5280, 3.0768) -- cycle;
\fill[blue!19.1, opacity=0.7] (2.8340, 0.5280, 3.0768) -- (2.8800, 0.5280, 3.0705) -- (2.8800, 0.5820, 3.0755) -- (2.8340, 0.5820, 3.0818) -- cycle;
\fill[blue!17.3, opacity=0.7] (2.8340, 0.5820, 3.0818) -- (2.8800, 0.5820, 3.0755) -- (2.8800, 0.6360, 3.0803) -- (2.8340, 0.6360, 3.0866) -- cycle;
\fill[blue!16.6, opacity=0.7] (2.8340, 0.6360, 3.0866) -- (2.8800, 0.6360, 3.0803) -- (2.8800, 0.6900, 3.0849) -- (2.8340, 0.6900, 3.0911) -- cycle;
\fill[blue!16.4, opacity=0.7] (2.8340, 0.6900, 3.0911) -- (2.8800, 0.6900, 3.0849) -- (2.8800, 0.7440, 3.0892) -- (2.8340, 0.7440, 3.0955) -- cycle;
\fill[blue!16.5, opacity=0.7] (2.8340, 0.7440, 3.0955) -- (2.8800, 0.7440, 3.0892) -- (2.8800, 0.7980, 3.0933) -- (2.8340, 0.7980, 3.0995) -- cycle;
\fill[blue!17.0, opacity=0.7] (2.8340, 0.7980, 3.0995) -- (2.8800, 0.7980, 3.0933) -- (2.8800, 0.8520, 3.0971) -- (2.8340, 0.8520, 3.1034) -- cycle;
\fill[blue!17.9, opacity=0.7] (2.8340, 0.8520, 3.1034) -- (2.8800, 0.8520, 3.0971) -- (2.8800, 0.9060, 3.1006) -- (2.8340, 0.9060, 3.1069) -- cycle;
\fill[blue!19.4, opacity=0.7] (2.8340, 0.9060, 3.1069) -- (2.8800, 0.9060, 3.1006) -- (2.8800, 0.9600, 3.1039) -- (2.8340, 0.9600, 3.1102) -- cycle;
\fill[blue!21.7, opacity=0.7] (2.8340, 0.9600, 3.1102) -- (2.8800, 0.9600, 3.1039) -- (2.8800, 1.0140, 3.1069) -- (2.8340, 1.0140, 3.1132) -- cycle;
\fill[blue!24.7, opacity=0.7] (2.8340, 1.0140, 3.1132) -- (2.8800, 1.0140, 3.1069) -- (2.8800, 1.0680, 3.1096) -- (2.8340, 1.0680, 3.1159) -- cycle;
\fill[blue!28.0, opacity=0.7] (2.8340, 1.0680, 3.1159) -- (2.8800, 1.0680, 3.1096) -- (2.8800, 1.1220, 3.1120) -- (2.8340, 1.1220, 3.1183) -- cycle;
\fill[blue!31.4, opacity=0.7] (2.8340, 1.1220, 3.1183) -- (2.8800, 1.1220, 3.1120) -- (2.8800, 1.1760, 3.1141) -- (2.8340, 1.1760, 3.1204) -- cycle;
\fill[blue!34.5, opacity=0.7] (2.8340, 1.1760, 3.1204) -- (2.8800, 1.1760, 3.1141) -- (2.8800, 1.2300, 3.1159) -- (2.8340, 1.2300, 3.1222) -- cycle;
\fill[blue!36.9, opacity=0.7] (2.8340, 1.2300, 3.1222) -- (2.8800, 1.2300, 3.1159) -- (2.8800, 1.2840, 3.1174) -- (2.8340, 1.2840, 3.1237) -- cycle;
\fill[blue!38.5, opacity=0.7] (2.8340, 1.2840, 3.1237) -- (2.8800, 1.2840, 3.1174) -- (2.8800, 1.3380, 3.1185) -- (2.8340, 1.3380, 3.1248) -- cycle;
\fill[blue!39.1, opacity=0.7] (2.8340, 1.3380, 3.1248) -- (2.8800, 1.3380, 3.1185) -- (2.8800, 1.3920, 3.1193) -- (2.8340, 1.3920, 3.1256) -- cycle;
\fill[blue!38.8, opacity=0.7] (2.8340, 1.3920, 3.1256) -- (2.8800, 1.3920, 3.1193) -- (2.8800, 1.4460, 3.1198) -- (2.8340, 1.4460, 3.1261) -- cycle;
\fill[blue!37.5, opacity=0.7] (2.8340, 1.4460, 3.1261) -- (2.8800, 1.4460, 3.1198) -- (2.8800, 1.5000, 3.1200) -- (2.8340, 1.5000, 3.1263) -- cycle;
\fill[blue!35.3, opacity=0.7] (2.8340, 1.5000, 3.1263) -- (2.8800, 1.5000, 3.1200) -- (2.8800, 1.5540, 3.1198) -- (2.8340, 1.5540, 3.1261) -- cycle;
\fill[blue!32.4, opacity=0.7] (2.8340, 1.5540, 3.1261) -- (2.8800, 1.5540, 3.1198) -- (2.8800, 1.6080, 3.1193) -- (2.8340, 1.6080, 3.1256) -- cycle;
\fill[blue!29.1, opacity=0.7] (2.8340, 1.6080, 3.1256) -- (2.8800, 1.6080, 3.1193) -- (2.8800, 1.6620, 3.1185) -- (2.8340, 1.6620, 3.1248) -- cycle;
\fill[blue!25.8, opacity=0.7] (2.8340, 1.6620, 3.1248) -- (2.8800, 1.6620, 3.1185) -- (2.8800, 1.7160, 3.1174) -- (2.8340, 1.7160, 3.1237) -- cycle;
\fill[blue!22.8, opacity=0.7] (2.8340, 1.7160, 3.1237) -- (2.8800, 1.7160, 3.1174) -- (2.8800, 1.7700, 3.1159) -- (2.8340, 1.7700, 3.1222) -- cycle;
\fill[blue!20.5, opacity=0.7] (2.8340, 1.7700, 3.1222) -- (2.8800, 1.7700, 3.1159) -- (2.8800, 1.8240, 3.1141) -- (2.8340, 1.8240, 3.1204) -- cycle;
\fill[blue!18.8, opacity=0.7] (2.8340, 1.8240, 3.1204) -- (2.8800, 1.8240, 3.1141) -- (2.8800, 1.8780, 3.1120) -- (2.8340, 1.8780, 3.1183) -- cycle;
\fill[blue!17.9, opacity=0.7] (2.8340, 1.8780, 3.1183) -- (2.8800, 1.8780, 3.1120) -- (2.8800, 1.9320, 3.1096) -- (2.8340, 1.9320, 3.1159) -- cycle;
\fill[blue!17.5, opacity=0.7] (2.8340, 1.9320, 3.1159) -- (2.8800, 1.9320, 3.1096) -- (2.8800, 1.9860, 3.1069) -- (2.8340, 1.9860, 3.1132) -- cycle;
\fill[blue!17.6, opacity=0.7] (2.8340, 1.9860, 3.1132) -- (2.8800, 1.9860, 3.1069) -- (2.8800, 2.0400, 3.1039) -- (2.8340, 2.0400, 3.1102) -- cycle;
\fill[blue!18.4, opacity=0.7] (2.8340, 2.0400, 3.1102) -- (2.8800, 2.0400, 3.1039) -- (2.8800, 2.0940, 3.1006) -- (2.8340, 2.0940, 3.1069) -- cycle;
\fill[blue!20.6, opacity=0.7] (2.8340, 2.0940, 3.1069) -- (2.8800, 2.0940, 3.1006) -- (2.8800, 2.1480, 3.0971) -- (2.8340, 2.1480, 3.1034) -- cycle;
\fill[blue!25.2, opacity=0.7] (2.8340, 2.1480, 3.1034) -- (2.8800, 2.1480, 3.0971) -- (2.8800, 2.2020, 3.0933) -- (2.8340, 2.2020, 3.0995) -- cycle;
\fill[blue!33.3, opacity=0.7] (2.8340, 2.2020, 3.0995) -- (2.8800, 2.2020, 3.0933) -- (2.8800, 2.2560, 3.0892) -- (2.8340, 2.2560, 3.0955) -- cycle;
\fill[blue!43.8, opacity=0.7] (2.8340, 2.2560, 3.0955) -- (2.8800, 2.2560, 3.0892) -- (2.8800, 2.3100, 3.0849) -- (2.8340, 2.3100, 3.0911) -- cycle;
\fill[blue!53.0, opacity=0.7] (2.8340, 2.3100, 3.0911) -- (2.8800, 2.3100, 3.0849) -- (2.8800, 2.3640, 3.0803) -- (2.8340, 2.3640, 3.0866) -- cycle;
\fill[blue!57.6, opacity=0.7] (2.8340, 2.3640, 3.0866) -- (2.8800, 2.3640, 3.0803) -- (2.8800, 2.4180, 3.0755) -- (2.8340, 2.4180, 3.0818) -- cycle;
\fill[blue!57.6, opacity=0.7] (2.8340, 2.4180, 3.0818) -- (2.8800, 2.4180, 3.0755) -- (2.8800, 2.4720, 3.0705) -- (2.8340, 2.4720, 3.0768) -- cycle;
\fill[blue!52.2, opacity=0.7] (2.8340, 2.4720, 3.0768) -- (2.8800, 2.4720, 3.0705) -- (2.8800, 2.5260, 3.0654) -- (2.8340, 2.5260, 3.0716) -- cycle;
\fill[blue!39.8, opacity=0.7] (2.8340, 2.5260, 3.0716) -- (2.8800, 2.5260, 3.0654) -- (2.8800, 2.5800, 3.0600) -- (2.8340, 2.5800, 3.0663) -- cycle;
\fill[blue!25.7, opacity=0.7] (2.8340, 2.5800, 3.0663) -- (2.8800, 2.5800, 3.0600) -- (2.8800, 2.6340, 3.0545) -- (2.8340, 2.6340, 3.0608) -- cycle;
\fill[blue!18.0, opacity=0.7] (2.8340, 2.6340, 3.0608) -- (2.8800, 2.6340, 3.0545) -- (2.8800, 2.6880, 3.0488) -- (2.8340, 2.6880, 3.0551) -- cycle;
\fill[blue!15.7, opacity=0.7] (2.8340, 2.6880, 3.0551) -- (2.8800, 2.6880, 3.0488) -- (2.8800, 2.7420, 3.0430) -- (2.8340, 2.7420, 3.0493) -- cycle;
\fill[blue!15.3, opacity=0.7] (2.8340, 2.7420, 3.0493) -- (2.8800, 2.7420, 3.0430) -- (2.8800, 2.7960, 3.0371) -- (2.8340, 2.7960, 3.0434) -- cycle;
\fill[blue!15.2, opacity=0.7] (2.8340, 2.7960, 3.0434) -- (2.8800, 2.7960, 3.0371) -- (2.8800, 2.8500, 3.0311) -- (2.8340, 2.8500, 3.0373) -- cycle;
\fill[blue!15.5, opacity=0.7] (2.8340, 2.8500, 3.0373) -- (2.8800, 2.8500, 3.0311) -- (2.8800, 2.9040, 3.0249) -- (2.8340, 2.9040, 3.0312) -- cycle;
\fill[blue!16.8, opacity=0.7] (2.8340, 2.9040, 3.0312) -- (2.8800, 2.9040, 3.0249) -- (2.8800, 2.9580, 3.0188) -- (2.8340, 2.9580, 3.0251) -- cycle;
\fill[blue!21.4, opacity=0.7] (2.8340, 2.9580, 3.0251) -- (2.8800, 2.9580, 3.0188) -- (2.8800, 3.0120, 3.0125) -- (2.8340, 3.0120, 3.0188) -- cycle;
\fill[blue!29.3, opacity=0.7] (2.8340, 3.0120, 3.0188) -- (2.8800, 3.0120, 3.0125) -- (2.8800, 3.0660, 3.0063) -- (2.8340, 3.0660, 3.0126) -- cycle;
\fill[blue!33.5, opacity=0.7] (2.8340, 3.0660, 3.0126) -- (2.8800, 3.0660, 3.0063) -- (2.8800, 3.1200, 3.0000) -- (2.8340, 3.1200, 3.0063) -- cycle;


% Back faces (lighter, behind)
\fill[blue!5, opacity=0.2] (0, \ysize, 0) -- (\xsize, \ysize, 0) -- (\xsize, \ysize, \zmax) -- (0, \ysize, \zmax) -- cycle;
\fill[blue!5, opacity=0.2] (0, 0, 0) -- (0, \ysize, 0) -- (0, \ysize, \zmax) -- (0, 0, \zmax) -- cycle;

\else
% Right figure: plain vertical faces
\fill[blue!5, opacity=0.3] (0, 0, 0) -- (\xsize, 0, 0) -- (\xsize, 0, \zmax) -- (0, 0, \zmax) -- cycle;
\fill[blue!5, opacity=0.3] (\xsize, 0, 0) -- (\xsize, \ysize, 0) -- (\xsize, \ysize, \zmax) -- (\xsize, 0, \zmax) -- cycle;
\fill[blue!5, opacity=0.2] (0, \ysize, 0) -- (\xsize, \ysize, 0) -- (\xsize, \ysize, \zmax) -- (0, \ysize, \zmax) -- cycle;
\fill[blue!5, opacity=0.2] (0, 0, 0) -- (0, \ysize, 0) -- (0, \ysize, \zmax) -- (0, 0, \zmax) -- cycle;
\fi

% Vertical edges only (no top/bottom frames) - curved like t-axis
\ifdim\xoffset pt=0pt\else
\draw[black!50, thick, dashed] (0, \ysize, 0)
    \foreach \zz in {0,0.1,...,\zmax} {
        -- ({0 + 0.5*\curvature*(1 - cos(\zz*180/\zmax))}, {\ysize + 0.5*\curvature*(1 - cos(\zz*180/\zmax))}, \zz)
    };
\fi
\draw[black!50, thick] (\xsize, 0, 0)
    \foreach \zz in {0,0.1,...,\zmax} {
        -- ({\xsize - 0.5*\curvature*(1 - cos(\zz*180/\zmax))}, {0 - 0.5*\curvature*(1 - cos(\zz*180/\zmax))}, \zz)
    };
\draw[black!50, thick] (0, 0, 0)
    \foreach \zz in {0,0.1,...,\zmax} {
        -- ({0 + 0.5*\curvature*(1 - cos(\zz*180/\zmax))}, {0 - 0.5*\curvature*(1 - cos(\zz*180/\zmax))}, \zz)
    };
\draw[black!50, thick] (\xsize, \ysize, 0)
    \foreach \zz in {0,0.1,...,\zmax} {
        -- ({\xsize - 0.5*\curvature*(1 - cos(\zz*180/\zmax))}, {\ysize + 0.5*\curvature*(1 - cos(\zz*180/\zmax))}, \zz)
    };

% Precomputed horizontal slice patches (right figure only - left figure patches included above)
\ifdim\xoffset pt=0pt\else
% Auto-generated by generate_logo_patches.py
% Slice 0 horizontal patches
\fill[blue!15.0, opacity=0.7] (0.0000, 0.0000, 0.0000) -- (0.0500, 0.0000, 0.0063) -- (0.0500, 0.0500, 0.0126) -- (0.0000, 0.0500, 0.0063) -- cycle;
\fill[blue!15.0, opacity=0.7] (0.0000, 0.0500, 0.0063) -- (0.0500, 0.0500, 0.0126) -- (0.0500, 0.1000, 0.0188) -- (0.0000, 0.1000, 0.0125) -- cycle;
\fill[blue!15.0, opacity=0.7] (0.0000, 0.1000, 0.0125) -- (0.0500, 0.1000, 0.0188) -- (0.0500, 0.1500, 0.0251) -- (0.0000, 0.1500, 0.0188) -- cycle;
\fill[blue!15.0, opacity=0.7] (0.0000, 0.1500, 0.0188) -- (0.0500, 0.1500, 0.0251) -- (0.0500, 0.2000, 0.0312) -- (0.0000, 0.2000, 0.0249) -- cycle;
\fill[blue!15.0, opacity=0.7] (0.0000, 0.2000, 0.0249) -- (0.0500, 0.2000, 0.0312) -- (0.0500, 0.2500, 0.0373) -- (0.0000, 0.2500, 0.0311) -- cycle;
\fill[blue!15.0, opacity=0.7] (0.0000, 0.2500, 0.0311) -- (0.0500, 0.2500, 0.0373) -- (0.0500, 0.3000, 0.0434) -- (0.0000, 0.3000, 0.0371) -- cycle;
\fill[blue!15.0, opacity=0.7] (0.0000, 0.3000, 0.0371) -- (0.0500, 0.3000, 0.0434) -- (0.0500, 0.3500, 0.0493) -- (0.0000, 0.3500, 0.0430) -- cycle;
\fill[blue!15.0, opacity=0.7] (0.0000, 0.3500, 0.0430) -- (0.0500, 0.3500, 0.0493) -- (0.0500, 0.4000, 0.0551) -- (0.0000, 0.4000, 0.0488) -- cycle;
\fill[blue!15.0, opacity=0.7] (0.0000, 0.4000, 0.0488) -- (0.0500, 0.4000, 0.0551) -- (0.0500, 0.4500, 0.0608) -- (0.0000, 0.4500, 0.0545) -- cycle;
\fill[blue!15.0, opacity=0.7] (0.0000, 0.4500, 0.0545) -- (0.0500, 0.4500, 0.0608) -- (0.0500, 0.5000, 0.0663) -- (0.0000, 0.5000, 0.0600) -- cycle;
\fill[blue!15.0, opacity=0.7] (0.0000, 0.5000, 0.0600) -- (0.0500, 0.5000, 0.0663) -- (0.0500, 0.5500, 0.0716) -- (0.0000, 0.5500, 0.0654) -- cycle;
\fill[blue!15.0, opacity=0.7] (0.0000, 0.5500, 0.0654) -- (0.0500, 0.5500, 0.0716) -- (0.0500, 0.6000, 0.0768) -- (0.0000, 0.6000, 0.0705) -- cycle;
\fill[blue!15.0, opacity=0.7] (0.0000, 0.6000, 0.0705) -- (0.0500, 0.6000, 0.0768) -- (0.0500, 0.6500, 0.0818) -- (0.0000, 0.6500, 0.0755) -- cycle;
\fill[blue!15.0, opacity=0.7] (0.0000, 0.6500, 0.0755) -- (0.0500, 0.6500, 0.0818) -- (0.0500, 0.7000, 0.0866) -- (0.0000, 0.7000, 0.0803) -- cycle;
\fill[blue!15.0, opacity=0.7] (0.0000, 0.7000, 0.0803) -- (0.0500, 0.7000, 0.0866) -- (0.0500, 0.7500, 0.0911) -- (0.0000, 0.7500, 0.0849) -- cycle;
\fill[blue!15.0, opacity=0.7] (0.0000, 0.7500, 0.0849) -- (0.0500, 0.7500, 0.0911) -- (0.0500, 0.8000, 0.0955) -- (0.0000, 0.8000, 0.0892) -- cycle;
\fill[blue!15.0, opacity=0.7] (0.0000, 0.8000, 0.0892) -- (0.0500, 0.8000, 0.0955) -- (0.0500, 0.8500, 0.0995) -- (0.0000, 0.8500, 0.0933) -- cycle;
\fill[blue!15.0, opacity=0.7] (0.0000, 0.8500, 0.0933) -- (0.0500, 0.8500, 0.0995) -- (0.0500, 0.9000, 0.1034) -- (0.0000, 0.9000, 0.0971) -- cycle;
\fill[blue!15.0, opacity=0.7] (0.0000, 0.9000, 0.0971) -- (0.0500, 0.9000, 0.1034) -- (0.0500, 0.9500, 0.1069) -- (0.0000, 0.9500, 0.1006) -- cycle;
\fill[blue!15.0, opacity=0.7] (0.0000, 0.9500, 0.1006) -- (0.0500, 0.9500, 0.1069) -- (0.0500, 1.0000, 0.1102) -- (0.0000, 1.0000, 0.1039) -- cycle;
\fill[blue!15.0, opacity=0.7] (0.0000, 1.0000, 0.1039) -- (0.0500, 1.0000, 0.1102) -- (0.0500, 1.0500, 0.1132) -- (0.0000, 1.0500, 0.1069) -- cycle;
\fill[blue!15.0, opacity=0.7] (0.0000, 1.0500, 0.1069) -- (0.0500, 1.0500, 0.1132) -- (0.0500, 1.1000, 0.1159) -- (0.0000, 1.1000, 0.1096) -- cycle;
\fill[blue!15.0, opacity=0.7] (0.0000, 1.1000, 0.1096) -- (0.0500, 1.1000, 0.1159) -- (0.0500, 1.1500, 0.1183) -- (0.0000, 1.1500, 0.1120) -- cycle;
\fill[blue!15.0, opacity=0.7] (0.0000, 1.1500, 0.1120) -- (0.0500, 1.1500, 0.1183) -- (0.0500, 1.2000, 0.1204) -- (0.0000, 1.2000, 0.1141) -- cycle;
\fill[blue!15.0, opacity=0.7] (0.0000, 1.2000, 0.1141) -- (0.0500, 1.2000, 0.1204) -- (0.0500, 1.2500, 0.1222) -- (0.0000, 1.2500, 0.1159) -- cycle;
\fill[blue!15.0, opacity=0.7] (0.0000, 1.2500, 0.1159) -- (0.0500, 1.2500, 0.1222) -- (0.0500, 1.3000, 0.1237) -- (0.0000, 1.3000, 0.1174) -- cycle;
\fill[blue!15.0, opacity=0.7] (0.0000, 1.3000, 0.1174) -- (0.0500, 1.3000, 0.1237) -- (0.0500, 1.3500, 0.1248) -- (0.0000, 1.3500, 0.1185) -- cycle;
\fill[blue!15.0, opacity=0.7] (0.0000, 1.3500, 0.1185) -- (0.0500, 1.3500, 0.1248) -- (0.0500, 1.4000, 0.1256) -- (0.0000, 1.4000, 0.1193) -- cycle;
\fill[blue!15.0, opacity=0.7] (0.0000, 1.4000, 0.1193) -- (0.0500, 1.4000, 0.1256) -- (0.0500, 1.4500, 0.1261) -- (0.0000, 1.4500, 0.1198) -- cycle;
\fill[blue!15.0, opacity=0.7] (0.0000, 1.4500, 0.1198) -- (0.0500, 1.4500, 0.1261) -- (0.0500, 1.5000, 0.1263) -- (0.0000, 1.5000, 0.1200) -- cycle;
\fill[blue!15.0, opacity=0.7] (0.0000, 1.5000, 0.1200) -- (0.0500, 1.5000, 0.1263) -- (0.0500, 1.5500, 0.1261) -- (0.0000, 1.5500, 0.1198) -- cycle;
\fill[blue!15.0, opacity=0.7] (0.0000, 1.5500, 0.1198) -- (0.0500, 1.5500, 0.1261) -- (0.0500, 1.6000, 0.1256) -- (0.0000, 1.6000, 0.1193) -- cycle;
\fill[blue!15.0, opacity=0.7] (0.0000, 1.6000, 0.1193) -- (0.0500, 1.6000, 0.1256) -- (0.0500, 1.6500, 0.1248) -- (0.0000, 1.6500, 0.1185) -- cycle;
\fill[blue!15.0, opacity=0.7] (0.0000, 1.6500, 0.1185) -- (0.0500, 1.6500, 0.1248) -- (0.0500, 1.7000, 0.1237) -- (0.0000, 1.7000, 0.1174) -- cycle;
\fill[blue!15.0, opacity=0.7] (0.0000, 1.7000, 0.1174) -- (0.0500, 1.7000, 0.1237) -- (0.0500, 1.7500, 0.1222) -- (0.0000, 1.7500, 0.1159) -- cycle;
\fill[blue!15.0, opacity=0.7] (0.0000, 1.7500, 0.1159) -- (0.0500, 1.7500, 0.1222) -- (0.0500, 1.8000, 0.1204) -- (0.0000, 1.8000, 0.1141) -- cycle;
\fill[blue!15.0, opacity=0.7] (0.0000, 1.8000, 0.1141) -- (0.0500, 1.8000, 0.1204) -- (0.0500, 1.8500, 0.1183) -- (0.0000, 1.8500, 0.1120) -- cycle;
\fill[blue!15.0, opacity=0.7] (0.0000, 1.8500, 0.1120) -- (0.0500, 1.8500, 0.1183) -- (0.0500, 1.9000, 0.1159) -- (0.0000, 1.9000, 0.1096) -- cycle;
\fill[blue!15.0, opacity=0.7] (0.0000, 1.9000, 0.1096) -- (0.0500, 1.9000, 0.1159) -- (0.0500, 1.9500, 0.1132) -- (0.0000, 1.9500, 0.1069) -- cycle;
\fill[blue!15.0, opacity=0.7] (0.0000, 1.9500, 0.1069) -- (0.0500, 1.9500, 0.1132) -- (0.0500, 2.0000, 0.1102) -- (0.0000, 2.0000, 0.1039) -- cycle;
\fill[blue!15.0, opacity=0.7] (0.0000, 2.0000, 0.1039) -- (0.0500, 2.0000, 0.1102) -- (0.0500, 2.0500, 0.1069) -- (0.0000, 2.0500, 0.1006) -- cycle;
\fill[blue!15.0, opacity=0.7] (0.0000, 2.0500, 0.1006) -- (0.0500, 2.0500, 0.1069) -- (0.0500, 2.1000, 0.1034) -- (0.0000, 2.1000, 0.0971) -- cycle;
\fill[blue!15.0, opacity=0.7] (0.0000, 2.1000, 0.0971) -- (0.0500, 2.1000, 0.1034) -- (0.0500, 2.1500, 0.0995) -- (0.0000, 2.1500, 0.0933) -- cycle;
\fill[blue!15.0, opacity=0.7] (0.0000, 2.1500, 0.0933) -- (0.0500, 2.1500, 0.0995) -- (0.0500, 2.2000, 0.0955) -- (0.0000, 2.2000, 0.0892) -- cycle;
\fill[blue!15.0, opacity=0.7] (0.0000, 2.2000, 0.0892) -- (0.0500, 2.2000, 0.0955) -- (0.0500, 2.2500, 0.0911) -- (0.0000, 2.2500, 0.0849) -- cycle;
\fill[blue!15.0, opacity=0.7] (0.0000, 2.2500, 0.0849) -- (0.0500, 2.2500, 0.0911) -- (0.0500, 2.3000, 0.0866) -- (0.0000, 2.3000, 0.0803) -- cycle;
\fill[blue!15.0, opacity=0.7] (0.0000, 2.3000, 0.0803) -- (0.0500, 2.3000, 0.0866) -- (0.0500, 2.3500, 0.0818) -- (0.0000, 2.3500, 0.0755) -- cycle;
\fill[blue!15.0, opacity=0.7] (0.0000, 2.3500, 0.0755) -- (0.0500, 2.3500, 0.0818) -- (0.0500, 2.4000, 0.0768) -- (0.0000, 2.4000, 0.0705) -- cycle;
\fill[blue!15.0, opacity=0.7] (0.0000, 2.4000, 0.0705) -- (0.0500, 2.4000, 0.0768) -- (0.0500, 2.4500, 0.0716) -- (0.0000, 2.4500, 0.0654) -- cycle;
\fill[blue!15.0, opacity=0.7] (0.0000, 2.4500, 0.0654) -- (0.0500, 2.4500, 0.0716) -- (0.0500, 2.5000, 0.0663) -- (0.0000, 2.5000, 0.0600) -- cycle;
\fill[blue!15.0, opacity=0.7] (0.0000, 2.5000, 0.0600) -- (0.0500, 2.5000, 0.0663) -- (0.0500, 2.5500, 0.0608) -- (0.0000, 2.5500, 0.0545) -- cycle;
\fill[blue!15.0, opacity=0.7] (0.0000, 2.5500, 0.0545) -- (0.0500, 2.5500, 0.0608) -- (0.0500, 2.6000, 0.0551) -- (0.0000, 2.6000, 0.0488) -- cycle;
\fill[blue!15.0, opacity=0.7] (0.0000, 2.6000, 0.0488) -- (0.0500, 2.6000, 0.0551) -- (0.0500, 2.6500, 0.0493) -- (0.0000, 2.6500, 0.0430) -- cycle;
\fill[blue!15.0, opacity=0.7] (0.0000, 2.6500, 0.0430) -- (0.0500, 2.6500, 0.0493) -- (0.0500, 2.7000, 0.0434) -- (0.0000, 2.7000, 0.0371) -- cycle;
\fill[blue!15.0, opacity=0.7] (0.0000, 2.7000, 0.0371) -- (0.0500, 2.7000, 0.0434) -- (0.0500, 2.7500, 0.0373) -- (0.0000, 2.7500, 0.0311) -- cycle;
\fill[blue!15.0, opacity=0.7] (0.0000, 2.7500, 0.0311) -- (0.0500, 2.7500, 0.0373) -- (0.0500, 2.8000, 0.0312) -- (0.0000, 2.8000, 0.0249) -- cycle;
\fill[blue!15.0, opacity=0.7] (0.0000, 2.8000, 0.0249) -- (0.0500, 2.8000, 0.0312) -- (0.0500, 2.8500, 0.0251) -- (0.0000, 2.8500, 0.0188) -- cycle;
\fill[blue!15.0, opacity=0.7] (0.0000, 2.8500, 0.0188) -- (0.0500, 2.8500, 0.0251) -- (0.0500, 2.9000, 0.0188) -- (0.0000, 2.9000, 0.0125) -- cycle;
\fill[blue!15.0, opacity=0.7] (0.0000, 2.9000, 0.0125) -- (0.0500, 2.9000, 0.0188) -- (0.0500, 2.9500, 0.0126) -- (0.0000, 2.9500, 0.0063) -- cycle;
\fill[blue!15.0, opacity=0.7] (0.0000, 2.9500, 0.0063) -- (0.0500, 2.9500, 0.0126) -- (0.0500, 3.0000, 0.0063) -- (0.0000, 3.0000, 0.0000) -- cycle;
\fill[blue!15.0, opacity=0.7] (0.0500, 0.0000, 0.0063) -- (0.1000, 0.0000, 0.0125) -- (0.1000, 0.0500, 0.0188) -- (0.0500, 0.0500, 0.0126) -- cycle;
\fill[blue!15.0, opacity=0.7] (0.0500, 0.0500, 0.0126) -- (0.1000, 0.0500, 0.0188) -- (0.1000, 0.1000, 0.0251) -- (0.0500, 0.1000, 0.0188) -- cycle;
\fill[blue!15.0, opacity=0.7] (0.0500, 0.1000, 0.0188) -- (0.1000, 0.1000, 0.0251) -- (0.1000, 0.1500, 0.0313) -- (0.0500, 0.1500, 0.0251) -- cycle;
\fill[blue!15.0, opacity=0.7] (0.0500, 0.1500, 0.0251) -- (0.1000, 0.1500, 0.0313) -- (0.1000, 0.2000, 0.0375) -- (0.0500, 0.2000, 0.0312) -- cycle;
\fill[blue!15.0, opacity=0.7] (0.0500, 0.2000, 0.0312) -- (0.1000, 0.2000, 0.0375) -- (0.1000, 0.2500, 0.0436) -- (0.0500, 0.2500, 0.0373) -- cycle;
\fill[blue!15.0, opacity=0.7] (0.0500, 0.2500, 0.0373) -- (0.1000, 0.2500, 0.0436) -- (0.1000, 0.3000, 0.0496) -- (0.0500, 0.3000, 0.0434) -- cycle;
\fill[blue!15.0, opacity=0.7] (0.0500, 0.3000, 0.0434) -- (0.1000, 0.3000, 0.0496) -- (0.1000, 0.3500, 0.0555) -- (0.0500, 0.3500, 0.0493) -- cycle;
\fill[blue!15.0, opacity=0.7] (0.0500, 0.3500, 0.0493) -- (0.1000, 0.3500, 0.0555) -- (0.1000, 0.4000, 0.0614) -- (0.0500, 0.4000, 0.0551) -- cycle;
\fill[blue!15.0, opacity=0.7] (0.0500, 0.4000, 0.0551) -- (0.1000, 0.4000, 0.0614) -- (0.1000, 0.4500, 0.0670) -- (0.0500, 0.4500, 0.0608) -- cycle;
\fill[blue!15.0, opacity=0.7] (0.0500, 0.4500, 0.0608) -- (0.1000, 0.4500, 0.0670) -- (0.1000, 0.5000, 0.0725) -- (0.0500, 0.5000, 0.0663) -- cycle;
\fill[blue!15.0, opacity=0.7] (0.0500, 0.5000, 0.0663) -- (0.1000, 0.5000, 0.0725) -- (0.1000, 0.5500, 0.0779) -- (0.0500, 0.5500, 0.0716) -- cycle;
\fill[blue!15.0, opacity=0.7] (0.0500, 0.5500, 0.0716) -- (0.1000, 0.5500, 0.0779) -- (0.1000, 0.6000, 0.0831) -- (0.0500, 0.6000, 0.0768) -- cycle;
\fill[blue!15.0, opacity=0.7] (0.0500, 0.6000, 0.0768) -- (0.1000, 0.6000, 0.0831) -- (0.1000, 0.6500, 0.0881) -- (0.0500, 0.6500, 0.0818) -- cycle;
\fill[blue!15.0, opacity=0.7] (0.0500, 0.6500, 0.0818) -- (0.1000, 0.6500, 0.0881) -- (0.1000, 0.7000, 0.0928) -- (0.0500, 0.7000, 0.0866) -- cycle;
\fill[blue!15.0, opacity=0.7] (0.0500, 0.7000, 0.0866) -- (0.1000, 0.7000, 0.0928) -- (0.1000, 0.7500, 0.0974) -- (0.0500, 0.7500, 0.0911) -- cycle;
\fill[blue!15.0, opacity=0.7] (0.0500, 0.7500, 0.0911) -- (0.1000, 0.7500, 0.0974) -- (0.1000, 0.8000, 0.1017) -- (0.0500, 0.8000, 0.0955) -- cycle;
\fill[blue!15.0, opacity=0.7] (0.0500, 0.8000, 0.0955) -- (0.1000, 0.8000, 0.1017) -- (0.1000, 0.8500, 0.1058) -- (0.0500, 0.8500, 0.0995) -- cycle;
\fill[blue!15.0, opacity=0.7] (0.0500, 0.8500, 0.0995) -- (0.1000, 0.8500, 0.1058) -- (0.1000, 0.9000, 0.1096) -- (0.0500, 0.9000, 0.1034) -- cycle;
\fill[blue!15.0, opacity=0.7] (0.0500, 0.9000, 0.1034) -- (0.1000, 0.9000, 0.1096) -- (0.1000, 0.9500, 0.1132) -- (0.0500, 0.9500, 0.1069) -- cycle;
\fill[blue!15.0, opacity=0.7] (0.0500, 0.9500, 0.1069) -- (0.1000, 0.9500, 0.1132) -- (0.1000, 1.0000, 0.1165) -- (0.0500, 1.0000, 0.1102) -- cycle;
\fill[blue!15.0, opacity=0.7] (0.0500, 1.0000, 0.1102) -- (0.1000, 1.0000, 0.1165) -- (0.1000, 1.0500, 0.1195) -- (0.0500, 1.0500, 0.1132) -- cycle;
\fill[blue!15.0, opacity=0.7] (0.0500, 1.0500, 0.1132) -- (0.1000, 1.0500, 0.1195) -- (0.1000, 1.1000, 0.1222) -- (0.0500, 1.1000, 0.1159) -- cycle;
\fill[blue!15.0, opacity=0.7] (0.0500, 1.1000, 0.1159) -- (0.1000, 1.1000, 0.1222) -- (0.1000, 1.1500, 0.1246) -- (0.0500, 1.1500, 0.1183) -- cycle;
\fill[blue!15.0, opacity=0.7] (0.0500, 1.1500, 0.1183) -- (0.1000, 1.1500, 0.1246) -- (0.1000, 1.2000, 0.1267) -- (0.0500, 1.2000, 0.1204) -- cycle;
\fill[blue!15.0, opacity=0.7] (0.0500, 1.2000, 0.1204) -- (0.1000, 1.2000, 0.1267) -- (0.1000, 1.2500, 0.1285) -- (0.0500, 1.2500, 0.1222) -- cycle;
\fill[blue!15.0, opacity=0.7] (0.0500, 1.2500, 0.1222) -- (0.1000, 1.2500, 0.1285) -- (0.1000, 1.3000, 0.1299) -- (0.0500, 1.3000, 0.1237) -- cycle;
\fill[blue!15.0, opacity=0.7] (0.0500, 1.3000, 0.1237) -- (0.1000, 1.3000, 0.1299) -- (0.1000, 1.3500, 0.1311) -- (0.0500, 1.3500, 0.1248) -- cycle;
\fill[blue!15.0, opacity=0.7] (0.0500, 1.3500, 0.1248) -- (0.1000, 1.3500, 0.1311) -- (0.1000, 1.4000, 0.1319) -- (0.0500, 1.4000, 0.1256) -- cycle;
\fill[blue!15.0, opacity=0.7] (0.0500, 1.4000, 0.1256) -- (0.1000, 1.4000, 0.1319) -- (0.1000, 1.4500, 0.1324) -- (0.0500, 1.4500, 0.1261) -- cycle;
\fill[blue!15.0, opacity=0.7] (0.0500, 1.4500, 0.1261) -- (0.1000, 1.4500, 0.1324) -- (0.1000, 1.5000, 0.1325) -- (0.0500, 1.5000, 0.1263) -- cycle;
\fill[blue!15.0, opacity=0.7] (0.0500, 1.5000, 0.1263) -- (0.1000, 1.5000, 0.1325) -- (0.1000, 1.5500, 0.1324) -- (0.0500, 1.5500, 0.1261) -- cycle;
\fill[blue!15.0, opacity=0.7] (0.0500, 1.5500, 0.1261) -- (0.1000, 1.5500, 0.1324) -- (0.1000, 1.6000, 0.1319) -- (0.0500, 1.6000, 0.1256) -- cycle;
\fill[blue!15.0, opacity=0.7] (0.0500, 1.6000, 0.1256) -- (0.1000, 1.6000, 0.1319) -- (0.1000, 1.6500, 0.1311) -- (0.0500, 1.6500, 0.1248) -- cycle;
\fill[blue!15.0, opacity=0.7] (0.0500, 1.6500, 0.1248) -- (0.1000, 1.6500, 0.1311) -- (0.1000, 1.7000, 0.1299) -- (0.0500, 1.7000, 0.1237) -- cycle;
\fill[blue!15.0, opacity=0.7] (0.0500, 1.7000, 0.1237) -- (0.1000, 1.7000, 0.1299) -- (0.1000, 1.7500, 0.1285) -- (0.0500, 1.7500, 0.1222) -- cycle;
\fill[blue!15.0, opacity=0.7] (0.0500, 1.7500, 0.1222) -- (0.1000, 1.7500, 0.1285) -- (0.1000, 1.8000, 0.1267) -- (0.0500, 1.8000, 0.1204) -- cycle;
\fill[blue!15.0, opacity=0.7] (0.0500, 1.8000, 0.1204) -- (0.1000, 1.8000, 0.1267) -- (0.1000, 1.8500, 0.1246) -- (0.0500, 1.8500, 0.1183) -- cycle;
\fill[blue!15.0, opacity=0.7] (0.0500, 1.8500, 0.1183) -- (0.1000, 1.8500, 0.1246) -- (0.1000, 1.9000, 0.1222) -- (0.0500, 1.9000, 0.1159) -- cycle;
\fill[blue!15.0, opacity=0.7] (0.0500, 1.9000, 0.1159) -- (0.1000, 1.9000, 0.1222) -- (0.1000, 1.9500, 0.1195) -- (0.0500, 1.9500, 0.1132) -- cycle;
\fill[blue!15.0, opacity=0.7] (0.0500, 1.9500, 0.1132) -- (0.1000, 1.9500, 0.1195) -- (0.1000, 2.0000, 0.1165) -- (0.0500, 2.0000, 0.1102) -- cycle;
\fill[blue!15.0, opacity=0.7] (0.0500, 2.0000, 0.1102) -- (0.1000, 2.0000, 0.1165) -- (0.1000, 2.0500, 0.1132) -- (0.0500, 2.0500, 0.1069) -- cycle;
\fill[blue!15.0, opacity=0.7] (0.0500, 2.0500, 0.1069) -- (0.1000, 2.0500, 0.1132) -- (0.1000, 2.1000, 0.1096) -- (0.0500, 2.1000, 0.1034) -- cycle;
\fill[blue!15.0, opacity=0.7] (0.0500, 2.1000, 0.1034) -- (0.1000, 2.1000, 0.1096) -- (0.1000, 2.1500, 0.1058) -- (0.0500, 2.1500, 0.0995) -- cycle;
\fill[blue!15.0, opacity=0.7] (0.0500, 2.1500, 0.0995) -- (0.1000, 2.1500, 0.1058) -- (0.1000, 2.2000, 0.1017) -- (0.0500, 2.2000, 0.0955) -- cycle;
\fill[blue!15.0, opacity=0.7] (0.0500, 2.2000, 0.0955) -- (0.1000, 2.2000, 0.1017) -- (0.1000, 2.2500, 0.0974) -- (0.0500, 2.2500, 0.0911) -- cycle;
\fill[blue!15.0, opacity=0.7] (0.0500, 2.2500, 0.0911) -- (0.1000, 2.2500, 0.0974) -- (0.1000, 2.3000, 0.0928) -- (0.0500, 2.3000, 0.0866) -- cycle;
\fill[blue!15.0, opacity=0.7] (0.0500, 2.3000, 0.0866) -- (0.1000, 2.3000, 0.0928) -- (0.1000, 2.3500, 0.0881) -- (0.0500, 2.3500, 0.0818) -- cycle;
\fill[blue!15.0, opacity=0.7] (0.0500, 2.3500, 0.0818) -- (0.1000, 2.3500, 0.0881) -- (0.1000, 2.4000, 0.0831) -- (0.0500, 2.4000, 0.0768) -- cycle;
\fill[blue!15.0, opacity=0.7] (0.0500, 2.4000, 0.0768) -- (0.1000, 2.4000, 0.0831) -- (0.1000, 2.4500, 0.0779) -- (0.0500, 2.4500, 0.0716) -- cycle;
\fill[blue!15.0, opacity=0.7] (0.0500, 2.4500, 0.0716) -- (0.1000, 2.4500, 0.0779) -- (0.1000, 2.5000, 0.0725) -- (0.0500, 2.5000, 0.0663) -- cycle;
\fill[blue!15.0, opacity=0.7] (0.0500, 2.5000, 0.0663) -- (0.1000, 2.5000, 0.0725) -- (0.1000, 2.5500, 0.0670) -- (0.0500, 2.5500, 0.0608) -- cycle;
\fill[blue!15.0, opacity=0.7] (0.0500, 2.5500, 0.0608) -- (0.1000, 2.5500, 0.0670) -- (0.1000, 2.6000, 0.0614) -- (0.0500, 2.6000, 0.0551) -- cycle;
\fill[blue!15.0, opacity=0.7] (0.0500, 2.6000, 0.0551) -- (0.1000, 2.6000, 0.0614) -- (0.1000, 2.6500, 0.0555) -- (0.0500, 2.6500, 0.0493) -- cycle;
\fill[blue!15.0, opacity=0.7] (0.0500, 2.6500, 0.0493) -- (0.1000, 2.6500, 0.0555) -- (0.1000, 2.7000, 0.0496) -- (0.0500, 2.7000, 0.0434) -- cycle;
\fill[blue!15.0, opacity=0.7] (0.0500, 2.7000, 0.0434) -- (0.1000, 2.7000, 0.0496) -- (0.1000, 2.7500, 0.0436) -- (0.0500, 2.7500, 0.0373) -- cycle;
\fill[blue!15.0, opacity=0.7] (0.0500, 2.7500, 0.0373) -- (0.1000, 2.7500, 0.0436) -- (0.1000, 2.8000, 0.0375) -- (0.0500, 2.8000, 0.0312) -- cycle;
\fill[blue!15.0, opacity=0.7] (0.0500, 2.8000, 0.0312) -- (0.1000, 2.8000, 0.0375) -- (0.1000, 2.8500, 0.0313) -- (0.0500, 2.8500, 0.0251) -- cycle;
\fill[blue!15.0, opacity=0.7] (0.0500, 2.8500, 0.0251) -- (0.1000, 2.8500, 0.0313) -- (0.1000, 2.9000, 0.0251) -- (0.0500, 2.9000, 0.0188) -- cycle;
\fill[blue!15.0, opacity=0.7] (0.0500, 2.9000, 0.0188) -- (0.1000, 2.9000, 0.0251) -- (0.1000, 2.9500, 0.0188) -- (0.0500, 2.9500, 0.0126) -- cycle;
\fill[blue!15.0, opacity=0.7] (0.0500, 2.9500, 0.0126) -- (0.1000, 2.9500, 0.0188) -- (0.1000, 3.0000, 0.0125) -- (0.0500, 3.0000, 0.0063) -- cycle;
\fill[blue!15.0, opacity=0.7] (0.1000, 0.0000, 0.0125) -- (0.1500, 0.0000, 0.0188) -- (0.1500, 0.0500, 0.0251) -- (0.1000, 0.0500, 0.0188) -- cycle;
\fill[blue!15.0, opacity=0.7] (0.1000, 0.0500, 0.0188) -- (0.1500, 0.0500, 0.0251) -- (0.1500, 0.1000, 0.0313) -- (0.1000, 0.1000, 0.0251) -- cycle;
\fill[blue!15.0, opacity=0.7] (0.1000, 0.1000, 0.0251) -- (0.1500, 0.1000, 0.0313) -- (0.1500, 0.1500, 0.0375) -- (0.1000, 0.1500, 0.0313) -- cycle;
\fill[blue!15.0, opacity=0.7] (0.1000, 0.1500, 0.0313) -- (0.1500, 0.1500, 0.0375) -- (0.1500, 0.2000, 0.0437) -- (0.1000, 0.2000, 0.0375) -- cycle;
\fill[blue!15.0, opacity=0.7] (0.1000, 0.2000, 0.0375) -- (0.1500, 0.2000, 0.0437) -- (0.1500, 0.2500, 0.0498) -- (0.1000, 0.2500, 0.0436) -- cycle;
\fill[blue!15.0, opacity=0.7] (0.1000, 0.2500, 0.0436) -- (0.1500, 0.2500, 0.0498) -- (0.1500, 0.3000, 0.0559) -- (0.1000, 0.3000, 0.0496) -- cycle;
\fill[blue!15.0, opacity=0.7] (0.1000, 0.3000, 0.0496) -- (0.1500, 0.3000, 0.0559) -- (0.1500, 0.3500, 0.0618) -- (0.1000, 0.3500, 0.0555) -- cycle;
\fill[blue!15.0, opacity=0.7] (0.1000, 0.3500, 0.0555) -- (0.1500, 0.3500, 0.0618) -- (0.1500, 0.4000, 0.0676) -- (0.1000, 0.4000, 0.0614) -- cycle;
\fill[blue!15.0, opacity=0.7] (0.1000, 0.4000, 0.0614) -- (0.1500, 0.4000, 0.0676) -- (0.1500, 0.4500, 0.0733) -- (0.1000, 0.4500, 0.0670) -- cycle;
\fill[blue!15.0, opacity=0.7] (0.1000, 0.4500, 0.0670) -- (0.1500, 0.4500, 0.0733) -- (0.1500, 0.5000, 0.0788) -- (0.1000, 0.5000, 0.0725) -- cycle;
\fill[blue!15.0, opacity=0.7] (0.1000, 0.5000, 0.0725) -- (0.1500, 0.5000, 0.0788) -- (0.1500, 0.5500, 0.0841) -- (0.1000, 0.5500, 0.0779) -- cycle;
\fill[blue!15.0, opacity=0.7] (0.1000, 0.5500, 0.0779) -- (0.1500, 0.5500, 0.0841) -- (0.1500, 0.6000, 0.0893) -- (0.1000, 0.6000, 0.0831) -- cycle;
\fill[blue!15.0, opacity=0.7] (0.1000, 0.6000, 0.0831) -- (0.1500, 0.6000, 0.0893) -- (0.1500, 0.6500, 0.0943) -- (0.1000, 0.6500, 0.0881) -- cycle;
\fill[blue!15.0, opacity=0.7] (0.1000, 0.6500, 0.0881) -- (0.1500, 0.6500, 0.0943) -- (0.1500, 0.7000, 0.0991) -- (0.1000, 0.7000, 0.0928) -- cycle;
\fill[blue!15.0, opacity=0.7] (0.1000, 0.7000, 0.0928) -- (0.1500, 0.7000, 0.0991) -- (0.1500, 0.7500, 0.1036) -- (0.1000, 0.7500, 0.0974) -- cycle;
\fill[blue!15.0, opacity=0.7] (0.1000, 0.7500, 0.0974) -- (0.1500, 0.7500, 0.1036) -- (0.1500, 0.8000, 0.1079) -- (0.1000, 0.8000, 0.1017) -- cycle;
\fill[blue!15.0, opacity=0.7] (0.1000, 0.8000, 0.1017) -- (0.1500, 0.8000, 0.1079) -- (0.1500, 0.8500, 0.1120) -- (0.1000, 0.8500, 0.1058) -- cycle;
\fill[blue!15.0, opacity=0.7] (0.1000, 0.8500, 0.1058) -- (0.1500, 0.8500, 0.1120) -- (0.1500, 0.9000, 0.1159) -- (0.1000, 0.9000, 0.1096) -- cycle;
\fill[blue!15.0, opacity=0.7] (0.1000, 0.9000, 0.1096) -- (0.1500, 0.9000, 0.1159) -- (0.1500, 0.9500, 0.1194) -- (0.1000, 0.9500, 0.1132) -- cycle;
\fill[blue!15.0, opacity=0.7] (0.1000, 0.9500, 0.1132) -- (0.1500, 0.9500, 0.1194) -- (0.1500, 1.0000, 0.1227) -- (0.1000, 1.0000, 0.1165) -- cycle;
\fill[blue!15.0, opacity=0.7] (0.1000, 1.0000, 0.1165) -- (0.1500, 1.0000, 0.1227) -- (0.1500, 1.0500, 0.1257) -- (0.1000, 1.0500, 0.1195) -- cycle;
\fill[blue!15.0, opacity=0.7] (0.1000, 1.0500, 0.1195) -- (0.1500, 1.0500, 0.1257) -- (0.1500, 1.1000, 0.1284) -- (0.1000, 1.1000, 0.1222) -- cycle;
\fill[blue!15.0, opacity=0.7] (0.1000, 1.1000, 0.1222) -- (0.1500, 1.1000, 0.1284) -- (0.1500, 1.1500, 0.1308) -- (0.1000, 1.1500, 0.1246) -- cycle;
\fill[blue!15.0, opacity=0.7] (0.1000, 1.1500, 0.1246) -- (0.1500, 1.1500, 0.1308) -- (0.1500, 1.2000, 0.1329) -- (0.1000, 1.2000, 0.1267) -- cycle;
\fill[blue!15.0, opacity=0.7] (0.1000, 1.2000, 0.1267) -- (0.1500, 1.2000, 0.1329) -- (0.1500, 1.2500, 0.1347) -- (0.1000, 1.2500, 0.1285) -- cycle;
\fill[blue!15.0, opacity=0.7] (0.1000, 1.2500, 0.1285) -- (0.1500, 1.2500, 0.1347) -- (0.1500, 1.3000, 0.1361) -- (0.1000, 1.3000, 0.1299) -- cycle;
\fill[blue!15.0, opacity=0.7] (0.1000, 1.3000, 0.1299) -- (0.1500, 1.3000, 0.1361) -- (0.1500, 1.3500, 0.1373) -- (0.1000, 1.3500, 0.1311) -- cycle;
\fill[blue!15.0, opacity=0.7] (0.1000, 1.3500, 0.1311) -- (0.1500, 1.3500, 0.1373) -- (0.1500, 1.4000, 0.1381) -- (0.1000, 1.4000, 0.1319) -- cycle;
\fill[blue!15.0, opacity=0.7] (0.1000, 1.4000, 0.1319) -- (0.1500, 1.4000, 0.1381) -- (0.1500, 1.4500, 0.1386) -- (0.1000, 1.4500, 0.1324) -- cycle;
\fill[blue!15.0, opacity=0.7] (0.1000, 1.4500, 0.1324) -- (0.1500, 1.4500, 0.1386) -- (0.1500, 1.5000, 0.1388) -- (0.1000, 1.5000, 0.1325) -- cycle;
\fill[blue!15.0, opacity=0.7] (0.1000, 1.5000, 0.1325) -- (0.1500, 1.5000, 0.1388) -- (0.1500, 1.5500, 0.1386) -- (0.1000, 1.5500, 0.1324) -- cycle;
\fill[blue!15.0, opacity=0.7] (0.1000, 1.5500, 0.1324) -- (0.1500, 1.5500, 0.1386) -- (0.1500, 1.6000, 0.1381) -- (0.1000, 1.6000, 0.1319) -- cycle;
\fill[blue!15.0, opacity=0.7] (0.1000, 1.6000, 0.1319) -- (0.1500, 1.6000, 0.1381) -- (0.1500, 1.6500, 0.1373) -- (0.1000, 1.6500, 0.1311) -- cycle;
\fill[blue!15.0, opacity=0.7] (0.1000, 1.6500, 0.1311) -- (0.1500, 1.6500, 0.1373) -- (0.1500, 1.7000, 0.1361) -- (0.1000, 1.7000, 0.1299) -- cycle;
\fill[blue!15.0, opacity=0.7] (0.1000, 1.7000, 0.1299) -- (0.1500, 1.7000, 0.1361) -- (0.1500, 1.7500, 0.1347) -- (0.1000, 1.7500, 0.1285) -- cycle;
\fill[blue!15.0, opacity=0.7] (0.1000, 1.7500, 0.1285) -- (0.1500, 1.7500, 0.1347) -- (0.1500, 1.8000, 0.1329) -- (0.1000, 1.8000, 0.1267) -- cycle;
\fill[blue!15.0, opacity=0.7] (0.1000, 1.8000, 0.1267) -- (0.1500, 1.8000, 0.1329) -- (0.1500, 1.8500, 0.1308) -- (0.1000, 1.8500, 0.1246) -- cycle;
\fill[blue!15.0, opacity=0.7] (0.1000, 1.8500, 0.1246) -- (0.1500, 1.8500, 0.1308) -- (0.1500, 1.9000, 0.1284) -- (0.1000, 1.9000, 0.1222) -- cycle;
\fill[blue!15.0, opacity=0.7] (0.1000, 1.9000, 0.1222) -- (0.1500, 1.9000, 0.1284) -- (0.1500, 1.9500, 0.1257) -- (0.1000, 1.9500, 0.1195) -- cycle;
\fill[blue!15.0, opacity=0.7] (0.1000, 1.9500, 0.1195) -- (0.1500, 1.9500, 0.1257) -- (0.1500, 2.0000, 0.1227) -- (0.1000, 2.0000, 0.1165) -- cycle;
\fill[blue!15.0, opacity=0.7] (0.1000, 2.0000, 0.1165) -- (0.1500, 2.0000, 0.1227) -- (0.1500, 2.0500, 0.1194) -- (0.1000, 2.0500, 0.1132) -- cycle;
\fill[blue!15.0, opacity=0.7] (0.1000, 2.0500, 0.1132) -- (0.1500, 2.0500, 0.1194) -- (0.1500, 2.1000, 0.1159) -- (0.1000, 2.1000, 0.1096) -- cycle;
\fill[blue!15.0, opacity=0.7] (0.1000, 2.1000, 0.1096) -- (0.1500, 2.1000, 0.1159) -- (0.1500, 2.1500, 0.1120) -- (0.1000, 2.1500, 0.1058) -- cycle;
\fill[blue!15.0, opacity=0.7] (0.1000, 2.1500, 0.1058) -- (0.1500, 2.1500, 0.1120) -- (0.1500, 2.2000, 0.1079) -- (0.1000, 2.2000, 0.1017) -- cycle;
\fill[blue!15.0, opacity=0.7] (0.1000, 2.2000, 0.1017) -- (0.1500, 2.2000, 0.1079) -- (0.1500, 2.2500, 0.1036) -- (0.1000, 2.2500, 0.0974) -- cycle;
\fill[blue!15.0, opacity=0.7] (0.1000, 2.2500, 0.0974) -- (0.1500, 2.2500, 0.1036) -- (0.1500, 2.3000, 0.0991) -- (0.1000, 2.3000, 0.0928) -- cycle;
\fill[blue!15.0, opacity=0.7] (0.1000, 2.3000, 0.0928) -- (0.1500, 2.3000, 0.0991) -- (0.1500, 2.3500, 0.0943) -- (0.1000, 2.3500, 0.0881) -- cycle;
\fill[blue!15.0, opacity=0.7] (0.1000, 2.3500, 0.0881) -- (0.1500, 2.3500, 0.0943) -- (0.1500, 2.4000, 0.0893) -- (0.1000, 2.4000, 0.0831) -- cycle;
\fill[blue!15.0, opacity=0.7] (0.1000, 2.4000, 0.0831) -- (0.1500, 2.4000, 0.0893) -- (0.1500, 2.4500, 0.0841) -- (0.1000, 2.4500, 0.0779) -- cycle;
\fill[blue!15.0, opacity=0.7] (0.1000, 2.4500, 0.0779) -- (0.1500, 2.4500, 0.0841) -- (0.1500, 2.5000, 0.0788) -- (0.1000, 2.5000, 0.0725) -- cycle;
\fill[blue!15.0, opacity=0.7] (0.1000, 2.5000, 0.0725) -- (0.1500, 2.5000, 0.0788) -- (0.1500, 2.5500, 0.0733) -- (0.1000, 2.5500, 0.0670) -- cycle;
\fill[blue!15.0, opacity=0.7] (0.1000, 2.5500, 0.0670) -- (0.1500, 2.5500, 0.0733) -- (0.1500, 2.6000, 0.0676) -- (0.1000, 2.6000, 0.0614) -- cycle;
\fill[blue!15.0, opacity=0.7] (0.1000, 2.6000, 0.0614) -- (0.1500, 2.6000, 0.0676) -- (0.1500, 2.6500, 0.0618) -- (0.1000, 2.6500, 0.0555) -- cycle;
\fill[blue!15.0, opacity=0.7] (0.1000, 2.6500, 0.0555) -- (0.1500, 2.6500, 0.0618) -- (0.1500, 2.7000, 0.0559) -- (0.1000, 2.7000, 0.0496) -- cycle;
\fill[blue!15.0, opacity=0.7] (0.1000, 2.7000, 0.0496) -- (0.1500, 2.7000, 0.0559) -- (0.1500, 2.7500, 0.0498) -- (0.1000, 2.7500, 0.0436) -- cycle;
\fill[blue!15.0, opacity=0.7] (0.1000, 2.7500, 0.0436) -- (0.1500, 2.7500, 0.0498) -- (0.1500, 2.8000, 0.0437) -- (0.1000, 2.8000, 0.0375) -- cycle;
\fill[blue!15.0, opacity=0.7] (0.1000, 2.8000, 0.0375) -- (0.1500, 2.8000, 0.0437) -- (0.1500, 2.8500, 0.0375) -- (0.1000, 2.8500, 0.0313) -- cycle;
\fill[blue!15.0, opacity=0.7] (0.1000, 2.8500, 0.0313) -- (0.1500, 2.8500, 0.0375) -- (0.1500, 2.9000, 0.0313) -- (0.1000, 2.9000, 0.0251) -- cycle;
\fill[blue!15.0, opacity=0.7] (0.1000, 2.9000, 0.0251) -- (0.1500, 2.9000, 0.0313) -- (0.1500, 2.9500, 0.0251) -- (0.1000, 2.9500, 0.0188) -- cycle;
\fill[blue!15.0, opacity=0.7] (0.1000, 2.9500, 0.0188) -- (0.1500, 2.9500, 0.0251) -- (0.1500, 3.0000, 0.0188) -- (0.1000, 3.0000, 0.0125) -- cycle;
\fill[blue!15.0, opacity=0.7] (0.1500, 0.0000, 0.0188) -- (0.2000, 0.0000, 0.0249) -- (0.2000, 0.0500, 0.0312) -- (0.1500, 0.0500, 0.0251) -- cycle;
\fill[blue!15.0, opacity=0.7] (0.1500, 0.0500, 0.0251) -- (0.2000, 0.0500, 0.0312) -- (0.2000, 0.1000, 0.0375) -- (0.1500, 0.1000, 0.0313) -- cycle;
\fill[blue!15.0, opacity=0.7] (0.1500, 0.1000, 0.0313) -- (0.2000, 0.1000, 0.0375) -- (0.2000, 0.1500, 0.0437) -- (0.1500, 0.1500, 0.0375) -- cycle;
\fill[blue!15.0, opacity=0.7] (0.1500, 0.1500, 0.0375) -- (0.2000, 0.1500, 0.0437) -- (0.2000, 0.2000, 0.0499) -- (0.1500, 0.2000, 0.0437) -- cycle;
\fill[blue!15.0, opacity=0.7] (0.1500, 0.2000, 0.0437) -- (0.2000, 0.2000, 0.0499) -- (0.2000, 0.2500, 0.0560) -- (0.1500, 0.2500, 0.0498) -- cycle;
\fill[blue!15.0, opacity=0.7] (0.1500, 0.2500, 0.0498) -- (0.2000, 0.2500, 0.0560) -- (0.2000, 0.3000, 0.0620) -- (0.1500, 0.3000, 0.0559) -- cycle;
\fill[blue!15.0, opacity=0.7] (0.1500, 0.3000, 0.0559) -- (0.2000, 0.3000, 0.0620) -- (0.2000, 0.3500, 0.0680) -- (0.1500, 0.3500, 0.0618) -- cycle;
\fill[blue!15.0, opacity=0.7] (0.1500, 0.3500, 0.0618) -- (0.2000, 0.3500, 0.0680) -- (0.2000, 0.4000, 0.0738) -- (0.1500, 0.4000, 0.0676) -- cycle;
\fill[blue!15.0, opacity=0.7] (0.1500, 0.4000, 0.0676) -- (0.2000, 0.4000, 0.0738) -- (0.2000, 0.4500, 0.0794) -- (0.1500, 0.4500, 0.0733) -- cycle;
\fill[blue!15.0, opacity=0.7] (0.1500, 0.4500, 0.0733) -- (0.2000, 0.4500, 0.0794) -- (0.2000, 0.5000, 0.0849) -- (0.1500, 0.5000, 0.0788) -- cycle;
\fill[blue!15.0, opacity=0.7] (0.1500, 0.5000, 0.0788) -- (0.2000, 0.5000, 0.0849) -- (0.2000, 0.5500, 0.0903) -- (0.1500, 0.5500, 0.0841) -- cycle;
\fill[blue!15.0, opacity=0.7] (0.1500, 0.5500, 0.0841) -- (0.2000, 0.5500, 0.0903) -- (0.2000, 0.6000, 0.0955) -- (0.1500, 0.6000, 0.0893) -- cycle;
\fill[blue!15.0, opacity=0.7] (0.1500, 0.6000, 0.0893) -- (0.2000, 0.6000, 0.0955) -- (0.2000, 0.6500, 0.1005) -- (0.1500, 0.6500, 0.0943) -- cycle;
\fill[blue!15.0, opacity=0.7] (0.1500, 0.6500, 0.0943) -- (0.2000, 0.6500, 0.1005) -- (0.2000, 0.7000, 0.1052) -- (0.1500, 0.7000, 0.0991) -- cycle;
\fill[blue!15.0, opacity=0.7] (0.1500, 0.7000, 0.0991) -- (0.2000, 0.7000, 0.1052) -- (0.2000, 0.7500, 0.1098) -- (0.1500, 0.7500, 0.1036) -- cycle;
\fill[blue!15.0, opacity=0.7] (0.1500, 0.7500, 0.1036) -- (0.2000, 0.7500, 0.1098) -- (0.2000, 0.8000, 0.1141) -- (0.1500, 0.8000, 0.1079) -- cycle;
\fill[blue!15.0, opacity=0.7] (0.1500, 0.8000, 0.1079) -- (0.2000, 0.8000, 0.1141) -- (0.2000, 0.8500, 0.1182) -- (0.1500, 0.8500, 0.1120) -- cycle;
\fill[blue!15.0, opacity=0.7] (0.1500, 0.8500, 0.1120) -- (0.2000, 0.8500, 0.1182) -- (0.2000, 0.9000, 0.1220) -- (0.1500, 0.9000, 0.1159) -- cycle;
\fill[blue!15.0, opacity=0.7] (0.1500, 0.9000, 0.1159) -- (0.2000, 0.9000, 0.1220) -- (0.2000, 0.9500, 0.1256) -- (0.1500, 0.9500, 0.1194) -- cycle;
\fill[blue!15.0, opacity=0.7] (0.1500, 0.9500, 0.1194) -- (0.2000, 0.9500, 0.1256) -- (0.2000, 1.0000, 0.1289) -- (0.1500, 1.0000, 0.1227) -- cycle;
\fill[blue!15.0, opacity=0.7] (0.1500, 1.0000, 0.1227) -- (0.2000, 1.0000, 0.1289) -- (0.2000, 1.0500, 0.1319) -- (0.1500, 1.0500, 0.1257) -- cycle;
\fill[blue!15.0, opacity=0.7] (0.1500, 1.0500, 0.1257) -- (0.2000, 1.0500, 0.1319) -- (0.2000, 1.1000, 0.1346) -- (0.1500, 1.1000, 0.1284) -- cycle;
\fill[blue!15.0, opacity=0.7] (0.1500, 1.1000, 0.1284) -- (0.2000, 1.1000, 0.1346) -- (0.2000, 1.1500, 0.1370) -- (0.1500, 1.1500, 0.1308) -- cycle;
\fill[blue!15.0, opacity=0.7] (0.1500, 1.1500, 0.1308) -- (0.2000, 1.1500, 0.1370) -- (0.2000, 1.2000, 0.1391) -- (0.1500, 1.2000, 0.1329) -- cycle;
\fill[blue!15.0, opacity=0.7] (0.1500, 1.2000, 0.1329) -- (0.2000, 1.2000, 0.1391) -- (0.2000, 1.2500, 0.1409) -- (0.1500, 1.2500, 0.1347) -- cycle;
\fill[blue!15.0, opacity=0.7] (0.1500, 1.2500, 0.1347) -- (0.2000, 1.2500, 0.1409) -- (0.2000, 1.3000, 0.1423) -- (0.1500, 1.3000, 0.1361) -- cycle;
\fill[blue!15.0, opacity=0.7] (0.1500, 1.3000, 0.1361) -- (0.2000, 1.3000, 0.1423) -- (0.2000, 1.3500, 0.1435) -- (0.1500, 1.3500, 0.1373) -- cycle;
\fill[blue!15.0, opacity=0.7] (0.1500, 1.3500, 0.1373) -- (0.2000, 1.3500, 0.1435) -- (0.2000, 1.4000, 0.1443) -- (0.1500, 1.4000, 0.1381) -- cycle;
\fill[blue!15.0, opacity=0.7] (0.1500, 1.4000, 0.1381) -- (0.2000, 1.4000, 0.1443) -- (0.2000, 1.4500, 0.1448) -- (0.1500, 1.4500, 0.1386) -- cycle;
\fill[blue!15.0, opacity=0.7] (0.1500, 1.4500, 0.1386) -- (0.2000, 1.4500, 0.1448) -- (0.2000, 1.5000, 0.1449) -- (0.1500, 1.5000, 0.1388) -- cycle;
\fill[blue!15.0, opacity=0.7] (0.1500, 1.5000, 0.1388) -- (0.2000, 1.5000, 0.1449) -- (0.2000, 1.5500, 0.1448) -- (0.1500, 1.5500, 0.1386) -- cycle;
\fill[blue!15.0, opacity=0.7] (0.1500, 1.5500, 0.1386) -- (0.2000, 1.5500, 0.1448) -- (0.2000, 1.6000, 0.1443) -- (0.1500, 1.6000, 0.1381) -- cycle;
\fill[blue!15.0, opacity=0.7] (0.1500, 1.6000, 0.1381) -- (0.2000, 1.6000, 0.1443) -- (0.2000, 1.6500, 0.1435) -- (0.1500, 1.6500, 0.1373) -- cycle;
\fill[blue!15.0, opacity=0.7] (0.1500, 1.6500, 0.1373) -- (0.2000, 1.6500, 0.1435) -- (0.2000, 1.7000, 0.1423) -- (0.1500, 1.7000, 0.1361) -- cycle;
\fill[blue!15.0, opacity=0.7] (0.1500, 1.7000, 0.1361) -- (0.2000, 1.7000, 0.1423) -- (0.2000, 1.7500, 0.1409) -- (0.1500, 1.7500, 0.1347) -- cycle;
\fill[blue!15.0, opacity=0.7] (0.1500, 1.7500, 0.1347) -- (0.2000, 1.7500, 0.1409) -- (0.2000, 1.8000, 0.1391) -- (0.1500, 1.8000, 0.1329) -- cycle;
\fill[blue!15.0, opacity=0.7] (0.1500, 1.8000, 0.1329) -- (0.2000, 1.8000, 0.1391) -- (0.2000, 1.8500, 0.1370) -- (0.1500, 1.8500, 0.1308) -- cycle;
\fill[blue!15.0, opacity=0.7] (0.1500, 1.8500, 0.1308) -- (0.2000, 1.8500, 0.1370) -- (0.2000, 1.9000, 0.1346) -- (0.1500, 1.9000, 0.1284) -- cycle;
\fill[blue!15.0, opacity=0.7] (0.1500, 1.9000, 0.1284) -- (0.2000, 1.9000, 0.1346) -- (0.2000, 1.9500, 0.1319) -- (0.1500, 1.9500, 0.1257) -- cycle;
\fill[blue!15.0, opacity=0.7] (0.1500, 1.9500, 0.1257) -- (0.2000, 1.9500, 0.1319) -- (0.2000, 2.0000, 0.1289) -- (0.1500, 2.0000, 0.1227) -- cycle;
\fill[blue!15.0, opacity=0.7] (0.1500, 2.0000, 0.1227) -- (0.2000, 2.0000, 0.1289) -- (0.2000, 2.0500, 0.1256) -- (0.1500, 2.0500, 0.1194) -- cycle;
\fill[blue!15.0, opacity=0.7] (0.1500, 2.0500, 0.1194) -- (0.2000, 2.0500, 0.1256) -- (0.2000, 2.1000, 0.1220) -- (0.1500, 2.1000, 0.1159) -- cycle;
\fill[blue!15.0, opacity=0.7] (0.1500, 2.1000, 0.1159) -- (0.2000, 2.1000, 0.1220) -- (0.2000, 2.1500, 0.1182) -- (0.1500, 2.1500, 0.1120) -- cycle;
\fill[blue!15.0, opacity=0.7] (0.1500, 2.1500, 0.1120) -- (0.2000, 2.1500, 0.1182) -- (0.2000, 2.2000, 0.1141) -- (0.1500, 2.2000, 0.1079) -- cycle;
\fill[blue!15.0, opacity=0.7] (0.1500, 2.2000, 0.1079) -- (0.2000, 2.2000, 0.1141) -- (0.2000, 2.2500, 0.1098) -- (0.1500, 2.2500, 0.1036) -- cycle;
\fill[blue!15.0, opacity=0.7] (0.1500, 2.2500, 0.1036) -- (0.2000, 2.2500, 0.1098) -- (0.2000, 2.3000, 0.1052) -- (0.1500, 2.3000, 0.0991) -- cycle;
\fill[blue!15.0, opacity=0.7] (0.1500, 2.3000, 0.0991) -- (0.2000, 2.3000, 0.1052) -- (0.2000, 2.3500, 0.1005) -- (0.1500, 2.3500, 0.0943) -- cycle;
\fill[blue!15.0, opacity=0.7] (0.1500, 2.3500, 0.0943) -- (0.2000, 2.3500, 0.1005) -- (0.2000, 2.4000, 0.0955) -- (0.1500, 2.4000, 0.0893) -- cycle;
\fill[blue!15.0, opacity=0.7] (0.1500, 2.4000, 0.0893) -- (0.2000, 2.4000, 0.0955) -- (0.2000, 2.4500, 0.0903) -- (0.1500, 2.4500, 0.0841) -- cycle;
\fill[blue!15.0, opacity=0.7] (0.1500, 2.4500, 0.0841) -- (0.2000, 2.4500, 0.0903) -- (0.2000, 2.5000, 0.0849) -- (0.1500, 2.5000, 0.0788) -- cycle;
\fill[blue!15.0, opacity=0.7] (0.1500, 2.5000, 0.0788) -- (0.2000, 2.5000, 0.0849) -- (0.2000, 2.5500, 0.0794) -- (0.1500, 2.5500, 0.0733) -- cycle;
\fill[blue!15.0, opacity=0.7] (0.1500, 2.5500, 0.0733) -- (0.2000, 2.5500, 0.0794) -- (0.2000, 2.6000, 0.0738) -- (0.1500, 2.6000, 0.0676) -- cycle;
\fill[blue!15.0, opacity=0.7] (0.1500, 2.6000, 0.0676) -- (0.2000, 2.6000, 0.0738) -- (0.2000, 2.6500, 0.0680) -- (0.1500, 2.6500, 0.0618) -- cycle;
\fill[blue!15.0, opacity=0.7] (0.1500, 2.6500, 0.0618) -- (0.2000, 2.6500, 0.0680) -- (0.2000, 2.7000, 0.0620) -- (0.1500, 2.7000, 0.0559) -- cycle;
\fill[blue!15.0, opacity=0.7] (0.1500, 2.7000, 0.0559) -- (0.2000, 2.7000, 0.0620) -- (0.2000, 2.7500, 0.0560) -- (0.1500, 2.7500, 0.0498) -- cycle;
\fill[blue!15.0, opacity=0.7] (0.1500, 2.7500, 0.0498) -- (0.2000, 2.7500, 0.0560) -- (0.2000, 2.8000, 0.0499) -- (0.1500, 2.8000, 0.0437) -- cycle;
\fill[blue!15.0, opacity=0.7] (0.1500, 2.8000, 0.0437) -- (0.2000, 2.8000, 0.0499) -- (0.2000, 2.8500, 0.0437) -- (0.1500, 2.8500, 0.0375) -- cycle;
\fill[blue!15.0, opacity=0.7] (0.1500, 2.8500, 0.0375) -- (0.2000, 2.8500, 0.0437) -- (0.2000, 2.9000, 0.0375) -- (0.1500, 2.9000, 0.0313) -- cycle;
\fill[blue!15.0, opacity=0.7] (0.1500, 2.9000, 0.0313) -- (0.2000, 2.9000, 0.0375) -- (0.2000, 2.9500, 0.0312) -- (0.1500, 2.9500, 0.0251) -- cycle;
\fill[blue!15.0, opacity=0.7] (0.1500, 2.9500, 0.0251) -- (0.2000, 2.9500, 0.0312) -- (0.2000, 3.0000, 0.0249) -- (0.1500, 3.0000, 0.0188) -- cycle;
\fill[blue!15.0, opacity=0.7] (0.2000, 0.0000, 0.0249) -- (0.2500, 0.0000, 0.0311) -- (0.2500, 0.0500, 0.0373) -- (0.2000, 0.0500, 0.0312) -- cycle;
\fill[blue!15.0, opacity=0.7] (0.2000, 0.0500, 0.0312) -- (0.2500, 0.0500, 0.0373) -- (0.2500, 0.1000, 0.0436) -- (0.2000, 0.1000, 0.0375) -- cycle;
\fill[blue!15.0, opacity=0.7] (0.2000, 0.1000, 0.0375) -- (0.2500, 0.1000, 0.0436) -- (0.2500, 0.1500, 0.0498) -- (0.2000, 0.1500, 0.0437) -- cycle;
\fill[blue!15.0, opacity=0.7] (0.2000, 0.1500, 0.0437) -- (0.2500, 0.1500, 0.0498) -- (0.2500, 0.2000, 0.0560) -- (0.2000, 0.2000, 0.0499) -- cycle;
\fill[blue!15.0, opacity=0.7] (0.2000, 0.2000, 0.0499) -- (0.2500, 0.2000, 0.0560) -- (0.2500, 0.2500, 0.0621) -- (0.2000, 0.2500, 0.0560) -- cycle;
\fill[blue!15.0, opacity=0.7] (0.2000, 0.2500, 0.0560) -- (0.2500, 0.2500, 0.0621) -- (0.2500, 0.3000, 0.0681) -- (0.2000, 0.3000, 0.0620) -- cycle;
\fill[blue!15.0, opacity=0.7] (0.2000, 0.3000, 0.0620) -- (0.2500, 0.3000, 0.0681) -- (0.2500, 0.3500, 0.0741) -- (0.2000, 0.3500, 0.0680) -- cycle;
\fill[blue!15.0, opacity=0.7] (0.2000, 0.3500, 0.0680) -- (0.2500, 0.3500, 0.0741) -- (0.2500, 0.4000, 0.0799) -- (0.2000, 0.4000, 0.0738) -- cycle;
\fill[blue!15.0, opacity=0.7] (0.2000, 0.4000, 0.0738) -- (0.2500, 0.4000, 0.0799) -- (0.2500, 0.4500, 0.0855) -- (0.2000, 0.4500, 0.0794) -- cycle;
\fill[blue!15.0, opacity=0.7] (0.2000, 0.4500, 0.0794) -- (0.2500, 0.4500, 0.0855) -- (0.2500, 0.5000, 0.0911) -- (0.2000, 0.5000, 0.0849) -- cycle;
\fill[blue!15.0, opacity=0.7] (0.2000, 0.5000, 0.0849) -- (0.2500, 0.5000, 0.0911) -- (0.2500, 0.5500, 0.0964) -- (0.2000, 0.5500, 0.0903) -- cycle;
\fill[blue!15.0, opacity=0.7] (0.2000, 0.5500, 0.0903) -- (0.2500, 0.5500, 0.0964) -- (0.2500, 0.6000, 0.1016) -- (0.2000, 0.6000, 0.0955) -- cycle;
\fill[blue!15.0, opacity=0.7] (0.2000, 0.6000, 0.0955) -- (0.2500, 0.6000, 0.1016) -- (0.2500, 0.6500, 0.1066) -- (0.2000, 0.6500, 0.1005) -- cycle;
\fill[blue!15.0, opacity=0.7] (0.2000, 0.6500, 0.1005) -- (0.2500, 0.6500, 0.1066) -- (0.2500, 0.7000, 0.1114) -- (0.2000, 0.7000, 0.1052) -- cycle;
\fill[blue!15.0, opacity=0.7] (0.2000, 0.7000, 0.1052) -- (0.2500, 0.7000, 0.1114) -- (0.2500, 0.7500, 0.1159) -- (0.2000, 0.7500, 0.1098) -- cycle;
\fill[blue!15.0, opacity=0.7] (0.2000, 0.7500, 0.1098) -- (0.2500, 0.7500, 0.1159) -- (0.2500, 0.8000, 0.1202) -- (0.2000, 0.8000, 0.1141) -- cycle;
\fill[blue!15.0, opacity=0.7] (0.2000, 0.8000, 0.1141) -- (0.2500, 0.8000, 0.1202) -- (0.2500, 0.8500, 0.1243) -- (0.2000, 0.8500, 0.1182) -- cycle;
\fill[blue!15.0, opacity=0.7] (0.2000, 0.8500, 0.1182) -- (0.2500, 0.8500, 0.1243) -- (0.2500, 0.9000, 0.1281) -- (0.2000, 0.9000, 0.1220) -- cycle;
\fill[blue!15.0, opacity=0.7] (0.2000, 0.9000, 0.1220) -- (0.2500, 0.9000, 0.1281) -- (0.2500, 0.9500, 0.1317) -- (0.2000, 0.9500, 0.1256) -- cycle;
\fill[blue!15.0, opacity=0.7] (0.2000, 0.9500, 0.1256) -- (0.2500, 0.9500, 0.1317) -- (0.2500, 1.0000, 0.1350) -- (0.2000, 1.0000, 0.1289) -- cycle;
\fill[blue!15.0, opacity=0.7] (0.2000, 1.0000, 0.1289) -- (0.2500, 1.0000, 0.1350) -- (0.2500, 1.0500, 0.1380) -- (0.2000, 1.0500, 0.1319) -- cycle;
\fill[blue!15.0, opacity=0.7] (0.2000, 1.0500, 0.1319) -- (0.2500, 1.0500, 0.1380) -- (0.2500, 1.1000, 0.1407) -- (0.2000, 1.1000, 0.1346) -- cycle;
\fill[blue!15.0, opacity=0.7] (0.2000, 1.1000, 0.1346) -- (0.2500, 1.1000, 0.1407) -- (0.2500, 1.1500, 0.1431) -- (0.2000, 1.1500, 0.1370) -- cycle;
\fill[blue!15.0, opacity=0.7] (0.2000, 1.1500, 0.1370) -- (0.2500, 1.1500, 0.1431) -- (0.2500, 1.2000, 0.1452) -- (0.2000, 1.2000, 0.1391) -- cycle;
\fill[blue!15.0, opacity=0.7] (0.2000, 1.2000, 0.1391) -- (0.2500, 1.2000, 0.1452) -- (0.2500, 1.2500, 0.1470) -- (0.2000, 1.2500, 0.1409) -- cycle;
\fill[blue!15.0, opacity=0.7] (0.2000, 1.2500, 0.1409) -- (0.2500, 1.2500, 0.1470) -- (0.2500, 1.3000, 0.1484) -- (0.2000, 1.3000, 0.1423) -- cycle;
\fill[blue!15.0, opacity=0.7] (0.2000, 1.3000, 0.1423) -- (0.2500, 1.3000, 0.1484) -- (0.2500, 1.3500, 0.1496) -- (0.2000, 1.3500, 0.1435) -- cycle;
\fill[blue!15.0, opacity=0.7] (0.2000, 1.3500, 0.1435) -- (0.2500, 1.3500, 0.1496) -- (0.2500, 1.4000, 0.1504) -- (0.2000, 1.4000, 0.1443) -- cycle;
\fill[blue!15.0, opacity=0.7] (0.2000, 1.4000, 0.1443) -- (0.2500, 1.4000, 0.1504) -- (0.2500, 1.4500, 0.1509) -- (0.2000, 1.4500, 0.1448) -- cycle;
\fill[blue!15.0, opacity=0.7] (0.2000, 1.4500, 0.1448) -- (0.2500, 1.4500, 0.1509) -- (0.2500, 1.5000, 0.1511) -- (0.2000, 1.5000, 0.1449) -- cycle;
\fill[blue!15.0, opacity=0.7] (0.2000, 1.5000, 0.1449) -- (0.2500, 1.5000, 0.1511) -- (0.2500, 1.5500, 0.1509) -- (0.2000, 1.5500, 0.1448) -- cycle;
\fill[blue!15.0, opacity=0.7] (0.2000, 1.5500, 0.1448) -- (0.2500, 1.5500, 0.1509) -- (0.2500, 1.6000, 0.1504) -- (0.2000, 1.6000, 0.1443) -- cycle;
\fill[blue!15.0, opacity=0.7] (0.2000, 1.6000, 0.1443) -- (0.2500, 1.6000, 0.1504) -- (0.2500, 1.6500, 0.1496) -- (0.2000, 1.6500, 0.1435) -- cycle;
\fill[blue!15.0, opacity=0.7] (0.2000, 1.6500, 0.1435) -- (0.2500, 1.6500, 0.1496) -- (0.2500, 1.7000, 0.1484) -- (0.2000, 1.7000, 0.1423) -- cycle;
\fill[blue!15.0, opacity=0.7] (0.2000, 1.7000, 0.1423) -- (0.2500, 1.7000, 0.1484) -- (0.2500, 1.7500, 0.1470) -- (0.2000, 1.7500, 0.1409) -- cycle;
\fill[blue!15.0, opacity=0.7] (0.2000, 1.7500, 0.1409) -- (0.2500, 1.7500, 0.1470) -- (0.2500, 1.8000, 0.1452) -- (0.2000, 1.8000, 0.1391) -- cycle;
\fill[blue!15.0, opacity=0.7] (0.2000, 1.8000, 0.1391) -- (0.2500, 1.8000, 0.1452) -- (0.2500, 1.8500, 0.1431) -- (0.2000, 1.8500, 0.1370) -- cycle;
\fill[blue!15.0, opacity=0.7] (0.2000, 1.8500, 0.1370) -- (0.2500, 1.8500, 0.1431) -- (0.2500, 1.9000, 0.1407) -- (0.2000, 1.9000, 0.1346) -- cycle;
\fill[blue!15.0, opacity=0.7] (0.2000, 1.9000, 0.1346) -- (0.2500, 1.9000, 0.1407) -- (0.2500, 1.9500, 0.1380) -- (0.2000, 1.9500, 0.1319) -- cycle;
\fill[blue!15.0, opacity=0.7] (0.2000, 1.9500, 0.1319) -- (0.2500, 1.9500, 0.1380) -- (0.2500, 2.0000, 0.1350) -- (0.2000, 2.0000, 0.1289) -- cycle;
\fill[blue!15.0, opacity=0.7] (0.2000, 2.0000, 0.1289) -- (0.2500, 2.0000, 0.1350) -- (0.2500, 2.0500, 0.1317) -- (0.2000, 2.0500, 0.1256) -- cycle;
\fill[blue!15.0, opacity=0.7] (0.2000, 2.0500, 0.1256) -- (0.2500, 2.0500, 0.1317) -- (0.2500, 2.1000, 0.1281) -- (0.2000, 2.1000, 0.1220) -- cycle;
\fill[blue!15.0, opacity=0.7] (0.2000, 2.1000, 0.1220) -- (0.2500, 2.1000, 0.1281) -- (0.2500, 2.1500, 0.1243) -- (0.2000, 2.1500, 0.1182) -- cycle;
\fill[blue!15.0, opacity=0.7] (0.2000, 2.1500, 0.1182) -- (0.2500, 2.1500, 0.1243) -- (0.2500, 2.2000, 0.1202) -- (0.2000, 2.2000, 0.1141) -- cycle;
\fill[blue!15.0, opacity=0.7] (0.2000, 2.2000, 0.1141) -- (0.2500, 2.2000, 0.1202) -- (0.2500, 2.2500, 0.1159) -- (0.2000, 2.2500, 0.1098) -- cycle;
\fill[blue!15.0, opacity=0.7] (0.2000, 2.2500, 0.1098) -- (0.2500, 2.2500, 0.1159) -- (0.2500, 2.3000, 0.1114) -- (0.2000, 2.3000, 0.1052) -- cycle;
\fill[blue!15.0, opacity=0.7] (0.2000, 2.3000, 0.1052) -- (0.2500, 2.3000, 0.1114) -- (0.2500, 2.3500, 0.1066) -- (0.2000, 2.3500, 0.1005) -- cycle;
\fill[blue!15.0, opacity=0.7] (0.2000, 2.3500, 0.1005) -- (0.2500, 2.3500, 0.1066) -- (0.2500, 2.4000, 0.1016) -- (0.2000, 2.4000, 0.0955) -- cycle;
\fill[blue!15.0, opacity=0.7] (0.2000, 2.4000, 0.0955) -- (0.2500, 2.4000, 0.1016) -- (0.2500, 2.4500, 0.0964) -- (0.2000, 2.4500, 0.0903) -- cycle;
\fill[blue!15.0, opacity=0.7] (0.2000, 2.4500, 0.0903) -- (0.2500, 2.4500, 0.0964) -- (0.2500, 2.5000, 0.0911) -- (0.2000, 2.5000, 0.0849) -- cycle;
\fill[blue!15.0, opacity=0.7] (0.2000, 2.5000, 0.0849) -- (0.2500, 2.5000, 0.0911) -- (0.2500, 2.5500, 0.0855) -- (0.2000, 2.5500, 0.0794) -- cycle;
\fill[blue!15.0, opacity=0.7] (0.2000, 2.5500, 0.0794) -- (0.2500, 2.5500, 0.0855) -- (0.2500, 2.6000, 0.0799) -- (0.2000, 2.6000, 0.0738) -- cycle;
\fill[blue!15.0, opacity=0.7] (0.2000, 2.6000, 0.0738) -- (0.2500, 2.6000, 0.0799) -- (0.2500, 2.6500, 0.0741) -- (0.2000, 2.6500, 0.0680) -- cycle;
\fill[blue!15.0, opacity=0.7] (0.2000, 2.6500, 0.0680) -- (0.2500, 2.6500, 0.0741) -- (0.2500, 2.7000, 0.0681) -- (0.2000, 2.7000, 0.0620) -- cycle;
\fill[blue!15.0, opacity=0.7] (0.2000, 2.7000, 0.0620) -- (0.2500, 2.7000, 0.0681) -- (0.2500, 2.7500, 0.0621) -- (0.2000, 2.7500, 0.0560) -- cycle;
\fill[blue!15.0, opacity=0.7] (0.2000, 2.7500, 0.0560) -- (0.2500, 2.7500, 0.0621) -- (0.2500, 2.8000, 0.0560) -- (0.2000, 2.8000, 0.0499) -- cycle;
\fill[blue!15.0, opacity=0.7] (0.2000, 2.8000, 0.0499) -- (0.2500, 2.8000, 0.0560) -- (0.2500, 2.8500, 0.0498) -- (0.2000, 2.8500, 0.0437) -- cycle;
\fill[blue!15.0, opacity=0.7] (0.2000, 2.8500, 0.0437) -- (0.2500, 2.8500, 0.0498) -- (0.2500, 2.9000, 0.0436) -- (0.2000, 2.9000, 0.0375) -- cycle;
\fill[blue!15.0, opacity=0.7] (0.2000, 2.9000, 0.0375) -- (0.2500, 2.9000, 0.0436) -- (0.2500, 2.9500, 0.0373) -- (0.2000, 2.9500, 0.0312) -- cycle;
\fill[blue!15.0, opacity=0.7] (0.2000, 2.9500, 0.0312) -- (0.2500, 2.9500, 0.0373) -- (0.2500, 3.0000, 0.0311) -- (0.2000, 3.0000, 0.0249) -- cycle;
\fill[blue!15.0, opacity=0.7] (0.2500, 0.0000, 0.0311) -- (0.3000, 0.0000, 0.0371) -- (0.3000, 0.0500, 0.0434) -- (0.2500, 0.0500, 0.0373) -- cycle;
\fill[blue!15.0, opacity=0.7] (0.2500, 0.0500, 0.0373) -- (0.3000, 0.0500, 0.0434) -- (0.3000, 0.1000, 0.0496) -- (0.2500, 0.1000, 0.0436) -- cycle;
\fill[blue!15.0, opacity=0.7] (0.2500, 0.1000, 0.0436) -- (0.3000, 0.1000, 0.0496) -- (0.3000, 0.1500, 0.0559) -- (0.2500, 0.1500, 0.0498) -- cycle;
\fill[blue!15.0, opacity=0.7] (0.2500, 0.1500, 0.0498) -- (0.3000, 0.1500, 0.0559) -- (0.3000, 0.2000, 0.0620) -- (0.2500, 0.2000, 0.0560) -- cycle;
\fill[blue!15.0, opacity=0.7] (0.2500, 0.2000, 0.0560) -- (0.3000, 0.2000, 0.0620) -- (0.3000, 0.2500, 0.0681) -- (0.2500, 0.2500, 0.0621) -- cycle;
\fill[blue!15.0, opacity=0.7] (0.2500, 0.2500, 0.0621) -- (0.3000, 0.2500, 0.0681) -- (0.3000, 0.3000, 0.0742) -- (0.2500, 0.3000, 0.0681) -- cycle;
\fill[blue!15.0, opacity=0.7] (0.2500, 0.3000, 0.0681) -- (0.3000, 0.3000, 0.0742) -- (0.3000, 0.3500, 0.0801) -- (0.2500, 0.3500, 0.0741) -- cycle;
\fill[blue!15.0, opacity=0.7] (0.2500, 0.3500, 0.0741) -- (0.3000, 0.3500, 0.0801) -- (0.3000, 0.4000, 0.0859) -- (0.2500, 0.4000, 0.0799) -- cycle;
\fill[blue!15.0, opacity=0.7] (0.2500, 0.4000, 0.0799) -- (0.3000, 0.4000, 0.0859) -- (0.3000, 0.4500, 0.0916) -- (0.2500, 0.4500, 0.0855) -- cycle;
\fill[blue!15.0, opacity=0.7] (0.2500, 0.4500, 0.0855) -- (0.3000, 0.4500, 0.0916) -- (0.3000, 0.5000, 0.0971) -- (0.2500, 0.5000, 0.0911) -- cycle;
\fill[blue!15.0, opacity=0.7] (0.2500, 0.5000, 0.0911) -- (0.3000, 0.5000, 0.0971) -- (0.3000, 0.5500, 0.1024) -- (0.2500, 0.5500, 0.0964) -- cycle;
\fill[blue!15.0, opacity=0.7] (0.2500, 0.5500, 0.0964) -- (0.3000, 0.5500, 0.1024) -- (0.3000, 0.6000, 0.1076) -- (0.2500, 0.6000, 0.1016) -- cycle;
\fill[blue!15.0, opacity=0.7] (0.2500, 0.6000, 0.1016) -- (0.3000, 0.6000, 0.1076) -- (0.3000, 0.6500, 0.1126) -- (0.2500, 0.6500, 0.1066) -- cycle;
\fill[blue!15.0, opacity=0.7] (0.2500, 0.6500, 0.1066) -- (0.3000, 0.6500, 0.1126) -- (0.3000, 0.7000, 0.1174) -- (0.2500, 0.7000, 0.1114) -- cycle;
\fill[blue!15.0, opacity=0.7] (0.2500, 0.7000, 0.1114) -- (0.3000, 0.7000, 0.1174) -- (0.3000, 0.7500, 0.1219) -- (0.2500, 0.7500, 0.1159) -- cycle;
\fill[blue!15.0, opacity=0.7] (0.2500, 0.7500, 0.1159) -- (0.3000, 0.7500, 0.1219) -- (0.3000, 0.8000, 0.1263) -- (0.2500, 0.8000, 0.1202) -- cycle;
\fill[blue!15.0, opacity=0.7] (0.2500, 0.8000, 0.1202) -- (0.3000, 0.8000, 0.1263) -- (0.3000, 0.8500, 0.1303) -- (0.2500, 0.8500, 0.1243) -- cycle;
\fill[blue!15.0, opacity=0.7] (0.2500, 0.8500, 0.1243) -- (0.3000, 0.8500, 0.1303) -- (0.3000, 0.9000, 0.1342) -- (0.2500, 0.9000, 0.1281) -- cycle;
\fill[blue!15.0, opacity=0.7] (0.2500, 0.9000, 0.1281) -- (0.3000, 0.9000, 0.1342) -- (0.3000, 0.9500, 0.1377) -- (0.2500, 0.9500, 0.1317) -- cycle;
\fill[blue!15.0, opacity=0.7] (0.2500, 0.9500, 0.1317) -- (0.3000, 0.9500, 0.1377) -- (0.3000, 1.0000, 0.1410) -- (0.2500, 1.0000, 0.1350) -- cycle;
\fill[blue!15.0, opacity=0.7] (0.2500, 1.0000, 0.1350) -- (0.3000, 1.0000, 0.1410) -- (0.3000, 1.0500, 0.1440) -- (0.2500, 1.0500, 0.1380) -- cycle;
\fill[blue!15.0, opacity=0.7] (0.2500, 1.0500, 0.1380) -- (0.3000, 1.0500, 0.1440) -- (0.3000, 1.1000, 0.1467) -- (0.2500, 1.1000, 0.1407) -- cycle;
\fill[blue!15.0, opacity=0.7] (0.2500, 1.1000, 0.1407) -- (0.3000, 1.1000, 0.1467) -- (0.3000, 1.1500, 0.1491) -- (0.2500, 1.1500, 0.1431) -- cycle;
\fill[blue!15.0, opacity=0.7] (0.2500, 1.1500, 0.1431) -- (0.3000, 1.1500, 0.1491) -- (0.3000, 1.2000, 0.1512) -- (0.2500, 1.2000, 0.1452) -- cycle;
\fill[blue!15.0, opacity=0.7] (0.2500, 1.2000, 0.1452) -- (0.3000, 1.2000, 0.1512) -- (0.3000, 1.2500, 0.1530) -- (0.2500, 1.2500, 0.1470) -- cycle;
\fill[blue!15.0, opacity=0.7] (0.2500, 1.2500, 0.1470) -- (0.3000, 1.2500, 0.1530) -- (0.3000, 1.3000, 0.1545) -- (0.2500, 1.3000, 0.1484) -- cycle;
\fill[blue!15.0, opacity=0.7] (0.2500, 1.3000, 0.1484) -- (0.3000, 1.3000, 0.1545) -- (0.3000, 1.3500, 0.1556) -- (0.2500, 1.3500, 0.1496) -- cycle;
\fill[blue!15.0, opacity=0.7] (0.2500, 1.3500, 0.1496) -- (0.3000, 1.3500, 0.1556) -- (0.3000, 1.4000, 0.1564) -- (0.2500, 1.4000, 0.1504) -- cycle;
\fill[blue!15.0, opacity=0.7] (0.2500, 1.4000, 0.1504) -- (0.3000, 1.4000, 0.1564) -- (0.3000, 1.4500, 0.1569) -- (0.2500, 1.4500, 0.1509) -- cycle;
\fill[blue!15.0, opacity=0.7] (0.2500, 1.4500, 0.1509) -- (0.3000, 1.4500, 0.1569) -- (0.3000, 1.5000, 0.1571) -- (0.2500, 1.5000, 0.1511) -- cycle;
\fill[blue!15.0, opacity=0.7] (0.2500, 1.5000, 0.1511) -- (0.3000, 1.5000, 0.1571) -- (0.3000, 1.5500, 0.1569) -- (0.2500, 1.5500, 0.1509) -- cycle;
\fill[blue!15.0, opacity=0.7] (0.2500, 1.5500, 0.1509) -- (0.3000, 1.5500, 0.1569) -- (0.3000, 1.6000, 0.1564) -- (0.2500, 1.6000, 0.1504) -- cycle;
\fill[blue!15.0, opacity=0.7] (0.2500, 1.6000, 0.1504) -- (0.3000, 1.6000, 0.1564) -- (0.3000, 1.6500, 0.1556) -- (0.2500, 1.6500, 0.1496) -- cycle;
\fill[blue!15.0, opacity=0.7] (0.2500, 1.6500, 0.1496) -- (0.3000, 1.6500, 0.1556) -- (0.3000, 1.7000, 0.1545) -- (0.2500, 1.7000, 0.1484) -- cycle;
\fill[blue!15.0, opacity=0.7] (0.2500, 1.7000, 0.1484) -- (0.3000, 1.7000, 0.1545) -- (0.3000, 1.7500, 0.1530) -- (0.2500, 1.7500, 0.1470) -- cycle;
\fill[blue!15.0, opacity=0.7] (0.2500, 1.7500, 0.1470) -- (0.3000, 1.7500, 0.1530) -- (0.3000, 1.8000, 0.1512) -- (0.2500, 1.8000, 0.1452) -- cycle;
\fill[blue!15.0, opacity=0.7] (0.2500, 1.8000, 0.1452) -- (0.3000, 1.8000, 0.1512) -- (0.3000, 1.8500, 0.1491) -- (0.2500, 1.8500, 0.1431) -- cycle;
\fill[blue!15.0, opacity=0.7] (0.2500, 1.8500, 0.1431) -- (0.3000, 1.8500, 0.1491) -- (0.3000, 1.9000, 0.1467) -- (0.2500, 1.9000, 0.1407) -- cycle;
\fill[blue!15.0, opacity=0.7] (0.2500, 1.9000, 0.1407) -- (0.3000, 1.9000, 0.1467) -- (0.3000, 1.9500, 0.1440) -- (0.2500, 1.9500, 0.1380) -- cycle;
\fill[blue!15.0, opacity=0.7] (0.2500, 1.9500, 0.1380) -- (0.3000, 1.9500, 0.1440) -- (0.3000, 2.0000, 0.1410) -- (0.2500, 2.0000, 0.1350) -- cycle;
\fill[blue!15.0, opacity=0.7] (0.2500, 2.0000, 0.1350) -- (0.3000, 2.0000, 0.1410) -- (0.3000, 2.0500, 0.1377) -- (0.2500, 2.0500, 0.1317) -- cycle;
\fill[blue!15.0, opacity=0.7] (0.2500, 2.0500, 0.1317) -- (0.3000, 2.0500, 0.1377) -- (0.3000, 2.1000, 0.1342) -- (0.2500, 2.1000, 0.1281) -- cycle;
\fill[blue!15.0, opacity=0.7] (0.2500, 2.1000, 0.1281) -- (0.3000, 2.1000, 0.1342) -- (0.3000, 2.1500, 0.1303) -- (0.2500, 2.1500, 0.1243) -- cycle;
\fill[blue!15.0, opacity=0.7] (0.2500, 2.1500, 0.1243) -- (0.3000, 2.1500, 0.1303) -- (0.3000, 2.2000, 0.1263) -- (0.2500, 2.2000, 0.1202) -- cycle;
\fill[blue!15.0, opacity=0.7] (0.2500, 2.2000, 0.1202) -- (0.3000, 2.2000, 0.1263) -- (0.3000, 2.2500, 0.1219) -- (0.2500, 2.2500, 0.1159) -- cycle;
\fill[blue!15.0, opacity=0.7] (0.2500, 2.2500, 0.1159) -- (0.3000, 2.2500, 0.1219) -- (0.3000, 2.3000, 0.1174) -- (0.2500, 2.3000, 0.1114) -- cycle;
\fill[blue!15.0, opacity=0.7] (0.2500, 2.3000, 0.1114) -- (0.3000, 2.3000, 0.1174) -- (0.3000, 2.3500, 0.1126) -- (0.2500, 2.3500, 0.1066) -- cycle;
\fill[blue!15.0, opacity=0.7] (0.2500, 2.3500, 0.1066) -- (0.3000, 2.3500, 0.1126) -- (0.3000, 2.4000, 0.1076) -- (0.2500, 2.4000, 0.1016) -- cycle;
\fill[blue!15.0, opacity=0.7] (0.2500, 2.4000, 0.1016) -- (0.3000, 2.4000, 0.1076) -- (0.3000, 2.4500, 0.1024) -- (0.2500, 2.4500, 0.0964) -- cycle;
\fill[blue!15.0, opacity=0.7] (0.2500, 2.4500, 0.0964) -- (0.3000, 2.4500, 0.1024) -- (0.3000, 2.5000, 0.0971) -- (0.2500, 2.5000, 0.0911) -- cycle;
\fill[blue!15.0, opacity=0.7] (0.2500, 2.5000, 0.0911) -- (0.3000, 2.5000, 0.0971) -- (0.3000, 2.5500, 0.0916) -- (0.2500, 2.5500, 0.0855) -- cycle;
\fill[blue!15.0, opacity=0.7] (0.2500, 2.5500, 0.0855) -- (0.3000, 2.5500, 0.0916) -- (0.3000, 2.6000, 0.0859) -- (0.2500, 2.6000, 0.0799) -- cycle;
\fill[blue!15.0, opacity=0.7] (0.2500, 2.6000, 0.0799) -- (0.3000, 2.6000, 0.0859) -- (0.3000, 2.6500, 0.0801) -- (0.2500, 2.6500, 0.0741) -- cycle;
\fill[blue!15.0, opacity=0.7] (0.2500, 2.6500, 0.0741) -- (0.3000, 2.6500, 0.0801) -- (0.3000, 2.7000, 0.0742) -- (0.2500, 2.7000, 0.0681) -- cycle;
\fill[blue!15.0, opacity=0.7] (0.2500, 2.7000, 0.0681) -- (0.3000, 2.7000, 0.0742) -- (0.3000, 2.7500, 0.0681) -- (0.2500, 2.7500, 0.0621) -- cycle;
\fill[blue!15.0, opacity=0.7] (0.2500, 2.7500, 0.0621) -- (0.3000, 2.7500, 0.0681) -- (0.3000, 2.8000, 0.0620) -- (0.2500, 2.8000, 0.0560) -- cycle;
\fill[blue!15.0, opacity=0.7] (0.2500, 2.8000, 0.0560) -- (0.3000, 2.8000, 0.0620) -- (0.3000, 2.8500, 0.0559) -- (0.2500, 2.8500, 0.0498) -- cycle;
\fill[blue!15.0, opacity=0.7] (0.2500, 2.8500, 0.0498) -- (0.3000, 2.8500, 0.0559) -- (0.3000, 2.9000, 0.0496) -- (0.2500, 2.9000, 0.0436) -- cycle;
\fill[blue!15.0, opacity=0.7] (0.2500, 2.9000, 0.0436) -- (0.3000, 2.9000, 0.0496) -- (0.3000, 2.9500, 0.0434) -- (0.2500, 2.9500, 0.0373) -- cycle;
\fill[blue!15.0, opacity=0.7] (0.2500, 2.9500, 0.0373) -- (0.3000, 2.9500, 0.0434) -- (0.3000, 3.0000, 0.0371) -- (0.2500, 3.0000, 0.0311) -- cycle;
\fill[blue!15.0, opacity=0.7] (0.3000, 0.0000, 0.0371) -- (0.3500, 0.0000, 0.0430) -- (0.3500, 0.0500, 0.0493) -- (0.3000, 0.0500, 0.0434) -- cycle;
\fill[blue!15.0, opacity=0.7] (0.3000, 0.0500, 0.0434) -- (0.3500, 0.0500, 0.0493) -- (0.3500, 0.1000, 0.0555) -- (0.3000, 0.1000, 0.0496) -- cycle;
\fill[blue!15.0, opacity=0.7] (0.3000, 0.1000, 0.0496) -- (0.3500, 0.1000, 0.0555) -- (0.3500, 0.1500, 0.0618) -- (0.3000, 0.1500, 0.0559) -- cycle;
\fill[blue!15.0, opacity=0.7] (0.3000, 0.1500, 0.0559) -- (0.3500, 0.1500, 0.0618) -- (0.3500, 0.2000, 0.0680) -- (0.3000, 0.2000, 0.0620) -- cycle;
\fill[blue!15.0, opacity=0.7] (0.3000, 0.2000, 0.0620) -- (0.3500, 0.2000, 0.0680) -- (0.3500, 0.2500, 0.0741) -- (0.3000, 0.2500, 0.0681) -- cycle;
\fill[blue!15.0, opacity=0.7] (0.3000, 0.2500, 0.0681) -- (0.3500, 0.2500, 0.0741) -- (0.3500, 0.3000, 0.0801) -- (0.3000, 0.3000, 0.0742) -- cycle;
\fill[blue!15.0, opacity=0.7] (0.3000, 0.3000, 0.0742) -- (0.3500, 0.3000, 0.0801) -- (0.3500, 0.3500, 0.0860) -- (0.3000, 0.3500, 0.0801) -- cycle;
\fill[blue!15.0, opacity=0.7] (0.3000, 0.3500, 0.0801) -- (0.3500, 0.3500, 0.0860) -- (0.3500, 0.4000, 0.0918) -- (0.3000, 0.4000, 0.0859) -- cycle;
\fill[blue!15.0, opacity=0.7] (0.3000, 0.4000, 0.0859) -- (0.3500, 0.4000, 0.0918) -- (0.3500, 0.4500, 0.0975) -- (0.3000, 0.4500, 0.0916) -- cycle;
\fill[blue!15.0, opacity=0.7] (0.3000, 0.4500, 0.0916) -- (0.3500, 0.4500, 0.0975) -- (0.3500, 0.5000, 0.1030) -- (0.3000, 0.5000, 0.0971) -- cycle;
\fill[blue!15.0, opacity=0.7] (0.3000, 0.5000, 0.0971) -- (0.3500, 0.5000, 0.1030) -- (0.3500, 0.5500, 0.1084) -- (0.3000, 0.5500, 0.1024) -- cycle;
\fill[blue!15.0, opacity=0.7] (0.3000, 0.5500, 0.1024) -- (0.3500, 0.5500, 0.1084) -- (0.3500, 0.6000, 0.1135) -- (0.3000, 0.6000, 0.1076) -- cycle;
\fill[blue!15.0, opacity=0.7] (0.3000, 0.6000, 0.1076) -- (0.3500, 0.6000, 0.1135) -- (0.3500, 0.6500, 0.1185) -- (0.3000, 0.6500, 0.1126) -- cycle;
\fill[blue!15.0, opacity=0.7] (0.3000, 0.6500, 0.1126) -- (0.3500, 0.6500, 0.1185) -- (0.3500, 0.7000, 0.1233) -- (0.3000, 0.7000, 0.1174) -- cycle;
\fill[blue!15.0, opacity=0.7] (0.3000, 0.7000, 0.1174) -- (0.3500, 0.7000, 0.1233) -- (0.3500, 0.7500, 0.1279) -- (0.3000, 0.7500, 0.1219) -- cycle;
\fill[blue!15.0, opacity=0.7] (0.3000, 0.7500, 0.1219) -- (0.3500, 0.7500, 0.1279) -- (0.3500, 0.8000, 0.1322) -- (0.3000, 0.8000, 0.1263) -- cycle;
\fill[blue!15.0, opacity=0.7] (0.3000, 0.8000, 0.1263) -- (0.3500, 0.8000, 0.1322) -- (0.3500, 0.8500, 0.1363) -- (0.3000, 0.8500, 0.1303) -- cycle;
\fill[blue!15.0, opacity=0.7] (0.3000, 0.8500, 0.1303) -- (0.3500, 0.8500, 0.1363) -- (0.3500, 0.9000, 0.1401) -- (0.3000, 0.9000, 0.1342) -- cycle;
\fill[blue!15.0, opacity=0.7] (0.3000, 0.9000, 0.1342) -- (0.3500, 0.9000, 0.1401) -- (0.3500, 0.9500, 0.1436) -- (0.3000, 0.9500, 0.1377) -- cycle;
\fill[blue!15.0, opacity=0.7] (0.3000, 0.9500, 0.1377) -- (0.3500, 0.9500, 0.1436) -- (0.3500, 1.0000, 0.1469) -- (0.3000, 1.0000, 0.1410) -- cycle;
\fill[blue!15.0, opacity=0.7] (0.3000, 1.0000, 0.1410) -- (0.3500, 1.0000, 0.1469) -- (0.3500, 1.0500, 0.1499) -- (0.3000, 1.0500, 0.1440) -- cycle;
\fill[blue!15.0, opacity=0.7] (0.3000, 1.0500, 0.1440) -- (0.3500, 1.0500, 0.1499) -- (0.3500, 1.1000, 0.1526) -- (0.3000, 1.1000, 0.1467) -- cycle;
\fill[blue!15.0, opacity=0.7] (0.3000, 1.1000, 0.1467) -- (0.3500, 1.1000, 0.1526) -- (0.3500, 1.1500, 0.1550) -- (0.3000, 1.1500, 0.1491) -- cycle;
\fill[blue!15.0, opacity=0.7] (0.3000, 1.1500, 0.1491) -- (0.3500, 1.1500, 0.1550) -- (0.3500, 1.2000, 0.1571) -- (0.3000, 1.2000, 0.1512) -- cycle;
\fill[blue!15.0, opacity=0.7] (0.3000, 1.2000, 0.1512) -- (0.3500, 1.2000, 0.1571) -- (0.3500, 1.2500, 0.1589) -- (0.3000, 1.2500, 0.1530) -- cycle;
\fill[blue!15.0, opacity=0.7] (0.3000, 1.2500, 0.1530) -- (0.3500, 1.2500, 0.1589) -- (0.3500, 1.3000, 0.1604) -- (0.3000, 1.3000, 0.1545) -- cycle;
\fill[blue!15.0, opacity=0.7] (0.3000, 1.3000, 0.1545) -- (0.3500, 1.3000, 0.1604) -- (0.3500, 1.3500, 0.1615) -- (0.3000, 1.3500, 0.1556) -- cycle;
\fill[blue!15.0, opacity=0.7] (0.3000, 1.3500, 0.1556) -- (0.3500, 1.3500, 0.1615) -- (0.3500, 1.4000, 0.1623) -- (0.3000, 1.4000, 0.1564) -- cycle;
\fill[blue!15.0, opacity=0.7] (0.3000, 1.4000, 0.1564) -- (0.3500, 1.4000, 0.1623) -- (0.3500, 1.4500, 0.1628) -- (0.3000, 1.4500, 0.1569) -- cycle;
\fill[blue!15.0, opacity=0.7] (0.3000, 1.4500, 0.1569) -- (0.3500, 1.4500, 0.1628) -- (0.3500, 1.5000, 0.1630) -- (0.3000, 1.5000, 0.1571) -- cycle;
\fill[blue!15.0, opacity=0.7] (0.3000, 1.5000, 0.1571) -- (0.3500, 1.5000, 0.1630) -- (0.3500, 1.5500, 0.1628) -- (0.3000, 1.5500, 0.1569) -- cycle;
\fill[blue!15.0, opacity=0.7] (0.3000, 1.5500, 0.1569) -- (0.3500, 1.5500, 0.1628) -- (0.3500, 1.6000, 0.1623) -- (0.3000, 1.6000, 0.1564) -- cycle;
\fill[blue!15.0, opacity=0.7] (0.3000, 1.6000, 0.1564) -- (0.3500, 1.6000, 0.1623) -- (0.3500, 1.6500, 0.1615) -- (0.3000, 1.6500, 0.1556) -- cycle;
\fill[blue!15.0, opacity=0.7] (0.3000, 1.6500, 0.1556) -- (0.3500, 1.6500, 0.1615) -- (0.3500, 1.7000, 0.1604) -- (0.3000, 1.7000, 0.1545) -- cycle;
\fill[blue!15.0, opacity=0.7] (0.3000, 1.7000, 0.1545) -- (0.3500, 1.7000, 0.1604) -- (0.3500, 1.7500, 0.1589) -- (0.3000, 1.7500, 0.1530) -- cycle;
\fill[blue!15.0, opacity=0.7] (0.3000, 1.7500, 0.1530) -- (0.3500, 1.7500, 0.1589) -- (0.3500, 1.8000, 0.1571) -- (0.3000, 1.8000, 0.1512) -- cycle;
\fill[blue!15.0, opacity=0.7] (0.3000, 1.8000, 0.1512) -- (0.3500, 1.8000, 0.1571) -- (0.3500, 1.8500, 0.1550) -- (0.3000, 1.8500, 0.1491) -- cycle;
\fill[blue!15.0, opacity=0.7] (0.3000, 1.8500, 0.1491) -- (0.3500, 1.8500, 0.1550) -- (0.3500, 1.9000, 0.1526) -- (0.3000, 1.9000, 0.1467) -- cycle;
\fill[blue!15.0, opacity=0.7] (0.3000, 1.9000, 0.1467) -- (0.3500, 1.9000, 0.1526) -- (0.3500, 1.9500, 0.1499) -- (0.3000, 1.9500, 0.1440) -- cycle;
\fill[blue!15.0, opacity=0.7] (0.3000, 1.9500, 0.1440) -- (0.3500, 1.9500, 0.1499) -- (0.3500, 2.0000, 0.1469) -- (0.3000, 2.0000, 0.1410) -- cycle;
\fill[blue!15.0, opacity=0.7] (0.3000, 2.0000, 0.1410) -- (0.3500, 2.0000, 0.1469) -- (0.3500, 2.0500, 0.1436) -- (0.3000, 2.0500, 0.1377) -- cycle;
\fill[blue!15.0, opacity=0.7] (0.3000, 2.0500, 0.1377) -- (0.3500, 2.0500, 0.1436) -- (0.3500, 2.1000, 0.1401) -- (0.3000, 2.1000, 0.1342) -- cycle;
\fill[blue!15.0, opacity=0.7] (0.3000, 2.1000, 0.1342) -- (0.3500, 2.1000, 0.1401) -- (0.3500, 2.1500, 0.1363) -- (0.3000, 2.1500, 0.1303) -- cycle;
\fill[blue!15.0, opacity=0.7] (0.3000, 2.1500, 0.1303) -- (0.3500, 2.1500, 0.1363) -- (0.3500, 2.2000, 0.1322) -- (0.3000, 2.2000, 0.1263) -- cycle;
\fill[blue!15.0, opacity=0.7] (0.3000, 2.2000, 0.1263) -- (0.3500, 2.2000, 0.1322) -- (0.3500, 2.2500, 0.1279) -- (0.3000, 2.2500, 0.1219) -- cycle;
\fill[blue!15.0, opacity=0.7] (0.3000, 2.2500, 0.1219) -- (0.3500, 2.2500, 0.1279) -- (0.3500, 2.3000, 0.1233) -- (0.3000, 2.3000, 0.1174) -- cycle;
\fill[blue!15.0, opacity=0.7] (0.3000, 2.3000, 0.1174) -- (0.3500, 2.3000, 0.1233) -- (0.3500, 2.3500, 0.1185) -- (0.3000, 2.3500, 0.1126) -- cycle;
\fill[blue!15.0, opacity=0.7] (0.3000, 2.3500, 0.1126) -- (0.3500, 2.3500, 0.1185) -- (0.3500, 2.4000, 0.1135) -- (0.3000, 2.4000, 0.1076) -- cycle;
\fill[blue!15.0, opacity=0.7] (0.3000, 2.4000, 0.1076) -- (0.3500, 2.4000, 0.1135) -- (0.3500, 2.4500, 0.1084) -- (0.3000, 2.4500, 0.1024) -- cycle;
\fill[blue!15.0, opacity=0.7] (0.3000, 2.4500, 0.1024) -- (0.3500, 2.4500, 0.1084) -- (0.3500, 2.5000, 0.1030) -- (0.3000, 2.5000, 0.0971) -- cycle;
\fill[blue!15.0, opacity=0.7] (0.3000, 2.5000, 0.0971) -- (0.3500, 2.5000, 0.1030) -- (0.3500, 2.5500, 0.0975) -- (0.3000, 2.5500, 0.0916) -- cycle;
\fill[blue!15.0, opacity=0.7] (0.3000, 2.5500, 0.0916) -- (0.3500, 2.5500, 0.0975) -- (0.3500, 2.6000, 0.0918) -- (0.3000, 2.6000, 0.0859) -- cycle;
\fill[blue!15.0, opacity=0.7] (0.3000, 2.6000, 0.0859) -- (0.3500, 2.6000, 0.0918) -- (0.3500, 2.6500, 0.0860) -- (0.3000, 2.6500, 0.0801) -- cycle;
\fill[blue!15.0, opacity=0.7] (0.3000, 2.6500, 0.0801) -- (0.3500, 2.6500, 0.0860) -- (0.3500, 2.7000, 0.0801) -- (0.3000, 2.7000, 0.0742) -- cycle;
\fill[blue!15.0, opacity=0.7] (0.3000, 2.7000, 0.0742) -- (0.3500, 2.7000, 0.0801) -- (0.3500, 2.7500, 0.0741) -- (0.3000, 2.7500, 0.0681) -- cycle;
\fill[blue!15.0, opacity=0.7] (0.3000, 2.7500, 0.0681) -- (0.3500, 2.7500, 0.0741) -- (0.3500, 2.8000, 0.0680) -- (0.3000, 2.8000, 0.0620) -- cycle;
\fill[blue!15.0, opacity=0.7] (0.3000, 2.8000, 0.0620) -- (0.3500, 2.8000, 0.0680) -- (0.3500, 2.8500, 0.0618) -- (0.3000, 2.8500, 0.0559) -- cycle;
\fill[blue!15.0, opacity=0.7] (0.3000, 2.8500, 0.0559) -- (0.3500, 2.8500, 0.0618) -- (0.3500, 2.9000, 0.0555) -- (0.3000, 2.9000, 0.0496) -- cycle;
\fill[blue!15.0, opacity=0.7] (0.3000, 2.9000, 0.0496) -- (0.3500, 2.9000, 0.0555) -- (0.3500, 2.9500, 0.0493) -- (0.3000, 2.9500, 0.0434) -- cycle;
\fill[blue!15.0, opacity=0.7] (0.3000, 2.9500, 0.0434) -- (0.3500, 2.9500, 0.0493) -- (0.3500, 3.0000, 0.0430) -- (0.3000, 3.0000, 0.0371) -- cycle;
\fill[blue!15.0, opacity=0.7] (0.3500, 0.0000, 0.0430) -- (0.4000, 0.0000, 0.0488) -- (0.4000, 0.0500, 0.0551) -- (0.3500, 0.0500, 0.0493) -- cycle;
\fill[blue!15.0, opacity=0.7] (0.3500, 0.0500, 0.0493) -- (0.4000, 0.0500, 0.0551) -- (0.4000, 0.1000, 0.0614) -- (0.3500, 0.1000, 0.0555) -- cycle;
\fill[blue!15.0, opacity=0.7] (0.3500, 0.1000, 0.0555) -- (0.4000, 0.1000, 0.0614) -- (0.4000, 0.1500, 0.0676) -- (0.3500, 0.1500, 0.0618) -- cycle;
\fill[blue!15.0, opacity=0.7] (0.3500, 0.1500, 0.0618) -- (0.4000, 0.1500, 0.0676) -- (0.4000, 0.2000, 0.0738) -- (0.3500, 0.2000, 0.0680) -- cycle;
\fill[blue!15.0, opacity=0.7] (0.3500, 0.2000, 0.0680) -- (0.4000, 0.2000, 0.0738) -- (0.4000, 0.2500, 0.0799) -- (0.3500, 0.2500, 0.0741) -- cycle;
\fill[blue!15.0, opacity=0.7] (0.3500, 0.2500, 0.0741) -- (0.4000, 0.2500, 0.0799) -- (0.4000, 0.3000, 0.0859) -- (0.3500, 0.3000, 0.0801) -- cycle;
\fill[blue!15.0, opacity=0.7] (0.3500, 0.3000, 0.0801) -- (0.4000, 0.3000, 0.0859) -- (0.4000, 0.3500, 0.0918) -- (0.3500, 0.3500, 0.0860) -- cycle;
\fill[blue!15.0, opacity=0.7] (0.3500, 0.3500, 0.0860) -- (0.4000, 0.3500, 0.0918) -- (0.4000, 0.4000, 0.0976) -- (0.3500, 0.4000, 0.0918) -- cycle;
\fill[blue!15.0, opacity=0.7] (0.3500, 0.4000, 0.0918) -- (0.4000, 0.4000, 0.0976) -- (0.4000, 0.4500, 0.1033) -- (0.3500, 0.4500, 0.0975) -- cycle;
\fill[blue!15.0, opacity=0.7] (0.3500, 0.4500, 0.0975) -- (0.4000, 0.4500, 0.1033) -- (0.4000, 0.5000, 0.1088) -- (0.3500, 0.5000, 0.1030) -- cycle;
\fill[blue!15.0, opacity=0.7] (0.3500, 0.5000, 0.1030) -- (0.4000, 0.5000, 0.1088) -- (0.4000, 0.5500, 0.1142) -- (0.3500, 0.5500, 0.1084) -- cycle;
\fill[blue!15.0, opacity=0.7] (0.3500, 0.5500, 0.1084) -- (0.4000, 0.5500, 0.1142) -- (0.4000, 0.6000, 0.1193) -- (0.3500, 0.6000, 0.1135) -- cycle;
\fill[blue!15.0, opacity=0.7] (0.3500, 0.6000, 0.1135) -- (0.4000, 0.6000, 0.1193) -- (0.4000, 0.6500, 0.1243) -- (0.3500, 0.6500, 0.1185) -- cycle;
\fill[blue!15.0, opacity=0.7] (0.3500, 0.6500, 0.1185) -- (0.4000, 0.6500, 0.1243) -- (0.4000, 0.7000, 0.1291) -- (0.3500, 0.7000, 0.1233) -- cycle;
\fill[blue!15.0, opacity=0.7] (0.3500, 0.7000, 0.1233) -- (0.4000, 0.7000, 0.1291) -- (0.4000, 0.7500, 0.1337) -- (0.3500, 0.7500, 0.1279) -- cycle;
\fill[blue!15.0, opacity=0.7] (0.3500, 0.7500, 0.1279) -- (0.4000, 0.7500, 0.1337) -- (0.4000, 0.8000, 0.1380) -- (0.3500, 0.8000, 0.1322) -- cycle;
\fill[blue!15.0, opacity=0.7] (0.3500, 0.8000, 0.1322) -- (0.4000, 0.8000, 0.1380) -- (0.4000, 0.8500, 0.1421) -- (0.3500, 0.8500, 0.1363) -- cycle;
\fill[blue!15.0, opacity=0.7] (0.3500, 0.8500, 0.1363) -- (0.4000, 0.8500, 0.1421) -- (0.4000, 0.9000, 0.1459) -- (0.3500, 0.9000, 0.1401) -- cycle;
\fill[blue!15.0, opacity=0.7] (0.3500, 0.9000, 0.1401) -- (0.4000, 0.9000, 0.1459) -- (0.4000, 0.9500, 0.1494) -- (0.3500, 0.9500, 0.1436) -- cycle;
\fill[blue!15.0, opacity=0.7] (0.3500, 0.9500, 0.1436) -- (0.4000, 0.9500, 0.1494) -- (0.4000, 1.0000, 0.1527) -- (0.3500, 1.0000, 0.1469) -- cycle;
\fill[blue!15.0, opacity=0.7] (0.3500, 1.0000, 0.1469) -- (0.4000, 1.0000, 0.1527) -- (0.4000, 1.0500, 0.1557) -- (0.3500, 1.0500, 0.1499) -- cycle;
\fill[blue!15.0, opacity=0.7] (0.3500, 1.0500, 0.1499) -- (0.4000, 1.0500, 0.1557) -- (0.4000, 1.1000, 0.1584) -- (0.3500, 1.1000, 0.1526) -- cycle;
\fill[blue!15.0, opacity=0.7] (0.3500, 1.1000, 0.1526) -- (0.4000, 1.1000, 0.1584) -- (0.4000, 1.1500, 0.1608) -- (0.3500, 1.1500, 0.1550) -- cycle;
\fill[blue!15.0, opacity=0.7] (0.3500, 1.1500, 0.1550) -- (0.4000, 1.1500, 0.1608) -- (0.4000, 1.2000, 0.1629) -- (0.3500, 1.2000, 0.1571) -- cycle;
\fill[blue!15.0, opacity=0.7] (0.3500, 1.2000, 0.1571) -- (0.4000, 1.2000, 0.1629) -- (0.4000, 1.2500, 0.1647) -- (0.3500, 1.2500, 0.1589) -- cycle;
\fill[blue!15.0, opacity=0.7] (0.3500, 1.2500, 0.1589) -- (0.4000, 1.2500, 0.1647) -- (0.4000, 1.3000, 0.1662) -- (0.3500, 1.3000, 0.1604) -- cycle;
\fill[blue!15.0, opacity=0.7] (0.3500, 1.3000, 0.1604) -- (0.4000, 1.3000, 0.1662) -- (0.4000, 1.3500, 0.1673) -- (0.3500, 1.3500, 0.1615) -- cycle;
\fill[blue!15.0, opacity=0.7] (0.3500, 1.3500, 0.1615) -- (0.4000, 1.3500, 0.1673) -- (0.4000, 1.4000, 0.1682) -- (0.3500, 1.4000, 0.1623) -- cycle;
\fill[blue!15.0, opacity=0.7] (0.3500, 1.4000, 0.1623) -- (0.4000, 1.4000, 0.1682) -- (0.4000, 1.4500, 0.1686) -- (0.3500, 1.4500, 0.1628) -- cycle;
\fill[blue!15.0, opacity=0.7] (0.3500, 1.4500, 0.1628) -- (0.4000, 1.4500, 0.1686) -- (0.4000, 1.5000, 0.1688) -- (0.3500, 1.5000, 0.1630) -- cycle;
\fill[blue!15.0, opacity=0.7] (0.3500, 1.5000, 0.1630) -- (0.4000, 1.5000, 0.1688) -- (0.4000, 1.5500, 0.1686) -- (0.3500, 1.5500, 0.1628) -- cycle;
\fill[blue!15.0, opacity=0.7] (0.3500, 1.5500, 0.1628) -- (0.4000, 1.5500, 0.1686) -- (0.4000, 1.6000, 0.1682) -- (0.3500, 1.6000, 0.1623) -- cycle;
\fill[blue!15.0, opacity=0.7] (0.3500, 1.6000, 0.1623) -- (0.4000, 1.6000, 0.1682) -- (0.4000, 1.6500, 0.1673) -- (0.3500, 1.6500, 0.1615) -- cycle;
\fill[blue!15.0, opacity=0.7] (0.3500, 1.6500, 0.1615) -- (0.4000, 1.6500, 0.1673) -- (0.4000, 1.7000, 0.1662) -- (0.3500, 1.7000, 0.1604) -- cycle;
\fill[blue!15.0, opacity=0.7] (0.3500, 1.7000, 0.1604) -- (0.4000, 1.7000, 0.1662) -- (0.4000, 1.7500, 0.1647) -- (0.3500, 1.7500, 0.1589) -- cycle;
\fill[blue!15.0, opacity=0.7] (0.3500, 1.7500, 0.1589) -- (0.4000, 1.7500, 0.1647) -- (0.4000, 1.8000, 0.1629) -- (0.3500, 1.8000, 0.1571) -- cycle;
\fill[blue!15.0, opacity=0.7] (0.3500, 1.8000, 0.1571) -- (0.4000, 1.8000, 0.1629) -- (0.4000, 1.8500, 0.1608) -- (0.3500, 1.8500, 0.1550) -- cycle;
\fill[blue!15.0, opacity=0.7] (0.3500, 1.8500, 0.1550) -- (0.4000, 1.8500, 0.1608) -- (0.4000, 1.9000, 0.1584) -- (0.3500, 1.9000, 0.1526) -- cycle;
\fill[blue!15.0, opacity=0.7] (0.3500, 1.9000, 0.1526) -- (0.4000, 1.9000, 0.1584) -- (0.4000, 1.9500, 0.1557) -- (0.3500, 1.9500, 0.1499) -- cycle;
\fill[blue!15.0, opacity=0.7] (0.3500, 1.9500, 0.1499) -- (0.4000, 1.9500, 0.1557) -- (0.4000, 2.0000, 0.1527) -- (0.3500, 2.0000, 0.1469) -- cycle;
\fill[blue!15.0, opacity=0.7] (0.3500, 2.0000, 0.1469) -- (0.4000, 2.0000, 0.1527) -- (0.4000, 2.0500, 0.1494) -- (0.3500, 2.0500, 0.1436) -- cycle;
\fill[blue!15.0, opacity=0.7] (0.3500, 2.0500, 0.1436) -- (0.4000, 2.0500, 0.1494) -- (0.4000, 2.1000, 0.1459) -- (0.3500, 2.1000, 0.1401) -- cycle;
\fill[blue!15.0, opacity=0.7] (0.3500, 2.1000, 0.1401) -- (0.4000, 2.1000, 0.1459) -- (0.4000, 2.1500, 0.1421) -- (0.3500, 2.1500, 0.1363) -- cycle;
\fill[blue!15.0, opacity=0.7] (0.3500, 2.1500, 0.1363) -- (0.4000, 2.1500, 0.1421) -- (0.4000, 2.2000, 0.1380) -- (0.3500, 2.2000, 0.1322) -- cycle;
\fill[blue!15.0, opacity=0.7] (0.3500, 2.2000, 0.1322) -- (0.4000, 2.2000, 0.1380) -- (0.4000, 2.2500, 0.1337) -- (0.3500, 2.2500, 0.1279) -- cycle;
\fill[blue!15.0, opacity=0.7] (0.3500, 2.2500, 0.1279) -- (0.4000, 2.2500, 0.1337) -- (0.4000, 2.3000, 0.1291) -- (0.3500, 2.3000, 0.1233) -- cycle;
\fill[blue!15.0, opacity=0.7] (0.3500, 2.3000, 0.1233) -- (0.4000, 2.3000, 0.1291) -- (0.4000, 2.3500, 0.1243) -- (0.3500, 2.3500, 0.1185) -- cycle;
\fill[blue!15.0, opacity=0.7] (0.3500, 2.3500, 0.1185) -- (0.4000, 2.3500, 0.1243) -- (0.4000, 2.4000, 0.1193) -- (0.3500, 2.4000, 0.1135) -- cycle;
\fill[blue!15.0, opacity=0.7] (0.3500, 2.4000, 0.1135) -- (0.4000, 2.4000, 0.1193) -- (0.4000, 2.4500, 0.1142) -- (0.3500, 2.4500, 0.1084) -- cycle;
\fill[blue!15.0, opacity=0.7] (0.3500, 2.4500, 0.1084) -- (0.4000, 2.4500, 0.1142) -- (0.4000, 2.5000, 0.1088) -- (0.3500, 2.5000, 0.1030) -- cycle;
\fill[blue!15.0, opacity=0.7] (0.3500, 2.5000, 0.1030) -- (0.4000, 2.5000, 0.1088) -- (0.4000, 2.5500, 0.1033) -- (0.3500, 2.5500, 0.0975) -- cycle;
\fill[blue!15.0, opacity=0.7] (0.3500, 2.5500, 0.0975) -- (0.4000, 2.5500, 0.1033) -- (0.4000, 2.6000, 0.0976) -- (0.3500, 2.6000, 0.0918) -- cycle;
\fill[blue!15.0, opacity=0.7] (0.3500, 2.6000, 0.0918) -- (0.4000, 2.6000, 0.0976) -- (0.4000, 2.6500, 0.0918) -- (0.3500, 2.6500, 0.0860) -- cycle;
\fill[blue!15.0, opacity=0.7] (0.3500, 2.6500, 0.0860) -- (0.4000, 2.6500, 0.0918) -- (0.4000, 2.7000, 0.0859) -- (0.3500, 2.7000, 0.0801) -- cycle;
\fill[blue!15.0, opacity=0.7] (0.3500, 2.7000, 0.0801) -- (0.4000, 2.7000, 0.0859) -- (0.4000, 2.7500, 0.0799) -- (0.3500, 2.7500, 0.0741) -- cycle;
\fill[blue!15.0, opacity=0.7] (0.3500, 2.7500, 0.0741) -- (0.4000, 2.7500, 0.0799) -- (0.4000, 2.8000, 0.0738) -- (0.3500, 2.8000, 0.0680) -- cycle;
\fill[blue!15.0, opacity=0.7] (0.3500, 2.8000, 0.0680) -- (0.4000, 2.8000, 0.0738) -- (0.4000, 2.8500, 0.0676) -- (0.3500, 2.8500, 0.0618) -- cycle;
\fill[blue!15.0, opacity=0.7] (0.3500, 2.8500, 0.0618) -- (0.4000, 2.8500, 0.0676) -- (0.4000, 2.9000, 0.0614) -- (0.3500, 2.9000, 0.0555) -- cycle;
\fill[blue!15.0, opacity=0.7] (0.3500, 2.9000, 0.0555) -- (0.4000, 2.9000, 0.0614) -- (0.4000, 2.9500, 0.0551) -- (0.3500, 2.9500, 0.0493) -- cycle;
\fill[blue!15.0, opacity=0.7] (0.3500, 2.9500, 0.0493) -- (0.4000, 2.9500, 0.0551) -- (0.4000, 3.0000, 0.0488) -- (0.3500, 3.0000, 0.0430) -- cycle;
\fill[blue!15.0, opacity=0.7] (0.4000, 0.0000, 0.0488) -- (0.4500, 0.0000, 0.0545) -- (0.4500, 0.0500, 0.0608) -- (0.4000, 0.0500, 0.0551) -- cycle;
\fill[blue!15.0, opacity=0.7] (0.4000, 0.0500, 0.0551) -- (0.4500, 0.0500, 0.0608) -- (0.4500, 0.1000, 0.0670) -- (0.4000, 0.1000, 0.0614) -- cycle;
\fill[blue!15.0, opacity=0.7] (0.4000, 0.1000, 0.0614) -- (0.4500, 0.1000, 0.0670) -- (0.4500, 0.1500, 0.0733) -- (0.4000, 0.1500, 0.0676) -- cycle;
\fill[blue!15.0, opacity=0.7] (0.4000, 0.1500, 0.0676) -- (0.4500, 0.1500, 0.0733) -- (0.4500, 0.2000, 0.0794) -- (0.4000, 0.2000, 0.0738) -- cycle;
\fill[blue!15.0, opacity=0.7] (0.4000, 0.2000, 0.0738) -- (0.4500, 0.2000, 0.0794) -- (0.4500, 0.2500, 0.0855) -- (0.4000, 0.2500, 0.0799) -- cycle;
\fill[blue!15.0, opacity=0.7] (0.4000, 0.2500, 0.0799) -- (0.4500, 0.2500, 0.0855) -- (0.4500, 0.3000, 0.0916) -- (0.4000, 0.3000, 0.0859) -- cycle;
\fill[blue!15.0, opacity=0.7] (0.4000, 0.3000, 0.0859) -- (0.4500, 0.3000, 0.0916) -- (0.4500, 0.3500, 0.0975) -- (0.4000, 0.3500, 0.0918) -- cycle;
\fill[blue!15.0, opacity=0.7] (0.4000, 0.3500, 0.0918) -- (0.4500, 0.3500, 0.0975) -- (0.4500, 0.4000, 0.1033) -- (0.4000, 0.4000, 0.0976) -- cycle;
\fill[blue!15.0, opacity=0.7] (0.4000, 0.4000, 0.0976) -- (0.4500, 0.4000, 0.1033) -- (0.4500, 0.4500, 0.1090) -- (0.4000, 0.4500, 0.1033) -- cycle;
\fill[blue!15.0, opacity=0.7] (0.4000, 0.4500, 0.1033) -- (0.4500, 0.4500, 0.1090) -- (0.4500, 0.5000, 0.1145) -- (0.4000, 0.5000, 0.1088) -- cycle;
\fill[blue!15.0, opacity=0.7] (0.4000, 0.5000, 0.1088) -- (0.4500, 0.5000, 0.1145) -- (0.4500, 0.5500, 0.1198) -- (0.4000, 0.5500, 0.1142) -- cycle;
\fill[blue!15.0, opacity=0.7] (0.4000, 0.5500, 0.1142) -- (0.4500, 0.5500, 0.1198) -- (0.4500, 0.6000, 0.1250) -- (0.4000, 0.6000, 0.1193) -- cycle;
\fill[blue!15.0, opacity=0.7] (0.4000, 0.6000, 0.1193) -- (0.4500, 0.6000, 0.1250) -- (0.4500, 0.6500, 0.1300) -- (0.4000, 0.6500, 0.1243) -- cycle;
\fill[blue!15.0, opacity=0.7] (0.4000, 0.6500, 0.1243) -- (0.4500, 0.6500, 0.1300) -- (0.4500, 0.7000, 0.1348) -- (0.4000, 0.7000, 0.1291) -- cycle;
\fill[blue!15.0, opacity=0.7] (0.4000, 0.7000, 0.1291) -- (0.4500, 0.7000, 0.1348) -- (0.4500, 0.7500, 0.1393) -- (0.4000, 0.7500, 0.1337) -- cycle;
\fill[blue!15.0, opacity=0.7] (0.4000, 0.7500, 0.1337) -- (0.4500, 0.7500, 0.1393) -- (0.4500, 0.8000, 0.1437) -- (0.4000, 0.8000, 0.1380) -- cycle;
\fill[blue!15.0, opacity=0.7] (0.4000, 0.8000, 0.1380) -- (0.4500, 0.8000, 0.1437) -- (0.4500, 0.8500, 0.1477) -- (0.4000, 0.8500, 0.1421) -- cycle;
\fill[blue!15.0, opacity=0.7] (0.4000, 0.8500, 0.1421) -- (0.4500, 0.8500, 0.1477) -- (0.4500, 0.9000, 0.1516) -- (0.4000, 0.9000, 0.1459) -- cycle;
\fill[blue!15.0, opacity=0.7] (0.4000, 0.9000, 0.1459) -- (0.4500, 0.9000, 0.1516) -- (0.4500, 0.9500, 0.1551) -- (0.4000, 0.9500, 0.1494) -- cycle;
\fill[blue!15.0, opacity=0.7] (0.4000, 0.9500, 0.1494) -- (0.4500, 0.9500, 0.1551) -- (0.4500, 1.0000, 0.1584) -- (0.4000, 1.0000, 0.1527) -- cycle;
\fill[blue!15.0, opacity=0.7] (0.4000, 1.0000, 0.1527) -- (0.4500, 1.0000, 0.1584) -- (0.4500, 1.0500, 0.1614) -- (0.4000, 1.0500, 0.1557) -- cycle;
\fill[blue!15.0, opacity=0.7] (0.4000, 1.0500, 0.1557) -- (0.4500, 1.0500, 0.1614) -- (0.4500, 1.1000, 0.1641) -- (0.4000, 1.1000, 0.1584) -- cycle;
\fill[blue!15.0, opacity=0.7] (0.4000, 1.1000, 0.1584) -- (0.4500, 1.1000, 0.1641) -- (0.4500, 1.1500, 0.1665) -- (0.4000, 1.1500, 0.1608) -- cycle;
\fill[blue!15.0, opacity=0.7] (0.4000, 1.1500, 0.1608) -- (0.4500, 1.1500, 0.1665) -- (0.4500, 1.2000, 0.1686) -- (0.4000, 1.2000, 0.1629) -- cycle;
\fill[blue!15.0, opacity=0.7] (0.4000, 1.2000, 0.1629) -- (0.4500, 1.2000, 0.1686) -- (0.4500, 1.2500, 0.1704) -- (0.4000, 1.2500, 0.1647) -- cycle;
\fill[blue!15.0, opacity=0.7] (0.4000, 1.2500, 0.1647) -- (0.4500, 1.2500, 0.1704) -- (0.4500, 1.3000, 0.1719) -- (0.4000, 1.3000, 0.1662) -- cycle;
\fill[blue!15.0, opacity=0.7] (0.4000, 1.3000, 0.1662) -- (0.4500, 1.3000, 0.1719) -- (0.4500, 1.3500, 0.1730) -- (0.4000, 1.3500, 0.1673) -- cycle;
\fill[blue!15.0, opacity=0.7] (0.4000, 1.3500, 0.1673) -- (0.4500, 1.3500, 0.1730) -- (0.4500, 1.4000, 0.1738) -- (0.4000, 1.4000, 0.1682) -- cycle;
\fill[blue!15.0, opacity=0.7] (0.4000, 1.4000, 0.1682) -- (0.4500, 1.4000, 0.1738) -- (0.4500, 1.4500, 0.1743) -- (0.4000, 1.4500, 0.1686) -- cycle;
\fill[blue!15.0, opacity=0.7] (0.4000, 1.4500, 0.1686) -- (0.4500, 1.4500, 0.1743) -- (0.4500, 1.5000, 0.1745) -- (0.4000, 1.5000, 0.1688) -- cycle;
\fill[blue!15.0, opacity=0.7] (0.4000, 1.5000, 0.1688) -- (0.4500, 1.5000, 0.1745) -- (0.4500, 1.5500, 0.1743) -- (0.4000, 1.5500, 0.1686) -- cycle;
\fill[blue!15.0, opacity=0.7] (0.4000, 1.5500, 0.1686) -- (0.4500, 1.5500, 0.1743) -- (0.4500, 1.6000, 0.1738) -- (0.4000, 1.6000, 0.1682) -- cycle;
\fill[blue!15.0, opacity=0.7] (0.4000, 1.6000, 0.1682) -- (0.4500, 1.6000, 0.1738) -- (0.4500, 1.6500, 0.1730) -- (0.4000, 1.6500, 0.1673) -- cycle;
\fill[blue!15.0, opacity=0.7] (0.4000, 1.6500, 0.1673) -- (0.4500, 1.6500, 0.1730) -- (0.4500, 1.7000, 0.1719) -- (0.4000, 1.7000, 0.1662) -- cycle;
\fill[blue!15.0, opacity=0.7] (0.4000, 1.7000, 0.1662) -- (0.4500, 1.7000, 0.1719) -- (0.4500, 1.7500, 0.1704) -- (0.4000, 1.7500, 0.1647) -- cycle;
\fill[blue!15.0, opacity=0.7] (0.4000, 1.7500, 0.1647) -- (0.4500, 1.7500, 0.1704) -- (0.4500, 1.8000, 0.1686) -- (0.4000, 1.8000, 0.1629) -- cycle;
\fill[blue!15.0, opacity=0.7] (0.4000, 1.8000, 0.1629) -- (0.4500, 1.8000, 0.1686) -- (0.4500, 1.8500, 0.1665) -- (0.4000, 1.8500, 0.1608) -- cycle;
\fill[blue!15.0, opacity=0.7] (0.4000, 1.8500, 0.1608) -- (0.4500, 1.8500, 0.1665) -- (0.4500, 1.9000, 0.1641) -- (0.4000, 1.9000, 0.1584) -- cycle;
\fill[blue!15.0, opacity=0.7] (0.4000, 1.9000, 0.1584) -- (0.4500, 1.9000, 0.1641) -- (0.4500, 1.9500, 0.1614) -- (0.4000, 1.9500, 0.1557) -- cycle;
\fill[blue!15.0, opacity=0.7] (0.4000, 1.9500, 0.1557) -- (0.4500, 1.9500, 0.1614) -- (0.4500, 2.0000, 0.1584) -- (0.4000, 2.0000, 0.1527) -- cycle;
\fill[blue!15.0, opacity=0.7] (0.4000, 2.0000, 0.1527) -- (0.4500, 2.0000, 0.1584) -- (0.4500, 2.0500, 0.1551) -- (0.4000, 2.0500, 0.1494) -- cycle;
\fill[blue!15.0, opacity=0.7] (0.4000, 2.0500, 0.1494) -- (0.4500, 2.0500, 0.1551) -- (0.4500, 2.1000, 0.1516) -- (0.4000, 2.1000, 0.1459) -- cycle;
\fill[blue!15.0, opacity=0.7] (0.4000, 2.1000, 0.1459) -- (0.4500, 2.1000, 0.1516) -- (0.4500, 2.1500, 0.1477) -- (0.4000, 2.1500, 0.1421) -- cycle;
\fill[blue!15.0, opacity=0.7] (0.4000, 2.1500, 0.1421) -- (0.4500, 2.1500, 0.1477) -- (0.4500, 2.2000, 0.1437) -- (0.4000, 2.2000, 0.1380) -- cycle;
\fill[blue!15.0, opacity=0.7] (0.4000, 2.2000, 0.1380) -- (0.4500, 2.2000, 0.1437) -- (0.4500, 2.2500, 0.1393) -- (0.4000, 2.2500, 0.1337) -- cycle;
\fill[blue!15.0, opacity=0.7] (0.4000, 2.2500, 0.1337) -- (0.4500, 2.2500, 0.1393) -- (0.4500, 2.3000, 0.1348) -- (0.4000, 2.3000, 0.1291) -- cycle;
\fill[blue!15.0, opacity=0.7] (0.4000, 2.3000, 0.1291) -- (0.4500, 2.3000, 0.1348) -- (0.4500, 2.3500, 0.1300) -- (0.4000, 2.3500, 0.1243) -- cycle;
\fill[blue!15.0, opacity=0.7] (0.4000, 2.3500, 0.1243) -- (0.4500, 2.3500, 0.1300) -- (0.4500, 2.4000, 0.1250) -- (0.4000, 2.4000, 0.1193) -- cycle;
\fill[blue!15.0, opacity=0.7] (0.4000, 2.4000, 0.1193) -- (0.4500, 2.4000, 0.1250) -- (0.4500, 2.4500, 0.1198) -- (0.4000, 2.4500, 0.1142) -- cycle;
\fill[blue!15.0, opacity=0.7] (0.4000, 2.4500, 0.1142) -- (0.4500, 2.4500, 0.1198) -- (0.4500, 2.5000, 0.1145) -- (0.4000, 2.5000, 0.1088) -- cycle;
\fill[blue!15.0, opacity=0.7] (0.4000, 2.5000, 0.1088) -- (0.4500, 2.5000, 0.1145) -- (0.4500, 2.5500, 0.1090) -- (0.4000, 2.5500, 0.1033) -- cycle;
\fill[blue!15.0, opacity=0.7] (0.4000, 2.5500, 0.1033) -- (0.4500, 2.5500, 0.1090) -- (0.4500, 2.6000, 0.1033) -- (0.4000, 2.6000, 0.0976) -- cycle;
\fill[blue!15.0, opacity=0.7] (0.4000, 2.6000, 0.0976) -- (0.4500, 2.6000, 0.1033) -- (0.4500, 2.6500, 0.0975) -- (0.4000, 2.6500, 0.0918) -- cycle;
\fill[blue!15.0, opacity=0.7] (0.4000, 2.6500, 0.0918) -- (0.4500, 2.6500, 0.0975) -- (0.4500, 2.7000, 0.0916) -- (0.4000, 2.7000, 0.0859) -- cycle;
\fill[blue!15.0, opacity=0.7] (0.4000, 2.7000, 0.0859) -- (0.4500, 2.7000, 0.0916) -- (0.4500, 2.7500, 0.0855) -- (0.4000, 2.7500, 0.0799) -- cycle;
\fill[blue!15.0, opacity=0.7] (0.4000, 2.7500, 0.0799) -- (0.4500, 2.7500, 0.0855) -- (0.4500, 2.8000, 0.0794) -- (0.4000, 2.8000, 0.0738) -- cycle;
\fill[blue!15.0, opacity=0.7] (0.4000, 2.8000, 0.0738) -- (0.4500, 2.8000, 0.0794) -- (0.4500, 2.8500, 0.0733) -- (0.4000, 2.8500, 0.0676) -- cycle;
\fill[blue!15.0, opacity=0.7] (0.4000, 2.8500, 0.0676) -- (0.4500, 2.8500, 0.0733) -- (0.4500, 2.9000, 0.0670) -- (0.4000, 2.9000, 0.0614) -- cycle;
\fill[blue!15.0, opacity=0.7] (0.4000, 2.9000, 0.0614) -- (0.4500, 2.9000, 0.0670) -- (0.4500, 2.9500, 0.0608) -- (0.4000, 2.9500, 0.0551) -- cycle;
\fill[blue!15.0, opacity=0.7] (0.4000, 2.9500, 0.0551) -- (0.4500, 2.9500, 0.0608) -- (0.4500, 3.0000, 0.0545) -- (0.4000, 3.0000, 0.0488) -- cycle;
\fill[blue!15.0, opacity=0.7] (0.4500, 0.0000, 0.0545) -- (0.5000, 0.0000, 0.0600) -- (0.5000, 0.0500, 0.0663) -- (0.4500, 0.0500, 0.0608) -- cycle;
\fill[blue!15.0, opacity=0.7] (0.4500, 0.0500, 0.0608) -- (0.5000, 0.0500, 0.0663) -- (0.5000, 0.1000, 0.0725) -- (0.4500, 0.1000, 0.0670) -- cycle;
\fill[blue!15.0, opacity=0.7] (0.4500, 0.1000, 0.0670) -- (0.5000, 0.1000, 0.0725) -- (0.5000, 0.1500, 0.0788) -- (0.4500, 0.1500, 0.0733) -- cycle;
\fill[blue!15.0, opacity=0.7] (0.4500, 0.1500, 0.0733) -- (0.5000, 0.1500, 0.0788) -- (0.5000, 0.2000, 0.0849) -- (0.4500, 0.2000, 0.0794) -- cycle;
\fill[blue!15.0, opacity=0.7] (0.4500, 0.2000, 0.0794) -- (0.5000, 0.2000, 0.0849) -- (0.5000, 0.2500, 0.0911) -- (0.4500, 0.2500, 0.0855) -- cycle;
\fill[blue!15.0, opacity=0.7] (0.4500, 0.2500, 0.0855) -- (0.5000, 0.2500, 0.0911) -- (0.5000, 0.3000, 0.0971) -- (0.4500, 0.3000, 0.0916) -- cycle;
\fill[blue!15.0, opacity=0.7] (0.4500, 0.3000, 0.0916) -- (0.5000, 0.3000, 0.0971) -- (0.5000, 0.3500, 0.1030) -- (0.4500, 0.3500, 0.0975) -- cycle;
\fill[blue!15.0, opacity=0.7] (0.4500, 0.3500, 0.0975) -- (0.5000, 0.3500, 0.1030) -- (0.5000, 0.4000, 0.1088) -- (0.4500, 0.4000, 0.1033) -- cycle;
\fill[blue!15.0, opacity=0.7] (0.4500, 0.4000, 0.1033) -- (0.5000, 0.4000, 0.1088) -- (0.5000, 0.4500, 0.1145) -- (0.4500, 0.4500, 0.1090) -- cycle;
\fill[blue!15.0, opacity=0.7] (0.4500, 0.4500, 0.1090) -- (0.5000, 0.4500, 0.1145) -- (0.5000, 0.5000, 0.1200) -- (0.4500, 0.5000, 0.1145) -- cycle;
\fill[blue!15.0, opacity=0.7] (0.4500, 0.5000, 0.1145) -- (0.5000, 0.5000, 0.1200) -- (0.5000, 0.5500, 0.1254) -- (0.4500, 0.5500, 0.1198) -- cycle;
\fill[blue!15.0, opacity=0.7] (0.4500, 0.5500, 0.1198) -- (0.5000, 0.5500, 0.1254) -- (0.5000, 0.6000, 0.1305) -- (0.4500, 0.6000, 0.1250) -- cycle;
\fill[blue!15.0, opacity=0.7] (0.4500, 0.6000, 0.1250) -- (0.5000, 0.6000, 0.1305) -- (0.5000, 0.6500, 0.1355) -- (0.4500, 0.6500, 0.1300) -- cycle;
\fill[blue!15.0, opacity=0.7] (0.4500, 0.6500, 0.1300) -- (0.5000, 0.6500, 0.1355) -- (0.5000, 0.7000, 0.1403) -- (0.4500, 0.7000, 0.1348) -- cycle;
\fill[blue!15.0, opacity=0.7] (0.4500, 0.7000, 0.1348) -- (0.5000, 0.7000, 0.1403) -- (0.5000, 0.7500, 0.1449) -- (0.4500, 0.7500, 0.1393) -- cycle;
\fill[blue!15.0, opacity=0.7] (0.4500, 0.7500, 0.1393) -- (0.5000, 0.7500, 0.1449) -- (0.5000, 0.8000, 0.1492) -- (0.4500, 0.8000, 0.1437) -- cycle;
\fill[blue!15.0, opacity=0.7] (0.4500, 0.8000, 0.1437) -- (0.5000, 0.8000, 0.1492) -- (0.5000, 0.8500, 0.1533) -- (0.4500, 0.8500, 0.1477) -- cycle;
\fill[blue!15.0, opacity=0.7] (0.4500, 0.8500, 0.1477) -- (0.5000, 0.8500, 0.1533) -- (0.5000, 0.9000, 0.1571) -- (0.4500, 0.9000, 0.1516) -- cycle;
\fill[blue!15.0, opacity=0.7] (0.4500, 0.9000, 0.1516) -- (0.5000, 0.9000, 0.1571) -- (0.5000, 0.9500, 0.1606) -- (0.4500, 0.9500, 0.1551) -- cycle;
\fill[blue!15.0, opacity=0.7] (0.4500, 0.9500, 0.1551) -- (0.5000, 0.9500, 0.1606) -- (0.5000, 1.0000, 0.1639) -- (0.4500, 1.0000, 0.1584) -- cycle;
\fill[blue!15.0, opacity=0.7] (0.4500, 1.0000, 0.1584) -- (0.5000, 1.0000, 0.1639) -- (0.5000, 1.0500, 0.1669) -- (0.4500, 1.0500, 0.1614) -- cycle;
\fill[blue!15.0, opacity=0.7] (0.4500, 1.0500, 0.1614) -- (0.5000, 1.0500, 0.1669) -- (0.5000, 1.1000, 0.1696) -- (0.4500, 1.1000, 0.1641) -- cycle;
\fill[blue!15.0, opacity=0.7] (0.4500, 1.1000, 0.1641) -- (0.5000, 1.1000, 0.1696) -- (0.5000, 1.1500, 0.1720) -- (0.4500, 1.1500, 0.1665) -- cycle;
\fill[blue!15.0, opacity=0.7] (0.4500, 1.1500, 0.1665) -- (0.5000, 1.1500, 0.1720) -- (0.5000, 1.2000, 0.1741) -- (0.4500, 1.2000, 0.1686) -- cycle;
\fill[blue!15.0, opacity=0.7] (0.4500, 1.2000, 0.1686) -- (0.5000, 1.2000, 0.1741) -- (0.5000, 1.2500, 0.1759) -- (0.4500, 1.2500, 0.1704) -- cycle;
\fill[blue!15.0, opacity=0.7] (0.4500, 1.2500, 0.1704) -- (0.5000, 1.2500, 0.1759) -- (0.5000, 1.3000, 0.1774) -- (0.4500, 1.3000, 0.1719) -- cycle;
\fill[blue!15.0, opacity=0.7] (0.4500, 1.3000, 0.1719) -- (0.5000, 1.3000, 0.1774) -- (0.5000, 1.3500, 0.1785) -- (0.4500, 1.3500, 0.1730) -- cycle;
\fill[blue!15.0, opacity=0.7] (0.4500, 1.3500, 0.1730) -- (0.5000, 1.3500, 0.1785) -- (0.5000, 1.4000, 0.1793) -- (0.4500, 1.4000, 0.1738) -- cycle;
\fill[blue!15.0, opacity=0.7] (0.4500, 1.4000, 0.1738) -- (0.5000, 1.4000, 0.1793) -- (0.5000, 1.4500, 0.1798) -- (0.4500, 1.4500, 0.1743) -- cycle;
\fill[blue!15.0, opacity=0.7] (0.4500, 1.4500, 0.1743) -- (0.5000, 1.4500, 0.1798) -- (0.5000, 1.5000, 0.1800) -- (0.4500, 1.5000, 0.1745) -- cycle;
\fill[blue!15.0, opacity=0.7] (0.4500, 1.5000, 0.1745) -- (0.5000, 1.5000, 0.1800) -- (0.5000, 1.5500, 0.1798) -- (0.4500, 1.5500, 0.1743) -- cycle;
\fill[blue!15.0, opacity=0.7] (0.4500, 1.5500, 0.1743) -- (0.5000, 1.5500, 0.1798) -- (0.5000, 1.6000, 0.1793) -- (0.4500, 1.6000, 0.1738) -- cycle;
\fill[blue!15.0, opacity=0.7] (0.4500, 1.6000, 0.1738) -- (0.5000, 1.6000, 0.1793) -- (0.5000, 1.6500, 0.1785) -- (0.4500, 1.6500, 0.1730) -- cycle;
\fill[blue!15.0, opacity=0.7] (0.4500, 1.6500, 0.1730) -- (0.5000, 1.6500, 0.1785) -- (0.5000, 1.7000, 0.1774) -- (0.4500, 1.7000, 0.1719) -- cycle;
\fill[blue!15.0, opacity=0.7] (0.4500, 1.7000, 0.1719) -- (0.5000, 1.7000, 0.1774) -- (0.5000, 1.7500, 0.1759) -- (0.4500, 1.7500, 0.1704) -- cycle;
\fill[blue!15.0, opacity=0.7] (0.4500, 1.7500, 0.1704) -- (0.5000, 1.7500, 0.1759) -- (0.5000, 1.8000, 0.1741) -- (0.4500, 1.8000, 0.1686) -- cycle;
\fill[blue!15.0, opacity=0.7] (0.4500, 1.8000, 0.1686) -- (0.5000, 1.8000, 0.1741) -- (0.5000, 1.8500, 0.1720) -- (0.4500, 1.8500, 0.1665) -- cycle;
\fill[blue!15.0, opacity=0.7] (0.4500, 1.8500, 0.1665) -- (0.5000, 1.8500, 0.1720) -- (0.5000, 1.9000, 0.1696) -- (0.4500, 1.9000, 0.1641) -- cycle;
\fill[blue!15.0, opacity=0.7] (0.4500, 1.9000, 0.1641) -- (0.5000, 1.9000, 0.1696) -- (0.5000, 1.9500, 0.1669) -- (0.4500, 1.9500, 0.1614) -- cycle;
\fill[blue!15.0, opacity=0.7] (0.4500, 1.9500, 0.1614) -- (0.5000, 1.9500, 0.1669) -- (0.5000, 2.0000, 0.1639) -- (0.4500, 2.0000, 0.1584) -- cycle;
\fill[blue!15.0, opacity=0.7] (0.4500, 2.0000, 0.1584) -- (0.5000, 2.0000, 0.1639) -- (0.5000, 2.0500, 0.1606) -- (0.4500, 2.0500, 0.1551) -- cycle;
\fill[blue!15.0, opacity=0.7] (0.4500, 2.0500, 0.1551) -- (0.5000, 2.0500, 0.1606) -- (0.5000, 2.1000, 0.1571) -- (0.4500, 2.1000, 0.1516) -- cycle;
\fill[blue!15.0, opacity=0.7] (0.4500, 2.1000, 0.1516) -- (0.5000, 2.1000, 0.1571) -- (0.5000, 2.1500, 0.1533) -- (0.4500, 2.1500, 0.1477) -- cycle;
\fill[blue!15.0, opacity=0.7] (0.4500, 2.1500, 0.1477) -- (0.5000, 2.1500, 0.1533) -- (0.5000, 2.2000, 0.1492) -- (0.4500, 2.2000, 0.1437) -- cycle;
\fill[blue!15.0, opacity=0.7] (0.4500, 2.2000, 0.1437) -- (0.5000, 2.2000, 0.1492) -- (0.5000, 2.2500, 0.1449) -- (0.4500, 2.2500, 0.1393) -- cycle;
\fill[blue!15.0, opacity=0.7] (0.4500, 2.2500, 0.1393) -- (0.5000, 2.2500, 0.1449) -- (0.5000, 2.3000, 0.1403) -- (0.4500, 2.3000, 0.1348) -- cycle;
\fill[blue!15.0, opacity=0.7] (0.4500, 2.3000, 0.1348) -- (0.5000, 2.3000, 0.1403) -- (0.5000, 2.3500, 0.1355) -- (0.4500, 2.3500, 0.1300) -- cycle;
\fill[blue!15.0, opacity=0.7] (0.4500, 2.3500, 0.1300) -- (0.5000, 2.3500, 0.1355) -- (0.5000, 2.4000, 0.1305) -- (0.4500, 2.4000, 0.1250) -- cycle;
\fill[blue!15.0, opacity=0.7] (0.4500, 2.4000, 0.1250) -- (0.5000, 2.4000, 0.1305) -- (0.5000, 2.4500, 0.1254) -- (0.4500, 2.4500, 0.1198) -- cycle;
\fill[blue!15.0, opacity=0.7] (0.4500, 2.4500, 0.1198) -- (0.5000, 2.4500, 0.1254) -- (0.5000, 2.5000, 0.1200) -- (0.4500, 2.5000, 0.1145) -- cycle;
\fill[blue!15.0, opacity=0.7] (0.4500, 2.5000, 0.1145) -- (0.5000, 2.5000, 0.1200) -- (0.5000, 2.5500, 0.1145) -- (0.4500, 2.5500, 0.1090) -- cycle;
\fill[blue!15.0, opacity=0.7] (0.4500, 2.5500, 0.1090) -- (0.5000, 2.5500, 0.1145) -- (0.5000, 2.6000, 0.1088) -- (0.4500, 2.6000, 0.1033) -- cycle;
\fill[blue!15.0, opacity=0.7] (0.4500, 2.6000, 0.1033) -- (0.5000, 2.6000, 0.1088) -- (0.5000, 2.6500, 0.1030) -- (0.4500, 2.6500, 0.0975) -- cycle;
\fill[blue!15.0, opacity=0.7] (0.4500, 2.6500, 0.0975) -- (0.5000, 2.6500, 0.1030) -- (0.5000, 2.7000, 0.0971) -- (0.4500, 2.7000, 0.0916) -- cycle;
\fill[blue!15.0, opacity=0.7] (0.4500, 2.7000, 0.0916) -- (0.5000, 2.7000, 0.0971) -- (0.5000, 2.7500, 0.0911) -- (0.4500, 2.7500, 0.0855) -- cycle;
\fill[blue!15.0, opacity=0.7] (0.4500, 2.7500, 0.0855) -- (0.5000, 2.7500, 0.0911) -- (0.5000, 2.8000, 0.0849) -- (0.4500, 2.8000, 0.0794) -- cycle;
\fill[blue!15.0, opacity=0.7] (0.4500, 2.8000, 0.0794) -- (0.5000, 2.8000, 0.0849) -- (0.5000, 2.8500, 0.0788) -- (0.4500, 2.8500, 0.0733) -- cycle;
\fill[blue!15.0, opacity=0.7] (0.4500, 2.8500, 0.0733) -- (0.5000, 2.8500, 0.0788) -- (0.5000, 2.9000, 0.0725) -- (0.4500, 2.9000, 0.0670) -- cycle;
\fill[blue!15.0, opacity=0.7] (0.4500, 2.9000, 0.0670) -- (0.5000, 2.9000, 0.0725) -- (0.5000, 2.9500, 0.0663) -- (0.4500, 2.9500, 0.0608) -- cycle;
\fill[blue!15.0, opacity=0.7] (0.4500, 2.9500, 0.0608) -- (0.5000, 2.9500, 0.0663) -- (0.5000, 3.0000, 0.0600) -- (0.4500, 3.0000, 0.0545) -- cycle;
\fill[blue!15.0, opacity=0.7] (0.5000, 0.0000, 0.0600) -- (0.5500, 0.0000, 0.0654) -- (0.5500, 0.0500, 0.0716) -- (0.5000, 0.0500, 0.0663) -- cycle;
\fill[blue!15.0, opacity=0.7] (0.5000, 0.0500, 0.0663) -- (0.5500, 0.0500, 0.0716) -- (0.5500, 0.1000, 0.0779) -- (0.5000, 0.1000, 0.0725) -- cycle;
\fill[blue!15.0, opacity=0.7] (0.5000, 0.1000, 0.0725) -- (0.5500, 0.1000, 0.0779) -- (0.5500, 0.1500, 0.0841) -- (0.5000, 0.1500, 0.0788) -- cycle;
\fill[blue!15.0, opacity=0.7] (0.5000, 0.1500, 0.0788) -- (0.5500, 0.1500, 0.0841) -- (0.5500, 0.2000, 0.0903) -- (0.5000, 0.2000, 0.0849) -- cycle;
\fill[blue!15.0, opacity=0.7] (0.5000, 0.2000, 0.0849) -- (0.5500, 0.2000, 0.0903) -- (0.5500, 0.2500, 0.0964) -- (0.5000, 0.2500, 0.0911) -- cycle;
\fill[blue!15.0, opacity=0.7] (0.5000, 0.2500, 0.0911) -- (0.5500, 0.2500, 0.0964) -- (0.5500, 0.3000, 0.1024) -- (0.5000, 0.3000, 0.0971) -- cycle;
\fill[blue!15.0, opacity=0.7] (0.5000, 0.3000, 0.0971) -- (0.5500, 0.3000, 0.1024) -- (0.5500, 0.3500, 0.1084) -- (0.5000, 0.3500, 0.1030) -- cycle;
\fill[blue!15.0, opacity=0.7] (0.5000, 0.3500, 0.1030) -- (0.5500, 0.3500, 0.1084) -- (0.5500, 0.4000, 0.1142) -- (0.5000, 0.4000, 0.1088) -- cycle;
\fill[blue!15.0, opacity=0.7] (0.5000, 0.4000, 0.1088) -- (0.5500, 0.4000, 0.1142) -- (0.5500, 0.4500, 0.1198) -- (0.5000, 0.4500, 0.1145) -- cycle;
\fill[blue!15.0, opacity=0.7] (0.5000, 0.4500, 0.1145) -- (0.5500, 0.4500, 0.1198) -- (0.5500, 0.5000, 0.1254) -- (0.5000, 0.5000, 0.1200) -- cycle;
\fill[blue!15.0, opacity=0.7] (0.5000, 0.5000, 0.1200) -- (0.5500, 0.5000, 0.1254) -- (0.5500, 0.5500, 0.1307) -- (0.5000, 0.5500, 0.1254) -- cycle;
\fill[blue!15.0, opacity=0.7] (0.5000, 0.5500, 0.1254) -- (0.5500, 0.5500, 0.1307) -- (0.5500, 0.6000, 0.1359) -- (0.5000, 0.6000, 0.1305) -- cycle;
\fill[blue!15.0, opacity=0.7] (0.5000, 0.6000, 0.1305) -- (0.5500, 0.6000, 0.1359) -- (0.5500, 0.6500, 0.1409) -- (0.5000, 0.6500, 0.1355) -- cycle;
\fill[blue!15.0, opacity=0.7] (0.5000, 0.6500, 0.1355) -- (0.5500, 0.6500, 0.1409) -- (0.5500, 0.7000, 0.1457) -- (0.5000, 0.7000, 0.1403) -- cycle;
\fill[blue!15.0, opacity=0.7] (0.5000, 0.7000, 0.1403) -- (0.5500, 0.7000, 0.1457) -- (0.5500, 0.7500, 0.1502) -- (0.5000, 0.7500, 0.1449) -- cycle;
\fill[blue!15.0, opacity=0.7] (0.5000, 0.7500, 0.1449) -- (0.5500, 0.7500, 0.1502) -- (0.5500, 0.8000, 0.1545) -- (0.5000, 0.8000, 0.1492) -- cycle;
\fill[blue!15.0, opacity=0.7] (0.5000, 0.8000, 0.1492) -- (0.5500, 0.8000, 0.1545) -- (0.5500, 0.8500, 0.1586) -- (0.5000, 0.8500, 0.1533) -- cycle;
\fill[blue!15.0, opacity=0.7] (0.5000, 0.8500, 0.1533) -- (0.5500, 0.8500, 0.1586) -- (0.5500, 0.9000, 0.1624) -- (0.5000, 0.9000, 0.1571) -- cycle;
\fill[blue!15.0, opacity=0.7] (0.5000, 0.9000, 0.1571) -- (0.5500, 0.9000, 0.1624) -- (0.5500, 0.9500, 0.1660) -- (0.5000, 0.9500, 0.1606) -- cycle;
\fill[blue!15.0, opacity=0.7] (0.5000, 0.9500, 0.1606) -- (0.5500, 0.9500, 0.1660) -- (0.5500, 1.0000, 0.1693) -- (0.5000, 1.0000, 0.1639) -- cycle;
\fill[blue!15.0, opacity=0.7] (0.5000, 1.0000, 0.1639) -- (0.5500, 1.0000, 0.1693) -- (0.5500, 1.0500, 0.1723) -- (0.5000, 1.0500, 0.1669) -- cycle;
\fill[blue!15.0, opacity=0.7] (0.5000, 1.0500, 0.1669) -- (0.5500, 1.0500, 0.1723) -- (0.5500, 1.1000, 0.1750) -- (0.5000, 1.1000, 0.1696) -- cycle;
\fill[blue!15.0, opacity=0.7] (0.5000, 1.1000, 0.1696) -- (0.5500, 1.1000, 0.1750) -- (0.5500, 1.1500, 0.1774) -- (0.5000, 1.1500, 0.1720) -- cycle;
\fill[blue!15.0, opacity=0.7] (0.5000, 1.1500, 0.1720) -- (0.5500, 1.1500, 0.1774) -- (0.5500, 1.2000, 0.1795) -- (0.5000, 1.2000, 0.1741) -- cycle;
\fill[blue!15.0, opacity=0.7] (0.5000, 1.2000, 0.1741) -- (0.5500, 1.2000, 0.1795) -- (0.5500, 1.2500, 0.1813) -- (0.5000, 1.2500, 0.1759) -- cycle;
\fill[blue!15.0, opacity=0.7] (0.5000, 1.2500, 0.1759) -- (0.5500, 1.2500, 0.1813) -- (0.5500, 1.3000, 0.1827) -- (0.5000, 1.3000, 0.1774) -- cycle;
\fill[blue!15.0, opacity=0.7] (0.5000, 1.3000, 0.1774) -- (0.5500, 1.3000, 0.1827) -- (0.5500, 1.3500, 0.1839) -- (0.5000, 1.3500, 0.1785) -- cycle;
\fill[blue!15.0, opacity=0.7] (0.5000, 1.3500, 0.1785) -- (0.5500, 1.3500, 0.1839) -- (0.5500, 1.4000, 0.1847) -- (0.5000, 1.4000, 0.1793) -- cycle;
\fill[blue!15.0, opacity=0.7] (0.5000, 1.4000, 0.1793) -- (0.5500, 1.4000, 0.1847) -- (0.5500, 1.4500, 0.1852) -- (0.5000, 1.4500, 0.1798) -- cycle;
\fill[blue!15.1, opacity=0.7] (0.5000, 1.4500, 0.1798) -- (0.5500, 1.4500, 0.1852) -- (0.5500, 1.5000, 0.1854) -- (0.5000, 1.5000, 0.1800) -- cycle;
\fill[blue!15.1, opacity=0.7] (0.5000, 1.5000, 0.1800) -- (0.5500, 1.5000, 0.1854) -- (0.5500, 1.5500, 0.1852) -- (0.5000, 1.5500, 0.1798) -- cycle;
\fill[blue!15.2, opacity=0.7] (0.5000, 1.5500, 0.1798) -- (0.5500, 1.5500, 0.1852) -- (0.5500, 1.6000, 0.1847) -- (0.5000, 1.6000, 0.1793) -- cycle;
\fill[blue!15.2, opacity=0.7] (0.5000, 1.6000, 0.1793) -- (0.5500, 1.6000, 0.1847) -- (0.5500, 1.6500, 0.1839) -- (0.5000, 1.6500, 0.1785) -- cycle;
\fill[blue!15.1, opacity=0.7] (0.5000, 1.6500, 0.1785) -- (0.5500, 1.6500, 0.1839) -- (0.5500, 1.7000, 0.1827) -- (0.5000, 1.7000, 0.1774) -- cycle;
\fill[blue!15.1, opacity=0.7] (0.5000, 1.7000, 0.1774) -- (0.5500, 1.7000, 0.1827) -- (0.5500, 1.7500, 0.1813) -- (0.5000, 1.7500, 0.1759) -- cycle;
\fill[blue!15.0, opacity=0.7] (0.5000, 1.7500, 0.1759) -- (0.5500, 1.7500, 0.1813) -- (0.5500, 1.8000, 0.1795) -- (0.5000, 1.8000, 0.1741) -- cycle;
\fill[blue!15.0, opacity=0.7] (0.5000, 1.8000, 0.1741) -- (0.5500, 1.8000, 0.1795) -- (0.5500, 1.8500, 0.1774) -- (0.5000, 1.8500, 0.1720) -- cycle;
\fill[blue!15.0, opacity=0.7] (0.5000, 1.8500, 0.1720) -- (0.5500, 1.8500, 0.1774) -- (0.5500, 1.9000, 0.1750) -- (0.5000, 1.9000, 0.1696) -- cycle;
\fill[blue!15.0, opacity=0.7] (0.5000, 1.9000, 0.1696) -- (0.5500, 1.9000, 0.1750) -- (0.5500, 1.9500, 0.1723) -- (0.5000, 1.9500, 0.1669) -- cycle;
\fill[blue!15.0, opacity=0.7] (0.5000, 1.9500, 0.1669) -- (0.5500, 1.9500, 0.1723) -- (0.5500, 2.0000, 0.1693) -- (0.5000, 2.0000, 0.1639) -- cycle;
\fill[blue!15.0, opacity=0.7] (0.5000, 2.0000, 0.1639) -- (0.5500, 2.0000, 0.1693) -- (0.5500, 2.0500, 0.1660) -- (0.5000, 2.0500, 0.1606) -- cycle;
\fill[blue!15.0, opacity=0.7] (0.5000, 2.0500, 0.1606) -- (0.5500, 2.0500, 0.1660) -- (0.5500, 2.1000, 0.1624) -- (0.5000, 2.1000, 0.1571) -- cycle;
\fill[blue!15.0, opacity=0.7] (0.5000, 2.1000, 0.1571) -- (0.5500, 2.1000, 0.1624) -- (0.5500, 2.1500, 0.1586) -- (0.5000, 2.1500, 0.1533) -- cycle;
\fill[blue!15.0, opacity=0.7] (0.5000, 2.1500, 0.1533) -- (0.5500, 2.1500, 0.1586) -- (0.5500, 2.2000, 0.1545) -- (0.5000, 2.2000, 0.1492) -- cycle;
\fill[blue!15.0, opacity=0.7] (0.5000, 2.2000, 0.1492) -- (0.5500, 2.2000, 0.1545) -- (0.5500, 2.2500, 0.1502) -- (0.5000, 2.2500, 0.1449) -- cycle;
\fill[blue!15.0, opacity=0.7] (0.5000, 2.2500, 0.1449) -- (0.5500, 2.2500, 0.1502) -- (0.5500, 2.3000, 0.1457) -- (0.5000, 2.3000, 0.1403) -- cycle;
\fill[blue!15.0, opacity=0.7] (0.5000, 2.3000, 0.1403) -- (0.5500, 2.3000, 0.1457) -- (0.5500, 2.3500, 0.1409) -- (0.5000, 2.3500, 0.1355) -- cycle;
\fill[blue!15.0, opacity=0.7] (0.5000, 2.3500, 0.1355) -- (0.5500, 2.3500, 0.1409) -- (0.5500, 2.4000, 0.1359) -- (0.5000, 2.4000, 0.1305) -- cycle;
\fill[blue!15.0, opacity=0.7] (0.5000, 2.4000, 0.1305) -- (0.5500, 2.4000, 0.1359) -- (0.5500, 2.4500, 0.1307) -- (0.5000, 2.4500, 0.1254) -- cycle;
\fill[blue!15.0, opacity=0.7] (0.5000, 2.4500, 0.1254) -- (0.5500, 2.4500, 0.1307) -- (0.5500, 2.5000, 0.1254) -- (0.5000, 2.5000, 0.1200) -- cycle;
\fill[blue!15.0, opacity=0.7] (0.5000, 2.5000, 0.1200) -- (0.5500, 2.5000, 0.1254) -- (0.5500, 2.5500, 0.1198) -- (0.5000, 2.5500, 0.1145) -- cycle;
\fill[blue!15.0, opacity=0.7] (0.5000, 2.5500, 0.1145) -- (0.5500, 2.5500, 0.1198) -- (0.5500, 2.6000, 0.1142) -- (0.5000, 2.6000, 0.1088) -- cycle;
\fill[blue!15.0, opacity=0.7] (0.5000, 2.6000, 0.1088) -- (0.5500, 2.6000, 0.1142) -- (0.5500, 2.6500, 0.1084) -- (0.5000, 2.6500, 0.1030) -- cycle;
\fill[blue!15.0, opacity=0.7] (0.5000, 2.6500, 0.1030) -- (0.5500, 2.6500, 0.1084) -- (0.5500, 2.7000, 0.1024) -- (0.5000, 2.7000, 0.0971) -- cycle;
\fill[blue!15.0, opacity=0.7] (0.5000, 2.7000, 0.0971) -- (0.5500, 2.7000, 0.1024) -- (0.5500, 2.7500, 0.0964) -- (0.5000, 2.7500, 0.0911) -- cycle;
\fill[blue!15.0, opacity=0.7] (0.5000, 2.7500, 0.0911) -- (0.5500, 2.7500, 0.0964) -- (0.5500, 2.8000, 0.0903) -- (0.5000, 2.8000, 0.0849) -- cycle;
\fill[blue!15.0, opacity=0.7] (0.5000, 2.8000, 0.0849) -- (0.5500, 2.8000, 0.0903) -- (0.5500, 2.8500, 0.0841) -- (0.5000, 2.8500, 0.0788) -- cycle;
\fill[blue!15.0, opacity=0.7] (0.5000, 2.8500, 0.0788) -- (0.5500, 2.8500, 0.0841) -- (0.5500, 2.9000, 0.0779) -- (0.5000, 2.9000, 0.0725) -- cycle;
\fill[blue!15.0, opacity=0.7] (0.5000, 2.9000, 0.0725) -- (0.5500, 2.9000, 0.0779) -- (0.5500, 2.9500, 0.0716) -- (0.5000, 2.9500, 0.0663) -- cycle;
\fill[blue!15.0, opacity=0.7] (0.5000, 2.9500, 0.0663) -- (0.5500, 2.9500, 0.0716) -- (0.5500, 3.0000, 0.0654) -- (0.5000, 3.0000, 0.0600) -- cycle;
\fill[blue!15.0, opacity=0.7] (0.5500, 0.0000, 0.0654) -- (0.6000, 0.0000, 0.0705) -- (0.6000, 0.0500, 0.0768) -- (0.5500, 0.0500, 0.0716) -- cycle;
\fill[blue!15.0, opacity=0.7] (0.5500, 0.0500, 0.0716) -- (0.6000, 0.0500, 0.0768) -- (0.6000, 0.1000, 0.0831) -- (0.5500, 0.1000, 0.0779) -- cycle;
\fill[blue!15.0, opacity=0.7] (0.5500, 0.1000, 0.0779) -- (0.6000, 0.1000, 0.0831) -- (0.6000, 0.1500, 0.0893) -- (0.5500, 0.1500, 0.0841) -- cycle;
\fill[blue!15.0, opacity=0.7] (0.5500, 0.1500, 0.0841) -- (0.6000, 0.1500, 0.0893) -- (0.6000, 0.2000, 0.0955) -- (0.5500, 0.2000, 0.0903) -- cycle;
\fill[blue!15.0, opacity=0.7] (0.5500, 0.2000, 0.0903) -- (0.6000, 0.2000, 0.0955) -- (0.6000, 0.2500, 0.1016) -- (0.5500, 0.2500, 0.0964) -- cycle;
\fill[blue!15.0, opacity=0.7] (0.5500, 0.2500, 0.0964) -- (0.6000, 0.2500, 0.1016) -- (0.6000, 0.3000, 0.1076) -- (0.5500, 0.3000, 0.1024) -- cycle;
\fill[blue!15.0, opacity=0.7] (0.5500, 0.3000, 0.1024) -- (0.6000, 0.3000, 0.1076) -- (0.6000, 0.3500, 0.1135) -- (0.5500, 0.3500, 0.1084) -- cycle;
\fill[blue!15.0, opacity=0.7] (0.5500, 0.3500, 0.1084) -- (0.6000, 0.3500, 0.1135) -- (0.6000, 0.4000, 0.1193) -- (0.5500, 0.4000, 0.1142) -- cycle;
\fill[blue!15.0, opacity=0.7] (0.5500, 0.4000, 0.1142) -- (0.6000, 0.4000, 0.1193) -- (0.6000, 0.4500, 0.1250) -- (0.5500, 0.4500, 0.1198) -- cycle;
\fill[blue!15.0, opacity=0.7] (0.5500, 0.4500, 0.1198) -- (0.6000, 0.4500, 0.1250) -- (0.6000, 0.5000, 0.1305) -- (0.5500, 0.5000, 0.1254) -- cycle;
\fill[blue!15.0, opacity=0.7] (0.5500, 0.5000, 0.1254) -- (0.6000, 0.5000, 0.1305) -- (0.6000, 0.5500, 0.1359) -- (0.5500, 0.5500, 0.1307) -- cycle;
\fill[blue!15.0, opacity=0.7] (0.5500, 0.5500, 0.1307) -- (0.6000, 0.5500, 0.1359) -- (0.6000, 0.6000, 0.1411) -- (0.5500, 0.6000, 0.1359) -- cycle;
\fill[blue!15.0, opacity=0.7] (0.5500, 0.6000, 0.1359) -- (0.6000, 0.6000, 0.1411) -- (0.6000, 0.6500, 0.1461) -- (0.5500, 0.6500, 0.1409) -- cycle;
\fill[blue!15.0, opacity=0.7] (0.5500, 0.6500, 0.1409) -- (0.6000, 0.6500, 0.1461) -- (0.6000, 0.7000, 0.1508) -- (0.5500, 0.7000, 0.1457) -- cycle;
\fill[blue!15.0, opacity=0.7] (0.5500, 0.7000, 0.1457) -- (0.6000, 0.7000, 0.1508) -- (0.6000, 0.7500, 0.1554) -- (0.5500, 0.7500, 0.1502) -- cycle;
\fill[blue!15.0, opacity=0.7] (0.5500, 0.7500, 0.1502) -- (0.6000, 0.7500, 0.1554) -- (0.6000, 0.8000, 0.1597) -- (0.5500, 0.8000, 0.1545) -- cycle;
\fill[blue!15.0, opacity=0.7] (0.5500, 0.8000, 0.1545) -- (0.6000, 0.8000, 0.1597) -- (0.6000, 0.8500, 0.1638) -- (0.5500, 0.8500, 0.1586) -- cycle;
\fill[blue!15.0, opacity=0.7] (0.5500, 0.8500, 0.1586) -- (0.6000, 0.8500, 0.1638) -- (0.6000, 0.9000, 0.1676) -- (0.5500, 0.9000, 0.1624) -- cycle;
\fill[blue!15.0, opacity=0.7] (0.5500, 0.9000, 0.1624) -- (0.6000, 0.9000, 0.1676) -- (0.6000, 0.9500, 0.1712) -- (0.5500, 0.9500, 0.1660) -- cycle;
\fill[blue!15.0, opacity=0.7] (0.5500, 0.9500, 0.1660) -- (0.6000, 0.9500, 0.1712) -- (0.6000, 1.0000, 0.1745) -- (0.5500, 1.0000, 0.1693) -- cycle;
\fill[blue!15.0, opacity=0.7] (0.5500, 1.0000, 0.1693) -- (0.6000, 1.0000, 0.1745) -- (0.6000, 1.0500, 0.1775) -- (0.5500, 1.0500, 0.1723) -- cycle;
\fill[blue!15.0, opacity=0.7] (0.5500, 1.0500, 0.1723) -- (0.6000, 1.0500, 0.1775) -- (0.6000, 1.1000, 0.1802) -- (0.5500, 1.1000, 0.1750) -- cycle;
\fill[blue!15.0, opacity=0.7] (0.5500, 1.1000, 0.1750) -- (0.6000, 1.1000, 0.1802) -- (0.6000, 1.1500, 0.1826) -- (0.5500, 1.1500, 0.1774) -- cycle;
\fill[blue!15.0, opacity=0.7] (0.5500, 1.1500, 0.1774) -- (0.6000, 1.1500, 0.1826) -- (0.6000, 1.2000, 0.1847) -- (0.5500, 1.2000, 0.1795) -- cycle;
\fill[blue!15.0, opacity=0.7] (0.5500, 1.2000, 0.1795) -- (0.6000, 1.2000, 0.1847) -- (0.6000, 1.2500, 0.1864) -- (0.5500, 1.2500, 0.1813) -- cycle;
\fill[blue!15.1, opacity=0.7] (0.5500, 1.2500, 0.1813) -- (0.6000, 1.2500, 0.1864) -- (0.6000, 1.3000, 0.1879) -- (0.5500, 1.3000, 0.1827) -- cycle;
\fill[blue!15.5, opacity=0.7] (0.5500, 1.3000, 0.1827) -- (0.6000, 1.3000, 0.1879) -- (0.6000, 1.3500, 0.1891) -- (0.5500, 1.3500, 0.1839) -- cycle;
\fill[blue!16.0, opacity=0.7] (0.5500, 1.3500, 0.1839) -- (0.6000, 1.3500, 0.1891) -- (0.6000, 1.4000, 0.1899) -- (0.5500, 1.4000, 0.1847) -- cycle;
\fill[blue!16.2, opacity=0.7] (0.5500, 1.4000, 0.1847) -- (0.6000, 1.4000, 0.1899) -- (0.6000, 1.4500, 0.1904) -- (0.5500, 1.4500, 0.1852) -- cycle;
\fill[blue!16.1, opacity=0.7] (0.5500, 1.4500, 0.1852) -- (0.6000, 1.4500, 0.1904) -- (0.6000, 1.5000, 0.1905) -- (0.5500, 1.5000, 0.1854) -- cycle;
\fill[blue!15.8, opacity=0.7] (0.5500, 1.5000, 0.1854) -- (0.6000, 1.5000, 0.1905) -- (0.6000, 1.5500, 0.1904) -- (0.5500, 1.5500, 0.1852) -- cycle;
\fill[blue!15.6, opacity=0.7] (0.5500, 1.5500, 0.1852) -- (0.6000, 1.5500, 0.1904) -- (0.6000, 1.6000, 0.1899) -- (0.5500, 1.6000, 0.1847) -- cycle;
\fill[blue!15.4, opacity=0.7] (0.5500, 1.6000, 0.1847) -- (0.6000, 1.6000, 0.1899) -- (0.6000, 1.6500, 0.1891) -- (0.5500, 1.6500, 0.1839) -- cycle;
\fill[blue!15.4, opacity=0.7] (0.5500, 1.6500, 0.1839) -- (0.6000, 1.6500, 0.1891) -- (0.6000, 1.7000, 0.1879) -- (0.5500, 1.7000, 0.1827) -- cycle;
\fill[blue!15.4, opacity=0.7] (0.5500, 1.7000, 0.1827) -- (0.6000, 1.7000, 0.1879) -- (0.6000, 1.7500, 0.1864) -- (0.5500, 1.7500, 0.1813) -- cycle;
\fill[blue!15.4, opacity=0.7] (0.5500, 1.7500, 0.1813) -- (0.6000, 1.7500, 0.1864) -- (0.6000, 1.8000, 0.1847) -- (0.5500, 1.8000, 0.1795) -- cycle;
\fill[blue!15.4, opacity=0.7] (0.5500, 1.8000, 0.1795) -- (0.6000, 1.8000, 0.1847) -- (0.6000, 1.8500, 0.1826) -- (0.5500, 1.8500, 0.1774) -- cycle;
\fill[blue!15.2, opacity=0.7] (0.5500, 1.8500, 0.1774) -- (0.6000, 1.8500, 0.1826) -- (0.6000, 1.9000, 0.1802) -- (0.5500, 1.9000, 0.1750) -- cycle;
\fill[blue!15.1, opacity=0.7] (0.5500, 1.9000, 0.1750) -- (0.6000, 1.9000, 0.1802) -- (0.6000, 1.9500, 0.1775) -- (0.5500, 1.9500, 0.1723) -- cycle;
\fill[blue!15.0, opacity=0.7] (0.5500, 1.9500, 0.1723) -- (0.6000, 1.9500, 0.1775) -- (0.6000, 2.0000, 0.1745) -- (0.5500, 2.0000, 0.1693) -- cycle;
\fill[blue!15.0, opacity=0.7] (0.5500, 2.0000, 0.1693) -- (0.6000, 2.0000, 0.1745) -- (0.6000, 2.0500, 0.1712) -- (0.5500, 2.0500, 0.1660) -- cycle;
\fill[blue!15.0, opacity=0.7] (0.5500, 2.0500, 0.1660) -- (0.6000, 2.0500, 0.1712) -- (0.6000, 2.1000, 0.1676) -- (0.5500, 2.1000, 0.1624) -- cycle;
\fill[blue!15.0, opacity=0.7] (0.5500, 2.1000, 0.1624) -- (0.6000, 2.1000, 0.1676) -- (0.6000, 2.1500, 0.1638) -- (0.5500, 2.1500, 0.1586) -- cycle;
\fill[blue!15.0, opacity=0.7] (0.5500, 2.1500, 0.1586) -- (0.6000, 2.1500, 0.1638) -- (0.6000, 2.2000, 0.1597) -- (0.5500, 2.2000, 0.1545) -- cycle;
\fill[blue!15.0, opacity=0.7] (0.5500, 2.2000, 0.1545) -- (0.6000, 2.2000, 0.1597) -- (0.6000, 2.2500, 0.1554) -- (0.5500, 2.2500, 0.1502) -- cycle;
\fill[blue!15.0, opacity=0.7] (0.5500, 2.2500, 0.1502) -- (0.6000, 2.2500, 0.1554) -- (0.6000, 2.3000, 0.1508) -- (0.5500, 2.3000, 0.1457) -- cycle;
\fill[blue!15.0, opacity=0.7] (0.5500, 2.3000, 0.1457) -- (0.6000, 2.3000, 0.1508) -- (0.6000, 2.3500, 0.1461) -- (0.5500, 2.3500, 0.1409) -- cycle;
\fill[blue!15.0, opacity=0.7] (0.5500, 2.3500, 0.1409) -- (0.6000, 2.3500, 0.1461) -- (0.6000, 2.4000, 0.1411) -- (0.5500, 2.4000, 0.1359) -- cycle;
\fill[blue!15.0, opacity=0.7] (0.5500, 2.4000, 0.1359) -- (0.6000, 2.4000, 0.1411) -- (0.6000, 2.4500, 0.1359) -- (0.5500, 2.4500, 0.1307) -- cycle;
\fill[blue!15.0, opacity=0.7] (0.5500, 2.4500, 0.1307) -- (0.6000, 2.4500, 0.1359) -- (0.6000, 2.5000, 0.1305) -- (0.5500, 2.5000, 0.1254) -- cycle;
\fill[blue!15.0, opacity=0.7] (0.5500, 2.5000, 0.1254) -- (0.6000, 2.5000, 0.1305) -- (0.6000, 2.5500, 0.1250) -- (0.5500, 2.5500, 0.1198) -- cycle;
\fill[blue!15.0, opacity=0.7] (0.5500, 2.5500, 0.1198) -- (0.6000, 2.5500, 0.1250) -- (0.6000, 2.6000, 0.1193) -- (0.5500, 2.6000, 0.1142) -- cycle;
\fill[blue!15.0, opacity=0.7] (0.5500, 2.6000, 0.1142) -- (0.6000, 2.6000, 0.1193) -- (0.6000, 2.6500, 0.1135) -- (0.5500, 2.6500, 0.1084) -- cycle;
\fill[blue!15.0, opacity=0.7] (0.5500, 2.6500, 0.1084) -- (0.6000, 2.6500, 0.1135) -- (0.6000, 2.7000, 0.1076) -- (0.5500, 2.7000, 0.1024) -- cycle;
\fill[blue!15.0, opacity=0.7] (0.5500, 2.7000, 0.1024) -- (0.6000, 2.7000, 0.1076) -- (0.6000, 2.7500, 0.1016) -- (0.5500, 2.7500, 0.0964) -- cycle;
\fill[blue!15.0, opacity=0.7] (0.5500, 2.7500, 0.0964) -- (0.6000, 2.7500, 0.1016) -- (0.6000, 2.8000, 0.0955) -- (0.5500, 2.8000, 0.0903) -- cycle;
\fill[blue!15.0, opacity=0.7] (0.5500, 2.8000, 0.0903) -- (0.6000, 2.8000, 0.0955) -- (0.6000, 2.8500, 0.0893) -- (0.5500, 2.8500, 0.0841) -- cycle;
\fill[blue!15.0, opacity=0.7] (0.5500, 2.8500, 0.0841) -- (0.6000, 2.8500, 0.0893) -- (0.6000, 2.9000, 0.0831) -- (0.5500, 2.9000, 0.0779) -- cycle;
\fill[blue!15.0, opacity=0.7] (0.5500, 2.9000, 0.0779) -- (0.6000, 2.9000, 0.0831) -- (0.6000, 2.9500, 0.0768) -- (0.5500, 2.9500, 0.0716) -- cycle;
\fill[blue!15.0, opacity=0.7] (0.5500, 2.9500, 0.0716) -- (0.6000, 2.9500, 0.0768) -- (0.6000, 3.0000, 0.0705) -- (0.5500, 3.0000, 0.0654) -- cycle;
\fill[blue!15.0, opacity=0.7] (0.6000, 0.0000, 0.0705) -- (0.6500, 0.0000, 0.0755) -- (0.6500, 0.0500, 0.0818) -- (0.6000, 0.0500, 0.0768) -- cycle;
\fill[blue!15.0, opacity=0.7] (0.6000, 0.0500, 0.0768) -- (0.6500, 0.0500, 0.0818) -- (0.6500, 0.1000, 0.0881) -- (0.6000, 0.1000, 0.0831) -- cycle;
\fill[blue!15.0, opacity=0.7] (0.6000, 0.1000, 0.0831) -- (0.6500, 0.1000, 0.0881) -- (0.6500, 0.1500, 0.0943) -- (0.6000, 0.1500, 0.0893) -- cycle;
\fill[blue!15.0, opacity=0.7] (0.6000, 0.1500, 0.0893) -- (0.6500, 0.1500, 0.0943) -- (0.6500, 0.2000, 0.1005) -- (0.6000, 0.2000, 0.0955) -- cycle;
\fill[blue!15.0, opacity=0.7] (0.6000, 0.2000, 0.0955) -- (0.6500, 0.2000, 0.1005) -- (0.6500, 0.2500, 0.1066) -- (0.6000, 0.2500, 0.1016) -- cycle;
\fill[blue!15.0, opacity=0.7] (0.6000, 0.2500, 0.1016) -- (0.6500, 0.2500, 0.1066) -- (0.6500, 0.3000, 0.1126) -- (0.6000, 0.3000, 0.1076) -- cycle;
\fill[blue!15.0, opacity=0.7] (0.6000, 0.3000, 0.1076) -- (0.6500, 0.3000, 0.1126) -- (0.6500, 0.3500, 0.1185) -- (0.6000, 0.3500, 0.1135) -- cycle;
\fill[blue!15.0, opacity=0.7] (0.6000, 0.3500, 0.1135) -- (0.6500, 0.3500, 0.1185) -- (0.6500, 0.4000, 0.1243) -- (0.6000, 0.4000, 0.1193) -- cycle;
\fill[blue!15.0, opacity=0.7] (0.6000, 0.4000, 0.1193) -- (0.6500, 0.4000, 0.1243) -- (0.6500, 0.4500, 0.1300) -- (0.6000, 0.4500, 0.1250) -- cycle;
\fill[blue!15.0, opacity=0.7] (0.6000, 0.4500, 0.1250) -- (0.6500, 0.4500, 0.1300) -- (0.6500, 0.5000, 0.1355) -- (0.6000, 0.5000, 0.1305) -- cycle;
\fill[blue!15.0, opacity=0.7] (0.6000, 0.5000, 0.1305) -- (0.6500, 0.5000, 0.1355) -- (0.6500, 0.5500, 0.1409) -- (0.6000, 0.5500, 0.1359) -- cycle;
\fill[blue!15.0, opacity=0.7] (0.6000, 0.5500, 0.1359) -- (0.6500, 0.5500, 0.1409) -- (0.6500, 0.6000, 0.1461) -- (0.6000, 0.6000, 0.1411) -- cycle;
\fill[blue!15.0, opacity=0.7] (0.6000, 0.6000, 0.1411) -- (0.6500, 0.6000, 0.1461) -- (0.6500, 0.6500, 0.1510) -- (0.6000, 0.6500, 0.1461) -- cycle;
\fill[blue!15.0, opacity=0.7] (0.6000, 0.6500, 0.1461) -- (0.6500, 0.6500, 0.1510) -- (0.6500, 0.7000, 0.1558) -- (0.6000, 0.7000, 0.1508) -- cycle;
\fill[blue!15.0, opacity=0.7] (0.6000, 0.7000, 0.1508) -- (0.6500, 0.7000, 0.1558) -- (0.6500, 0.7500, 0.1604) -- (0.6000, 0.7500, 0.1554) -- cycle;
\fill[blue!15.0, opacity=0.7] (0.6000, 0.7500, 0.1554) -- (0.6500, 0.7500, 0.1604) -- (0.6500, 0.8000, 0.1647) -- (0.6000, 0.8000, 0.1597) -- cycle;
\fill[blue!15.0, opacity=0.7] (0.6000, 0.8000, 0.1597) -- (0.6500, 0.8000, 0.1647) -- (0.6500, 0.8500, 0.1688) -- (0.6000, 0.8500, 0.1638) -- cycle;
\fill[blue!15.0, opacity=0.7] (0.6000, 0.8500, 0.1638) -- (0.6500, 0.8500, 0.1688) -- (0.6500, 0.9000, 0.1726) -- (0.6000, 0.9000, 0.1676) -- cycle;
\fill[blue!15.0, opacity=0.7] (0.6000, 0.9000, 0.1676) -- (0.6500, 0.9000, 0.1726) -- (0.6500, 0.9500, 0.1762) -- (0.6000, 0.9500, 0.1712) -- cycle;
\fill[blue!15.0, opacity=0.7] (0.6000, 0.9500, 0.1712) -- (0.6500, 0.9500, 0.1762) -- (0.6500, 1.0000, 0.1794) -- (0.6000, 1.0000, 0.1745) -- cycle;
\fill[blue!15.0, opacity=0.7] (0.6000, 1.0000, 0.1745) -- (0.6500, 1.0000, 0.1794) -- (0.6500, 1.0500, 0.1824) -- (0.6000, 1.0500, 0.1775) -- cycle;
\fill[blue!15.0, opacity=0.7] (0.6000, 1.0500, 0.1775) -- (0.6500, 1.0500, 0.1824) -- (0.6500, 1.1000, 0.1851) -- (0.6000, 1.1000, 0.1802) -- cycle;
\fill[blue!15.0, opacity=0.7] (0.6000, 1.1000, 0.1802) -- (0.6500, 1.1000, 0.1851) -- (0.6500, 1.1500, 0.1875) -- (0.6000, 1.1500, 0.1826) -- cycle;
\fill[blue!15.2, opacity=0.7] (0.6000, 1.1500, 0.1826) -- (0.6500, 1.1500, 0.1875) -- (0.6500, 1.2000, 0.1896) -- (0.6000, 1.2000, 0.1847) -- cycle;
\fill[blue!16.2, opacity=0.7] (0.6000, 1.2000, 0.1847) -- (0.6500, 1.2000, 0.1896) -- (0.6500, 1.2500, 0.1914) -- (0.6000, 1.2500, 0.1864) -- cycle;
\fill[blue!17.0, opacity=0.7] (0.6000, 1.2500, 0.1864) -- (0.6500, 1.2500, 0.1914) -- (0.6500, 1.3000, 0.1929) -- (0.6000, 1.3000, 0.1879) -- cycle;
\fill[blue!16.4, opacity=0.7] (0.6000, 1.3000, 0.1879) -- (0.6500, 1.3000, 0.1929) -- (0.6500, 1.3500, 0.1940) -- (0.6000, 1.3500, 0.1891) -- cycle;
\fill[blue!15.5, opacity=0.7] (0.6000, 1.3500, 0.1891) -- (0.6500, 1.3500, 0.1940) -- (0.6500, 1.4000, 0.1949) -- (0.6000, 1.4000, 0.1899) -- cycle;
\fill[blue!15.1, opacity=0.7] (0.6000, 1.4000, 0.1899) -- (0.6500, 1.4000, 0.1949) -- (0.6500, 1.4500, 0.1954) -- (0.6000, 1.4500, 0.1904) -- cycle;
\fill[blue!15.0, opacity=0.7] (0.6000, 1.4500, 0.1904) -- (0.6500, 1.4500, 0.1954) -- (0.6500, 1.5000, 0.1955) -- (0.6000, 1.5000, 0.1905) -- cycle;
\fill[blue!15.0, opacity=0.7] (0.6000, 1.5000, 0.1905) -- (0.6500, 1.5000, 0.1955) -- (0.6500, 1.5500, 0.1954) -- (0.6000, 1.5500, 0.1904) -- cycle;
\fill[blue!15.0, opacity=0.7] (0.6000, 1.5500, 0.1904) -- (0.6500, 1.5500, 0.1954) -- (0.6500, 1.6000, 0.1949) -- (0.6000, 1.6000, 0.1899) -- cycle;
\fill[blue!15.0, opacity=0.7] (0.6000, 1.6000, 0.1899) -- (0.6500, 1.6000, 0.1949) -- (0.6500, 1.6500, 0.1940) -- (0.6000, 1.6500, 0.1891) -- cycle;
\fill[blue!15.0, opacity=0.7] (0.6000, 1.6500, 0.1891) -- (0.6500, 1.6500, 0.1940) -- (0.6500, 1.7000, 0.1929) -- (0.6000, 1.7000, 0.1879) -- cycle;
\fill[blue!15.0, opacity=0.7] (0.6000, 1.7000, 0.1879) -- (0.6500, 1.7000, 0.1929) -- (0.6500, 1.7500, 0.1914) -- (0.6000, 1.7500, 0.1864) -- cycle;
\fill[blue!15.0, opacity=0.7] (0.6000, 1.7500, 0.1864) -- (0.6500, 1.7500, 0.1914) -- (0.6500, 1.8000, 0.1896) -- (0.6000, 1.8000, 0.1847) -- cycle;
\fill[blue!15.0, opacity=0.7] (0.6000, 1.8000, 0.1847) -- (0.6500, 1.8000, 0.1896) -- (0.6500, 1.8500, 0.1875) -- (0.6000, 1.8500, 0.1826) -- cycle;
\fill[blue!15.0, opacity=0.7] (0.6000, 1.8500, 0.1826) -- (0.6500, 1.8500, 0.1875) -- (0.6500, 1.9000, 0.1851) -- (0.6000, 1.9000, 0.1802) -- cycle;
\fill[blue!15.1, opacity=0.7] (0.6000, 1.9000, 0.1802) -- (0.6500, 1.9000, 0.1851) -- (0.6500, 1.9500, 0.1824) -- (0.6000, 1.9500, 0.1775) -- cycle;
\fill[blue!15.2, opacity=0.7] (0.6000, 1.9500, 0.1775) -- (0.6500, 1.9500, 0.1824) -- (0.6500, 2.0000, 0.1794) -- (0.6000, 2.0000, 0.1745) -- cycle;
\fill[blue!15.1, opacity=0.7] (0.6000, 2.0000, 0.1745) -- (0.6500, 2.0000, 0.1794) -- (0.6500, 2.0500, 0.1762) -- (0.6000, 2.0500, 0.1712) -- cycle;
\fill[blue!15.0, opacity=0.7] (0.6000, 2.0500, 0.1712) -- (0.6500, 2.0500, 0.1762) -- (0.6500, 2.1000, 0.1726) -- (0.6000, 2.1000, 0.1676) -- cycle;
\fill[blue!15.0, opacity=0.7] (0.6000, 2.1000, 0.1676) -- (0.6500, 2.1000, 0.1726) -- (0.6500, 2.1500, 0.1688) -- (0.6000, 2.1500, 0.1638) -- cycle;
\fill[blue!15.0, opacity=0.7] (0.6000, 2.1500, 0.1638) -- (0.6500, 2.1500, 0.1688) -- (0.6500, 2.2000, 0.1647) -- (0.6000, 2.2000, 0.1597) -- cycle;
\fill[blue!15.0, opacity=0.7] (0.6000, 2.2000, 0.1597) -- (0.6500, 2.2000, 0.1647) -- (0.6500, 2.2500, 0.1604) -- (0.6000, 2.2500, 0.1554) -- cycle;
\fill[blue!15.0, opacity=0.7] (0.6000, 2.2500, 0.1554) -- (0.6500, 2.2500, 0.1604) -- (0.6500, 2.3000, 0.1558) -- (0.6000, 2.3000, 0.1508) -- cycle;
\fill[blue!15.0, opacity=0.7] (0.6000, 2.3000, 0.1508) -- (0.6500, 2.3000, 0.1558) -- (0.6500, 2.3500, 0.1510) -- (0.6000, 2.3500, 0.1461) -- cycle;
\fill[blue!15.0, opacity=0.7] (0.6000, 2.3500, 0.1461) -- (0.6500, 2.3500, 0.1510) -- (0.6500, 2.4000, 0.1461) -- (0.6000, 2.4000, 0.1411) -- cycle;
\fill[blue!15.0, opacity=0.7] (0.6000, 2.4000, 0.1411) -- (0.6500, 2.4000, 0.1461) -- (0.6500, 2.4500, 0.1409) -- (0.6000, 2.4500, 0.1359) -- cycle;
\fill[blue!15.0, opacity=0.7] (0.6000, 2.4500, 0.1359) -- (0.6500, 2.4500, 0.1409) -- (0.6500, 2.5000, 0.1355) -- (0.6000, 2.5000, 0.1305) -- cycle;
\fill[blue!15.0, opacity=0.7] (0.6000, 2.5000, 0.1305) -- (0.6500, 2.5000, 0.1355) -- (0.6500, 2.5500, 0.1300) -- (0.6000, 2.5500, 0.1250) -- cycle;
\fill[blue!15.0, opacity=0.7] (0.6000, 2.5500, 0.1250) -- (0.6500, 2.5500, 0.1300) -- (0.6500, 2.6000, 0.1243) -- (0.6000, 2.6000, 0.1193) -- cycle;
\fill[blue!15.0, opacity=0.7] (0.6000, 2.6000, 0.1193) -- (0.6500, 2.6000, 0.1243) -- (0.6500, 2.6500, 0.1185) -- (0.6000, 2.6500, 0.1135) -- cycle;
\fill[blue!15.0, opacity=0.7] (0.6000, 2.6500, 0.1135) -- (0.6500, 2.6500, 0.1185) -- (0.6500, 2.7000, 0.1126) -- (0.6000, 2.7000, 0.1076) -- cycle;
\fill[blue!15.0, opacity=0.7] (0.6000, 2.7000, 0.1076) -- (0.6500, 2.7000, 0.1126) -- (0.6500, 2.7500, 0.1066) -- (0.6000, 2.7500, 0.1016) -- cycle;
\fill[blue!15.0, opacity=0.7] (0.6000, 2.7500, 0.1016) -- (0.6500, 2.7500, 0.1066) -- (0.6500, 2.8000, 0.1005) -- (0.6000, 2.8000, 0.0955) -- cycle;
\fill[blue!15.0, opacity=0.7] (0.6000, 2.8000, 0.0955) -- (0.6500, 2.8000, 0.1005) -- (0.6500, 2.8500, 0.0943) -- (0.6000, 2.8500, 0.0893) -- cycle;
\fill[blue!15.0, opacity=0.7] (0.6000, 2.8500, 0.0893) -- (0.6500, 2.8500, 0.0943) -- (0.6500, 2.9000, 0.0881) -- (0.6000, 2.9000, 0.0831) -- cycle;
\fill[blue!15.0, opacity=0.7] (0.6000, 2.9000, 0.0831) -- (0.6500, 2.9000, 0.0881) -- (0.6500, 2.9500, 0.0818) -- (0.6000, 2.9500, 0.0768) -- cycle;
\fill[blue!15.0, opacity=0.7] (0.6000, 2.9500, 0.0768) -- (0.6500, 2.9500, 0.0818) -- (0.6500, 3.0000, 0.0755) -- (0.6000, 3.0000, 0.0705) -- cycle;
\fill[blue!15.0, opacity=0.7] (0.6500, 0.0000, 0.0755) -- (0.7000, 0.0000, 0.0803) -- (0.7000, 0.0500, 0.0866) -- (0.6500, 0.0500, 0.0818) -- cycle;
\fill[blue!15.0, opacity=0.7] (0.6500, 0.0500, 0.0818) -- (0.7000, 0.0500, 0.0866) -- (0.7000, 0.1000, 0.0928) -- (0.6500, 0.1000, 0.0881) -- cycle;
\fill[blue!15.0, opacity=0.7] (0.6500, 0.1000, 0.0881) -- (0.7000, 0.1000, 0.0928) -- (0.7000, 0.1500, 0.0991) -- (0.6500, 0.1500, 0.0943) -- cycle;
\fill[blue!15.0, opacity=0.7] (0.6500, 0.1500, 0.0943) -- (0.7000, 0.1500, 0.0991) -- (0.7000, 0.2000, 0.1052) -- (0.6500, 0.2000, 0.1005) -- cycle;
\fill[blue!15.0, opacity=0.7] (0.6500, 0.2000, 0.1005) -- (0.7000, 0.2000, 0.1052) -- (0.7000, 0.2500, 0.1114) -- (0.6500, 0.2500, 0.1066) -- cycle;
\fill[blue!15.0, opacity=0.7] (0.6500, 0.2500, 0.1066) -- (0.7000, 0.2500, 0.1114) -- (0.7000, 0.3000, 0.1174) -- (0.6500, 0.3000, 0.1126) -- cycle;
\fill[blue!15.0, opacity=0.7] (0.6500, 0.3000, 0.1126) -- (0.7000, 0.3000, 0.1174) -- (0.7000, 0.3500, 0.1233) -- (0.6500, 0.3500, 0.1185) -- cycle;
\fill[blue!15.0, opacity=0.7] (0.6500, 0.3500, 0.1185) -- (0.7000, 0.3500, 0.1233) -- (0.7000, 0.4000, 0.1291) -- (0.6500, 0.4000, 0.1243) -- cycle;
\fill[blue!15.0, opacity=0.7] (0.6500, 0.4000, 0.1243) -- (0.7000, 0.4000, 0.1291) -- (0.7000, 0.4500, 0.1348) -- (0.6500, 0.4500, 0.1300) -- cycle;
\fill[blue!15.0, opacity=0.7] (0.6500, 0.4500, 0.1300) -- (0.7000, 0.4500, 0.1348) -- (0.7000, 0.5000, 0.1403) -- (0.6500, 0.5000, 0.1355) -- cycle;
\fill[blue!15.0, opacity=0.7] (0.6500, 0.5000, 0.1355) -- (0.7000, 0.5000, 0.1403) -- (0.7000, 0.5500, 0.1457) -- (0.6500, 0.5500, 0.1409) -- cycle;
\fill[blue!15.0, opacity=0.7] (0.6500, 0.5500, 0.1409) -- (0.7000, 0.5500, 0.1457) -- (0.7000, 0.6000, 0.1508) -- (0.6500, 0.6000, 0.1461) -- cycle;
\fill[blue!15.0, opacity=0.7] (0.6500, 0.6000, 0.1461) -- (0.7000, 0.6000, 0.1508) -- (0.7000, 0.6500, 0.1558) -- (0.6500, 0.6500, 0.1510) -- cycle;
\fill[blue!15.0, opacity=0.7] (0.6500, 0.6500, 0.1510) -- (0.7000, 0.6500, 0.1558) -- (0.7000, 0.7000, 0.1606) -- (0.6500, 0.7000, 0.1558) -- cycle;
\fill[blue!15.0, opacity=0.7] (0.6500, 0.7000, 0.1558) -- (0.7000, 0.7000, 0.1606) -- (0.7000, 0.7500, 0.1651) -- (0.6500, 0.7500, 0.1604) -- cycle;
\fill[blue!15.0, opacity=0.7] (0.6500, 0.7500, 0.1604) -- (0.7000, 0.7500, 0.1651) -- (0.7000, 0.8000, 0.1695) -- (0.6500, 0.8000, 0.1647) -- cycle;
\fill[blue!15.0, opacity=0.7] (0.6500, 0.8000, 0.1647) -- (0.7000, 0.8000, 0.1695) -- (0.7000, 0.8500, 0.1736) -- (0.6500, 0.8500, 0.1688) -- cycle;
\fill[blue!15.0, opacity=0.7] (0.6500, 0.8500, 0.1688) -- (0.7000, 0.8500, 0.1736) -- (0.7000, 0.9000, 0.1774) -- (0.6500, 0.9000, 0.1726) -- cycle;
\fill[blue!15.0, opacity=0.7] (0.6500, 0.9000, 0.1726) -- (0.7000, 0.9000, 0.1774) -- (0.7000, 0.9500, 0.1809) -- (0.6500, 0.9500, 0.1762) -- cycle;
\fill[blue!15.0, opacity=0.7] (0.6500, 0.9500, 0.1762) -- (0.7000, 0.9500, 0.1809) -- (0.7000, 1.0000, 0.1842) -- (0.6500, 1.0000, 0.1794) -- cycle;
\fill[blue!15.0, opacity=0.7] (0.6500, 1.0000, 0.1794) -- (0.7000, 1.0000, 0.1842) -- (0.7000, 1.0500, 0.1872) -- (0.6500, 1.0500, 0.1824) -- cycle;
\fill[blue!15.1, opacity=0.7] (0.6500, 1.0500, 0.1824) -- (0.7000, 1.0500, 0.1872) -- (0.7000, 1.1000, 0.1899) -- (0.6500, 1.1000, 0.1851) -- cycle;
\fill[blue!16.2, opacity=0.7] (0.6500, 1.1000, 0.1851) -- (0.7000, 1.1000, 0.1899) -- (0.7000, 1.1500, 0.1923) -- (0.6500, 1.1500, 0.1875) -- cycle;
\fill[blue!17.8, opacity=0.7] (0.6500, 1.1500, 0.1875) -- (0.7000, 1.1500, 0.1923) -- (0.7000, 1.2000, 0.1944) -- (0.6500, 1.2000, 0.1896) -- cycle;
\fill[blue!16.5, opacity=0.7] (0.6500, 1.2000, 0.1896) -- (0.7000, 1.2000, 0.1944) -- (0.7000, 1.2500, 0.1962) -- (0.6500, 1.2500, 0.1914) -- cycle;
\fill[blue!15.2, opacity=0.7] (0.6500, 1.2500, 0.1914) -- (0.7000, 1.2500, 0.1962) -- (0.7000, 1.3000, 0.1977) -- (0.6500, 1.3000, 0.1929) -- cycle;
\fill[blue!15.0, opacity=0.7] (0.6500, 1.3000, 0.1929) -- (0.7000, 1.3000, 0.1977) -- (0.7000, 1.3500, 0.1988) -- (0.6500, 1.3500, 0.1940) -- cycle;
\fill[blue!15.0, opacity=0.7] (0.6500, 1.3500, 0.1940) -- (0.7000, 1.3500, 0.1988) -- (0.7000, 1.4000, 0.1996) -- (0.6500, 1.4000, 0.1949) -- cycle;
\fill[blue!15.0, opacity=0.7] (0.6500, 1.4000, 0.1949) -- (0.7000, 1.4000, 0.1996) -- (0.7000, 1.4500, 0.2001) -- (0.6500, 1.4500, 0.1954) -- cycle;
\fill[blue!15.0, opacity=0.7] (0.6500, 1.4500, 0.1954) -- (0.7000, 1.4500, 0.2001) -- (0.7000, 1.5000, 0.2003) -- (0.6500, 1.5000, 0.1955) -- cycle;
\fill[blue!15.0, opacity=0.7] (0.6500, 1.5000, 0.1955) -- (0.7000, 1.5000, 0.2003) -- (0.7000, 1.5500, 0.2001) -- (0.6500, 1.5500, 0.1954) -- cycle;
\fill[blue!15.0, opacity=0.7] (0.6500, 1.5500, 0.1954) -- (0.7000, 1.5500, 0.2001) -- (0.7000, 1.6000, 0.1996) -- (0.6500, 1.6000, 0.1949) -- cycle;
\fill[blue!15.0, opacity=0.7] (0.6500, 1.6000, 0.1949) -- (0.7000, 1.6000, 0.1996) -- (0.7000, 1.6500, 0.1988) -- (0.6500, 1.6500, 0.1940) -- cycle;
\fill[blue!15.0, opacity=0.7] (0.6500, 1.6500, 0.1940) -- (0.7000, 1.6500, 0.1988) -- (0.7000, 1.7000, 0.1977) -- (0.6500, 1.7000, 0.1929) -- cycle;
\fill[blue!15.0, opacity=0.7] (0.6500, 1.7000, 0.1929) -- (0.7000, 1.7000, 0.1977) -- (0.7000, 1.7500, 0.1962) -- (0.6500, 1.7500, 0.1914) -- cycle;
\fill[blue!15.0, opacity=0.7] (0.6500, 1.7500, 0.1914) -- (0.7000, 1.7500, 0.1962) -- (0.7000, 1.8000, 0.1944) -- (0.6500, 1.8000, 0.1896) -- cycle;
\fill[blue!15.0, opacity=0.7] (0.6500, 1.8000, 0.1896) -- (0.7000, 1.8000, 0.1944) -- (0.7000, 1.8500, 0.1923) -- (0.6500, 1.8500, 0.1875) -- cycle;
\fill[blue!15.0, opacity=0.7] (0.6500, 1.8500, 0.1875) -- (0.7000, 1.8500, 0.1923) -- (0.7000, 1.9000, 0.1899) -- (0.6500, 1.9000, 0.1851) -- cycle;
\fill[blue!15.0, opacity=0.7] (0.6500, 1.9000, 0.1851) -- (0.7000, 1.9000, 0.1899) -- (0.7000, 1.9500, 0.1872) -- (0.6500, 1.9500, 0.1824) -- cycle;
\fill[blue!15.0, opacity=0.7] (0.6500, 1.9500, 0.1824) -- (0.7000, 1.9500, 0.1872) -- (0.7000, 2.0000, 0.1842) -- (0.6500, 2.0000, 0.1794) -- cycle;
\fill[blue!15.0, opacity=0.7] (0.6500, 2.0000, 0.1794) -- (0.7000, 2.0000, 0.1842) -- (0.7000, 2.0500, 0.1809) -- (0.6500, 2.0500, 0.1762) -- cycle;
\fill[blue!15.1, opacity=0.7] (0.6500, 2.0500, 0.1762) -- (0.7000, 2.0500, 0.1809) -- (0.7000, 2.1000, 0.1774) -- (0.6500, 2.1000, 0.1726) -- cycle;
\fill[blue!15.1, opacity=0.7] (0.6500, 2.1000, 0.1726) -- (0.7000, 2.1000, 0.1774) -- (0.7000, 2.1500, 0.1736) -- (0.6500, 2.1500, 0.1688) -- cycle;
\fill[blue!15.0, opacity=0.7] (0.6500, 2.1500, 0.1688) -- (0.7000, 2.1500, 0.1736) -- (0.7000, 2.2000, 0.1695) -- (0.6500, 2.2000, 0.1647) -- cycle;
\fill[blue!15.0, opacity=0.7] (0.6500, 2.2000, 0.1647) -- (0.7000, 2.2000, 0.1695) -- (0.7000, 2.2500, 0.1651) -- (0.6500, 2.2500, 0.1604) -- cycle;
\fill[blue!15.0, opacity=0.7] (0.6500, 2.2500, 0.1604) -- (0.7000, 2.2500, 0.1651) -- (0.7000, 2.3000, 0.1606) -- (0.6500, 2.3000, 0.1558) -- cycle;
\fill[blue!15.0, opacity=0.7] (0.6500, 2.3000, 0.1558) -- (0.7000, 2.3000, 0.1606) -- (0.7000, 2.3500, 0.1558) -- (0.6500, 2.3500, 0.1510) -- cycle;
\fill[blue!15.0, opacity=0.7] (0.6500, 2.3500, 0.1510) -- (0.7000, 2.3500, 0.1558) -- (0.7000, 2.4000, 0.1508) -- (0.6500, 2.4000, 0.1461) -- cycle;
\fill[blue!15.0, opacity=0.7] (0.6500, 2.4000, 0.1461) -- (0.7000, 2.4000, 0.1508) -- (0.7000, 2.4500, 0.1457) -- (0.6500, 2.4500, 0.1409) -- cycle;
\fill[blue!15.0, opacity=0.7] (0.6500, 2.4500, 0.1409) -- (0.7000, 2.4500, 0.1457) -- (0.7000, 2.5000, 0.1403) -- (0.6500, 2.5000, 0.1355) -- cycle;
\fill[blue!15.0, opacity=0.7] (0.6500, 2.5000, 0.1355) -- (0.7000, 2.5000, 0.1403) -- (0.7000, 2.5500, 0.1348) -- (0.6500, 2.5500, 0.1300) -- cycle;
\fill[blue!15.0, opacity=0.7] (0.6500, 2.5500, 0.1300) -- (0.7000, 2.5500, 0.1348) -- (0.7000, 2.6000, 0.1291) -- (0.6500, 2.6000, 0.1243) -- cycle;
\fill[blue!15.0, opacity=0.7] (0.6500, 2.6000, 0.1243) -- (0.7000, 2.6000, 0.1291) -- (0.7000, 2.6500, 0.1233) -- (0.6500, 2.6500, 0.1185) -- cycle;
\fill[blue!15.0, opacity=0.7] (0.6500, 2.6500, 0.1185) -- (0.7000, 2.6500, 0.1233) -- (0.7000, 2.7000, 0.1174) -- (0.6500, 2.7000, 0.1126) -- cycle;
\fill[blue!15.0, opacity=0.7] (0.6500, 2.7000, 0.1126) -- (0.7000, 2.7000, 0.1174) -- (0.7000, 2.7500, 0.1114) -- (0.6500, 2.7500, 0.1066) -- cycle;
\fill[blue!15.0, opacity=0.7] (0.6500, 2.7500, 0.1066) -- (0.7000, 2.7500, 0.1114) -- (0.7000, 2.8000, 0.1052) -- (0.6500, 2.8000, 0.1005) -- cycle;
\fill[blue!15.0, opacity=0.7] (0.6500, 2.8000, 0.1005) -- (0.7000, 2.8000, 0.1052) -- (0.7000, 2.8500, 0.0991) -- (0.6500, 2.8500, 0.0943) -- cycle;
\fill[blue!15.0, opacity=0.7] (0.6500, 2.8500, 0.0943) -- (0.7000, 2.8500, 0.0991) -- (0.7000, 2.9000, 0.0928) -- (0.6500, 2.9000, 0.0881) -- cycle;
\fill[blue!15.0, opacity=0.7] (0.6500, 2.9000, 0.0881) -- (0.7000, 2.9000, 0.0928) -- (0.7000, 2.9500, 0.0866) -- (0.6500, 2.9500, 0.0818) -- cycle;
\fill[blue!15.0, opacity=0.7] (0.6500, 2.9500, 0.0818) -- (0.7000, 2.9500, 0.0866) -- (0.7000, 3.0000, 0.0803) -- (0.6500, 3.0000, 0.0755) -- cycle;
\fill[blue!15.0, opacity=0.7] (0.7000, 0.0000, 0.0803) -- (0.7500, 0.0000, 0.0849) -- (0.7500, 0.0500, 0.0911) -- (0.7000, 0.0500, 0.0866) -- cycle;
\fill[blue!15.0, opacity=0.7] (0.7000, 0.0500, 0.0866) -- (0.7500, 0.0500, 0.0911) -- (0.7500, 0.1000, 0.0974) -- (0.7000, 0.1000, 0.0928) -- cycle;
\fill[blue!15.0, opacity=0.7] (0.7000, 0.1000, 0.0928) -- (0.7500, 0.1000, 0.0974) -- (0.7500, 0.1500, 0.1036) -- (0.7000, 0.1500, 0.0991) -- cycle;
\fill[blue!15.0, opacity=0.7] (0.7000, 0.1500, 0.0991) -- (0.7500, 0.1500, 0.1036) -- (0.7500, 0.2000, 0.1098) -- (0.7000, 0.2000, 0.1052) -- cycle;
\fill[blue!15.0, opacity=0.7] (0.7000, 0.2000, 0.1052) -- (0.7500, 0.2000, 0.1098) -- (0.7500, 0.2500, 0.1159) -- (0.7000, 0.2500, 0.1114) -- cycle;
\fill[blue!15.0, opacity=0.7] (0.7000, 0.2500, 0.1114) -- (0.7500, 0.2500, 0.1159) -- (0.7500, 0.3000, 0.1219) -- (0.7000, 0.3000, 0.1174) -- cycle;
\fill[blue!15.0, opacity=0.7] (0.7000, 0.3000, 0.1174) -- (0.7500, 0.3000, 0.1219) -- (0.7500, 0.3500, 0.1279) -- (0.7000, 0.3500, 0.1233) -- cycle;
\fill[blue!15.0, opacity=0.7] (0.7000, 0.3500, 0.1233) -- (0.7500, 0.3500, 0.1279) -- (0.7500, 0.4000, 0.1337) -- (0.7000, 0.4000, 0.1291) -- cycle;
\fill[blue!15.0, opacity=0.7] (0.7000, 0.4000, 0.1291) -- (0.7500, 0.4000, 0.1337) -- (0.7500, 0.4500, 0.1393) -- (0.7000, 0.4500, 0.1348) -- cycle;
\fill[blue!15.0, opacity=0.7] (0.7000, 0.4500, 0.1348) -- (0.7500, 0.4500, 0.1393) -- (0.7500, 0.5000, 0.1449) -- (0.7000, 0.5000, 0.1403) -- cycle;
\fill[blue!15.0, opacity=0.7] (0.7000, 0.5000, 0.1403) -- (0.7500, 0.5000, 0.1449) -- (0.7500, 0.5500, 0.1502) -- (0.7000, 0.5500, 0.1457) -- cycle;
\fill[blue!15.0, opacity=0.7] (0.7000, 0.5500, 0.1457) -- (0.7500, 0.5500, 0.1502) -- (0.7500, 0.6000, 0.1554) -- (0.7000, 0.6000, 0.1508) -- cycle;
\fill[blue!15.0, opacity=0.7] (0.7000, 0.6000, 0.1508) -- (0.7500, 0.6000, 0.1554) -- (0.7500, 0.6500, 0.1604) -- (0.7000, 0.6500, 0.1558) -- cycle;
\fill[blue!15.0, opacity=0.7] (0.7000, 0.6500, 0.1558) -- (0.7500, 0.6500, 0.1604) -- (0.7500, 0.7000, 0.1651) -- (0.7000, 0.7000, 0.1606) -- cycle;
\fill[blue!15.0, opacity=0.7] (0.7000, 0.7000, 0.1606) -- (0.7500, 0.7000, 0.1651) -- (0.7500, 0.7500, 0.1697) -- (0.7000, 0.7500, 0.1651) -- cycle;
\fill[blue!15.0, opacity=0.7] (0.7000, 0.7500, 0.1651) -- (0.7500, 0.7500, 0.1697) -- (0.7500, 0.8000, 0.1740) -- (0.7000, 0.8000, 0.1695) -- cycle;
\fill[blue!15.0, opacity=0.7] (0.7000, 0.8000, 0.1695) -- (0.7500, 0.8000, 0.1740) -- (0.7500, 0.8500, 0.1781) -- (0.7000, 0.8500, 0.1736) -- cycle;
\fill[blue!15.0, opacity=0.7] (0.7000, 0.8500, 0.1736) -- (0.7500, 0.8500, 0.1781) -- (0.7500, 0.9000, 0.1819) -- (0.7000, 0.9000, 0.1774) -- cycle;
\fill[blue!15.0, opacity=0.7] (0.7000, 0.9000, 0.1774) -- (0.7500, 0.9000, 0.1819) -- (0.7500, 0.9500, 0.1855) -- (0.7000, 0.9500, 0.1809) -- cycle;
\fill[blue!15.0, opacity=0.7] (0.7000, 0.9500, 0.1809) -- (0.7500, 0.9500, 0.1855) -- (0.7500, 1.0000, 0.1888) -- (0.7000, 1.0000, 0.1842) -- cycle;
\fill[blue!15.3, opacity=0.7] (0.7000, 1.0000, 0.1842) -- (0.7500, 1.0000, 0.1888) -- (0.7500, 1.0500, 0.1918) -- (0.7000, 1.0500, 0.1872) -- cycle;
\fill[blue!17.8, opacity=0.7] (0.7000, 1.0500, 0.1872) -- (0.7500, 1.0500, 0.1918) -- (0.7500, 1.1000, 0.1945) -- (0.7000, 1.1000, 0.1899) -- cycle;
\fill[blue!17.9, opacity=0.7] (0.7000, 1.1000, 0.1899) -- (0.7500, 1.1000, 0.1945) -- (0.7500, 1.1500, 0.1969) -- (0.7000, 1.1500, 0.1923) -- cycle;
\fill[blue!15.5, opacity=0.7] (0.7000, 1.1500, 0.1923) -- (0.7500, 1.1500, 0.1969) -- (0.7500, 1.2000, 0.1990) -- (0.7000, 1.2000, 0.1944) -- cycle;
\fill[blue!15.0, opacity=0.7] (0.7000, 1.2000, 0.1944) -- (0.7500, 1.2000, 0.1990) -- (0.7500, 1.2500, 0.2008) -- (0.7000, 1.2500, 0.1962) -- cycle;
\fill[blue!15.0, opacity=0.7] (0.7000, 1.2500, 0.1962) -- (0.7500, 1.2500, 0.2008) -- (0.7500, 1.3000, 0.2022) -- (0.7000, 1.3000, 0.1977) -- cycle;
\fill[blue!15.0, opacity=0.7] (0.7000, 1.3000, 0.1977) -- (0.7500, 1.3000, 0.2022) -- (0.7500, 1.3500, 0.2034) -- (0.7000, 1.3500, 0.1988) -- cycle;
\fill[blue!15.0, opacity=0.7] (0.7000, 1.3500, 0.1988) -- (0.7500, 1.3500, 0.2034) -- (0.7500, 1.4000, 0.2042) -- (0.7000, 1.4000, 0.1996) -- cycle;
\fill[blue!15.0, opacity=0.7] (0.7000, 1.4000, 0.1996) -- (0.7500, 1.4000, 0.2042) -- (0.7500, 1.4500, 0.2047) -- (0.7000, 1.4500, 0.2001) -- cycle;
\fill[blue!15.0, opacity=0.7] (0.7000, 1.4500, 0.2001) -- (0.7500, 1.4500, 0.2047) -- (0.7500, 1.5000, 0.2049) -- (0.7000, 1.5000, 0.2003) -- cycle;
\fill[blue!15.0, opacity=0.7] (0.7000, 1.5000, 0.2003) -- (0.7500, 1.5000, 0.2049) -- (0.7500, 1.5500, 0.2047) -- (0.7000, 1.5500, 0.2001) -- cycle;
\fill[blue!15.0, opacity=0.7] (0.7000, 1.5500, 0.2001) -- (0.7500, 1.5500, 0.2047) -- (0.7500, 1.6000, 0.2042) -- (0.7000, 1.6000, 0.1996) -- cycle;
\fill[blue!15.0, opacity=0.7] (0.7000, 1.6000, 0.1996) -- (0.7500, 1.6000, 0.2042) -- (0.7500, 1.6500, 0.2034) -- (0.7000, 1.6500, 0.1988) -- cycle;
\fill[blue!15.0, opacity=0.7] (0.7000, 1.6500, 0.1988) -- (0.7500, 1.6500, 0.2034) -- (0.7500, 1.7000, 0.2022) -- (0.7000, 1.7000, 0.1977) -- cycle;
\fill[blue!15.0, opacity=0.7] (0.7000, 1.7000, 0.1977) -- (0.7500, 1.7000, 0.2022) -- (0.7500, 1.7500, 0.2008) -- (0.7000, 1.7500, 0.1962) -- cycle;
\fill[blue!15.0, opacity=0.7] (0.7000, 1.7500, 0.1962) -- (0.7500, 1.7500, 0.2008) -- (0.7500, 1.8000, 0.1990) -- (0.7000, 1.8000, 0.1944) -- cycle;
\fill[blue!15.0, opacity=0.7] (0.7000, 1.8000, 0.1944) -- (0.7500, 1.8000, 0.1990) -- (0.7500, 1.8500, 0.1969) -- (0.7000, 1.8500, 0.1923) -- cycle;
\fill[blue!15.0, opacity=0.7] (0.7000, 1.8500, 0.1923) -- (0.7500, 1.8500, 0.1969) -- (0.7500, 1.9000, 0.1945) -- (0.7000, 1.9000, 0.1899) -- cycle;
\fill[blue!15.0, opacity=0.7] (0.7000, 1.9000, 0.1899) -- (0.7500, 1.9000, 0.1945) -- (0.7500, 1.9500, 0.1918) -- (0.7000, 1.9500, 0.1872) -- cycle;
\fill[blue!15.0, opacity=0.7] (0.7000, 1.9500, 0.1872) -- (0.7500, 1.9500, 0.1918) -- (0.7500, 2.0000, 0.1888) -- (0.7000, 2.0000, 0.1842) -- cycle;
\fill[blue!15.0, opacity=0.7] (0.7000, 2.0000, 0.1842) -- (0.7500, 2.0000, 0.1888) -- (0.7500, 2.0500, 0.1855) -- (0.7000, 2.0500, 0.1809) -- cycle;
\fill[blue!15.0, opacity=0.7] (0.7000, 2.0500, 0.1809) -- (0.7500, 2.0500, 0.1855) -- (0.7500, 2.1000, 0.1819) -- (0.7000, 2.1000, 0.1774) -- cycle;
\fill[blue!15.0, opacity=0.7] (0.7000, 2.1000, 0.1774) -- (0.7500, 2.1000, 0.1819) -- (0.7500, 2.1500, 0.1781) -- (0.7000, 2.1500, 0.1736) -- cycle;
\fill[blue!15.1, opacity=0.7] (0.7000, 2.1500, 0.1736) -- (0.7500, 2.1500, 0.1781) -- (0.7500, 2.2000, 0.1740) -- (0.7000, 2.2000, 0.1695) -- cycle;
\fill[blue!15.0, opacity=0.7] (0.7000, 2.2000, 0.1695) -- (0.7500, 2.2000, 0.1740) -- (0.7500, 2.2500, 0.1697) -- (0.7000, 2.2500, 0.1651) -- cycle;
\fill[blue!15.0, opacity=0.7] (0.7000, 2.2500, 0.1651) -- (0.7500, 2.2500, 0.1697) -- (0.7500, 2.3000, 0.1651) -- (0.7000, 2.3000, 0.1606) -- cycle;
\fill[blue!15.0, opacity=0.7] (0.7000, 2.3000, 0.1606) -- (0.7500, 2.3000, 0.1651) -- (0.7500, 2.3500, 0.1604) -- (0.7000, 2.3500, 0.1558) -- cycle;
\fill[blue!15.0, opacity=0.7] (0.7000, 2.3500, 0.1558) -- (0.7500, 2.3500, 0.1604) -- (0.7500, 2.4000, 0.1554) -- (0.7000, 2.4000, 0.1508) -- cycle;
\fill[blue!15.0, opacity=0.7] (0.7000, 2.4000, 0.1508) -- (0.7500, 2.4000, 0.1554) -- (0.7500, 2.4500, 0.1502) -- (0.7000, 2.4500, 0.1457) -- cycle;
\fill[blue!15.0, opacity=0.7] (0.7000, 2.4500, 0.1457) -- (0.7500, 2.4500, 0.1502) -- (0.7500, 2.5000, 0.1449) -- (0.7000, 2.5000, 0.1403) -- cycle;
\fill[blue!15.0, opacity=0.7] (0.7000, 2.5000, 0.1403) -- (0.7500, 2.5000, 0.1449) -- (0.7500, 2.5500, 0.1393) -- (0.7000, 2.5500, 0.1348) -- cycle;
\fill[blue!15.0, opacity=0.7] (0.7000, 2.5500, 0.1348) -- (0.7500, 2.5500, 0.1393) -- (0.7500, 2.6000, 0.1337) -- (0.7000, 2.6000, 0.1291) -- cycle;
\fill[blue!15.0, opacity=0.7] (0.7000, 2.6000, 0.1291) -- (0.7500, 2.6000, 0.1337) -- (0.7500, 2.6500, 0.1279) -- (0.7000, 2.6500, 0.1233) -- cycle;
\fill[blue!15.0, opacity=0.7] (0.7000, 2.6500, 0.1233) -- (0.7500, 2.6500, 0.1279) -- (0.7500, 2.7000, 0.1219) -- (0.7000, 2.7000, 0.1174) -- cycle;
\fill[blue!15.0, opacity=0.7] (0.7000, 2.7000, 0.1174) -- (0.7500, 2.7000, 0.1219) -- (0.7500, 2.7500, 0.1159) -- (0.7000, 2.7500, 0.1114) -- cycle;
\fill[blue!15.0, opacity=0.7] (0.7000, 2.7500, 0.1114) -- (0.7500, 2.7500, 0.1159) -- (0.7500, 2.8000, 0.1098) -- (0.7000, 2.8000, 0.1052) -- cycle;
\fill[blue!15.0, opacity=0.7] (0.7000, 2.8000, 0.1052) -- (0.7500, 2.8000, 0.1098) -- (0.7500, 2.8500, 0.1036) -- (0.7000, 2.8500, 0.0991) -- cycle;
\fill[blue!15.0, opacity=0.7] (0.7000, 2.8500, 0.0991) -- (0.7500, 2.8500, 0.1036) -- (0.7500, 2.9000, 0.0974) -- (0.7000, 2.9000, 0.0928) -- cycle;
\fill[blue!15.0, opacity=0.7] (0.7000, 2.9000, 0.0928) -- (0.7500, 2.9000, 0.0974) -- (0.7500, 2.9500, 0.0911) -- (0.7000, 2.9500, 0.0866) -- cycle;
\fill[blue!15.0, opacity=0.7] (0.7000, 2.9500, 0.0866) -- (0.7500, 2.9500, 0.0911) -- (0.7500, 3.0000, 0.0849) -- (0.7000, 3.0000, 0.0803) -- cycle;
\fill[blue!15.0, opacity=0.7] (0.7500, 0.0000, 0.0849) -- (0.8000, 0.0000, 0.0892) -- (0.8000, 0.0500, 0.0955) -- (0.7500, 0.0500, 0.0911) -- cycle;
\fill[blue!15.0, opacity=0.7] (0.7500, 0.0500, 0.0911) -- (0.8000, 0.0500, 0.0955) -- (0.8000, 0.1000, 0.1017) -- (0.7500, 0.1000, 0.0974) -- cycle;
\fill[blue!15.0, opacity=0.7] (0.7500, 0.1000, 0.0974) -- (0.8000, 0.1000, 0.1017) -- (0.8000, 0.1500, 0.1079) -- (0.7500, 0.1500, 0.1036) -- cycle;
\fill[blue!15.0, opacity=0.7] (0.7500, 0.1500, 0.1036) -- (0.8000, 0.1500, 0.1079) -- (0.8000, 0.2000, 0.1141) -- (0.7500, 0.2000, 0.1098) -- cycle;
\fill[blue!15.0, opacity=0.7] (0.7500, 0.2000, 0.1098) -- (0.8000, 0.2000, 0.1141) -- (0.8000, 0.2500, 0.1202) -- (0.7500, 0.2500, 0.1159) -- cycle;
\fill[blue!15.0, opacity=0.7] (0.7500, 0.2500, 0.1159) -- (0.8000, 0.2500, 0.1202) -- (0.8000, 0.3000, 0.1263) -- (0.7500, 0.3000, 0.1219) -- cycle;
\fill[blue!15.0, opacity=0.7] (0.7500, 0.3000, 0.1219) -- (0.8000, 0.3000, 0.1263) -- (0.8000, 0.3500, 0.1322) -- (0.7500, 0.3500, 0.1279) -- cycle;
\fill[blue!15.0, opacity=0.7] (0.7500, 0.3500, 0.1279) -- (0.8000, 0.3500, 0.1322) -- (0.8000, 0.4000, 0.1380) -- (0.7500, 0.4000, 0.1337) -- cycle;
\fill[blue!15.0, opacity=0.7] (0.7500, 0.4000, 0.1337) -- (0.8000, 0.4000, 0.1380) -- (0.8000, 0.4500, 0.1437) -- (0.7500, 0.4500, 0.1393) -- cycle;
\fill[blue!15.0, opacity=0.7] (0.7500, 0.4500, 0.1393) -- (0.8000, 0.4500, 0.1437) -- (0.8000, 0.5000, 0.1492) -- (0.7500, 0.5000, 0.1449) -- cycle;
\fill[blue!15.0, opacity=0.7] (0.7500, 0.5000, 0.1449) -- (0.8000, 0.5000, 0.1492) -- (0.8000, 0.5500, 0.1545) -- (0.7500, 0.5500, 0.1502) -- cycle;
\fill[blue!15.0, opacity=0.7] (0.7500, 0.5500, 0.1502) -- (0.8000, 0.5500, 0.1545) -- (0.8000, 0.6000, 0.1597) -- (0.7500, 0.6000, 0.1554) -- cycle;
\fill[blue!15.0, opacity=0.7] (0.7500, 0.6000, 0.1554) -- (0.8000, 0.6000, 0.1597) -- (0.8000, 0.6500, 0.1647) -- (0.7500, 0.6500, 0.1604) -- cycle;
\fill[blue!15.0, opacity=0.7] (0.7500, 0.6500, 0.1604) -- (0.8000, 0.6500, 0.1647) -- (0.8000, 0.7000, 0.1695) -- (0.7500, 0.7000, 0.1651) -- cycle;
\fill[blue!15.0, opacity=0.7] (0.7500, 0.7000, 0.1651) -- (0.8000, 0.7000, 0.1695) -- (0.8000, 0.7500, 0.1740) -- (0.7500, 0.7500, 0.1697) -- cycle;
\fill[blue!15.0, opacity=0.7] (0.7500, 0.7500, 0.1697) -- (0.8000, 0.7500, 0.1740) -- (0.8000, 0.8000, 0.1784) -- (0.7500, 0.8000, 0.1740) -- cycle;
\fill[blue!15.0, opacity=0.7] (0.7500, 0.8000, 0.1740) -- (0.8000, 0.8000, 0.1784) -- (0.8000, 0.8500, 0.1824) -- (0.7500, 0.8500, 0.1781) -- cycle;
\fill[blue!15.0, opacity=0.7] (0.7500, 0.8500, 0.1781) -- (0.8000, 0.8500, 0.1824) -- (0.8000, 0.9000, 0.1863) -- (0.7500, 0.9000, 0.1819) -- cycle;
\fill[blue!15.0, opacity=0.7] (0.7500, 0.9000, 0.1819) -- (0.8000, 0.9000, 0.1863) -- (0.8000, 0.9500, 0.1898) -- (0.7500, 0.9500, 0.1855) -- cycle;
\fill[blue!15.5, opacity=0.7] (0.7500, 0.9500, 0.1855) -- (0.8000, 0.9500, 0.1898) -- (0.8000, 1.0000, 0.1931) -- (0.7500, 1.0000, 0.1888) -- cycle;
\fill[blue!19.2, opacity=0.7] (0.7500, 1.0000, 0.1888) -- (0.8000, 1.0000, 0.1931) -- (0.8000, 1.0500, 0.1961) -- (0.7500, 1.0500, 0.1918) -- cycle;
\fill[blue!17.5, opacity=0.7] (0.7500, 1.0500, 0.1918) -- (0.8000, 1.0500, 0.1961) -- (0.8000, 1.1000, 0.1988) -- (0.7500, 1.1000, 0.1945) -- cycle;
\fill[blue!15.1, opacity=0.7] (0.7500, 1.1000, 0.1945) -- (0.8000, 1.1000, 0.1988) -- (0.8000, 1.1500, 0.2012) -- (0.7500, 1.1500, 0.1969) -- cycle;
\fill[blue!15.0, opacity=0.7] (0.7500, 1.1500, 0.1969) -- (0.8000, 1.1500, 0.2012) -- (0.8000, 1.2000, 0.2033) -- (0.7500, 1.2000, 0.1990) -- cycle;
\fill[blue!15.0, opacity=0.7] (0.7500, 1.2000, 0.1990) -- (0.8000, 1.2000, 0.2033) -- (0.8000, 1.2500, 0.2051) -- (0.7500, 1.2500, 0.2008) -- cycle;
\fill[blue!15.0, opacity=0.7] (0.7500, 1.2500, 0.2008) -- (0.8000, 1.2500, 0.2051) -- (0.8000, 1.3000, 0.2066) -- (0.7500, 1.3000, 0.2022) -- cycle;
\fill[blue!15.0, opacity=0.7] (0.7500, 1.3000, 0.2022) -- (0.8000, 1.3000, 0.2066) -- (0.8000, 1.3500, 0.2077) -- (0.7500, 1.3500, 0.2034) -- cycle;
\fill[blue!15.0, opacity=0.7] (0.7500, 1.3500, 0.2034) -- (0.8000, 1.3500, 0.2077) -- (0.8000, 1.4000, 0.2085) -- (0.7500, 1.4000, 0.2042) -- cycle;
\fill[blue!15.0, opacity=0.7] (0.7500, 1.4000, 0.2042) -- (0.8000, 1.4000, 0.2085) -- (0.8000, 1.4500, 0.2090) -- (0.7500, 1.4500, 0.2047) -- cycle;
\fill[blue!15.0, opacity=0.7] (0.7500, 1.4500, 0.2047) -- (0.8000, 1.4500, 0.2090) -- (0.8000, 1.5000, 0.2092) -- (0.7500, 1.5000, 0.2049) -- cycle;
\fill[blue!15.0, opacity=0.7] (0.7500, 1.5000, 0.2049) -- (0.8000, 1.5000, 0.2092) -- (0.8000, 1.5500, 0.2090) -- (0.7500, 1.5500, 0.2047) -- cycle;
\fill[blue!15.0, opacity=0.7] (0.7500, 1.5500, 0.2047) -- (0.8000, 1.5500, 0.2090) -- (0.8000, 1.6000, 0.2085) -- (0.7500, 1.6000, 0.2042) -- cycle;
\fill[blue!15.0, opacity=0.7] (0.7500, 1.6000, 0.2042) -- (0.8000, 1.6000, 0.2085) -- (0.8000, 1.6500, 0.2077) -- (0.7500, 1.6500, 0.2034) -- cycle;
\fill[blue!15.0, opacity=0.7] (0.7500, 1.6500, 0.2034) -- (0.8000, 1.6500, 0.2077) -- (0.8000, 1.7000, 0.2066) -- (0.7500, 1.7000, 0.2022) -- cycle;
\fill[blue!15.0, opacity=0.7] (0.7500, 1.7000, 0.2022) -- (0.8000, 1.7000, 0.2066) -- (0.8000, 1.7500, 0.2051) -- (0.7500, 1.7500, 0.2008) -- cycle;
\fill[blue!15.0, opacity=0.7] (0.7500, 1.7500, 0.2008) -- (0.8000, 1.7500, 0.2051) -- (0.8000, 1.8000, 0.2033) -- (0.7500, 1.8000, 0.1990) -- cycle;
\fill[blue!15.0, opacity=0.7] (0.7500, 1.8000, 0.1990) -- (0.8000, 1.8000, 0.2033) -- (0.8000, 1.8500, 0.2012) -- (0.7500, 1.8500, 0.1969) -- cycle;
\fill[blue!15.0, opacity=0.7] (0.7500, 1.8500, 0.1969) -- (0.8000, 1.8500, 0.2012) -- (0.8000, 1.9000, 0.1988) -- (0.7500, 1.9000, 0.1945) -- cycle;
\fill[blue!15.0, opacity=0.7] (0.7500, 1.9000, 0.1945) -- (0.8000, 1.9000, 0.1988) -- (0.8000, 1.9500, 0.1961) -- (0.7500, 1.9500, 0.1918) -- cycle;
\fill[blue!15.0, opacity=0.7] (0.7500, 1.9500, 0.1918) -- (0.8000, 1.9500, 0.1961) -- (0.8000, 2.0000, 0.1931) -- (0.7500, 2.0000, 0.1888) -- cycle;
\fill[blue!15.0, opacity=0.7] (0.7500, 2.0000, 0.1888) -- (0.8000, 2.0000, 0.1931) -- (0.8000, 2.0500, 0.1898) -- (0.7500, 2.0500, 0.1855) -- cycle;
\fill[blue!15.0, opacity=0.7] (0.7500, 2.0500, 0.1855) -- (0.8000, 2.0500, 0.1898) -- (0.8000, 2.1000, 0.1863) -- (0.7500, 2.1000, 0.1819) -- cycle;
\fill[blue!15.0, opacity=0.7] (0.7500, 2.1000, 0.1819) -- (0.8000, 2.1000, 0.1863) -- (0.8000, 2.1500, 0.1824) -- (0.7500, 2.1500, 0.1781) -- cycle;
\fill[blue!15.0, opacity=0.7] (0.7500, 2.1500, 0.1781) -- (0.8000, 2.1500, 0.1824) -- (0.8000, 2.2000, 0.1784) -- (0.7500, 2.2000, 0.1740) -- cycle;
\fill[blue!15.1, opacity=0.7] (0.7500, 2.2000, 0.1740) -- (0.8000, 2.2000, 0.1784) -- (0.8000, 2.2500, 0.1740) -- (0.7500, 2.2500, 0.1697) -- cycle;
\fill[blue!15.0, opacity=0.7] (0.7500, 2.2500, 0.1697) -- (0.8000, 2.2500, 0.1740) -- (0.8000, 2.3000, 0.1695) -- (0.7500, 2.3000, 0.1651) -- cycle;
\fill[blue!15.0, opacity=0.7] (0.7500, 2.3000, 0.1651) -- (0.8000, 2.3000, 0.1695) -- (0.8000, 2.3500, 0.1647) -- (0.7500, 2.3500, 0.1604) -- cycle;
\fill[blue!15.0, opacity=0.7] (0.7500, 2.3500, 0.1604) -- (0.8000, 2.3500, 0.1647) -- (0.8000, 2.4000, 0.1597) -- (0.7500, 2.4000, 0.1554) -- cycle;
\fill[blue!15.0, opacity=0.7] (0.7500, 2.4000, 0.1554) -- (0.8000, 2.4000, 0.1597) -- (0.8000, 2.4500, 0.1545) -- (0.7500, 2.4500, 0.1502) -- cycle;
\fill[blue!15.0, opacity=0.7] (0.7500, 2.4500, 0.1502) -- (0.8000, 2.4500, 0.1545) -- (0.8000, 2.5000, 0.1492) -- (0.7500, 2.5000, 0.1449) -- cycle;
\fill[blue!15.0, opacity=0.7] (0.7500, 2.5000, 0.1449) -- (0.8000, 2.5000, 0.1492) -- (0.8000, 2.5500, 0.1437) -- (0.7500, 2.5500, 0.1393) -- cycle;
\fill[blue!15.0, opacity=0.7] (0.7500, 2.5500, 0.1393) -- (0.8000, 2.5500, 0.1437) -- (0.8000, 2.6000, 0.1380) -- (0.7500, 2.6000, 0.1337) -- cycle;
\fill[blue!15.0, opacity=0.7] (0.7500, 2.6000, 0.1337) -- (0.8000, 2.6000, 0.1380) -- (0.8000, 2.6500, 0.1322) -- (0.7500, 2.6500, 0.1279) -- cycle;
\fill[blue!15.0, opacity=0.7] (0.7500, 2.6500, 0.1279) -- (0.8000, 2.6500, 0.1322) -- (0.8000, 2.7000, 0.1263) -- (0.7500, 2.7000, 0.1219) -- cycle;
\fill[blue!15.0, opacity=0.7] (0.7500, 2.7000, 0.1219) -- (0.8000, 2.7000, 0.1263) -- (0.8000, 2.7500, 0.1202) -- (0.7500, 2.7500, 0.1159) -- cycle;
\fill[blue!15.0, opacity=0.7] (0.7500, 2.7500, 0.1159) -- (0.8000, 2.7500, 0.1202) -- (0.8000, 2.8000, 0.1141) -- (0.7500, 2.8000, 0.1098) -- cycle;
\fill[blue!15.0, opacity=0.7] (0.7500, 2.8000, 0.1098) -- (0.8000, 2.8000, 0.1141) -- (0.8000, 2.8500, 0.1079) -- (0.7500, 2.8500, 0.1036) -- cycle;
\fill[blue!15.0, opacity=0.7] (0.7500, 2.8500, 0.1036) -- (0.8000, 2.8500, 0.1079) -- (0.8000, 2.9000, 0.1017) -- (0.7500, 2.9000, 0.0974) -- cycle;
\fill[blue!15.0, opacity=0.7] (0.7500, 2.9000, 0.0974) -- (0.8000, 2.9000, 0.1017) -- (0.8000, 2.9500, 0.0955) -- (0.7500, 2.9500, 0.0911) -- cycle;
\fill[blue!15.0, opacity=0.7] (0.7500, 2.9500, 0.0911) -- (0.8000, 2.9500, 0.0955) -- (0.8000, 3.0000, 0.0892) -- (0.7500, 3.0000, 0.0849) -- cycle;
\fill[blue!15.0, opacity=0.7] (0.8000, 0.0000, 0.0892) -- (0.8500, 0.0000, 0.0933) -- (0.8500, 0.0500, 0.0995) -- (0.8000, 0.0500, 0.0955) -- cycle;
\fill[blue!15.0, opacity=0.7] (0.8000, 0.0500, 0.0955) -- (0.8500, 0.0500, 0.0995) -- (0.8500, 0.1000, 0.1058) -- (0.8000, 0.1000, 0.1017) -- cycle;
\fill[blue!15.0, opacity=0.7] (0.8000, 0.1000, 0.1017) -- (0.8500, 0.1000, 0.1058) -- (0.8500, 0.1500, 0.1120) -- (0.8000, 0.1500, 0.1079) -- cycle;
\fill[blue!15.0, opacity=0.7] (0.8000, 0.1500, 0.1079) -- (0.8500, 0.1500, 0.1120) -- (0.8500, 0.2000, 0.1182) -- (0.8000, 0.2000, 0.1141) -- cycle;
\fill[blue!15.0, opacity=0.7] (0.8000, 0.2000, 0.1141) -- (0.8500, 0.2000, 0.1182) -- (0.8500, 0.2500, 0.1243) -- (0.8000, 0.2500, 0.1202) -- cycle;
\fill[blue!15.0, opacity=0.7] (0.8000, 0.2500, 0.1202) -- (0.8500, 0.2500, 0.1243) -- (0.8500, 0.3000, 0.1303) -- (0.8000, 0.3000, 0.1263) -- cycle;
\fill[blue!15.0, opacity=0.7] (0.8000, 0.3000, 0.1263) -- (0.8500, 0.3000, 0.1303) -- (0.8500, 0.3500, 0.1363) -- (0.8000, 0.3500, 0.1322) -- cycle;
\fill[blue!15.0, opacity=0.7] (0.8000, 0.3500, 0.1322) -- (0.8500, 0.3500, 0.1363) -- (0.8500, 0.4000, 0.1421) -- (0.8000, 0.4000, 0.1380) -- cycle;
\fill[blue!15.0, opacity=0.7] (0.8000, 0.4000, 0.1380) -- (0.8500, 0.4000, 0.1421) -- (0.8500, 0.4500, 0.1477) -- (0.8000, 0.4500, 0.1437) -- cycle;
\fill[blue!15.0, opacity=0.7] (0.8000, 0.4500, 0.1437) -- (0.8500, 0.4500, 0.1477) -- (0.8500, 0.5000, 0.1533) -- (0.8000, 0.5000, 0.1492) -- cycle;
\fill[blue!15.0, opacity=0.7] (0.8000, 0.5000, 0.1492) -- (0.8500, 0.5000, 0.1533) -- (0.8500, 0.5500, 0.1586) -- (0.8000, 0.5500, 0.1545) -- cycle;
\fill[blue!15.0, opacity=0.7] (0.8000, 0.5500, 0.1545) -- (0.8500, 0.5500, 0.1586) -- (0.8500, 0.6000, 0.1638) -- (0.8000, 0.6000, 0.1597) -- cycle;
\fill[blue!15.0, opacity=0.7] (0.8000, 0.6000, 0.1597) -- (0.8500, 0.6000, 0.1638) -- (0.8500, 0.6500, 0.1688) -- (0.8000, 0.6500, 0.1647) -- cycle;
\fill[blue!15.0, opacity=0.7] (0.8000, 0.6500, 0.1647) -- (0.8500, 0.6500, 0.1688) -- (0.8500, 0.7000, 0.1736) -- (0.8000, 0.7000, 0.1695) -- cycle;
\fill[blue!15.0, opacity=0.7] (0.8000, 0.7000, 0.1695) -- (0.8500, 0.7000, 0.1736) -- (0.8500, 0.7500, 0.1781) -- (0.8000, 0.7500, 0.1740) -- cycle;
\fill[blue!15.0, opacity=0.7] (0.8000, 0.7500, 0.1740) -- (0.8500, 0.7500, 0.1781) -- (0.8500, 0.8000, 0.1824) -- (0.8000, 0.8000, 0.1784) -- cycle;
\fill[blue!15.0, opacity=0.7] (0.8000, 0.8000, 0.1784) -- (0.8500, 0.8000, 0.1824) -- (0.8500, 0.8500, 0.1865) -- (0.8000, 0.8500, 0.1824) -- cycle;
\fill[blue!15.0, opacity=0.7] (0.8000, 0.8500, 0.1824) -- (0.8500, 0.8500, 0.1865) -- (0.8500, 0.9000, 0.1903) -- (0.8000, 0.9000, 0.1863) -- cycle;
\fill[blue!15.5, opacity=0.7] (0.8000, 0.9000, 0.1863) -- (0.8500, 0.9000, 0.1903) -- (0.8500, 0.9500, 0.1939) -- (0.8000, 0.9500, 0.1898) -- cycle;
\fill[blue!20.2, opacity=0.7] (0.8000, 0.9500, 0.1898) -- (0.8500, 0.9500, 0.1939) -- (0.8500, 1.0000, 0.1972) -- (0.8000, 1.0000, 0.1931) -- cycle;
\fill[blue!17.5, opacity=0.7] (0.8000, 1.0000, 0.1931) -- (0.8500, 1.0000, 0.1972) -- (0.8500, 1.0500, 0.2002) -- (0.8000, 1.0500, 0.1961) -- cycle;
\fill[blue!15.1, opacity=0.7] (0.8000, 1.0500, 0.1961) -- (0.8500, 1.0500, 0.2002) -- (0.8500, 1.1000, 0.2029) -- (0.8000, 1.1000, 0.1988) -- cycle;
\fill[blue!15.0, opacity=0.7] (0.8000, 1.1000, 0.1988) -- (0.8500, 1.1000, 0.2029) -- (0.8500, 1.1500, 0.2053) -- (0.8000, 1.1500, 0.2012) -- cycle;
\fill[blue!15.0, opacity=0.7] (0.8000, 1.1500, 0.2012) -- (0.8500, 1.1500, 0.2053) -- (0.8500, 1.2000, 0.2074) -- (0.8000, 1.2000, 0.2033) -- cycle;
\fill[blue!15.0, opacity=0.7] (0.8000, 1.2000, 0.2033) -- (0.8500, 1.2000, 0.2074) -- (0.8500, 1.2500, 0.2092) -- (0.8000, 1.2500, 0.2051) -- cycle;
\fill[blue!15.0, opacity=0.7] (0.8000, 1.2500, 0.2051) -- (0.8500, 1.2500, 0.2092) -- (0.8500, 1.3000, 0.2106) -- (0.8000, 1.3000, 0.2066) -- cycle;
\fill[blue!15.0, opacity=0.7] (0.8000, 1.3000, 0.2066) -- (0.8500, 1.3000, 0.2106) -- (0.8500, 1.3500, 0.2118) -- (0.8000, 1.3500, 0.2077) -- cycle;
\fill[blue!15.0, opacity=0.7] (0.8000, 1.3500, 0.2077) -- (0.8500, 1.3500, 0.2118) -- (0.8500, 1.4000, 0.2126) -- (0.8000, 1.4000, 0.2085) -- cycle;
\fill[blue!15.0, opacity=0.7] (0.8000, 1.4000, 0.2085) -- (0.8500, 1.4000, 0.2126) -- (0.8500, 1.4500, 0.2131) -- (0.8000, 1.4500, 0.2090) -- cycle;
\fill[blue!15.0, opacity=0.7] (0.8000, 1.4500, 0.2090) -- (0.8500, 1.4500, 0.2131) -- (0.8500, 1.5000, 0.2133) -- (0.8000, 1.5000, 0.2092) -- cycle;
\fill[blue!15.0, opacity=0.7] (0.8000, 1.5000, 0.2092) -- (0.8500, 1.5000, 0.2133) -- (0.8500, 1.5500, 0.2131) -- (0.8000, 1.5500, 0.2090) -- cycle;
\fill[blue!15.0, opacity=0.7] (0.8000, 1.5500, 0.2090) -- (0.8500, 1.5500, 0.2131) -- (0.8500, 1.6000, 0.2126) -- (0.8000, 1.6000, 0.2085) -- cycle;
\fill[blue!15.0, opacity=0.7] (0.8000, 1.6000, 0.2085) -- (0.8500, 1.6000, 0.2126) -- (0.8500, 1.6500, 0.2118) -- (0.8000, 1.6500, 0.2077) -- cycle;
\fill[blue!15.0, opacity=0.7] (0.8000, 1.6500, 0.2077) -- (0.8500, 1.6500, 0.2118) -- (0.8500, 1.7000, 0.2106) -- (0.8000, 1.7000, 0.2066) -- cycle;
\fill[blue!15.0, opacity=0.7] (0.8000, 1.7000, 0.2066) -- (0.8500, 1.7000, 0.2106) -- (0.8500, 1.7500, 0.2092) -- (0.8000, 1.7500, 0.2051) -- cycle;
\fill[blue!15.0, opacity=0.7] (0.8000, 1.7500, 0.2051) -- (0.8500, 1.7500, 0.2092) -- (0.8500, 1.8000, 0.2074) -- (0.8000, 1.8000, 0.2033) -- cycle;
\fill[blue!15.0, opacity=0.7] (0.8000, 1.8000, 0.2033) -- (0.8500, 1.8000, 0.2074) -- (0.8500, 1.8500, 0.2053) -- (0.8000, 1.8500, 0.2012) -- cycle;
\fill[blue!15.0, opacity=0.7] (0.8000, 1.8500, 0.2012) -- (0.8500, 1.8500, 0.2053) -- (0.8500, 1.9000, 0.2029) -- (0.8000, 1.9000, 0.1988) -- cycle;
\fill[blue!15.0, opacity=0.7] (0.8000, 1.9000, 0.1988) -- (0.8500, 1.9000, 0.2029) -- (0.8500, 1.9500, 0.2002) -- (0.8000, 1.9500, 0.1961) -- cycle;
\fill[blue!15.0, opacity=0.7] (0.8000, 1.9500, 0.1961) -- (0.8500, 1.9500, 0.2002) -- (0.8500, 2.0000, 0.1972) -- (0.8000, 2.0000, 0.1931) -- cycle;
\fill[blue!15.0, opacity=0.7] (0.8000, 2.0000, 0.1931) -- (0.8500, 2.0000, 0.1972) -- (0.8500, 2.0500, 0.1939) -- (0.8000, 2.0500, 0.1898) -- cycle;
\fill[blue!15.0, opacity=0.7] (0.8000, 2.0500, 0.1898) -- (0.8500, 2.0500, 0.1939) -- (0.8500, 2.1000, 0.1903) -- (0.8000, 2.1000, 0.1863) -- cycle;
\fill[blue!15.0, opacity=0.7] (0.8000, 2.1000, 0.1863) -- (0.8500, 2.1000, 0.1903) -- (0.8500, 2.1500, 0.1865) -- (0.8000, 2.1500, 0.1824) -- cycle;
\fill[blue!15.0, opacity=0.7] (0.8000, 2.1500, 0.1824) -- (0.8500, 2.1500, 0.1865) -- (0.8500, 2.2000, 0.1824) -- (0.8000, 2.2000, 0.1784) -- cycle;
\fill[blue!15.0, opacity=0.7] (0.8000, 2.2000, 0.1784) -- (0.8500, 2.2000, 0.1824) -- (0.8500, 2.2500, 0.1781) -- (0.8000, 2.2500, 0.1740) -- cycle;
\fill[blue!15.0, opacity=0.7] (0.8000, 2.2500, 0.1740) -- (0.8500, 2.2500, 0.1781) -- (0.8500, 2.3000, 0.1736) -- (0.8000, 2.3000, 0.1695) -- cycle;
\fill[blue!15.0, opacity=0.7] (0.8000, 2.3000, 0.1695) -- (0.8500, 2.3000, 0.1736) -- (0.8500, 2.3500, 0.1688) -- (0.8000, 2.3500, 0.1647) -- cycle;
\fill[blue!15.0, opacity=0.7] (0.8000, 2.3500, 0.1647) -- (0.8500, 2.3500, 0.1688) -- (0.8500, 2.4000, 0.1638) -- (0.8000, 2.4000, 0.1597) -- cycle;
\fill[blue!15.0, opacity=0.7] (0.8000, 2.4000, 0.1597) -- (0.8500, 2.4000, 0.1638) -- (0.8500, 2.4500, 0.1586) -- (0.8000, 2.4500, 0.1545) -- cycle;
\fill[blue!15.0, opacity=0.7] (0.8000, 2.4500, 0.1545) -- (0.8500, 2.4500, 0.1586) -- (0.8500, 2.5000, 0.1533) -- (0.8000, 2.5000, 0.1492) -- cycle;
\fill[blue!15.0, opacity=0.7] (0.8000, 2.5000, 0.1492) -- (0.8500, 2.5000, 0.1533) -- (0.8500, 2.5500, 0.1477) -- (0.8000, 2.5500, 0.1437) -- cycle;
\fill[blue!15.0, opacity=0.7] (0.8000, 2.5500, 0.1437) -- (0.8500, 2.5500, 0.1477) -- (0.8500, 2.6000, 0.1421) -- (0.8000, 2.6000, 0.1380) -- cycle;
\fill[blue!15.0, opacity=0.7] (0.8000, 2.6000, 0.1380) -- (0.8500, 2.6000, 0.1421) -- (0.8500, 2.6500, 0.1363) -- (0.8000, 2.6500, 0.1322) -- cycle;
\fill[blue!15.0, opacity=0.7] (0.8000, 2.6500, 0.1322) -- (0.8500, 2.6500, 0.1363) -- (0.8500, 2.7000, 0.1303) -- (0.8000, 2.7000, 0.1263) -- cycle;
\fill[blue!15.0, opacity=0.7] (0.8000, 2.7000, 0.1263) -- (0.8500, 2.7000, 0.1303) -- (0.8500, 2.7500, 0.1243) -- (0.8000, 2.7500, 0.1202) -- cycle;
\fill[blue!15.0, opacity=0.7] (0.8000, 2.7500, 0.1202) -- (0.8500, 2.7500, 0.1243) -- (0.8500, 2.8000, 0.1182) -- (0.8000, 2.8000, 0.1141) -- cycle;
\fill[blue!15.0, opacity=0.7] (0.8000, 2.8000, 0.1141) -- (0.8500, 2.8000, 0.1182) -- (0.8500, 2.8500, 0.1120) -- (0.8000, 2.8500, 0.1079) -- cycle;
\fill[blue!15.0, opacity=0.7] (0.8000, 2.8500, 0.1079) -- (0.8500, 2.8500, 0.1120) -- (0.8500, 2.9000, 0.1058) -- (0.8000, 2.9000, 0.1017) -- cycle;
\fill[blue!15.0, opacity=0.7] (0.8000, 2.9000, 0.1017) -- (0.8500, 2.9000, 0.1058) -- (0.8500, 2.9500, 0.0995) -- (0.8000, 2.9500, 0.0955) -- cycle;
\fill[blue!15.0, opacity=0.7] (0.8000, 2.9500, 0.0955) -- (0.8500, 2.9500, 0.0995) -- (0.8500, 3.0000, 0.0933) -- (0.8000, 3.0000, 0.0892) -- cycle;
\fill[blue!15.0, opacity=0.7] (0.8500, 0.0000, 0.0933) -- (0.9000, 0.0000, 0.0971) -- (0.9000, 0.0500, 0.1034) -- (0.8500, 0.0500, 0.0995) -- cycle;
\fill[blue!15.0, opacity=0.7] (0.8500, 0.0500, 0.0995) -- (0.9000, 0.0500, 0.1034) -- (0.9000, 0.1000, 0.1096) -- (0.8500, 0.1000, 0.1058) -- cycle;
\fill[blue!15.0, opacity=0.7] (0.8500, 0.1000, 0.1058) -- (0.9000, 0.1000, 0.1096) -- (0.9000, 0.1500, 0.1159) -- (0.8500, 0.1500, 0.1120) -- cycle;
\fill[blue!15.0, opacity=0.7] (0.8500, 0.1500, 0.1120) -- (0.9000, 0.1500, 0.1159) -- (0.9000, 0.2000, 0.1220) -- (0.8500, 0.2000, 0.1182) -- cycle;
\fill[blue!15.0, opacity=0.7] (0.8500, 0.2000, 0.1182) -- (0.9000, 0.2000, 0.1220) -- (0.9000, 0.2500, 0.1281) -- (0.8500, 0.2500, 0.1243) -- cycle;
\fill[blue!15.0, opacity=0.7] (0.8500, 0.2500, 0.1243) -- (0.9000, 0.2500, 0.1281) -- (0.9000, 0.3000, 0.1342) -- (0.8500, 0.3000, 0.1303) -- cycle;
\fill[blue!15.0, opacity=0.7] (0.8500, 0.3000, 0.1303) -- (0.9000, 0.3000, 0.1342) -- (0.9000, 0.3500, 0.1401) -- (0.8500, 0.3500, 0.1363) -- cycle;
\fill[blue!15.0, opacity=0.7] (0.8500, 0.3500, 0.1363) -- (0.9000, 0.3500, 0.1401) -- (0.9000, 0.4000, 0.1459) -- (0.8500, 0.4000, 0.1421) -- cycle;
\fill[blue!15.0, opacity=0.7] (0.8500, 0.4000, 0.1421) -- (0.9000, 0.4000, 0.1459) -- (0.9000, 0.4500, 0.1516) -- (0.8500, 0.4500, 0.1477) -- cycle;
\fill[blue!15.0, opacity=0.7] (0.8500, 0.4500, 0.1477) -- (0.9000, 0.4500, 0.1516) -- (0.9000, 0.5000, 0.1571) -- (0.8500, 0.5000, 0.1533) -- cycle;
\fill[blue!15.0, opacity=0.7] (0.8500, 0.5000, 0.1533) -- (0.9000, 0.5000, 0.1571) -- (0.9000, 0.5500, 0.1624) -- (0.8500, 0.5500, 0.1586) -- cycle;
\fill[blue!15.0, opacity=0.7] (0.8500, 0.5500, 0.1586) -- (0.9000, 0.5500, 0.1624) -- (0.9000, 0.6000, 0.1676) -- (0.8500, 0.6000, 0.1638) -- cycle;
\fill[blue!15.0, opacity=0.7] (0.8500, 0.6000, 0.1638) -- (0.9000, 0.6000, 0.1676) -- (0.9000, 0.6500, 0.1726) -- (0.8500, 0.6500, 0.1688) -- cycle;
\fill[blue!15.0, opacity=0.7] (0.8500, 0.6500, 0.1688) -- (0.9000, 0.6500, 0.1726) -- (0.9000, 0.7000, 0.1774) -- (0.8500, 0.7000, 0.1736) -- cycle;
\fill[blue!15.0, opacity=0.7] (0.8500, 0.7000, 0.1736) -- (0.9000, 0.7000, 0.1774) -- (0.9000, 0.7500, 0.1819) -- (0.8500, 0.7500, 0.1781) -- cycle;
\fill[blue!15.0, opacity=0.7] (0.8500, 0.7500, 0.1781) -- (0.9000, 0.7500, 0.1819) -- (0.9000, 0.8000, 0.1863) -- (0.8500, 0.8000, 0.1824) -- cycle;
\fill[blue!15.0, opacity=0.7] (0.8500, 0.8000, 0.1824) -- (0.9000, 0.8000, 0.1863) -- (0.9000, 0.8500, 0.1903) -- (0.8500, 0.8500, 0.1865) -- cycle;
\fill[blue!15.2, opacity=0.7] (0.8500, 0.8500, 0.1865) -- (0.9000, 0.8500, 0.1903) -- (0.9000, 0.9000, 0.1942) -- (0.8500, 0.9000, 0.1903) -- cycle;
\fill[blue!20.6, opacity=0.7] (0.8500, 0.9000, 0.1903) -- (0.9000, 0.9000, 0.1942) -- (0.9000, 0.9500, 0.1977) -- (0.8500, 0.9500, 0.1939) -- cycle;
\fill[blue!18.5, opacity=0.7] (0.8500, 0.9500, 0.1939) -- (0.9000, 0.9500, 0.1977) -- (0.9000, 1.0000, 0.2010) -- (0.8500, 1.0000, 0.1972) -- cycle;
\fill[blue!15.1, opacity=0.7] (0.8500, 1.0000, 0.1972) -- (0.9000, 1.0000, 0.2010) -- (0.9000, 1.0500, 0.2040) -- (0.8500, 1.0500, 0.2002) -- cycle;
\fill[blue!15.0, opacity=0.7] (0.8500, 1.0500, 0.2002) -- (0.9000, 1.0500, 0.2040) -- (0.9000, 1.1000, 0.2067) -- (0.8500, 1.1000, 0.2029) -- cycle;
\fill[blue!15.0, opacity=0.7] (0.8500, 1.1000, 0.2029) -- (0.9000, 1.1000, 0.2067) -- (0.9000, 1.1500, 0.2091) -- (0.8500, 1.1500, 0.2053) -- cycle;
\fill[blue!15.0, opacity=0.7] (0.8500, 1.1500, 0.2053) -- (0.9000, 1.1500, 0.2091) -- (0.9000, 1.2000, 0.2112) -- (0.8500, 1.2000, 0.2074) -- cycle;
\fill[blue!15.0, opacity=0.7] (0.8500, 1.2000, 0.2074) -- (0.9000, 1.2000, 0.2112) -- (0.9000, 1.2500, 0.2130) -- (0.8500, 1.2500, 0.2092) -- cycle;
\fill[blue!15.0, opacity=0.7] (0.8500, 1.2500, 0.2092) -- (0.9000, 1.2500, 0.2130) -- (0.9000, 1.3000, 0.2145) -- (0.8500, 1.3000, 0.2106) -- cycle;
\fill[blue!15.0, opacity=0.7] (0.8500, 1.3000, 0.2106) -- (0.9000, 1.3000, 0.2145) -- (0.9000, 1.3500, 0.2156) -- (0.8500, 1.3500, 0.2118) -- cycle;
\fill[blue!15.0, opacity=0.7] (0.8500, 1.3500, 0.2118) -- (0.9000, 1.3500, 0.2156) -- (0.9000, 1.4000, 0.2164) -- (0.8500, 1.4000, 0.2126) -- cycle;
\fill[blue!15.0, opacity=0.7] (0.8500, 1.4000, 0.2126) -- (0.9000, 1.4000, 0.2164) -- (0.9000, 1.4500, 0.2169) -- (0.8500, 1.4500, 0.2131) -- cycle;
\fill[blue!15.0, opacity=0.7] (0.8500, 1.4500, 0.2131) -- (0.9000, 1.4500, 0.2169) -- (0.9000, 1.5000, 0.2171) -- (0.8500, 1.5000, 0.2133) -- cycle;
\fill[blue!15.0, opacity=0.7] (0.8500, 1.5000, 0.2133) -- (0.9000, 1.5000, 0.2171) -- (0.9000, 1.5500, 0.2169) -- (0.8500, 1.5500, 0.2131) -- cycle;
\fill[blue!15.1, opacity=0.7] (0.8500, 1.5500, 0.2131) -- (0.9000, 1.5500, 0.2169) -- (0.9000, 1.6000, 0.2164) -- (0.8500, 1.6000, 0.2126) -- cycle;
\fill[blue!15.2, opacity=0.7] (0.8500, 1.6000, 0.2126) -- (0.9000, 1.6000, 0.2164) -- (0.9000, 1.6500, 0.2156) -- (0.8500, 1.6500, 0.2118) -- cycle;
\fill[blue!15.1, opacity=0.7] (0.8500, 1.6500, 0.2118) -- (0.9000, 1.6500, 0.2156) -- (0.9000, 1.7000, 0.2145) -- (0.8500, 1.7000, 0.2106) -- cycle;
\fill[blue!15.0, opacity=0.7] (0.8500, 1.7000, 0.2106) -- (0.9000, 1.7000, 0.2145) -- (0.9000, 1.7500, 0.2130) -- (0.8500, 1.7500, 0.2092) -- cycle;
\fill[blue!15.0, opacity=0.7] (0.8500, 1.7500, 0.2092) -- (0.9000, 1.7500, 0.2130) -- (0.9000, 1.8000, 0.2112) -- (0.8500, 1.8000, 0.2074) -- cycle;
\fill[blue!15.0, opacity=0.7] (0.8500, 1.8000, 0.2074) -- (0.9000, 1.8000, 0.2112) -- (0.9000, 1.8500, 0.2091) -- (0.8500, 1.8500, 0.2053) -- cycle;
\fill[blue!15.0, opacity=0.7] (0.8500, 1.8500, 0.2053) -- (0.9000, 1.8500, 0.2091) -- (0.9000, 1.9000, 0.2067) -- (0.8500, 1.9000, 0.2029) -- cycle;
\fill[blue!15.0, opacity=0.7] (0.8500, 1.9000, 0.2029) -- (0.9000, 1.9000, 0.2067) -- (0.9000, 1.9500, 0.2040) -- (0.8500, 1.9500, 0.2002) -- cycle;
\fill[blue!15.0, opacity=0.7] (0.8500, 1.9500, 0.2002) -- (0.9000, 1.9500, 0.2040) -- (0.9000, 2.0000, 0.2010) -- (0.8500, 2.0000, 0.1972) -- cycle;
\fill[blue!15.0, opacity=0.7] (0.8500, 2.0000, 0.1972) -- (0.9000, 2.0000, 0.2010) -- (0.9000, 2.0500, 0.1977) -- (0.8500, 2.0500, 0.1939) -- cycle;
\fill[blue!15.0, opacity=0.7] (0.8500, 2.0500, 0.1939) -- (0.9000, 2.0500, 0.1977) -- (0.9000, 2.1000, 0.1942) -- (0.8500, 2.1000, 0.1903) -- cycle;
\fill[blue!15.0, opacity=0.7] (0.8500, 2.1000, 0.1903) -- (0.9000, 2.1000, 0.1942) -- (0.9000, 2.1500, 0.1903) -- (0.8500, 2.1500, 0.1865) -- cycle;
\fill[blue!15.0, opacity=0.7] (0.8500, 2.1500, 0.1865) -- (0.9000, 2.1500, 0.1903) -- (0.9000, 2.2000, 0.1863) -- (0.8500, 2.2000, 0.1824) -- cycle;
\fill[blue!15.0, opacity=0.7] (0.8500, 2.2000, 0.1824) -- (0.9000, 2.2000, 0.1863) -- (0.9000, 2.2500, 0.1819) -- (0.8500, 2.2500, 0.1781) -- cycle;
\fill[blue!15.0, opacity=0.7] (0.8500, 2.2500, 0.1781) -- (0.9000, 2.2500, 0.1819) -- (0.9000, 2.3000, 0.1774) -- (0.8500, 2.3000, 0.1736) -- cycle;
\fill[blue!15.0, opacity=0.7] (0.8500, 2.3000, 0.1736) -- (0.9000, 2.3000, 0.1774) -- (0.9000, 2.3500, 0.1726) -- (0.8500, 2.3500, 0.1688) -- cycle;
\fill[blue!15.0, opacity=0.7] (0.8500, 2.3500, 0.1688) -- (0.9000, 2.3500, 0.1726) -- (0.9000, 2.4000, 0.1676) -- (0.8500, 2.4000, 0.1638) -- cycle;
\fill[blue!15.0, opacity=0.7] (0.8500, 2.4000, 0.1638) -- (0.9000, 2.4000, 0.1676) -- (0.9000, 2.4500, 0.1624) -- (0.8500, 2.4500, 0.1586) -- cycle;
\fill[blue!15.0, opacity=0.7] (0.8500, 2.4500, 0.1586) -- (0.9000, 2.4500, 0.1624) -- (0.9000, 2.5000, 0.1571) -- (0.8500, 2.5000, 0.1533) -- cycle;
\fill[blue!15.0, opacity=0.7] (0.8500, 2.5000, 0.1533) -- (0.9000, 2.5000, 0.1571) -- (0.9000, 2.5500, 0.1516) -- (0.8500, 2.5500, 0.1477) -- cycle;
\fill[blue!15.0, opacity=0.7] (0.8500, 2.5500, 0.1477) -- (0.9000, 2.5500, 0.1516) -- (0.9000, 2.6000, 0.1459) -- (0.8500, 2.6000, 0.1421) -- cycle;
\fill[blue!15.0, opacity=0.7] (0.8500, 2.6000, 0.1421) -- (0.9000, 2.6000, 0.1459) -- (0.9000, 2.6500, 0.1401) -- (0.8500, 2.6500, 0.1363) -- cycle;
\fill[blue!15.0, opacity=0.7] (0.8500, 2.6500, 0.1363) -- (0.9000, 2.6500, 0.1401) -- (0.9000, 2.7000, 0.1342) -- (0.8500, 2.7000, 0.1303) -- cycle;
\fill[blue!15.0, opacity=0.7] (0.8500, 2.7000, 0.1303) -- (0.9000, 2.7000, 0.1342) -- (0.9000, 2.7500, 0.1281) -- (0.8500, 2.7500, 0.1243) -- cycle;
\fill[blue!15.0, opacity=0.7] (0.8500, 2.7500, 0.1243) -- (0.9000, 2.7500, 0.1281) -- (0.9000, 2.8000, 0.1220) -- (0.8500, 2.8000, 0.1182) -- cycle;
\fill[blue!15.0, opacity=0.7] (0.8500, 2.8000, 0.1182) -- (0.9000, 2.8000, 0.1220) -- (0.9000, 2.8500, 0.1159) -- (0.8500, 2.8500, 0.1120) -- cycle;
\fill[blue!15.0, opacity=0.7] (0.8500, 2.8500, 0.1120) -- (0.9000, 2.8500, 0.1159) -- (0.9000, 2.9000, 0.1096) -- (0.8500, 2.9000, 0.1058) -- cycle;
\fill[blue!15.0, opacity=0.7] (0.8500, 2.9000, 0.1058) -- (0.9000, 2.9000, 0.1096) -- (0.9000, 2.9500, 0.1034) -- (0.8500, 2.9500, 0.0995) -- cycle;
\fill[blue!15.0, opacity=0.7] (0.8500, 2.9500, 0.0995) -- (0.9000, 2.9500, 0.1034) -- (0.9000, 3.0000, 0.0971) -- (0.8500, 3.0000, 0.0933) -- cycle;
\fill[blue!15.0, opacity=0.7] (0.9000, 0.0000, 0.0971) -- (0.9500, 0.0000, 0.1006) -- (0.9500, 0.0500, 0.1069) -- (0.9000, 0.0500, 0.1034) -- cycle;
\fill[blue!15.0, opacity=0.7] (0.9000, 0.0500, 0.1034) -- (0.9500, 0.0500, 0.1069) -- (0.9500, 0.1000, 0.1132) -- (0.9000, 0.1000, 0.1096) -- cycle;
\fill[blue!15.0, opacity=0.7] (0.9000, 0.1000, 0.1096) -- (0.9500, 0.1000, 0.1132) -- (0.9500, 0.1500, 0.1194) -- (0.9000, 0.1500, 0.1159) -- cycle;
\fill[blue!15.0, opacity=0.7] (0.9000, 0.1500, 0.1159) -- (0.9500, 0.1500, 0.1194) -- (0.9500, 0.2000, 0.1256) -- (0.9000, 0.2000, 0.1220) -- cycle;
\fill[blue!15.0, opacity=0.7] (0.9000, 0.2000, 0.1220) -- (0.9500, 0.2000, 0.1256) -- (0.9500, 0.2500, 0.1317) -- (0.9000, 0.2500, 0.1281) -- cycle;
\fill[blue!15.0, opacity=0.7] (0.9000, 0.2500, 0.1281) -- (0.9500, 0.2500, 0.1317) -- (0.9500, 0.3000, 0.1377) -- (0.9000, 0.3000, 0.1342) -- cycle;
\fill[blue!15.0, opacity=0.7] (0.9000, 0.3000, 0.1342) -- (0.9500, 0.3000, 0.1377) -- (0.9500, 0.3500, 0.1436) -- (0.9000, 0.3500, 0.1401) -- cycle;
\fill[blue!15.0, opacity=0.7] (0.9000, 0.3500, 0.1401) -- (0.9500, 0.3500, 0.1436) -- (0.9500, 0.4000, 0.1494) -- (0.9000, 0.4000, 0.1459) -- cycle;
\fill[blue!15.0, opacity=0.7] (0.9000, 0.4000, 0.1459) -- (0.9500, 0.4000, 0.1494) -- (0.9500, 0.4500, 0.1551) -- (0.9000, 0.4500, 0.1516) -- cycle;
\fill[blue!15.0, opacity=0.7] (0.9000, 0.4500, 0.1516) -- (0.9500, 0.4500, 0.1551) -- (0.9500, 0.5000, 0.1606) -- (0.9000, 0.5000, 0.1571) -- cycle;
\fill[blue!15.0, opacity=0.7] (0.9000, 0.5000, 0.1571) -- (0.9500, 0.5000, 0.1606) -- (0.9500, 0.5500, 0.1660) -- (0.9000, 0.5500, 0.1624) -- cycle;
\fill[blue!15.0, opacity=0.7] (0.9000, 0.5500, 0.1624) -- (0.9500, 0.5500, 0.1660) -- (0.9500, 0.6000, 0.1712) -- (0.9000, 0.6000, 0.1676) -- cycle;
\fill[blue!15.0, opacity=0.7] (0.9000, 0.6000, 0.1676) -- (0.9500, 0.6000, 0.1712) -- (0.9500, 0.6500, 0.1762) -- (0.9000, 0.6500, 0.1726) -- cycle;
\fill[blue!15.0, opacity=0.7] (0.9000, 0.6500, 0.1726) -- (0.9500, 0.6500, 0.1762) -- (0.9500, 0.7000, 0.1809) -- (0.9000, 0.7000, 0.1774) -- cycle;
\fill[blue!15.0, opacity=0.7] (0.9000, 0.7000, 0.1774) -- (0.9500, 0.7000, 0.1809) -- (0.9500, 0.7500, 0.1855) -- (0.9000, 0.7500, 0.1819) -- cycle;
\fill[blue!15.0, opacity=0.7] (0.9000, 0.7500, 0.1819) -- (0.9500, 0.7500, 0.1855) -- (0.9500, 0.8000, 0.1898) -- (0.9000, 0.8000, 0.1863) -- cycle;
\fill[blue!15.0, opacity=0.7] (0.9000, 0.8000, 0.1863) -- (0.9500, 0.8000, 0.1898) -- (0.9500, 0.8500, 0.1939) -- (0.9000, 0.8500, 0.1903) -- cycle;
\fill[blue!19.3, opacity=0.7] (0.9000, 0.8500, 0.1903) -- (0.9500, 0.8500, 0.1939) -- (0.9500, 0.9000, 0.1977) -- (0.9000, 0.9000, 0.1942) -- cycle;
\fill[blue!21.0, opacity=0.7] (0.9000, 0.9000, 0.1942) -- (0.9500, 0.9000, 0.1977) -- (0.9500, 0.9500, 0.2013) -- (0.9000, 0.9500, 0.1977) -- cycle;
\fill[blue!15.2, opacity=0.7] (0.9000, 0.9500, 0.1977) -- (0.9500, 0.9500, 0.2013) -- (0.9500, 1.0000, 0.2046) -- (0.9000, 1.0000, 0.2010) -- cycle;
\fill[blue!15.0, opacity=0.7] (0.9000, 1.0000, 0.2010) -- (0.9500, 1.0000, 0.2046) -- (0.9500, 1.0500, 0.2076) -- (0.9000, 1.0500, 0.2040) -- cycle;
\fill[blue!15.0, opacity=0.7] (0.9000, 1.0500, 0.2040) -- (0.9500, 1.0500, 0.2076) -- (0.9500, 1.1000, 0.2103) -- (0.9000, 1.1000, 0.2067) -- cycle;
\fill[blue!15.0, opacity=0.7] (0.9000, 1.1000, 0.2067) -- (0.9500, 1.1000, 0.2103) -- (0.9500, 1.1500, 0.2127) -- (0.9000, 1.1500, 0.2091) -- cycle;
\fill[blue!15.0, opacity=0.7] (0.9000, 1.1500, 0.2091) -- (0.9500, 1.1500, 0.2127) -- (0.9500, 1.2000, 0.2148) -- (0.9000, 1.2000, 0.2112) -- cycle;
\fill[blue!15.0, opacity=0.7] (0.9000, 1.2000, 0.2112) -- (0.9500, 1.2000, 0.2148) -- (0.9500, 1.2500, 0.2166) -- (0.9000, 1.2500, 0.2130) -- cycle;
\fill[blue!15.0, opacity=0.7] (0.9000, 1.2500, 0.2130) -- (0.9500, 1.2500, 0.2166) -- (0.9500, 1.3000, 0.2180) -- (0.9000, 1.3000, 0.2145) -- cycle;
\fill[blue!15.0, opacity=0.7] (0.9000, 1.3000, 0.2145) -- (0.9500, 1.3000, 0.2180) -- (0.9500, 1.3500, 0.2192) -- (0.9000, 1.3500, 0.2156) -- cycle;
\fill[blue!15.2, opacity=0.7] (0.9000, 1.3500, 0.2156) -- (0.9500, 1.3500, 0.2192) -- (0.9500, 1.4000, 0.2200) -- (0.9000, 1.4000, 0.2164) -- cycle;
\fill[blue!20.2, opacity=0.7] (0.9000, 1.4000, 0.2164) -- (0.9500, 1.4000, 0.2200) -- (0.9500, 1.4500, 0.2205) -- (0.9000, 1.4500, 0.2169) -- cycle;
\fill[blue!39.9, opacity=0.7] (0.9000, 1.4500, 0.2169) -- (0.9500, 1.4500, 0.2205) -- (0.9500, 1.5000, 0.2206) -- (0.9000, 1.5000, 0.2171) -- cycle;
\fill[blue!63.9, opacity=0.7] (0.9000, 1.5000, 0.2171) -- (0.9500, 1.5000, 0.2206) -- (0.9500, 1.5500, 0.2205) -- (0.9000, 1.5500, 0.2169) -- cycle;
\fill[blue!77.5, opacity=0.7] (0.9000, 1.5500, 0.2169) -- (0.9500, 1.5500, 0.2205) -- (0.9500, 1.6000, 0.2200) -- (0.9000, 1.6000, 0.2164) -- cycle;
\fill[blue!78.7, opacity=0.7] (0.9000, 1.6000, 0.2164) -- (0.9500, 1.6000, 0.2200) -- (0.9500, 1.6500, 0.2192) -- (0.9000, 1.6500, 0.2156) -- cycle;
\fill[blue!68.9, opacity=0.7] (0.9000, 1.6500, 0.2156) -- (0.9500, 1.6500, 0.2192) -- (0.9500, 1.7000, 0.2180) -- (0.9000, 1.7000, 0.2145) -- cycle;
\fill[blue!50.0, opacity=0.7] (0.9000, 1.7000, 0.2145) -- (0.9500, 1.7000, 0.2180) -- (0.9500, 1.7500, 0.2166) -- (0.9000, 1.7500, 0.2130) -- cycle;
\fill[blue!28.9, opacity=0.7] (0.9000, 1.7500, 0.2130) -- (0.9500, 1.7500, 0.2166) -- (0.9500, 1.8000, 0.2148) -- (0.9000, 1.8000, 0.2112) -- cycle;
\fill[blue!17.2, opacity=0.7] (0.9000, 1.8000, 0.2112) -- (0.9500, 1.8000, 0.2148) -- (0.9500, 1.8500, 0.2127) -- (0.9000, 1.8500, 0.2091) -- cycle;
\fill[blue!15.1, opacity=0.7] (0.9000, 1.8500, 0.2091) -- (0.9500, 1.8500, 0.2127) -- (0.9500, 1.9000, 0.2103) -- (0.9000, 1.9000, 0.2067) -- cycle;
\fill[blue!15.0, opacity=0.7] (0.9000, 1.9000, 0.2067) -- (0.9500, 1.9000, 0.2103) -- (0.9500, 1.9500, 0.2076) -- (0.9000, 1.9500, 0.2040) -- cycle;
\fill[blue!15.0, opacity=0.7] (0.9000, 1.9500, 0.2040) -- (0.9500, 1.9500, 0.2076) -- (0.9500, 2.0000, 0.2046) -- (0.9000, 2.0000, 0.2010) -- cycle;
\fill[blue!15.0, opacity=0.7] (0.9000, 2.0000, 0.2010) -- (0.9500, 2.0000, 0.2046) -- (0.9500, 2.0500, 0.2013) -- (0.9000, 2.0500, 0.1977) -- cycle;
\fill[blue!15.0, opacity=0.7] (0.9000, 2.0500, 0.1977) -- (0.9500, 2.0500, 0.2013) -- (0.9500, 2.1000, 0.1977) -- (0.9000, 2.1000, 0.1942) -- cycle;
\fill[blue!15.0, opacity=0.7] (0.9000, 2.1000, 0.1942) -- (0.9500, 2.1000, 0.1977) -- (0.9500, 2.1500, 0.1939) -- (0.9000, 2.1500, 0.1903) -- cycle;
\fill[blue!15.0, opacity=0.7] (0.9000, 2.1500, 0.1903) -- (0.9500, 2.1500, 0.1939) -- (0.9500, 2.2000, 0.1898) -- (0.9000, 2.2000, 0.1863) -- cycle;
\fill[blue!15.0, opacity=0.7] (0.9000, 2.2000, 0.1863) -- (0.9500, 2.2000, 0.1898) -- (0.9500, 2.2500, 0.1855) -- (0.9000, 2.2500, 0.1819) -- cycle;
\fill[blue!15.0, opacity=0.7] (0.9000, 2.2500, 0.1819) -- (0.9500, 2.2500, 0.1855) -- (0.9500, 2.3000, 0.1809) -- (0.9000, 2.3000, 0.1774) -- cycle;
\fill[blue!15.0, opacity=0.7] (0.9000, 2.3000, 0.1774) -- (0.9500, 2.3000, 0.1809) -- (0.9500, 2.3500, 0.1762) -- (0.9000, 2.3500, 0.1726) -- cycle;
\fill[blue!15.0, opacity=0.7] (0.9000, 2.3500, 0.1726) -- (0.9500, 2.3500, 0.1762) -- (0.9500, 2.4000, 0.1712) -- (0.9000, 2.4000, 0.1676) -- cycle;
\fill[blue!15.0, opacity=0.7] (0.9000, 2.4000, 0.1676) -- (0.9500, 2.4000, 0.1712) -- (0.9500, 2.4500, 0.1660) -- (0.9000, 2.4500, 0.1624) -- cycle;
\fill[blue!15.0, opacity=0.7] (0.9000, 2.4500, 0.1624) -- (0.9500, 2.4500, 0.1660) -- (0.9500, 2.5000, 0.1606) -- (0.9000, 2.5000, 0.1571) -- cycle;
\fill[blue!15.0, opacity=0.7] (0.9000, 2.5000, 0.1571) -- (0.9500, 2.5000, 0.1606) -- (0.9500, 2.5500, 0.1551) -- (0.9000, 2.5500, 0.1516) -- cycle;
\fill[blue!15.0, opacity=0.7] (0.9000, 2.5500, 0.1516) -- (0.9500, 2.5500, 0.1551) -- (0.9500, 2.6000, 0.1494) -- (0.9000, 2.6000, 0.1459) -- cycle;
\fill[blue!15.0, opacity=0.7] (0.9000, 2.6000, 0.1459) -- (0.9500, 2.6000, 0.1494) -- (0.9500, 2.6500, 0.1436) -- (0.9000, 2.6500, 0.1401) -- cycle;
\fill[blue!15.0, opacity=0.7] (0.9000, 2.6500, 0.1401) -- (0.9500, 2.6500, 0.1436) -- (0.9500, 2.7000, 0.1377) -- (0.9000, 2.7000, 0.1342) -- cycle;
\fill[blue!15.0, opacity=0.7] (0.9000, 2.7000, 0.1342) -- (0.9500, 2.7000, 0.1377) -- (0.9500, 2.7500, 0.1317) -- (0.9000, 2.7500, 0.1281) -- cycle;
\fill[blue!15.0, opacity=0.7] (0.9000, 2.7500, 0.1281) -- (0.9500, 2.7500, 0.1317) -- (0.9500, 2.8000, 0.1256) -- (0.9000, 2.8000, 0.1220) -- cycle;
\fill[blue!15.0, opacity=0.7] (0.9000, 2.8000, 0.1220) -- (0.9500, 2.8000, 0.1256) -- (0.9500, 2.8500, 0.1194) -- (0.9000, 2.8500, 0.1159) -- cycle;
\fill[blue!15.0, opacity=0.7] (0.9000, 2.8500, 0.1159) -- (0.9500, 2.8500, 0.1194) -- (0.9500, 2.9000, 0.1132) -- (0.9000, 2.9000, 0.1096) -- cycle;
\fill[blue!15.0, opacity=0.7] (0.9000, 2.9000, 0.1096) -- (0.9500, 2.9000, 0.1132) -- (0.9500, 2.9500, 0.1069) -- (0.9000, 2.9500, 0.1034) -- cycle;
\fill[blue!15.0, opacity=0.7] (0.9000, 2.9500, 0.1034) -- (0.9500, 2.9500, 0.1069) -- (0.9500, 3.0000, 0.1006) -- (0.9000, 3.0000, 0.0971) -- cycle;
\fill[blue!15.0, opacity=0.7] (0.9500, 0.0000, 0.1006) -- (1.0000, 0.0000, 0.1039) -- (1.0000, 0.0500, 0.1102) -- (0.9500, 0.0500, 0.1069) -- cycle;
\fill[blue!15.0, opacity=0.7] (0.9500, 0.0500, 0.1069) -- (1.0000, 0.0500, 0.1102) -- (1.0000, 0.1000, 0.1165) -- (0.9500, 0.1000, 0.1132) -- cycle;
\fill[blue!15.0, opacity=0.7] (0.9500, 0.1000, 0.1132) -- (1.0000, 0.1000, 0.1165) -- (1.0000, 0.1500, 0.1227) -- (0.9500, 0.1500, 0.1194) -- cycle;
\fill[blue!15.0, opacity=0.7] (0.9500, 0.1500, 0.1194) -- (1.0000, 0.1500, 0.1227) -- (1.0000, 0.2000, 0.1289) -- (0.9500, 0.2000, 0.1256) -- cycle;
\fill[blue!15.0, opacity=0.7] (0.9500, 0.2000, 0.1256) -- (1.0000, 0.2000, 0.1289) -- (1.0000, 0.2500, 0.1350) -- (0.9500, 0.2500, 0.1317) -- cycle;
\fill[blue!15.0, opacity=0.7] (0.9500, 0.2500, 0.1317) -- (1.0000, 0.2500, 0.1350) -- (1.0000, 0.3000, 0.1410) -- (0.9500, 0.3000, 0.1377) -- cycle;
\fill[blue!15.0, opacity=0.7] (0.9500, 0.3000, 0.1377) -- (1.0000, 0.3000, 0.1410) -- (1.0000, 0.3500, 0.1469) -- (0.9500, 0.3500, 0.1436) -- cycle;
\fill[blue!15.0, opacity=0.7] (0.9500, 0.3500, 0.1436) -- (1.0000, 0.3500, 0.1469) -- (1.0000, 0.4000, 0.1527) -- (0.9500, 0.4000, 0.1494) -- cycle;
\fill[blue!15.0, opacity=0.7] (0.9500, 0.4000, 0.1494) -- (1.0000, 0.4000, 0.1527) -- (1.0000, 0.4500, 0.1584) -- (0.9500, 0.4500, 0.1551) -- cycle;
\fill[blue!15.0, opacity=0.7] (0.9500, 0.4500, 0.1551) -- (1.0000, 0.4500, 0.1584) -- (1.0000, 0.5000, 0.1639) -- (0.9500, 0.5000, 0.1606) -- cycle;
\fill[blue!15.0, opacity=0.7] (0.9500, 0.5000, 0.1606) -- (1.0000, 0.5000, 0.1639) -- (1.0000, 0.5500, 0.1693) -- (0.9500, 0.5500, 0.1660) -- cycle;
\fill[blue!15.0, opacity=0.7] (0.9500, 0.5500, 0.1660) -- (1.0000, 0.5500, 0.1693) -- (1.0000, 0.6000, 0.1745) -- (0.9500, 0.6000, 0.1712) -- cycle;
\fill[blue!15.0, opacity=0.7] (0.9500, 0.6000, 0.1712) -- (1.0000, 0.6000, 0.1745) -- (1.0000, 0.6500, 0.1794) -- (0.9500, 0.6500, 0.1762) -- cycle;
\fill[blue!15.0, opacity=0.7] (0.9500, 0.6500, 0.1762) -- (1.0000, 0.6500, 0.1794) -- (1.0000, 0.7000, 0.1842) -- (0.9500, 0.7000, 0.1809) -- cycle;
\fill[blue!15.0, opacity=0.7] (0.9500, 0.7000, 0.1809) -- (1.0000, 0.7000, 0.1842) -- (1.0000, 0.7500, 0.1888) -- (0.9500, 0.7500, 0.1855) -- cycle;
\fill[blue!15.0, opacity=0.7] (0.9500, 0.7500, 0.1855) -- (1.0000, 0.7500, 0.1888) -- (1.0000, 0.8000, 0.1931) -- (0.9500, 0.8000, 0.1898) -- cycle;
\fill[blue!16.5, opacity=0.7] (0.9500, 0.8000, 0.1898) -- (1.0000, 0.8000, 0.1931) -- (1.0000, 0.8500, 0.1972) -- (0.9500, 0.8500, 0.1939) -- cycle;
\fill[blue!24.1, opacity=0.7] (0.9500, 0.8500, 0.1939) -- (1.0000, 0.8500, 0.1972) -- (1.0000, 0.9000, 0.2010) -- (0.9500, 0.9000, 0.1977) -- cycle;
\fill[blue!16.2, opacity=0.7] (0.9500, 0.9000, 0.1977) -- (1.0000, 0.9000, 0.2010) -- (1.0000, 0.9500, 0.2046) -- (0.9500, 0.9500, 0.2013) -- cycle;
\fill[blue!15.0, opacity=0.7] (0.9500, 0.9500, 0.2013) -- (1.0000, 0.9500, 0.2046) -- (1.0000, 1.0000, 0.2078) -- (0.9500, 1.0000, 0.2046) -- cycle;
\fill[blue!15.0, opacity=0.7] (0.9500, 1.0000, 0.2046) -- (1.0000, 1.0000, 0.2078) -- (1.0000, 1.0500, 0.2108) -- (0.9500, 1.0500, 0.2076) -- cycle;
\fill[blue!15.0, opacity=0.7] (0.9500, 1.0500, 0.2076) -- (1.0000, 1.0500, 0.2108) -- (1.0000, 1.1000, 0.2135) -- (0.9500, 1.1000, 0.2103) -- cycle;
\fill[blue!15.0, opacity=0.7] (0.9500, 1.1000, 0.2103) -- (1.0000, 1.1000, 0.2135) -- (1.0000, 1.1500, 0.2160) -- (0.9500, 1.1500, 0.2127) -- cycle;
\fill[blue!15.0, opacity=0.7] (0.9500, 1.1500, 0.2127) -- (1.0000, 1.1500, 0.2160) -- (1.0000, 1.2000, 0.2180) -- (0.9500, 1.2000, 0.2148) -- cycle;
\fill[blue!15.0, opacity=0.7] (0.9500, 1.2000, 0.2148) -- (1.0000, 1.2000, 0.2180) -- (1.0000, 1.2500, 0.2198) -- (0.9500, 1.2500, 0.2166) -- cycle;
\fill[blue!15.1, opacity=0.7] (0.9500, 1.2500, 0.2166) -- (1.0000, 1.2500, 0.2198) -- (1.0000, 1.3000, 0.2213) -- (0.9500, 1.3000, 0.2180) -- cycle;
\fill[blue!24.0, opacity=0.7] (0.9500, 1.3000, 0.2180) -- (1.0000, 1.3000, 0.2213) -- (1.0000, 1.3500, 0.2224) -- (0.9500, 1.3500, 0.2192) -- cycle;
\fill[blue!79.0, opacity=0.7] (0.9500, 1.3500, 0.2192) -- (1.0000, 1.3500, 0.2224) -- (1.0000, 1.4000, 0.2233) -- (0.9500, 1.4000, 0.2200) -- cycle;
\fill[blue!38.5!black, opacity=0.7] (0.9500, 1.4000, 0.2200) -- (1.0000, 1.4000, 0.2233) -- (1.0000, 1.4500, 0.2238) -- (0.9500, 1.4500, 0.2205) -- cycle;
\fill[blue!11.8!black, opacity=0.7] (0.9500, 1.4500, 0.2205) -- (1.0000, 1.4500, 0.2238) -- (1.0000, 1.5000, 0.2239) -- (0.9500, 1.5000, 0.2206) -- cycle;
\fill[blue!10.6!black, opacity=0.7] (0.9500, 1.5000, 0.2206) -- (1.0000, 1.5000, 0.2239) -- (1.0000, 1.5500, 0.2238) -- (0.9500, 1.5500, 0.2205) -- cycle;
\fill[blue!16.6!black, opacity=0.7] (0.9500, 1.5500, 0.2205) -- (1.0000, 1.5500, 0.2238) -- (1.0000, 1.6000, 0.2233) -- (0.9500, 1.6000, 0.2200) -- cycle;
\fill[blue!26.0!black, opacity=0.7] (0.9500, 1.6000, 0.2200) -- (1.0000, 1.6000, 0.2233) -- (1.0000, 1.6500, 0.2224) -- (0.9500, 1.6500, 0.2192) -- cycle;
\fill[blue!34.9!black, opacity=0.7] (0.9500, 1.6500, 0.2192) -- (1.0000, 1.6500, 0.2224) -- (1.0000, 1.7000, 0.2213) -- (0.9500, 1.7000, 0.2180) -- cycle;
\fill[blue!45.0!black, opacity=0.7] (0.9500, 1.7000, 0.2180) -- (1.0000, 1.7000, 0.2213) -- (1.0000, 1.7500, 0.2198) -- (0.9500, 1.7500, 0.2166) -- cycle;
\fill[blue!67.7!black, opacity=0.7] (0.9500, 1.7500, 0.2166) -- (1.0000, 1.7500, 0.2198) -- (1.0000, 1.8000, 0.2180) -- (0.9500, 1.8000, 0.2148) -- cycle;
\fill[blue!89.6, opacity=0.7] (0.9500, 1.8000, 0.2148) -- (1.0000, 1.8000, 0.2180) -- (1.0000, 1.8500, 0.2160) -- (0.9500, 1.8500, 0.2127) -- cycle;
\fill[blue!48.2, opacity=0.7] (0.9500, 1.8500, 0.2127) -- (1.0000, 1.8500, 0.2160) -- (1.0000, 1.9000, 0.2135) -- (0.9500, 1.9000, 0.2103) -- cycle;
\fill[blue!18.9, opacity=0.7] (0.9500, 1.9000, 0.2103) -- (1.0000, 1.9000, 0.2135) -- (1.0000, 1.9500, 0.2108) -- (0.9500, 1.9500, 0.2076) -- cycle;
\fill[blue!15.0, opacity=0.7] (0.9500, 1.9500, 0.2076) -- (1.0000, 1.9500, 0.2108) -- (1.0000, 2.0000, 0.2078) -- (0.9500, 2.0000, 0.2046) -- cycle;
\fill[blue!15.0, opacity=0.7] (0.9500, 2.0000, 0.2046) -- (1.0000, 2.0000, 0.2078) -- (1.0000, 2.0500, 0.2046) -- (0.9500, 2.0500, 0.2013) -- cycle;
\fill[blue!15.0, opacity=0.7] (0.9500, 2.0500, 0.2013) -- (1.0000, 2.0500, 0.2046) -- (1.0000, 2.1000, 0.2010) -- (0.9500, 2.1000, 0.1977) -- cycle;
\fill[blue!15.0, opacity=0.7] (0.9500, 2.1000, 0.1977) -- (1.0000, 2.1000, 0.2010) -- (1.0000, 2.1500, 0.1972) -- (0.9500, 2.1500, 0.1939) -- cycle;
\fill[blue!15.0, opacity=0.7] (0.9500, 2.1500, 0.1939) -- (1.0000, 2.1500, 0.1972) -- (1.0000, 2.2000, 0.1931) -- (0.9500, 2.2000, 0.1898) -- cycle;
\fill[blue!15.0, opacity=0.7] (0.9500, 2.2000, 0.1898) -- (1.0000, 2.2000, 0.1931) -- (1.0000, 2.2500, 0.1888) -- (0.9500, 2.2500, 0.1855) -- cycle;
\fill[blue!15.0, opacity=0.7] (0.9500, 2.2500, 0.1855) -- (1.0000, 2.2500, 0.1888) -- (1.0000, 2.3000, 0.1842) -- (0.9500, 2.3000, 0.1809) -- cycle;
\fill[blue!15.0, opacity=0.7] (0.9500, 2.3000, 0.1809) -- (1.0000, 2.3000, 0.1842) -- (1.0000, 2.3500, 0.1794) -- (0.9500, 2.3500, 0.1762) -- cycle;
\fill[blue!15.0, opacity=0.7] (0.9500, 2.3500, 0.1762) -- (1.0000, 2.3500, 0.1794) -- (1.0000, 2.4000, 0.1745) -- (0.9500, 2.4000, 0.1712) -- cycle;
\fill[blue!15.0, opacity=0.7] (0.9500, 2.4000, 0.1712) -- (1.0000, 2.4000, 0.1745) -- (1.0000, 2.4500, 0.1693) -- (0.9500, 2.4500, 0.1660) -- cycle;
\fill[blue!15.0, opacity=0.7] (0.9500, 2.4500, 0.1660) -- (1.0000, 2.4500, 0.1693) -- (1.0000, 2.5000, 0.1639) -- (0.9500, 2.5000, 0.1606) -- cycle;
\fill[blue!15.0, opacity=0.7] (0.9500, 2.5000, 0.1606) -- (1.0000, 2.5000, 0.1639) -- (1.0000, 2.5500, 0.1584) -- (0.9500, 2.5500, 0.1551) -- cycle;
\fill[blue!15.0, opacity=0.7] (0.9500, 2.5500, 0.1551) -- (1.0000, 2.5500, 0.1584) -- (1.0000, 2.6000, 0.1527) -- (0.9500, 2.6000, 0.1494) -- cycle;
\fill[blue!15.0, opacity=0.7] (0.9500, 2.6000, 0.1494) -- (1.0000, 2.6000, 0.1527) -- (1.0000, 2.6500, 0.1469) -- (0.9500, 2.6500, 0.1436) -- cycle;
\fill[blue!15.0, opacity=0.7] (0.9500, 2.6500, 0.1436) -- (1.0000, 2.6500, 0.1469) -- (1.0000, 2.7000, 0.1410) -- (0.9500, 2.7000, 0.1377) -- cycle;
\fill[blue!15.0, opacity=0.7] (0.9500, 2.7000, 0.1377) -- (1.0000, 2.7000, 0.1410) -- (1.0000, 2.7500, 0.1350) -- (0.9500, 2.7500, 0.1317) -- cycle;
\fill[blue!15.0, opacity=0.7] (0.9500, 2.7500, 0.1317) -- (1.0000, 2.7500, 0.1350) -- (1.0000, 2.8000, 0.1289) -- (0.9500, 2.8000, 0.1256) -- cycle;
\fill[blue!15.0, opacity=0.7] (0.9500, 2.8000, 0.1256) -- (1.0000, 2.8000, 0.1289) -- (1.0000, 2.8500, 0.1227) -- (0.9500, 2.8500, 0.1194) -- cycle;
\fill[blue!15.0, opacity=0.7] (0.9500, 2.8500, 0.1194) -- (1.0000, 2.8500, 0.1227) -- (1.0000, 2.9000, 0.1165) -- (0.9500, 2.9000, 0.1132) -- cycle;
\fill[blue!15.0, opacity=0.7] (0.9500, 2.9000, 0.1132) -- (1.0000, 2.9000, 0.1165) -- (1.0000, 2.9500, 0.1102) -- (0.9500, 2.9500, 0.1069) -- cycle;
\fill[blue!15.0, opacity=0.7] (0.9500, 2.9500, 0.1069) -- (1.0000, 2.9500, 0.1102) -- (1.0000, 3.0000, 0.1039) -- (0.9500, 3.0000, 0.1006) -- cycle;
\fill[blue!15.0, opacity=0.7] (1.0000, 0.0000, 0.1039) -- (1.0500, 0.0000, 0.1069) -- (1.0500, 0.0500, 0.1132) -- (1.0000, 0.0500, 0.1102) -- cycle;
\fill[blue!15.0, opacity=0.7] (1.0000, 0.0500, 0.1102) -- (1.0500, 0.0500, 0.1132) -- (1.0500, 0.1000, 0.1195) -- (1.0000, 0.1000, 0.1165) -- cycle;
\fill[blue!15.0, opacity=0.7] (1.0000, 0.1000, 0.1165) -- (1.0500, 0.1000, 0.1195) -- (1.0500, 0.1500, 0.1257) -- (1.0000, 0.1500, 0.1227) -- cycle;
\fill[blue!15.0, opacity=0.7] (1.0000, 0.1500, 0.1227) -- (1.0500, 0.1500, 0.1257) -- (1.0500, 0.2000, 0.1319) -- (1.0000, 0.2000, 0.1289) -- cycle;
\fill[blue!15.0, opacity=0.7] (1.0000, 0.2000, 0.1289) -- (1.0500, 0.2000, 0.1319) -- (1.0500, 0.2500, 0.1380) -- (1.0000, 0.2500, 0.1350) -- cycle;
\fill[blue!15.0, opacity=0.7] (1.0000, 0.2500, 0.1350) -- (1.0500, 0.2500, 0.1380) -- (1.0500, 0.3000, 0.1440) -- (1.0000, 0.3000, 0.1410) -- cycle;
\fill[blue!15.0, opacity=0.7] (1.0000, 0.3000, 0.1410) -- (1.0500, 0.3000, 0.1440) -- (1.0500, 0.3500, 0.1499) -- (1.0000, 0.3500, 0.1469) -- cycle;
\fill[blue!15.0, opacity=0.7] (1.0000, 0.3500, 0.1469) -- (1.0500, 0.3500, 0.1499) -- (1.0500, 0.4000, 0.1557) -- (1.0000, 0.4000, 0.1527) -- cycle;
\fill[blue!15.0, opacity=0.7] (1.0000, 0.4000, 0.1527) -- (1.0500, 0.4000, 0.1557) -- (1.0500, 0.4500, 0.1614) -- (1.0000, 0.4500, 0.1584) -- cycle;
\fill[blue!15.0, opacity=0.7] (1.0000, 0.4500, 0.1584) -- (1.0500, 0.4500, 0.1614) -- (1.0500, 0.5000, 0.1669) -- (1.0000, 0.5000, 0.1639) -- cycle;
\fill[blue!15.0, opacity=0.7] (1.0000, 0.5000, 0.1639) -- (1.0500, 0.5000, 0.1669) -- (1.0500, 0.5500, 0.1723) -- (1.0000, 0.5500, 0.1693) -- cycle;
\fill[blue!15.0, opacity=0.7] (1.0000, 0.5500, 0.1693) -- (1.0500, 0.5500, 0.1723) -- (1.0500, 0.6000, 0.1775) -- (1.0000, 0.6000, 0.1745) -- cycle;
\fill[blue!15.0, opacity=0.7] (1.0000, 0.6000, 0.1745) -- (1.0500, 0.6000, 0.1775) -- (1.0500, 0.6500, 0.1824) -- (1.0000, 0.6500, 0.1794) -- cycle;
\fill[blue!15.0, opacity=0.7] (1.0000, 0.6500, 0.1794) -- (1.0500, 0.6500, 0.1824) -- (1.0500, 0.7000, 0.1872) -- (1.0000, 0.7000, 0.1842) -- cycle;
\fill[blue!15.0, opacity=0.7] (1.0000, 0.7000, 0.1842) -- (1.0500, 0.7000, 0.1872) -- (1.0500, 0.7500, 0.1918) -- (1.0000, 0.7500, 0.1888) -- cycle;
\fill[blue!15.1, opacity=0.7] (1.0000, 0.7500, 0.1888) -- (1.0500, 0.7500, 0.1918) -- (1.0500, 0.8000, 0.1961) -- (1.0000, 0.8000, 0.1931) -- cycle;
\fill[blue!22.8, opacity=0.7] (1.0000, 0.8000, 0.1931) -- (1.0500, 0.8000, 0.1961) -- (1.0500, 0.8500, 0.2002) -- (1.0000, 0.8500, 0.1972) -- cycle;
\fill[blue!20.6, opacity=0.7] (1.0000, 0.8500, 0.1972) -- (1.0500, 0.8500, 0.2002) -- (1.0500, 0.9000, 0.2040) -- (1.0000, 0.9000, 0.2010) -- cycle;
\fill[blue!15.0, opacity=0.7] (1.0000, 0.9000, 0.2010) -- (1.0500, 0.9000, 0.2040) -- (1.0500, 0.9500, 0.2076) -- (1.0000, 0.9500, 0.2046) -- cycle;
\fill[blue!15.0, opacity=0.7] (1.0000, 0.9500, 0.2046) -- (1.0500, 0.9500, 0.2076) -- (1.0500, 1.0000, 0.2108) -- (1.0000, 1.0000, 0.2078) -- cycle;
\fill[blue!15.0, opacity=0.7] (1.0000, 1.0000, 0.2078) -- (1.0500, 1.0000, 0.2108) -- (1.0500, 1.0500, 0.2138) -- (1.0000, 1.0500, 0.2108) -- cycle;
\fill[blue!15.0, opacity=0.7] (1.0000, 1.0500, 0.2108) -- (1.0500, 1.0500, 0.2138) -- (1.0500, 1.1000, 0.2165) -- (1.0000, 1.1000, 0.2135) -- cycle;
\fill[blue!15.0, opacity=0.7] (1.0000, 1.1000, 0.2135) -- (1.0500, 1.1000, 0.2165) -- (1.0500, 1.1500, 0.2190) -- (1.0000, 1.1500, 0.2160) -- cycle;
\fill[blue!15.0, opacity=0.7] (1.0000, 1.1500, 0.2160) -- (1.0500, 1.1500, 0.2190) -- (1.0500, 1.2000, 0.2210) -- (1.0000, 1.2000, 0.2180) -- cycle;
\fill[blue!15.4, opacity=0.7] (1.0000, 1.2000, 0.2180) -- (1.0500, 1.2000, 0.2210) -- (1.0500, 1.2500, 0.2228) -- (1.0000, 1.2500, 0.2198) -- cycle;
\fill[blue!51.0, opacity=0.7] (1.0000, 1.2500, 0.2198) -- (1.0500, 1.2500, 0.2228) -- (1.0500, 1.3000, 0.2243) -- (1.0000, 1.3000, 0.2213) -- cycle;
\fill[blue!28.0!black, opacity=0.7] (1.0000, 1.3000, 0.2213) -- (1.0500, 1.3000, 0.2243) -- (1.0500, 1.3500, 0.2254) -- (1.0000, 1.3500, 0.2224) -- cycle;
\fill[blue!5.6!black, opacity=0.7] (1.0000, 1.3500, 0.2224) -- (1.0500, 1.3500, 0.2254) -- (1.0500, 1.4000, 0.2263) -- (1.0000, 1.4000, 0.2233) -- cycle;
\fill[blue!5.4!black, opacity=0.7] (1.0000, 1.4000, 0.2233) -- (1.0500, 1.4000, 0.2263) -- (1.0500, 1.4500, 0.2268) -- (1.0000, 1.4500, 0.2238) -- cycle;
\fill[blue!38.6!black, opacity=0.7] (1.0000, 1.4500, 0.2238) -- (1.0500, 1.4500, 0.2268) -- (1.0500, 1.5000, 0.2269) -- (1.0000, 1.5000, 0.2239) -- cycle;
\fill[blue!91.4, opacity=0.7] (1.0000, 1.5000, 0.2239) -- (1.0500, 1.5000, 0.2269) -- (1.0500, 1.5500, 0.2268) -- (1.0000, 1.5500, 0.2238) -- cycle;
\fill[blue!63.5, opacity=0.7] (1.0000, 1.5500, 0.2238) -- (1.0500, 1.5500, 0.2268) -- (1.0500, 1.6000, 0.2263) -- (1.0000, 1.6000, 0.2233) -- cycle;
\fill[blue!50.6, opacity=0.7] (1.0000, 1.6000, 0.2233) -- (1.0500, 1.6000, 0.2263) -- (1.0500, 1.6500, 0.2254) -- (1.0000, 1.6500, 0.2224) -- cycle;
\fill[blue!50.3, opacity=0.7] (1.0000, 1.6500, 0.2224) -- (1.0500, 1.6500, 0.2254) -- (1.0500, 1.7000, 0.2243) -- (1.0000, 1.7000, 0.2213) -- cycle;
\fill[blue!60.8, opacity=0.7] (1.0000, 1.7000, 0.2213) -- (1.0500, 1.7000, 0.2243) -- (1.0500, 1.7500, 0.2228) -- (1.0000, 1.7500, 0.2198) -- cycle;
\fill[blue!80.5, opacity=0.7] (1.0000, 1.7500, 0.2198) -- (1.0500, 1.7500, 0.2228) -- (1.0500, 1.8000, 0.2210) -- (1.0000, 1.8000, 0.2180) -- cycle;
\fill[blue!99.7!black, opacity=0.7] (1.0000, 1.8000, 0.2180) -- (1.0500, 1.8000, 0.2210) -- (1.0500, 1.8500, 0.2190) -- (1.0000, 1.8500, 0.2160) -- cycle;
\fill[blue!85.6!black, opacity=0.7] (1.0000, 1.8500, 0.2160) -- (1.0500, 1.8500, 0.2190) -- (1.0500, 1.9000, 0.2165) -- (1.0000, 1.9000, 0.2135) -- cycle;
\fill[blue!82.9, opacity=0.7] (1.0000, 1.9000, 0.2135) -- (1.0500, 1.9000, 0.2165) -- (1.0500, 1.9500, 0.2138) -- (1.0000, 1.9500, 0.2108) -- cycle;
\fill[blue!33.5, opacity=0.7] (1.0000, 1.9500, 0.2108) -- (1.0500, 1.9500, 0.2138) -- (1.0500, 2.0000, 0.2108) -- (1.0000, 2.0000, 0.2078) -- cycle;
\fill[blue!15.4, opacity=0.7] (1.0000, 2.0000, 0.2078) -- (1.0500, 2.0000, 0.2108) -- (1.0500, 2.0500, 0.2076) -- (1.0000, 2.0500, 0.2046) -- cycle;
\fill[blue!15.0, opacity=0.7] (1.0000, 2.0500, 0.2046) -- (1.0500, 2.0500, 0.2076) -- (1.0500, 2.1000, 0.2040) -- (1.0000, 2.1000, 0.2010) -- cycle;
\fill[blue!15.0, opacity=0.7] (1.0000, 2.1000, 0.2010) -- (1.0500, 2.1000, 0.2040) -- (1.0500, 2.1500, 0.2002) -- (1.0000, 2.1500, 0.1972) -- cycle;
\fill[blue!15.0, opacity=0.7] (1.0000, 2.1500, 0.1972) -- (1.0500, 2.1500, 0.2002) -- (1.0500, 2.2000, 0.1961) -- (1.0000, 2.2000, 0.1931) -- cycle;
\fill[blue!15.0, opacity=0.7] (1.0000, 2.2000, 0.1931) -- (1.0500, 2.2000, 0.1961) -- (1.0500, 2.2500, 0.1918) -- (1.0000, 2.2500, 0.1888) -- cycle;
\fill[blue!15.0, opacity=0.7] (1.0000, 2.2500, 0.1888) -- (1.0500, 2.2500, 0.1918) -- (1.0500, 2.3000, 0.1872) -- (1.0000, 2.3000, 0.1842) -- cycle;
\fill[blue!15.0, opacity=0.7] (1.0000, 2.3000, 0.1842) -- (1.0500, 2.3000, 0.1872) -- (1.0500, 2.3500, 0.1824) -- (1.0000, 2.3500, 0.1794) -- cycle;
\fill[blue!15.0, opacity=0.7] (1.0000, 2.3500, 0.1794) -- (1.0500, 2.3500, 0.1824) -- (1.0500, 2.4000, 0.1775) -- (1.0000, 2.4000, 0.1745) -- cycle;
\fill[blue!15.0, opacity=0.7] (1.0000, 2.4000, 0.1745) -- (1.0500, 2.4000, 0.1775) -- (1.0500, 2.4500, 0.1723) -- (1.0000, 2.4500, 0.1693) -- cycle;
\fill[blue!15.0, opacity=0.7] (1.0000, 2.4500, 0.1693) -- (1.0500, 2.4500, 0.1723) -- (1.0500, 2.5000, 0.1669) -- (1.0000, 2.5000, 0.1639) -- cycle;
\fill[blue!15.0, opacity=0.7] (1.0000, 2.5000, 0.1639) -- (1.0500, 2.5000, 0.1669) -- (1.0500, 2.5500, 0.1614) -- (1.0000, 2.5500, 0.1584) -- cycle;
\fill[blue!15.0, opacity=0.7] (1.0000, 2.5500, 0.1584) -- (1.0500, 2.5500, 0.1614) -- (1.0500, 2.6000, 0.1557) -- (1.0000, 2.6000, 0.1527) -- cycle;
\fill[blue!15.0, opacity=0.7] (1.0000, 2.6000, 0.1527) -- (1.0500, 2.6000, 0.1557) -- (1.0500, 2.6500, 0.1499) -- (1.0000, 2.6500, 0.1469) -- cycle;
\fill[blue!15.0, opacity=0.7] (1.0000, 2.6500, 0.1469) -- (1.0500, 2.6500, 0.1499) -- (1.0500, 2.7000, 0.1440) -- (1.0000, 2.7000, 0.1410) -- cycle;
\fill[blue!15.0, opacity=0.7] (1.0000, 2.7000, 0.1410) -- (1.0500, 2.7000, 0.1440) -- (1.0500, 2.7500, 0.1380) -- (1.0000, 2.7500, 0.1350) -- cycle;
\fill[blue!15.0, opacity=0.7] (1.0000, 2.7500, 0.1350) -- (1.0500, 2.7500, 0.1380) -- (1.0500, 2.8000, 0.1319) -- (1.0000, 2.8000, 0.1289) -- cycle;
\fill[blue!15.0, opacity=0.7] (1.0000, 2.8000, 0.1289) -- (1.0500, 2.8000, 0.1319) -- (1.0500, 2.8500, 0.1257) -- (1.0000, 2.8500, 0.1227) -- cycle;
\fill[blue!15.0, opacity=0.7] (1.0000, 2.8500, 0.1227) -- (1.0500, 2.8500, 0.1257) -- (1.0500, 2.9000, 0.1195) -- (1.0000, 2.9000, 0.1165) -- cycle;
\fill[blue!15.0, opacity=0.7] (1.0000, 2.9000, 0.1165) -- (1.0500, 2.9000, 0.1195) -- (1.0500, 2.9500, 0.1132) -- (1.0000, 2.9500, 0.1102) -- cycle;
\fill[blue!15.0, opacity=0.7] (1.0000, 2.9500, 0.1102) -- (1.0500, 2.9500, 0.1132) -- (1.0500, 3.0000, 0.1069) -- (1.0000, 3.0000, 0.1039) -- cycle;
\fill[blue!15.0, opacity=0.7] (1.0500, 0.0000, 0.1069) -- (1.1000, 0.0000, 0.1096) -- (1.1000, 0.0500, 0.1159) -- (1.0500, 0.0500, 0.1132) -- cycle;
\fill[blue!15.0, opacity=0.7] (1.0500, 0.0500, 0.1132) -- (1.1000, 0.0500, 0.1159) -- (1.1000, 0.1000, 0.1222) -- (1.0500, 0.1000, 0.1195) -- cycle;
\fill[blue!15.0, opacity=0.7] (1.0500, 0.1000, 0.1195) -- (1.1000, 0.1000, 0.1222) -- (1.1000, 0.1500, 0.1284) -- (1.0500, 0.1500, 0.1257) -- cycle;
\fill[blue!15.0, opacity=0.7] (1.0500, 0.1500, 0.1257) -- (1.1000, 0.1500, 0.1284) -- (1.1000, 0.2000, 0.1346) -- (1.0500, 0.2000, 0.1319) -- cycle;
\fill[blue!15.0, opacity=0.7] (1.0500, 0.2000, 0.1319) -- (1.1000, 0.2000, 0.1346) -- (1.1000, 0.2500, 0.1407) -- (1.0500, 0.2500, 0.1380) -- cycle;
\fill[blue!15.0, opacity=0.7] (1.0500, 0.2500, 0.1380) -- (1.1000, 0.2500, 0.1407) -- (1.1000, 0.3000, 0.1467) -- (1.0500, 0.3000, 0.1440) -- cycle;
\fill[blue!15.0, opacity=0.7] (1.0500, 0.3000, 0.1440) -- (1.1000, 0.3000, 0.1467) -- (1.1000, 0.3500, 0.1526) -- (1.0500, 0.3500, 0.1499) -- cycle;
\fill[blue!15.0, opacity=0.7] (1.0500, 0.3500, 0.1499) -- (1.1000, 0.3500, 0.1526) -- (1.1000, 0.4000, 0.1584) -- (1.0500, 0.4000, 0.1557) -- cycle;
\fill[blue!15.0, opacity=0.7] (1.0500, 0.4000, 0.1557) -- (1.1000, 0.4000, 0.1584) -- (1.1000, 0.4500, 0.1641) -- (1.0500, 0.4500, 0.1614) -- cycle;
\fill[blue!15.0, opacity=0.7] (1.0500, 0.4500, 0.1614) -- (1.1000, 0.4500, 0.1641) -- (1.1000, 0.5000, 0.1696) -- (1.0500, 0.5000, 0.1669) -- cycle;
\fill[blue!15.0, opacity=0.7] (1.0500, 0.5000, 0.1669) -- (1.1000, 0.5000, 0.1696) -- (1.1000, 0.5500, 0.1750) -- (1.0500, 0.5500, 0.1723) -- cycle;
\fill[blue!15.0, opacity=0.7] (1.0500, 0.5500, 0.1723) -- (1.1000, 0.5500, 0.1750) -- (1.1000, 0.6000, 0.1802) -- (1.0500, 0.6000, 0.1775) -- cycle;
\fill[blue!15.0, opacity=0.7] (1.0500, 0.6000, 0.1775) -- (1.1000, 0.6000, 0.1802) -- (1.1000, 0.6500, 0.1851) -- (1.0500, 0.6500, 0.1824) -- cycle;
\fill[blue!15.0, opacity=0.7] (1.0500, 0.6500, 0.1824) -- (1.1000, 0.6500, 0.1851) -- (1.1000, 0.7000, 0.1899) -- (1.0500, 0.7000, 0.1872) -- cycle;
\fill[blue!15.0, opacity=0.7] (1.0500, 0.7000, 0.1872) -- (1.1000, 0.7000, 0.1899) -- (1.1000, 0.7500, 0.1945) -- (1.0500, 0.7500, 0.1918) -- cycle;
\fill[blue!16.7, opacity=0.7] (1.0500, 0.7500, 0.1918) -- (1.1000, 0.7500, 0.1945) -- (1.1000, 0.8000, 0.1988) -- (1.0500, 0.8000, 0.1961) -- cycle;
\fill[blue!27.1, opacity=0.7] (1.0500, 0.8000, 0.1961) -- (1.1000, 0.8000, 0.1988) -- (1.1000, 0.8500, 0.2029) -- (1.0500, 0.8500, 0.2002) -- cycle;
\fill[blue!16.3, opacity=0.7] (1.0500, 0.8500, 0.2002) -- (1.1000, 0.8500, 0.2029) -- (1.1000, 0.9000, 0.2067) -- (1.0500, 0.9000, 0.2040) -- cycle;
\fill[blue!15.0, opacity=0.7] (1.0500, 0.9000, 0.2040) -- (1.1000, 0.9000, 0.2067) -- (1.1000, 0.9500, 0.2103) -- (1.0500, 0.9500, 0.2076) -- cycle;
\fill[blue!15.0, opacity=0.7] (1.0500, 0.9500, 0.2076) -- (1.1000, 0.9500, 0.2103) -- (1.1000, 1.0000, 0.2135) -- (1.0500, 1.0000, 0.2108) -- cycle;
\fill[blue!15.0, opacity=0.7] (1.0500, 1.0000, 0.2108) -- (1.1000, 1.0000, 0.2135) -- (1.1000, 1.0500, 0.2165) -- (1.0500, 1.0500, 0.2138) -- cycle;
\fill[blue!15.0, opacity=0.7] (1.0500, 1.0500, 0.2138) -- (1.1000, 1.0500, 0.2165) -- (1.1000, 1.1000, 0.2193) -- (1.0500, 1.1000, 0.2165) -- cycle;
\fill[blue!15.0, opacity=0.7] (1.0500, 1.1000, 0.2165) -- (1.1000, 1.1000, 0.2193) -- (1.1000, 1.1500, 0.2217) -- (1.0500, 1.1500, 0.2190) -- cycle;
\fill[blue!15.3, opacity=0.7] (1.0500, 1.1500, 0.2190) -- (1.1000, 1.1500, 0.2217) -- (1.1000, 1.2000, 0.2238) -- (1.0500, 1.2000, 0.2210) -- cycle;
\fill[blue!63.9, opacity=0.7] (1.0500, 1.2000, 0.2210) -- (1.1000, 1.2000, 0.2238) -- (1.1000, 1.2500, 0.2255) -- (1.0500, 1.2500, 0.2228) -- cycle;
\fill[blue!5.6!black, opacity=0.7] (1.0500, 1.2500, 0.2228) -- (1.1000, 1.2500, 0.2255) -- (1.1000, 1.3000, 0.2270) -- (1.0500, 1.3000, 0.2243) -- cycle;
\fill[blue!13.2!black, opacity=0.7] (1.0500, 1.3000, 0.2243) -- (1.1000, 1.3000, 0.2270) -- (1.1000, 1.3500, 0.2281) -- (1.0500, 1.3500, 0.2254) -- cycle;
\fill[blue!18.0!black, opacity=0.7] (1.0500, 1.3500, 0.2254) -- (1.1000, 1.3500, 0.2281) -- (1.1000, 1.4000, 0.2290) -- (1.0500, 1.4000, 0.2263) -- cycle;
\fill[blue!71.6, opacity=0.7] (1.0500, 1.4000, 0.2263) -- (1.1000, 1.4000, 0.2290) -- (1.1000, 1.4500, 0.2295) -- (1.0500, 1.4500, 0.2268) -- cycle;
\fill[blue!24.9, opacity=0.7] (1.0500, 1.4500, 0.2268) -- (1.1000, 1.4500, 0.2295) -- (1.1000, 1.5000, 0.2296) -- (1.0500, 1.5000, 0.2269) -- cycle;
\fill[blue!16.1, opacity=0.7] (1.0500, 1.5000, 0.2269) -- (1.1000, 1.5000, 0.2296) -- (1.1000, 1.5500, 0.2295) -- (1.0500, 1.5500, 0.2268) -- cycle;
\fill[blue!15.2, opacity=0.7] (1.0500, 1.5500, 0.2268) -- (1.1000, 1.5500, 0.2295) -- (1.1000, 1.6000, 0.2290) -- (1.0500, 1.6000, 0.2263) -- cycle;
\fill[blue!15.1, opacity=0.7] (1.0500, 1.6000, 0.2263) -- (1.1000, 1.6000, 0.2290) -- (1.1000, 1.6500, 0.2281) -- (1.0500, 1.6500, 0.2254) -- cycle;
\fill[blue!15.1, opacity=0.7] (1.0500, 1.6500, 0.2254) -- (1.1000, 1.6500, 0.2281) -- (1.1000, 1.7000, 0.2270) -- (1.0500, 1.7000, 0.2243) -- cycle;
\fill[blue!15.2, opacity=0.7] (1.0500, 1.7000, 0.2243) -- (1.1000, 1.7000, 0.2270) -- (1.1000, 1.7500, 0.2255) -- (1.0500, 1.7500, 0.2228) -- cycle;
\fill[blue!16.3, opacity=0.7] (1.0500, 1.7500, 0.2228) -- (1.1000, 1.7500, 0.2255) -- (1.1000, 1.8000, 0.2238) -- (1.0500, 1.8000, 0.2210) -- cycle;
\fill[blue!23.9, opacity=0.7] (1.0500, 1.8000, 0.2210) -- (1.1000, 1.8000, 0.2238) -- (1.1000, 1.8500, 0.2217) -- (1.0500, 1.8500, 0.2190) -- cycle;
\fill[blue!52.1, opacity=0.7] (1.0500, 1.8500, 0.2190) -- (1.1000, 1.8500, 0.2217) -- (1.1000, 1.9000, 0.2193) -- (1.0500, 1.9000, 0.2165) -- cycle;
\fill[blue!88.2, opacity=0.7] (1.0500, 1.9000, 0.2165) -- (1.1000, 1.9000, 0.2193) -- (1.1000, 1.9500, 0.2165) -- (1.0500, 1.9500, 0.2138) -- cycle;
\fill[blue!90.7, opacity=0.7] (1.0500, 1.9500, 0.2138) -- (1.1000, 1.9500, 0.2165) -- (1.1000, 2.0000, 0.2135) -- (1.0500, 2.0000, 0.2108) -- cycle;
\fill[blue!45.2, opacity=0.7] (1.0500, 2.0000, 0.2108) -- (1.1000, 2.0000, 0.2135) -- (1.1000, 2.0500, 0.2103) -- (1.0500, 2.0500, 0.2076) -- cycle;
\fill[blue!15.8, opacity=0.7] (1.0500, 2.0500, 0.2076) -- (1.1000, 2.0500, 0.2103) -- (1.1000, 2.1000, 0.2067) -- (1.0500, 2.1000, 0.2040) -- cycle;
\fill[blue!15.0, opacity=0.7] (1.0500, 2.1000, 0.2040) -- (1.1000, 2.1000, 0.2067) -- (1.1000, 2.1500, 0.2029) -- (1.0500, 2.1500, 0.2002) -- cycle;
\fill[blue!15.0, opacity=0.7] (1.0500, 2.1500, 0.2002) -- (1.1000, 2.1500, 0.2029) -- (1.1000, 2.2000, 0.1988) -- (1.0500, 2.2000, 0.1961) -- cycle;
\fill[blue!15.0, opacity=0.7] (1.0500, 2.2000, 0.1961) -- (1.1000, 2.2000, 0.1988) -- (1.1000, 2.2500, 0.1945) -- (1.0500, 2.2500, 0.1918) -- cycle;
\fill[blue!15.0, opacity=0.7] (1.0500, 2.2500, 0.1918) -- (1.1000, 2.2500, 0.1945) -- (1.1000, 2.3000, 0.1899) -- (1.0500, 2.3000, 0.1872) -- cycle;
\fill[blue!15.0, opacity=0.7] (1.0500, 2.3000, 0.1872) -- (1.1000, 2.3000, 0.1899) -- (1.1000, 2.3500, 0.1851) -- (1.0500, 2.3500, 0.1824) -- cycle;
\fill[blue!15.0, opacity=0.7] (1.0500, 2.3500, 0.1824) -- (1.1000, 2.3500, 0.1851) -- (1.1000, 2.4000, 0.1802) -- (1.0500, 2.4000, 0.1775) -- cycle;
\fill[blue!15.0, opacity=0.7] (1.0500, 2.4000, 0.1775) -- (1.1000, 2.4000, 0.1802) -- (1.1000, 2.4500, 0.1750) -- (1.0500, 2.4500, 0.1723) -- cycle;
\fill[blue!15.0, opacity=0.7] (1.0500, 2.4500, 0.1723) -- (1.1000, 2.4500, 0.1750) -- (1.1000, 2.5000, 0.1696) -- (1.0500, 2.5000, 0.1669) -- cycle;
\fill[blue!15.0, opacity=0.7] (1.0500, 2.5000, 0.1669) -- (1.1000, 2.5000, 0.1696) -- (1.1000, 2.5500, 0.1641) -- (1.0500, 2.5500, 0.1614) -- cycle;
\fill[blue!15.0, opacity=0.7] (1.0500, 2.5500, 0.1614) -- (1.1000, 2.5500, 0.1641) -- (1.1000, 2.6000, 0.1584) -- (1.0500, 2.6000, 0.1557) -- cycle;
\fill[blue!15.0, opacity=0.7] (1.0500, 2.6000, 0.1557) -- (1.1000, 2.6000, 0.1584) -- (1.1000, 2.6500, 0.1526) -- (1.0500, 2.6500, 0.1499) -- cycle;
\fill[blue!15.0, opacity=0.7] (1.0500, 2.6500, 0.1499) -- (1.1000, 2.6500, 0.1526) -- (1.1000, 2.7000, 0.1467) -- (1.0500, 2.7000, 0.1440) -- cycle;
\fill[blue!15.0, opacity=0.7] (1.0500, 2.7000, 0.1440) -- (1.1000, 2.7000, 0.1467) -- (1.1000, 2.7500, 0.1407) -- (1.0500, 2.7500, 0.1380) -- cycle;
\fill[blue!15.0, opacity=0.7] (1.0500, 2.7500, 0.1380) -- (1.1000, 2.7500, 0.1407) -- (1.1000, 2.8000, 0.1346) -- (1.0500, 2.8000, 0.1319) -- cycle;
\fill[blue!15.0, opacity=0.7] (1.0500, 2.8000, 0.1319) -- (1.1000, 2.8000, 0.1346) -- (1.1000, 2.8500, 0.1284) -- (1.0500, 2.8500, 0.1257) -- cycle;
\fill[blue!15.0, opacity=0.7] (1.0500, 2.8500, 0.1257) -- (1.1000, 2.8500, 0.1284) -- (1.1000, 2.9000, 0.1222) -- (1.0500, 2.9000, 0.1195) -- cycle;
\fill[blue!15.0, opacity=0.7] (1.0500, 2.9000, 0.1195) -- (1.1000, 2.9000, 0.1222) -- (1.1000, 2.9500, 0.1159) -- (1.0500, 2.9500, 0.1132) -- cycle;
\fill[blue!15.0, opacity=0.7] (1.0500, 2.9500, 0.1132) -- (1.1000, 2.9500, 0.1159) -- (1.1000, 3.0000, 0.1096) -- (1.0500, 3.0000, 0.1069) -- cycle;
\fill[blue!15.0, opacity=0.7] (1.1000, 0.0000, 0.1096) -- (1.1500, 0.0000, 0.1120) -- (1.1500, 0.0500, 0.1183) -- (1.1000, 0.0500, 0.1159) -- cycle;
\fill[blue!15.0, opacity=0.7] (1.1000, 0.0500, 0.1159) -- (1.1500, 0.0500, 0.1183) -- (1.1500, 0.1000, 0.1246) -- (1.1000, 0.1000, 0.1222) -- cycle;
\fill[blue!15.0, opacity=0.7] (1.1000, 0.1000, 0.1222) -- (1.1500, 0.1000, 0.1246) -- (1.1500, 0.1500, 0.1308) -- (1.1000, 0.1500, 0.1284) -- cycle;
\fill[blue!15.0, opacity=0.7] (1.1000, 0.1500, 0.1284) -- (1.1500, 0.1500, 0.1308) -- (1.1500, 0.2000, 0.1370) -- (1.1000, 0.2000, 0.1346) -- cycle;
\fill[blue!15.0, opacity=0.7] (1.1000, 0.2000, 0.1346) -- (1.1500, 0.2000, 0.1370) -- (1.1500, 0.2500, 0.1431) -- (1.1000, 0.2500, 0.1407) -- cycle;
\fill[blue!15.0, opacity=0.7] (1.1000, 0.2500, 0.1407) -- (1.1500, 0.2500, 0.1431) -- (1.1500, 0.3000, 0.1491) -- (1.1000, 0.3000, 0.1467) -- cycle;
\fill[blue!15.0, opacity=0.7] (1.1000, 0.3000, 0.1467) -- (1.1500, 0.3000, 0.1491) -- (1.1500, 0.3500, 0.1550) -- (1.1000, 0.3500, 0.1526) -- cycle;
\fill[blue!15.0, opacity=0.7] (1.1000, 0.3500, 0.1526) -- (1.1500, 0.3500, 0.1550) -- (1.1500, 0.4000, 0.1608) -- (1.1000, 0.4000, 0.1584) -- cycle;
\fill[blue!15.0, opacity=0.7] (1.1000, 0.4000, 0.1584) -- (1.1500, 0.4000, 0.1608) -- (1.1500, 0.4500, 0.1665) -- (1.1000, 0.4500, 0.1641) -- cycle;
\fill[blue!15.0, opacity=0.7] (1.1000, 0.4500, 0.1641) -- (1.1500, 0.4500, 0.1665) -- (1.1500, 0.5000, 0.1720) -- (1.1000, 0.5000, 0.1696) -- cycle;
\fill[blue!15.0, opacity=0.7] (1.1000, 0.5000, 0.1696) -- (1.1500, 0.5000, 0.1720) -- (1.1500, 0.5500, 0.1774) -- (1.1000, 0.5500, 0.1750) -- cycle;
\fill[blue!15.0, opacity=0.7] (1.1000, 0.5500, 0.1750) -- (1.1500, 0.5500, 0.1774) -- (1.1500, 0.6000, 0.1826) -- (1.1000, 0.6000, 0.1802) -- cycle;
\fill[blue!15.0, opacity=0.7] (1.1000, 0.6000, 0.1802) -- (1.1500, 0.6000, 0.1826) -- (1.1500, 0.6500, 0.1875) -- (1.1000, 0.6500, 0.1851) -- cycle;
\fill[blue!15.0, opacity=0.7] (1.1000, 0.6500, 0.1851) -- (1.1500, 0.6500, 0.1875) -- (1.1500, 0.7000, 0.1923) -- (1.1000, 0.7000, 0.1899) -- cycle;
\fill[blue!15.0, opacity=0.7] (1.1000, 0.7000, 0.1899) -- (1.1500, 0.7000, 0.1923) -- (1.1500, 0.7500, 0.1969) -- (1.1000, 0.7500, 0.1945) -- cycle;
\fill[blue!22.0, opacity=0.7] (1.1000, 0.7500, 0.1945) -- (1.1500, 0.7500, 0.1969) -- (1.1500, 0.8000, 0.2012) -- (1.1000, 0.8000, 0.1988) -- cycle;
\fill[blue!24.8, opacity=0.7] (1.1000, 0.8000, 0.1988) -- (1.1500, 0.8000, 0.2012) -- (1.1500, 0.8500, 0.2053) -- (1.1000, 0.8500, 0.2029) -- cycle;
\fill[blue!15.1, opacity=0.7] (1.1000, 0.8500, 0.2029) -- (1.1500, 0.8500, 0.2053) -- (1.1500, 0.9000, 0.2091) -- (1.1000, 0.9000, 0.2067) -- cycle;
\fill[blue!15.0, opacity=0.7] (1.1000, 0.9000, 0.2067) -- (1.1500, 0.9000, 0.2091) -- (1.1500, 0.9500, 0.2127) -- (1.1000, 0.9500, 0.2103) -- cycle;
\fill[blue!15.0, opacity=0.7] (1.1000, 0.9500, 0.2103) -- (1.1500, 0.9500, 0.2127) -- (1.1500, 1.0000, 0.2160) -- (1.1000, 1.0000, 0.2135) -- cycle;
\fill[blue!15.0, opacity=0.7] (1.1000, 1.0000, 0.2135) -- (1.1500, 1.0000, 0.2160) -- (1.1500, 1.0500, 0.2190) -- (1.1000, 1.0500, 0.2165) -- cycle;
\fill[blue!15.0, opacity=0.7] (1.1000, 1.0500, 0.2165) -- (1.1500, 1.0500, 0.2190) -- (1.1500, 1.1000, 0.2217) -- (1.1000, 1.1000, 0.2193) -- cycle;
\fill[blue!15.0, opacity=0.7] (1.1000, 1.1000, 0.2193) -- (1.1500, 1.1000, 0.2217) -- (1.1500, 1.1500, 0.2241) -- (1.1000, 1.1500, 0.2217) -- cycle;
\fill[blue!46.0, opacity=0.7] (1.1000, 1.1500, 0.2217) -- (1.1500, 1.1500, 0.2241) -- (1.1500, 1.2000, 0.2262) -- (1.1000, 1.2000, 0.2238) -- cycle;
\fill[blue!5.0!black, opacity=0.7] (1.1000, 1.2000, 0.2238) -- (1.1500, 1.2000, 0.2262) -- (1.1500, 1.2500, 0.2279) -- (1.1000, 1.2500, 0.2255) -- cycle;
\fill[blue!25.2!black, opacity=0.7] (1.1000, 1.2500, 0.2255) -- (1.1500, 1.2500, 0.2279) -- (1.1500, 1.3000, 0.2294) -- (1.1000, 1.3000, 0.2270) -- cycle;
\fill[blue!28.9!black, opacity=0.7] (1.1000, 1.3000, 0.2270) -- (1.1500, 1.3000, 0.2294) -- (1.1500, 1.3500, 0.2306) -- (1.1000, 1.3500, 0.2281) -- cycle;
\fill[blue!39.7, opacity=0.7] (1.1000, 1.3500, 0.2281) -- (1.1500, 1.3500, 0.2306) -- (1.1500, 1.4000, 0.2314) -- (1.1000, 1.4000, 0.2290) -- cycle;
\fill[blue!15.6, opacity=0.7] (1.1000, 1.4000, 0.2290) -- (1.1500, 1.4000, 0.2314) -- (1.1500, 1.4500, 0.2319) -- (1.1000, 1.4500, 0.2295) -- cycle;
\fill[blue!15.0, opacity=0.7] (1.1000, 1.4500, 0.2295) -- (1.1500, 1.4500, 0.2319) -- (1.1500, 1.5000, 0.2320) -- (1.1000, 1.5000, 0.2296) -- cycle;
\fill[blue!15.0, opacity=0.7] (1.1000, 1.5000, 0.2296) -- (1.1500, 1.5000, 0.2320) -- (1.1500, 1.5500, 0.2319) -- (1.1000, 1.5500, 0.2295) -- cycle;
\fill[blue!15.0, opacity=0.7] (1.1000, 1.5500, 0.2295) -- (1.1500, 1.5500, 0.2319) -- (1.1500, 1.6000, 0.2314) -- (1.1000, 1.6000, 0.2290) -- cycle;
\fill[blue!15.0, opacity=0.7] (1.1000, 1.6000, 0.2290) -- (1.1500, 1.6000, 0.2314) -- (1.1500, 1.6500, 0.2306) -- (1.1000, 1.6500, 0.2281) -- cycle;
\fill[blue!15.0, opacity=0.7] (1.1000, 1.6500, 0.2281) -- (1.1500, 1.6500, 0.2306) -- (1.1500, 1.7000, 0.2294) -- (1.1000, 1.7000, 0.2270) -- cycle;
\fill[blue!15.0, opacity=0.7] (1.1000, 1.7000, 0.2270) -- (1.1500, 1.7000, 0.2294) -- (1.1500, 1.7500, 0.2279) -- (1.1000, 1.7500, 0.2255) -- cycle;
\fill[blue!15.0, opacity=0.7] (1.1000, 1.7500, 0.2255) -- (1.1500, 1.7500, 0.2279) -- (1.1500, 1.8000, 0.2262) -- (1.1000, 1.8000, 0.2238) -- cycle;
\fill[blue!15.0, opacity=0.7] (1.1000, 1.8000, 0.2238) -- (1.1500, 1.8000, 0.2262) -- (1.1500, 1.8500, 0.2241) -- (1.1000, 1.8500, 0.2217) -- cycle;
\fill[blue!15.4, opacity=0.7] (1.1000, 1.8500, 0.2217) -- (1.1500, 1.8500, 0.2241) -- (1.1500, 1.9000, 0.2217) -- (1.1000, 1.9000, 0.2193) -- cycle;
\fill[blue!24.8, opacity=0.7] (1.1000, 1.9000, 0.2193) -- (1.1500, 1.9000, 0.2217) -- (1.1500, 1.9500, 0.2190) -- (1.1000, 1.9500, 0.2165) -- cycle;
\fill[blue!65.5, opacity=0.7] (1.1000, 1.9500, 0.2165) -- (1.1500, 1.9500, 0.2190) -- (1.1500, 2.0000, 0.2160) -- (1.1000, 2.0000, 0.2135) -- cycle;
\fill[blue!85.9, opacity=0.7] (1.1000, 2.0000, 0.2135) -- (1.1500, 2.0000, 0.2160) -- (1.1500, 2.0500, 0.2127) -- (1.1000, 2.0500, 0.2103) -- cycle;
\fill[blue!44.5, opacity=0.7] (1.1000, 2.0500, 0.2103) -- (1.1500, 2.0500, 0.2127) -- (1.1500, 2.1000, 0.2091) -- (1.1000, 2.1000, 0.2067) -- cycle;
\fill[blue!15.5, opacity=0.7] (1.1000, 2.1000, 0.2067) -- (1.1500, 2.1000, 0.2091) -- (1.1500, 2.1500, 0.2053) -- (1.1000, 2.1500, 0.2029) -- cycle;
\fill[blue!15.0, opacity=0.7] (1.1000, 2.1500, 0.2029) -- (1.1500, 2.1500, 0.2053) -- (1.1500, 2.2000, 0.2012) -- (1.1000, 2.2000, 0.1988) -- cycle;
\fill[blue!15.0, opacity=0.7] (1.1000, 2.2000, 0.1988) -- (1.1500, 2.2000, 0.2012) -- (1.1500, 2.2500, 0.1969) -- (1.1000, 2.2500, 0.1945) -- cycle;
\fill[blue!15.0, opacity=0.7] (1.1000, 2.2500, 0.1945) -- (1.1500, 2.2500, 0.1969) -- (1.1500, 2.3000, 0.1923) -- (1.1000, 2.3000, 0.1899) -- cycle;
\fill[blue!15.0, opacity=0.7] (1.1000, 2.3000, 0.1899) -- (1.1500, 2.3000, 0.1923) -- (1.1500, 2.3500, 0.1875) -- (1.1000, 2.3500, 0.1851) -- cycle;
\fill[blue!15.0, opacity=0.7] (1.1000, 2.3500, 0.1851) -- (1.1500, 2.3500, 0.1875) -- (1.1500, 2.4000, 0.1826) -- (1.1000, 2.4000, 0.1802) -- cycle;
\fill[blue!15.0, opacity=0.7] (1.1000, 2.4000, 0.1802) -- (1.1500, 2.4000, 0.1826) -- (1.1500, 2.4500, 0.1774) -- (1.1000, 2.4500, 0.1750) -- cycle;
\fill[blue!15.0, opacity=0.7] (1.1000, 2.4500, 0.1750) -- (1.1500, 2.4500, 0.1774) -- (1.1500, 2.5000, 0.1720) -- (1.1000, 2.5000, 0.1696) -- cycle;
\fill[blue!15.0, opacity=0.7] (1.1000, 2.5000, 0.1696) -- (1.1500, 2.5000, 0.1720) -- (1.1500, 2.5500, 0.1665) -- (1.1000, 2.5500, 0.1641) -- cycle;
\fill[blue!15.0, opacity=0.7] (1.1000, 2.5500, 0.1641) -- (1.1500, 2.5500, 0.1665) -- (1.1500, 2.6000, 0.1608) -- (1.1000, 2.6000, 0.1584) -- cycle;
\fill[blue!15.0, opacity=0.7] (1.1000, 2.6000, 0.1584) -- (1.1500, 2.6000, 0.1608) -- (1.1500, 2.6500, 0.1550) -- (1.1000, 2.6500, 0.1526) -- cycle;
\fill[blue!15.0, opacity=0.7] (1.1000, 2.6500, 0.1526) -- (1.1500, 2.6500, 0.1550) -- (1.1500, 2.7000, 0.1491) -- (1.1000, 2.7000, 0.1467) -- cycle;
\fill[blue!15.0, opacity=0.7] (1.1000, 2.7000, 0.1467) -- (1.1500, 2.7000, 0.1491) -- (1.1500, 2.7500, 0.1431) -- (1.1000, 2.7500, 0.1407) -- cycle;
\fill[blue!15.0, opacity=0.7] (1.1000, 2.7500, 0.1407) -- (1.1500, 2.7500, 0.1431) -- (1.1500, 2.8000, 0.1370) -- (1.1000, 2.8000, 0.1346) -- cycle;
\fill[blue!15.0, opacity=0.7] (1.1000, 2.8000, 0.1346) -- (1.1500, 2.8000, 0.1370) -- (1.1500, 2.8500, 0.1308) -- (1.1000, 2.8500, 0.1284) -- cycle;
\fill[blue!15.0, opacity=0.7] (1.1000, 2.8500, 0.1284) -- (1.1500, 2.8500, 0.1308) -- (1.1500, 2.9000, 0.1246) -- (1.1000, 2.9000, 0.1222) -- cycle;
\fill[blue!15.0, opacity=0.7] (1.1000, 2.9000, 0.1222) -- (1.1500, 2.9000, 0.1246) -- (1.1500, 2.9500, 0.1183) -- (1.1000, 2.9500, 0.1159) -- cycle;
\fill[blue!15.0, opacity=0.7] (1.1000, 2.9500, 0.1159) -- (1.1500, 2.9500, 0.1183) -- (1.1500, 3.0000, 0.1120) -- (1.1000, 3.0000, 0.1096) -- cycle;
\fill[blue!15.0, opacity=0.7] (1.1500, 0.0000, 0.1120) -- (1.2000, 0.0000, 0.1141) -- (1.2000, 0.0500, 0.1204) -- (1.1500, 0.0500, 0.1183) -- cycle;
\fill[blue!15.0, opacity=0.7] (1.1500, 0.0500, 0.1183) -- (1.2000, 0.0500, 0.1204) -- (1.2000, 0.1000, 0.1267) -- (1.1500, 0.1000, 0.1246) -- cycle;
\fill[blue!15.0, opacity=0.7] (1.1500, 0.1000, 0.1246) -- (1.2000, 0.1000, 0.1267) -- (1.2000, 0.1500, 0.1329) -- (1.1500, 0.1500, 0.1308) -- cycle;
\fill[blue!15.0, opacity=0.7] (1.1500, 0.1500, 0.1308) -- (1.2000, 0.1500, 0.1329) -- (1.2000, 0.2000, 0.1391) -- (1.1500, 0.2000, 0.1370) -- cycle;
\fill[blue!15.0, opacity=0.7] (1.1500, 0.2000, 0.1370) -- (1.2000, 0.2000, 0.1391) -- (1.2000, 0.2500, 0.1452) -- (1.1500, 0.2500, 0.1431) -- cycle;
\fill[blue!15.0, opacity=0.7] (1.1500, 0.2500, 0.1431) -- (1.2000, 0.2500, 0.1452) -- (1.2000, 0.3000, 0.1512) -- (1.1500, 0.3000, 0.1491) -- cycle;
\fill[blue!15.0, opacity=0.7] (1.1500, 0.3000, 0.1491) -- (1.2000, 0.3000, 0.1512) -- (1.2000, 0.3500, 0.1571) -- (1.1500, 0.3500, 0.1550) -- cycle;
\fill[blue!15.0, opacity=0.7] (1.1500, 0.3500, 0.1550) -- (1.2000, 0.3500, 0.1571) -- (1.2000, 0.4000, 0.1629) -- (1.1500, 0.4000, 0.1608) -- cycle;
\fill[blue!15.0, opacity=0.7] (1.1500, 0.4000, 0.1608) -- (1.2000, 0.4000, 0.1629) -- (1.2000, 0.4500, 0.1686) -- (1.1500, 0.4500, 0.1665) -- cycle;
\fill[blue!15.0, opacity=0.7] (1.1500, 0.4500, 0.1665) -- (1.2000, 0.4500, 0.1686) -- (1.2000, 0.5000, 0.1741) -- (1.1500, 0.5000, 0.1720) -- cycle;
\fill[blue!15.0, opacity=0.7] (1.1500, 0.5000, 0.1720) -- (1.2000, 0.5000, 0.1741) -- (1.2000, 0.5500, 0.1795) -- (1.1500, 0.5500, 0.1774) -- cycle;
\fill[blue!15.0, opacity=0.7] (1.1500, 0.5500, 0.1774) -- (1.2000, 0.5500, 0.1795) -- (1.2000, 0.6000, 0.1847) -- (1.1500, 0.6000, 0.1826) -- cycle;
\fill[blue!15.0, opacity=0.7] (1.1500, 0.6000, 0.1826) -- (1.2000, 0.6000, 0.1847) -- (1.2000, 0.6500, 0.1896) -- (1.1500, 0.6500, 0.1875) -- cycle;
\fill[blue!15.0, opacity=0.7] (1.1500, 0.6500, 0.1875) -- (1.2000, 0.6500, 0.1896) -- (1.2000, 0.7000, 0.1944) -- (1.1500, 0.7000, 0.1923) -- cycle;
\fill[blue!15.2, opacity=0.7] (1.1500, 0.7000, 0.1923) -- (1.2000, 0.7000, 0.1944) -- (1.2000, 0.7500, 0.1990) -- (1.1500, 0.7500, 0.1969) -- cycle;
\fill[blue!28.6, opacity=0.7] (1.1500, 0.7500, 0.1969) -- (1.2000, 0.7500, 0.1990) -- (1.2000, 0.8000, 0.2033) -- (1.1500, 0.8000, 0.2012) -- cycle;
\fill[blue!20.4, opacity=0.7] (1.1500, 0.8000, 0.2012) -- (1.2000, 0.8000, 0.2033) -- (1.2000, 0.8500, 0.2074) -- (1.1500, 0.8500, 0.2053) -- cycle;
\fill[blue!15.0, opacity=0.7] (1.1500, 0.8500, 0.2053) -- (1.2000, 0.8500, 0.2074) -- (1.2000, 0.9000, 0.2112) -- (1.1500, 0.9000, 0.2091) -- cycle;
\fill[blue!15.0, opacity=0.7] (1.1500, 0.9000, 0.2091) -- (1.2000, 0.9000, 0.2112) -- (1.2000, 0.9500, 0.2148) -- (1.1500, 0.9500, 0.2127) -- cycle;
\fill[blue!15.0, opacity=0.7] (1.1500, 0.9500, 0.2127) -- (1.2000, 0.9500, 0.2148) -- (1.2000, 1.0000, 0.2180) -- (1.1500, 1.0000, 0.2160) -- cycle;
\fill[blue!15.0, opacity=0.7] (1.1500, 1.0000, 0.2160) -- (1.2000, 1.0000, 0.2180) -- (1.2000, 1.0500, 0.2210) -- (1.1500, 1.0500, 0.2190) -- cycle;
\fill[blue!15.0, opacity=0.7] (1.1500, 1.0500, 0.2190) -- (1.2000, 1.0500, 0.2210) -- (1.2000, 1.1000, 0.2238) -- (1.1500, 1.1000, 0.2217) -- cycle;
\fill[blue!19.4, opacity=0.7] (1.1500, 1.1000, 0.2217) -- (1.2000, 1.1000, 0.2238) -- (1.2000, 1.1500, 0.2262) -- (1.1500, 1.1500, 0.2241) -- cycle;
\fill[blue!24.3!black, opacity=0.7] (1.1500, 1.1500, 0.2241) -- (1.2000, 1.1500, 0.2262) -- (1.2000, 1.2000, 0.2283) -- (1.1500, 1.2000, 0.2262) -- cycle;
\fill[blue!44.8!black, opacity=0.7] (1.1500, 1.2000, 0.2262) -- (1.2000, 1.2000, 0.2283) -- (1.2000, 1.2500, 0.2300) -- (1.1500, 1.2500, 0.2279) -- cycle;
\fill[blue!7.4!black, opacity=0.7] (1.1500, 1.2500, 0.2279) -- (1.2000, 1.2500, 0.2300) -- (1.2000, 1.3000, 0.2315) -- (1.1500, 1.3000, 0.2294) -- cycle;
\fill[blue!40.6, opacity=0.7] (1.1500, 1.3000, 0.2294) -- (1.2000, 1.3000, 0.2315) -- (1.2000, 1.3500, 0.2326) -- (1.1500, 1.3500, 0.2306) -- cycle;
\fill[blue!15.2, opacity=0.7] (1.1500, 1.3500, 0.2306) -- (1.2000, 1.3500, 0.2326) -- (1.2000, 1.4000, 0.2335) -- (1.1500, 1.4000, 0.2314) -- cycle;
\fill[blue!15.0, opacity=0.7] (1.1500, 1.4000, 0.2314) -- (1.2000, 1.4000, 0.2335) -- (1.2000, 1.4500, 0.2340) -- (1.1500, 1.4500, 0.2319) -- cycle;
\fill[blue!15.0, opacity=0.7] (1.1500, 1.4500, 0.2319) -- (1.2000, 1.4500, 0.2340) -- (1.2000, 1.5000, 0.2341) -- (1.1500, 1.5000, 0.2320) -- cycle;
\fill[blue!15.0, opacity=0.7] (1.1500, 1.5000, 0.2320) -- (1.2000, 1.5000, 0.2341) -- (1.2000, 1.5500, 0.2340) -- (1.1500, 1.5500, 0.2319) -- cycle;
\fill[blue!15.0, opacity=0.7] (1.1500, 1.5500, 0.2319) -- (1.2000, 1.5500, 0.2340) -- (1.2000, 1.6000, 0.2335) -- (1.1500, 1.6000, 0.2314) -- cycle;
\fill[blue!15.0, opacity=0.7] (1.1500, 1.6000, 0.2314) -- (1.2000, 1.6000, 0.2335) -- (1.2000, 1.6500, 0.2326) -- (1.1500, 1.6500, 0.2306) -- cycle;
\fill[blue!15.0, opacity=0.7] (1.1500, 1.6500, 0.2306) -- (1.2000, 1.6500, 0.2326) -- (1.2000, 1.7000, 0.2315) -- (1.1500, 1.7000, 0.2294) -- cycle;
\fill[blue!15.0, opacity=0.7] (1.1500, 1.7000, 0.2294) -- (1.2000, 1.7000, 0.2315) -- (1.2000, 1.7500, 0.2300) -- (1.1500, 1.7500, 0.2279) -- cycle;
\fill[blue!15.0, opacity=0.7] (1.1500, 1.7500, 0.2279) -- (1.2000, 1.7500, 0.2300) -- (1.2000, 1.8000, 0.2283) -- (1.1500, 1.8000, 0.2262) -- cycle;
\fill[blue!15.0, opacity=0.7] (1.1500, 1.8000, 0.2262) -- (1.2000, 1.8000, 0.2283) -- (1.2000, 1.8500, 0.2262) -- (1.1500, 1.8500, 0.2241) -- cycle;
\fill[blue!15.0, opacity=0.7] (1.1500, 1.8500, 0.2241) -- (1.2000, 1.8500, 0.2262) -- (1.2000, 1.9000, 0.2238) -- (1.1500, 1.9000, 0.2217) -- cycle;
\fill[blue!15.0, opacity=0.7] (1.1500, 1.9000, 0.2217) -- (1.2000, 1.9000, 0.2238) -- (1.2000, 1.9500, 0.2210) -- (1.1500, 1.9500, 0.2190) -- cycle;
\fill[blue!18.5, opacity=0.7] (1.1500, 1.9500, 0.2190) -- (1.2000, 1.9500, 0.2210) -- (1.2000, 2.0000, 0.2180) -- (1.1500, 2.0000, 0.2160) -- cycle;
\fill[blue!54.5, opacity=0.7] (1.1500, 2.0000, 0.2160) -- (1.2000, 2.0000, 0.2180) -- (1.2000, 2.0500, 0.2148) -- (1.1500, 2.0500, 0.2127) -- cycle;
\fill[blue!78.3, opacity=0.7] (1.1500, 2.0500, 0.2127) -- (1.2000, 2.0500, 0.2148) -- (1.2000, 2.1000, 0.2112) -- (1.1500, 2.1000, 0.2091) -- cycle;
\fill[blue!33.8, opacity=0.7] (1.1500, 2.1000, 0.2091) -- (1.2000, 2.1000, 0.2112) -- (1.2000, 2.1500, 0.2074) -- (1.1500, 2.1500, 0.2053) -- cycle;
\fill[blue!15.1, opacity=0.7] (1.1500, 2.1500, 0.2053) -- (1.2000, 2.1500, 0.2074) -- (1.2000, 2.2000, 0.2033) -- (1.1500, 2.2000, 0.2012) -- cycle;
\fill[blue!15.0, opacity=0.7] (1.1500, 2.2000, 0.2012) -- (1.2000, 2.2000, 0.2033) -- (1.2000, 2.2500, 0.1990) -- (1.1500, 2.2500, 0.1969) -- cycle;
\fill[blue!15.0, opacity=0.7] (1.1500, 2.2500, 0.1969) -- (1.2000, 2.2500, 0.1990) -- (1.2000, 2.3000, 0.1944) -- (1.1500, 2.3000, 0.1923) -- cycle;
\fill[blue!15.0, opacity=0.7] (1.1500, 2.3000, 0.1923) -- (1.2000, 2.3000, 0.1944) -- (1.2000, 2.3500, 0.1896) -- (1.1500, 2.3500, 0.1875) -- cycle;
\fill[blue!15.0, opacity=0.7] (1.1500, 2.3500, 0.1875) -- (1.2000, 2.3500, 0.1896) -- (1.2000, 2.4000, 0.1847) -- (1.1500, 2.4000, 0.1826) -- cycle;
\fill[blue!15.0, opacity=0.7] (1.1500, 2.4000, 0.1826) -- (1.2000, 2.4000, 0.1847) -- (1.2000, 2.4500, 0.1795) -- (1.1500, 2.4500, 0.1774) -- cycle;
\fill[blue!15.0, opacity=0.7] (1.1500, 2.4500, 0.1774) -- (1.2000, 2.4500, 0.1795) -- (1.2000, 2.5000, 0.1741) -- (1.1500, 2.5000, 0.1720) -- cycle;
\fill[blue!15.0, opacity=0.7] (1.1500, 2.5000, 0.1720) -- (1.2000, 2.5000, 0.1741) -- (1.2000, 2.5500, 0.1686) -- (1.1500, 2.5500, 0.1665) -- cycle;
\fill[blue!15.0, opacity=0.7] (1.1500, 2.5500, 0.1665) -- (1.2000, 2.5500, 0.1686) -- (1.2000, 2.6000, 0.1629) -- (1.1500, 2.6000, 0.1608) -- cycle;
\fill[blue!15.0, opacity=0.7] (1.1500, 2.6000, 0.1608) -- (1.2000, 2.6000, 0.1629) -- (1.2000, 2.6500, 0.1571) -- (1.1500, 2.6500, 0.1550) -- cycle;
\fill[blue!15.0, opacity=0.7] (1.1500, 2.6500, 0.1550) -- (1.2000, 2.6500, 0.1571) -- (1.2000, 2.7000, 0.1512) -- (1.1500, 2.7000, 0.1491) -- cycle;
\fill[blue!15.0, opacity=0.7] (1.1500, 2.7000, 0.1491) -- (1.2000, 2.7000, 0.1512) -- (1.2000, 2.7500, 0.1452) -- (1.1500, 2.7500, 0.1431) -- cycle;
\fill[blue!15.0, opacity=0.7] (1.1500, 2.7500, 0.1431) -- (1.2000, 2.7500, 0.1452) -- (1.2000, 2.8000, 0.1391) -- (1.1500, 2.8000, 0.1370) -- cycle;
\fill[blue!15.0, opacity=0.7] (1.1500, 2.8000, 0.1370) -- (1.2000, 2.8000, 0.1391) -- (1.2000, 2.8500, 0.1329) -- (1.1500, 2.8500, 0.1308) -- cycle;
\fill[blue!15.0, opacity=0.7] (1.1500, 2.8500, 0.1308) -- (1.2000, 2.8500, 0.1329) -- (1.2000, 2.9000, 0.1267) -- (1.1500, 2.9000, 0.1246) -- cycle;
\fill[blue!15.0, opacity=0.7] (1.1500, 2.9000, 0.1246) -- (1.2000, 2.9000, 0.1267) -- (1.2000, 2.9500, 0.1204) -- (1.1500, 2.9500, 0.1183) -- cycle;
\fill[blue!15.0, opacity=0.7] (1.1500, 2.9500, 0.1183) -- (1.2000, 2.9500, 0.1204) -- (1.2000, 3.0000, 0.1141) -- (1.1500, 3.0000, 0.1120) -- cycle;
\fill[blue!15.0, opacity=0.7] (1.2000, 0.0000, 0.1141) -- (1.2500, 0.0000, 0.1159) -- (1.2500, 0.0500, 0.1222) -- (1.2000, 0.0500, 0.1204) -- cycle;
\fill[blue!15.0, opacity=0.7] (1.2000, 0.0500, 0.1204) -- (1.2500, 0.0500, 0.1222) -- (1.2500, 0.1000, 0.1285) -- (1.2000, 0.1000, 0.1267) -- cycle;
\fill[blue!15.0, opacity=0.7] (1.2000, 0.1000, 0.1267) -- (1.2500, 0.1000, 0.1285) -- (1.2500, 0.1500, 0.1347) -- (1.2000, 0.1500, 0.1329) -- cycle;
\fill[blue!15.0, opacity=0.7] (1.2000, 0.1500, 0.1329) -- (1.2500, 0.1500, 0.1347) -- (1.2500, 0.2000, 0.1409) -- (1.2000, 0.2000, 0.1391) -- cycle;
\fill[blue!15.0, opacity=0.7] (1.2000, 0.2000, 0.1391) -- (1.2500, 0.2000, 0.1409) -- (1.2500, 0.2500, 0.1470) -- (1.2000, 0.2500, 0.1452) -- cycle;
\fill[blue!15.0, opacity=0.7] (1.2000, 0.2500, 0.1452) -- (1.2500, 0.2500, 0.1470) -- (1.2500, 0.3000, 0.1530) -- (1.2000, 0.3000, 0.1512) -- cycle;
\fill[blue!15.0, opacity=0.7] (1.2000, 0.3000, 0.1512) -- (1.2500, 0.3000, 0.1530) -- (1.2500, 0.3500, 0.1589) -- (1.2000, 0.3500, 0.1571) -- cycle;
\fill[blue!15.0, opacity=0.7] (1.2000, 0.3500, 0.1571) -- (1.2500, 0.3500, 0.1589) -- (1.2500, 0.4000, 0.1647) -- (1.2000, 0.4000, 0.1629) -- cycle;
\fill[blue!15.0, opacity=0.7] (1.2000, 0.4000, 0.1629) -- (1.2500, 0.4000, 0.1647) -- (1.2500, 0.4500, 0.1704) -- (1.2000, 0.4500, 0.1686) -- cycle;
\fill[blue!15.0, opacity=0.7] (1.2000, 0.4500, 0.1686) -- (1.2500, 0.4500, 0.1704) -- (1.2500, 0.5000, 0.1759) -- (1.2000, 0.5000, 0.1741) -- cycle;
\fill[blue!15.0, opacity=0.7] (1.2000, 0.5000, 0.1741) -- (1.2500, 0.5000, 0.1759) -- (1.2500, 0.5500, 0.1813) -- (1.2000, 0.5500, 0.1795) -- cycle;
\fill[blue!15.0, opacity=0.7] (1.2000, 0.5500, 0.1795) -- (1.2500, 0.5500, 0.1813) -- (1.2500, 0.6000, 0.1864) -- (1.2000, 0.6000, 0.1847) -- cycle;
\fill[blue!15.0, opacity=0.7] (1.2000, 0.6000, 0.1847) -- (1.2500, 0.6000, 0.1864) -- (1.2500, 0.6500, 0.1914) -- (1.2000, 0.6500, 0.1896) -- cycle;
\fill[blue!15.0, opacity=0.7] (1.2000, 0.6500, 0.1896) -- (1.2500, 0.6500, 0.1914) -- (1.2500, 0.7000, 0.1962) -- (1.2000, 0.7000, 0.1944) -- cycle;
\fill[blue!16.3, opacity=0.7] (1.2000, 0.7000, 0.1944) -- (1.2500, 0.7000, 0.1962) -- (1.2500, 0.7500, 0.2008) -- (1.2000, 0.7500, 0.1990) -- cycle;
\fill[blue!32.7, opacity=0.7] (1.2000, 0.7500, 0.1990) -- (1.2500, 0.7500, 0.2008) -- (1.2500, 0.8000, 0.2051) -- (1.2000, 0.8000, 0.2033) -- cycle;
\fill[blue!17.5, opacity=0.7] (1.2000, 0.8000, 0.2033) -- (1.2500, 0.8000, 0.2051) -- (1.2500, 0.8500, 0.2092) -- (1.2000, 0.8500, 0.2074) -- cycle;
\fill[blue!15.0, opacity=0.7] (1.2000, 0.8500, 0.2074) -- (1.2500, 0.8500, 0.2092) -- (1.2500, 0.9000, 0.2130) -- (1.2000, 0.9000, 0.2112) -- cycle;
\fill[blue!15.0, opacity=0.7] (1.2000, 0.9000, 0.2112) -- (1.2500, 0.9000, 0.2130) -- (1.2500, 0.9500, 0.2166) -- (1.2000, 0.9500, 0.2148) -- cycle;
\fill[blue!15.0, opacity=0.7] (1.2000, 0.9500, 0.2148) -- (1.2500, 0.9500, 0.2166) -- (1.2500, 1.0000, 0.2198) -- (1.2000, 1.0000, 0.2180) -- cycle;
\fill[blue!15.0, opacity=0.7] (1.2000, 1.0000, 0.2180) -- (1.2500, 1.0000, 0.2198) -- (1.2500, 1.0500, 0.2228) -- (1.2000, 1.0500, 0.2210) -- cycle;
\fill[blue!15.0, opacity=0.7] (1.2000, 1.0500, 0.2210) -- (1.2500, 1.0500, 0.2228) -- (1.2500, 1.1000, 0.2255) -- (1.2000, 1.1000, 0.2238) -- cycle;
\fill[blue!62.5, opacity=0.7] (1.2000, 1.1000, 0.2238) -- (1.2500, 1.1000, 0.2255) -- (1.2500, 1.1500, 0.2279) -- (1.2000, 1.1500, 0.2262) -- cycle;
\fill[blue!29.1!black, opacity=0.7] (1.2000, 1.1500, 0.2262) -- (1.2500, 1.1500, 0.2279) -- (1.2500, 1.2000, 0.2300) -- (1.2000, 1.2000, 0.2283) -- cycle;
\fill[blue!33.5!black, opacity=0.7] (1.2000, 1.2000, 0.2283) -- (1.2500, 1.2000, 0.2300) -- (1.2500, 1.2500, 0.2318) -- (1.2000, 1.2500, 0.2300) -- cycle;
\fill[blue!83.0, opacity=0.7] (1.2000, 1.2500, 0.2300) -- (1.2500, 1.2500, 0.2318) -- (1.2500, 1.3000, 0.2333) -- (1.2000, 1.3000, 0.2315) -- cycle;
\fill[blue!15.7, opacity=0.7] (1.2000, 1.3000, 0.2315) -- (1.2500, 1.3000, 0.2333) -- (1.2500, 1.3500, 0.2344) -- (1.2000, 1.3500, 0.2326) -- cycle;
\fill[blue!15.0, opacity=0.7] (1.2000, 1.3500, 0.2326) -- (1.2500, 1.3500, 0.2344) -- (1.2500, 1.4000, 0.2353) -- (1.2000, 1.4000, 0.2335) -- cycle;
\fill[blue!15.0, opacity=0.7] (1.2000, 1.4000, 0.2335) -- (1.2500, 1.4000, 0.2353) -- (1.2500, 1.4500, 0.2357) -- (1.2000, 1.4500, 0.2340) -- cycle;
\fill[blue!15.0, opacity=0.7] (1.2000, 1.4500, 0.2340) -- (1.2500, 1.4500, 0.2357) -- (1.2500, 1.5000, 0.2359) -- (1.2000, 1.5000, 0.2341) -- cycle;
\fill[blue!15.0, opacity=0.7] (1.2000, 1.5000, 0.2341) -- (1.2500, 1.5000, 0.2359) -- (1.2500, 1.5500, 0.2357) -- (1.2000, 1.5500, 0.2340) -- cycle;
\fill[blue!15.0, opacity=0.7] (1.2000, 1.5500, 0.2340) -- (1.2500, 1.5500, 0.2357) -- (1.2500, 1.6000, 0.2353) -- (1.2000, 1.6000, 0.2335) -- cycle;
\fill[blue!15.1, opacity=0.7] (1.2000, 1.6000, 0.2335) -- (1.2500, 1.6000, 0.2353) -- (1.2500, 1.6500, 0.2344) -- (1.2000, 1.6500, 0.2326) -- cycle;
\fill[blue!15.0, opacity=0.7] (1.2000, 1.6500, 0.2326) -- (1.2500, 1.6500, 0.2344) -- (1.2500, 1.7000, 0.2333) -- (1.2000, 1.7000, 0.2315) -- cycle;
\fill[blue!15.0, opacity=0.7] (1.2000, 1.7000, 0.2315) -- (1.2500, 1.7000, 0.2333) -- (1.2500, 1.7500, 0.2318) -- (1.2000, 1.7500, 0.2300) -- cycle;
\fill[blue!15.0, opacity=0.7] (1.2000, 1.7500, 0.2300) -- (1.2500, 1.7500, 0.2318) -- (1.2500, 1.8000, 0.2300) -- (1.2000, 1.8000, 0.2283) -- cycle;
\fill[blue!15.0, opacity=0.7] (1.2000, 1.8000, 0.2283) -- (1.2500, 1.8000, 0.2300) -- (1.2500, 1.8500, 0.2279) -- (1.2000, 1.8500, 0.2262) -- cycle;
\fill[blue!15.0, opacity=0.7] (1.2000, 1.8500, 0.2262) -- (1.2500, 1.8500, 0.2279) -- (1.2500, 1.9000, 0.2255) -- (1.2000, 1.9000, 0.2238) -- cycle;
\fill[blue!15.0, opacity=0.7] (1.2000, 1.9000, 0.2238) -- (1.2500, 1.9000, 0.2255) -- (1.2500, 1.9500, 0.2228) -- (1.2000, 1.9500, 0.2210) -- cycle;
\fill[blue!15.0, opacity=0.7] (1.2000, 1.9500, 0.2210) -- (1.2500, 1.9500, 0.2228) -- (1.2500, 2.0000, 0.2198) -- (1.2000, 2.0000, 0.2180) -- cycle;
\fill[blue!18.0, opacity=0.7] (1.2000, 2.0000, 0.2180) -- (1.2500, 2.0000, 0.2198) -- (1.2500, 2.0500, 0.2166) -- (1.2000, 2.0500, 0.2148) -- cycle;
\fill[blue!55.7, opacity=0.7] (1.2000, 2.0500, 0.2148) -- (1.2500, 2.0500, 0.2166) -- (1.2500, 2.1000, 0.2130) -- (1.2000, 2.1000, 0.2112) -- cycle;
\fill[blue!67.0, opacity=0.7] (1.2000, 2.1000, 0.2112) -- (1.2500, 2.1000, 0.2130) -- (1.2500, 2.1500, 0.2092) -- (1.2000, 2.1500, 0.2074) -- cycle;
\fill[blue!20.7, opacity=0.7] (1.2000, 2.1500, 0.2074) -- (1.2500, 2.1500, 0.2092) -- (1.2500, 2.2000, 0.2051) -- (1.2000, 2.2000, 0.2033) -- cycle;
\fill[blue!15.0, opacity=0.7] (1.2000, 2.2000, 0.2033) -- (1.2500, 2.2000, 0.2051) -- (1.2500, 2.2500, 0.2008) -- (1.2000, 2.2500, 0.1990) -- cycle;
\fill[blue!15.0, opacity=0.7] (1.2000, 2.2500, 0.1990) -- (1.2500, 2.2500, 0.2008) -- (1.2500, 2.3000, 0.1962) -- (1.2000, 2.3000, 0.1944) -- cycle;
\fill[blue!15.0, opacity=0.7] (1.2000, 2.3000, 0.1944) -- (1.2500, 2.3000, 0.1962) -- (1.2500, 2.3500, 0.1914) -- (1.2000, 2.3500, 0.1896) -- cycle;
\fill[blue!15.0, opacity=0.7] (1.2000, 2.3500, 0.1896) -- (1.2500, 2.3500, 0.1914) -- (1.2500, 2.4000, 0.1864) -- (1.2000, 2.4000, 0.1847) -- cycle;
\fill[blue!15.0, opacity=0.7] (1.2000, 2.4000, 0.1847) -- (1.2500, 2.4000, 0.1864) -- (1.2500, 2.4500, 0.1813) -- (1.2000, 2.4500, 0.1795) -- cycle;
\fill[blue!15.0, opacity=0.7] (1.2000, 2.4500, 0.1795) -- (1.2500, 2.4500, 0.1813) -- (1.2500, 2.5000, 0.1759) -- (1.2000, 2.5000, 0.1741) -- cycle;
\fill[blue!15.0, opacity=0.7] (1.2000, 2.5000, 0.1741) -- (1.2500, 2.5000, 0.1759) -- (1.2500, 2.5500, 0.1704) -- (1.2000, 2.5500, 0.1686) -- cycle;
\fill[blue!15.0, opacity=0.7] (1.2000, 2.5500, 0.1686) -- (1.2500, 2.5500, 0.1704) -- (1.2500, 2.6000, 0.1647) -- (1.2000, 2.6000, 0.1629) -- cycle;
\fill[blue!15.0, opacity=0.7] (1.2000, 2.6000, 0.1629) -- (1.2500, 2.6000, 0.1647) -- (1.2500, 2.6500, 0.1589) -- (1.2000, 2.6500, 0.1571) -- cycle;
\fill[blue!15.0, opacity=0.7] (1.2000, 2.6500, 0.1571) -- (1.2500, 2.6500, 0.1589) -- (1.2500, 2.7000, 0.1530) -- (1.2000, 2.7000, 0.1512) -- cycle;
\fill[blue!15.0, opacity=0.7] (1.2000, 2.7000, 0.1512) -- (1.2500, 2.7000, 0.1530) -- (1.2500, 2.7500, 0.1470) -- (1.2000, 2.7500, 0.1452) -- cycle;
\fill[blue!15.0, opacity=0.7] (1.2000, 2.7500, 0.1452) -- (1.2500, 2.7500, 0.1470) -- (1.2500, 2.8000, 0.1409) -- (1.2000, 2.8000, 0.1391) -- cycle;
\fill[blue!15.0, opacity=0.7] (1.2000, 2.8000, 0.1391) -- (1.2500, 2.8000, 0.1409) -- (1.2500, 2.8500, 0.1347) -- (1.2000, 2.8500, 0.1329) -- cycle;
\fill[blue!15.0, opacity=0.7] (1.2000, 2.8500, 0.1329) -- (1.2500, 2.8500, 0.1347) -- (1.2500, 2.9000, 0.1285) -- (1.2000, 2.9000, 0.1267) -- cycle;
\fill[blue!15.0, opacity=0.7] (1.2000, 2.9000, 0.1267) -- (1.2500, 2.9000, 0.1285) -- (1.2500, 2.9500, 0.1222) -- (1.2000, 2.9500, 0.1204) -- cycle;
\fill[blue!15.0, opacity=0.7] (1.2000, 2.9500, 0.1204) -- (1.2500, 2.9500, 0.1222) -- (1.2500, 3.0000, 0.1159) -- (1.2000, 3.0000, 0.1141) -- cycle;
\fill[blue!15.0, opacity=0.7] (1.2500, 0.0000, 0.1159) -- (1.3000, 0.0000, 0.1174) -- (1.3000, 0.0500, 0.1237) -- (1.2500, 0.0500, 0.1222) -- cycle;
\fill[blue!15.0, opacity=0.7] (1.2500, 0.0500, 0.1222) -- (1.3000, 0.0500, 0.1237) -- (1.3000, 0.1000, 0.1299) -- (1.2500, 0.1000, 0.1285) -- cycle;
\fill[blue!15.0, opacity=0.7] (1.2500, 0.1000, 0.1285) -- (1.3000, 0.1000, 0.1299) -- (1.3000, 0.1500, 0.1361) -- (1.2500, 0.1500, 0.1347) -- cycle;
\fill[blue!15.0, opacity=0.7] (1.2500, 0.1500, 0.1347) -- (1.3000, 0.1500, 0.1361) -- (1.3000, 0.2000, 0.1423) -- (1.2500, 0.2000, 0.1409) -- cycle;
\fill[blue!15.0, opacity=0.7] (1.2500, 0.2000, 0.1409) -- (1.3000, 0.2000, 0.1423) -- (1.3000, 0.2500, 0.1484) -- (1.2500, 0.2500, 0.1470) -- cycle;
\fill[blue!15.0, opacity=0.7] (1.2500, 0.2500, 0.1470) -- (1.3000, 0.2500, 0.1484) -- (1.3000, 0.3000, 0.1545) -- (1.2500, 0.3000, 0.1530) -- cycle;
\fill[blue!15.0, opacity=0.7] (1.2500, 0.3000, 0.1530) -- (1.3000, 0.3000, 0.1545) -- (1.3000, 0.3500, 0.1604) -- (1.2500, 0.3500, 0.1589) -- cycle;
\fill[blue!15.0, opacity=0.7] (1.2500, 0.3500, 0.1589) -- (1.3000, 0.3500, 0.1604) -- (1.3000, 0.4000, 0.1662) -- (1.2500, 0.4000, 0.1647) -- cycle;
\fill[blue!15.0, opacity=0.7] (1.2500, 0.4000, 0.1647) -- (1.3000, 0.4000, 0.1662) -- (1.3000, 0.4500, 0.1719) -- (1.2500, 0.4500, 0.1704) -- cycle;
\fill[blue!15.0, opacity=0.7] (1.2500, 0.4500, 0.1704) -- (1.3000, 0.4500, 0.1719) -- (1.3000, 0.5000, 0.1774) -- (1.2500, 0.5000, 0.1759) -- cycle;
\fill[blue!15.0, opacity=0.7] (1.2500, 0.5000, 0.1759) -- (1.3000, 0.5000, 0.1774) -- (1.3000, 0.5500, 0.1827) -- (1.2500, 0.5500, 0.1813) -- cycle;
\fill[blue!15.0, opacity=0.7] (1.2500, 0.5500, 0.1813) -- (1.3000, 0.5500, 0.1827) -- (1.3000, 0.6000, 0.1879) -- (1.2500, 0.6000, 0.1864) -- cycle;
\fill[blue!15.0, opacity=0.7] (1.2500, 0.6000, 0.1864) -- (1.3000, 0.6000, 0.1879) -- (1.3000, 0.6500, 0.1929) -- (1.2500, 0.6500, 0.1914) -- cycle;
\fill[blue!15.0, opacity=0.7] (1.2500, 0.6500, 0.1914) -- (1.3000, 0.6500, 0.1929) -- (1.3000, 0.7000, 0.1977) -- (1.2500, 0.7000, 0.1962) -- cycle;
\fill[blue!18.3, opacity=0.7] (1.2500, 0.7000, 0.1962) -- (1.3000, 0.7000, 0.1977) -- (1.3000, 0.7500, 0.2022) -- (1.2500, 0.7500, 0.2008) -- cycle;
\fill[blue!34.6, opacity=0.7] (1.2500, 0.7500, 0.2008) -- (1.3000, 0.7500, 0.2022) -- (1.3000, 0.8000, 0.2066) -- (1.2500, 0.8000, 0.2051) -- cycle;
\fill[blue!16.3, opacity=0.7] (1.2500, 0.8000, 0.2051) -- (1.3000, 0.8000, 0.2066) -- (1.3000, 0.8500, 0.2106) -- (1.2500, 0.8500, 0.2092) -- cycle;
\fill[blue!15.0, opacity=0.7] (1.2500, 0.8500, 0.2092) -- (1.3000, 0.8500, 0.2106) -- (1.3000, 0.9000, 0.2145) -- (1.2500, 0.9000, 0.2130) -- cycle;
\fill[blue!15.0, opacity=0.7] (1.2500, 0.9000, 0.2130) -- (1.3000, 0.9000, 0.2145) -- (1.3000, 0.9500, 0.2180) -- (1.2500, 0.9500, 0.2166) -- cycle;
\fill[blue!15.0, opacity=0.7] (1.2500, 0.9500, 0.2166) -- (1.3000, 0.9500, 0.2180) -- (1.3000, 1.0000, 0.2213) -- (1.2500, 1.0000, 0.2198) -- cycle;
\fill[blue!15.0, opacity=0.7] (1.2500, 1.0000, 0.2198) -- (1.3000, 1.0000, 0.2213) -- (1.3000, 1.0500, 0.2243) -- (1.2500, 1.0500, 0.2228) -- cycle;
\fill[blue!15.6, opacity=0.7] (1.2500, 1.0500, 0.2228) -- (1.3000, 1.0500, 0.2243) -- (1.3000, 1.1000, 0.2270) -- (1.2500, 1.1000, 0.2255) -- cycle;
\fill[blue!48.8!black, opacity=0.7] (1.2500, 1.1000, 0.2255) -- (1.3000, 1.1000, 0.2270) -- (1.3000, 1.1500, 0.2294) -- (1.2500, 1.1500, 0.2279) -- cycle;
\fill[blue!80.9!black, opacity=0.7] (1.2500, 1.1500, 0.2279) -- (1.3000, 1.1500, 0.2294) -- (1.3000, 1.2000, 0.2315) -- (1.2500, 1.2000, 0.2300) -- cycle;
\fill[blue!5.7!black, opacity=0.7] (1.2500, 1.2000, 0.2300) -- (1.3000, 1.2000, 0.2315) -- (1.3000, 1.2500, 0.2333) -- (1.2500, 1.2500, 0.2318) -- cycle;
\fill[blue!30.3, opacity=0.7] (1.2500, 1.2500, 0.2318) -- (1.3000, 1.2500, 0.2333) -- (1.3000, 1.3000, 0.2348) -- (1.2500, 1.3000, 0.2333) -- cycle;
\fill[blue!15.0, opacity=0.7] (1.2500, 1.3000, 0.2333) -- (1.3000, 1.3000, 0.2348) -- (1.3000, 1.3500, 0.2359) -- (1.2500, 1.3500, 0.2344) -- cycle;
\fill[blue!15.0, opacity=0.7] (1.2500, 1.3500, 0.2344) -- (1.3000, 1.3500, 0.2359) -- (1.3000, 1.4000, 0.2367) -- (1.2500, 1.4000, 0.2353) -- cycle;
\fill[blue!15.0, opacity=0.7] (1.2500, 1.4000, 0.2353) -- (1.3000, 1.4000, 0.2367) -- (1.3000, 1.4500, 0.2372) -- (1.2500, 1.4500, 0.2357) -- cycle;
\fill[blue!15.7, opacity=0.7] (1.2500, 1.4500, 0.2357) -- (1.3000, 1.4500, 0.2372) -- (1.3000, 1.5000, 0.2374) -- (1.2500, 1.5000, 0.2359) -- cycle;
\fill[blue!42.8, opacity=0.7] (1.2500, 1.5000, 0.2359) -- (1.3000, 1.5000, 0.2374) -- (1.3000, 1.5500, 0.2372) -- (1.2500, 1.5500, 0.2357) -- cycle;
\fill[blue!97.5, opacity=0.7] (1.2500, 1.5500, 0.2357) -- (1.3000, 1.5500, 0.2372) -- (1.3000, 1.6000, 0.2367) -- (1.2500, 1.6000, 0.2353) -- cycle;
\fill[blue!90.3!black, opacity=0.7] (1.2500, 1.6000, 0.2353) -- (1.3000, 1.6000, 0.2367) -- (1.3000, 1.6500, 0.2359) -- (1.2500, 1.6500, 0.2344) -- cycle;
\fill[blue!63.5, opacity=0.7] (1.2500, 1.6500, 0.2344) -- (1.3000, 1.6500, 0.2359) -- (1.3000, 1.7000, 0.2348) -- (1.2500, 1.7000, 0.2333) -- cycle;
\fill[blue!20.8, opacity=0.7] (1.2500, 1.7000, 0.2333) -- (1.3000, 1.7000, 0.2348) -- (1.3000, 1.7500, 0.2333) -- (1.2500, 1.7500, 0.2318) -- cycle;
\fill[blue!15.1, opacity=0.7] (1.2500, 1.7500, 0.2318) -- (1.3000, 1.7500, 0.2333) -- (1.3000, 1.8000, 0.2315) -- (1.2500, 1.8000, 0.2300) -- cycle;
\fill[blue!15.0, opacity=0.7] (1.2500, 1.8000, 0.2300) -- (1.3000, 1.8000, 0.2315) -- (1.3000, 1.8500, 0.2294) -- (1.2500, 1.8500, 0.2279) -- cycle;
\fill[blue!15.0, opacity=0.7] (1.2500, 1.8500, 0.2279) -- (1.3000, 1.8500, 0.2294) -- (1.3000, 1.9000, 0.2270) -- (1.2500, 1.9000, 0.2255) -- cycle;
\fill[blue!15.0, opacity=0.7] (1.2500, 1.9000, 0.2255) -- (1.3000, 1.9000, 0.2270) -- (1.3000, 1.9500, 0.2243) -- (1.2500, 1.9500, 0.2228) -- cycle;
\fill[blue!15.0, opacity=0.7] (1.2500, 1.9500, 0.2228) -- (1.3000, 1.9500, 0.2243) -- (1.3000, 2.0000, 0.2213) -- (1.2500, 2.0000, 0.2198) -- cycle;
\fill[blue!15.0, opacity=0.7] (1.2500, 2.0000, 0.2198) -- (1.3000, 2.0000, 0.2213) -- (1.3000, 2.0500, 0.2180) -- (1.2500, 2.0500, 0.2166) -- cycle;
\fill[blue!21.5, opacity=0.7] (1.2500, 2.0500, 0.2166) -- (1.3000, 2.0500, 0.2180) -- (1.3000, 2.1000, 0.2145) -- (1.2500, 2.1000, 0.2130) -- cycle;
\fill[blue!62.9, opacity=0.7] (1.2500, 2.1000, 0.2130) -- (1.3000, 2.1000, 0.2145) -- (1.3000, 2.1500, 0.2106) -- (1.2500, 2.1500, 0.2092) -- cycle;
\fill[blue!44.8, opacity=0.7] (1.2500, 2.1500, 0.2092) -- (1.3000, 2.1500, 0.2106) -- (1.3000, 2.2000, 0.2066) -- (1.2500, 2.2000, 0.2051) -- cycle;
\fill[blue!15.3, opacity=0.7] (1.2500, 2.2000, 0.2051) -- (1.3000, 2.2000, 0.2066) -- (1.3000, 2.2500, 0.2022) -- (1.2500, 2.2500, 0.2008) -- cycle;
\fill[blue!15.0, opacity=0.7] (1.2500, 2.2500, 0.2008) -- (1.3000, 2.2500, 0.2022) -- (1.3000, 2.3000, 0.1977) -- (1.2500, 2.3000, 0.1962) -- cycle;
\fill[blue!15.0, opacity=0.7] (1.2500, 2.3000, 0.1962) -- (1.3000, 2.3000, 0.1977) -- (1.3000, 2.3500, 0.1929) -- (1.2500, 2.3500, 0.1914) -- cycle;
\fill[blue!15.0, opacity=0.7] (1.2500, 2.3500, 0.1914) -- (1.3000, 2.3500, 0.1929) -- (1.3000, 2.4000, 0.1879) -- (1.2500, 2.4000, 0.1864) -- cycle;
\fill[blue!15.0, opacity=0.7] (1.2500, 2.4000, 0.1864) -- (1.3000, 2.4000, 0.1879) -- (1.3000, 2.4500, 0.1827) -- (1.2500, 2.4500, 0.1813) -- cycle;
\fill[blue!15.0, opacity=0.7] (1.2500, 2.4500, 0.1813) -- (1.3000, 2.4500, 0.1827) -- (1.3000, 2.5000, 0.1774) -- (1.2500, 2.5000, 0.1759) -- cycle;
\fill[blue!15.0, opacity=0.7] (1.2500, 2.5000, 0.1759) -- (1.3000, 2.5000, 0.1774) -- (1.3000, 2.5500, 0.1719) -- (1.2500, 2.5500, 0.1704) -- cycle;
\fill[blue!15.0, opacity=0.7] (1.2500, 2.5500, 0.1704) -- (1.3000, 2.5500, 0.1719) -- (1.3000, 2.6000, 0.1662) -- (1.2500, 2.6000, 0.1647) -- cycle;
\fill[blue!15.0, opacity=0.7] (1.2500, 2.6000, 0.1647) -- (1.3000, 2.6000, 0.1662) -- (1.3000, 2.6500, 0.1604) -- (1.2500, 2.6500, 0.1589) -- cycle;
\fill[blue!15.0, opacity=0.7] (1.2500, 2.6500, 0.1589) -- (1.3000, 2.6500, 0.1604) -- (1.3000, 2.7000, 0.1545) -- (1.2500, 2.7000, 0.1530) -- cycle;
\fill[blue!15.0, opacity=0.7] (1.2500, 2.7000, 0.1530) -- (1.3000, 2.7000, 0.1545) -- (1.3000, 2.7500, 0.1484) -- (1.2500, 2.7500, 0.1470) -- cycle;
\fill[blue!15.0, opacity=0.7] (1.2500, 2.7500, 0.1470) -- (1.3000, 2.7500, 0.1484) -- (1.3000, 2.8000, 0.1423) -- (1.2500, 2.8000, 0.1409) -- cycle;
\fill[blue!15.0, opacity=0.7] (1.2500, 2.8000, 0.1409) -- (1.3000, 2.8000, 0.1423) -- (1.3000, 2.8500, 0.1361) -- (1.2500, 2.8500, 0.1347) -- cycle;
\fill[blue!15.0, opacity=0.7] (1.2500, 2.8500, 0.1347) -- (1.3000, 2.8500, 0.1361) -- (1.3000, 2.9000, 0.1299) -- (1.2500, 2.9000, 0.1285) -- cycle;
\fill[blue!15.0, opacity=0.7] (1.2500, 2.9000, 0.1285) -- (1.3000, 2.9000, 0.1299) -- (1.3000, 2.9500, 0.1237) -- (1.2500, 2.9500, 0.1222) -- cycle;
\fill[blue!15.0, opacity=0.7] (1.2500, 2.9500, 0.1222) -- (1.3000, 2.9500, 0.1237) -- (1.3000, 3.0000, 0.1174) -- (1.2500, 3.0000, 0.1159) -- cycle;
\fill[blue!15.0, opacity=0.7] (1.3000, 0.0000, 0.1174) -- (1.3500, 0.0000, 0.1185) -- (1.3500, 0.0500, 0.1248) -- (1.3000, 0.0500, 0.1237) -- cycle;
\fill[blue!15.0, opacity=0.7] (1.3000, 0.0500, 0.1237) -- (1.3500, 0.0500, 0.1248) -- (1.3500, 0.1000, 0.1311) -- (1.3000, 0.1000, 0.1299) -- cycle;
\fill[blue!15.0, opacity=0.7] (1.3000, 0.1000, 0.1299) -- (1.3500, 0.1000, 0.1311) -- (1.3500, 0.1500, 0.1373) -- (1.3000, 0.1500, 0.1361) -- cycle;
\fill[blue!15.0, opacity=0.7] (1.3000, 0.1500, 0.1361) -- (1.3500, 0.1500, 0.1373) -- (1.3500, 0.2000, 0.1435) -- (1.3000, 0.2000, 0.1423) -- cycle;
\fill[blue!15.0, opacity=0.7] (1.3000, 0.2000, 0.1423) -- (1.3500, 0.2000, 0.1435) -- (1.3500, 0.2500, 0.1496) -- (1.3000, 0.2500, 0.1484) -- cycle;
\fill[blue!15.0, opacity=0.7] (1.3000, 0.2500, 0.1484) -- (1.3500, 0.2500, 0.1496) -- (1.3500, 0.3000, 0.1556) -- (1.3000, 0.3000, 0.1545) -- cycle;
\fill[blue!15.0, opacity=0.7] (1.3000, 0.3000, 0.1545) -- (1.3500, 0.3000, 0.1556) -- (1.3500, 0.3500, 0.1615) -- (1.3000, 0.3500, 0.1604) -- cycle;
\fill[blue!15.0, opacity=0.7] (1.3000, 0.3500, 0.1604) -- (1.3500, 0.3500, 0.1615) -- (1.3500, 0.4000, 0.1673) -- (1.3000, 0.4000, 0.1662) -- cycle;
\fill[blue!15.0, opacity=0.7] (1.3000, 0.4000, 0.1662) -- (1.3500, 0.4000, 0.1673) -- (1.3500, 0.4500, 0.1730) -- (1.3000, 0.4500, 0.1719) -- cycle;
\fill[blue!15.0, opacity=0.7] (1.3000, 0.4500, 0.1719) -- (1.3500, 0.4500, 0.1730) -- (1.3500, 0.5000, 0.1785) -- (1.3000, 0.5000, 0.1774) -- cycle;
\fill[blue!15.0, opacity=0.7] (1.3000, 0.5000, 0.1774) -- (1.3500, 0.5000, 0.1785) -- (1.3500, 0.5500, 0.1839) -- (1.3000, 0.5500, 0.1827) -- cycle;
\fill[blue!15.0, opacity=0.7] (1.3000, 0.5500, 0.1827) -- (1.3500, 0.5500, 0.1839) -- (1.3500, 0.6000, 0.1891) -- (1.3000, 0.6000, 0.1879) -- cycle;
\fill[blue!15.0, opacity=0.7] (1.3000, 0.6000, 0.1879) -- (1.3500, 0.6000, 0.1891) -- (1.3500, 0.6500, 0.1940) -- (1.3000, 0.6500, 0.1929) -- cycle;
\fill[blue!15.0, opacity=0.7] (1.3000, 0.6500, 0.1929) -- (1.3500, 0.6500, 0.1940) -- (1.3500, 0.7000, 0.1988) -- (1.3000, 0.7000, 0.1977) -- cycle;
\fill[blue!20.6, opacity=0.7] (1.3000, 0.7000, 0.1977) -- (1.3500, 0.7000, 0.1988) -- (1.3500, 0.7500, 0.2034) -- (1.3000, 0.7500, 0.2022) -- cycle;
\fill[blue!35.8, opacity=0.7] (1.3000, 0.7500, 0.2022) -- (1.3500, 0.7500, 0.2034) -- (1.3500, 0.8000, 0.2077) -- (1.3000, 0.8000, 0.2066) -- cycle;
\fill[blue!15.8, opacity=0.7] (1.3000, 0.8000, 0.2066) -- (1.3500, 0.8000, 0.2077) -- (1.3500, 0.8500, 0.2118) -- (1.3000, 0.8500, 0.2106) -- cycle;
\fill[blue!15.0, opacity=0.7] (1.3000, 0.8500, 0.2106) -- (1.3500, 0.8500, 0.2118) -- (1.3500, 0.9000, 0.2156) -- (1.3000, 0.9000, 0.2145) -- cycle;
\fill[blue!15.0, opacity=0.7] (1.3000, 0.9000, 0.2145) -- (1.3500, 0.9000, 0.2156) -- (1.3500, 0.9500, 0.2192) -- (1.3000, 0.9500, 0.2180) -- cycle;
\fill[blue!15.0, opacity=0.7] (1.3000, 0.9500, 0.2180) -- (1.3500, 0.9500, 0.2192) -- (1.3500, 1.0000, 0.2224) -- (1.3000, 1.0000, 0.2213) -- cycle;
\fill[blue!15.0, opacity=0.7] (1.3000, 1.0000, 0.2213) -- (1.3500, 1.0000, 0.2224) -- (1.3500, 1.0500, 0.2254) -- (1.3000, 1.0500, 0.2243) -- cycle;
\fill[blue!18.6, opacity=0.7] (1.3000, 1.0500, 0.2243) -- (1.3500, 1.0500, 0.2254) -- (1.3500, 1.1000, 0.2281) -- (1.3000, 1.1000, 0.2270) -- cycle;
\fill[blue!5.4!black, opacity=0.7] (1.3000, 1.1000, 0.2270) -- (1.3500, 1.1000, 0.2281) -- (1.3500, 1.1500, 0.2306) -- (1.3000, 1.1500, 0.2294) -- cycle;
\fill[blue!97.6, opacity=0.7] (1.3000, 1.1500, 0.2294) -- (1.3500, 1.1500, 0.2306) -- (1.3500, 1.2000, 0.2326) -- (1.3000, 1.2000, 0.2315) -- cycle;
\fill[blue!16.1!black, opacity=0.7] (1.3000, 1.2000, 0.2315) -- (1.3500, 1.2000, 0.2326) -- (1.3500, 1.2500, 0.2344) -- (1.3000, 1.2500, 0.2333) -- cycle;
\fill[blue!18.4, opacity=0.7] (1.3000, 1.2500, 0.2333) -- (1.3500, 1.2500, 0.2344) -- (1.3500, 1.3000, 0.2359) -- (1.3000, 1.3000, 0.2348) -- cycle;
\fill[blue!15.0, opacity=0.7] (1.3000, 1.3000, 0.2348) -- (1.3500, 1.3000, 0.2359) -- (1.3500, 1.3500, 0.2370) -- (1.3000, 1.3500, 0.2359) -- cycle;
\fill[blue!15.0, opacity=0.7] (1.3000, 1.3500, 0.2359) -- (1.3500, 1.3500, 0.2370) -- (1.3500, 1.4000, 0.2379) -- (1.3000, 1.4000, 0.2367) -- cycle;
\fill[blue!15.9, opacity=0.7] (1.3000, 1.4000, 0.2367) -- (1.3500, 1.4000, 0.2379) -- (1.3500, 1.4500, 0.2384) -- (1.3000, 1.4500, 0.2372) -- cycle;
\fill[blue!46.0!black, opacity=0.7] (1.3000, 1.4500, 0.2372) -- (1.3500, 1.4500, 0.2384) -- (1.3500, 1.5000, 0.2385) -- (1.3000, 1.5000, 0.2374) -- cycle;
\fill[blue!72.6, opacity=0.7] (1.3000, 1.5000, 0.2374) -- (1.3500, 1.5000, 0.2385) -- (1.3500, 1.5500, 0.2384) -- (1.3000, 1.5500, 0.2372) -- cycle;
\fill[blue!39.1, opacity=0.7] (1.3000, 1.5500, 0.2372) -- (1.3500, 1.5500, 0.2384) -- (1.3500, 1.6000, 0.2379) -- (1.3000, 1.6000, 0.2367) -- cycle;
\fill[blue!42.9, opacity=0.7] (1.3000, 1.6000, 0.2367) -- (1.3500, 1.6000, 0.2379) -- (1.3500, 1.6500, 0.2370) -- (1.3000, 1.6500, 0.2359) -- cycle;
\fill[blue!72.3, opacity=0.7] (1.3000, 1.6500, 0.2359) -- (1.3500, 1.6500, 0.2370) -- (1.3500, 1.7000, 0.2359) -- (1.3000, 1.7000, 0.2348) -- cycle;
\fill[blue!12.7!black, opacity=0.7] (1.3000, 1.7000, 0.2348) -- (1.3500, 1.7000, 0.2359) -- (1.3500, 1.7500, 0.2344) -- (1.3000, 1.7500, 0.2333) -- cycle;
\fill[blue!60.9, opacity=0.7] (1.3000, 1.7500, 0.2333) -- (1.3500, 1.7500, 0.2344) -- (1.3500, 1.8000, 0.2326) -- (1.3000, 1.8000, 0.2315) -- cycle;
\fill[blue!15.3, opacity=0.7] (1.3000, 1.8000, 0.2315) -- (1.3500, 1.8000, 0.2326) -- (1.3500, 1.8500, 0.2306) -- (1.3000, 1.8500, 0.2294) -- cycle;
\fill[blue!15.0, opacity=0.7] (1.3000, 1.8500, 0.2294) -- (1.3500, 1.8500, 0.2306) -- (1.3500, 1.9000, 0.2281) -- (1.3000, 1.9000, 0.2270) -- cycle;
\fill[blue!15.0, opacity=0.7] (1.3000, 1.9000, 0.2270) -- (1.3500, 1.9000, 0.2281) -- (1.3500, 1.9500, 0.2254) -- (1.3000, 1.9500, 0.2243) -- cycle;
\fill[blue!15.0, opacity=0.7] (1.3000, 1.9500, 0.2243) -- (1.3500, 1.9500, 0.2254) -- (1.3500, 2.0000, 0.2224) -- (1.3000, 2.0000, 0.2213) -- cycle;
\fill[blue!15.0, opacity=0.7] (1.3000, 2.0000, 0.2213) -- (1.3500, 2.0000, 0.2224) -- (1.3500, 2.0500, 0.2192) -- (1.3000, 2.0500, 0.2180) -- cycle;
\fill[blue!15.2, opacity=0.7] (1.3000, 2.0500, 0.2180) -- (1.3500, 2.0500, 0.2192) -- (1.3500, 2.1000, 0.2156) -- (1.3000, 2.1000, 0.2145) -- cycle;
\fill[blue!34.3, opacity=0.7] (1.3000, 2.1000, 0.2145) -- (1.3500, 2.1000, 0.2156) -- (1.3500, 2.1500, 0.2118) -- (1.3000, 2.1500, 0.2106) -- cycle;
\fill[blue!60.8, opacity=0.7] (1.3000, 2.1500, 0.2106) -- (1.3500, 2.1500, 0.2118) -- (1.3500, 2.2000, 0.2077) -- (1.3000, 2.2000, 0.2066) -- cycle;
\fill[blue!20.4, opacity=0.7] (1.3000, 2.2000, 0.2066) -- (1.3500, 2.2000, 0.2077) -- (1.3500, 2.2500, 0.2034) -- (1.3000, 2.2500, 0.2022) -- cycle;
\fill[blue!15.0, opacity=0.7] (1.3000, 2.2500, 0.2022) -- (1.3500, 2.2500, 0.2034) -- (1.3500, 2.3000, 0.1988) -- (1.3000, 2.3000, 0.1977) -- cycle;
\fill[blue!15.0, opacity=0.7] (1.3000, 2.3000, 0.1977) -- (1.3500, 2.3000, 0.1988) -- (1.3500, 2.3500, 0.1940) -- (1.3000, 2.3500, 0.1929) -- cycle;
\fill[blue!15.0, opacity=0.7] (1.3000, 2.3500, 0.1929) -- (1.3500, 2.3500, 0.1940) -- (1.3500, 2.4000, 0.1891) -- (1.3000, 2.4000, 0.1879) -- cycle;
\fill[blue!15.0, opacity=0.7] (1.3000, 2.4000, 0.1879) -- (1.3500, 2.4000, 0.1891) -- (1.3500, 2.4500, 0.1839) -- (1.3000, 2.4500, 0.1827) -- cycle;
\fill[blue!15.0, opacity=0.7] (1.3000, 2.4500, 0.1827) -- (1.3500, 2.4500, 0.1839) -- (1.3500, 2.5000, 0.1785) -- (1.3000, 2.5000, 0.1774) -- cycle;
\fill[blue!15.0, opacity=0.7] (1.3000, 2.5000, 0.1774) -- (1.3500, 2.5000, 0.1785) -- (1.3500, 2.5500, 0.1730) -- (1.3000, 2.5500, 0.1719) -- cycle;
\fill[blue!15.0, opacity=0.7] (1.3000, 2.5500, 0.1719) -- (1.3500, 2.5500, 0.1730) -- (1.3500, 2.6000, 0.1673) -- (1.3000, 2.6000, 0.1662) -- cycle;
\fill[blue!15.0, opacity=0.7] (1.3000, 2.6000, 0.1662) -- (1.3500, 2.6000, 0.1673) -- (1.3500, 2.6500, 0.1615) -- (1.3000, 2.6500, 0.1604) -- cycle;
\fill[blue!15.0, opacity=0.7] (1.3000, 2.6500, 0.1604) -- (1.3500, 2.6500, 0.1615) -- (1.3500, 2.7000, 0.1556) -- (1.3000, 2.7000, 0.1545) -- cycle;
\fill[blue!15.0, opacity=0.7] (1.3000, 2.7000, 0.1545) -- (1.3500, 2.7000, 0.1556) -- (1.3500, 2.7500, 0.1496) -- (1.3000, 2.7500, 0.1484) -- cycle;
\fill[blue!15.0, opacity=0.7] (1.3000, 2.7500, 0.1484) -- (1.3500, 2.7500, 0.1496) -- (1.3500, 2.8000, 0.1435) -- (1.3000, 2.8000, 0.1423) -- cycle;
\fill[blue!15.0, opacity=0.7] (1.3000, 2.8000, 0.1423) -- (1.3500, 2.8000, 0.1435) -- (1.3500, 2.8500, 0.1373) -- (1.3000, 2.8500, 0.1361) -- cycle;
\fill[blue!15.0, opacity=0.7] (1.3000, 2.8500, 0.1361) -- (1.3500, 2.8500, 0.1373) -- (1.3500, 2.9000, 0.1311) -- (1.3000, 2.9000, 0.1299) -- cycle;
\fill[blue!15.0, opacity=0.7] (1.3000, 2.9000, 0.1299) -- (1.3500, 2.9000, 0.1311) -- (1.3500, 2.9500, 0.1248) -- (1.3000, 2.9500, 0.1237) -- cycle;
\fill[blue!15.0, opacity=0.7] (1.3000, 2.9500, 0.1237) -- (1.3500, 2.9500, 0.1248) -- (1.3500, 3.0000, 0.1185) -- (1.3000, 3.0000, 0.1174) -- cycle;
\fill[blue!15.0, opacity=0.7] (1.3500, 0.0000, 0.1185) -- (1.4000, 0.0000, 0.1193) -- (1.4000, 0.0500, 0.1256) -- (1.3500, 0.0500, 0.1248) -- cycle;
\fill[blue!15.0, opacity=0.7] (1.3500, 0.0500, 0.1248) -- (1.4000, 0.0500, 0.1256) -- (1.4000, 0.1000, 0.1319) -- (1.3500, 0.1000, 0.1311) -- cycle;
\fill[blue!15.0, opacity=0.7] (1.3500, 0.1000, 0.1311) -- (1.4000, 0.1000, 0.1319) -- (1.4000, 0.1500, 0.1381) -- (1.3500, 0.1500, 0.1373) -- cycle;
\fill[blue!15.0, opacity=0.7] (1.3500, 0.1500, 0.1373) -- (1.4000, 0.1500, 0.1381) -- (1.4000, 0.2000, 0.1443) -- (1.3500, 0.2000, 0.1435) -- cycle;
\fill[blue!15.0, opacity=0.7] (1.3500, 0.2000, 0.1435) -- (1.4000, 0.2000, 0.1443) -- (1.4000, 0.2500, 0.1504) -- (1.3500, 0.2500, 0.1496) -- cycle;
\fill[blue!15.0, opacity=0.7] (1.3500, 0.2500, 0.1496) -- (1.4000, 0.2500, 0.1504) -- (1.4000, 0.3000, 0.1564) -- (1.3500, 0.3000, 0.1556) -- cycle;
\fill[blue!15.0, opacity=0.7] (1.3500, 0.3000, 0.1556) -- (1.4000, 0.3000, 0.1564) -- (1.4000, 0.3500, 0.1623) -- (1.3500, 0.3500, 0.1615) -- cycle;
\fill[blue!15.0, opacity=0.7] (1.3500, 0.3500, 0.1615) -- (1.4000, 0.3500, 0.1623) -- (1.4000, 0.4000, 0.1682) -- (1.3500, 0.4000, 0.1673) -- cycle;
\fill[blue!15.0, opacity=0.7] (1.3500, 0.4000, 0.1673) -- (1.4000, 0.4000, 0.1682) -- (1.4000, 0.4500, 0.1738) -- (1.3500, 0.4500, 0.1730) -- cycle;
\fill[blue!15.0, opacity=0.7] (1.3500, 0.4500, 0.1730) -- (1.4000, 0.4500, 0.1738) -- (1.4000, 0.5000, 0.1793) -- (1.3500, 0.5000, 0.1785) -- cycle;
\fill[blue!15.0, opacity=0.7] (1.3500, 0.5000, 0.1785) -- (1.4000, 0.5000, 0.1793) -- (1.4000, 0.5500, 0.1847) -- (1.3500, 0.5500, 0.1839) -- cycle;
\fill[blue!15.0, opacity=0.7] (1.3500, 0.5500, 0.1839) -- (1.4000, 0.5500, 0.1847) -- (1.4000, 0.6000, 0.1899) -- (1.3500, 0.6000, 0.1891) -- cycle;
\fill[blue!15.0, opacity=0.7] (1.3500, 0.6000, 0.1891) -- (1.4000, 0.6000, 0.1899) -- (1.4000, 0.6500, 0.1949) -- (1.3500, 0.6500, 0.1940) -- cycle;
\fill[blue!15.0, opacity=0.7] (1.3500, 0.6500, 0.1940) -- (1.4000, 0.6500, 0.1949) -- (1.4000, 0.7000, 0.1996) -- (1.3500, 0.7000, 0.1988) -- cycle;
\fill[blue!22.1, opacity=0.7] (1.3500, 0.7000, 0.1988) -- (1.4000, 0.7000, 0.1996) -- (1.4000, 0.7500, 0.2042) -- (1.3500, 0.7500, 0.2034) -- cycle;
\fill[blue!37.9, opacity=0.7] (1.3500, 0.7500, 0.2034) -- (1.4000, 0.7500, 0.2042) -- (1.4000, 0.8000, 0.2085) -- (1.3500, 0.8000, 0.2077) -- cycle;
\fill[blue!15.9, opacity=0.7] (1.3500, 0.8000, 0.2077) -- (1.4000, 0.8000, 0.2085) -- (1.4000, 0.8500, 0.2126) -- (1.3500, 0.8500, 0.2118) -- cycle;
\fill[blue!15.0, opacity=0.7] (1.3500, 0.8500, 0.2118) -- (1.4000, 0.8500, 0.2126) -- (1.4000, 0.9000, 0.2164) -- (1.3500, 0.9000, 0.2156) -- cycle;
\fill[blue!15.0, opacity=0.7] (1.3500, 0.9000, 0.2156) -- (1.4000, 0.9000, 0.2164) -- (1.4000, 0.9500, 0.2200) -- (1.3500, 0.9500, 0.2192) -- cycle;
\fill[blue!15.0, opacity=0.7] (1.3500, 0.9500, 0.2192) -- (1.4000, 0.9500, 0.2200) -- (1.4000, 1.0000, 0.2233) -- (1.3500, 1.0000, 0.2224) -- cycle;
\fill[blue!15.0, opacity=0.7] (1.3500, 1.0000, 0.2224) -- (1.4000, 1.0000, 0.2233) -- (1.4000, 1.0500, 0.2263) -- (1.3500, 1.0500, 0.2254) -- cycle;
\fill[blue!21.9, opacity=0.7] (1.3500, 1.0500, 0.2254) -- (1.4000, 1.0500, 0.2263) -- (1.4000, 1.1000, 0.2290) -- (1.3500, 1.1000, 0.2281) -- cycle;
\fill[blue!9.8!black, opacity=0.7] (1.3500, 1.1000, 0.2281) -- (1.4000, 1.1000, 0.2290) -- (1.4000, 1.1500, 0.2314) -- (1.3500, 1.1500, 0.2306) -- cycle;
\fill[blue!89.7, opacity=0.7] (1.3500, 1.1500, 0.2306) -- (1.4000, 1.1500, 0.2314) -- (1.4000, 1.2000, 0.2335) -- (1.3500, 1.2000, 0.2326) -- cycle;
\fill[blue!21.7!black, opacity=0.7] (1.3500, 1.2000, 0.2326) -- (1.4000, 1.2000, 0.2335) -- (1.4000, 1.2500, 0.2353) -- (1.3500, 1.2500, 0.2344) -- cycle;
\fill[blue!17.1, opacity=0.7] (1.3500, 1.2500, 0.2344) -- (1.4000, 1.2500, 0.2353) -- (1.4000, 1.3000, 0.2367) -- (1.3500, 1.3000, 0.2359) -- cycle;
\fill[blue!15.0, opacity=0.7] (1.3500, 1.3000, 0.2359) -- (1.4000, 1.3000, 0.2367) -- (1.4000, 1.3500, 0.2379) -- (1.3500, 1.3500, 0.2370) -- cycle;
\fill[blue!15.0, opacity=0.7] (1.3500, 1.3500, 0.2370) -- (1.4000, 1.3500, 0.2379) -- (1.4000, 1.4000, 0.2387) -- (1.3500, 1.4000, 0.2379) -- cycle;
\fill[blue!55.9, opacity=0.7] (1.3500, 1.4000, 0.2379) -- (1.4000, 1.4000, 0.2387) -- (1.4000, 1.4500, 0.2392) -- (1.3500, 1.4500, 0.2384) -- cycle;
\fill[blue!42.7, opacity=0.7] (1.3500, 1.4500, 0.2384) -- (1.4000, 1.4500, 0.2392) -- (1.4000, 1.5000, 0.2393) -- (1.3500, 1.5000, 0.2385) -- cycle;
\fill[blue!18.8, opacity=0.7] (1.3500, 1.5000, 0.2385) -- (1.4000, 1.5000, 0.2393) -- (1.4000, 1.5500, 0.2392) -- (1.3500, 1.5500, 0.2384) -- cycle;
\fill[blue!27.6, opacity=0.7] (1.3500, 1.5500, 0.2384) -- (1.4000, 1.5500, 0.2392) -- (1.4000, 1.6000, 0.2387) -- (1.3500, 1.6000, 0.2379) -- cycle;
\fill[blue!36.1, opacity=0.7] (1.3500, 1.6000, 0.2379) -- (1.4000, 1.6000, 0.2387) -- (1.4000, 1.6500, 0.2379) -- (1.3500, 1.6500, 0.2370) -- cycle;
\fill[blue!34.7, opacity=0.7] (1.3500, 1.6500, 0.2370) -- (1.4000, 1.6500, 0.2379) -- (1.4000, 1.7000, 0.2367) -- (1.3500, 1.7000, 0.2359) -- cycle;
\fill[blue!45.2, opacity=0.7] (1.3500, 1.7000, 0.2359) -- (1.4000, 1.7000, 0.2367) -- (1.4000, 1.7500, 0.2353) -- (1.3500, 1.7500, 0.2344) -- cycle;
\fill[blue!62.1!black, opacity=0.7] (1.3500, 1.7500, 0.2344) -- (1.4000, 1.7500, 0.2353) -- (1.4000, 1.8000, 0.2335) -- (1.3500, 1.8000, 0.2326) -- cycle;
\fill[blue!65.1, opacity=0.7] (1.3500, 1.8000, 0.2326) -- (1.4000, 1.8000, 0.2335) -- (1.4000, 1.8500, 0.2314) -- (1.3500, 1.8500, 0.2306) -- cycle;
\fill[blue!15.1, opacity=0.7] (1.3500, 1.8500, 0.2306) -- (1.4000, 1.8500, 0.2314) -- (1.4000, 1.9000, 0.2290) -- (1.3500, 1.9000, 0.2281) -- cycle;
\fill[blue!15.0, opacity=0.7] (1.3500, 1.9000, 0.2281) -- (1.4000, 1.9000, 0.2290) -- (1.4000, 1.9500, 0.2263) -- (1.3500, 1.9500, 0.2254) -- cycle;
\fill[blue!15.0, opacity=0.7] (1.3500, 1.9500, 0.2254) -- (1.4000, 1.9500, 0.2263) -- (1.4000, 2.0000, 0.2233) -- (1.3500, 2.0000, 0.2224) -- cycle;
\fill[blue!15.0, opacity=0.7] (1.3500, 2.0000, 0.2224) -- (1.4000, 2.0000, 0.2233) -- (1.4000, 2.0500, 0.2200) -- (1.3500, 2.0500, 0.2192) -- cycle;
\fill[blue!15.0, opacity=0.7] (1.3500, 2.0500, 0.2192) -- (1.4000, 2.0500, 0.2200) -- (1.4000, 2.1000, 0.2164) -- (1.3500, 2.1000, 0.2156) -- cycle;
\fill[blue!18.0, opacity=0.7] (1.3500, 2.1000, 0.2156) -- (1.4000, 2.1000, 0.2164) -- (1.4000, 2.1500, 0.2126) -- (1.3500, 2.1500, 0.2118) -- cycle;
\fill[blue!53.7, opacity=0.7] (1.3500, 2.1500, 0.2118) -- (1.4000, 2.1500, 0.2126) -- (1.4000, 2.2000, 0.2085) -- (1.3500, 2.2000, 0.2077) -- cycle;
\fill[blue!34.2, opacity=0.7] (1.3500, 2.2000, 0.2077) -- (1.4000, 2.2000, 0.2085) -- (1.4000, 2.2500, 0.2042) -- (1.3500, 2.2500, 0.2034) -- cycle;
\fill[blue!15.1, opacity=0.7] (1.3500, 2.2500, 0.2034) -- (1.4000, 2.2500, 0.2042) -- (1.4000, 2.3000, 0.1996) -- (1.3500, 2.3000, 0.1988) -- cycle;
\fill[blue!15.0, opacity=0.7] (1.3500, 2.3000, 0.1988) -- (1.4000, 2.3000, 0.1996) -- (1.4000, 2.3500, 0.1949) -- (1.3500, 2.3500, 0.1940) -- cycle;
\fill[blue!15.0, opacity=0.7] (1.3500, 2.3500, 0.1940) -- (1.4000, 2.3500, 0.1949) -- (1.4000, 2.4000, 0.1899) -- (1.3500, 2.4000, 0.1891) -- cycle;
\fill[blue!15.0, opacity=0.7] (1.3500, 2.4000, 0.1891) -- (1.4000, 2.4000, 0.1899) -- (1.4000, 2.4500, 0.1847) -- (1.3500, 2.4500, 0.1839) -- cycle;
\fill[blue!15.0, opacity=0.7] (1.3500, 2.4500, 0.1839) -- (1.4000, 2.4500, 0.1847) -- (1.4000, 2.5000, 0.1793) -- (1.3500, 2.5000, 0.1785) -- cycle;
\fill[blue!15.0, opacity=0.7] (1.3500, 2.5000, 0.1785) -- (1.4000, 2.5000, 0.1793) -- (1.4000, 2.5500, 0.1738) -- (1.3500, 2.5500, 0.1730) -- cycle;
\fill[blue!15.0, opacity=0.7] (1.3500, 2.5500, 0.1730) -- (1.4000, 2.5500, 0.1738) -- (1.4000, 2.6000, 0.1682) -- (1.3500, 2.6000, 0.1673) -- cycle;
\fill[blue!15.0, opacity=0.7] (1.3500, 2.6000, 0.1673) -- (1.4000, 2.6000, 0.1682) -- (1.4000, 2.6500, 0.1623) -- (1.3500, 2.6500, 0.1615) -- cycle;
\fill[blue!15.0, opacity=0.7] (1.3500, 2.6500, 0.1615) -- (1.4000, 2.6500, 0.1623) -- (1.4000, 2.7000, 0.1564) -- (1.3500, 2.7000, 0.1556) -- cycle;
\fill[blue!15.0, opacity=0.7] (1.3500, 2.7000, 0.1556) -- (1.4000, 2.7000, 0.1564) -- (1.4000, 2.7500, 0.1504) -- (1.3500, 2.7500, 0.1496) -- cycle;
\fill[blue!15.0, opacity=0.7] (1.3500, 2.7500, 0.1496) -- (1.4000, 2.7500, 0.1504) -- (1.4000, 2.8000, 0.1443) -- (1.3500, 2.8000, 0.1435) -- cycle;
\fill[blue!15.0, opacity=0.7] (1.3500, 2.8000, 0.1435) -- (1.4000, 2.8000, 0.1443) -- (1.4000, 2.8500, 0.1381) -- (1.3500, 2.8500, 0.1373) -- cycle;
\fill[blue!15.0, opacity=0.7] (1.3500, 2.8500, 0.1373) -- (1.4000, 2.8500, 0.1381) -- (1.4000, 2.9000, 0.1319) -- (1.3500, 2.9000, 0.1311) -- cycle;
\fill[blue!15.0, opacity=0.7] (1.3500, 2.9000, 0.1311) -- (1.4000, 2.9000, 0.1319) -- (1.4000, 2.9500, 0.1256) -- (1.3500, 2.9500, 0.1248) -- cycle;
\fill[blue!15.0, opacity=0.7] (1.3500, 2.9500, 0.1248) -- (1.4000, 2.9500, 0.1256) -- (1.4000, 3.0000, 0.1193) -- (1.3500, 3.0000, 0.1185) -- cycle;
\fill[blue!15.0, opacity=0.7] (1.4000, 0.0000, 0.1193) -- (1.4500, 0.0000, 0.1198) -- (1.4500, 0.0500, 0.1261) -- (1.4000, 0.0500, 0.1256) -- cycle;
\fill[blue!15.0, opacity=0.7] (1.4000, 0.0500, 0.1256) -- (1.4500, 0.0500, 0.1261) -- (1.4500, 0.1000, 0.1324) -- (1.4000, 0.1000, 0.1319) -- cycle;
\fill[blue!15.0, opacity=0.7] (1.4000, 0.1000, 0.1319) -- (1.4500, 0.1000, 0.1324) -- (1.4500, 0.1500, 0.1386) -- (1.4000, 0.1500, 0.1381) -- cycle;
\fill[blue!15.0, opacity=0.7] (1.4000, 0.1500, 0.1381) -- (1.4500, 0.1500, 0.1386) -- (1.4500, 0.2000, 0.1448) -- (1.4000, 0.2000, 0.1443) -- cycle;
\fill[blue!15.0, opacity=0.7] (1.4000, 0.2000, 0.1443) -- (1.4500, 0.2000, 0.1448) -- (1.4500, 0.2500, 0.1509) -- (1.4000, 0.2500, 0.1504) -- cycle;
\fill[blue!15.0, opacity=0.7] (1.4000, 0.2500, 0.1504) -- (1.4500, 0.2500, 0.1509) -- (1.4500, 0.3000, 0.1569) -- (1.4000, 0.3000, 0.1564) -- cycle;
\fill[blue!15.0, opacity=0.7] (1.4000, 0.3000, 0.1564) -- (1.4500, 0.3000, 0.1569) -- (1.4500, 0.3500, 0.1628) -- (1.4000, 0.3500, 0.1623) -- cycle;
\fill[blue!15.0, opacity=0.7] (1.4000, 0.3500, 0.1623) -- (1.4500, 0.3500, 0.1628) -- (1.4500, 0.4000, 0.1686) -- (1.4000, 0.4000, 0.1682) -- cycle;
\fill[blue!15.0, opacity=0.7] (1.4000, 0.4000, 0.1682) -- (1.4500, 0.4000, 0.1686) -- (1.4500, 0.4500, 0.1743) -- (1.4000, 0.4500, 0.1738) -- cycle;
\fill[blue!15.0, opacity=0.7] (1.4000, 0.4500, 0.1738) -- (1.4500, 0.4500, 0.1743) -- (1.4500, 0.5000, 0.1798) -- (1.4000, 0.5000, 0.1793) -- cycle;
\fill[blue!15.0, opacity=0.7] (1.4000, 0.5000, 0.1793) -- (1.4500, 0.5000, 0.1798) -- (1.4500, 0.5500, 0.1852) -- (1.4000, 0.5500, 0.1847) -- cycle;
\fill[blue!15.0, opacity=0.7] (1.4000, 0.5500, 0.1847) -- (1.4500, 0.5500, 0.1852) -- (1.4500, 0.6000, 0.1904) -- (1.4000, 0.6000, 0.1899) -- cycle;
\fill[blue!15.0, opacity=0.7] (1.4000, 0.6000, 0.1899) -- (1.4500, 0.6000, 0.1904) -- (1.4500, 0.6500, 0.1954) -- (1.4000, 0.6500, 0.1949) -- cycle;
\fill[blue!15.0, opacity=0.7] (1.4000, 0.6500, 0.1949) -- (1.4500, 0.6500, 0.1954) -- (1.4500, 0.7000, 0.2001) -- (1.4000, 0.7000, 0.1996) -- cycle;
\fill[blue!22.0, opacity=0.7] (1.4000, 0.7000, 0.1996) -- (1.4500, 0.7000, 0.2001) -- (1.4500, 0.7500, 0.2047) -- (1.4000, 0.7500, 0.2042) -- cycle;
\fill[blue!41.5, opacity=0.7] (1.4000, 0.7500, 0.2042) -- (1.4500, 0.7500, 0.2047) -- (1.4500, 0.8000, 0.2090) -- (1.4000, 0.8000, 0.2085) -- cycle;
\fill[blue!16.4, opacity=0.7] (1.4000, 0.8000, 0.2085) -- (1.4500, 0.8000, 0.2090) -- (1.4500, 0.8500, 0.2131) -- (1.4000, 0.8500, 0.2126) -- cycle;
\fill[blue!15.0, opacity=0.7] (1.4000, 0.8500, 0.2126) -- (1.4500, 0.8500, 0.2131) -- (1.4500, 0.9000, 0.2169) -- (1.4000, 0.9000, 0.2164) -- cycle;
\fill[blue!15.0, opacity=0.7] (1.4000, 0.9000, 0.2164) -- (1.4500, 0.9000, 0.2169) -- (1.4500, 0.9500, 0.2205) -- (1.4000, 0.9500, 0.2200) -- cycle;
\fill[blue!15.0, opacity=0.7] (1.4000, 0.9500, 0.2200) -- (1.4500, 0.9500, 0.2205) -- (1.4500, 1.0000, 0.2238) -- (1.4000, 1.0000, 0.2233) -- cycle;
\fill[blue!15.0, opacity=0.7] (1.4000, 1.0000, 0.2233) -- (1.4500, 1.0000, 0.2238) -- (1.4500, 1.0500, 0.2268) -- (1.4000, 1.0500, 0.2263) -- cycle;
\fill[blue!20.4, opacity=0.7] (1.4000, 1.0500, 0.2263) -- (1.4500, 1.0500, 0.2268) -- (1.4500, 1.1000, 0.2295) -- (1.4000, 1.1000, 0.2290) -- cycle;
\fill[blue!8.7!black, opacity=0.7] (1.4000, 1.1000, 0.2290) -- (1.4500, 1.1000, 0.2295) -- (1.4500, 1.1500, 0.2319) -- (1.4000, 1.1500, 0.2314) -- cycle;
\fill[blue!79.1, opacity=0.7] (1.4000, 1.1500, 0.2314) -- (1.4500, 1.1500, 0.2319) -- (1.4500, 1.2000, 0.2340) -- (1.4000, 1.2000, 0.2335) -- cycle;
\fill[blue!5.0!black, opacity=0.7] (1.4000, 1.2000, 0.2335) -- (1.4500, 1.2000, 0.2340) -- (1.4500, 1.2500, 0.2357) -- (1.4000, 1.2500, 0.2353) -- cycle;
\fill[blue!20.7, opacity=0.7] (1.4000, 1.2500, 0.2353) -- (1.4500, 1.2500, 0.2357) -- (1.4500, 1.3000, 0.2372) -- (1.4000, 1.3000, 0.2367) -- cycle;
\fill[blue!15.0, opacity=0.7] (1.4000, 1.3000, 0.2367) -- (1.4500, 1.3000, 0.2372) -- (1.4500, 1.3500, 0.2384) -- (1.4000, 1.3500, 0.2379) -- cycle;
\fill[blue!15.0, opacity=0.7] (1.4000, 1.3500, 0.2379) -- (1.4500, 1.3500, 0.2384) -- (1.4500, 1.4000, 0.2392) -- (1.4000, 1.4000, 0.2387) -- cycle;
\fill[blue!95.7!black, opacity=0.7] (1.4000, 1.4000, 0.2387) -- (1.4500, 1.4000, 0.2392) -- (1.4500, 1.4500, 0.2397) -- (1.4000, 1.4500, 0.2392) -- cycle;
\fill[blue!17.0, opacity=0.7] (1.4000, 1.4500, 0.2392) -- (1.4500, 1.4500, 0.2397) -- (1.4500, 1.5000, 0.2398) -- (1.4000, 1.5000, 0.2393) -- cycle;
\fill[blue!69.5, opacity=0.7] (1.4000, 1.5000, 0.2393) -- (1.4500, 1.5000, 0.2398) -- (1.4500, 1.5500, 0.2397) -- (1.4000, 1.5500, 0.2392) -- cycle;
\fill[blue!70.1!black, opacity=0.7] (1.4000, 1.5500, 0.2392) -- (1.4500, 1.5500, 0.2397) -- (1.4500, 1.6000, 0.2392) -- (1.4000, 1.6000, 0.2387) -- cycle;
\fill[blue!55.1!black, opacity=0.7] (1.4000, 1.6000, 0.2387) -- (1.4500, 1.6000, 0.2392) -- (1.4500, 1.6500, 0.2384) -- (1.4000, 1.6500, 0.2379) -- cycle;
\fill[blue!32.1!black, opacity=0.7] (1.4000, 1.6500, 0.2379) -- (1.4500, 1.6500, 0.2384) -- (1.4500, 1.7000, 0.2372) -- (1.4000, 1.7000, 0.2367) -- cycle;
\fill[blue!60.2, opacity=0.7] (1.4000, 1.7000, 0.2367) -- (1.4500, 1.7000, 0.2372) -- (1.4500, 1.7500, 0.2357) -- (1.4000, 1.7500, 0.2353) -- cycle;
\fill[blue!54.2, opacity=0.7] (1.4000, 1.7500, 0.2353) -- (1.4500, 1.7500, 0.2357) -- (1.4500, 1.8000, 0.2340) -- (1.4000, 1.8000, 0.2335) -- cycle;
\fill[blue!14.1!black, opacity=0.7] (1.4000, 1.8000, 0.2335) -- (1.4500, 1.8000, 0.2340) -- (1.4500, 1.8500, 0.2319) -- (1.4000, 1.8500, 0.2314) -- cycle;
\fill[blue!26.6, opacity=0.7] (1.4000, 1.8500, 0.2314) -- (1.4500, 1.8500, 0.2319) -- (1.4500, 1.9000, 0.2295) -- (1.4000, 1.9000, 0.2290) -- cycle;
\fill[blue!15.0, opacity=0.7] (1.4000, 1.9000, 0.2290) -- (1.4500, 1.9000, 0.2295) -- (1.4500, 1.9500, 0.2268) -- (1.4000, 1.9500, 0.2263) -- cycle;
\fill[blue!15.0, opacity=0.7] (1.4000, 1.9500, 0.2263) -- (1.4500, 1.9500, 0.2268) -- (1.4500, 2.0000, 0.2238) -- (1.4000, 2.0000, 0.2233) -- cycle;
\fill[blue!15.0, opacity=0.7] (1.4000, 2.0000, 0.2233) -- (1.4500, 2.0000, 0.2238) -- (1.4500, 2.0500, 0.2205) -- (1.4000, 2.0500, 0.2200) -- cycle;
\fill[blue!15.0, opacity=0.7] (1.4000, 2.0500, 0.2200) -- (1.4500, 2.0500, 0.2205) -- (1.4500, 2.1000, 0.2169) -- (1.4000, 2.1000, 0.2164) -- cycle;
\fill[blue!15.2, opacity=0.7] (1.4000, 2.1000, 0.2164) -- (1.4500, 2.1000, 0.2169) -- (1.4500, 2.1500, 0.2131) -- (1.4000, 2.1500, 0.2126) -- cycle;
\fill[blue!36.6, opacity=0.7] (1.4000, 2.1500, 0.2126) -- (1.4500, 2.1500, 0.2131) -- (1.4500, 2.2000, 0.2090) -- (1.4000, 2.2000, 0.2085) -- cycle;
\fill[blue!45.9, opacity=0.7] (1.4000, 2.2000, 0.2085) -- (1.4500, 2.2000, 0.2090) -- (1.4500, 2.2500, 0.2047) -- (1.4000, 2.2500, 0.2042) -- cycle;
\fill[blue!15.6, opacity=0.7] (1.4000, 2.2500, 0.2042) -- (1.4500, 2.2500, 0.2047) -- (1.4500, 2.3000, 0.2001) -- (1.4000, 2.3000, 0.1996) -- cycle;
\fill[blue!15.0, opacity=0.7] (1.4000, 2.3000, 0.1996) -- (1.4500, 2.3000, 0.2001) -- (1.4500, 2.3500, 0.1954) -- (1.4000, 2.3500, 0.1949) -- cycle;
\fill[blue!15.0, opacity=0.7] (1.4000, 2.3500, 0.1949) -- (1.4500, 2.3500, 0.1954) -- (1.4500, 2.4000, 0.1904) -- (1.4000, 2.4000, 0.1899) -- cycle;
\fill[blue!15.0, opacity=0.7] (1.4000, 2.4000, 0.1899) -- (1.4500, 2.4000, 0.1904) -- (1.4500, 2.4500, 0.1852) -- (1.4000, 2.4500, 0.1847) -- cycle;
\fill[blue!15.0, opacity=0.7] (1.4000, 2.4500, 0.1847) -- (1.4500, 2.4500, 0.1852) -- (1.4500, 2.5000, 0.1798) -- (1.4000, 2.5000, 0.1793) -- cycle;
\fill[blue!15.0, opacity=0.7] (1.4000, 2.5000, 0.1793) -- (1.4500, 2.5000, 0.1798) -- (1.4500, 2.5500, 0.1743) -- (1.4000, 2.5500, 0.1738) -- cycle;
\fill[blue!15.0, opacity=0.7] (1.4000, 2.5500, 0.1738) -- (1.4500, 2.5500, 0.1743) -- (1.4500, 2.6000, 0.1686) -- (1.4000, 2.6000, 0.1682) -- cycle;
\fill[blue!15.0, opacity=0.7] (1.4000, 2.6000, 0.1682) -- (1.4500, 2.6000, 0.1686) -- (1.4500, 2.6500, 0.1628) -- (1.4000, 2.6500, 0.1623) -- cycle;
\fill[blue!15.0, opacity=0.7] (1.4000, 2.6500, 0.1623) -- (1.4500, 2.6500, 0.1628) -- (1.4500, 2.7000, 0.1569) -- (1.4000, 2.7000, 0.1564) -- cycle;
\fill[blue!15.0, opacity=0.7] (1.4000, 2.7000, 0.1564) -- (1.4500, 2.7000, 0.1569) -- (1.4500, 2.7500, 0.1509) -- (1.4000, 2.7500, 0.1504) -- cycle;
\fill[blue!15.0, opacity=0.7] (1.4000, 2.7500, 0.1504) -- (1.4500, 2.7500, 0.1509) -- (1.4500, 2.8000, 0.1448) -- (1.4000, 2.8000, 0.1443) -- cycle;
\fill[blue!15.0, opacity=0.7] (1.4000, 2.8000, 0.1443) -- (1.4500, 2.8000, 0.1448) -- (1.4500, 2.8500, 0.1386) -- (1.4000, 2.8500, 0.1381) -- cycle;
\fill[blue!15.0, opacity=0.7] (1.4000, 2.8500, 0.1381) -- (1.4500, 2.8500, 0.1386) -- (1.4500, 2.9000, 0.1324) -- (1.4000, 2.9000, 0.1319) -- cycle;
\fill[blue!15.0, opacity=0.7] (1.4000, 2.9000, 0.1319) -- (1.4500, 2.9000, 0.1324) -- (1.4500, 2.9500, 0.1261) -- (1.4000, 2.9500, 0.1256) -- cycle;
\fill[blue!15.0, opacity=0.7] (1.4000, 2.9500, 0.1256) -- (1.4500, 2.9500, 0.1261) -- (1.4500, 3.0000, 0.1198) -- (1.4000, 3.0000, 0.1193) -- cycle;
\fill[blue!15.0, opacity=0.7] (1.4500, 0.0000, 0.1198) -- (1.5000, 0.0000, 0.1200) -- (1.5000, 0.0500, 0.1263) -- (1.4500, 0.0500, 0.1261) -- cycle;
\fill[blue!15.0, opacity=0.7] (1.4500, 0.0500, 0.1261) -- (1.5000, 0.0500, 0.1263) -- (1.5000, 0.1000, 0.1325) -- (1.4500, 0.1000, 0.1324) -- cycle;
\fill[blue!15.0, opacity=0.7] (1.4500, 0.1000, 0.1324) -- (1.5000, 0.1000, 0.1325) -- (1.5000, 0.1500, 0.1388) -- (1.4500, 0.1500, 0.1386) -- cycle;
\fill[blue!15.0, opacity=0.7] (1.4500, 0.1500, 0.1386) -- (1.5000, 0.1500, 0.1388) -- (1.5000, 0.2000, 0.1449) -- (1.4500, 0.2000, 0.1448) -- cycle;
\fill[blue!15.0, opacity=0.7] (1.4500, 0.2000, 0.1448) -- (1.5000, 0.2000, 0.1449) -- (1.5000, 0.2500, 0.1511) -- (1.4500, 0.2500, 0.1509) -- cycle;
\fill[blue!15.0, opacity=0.7] (1.4500, 0.2500, 0.1509) -- (1.5000, 0.2500, 0.1511) -- (1.5000, 0.3000, 0.1571) -- (1.4500, 0.3000, 0.1569) -- cycle;
\fill[blue!15.0, opacity=0.7] (1.4500, 0.3000, 0.1569) -- (1.5000, 0.3000, 0.1571) -- (1.5000, 0.3500, 0.1630) -- (1.4500, 0.3500, 0.1628) -- cycle;
\fill[blue!15.0, opacity=0.7] (1.4500, 0.3500, 0.1628) -- (1.5000, 0.3500, 0.1630) -- (1.5000, 0.4000, 0.1688) -- (1.4500, 0.4000, 0.1686) -- cycle;
\fill[blue!15.0, opacity=0.7] (1.4500, 0.4000, 0.1686) -- (1.5000, 0.4000, 0.1688) -- (1.5000, 0.4500, 0.1745) -- (1.4500, 0.4500, 0.1743) -- cycle;
\fill[blue!15.0, opacity=0.7] (1.4500, 0.4500, 0.1743) -- (1.5000, 0.4500, 0.1745) -- (1.5000, 0.5000, 0.1800) -- (1.4500, 0.5000, 0.1798) -- cycle;
\fill[blue!15.0, opacity=0.7] (1.4500, 0.5000, 0.1798) -- (1.5000, 0.5000, 0.1800) -- (1.5000, 0.5500, 0.1854) -- (1.4500, 0.5500, 0.1852) -- cycle;
\fill[blue!15.0, opacity=0.7] (1.4500, 0.5500, 0.1852) -- (1.5000, 0.5500, 0.1854) -- (1.5000, 0.6000, 0.1905) -- (1.4500, 0.6000, 0.1904) -- cycle;
\fill[blue!15.0, opacity=0.7] (1.4500, 0.6000, 0.1904) -- (1.5000, 0.6000, 0.1905) -- (1.5000, 0.6500, 0.1955) -- (1.4500, 0.6500, 0.1954) -- cycle;
\fill[blue!15.0, opacity=0.7] (1.4500, 0.6500, 0.1954) -- (1.5000, 0.6500, 0.1955) -- (1.5000, 0.7000, 0.2003) -- (1.4500, 0.7000, 0.2001) -- cycle;
\fill[blue!20.2, opacity=0.7] (1.4500, 0.7000, 0.2001) -- (1.5000, 0.7000, 0.2003) -- (1.5000, 0.7500, 0.2049) -- (1.4500, 0.7500, 0.2047) -- cycle;
\fill[blue!46.0, opacity=0.7] (1.4500, 0.7500, 0.2047) -- (1.5000, 0.7500, 0.2049) -- (1.5000, 0.8000, 0.2092) -- (1.4500, 0.8000, 0.2090) -- cycle;
\fill[blue!18.3, opacity=0.7] (1.4500, 0.8000, 0.2090) -- (1.5000, 0.8000, 0.2092) -- (1.5000, 0.8500, 0.2133) -- (1.4500, 0.8500, 0.2131) -- cycle;
\fill[blue!15.0, opacity=0.7] (1.4500, 0.8500, 0.2131) -- (1.5000, 0.8500, 0.2133) -- (1.5000, 0.9000, 0.2171) -- (1.4500, 0.9000, 0.2169) -- cycle;
\fill[blue!15.0, opacity=0.7] (1.4500, 0.9000, 0.2169) -- (1.5000, 0.9000, 0.2171) -- (1.5000, 0.9500, 0.2206) -- (1.4500, 0.9500, 0.2205) -- cycle;
\fill[blue!15.0, opacity=0.7] (1.4500, 0.9500, 0.2205) -- (1.5000, 0.9500, 0.2206) -- (1.5000, 1.0000, 0.2239) -- (1.4500, 1.0000, 0.2238) -- cycle;
\fill[blue!15.0, opacity=0.7] (1.4500, 1.0000, 0.2238) -- (1.5000, 1.0000, 0.2239) -- (1.5000, 1.0500, 0.2269) -- (1.4500, 1.0500, 0.2268) -- cycle;
\fill[blue!16.5, opacity=0.7] (1.4500, 1.0500, 0.2268) -- (1.5000, 1.0500, 0.2269) -- (1.5000, 1.1000, 0.2296) -- (1.4500, 1.1000, 0.2295) -- cycle;
\fill[blue!9.8!black, opacity=0.7] (1.4500, 1.1000, 0.2295) -- (1.5000, 1.1000, 0.2296) -- (1.5000, 1.1500, 0.2320) -- (1.4500, 1.1500, 0.2319) -- cycle;
\fill[blue!72.0, opacity=0.7] (1.4500, 1.1500, 0.2319) -- (1.5000, 1.1500, 0.2320) -- (1.5000, 1.2000, 0.2341) -- (1.4500, 1.2000, 0.2340) -- cycle;
\fill[blue!62.0!black, opacity=0.7] (1.4500, 1.2000, 0.2340) -- (1.5000, 1.2000, 0.2341) -- (1.5000, 1.2500, 0.2359) -- (1.4500, 1.2500, 0.2357) -- cycle;
\fill[blue!55.3, opacity=0.7] (1.4500, 1.2500, 0.2357) -- (1.5000, 1.2500, 0.2359) -- (1.5000, 1.3000, 0.2374) -- (1.4500, 1.3000, 0.2372) -- cycle;
\fill[blue!15.1, opacity=0.7] (1.4500, 1.3000, 0.2372) -- (1.5000, 1.3000, 0.2374) -- (1.5000, 1.3500, 0.2385) -- (1.4500, 1.3500, 0.2384) -- cycle;
\fill[blue!15.0, opacity=0.7] (1.4500, 1.3500, 0.2384) -- (1.5000, 1.3500, 0.2385) -- (1.5000, 1.4000, 0.2393) -- (1.4500, 1.4000, 0.2392) -- cycle;
\fill[blue!22.1, opacity=0.7] (1.4500, 1.4000, 0.2392) -- (1.5000, 1.4000, 0.2393) -- (1.5000, 1.4500, 0.2398) -- (1.4500, 1.4500, 0.2397) -- cycle;
\fill[blue!17.4, opacity=0.7] (1.4500, 1.4500, 0.2397) -- (1.5000, 1.4500, 0.2398) -- (1.5000, 1.5000, 0.2400) -- (1.4500, 1.5000, 0.2398) -- cycle;
\fill[blue!15.5, opacity=0.7] (1.4500, 1.5000, 0.2398) -- (1.5000, 1.5000, 0.2400) -- (1.5000, 1.5500, 0.2398) -- (1.4500, 1.5500, 0.2397) -- cycle;
\fill[blue!15.0, opacity=0.7] (1.4500, 1.5500, 0.2397) -- (1.5000, 1.5500, 0.2398) -- (1.5000, 1.6000, 0.2393) -- (1.4500, 1.6000, 0.2392) -- cycle;
\fill[blue!15.2, opacity=0.7] (1.4500, 1.6000, 0.2392) -- (1.5000, 1.6000, 0.2393) -- (1.5000, 1.6500, 0.2385) -- (1.4500, 1.6500, 0.2384) -- cycle;
\fill[blue!32.3, opacity=0.7] (1.4500, 1.6500, 0.2384) -- (1.5000, 1.6500, 0.2385) -- (1.5000, 1.7000, 0.2374) -- (1.4500, 1.7000, 0.2372) -- cycle;
\fill[blue!5.0!black, opacity=0.7] (1.4500, 1.7000, 0.2372) -- (1.5000, 1.7000, 0.2374) -- (1.5000, 1.7500, 0.2359) -- (1.4500, 1.7500, 0.2357) -- cycle;
\fill[blue!65.9, opacity=0.7] (1.4500, 1.7500, 0.2357) -- (1.5000, 1.7500, 0.2359) -- (1.5000, 1.8000, 0.2341) -- (1.4500, 1.8000, 0.2340) -- cycle;
\fill[blue!85.8, opacity=0.7] (1.4500, 1.8000, 0.2340) -- (1.5000, 1.8000, 0.2341) -- (1.5000, 1.8500, 0.2320) -- (1.4500, 1.8500, 0.2319) -- cycle;
\fill[blue!94.3, opacity=0.7] (1.4500, 1.8500, 0.2319) -- (1.5000, 1.8500, 0.2320) -- (1.5000, 1.9000, 0.2296) -- (1.4500, 1.9000, 0.2295) -- cycle;
\fill[blue!15.1, opacity=0.7] (1.4500, 1.9000, 0.2295) -- (1.5000, 1.9000, 0.2296) -- (1.5000, 1.9500, 0.2269) -- (1.4500, 1.9500, 0.2268) -- cycle;
\fill[blue!15.0, opacity=0.7] (1.4500, 1.9500, 0.2268) -- (1.5000, 1.9500, 0.2269) -- (1.5000, 2.0000, 0.2239) -- (1.4500, 2.0000, 0.2238) -- cycle;
\fill[blue!15.0, opacity=0.7] (1.4500, 2.0000, 0.2238) -- (1.5000, 2.0000, 0.2239) -- (1.5000, 2.0500, 0.2206) -- (1.4500, 2.0500, 0.2205) -- cycle;
\fill[blue!15.0, opacity=0.7] (1.4500, 2.0500, 0.2205) -- (1.5000, 2.0500, 0.2206) -- (1.5000, 2.1000, 0.2171) -- (1.4500, 2.1000, 0.2169) -- cycle;
\fill[blue!15.0, opacity=0.7] (1.4500, 2.1000, 0.2169) -- (1.5000, 2.1000, 0.2171) -- (1.5000, 2.1500, 0.2133) -- (1.4500, 2.1500, 0.2131) -- cycle;
\fill[blue!23.9, opacity=0.7] (1.4500, 2.1500, 0.2131) -- (1.5000, 2.1500, 0.2133) -- (1.5000, 2.2000, 0.2092) -- (1.4500, 2.2000, 0.2090) -- cycle;
\fill[blue!48.9, opacity=0.7] (1.4500, 2.2000, 0.2090) -- (1.5000, 2.2000, 0.2092) -- (1.5000, 2.2500, 0.2049) -- (1.4500, 2.2500, 0.2047) -- cycle;
\fill[blue!17.6, opacity=0.7] (1.4500, 2.2500, 0.2047) -- (1.5000, 2.2500, 0.2049) -- (1.5000, 2.3000, 0.2003) -- (1.4500, 2.3000, 0.2001) -- cycle;
\fill[blue!15.0, opacity=0.7] (1.4500, 2.3000, 0.2001) -- (1.5000, 2.3000, 0.2003) -- (1.5000, 2.3500, 0.1955) -- (1.4500, 2.3500, 0.1954) -- cycle;
\fill[blue!15.0, opacity=0.7] (1.4500, 2.3500, 0.1954) -- (1.5000, 2.3500, 0.1955) -- (1.5000, 2.4000, 0.1905) -- (1.4500, 2.4000, 0.1904) -- cycle;
\fill[blue!15.0, opacity=0.7] (1.4500, 2.4000, 0.1904) -- (1.5000, 2.4000, 0.1905) -- (1.5000, 2.4500, 0.1854) -- (1.4500, 2.4500, 0.1852) -- cycle;
\fill[blue!15.0, opacity=0.7] (1.4500, 2.4500, 0.1852) -- (1.5000, 2.4500, 0.1854) -- (1.5000, 2.5000, 0.1800) -- (1.4500, 2.5000, 0.1798) -- cycle;
\fill[blue!15.0, opacity=0.7] (1.4500, 2.5000, 0.1798) -- (1.5000, 2.5000, 0.1800) -- (1.5000, 2.5500, 0.1745) -- (1.4500, 2.5500, 0.1743) -- cycle;
\fill[blue!15.0, opacity=0.7] (1.4500, 2.5500, 0.1743) -- (1.5000, 2.5500, 0.1745) -- (1.5000, 2.6000, 0.1688) -- (1.4500, 2.6000, 0.1686) -- cycle;
\fill[blue!15.0, opacity=0.7] (1.4500, 2.6000, 0.1686) -- (1.5000, 2.6000, 0.1688) -- (1.5000, 2.6500, 0.1630) -- (1.4500, 2.6500, 0.1628) -- cycle;
\fill[blue!15.0, opacity=0.7] (1.4500, 2.6500, 0.1628) -- (1.5000, 2.6500, 0.1630) -- (1.5000, 2.7000, 0.1571) -- (1.4500, 2.7000, 0.1569) -- cycle;
\fill[blue!15.0, opacity=0.7] (1.4500, 2.7000, 0.1569) -- (1.5000, 2.7000, 0.1571) -- (1.5000, 2.7500, 0.1511) -- (1.4500, 2.7500, 0.1509) -- cycle;
\fill[blue!15.0, opacity=0.7] (1.4500, 2.7500, 0.1509) -- (1.5000, 2.7500, 0.1511) -- (1.5000, 2.8000, 0.1449) -- (1.4500, 2.8000, 0.1448) -- cycle;
\fill[blue!15.0, opacity=0.7] (1.4500, 2.8000, 0.1448) -- (1.5000, 2.8000, 0.1449) -- (1.5000, 2.8500, 0.1388) -- (1.4500, 2.8500, 0.1386) -- cycle;
\fill[blue!15.0, opacity=0.7] (1.4500, 2.8500, 0.1386) -- (1.5000, 2.8500, 0.1388) -- (1.5000, 2.9000, 0.1325) -- (1.4500, 2.9000, 0.1324) -- cycle;
\fill[blue!15.0, opacity=0.7] (1.4500, 2.9000, 0.1324) -- (1.5000, 2.9000, 0.1325) -- (1.5000, 2.9500, 0.1263) -- (1.4500, 2.9500, 0.1261) -- cycle;
\fill[blue!15.0, opacity=0.7] (1.4500, 2.9500, 0.1261) -- (1.5000, 2.9500, 0.1263) -- (1.5000, 3.0000, 0.1200) -- (1.4500, 3.0000, 0.1198) -- cycle;
\fill[blue!15.0, opacity=0.7] (1.5000, 0.0000, 0.1200) -- (1.5500, 0.0000, 0.1198) -- (1.5500, 0.0500, 0.1261) -- (1.5000, 0.0500, 0.1263) -- cycle;
\fill[blue!15.0, opacity=0.7] (1.5000, 0.0500, 0.1263) -- (1.5500, 0.0500, 0.1261) -- (1.5500, 0.1000, 0.1324) -- (1.5000, 0.1000, 0.1325) -- cycle;
\fill[blue!15.0, opacity=0.7] (1.5000, 0.1000, 0.1325) -- (1.5500, 0.1000, 0.1324) -- (1.5500, 0.1500, 0.1386) -- (1.5000, 0.1500, 0.1388) -- cycle;
\fill[blue!15.0, opacity=0.7] (1.5000, 0.1500, 0.1388) -- (1.5500, 0.1500, 0.1386) -- (1.5500, 0.2000, 0.1448) -- (1.5000, 0.2000, 0.1449) -- cycle;
\fill[blue!15.0, opacity=0.7] (1.5000, 0.2000, 0.1449) -- (1.5500, 0.2000, 0.1448) -- (1.5500, 0.2500, 0.1509) -- (1.5000, 0.2500, 0.1511) -- cycle;
\fill[blue!15.0, opacity=0.7] (1.5000, 0.2500, 0.1511) -- (1.5500, 0.2500, 0.1509) -- (1.5500, 0.3000, 0.1569) -- (1.5000, 0.3000, 0.1571) -- cycle;
\fill[blue!15.0, opacity=0.7] (1.5000, 0.3000, 0.1571) -- (1.5500, 0.3000, 0.1569) -- (1.5500, 0.3500, 0.1628) -- (1.5000, 0.3500, 0.1630) -- cycle;
\fill[blue!15.0, opacity=0.7] (1.5000, 0.3500, 0.1630) -- (1.5500, 0.3500, 0.1628) -- (1.5500, 0.4000, 0.1686) -- (1.5000, 0.4000, 0.1688) -- cycle;
\fill[blue!15.0, opacity=0.7] (1.5000, 0.4000, 0.1688) -- (1.5500, 0.4000, 0.1686) -- (1.5500, 0.4500, 0.1743) -- (1.5000, 0.4500, 0.1745) -- cycle;
\fill[blue!15.0, opacity=0.7] (1.5000, 0.4500, 0.1745) -- (1.5500, 0.4500, 0.1743) -- (1.5500, 0.5000, 0.1798) -- (1.5000, 0.5000, 0.1800) -- cycle;
\fill[blue!15.0, opacity=0.7] (1.5000, 0.5000, 0.1800) -- (1.5500, 0.5000, 0.1798) -- (1.5500, 0.5500, 0.1852) -- (1.5000, 0.5500, 0.1854) -- cycle;
\fill[blue!15.0, opacity=0.7] (1.5000, 0.5500, 0.1854) -- (1.5500, 0.5500, 0.1852) -- (1.5500, 0.6000, 0.1904) -- (1.5000, 0.6000, 0.1905) -- cycle;
\fill[blue!15.0, opacity=0.7] (1.5000, 0.6000, 0.1905) -- (1.5500, 0.6000, 0.1904) -- (1.5500, 0.6500, 0.1954) -- (1.5000, 0.6500, 0.1955) -- cycle;
\fill[blue!15.0, opacity=0.7] (1.5000, 0.6500, 0.1955) -- (1.5500, 0.6500, 0.1954) -- (1.5500, 0.7000, 0.2001) -- (1.5000, 0.7000, 0.2003) -- cycle;
\fill[blue!17.6, opacity=0.7] (1.5000, 0.7000, 0.2003) -- (1.5500, 0.7000, 0.2001) -- (1.5500, 0.7500, 0.2047) -- (1.5000, 0.7500, 0.2049) -- cycle;
\fill[blue!48.9, opacity=0.7] (1.5000, 0.7500, 0.2049) -- (1.5500, 0.7500, 0.2047) -- (1.5500, 0.8000, 0.2090) -- (1.5000, 0.8000, 0.2092) -- cycle;
\fill[blue!23.9, opacity=0.7] (1.5000, 0.8000, 0.2092) -- (1.5500, 0.8000, 0.2090) -- (1.5500, 0.8500, 0.2131) -- (1.5000, 0.8500, 0.2133) -- cycle;
\fill[blue!15.0, opacity=0.7] (1.5000, 0.8500, 0.2133) -- (1.5500, 0.8500, 0.2131) -- (1.5500, 0.9000, 0.2169) -- (1.5000, 0.9000, 0.2171) -- cycle;
\fill[blue!15.0, opacity=0.7] (1.5000, 0.9000, 0.2171) -- (1.5500, 0.9000, 0.2169) -- (1.5500, 0.9500, 0.2205) -- (1.5000, 0.9500, 0.2206) -- cycle;
\fill[blue!15.0, opacity=0.7] (1.5000, 0.9500, 0.2206) -- (1.5500, 0.9500, 0.2205) -- (1.5500, 1.0000, 0.2238) -- (1.5000, 1.0000, 0.2239) -- cycle;
\fill[blue!15.0, opacity=0.7] (1.5000, 1.0000, 0.2239) -- (1.5500, 1.0000, 0.2238) -- (1.5500, 1.0500, 0.2268) -- (1.5000, 1.0500, 0.2269) -- cycle;
\fill[blue!15.1, opacity=0.7] (1.5000, 1.0500, 0.2269) -- (1.5500, 1.0500, 0.2268) -- (1.5500, 1.1000, 0.2295) -- (1.5000, 1.1000, 0.2296) -- cycle;
\fill[blue!94.3, opacity=0.7] (1.5000, 1.1000, 0.2296) -- (1.5500, 1.1000, 0.2295) -- (1.5500, 1.1500, 0.2319) -- (1.5000, 1.1500, 0.2320) -- cycle;
\fill[blue!85.8, opacity=0.7] (1.5000, 1.1500, 0.2320) -- (1.5500, 1.1500, 0.2319) -- (1.5500, 1.2000, 0.2340) -- (1.5000, 1.2000, 0.2341) -- cycle;
\fill[blue!65.9, opacity=0.7] (1.5000, 1.2000, 0.2341) -- (1.5500, 1.2000, 0.2340) -- (1.5500, 1.2500, 0.2357) -- (1.5000, 1.2500, 0.2359) -- cycle;
\fill[blue!5.0!black, opacity=0.7] (1.5000, 1.2500, 0.2359) -- (1.5500, 1.2500, 0.2357) -- (1.5500, 1.3000, 0.2372) -- (1.5000, 1.3000, 0.2374) -- cycle;
\fill[blue!32.3, opacity=0.7] (1.5000, 1.3000, 0.2374) -- (1.5500, 1.3000, 0.2372) -- (1.5500, 1.3500, 0.2384) -- (1.5000, 1.3500, 0.2385) -- cycle;
\fill[blue!15.2, opacity=0.7] (1.5000, 1.3500, 0.2385) -- (1.5500, 1.3500, 0.2384) -- (1.5500, 1.4000, 0.2392) -- (1.5000, 1.4000, 0.2393) -- cycle;
\fill[blue!15.0, opacity=0.7] (1.5000, 1.4000, 0.2393) -- (1.5500, 1.4000, 0.2392) -- (1.5500, 1.4500, 0.2397) -- (1.5000, 1.4500, 0.2398) -- cycle;
\fill[blue!15.5, opacity=0.7] (1.5000, 1.4500, 0.2398) -- (1.5500, 1.4500, 0.2397) -- (1.5500, 1.5000, 0.2398) -- (1.5000, 1.5000, 0.2400) -- cycle;
\fill[blue!17.4, opacity=0.7] (1.5000, 1.5000, 0.2400) -- (1.5500, 1.5000, 0.2398) -- (1.5500, 1.5500, 0.2397) -- (1.5000, 1.5500, 0.2398) -- cycle;
\fill[blue!22.1, opacity=0.7] (1.5000, 1.5500, 0.2398) -- (1.5500, 1.5500, 0.2397) -- (1.5500, 1.6000, 0.2392) -- (1.5000, 1.6000, 0.2393) -- cycle;
\fill[blue!15.0, opacity=0.7] (1.5000, 1.6000, 0.2393) -- (1.5500, 1.6000, 0.2392) -- (1.5500, 1.6500, 0.2384) -- (1.5000, 1.6500, 0.2385) -- cycle;
\fill[blue!15.1, opacity=0.7] (1.5000, 1.6500, 0.2385) -- (1.5500, 1.6500, 0.2384) -- (1.5500, 1.7000, 0.2372) -- (1.5000, 1.7000, 0.2374) -- cycle;
\fill[blue!55.3, opacity=0.7] (1.5000, 1.7000, 0.2374) -- (1.5500, 1.7000, 0.2372) -- (1.5500, 1.7500, 0.2357) -- (1.5000, 1.7500, 0.2359) -- cycle;
\fill[blue!62.0!black, opacity=0.7] (1.5000, 1.7500, 0.2359) -- (1.5500, 1.7500, 0.2357) -- (1.5500, 1.8000, 0.2340) -- (1.5000, 1.8000, 0.2341) -- cycle;
\fill[blue!72.0, opacity=0.7] (1.5000, 1.8000, 0.2341) -- (1.5500, 1.8000, 0.2340) -- (1.5500, 1.8500, 0.2319) -- (1.5000, 1.8500, 0.2320) -- cycle;
\fill[blue!9.8!black, opacity=0.7] (1.5000, 1.8500, 0.2320) -- (1.5500, 1.8500, 0.2319) -- (1.5500, 1.9000, 0.2295) -- (1.5000, 1.9000, 0.2296) -- cycle;
\fill[blue!16.5, opacity=0.7] (1.5000, 1.9000, 0.2296) -- (1.5500, 1.9000, 0.2295) -- (1.5500, 1.9500, 0.2268) -- (1.5000, 1.9500, 0.2269) -- cycle;
\fill[blue!15.0, opacity=0.7] (1.5000, 1.9500, 0.2269) -- (1.5500, 1.9500, 0.2268) -- (1.5500, 2.0000, 0.2238) -- (1.5000, 2.0000, 0.2239) -- cycle;
\fill[blue!15.0, opacity=0.7] (1.5000, 2.0000, 0.2239) -- (1.5500, 2.0000, 0.2238) -- (1.5500, 2.0500, 0.2205) -- (1.5000, 2.0500, 0.2206) -- cycle;
\fill[blue!15.0, opacity=0.7] (1.5000, 2.0500, 0.2206) -- (1.5500, 2.0500, 0.2205) -- (1.5500, 2.1000, 0.2169) -- (1.5000, 2.1000, 0.2171) -- cycle;
\fill[blue!15.0, opacity=0.7] (1.5000, 2.1000, 0.2171) -- (1.5500, 2.1000, 0.2169) -- (1.5500, 2.1500, 0.2131) -- (1.5000, 2.1500, 0.2133) -- cycle;
\fill[blue!18.3, opacity=0.7] (1.5000, 2.1500, 0.2133) -- (1.5500, 2.1500, 0.2131) -- (1.5500, 2.2000, 0.2090) -- (1.5000, 2.2000, 0.2092) -- cycle;
\fill[blue!46.0, opacity=0.7] (1.5000, 2.2000, 0.2092) -- (1.5500, 2.2000, 0.2090) -- (1.5500, 2.2500, 0.2047) -- (1.5000, 2.2500, 0.2049) -- cycle;
\fill[blue!20.2, opacity=0.7] (1.5000, 2.2500, 0.2049) -- (1.5500, 2.2500, 0.2047) -- (1.5500, 2.3000, 0.2001) -- (1.5000, 2.3000, 0.2003) -- cycle;
\fill[blue!15.0, opacity=0.7] (1.5000, 2.3000, 0.2003) -- (1.5500, 2.3000, 0.2001) -- (1.5500, 2.3500, 0.1954) -- (1.5000, 2.3500, 0.1955) -- cycle;
\fill[blue!15.0, opacity=0.7] (1.5000, 2.3500, 0.1955) -- (1.5500, 2.3500, 0.1954) -- (1.5500, 2.4000, 0.1904) -- (1.5000, 2.4000, 0.1905) -- cycle;
\fill[blue!15.0, opacity=0.7] (1.5000, 2.4000, 0.1905) -- (1.5500, 2.4000, 0.1904) -- (1.5500, 2.4500, 0.1852) -- (1.5000, 2.4500, 0.1854) -- cycle;
\fill[blue!15.0, opacity=0.7] (1.5000, 2.4500, 0.1854) -- (1.5500, 2.4500, 0.1852) -- (1.5500, 2.5000, 0.1798) -- (1.5000, 2.5000, 0.1800) -- cycle;
\fill[blue!15.0, opacity=0.7] (1.5000, 2.5000, 0.1800) -- (1.5500, 2.5000, 0.1798) -- (1.5500, 2.5500, 0.1743) -- (1.5000, 2.5500, 0.1745) -- cycle;
\fill[blue!15.0, opacity=0.7] (1.5000, 2.5500, 0.1745) -- (1.5500, 2.5500, 0.1743) -- (1.5500, 2.6000, 0.1686) -- (1.5000, 2.6000, 0.1688) -- cycle;
\fill[blue!15.0, opacity=0.7] (1.5000, 2.6000, 0.1688) -- (1.5500, 2.6000, 0.1686) -- (1.5500, 2.6500, 0.1628) -- (1.5000, 2.6500, 0.1630) -- cycle;
\fill[blue!15.0, opacity=0.7] (1.5000, 2.6500, 0.1630) -- (1.5500, 2.6500, 0.1628) -- (1.5500, 2.7000, 0.1569) -- (1.5000, 2.7000, 0.1571) -- cycle;
\fill[blue!15.0, opacity=0.7] (1.5000, 2.7000, 0.1571) -- (1.5500, 2.7000, 0.1569) -- (1.5500, 2.7500, 0.1509) -- (1.5000, 2.7500, 0.1511) -- cycle;
\fill[blue!15.0, opacity=0.7] (1.5000, 2.7500, 0.1511) -- (1.5500, 2.7500, 0.1509) -- (1.5500, 2.8000, 0.1448) -- (1.5000, 2.8000, 0.1449) -- cycle;
\fill[blue!15.0, opacity=0.7] (1.5000, 2.8000, 0.1449) -- (1.5500, 2.8000, 0.1448) -- (1.5500, 2.8500, 0.1386) -- (1.5000, 2.8500, 0.1388) -- cycle;
\fill[blue!15.0, opacity=0.7] (1.5000, 2.8500, 0.1388) -- (1.5500, 2.8500, 0.1386) -- (1.5500, 2.9000, 0.1324) -- (1.5000, 2.9000, 0.1325) -- cycle;
\fill[blue!15.0, opacity=0.7] (1.5000, 2.9000, 0.1325) -- (1.5500, 2.9000, 0.1324) -- (1.5500, 2.9500, 0.1261) -- (1.5000, 2.9500, 0.1263) -- cycle;
\fill[blue!15.0, opacity=0.7] (1.5000, 2.9500, 0.1263) -- (1.5500, 2.9500, 0.1261) -- (1.5500, 3.0000, 0.1198) -- (1.5000, 3.0000, 0.1200) -- cycle;
\fill[blue!15.0, opacity=0.7] (1.5500, 0.0000, 0.1198) -- (1.6000, 0.0000, 0.1193) -- (1.6000, 0.0500, 0.1256) -- (1.5500, 0.0500, 0.1261) -- cycle;
\fill[blue!15.0, opacity=0.7] (1.5500, 0.0500, 0.1261) -- (1.6000, 0.0500, 0.1256) -- (1.6000, 0.1000, 0.1319) -- (1.5500, 0.1000, 0.1324) -- cycle;
\fill[blue!15.0, opacity=0.7] (1.5500, 0.1000, 0.1324) -- (1.6000, 0.1000, 0.1319) -- (1.6000, 0.1500, 0.1381) -- (1.5500, 0.1500, 0.1386) -- cycle;
\fill[blue!15.0, opacity=0.7] (1.5500, 0.1500, 0.1386) -- (1.6000, 0.1500, 0.1381) -- (1.6000, 0.2000, 0.1443) -- (1.5500, 0.2000, 0.1448) -- cycle;
\fill[blue!15.0, opacity=0.7] (1.5500, 0.2000, 0.1448) -- (1.6000, 0.2000, 0.1443) -- (1.6000, 0.2500, 0.1504) -- (1.5500, 0.2500, 0.1509) -- cycle;
\fill[blue!15.0, opacity=0.7] (1.5500, 0.2500, 0.1509) -- (1.6000, 0.2500, 0.1504) -- (1.6000, 0.3000, 0.1564) -- (1.5500, 0.3000, 0.1569) -- cycle;
\fill[blue!15.0, opacity=0.7] (1.5500, 0.3000, 0.1569) -- (1.6000, 0.3000, 0.1564) -- (1.6000, 0.3500, 0.1623) -- (1.5500, 0.3500, 0.1628) -- cycle;
\fill[blue!15.0, opacity=0.7] (1.5500, 0.3500, 0.1628) -- (1.6000, 0.3500, 0.1623) -- (1.6000, 0.4000, 0.1682) -- (1.5500, 0.4000, 0.1686) -- cycle;
\fill[blue!15.0, opacity=0.7] (1.5500, 0.4000, 0.1686) -- (1.6000, 0.4000, 0.1682) -- (1.6000, 0.4500, 0.1738) -- (1.5500, 0.4500, 0.1743) -- cycle;
\fill[blue!15.0, opacity=0.7] (1.5500, 0.4500, 0.1743) -- (1.6000, 0.4500, 0.1738) -- (1.6000, 0.5000, 0.1793) -- (1.5500, 0.5000, 0.1798) -- cycle;
\fill[blue!15.0, opacity=0.7] (1.5500, 0.5000, 0.1798) -- (1.6000, 0.5000, 0.1793) -- (1.6000, 0.5500, 0.1847) -- (1.5500, 0.5500, 0.1852) -- cycle;
\fill[blue!15.0, opacity=0.7] (1.5500, 0.5500, 0.1852) -- (1.6000, 0.5500, 0.1847) -- (1.6000, 0.6000, 0.1899) -- (1.5500, 0.6000, 0.1904) -- cycle;
\fill[blue!15.0, opacity=0.7] (1.5500, 0.6000, 0.1904) -- (1.6000, 0.6000, 0.1899) -- (1.6000, 0.6500, 0.1949) -- (1.5500, 0.6500, 0.1954) -- cycle;
\fill[blue!15.0, opacity=0.7] (1.5500, 0.6500, 0.1954) -- (1.6000, 0.6500, 0.1949) -- (1.6000, 0.7000, 0.1996) -- (1.5500, 0.7000, 0.2001) -- cycle;
\fill[blue!15.6, opacity=0.7] (1.5500, 0.7000, 0.2001) -- (1.6000, 0.7000, 0.1996) -- (1.6000, 0.7500, 0.2042) -- (1.5500, 0.7500, 0.2047) -- cycle;
\fill[blue!45.9, opacity=0.7] (1.5500, 0.7500, 0.2047) -- (1.6000, 0.7500, 0.2042) -- (1.6000, 0.8000, 0.2085) -- (1.5500, 0.8000, 0.2090) -- cycle;
\fill[blue!36.6, opacity=0.7] (1.5500, 0.8000, 0.2090) -- (1.6000, 0.8000, 0.2085) -- (1.6000, 0.8500, 0.2126) -- (1.5500, 0.8500, 0.2131) -- cycle;
\fill[blue!15.2, opacity=0.7] (1.5500, 0.8500, 0.2131) -- (1.6000, 0.8500, 0.2126) -- (1.6000, 0.9000, 0.2164) -- (1.5500, 0.9000, 0.2169) -- cycle;
\fill[blue!15.0, opacity=0.7] (1.5500, 0.9000, 0.2169) -- (1.6000, 0.9000, 0.2164) -- (1.6000, 0.9500, 0.2200) -- (1.5500, 0.9500, 0.2205) -- cycle;
\fill[blue!15.0, opacity=0.7] (1.5500, 0.9500, 0.2205) -- (1.6000, 0.9500, 0.2200) -- (1.6000, 1.0000, 0.2233) -- (1.5500, 1.0000, 0.2238) -- cycle;
\fill[blue!15.0, opacity=0.7] (1.5500, 1.0000, 0.2238) -- (1.6000, 1.0000, 0.2233) -- (1.6000, 1.0500, 0.2263) -- (1.5500, 1.0500, 0.2268) -- cycle;
\fill[blue!15.0, opacity=0.7] (1.5500, 1.0500, 0.2268) -- (1.6000, 1.0500, 0.2263) -- (1.6000, 1.1000, 0.2290) -- (1.5500, 1.1000, 0.2295) -- cycle;
\fill[blue!26.6, opacity=0.7] (1.5500, 1.1000, 0.2295) -- (1.6000, 1.1000, 0.2290) -- (1.6000, 1.1500, 0.2314) -- (1.5500, 1.1500, 0.2319) -- cycle;
\fill[blue!14.1!black, opacity=0.7] (1.5500, 1.1500, 0.2319) -- (1.6000, 1.1500, 0.2314) -- (1.6000, 1.2000, 0.2335) -- (1.5500, 1.2000, 0.2340) -- cycle;
\fill[blue!54.2, opacity=0.7] (1.5500, 1.2000, 0.2340) -- (1.6000, 1.2000, 0.2335) -- (1.6000, 1.2500, 0.2353) -- (1.5500, 1.2500, 0.2357) -- cycle;
\fill[blue!60.2, opacity=0.7] (1.5500, 1.2500, 0.2357) -- (1.6000, 1.2500, 0.2353) -- (1.6000, 1.3000, 0.2367) -- (1.5500, 1.3000, 0.2372) -- cycle;
\fill[blue!32.1!black, opacity=0.7] (1.5500, 1.3000, 0.2372) -- (1.6000, 1.3000, 0.2367) -- (1.6000, 1.3500, 0.2379) -- (1.5500, 1.3500, 0.2384) -- cycle;
\fill[blue!55.1!black, opacity=0.7] (1.5500, 1.3500, 0.2384) -- (1.6000, 1.3500, 0.2379) -- (1.6000, 1.4000, 0.2387) -- (1.5500, 1.4000, 0.2392) -- cycle;
\fill[blue!70.1!black, opacity=0.7] (1.5500, 1.4000, 0.2392) -- (1.6000, 1.4000, 0.2387) -- (1.6000, 1.4500, 0.2392) -- (1.5500, 1.4500, 0.2397) -- cycle;
\fill[blue!69.5, opacity=0.7] (1.5500, 1.4500, 0.2397) -- (1.6000, 1.4500, 0.2392) -- (1.6000, 1.5000, 0.2393) -- (1.5500, 1.5000, 0.2398) -- cycle;
\fill[blue!17.0, opacity=0.7] (1.5500, 1.5000, 0.2398) -- (1.6000, 1.5000, 0.2393) -- (1.6000, 1.5500, 0.2392) -- (1.5500, 1.5500, 0.2397) -- cycle;
\fill[blue!95.7!black, opacity=0.7] (1.5500, 1.5500, 0.2397) -- (1.6000, 1.5500, 0.2392) -- (1.6000, 1.6000, 0.2387) -- (1.5500, 1.6000, 0.2392) -- cycle;
\fill[blue!15.0, opacity=0.7] (1.5500, 1.6000, 0.2392) -- (1.6000, 1.6000, 0.2387) -- (1.6000, 1.6500, 0.2379) -- (1.5500, 1.6500, 0.2384) -- cycle;
\fill[blue!15.0, opacity=0.7] (1.5500, 1.6500, 0.2384) -- (1.6000, 1.6500, 0.2379) -- (1.6000, 1.7000, 0.2367) -- (1.5500, 1.7000, 0.2372) -- cycle;
\fill[blue!20.7, opacity=0.7] (1.5500, 1.7000, 0.2372) -- (1.6000, 1.7000, 0.2367) -- (1.6000, 1.7500, 0.2353) -- (1.5500, 1.7500, 0.2357) -- cycle;
\fill[blue!5.0!black, opacity=0.7] (1.5500, 1.7500, 0.2357) -- (1.6000, 1.7500, 0.2353) -- (1.6000, 1.8000, 0.2335) -- (1.5500, 1.8000, 0.2340) -- cycle;
\fill[blue!79.1, opacity=0.7] (1.5500, 1.8000, 0.2340) -- (1.6000, 1.8000, 0.2335) -- (1.6000, 1.8500, 0.2314) -- (1.5500, 1.8500, 0.2319) -- cycle;
\fill[blue!8.7!black, opacity=0.7] (1.5500, 1.8500, 0.2319) -- (1.6000, 1.8500, 0.2314) -- (1.6000, 1.9000, 0.2290) -- (1.5500, 1.9000, 0.2295) -- cycle;
\fill[blue!20.4, opacity=0.7] (1.5500, 1.9000, 0.2295) -- (1.6000, 1.9000, 0.2290) -- (1.6000, 1.9500, 0.2263) -- (1.5500, 1.9500, 0.2268) -- cycle;
\fill[blue!15.0, opacity=0.7] (1.5500, 1.9500, 0.2268) -- (1.6000, 1.9500, 0.2263) -- (1.6000, 2.0000, 0.2233) -- (1.5500, 2.0000, 0.2238) -- cycle;
\fill[blue!15.0, opacity=0.7] (1.5500, 2.0000, 0.2238) -- (1.6000, 2.0000, 0.2233) -- (1.6000, 2.0500, 0.2200) -- (1.5500, 2.0500, 0.2205) -- cycle;
\fill[blue!15.0, opacity=0.7] (1.5500, 2.0500, 0.2205) -- (1.6000, 2.0500, 0.2200) -- (1.6000, 2.1000, 0.2164) -- (1.5500, 2.1000, 0.2169) -- cycle;
\fill[blue!15.0, opacity=0.7] (1.5500, 2.1000, 0.2169) -- (1.6000, 2.1000, 0.2164) -- (1.6000, 2.1500, 0.2126) -- (1.5500, 2.1500, 0.2131) -- cycle;
\fill[blue!16.4, opacity=0.7] (1.5500, 2.1500, 0.2131) -- (1.6000, 2.1500, 0.2126) -- (1.6000, 2.2000, 0.2085) -- (1.5500, 2.2000, 0.2090) -- cycle;
\fill[blue!41.5, opacity=0.7] (1.5500, 2.2000, 0.2090) -- (1.6000, 2.2000, 0.2085) -- (1.6000, 2.2500, 0.2042) -- (1.5500, 2.2500, 0.2047) -- cycle;
\fill[blue!22.0, opacity=0.7] (1.5500, 2.2500, 0.2047) -- (1.6000, 2.2500, 0.2042) -- (1.6000, 2.3000, 0.1996) -- (1.5500, 2.3000, 0.2001) -- cycle;
\fill[blue!15.0, opacity=0.7] (1.5500, 2.3000, 0.2001) -- (1.6000, 2.3000, 0.1996) -- (1.6000, 2.3500, 0.1949) -- (1.5500, 2.3500, 0.1954) -- cycle;
\fill[blue!15.0, opacity=0.7] (1.5500, 2.3500, 0.1954) -- (1.6000, 2.3500, 0.1949) -- (1.6000, 2.4000, 0.1899) -- (1.5500, 2.4000, 0.1904) -- cycle;
\fill[blue!15.0, opacity=0.7] (1.5500, 2.4000, 0.1904) -- (1.6000, 2.4000, 0.1899) -- (1.6000, 2.4500, 0.1847) -- (1.5500, 2.4500, 0.1852) -- cycle;
\fill[blue!15.0, opacity=0.7] (1.5500, 2.4500, 0.1852) -- (1.6000, 2.4500, 0.1847) -- (1.6000, 2.5000, 0.1793) -- (1.5500, 2.5000, 0.1798) -- cycle;
\fill[blue!15.0, opacity=0.7] (1.5500, 2.5000, 0.1798) -- (1.6000, 2.5000, 0.1793) -- (1.6000, 2.5500, 0.1738) -- (1.5500, 2.5500, 0.1743) -- cycle;
\fill[blue!15.0, opacity=0.7] (1.5500, 2.5500, 0.1743) -- (1.6000, 2.5500, 0.1738) -- (1.6000, 2.6000, 0.1682) -- (1.5500, 2.6000, 0.1686) -- cycle;
\fill[blue!15.0, opacity=0.7] (1.5500, 2.6000, 0.1686) -- (1.6000, 2.6000, 0.1682) -- (1.6000, 2.6500, 0.1623) -- (1.5500, 2.6500, 0.1628) -- cycle;
\fill[blue!15.0, opacity=0.7] (1.5500, 2.6500, 0.1628) -- (1.6000, 2.6500, 0.1623) -- (1.6000, 2.7000, 0.1564) -- (1.5500, 2.7000, 0.1569) -- cycle;
\fill[blue!15.0, opacity=0.7] (1.5500, 2.7000, 0.1569) -- (1.6000, 2.7000, 0.1564) -- (1.6000, 2.7500, 0.1504) -- (1.5500, 2.7500, 0.1509) -- cycle;
\fill[blue!15.0, opacity=0.7] (1.5500, 2.7500, 0.1509) -- (1.6000, 2.7500, 0.1504) -- (1.6000, 2.8000, 0.1443) -- (1.5500, 2.8000, 0.1448) -- cycle;
\fill[blue!15.0, opacity=0.7] (1.5500, 2.8000, 0.1448) -- (1.6000, 2.8000, 0.1443) -- (1.6000, 2.8500, 0.1381) -- (1.5500, 2.8500, 0.1386) -- cycle;
\fill[blue!15.0, opacity=0.7] (1.5500, 2.8500, 0.1386) -- (1.6000, 2.8500, 0.1381) -- (1.6000, 2.9000, 0.1319) -- (1.5500, 2.9000, 0.1324) -- cycle;
\fill[blue!15.0, opacity=0.7] (1.5500, 2.9000, 0.1324) -- (1.6000, 2.9000, 0.1319) -- (1.6000, 2.9500, 0.1256) -- (1.5500, 2.9500, 0.1261) -- cycle;
\fill[blue!15.0, opacity=0.7] (1.5500, 2.9500, 0.1261) -- (1.6000, 2.9500, 0.1256) -- (1.6000, 3.0000, 0.1193) -- (1.5500, 3.0000, 0.1198) -- cycle;
\fill[blue!15.0, opacity=0.7] (1.6000, 0.0000, 0.1193) -- (1.6500, 0.0000, 0.1185) -- (1.6500, 0.0500, 0.1248) -- (1.6000, 0.0500, 0.1256) -- cycle;
\fill[blue!15.0, opacity=0.7] (1.6000, 0.0500, 0.1256) -- (1.6500, 0.0500, 0.1248) -- (1.6500, 0.1000, 0.1311) -- (1.6000, 0.1000, 0.1319) -- cycle;
\fill[blue!15.0, opacity=0.7] (1.6000, 0.1000, 0.1319) -- (1.6500, 0.1000, 0.1311) -- (1.6500, 0.1500, 0.1373) -- (1.6000, 0.1500, 0.1381) -- cycle;
\fill[blue!15.0, opacity=0.7] (1.6000, 0.1500, 0.1381) -- (1.6500, 0.1500, 0.1373) -- (1.6500, 0.2000, 0.1435) -- (1.6000, 0.2000, 0.1443) -- cycle;
\fill[blue!15.0, opacity=0.7] (1.6000, 0.2000, 0.1443) -- (1.6500, 0.2000, 0.1435) -- (1.6500, 0.2500, 0.1496) -- (1.6000, 0.2500, 0.1504) -- cycle;
\fill[blue!15.0, opacity=0.7] (1.6000, 0.2500, 0.1504) -- (1.6500, 0.2500, 0.1496) -- (1.6500, 0.3000, 0.1556) -- (1.6000, 0.3000, 0.1564) -- cycle;
\fill[blue!15.0, opacity=0.7] (1.6000, 0.3000, 0.1564) -- (1.6500, 0.3000, 0.1556) -- (1.6500, 0.3500, 0.1615) -- (1.6000, 0.3500, 0.1623) -- cycle;
\fill[blue!15.0, opacity=0.7] (1.6000, 0.3500, 0.1623) -- (1.6500, 0.3500, 0.1615) -- (1.6500, 0.4000, 0.1673) -- (1.6000, 0.4000, 0.1682) -- cycle;
\fill[blue!15.0, opacity=0.7] (1.6000, 0.4000, 0.1682) -- (1.6500, 0.4000, 0.1673) -- (1.6500, 0.4500, 0.1730) -- (1.6000, 0.4500, 0.1738) -- cycle;
\fill[blue!15.0, opacity=0.7] (1.6000, 0.4500, 0.1738) -- (1.6500, 0.4500, 0.1730) -- (1.6500, 0.5000, 0.1785) -- (1.6000, 0.5000, 0.1793) -- cycle;
\fill[blue!15.0, opacity=0.7] (1.6000, 0.5000, 0.1793) -- (1.6500, 0.5000, 0.1785) -- (1.6500, 0.5500, 0.1839) -- (1.6000, 0.5500, 0.1847) -- cycle;
\fill[blue!15.0, opacity=0.7] (1.6000, 0.5500, 0.1847) -- (1.6500, 0.5500, 0.1839) -- (1.6500, 0.6000, 0.1891) -- (1.6000, 0.6000, 0.1899) -- cycle;
\fill[blue!15.0, opacity=0.7] (1.6000, 0.6000, 0.1899) -- (1.6500, 0.6000, 0.1891) -- (1.6500, 0.6500, 0.1940) -- (1.6000, 0.6500, 0.1949) -- cycle;
\fill[blue!15.0, opacity=0.7] (1.6000, 0.6500, 0.1949) -- (1.6500, 0.6500, 0.1940) -- (1.6500, 0.7000, 0.1988) -- (1.6000, 0.7000, 0.1996) -- cycle;
\fill[blue!15.1, opacity=0.7] (1.6000, 0.7000, 0.1996) -- (1.6500, 0.7000, 0.1988) -- (1.6500, 0.7500, 0.2034) -- (1.6000, 0.7500, 0.2042) -- cycle;
\fill[blue!34.2, opacity=0.7] (1.6000, 0.7500, 0.2042) -- (1.6500, 0.7500, 0.2034) -- (1.6500, 0.8000, 0.2077) -- (1.6000, 0.8000, 0.2085) -- cycle;
\fill[blue!53.7, opacity=0.7] (1.6000, 0.8000, 0.2085) -- (1.6500, 0.8000, 0.2077) -- (1.6500, 0.8500, 0.2118) -- (1.6000, 0.8500, 0.2126) -- cycle;
\fill[blue!18.0, opacity=0.7] (1.6000, 0.8500, 0.2126) -- (1.6500, 0.8500, 0.2118) -- (1.6500, 0.9000, 0.2156) -- (1.6000, 0.9000, 0.2164) -- cycle;
\fill[blue!15.0, opacity=0.7] (1.6000, 0.9000, 0.2164) -- (1.6500, 0.9000, 0.2156) -- (1.6500, 0.9500, 0.2192) -- (1.6000, 0.9500, 0.2200) -- cycle;
\fill[blue!15.0, opacity=0.7] (1.6000, 0.9500, 0.2200) -- (1.6500, 0.9500, 0.2192) -- (1.6500, 1.0000, 0.2224) -- (1.6000, 1.0000, 0.2233) -- cycle;
\fill[blue!15.0, opacity=0.7] (1.6000, 1.0000, 0.2233) -- (1.6500, 1.0000, 0.2224) -- (1.6500, 1.0500, 0.2254) -- (1.6000, 1.0500, 0.2263) -- cycle;
\fill[blue!15.0, opacity=0.7] (1.6000, 1.0500, 0.2263) -- (1.6500, 1.0500, 0.2254) -- (1.6500, 1.1000, 0.2281) -- (1.6000, 1.1000, 0.2290) -- cycle;
\fill[blue!15.1, opacity=0.7] (1.6000, 1.1000, 0.2290) -- (1.6500, 1.1000, 0.2281) -- (1.6500, 1.1500, 0.2306) -- (1.6000, 1.1500, 0.2314) -- cycle;
\fill[blue!65.1, opacity=0.7] (1.6000, 1.1500, 0.2314) -- (1.6500, 1.1500, 0.2306) -- (1.6500, 1.2000, 0.2326) -- (1.6000, 1.2000, 0.2335) -- cycle;
\fill[blue!62.1!black, opacity=0.7] (1.6000, 1.2000, 0.2335) -- (1.6500, 1.2000, 0.2326) -- (1.6500, 1.2500, 0.2344) -- (1.6000, 1.2500, 0.2353) -- cycle;
\fill[blue!45.2, opacity=0.7] (1.6000, 1.2500, 0.2353) -- (1.6500, 1.2500, 0.2344) -- (1.6500, 1.3000, 0.2359) -- (1.6000, 1.3000, 0.2367) -- cycle;
\fill[blue!34.7, opacity=0.7] (1.6000, 1.3000, 0.2367) -- (1.6500, 1.3000, 0.2359) -- (1.6500, 1.3500, 0.2370) -- (1.6000, 1.3500, 0.2379) -- cycle;
\fill[blue!36.1, opacity=0.7] (1.6000, 1.3500, 0.2379) -- (1.6500, 1.3500, 0.2370) -- (1.6500, 1.4000, 0.2379) -- (1.6000, 1.4000, 0.2387) -- cycle;
\fill[blue!27.6, opacity=0.7] (1.6000, 1.4000, 0.2387) -- (1.6500, 1.4000, 0.2379) -- (1.6500, 1.4500, 0.2384) -- (1.6000, 1.4500, 0.2392) -- cycle;
\fill[blue!18.8, opacity=0.7] (1.6000, 1.4500, 0.2392) -- (1.6500, 1.4500, 0.2384) -- (1.6500, 1.5000, 0.2385) -- (1.6000, 1.5000, 0.2393) -- cycle;
\fill[blue!42.7, opacity=0.7] (1.6000, 1.5000, 0.2393) -- (1.6500, 1.5000, 0.2385) -- (1.6500, 1.5500, 0.2384) -- (1.6000, 1.5500, 0.2392) -- cycle;
\fill[blue!55.9, opacity=0.7] (1.6000, 1.5500, 0.2392) -- (1.6500, 1.5500, 0.2384) -- (1.6500, 1.6000, 0.2379) -- (1.6000, 1.6000, 0.2387) -- cycle;
\fill[blue!15.0, opacity=0.7] (1.6000, 1.6000, 0.2387) -- (1.6500, 1.6000, 0.2379) -- (1.6500, 1.6500, 0.2370) -- (1.6000, 1.6500, 0.2379) -- cycle;
\fill[blue!15.0, opacity=0.7] (1.6000, 1.6500, 0.2379) -- (1.6500, 1.6500, 0.2370) -- (1.6500, 1.7000, 0.2359) -- (1.6000, 1.7000, 0.2367) -- cycle;
\fill[blue!17.1, opacity=0.7] (1.6000, 1.7000, 0.2367) -- (1.6500, 1.7000, 0.2359) -- (1.6500, 1.7500, 0.2344) -- (1.6000, 1.7500, 0.2353) -- cycle;
\fill[blue!21.7!black, opacity=0.7] (1.6000, 1.7500, 0.2353) -- (1.6500, 1.7500, 0.2344) -- (1.6500, 1.8000, 0.2326) -- (1.6000, 1.8000, 0.2335) -- cycle;
\fill[blue!89.7, opacity=0.7] (1.6000, 1.8000, 0.2335) -- (1.6500, 1.8000, 0.2326) -- (1.6500, 1.8500, 0.2306) -- (1.6000, 1.8500, 0.2314) -- cycle;
\fill[blue!9.8!black, opacity=0.7] (1.6000, 1.8500, 0.2314) -- (1.6500, 1.8500, 0.2306) -- (1.6500, 1.9000, 0.2281) -- (1.6000, 1.9000, 0.2290) -- cycle;
\fill[blue!21.9, opacity=0.7] (1.6000, 1.9000, 0.2290) -- (1.6500, 1.9000, 0.2281) -- (1.6500, 1.9500, 0.2254) -- (1.6000, 1.9500, 0.2263) -- cycle;
\fill[blue!15.0, opacity=0.7] (1.6000, 1.9500, 0.2263) -- (1.6500, 1.9500, 0.2254) -- (1.6500, 2.0000, 0.2224) -- (1.6000, 2.0000, 0.2233) -- cycle;
\fill[blue!15.0, opacity=0.7] (1.6000, 2.0000, 0.2233) -- (1.6500, 2.0000, 0.2224) -- (1.6500, 2.0500, 0.2192) -- (1.6000, 2.0500, 0.2200) -- cycle;
\fill[blue!15.0, opacity=0.7] (1.6000, 2.0500, 0.2200) -- (1.6500, 2.0500, 0.2192) -- (1.6500, 2.1000, 0.2156) -- (1.6000, 2.1000, 0.2164) -- cycle;
\fill[blue!15.0, opacity=0.7] (1.6000, 2.1000, 0.2164) -- (1.6500, 2.1000, 0.2156) -- (1.6500, 2.1500, 0.2118) -- (1.6000, 2.1500, 0.2126) -- cycle;
\fill[blue!15.9, opacity=0.7] (1.6000, 2.1500, 0.2126) -- (1.6500, 2.1500, 0.2118) -- (1.6500, 2.2000, 0.2077) -- (1.6000, 2.2000, 0.2085) -- cycle;
\fill[blue!37.9, opacity=0.7] (1.6000, 2.2000, 0.2085) -- (1.6500, 2.2000, 0.2077) -- (1.6500, 2.2500, 0.2034) -- (1.6000, 2.2500, 0.2042) -- cycle;
\fill[blue!22.1, opacity=0.7] (1.6000, 2.2500, 0.2042) -- (1.6500, 2.2500, 0.2034) -- (1.6500, 2.3000, 0.1988) -- (1.6000, 2.3000, 0.1996) -- cycle;
\fill[blue!15.0, opacity=0.7] (1.6000, 2.3000, 0.1996) -- (1.6500, 2.3000, 0.1988) -- (1.6500, 2.3500, 0.1940) -- (1.6000, 2.3500, 0.1949) -- cycle;
\fill[blue!15.0, opacity=0.7] (1.6000, 2.3500, 0.1949) -- (1.6500, 2.3500, 0.1940) -- (1.6500, 2.4000, 0.1891) -- (1.6000, 2.4000, 0.1899) -- cycle;
\fill[blue!15.0, opacity=0.7] (1.6000, 2.4000, 0.1899) -- (1.6500, 2.4000, 0.1891) -- (1.6500, 2.4500, 0.1839) -- (1.6000, 2.4500, 0.1847) -- cycle;
\fill[blue!15.0, opacity=0.7] (1.6000, 2.4500, 0.1847) -- (1.6500, 2.4500, 0.1839) -- (1.6500, 2.5000, 0.1785) -- (1.6000, 2.5000, 0.1793) -- cycle;
\fill[blue!15.0, opacity=0.7] (1.6000, 2.5000, 0.1793) -- (1.6500, 2.5000, 0.1785) -- (1.6500, 2.5500, 0.1730) -- (1.6000, 2.5500, 0.1738) -- cycle;
\fill[blue!15.0, opacity=0.7] (1.6000, 2.5500, 0.1738) -- (1.6500, 2.5500, 0.1730) -- (1.6500, 2.6000, 0.1673) -- (1.6000, 2.6000, 0.1682) -- cycle;
\fill[blue!15.0, opacity=0.7] (1.6000, 2.6000, 0.1682) -- (1.6500, 2.6000, 0.1673) -- (1.6500, 2.6500, 0.1615) -- (1.6000, 2.6500, 0.1623) -- cycle;
\fill[blue!15.0, opacity=0.7] (1.6000, 2.6500, 0.1623) -- (1.6500, 2.6500, 0.1615) -- (1.6500, 2.7000, 0.1556) -- (1.6000, 2.7000, 0.1564) -- cycle;
\fill[blue!15.0, opacity=0.7] (1.6000, 2.7000, 0.1564) -- (1.6500, 2.7000, 0.1556) -- (1.6500, 2.7500, 0.1496) -- (1.6000, 2.7500, 0.1504) -- cycle;
\fill[blue!15.0, opacity=0.7] (1.6000, 2.7500, 0.1504) -- (1.6500, 2.7500, 0.1496) -- (1.6500, 2.8000, 0.1435) -- (1.6000, 2.8000, 0.1443) -- cycle;
\fill[blue!15.0, opacity=0.7] (1.6000, 2.8000, 0.1443) -- (1.6500, 2.8000, 0.1435) -- (1.6500, 2.8500, 0.1373) -- (1.6000, 2.8500, 0.1381) -- cycle;
\fill[blue!15.0, opacity=0.7] (1.6000, 2.8500, 0.1381) -- (1.6500, 2.8500, 0.1373) -- (1.6500, 2.9000, 0.1311) -- (1.6000, 2.9000, 0.1319) -- cycle;
\fill[blue!15.0, opacity=0.7] (1.6000, 2.9000, 0.1319) -- (1.6500, 2.9000, 0.1311) -- (1.6500, 2.9500, 0.1248) -- (1.6000, 2.9500, 0.1256) -- cycle;
\fill[blue!15.0, opacity=0.7] (1.6000, 2.9500, 0.1256) -- (1.6500, 2.9500, 0.1248) -- (1.6500, 3.0000, 0.1185) -- (1.6000, 3.0000, 0.1193) -- cycle;
\fill[blue!15.0, opacity=0.7] (1.6500, 0.0000, 0.1185) -- (1.7000, 0.0000, 0.1174) -- (1.7000, 0.0500, 0.1237) -- (1.6500, 0.0500, 0.1248) -- cycle;
\fill[blue!15.0, opacity=0.7] (1.6500, 0.0500, 0.1248) -- (1.7000, 0.0500, 0.1237) -- (1.7000, 0.1000, 0.1299) -- (1.6500, 0.1000, 0.1311) -- cycle;
\fill[blue!15.0, opacity=0.7] (1.6500, 0.1000, 0.1311) -- (1.7000, 0.1000, 0.1299) -- (1.7000, 0.1500, 0.1361) -- (1.6500, 0.1500, 0.1373) -- cycle;
\fill[blue!15.0, opacity=0.7] (1.6500, 0.1500, 0.1373) -- (1.7000, 0.1500, 0.1361) -- (1.7000, 0.2000, 0.1423) -- (1.6500, 0.2000, 0.1435) -- cycle;
\fill[blue!15.0, opacity=0.7] (1.6500, 0.2000, 0.1435) -- (1.7000, 0.2000, 0.1423) -- (1.7000, 0.2500, 0.1484) -- (1.6500, 0.2500, 0.1496) -- cycle;
\fill[blue!15.0, opacity=0.7] (1.6500, 0.2500, 0.1496) -- (1.7000, 0.2500, 0.1484) -- (1.7000, 0.3000, 0.1545) -- (1.6500, 0.3000, 0.1556) -- cycle;
\fill[blue!15.0, opacity=0.7] (1.6500, 0.3000, 0.1556) -- (1.7000, 0.3000, 0.1545) -- (1.7000, 0.3500, 0.1604) -- (1.6500, 0.3500, 0.1615) -- cycle;
\fill[blue!15.0, opacity=0.7] (1.6500, 0.3500, 0.1615) -- (1.7000, 0.3500, 0.1604) -- (1.7000, 0.4000, 0.1662) -- (1.6500, 0.4000, 0.1673) -- cycle;
\fill[blue!15.0, opacity=0.7] (1.6500, 0.4000, 0.1673) -- (1.7000, 0.4000, 0.1662) -- (1.7000, 0.4500, 0.1719) -- (1.6500, 0.4500, 0.1730) -- cycle;
\fill[blue!15.0, opacity=0.7] (1.6500, 0.4500, 0.1730) -- (1.7000, 0.4500, 0.1719) -- (1.7000, 0.5000, 0.1774) -- (1.6500, 0.5000, 0.1785) -- cycle;
\fill[blue!15.0, opacity=0.7] (1.6500, 0.5000, 0.1785) -- (1.7000, 0.5000, 0.1774) -- (1.7000, 0.5500, 0.1827) -- (1.6500, 0.5500, 0.1839) -- cycle;
\fill[blue!15.0, opacity=0.7] (1.6500, 0.5500, 0.1839) -- (1.7000, 0.5500, 0.1827) -- (1.7000, 0.6000, 0.1879) -- (1.6500, 0.6000, 0.1891) -- cycle;
\fill[blue!15.0, opacity=0.7] (1.6500, 0.6000, 0.1891) -- (1.7000, 0.6000, 0.1879) -- (1.7000, 0.6500, 0.1929) -- (1.6500, 0.6500, 0.1940) -- cycle;
\fill[blue!15.0, opacity=0.7] (1.6500, 0.6500, 0.1940) -- (1.7000, 0.6500, 0.1929) -- (1.7000, 0.7000, 0.1977) -- (1.6500, 0.7000, 0.1988) -- cycle;
\fill[blue!15.0, opacity=0.7] (1.6500, 0.7000, 0.1988) -- (1.7000, 0.7000, 0.1977) -- (1.7000, 0.7500, 0.2022) -- (1.6500, 0.7500, 0.2034) -- cycle;
\fill[blue!20.4, opacity=0.7] (1.6500, 0.7500, 0.2034) -- (1.7000, 0.7500, 0.2022) -- (1.7000, 0.8000, 0.2066) -- (1.6500, 0.8000, 0.2077) -- cycle;
\fill[blue!60.8, opacity=0.7] (1.6500, 0.8000, 0.2077) -- (1.7000, 0.8000, 0.2066) -- (1.7000, 0.8500, 0.2106) -- (1.6500, 0.8500, 0.2118) -- cycle;
\fill[blue!34.3, opacity=0.7] (1.6500, 0.8500, 0.2118) -- (1.7000, 0.8500, 0.2106) -- (1.7000, 0.9000, 0.2145) -- (1.6500, 0.9000, 0.2156) -- cycle;
\fill[blue!15.2, opacity=0.7] (1.6500, 0.9000, 0.2156) -- (1.7000, 0.9000, 0.2145) -- (1.7000, 0.9500, 0.2180) -- (1.6500, 0.9500, 0.2192) -- cycle;
\fill[blue!15.0, opacity=0.7] (1.6500, 0.9500, 0.2192) -- (1.7000, 0.9500, 0.2180) -- (1.7000, 1.0000, 0.2213) -- (1.6500, 1.0000, 0.2224) -- cycle;
\fill[blue!15.0, opacity=0.7] (1.6500, 1.0000, 0.2224) -- (1.7000, 1.0000, 0.2213) -- (1.7000, 1.0500, 0.2243) -- (1.6500, 1.0500, 0.2254) -- cycle;
\fill[blue!15.0, opacity=0.7] (1.6500, 1.0500, 0.2254) -- (1.7000, 1.0500, 0.2243) -- (1.7000, 1.1000, 0.2270) -- (1.6500, 1.1000, 0.2281) -- cycle;
\fill[blue!15.0, opacity=0.7] (1.6500, 1.1000, 0.2281) -- (1.7000, 1.1000, 0.2270) -- (1.7000, 1.1500, 0.2294) -- (1.6500, 1.1500, 0.2306) -- cycle;
\fill[blue!15.3, opacity=0.7] (1.6500, 1.1500, 0.2306) -- (1.7000, 1.1500, 0.2294) -- (1.7000, 1.2000, 0.2315) -- (1.6500, 1.2000, 0.2326) -- cycle;
\fill[blue!60.9, opacity=0.7] (1.6500, 1.2000, 0.2326) -- (1.7000, 1.2000, 0.2315) -- (1.7000, 1.2500, 0.2333) -- (1.6500, 1.2500, 0.2344) -- cycle;
\fill[blue!12.7!black, opacity=0.7] (1.6500, 1.2500, 0.2344) -- (1.7000, 1.2500, 0.2333) -- (1.7000, 1.3000, 0.2348) -- (1.6500, 1.3000, 0.2359) -- cycle;
\fill[blue!72.3, opacity=0.7] (1.6500, 1.3000, 0.2359) -- (1.7000, 1.3000, 0.2348) -- (1.7000, 1.3500, 0.2359) -- (1.6500, 1.3500, 0.2370) -- cycle;
\fill[blue!42.9, opacity=0.7] (1.6500, 1.3500, 0.2370) -- (1.7000, 1.3500, 0.2359) -- (1.7000, 1.4000, 0.2367) -- (1.6500, 1.4000, 0.2379) -- cycle;
\fill[blue!39.1, opacity=0.7] (1.6500, 1.4000, 0.2379) -- (1.7000, 1.4000, 0.2367) -- (1.7000, 1.4500, 0.2372) -- (1.6500, 1.4500, 0.2384) -- cycle;
\fill[blue!72.6, opacity=0.7] (1.6500, 1.4500, 0.2384) -- (1.7000, 1.4500, 0.2372) -- (1.7000, 1.5000, 0.2374) -- (1.6500, 1.5000, 0.2385) -- cycle;
\fill[blue!46.0!black, opacity=0.7] (1.6500, 1.5000, 0.2385) -- (1.7000, 1.5000, 0.2374) -- (1.7000, 1.5500, 0.2372) -- (1.6500, 1.5500, 0.2384) -- cycle;
\fill[blue!15.9, opacity=0.7] (1.6500, 1.5500, 0.2384) -- (1.7000, 1.5500, 0.2372) -- (1.7000, 1.6000, 0.2367) -- (1.6500, 1.6000, 0.2379) -- cycle;
\fill[blue!15.0, opacity=0.7] (1.6500, 1.6000, 0.2379) -- (1.7000, 1.6000, 0.2367) -- (1.7000, 1.6500, 0.2359) -- (1.6500, 1.6500, 0.2370) -- cycle;
\fill[blue!15.0, opacity=0.7] (1.6500, 1.6500, 0.2370) -- (1.7000, 1.6500, 0.2359) -- (1.7000, 1.7000, 0.2348) -- (1.6500, 1.7000, 0.2359) -- cycle;
\fill[blue!18.4, opacity=0.7] (1.6500, 1.7000, 0.2359) -- (1.7000, 1.7000, 0.2348) -- (1.7000, 1.7500, 0.2333) -- (1.6500, 1.7500, 0.2344) -- cycle;
\fill[blue!16.1!black, opacity=0.7] (1.6500, 1.7500, 0.2344) -- (1.7000, 1.7500, 0.2333) -- (1.7000, 1.8000, 0.2315) -- (1.6500, 1.8000, 0.2326) -- cycle;
\fill[blue!97.6, opacity=0.7] (1.6500, 1.8000, 0.2326) -- (1.7000, 1.8000, 0.2315) -- (1.7000, 1.8500, 0.2294) -- (1.6500, 1.8500, 0.2306) -- cycle;
\fill[blue!5.4!black, opacity=0.7] (1.6500, 1.8500, 0.2306) -- (1.7000, 1.8500, 0.2294) -- (1.7000, 1.9000, 0.2270) -- (1.6500, 1.9000, 0.2281) -- cycle;
\fill[blue!18.6, opacity=0.7] (1.6500, 1.9000, 0.2281) -- (1.7000, 1.9000, 0.2270) -- (1.7000, 1.9500, 0.2243) -- (1.6500, 1.9500, 0.2254) -- cycle;
\fill[blue!15.0, opacity=0.7] (1.6500, 1.9500, 0.2254) -- (1.7000, 1.9500, 0.2243) -- (1.7000, 2.0000, 0.2213) -- (1.6500, 2.0000, 0.2224) -- cycle;
\fill[blue!15.0, opacity=0.7] (1.6500, 2.0000, 0.2224) -- (1.7000, 2.0000, 0.2213) -- (1.7000, 2.0500, 0.2180) -- (1.6500, 2.0500, 0.2192) -- cycle;
\fill[blue!15.0, opacity=0.7] (1.6500, 2.0500, 0.2192) -- (1.7000, 2.0500, 0.2180) -- (1.7000, 2.1000, 0.2145) -- (1.6500, 2.1000, 0.2156) -- cycle;
\fill[blue!15.0, opacity=0.7] (1.6500, 2.1000, 0.2156) -- (1.7000, 2.1000, 0.2145) -- (1.7000, 2.1500, 0.2106) -- (1.6500, 2.1500, 0.2118) -- cycle;
\fill[blue!15.8, opacity=0.7] (1.6500, 2.1500, 0.2118) -- (1.7000, 2.1500, 0.2106) -- (1.7000, 2.2000, 0.2066) -- (1.6500, 2.2000, 0.2077) -- cycle;
\fill[blue!35.8, opacity=0.7] (1.6500, 2.2000, 0.2077) -- (1.7000, 2.2000, 0.2066) -- (1.7000, 2.2500, 0.2022) -- (1.6500, 2.2500, 0.2034) -- cycle;
\fill[blue!20.6, opacity=0.7] (1.6500, 2.2500, 0.2034) -- (1.7000, 2.2500, 0.2022) -- (1.7000, 2.3000, 0.1977) -- (1.6500, 2.3000, 0.1988) -- cycle;
\fill[blue!15.0, opacity=0.7] (1.6500, 2.3000, 0.1988) -- (1.7000, 2.3000, 0.1977) -- (1.7000, 2.3500, 0.1929) -- (1.6500, 2.3500, 0.1940) -- cycle;
\fill[blue!15.0, opacity=0.7] (1.6500, 2.3500, 0.1940) -- (1.7000, 2.3500, 0.1929) -- (1.7000, 2.4000, 0.1879) -- (1.6500, 2.4000, 0.1891) -- cycle;
\fill[blue!15.0, opacity=0.7] (1.6500, 2.4000, 0.1891) -- (1.7000, 2.4000, 0.1879) -- (1.7000, 2.4500, 0.1827) -- (1.6500, 2.4500, 0.1839) -- cycle;
\fill[blue!15.0, opacity=0.7] (1.6500, 2.4500, 0.1839) -- (1.7000, 2.4500, 0.1827) -- (1.7000, 2.5000, 0.1774) -- (1.6500, 2.5000, 0.1785) -- cycle;
\fill[blue!15.0, opacity=0.7] (1.6500, 2.5000, 0.1785) -- (1.7000, 2.5000, 0.1774) -- (1.7000, 2.5500, 0.1719) -- (1.6500, 2.5500, 0.1730) -- cycle;
\fill[blue!15.0, opacity=0.7] (1.6500, 2.5500, 0.1730) -- (1.7000, 2.5500, 0.1719) -- (1.7000, 2.6000, 0.1662) -- (1.6500, 2.6000, 0.1673) -- cycle;
\fill[blue!15.0, opacity=0.7] (1.6500, 2.6000, 0.1673) -- (1.7000, 2.6000, 0.1662) -- (1.7000, 2.6500, 0.1604) -- (1.6500, 2.6500, 0.1615) -- cycle;
\fill[blue!15.0, opacity=0.7] (1.6500, 2.6500, 0.1615) -- (1.7000, 2.6500, 0.1604) -- (1.7000, 2.7000, 0.1545) -- (1.6500, 2.7000, 0.1556) -- cycle;
\fill[blue!15.0, opacity=0.7] (1.6500, 2.7000, 0.1556) -- (1.7000, 2.7000, 0.1545) -- (1.7000, 2.7500, 0.1484) -- (1.6500, 2.7500, 0.1496) -- cycle;
\fill[blue!15.0, opacity=0.7] (1.6500, 2.7500, 0.1496) -- (1.7000, 2.7500, 0.1484) -- (1.7000, 2.8000, 0.1423) -- (1.6500, 2.8000, 0.1435) -- cycle;
\fill[blue!15.0, opacity=0.7] (1.6500, 2.8000, 0.1435) -- (1.7000, 2.8000, 0.1423) -- (1.7000, 2.8500, 0.1361) -- (1.6500, 2.8500, 0.1373) -- cycle;
\fill[blue!15.0, opacity=0.7] (1.6500, 2.8500, 0.1373) -- (1.7000, 2.8500, 0.1361) -- (1.7000, 2.9000, 0.1299) -- (1.6500, 2.9000, 0.1311) -- cycle;
\fill[blue!15.0, opacity=0.7] (1.6500, 2.9000, 0.1311) -- (1.7000, 2.9000, 0.1299) -- (1.7000, 2.9500, 0.1237) -- (1.6500, 2.9500, 0.1248) -- cycle;
\fill[blue!15.0, opacity=0.7] (1.6500, 2.9500, 0.1248) -- (1.7000, 2.9500, 0.1237) -- (1.7000, 3.0000, 0.1174) -- (1.6500, 3.0000, 0.1185) -- cycle;
\fill[blue!15.0, opacity=0.7] (1.7000, 0.0000, 0.1174) -- (1.7500, 0.0000, 0.1159) -- (1.7500, 0.0500, 0.1222) -- (1.7000, 0.0500, 0.1237) -- cycle;
\fill[blue!15.0, opacity=0.7] (1.7000, 0.0500, 0.1237) -- (1.7500, 0.0500, 0.1222) -- (1.7500, 0.1000, 0.1285) -- (1.7000, 0.1000, 0.1299) -- cycle;
\fill[blue!15.0, opacity=0.7] (1.7000, 0.1000, 0.1299) -- (1.7500, 0.1000, 0.1285) -- (1.7500, 0.1500, 0.1347) -- (1.7000, 0.1500, 0.1361) -- cycle;
\fill[blue!15.0, opacity=0.7] (1.7000, 0.1500, 0.1361) -- (1.7500, 0.1500, 0.1347) -- (1.7500, 0.2000, 0.1409) -- (1.7000, 0.2000, 0.1423) -- cycle;
\fill[blue!15.0, opacity=0.7] (1.7000, 0.2000, 0.1423) -- (1.7500, 0.2000, 0.1409) -- (1.7500, 0.2500, 0.1470) -- (1.7000, 0.2500, 0.1484) -- cycle;
\fill[blue!15.0, opacity=0.7] (1.7000, 0.2500, 0.1484) -- (1.7500, 0.2500, 0.1470) -- (1.7500, 0.3000, 0.1530) -- (1.7000, 0.3000, 0.1545) -- cycle;
\fill[blue!15.0, opacity=0.7] (1.7000, 0.3000, 0.1545) -- (1.7500, 0.3000, 0.1530) -- (1.7500, 0.3500, 0.1589) -- (1.7000, 0.3500, 0.1604) -- cycle;
\fill[blue!15.0, opacity=0.7] (1.7000, 0.3500, 0.1604) -- (1.7500, 0.3500, 0.1589) -- (1.7500, 0.4000, 0.1647) -- (1.7000, 0.4000, 0.1662) -- cycle;
\fill[blue!15.0, opacity=0.7] (1.7000, 0.4000, 0.1662) -- (1.7500, 0.4000, 0.1647) -- (1.7500, 0.4500, 0.1704) -- (1.7000, 0.4500, 0.1719) -- cycle;
\fill[blue!15.0, opacity=0.7] (1.7000, 0.4500, 0.1719) -- (1.7500, 0.4500, 0.1704) -- (1.7500, 0.5000, 0.1759) -- (1.7000, 0.5000, 0.1774) -- cycle;
\fill[blue!15.0, opacity=0.7] (1.7000, 0.5000, 0.1774) -- (1.7500, 0.5000, 0.1759) -- (1.7500, 0.5500, 0.1813) -- (1.7000, 0.5500, 0.1827) -- cycle;
\fill[blue!15.0, opacity=0.7] (1.7000, 0.5500, 0.1827) -- (1.7500, 0.5500, 0.1813) -- (1.7500, 0.6000, 0.1864) -- (1.7000, 0.6000, 0.1879) -- cycle;
\fill[blue!15.0, opacity=0.7] (1.7000, 0.6000, 0.1879) -- (1.7500, 0.6000, 0.1864) -- (1.7500, 0.6500, 0.1914) -- (1.7000, 0.6500, 0.1929) -- cycle;
\fill[blue!15.0, opacity=0.7] (1.7000, 0.6500, 0.1929) -- (1.7500, 0.6500, 0.1914) -- (1.7500, 0.7000, 0.1962) -- (1.7000, 0.7000, 0.1977) -- cycle;
\fill[blue!15.0, opacity=0.7] (1.7000, 0.7000, 0.1977) -- (1.7500, 0.7000, 0.1962) -- (1.7500, 0.7500, 0.2008) -- (1.7000, 0.7500, 0.2022) -- cycle;
\fill[blue!15.3, opacity=0.7] (1.7000, 0.7500, 0.2022) -- (1.7500, 0.7500, 0.2008) -- (1.7500, 0.8000, 0.2051) -- (1.7000, 0.8000, 0.2066) -- cycle;
\fill[blue!44.8, opacity=0.7] (1.7000, 0.8000, 0.2066) -- (1.7500, 0.8000, 0.2051) -- (1.7500, 0.8500, 0.2092) -- (1.7000, 0.8500, 0.2106) -- cycle;
\fill[blue!62.9, opacity=0.7] (1.7000, 0.8500, 0.2106) -- (1.7500, 0.8500, 0.2092) -- (1.7500, 0.9000, 0.2130) -- (1.7000, 0.9000, 0.2145) -- cycle;
\fill[blue!21.5, opacity=0.7] (1.7000, 0.9000, 0.2145) -- (1.7500, 0.9000, 0.2130) -- (1.7500, 0.9500, 0.2166) -- (1.7000, 0.9500, 0.2180) -- cycle;
\fill[blue!15.0, opacity=0.7] (1.7000, 0.9500, 0.2180) -- (1.7500, 0.9500, 0.2166) -- (1.7500, 1.0000, 0.2198) -- (1.7000, 1.0000, 0.2213) -- cycle;
\fill[blue!15.0, opacity=0.7] (1.7000, 1.0000, 0.2213) -- (1.7500, 1.0000, 0.2198) -- (1.7500, 1.0500, 0.2228) -- (1.7000, 1.0500, 0.2243) -- cycle;
\fill[blue!15.0, opacity=0.7] (1.7000, 1.0500, 0.2243) -- (1.7500, 1.0500, 0.2228) -- (1.7500, 1.1000, 0.2255) -- (1.7000, 1.1000, 0.2270) -- cycle;
\fill[blue!15.0, opacity=0.7] (1.7000, 1.1000, 0.2270) -- (1.7500, 1.1000, 0.2255) -- (1.7500, 1.1500, 0.2279) -- (1.7000, 1.1500, 0.2294) -- cycle;
\fill[blue!15.0, opacity=0.7] (1.7000, 1.1500, 0.2294) -- (1.7500, 1.1500, 0.2279) -- (1.7500, 1.2000, 0.2300) -- (1.7000, 1.2000, 0.2315) -- cycle;
\fill[blue!15.1, opacity=0.7] (1.7000, 1.2000, 0.2315) -- (1.7500, 1.2000, 0.2300) -- (1.7500, 1.2500, 0.2318) -- (1.7000, 1.2500, 0.2333) -- cycle;
\fill[blue!20.8, opacity=0.7] (1.7000, 1.2500, 0.2333) -- (1.7500, 1.2500, 0.2318) -- (1.7500, 1.3000, 0.2333) -- (1.7000, 1.3000, 0.2348) -- cycle;
\fill[blue!63.5, opacity=0.7] (1.7000, 1.3000, 0.2348) -- (1.7500, 1.3000, 0.2333) -- (1.7500, 1.3500, 0.2344) -- (1.7000, 1.3500, 0.2359) -- cycle;
\fill[blue!90.3!black, opacity=0.7] (1.7000, 1.3500, 0.2359) -- (1.7500, 1.3500, 0.2344) -- (1.7500, 1.4000, 0.2353) -- (1.7000, 1.4000, 0.2367) -- cycle;
\fill[blue!97.5, opacity=0.7] (1.7000, 1.4000, 0.2367) -- (1.7500, 1.4000, 0.2353) -- (1.7500, 1.4500, 0.2357) -- (1.7000, 1.4500, 0.2372) -- cycle;
\fill[blue!42.8, opacity=0.7] (1.7000, 1.4500, 0.2372) -- (1.7500, 1.4500, 0.2357) -- (1.7500, 1.5000, 0.2359) -- (1.7000, 1.5000, 0.2374) -- cycle;
\fill[blue!15.7, opacity=0.7] (1.7000, 1.5000, 0.2374) -- (1.7500, 1.5000, 0.2359) -- (1.7500, 1.5500, 0.2357) -- (1.7000, 1.5500, 0.2372) -- cycle;
\fill[blue!15.0, opacity=0.7] (1.7000, 1.5500, 0.2372) -- (1.7500, 1.5500, 0.2357) -- (1.7500, 1.6000, 0.2353) -- (1.7000, 1.6000, 0.2367) -- cycle;
\fill[blue!15.0, opacity=0.7] (1.7000, 1.6000, 0.2367) -- (1.7500, 1.6000, 0.2353) -- (1.7500, 1.6500, 0.2344) -- (1.7000, 1.6500, 0.2359) -- cycle;
\fill[blue!15.0, opacity=0.7] (1.7000, 1.6500, 0.2359) -- (1.7500, 1.6500, 0.2344) -- (1.7500, 1.7000, 0.2333) -- (1.7000, 1.7000, 0.2348) -- cycle;
\fill[blue!30.3, opacity=0.7] (1.7000, 1.7000, 0.2348) -- (1.7500, 1.7000, 0.2333) -- (1.7500, 1.7500, 0.2318) -- (1.7000, 1.7500, 0.2333) -- cycle;
\fill[blue!5.7!black, opacity=0.7] (1.7000, 1.7500, 0.2333) -- (1.7500, 1.7500, 0.2318) -- (1.7500, 1.8000, 0.2300) -- (1.7000, 1.8000, 0.2315) -- cycle;
\fill[blue!80.9!black, opacity=0.7] (1.7000, 1.8000, 0.2315) -- (1.7500, 1.8000, 0.2300) -- (1.7500, 1.8500, 0.2279) -- (1.7000, 1.8500, 0.2294) -- cycle;
\fill[blue!48.8!black, opacity=0.7] (1.7000, 1.8500, 0.2294) -- (1.7500, 1.8500, 0.2279) -- (1.7500, 1.9000, 0.2255) -- (1.7000, 1.9000, 0.2270) -- cycle;
\fill[blue!15.6, opacity=0.7] (1.7000, 1.9000, 0.2270) -- (1.7500, 1.9000, 0.2255) -- (1.7500, 1.9500, 0.2228) -- (1.7000, 1.9500, 0.2243) -- cycle;
\fill[blue!15.0, opacity=0.7] (1.7000, 1.9500, 0.2243) -- (1.7500, 1.9500, 0.2228) -- (1.7500, 2.0000, 0.2198) -- (1.7000, 2.0000, 0.2213) -- cycle;
\fill[blue!15.0, opacity=0.7] (1.7000, 2.0000, 0.2213) -- (1.7500, 2.0000, 0.2198) -- (1.7500, 2.0500, 0.2166) -- (1.7000, 2.0500, 0.2180) -- cycle;
\fill[blue!15.0, opacity=0.7] (1.7000, 2.0500, 0.2180) -- (1.7500, 2.0500, 0.2166) -- (1.7500, 2.1000, 0.2130) -- (1.7000, 2.1000, 0.2145) -- cycle;
\fill[blue!15.0, opacity=0.7] (1.7000, 2.1000, 0.2145) -- (1.7500, 2.1000, 0.2130) -- (1.7500, 2.1500, 0.2092) -- (1.7000, 2.1500, 0.2106) -- cycle;
\fill[blue!16.3, opacity=0.7] (1.7000, 2.1500, 0.2106) -- (1.7500, 2.1500, 0.2092) -- (1.7500, 2.2000, 0.2051) -- (1.7000, 2.2000, 0.2066) -- cycle;
\fill[blue!34.6, opacity=0.7] (1.7000, 2.2000, 0.2066) -- (1.7500, 2.2000, 0.2051) -- (1.7500, 2.2500, 0.2008) -- (1.7000, 2.2500, 0.2022) -- cycle;
\fill[blue!18.3, opacity=0.7] (1.7000, 2.2500, 0.2022) -- (1.7500, 2.2500, 0.2008) -- (1.7500, 2.3000, 0.1962) -- (1.7000, 2.3000, 0.1977) -- cycle;
\fill[blue!15.0, opacity=0.7] (1.7000, 2.3000, 0.1977) -- (1.7500, 2.3000, 0.1962) -- (1.7500, 2.3500, 0.1914) -- (1.7000, 2.3500, 0.1929) -- cycle;
\fill[blue!15.0, opacity=0.7] (1.7000, 2.3500, 0.1929) -- (1.7500, 2.3500, 0.1914) -- (1.7500, 2.4000, 0.1864) -- (1.7000, 2.4000, 0.1879) -- cycle;
\fill[blue!15.0, opacity=0.7] (1.7000, 2.4000, 0.1879) -- (1.7500, 2.4000, 0.1864) -- (1.7500, 2.4500, 0.1813) -- (1.7000, 2.4500, 0.1827) -- cycle;
\fill[blue!15.0, opacity=0.7] (1.7000, 2.4500, 0.1827) -- (1.7500, 2.4500, 0.1813) -- (1.7500, 2.5000, 0.1759) -- (1.7000, 2.5000, 0.1774) -- cycle;
\fill[blue!15.0, opacity=0.7] (1.7000, 2.5000, 0.1774) -- (1.7500, 2.5000, 0.1759) -- (1.7500, 2.5500, 0.1704) -- (1.7000, 2.5500, 0.1719) -- cycle;
\fill[blue!15.0, opacity=0.7] (1.7000, 2.5500, 0.1719) -- (1.7500, 2.5500, 0.1704) -- (1.7500, 2.6000, 0.1647) -- (1.7000, 2.6000, 0.1662) -- cycle;
\fill[blue!15.0, opacity=0.7] (1.7000, 2.6000, 0.1662) -- (1.7500, 2.6000, 0.1647) -- (1.7500, 2.6500, 0.1589) -- (1.7000, 2.6500, 0.1604) -- cycle;
\fill[blue!15.0, opacity=0.7] (1.7000, 2.6500, 0.1604) -- (1.7500, 2.6500, 0.1589) -- (1.7500, 2.7000, 0.1530) -- (1.7000, 2.7000, 0.1545) -- cycle;
\fill[blue!15.0, opacity=0.7] (1.7000, 2.7000, 0.1545) -- (1.7500, 2.7000, 0.1530) -- (1.7500, 2.7500, 0.1470) -- (1.7000, 2.7500, 0.1484) -- cycle;
\fill[blue!15.0, opacity=0.7] (1.7000, 2.7500, 0.1484) -- (1.7500, 2.7500, 0.1470) -- (1.7500, 2.8000, 0.1409) -- (1.7000, 2.8000, 0.1423) -- cycle;
\fill[blue!15.0, opacity=0.7] (1.7000, 2.8000, 0.1423) -- (1.7500, 2.8000, 0.1409) -- (1.7500, 2.8500, 0.1347) -- (1.7000, 2.8500, 0.1361) -- cycle;
\fill[blue!15.0, opacity=0.7] (1.7000, 2.8500, 0.1361) -- (1.7500, 2.8500, 0.1347) -- (1.7500, 2.9000, 0.1285) -- (1.7000, 2.9000, 0.1299) -- cycle;
\fill[blue!15.0, opacity=0.7] (1.7000, 2.9000, 0.1299) -- (1.7500, 2.9000, 0.1285) -- (1.7500, 2.9500, 0.1222) -- (1.7000, 2.9500, 0.1237) -- cycle;
\fill[blue!15.0, opacity=0.7] (1.7000, 2.9500, 0.1237) -- (1.7500, 2.9500, 0.1222) -- (1.7500, 3.0000, 0.1159) -- (1.7000, 3.0000, 0.1174) -- cycle;
\fill[blue!15.0, opacity=0.7] (1.7500, 0.0000, 0.1159) -- (1.8000, 0.0000, 0.1141) -- (1.8000, 0.0500, 0.1204) -- (1.7500, 0.0500, 0.1222) -- cycle;
\fill[blue!15.0, opacity=0.7] (1.7500, 0.0500, 0.1222) -- (1.8000, 0.0500, 0.1204) -- (1.8000, 0.1000, 0.1267) -- (1.7500, 0.1000, 0.1285) -- cycle;
\fill[blue!15.0, opacity=0.7] (1.7500, 0.1000, 0.1285) -- (1.8000, 0.1000, 0.1267) -- (1.8000, 0.1500, 0.1329) -- (1.7500, 0.1500, 0.1347) -- cycle;
\fill[blue!15.0, opacity=0.7] (1.7500, 0.1500, 0.1347) -- (1.8000, 0.1500, 0.1329) -- (1.8000, 0.2000, 0.1391) -- (1.7500, 0.2000, 0.1409) -- cycle;
\fill[blue!15.0, opacity=0.7] (1.7500, 0.2000, 0.1409) -- (1.8000, 0.2000, 0.1391) -- (1.8000, 0.2500, 0.1452) -- (1.7500, 0.2500, 0.1470) -- cycle;
\fill[blue!15.0, opacity=0.7] (1.7500, 0.2500, 0.1470) -- (1.8000, 0.2500, 0.1452) -- (1.8000, 0.3000, 0.1512) -- (1.7500, 0.3000, 0.1530) -- cycle;
\fill[blue!15.0, opacity=0.7] (1.7500, 0.3000, 0.1530) -- (1.8000, 0.3000, 0.1512) -- (1.8000, 0.3500, 0.1571) -- (1.7500, 0.3500, 0.1589) -- cycle;
\fill[blue!15.0, opacity=0.7] (1.7500, 0.3500, 0.1589) -- (1.8000, 0.3500, 0.1571) -- (1.8000, 0.4000, 0.1629) -- (1.7500, 0.4000, 0.1647) -- cycle;
\fill[blue!15.0, opacity=0.7] (1.7500, 0.4000, 0.1647) -- (1.8000, 0.4000, 0.1629) -- (1.8000, 0.4500, 0.1686) -- (1.7500, 0.4500, 0.1704) -- cycle;
\fill[blue!15.0, opacity=0.7] (1.7500, 0.4500, 0.1704) -- (1.8000, 0.4500, 0.1686) -- (1.8000, 0.5000, 0.1741) -- (1.7500, 0.5000, 0.1759) -- cycle;
\fill[blue!15.0, opacity=0.7] (1.7500, 0.5000, 0.1759) -- (1.8000, 0.5000, 0.1741) -- (1.8000, 0.5500, 0.1795) -- (1.7500, 0.5500, 0.1813) -- cycle;
\fill[blue!15.0, opacity=0.7] (1.7500, 0.5500, 0.1813) -- (1.8000, 0.5500, 0.1795) -- (1.8000, 0.6000, 0.1847) -- (1.7500, 0.6000, 0.1864) -- cycle;
\fill[blue!15.0, opacity=0.7] (1.7500, 0.6000, 0.1864) -- (1.8000, 0.6000, 0.1847) -- (1.8000, 0.6500, 0.1896) -- (1.7500, 0.6500, 0.1914) -- cycle;
\fill[blue!15.0, opacity=0.7] (1.7500, 0.6500, 0.1914) -- (1.8000, 0.6500, 0.1896) -- (1.8000, 0.7000, 0.1944) -- (1.7500, 0.7000, 0.1962) -- cycle;
\fill[blue!15.0, opacity=0.7] (1.7500, 0.7000, 0.1962) -- (1.8000, 0.7000, 0.1944) -- (1.8000, 0.7500, 0.1990) -- (1.7500, 0.7500, 0.2008) -- cycle;
\fill[blue!15.0, opacity=0.7] (1.7500, 0.7500, 0.2008) -- (1.8000, 0.7500, 0.1990) -- (1.8000, 0.8000, 0.2033) -- (1.7500, 0.8000, 0.2051) -- cycle;
\fill[blue!20.7, opacity=0.7] (1.7500, 0.8000, 0.2051) -- (1.8000, 0.8000, 0.2033) -- (1.8000, 0.8500, 0.2074) -- (1.7500, 0.8500, 0.2092) -- cycle;
\fill[blue!67.0, opacity=0.7] (1.7500, 0.8500, 0.2092) -- (1.8000, 0.8500, 0.2074) -- (1.8000, 0.9000, 0.2112) -- (1.7500, 0.9000, 0.2130) -- cycle;
\fill[blue!55.7, opacity=0.7] (1.7500, 0.9000, 0.2130) -- (1.8000, 0.9000, 0.2112) -- (1.8000, 0.9500, 0.2148) -- (1.7500, 0.9500, 0.2166) -- cycle;
\fill[blue!18.0, opacity=0.7] (1.7500, 0.9500, 0.2166) -- (1.8000, 0.9500, 0.2148) -- (1.8000, 1.0000, 0.2180) -- (1.7500, 1.0000, 0.2198) -- cycle;
\fill[blue!15.0, opacity=0.7] (1.7500, 1.0000, 0.2198) -- (1.8000, 1.0000, 0.2180) -- (1.8000, 1.0500, 0.2210) -- (1.7500, 1.0500, 0.2228) -- cycle;
\fill[blue!15.0, opacity=0.7] (1.7500, 1.0500, 0.2228) -- (1.8000, 1.0500, 0.2210) -- (1.8000, 1.1000, 0.2238) -- (1.7500, 1.1000, 0.2255) -- cycle;
\fill[blue!15.0, opacity=0.7] (1.7500, 1.1000, 0.2255) -- (1.8000, 1.1000, 0.2238) -- (1.8000, 1.1500, 0.2262) -- (1.7500, 1.1500, 0.2279) -- cycle;
\fill[blue!15.0, opacity=0.7] (1.7500, 1.1500, 0.2279) -- (1.8000, 1.1500, 0.2262) -- (1.8000, 1.2000, 0.2283) -- (1.7500, 1.2000, 0.2300) -- cycle;
\fill[blue!15.0, opacity=0.7] (1.7500, 1.2000, 0.2300) -- (1.8000, 1.2000, 0.2283) -- (1.8000, 1.2500, 0.2300) -- (1.7500, 1.2500, 0.2318) -- cycle;
\fill[blue!15.0, opacity=0.7] (1.7500, 1.2500, 0.2318) -- (1.8000, 1.2500, 0.2300) -- (1.8000, 1.3000, 0.2315) -- (1.7500, 1.3000, 0.2333) -- cycle;
\fill[blue!15.0, opacity=0.7] (1.7500, 1.3000, 0.2333) -- (1.8000, 1.3000, 0.2315) -- (1.8000, 1.3500, 0.2326) -- (1.7500, 1.3500, 0.2344) -- cycle;
\fill[blue!15.1, opacity=0.7] (1.7500, 1.3500, 0.2344) -- (1.8000, 1.3500, 0.2326) -- (1.8000, 1.4000, 0.2335) -- (1.7500, 1.4000, 0.2353) -- cycle;
\fill[blue!15.0, opacity=0.7] (1.7500, 1.4000, 0.2353) -- (1.8000, 1.4000, 0.2335) -- (1.8000, 1.4500, 0.2340) -- (1.7500, 1.4500, 0.2357) -- cycle;
\fill[blue!15.0, opacity=0.7] (1.7500, 1.4500, 0.2357) -- (1.8000, 1.4500, 0.2340) -- (1.8000, 1.5000, 0.2341) -- (1.7500, 1.5000, 0.2359) -- cycle;
\fill[blue!15.0, opacity=0.7] (1.7500, 1.5000, 0.2359) -- (1.8000, 1.5000, 0.2341) -- (1.8000, 1.5500, 0.2340) -- (1.7500, 1.5500, 0.2357) -- cycle;
\fill[blue!15.0, opacity=0.7] (1.7500, 1.5500, 0.2357) -- (1.8000, 1.5500, 0.2340) -- (1.8000, 1.6000, 0.2335) -- (1.7500, 1.6000, 0.2353) -- cycle;
\fill[blue!15.0, opacity=0.7] (1.7500, 1.6000, 0.2353) -- (1.8000, 1.6000, 0.2335) -- (1.8000, 1.6500, 0.2326) -- (1.7500, 1.6500, 0.2344) -- cycle;
\fill[blue!15.7, opacity=0.7] (1.7500, 1.6500, 0.2344) -- (1.8000, 1.6500, 0.2326) -- (1.8000, 1.7000, 0.2315) -- (1.7500, 1.7000, 0.2333) -- cycle;
\fill[blue!83.0, opacity=0.7] (1.7500, 1.7000, 0.2333) -- (1.8000, 1.7000, 0.2315) -- (1.8000, 1.7500, 0.2300) -- (1.7500, 1.7500, 0.2318) -- cycle;
\fill[blue!33.5!black, opacity=0.7] (1.7500, 1.7500, 0.2318) -- (1.8000, 1.7500, 0.2300) -- (1.8000, 1.8000, 0.2283) -- (1.7500, 1.8000, 0.2300) -- cycle;
\fill[blue!29.1!black, opacity=0.7] (1.7500, 1.8000, 0.2300) -- (1.8000, 1.8000, 0.2283) -- (1.8000, 1.8500, 0.2262) -- (1.7500, 1.8500, 0.2279) -- cycle;
\fill[blue!62.5, opacity=0.7] (1.7500, 1.8500, 0.2279) -- (1.8000, 1.8500, 0.2262) -- (1.8000, 1.9000, 0.2238) -- (1.7500, 1.9000, 0.2255) -- cycle;
\fill[blue!15.0, opacity=0.7] (1.7500, 1.9000, 0.2255) -- (1.8000, 1.9000, 0.2238) -- (1.8000, 1.9500, 0.2210) -- (1.7500, 1.9500, 0.2228) -- cycle;
\fill[blue!15.0, opacity=0.7] (1.7500, 1.9500, 0.2228) -- (1.8000, 1.9500, 0.2210) -- (1.8000, 2.0000, 0.2180) -- (1.7500, 2.0000, 0.2198) -- cycle;
\fill[blue!15.0, opacity=0.7] (1.7500, 2.0000, 0.2198) -- (1.8000, 2.0000, 0.2180) -- (1.8000, 2.0500, 0.2148) -- (1.7500, 2.0500, 0.2166) -- cycle;
\fill[blue!15.0, opacity=0.7] (1.7500, 2.0500, 0.2166) -- (1.8000, 2.0500, 0.2148) -- (1.8000, 2.1000, 0.2112) -- (1.7500, 2.1000, 0.2130) -- cycle;
\fill[blue!15.0, opacity=0.7] (1.7500, 2.1000, 0.2130) -- (1.8000, 2.1000, 0.2112) -- (1.8000, 2.1500, 0.2074) -- (1.7500, 2.1500, 0.2092) -- cycle;
\fill[blue!17.5, opacity=0.7] (1.7500, 2.1500, 0.2092) -- (1.8000, 2.1500, 0.2074) -- (1.8000, 2.2000, 0.2033) -- (1.7500, 2.2000, 0.2051) -- cycle;
\fill[blue!32.7, opacity=0.7] (1.7500, 2.2000, 0.2051) -- (1.8000, 2.2000, 0.2033) -- (1.8000, 2.2500, 0.1990) -- (1.7500, 2.2500, 0.2008) -- cycle;
\fill[blue!16.3, opacity=0.7] (1.7500, 2.2500, 0.2008) -- (1.8000, 2.2500, 0.1990) -- (1.8000, 2.3000, 0.1944) -- (1.7500, 2.3000, 0.1962) -- cycle;
\fill[blue!15.0, opacity=0.7] (1.7500, 2.3000, 0.1962) -- (1.8000, 2.3000, 0.1944) -- (1.8000, 2.3500, 0.1896) -- (1.7500, 2.3500, 0.1914) -- cycle;
\fill[blue!15.0, opacity=0.7] (1.7500, 2.3500, 0.1914) -- (1.8000, 2.3500, 0.1896) -- (1.8000, 2.4000, 0.1847) -- (1.7500, 2.4000, 0.1864) -- cycle;
\fill[blue!15.0, opacity=0.7] (1.7500, 2.4000, 0.1864) -- (1.8000, 2.4000, 0.1847) -- (1.8000, 2.4500, 0.1795) -- (1.7500, 2.4500, 0.1813) -- cycle;
\fill[blue!15.0, opacity=0.7] (1.7500, 2.4500, 0.1813) -- (1.8000, 2.4500, 0.1795) -- (1.8000, 2.5000, 0.1741) -- (1.7500, 2.5000, 0.1759) -- cycle;
\fill[blue!15.0, opacity=0.7] (1.7500, 2.5000, 0.1759) -- (1.8000, 2.5000, 0.1741) -- (1.8000, 2.5500, 0.1686) -- (1.7500, 2.5500, 0.1704) -- cycle;
\fill[blue!15.0, opacity=0.7] (1.7500, 2.5500, 0.1704) -- (1.8000, 2.5500, 0.1686) -- (1.8000, 2.6000, 0.1629) -- (1.7500, 2.6000, 0.1647) -- cycle;
\fill[blue!15.0, opacity=0.7] (1.7500, 2.6000, 0.1647) -- (1.8000, 2.6000, 0.1629) -- (1.8000, 2.6500, 0.1571) -- (1.7500, 2.6500, 0.1589) -- cycle;
\fill[blue!15.0, opacity=0.7] (1.7500, 2.6500, 0.1589) -- (1.8000, 2.6500, 0.1571) -- (1.8000, 2.7000, 0.1512) -- (1.7500, 2.7000, 0.1530) -- cycle;
\fill[blue!15.0, opacity=0.7] (1.7500, 2.7000, 0.1530) -- (1.8000, 2.7000, 0.1512) -- (1.8000, 2.7500, 0.1452) -- (1.7500, 2.7500, 0.1470) -- cycle;
\fill[blue!15.0, opacity=0.7] (1.7500, 2.7500, 0.1470) -- (1.8000, 2.7500, 0.1452) -- (1.8000, 2.8000, 0.1391) -- (1.7500, 2.8000, 0.1409) -- cycle;
\fill[blue!15.0, opacity=0.7] (1.7500, 2.8000, 0.1409) -- (1.8000, 2.8000, 0.1391) -- (1.8000, 2.8500, 0.1329) -- (1.7500, 2.8500, 0.1347) -- cycle;
\fill[blue!15.0, opacity=0.7] (1.7500, 2.8500, 0.1347) -- (1.8000, 2.8500, 0.1329) -- (1.8000, 2.9000, 0.1267) -- (1.7500, 2.9000, 0.1285) -- cycle;
\fill[blue!15.0, opacity=0.7] (1.7500, 2.9000, 0.1285) -- (1.8000, 2.9000, 0.1267) -- (1.8000, 2.9500, 0.1204) -- (1.7500, 2.9500, 0.1222) -- cycle;
\fill[blue!15.0, opacity=0.7] (1.7500, 2.9500, 0.1222) -- (1.8000, 2.9500, 0.1204) -- (1.8000, 3.0000, 0.1141) -- (1.7500, 3.0000, 0.1159) -- cycle;
\fill[blue!15.0, opacity=0.7] (1.8000, 0.0000, 0.1141) -- (1.8500, 0.0000, 0.1120) -- (1.8500, 0.0500, 0.1183) -- (1.8000, 0.0500, 0.1204) -- cycle;
\fill[blue!15.0, opacity=0.7] (1.8000, 0.0500, 0.1204) -- (1.8500, 0.0500, 0.1183) -- (1.8500, 0.1000, 0.1246) -- (1.8000, 0.1000, 0.1267) -- cycle;
\fill[blue!15.0, opacity=0.7] (1.8000, 0.1000, 0.1267) -- (1.8500, 0.1000, 0.1246) -- (1.8500, 0.1500, 0.1308) -- (1.8000, 0.1500, 0.1329) -- cycle;
\fill[blue!15.0, opacity=0.7] (1.8000, 0.1500, 0.1329) -- (1.8500, 0.1500, 0.1308) -- (1.8500, 0.2000, 0.1370) -- (1.8000, 0.2000, 0.1391) -- cycle;
\fill[blue!15.0, opacity=0.7] (1.8000, 0.2000, 0.1391) -- (1.8500, 0.2000, 0.1370) -- (1.8500, 0.2500, 0.1431) -- (1.8000, 0.2500, 0.1452) -- cycle;
\fill[blue!15.0, opacity=0.7] (1.8000, 0.2500, 0.1452) -- (1.8500, 0.2500, 0.1431) -- (1.8500, 0.3000, 0.1491) -- (1.8000, 0.3000, 0.1512) -- cycle;
\fill[blue!15.0, opacity=0.7] (1.8000, 0.3000, 0.1512) -- (1.8500, 0.3000, 0.1491) -- (1.8500, 0.3500, 0.1550) -- (1.8000, 0.3500, 0.1571) -- cycle;
\fill[blue!15.0, opacity=0.7] (1.8000, 0.3500, 0.1571) -- (1.8500, 0.3500, 0.1550) -- (1.8500, 0.4000, 0.1608) -- (1.8000, 0.4000, 0.1629) -- cycle;
\fill[blue!15.0, opacity=0.7] (1.8000, 0.4000, 0.1629) -- (1.8500, 0.4000, 0.1608) -- (1.8500, 0.4500, 0.1665) -- (1.8000, 0.4500, 0.1686) -- cycle;
\fill[blue!15.0, opacity=0.7] (1.8000, 0.4500, 0.1686) -- (1.8500, 0.4500, 0.1665) -- (1.8500, 0.5000, 0.1720) -- (1.8000, 0.5000, 0.1741) -- cycle;
\fill[blue!15.0, opacity=0.7] (1.8000, 0.5000, 0.1741) -- (1.8500, 0.5000, 0.1720) -- (1.8500, 0.5500, 0.1774) -- (1.8000, 0.5500, 0.1795) -- cycle;
\fill[blue!15.0, opacity=0.7] (1.8000, 0.5500, 0.1795) -- (1.8500, 0.5500, 0.1774) -- (1.8500, 0.6000, 0.1826) -- (1.8000, 0.6000, 0.1847) -- cycle;
\fill[blue!15.0, opacity=0.7] (1.8000, 0.6000, 0.1847) -- (1.8500, 0.6000, 0.1826) -- (1.8500, 0.6500, 0.1875) -- (1.8000, 0.6500, 0.1896) -- cycle;
\fill[blue!15.0, opacity=0.7] (1.8000, 0.6500, 0.1896) -- (1.8500, 0.6500, 0.1875) -- (1.8500, 0.7000, 0.1923) -- (1.8000, 0.7000, 0.1944) -- cycle;
\fill[blue!15.0, opacity=0.7] (1.8000, 0.7000, 0.1944) -- (1.8500, 0.7000, 0.1923) -- (1.8500, 0.7500, 0.1969) -- (1.8000, 0.7500, 0.1990) -- cycle;
\fill[blue!15.0, opacity=0.7] (1.8000, 0.7500, 0.1990) -- (1.8500, 0.7500, 0.1969) -- (1.8500, 0.8000, 0.2012) -- (1.8000, 0.8000, 0.2033) -- cycle;
\fill[blue!15.1, opacity=0.7] (1.8000, 0.8000, 0.2033) -- (1.8500, 0.8000, 0.2012) -- (1.8500, 0.8500, 0.2053) -- (1.8000, 0.8500, 0.2074) -- cycle;
\fill[blue!33.8, opacity=0.7] (1.8000, 0.8500, 0.2074) -- (1.8500, 0.8500, 0.2053) -- (1.8500, 0.9000, 0.2091) -- (1.8000, 0.9000, 0.2112) -- cycle;
\fill[blue!78.3, opacity=0.7] (1.8000, 0.9000, 0.2112) -- (1.8500, 0.9000, 0.2091) -- (1.8500, 0.9500, 0.2127) -- (1.8000, 0.9500, 0.2148) -- cycle;
\fill[blue!54.5, opacity=0.7] (1.8000, 0.9500, 0.2148) -- (1.8500, 0.9500, 0.2127) -- (1.8500, 1.0000, 0.2160) -- (1.8000, 1.0000, 0.2180) -- cycle;
\fill[blue!18.5, opacity=0.7] (1.8000, 1.0000, 0.2180) -- (1.8500, 1.0000, 0.2160) -- (1.8500, 1.0500, 0.2190) -- (1.8000, 1.0500, 0.2210) -- cycle;
\fill[blue!15.0, opacity=0.7] (1.8000, 1.0500, 0.2210) -- (1.8500, 1.0500, 0.2190) -- (1.8500, 1.1000, 0.2217) -- (1.8000, 1.1000, 0.2238) -- cycle;
\fill[blue!15.0, opacity=0.7] (1.8000, 1.1000, 0.2238) -- (1.8500, 1.1000, 0.2217) -- (1.8500, 1.1500, 0.2241) -- (1.8000, 1.1500, 0.2262) -- cycle;
\fill[blue!15.0, opacity=0.7] (1.8000, 1.1500, 0.2262) -- (1.8500, 1.1500, 0.2241) -- (1.8500, 1.2000, 0.2262) -- (1.8000, 1.2000, 0.2283) -- cycle;
\fill[blue!15.0, opacity=0.7] (1.8000, 1.2000, 0.2283) -- (1.8500, 1.2000, 0.2262) -- (1.8500, 1.2500, 0.2279) -- (1.8000, 1.2500, 0.2300) -- cycle;
\fill[blue!15.0, opacity=0.7] (1.8000, 1.2500, 0.2300) -- (1.8500, 1.2500, 0.2279) -- (1.8500, 1.3000, 0.2294) -- (1.8000, 1.3000, 0.2315) -- cycle;
\fill[blue!15.0, opacity=0.7] (1.8000, 1.3000, 0.2315) -- (1.8500, 1.3000, 0.2294) -- (1.8500, 1.3500, 0.2306) -- (1.8000, 1.3500, 0.2326) -- cycle;
\fill[blue!15.0, opacity=0.7] (1.8000, 1.3500, 0.2326) -- (1.8500, 1.3500, 0.2306) -- (1.8500, 1.4000, 0.2314) -- (1.8000, 1.4000, 0.2335) -- cycle;
\fill[blue!15.0, opacity=0.7] (1.8000, 1.4000, 0.2335) -- (1.8500, 1.4000, 0.2314) -- (1.8500, 1.4500, 0.2319) -- (1.8000, 1.4500, 0.2340) -- cycle;
\fill[blue!15.0, opacity=0.7] (1.8000, 1.4500, 0.2340) -- (1.8500, 1.4500, 0.2319) -- (1.8500, 1.5000, 0.2320) -- (1.8000, 1.5000, 0.2341) -- cycle;
\fill[blue!15.0, opacity=0.7] (1.8000, 1.5000, 0.2341) -- (1.8500, 1.5000, 0.2320) -- (1.8500, 1.5500, 0.2319) -- (1.8000, 1.5500, 0.2340) -- cycle;
\fill[blue!15.0, opacity=0.7] (1.8000, 1.5500, 0.2340) -- (1.8500, 1.5500, 0.2319) -- (1.8500, 1.6000, 0.2314) -- (1.8000, 1.6000, 0.2335) -- cycle;
\fill[blue!15.2, opacity=0.7] (1.8000, 1.6000, 0.2335) -- (1.8500, 1.6000, 0.2314) -- (1.8500, 1.6500, 0.2306) -- (1.8000, 1.6500, 0.2326) -- cycle;
\fill[blue!40.6, opacity=0.7] (1.8000, 1.6500, 0.2326) -- (1.8500, 1.6500, 0.2306) -- (1.8500, 1.7000, 0.2294) -- (1.8000, 1.7000, 0.2315) -- cycle;
\fill[blue!7.4!black, opacity=0.7] (1.8000, 1.7000, 0.2315) -- (1.8500, 1.7000, 0.2294) -- (1.8500, 1.7500, 0.2279) -- (1.8000, 1.7500, 0.2300) -- cycle;
\fill[blue!44.8!black, opacity=0.7] (1.8000, 1.7500, 0.2300) -- (1.8500, 1.7500, 0.2279) -- (1.8500, 1.8000, 0.2262) -- (1.8000, 1.8000, 0.2283) -- cycle;
\fill[blue!24.3!black, opacity=0.7] (1.8000, 1.8000, 0.2283) -- (1.8500, 1.8000, 0.2262) -- (1.8500, 1.8500, 0.2241) -- (1.8000, 1.8500, 0.2262) -- cycle;
\fill[blue!19.4, opacity=0.7] (1.8000, 1.8500, 0.2262) -- (1.8500, 1.8500, 0.2241) -- (1.8500, 1.9000, 0.2217) -- (1.8000, 1.9000, 0.2238) -- cycle;
\fill[blue!15.0, opacity=0.7] (1.8000, 1.9000, 0.2238) -- (1.8500, 1.9000, 0.2217) -- (1.8500, 1.9500, 0.2190) -- (1.8000, 1.9500, 0.2210) -- cycle;
\fill[blue!15.0, opacity=0.7] (1.8000, 1.9500, 0.2210) -- (1.8500, 1.9500, 0.2190) -- (1.8500, 2.0000, 0.2160) -- (1.8000, 2.0000, 0.2180) -- cycle;
\fill[blue!15.0, opacity=0.7] (1.8000, 2.0000, 0.2180) -- (1.8500, 2.0000, 0.2160) -- (1.8500, 2.0500, 0.2127) -- (1.8000, 2.0500, 0.2148) -- cycle;
\fill[blue!15.0, opacity=0.7] (1.8000, 2.0500, 0.2148) -- (1.8500, 2.0500, 0.2127) -- (1.8500, 2.1000, 0.2091) -- (1.8000, 2.1000, 0.2112) -- cycle;
\fill[blue!15.0, opacity=0.7] (1.8000, 2.1000, 0.2112) -- (1.8500, 2.1000, 0.2091) -- (1.8500, 2.1500, 0.2053) -- (1.8000, 2.1500, 0.2074) -- cycle;
\fill[blue!20.4, opacity=0.7] (1.8000, 2.1500, 0.2074) -- (1.8500, 2.1500, 0.2053) -- (1.8500, 2.2000, 0.2012) -- (1.8000, 2.2000, 0.2033) -- cycle;
\fill[blue!28.6, opacity=0.7] (1.8000, 2.2000, 0.2033) -- (1.8500, 2.2000, 0.2012) -- (1.8500, 2.2500, 0.1969) -- (1.8000, 2.2500, 0.1990) -- cycle;
\fill[blue!15.2, opacity=0.7] (1.8000, 2.2500, 0.1990) -- (1.8500, 2.2500, 0.1969) -- (1.8500, 2.3000, 0.1923) -- (1.8000, 2.3000, 0.1944) -- cycle;
\fill[blue!15.0, opacity=0.7] (1.8000, 2.3000, 0.1944) -- (1.8500, 2.3000, 0.1923) -- (1.8500, 2.3500, 0.1875) -- (1.8000, 2.3500, 0.1896) -- cycle;
\fill[blue!15.0, opacity=0.7] (1.8000, 2.3500, 0.1896) -- (1.8500, 2.3500, 0.1875) -- (1.8500, 2.4000, 0.1826) -- (1.8000, 2.4000, 0.1847) -- cycle;
\fill[blue!15.0, opacity=0.7] (1.8000, 2.4000, 0.1847) -- (1.8500, 2.4000, 0.1826) -- (1.8500, 2.4500, 0.1774) -- (1.8000, 2.4500, 0.1795) -- cycle;
\fill[blue!15.0, opacity=0.7] (1.8000, 2.4500, 0.1795) -- (1.8500, 2.4500, 0.1774) -- (1.8500, 2.5000, 0.1720) -- (1.8000, 2.5000, 0.1741) -- cycle;
\fill[blue!15.0, opacity=0.7] (1.8000, 2.5000, 0.1741) -- (1.8500, 2.5000, 0.1720) -- (1.8500, 2.5500, 0.1665) -- (1.8000, 2.5500, 0.1686) -- cycle;
\fill[blue!15.0, opacity=0.7] (1.8000, 2.5500, 0.1686) -- (1.8500, 2.5500, 0.1665) -- (1.8500, 2.6000, 0.1608) -- (1.8000, 2.6000, 0.1629) -- cycle;
\fill[blue!15.0, opacity=0.7] (1.8000, 2.6000, 0.1629) -- (1.8500, 2.6000, 0.1608) -- (1.8500, 2.6500, 0.1550) -- (1.8000, 2.6500, 0.1571) -- cycle;
\fill[blue!15.0, opacity=0.7] (1.8000, 2.6500, 0.1571) -- (1.8500, 2.6500, 0.1550) -- (1.8500, 2.7000, 0.1491) -- (1.8000, 2.7000, 0.1512) -- cycle;
\fill[blue!15.0, opacity=0.7] (1.8000, 2.7000, 0.1512) -- (1.8500, 2.7000, 0.1491) -- (1.8500, 2.7500, 0.1431) -- (1.8000, 2.7500, 0.1452) -- cycle;
\fill[blue!15.0, opacity=0.7] (1.8000, 2.7500, 0.1452) -- (1.8500, 2.7500, 0.1431) -- (1.8500, 2.8000, 0.1370) -- (1.8000, 2.8000, 0.1391) -- cycle;
\fill[blue!15.0, opacity=0.7] (1.8000, 2.8000, 0.1391) -- (1.8500, 2.8000, 0.1370) -- (1.8500, 2.8500, 0.1308) -- (1.8000, 2.8500, 0.1329) -- cycle;
\fill[blue!15.0, opacity=0.7] (1.8000, 2.8500, 0.1329) -- (1.8500, 2.8500, 0.1308) -- (1.8500, 2.9000, 0.1246) -- (1.8000, 2.9000, 0.1267) -- cycle;
\fill[blue!15.0, opacity=0.7] (1.8000, 2.9000, 0.1267) -- (1.8500, 2.9000, 0.1246) -- (1.8500, 2.9500, 0.1183) -- (1.8000, 2.9500, 0.1204) -- cycle;
\fill[blue!15.0, opacity=0.7] (1.8000, 2.9500, 0.1204) -- (1.8500, 2.9500, 0.1183) -- (1.8500, 3.0000, 0.1120) -- (1.8000, 3.0000, 0.1141) -- cycle;
\fill[blue!15.0, opacity=0.7] (1.8500, 0.0000, 0.1120) -- (1.9000, 0.0000, 0.1096) -- (1.9000, 0.0500, 0.1159) -- (1.8500, 0.0500, 0.1183) -- cycle;
\fill[blue!15.0, opacity=0.7] (1.8500, 0.0500, 0.1183) -- (1.9000, 0.0500, 0.1159) -- (1.9000, 0.1000, 0.1222) -- (1.8500, 0.1000, 0.1246) -- cycle;
\fill[blue!15.0, opacity=0.7] (1.8500, 0.1000, 0.1246) -- (1.9000, 0.1000, 0.1222) -- (1.9000, 0.1500, 0.1284) -- (1.8500, 0.1500, 0.1308) -- cycle;
\fill[blue!15.0, opacity=0.7] (1.8500, 0.1500, 0.1308) -- (1.9000, 0.1500, 0.1284) -- (1.9000, 0.2000, 0.1346) -- (1.8500, 0.2000, 0.1370) -- cycle;
\fill[blue!15.0, opacity=0.7] (1.8500, 0.2000, 0.1370) -- (1.9000, 0.2000, 0.1346) -- (1.9000, 0.2500, 0.1407) -- (1.8500, 0.2500, 0.1431) -- cycle;
\fill[blue!15.0, opacity=0.7] (1.8500, 0.2500, 0.1431) -- (1.9000, 0.2500, 0.1407) -- (1.9000, 0.3000, 0.1467) -- (1.8500, 0.3000, 0.1491) -- cycle;
\fill[blue!15.0, opacity=0.7] (1.8500, 0.3000, 0.1491) -- (1.9000, 0.3000, 0.1467) -- (1.9000, 0.3500, 0.1526) -- (1.8500, 0.3500, 0.1550) -- cycle;
\fill[blue!15.0, opacity=0.7] (1.8500, 0.3500, 0.1550) -- (1.9000, 0.3500, 0.1526) -- (1.9000, 0.4000, 0.1584) -- (1.8500, 0.4000, 0.1608) -- cycle;
\fill[blue!15.0, opacity=0.7] (1.8500, 0.4000, 0.1608) -- (1.9000, 0.4000, 0.1584) -- (1.9000, 0.4500, 0.1641) -- (1.8500, 0.4500, 0.1665) -- cycle;
\fill[blue!15.0, opacity=0.7] (1.8500, 0.4500, 0.1665) -- (1.9000, 0.4500, 0.1641) -- (1.9000, 0.5000, 0.1696) -- (1.8500, 0.5000, 0.1720) -- cycle;
\fill[blue!15.0, opacity=0.7] (1.8500, 0.5000, 0.1720) -- (1.9000, 0.5000, 0.1696) -- (1.9000, 0.5500, 0.1750) -- (1.8500, 0.5500, 0.1774) -- cycle;
\fill[blue!15.0, opacity=0.7] (1.8500, 0.5500, 0.1774) -- (1.9000, 0.5500, 0.1750) -- (1.9000, 0.6000, 0.1802) -- (1.8500, 0.6000, 0.1826) -- cycle;
\fill[blue!15.0, opacity=0.7] (1.8500, 0.6000, 0.1826) -- (1.9000, 0.6000, 0.1802) -- (1.9000, 0.6500, 0.1851) -- (1.8500, 0.6500, 0.1875) -- cycle;
\fill[blue!15.0, opacity=0.7] (1.8500, 0.6500, 0.1875) -- (1.9000, 0.6500, 0.1851) -- (1.9000, 0.7000, 0.1899) -- (1.8500, 0.7000, 0.1923) -- cycle;
\fill[blue!15.0, opacity=0.7] (1.8500, 0.7000, 0.1923) -- (1.9000, 0.7000, 0.1899) -- (1.9000, 0.7500, 0.1945) -- (1.8500, 0.7500, 0.1969) -- cycle;
\fill[blue!15.0, opacity=0.7] (1.8500, 0.7500, 0.1969) -- (1.9000, 0.7500, 0.1945) -- (1.9000, 0.8000, 0.1988) -- (1.8500, 0.8000, 0.2012) -- cycle;
\fill[blue!15.0, opacity=0.7] (1.8500, 0.8000, 0.2012) -- (1.9000, 0.8000, 0.1988) -- (1.9000, 0.8500, 0.2029) -- (1.8500, 0.8500, 0.2053) -- cycle;
\fill[blue!15.5, opacity=0.7] (1.8500, 0.8500, 0.2053) -- (1.9000, 0.8500, 0.2029) -- (1.9000, 0.9000, 0.2067) -- (1.8500, 0.9000, 0.2091) -- cycle;
\fill[blue!44.5, opacity=0.7] (1.8500, 0.9000, 0.2091) -- (1.9000, 0.9000, 0.2067) -- (1.9000, 0.9500, 0.2103) -- (1.8500, 0.9500, 0.2127) -- cycle;
\fill[blue!85.9, opacity=0.7] (1.8500, 0.9500, 0.2127) -- (1.9000, 0.9500, 0.2103) -- (1.9000, 1.0000, 0.2135) -- (1.8500, 1.0000, 0.2160) -- cycle;
\fill[blue!65.5, opacity=0.7] (1.8500, 1.0000, 0.2160) -- (1.9000, 1.0000, 0.2135) -- (1.9000, 1.0500, 0.2165) -- (1.8500, 1.0500, 0.2190) -- cycle;
\fill[blue!24.8, opacity=0.7] (1.8500, 1.0500, 0.2190) -- (1.9000, 1.0500, 0.2165) -- (1.9000, 1.1000, 0.2193) -- (1.8500, 1.1000, 0.2217) -- cycle;
\fill[blue!15.4, opacity=0.7] (1.8500, 1.1000, 0.2217) -- (1.9000, 1.1000, 0.2193) -- (1.9000, 1.1500, 0.2217) -- (1.8500, 1.1500, 0.2241) -- cycle;
\fill[blue!15.0, opacity=0.7] (1.8500, 1.1500, 0.2241) -- (1.9000, 1.1500, 0.2217) -- (1.9000, 1.2000, 0.2238) -- (1.8500, 1.2000, 0.2262) -- cycle;
\fill[blue!15.0, opacity=0.7] (1.8500, 1.2000, 0.2262) -- (1.9000, 1.2000, 0.2238) -- (1.9000, 1.2500, 0.2255) -- (1.8500, 1.2500, 0.2279) -- cycle;
\fill[blue!15.0, opacity=0.7] (1.8500, 1.2500, 0.2279) -- (1.9000, 1.2500, 0.2255) -- (1.9000, 1.3000, 0.2270) -- (1.8500, 1.3000, 0.2294) -- cycle;
\fill[blue!15.0, opacity=0.7] (1.8500, 1.3000, 0.2294) -- (1.9000, 1.3000, 0.2270) -- (1.9000, 1.3500, 0.2281) -- (1.8500, 1.3500, 0.2306) -- cycle;
\fill[blue!15.0, opacity=0.7] (1.8500, 1.3500, 0.2306) -- (1.9000, 1.3500, 0.2281) -- (1.9000, 1.4000, 0.2290) -- (1.8500, 1.4000, 0.2314) -- cycle;
\fill[blue!15.0, opacity=0.7] (1.8500, 1.4000, 0.2314) -- (1.9000, 1.4000, 0.2290) -- (1.9000, 1.4500, 0.2295) -- (1.8500, 1.4500, 0.2319) -- cycle;
\fill[blue!15.0, opacity=0.7] (1.8500, 1.4500, 0.2319) -- (1.9000, 1.4500, 0.2295) -- (1.9000, 1.5000, 0.2296) -- (1.8500, 1.5000, 0.2320) -- cycle;
\fill[blue!15.0, opacity=0.7] (1.8500, 1.5000, 0.2320) -- (1.9000, 1.5000, 0.2296) -- (1.9000, 1.5500, 0.2295) -- (1.8500, 1.5500, 0.2319) -- cycle;
\fill[blue!15.6, opacity=0.7] (1.8500, 1.5500, 0.2319) -- (1.9000, 1.5500, 0.2295) -- (1.9000, 1.6000, 0.2290) -- (1.8500, 1.6000, 0.2314) -- cycle;
\fill[blue!39.7, opacity=0.7] (1.8500, 1.6000, 0.2314) -- (1.9000, 1.6000, 0.2290) -- (1.9000, 1.6500, 0.2281) -- (1.8500, 1.6500, 0.2306) -- cycle;
\fill[blue!28.9!black, opacity=0.7] (1.8500, 1.6500, 0.2306) -- (1.9000, 1.6500, 0.2281) -- (1.9000, 1.7000, 0.2270) -- (1.8500, 1.7000, 0.2294) -- cycle;
\fill[blue!25.2!black, opacity=0.7] (1.8500, 1.7000, 0.2294) -- (1.9000, 1.7000, 0.2270) -- (1.9000, 1.7500, 0.2255) -- (1.8500, 1.7500, 0.2279) -- cycle;
\fill[blue!5.0!black, opacity=0.7] (1.8500, 1.7500, 0.2279) -- (1.9000, 1.7500, 0.2255) -- (1.9000, 1.8000, 0.2238) -- (1.8500, 1.8000, 0.2262) -- cycle;
\fill[blue!46.0, opacity=0.7] (1.8500, 1.8000, 0.2262) -- (1.9000, 1.8000, 0.2238) -- (1.9000, 1.8500, 0.2217) -- (1.8500, 1.8500, 0.2241) -- cycle;
\fill[blue!15.0, opacity=0.7] (1.8500, 1.8500, 0.2241) -- (1.9000, 1.8500, 0.2217) -- (1.9000, 1.9000, 0.2193) -- (1.8500, 1.9000, 0.2217) -- cycle;
\fill[blue!15.0, opacity=0.7] (1.8500, 1.9000, 0.2217) -- (1.9000, 1.9000, 0.2193) -- (1.9000, 1.9500, 0.2165) -- (1.8500, 1.9500, 0.2190) -- cycle;
\fill[blue!15.0, opacity=0.7] (1.8500, 1.9500, 0.2190) -- (1.9000, 1.9500, 0.2165) -- (1.9000, 2.0000, 0.2135) -- (1.8500, 2.0000, 0.2160) -- cycle;
\fill[blue!15.0, opacity=0.7] (1.8500, 2.0000, 0.2160) -- (1.9000, 2.0000, 0.2135) -- (1.9000, 2.0500, 0.2103) -- (1.8500, 2.0500, 0.2127) -- cycle;
\fill[blue!15.0, opacity=0.7] (1.8500, 2.0500, 0.2127) -- (1.9000, 2.0500, 0.2103) -- (1.9000, 2.1000, 0.2067) -- (1.8500, 2.1000, 0.2091) -- cycle;
\fill[blue!15.1, opacity=0.7] (1.8500, 2.1000, 0.2091) -- (1.9000, 2.1000, 0.2067) -- (1.9000, 2.1500, 0.2029) -- (1.8500, 2.1500, 0.2053) -- cycle;
\fill[blue!24.8, opacity=0.7] (1.8500, 2.1500, 0.2053) -- (1.9000, 2.1500, 0.2029) -- (1.9000, 2.2000, 0.1988) -- (1.8500, 2.2000, 0.2012) -- cycle;
\fill[blue!22.0, opacity=0.7] (1.8500, 2.2000, 0.2012) -- (1.9000, 2.2000, 0.1988) -- (1.9000, 2.2500, 0.1945) -- (1.8500, 2.2500, 0.1969) -- cycle;
\fill[blue!15.0, opacity=0.7] (1.8500, 2.2500, 0.1969) -- (1.9000, 2.2500, 0.1945) -- (1.9000, 2.3000, 0.1899) -- (1.8500, 2.3000, 0.1923) -- cycle;
\fill[blue!15.0, opacity=0.7] (1.8500, 2.3000, 0.1923) -- (1.9000, 2.3000, 0.1899) -- (1.9000, 2.3500, 0.1851) -- (1.8500, 2.3500, 0.1875) -- cycle;
\fill[blue!15.0, opacity=0.7] (1.8500, 2.3500, 0.1875) -- (1.9000, 2.3500, 0.1851) -- (1.9000, 2.4000, 0.1802) -- (1.8500, 2.4000, 0.1826) -- cycle;
\fill[blue!15.0, opacity=0.7] (1.8500, 2.4000, 0.1826) -- (1.9000, 2.4000, 0.1802) -- (1.9000, 2.4500, 0.1750) -- (1.8500, 2.4500, 0.1774) -- cycle;
\fill[blue!15.0, opacity=0.7] (1.8500, 2.4500, 0.1774) -- (1.9000, 2.4500, 0.1750) -- (1.9000, 2.5000, 0.1696) -- (1.8500, 2.5000, 0.1720) -- cycle;
\fill[blue!15.0, opacity=0.7] (1.8500, 2.5000, 0.1720) -- (1.9000, 2.5000, 0.1696) -- (1.9000, 2.5500, 0.1641) -- (1.8500, 2.5500, 0.1665) -- cycle;
\fill[blue!15.0, opacity=0.7] (1.8500, 2.5500, 0.1665) -- (1.9000, 2.5500, 0.1641) -- (1.9000, 2.6000, 0.1584) -- (1.8500, 2.6000, 0.1608) -- cycle;
\fill[blue!15.0, opacity=0.7] (1.8500, 2.6000, 0.1608) -- (1.9000, 2.6000, 0.1584) -- (1.9000, 2.6500, 0.1526) -- (1.8500, 2.6500, 0.1550) -- cycle;
\fill[blue!15.0, opacity=0.7] (1.8500, 2.6500, 0.1550) -- (1.9000, 2.6500, 0.1526) -- (1.9000, 2.7000, 0.1467) -- (1.8500, 2.7000, 0.1491) -- cycle;
\fill[blue!15.0, opacity=0.7] (1.8500, 2.7000, 0.1491) -- (1.9000, 2.7000, 0.1467) -- (1.9000, 2.7500, 0.1407) -- (1.8500, 2.7500, 0.1431) -- cycle;
\fill[blue!15.0, opacity=0.7] (1.8500, 2.7500, 0.1431) -- (1.9000, 2.7500, 0.1407) -- (1.9000, 2.8000, 0.1346) -- (1.8500, 2.8000, 0.1370) -- cycle;
\fill[blue!15.0, opacity=0.7] (1.8500, 2.8000, 0.1370) -- (1.9000, 2.8000, 0.1346) -- (1.9000, 2.8500, 0.1284) -- (1.8500, 2.8500, 0.1308) -- cycle;
\fill[blue!15.0, opacity=0.7] (1.8500, 2.8500, 0.1308) -- (1.9000, 2.8500, 0.1284) -- (1.9000, 2.9000, 0.1222) -- (1.8500, 2.9000, 0.1246) -- cycle;
\fill[blue!15.0, opacity=0.7] (1.8500, 2.9000, 0.1246) -- (1.9000, 2.9000, 0.1222) -- (1.9000, 2.9500, 0.1159) -- (1.8500, 2.9500, 0.1183) -- cycle;
\fill[blue!15.0, opacity=0.7] (1.8500, 2.9500, 0.1183) -- (1.9000, 2.9500, 0.1159) -- (1.9000, 3.0000, 0.1096) -- (1.8500, 3.0000, 0.1120) -- cycle;
\fill[blue!15.0, opacity=0.7] (1.9000, 0.0000, 0.1096) -- (1.9500, 0.0000, 0.1069) -- (1.9500, 0.0500, 0.1132) -- (1.9000, 0.0500, 0.1159) -- cycle;
\fill[blue!15.0, opacity=0.7] (1.9000, 0.0500, 0.1159) -- (1.9500, 0.0500, 0.1132) -- (1.9500, 0.1000, 0.1195) -- (1.9000, 0.1000, 0.1222) -- cycle;
\fill[blue!15.0, opacity=0.7] (1.9000, 0.1000, 0.1222) -- (1.9500, 0.1000, 0.1195) -- (1.9500, 0.1500, 0.1257) -- (1.9000, 0.1500, 0.1284) -- cycle;
\fill[blue!15.0, opacity=0.7] (1.9000, 0.1500, 0.1284) -- (1.9500, 0.1500, 0.1257) -- (1.9500, 0.2000, 0.1319) -- (1.9000, 0.2000, 0.1346) -- cycle;
\fill[blue!15.0, opacity=0.7] (1.9000, 0.2000, 0.1346) -- (1.9500, 0.2000, 0.1319) -- (1.9500, 0.2500, 0.1380) -- (1.9000, 0.2500, 0.1407) -- cycle;
\fill[blue!15.0, opacity=0.7] (1.9000, 0.2500, 0.1407) -- (1.9500, 0.2500, 0.1380) -- (1.9500, 0.3000, 0.1440) -- (1.9000, 0.3000, 0.1467) -- cycle;
\fill[blue!15.0, opacity=0.7] (1.9000, 0.3000, 0.1467) -- (1.9500, 0.3000, 0.1440) -- (1.9500, 0.3500, 0.1499) -- (1.9000, 0.3500, 0.1526) -- cycle;
\fill[blue!15.0, opacity=0.7] (1.9000, 0.3500, 0.1526) -- (1.9500, 0.3500, 0.1499) -- (1.9500, 0.4000, 0.1557) -- (1.9000, 0.4000, 0.1584) -- cycle;
\fill[blue!15.0, opacity=0.7] (1.9000, 0.4000, 0.1584) -- (1.9500, 0.4000, 0.1557) -- (1.9500, 0.4500, 0.1614) -- (1.9000, 0.4500, 0.1641) -- cycle;
\fill[blue!15.0, opacity=0.7] (1.9000, 0.4500, 0.1641) -- (1.9500, 0.4500, 0.1614) -- (1.9500, 0.5000, 0.1669) -- (1.9000, 0.5000, 0.1696) -- cycle;
\fill[blue!15.0, opacity=0.7] (1.9000, 0.5000, 0.1696) -- (1.9500, 0.5000, 0.1669) -- (1.9500, 0.5500, 0.1723) -- (1.9000, 0.5500, 0.1750) -- cycle;
\fill[blue!15.0, opacity=0.7] (1.9000, 0.5500, 0.1750) -- (1.9500, 0.5500, 0.1723) -- (1.9500, 0.6000, 0.1775) -- (1.9000, 0.6000, 0.1802) -- cycle;
\fill[blue!15.0, opacity=0.7] (1.9000, 0.6000, 0.1802) -- (1.9500, 0.6000, 0.1775) -- (1.9500, 0.6500, 0.1824) -- (1.9000, 0.6500, 0.1851) -- cycle;
\fill[blue!15.0, opacity=0.7] (1.9000, 0.6500, 0.1851) -- (1.9500, 0.6500, 0.1824) -- (1.9500, 0.7000, 0.1872) -- (1.9000, 0.7000, 0.1899) -- cycle;
\fill[blue!15.0, opacity=0.7] (1.9000, 0.7000, 0.1899) -- (1.9500, 0.7000, 0.1872) -- (1.9500, 0.7500, 0.1918) -- (1.9000, 0.7500, 0.1945) -- cycle;
\fill[blue!15.0, opacity=0.7] (1.9000, 0.7500, 0.1945) -- (1.9500, 0.7500, 0.1918) -- (1.9500, 0.8000, 0.1961) -- (1.9000, 0.8000, 0.1988) -- cycle;
\fill[blue!15.0, opacity=0.7] (1.9000, 0.8000, 0.1988) -- (1.9500, 0.8000, 0.1961) -- (1.9500, 0.8500, 0.2002) -- (1.9000, 0.8500, 0.2029) -- cycle;
\fill[blue!15.0, opacity=0.7] (1.9000, 0.8500, 0.2029) -- (1.9500, 0.8500, 0.2002) -- (1.9500, 0.9000, 0.2040) -- (1.9000, 0.9000, 0.2067) -- cycle;
\fill[blue!15.8, opacity=0.7] (1.9000, 0.9000, 0.2067) -- (1.9500, 0.9000, 0.2040) -- (1.9500, 0.9500, 0.2076) -- (1.9000, 0.9500, 0.2103) -- cycle;
\fill[blue!45.2, opacity=0.7] (1.9000, 0.9500, 0.2103) -- (1.9500, 0.9500, 0.2076) -- (1.9500, 1.0000, 0.2108) -- (1.9000, 1.0000, 0.2135) -- cycle;
\fill[blue!90.7, opacity=0.7] (1.9000, 1.0000, 0.2135) -- (1.9500, 1.0000, 0.2108) -- (1.9500, 1.0500, 0.2138) -- (1.9000, 1.0500, 0.2165) -- cycle;
\fill[blue!88.2, opacity=0.7] (1.9000, 1.0500, 0.2165) -- (1.9500, 1.0500, 0.2138) -- (1.9500, 1.1000, 0.2165) -- (1.9000, 1.1000, 0.2193) -- cycle;
\fill[blue!52.1, opacity=0.7] (1.9000, 1.1000, 0.2193) -- (1.9500, 1.1000, 0.2165) -- (1.9500, 1.1500, 0.2190) -- (1.9000, 1.1500, 0.2217) -- cycle;
\fill[blue!23.9, opacity=0.7] (1.9000, 1.1500, 0.2217) -- (1.9500, 1.1500, 0.2190) -- (1.9500, 1.2000, 0.2210) -- (1.9000, 1.2000, 0.2238) -- cycle;
\fill[blue!16.3, opacity=0.7] (1.9000, 1.2000, 0.2238) -- (1.9500, 1.2000, 0.2210) -- (1.9500, 1.2500, 0.2228) -- (1.9000, 1.2500, 0.2255) -- cycle;
\fill[blue!15.2, opacity=0.7] (1.9000, 1.2500, 0.2255) -- (1.9500, 1.2500, 0.2228) -- (1.9500, 1.3000, 0.2243) -- (1.9000, 1.3000, 0.2270) -- cycle;
\fill[blue!15.1, opacity=0.7] (1.9000, 1.3000, 0.2270) -- (1.9500, 1.3000, 0.2243) -- (1.9500, 1.3500, 0.2254) -- (1.9000, 1.3500, 0.2281) -- cycle;
\fill[blue!15.1, opacity=0.7] (1.9000, 1.3500, 0.2281) -- (1.9500, 1.3500, 0.2254) -- (1.9500, 1.4000, 0.2263) -- (1.9000, 1.4000, 0.2290) -- cycle;
\fill[blue!15.2, opacity=0.7] (1.9000, 1.4000, 0.2290) -- (1.9500, 1.4000, 0.2263) -- (1.9500, 1.4500, 0.2268) -- (1.9000, 1.4500, 0.2295) -- cycle;
\fill[blue!16.1, opacity=0.7] (1.9000, 1.4500, 0.2295) -- (1.9500, 1.4500, 0.2268) -- (1.9500, 1.5000, 0.2269) -- (1.9000, 1.5000, 0.2296) -- cycle;
\fill[blue!24.9, opacity=0.7] (1.9000, 1.5000, 0.2296) -- (1.9500, 1.5000, 0.2269) -- (1.9500, 1.5500, 0.2268) -- (1.9000, 1.5500, 0.2295) -- cycle;
\fill[blue!71.6, opacity=0.7] (1.9000, 1.5500, 0.2295) -- (1.9500, 1.5500, 0.2268) -- (1.9500, 1.6000, 0.2263) -- (1.9000, 1.6000, 0.2290) -- cycle;
\fill[blue!18.0!black, opacity=0.7] (1.9000, 1.6000, 0.2290) -- (1.9500, 1.6000, 0.2263) -- (1.9500, 1.6500, 0.2254) -- (1.9000, 1.6500, 0.2281) -- cycle;
\fill[blue!13.2!black, opacity=0.7] (1.9000, 1.6500, 0.2281) -- (1.9500, 1.6500, 0.2254) -- (1.9500, 1.7000, 0.2243) -- (1.9000, 1.7000, 0.2270) -- cycle;
\fill[blue!5.6!black, opacity=0.7] (1.9000, 1.7000, 0.2270) -- (1.9500, 1.7000, 0.2243) -- (1.9500, 1.7500, 0.2228) -- (1.9000, 1.7500, 0.2255) -- cycle;
\fill[blue!63.9, opacity=0.7] (1.9000, 1.7500, 0.2255) -- (1.9500, 1.7500, 0.2228) -- (1.9500, 1.8000, 0.2210) -- (1.9000, 1.8000, 0.2238) -- cycle;
\fill[blue!15.3, opacity=0.7] (1.9000, 1.8000, 0.2238) -- (1.9500, 1.8000, 0.2210) -- (1.9500, 1.8500, 0.2190) -- (1.9000, 1.8500, 0.2217) -- cycle;
\fill[blue!15.0, opacity=0.7] (1.9000, 1.8500, 0.2217) -- (1.9500, 1.8500, 0.2190) -- (1.9500, 1.9000, 0.2165) -- (1.9000, 1.9000, 0.2193) -- cycle;
\fill[blue!15.0, opacity=0.7] (1.9000, 1.9000, 0.2193) -- (1.9500, 1.9000, 0.2165) -- (1.9500, 1.9500, 0.2138) -- (1.9000, 1.9500, 0.2165) -- cycle;
\fill[blue!15.0, opacity=0.7] (1.9000, 1.9500, 0.2165) -- (1.9500, 1.9500, 0.2138) -- (1.9500, 2.0000, 0.2108) -- (1.9000, 2.0000, 0.2135) -- cycle;
\fill[blue!15.0, opacity=0.7] (1.9000, 2.0000, 0.2135) -- (1.9500, 2.0000, 0.2108) -- (1.9500, 2.0500, 0.2076) -- (1.9000, 2.0500, 0.2103) -- cycle;
\fill[blue!15.0, opacity=0.7] (1.9000, 2.0500, 0.2103) -- (1.9500, 2.0500, 0.2076) -- (1.9500, 2.1000, 0.2040) -- (1.9000, 2.1000, 0.2067) -- cycle;
\fill[blue!16.3, opacity=0.7] (1.9000, 2.1000, 0.2067) -- (1.9500, 2.1000, 0.2040) -- (1.9500, 2.1500, 0.2002) -- (1.9000, 2.1500, 0.2029) -- cycle;
\fill[blue!27.1, opacity=0.7] (1.9000, 2.1500, 0.2029) -- (1.9500, 2.1500, 0.2002) -- (1.9500, 2.2000, 0.1961) -- (1.9000, 2.2000, 0.1988) -- cycle;
\fill[blue!16.7, opacity=0.7] (1.9000, 2.2000, 0.1988) -- (1.9500, 2.2000, 0.1961) -- (1.9500, 2.2500, 0.1918) -- (1.9000, 2.2500, 0.1945) -- cycle;
\fill[blue!15.0, opacity=0.7] (1.9000, 2.2500, 0.1945) -- (1.9500, 2.2500, 0.1918) -- (1.9500, 2.3000, 0.1872) -- (1.9000, 2.3000, 0.1899) -- cycle;
\fill[blue!15.0, opacity=0.7] (1.9000, 2.3000, 0.1899) -- (1.9500, 2.3000, 0.1872) -- (1.9500, 2.3500, 0.1824) -- (1.9000, 2.3500, 0.1851) -- cycle;
\fill[blue!15.0, opacity=0.7] (1.9000, 2.3500, 0.1851) -- (1.9500, 2.3500, 0.1824) -- (1.9500, 2.4000, 0.1775) -- (1.9000, 2.4000, 0.1802) -- cycle;
\fill[blue!15.0, opacity=0.7] (1.9000, 2.4000, 0.1802) -- (1.9500, 2.4000, 0.1775) -- (1.9500, 2.4500, 0.1723) -- (1.9000, 2.4500, 0.1750) -- cycle;
\fill[blue!15.0, opacity=0.7] (1.9000, 2.4500, 0.1750) -- (1.9500, 2.4500, 0.1723) -- (1.9500, 2.5000, 0.1669) -- (1.9000, 2.5000, 0.1696) -- cycle;
\fill[blue!15.0, opacity=0.7] (1.9000, 2.5000, 0.1696) -- (1.9500, 2.5000, 0.1669) -- (1.9500, 2.5500, 0.1614) -- (1.9000, 2.5500, 0.1641) -- cycle;
\fill[blue!15.0, opacity=0.7] (1.9000, 2.5500, 0.1641) -- (1.9500, 2.5500, 0.1614) -- (1.9500, 2.6000, 0.1557) -- (1.9000, 2.6000, 0.1584) -- cycle;
\fill[blue!15.0, opacity=0.7] (1.9000, 2.6000, 0.1584) -- (1.9500, 2.6000, 0.1557) -- (1.9500, 2.6500, 0.1499) -- (1.9000, 2.6500, 0.1526) -- cycle;
\fill[blue!15.0, opacity=0.7] (1.9000, 2.6500, 0.1526) -- (1.9500, 2.6500, 0.1499) -- (1.9500, 2.7000, 0.1440) -- (1.9000, 2.7000, 0.1467) -- cycle;
\fill[blue!15.0, opacity=0.7] (1.9000, 2.7000, 0.1467) -- (1.9500, 2.7000, 0.1440) -- (1.9500, 2.7500, 0.1380) -- (1.9000, 2.7500, 0.1407) -- cycle;
\fill[blue!15.0, opacity=0.7] (1.9000, 2.7500, 0.1407) -- (1.9500, 2.7500, 0.1380) -- (1.9500, 2.8000, 0.1319) -- (1.9000, 2.8000, 0.1346) -- cycle;
\fill[blue!15.0, opacity=0.7] (1.9000, 2.8000, 0.1346) -- (1.9500, 2.8000, 0.1319) -- (1.9500, 2.8500, 0.1257) -- (1.9000, 2.8500, 0.1284) -- cycle;
\fill[blue!15.0, opacity=0.7] (1.9000, 2.8500, 0.1284) -- (1.9500, 2.8500, 0.1257) -- (1.9500, 2.9000, 0.1195) -- (1.9000, 2.9000, 0.1222) -- cycle;
\fill[blue!15.0, opacity=0.7] (1.9000, 2.9000, 0.1222) -- (1.9500, 2.9000, 0.1195) -- (1.9500, 2.9500, 0.1132) -- (1.9000, 2.9500, 0.1159) -- cycle;
\fill[blue!15.0, opacity=0.7] (1.9000, 2.9500, 0.1159) -- (1.9500, 2.9500, 0.1132) -- (1.9500, 3.0000, 0.1069) -- (1.9000, 3.0000, 0.1096) -- cycle;
\fill[blue!15.0, opacity=0.7] (1.9500, 0.0000, 0.1069) -- (2.0000, 0.0000, 0.1039) -- (2.0000, 0.0500, 0.1102) -- (1.9500, 0.0500, 0.1132) -- cycle;
\fill[blue!15.0, opacity=0.7] (1.9500, 0.0500, 0.1132) -- (2.0000, 0.0500, 0.1102) -- (2.0000, 0.1000, 0.1165) -- (1.9500, 0.1000, 0.1195) -- cycle;
\fill[blue!15.0, opacity=0.7] (1.9500, 0.1000, 0.1195) -- (2.0000, 0.1000, 0.1165) -- (2.0000, 0.1500, 0.1227) -- (1.9500, 0.1500, 0.1257) -- cycle;
\fill[blue!15.0, opacity=0.7] (1.9500, 0.1500, 0.1257) -- (2.0000, 0.1500, 0.1227) -- (2.0000, 0.2000, 0.1289) -- (1.9500, 0.2000, 0.1319) -- cycle;
\fill[blue!15.0, opacity=0.7] (1.9500, 0.2000, 0.1319) -- (2.0000, 0.2000, 0.1289) -- (2.0000, 0.2500, 0.1350) -- (1.9500, 0.2500, 0.1380) -- cycle;
\fill[blue!15.0, opacity=0.7] (1.9500, 0.2500, 0.1380) -- (2.0000, 0.2500, 0.1350) -- (2.0000, 0.3000, 0.1410) -- (1.9500, 0.3000, 0.1440) -- cycle;
\fill[blue!15.0, opacity=0.7] (1.9500, 0.3000, 0.1440) -- (2.0000, 0.3000, 0.1410) -- (2.0000, 0.3500, 0.1469) -- (1.9500, 0.3500, 0.1499) -- cycle;
\fill[blue!15.0, opacity=0.7] (1.9500, 0.3500, 0.1499) -- (2.0000, 0.3500, 0.1469) -- (2.0000, 0.4000, 0.1527) -- (1.9500, 0.4000, 0.1557) -- cycle;
\fill[blue!15.0, opacity=0.7] (1.9500, 0.4000, 0.1557) -- (2.0000, 0.4000, 0.1527) -- (2.0000, 0.4500, 0.1584) -- (1.9500, 0.4500, 0.1614) -- cycle;
\fill[blue!15.0, opacity=0.7] (1.9500, 0.4500, 0.1614) -- (2.0000, 0.4500, 0.1584) -- (2.0000, 0.5000, 0.1639) -- (1.9500, 0.5000, 0.1669) -- cycle;
\fill[blue!15.0, opacity=0.7] (1.9500, 0.5000, 0.1669) -- (2.0000, 0.5000, 0.1639) -- (2.0000, 0.5500, 0.1693) -- (1.9500, 0.5500, 0.1723) -- cycle;
\fill[blue!15.0, opacity=0.7] (1.9500, 0.5500, 0.1723) -- (2.0000, 0.5500, 0.1693) -- (2.0000, 0.6000, 0.1745) -- (1.9500, 0.6000, 0.1775) -- cycle;
\fill[blue!15.0, opacity=0.7] (1.9500, 0.6000, 0.1775) -- (2.0000, 0.6000, 0.1745) -- (2.0000, 0.6500, 0.1794) -- (1.9500, 0.6500, 0.1824) -- cycle;
\fill[blue!15.0, opacity=0.7] (1.9500, 0.6500, 0.1824) -- (2.0000, 0.6500, 0.1794) -- (2.0000, 0.7000, 0.1842) -- (1.9500, 0.7000, 0.1872) -- cycle;
\fill[blue!15.0, opacity=0.7] (1.9500, 0.7000, 0.1872) -- (2.0000, 0.7000, 0.1842) -- (2.0000, 0.7500, 0.1888) -- (1.9500, 0.7500, 0.1918) -- cycle;
\fill[blue!15.0, opacity=0.7] (1.9500, 0.7500, 0.1918) -- (2.0000, 0.7500, 0.1888) -- (2.0000, 0.8000, 0.1931) -- (1.9500, 0.8000, 0.1961) -- cycle;
\fill[blue!15.0, opacity=0.7] (1.9500, 0.8000, 0.1961) -- (2.0000, 0.8000, 0.1931) -- (2.0000, 0.8500, 0.1972) -- (1.9500, 0.8500, 0.2002) -- cycle;
\fill[blue!15.0, opacity=0.7] (1.9500, 0.8500, 0.2002) -- (2.0000, 0.8500, 0.1972) -- (2.0000, 0.9000, 0.2010) -- (1.9500, 0.9000, 0.2040) -- cycle;
\fill[blue!15.0, opacity=0.7] (1.9500, 0.9000, 0.2040) -- (2.0000, 0.9000, 0.2010) -- (2.0000, 0.9500, 0.2046) -- (1.9500, 0.9500, 0.2076) -- cycle;
\fill[blue!15.4, opacity=0.7] (1.9500, 0.9500, 0.2076) -- (2.0000, 0.9500, 0.2046) -- (2.0000, 1.0000, 0.2078) -- (1.9500, 1.0000, 0.2108) -- cycle;
\fill[blue!33.5, opacity=0.7] (1.9500, 1.0000, 0.2108) -- (2.0000, 1.0000, 0.2078) -- (2.0000, 1.0500, 0.2108) -- (1.9500, 1.0500, 0.2138) -- cycle;
\fill[blue!82.9, opacity=0.7] (1.9500, 1.0500, 0.2138) -- (2.0000, 1.0500, 0.2108) -- (2.0000, 1.1000, 0.2135) -- (1.9500, 1.1000, 0.2165) -- cycle;
\fill[blue!85.6!black, opacity=0.7] (1.9500, 1.1000, 0.2165) -- (2.0000, 1.1000, 0.2135) -- (2.0000, 1.1500, 0.2160) -- (1.9500, 1.1500, 0.2190) -- cycle;
\fill[blue!99.7!black, opacity=0.7] (1.9500, 1.1500, 0.2190) -- (2.0000, 1.1500, 0.2160) -- (2.0000, 1.2000, 0.2180) -- (1.9500, 1.2000, 0.2210) -- cycle;
\fill[blue!80.5, opacity=0.7] (1.9500, 1.2000, 0.2210) -- (2.0000, 1.2000, 0.2180) -- (2.0000, 1.2500, 0.2198) -- (1.9500, 1.2500, 0.2228) -- cycle;
\fill[blue!60.8, opacity=0.7] (1.9500, 1.2500, 0.2228) -- (2.0000, 1.2500, 0.2198) -- (2.0000, 1.3000, 0.2213) -- (1.9500, 1.3000, 0.2243) -- cycle;
\fill[blue!50.3, opacity=0.7] (1.9500, 1.3000, 0.2243) -- (2.0000, 1.3000, 0.2213) -- (2.0000, 1.3500, 0.2224) -- (1.9500, 1.3500, 0.2254) -- cycle;
\fill[blue!50.6, opacity=0.7] (1.9500, 1.3500, 0.2254) -- (2.0000, 1.3500, 0.2224) -- (2.0000, 1.4000, 0.2233) -- (1.9500, 1.4000, 0.2263) -- cycle;
\fill[blue!63.5, opacity=0.7] (1.9500, 1.4000, 0.2263) -- (2.0000, 1.4000, 0.2233) -- (2.0000, 1.4500, 0.2238) -- (1.9500, 1.4500, 0.2268) -- cycle;
\fill[blue!91.4, opacity=0.7] (1.9500, 1.4500, 0.2268) -- (2.0000, 1.4500, 0.2238) -- (2.0000, 1.5000, 0.2239) -- (1.9500, 1.5000, 0.2269) -- cycle;
\fill[blue!38.6!black, opacity=0.7] (1.9500, 1.5000, 0.2269) -- (2.0000, 1.5000, 0.2239) -- (2.0000, 1.5500, 0.2238) -- (1.9500, 1.5500, 0.2268) -- cycle;
\fill[blue!5.4!black, opacity=0.7] (1.9500, 1.5500, 0.2268) -- (2.0000, 1.5500, 0.2238) -- (2.0000, 1.6000, 0.2233) -- (1.9500, 1.6000, 0.2263) -- cycle;
\fill[blue!5.6!black, opacity=0.7] (1.9500, 1.6000, 0.2263) -- (2.0000, 1.6000, 0.2233) -- (2.0000, 1.6500, 0.2224) -- (1.9500, 1.6500, 0.2254) -- cycle;
\fill[blue!28.0!black, opacity=0.7] (1.9500, 1.6500, 0.2254) -- (2.0000, 1.6500, 0.2224) -- (2.0000, 1.7000, 0.2213) -- (1.9500, 1.7000, 0.2243) -- cycle;
\fill[blue!51.0, opacity=0.7] (1.9500, 1.7000, 0.2243) -- (2.0000, 1.7000, 0.2213) -- (2.0000, 1.7500, 0.2198) -- (1.9500, 1.7500, 0.2228) -- cycle;
\fill[blue!15.4, opacity=0.7] (1.9500, 1.7500, 0.2228) -- (2.0000, 1.7500, 0.2198) -- (2.0000, 1.8000, 0.2180) -- (1.9500, 1.8000, 0.2210) -- cycle;
\fill[blue!15.0, opacity=0.7] (1.9500, 1.8000, 0.2210) -- (2.0000, 1.8000, 0.2180) -- (2.0000, 1.8500, 0.2160) -- (1.9500, 1.8500, 0.2190) -- cycle;
\fill[blue!15.0, opacity=0.7] (1.9500, 1.8500, 0.2190) -- (2.0000, 1.8500, 0.2160) -- (2.0000, 1.9000, 0.2135) -- (1.9500, 1.9000, 0.2165) -- cycle;
\fill[blue!15.0, opacity=0.7] (1.9500, 1.9000, 0.2165) -- (2.0000, 1.9000, 0.2135) -- (2.0000, 1.9500, 0.2108) -- (1.9500, 1.9500, 0.2138) -- cycle;
\fill[blue!15.0, opacity=0.7] (1.9500, 1.9500, 0.2138) -- (2.0000, 1.9500, 0.2108) -- (2.0000, 2.0000, 0.2078) -- (1.9500, 2.0000, 0.2108) -- cycle;
\fill[blue!15.0, opacity=0.7] (1.9500, 2.0000, 0.2108) -- (2.0000, 2.0000, 0.2078) -- (2.0000, 2.0500, 0.2046) -- (1.9500, 2.0500, 0.2076) -- cycle;
\fill[blue!15.0, opacity=0.7] (1.9500, 2.0500, 0.2076) -- (2.0000, 2.0500, 0.2046) -- (2.0000, 2.1000, 0.2010) -- (1.9500, 2.1000, 0.2040) -- cycle;
\fill[blue!20.6, opacity=0.7] (1.9500, 2.1000, 0.2040) -- (2.0000, 2.1000, 0.2010) -- (2.0000, 2.1500, 0.1972) -- (1.9500, 2.1500, 0.2002) -- cycle;
\fill[blue!22.8, opacity=0.7] (1.9500, 2.1500, 0.2002) -- (2.0000, 2.1500, 0.1972) -- (2.0000, 2.2000, 0.1931) -- (1.9500, 2.2000, 0.1961) -- cycle;
\fill[blue!15.1, opacity=0.7] (1.9500, 2.2000, 0.1961) -- (2.0000, 2.2000, 0.1931) -- (2.0000, 2.2500, 0.1888) -- (1.9500, 2.2500, 0.1918) -- cycle;
\fill[blue!15.0, opacity=0.7] (1.9500, 2.2500, 0.1918) -- (2.0000, 2.2500, 0.1888) -- (2.0000, 2.3000, 0.1842) -- (1.9500, 2.3000, 0.1872) -- cycle;
\fill[blue!15.0, opacity=0.7] (1.9500, 2.3000, 0.1872) -- (2.0000, 2.3000, 0.1842) -- (2.0000, 2.3500, 0.1794) -- (1.9500, 2.3500, 0.1824) -- cycle;
\fill[blue!15.0, opacity=0.7] (1.9500, 2.3500, 0.1824) -- (2.0000, 2.3500, 0.1794) -- (2.0000, 2.4000, 0.1745) -- (1.9500, 2.4000, 0.1775) -- cycle;
\fill[blue!15.0, opacity=0.7] (1.9500, 2.4000, 0.1775) -- (2.0000, 2.4000, 0.1745) -- (2.0000, 2.4500, 0.1693) -- (1.9500, 2.4500, 0.1723) -- cycle;
\fill[blue!15.0, opacity=0.7] (1.9500, 2.4500, 0.1723) -- (2.0000, 2.4500, 0.1693) -- (2.0000, 2.5000, 0.1639) -- (1.9500, 2.5000, 0.1669) -- cycle;
\fill[blue!15.0, opacity=0.7] (1.9500, 2.5000, 0.1669) -- (2.0000, 2.5000, 0.1639) -- (2.0000, 2.5500, 0.1584) -- (1.9500, 2.5500, 0.1614) -- cycle;
\fill[blue!15.0, opacity=0.7] (1.9500, 2.5500, 0.1614) -- (2.0000, 2.5500, 0.1584) -- (2.0000, 2.6000, 0.1527) -- (1.9500, 2.6000, 0.1557) -- cycle;
\fill[blue!15.0, opacity=0.7] (1.9500, 2.6000, 0.1557) -- (2.0000, 2.6000, 0.1527) -- (2.0000, 2.6500, 0.1469) -- (1.9500, 2.6500, 0.1499) -- cycle;
\fill[blue!15.0, opacity=0.7] (1.9500, 2.6500, 0.1499) -- (2.0000, 2.6500, 0.1469) -- (2.0000, 2.7000, 0.1410) -- (1.9500, 2.7000, 0.1440) -- cycle;
\fill[blue!15.0, opacity=0.7] (1.9500, 2.7000, 0.1440) -- (2.0000, 2.7000, 0.1410) -- (2.0000, 2.7500, 0.1350) -- (1.9500, 2.7500, 0.1380) -- cycle;
\fill[blue!15.0, opacity=0.7] (1.9500, 2.7500, 0.1380) -- (2.0000, 2.7500, 0.1350) -- (2.0000, 2.8000, 0.1289) -- (1.9500, 2.8000, 0.1319) -- cycle;
\fill[blue!15.0, opacity=0.7] (1.9500, 2.8000, 0.1319) -- (2.0000, 2.8000, 0.1289) -- (2.0000, 2.8500, 0.1227) -- (1.9500, 2.8500, 0.1257) -- cycle;
\fill[blue!15.0, opacity=0.7] (1.9500, 2.8500, 0.1257) -- (2.0000, 2.8500, 0.1227) -- (2.0000, 2.9000, 0.1165) -- (1.9500, 2.9000, 0.1195) -- cycle;
\fill[blue!15.0, opacity=0.7] (1.9500, 2.9000, 0.1195) -- (2.0000, 2.9000, 0.1165) -- (2.0000, 2.9500, 0.1102) -- (1.9500, 2.9500, 0.1132) -- cycle;
\fill[blue!15.0, opacity=0.7] (1.9500, 2.9500, 0.1132) -- (2.0000, 2.9500, 0.1102) -- (2.0000, 3.0000, 0.1039) -- (1.9500, 3.0000, 0.1069) -- cycle;
\fill[blue!15.0, opacity=0.7] (2.0000, 0.0000, 0.1039) -- (2.0500, 0.0000, 0.1006) -- (2.0500, 0.0500, 0.1069) -- (2.0000, 0.0500, 0.1102) -- cycle;
\fill[blue!15.0, opacity=0.7] (2.0000, 0.0500, 0.1102) -- (2.0500, 0.0500, 0.1069) -- (2.0500, 0.1000, 0.1132) -- (2.0000, 0.1000, 0.1165) -- cycle;
\fill[blue!15.0, opacity=0.7] (2.0000, 0.1000, 0.1165) -- (2.0500, 0.1000, 0.1132) -- (2.0500, 0.1500, 0.1194) -- (2.0000, 0.1500, 0.1227) -- cycle;
\fill[blue!15.0, opacity=0.7] (2.0000, 0.1500, 0.1227) -- (2.0500, 0.1500, 0.1194) -- (2.0500, 0.2000, 0.1256) -- (2.0000, 0.2000, 0.1289) -- cycle;
\fill[blue!15.0, opacity=0.7] (2.0000, 0.2000, 0.1289) -- (2.0500, 0.2000, 0.1256) -- (2.0500, 0.2500, 0.1317) -- (2.0000, 0.2500, 0.1350) -- cycle;
\fill[blue!15.0, opacity=0.7] (2.0000, 0.2500, 0.1350) -- (2.0500, 0.2500, 0.1317) -- (2.0500, 0.3000, 0.1377) -- (2.0000, 0.3000, 0.1410) -- cycle;
\fill[blue!15.0, opacity=0.7] (2.0000, 0.3000, 0.1410) -- (2.0500, 0.3000, 0.1377) -- (2.0500, 0.3500, 0.1436) -- (2.0000, 0.3500, 0.1469) -- cycle;
\fill[blue!15.0, opacity=0.7] (2.0000, 0.3500, 0.1469) -- (2.0500, 0.3500, 0.1436) -- (2.0500, 0.4000, 0.1494) -- (2.0000, 0.4000, 0.1527) -- cycle;
\fill[blue!15.0, opacity=0.7] (2.0000, 0.4000, 0.1527) -- (2.0500, 0.4000, 0.1494) -- (2.0500, 0.4500, 0.1551) -- (2.0000, 0.4500, 0.1584) -- cycle;
\fill[blue!15.0, opacity=0.7] (2.0000, 0.4500, 0.1584) -- (2.0500, 0.4500, 0.1551) -- (2.0500, 0.5000, 0.1606) -- (2.0000, 0.5000, 0.1639) -- cycle;
\fill[blue!15.0, opacity=0.7] (2.0000, 0.5000, 0.1639) -- (2.0500, 0.5000, 0.1606) -- (2.0500, 0.5500, 0.1660) -- (2.0000, 0.5500, 0.1693) -- cycle;
\fill[blue!15.0, opacity=0.7] (2.0000, 0.5500, 0.1693) -- (2.0500, 0.5500, 0.1660) -- (2.0500, 0.6000, 0.1712) -- (2.0000, 0.6000, 0.1745) -- cycle;
\fill[blue!15.0, opacity=0.7] (2.0000, 0.6000, 0.1745) -- (2.0500, 0.6000, 0.1712) -- (2.0500, 0.6500, 0.1762) -- (2.0000, 0.6500, 0.1794) -- cycle;
\fill[blue!15.0, opacity=0.7] (2.0000, 0.6500, 0.1794) -- (2.0500, 0.6500, 0.1762) -- (2.0500, 0.7000, 0.1809) -- (2.0000, 0.7000, 0.1842) -- cycle;
\fill[blue!15.0, opacity=0.7] (2.0000, 0.7000, 0.1842) -- (2.0500, 0.7000, 0.1809) -- (2.0500, 0.7500, 0.1855) -- (2.0000, 0.7500, 0.1888) -- cycle;
\fill[blue!15.0, opacity=0.7] (2.0000, 0.7500, 0.1888) -- (2.0500, 0.7500, 0.1855) -- (2.0500, 0.8000, 0.1898) -- (2.0000, 0.8000, 0.1931) -- cycle;
\fill[blue!15.0, opacity=0.7] (2.0000, 0.8000, 0.1931) -- (2.0500, 0.8000, 0.1898) -- (2.0500, 0.8500, 0.1939) -- (2.0000, 0.8500, 0.1972) -- cycle;
\fill[blue!15.0, opacity=0.7] (2.0000, 0.8500, 0.1972) -- (2.0500, 0.8500, 0.1939) -- (2.0500, 0.9000, 0.1977) -- (2.0000, 0.9000, 0.2010) -- cycle;
\fill[blue!15.0, opacity=0.7] (2.0000, 0.9000, 0.2010) -- (2.0500, 0.9000, 0.1977) -- (2.0500, 0.9500, 0.2013) -- (2.0000, 0.9500, 0.2046) -- cycle;
\fill[blue!15.0, opacity=0.7] (2.0000, 0.9500, 0.2046) -- (2.0500, 0.9500, 0.2013) -- (2.0500, 1.0000, 0.2046) -- (2.0000, 1.0000, 0.2078) -- cycle;
\fill[blue!15.0, opacity=0.7] (2.0000, 1.0000, 0.2078) -- (2.0500, 1.0000, 0.2046) -- (2.0500, 1.0500, 0.2076) -- (2.0000, 1.0500, 0.2108) -- cycle;
\fill[blue!18.9, opacity=0.7] (2.0000, 1.0500, 0.2108) -- (2.0500, 1.0500, 0.2076) -- (2.0500, 1.1000, 0.2103) -- (2.0000, 1.1000, 0.2135) -- cycle;
\fill[blue!48.2, opacity=0.7] (2.0000, 1.1000, 0.2135) -- (2.0500, 1.1000, 0.2103) -- (2.0500, 1.1500, 0.2127) -- (2.0000, 1.1500, 0.2160) -- cycle;
\fill[blue!89.6, opacity=0.7] (2.0000, 1.1500, 0.2160) -- (2.0500, 1.1500, 0.2127) -- (2.0500, 1.2000, 0.2148) -- (2.0000, 1.2000, 0.2180) -- cycle;
\fill[blue!67.7!black, opacity=0.7] (2.0000, 1.2000, 0.2180) -- (2.0500, 1.2000, 0.2148) -- (2.0500, 1.2500, 0.2166) -- (2.0000, 1.2500, 0.2198) -- cycle;
\fill[blue!45.0!black, opacity=0.7] (2.0000, 1.2500, 0.2198) -- (2.0500, 1.2500, 0.2166) -- (2.0500, 1.3000, 0.2180) -- (2.0000, 1.3000, 0.2213) -- cycle;
\fill[blue!34.9!black, opacity=0.7] (2.0000, 1.3000, 0.2213) -- (2.0500, 1.3000, 0.2180) -- (2.0500, 1.3500, 0.2192) -- (2.0000, 1.3500, 0.2224) -- cycle;
\fill[blue!26.0!black, opacity=0.7] (2.0000, 1.3500, 0.2224) -- (2.0500, 1.3500, 0.2192) -- (2.0500, 1.4000, 0.2200) -- (2.0000, 1.4000, 0.2233) -- cycle;
\fill[blue!16.6!black, opacity=0.7] (2.0000, 1.4000, 0.2233) -- (2.0500, 1.4000, 0.2200) -- (2.0500, 1.4500, 0.2205) -- (2.0000, 1.4500, 0.2238) -- cycle;
\fill[blue!10.6!black, opacity=0.7] (2.0000, 1.4500, 0.2238) -- (2.0500, 1.4500, 0.2205) -- (2.0500, 1.5000, 0.2206) -- (2.0000, 1.5000, 0.2239) -- cycle;
\fill[blue!11.8!black, opacity=0.7] (2.0000, 1.5000, 0.2239) -- (2.0500, 1.5000, 0.2206) -- (2.0500, 1.5500, 0.2205) -- (2.0000, 1.5500, 0.2238) -- cycle;
\fill[blue!38.5!black, opacity=0.7] (2.0000, 1.5500, 0.2238) -- (2.0500, 1.5500, 0.2205) -- (2.0500, 1.6000, 0.2200) -- (2.0000, 1.6000, 0.2233) -- cycle;
\fill[blue!79.0, opacity=0.7] (2.0000, 1.6000, 0.2233) -- (2.0500, 1.6000, 0.2200) -- (2.0500, 1.6500, 0.2192) -- (2.0000, 1.6500, 0.2224) -- cycle;
\fill[blue!24.0, opacity=0.7] (2.0000, 1.6500, 0.2224) -- (2.0500, 1.6500, 0.2192) -- (2.0500, 1.7000, 0.2180) -- (2.0000, 1.7000, 0.2213) -- cycle;
\fill[blue!15.1, opacity=0.7] (2.0000, 1.7000, 0.2213) -- (2.0500, 1.7000, 0.2180) -- (2.0500, 1.7500, 0.2166) -- (2.0000, 1.7500, 0.2198) -- cycle;
\fill[blue!15.0, opacity=0.7] (2.0000, 1.7500, 0.2198) -- (2.0500, 1.7500, 0.2166) -- (2.0500, 1.8000, 0.2148) -- (2.0000, 1.8000, 0.2180) -- cycle;
\fill[blue!15.0, opacity=0.7] (2.0000, 1.8000, 0.2180) -- (2.0500, 1.8000, 0.2148) -- (2.0500, 1.8500, 0.2127) -- (2.0000, 1.8500, 0.2160) -- cycle;
\fill[blue!15.0, opacity=0.7] (2.0000, 1.8500, 0.2160) -- (2.0500, 1.8500, 0.2127) -- (2.0500, 1.9000, 0.2103) -- (2.0000, 1.9000, 0.2135) -- cycle;
\fill[blue!15.0, opacity=0.7] (2.0000, 1.9000, 0.2135) -- (2.0500, 1.9000, 0.2103) -- (2.0500, 1.9500, 0.2076) -- (2.0000, 1.9500, 0.2108) -- cycle;
\fill[blue!15.0, opacity=0.7] (2.0000, 1.9500, 0.2108) -- (2.0500, 1.9500, 0.2076) -- (2.0500, 2.0000, 0.2046) -- (2.0000, 2.0000, 0.2078) -- cycle;
\fill[blue!15.0, opacity=0.7] (2.0000, 2.0000, 0.2078) -- (2.0500, 2.0000, 0.2046) -- (2.0500, 2.0500, 0.2013) -- (2.0000, 2.0500, 0.2046) -- cycle;
\fill[blue!16.2, opacity=0.7] (2.0000, 2.0500, 0.2046) -- (2.0500, 2.0500, 0.2013) -- (2.0500, 2.1000, 0.1977) -- (2.0000, 2.1000, 0.2010) -- cycle;
\fill[blue!24.1, opacity=0.7] (2.0000, 2.1000, 0.2010) -- (2.0500, 2.1000, 0.1977) -- (2.0500, 2.1500, 0.1939) -- (2.0000, 2.1500, 0.1972) -- cycle;
\fill[blue!16.5, opacity=0.7] (2.0000, 2.1500, 0.1972) -- (2.0500, 2.1500, 0.1939) -- (2.0500, 2.2000, 0.1898) -- (2.0000, 2.2000, 0.1931) -- cycle;
\fill[blue!15.0, opacity=0.7] (2.0000, 2.2000, 0.1931) -- (2.0500, 2.2000, 0.1898) -- (2.0500, 2.2500, 0.1855) -- (2.0000, 2.2500, 0.1888) -- cycle;
\fill[blue!15.0, opacity=0.7] (2.0000, 2.2500, 0.1888) -- (2.0500, 2.2500, 0.1855) -- (2.0500, 2.3000, 0.1809) -- (2.0000, 2.3000, 0.1842) -- cycle;
\fill[blue!15.0, opacity=0.7] (2.0000, 2.3000, 0.1842) -- (2.0500, 2.3000, 0.1809) -- (2.0500, 2.3500, 0.1762) -- (2.0000, 2.3500, 0.1794) -- cycle;
\fill[blue!15.0, opacity=0.7] (2.0000, 2.3500, 0.1794) -- (2.0500, 2.3500, 0.1762) -- (2.0500, 2.4000, 0.1712) -- (2.0000, 2.4000, 0.1745) -- cycle;
\fill[blue!15.0, opacity=0.7] (2.0000, 2.4000, 0.1745) -- (2.0500, 2.4000, 0.1712) -- (2.0500, 2.4500, 0.1660) -- (2.0000, 2.4500, 0.1693) -- cycle;
\fill[blue!15.0, opacity=0.7] (2.0000, 2.4500, 0.1693) -- (2.0500, 2.4500, 0.1660) -- (2.0500, 2.5000, 0.1606) -- (2.0000, 2.5000, 0.1639) -- cycle;
\fill[blue!15.0, opacity=0.7] (2.0000, 2.5000, 0.1639) -- (2.0500, 2.5000, 0.1606) -- (2.0500, 2.5500, 0.1551) -- (2.0000, 2.5500, 0.1584) -- cycle;
\fill[blue!15.0, opacity=0.7] (2.0000, 2.5500, 0.1584) -- (2.0500, 2.5500, 0.1551) -- (2.0500, 2.6000, 0.1494) -- (2.0000, 2.6000, 0.1527) -- cycle;
\fill[blue!15.0, opacity=0.7] (2.0000, 2.6000, 0.1527) -- (2.0500, 2.6000, 0.1494) -- (2.0500, 2.6500, 0.1436) -- (2.0000, 2.6500, 0.1469) -- cycle;
\fill[blue!15.0, opacity=0.7] (2.0000, 2.6500, 0.1469) -- (2.0500, 2.6500, 0.1436) -- (2.0500, 2.7000, 0.1377) -- (2.0000, 2.7000, 0.1410) -- cycle;
\fill[blue!15.0, opacity=0.7] (2.0000, 2.7000, 0.1410) -- (2.0500, 2.7000, 0.1377) -- (2.0500, 2.7500, 0.1317) -- (2.0000, 2.7500, 0.1350) -- cycle;
\fill[blue!15.0, opacity=0.7] (2.0000, 2.7500, 0.1350) -- (2.0500, 2.7500, 0.1317) -- (2.0500, 2.8000, 0.1256) -- (2.0000, 2.8000, 0.1289) -- cycle;
\fill[blue!15.0, opacity=0.7] (2.0000, 2.8000, 0.1289) -- (2.0500, 2.8000, 0.1256) -- (2.0500, 2.8500, 0.1194) -- (2.0000, 2.8500, 0.1227) -- cycle;
\fill[blue!15.0, opacity=0.7] (2.0000, 2.8500, 0.1227) -- (2.0500, 2.8500, 0.1194) -- (2.0500, 2.9000, 0.1132) -- (2.0000, 2.9000, 0.1165) -- cycle;
\fill[blue!15.0, opacity=0.7] (2.0000, 2.9000, 0.1165) -- (2.0500, 2.9000, 0.1132) -- (2.0500, 2.9500, 0.1069) -- (2.0000, 2.9500, 0.1102) -- cycle;
\fill[blue!15.0, opacity=0.7] (2.0000, 2.9500, 0.1102) -- (2.0500, 2.9500, 0.1069) -- (2.0500, 3.0000, 0.1006) -- (2.0000, 3.0000, 0.1039) -- cycle;
\fill[blue!15.0, opacity=0.7] (2.0500, 0.0000, 0.1006) -- (2.1000, 0.0000, 0.0971) -- (2.1000, 0.0500, 0.1034) -- (2.0500, 0.0500, 0.1069) -- cycle;
\fill[blue!15.0, opacity=0.7] (2.0500, 0.0500, 0.1069) -- (2.1000, 0.0500, 0.1034) -- (2.1000, 0.1000, 0.1096) -- (2.0500, 0.1000, 0.1132) -- cycle;
\fill[blue!15.0, opacity=0.7] (2.0500, 0.1000, 0.1132) -- (2.1000, 0.1000, 0.1096) -- (2.1000, 0.1500, 0.1159) -- (2.0500, 0.1500, 0.1194) -- cycle;
\fill[blue!15.0, opacity=0.7] (2.0500, 0.1500, 0.1194) -- (2.1000, 0.1500, 0.1159) -- (2.1000, 0.2000, 0.1220) -- (2.0500, 0.2000, 0.1256) -- cycle;
\fill[blue!15.0, opacity=0.7] (2.0500, 0.2000, 0.1256) -- (2.1000, 0.2000, 0.1220) -- (2.1000, 0.2500, 0.1281) -- (2.0500, 0.2500, 0.1317) -- cycle;
\fill[blue!15.0, opacity=0.7] (2.0500, 0.2500, 0.1317) -- (2.1000, 0.2500, 0.1281) -- (2.1000, 0.3000, 0.1342) -- (2.0500, 0.3000, 0.1377) -- cycle;
\fill[blue!15.0, opacity=0.7] (2.0500, 0.3000, 0.1377) -- (2.1000, 0.3000, 0.1342) -- (2.1000, 0.3500, 0.1401) -- (2.0500, 0.3500, 0.1436) -- cycle;
\fill[blue!15.0, opacity=0.7] (2.0500, 0.3500, 0.1436) -- (2.1000, 0.3500, 0.1401) -- (2.1000, 0.4000, 0.1459) -- (2.0500, 0.4000, 0.1494) -- cycle;
\fill[blue!15.0, opacity=0.7] (2.0500, 0.4000, 0.1494) -- (2.1000, 0.4000, 0.1459) -- (2.1000, 0.4500, 0.1516) -- (2.0500, 0.4500, 0.1551) -- cycle;
\fill[blue!15.0, opacity=0.7] (2.0500, 0.4500, 0.1551) -- (2.1000, 0.4500, 0.1516) -- (2.1000, 0.5000, 0.1571) -- (2.0500, 0.5000, 0.1606) -- cycle;
\fill[blue!15.0, opacity=0.7] (2.0500, 0.5000, 0.1606) -- (2.1000, 0.5000, 0.1571) -- (2.1000, 0.5500, 0.1624) -- (2.0500, 0.5500, 0.1660) -- cycle;
\fill[blue!15.0, opacity=0.7] (2.0500, 0.5500, 0.1660) -- (2.1000, 0.5500, 0.1624) -- (2.1000, 0.6000, 0.1676) -- (2.0500, 0.6000, 0.1712) -- cycle;
\fill[blue!15.0, opacity=0.7] (2.0500, 0.6000, 0.1712) -- (2.1000, 0.6000, 0.1676) -- (2.1000, 0.6500, 0.1726) -- (2.0500, 0.6500, 0.1762) -- cycle;
\fill[blue!15.0, opacity=0.7] (2.0500, 0.6500, 0.1762) -- (2.1000, 0.6500, 0.1726) -- (2.1000, 0.7000, 0.1774) -- (2.0500, 0.7000, 0.1809) -- cycle;
\fill[blue!15.0, opacity=0.7] (2.0500, 0.7000, 0.1809) -- (2.1000, 0.7000, 0.1774) -- (2.1000, 0.7500, 0.1819) -- (2.0500, 0.7500, 0.1855) -- cycle;
\fill[blue!15.0, opacity=0.7] (2.0500, 0.7500, 0.1855) -- (2.1000, 0.7500, 0.1819) -- (2.1000, 0.8000, 0.1863) -- (2.0500, 0.8000, 0.1898) -- cycle;
\fill[blue!15.0, opacity=0.7] (2.0500, 0.8000, 0.1898) -- (2.1000, 0.8000, 0.1863) -- (2.1000, 0.8500, 0.1903) -- (2.0500, 0.8500, 0.1939) -- cycle;
\fill[blue!15.0, opacity=0.7] (2.0500, 0.8500, 0.1939) -- (2.1000, 0.8500, 0.1903) -- (2.1000, 0.9000, 0.1942) -- (2.0500, 0.9000, 0.1977) -- cycle;
\fill[blue!15.0, opacity=0.7] (2.0500, 0.9000, 0.1977) -- (2.1000, 0.9000, 0.1942) -- (2.1000, 0.9500, 0.1977) -- (2.0500, 0.9500, 0.2013) -- cycle;
\fill[blue!15.0, opacity=0.7] (2.0500, 0.9500, 0.2013) -- (2.1000, 0.9500, 0.1977) -- (2.1000, 1.0000, 0.2010) -- (2.0500, 1.0000, 0.2046) -- cycle;
\fill[blue!15.0, opacity=0.7] (2.0500, 1.0000, 0.2046) -- (2.1000, 1.0000, 0.2010) -- (2.1000, 1.0500, 0.2040) -- (2.0500, 1.0500, 0.2076) -- cycle;
\fill[blue!15.0, opacity=0.7] (2.0500, 1.0500, 0.2076) -- (2.1000, 1.0500, 0.2040) -- (2.1000, 1.1000, 0.2067) -- (2.0500, 1.1000, 0.2103) -- cycle;
\fill[blue!15.1, opacity=0.7] (2.0500, 1.1000, 0.2103) -- (2.1000, 1.1000, 0.2067) -- (2.1000, 1.1500, 0.2091) -- (2.0500, 1.1500, 0.2127) -- cycle;
\fill[blue!17.2, opacity=0.7] (2.0500, 1.1500, 0.2127) -- (2.1000, 1.1500, 0.2091) -- (2.1000, 1.2000, 0.2112) -- (2.0500, 1.2000, 0.2148) -- cycle;
\fill[blue!28.9, opacity=0.7] (2.0500, 1.2000, 0.2148) -- (2.1000, 1.2000, 0.2112) -- (2.1000, 1.2500, 0.2130) -- (2.0500, 1.2500, 0.2166) -- cycle;
\fill[blue!50.0, opacity=0.7] (2.0500, 1.2500, 0.2166) -- (2.1000, 1.2500, 0.2130) -- (2.1000, 1.3000, 0.2145) -- (2.0500, 1.3000, 0.2180) -- cycle;
\fill[blue!68.9, opacity=0.7] (2.0500, 1.3000, 0.2180) -- (2.1000, 1.3000, 0.2145) -- (2.1000, 1.3500, 0.2156) -- (2.0500, 1.3500, 0.2192) -- cycle;
\fill[blue!78.7, opacity=0.7] (2.0500, 1.3500, 0.2192) -- (2.1000, 1.3500, 0.2156) -- (2.1000, 1.4000, 0.2164) -- (2.0500, 1.4000, 0.2200) -- cycle;
\fill[blue!77.5, opacity=0.7] (2.0500, 1.4000, 0.2200) -- (2.1000, 1.4000, 0.2164) -- (2.1000, 1.4500, 0.2169) -- (2.0500, 1.4500, 0.2205) -- cycle;
\fill[blue!63.9, opacity=0.7] (2.0500, 1.4500, 0.2205) -- (2.1000, 1.4500, 0.2169) -- (2.1000, 1.5000, 0.2171) -- (2.0500, 1.5000, 0.2206) -- cycle;
\fill[blue!39.9, opacity=0.7] (2.0500, 1.5000, 0.2206) -- (2.1000, 1.5000, 0.2171) -- (2.1000, 1.5500, 0.2169) -- (2.0500, 1.5500, 0.2205) -- cycle;
\fill[blue!20.2, opacity=0.7] (2.0500, 1.5500, 0.2205) -- (2.1000, 1.5500, 0.2169) -- (2.1000, 1.6000, 0.2164) -- (2.0500, 1.6000, 0.2200) -- cycle;
\fill[blue!15.2, opacity=0.7] (2.0500, 1.6000, 0.2200) -- (2.1000, 1.6000, 0.2164) -- (2.1000, 1.6500, 0.2156) -- (2.0500, 1.6500, 0.2192) -- cycle;
\fill[blue!15.0, opacity=0.7] (2.0500, 1.6500, 0.2192) -- (2.1000, 1.6500, 0.2156) -- (2.1000, 1.7000, 0.2145) -- (2.0500, 1.7000, 0.2180) -- cycle;
\fill[blue!15.0, opacity=0.7] (2.0500, 1.7000, 0.2180) -- (2.1000, 1.7000, 0.2145) -- (2.1000, 1.7500, 0.2130) -- (2.0500, 1.7500, 0.2166) -- cycle;
\fill[blue!15.0, opacity=0.7] (2.0500, 1.7500, 0.2166) -- (2.1000, 1.7500, 0.2130) -- (2.1000, 1.8000, 0.2112) -- (2.0500, 1.8000, 0.2148) -- cycle;
\fill[blue!15.0, opacity=0.7] (2.0500, 1.8000, 0.2148) -- (2.1000, 1.8000, 0.2112) -- (2.1000, 1.8500, 0.2091) -- (2.0500, 1.8500, 0.2127) -- cycle;
\fill[blue!15.0, opacity=0.7] (2.0500, 1.8500, 0.2127) -- (2.1000, 1.8500, 0.2091) -- (2.1000, 1.9000, 0.2067) -- (2.0500, 1.9000, 0.2103) -- cycle;
\fill[blue!15.0, opacity=0.7] (2.0500, 1.9000, 0.2103) -- (2.1000, 1.9000, 0.2067) -- (2.1000, 1.9500, 0.2040) -- (2.0500, 1.9500, 0.2076) -- cycle;
\fill[blue!15.0, opacity=0.7] (2.0500, 1.9500, 0.2076) -- (2.1000, 1.9500, 0.2040) -- (2.1000, 2.0000, 0.2010) -- (2.0500, 2.0000, 0.2046) -- cycle;
\fill[blue!15.2, opacity=0.7] (2.0500, 2.0000, 0.2046) -- (2.1000, 2.0000, 0.2010) -- (2.1000, 2.0500, 0.1977) -- (2.0500, 2.0500, 0.2013) -- cycle;
\fill[blue!21.0, opacity=0.7] (2.0500, 2.0500, 0.2013) -- (2.1000, 2.0500, 0.1977) -- (2.1000, 2.1000, 0.1942) -- (2.0500, 2.1000, 0.1977) -- cycle;
\fill[blue!19.3, opacity=0.7] (2.0500, 2.1000, 0.1977) -- (2.1000, 2.1000, 0.1942) -- (2.1000, 2.1500, 0.1903) -- (2.0500, 2.1500, 0.1939) -- cycle;
\fill[blue!15.0, opacity=0.7] (2.0500, 2.1500, 0.1939) -- (2.1000, 2.1500, 0.1903) -- (2.1000, 2.2000, 0.1863) -- (2.0500, 2.2000, 0.1898) -- cycle;
\fill[blue!15.0, opacity=0.7] (2.0500, 2.2000, 0.1898) -- (2.1000, 2.2000, 0.1863) -- (2.1000, 2.2500, 0.1819) -- (2.0500, 2.2500, 0.1855) -- cycle;
\fill[blue!15.0, opacity=0.7] (2.0500, 2.2500, 0.1855) -- (2.1000, 2.2500, 0.1819) -- (2.1000, 2.3000, 0.1774) -- (2.0500, 2.3000, 0.1809) -- cycle;
\fill[blue!15.0, opacity=0.7] (2.0500, 2.3000, 0.1809) -- (2.1000, 2.3000, 0.1774) -- (2.1000, 2.3500, 0.1726) -- (2.0500, 2.3500, 0.1762) -- cycle;
\fill[blue!15.0, opacity=0.7] (2.0500, 2.3500, 0.1762) -- (2.1000, 2.3500, 0.1726) -- (2.1000, 2.4000, 0.1676) -- (2.0500, 2.4000, 0.1712) -- cycle;
\fill[blue!15.0, opacity=0.7] (2.0500, 2.4000, 0.1712) -- (2.1000, 2.4000, 0.1676) -- (2.1000, 2.4500, 0.1624) -- (2.0500, 2.4500, 0.1660) -- cycle;
\fill[blue!15.0, opacity=0.7] (2.0500, 2.4500, 0.1660) -- (2.1000, 2.4500, 0.1624) -- (2.1000, 2.5000, 0.1571) -- (2.0500, 2.5000, 0.1606) -- cycle;
\fill[blue!15.0, opacity=0.7] (2.0500, 2.5000, 0.1606) -- (2.1000, 2.5000, 0.1571) -- (2.1000, 2.5500, 0.1516) -- (2.0500, 2.5500, 0.1551) -- cycle;
\fill[blue!15.0, opacity=0.7] (2.0500, 2.5500, 0.1551) -- (2.1000, 2.5500, 0.1516) -- (2.1000, 2.6000, 0.1459) -- (2.0500, 2.6000, 0.1494) -- cycle;
\fill[blue!15.0, opacity=0.7] (2.0500, 2.6000, 0.1494) -- (2.1000, 2.6000, 0.1459) -- (2.1000, 2.6500, 0.1401) -- (2.0500, 2.6500, 0.1436) -- cycle;
\fill[blue!15.0, opacity=0.7] (2.0500, 2.6500, 0.1436) -- (2.1000, 2.6500, 0.1401) -- (2.1000, 2.7000, 0.1342) -- (2.0500, 2.7000, 0.1377) -- cycle;
\fill[blue!15.0, opacity=0.7] (2.0500, 2.7000, 0.1377) -- (2.1000, 2.7000, 0.1342) -- (2.1000, 2.7500, 0.1281) -- (2.0500, 2.7500, 0.1317) -- cycle;
\fill[blue!15.0, opacity=0.7] (2.0500, 2.7500, 0.1317) -- (2.1000, 2.7500, 0.1281) -- (2.1000, 2.8000, 0.1220) -- (2.0500, 2.8000, 0.1256) -- cycle;
\fill[blue!15.0, opacity=0.7] (2.0500, 2.8000, 0.1256) -- (2.1000, 2.8000, 0.1220) -- (2.1000, 2.8500, 0.1159) -- (2.0500, 2.8500, 0.1194) -- cycle;
\fill[blue!15.0, opacity=0.7] (2.0500, 2.8500, 0.1194) -- (2.1000, 2.8500, 0.1159) -- (2.1000, 2.9000, 0.1096) -- (2.0500, 2.9000, 0.1132) -- cycle;
\fill[blue!15.0, opacity=0.7] (2.0500, 2.9000, 0.1132) -- (2.1000, 2.9000, 0.1096) -- (2.1000, 2.9500, 0.1034) -- (2.0500, 2.9500, 0.1069) -- cycle;
\fill[blue!15.0, opacity=0.7] (2.0500, 2.9500, 0.1069) -- (2.1000, 2.9500, 0.1034) -- (2.1000, 3.0000, 0.0971) -- (2.0500, 3.0000, 0.1006) -- cycle;
\fill[blue!15.0, opacity=0.7] (2.1000, 0.0000, 0.0971) -- (2.1500, 0.0000, 0.0933) -- (2.1500, 0.0500, 0.0995) -- (2.1000, 0.0500, 0.1034) -- cycle;
\fill[blue!15.0, opacity=0.7] (2.1000, 0.0500, 0.1034) -- (2.1500, 0.0500, 0.0995) -- (2.1500, 0.1000, 0.1058) -- (2.1000, 0.1000, 0.1096) -- cycle;
\fill[blue!15.0, opacity=0.7] (2.1000, 0.1000, 0.1096) -- (2.1500, 0.1000, 0.1058) -- (2.1500, 0.1500, 0.1120) -- (2.1000, 0.1500, 0.1159) -- cycle;
\fill[blue!15.0, opacity=0.7] (2.1000, 0.1500, 0.1159) -- (2.1500, 0.1500, 0.1120) -- (2.1500, 0.2000, 0.1182) -- (2.1000, 0.2000, 0.1220) -- cycle;
\fill[blue!15.0, opacity=0.7] (2.1000, 0.2000, 0.1220) -- (2.1500, 0.2000, 0.1182) -- (2.1500, 0.2500, 0.1243) -- (2.1000, 0.2500, 0.1281) -- cycle;
\fill[blue!15.0, opacity=0.7] (2.1000, 0.2500, 0.1281) -- (2.1500, 0.2500, 0.1243) -- (2.1500, 0.3000, 0.1303) -- (2.1000, 0.3000, 0.1342) -- cycle;
\fill[blue!15.0, opacity=0.7] (2.1000, 0.3000, 0.1342) -- (2.1500, 0.3000, 0.1303) -- (2.1500, 0.3500, 0.1363) -- (2.1000, 0.3500, 0.1401) -- cycle;
\fill[blue!15.0, opacity=0.7] (2.1000, 0.3500, 0.1401) -- (2.1500, 0.3500, 0.1363) -- (2.1500, 0.4000, 0.1421) -- (2.1000, 0.4000, 0.1459) -- cycle;
\fill[blue!15.0, opacity=0.7] (2.1000, 0.4000, 0.1459) -- (2.1500, 0.4000, 0.1421) -- (2.1500, 0.4500, 0.1477) -- (2.1000, 0.4500, 0.1516) -- cycle;
\fill[blue!15.0, opacity=0.7] (2.1000, 0.4500, 0.1516) -- (2.1500, 0.4500, 0.1477) -- (2.1500, 0.5000, 0.1533) -- (2.1000, 0.5000, 0.1571) -- cycle;
\fill[blue!15.0, opacity=0.7] (2.1000, 0.5000, 0.1571) -- (2.1500, 0.5000, 0.1533) -- (2.1500, 0.5500, 0.1586) -- (2.1000, 0.5500, 0.1624) -- cycle;
\fill[blue!15.0, opacity=0.7] (2.1000, 0.5500, 0.1624) -- (2.1500, 0.5500, 0.1586) -- (2.1500, 0.6000, 0.1638) -- (2.1000, 0.6000, 0.1676) -- cycle;
\fill[blue!15.0, opacity=0.7] (2.1000, 0.6000, 0.1676) -- (2.1500, 0.6000, 0.1638) -- (2.1500, 0.6500, 0.1688) -- (2.1000, 0.6500, 0.1726) -- cycle;
\fill[blue!15.0, opacity=0.7] (2.1000, 0.6500, 0.1726) -- (2.1500, 0.6500, 0.1688) -- (2.1500, 0.7000, 0.1736) -- (2.1000, 0.7000, 0.1774) -- cycle;
\fill[blue!15.0, opacity=0.7] (2.1000, 0.7000, 0.1774) -- (2.1500, 0.7000, 0.1736) -- (2.1500, 0.7500, 0.1781) -- (2.1000, 0.7500, 0.1819) -- cycle;
\fill[blue!15.0, opacity=0.7] (2.1000, 0.7500, 0.1819) -- (2.1500, 0.7500, 0.1781) -- (2.1500, 0.8000, 0.1824) -- (2.1000, 0.8000, 0.1863) -- cycle;
\fill[blue!15.0, opacity=0.7] (2.1000, 0.8000, 0.1863) -- (2.1500, 0.8000, 0.1824) -- (2.1500, 0.8500, 0.1865) -- (2.1000, 0.8500, 0.1903) -- cycle;
\fill[blue!15.0, opacity=0.7] (2.1000, 0.8500, 0.1903) -- (2.1500, 0.8500, 0.1865) -- (2.1500, 0.9000, 0.1903) -- (2.1000, 0.9000, 0.1942) -- cycle;
\fill[blue!15.0, opacity=0.7] (2.1000, 0.9000, 0.1942) -- (2.1500, 0.9000, 0.1903) -- (2.1500, 0.9500, 0.1939) -- (2.1000, 0.9500, 0.1977) -- cycle;
\fill[blue!15.0, opacity=0.7] (2.1000, 0.9500, 0.1977) -- (2.1500, 0.9500, 0.1939) -- (2.1500, 1.0000, 0.1972) -- (2.1000, 1.0000, 0.2010) -- cycle;
\fill[blue!15.0, opacity=0.7] (2.1000, 1.0000, 0.2010) -- (2.1500, 1.0000, 0.1972) -- (2.1500, 1.0500, 0.2002) -- (2.1000, 1.0500, 0.2040) -- cycle;
\fill[blue!15.0, opacity=0.7] (2.1000, 1.0500, 0.2040) -- (2.1500, 1.0500, 0.2002) -- (2.1500, 1.1000, 0.2029) -- (2.1000, 1.1000, 0.2067) -- cycle;
\fill[blue!15.0, opacity=0.7] (2.1000, 1.1000, 0.2067) -- (2.1500, 1.1000, 0.2029) -- (2.1500, 1.1500, 0.2053) -- (2.1000, 1.1500, 0.2091) -- cycle;
\fill[blue!15.0, opacity=0.7] (2.1000, 1.1500, 0.2091) -- (2.1500, 1.1500, 0.2053) -- (2.1500, 1.2000, 0.2074) -- (2.1000, 1.2000, 0.2112) -- cycle;
\fill[blue!15.0, opacity=0.7] (2.1000, 1.2000, 0.2112) -- (2.1500, 1.2000, 0.2074) -- (2.1500, 1.2500, 0.2092) -- (2.1000, 1.2500, 0.2130) -- cycle;
\fill[blue!15.0, opacity=0.7] (2.1000, 1.2500, 0.2130) -- (2.1500, 1.2500, 0.2092) -- (2.1500, 1.3000, 0.2106) -- (2.1000, 1.3000, 0.2145) -- cycle;
\fill[blue!15.1, opacity=0.7] (2.1000, 1.3000, 0.2145) -- (2.1500, 1.3000, 0.2106) -- (2.1500, 1.3500, 0.2118) -- (2.1000, 1.3500, 0.2156) -- cycle;
\fill[blue!15.2, opacity=0.7] (2.1000, 1.3500, 0.2156) -- (2.1500, 1.3500, 0.2118) -- (2.1500, 1.4000, 0.2126) -- (2.1000, 1.4000, 0.2164) -- cycle;
\fill[blue!15.1, opacity=0.7] (2.1000, 1.4000, 0.2164) -- (2.1500, 1.4000, 0.2126) -- (2.1500, 1.4500, 0.2131) -- (2.1000, 1.4500, 0.2169) -- cycle;
\fill[blue!15.0, opacity=0.7] (2.1000, 1.4500, 0.2169) -- (2.1500, 1.4500, 0.2131) -- (2.1500, 1.5000, 0.2133) -- (2.1000, 1.5000, 0.2171) -- cycle;
\fill[blue!15.0, opacity=0.7] (2.1000, 1.5000, 0.2171) -- (2.1500, 1.5000, 0.2133) -- (2.1500, 1.5500, 0.2131) -- (2.1000, 1.5500, 0.2169) -- cycle;
\fill[blue!15.0, opacity=0.7] (2.1000, 1.5500, 0.2169) -- (2.1500, 1.5500, 0.2131) -- (2.1500, 1.6000, 0.2126) -- (2.1000, 1.6000, 0.2164) -- cycle;
\fill[blue!15.0, opacity=0.7] (2.1000, 1.6000, 0.2164) -- (2.1500, 1.6000, 0.2126) -- (2.1500, 1.6500, 0.2118) -- (2.1000, 1.6500, 0.2156) -- cycle;
\fill[blue!15.0, opacity=0.7] (2.1000, 1.6500, 0.2156) -- (2.1500, 1.6500, 0.2118) -- (2.1500, 1.7000, 0.2106) -- (2.1000, 1.7000, 0.2145) -- cycle;
\fill[blue!15.0, opacity=0.7] (2.1000, 1.7000, 0.2145) -- (2.1500, 1.7000, 0.2106) -- (2.1500, 1.7500, 0.2092) -- (2.1000, 1.7500, 0.2130) -- cycle;
\fill[blue!15.0, opacity=0.7] (2.1000, 1.7500, 0.2130) -- (2.1500, 1.7500, 0.2092) -- (2.1500, 1.8000, 0.2074) -- (2.1000, 1.8000, 0.2112) -- cycle;
\fill[blue!15.0, opacity=0.7] (2.1000, 1.8000, 0.2112) -- (2.1500, 1.8000, 0.2074) -- (2.1500, 1.8500, 0.2053) -- (2.1000, 1.8500, 0.2091) -- cycle;
\fill[blue!15.0, opacity=0.7] (2.1000, 1.8500, 0.2091) -- (2.1500, 1.8500, 0.2053) -- (2.1500, 1.9000, 0.2029) -- (2.1000, 1.9000, 0.2067) -- cycle;
\fill[blue!15.0, opacity=0.7] (2.1000, 1.9000, 0.2067) -- (2.1500, 1.9000, 0.2029) -- (2.1500, 1.9500, 0.2002) -- (2.1000, 1.9500, 0.2040) -- cycle;
\fill[blue!15.1, opacity=0.7] (2.1000, 1.9500, 0.2040) -- (2.1500, 1.9500, 0.2002) -- (2.1500, 2.0000, 0.1972) -- (2.1000, 2.0000, 0.2010) -- cycle;
\fill[blue!18.5, opacity=0.7] (2.1000, 2.0000, 0.2010) -- (2.1500, 2.0000, 0.1972) -- (2.1500, 2.0500, 0.1939) -- (2.1000, 2.0500, 0.1977) -- cycle;
\fill[blue!20.6, opacity=0.7] (2.1000, 2.0500, 0.1977) -- (2.1500, 2.0500, 0.1939) -- (2.1500, 2.1000, 0.1903) -- (2.1000, 2.1000, 0.1942) -- cycle;
\fill[blue!15.2, opacity=0.7] (2.1000, 2.1000, 0.1942) -- (2.1500, 2.1000, 0.1903) -- (2.1500, 2.1500, 0.1865) -- (2.1000, 2.1500, 0.1903) -- cycle;
\fill[blue!15.0, opacity=0.7] (2.1000, 2.1500, 0.1903) -- (2.1500, 2.1500, 0.1865) -- (2.1500, 2.2000, 0.1824) -- (2.1000, 2.2000, 0.1863) -- cycle;
\fill[blue!15.0, opacity=0.7] (2.1000, 2.2000, 0.1863) -- (2.1500, 2.2000, 0.1824) -- (2.1500, 2.2500, 0.1781) -- (2.1000, 2.2500, 0.1819) -- cycle;
\fill[blue!15.0, opacity=0.7] (2.1000, 2.2500, 0.1819) -- (2.1500, 2.2500, 0.1781) -- (2.1500, 2.3000, 0.1736) -- (2.1000, 2.3000, 0.1774) -- cycle;
\fill[blue!15.0, opacity=0.7] (2.1000, 2.3000, 0.1774) -- (2.1500, 2.3000, 0.1736) -- (2.1500, 2.3500, 0.1688) -- (2.1000, 2.3500, 0.1726) -- cycle;
\fill[blue!15.0, opacity=0.7] (2.1000, 2.3500, 0.1726) -- (2.1500, 2.3500, 0.1688) -- (2.1500, 2.4000, 0.1638) -- (2.1000, 2.4000, 0.1676) -- cycle;
\fill[blue!15.0, opacity=0.7] (2.1000, 2.4000, 0.1676) -- (2.1500, 2.4000, 0.1638) -- (2.1500, 2.4500, 0.1586) -- (2.1000, 2.4500, 0.1624) -- cycle;
\fill[blue!15.0, opacity=0.7] (2.1000, 2.4500, 0.1624) -- (2.1500, 2.4500, 0.1586) -- (2.1500, 2.5000, 0.1533) -- (2.1000, 2.5000, 0.1571) -- cycle;
\fill[blue!15.0, opacity=0.7] (2.1000, 2.5000, 0.1571) -- (2.1500, 2.5000, 0.1533) -- (2.1500, 2.5500, 0.1477) -- (2.1000, 2.5500, 0.1516) -- cycle;
\fill[blue!15.0, opacity=0.7] (2.1000, 2.5500, 0.1516) -- (2.1500, 2.5500, 0.1477) -- (2.1500, 2.6000, 0.1421) -- (2.1000, 2.6000, 0.1459) -- cycle;
\fill[blue!15.0, opacity=0.7] (2.1000, 2.6000, 0.1459) -- (2.1500, 2.6000, 0.1421) -- (2.1500, 2.6500, 0.1363) -- (2.1000, 2.6500, 0.1401) -- cycle;
\fill[blue!15.0, opacity=0.7] (2.1000, 2.6500, 0.1401) -- (2.1500, 2.6500, 0.1363) -- (2.1500, 2.7000, 0.1303) -- (2.1000, 2.7000, 0.1342) -- cycle;
\fill[blue!15.0, opacity=0.7] (2.1000, 2.7000, 0.1342) -- (2.1500, 2.7000, 0.1303) -- (2.1500, 2.7500, 0.1243) -- (2.1000, 2.7500, 0.1281) -- cycle;
\fill[blue!15.0, opacity=0.7] (2.1000, 2.7500, 0.1281) -- (2.1500, 2.7500, 0.1243) -- (2.1500, 2.8000, 0.1182) -- (2.1000, 2.8000, 0.1220) -- cycle;
\fill[blue!15.0, opacity=0.7] (2.1000, 2.8000, 0.1220) -- (2.1500, 2.8000, 0.1182) -- (2.1500, 2.8500, 0.1120) -- (2.1000, 2.8500, 0.1159) -- cycle;
\fill[blue!15.0, opacity=0.7] (2.1000, 2.8500, 0.1159) -- (2.1500, 2.8500, 0.1120) -- (2.1500, 2.9000, 0.1058) -- (2.1000, 2.9000, 0.1096) -- cycle;
\fill[blue!15.0, opacity=0.7] (2.1000, 2.9000, 0.1096) -- (2.1500, 2.9000, 0.1058) -- (2.1500, 2.9500, 0.0995) -- (2.1000, 2.9500, 0.1034) -- cycle;
\fill[blue!15.0, opacity=0.7] (2.1000, 2.9500, 0.1034) -- (2.1500, 2.9500, 0.0995) -- (2.1500, 3.0000, 0.0933) -- (2.1000, 3.0000, 0.0971) -- cycle;
\fill[blue!15.0, opacity=0.7] (2.1500, 0.0000, 0.0933) -- (2.2000, 0.0000, 0.0892) -- (2.2000, 0.0500, 0.0955) -- (2.1500, 0.0500, 0.0995) -- cycle;
\fill[blue!15.0, opacity=0.7] (2.1500, 0.0500, 0.0995) -- (2.2000, 0.0500, 0.0955) -- (2.2000, 0.1000, 0.1017) -- (2.1500, 0.1000, 0.1058) -- cycle;
\fill[blue!15.0, opacity=0.7] (2.1500, 0.1000, 0.1058) -- (2.2000, 0.1000, 0.1017) -- (2.2000, 0.1500, 0.1079) -- (2.1500, 0.1500, 0.1120) -- cycle;
\fill[blue!15.0, opacity=0.7] (2.1500, 0.1500, 0.1120) -- (2.2000, 0.1500, 0.1079) -- (2.2000, 0.2000, 0.1141) -- (2.1500, 0.2000, 0.1182) -- cycle;
\fill[blue!15.0, opacity=0.7] (2.1500, 0.2000, 0.1182) -- (2.2000, 0.2000, 0.1141) -- (2.2000, 0.2500, 0.1202) -- (2.1500, 0.2500, 0.1243) -- cycle;
\fill[blue!15.0, opacity=0.7] (2.1500, 0.2500, 0.1243) -- (2.2000, 0.2500, 0.1202) -- (2.2000, 0.3000, 0.1263) -- (2.1500, 0.3000, 0.1303) -- cycle;
\fill[blue!15.0, opacity=0.7] (2.1500, 0.3000, 0.1303) -- (2.2000, 0.3000, 0.1263) -- (2.2000, 0.3500, 0.1322) -- (2.1500, 0.3500, 0.1363) -- cycle;
\fill[blue!15.0, opacity=0.7] (2.1500, 0.3500, 0.1363) -- (2.2000, 0.3500, 0.1322) -- (2.2000, 0.4000, 0.1380) -- (2.1500, 0.4000, 0.1421) -- cycle;
\fill[blue!15.0, opacity=0.7] (2.1500, 0.4000, 0.1421) -- (2.2000, 0.4000, 0.1380) -- (2.2000, 0.4500, 0.1437) -- (2.1500, 0.4500, 0.1477) -- cycle;
\fill[blue!15.0, opacity=0.7] (2.1500, 0.4500, 0.1477) -- (2.2000, 0.4500, 0.1437) -- (2.2000, 0.5000, 0.1492) -- (2.1500, 0.5000, 0.1533) -- cycle;
\fill[blue!15.0, opacity=0.7] (2.1500, 0.5000, 0.1533) -- (2.2000, 0.5000, 0.1492) -- (2.2000, 0.5500, 0.1545) -- (2.1500, 0.5500, 0.1586) -- cycle;
\fill[blue!15.0, opacity=0.7] (2.1500, 0.5500, 0.1586) -- (2.2000, 0.5500, 0.1545) -- (2.2000, 0.6000, 0.1597) -- (2.1500, 0.6000, 0.1638) -- cycle;
\fill[blue!15.0, opacity=0.7] (2.1500, 0.6000, 0.1638) -- (2.2000, 0.6000, 0.1597) -- (2.2000, 0.6500, 0.1647) -- (2.1500, 0.6500, 0.1688) -- cycle;
\fill[blue!15.0, opacity=0.7] (2.1500, 0.6500, 0.1688) -- (2.2000, 0.6500, 0.1647) -- (2.2000, 0.7000, 0.1695) -- (2.1500, 0.7000, 0.1736) -- cycle;
\fill[blue!15.0, opacity=0.7] (2.1500, 0.7000, 0.1736) -- (2.2000, 0.7000, 0.1695) -- (2.2000, 0.7500, 0.1740) -- (2.1500, 0.7500, 0.1781) -- cycle;
\fill[blue!15.0, opacity=0.7] (2.1500, 0.7500, 0.1781) -- (2.2000, 0.7500, 0.1740) -- (2.2000, 0.8000, 0.1784) -- (2.1500, 0.8000, 0.1824) -- cycle;
\fill[blue!15.0, opacity=0.7] (2.1500, 0.8000, 0.1824) -- (2.2000, 0.8000, 0.1784) -- (2.2000, 0.8500, 0.1824) -- (2.1500, 0.8500, 0.1865) -- cycle;
\fill[blue!15.0, opacity=0.7] (2.1500, 0.8500, 0.1865) -- (2.2000, 0.8500, 0.1824) -- (2.2000, 0.9000, 0.1863) -- (2.1500, 0.9000, 0.1903) -- cycle;
\fill[blue!15.0, opacity=0.7] (2.1500, 0.9000, 0.1903) -- (2.2000, 0.9000, 0.1863) -- (2.2000, 0.9500, 0.1898) -- (2.1500, 0.9500, 0.1939) -- cycle;
\fill[blue!15.0, opacity=0.7] (2.1500, 0.9500, 0.1939) -- (2.2000, 0.9500, 0.1898) -- (2.2000, 1.0000, 0.1931) -- (2.1500, 1.0000, 0.1972) -- cycle;
\fill[blue!15.0, opacity=0.7] (2.1500, 1.0000, 0.1972) -- (2.2000, 1.0000, 0.1931) -- (2.2000, 1.0500, 0.1961) -- (2.1500, 1.0500, 0.2002) -- cycle;
\fill[blue!15.0, opacity=0.7] (2.1500, 1.0500, 0.2002) -- (2.2000, 1.0500, 0.1961) -- (2.2000, 1.1000, 0.1988) -- (2.1500, 1.1000, 0.2029) -- cycle;
\fill[blue!15.0, opacity=0.7] (2.1500, 1.1000, 0.2029) -- (2.2000, 1.1000, 0.1988) -- (2.2000, 1.1500, 0.2012) -- (2.1500, 1.1500, 0.2053) -- cycle;
\fill[blue!15.0, opacity=0.7] (2.1500, 1.1500, 0.2053) -- (2.2000, 1.1500, 0.2012) -- (2.2000, 1.2000, 0.2033) -- (2.1500, 1.2000, 0.2074) -- cycle;
\fill[blue!15.0, opacity=0.7] (2.1500, 1.2000, 0.2074) -- (2.2000, 1.2000, 0.2033) -- (2.2000, 1.2500, 0.2051) -- (2.1500, 1.2500, 0.2092) -- cycle;
\fill[blue!15.0, opacity=0.7] (2.1500, 1.2500, 0.2092) -- (2.2000, 1.2500, 0.2051) -- (2.2000, 1.3000, 0.2066) -- (2.1500, 1.3000, 0.2106) -- cycle;
\fill[blue!15.0, opacity=0.7] (2.1500, 1.3000, 0.2106) -- (2.2000, 1.3000, 0.2066) -- (2.2000, 1.3500, 0.2077) -- (2.1500, 1.3500, 0.2118) -- cycle;
\fill[blue!15.0, opacity=0.7] (2.1500, 1.3500, 0.2118) -- (2.2000, 1.3500, 0.2077) -- (2.2000, 1.4000, 0.2085) -- (2.1500, 1.4000, 0.2126) -- cycle;
\fill[blue!15.0, opacity=0.7] (2.1500, 1.4000, 0.2126) -- (2.2000, 1.4000, 0.2085) -- (2.2000, 1.4500, 0.2090) -- (2.1500, 1.4500, 0.2131) -- cycle;
\fill[blue!15.0, opacity=0.7] (2.1500, 1.4500, 0.2131) -- (2.2000, 1.4500, 0.2090) -- (2.2000, 1.5000, 0.2092) -- (2.1500, 1.5000, 0.2133) -- cycle;
\fill[blue!15.0, opacity=0.7] (2.1500, 1.5000, 0.2133) -- (2.2000, 1.5000, 0.2092) -- (2.2000, 1.5500, 0.2090) -- (2.1500, 1.5500, 0.2131) -- cycle;
\fill[blue!15.0, opacity=0.7] (2.1500, 1.5500, 0.2131) -- (2.2000, 1.5500, 0.2090) -- (2.2000, 1.6000, 0.2085) -- (2.1500, 1.6000, 0.2126) -- cycle;
\fill[blue!15.0, opacity=0.7] (2.1500, 1.6000, 0.2126) -- (2.2000, 1.6000, 0.2085) -- (2.2000, 1.6500, 0.2077) -- (2.1500, 1.6500, 0.2118) -- cycle;
\fill[blue!15.0, opacity=0.7] (2.1500, 1.6500, 0.2118) -- (2.2000, 1.6500, 0.2077) -- (2.2000, 1.7000, 0.2066) -- (2.1500, 1.7000, 0.2106) -- cycle;
\fill[blue!15.0, opacity=0.7] (2.1500, 1.7000, 0.2106) -- (2.2000, 1.7000, 0.2066) -- (2.2000, 1.7500, 0.2051) -- (2.1500, 1.7500, 0.2092) -- cycle;
\fill[blue!15.0, opacity=0.7] (2.1500, 1.7500, 0.2092) -- (2.2000, 1.7500, 0.2051) -- (2.2000, 1.8000, 0.2033) -- (2.1500, 1.8000, 0.2074) -- cycle;
\fill[blue!15.0, opacity=0.7] (2.1500, 1.8000, 0.2074) -- (2.2000, 1.8000, 0.2033) -- (2.2000, 1.8500, 0.2012) -- (2.1500, 1.8500, 0.2053) -- cycle;
\fill[blue!15.0, opacity=0.7] (2.1500, 1.8500, 0.2053) -- (2.2000, 1.8500, 0.2012) -- (2.2000, 1.9000, 0.1988) -- (2.1500, 1.9000, 0.2029) -- cycle;
\fill[blue!15.1, opacity=0.7] (2.1500, 1.9000, 0.2029) -- (2.2000, 1.9000, 0.1988) -- (2.2000, 1.9500, 0.1961) -- (2.1500, 1.9500, 0.2002) -- cycle;
\fill[blue!17.5, opacity=0.7] (2.1500, 1.9500, 0.2002) -- (2.2000, 1.9500, 0.1961) -- (2.2000, 2.0000, 0.1931) -- (2.1500, 2.0000, 0.1972) -- cycle;
\fill[blue!20.2, opacity=0.7] (2.1500, 2.0000, 0.1972) -- (2.2000, 2.0000, 0.1931) -- (2.2000, 2.0500, 0.1898) -- (2.1500, 2.0500, 0.1939) -- cycle;
\fill[blue!15.5, opacity=0.7] (2.1500, 2.0500, 0.1939) -- (2.2000, 2.0500, 0.1898) -- (2.2000, 2.1000, 0.1863) -- (2.1500, 2.1000, 0.1903) -- cycle;
\fill[blue!15.0, opacity=0.7] (2.1500, 2.1000, 0.1903) -- (2.2000, 2.1000, 0.1863) -- (2.2000, 2.1500, 0.1824) -- (2.1500, 2.1500, 0.1865) -- cycle;
\fill[blue!15.0, opacity=0.7] (2.1500, 2.1500, 0.1865) -- (2.2000, 2.1500, 0.1824) -- (2.2000, 2.2000, 0.1784) -- (2.1500, 2.2000, 0.1824) -- cycle;
\fill[blue!15.0, opacity=0.7] (2.1500, 2.2000, 0.1824) -- (2.2000, 2.2000, 0.1784) -- (2.2000, 2.2500, 0.1740) -- (2.1500, 2.2500, 0.1781) -- cycle;
\fill[blue!15.0, opacity=0.7] (2.1500, 2.2500, 0.1781) -- (2.2000, 2.2500, 0.1740) -- (2.2000, 2.3000, 0.1695) -- (2.1500, 2.3000, 0.1736) -- cycle;
\fill[blue!15.0, opacity=0.7] (2.1500, 2.3000, 0.1736) -- (2.2000, 2.3000, 0.1695) -- (2.2000, 2.3500, 0.1647) -- (2.1500, 2.3500, 0.1688) -- cycle;
\fill[blue!15.0, opacity=0.7] (2.1500, 2.3500, 0.1688) -- (2.2000, 2.3500, 0.1647) -- (2.2000, 2.4000, 0.1597) -- (2.1500, 2.4000, 0.1638) -- cycle;
\fill[blue!15.0, opacity=0.7] (2.1500, 2.4000, 0.1638) -- (2.2000, 2.4000, 0.1597) -- (2.2000, 2.4500, 0.1545) -- (2.1500, 2.4500, 0.1586) -- cycle;
\fill[blue!15.0, opacity=0.7] (2.1500, 2.4500, 0.1586) -- (2.2000, 2.4500, 0.1545) -- (2.2000, 2.5000, 0.1492) -- (2.1500, 2.5000, 0.1533) -- cycle;
\fill[blue!15.0, opacity=0.7] (2.1500, 2.5000, 0.1533) -- (2.2000, 2.5000, 0.1492) -- (2.2000, 2.5500, 0.1437) -- (2.1500, 2.5500, 0.1477) -- cycle;
\fill[blue!15.0, opacity=0.7] (2.1500, 2.5500, 0.1477) -- (2.2000, 2.5500, 0.1437) -- (2.2000, 2.6000, 0.1380) -- (2.1500, 2.6000, 0.1421) -- cycle;
\fill[blue!15.0, opacity=0.7] (2.1500, 2.6000, 0.1421) -- (2.2000, 2.6000, 0.1380) -- (2.2000, 2.6500, 0.1322) -- (2.1500, 2.6500, 0.1363) -- cycle;
\fill[blue!15.0, opacity=0.7] (2.1500, 2.6500, 0.1363) -- (2.2000, 2.6500, 0.1322) -- (2.2000, 2.7000, 0.1263) -- (2.1500, 2.7000, 0.1303) -- cycle;
\fill[blue!15.0, opacity=0.7] (2.1500, 2.7000, 0.1303) -- (2.2000, 2.7000, 0.1263) -- (2.2000, 2.7500, 0.1202) -- (2.1500, 2.7500, 0.1243) -- cycle;
\fill[blue!15.0, opacity=0.7] (2.1500, 2.7500, 0.1243) -- (2.2000, 2.7500, 0.1202) -- (2.2000, 2.8000, 0.1141) -- (2.1500, 2.8000, 0.1182) -- cycle;
\fill[blue!15.0, opacity=0.7] (2.1500, 2.8000, 0.1182) -- (2.2000, 2.8000, 0.1141) -- (2.2000, 2.8500, 0.1079) -- (2.1500, 2.8500, 0.1120) -- cycle;
\fill[blue!15.0, opacity=0.7] (2.1500, 2.8500, 0.1120) -- (2.2000, 2.8500, 0.1079) -- (2.2000, 2.9000, 0.1017) -- (2.1500, 2.9000, 0.1058) -- cycle;
\fill[blue!15.0, opacity=0.7] (2.1500, 2.9000, 0.1058) -- (2.2000, 2.9000, 0.1017) -- (2.2000, 2.9500, 0.0955) -- (2.1500, 2.9500, 0.0995) -- cycle;
\fill[blue!15.0, opacity=0.7] (2.1500, 2.9500, 0.0995) -- (2.2000, 2.9500, 0.0955) -- (2.2000, 3.0000, 0.0892) -- (2.1500, 3.0000, 0.0933) -- cycle;
\fill[blue!15.0, opacity=0.7] (2.2000, 0.0000, 0.0892) -- (2.2500, 0.0000, 0.0849) -- (2.2500, 0.0500, 0.0911) -- (2.2000, 0.0500, 0.0955) -- cycle;
\fill[blue!15.0, opacity=0.7] (2.2000, 0.0500, 0.0955) -- (2.2500, 0.0500, 0.0911) -- (2.2500, 0.1000, 0.0974) -- (2.2000, 0.1000, 0.1017) -- cycle;
\fill[blue!15.0, opacity=0.7] (2.2000, 0.1000, 0.1017) -- (2.2500, 0.1000, 0.0974) -- (2.2500, 0.1500, 0.1036) -- (2.2000, 0.1500, 0.1079) -- cycle;
\fill[blue!15.0, opacity=0.7] (2.2000, 0.1500, 0.1079) -- (2.2500, 0.1500, 0.1036) -- (2.2500, 0.2000, 0.1098) -- (2.2000, 0.2000, 0.1141) -- cycle;
\fill[blue!15.0, opacity=0.7] (2.2000, 0.2000, 0.1141) -- (2.2500, 0.2000, 0.1098) -- (2.2500, 0.2500, 0.1159) -- (2.2000, 0.2500, 0.1202) -- cycle;
\fill[blue!15.0, opacity=0.7] (2.2000, 0.2500, 0.1202) -- (2.2500, 0.2500, 0.1159) -- (2.2500, 0.3000, 0.1219) -- (2.2000, 0.3000, 0.1263) -- cycle;
\fill[blue!15.0, opacity=0.7] (2.2000, 0.3000, 0.1263) -- (2.2500, 0.3000, 0.1219) -- (2.2500, 0.3500, 0.1279) -- (2.2000, 0.3500, 0.1322) -- cycle;
\fill[blue!15.0, opacity=0.7] (2.2000, 0.3500, 0.1322) -- (2.2500, 0.3500, 0.1279) -- (2.2500, 0.4000, 0.1337) -- (2.2000, 0.4000, 0.1380) -- cycle;
\fill[blue!15.0, opacity=0.7] (2.2000, 0.4000, 0.1380) -- (2.2500, 0.4000, 0.1337) -- (2.2500, 0.4500, 0.1393) -- (2.2000, 0.4500, 0.1437) -- cycle;
\fill[blue!15.0, opacity=0.7] (2.2000, 0.4500, 0.1437) -- (2.2500, 0.4500, 0.1393) -- (2.2500, 0.5000, 0.1449) -- (2.2000, 0.5000, 0.1492) -- cycle;
\fill[blue!15.0, opacity=0.7] (2.2000, 0.5000, 0.1492) -- (2.2500, 0.5000, 0.1449) -- (2.2500, 0.5500, 0.1502) -- (2.2000, 0.5500, 0.1545) -- cycle;
\fill[blue!15.0, opacity=0.7] (2.2000, 0.5500, 0.1545) -- (2.2500, 0.5500, 0.1502) -- (2.2500, 0.6000, 0.1554) -- (2.2000, 0.6000, 0.1597) -- cycle;
\fill[blue!15.0, opacity=0.7] (2.2000, 0.6000, 0.1597) -- (2.2500, 0.6000, 0.1554) -- (2.2500, 0.6500, 0.1604) -- (2.2000, 0.6500, 0.1647) -- cycle;
\fill[blue!15.0, opacity=0.7] (2.2000, 0.6500, 0.1647) -- (2.2500, 0.6500, 0.1604) -- (2.2500, 0.7000, 0.1651) -- (2.2000, 0.7000, 0.1695) -- cycle;
\fill[blue!15.0, opacity=0.7] (2.2000, 0.7000, 0.1695) -- (2.2500, 0.7000, 0.1651) -- (2.2500, 0.7500, 0.1697) -- (2.2000, 0.7500, 0.1740) -- cycle;
\fill[blue!15.1, opacity=0.7] (2.2000, 0.7500, 0.1740) -- (2.2500, 0.7500, 0.1697) -- (2.2500, 0.8000, 0.1740) -- (2.2000, 0.8000, 0.1784) -- cycle;
\fill[blue!15.0, opacity=0.7] (2.2000, 0.8000, 0.1784) -- (2.2500, 0.8000, 0.1740) -- (2.2500, 0.8500, 0.1781) -- (2.2000, 0.8500, 0.1824) -- cycle;
\fill[blue!15.0, opacity=0.7] (2.2000, 0.8500, 0.1824) -- (2.2500, 0.8500, 0.1781) -- (2.2500, 0.9000, 0.1819) -- (2.2000, 0.9000, 0.1863) -- cycle;
\fill[blue!15.0, opacity=0.7] (2.2000, 0.9000, 0.1863) -- (2.2500, 0.9000, 0.1819) -- (2.2500, 0.9500, 0.1855) -- (2.2000, 0.9500, 0.1898) -- cycle;
\fill[blue!15.0, opacity=0.7] (2.2000, 0.9500, 0.1898) -- (2.2500, 0.9500, 0.1855) -- (2.2500, 1.0000, 0.1888) -- (2.2000, 1.0000, 0.1931) -- cycle;
\fill[blue!15.0, opacity=0.7] (2.2000, 1.0000, 0.1931) -- (2.2500, 1.0000, 0.1888) -- (2.2500, 1.0500, 0.1918) -- (2.2000, 1.0500, 0.1961) -- cycle;
\fill[blue!15.0, opacity=0.7] (2.2000, 1.0500, 0.1961) -- (2.2500, 1.0500, 0.1918) -- (2.2500, 1.1000, 0.1945) -- (2.2000, 1.1000, 0.1988) -- cycle;
\fill[blue!15.0, opacity=0.7] (2.2000, 1.1000, 0.1988) -- (2.2500, 1.1000, 0.1945) -- (2.2500, 1.1500, 0.1969) -- (2.2000, 1.1500, 0.2012) -- cycle;
\fill[blue!15.0, opacity=0.7] (2.2000, 1.1500, 0.2012) -- (2.2500, 1.1500, 0.1969) -- (2.2500, 1.2000, 0.1990) -- (2.2000, 1.2000, 0.2033) -- cycle;
\fill[blue!15.0, opacity=0.7] (2.2000, 1.2000, 0.2033) -- (2.2500, 1.2000, 0.1990) -- (2.2500, 1.2500, 0.2008) -- (2.2000, 1.2500, 0.2051) -- cycle;
\fill[blue!15.0, opacity=0.7] (2.2000, 1.2500, 0.2051) -- (2.2500, 1.2500, 0.2008) -- (2.2500, 1.3000, 0.2022) -- (2.2000, 1.3000, 0.2066) -- cycle;
\fill[blue!15.0, opacity=0.7] (2.2000, 1.3000, 0.2066) -- (2.2500, 1.3000, 0.2022) -- (2.2500, 1.3500, 0.2034) -- (2.2000, 1.3500, 0.2077) -- cycle;
\fill[blue!15.0, opacity=0.7] (2.2000, 1.3500, 0.2077) -- (2.2500, 1.3500, 0.2034) -- (2.2500, 1.4000, 0.2042) -- (2.2000, 1.4000, 0.2085) -- cycle;
\fill[blue!15.0, opacity=0.7] (2.2000, 1.4000, 0.2085) -- (2.2500, 1.4000, 0.2042) -- (2.2500, 1.4500, 0.2047) -- (2.2000, 1.4500, 0.2090) -- cycle;
\fill[blue!15.0, opacity=0.7] (2.2000, 1.4500, 0.2090) -- (2.2500, 1.4500, 0.2047) -- (2.2500, 1.5000, 0.2049) -- (2.2000, 1.5000, 0.2092) -- cycle;
\fill[blue!15.0, opacity=0.7] (2.2000, 1.5000, 0.2092) -- (2.2500, 1.5000, 0.2049) -- (2.2500, 1.5500, 0.2047) -- (2.2000, 1.5500, 0.2090) -- cycle;
\fill[blue!15.0, opacity=0.7] (2.2000, 1.5500, 0.2090) -- (2.2500, 1.5500, 0.2047) -- (2.2500, 1.6000, 0.2042) -- (2.2000, 1.6000, 0.2085) -- cycle;
\fill[blue!15.0, opacity=0.7] (2.2000, 1.6000, 0.2085) -- (2.2500, 1.6000, 0.2042) -- (2.2500, 1.6500, 0.2034) -- (2.2000, 1.6500, 0.2077) -- cycle;
\fill[blue!15.0, opacity=0.7] (2.2000, 1.6500, 0.2077) -- (2.2500, 1.6500, 0.2034) -- (2.2500, 1.7000, 0.2022) -- (2.2000, 1.7000, 0.2066) -- cycle;
\fill[blue!15.0, opacity=0.7] (2.2000, 1.7000, 0.2066) -- (2.2500, 1.7000, 0.2022) -- (2.2500, 1.7500, 0.2008) -- (2.2000, 1.7500, 0.2051) -- cycle;
\fill[blue!15.0, opacity=0.7] (2.2000, 1.7500, 0.2051) -- (2.2500, 1.7500, 0.2008) -- (2.2500, 1.8000, 0.1990) -- (2.2000, 1.8000, 0.2033) -- cycle;
\fill[blue!15.0, opacity=0.7] (2.2000, 1.8000, 0.2033) -- (2.2500, 1.8000, 0.1990) -- (2.2500, 1.8500, 0.1969) -- (2.2000, 1.8500, 0.2012) -- cycle;
\fill[blue!15.1, opacity=0.7] (2.2000, 1.8500, 0.2012) -- (2.2500, 1.8500, 0.1969) -- (2.2500, 1.9000, 0.1945) -- (2.2000, 1.9000, 0.1988) -- cycle;
\fill[blue!17.5, opacity=0.7] (2.2000, 1.9000, 0.1988) -- (2.2500, 1.9000, 0.1945) -- (2.2500, 1.9500, 0.1918) -- (2.2000, 1.9500, 0.1961) -- cycle;
\fill[blue!19.2, opacity=0.7] (2.2000, 1.9500, 0.1961) -- (2.2500, 1.9500, 0.1918) -- (2.2500, 2.0000, 0.1888) -- (2.2000, 2.0000, 0.1931) -- cycle;
\fill[blue!15.5, opacity=0.7] (2.2000, 2.0000, 0.1931) -- (2.2500, 2.0000, 0.1888) -- (2.2500, 2.0500, 0.1855) -- (2.2000, 2.0500, 0.1898) -- cycle;
\fill[blue!15.0, opacity=0.7] (2.2000, 2.0500, 0.1898) -- (2.2500, 2.0500, 0.1855) -- (2.2500, 2.1000, 0.1819) -- (2.2000, 2.1000, 0.1863) -- cycle;
\fill[blue!15.0, opacity=0.7] (2.2000, 2.1000, 0.1863) -- (2.2500, 2.1000, 0.1819) -- (2.2500, 2.1500, 0.1781) -- (2.2000, 2.1500, 0.1824) -- cycle;
\fill[blue!15.0, opacity=0.7] (2.2000, 2.1500, 0.1824) -- (2.2500, 2.1500, 0.1781) -- (2.2500, 2.2000, 0.1740) -- (2.2000, 2.2000, 0.1784) -- cycle;
\fill[blue!15.0, opacity=0.7] (2.2000, 2.2000, 0.1784) -- (2.2500, 2.2000, 0.1740) -- (2.2500, 2.2500, 0.1697) -- (2.2000, 2.2500, 0.1740) -- cycle;
\fill[blue!15.0, opacity=0.7] (2.2000, 2.2500, 0.1740) -- (2.2500, 2.2500, 0.1697) -- (2.2500, 2.3000, 0.1651) -- (2.2000, 2.3000, 0.1695) -- cycle;
\fill[blue!15.0, opacity=0.7] (2.2000, 2.3000, 0.1695) -- (2.2500, 2.3000, 0.1651) -- (2.2500, 2.3500, 0.1604) -- (2.2000, 2.3500, 0.1647) -- cycle;
\fill[blue!15.0, opacity=0.7] (2.2000, 2.3500, 0.1647) -- (2.2500, 2.3500, 0.1604) -- (2.2500, 2.4000, 0.1554) -- (2.2000, 2.4000, 0.1597) -- cycle;
\fill[blue!15.0, opacity=0.7] (2.2000, 2.4000, 0.1597) -- (2.2500, 2.4000, 0.1554) -- (2.2500, 2.4500, 0.1502) -- (2.2000, 2.4500, 0.1545) -- cycle;
\fill[blue!15.0, opacity=0.7] (2.2000, 2.4500, 0.1545) -- (2.2500, 2.4500, 0.1502) -- (2.2500, 2.5000, 0.1449) -- (2.2000, 2.5000, 0.1492) -- cycle;
\fill[blue!15.0, opacity=0.7] (2.2000, 2.5000, 0.1492) -- (2.2500, 2.5000, 0.1449) -- (2.2500, 2.5500, 0.1393) -- (2.2000, 2.5500, 0.1437) -- cycle;
\fill[blue!15.0, opacity=0.7] (2.2000, 2.5500, 0.1437) -- (2.2500, 2.5500, 0.1393) -- (2.2500, 2.6000, 0.1337) -- (2.2000, 2.6000, 0.1380) -- cycle;
\fill[blue!15.0, opacity=0.7] (2.2000, 2.6000, 0.1380) -- (2.2500, 2.6000, 0.1337) -- (2.2500, 2.6500, 0.1279) -- (2.2000, 2.6500, 0.1322) -- cycle;
\fill[blue!15.0, opacity=0.7] (2.2000, 2.6500, 0.1322) -- (2.2500, 2.6500, 0.1279) -- (2.2500, 2.7000, 0.1219) -- (2.2000, 2.7000, 0.1263) -- cycle;
\fill[blue!15.0, opacity=0.7] (2.2000, 2.7000, 0.1263) -- (2.2500, 2.7000, 0.1219) -- (2.2500, 2.7500, 0.1159) -- (2.2000, 2.7500, 0.1202) -- cycle;
\fill[blue!15.0, opacity=0.7] (2.2000, 2.7500, 0.1202) -- (2.2500, 2.7500, 0.1159) -- (2.2500, 2.8000, 0.1098) -- (2.2000, 2.8000, 0.1141) -- cycle;
\fill[blue!15.0, opacity=0.7] (2.2000, 2.8000, 0.1141) -- (2.2500, 2.8000, 0.1098) -- (2.2500, 2.8500, 0.1036) -- (2.2000, 2.8500, 0.1079) -- cycle;
\fill[blue!15.0, opacity=0.7] (2.2000, 2.8500, 0.1079) -- (2.2500, 2.8500, 0.1036) -- (2.2500, 2.9000, 0.0974) -- (2.2000, 2.9000, 0.1017) -- cycle;
\fill[blue!15.0, opacity=0.7] (2.2000, 2.9000, 0.1017) -- (2.2500, 2.9000, 0.0974) -- (2.2500, 2.9500, 0.0911) -- (2.2000, 2.9500, 0.0955) -- cycle;
\fill[blue!15.0, opacity=0.7] (2.2000, 2.9500, 0.0955) -- (2.2500, 2.9500, 0.0911) -- (2.2500, 3.0000, 0.0849) -- (2.2000, 3.0000, 0.0892) -- cycle;
\fill[blue!15.0, opacity=0.7] (2.2500, 0.0000, 0.0849) -- (2.3000, 0.0000, 0.0803) -- (2.3000, 0.0500, 0.0866) -- (2.2500, 0.0500, 0.0911) -- cycle;
\fill[blue!15.0, opacity=0.7] (2.2500, 0.0500, 0.0911) -- (2.3000, 0.0500, 0.0866) -- (2.3000, 0.1000, 0.0928) -- (2.2500, 0.1000, 0.0974) -- cycle;
\fill[blue!15.0, opacity=0.7] (2.2500, 0.1000, 0.0974) -- (2.3000, 0.1000, 0.0928) -- (2.3000, 0.1500, 0.0991) -- (2.2500, 0.1500, 0.1036) -- cycle;
\fill[blue!15.0, opacity=0.7] (2.2500, 0.1500, 0.1036) -- (2.3000, 0.1500, 0.0991) -- (2.3000, 0.2000, 0.1052) -- (2.2500, 0.2000, 0.1098) -- cycle;
\fill[blue!15.0, opacity=0.7] (2.2500, 0.2000, 0.1098) -- (2.3000, 0.2000, 0.1052) -- (2.3000, 0.2500, 0.1114) -- (2.2500, 0.2500, 0.1159) -- cycle;
\fill[blue!15.0, opacity=0.7] (2.2500, 0.2500, 0.1159) -- (2.3000, 0.2500, 0.1114) -- (2.3000, 0.3000, 0.1174) -- (2.2500, 0.3000, 0.1219) -- cycle;
\fill[blue!15.0, opacity=0.7] (2.2500, 0.3000, 0.1219) -- (2.3000, 0.3000, 0.1174) -- (2.3000, 0.3500, 0.1233) -- (2.2500, 0.3500, 0.1279) -- cycle;
\fill[blue!15.0, opacity=0.7] (2.2500, 0.3500, 0.1279) -- (2.3000, 0.3500, 0.1233) -- (2.3000, 0.4000, 0.1291) -- (2.2500, 0.4000, 0.1337) -- cycle;
\fill[blue!15.0, opacity=0.7] (2.2500, 0.4000, 0.1337) -- (2.3000, 0.4000, 0.1291) -- (2.3000, 0.4500, 0.1348) -- (2.2500, 0.4500, 0.1393) -- cycle;
\fill[blue!15.0, opacity=0.7] (2.2500, 0.4500, 0.1393) -- (2.3000, 0.4500, 0.1348) -- (2.3000, 0.5000, 0.1403) -- (2.2500, 0.5000, 0.1449) -- cycle;
\fill[blue!15.0, opacity=0.7] (2.2500, 0.5000, 0.1449) -- (2.3000, 0.5000, 0.1403) -- (2.3000, 0.5500, 0.1457) -- (2.2500, 0.5500, 0.1502) -- cycle;
\fill[blue!15.0, opacity=0.7] (2.2500, 0.5500, 0.1502) -- (2.3000, 0.5500, 0.1457) -- (2.3000, 0.6000, 0.1508) -- (2.2500, 0.6000, 0.1554) -- cycle;
\fill[blue!15.0, opacity=0.7] (2.2500, 0.6000, 0.1554) -- (2.3000, 0.6000, 0.1508) -- (2.3000, 0.6500, 0.1558) -- (2.2500, 0.6500, 0.1604) -- cycle;
\fill[blue!15.0, opacity=0.7] (2.2500, 0.6500, 0.1604) -- (2.3000, 0.6500, 0.1558) -- (2.3000, 0.7000, 0.1606) -- (2.2500, 0.7000, 0.1651) -- cycle;
\fill[blue!15.0, opacity=0.7] (2.2500, 0.7000, 0.1651) -- (2.3000, 0.7000, 0.1606) -- (2.3000, 0.7500, 0.1651) -- (2.2500, 0.7500, 0.1697) -- cycle;
\fill[blue!15.0, opacity=0.7] (2.2500, 0.7500, 0.1697) -- (2.3000, 0.7500, 0.1651) -- (2.3000, 0.8000, 0.1695) -- (2.2500, 0.8000, 0.1740) -- cycle;
\fill[blue!15.1, opacity=0.7] (2.2500, 0.8000, 0.1740) -- (2.3000, 0.8000, 0.1695) -- (2.3000, 0.8500, 0.1736) -- (2.2500, 0.8500, 0.1781) -- cycle;
\fill[blue!15.0, opacity=0.7] (2.2500, 0.8500, 0.1781) -- (2.3000, 0.8500, 0.1736) -- (2.3000, 0.9000, 0.1774) -- (2.2500, 0.9000, 0.1819) -- cycle;
\fill[blue!15.0, opacity=0.7] (2.2500, 0.9000, 0.1819) -- (2.3000, 0.9000, 0.1774) -- (2.3000, 0.9500, 0.1809) -- (2.2500, 0.9500, 0.1855) -- cycle;
\fill[blue!15.0, opacity=0.7] (2.2500, 0.9500, 0.1855) -- (2.3000, 0.9500, 0.1809) -- (2.3000, 1.0000, 0.1842) -- (2.2500, 1.0000, 0.1888) -- cycle;
\fill[blue!15.0, opacity=0.7] (2.2500, 1.0000, 0.1888) -- (2.3000, 1.0000, 0.1842) -- (2.3000, 1.0500, 0.1872) -- (2.2500, 1.0500, 0.1918) -- cycle;
\fill[blue!15.0, opacity=0.7] (2.2500, 1.0500, 0.1918) -- (2.3000, 1.0500, 0.1872) -- (2.3000, 1.1000, 0.1899) -- (2.2500, 1.1000, 0.1945) -- cycle;
\fill[blue!15.0, opacity=0.7] (2.2500, 1.1000, 0.1945) -- (2.3000, 1.1000, 0.1899) -- (2.3000, 1.1500, 0.1923) -- (2.2500, 1.1500, 0.1969) -- cycle;
\fill[blue!15.0, opacity=0.7] (2.2500, 1.1500, 0.1969) -- (2.3000, 1.1500, 0.1923) -- (2.3000, 1.2000, 0.1944) -- (2.2500, 1.2000, 0.1990) -- cycle;
\fill[blue!15.0, opacity=0.7] (2.2500, 1.2000, 0.1990) -- (2.3000, 1.2000, 0.1944) -- (2.3000, 1.2500, 0.1962) -- (2.2500, 1.2500, 0.2008) -- cycle;
\fill[blue!15.0, opacity=0.7] (2.2500, 1.2500, 0.2008) -- (2.3000, 1.2500, 0.1962) -- (2.3000, 1.3000, 0.1977) -- (2.2500, 1.3000, 0.2022) -- cycle;
\fill[blue!15.0, opacity=0.7] (2.2500, 1.3000, 0.2022) -- (2.3000, 1.3000, 0.1977) -- (2.3000, 1.3500, 0.1988) -- (2.2500, 1.3500, 0.2034) -- cycle;
\fill[blue!15.0, opacity=0.7] (2.2500, 1.3500, 0.2034) -- (2.3000, 1.3500, 0.1988) -- (2.3000, 1.4000, 0.1996) -- (2.2500, 1.4000, 0.2042) -- cycle;
\fill[blue!15.0, opacity=0.7] (2.2500, 1.4000, 0.2042) -- (2.3000, 1.4000, 0.1996) -- (2.3000, 1.4500, 0.2001) -- (2.2500, 1.4500, 0.2047) -- cycle;
\fill[blue!15.0, opacity=0.7] (2.2500, 1.4500, 0.2047) -- (2.3000, 1.4500, 0.2001) -- (2.3000, 1.5000, 0.2003) -- (2.2500, 1.5000, 0.2049) -- cycle;
\fill[blue!15.0, opacity=0.7] (2.2500, 1.5000, 0.2049) -- (2.3000, 1.5000, 0.2003) -- (2.3000, 1.5500, 0.2001) -- (2.2500, 1.5500, 0.2047) -- cycle;
\fill[blue!15.0, opacity=0.7] (2.2500, 1.5500, 0.2047) -- (2.3000, 1.5500, 0.2001) -- (2.3000, 1.6000, 0.1996) -- (2.2500, 1.6000, 0.2042) -- cycle;
\fill[blue!15.0, opacity=0.7] (2.2500, 1.6000, 0.2042) -- (2.3000, 1.6000, 0.1996) -- (2.3000, 1.6500, 0.1988) -- (2.2500, 1.6500, 0.2034) -- cycle;
\fill[blue!15.0, opacity=0.7] (2.2500, 1.6500, 0.2034) -- (2.3000, 1.6500, 0.1988) -- (2.3000, 1.7000, 0.1977) -- (2.2500, 1.7000, 0.2022) -- cycle;
\fill[blue!15.0, opacity=0.7] (2.2500, 1.7000, 0.2022) -- (2.3000, 1.7000, 0.1977) -- (2.3000, 1.7500, 0.1962) -- (2.2500, 1.7500, 0.2008) -- cycle;
\fill[blue!15.0, opacity=0.7] (2.2500, 1.7500, 0.2008) -- (2.3000, 1.7500, 0.1962) -- (2.3000, 1.8000, 0.1944) -- (2.2500, 1.8000, 0.1990) -- cycle;
\fill[blue!15.5, opacity=0.7] (2.2500, 1.8000, 0.1990) -- (2.3000, 1.8000, 0.1944) -- (2.3000, 1.8500, 0.1923) -- (2.2500, 1.8500, 0.1969) -- cycle;
\fill[blue!17.9, opacity=0.7] (2.2500, 1.8500, 0.1969) -- (2.3000, 1.8500, 0.1923) -- (2.3000, 1.9000, 0.1899) -- (2.2500, 1.9000, 0.1945) -- cycle;
\fill[blue!17.8, opacity=0.7] (2.2500, 1.9000, 0.1945) -- (2.3000, 1.9000, 0.1899) -- (2.3000, 1.9500, 0.1872) -- (2.2500, 1.9500, 0.1918) -- cycle;
\fill[blue!15.3, opacity=0.7] (2.2500, 1.9500, 0.1918) -- (2.3000, 1.9500, 0.1872) -- (2.3000, 2.0000, 0.1842) -- (2.2500, 2.0000, 0.1888) -- cycle;
\fill[blue!15.0, opacity=0.7] (2.2500, 2.0000, 0.1888) -- (2.3000, 2.0000, 0.1842) -- (2.3000, 2.0500, 0.1809) -- (2.2500, 2.0500, 0.1855) -- cycle;
\fill[blue!15.0, opacity=0.7] (2.2500, 2.0500, 0.1855) -- (2.3000, 2.0500, 0.1809) -- (2.3000, 2.1000, 0.1774) -- (2.2500, 2.1000, 0.1819) -- cycle;
\fill[blue!15.0, opacity=0.7] (2.2500, 2.1000, 0.1819) -- (2.3000, 2.1000, 0.1774) -- (2.3000, 2.1500, 0.1736) -- (2.2500, 2.1500, 0.1781) -- cycle;
\fill[blue!15.0, opacity=0.7] (2.2500, 2.1500, 0.1781) -- (2.3000, 2.1500, 0.1736) -- (2.3000, 2.2000, 0.1695) -- (2.2500, 2.2000, 0.1740) -- cycle;
\fill[blue!15.0, opacity=0.7] (2.2500, 2.2000, 0.1740) -- (2.3000, 2.2000, 0.1695) -- (2.3000, 2.2500, 0.1651) -- (2.2500, 2.2500, 0.1697) -- cycle;
\fill[blue!15.0, opacity=0.7] (2.2500, 2.2500, 0.1697) -- (2.3000, 2.2500, 0.1651) -- (2.3000, 2.3000, 0.1606) -- (2.2500, 2.3000, 0.1651) -- cycle;
\fill[blue!15.0, opacity=0.7] (2.2500, 2.3000, 0.1651) -- (2.3000, 2.3000, 0.1606) -- (2.3000, 2.3500, 0.1558) -- (2.2500, 2.3500, 0.1604) -- cycle;
\fill[blue!15.0, opacity=0.7] (2.2500, 2.3500, 0.1604) -- (2.3000, 2.3500, 0.1558) -- (2.3000, 2.4000, 0.1508) -- (2.2500, 2.4000, 0.1554) -- cycle;
\fill[blue!15.0, opacity=0.7] (2.2500, 2.4000, 0.1554) -- (2.3000, 2.4000, 0.1508) -- (2.3000, 2.4500, 0.1457) -- (2.2500, 2.4500, 0.1502) -- cycle;
\fill[blue!15.0, opacity=0.7] (2.2500, 2.4500, 0.1502) -- (2.3000, 2.4500, 0.1457) -- (2.3000, 2.5000, 0.1403) -- (2.2500, 2.5000, 0.1449) -- cycle;
\fill[blue!15.0, opacity=0.7] (2.2500, 2.5000, 0.1449) -- (2.3000, 2.5000, 0.1403) -- (2.3000, 2.5500, 0.1348) -- (2.2500, 2.5500, 0.1393) -- cycle;
\fill[blue!15.0, opacity=0.7] (2.2500, 2.5500, 0.1393) -- (2.3000, 2.5500, 0.1348) -- (2.3000, 2.6000, 0.1291) -- (2.2500, 2.6000, 0.1337) -- cycle;
\fill[blue!15.0, opacity=0.7] (2.2500, 2.6000, 0.1337) -- (2.3000, 2.6000, 0.1291) -- (2.3000, 2.6500, 0.1233) -- (2.2500, 2.6500, 0.1279) -- cycle;
\fill[blue!15.0, opacity=0.7] (2.2500, 2.6500, 0.1279) -- (2.3000, 2.6500, 0.1233) -- (2.3000, 2.7000, 0.1174) -- (2.2500, 2.7000, 0.1219) -- cycle;
\fill[blue!15.0, opacity=0.7] (2.2500, 2.7000, 0.1219) -- (2.3000, 2.7000, 0.1174) -- (2.3000, 2.7500, 0.1114) -- (2.2500, 2.7500, 0.1159) -- cycle;
\fill[blue!15.0, opacity=0.7] (2.2500, 2.7500, 0.1159) -- (2.3000, 2.7500, 0.1114) -- (2.3000, 2.8000, 0.1052) -- (2.2500, 2.8000, 0.1098) -- cycle;
\fill[blue!15.0, opacity=0.7] (2.2500, 2.8000, 0.1098) -- (2.3000, 2.8000, 0.1052) -- (2.3000, 2.8500, 0.0991) -- (2.2500, 2.8500, 0.1036) -- cycle;
\fill[blue!15.0, opacity=0.7] (2.2500, 2.8500, 0.1036) -- (2.3000, 2.8500, 0.0991) -- (2.3000, 2.9000, 0.0928) -- (2.2500, 2.9000, 0.0974) -- cycle;
\fill[blue!15.0, opacity=0.7] (2.2500, 2.9000, 0.0974) -- (2.3000, 2.9000, 0.0928) -- (2.3000, 2.9500, 0.0866) -- (2.2500, 2.9500, 0.0911) -- cycle;
\fill[blue!15.0, opacity=0.7] (2.2500, 2.9500, 0.0911) -- (2.3000, 2.9500, 0.0866) -- (2.3000, 3.0000, 0.0803) -- (2.2500, 3.0000, 0.0849) -- cycle;
\fill[blue!15.0, opacity=0.7] (2.3000, 0.0000, 0.0803) -- (2.3500, 0.0000, 0.0755) -- (2.3500, 0.0500, 0.0818) -- (2.3000, 0.0500, 0.0866) -- cycle;
\fill[blue!15.0, opacity=0.7] (2.3000, 0.0500, 0.0866) -- (2.3500, 0.0500, 0.0818) -- (2.3500, 0.1000, 0.0881) -- (2.3000, 0.1000, 0.0928) -- cycle;
\fill[blue!15.0, opacity=0.7] (2.3000, 0.1000, 0.0928) -- (2.3500, 0.1000, 0.0881) -- (2.3500, 0.1500, 0.0943) -- (2.3000, 0.1500, 0.0991) -- cycle;
\fill[blue!15.0, opacity=0.7] (2.3000, 0.1500, 0.0991) -- (2.3500, 0.1500, 0.0943) -- (2.3500, 0.2000, 0.1005) -- (2.3000, 0.2000, 0.1052) -- cycle;
\fill[blue!15.0, opacity=0.7] (2.3000, 0.2000, 0.1052) -- (2.3500, 0.2000, 0.1005) -- (2.3500, 0.2500, 0.1066) -- (2.3000, 0.2500, 0.1114) -- cycle;
\fill[blue!15.0, opacity=0.7] (2.3000, 0.2500, 0.1114) -- (2.3500, 0.2500, 0.1066) -- (2.3500, 0.3000, 0.1126) -- (2.3000, 0.3000, 0.1174) -- cycle;
\fill[blue!15.0, opacity=0.7] (2.3000, 0.3000, 0.1174) -- (2.3500, 0.3000, 0.1126) -- (2.3500, 0.3500, 0.1185) -- (2.3000, 0.3500, 0.1233) -- cycle;
\fill[blue!15.0, opacity=0.7] (2.3000, 0.3500, 0.1233) -- (2.3500, 0.3500, 0.1185) -- (2.3500, 0.4000, 0.1243) -- (2.3000, 0.4000, 0.1291) -- cycle;
\fill[blue!15.0, opacity=0.7] (2.3000, 0.4000, 0.1291) -- (2.3500, 0.4000, 0.1243) -- (2.3500, 0.4500, 0.1300) -- (2.3000, 0.4500, 0.1348) -- cycle;
\fill[blue!15.0, opacity=0.7] (2.3000, 0.4500, 0.1348) -- (2.3500, 0.4500, 0.1300) -- (2.3500, 0.5000, 0.1355) -- (2.3000, 0.5000, 0.1403) -- cycle;
\fill[blue!15.0, opacity=0.7] (2.3000, 0.5000, 0.1403) -- (2.3500, 0.5000, 0.1355) -- (2.3500, 0.5500, 0.1409) -- (2.3000, 0.5500, 0.1457) -- cycle;
\fill[blue!15.0, opacity=0.7] (2.3000, 0.5500, 0.1457) -- (2.3500, 0.5500, 0.1409) -- (2.3500, 0.6000, 0.1461) -- (2.3000, 0.6000, 0.1508) -- cycle;
\fill[blue!15.0, opacity=0.7] (2.3000, 0.6000, 0.1508) -- (2.3500, 0.6000, 0.1461) -- (2.3500, 0.6500, 0.1510) -- (2.3000, 0.6500, 0.1558) -- cycle;
\fill[blue!15.0, opacity=0.7] (2.3000, 0.6500, 0.1558) -- (2.3500, 0.6500, 0.1510) -- (2.3500, 0.7000, 0.1558) -- (2.3000, 0.7000, 0.1606) -- cycle;
\fill[blue!15.0, opacity=0.7] (2.3000, 0.7000, 0.1606) -- (2.3500, 0.7000, 0.1558) -- (2.3500, 0.7500, 0.1604) -- (2.3000, 0.7500, 0.1651) -- cycle;
\fill[blue!15.0, opacity=0.7] (2.3000, 0.7500, 0.1651) -- (2.3500, 0.7500, 0.1604) -- (2.3500, 0.8000, 0.1647) -- (2.3000, 0.8000, 0.1695) -- cycle;
\fill[blue!15.0, opacity=0.7] (2.3000, 0.8000, 0.1695) -- (2.3500, 0.8000, 0.1647) -- (2.3500, 0.8500, 0.1688) -- (2.3000, 0.8500, 0.1736) -- cycle;
\fill[blue!15.1, opacity=0.7] (2.3000, 0.8500, 0.1736) -- (2.3500, 0.8500, 0.1688) -- (2.3500, 0.9000, 0.1726) -- (2.3000, 0.9000, 0.1774) -- cycle;
\fill[blue!15.1, opacity=0.7] (2.3000, 0.9000, 0.1774) -- (2.3500, 0.9000, 0.1726) -- (2.3500, 0.9500, 0.1762) -- (2.3000, 0.9500, 0.1809) -- cycle;
\fill[blue!15.0, opacity=0.7] (2.3000, 0.9500, 0.1809) -- (2.3500, 0.9500, 0.1762) -- (2.3500, 1.0000, 0.1794) -- (2.3000, 1.0000, 0.1842) -- cycle;
\fill[blue!15.0, opacity=0.7] (2.3000, 1.0000, 0.1842) -- (2.3500, 1.0000, 0.1794) -- (2.3500, 1.0500, 0.1824) -- (2.3000, 1.0500, 0.1872) -- cycle;
\fill[blue!15.0, opacity=0.7] (2.3000, 1.0500, 0.1872) -- (2.3500, 1.0500, 0.1824) -- (2.3500, 1.1000, 0.1851) -- (2.3000, 1.1000, 0.1899) -- cycle;
\fill[blue!15.0, opacity=0.7] (2.3000, 1.1000, 0.1899) -- (2.3500, 1.1000, 0.1851) -- (2.3500, 1.1500, 0.1875) -- (2.3000, 1.1500, 0.1923) -- cycle;
\fill[blue!15.0, opacity=0.7] (2.3000, 1.1500, 0.1923) -- (2.3500, 1.1500, 0.1875) -- (2.3500, 1.2000, 0.1896) -- (2.3000, 1.2000, 0.1944) -- cycle;
\fill[blue!15.0, opacity=0.7] (2.3000, 1.2000, 0.1944) -- (2.3500, 1.2000, 0.1896) -- (2.3500, 1.2500, 0.1914) -- (2.3000, 1.2500, 0.1962) -- cycle;
\fill[blue!15.0, opacity=0.7] (2.3000, 1.2500, 0.1962) -- (2.3500, 1.2500, 0.1914) -- (2.3500, 1.3000, 0.1929) -- (2.3000, 1.3000, 0.1977) -- cycle;
\fill[blue!15.0, opacity=0.7] (2.3000, 1.3000, 0.1977) -- (2.3500, 1.3000, 0.1929) -- (2.3500, 1.3500, 0.1940) -- (2.3000, 1.3500, 0.1988) -- cycle;
\fill[blue!15.0, opacity=0.7] (2.3000, 1.3500, 0.1988) -- (2.3500, 1.3500, 0.1940) -- (2.3500, 1.4000, 0.1949) -- (2.3000, 1.4000, 0.1996) -- cycle;
\fill[blue!15.0, opacity=0.7] (2.3000, 1.4000, 0.1996) -- (2.3500, 1.4000, 0.1949) -- (2.3500, 1.4500, 0.1954) -- (2.3000, 1.4500, 0.2001) -- cycle;
\fill[blue!15.0, opacity=0.7] (2.3000, 1.4500, 0.2001) -- (2.3500, 1.4500, 0.1954) -- (2.3500, 1.5000, 0.1955) -- (2.3000, 1.5000, 0.2003) -- cycle;
\fill[blue!15.0, opacity=0.7] (2.3000, 1.5000, 0.2003) -- (2.3500, 1.5000, 0.1955) -- (2.3500, 1.5500, 0.1954) -- (2.3000, 1.5500, 0.2001) -- cycle;
\fill[blue!15.0, opacity=0.7] (2.3000, 1.5500, 0.2001) -- (2.3500, 1.5500, 0.1954) -- (2.3500, 1.6000, 0.1949) -- (2.3000, 1.6000, 0.1996) -- cycle;
\fill[blue!15.0, opacity=0.7] (2.3000, 1.6000, 0.1996) -- (2.3500, 1.6000, 0.1949) -- (2.3500, 1.6500, 0.1940) -- (2.3000, 1.6500, 0.1988) -- cycle;
\fill[blue!15.0, opacity=0.7] (2.3000, 1.6500, 0.1988) -- (2.3500, 1.6500, 0.1940) -- (2.3500, 1.7000, 0.1929) -- (2.3000, 1.7000, 0.1977) -- cycle;
\fill[blue!15.2, opacity=0.7] (2.3000, 1.7000, 0.1977) -- (2.3500, 1.7000, 0.1929) -- (2.3500, 1.7500, 0.1914) -- (2.3000, 1.7500, 0.1962) -- cycle;
\fill[blue!16.5, opacity=0.7] (2.3000, 1.7500, 0.1962) -- (2.3500, 1.7500, 0.1914) -- (2.3500, 1.8000, 0.1896) -- (2.3000, 1.8000, 0.1944) -- cycle;
\fill[blue!17.8, opacity=0.7] (2.3000, 1.8000, 0.1944) -- (2.3500, 1.8000, 0.1896) -- (2.3500, 1.8500, 0.1875) -- (2.3000, 1.8500, 0.1923) -- cycle;
\fill[blue!16.2, opacity=0.7] (2.3000, 1.8500, 0.1923) -- (2.3500, 1.8500, 0.1875) -- (2.3500, 1.9000, 0.1851) -- (2.3000, 1.9000, 0.1899) -- cycle;
\fill[blue!15.1, opacity=0.7] (2.3000, 1.9000, 0.1899) -- (2.3500, 1.9000, 0.1851) -- (2.3500, 1.9500, 0.1824) -- (2.3000, 1.9500, 0.1872) -- cycle;
\fill[blue!15.0, opacity=0.7] (2.3000, 1.9500, 0.1872) -- (2.3500, 1.9500, 0.1824) -- (2.3500, 2.0000, 0.1794) -- (2.3000, 2.0000, 0.1842) -- cycle;
\fill[blue!15.0, opacity=0.7] (2.3000, 2.0000, 0.1842) -- (2.3500, 2.0000, 0.1794) -- (2.3500, 2.0500, 0.1762) -- (2.3000, 2.0500, 0.1809) -- cycle;
\fill[blue!15.0, opacity=0.7] (2.3000, 2.0500, 0.1809) -- (2.3500, 2.0500, 0.1762) -- (2.3500, 2.1000, 0.1726) -- (2.3000, 2.1000, 0.1774) -- cycle;
\fill[blue!15.0, opacity=0.7] (2.3000, 2.1000, 0.1774) -- (2.3500, 2.1000, 0.1726) -- (2.3500, 2.1500, 0.1688) -- (2.3000, 2.1500, 0.1736) -- cycle;
\fill[blue!15.0, opacity=0.7] (2.3000, 2.1500, 0.1736) -- (2.3500, 2.1500, 0.1688) -- (2.3500, 2.2000, 0.1647) -- (2.3000, 2.2000, 0.1695) -- cycle;
\fill[blue!15.0, opacity=0.7] (2.3000, 2.2000, 0.1695) -- (2.3500, 2.2000, 0.1647) -- (2.3500, 2.2500, 0.1604) -- (2.3000, 2.2500, 0.1651) -- cycle;
\fill[blue!15.0, opacity=0.7] (2.3000, 2.2500, 0.1651) -- (2.3500, 2.2500, 0.1604) -- (2.3500, 2.3000, 0.1558) -- (2.3000, 2.3000, 0.1606) -- cycle;
\fill[blue!15.0, opacity=0.7] (2.3000, 2.3000, 0.1606) -- (2.3500, 2.3000, 0.1558) -- (2.3500, 2.3500, 0.1510) -- (2.3000, 2.3500, 0.1558) -- cycle;
\fill[blue!15.0, opacity=0.7] (2.3000, 2.3500, 0.1558) -- (2.3500, 2.3500, 0.1510) -- (2.3500, 2.4000, 0.1461) -- (2.3000, 2.4000, 0.1508) -- cycle;
\fill[blue!15.0, opacity=0.7] (2.3000, 2.4000, 0.1508) -- (2.3500, 2.4000, 0.1461) -- (2.3500, 2.4500, 0.1409) -- (2.3000, 2.4500, 0.1457) -- cycle;
\fill[blue!15.0, opacity=0.7] (2.3000, 2.4500, 0.1457) -- (2.3500, 2.4500, 0.1409) -- (2.3500, 2.5000, 0.1355) -- (2.3000, 2.5000, 0.1403) -- cycle;
\fill[blue!15.0, opacity=0.7] (2.3000, 2.5000, 0.1403) -- (2.3500, 2.5000, 0.1355) -- (2.3500, 2.5500, 0.1300) -- (2.3000, 2.5500, 0.1348) -- cycle;
\fill[blue!15.0, opacity=0.7] (2.3000, 2.5500, 0.1348) -- (2.3500, 2.5500, 0.1300) -- (2.3500, 2.6000, 0.1243) -- (2.3000, 2.6000, 0.1291) -- cycle;
\fill[blue!15.0, opacity=0.7] (2.3000, 2.6000, 0.1291) -- (2.3500, 2.6000, 0.1243) -- (2.3500, 2.6500, 0.1185) -- (2.3000, 2.6500, 0.1233) -- cycle;
\fill[blue!15.0, opacity=0.7] (2.3000, 2.6500, 0.1233) -- (2.3500, 2.6500, 0.1185) -- (2.3500, 2.7000, 0.1126) -- (2.3000, 2.7000, 0.1174) -- cycle;
\fill[blue!15.0, opacity=0.7] (2.3000, 2.7000, 0.1174) -- (2.3500, 2.7000, 0.1126) -- (2.3500, 2.7500, 0.1066) -- (2.3000, 2.7500, 0.1114) -- cycle;
\fill[blue!15.0, opacity=0.7] (2.3000, 2.7500, 0.1114) -- (2.3500, 2.7500, 0.1066) -- (2.3500, 2.8000, 0.1005) -- (2.3000, 2.8000, 0.1052) -- cycle;
\fill[blue!15.0, opacity=0.7] (2.3000, 2.8000, 0.1052) -- (2.3500, 2.8000, 0.1005) -- (2.3500, 2.8500, 0.0943) -- (2.3000, 2.8500, 0.0991) -- cycle;
\fill[blue!15.0, opacity=0.7] (2.3000, 2.8500, 0.0991) -- (2.3500, 2.8500, 0.0943) -- (2.3500, 2.9000, 0.0881) -- (2.3000, 2.9000, 0.0928) -- cycle;
\fill[blue!15.0, opacity=0.7] (2.3000, 2.9000, 0.0928) -- (2.3500, 2.9000, 0.0881) -- (2.3500, 2.9500, 0.0818) -- (2.3000, 2.9500, 0.0866) -- cycle;
\fill[blue!15.0, opacity=0.7] (2.3000, 2.9500, 0.0866) -- (2.3500, 2.9500, 0.0818) -- (2.3500, 3.0000, 0.0755) -- (2.3000, 3.0000, 0.0803) -- cycle;
\fill[blue!15.0, opacity=0.7] (2.3500, 0.0000, 0.0755) -- (2.4000, 0.0000, 0.0705) -- (2.4000, 0.0500, 0.0768) -- (2.3500, 0.0500, 0.0818) -- cycle;
\fill[blue!15.0, opacity=0.7] (2.3500, 0.0500, 0.0818) -- (2.4000, 0.0500, 0.0768) -- (2.4000, 0.1000, 0.0831) -- (2.3500, 0.1000, 0.0881) -- cycle;
\fill[blue!15.0, opacity=0.7] (2.3500, 0.1000, 0.0881) -- (2.4000, 0.1000, 0.0831) -- (2.4000, 0.1500, 0.0893) -- (2.3500, 0.1500, 0.0943) -- cycle;
\fill[blue!15.0, opacity=0.7] (2.3500, 0.1500, 0.0943) -- (2.4000, 0.1500, 0.0893) -- (2.4000, 0.2000, 0.0955) -- (2.3500, 0.2000, 0.1005) -- cycle;
\fill[blue!15.0, opacity=0.7] (2.3500, 0.2000, 0.1005) -- (2.4000, 0.2000, 0.0955) -- (2.4000, 0.2500, 0.1016) -- (2.3500, 0.2500, 0.1066) -- cycle;
\fill[blue!15.0, opacity=0.7] (2.3500, 0.2500, 0.1066) -- (2.4000, 0.2500, 0.1016) -- (2.4000, 0.3000, 0.1076) -- (2.3500, 0.3000, 0.1126) -- cycle;
\fill[blue!15.0, opacity=0.7] (2.3500, 0.3000, 0.1126) -- (2.4000, 0.3000, 0.1076) -- (2.4000, 0.3500, 0.1135) -- (2.3500, 0.3500, 0.1185) -- cycle;
\fill[blue!15.0, opacity=0.7] (2.3500, 0.3500, 0.1185) -- (2.4000, 0.3500, 0.1135) -- (2.4000, 0.4000, 0.1193) -- (2.3500, 0.4000, 0.1243) -- cycle;
\fill[blue!15.0, opacity=0.7] (2.3500, 0.4000, 0.1243) -- (2.4000, 0.4000, 0.1193) -- (2.4000, 0.4500, 0.1250) -- (2.3500, 0.4500, 0.1300) -- cycle;
\fill[blue!15.0, opacity=0.7] (2.3500, 0.4500, 0.1300) -- (2.4000, 0.4500, 0.1250) -- (2.4000, 0.5000, 0.1305) -- (2.3500, 0.5000, 0.1355) -- cycle;
\fill[blue!15.0, opacity=0.7] (2.3500, 0.5000, 0.1355) -- (2.4000, 0.5000, 0.1305) -- (2.4000, 0.5500, 0.1359) -- (2.3500, 0.5500, 0.1409) -- cycle;
\fill[blue!15.0, opacity=0.7] (2.3500, 0.5500, 0.1409) -- (2.4000, 0.5500, 0.1359) -- (2.4000, 0.6000, 0.1411) -- (2.3500, 0.6000, 0.1461) -- cycle;
\fill[blue!15.0, opacity=0.7] (2.3500, 0.6000, 0.1461) -- (2.4000, 0.6000, 0.1411) -- (2.4000, 0.6500, 0.1461) -- (2.3500, 0.6500, 0.1510) -- cycle;
\fill[blue!15.0, opacity=0.7] (2.3500, 0.6500, 0.1510) -- (2.4000, 0.6500, 0.1461) -- (2.4000, 0.7000, 0.1508) -- (2.3500, 0.7000, 0.1558) -- cycle;
\fill[blue!15.0, opacity=0.7] (2.3500, 0.7000, 0.1558) -- (2.4000, 0.7000, 0.1508) -- (2.4000, 0.7500, 0.1554) -- (2.3500, 0.7500, 0.1604) -- cycle;
\fill[blue!15.0, opacity=0.7] (2.3500, 0.7500, 0.1604) -- (2.4000, 0.7500, 0.1554) -- (2.4000, 0.8000, 0.1597) -- (2.3500, 0.8000, 0.1647) -- cycle;
\fill[blue!15.0, opacity=0.7] (2.3500, 0.8000, 0.1647) -- (2.4000, 0.8000, 0.1597) -- (2.4000, 0.8500, 0.1638) -- (2.3500, 0.8500, 0.1688) -- cycle;
\fill[blue!15.0, opacity=0.7] (2.3500, 0.8500, 0.1688) -- (2.4000, 0.8500, 0.1638) -- (2.4000, 0.9000, 0.1676) -- (2.3500, 0.9000, 0.1726) -- cycle;
\fill[blue!15.0, opacity=0.7] (2.3500, 0.9000, 0.1726) -- (2.4000, 0.9000, 0.1676) -- (2.4000, 0.9500, 0.1712) -- (2.3500, 0.9500, 0.1762) -- cycle;
\fill[blue!15.1, opacity=0.7] (2.3500, 0.9500, 0.1762) -- (2.4000, 0.9500, 0.1712) -- (2.4000, 1.0000, 0.1745) -- (2.3500, 1.0000, 0.1794) -- cycle;
\fill[blue!15.2, opacity=0.7] (2.3500, 1.0000, 0.1794) -- (2.4000, 1.0000, 0.1745) -- (2.4000, 1.0500, 0.1775) -- (2.3500, 1.0500, 0.1824) -- cycle;
\fill[blue!15.1, opacity=0.7] (2.3500, 1.0500, 0.1824) -- (2.4000, 1.0500, 0.1775) -- (2.4000, 1.1000, 0.1802) -- (2.3500, 1.1000, 0.1851) -- cycle;
\fill[blue!15.0, opacity=0.7] (2.3500, 1.1000, 0.1851) -- (2.4000, 1.1000, 0.1802) -- (2.4000, 1.1500, 0.1826) -- (2.3500, 1.1500, 0.1875) -- cycle;
\fill[blue!15.0, opacity=0.7] (2.3500, 1.1500, 0.1875) -- (2.4000, 1.1500, 0.1826) -- (2.4000, 1.2000, 0.1847) -- (2.3500, 1.2000, 0.1896) -- cycle;
\fill[blue!15.0, opacity=0.7] (2.3500, 1.2000, 0.1896) -- (2.4000, 1.2000, 0.1847) -- (2.4000, 1.2500, 0.1864) -- (2.3500, 1.2500, 0.1914) -- cycle;
\fill[blue!15.0, opacity=0.7] (2.3500, 1.2500, 0.1914) -- (2.4000, 1.2500, 0.1864) -- (2.4000, 1.3000, 0.1879) -- (2.3500, 1.3000, 0.1929) -- cycle;
\fill[blue!15.0, opacity=0.7] (2.3500, 1.3000, 0.1929) -- (2.4000, 1.3000, 0.1879) -- (2.4000, 1.3500, 0.1891) -- (2.3500, 1.3500, 0.1940) -- cycle;
\fill[blue!15.0, opacity=0.7] (2.3500, 1.3500, 0.1940) -- (2.4000, 1.3500, 0.1891) -- (2.4000, 1.4000, 0.1899) -- (2.3500, 1.4000, 0.1949) -- cycle;
\fill[blue!15.0, opacity=0.7] (2.3500, 1.4000, 0.1949) -- (2.4000, 1.4000, 0.1899) -- (2.4000, 1.4500, 0.1904) -- (2.3500, 1.4500, 0.1954) -- cycle;
\fill[blue!15.0, opacity=0.7] (2.3500, 1.4500, 0.1954) -- (2.4000, 1.4500, 0.1904) -- (2.4000, 1.5000, 0.1905) -- (2.3500, 1.5000, 0.1955) -- cycle;
\fill[blue!15.0, opacity=0.7] (2.3500, 1.5000, 0.1955) -- (2.4000, 1.5000, 0.1905) -- (2.4000, 1.5500, 0.1904) -- (2.3500, 1.5500, 0.1954) -- cycle;
\fill[blue!15.1, opacity=0.7] (2.3500, 1.5500, 0.1954) -- (2.4000, 1.5500, 0.1904) -- (2.4000, 1.6000, 0.1899) -- (2.3500, 1.6000, 0.1949) -- cycle;
\fill[blue!15.5, opacity=0.7] (2.3500, 1.6000, 0.1949) -- (2.4000, 1.6000, 0.1899) -- (2.4000, 1.6500, 0.1891) -- (2.3500, 1.6500, 0.1940) -- cycle;
\fill[blue!16.4, opacity=0.7] (2.3500, 1.6500, 0.1940) -- (2.4000, 1.6500, 0.1891) -- (2.4000, 1.7000, 0.1879) -- (2.3500, 1.7000, 0.1929) -- cycle;
\fill[blue!17.0, opacity=0.7] (2.3500, 1.7000, 0.1929) -- (2.4000, 1.7000, 0.1879) -- (2.4000, 1.7500, 0.1864) -- (2.3500, 1.7500, 0.1914) -- cycle;
\fill[blue!16.2, opacity=0.7] (2.3500, 1.7500, 0.1914) -- (2.4000, 1.7500, 0.1864) -- (2.4000, 1.8000, 0.1847) -- (2.3500, 1.8000, 0.1896) -- cycle;
\fill[blue!15.2, opacity=0.7] (2.3500, 1.8000, 0.1896) -- (2.4000, 1.8000, 0.1847) -- (2.4000, 1.8500, 0.1826) -- (2.3500, 1.8500, 0.1875) -- cycle;
\fill[blue!15.0, opacity=0.7] (2.3500, 1.8500, 0.1875) -- (2.4000, 1.8500, 0.1826) -- (2.4000, 1.9000, 0.1802) -- (2.3500, 1.9000, 0.1851) -- cycle;
\fill[blue!15.0, opacity=0.7] (2.3500, 1.9000, 0.1851) -- (2.4000, 1.9000, 0.1802) -- (2.4000, 1.9500, 0.1775) -- (2.3500, 1.9500, 0.1824) -- cycle;
\fill[blue!15.0, opacity=0.7] (2.3500, 1.9500, 0.1824) -- (2.4000, 1.9500, 0.1775) -- (2.4000, 2.0000, 0.1745) -- (2.3500, 2.0000, 0.1794) -- cycle;
\fill[blue!15.0, opacity=0.7] (2.3500, 2.0000, 0.1794) -- (2.4000, 2.0000, 0.1745) -- (2.4000, 2.0500, 0.1712) -- (2.3500, 2.0500, 0.1762) -- cycle;
\fill[blue!15.0, opacity=0.7] (2.3500, 2.0500, 0.1762) -- (2.4000, 2.0500, 0.1712) -- (2.4000, 2.1000, 0.1676) -- (2.3500, 2.1000, 0.1726) -- cycle;
\fill[blue!15.0, opacity=0.7] (2.3500, 2.1000, 0.1726) -- (2.4000, 2.1000, 0.1676) -- (2.4000, 2.1500, 0.1638) -- (2.3500, 2.1500, 0.1688) -- cycle;
\fill[blue!15.0, opacity=0.7] (2.3500, 2.1500, 0.1688) -- (2.4000, 2.1500, 0.1638) -- (2.4000, 2.2000, 0.1597) -- (2.3500, 2.2000, 0.1647) -- cycle;
\fill[blue!15.0, opacity=0.7] (2.3500, 2.2000, 0.1647) -- (2.4000, 2.2000, 0.1597) -- (2.4000, 2.2500, 0.1554) -- (2.3500, 2.2500, 0.1604) -- cycle;
\fill[blue!15.0, opacity=0.7] (2.3500, 2.2500, 0.1604) -- (2.4000, 2.2500, 0.1554) -- (2.4000, 2.3000, 0.1508) -- (2.3500, 2.3000, 0.1558) -- cycle;
\fill[blue!15.0, opacity=0.7] (2.3500, 2.3000, 0.1558) -- (2.4000, 2.3000, 0.1508) -- (2.4000, 2.3500, 0.1461) -- (2.3500, 2.3500, 0.1510) -- cycle;
\fill[blue!15.0, opacity=0.7] (2.3500, 2.3500, 0.1510) -- (2.4000, 2.3500, 0.1461) -- (2.4000, 2.4000, 0.1411) -- (2.3500, 2.4000, 0.1461) -- cycle;
\fill[blue!15.0, opacity=0.7] (2.3500, 2.4000, 0.1461) -- (2.4000, 2.4000, 0.1411) -- (2.4000, 2.4500, 0.1359) -- (2.3500, 2.4500, 0.1409) -- cycle;
\fill[blue!15.0, opacity=0.7] (2.3500, 2.4500, 0.1409) -- (2.4000, 2.4500, 0.1359) -- (2.4000, 2.5000, 0.1305) -- (2.3500, 2.5000, 0.1355) -- cycle;
\fill[blue!15.0, opacity=0.7] (2.3500, 2.5000, 0.1355) -- (2.4000, 2.5000, 0.1305) -- (2.4000, 2.5500, 0.1250) -- (2.3500, 2.5500, 0.1300) -- cycle;
\fill[blue!15.0, opacity=0.7] (2.3500, 2.5500, 0.1300) -- (2.4000, 2.5500, 0.1250) -- (2.4000, 2.6000, 0.1193) -- (2.3500, 2.6000, 0.1243) -- cycle;
\fill[blue!15.0, opacity=0.7] (2.3500, 2.6000, 0.1243) -- (2.4000, 2.6000, 0.1193) -- (2.4000, 2.6500, 0.1135) -- (2.3500, 2.6500, 0.1185) -- cycle;
\fill[blue!15.0, opacity=0.7] (2.3500, 2.6500, 0.1185) -- (2.4000, 2.6500, 0.1135) -- (2.4000, 2.7000, 0.1076) -- (2.3500, 2.7000, 0.1126) -- cycle;
\fill[blue!15.0, opacity=0.7] (2.3500, 2.7000, 0.1126) -- (2.4000, 2.7000, 0.1076) -- (2.4000, 2.7500, 0.1016) -- (2.3500, 2.7500, 0.1066) -- cycle;
\fill[blue!15.0, opacity=0.7] (2.3500, 2.7500, 0.1066) -- (2.4000, 2.7500, 0.1016) -- (2.4000, 2.8000, 0.0955) -- (2.3500, 2.8000, 0.1005) -- cycle;
\fill[blue!15.0, opacity=0.7] (2.3500, 2.8000, 0.1005) -- (2.4000, 2.8000, 0.0955) -- (2.4000, 2.8500, 0.0893) -- (2.3500, 2.8500, 0.0943) -- cycle;
\fill[blue!15.0, opacity=0.7] (2.3500, 2.8500, 0.0943) -- (2.4000, 2.8500, 0.0893) -- (2.4000, 2.9000, 0.0831) -- (2.3500, 2.9000, 0.0881) -- cycle;
\fill[blue!15.0, opacity=0.7] (2.3500, 2.9000, 0.0881) -- (2.4000, 2.9000, 0.0831) -- (2.4000, 2.9500, 0.0768) -- (2.3500, 2.9500, 0.0818) -- cycle;
\fill[blue!15.0, opacity=0.7] (2.3500, 2.9500, 0.0818) -- (2.4000, 2.9500, 0.0768) -- (2.4000, 3.0000, 0.0705) -- (2.3500, 3.0000, 0.0755) -- cycle;
\fill[blue!15.0, opacity=0.7] (2.4000, 0.0000, 0.0705) -- (2.4500, 0.0000, 0.0654) -- (2.4500, 0.0500, 0.0716) -- (2.4000, 0.0500, 0.0768) -- cycle;
\fill[blue!15.0, opacity=0.7] (2.4000, 0.0500, 0.0768) -- (2.4500, 0.0500, 0.0716) -- (2.4500, 0.1000, 0.0779) -- (2.4000, 0.1000, 0.0831) -- cycle;
\fill[blue!15.0, opacity=0.7] (2.4000, 0.1000, 0.0831) -- (2.4500, 0.1000, 0.0779) -- (2.4500, 0.1500, 0.0841) -- (2.4000, 0.1500, 0.0893) -- cycle;
\fill[blue!15.0, opacity=0.7] (2.4000, 0.1500, 0.0893) -- (2.4500, 0.1500, 0.0841) -- (2.4500, 0.2000, 0.0903) -- (2.4000, 0.2000, 0.0955) -- cycle;
\fill[blue!15.0, opacity=0.7] (2.4000, 0.2000, 0.0955) -- (2.4500, 0.2000, 0.0903) -- (2.4500, 0.2500, 0.0964) -- (2.4000, 0.2500, 0.1016) -- cycle;
\fill[blue!15.0, opacity=0.7] (2.4000, 0.2500, 0.1016) -- (2.4500, 0.2500, 0.0964) -- (2.4500, 0.3000, 0.1024) -- (2.4000, 0.3000, 0.1076) -- cycle;
\fill[blue!15.0, opacity=0.7] (2.4000, 0.3000, 0.1076) -- (2.4500, 0.3000, 0.1024) -- (2.4500, 0.3500, 0.1084) -- (2.4000, 0.3500, 0.1135) -- cycle;
\fill[blue!15.0, opacity=0.7] (2.4000, 0.3500, 0.1135) -- (2.4500, 0.3500, 0.1084) -- (2.4500, 0.4000, 0.1142) -- (2.4000, 0.4000, 0.1193) -- cycle;
\fill[blue!15.0, opacity=0.7] (2.4000, 0.4000, 0.1193) -- (2.4500, 0.4000, 0.1142) -- (2.4500, 0.4500, 0.1198) -- (2.4000, 0.4500, 0.1250) -- cycle;
\fill[blue!15.0, opacity=0.7] (2.4000, 0.4500, 0.1250) -- (2.4500, 0.4500, 0.1198) -- (2.4500, 0.5000, 0.1254) -- (2.4000, 0.5000, 0.1305) -- cycle;
\fill[blue!15.0, opacity=0.7] (2.4000, 0.5000, 0.1305) -- (2.4500, 0.5000, 0.1254) -- (2.4500, 0.5500, 0.1307) -- (2.4000, 0.5500, 0.1359) -- cycle;
\fill[blue!15.0, opacity=0.7] (2.4000, 0.5500, 0.1359) -- (2.4500, 0.5500, 0.1307) -- (2.4500, 0.6000, 0.1359) -- (2.4000, 0.6000, 0.1411) -- cycle;
\fill[blue!15.0, opacity=0.7] (2.4000, 0.6000, 0.1411) -- (2.4500, 0.6000, 0.1359) -- (2.4500, 0.6500, 0.1409) -- (2.4000, 0.6500, 0.1461) -- cycle;
\fill[blue!15.0, opacity=0.7] (2.4000, 0.6500, 0.1461) -- (2.4500, 0.6500, 0.1409) -- (2.4500, 0.7000, 0.1457) -- (2.4000, 0.7000, 0.1508) -- cycle;
\fill[blue!15.0, opacity=0.7] (2.4000, 0.7000, 0.1508) -- (2.4500, 0.7000, 0.1457) -- (2.4500, 0.7500, 0.1502) -- (2.4000, 0.7500, 0.1554) -- cycle;
\fill[blue!15.0, opacity=0.7] (2.4000, 0.7500, 0.1554) -- (2.4500, 0.7500, 0.1502) -- (2.4500, 0.8000, 0.1545) -- (2.4000, 0.8000, 0.1597) -- cycle;
\fill[blue!15.0, opacity=0.7] (2.4000, 0.8000, 0.1597) -- (2.4500, 0.8000, 0.1545) -- (2.4500, 0.8500, 0.1586) -- (2.4000, 0.8500, 0.1638) -- cycle;
\fill[blue!15.0, opacity=0.7] (2.4000, 0.8500, 0.1638) -- (2.4500, 0.8500, 0.1586) -- (2.4500, 0.9000, 0.1624) -- (2.4000, 0.9000, 0.1676) -- cycle;
\fill[blue!15.0, opacity=0.7] (2.4000, 0.9000, 0.1676) -- (2.4500, 0.9000, 0.1624) -- (2.4500, 0.9500, 0.1660) -- (2.4000, 0.9500, 0.1712) -- cycle;
\fill[blue!15.0, opacity=0.7] (2.4000, 0.9500, 0.1712) -- (2.4500, 0.9500, 0.1660) -- (2.4500, 1.0000, 0.1693) -- (2.4000, 1.0000, 0.1745) -- cycle;
\fill[blue!15.0, opacity=0.7] (2.4000, 1.0000, 0.1745) -- (2.4500, 1.0000, 0.1693) -- (2.4500, 1.0500, 0.1723) -- (2.4000, 1.0500, 0.1775) -- cycle;
\fill[blue!15.1, opacity=0.7] (2.4000, 1.0500, 0.1775) -- (2.4500, 1.0500, 0.1723) -- (2.4500, 1.1000, 0.1750) -- (2.4000, 1.1000, 0.1802) -- cycle;
\fill[blue!15.2, opacity=0.7] (2.4000, 1.1000, 0.1802) -- (2.4500, 1.1000, 0.1750) -- (2.4500, 1.1500, 0.1774) -- (2.4000, 1.1500, 0.1826) -- cycle;
\fill[blue!15.4, opacity=0.7] (2.4000, 1.1500, 0.1826) -- (2.4500, 1.1500, 0.1774) -- (2.4500, 1.2000, 0.1795) -- (2.4000, 1.2000, 0.1847) -- cycle;
\fill[blue!15.4, opacity=0.7] (2.4000, 1.2000, 0.1847) -- (2.4500, 1.2000, 0.1795) -- (2.4500, 1.2500, 0.1813) -- (2.4000, 1.2500, 0.1864) -- cycle;
\fill[blue!15.4, opacity=0.7] (2.4000, 1.2500, 0.1864) -- (2.4500, 1.2500, 0.1813) -- (2.4500, 1.3000, 0.1827) -- (2.4000, 1.3000, 0.1879) -- cycle;
\fill[blue!15.4, opacity=0.7] (2.4000, 1.3000, 0.1879) -- (2.4500, 1.3000, 0.1827) -- (2.4500, 1.3500, 0.1839) -- (2.4000, 1.3500, 0.1891) -- cycle;
\fill[blue!15.4, opacity=0.7] (2.4000, 1.3500, 0.1891) -- (2.4500, 1.3500, 0.1839) -- (2.4500, 1.4000, 0.1847) -- (2.4000, 1.4000, 0.1899) -- cycle;
\fill[blue!15.6, opacity=0.7] (2.4000, 1.4000, 0.1899) -- (2.4500, 1.4000, 0.1847) -- (2.4500, 1.4500, 0.1852) -- (2.4000, 1.4500, 0.1904) -- cycle;
\fill[blue!15.8, opacity=0.7] (2.4000, 1.4500, 0.1904) -- (2.4500, 1.4500, 0.1852) -- (2.4500, 1.5000, 0.1854) -- (2.4000, 1.5000, 0.1905) -- cycle;
\fill[blue!16.1, opacity=0.7] (2.4000, 1.5000, 0.1905) -- (2.4500, 1.5000, 0.1854) -- (2.4500, 1.5500, 0.1852) -- (2.4000, 1.5500, 0.1904) -- cycle;
\fill[blue!16.2, opacity=0.7] (2.4000, 1.5500, 0.1904) -- (2.4500, 1.5500, 0.1852) -- (2.4500, 1.6000, 0.1847) -- (2.4000, 1.6000, 0.1899) -- cycle;
\fill[blue!16.0, opacity=0.7] (2.4000, 1.6000, 0.1899) -- (2.4500, 1.6000, 0.1847) -- (2.4500, 1.6500, 0.1839) -- (2.4000, 1.6500, 0.1891) -- cycle;
\fill[blue!15.5, opacity=0.7] (2.4000, 1.6500, 0.1891) -- (2.4500, 1.6500, 0.1839) -- (2.4500, 1.7000, 0.1827) -- (2.4000, 1.7000, 0.1879) -- cycle;
\fill[blue!15.1, opacity=0.7] (2.4000, 1.7000, 0.1879) -- (2.4500, 1.7000, 0.1827) -- (2.4500, 1.7500, 0.1813) -- (2.4000, 1.7500, 0.1864) -- cycle;
\fill[blue!15.0, opacity=0.7] (2.4000, 1.7500, 0.1864) -- (2.4500, 1.7500, 0.1813) -- (2.4500, 1.8000, 0.1795) -- (2.4000, 1.8000, 0.1847) -- cycle;
\fill[blue!15.0, opacity=0.7] (2.4000, 1.8000, 0.1847) -- (2.4500, 1.8000, 0.1795) -- (2.4500, 1.8500, 0.1774) -- (2.4000, 1.8500, 0.1826) -- cycle;
\fill[blue!15.0, opacity=0.7] (2.4000, 1.8500, 0.1826) -- (2.4500, 1.8500, 0.1774) -- (2.4500, 1.9000, 0.1750) -- (2.4000, 1.9000, 0.1802) -- cycle;
\fill[blue!15.0, opacity=0.7] (2.4000, 1.9000, 0.1802) -- (2.4500, 1.9000, 0.1750) -- (2.4500, 1.9500, 0.1723) -- (2.4000, 1.9500, 0.1775) -- cycle;
\fill[blue!15.0, opacity=0.7] (2.4000, 1.9500, 0.1775) -- (2.4500, 1.9500, 0.1723) -- (2.4500, 2.0000, 0.1693) -- (2.4000, 2.0000, 0.1745) -- cycle;
\fill[blue!15.0, opacity=0.7] (2.4000, 2.0000, 0.1745) -- (2.4500, 2.0000, 0.1693) -- (2.4500, 2.0500, 0.1660) -- (2.4000, 2.0500, 0.1712) -- cycle;
\fill[blue!15.0, opacity=0.7] (2.4000, 2.0500, 0.1712) -- (2.4500, 2.0500, 0.1660) -- (2.4500, 2.1000, 0.1624) -- (2.4000, 2.1000, 0.1676) -- cycle;
\fill[blue!15.0, opacity=0.7] (2.4000, 2.1000, 0.1676) -- (2.4500, 2.1000, 0.1624) -- (2.4500, 2.1500, 0.1586) -- (2.4000, 2.1500, 0.1638) -- cycle;
\fill[blue!15.0, opacity=0.7] (2.4000, 2.1500, 0.1638) -- (2.4500, 2.1500, 0.1586) -- (2.4500, 2.2000, 0.1545) -- (2.4000, 2.2000, 0.1597) -- cycle;
\fill[blue!15.0, opacity=0.7] (2.4000, 2.2000, 0.1597) -- (2.4500, 2.2000, 0.1545) -- (2.4500, 2.2500, 0.1502) -- (2.4000, 2.2500, 0.1554) -- cycle;
\fill[blue!15.0, opacity=0.7] (2.4000, 2.2500, 0.1554) -- (2.4500, 2.2500, 0.1502) -- (2.4500, 2.3000, 0.1457) -- (2.4000, 2.3000, 0.1508) -- cycle;
\fill[blue!15.0, opacity=0.7] (2.4000, 2.3000, 0.1508) -- (2.4500, 2.3000, 0.1457) -- (2.4500, 2.3500, 0.1409) -- (2.4000, 2.3500, 0.1461) -- cycle;
\fill[blue!15.0, opacity=0.7] (2.4000, 2.3500, 0.1461) -- (2.4500, 2.3500, 0.1409) -- (2.4500, 2.4000, 0.1359) -- (2.4000, 2.4000, 0.1411) -- cycle;
\fill[blue!15.0, opacity=0.7] (2.4000, 2.4000, 0.1411) -- (2.4500, 2.4000, 0.1359) -- (2.4500, 2.4500, 0.1307) -- (2.4000, 2.4500, 0.1359) -- cycle;
\fill[blue!15.0, opacity=0.7] (2.4000, 2.4500, 0.1359) -- (2.4500, 2.4500, 0.1307) -- (2.4500, 2.5000, 0.1254) -- (2.4000, 2.5000, 0.1305) -- cycle;
\fill[blue!15.0, opacity=0.7] (2.4000, 2.5000, 0.1305) -- (2.4500, 2.5000, 0.1254) -- (2.4500, 2.5500, 0.1198) -- (2.4000, 2.5500, 0.1250) -- cycle;
\fill[blue!15.0, opacity=0.7] (2.4000, 2.5500, 0.1250) -- (2.4500, 2.5500, 0.1198) -- (2.4500, 2.6000, 0.1142) -- (2.4000, 2.6000, 0.1193) -- cycle;
\fill[blue!15.0, opacity=0.7] (2.4000, 2.6000, 0.1193) -- (2.4500, 2.6000, 0.1142) -- (2.4500, 2.6500, 0.1084) -- (2.4000, 2.6500, 0.1135) -- cycle;
\fill[blue!15.0, opacity=0.7] (2.4000, 2.6500, 0.1135) -- (2.4500, 2.6500, 0.1084) -- (2.4500, 2.7000, 0.1024) -- (2.4000, 2.7000, 0.1076) -- cycle;
\fill[blue!15.0, opacity=0.7] (2.4000, 2.7000, 0.1076) -- (2.4500, 2.7000, 0.1024) -- (2.4500, 2.7500, 0.0964) -- (2.4000, 2.7500, 0.1016) -- cycle;
\fill[blue!15.0, opacity=0.7] (2.4000, 2.7500, 0.1016) -- (2.4500, 2.7500, 0.0964) -- (2.4500, 2.8000, 0.0903) -- (2.4000, 2.8000, 0.0955) -- cycle;
\fill[blue!15.0, opacity=0.7] (2.4000, 2.8000, 0.0955) -- (2.4500, 2.8000, 0.0903) -- (2.4500, 2.8500, 0.0841) -- (2.4000, 2.8500, 0.0893) -- cycle;
\fill[blue!15.0, opacity=0.7] (2.4000, 2.8500, 0.0893) -- (2.4500, 2.8500, 0.0841) -- (2.4500, 2.9000, 0.0779) -- (2.4000, 2.9000, 0.0831) -- cycle;
\fill[blue!15.0, opacity=0.7] (2.4000, 2.9000, 0.0831) -- (2.4500, 2.9000, 0.0779) -- (2.4500, 2.9500, 0.0716) -- (2.4000, 2.9500, 0.0768) -- cycle;
\fill[blue!15.0, opacity=0.7] (2.4000, 2.9500, 0.0768) -- (2.4500, 2.9500, 0.0716) -- (2.4500, 3.0000, 0.0654) -- (2.4000, 3.0000, 0.0705) -- cycle;
\fill[blue!15.0, opacity=0.7] (2.4500, 0.0000, 0.0654) -- (2.5000, 0.0000, 0.0600) -- (2.5000, 0.0500, 0.0663) -- (2.4500, 0.0500, 0.0716) -- cycle;
\fill[blue!15.0, opacity=0.7] (2.4500, 0.0500, 0.0716) -- (2.5000, 0.0500, 0.0663) -- (2.5000, 0.1000, 0.0725) -- (2.4500, 0.1000, 0.0779) -- cycle;
\fill[blue!15.0, opacity=0.7] (2.4500, 0.1000, 0.0779) -- (2.5000, 0.1000, 0.0725) -- (2.5000, 0.1500, 0.0788) -- (2.4500, 0.1500, 0.0841) -- cycle;
\fill[blue!15.0, opacity=0.7] (2.4500, 0.1500, 0.0841) -- (2.5000, 0.1500, 0.0788) -- (2.5000, 0.2000, 0.0849) -- (2.4500, 0.2000, 0.0903) -- cycle;
\fill[blue!15.0, opacity=0.7] (2.4500, 0.2000, 0.0903) -- (2.5000, 0.2000, 0.0849) -- (2.5000, 0.2500, 0.0911) -- (2.4500, 0.2500, 0.0964) -- cycle;
\fill[blue!15.0, opacity=0.7] (2.4500, 0.2500, 0.0964) -- (2.5000, 0.2500, 0.0911) -- (2.5000, 0.3000, 0.0971) -- (2.4500, 0.3000, 0.1024) -- cycle;
\fill[blue!15.0, opacity=0.7] (2.4500, 0.3000, 0.1024) -- (2.5000, 0.3000, 0.0971) -- (2.5000, 0.3500, 0.1030) -- (2.4500, 0.3500, 0.1084) -- cycle;
\fill[blue!15.0, opacity=0.7] (2.4500, 0.3500, 0.1084) -- (2.5000, 0.3500, 0.1030) -- (2.5000, 0.4000, 0.1088) -- (2.4500, 0.4000, 0.1142) -- cycle;
\fill[blue!15.0, opacity=0.7] (2.4500, 0.4000, 0.1142) -- (2.5000, 0.4000, 0.1088) -- (2.5000, 0.4500, 0.1145) -- (2.4500, 0.4500, 0.1198) -- cycle;
\fill[blue!15.0, opacity=0.7] (2.4500, 0.4500, 0.1198) -- (2.5000, 0.4500, 0.1145) -- (2.5000, 0.5000, 0.1200) -- (2.4500, 0.5000, 0.1254) -- cycle;
\fill[blue!15.0, opacity=0.7] (2.4500, 0.5000, 0.1254) -- (2.5000, 0.5000, 0.1200) -- (2.5000, 0.5500, 0.1254) -- (2.4500, 0.5500, 0.1307) -- cycle;
\fill[blue!15.0, opacity=0.7] (2.4500, 0.5500, 0.1307) -- (2.5000, 0.5500, 0.1254) -- (2.5000, 0.6000, 0.1305) -- (2.4500, 0.6000, 0.1359) -- cycle;
\fill[blue!15.0, opacity=0.7] (2.4500, 0.6000, 0.1359) -- (2.5000, 0.6000, 0.1305) -- (2.5000, 0.6500, 0.1355) -- (2.4500, 0.6500, 0.1409) -- cycle;
\fill[blue!15.0, opacity=0.7] (2.4500, 0.6500, 0.1409) -- (2.5000, 0.6500, 0.1355) -- (2.5000, 0.7000, 0.1403) -- (2.4500, 0.7000, 0.1457) -- cycle;
\fill[blue!15.0, opacity=0.7] (2.4500, 0.7000, 0.1457) -- (2.5000, 0.7000, 0.1403) -- (2.5000, 0.7500, 0.1449) -- (2.4500, 0.7500, 0.1502) -- cycle;
\fill[blue!15.0, opacity=0.7] (2.4500, 0.7500, 0.1502) -- (2.5000, 0.7500, 0.1449) -- (2.5000, 0.8000, 0.1492) -- (2.4500, 0.8000, 0.1545) -- cycle;
\fill[blue!15.0, opacity=0.7] (2.4500, 0.8000, 0.1545) -- (2.5000, 0.8000, 0.1492) -- (2.5000, 0.8500, 0.1533) -- (2.4500, 0.8500, 0.1586) -- cycle;
\fill[blue!15.0, opacity=0.7] (2.4500, 0.8500, 0.1586) -- (2.5000, 0.8500, 0.1533) -- (2.5000, 0.9000, 0.1571) -- (2.4500, 0.9000, 0.1624) -- cycle;
\fill[blue!15.0, opacity=0.7] (2.4500, 0.9000, 0.1624) -- (2.5000, 0.9000, 0.1571) -- (2.5000, 0.9500, 0.1606) -- (2.4500, 0.9500, 0.1660) -- cycle;
\fill[blue!15.0, opacity=0.7] (2.4500, 0.9500, 0.1660) -- (2.5000, 0.9500, 0.1606) -- (2.5000, 1.0000, 0.1639) -- (2.4500, 1.0000, 0.1693) -- cycle;
\fill[blue!15.0, opacity=0.7] (2.4500, 1.0000, 0.1693) -- (2.5000, 1.0000, 0.1639) -- (2.5000, 1.0500, 0.1669) -- (2.4500, 1.0500, 0.1723) -- cycle;
\fill[blue!15.0, opacity=0.7] (2.4500, 1.0500, 0.1723) -- (2.5000, 1.0500, 0.1669) -- (2.5000, 1.1000, 0.1696) -- (2.4500, 1.1000, 0.1750) -- cycle;
\fill[blue!15.0, opacity=0.7] (2.4500, 1.1000, 0.1750) -- (2.5000, 1.1000, 0.1696) -- (2.5000, 1.1500, 0.1720) -- (2.4500, 1.1500, 0.1774) -- cycle;
\fill[blue!15.0, opacity=0.7] (2.4500, 1.1500, 0.1774) -- (2.5000, 1.1500, 0.1720) -- (2.5000, 1.2000, 0.1741) -- (2.4500, 1.2000, 0.1795) -- cycle;
\fill[blue!15.0, opacity=0.7] (2.4500, 1.2000, 0.1795) -- (2.5000, 1.2000, 0.1741) -- (2.5000, 1.2500, 0.1759) -- (2.4500, 1.2500, 0.1813) -- cycle;
\fill[blue!15.1, opacity=0.7] (2.4500, 1.2500, 0.1813) -- (2.5000, 1.2500, 0.1759) -- (2.5000, 1.3000, 0.1774) -- (2.4500, 1.3000, 0.1827) -- cycle;
\fill[blue!15.1, opacity=0.7] (2.4500, 1.3000, 0.1827) -- (2.5000, 1.3000, 0.1774) -- (2.5000, 1.3500, 0.1785) -- (2.4500, 1.3500, 0.1839) -- cycle;
\fill[blue!15.2, opacity=0.7] (2.4500, 1.3500, 0.1839) -- (2.5000, 1.3500, 0.1785) -- (2.5000, 1.4000, 0.1793) -- (2.4500, 1.4000, 0.1847) -- cycle;
\fill[blue!15.2, opacity=0.7] (2.4500, 1.4000, 0.1847) -- (2.5000, 1.4000, 0.1793) -- (2.5000, 1.4500, 0.1798) -- (2.4500, 1.4500, 0.1852) -- cycle;
\fill[blue!15.1, opacity=0.7] (2.4500, 1.4500, 0.1852) -- (2.5000, 1.4500, 0.1798) -- (2.5000, 1.5000, 0.1800) -- (2.4500, 1.5000, 0.1854) -- cycle;
\fill[blue!15.1, opacity=0.7] (2.4500, 1.5000, 0.1854) -- (2.5000, 1.5000, 0.1800) -- (2.5000, 1.5500, 0.1798) -- (2.4500, 1.5500, 0.1852) -- cycle;
\fill[blue!15.0, opacity=0.7] (2.4500, 1.5500, 0.1852) -- (2.5000, 1.5500, 0.1798) -- (2.5000, 1.6000, 0.1793) -- (2.4500, 1.6000, 0.1847) -- cycle;
\fill[blue!15.0, opacity=0.7] (2.4500, 1.6000, 0.1847) -- (2.5000, 1.6000, 0.1793) -- (2.5000, 1.6500, 0.1785) -- (2.4500, 1.6500, 0.1839) -- cycle;
\fill[blue!15.0, opacity=0.7] (2.4500, 1.6500, 0.1839) -- (2.5000, 1.6500, 0.1785) -- (2.5000, 1.7000, 0.1774) -- (2.4500, 1.7000, 0.1827) -- cycle;
\fill[blue!15.0, opacity=0.7] (2.4500, 1.7000, 0.1827) -- (2.5000, 1.7000, 0.1774) -- (2.5000, 1.7500, 0.1759) -- (2.4500, 1.7500, 0.1813) -- cycle;
\fill[blue!15.0, opacity=0.7] (2.4500, 1.7500, 0.1813) -- (2.5000, 1.7500, 0.1759) -- (2.5000, 1.8000, 0.1741) -- (2.4500, 1.8000, 0.1795) -- cycle;
\fill[blue!15.0, opacity=0.7] (2.4500, 1.8000, 0.1795) -- (2.5000, 1.8000, 0.1741) -- (2.5000, 1.8500, 0.1720) -- (2.4500, 1.8500, 0.1774) -- cycle;
\fill[blue!15.0, opacity=0.7] (2.4500, 1.8500, 0.1774) -- (2.5000, 1.8500, 0.1720) -- (2.5000, 1.9000, 0.1696) -- (2.4500, 1.9000, 0.1750) -- cycle;
\fill[blue!15.0, opacity=0.7] (2.4500, 1.9000, 0.1750) -- (2.5000, 1.9000, 0.1696) -- (2.5000, 1.9500, 0.1669) -- (2.4500, 1.9500, 0.1723) -- cycle;
\fill[blue!15.0, opacity=0.7] (2.4500, 1.9500, 0.1723) -- (2.5000, 1.9500, 0.1669) -- (2.5000, 2.0000, 0.1639) -- (2.4500, 2.0000, 0.1693) -- cycle;
\fill[blue!15.0, opacity=0.7] (2.4500, 2.0000, 0.1693) -- (2.5000, 2.0000, 0.1639) -- (2.5000, 2.0500, 0.1606) -- (2.4500, 2.0500, 0.1660) -- cycle;
\fill[blue!15.0, opacity=0.7] (2.4500, 2.0500, 0.1660) -- (2.5000, 2.0500, 0.1606) -- (2.5000, 2.1000, 0.1571) -- (2.4500, 2.1000, 0.1624) -- cycle;
\fill[blue!15.0, opacity=0.7] (2.4500, 2.1000, 0.1624) -- (2.5000, 2.1000, 0.1571) -- (2.5000, 2.1500, 0.1533) -- (2.4500, 2.1500, 0.1586) -- cycle;
\fill[blue!15.0, opacity=0.7] (2.4500, 2.1500, 0.1586) -- (2.5000, 2.1500, 0.1533) -- (2.5000, 2.2000, 0.1492) -- (2.4500, 2.2000, 0.1545) -- cycle;
\fill[blue!15.0, opacity=0.7] (2.4500, 2.2000, 0.1545) -- (2.5000, 2.2000, 0.1492) -- (2.5000, 2.2500, 0.1449) -- (2.4500, 2.2500, 0.1502) -- cycle;
\fill[blue!15.0, opacity=0.7] (2.4500, 2.2500, 0.1502) -- (2.5000, 2.2500, 0.1449) -- (2.5000, 2.3000, 0.1403) -- (2.4500, 2.3000, 0.1457) -- cycle;
\fill[blue!15.0, opacity=0.7] (2.4500, 2.3000, 0.1457) -- (2.5000, 2.3000, 0.1403) -- (2.5000, 2.3500, 0.1355) -- (2.4500, 2.3500, 0.1409) -- cycle;
\fill[blue!15.0, opacity=0.7] (2.4500, 2.3500, 0.1409) -- (2.5000, 2.3500, 0.1355) -- (2.5000, 2.4000, 0.1305) -- (2.4500, 2.4000, 0.1359) -- cycle;
\fill[blue!15.0, opacity=0.7] (2.4500, 2.4000, 0.1359) -- (2.5000, 2.4000, 0.1305) -- (2.5000, 2.4500, 0.1254) -- (2.4500, 2.4500, 0.1307) -- cycle;
\fill[blue!15.0, opacity=0.7] (2.4500, 2.4500, 0.1307) -- (2.5000, 2.4500, 0.1254) -- (2.5000, 2.5000, 0.1200) -- (2.4500, 2.5000, 0.1254) -- cycle;
\fill[blue!15.0, opacity=0.7] (2.4500, 2.5000, 0.1254) -- (2.5000, 2.5000, 0.1200) -- (2.5000, 2.5500, 0.1145) -- (2.4500, 2.5500, 0.1198) -- cycle;
\fill[blue!15.0, opacity=0.7] (2.4500, 2.5500, 0.1198) -- (2.5000, 2.5500, 0.1145) -- (2.5000, 2.6000, 0.1088) -- (2.4500, 2.6000, 0.1142) -- cycle;
\fill[blue!15.0, opacity=0.7] (2.4500, 2.6000, 0.1142) -- (2.5000, 2.6000, 0.1088) -- (2.5000, 2.6500, 0.1030) -- (2.4500, 2.6500, 0.1084) -- cycle;
\fill[blue!15.0, opacity=0.7] (2.4500, 2.6500, 0.1084) -- (2.5000, 2.6500, 0.1030) -- (2.5000, 2.7000, 0.0971) -- (2.4500, 2.7000, 0.1024) -- cycle;
\fill[blue!15.0, opacity=0.7] (2.4500, 2.7000, 0.1024) -- (2.5000, 2.7000, 0.0971) -- (2.5000, 2.7500, 0.0911) -- (2.4500, 2.7500, 0.0964) -- cycle;
\fill[blue!15.0, opacity=0.7] (2.4500, 2.7500, 0.0964) -- (2.5000, 2.7500, 0.0911) -- (2.5000, 2.8000, 0.0849) -- (2.4500, 2.8000, 0.0903) -- cycle;
\fill[blue!15.0, opacity=0.7] (2.4500, 2.8000, 0.0903) -- (2.5000, 2.8000, 0.0849) -- (2.5000, 2.8500, 0.0788) -- (2.4500, 2.8500, 0.0841) -- cycle;
\fill[blue!15.0, opacity=0.7] (2.4500, 2.8500, 0.0841) -- (2.5000, 2.8500, 0.0788) -- (2.5000, 2.9000, 0.0725) -- (2.4500, 2.9000, 0.0779) -- cycle;
\fill[blue!15.0, opacity=0.7] (2.4500, 2.9000, 0.0779) -- (2.5000, 2.9000, 0.0725) -- (2.5000, 2.9500, 0.0663) -- (2.4500, 2.9500, 0.0716) -- cycle;
\fill[blue!15.0, opacity=0.7] (2.4500, 2.9500, 0.0716) -- (2.5000, 2.9500, 0.0663) -- (2.5000, 3.0000, 0.0600) -- (2.4500, 3.0000, 0.0654) -- cycle;
\fill[blue!15.0, opacity=0.7] (2.5000, 0.0000, 0.0600) -- (2.5500, 0.0000, 0.0545) -- (2.5500, 0.0500, 0.0608) -- (2.5000, 0.0500, 0.0663) -- cycle;
\fill[blue!15.0, opacity=0.7] (2.5000, 0.0500, 0.0663) -- (2.5500, 0.0500, 0.0608) -- (2.5500, 0.1000, 0.0670) -- (2.5000, 0.1000, 0.0725) -- cycle;
\fill[blue!15.0, opacity=0.7] (2.5000, 0.1000, 0.0725) -- (2.5500, 0.1000, 0.0670) -- (2.5500, 0.1500, 0.0733) -- (2.5000, 0.1500, 0.0788) -- cycle;
\fill[blue!15.0, opacity=0.7] (2.5000, 0.1500, 0.0788) -- (2.5500, 0.1500, 0.0733) -- (2.5500, 0.2000, 0.0794) -- (2.5000, 0.2000, 0.0849) -- cycle;
\fill[blue!15.0, opacity=0.7] (2.5000, 0.2000, 0.0849) -- (2.5500, 0.2000, 0.0794) -- (2.5500, 0.2500, 0.0855) -- (2.5000, 0.2500, 0.0911) -- cycle;
\fill[blue!15.0, opacity=0.7] (2.5000, 0.2500, 0.0911) -- (2.5500, 0.2500, 0.0855) -- (2.5500, 0.3000, 0.0916) -- (2.5000, 0.3000, 0.0971) -- cycle;
\fill[blue!15.0, opacity=0.7] (2.5000, 0.3000, 0.0971) -- (2.5500, 0.3000, 0.0916) -- (2.5500, 0.3500, 0.0975) -- (2.5000, 0.3500, 0.1030) -- cycle;
\fill[blue!15.0, opacity=0.7] (2.5000, 0.3500, 0.1030) -- (2.5500, 0.3500, 0.0975) -- (2.5500, 0.4000, 0.1033) -- (2.5000, 0.4000, 0.1088) -- cycle;
\fill[blue!15.0, opacity=0.7] (2.5000, 0.4000, 0.1088) -- (2.5500, 0.4000, 0.1033) -- (2.5500, 0.4500, 0.1090) -- (2.5000, 0.4500, 0.1145) -- cycle;
\fill[blue!15.0, opacity=0.7] (2.5000, 0.4500, 0.1145) -- (2.5500, 0.4500, 0.1090) -- (2.5500, 0.5000, 0.1145) -- (2.5000, 0.5000, 0.1200) -- cycle;
\fill[blue!15.0, opacity=0.7] (2.5000, 0.5000, 0.1200) -- (2.5500, 0.5000, 0.1145) -- (2.5500, 0.5500, 0.1198) -- (2.5000, 0.5500, 0.1254) -- cycle;
\fill[blue!15.0, opacity=0.7] (2.5000, 0.5500, 0.1254) -- (2.5500, 0.5500, 0.1198) -- (2.5500, 0.6000, 0.1250) -- (2.5000, 0.6000, 0.1305) -- cycle;
\fill[blue!15.0, opacity=0.7] (2.5000, 0.6000, 0.1305) -- (2.5500, 0.6000, 0.1250) -- (2.5500, 0.6500, 0.1300) -- (2.5000, 0.6500, 0.1355) -- cycle;
\fill[blue!15.0, opacity=0.7] (2.5000, 0.6500, 0.1355) -- (2.5500, 0.6500, 0.1300) -- (2.5500, 0.7000, 0.1348) -- (2.5000, 0.7000, 0.1403) -- cycle;
\fill[blue!15.0, opacity=0.7] (2.5000, 0.7000, 0.1403) -- (2.5500, 0.7000, 0.1348) -- (2.5500, 0.7500, 0.1393) -- (2.5000, 0.7500, 0.1449) -- cycle;
\fill[blue!15.0, opacity=0.7] (2.5000, 0.7500, 0.1449) -- (2.5500, 0.7500, 0.1393) -- (2.5500, 0.8000, 0.1437) -- (2.5000, 0.8000, 0.1492) -- cycle;
\fill[blue!15.0, opacity=0.7] (2.5000, 0.8000, 0.1492) -- (2.5500, 0.8000, 0.1437) -- (2.5500, 0.8500, 0.1477) -- (2.5000, 0.8500, 0.1533) -- cycle;
\fill[blue!15.0, opacity=0.7] (2.5000, 0.8500, 0.1533) -- (2.5500, 0.8500, 0.1477) -- (2.5500, 0.9000, 0.1516) -- (2.5000, 0.9000, 0.1571) -- cycle;
\fill[blue!15.0, opacity=0.7] (2.5000, 0.9000, 0.1571) -- (2.5500, 0.9000, 0.1516) -- (2.5500, 0.9500, 0.1551) -- (2.5000, 0.9500, 0.1606) -- cycle;
\fill[blue!15.0, opacity=0.7] (2.5000, 0.9500, 0.1606) -- (2.5500, 0.9500, 0.1551) -- (2.5500, 1.0000, 0.1584) -- (2.5000, 1.0000, 0.1639) -- cycle;
\fill[blue!15.0, opacity=0.7] (2.5000, 1.0000, 0.1639) -- (2.5500, 1.0000, 0.1584) -- (2.5500, 1.0500, 0.1614) -- (2.5000, 1.0500, 0.1669) -- cycle;
\fill[blue!15.0, opacity=0.7] (2.5000, 1.0500, 0.1669) -- (2.5500, 1.0500, 0.1614) -- (2.5500, 1.1000, 0.1641) -- (2.5000, 1.1000, 0.1696) -- cycle;
\fill[blue!15.0, opacity=0.7] (2.5000, 1.1000, 0.1696) -- (2.5500, 1.1000, 0.1641) -- (2.5500, 1.1500, 0.1665) -- (2.5000, 1.1500, 0.1720) -- cycle;
\fill[blue!15.0, opacity=0.7] (2.5000, 1.1500, 0.1720) -- (2.5500, 1.1500, 0.1665) -- (2.5500, 1.2000, 0.1686) -- (2.5000, 1.2000, 0.1741) -- cycle;
\fill[blue!15.0, opacity=0.7] (2.5000, 1.2000, 0.1741) -- (2.5500, 1.2000, 0.1686) -- (2.5500, 1.2500, 0.1704) -- (2.5000, 1.2500, 0.1759) -- cycle;
\fill[blue!15.0, opacity=0.7] (2.5000, 1.2500, 0.1759) -- (2.5500, 1.2500, 0.1704) -- (2.5500, 1.3000, 0.1719) -- (2.5000, 1.3000, 0.1774) -- cycle;
\fill[blue!15.0, opacity=0.7] (2.5000, 1.3000, 0.1774) -- (2.5500, 1.3000, 0.1719) -- (2.5500, 1.3500, 0.1730) -- (2.5000, 1.3500, 0.1785) -- cycle;
\fill[blue!15.0, opacity=0.7] (2.5000, 1.3500, 0.1785) -- (2.5500, 1.3500, 0.1730) -- (2.5500, 1.4000, 0.1738) -- (2.5000, 1.4000, 0.1793) -- cycle;
\fill[blue!15.0, opacity=0.7] (2.5000, 1.4000, 0.1793) -- (2.5500, 1.4000, 0.1738) -- (2.5500, 1.4500, 0.1743) -- (2.5000, 1.4500, 0.1798) -- cycle;
\fill[blue!15.0, opacity=0.7] (2.5000, 1.4500, 0.1798) -- (2.5500, 1.4500, 0.1743) -- (2.5500, 1.5000, 0.1745) -- (2.5000, 1.5000, 0.1800) -- cycle;
\fill[blue!15.0, opacity=0.7] (2.5000, 1.5000, 0.1800) -- (2.5500, 1.5000, 0.1745) -- (2.5500, 1.5500, 0.1743) -- (2.5000, 1.5500, 0.1798) -- cycle;
\fill[blue!15.0, opacity=0.7] (2.5000, 1.5500, 0.1798) -- (2.5500, 1.5500, 0.1743) -- (2.5500, 1.6000, 0.1738) -- (2.5000, 1.6000, 0.1793) -- cycle;
\fill[blue!15.0, opacity=0.7] (2.5000, 1.6000, 0.1793) -- (2.5500, 1.6000, 0.1738) -- (2.5500, 1.6500, 0.1730) -- (2.5000, 1.6500, 0.1785) -- cycle;
\fill[blue!15.0, opacity=0.7] (2.5000, 1.6500, 0.1785) -- (2.5500, 1.6500, 0.1730) -- (2.5500, 1.7000, 0.1719) -- (2.5000, 1.7000, 0.1774) -- cycle;
\fill[blue!15.0, opacity=0.7] (2.5000, 1.7000, 0.1774) -- (2.5500, 1.7000, 0.1719) -- (2.5500, 1.7500, 0.1704) -- (2.5000, 1.7500, 0.1759) -- cycle;
\fill[blue!15.0, opacity=0.7] (2.5000, 1.7500, 0.1759) -- (2.5500, 1.7500, 0.1704) -- (2.5500, 1.8000, 0.1686) -- (2.5000, 1.8000, 0.1741) -- cycle;
\fill[blue!15.0, opacity=0.7] (2.5000, 1.8000, 0.1741) -- (2.5500, 1.8000, 0.1686) -- (2.5500, 1.8500, 0.1665) -- (2.5000, 1.8500, 0.1720) -- cycle;
\fill[blue!15.0, opacity=0.7] (2.5000, 1.8500, 0.1720) -- (2.5500, 1.8500, 0.1665) -- (2.5500, 1.9000, 0.1641) -- (2.5000, 1.9000, 0.1696) -- cycle;
\fill[blue!15.0, opacity=0.7] (2.5000, 1.9000, 0.1696) -- (2.5500, 1.9000, 0.1641) -- (2.5500, 1.9500, 0.1614) -- (2.5000, 1.9500, 0.1669) -- cycle;
\fill[blue!15.0, opacity=0.7] (2.5000, 1.9500, 0.1669) -- (2.5500, 1.9500, 0.1614) -- (2.5500, 2.0000, 0.1584) -- (2.5000, 2.0000, 0.1639) -- cycle;
\fill[blue!15.0, opacity=0.7] (2.5000, 2.0000, 0.1639) -- (2.5500, 2.0000, 0.1584) -- (2.5500, 2.0500, 0.1551) -- (2.5000, 2.0500, 0.1606) -- cycle;
\fill[blue!15.0, opacity=0.7] (2.5000, 2.0500, 0.1606) -- (2.5500, 2.0500, 0.1551) -- (2.5500, 2.1000, 0.1516) -- (2.5000, 2.1000, 0.1571) -- cycle;
\fill[blue!15.0, opacity=0.7] (2.5000, 2.1000, 0.1571) -- (2.5500, 2.1000, 0.1516) -- (2.5500, 2.1500, 0.1477) -- (2.5000, 2.1500, 0.1533) -- cycle;
\fill[blue!15.0, opacity=0.7] (2.5000, 2.1500, 0.1533) -- (2.5500, 2.1500, 0.1477) -- (2.5500, 2.2000, 0.1437) -- (2.5000, 2.2000, 0.1492) -- cycle;
\fill[blue!15.0, opacity=0.7] (2.5000, 2.2000, 0.1492) -- (2.5500, 2.2000, 0.1437) -- (2.5500, 2.2500, 0.1393) -- (2.5000, 2.2500, 0.1449) -- cycle;
\fill[blue!15.0, opacity=0.7] (2.5000, 2.2500, 0.1449) -- (2.5500, 2.2500, 0.1393) -- (2.5500, 2.3000, 0.1348) -- (2.5000, 2.3000, 0.1403) -- cycle;
\fill[blue!15.0, opacity=0.7] (2.5000, 2.3000, 0.1403) -- (2.5500, 2.3000, 0.1348) -- (2.5500, 2.3500, 0.1300) -- (2.5000, 2.3500, 0.1355) -- cycle;
\fill[blue!15.0, opacity=0.7] (2.5000, 2.3500, 0.1355) -- (2.5500, 2.3500, 0.1300) -- (2.5500, 2.4000, 0.1250) -- (2.5000, 2.4000, 0.1305) -- cycle;
\fill[blue!15.0, opacity=0.7] (2.5000, 2.4000, 0.1305) -- (2.5500, 2.4000, 0.1250) -- (2.5500, 2.4500, 0.1198) -- (2.5000, 2.4500, 0.1254) -- cycle;
\fill[blue!15.0, opacity=0.7] (2.5000, 2.4500, 0.1254) -- (2.5500, 2.4500, 0.1198) -- (2.5500, 2.5000, 0.1145) -- (2.5000, 2.5000, 0.1200) -- cycle;
\fill[blue!15.0, opacity=0.7] (2.5000, 2.5000, 0.1200) -- (2.5500, 2.5000, 0.1145) -- (2.5500, 2.5500, 0.1090) -- (2.5000, 2.5500, 0.1145) -- cycle;
\fill[blue!15.0, opacity=0.7] (2.5000, 2.5500, 0.1145) -- (2.5500, 2.5500, 0.1090) -- (2.5500, 2.6000, 0.1033) -- (2.5000, 2.6000, 0.1088) -- cycle;
\fill[blue!15.0, opacity=0.7] (2.5000, 2.6000, 0.1088) -- (2.5500, 2.6000, 0.1033) -- (2.5500, 2.6500, 0.0975) -- (2.5000, 2.6500, 0.1030) -- cycle;
\fill[blue!15.0, opacity=0.7] (2.5000, 2.6500, 0.1030) -- (2.5500, 2.6500, 0.0975) -- (2.5500, 2.7000, 0.0916) -- (2.5000, 2.7000, 0.0971) -- cycle;
\fill[blue!15.0, opacity=0.7] (2.5000, 2.7000, 0.0971) -- (2.5500, 2.7000, 0.0916) -- (2.5500, 2.7500, 0.0855) -- (2.5000, 2.7500, 0.0911) -- cycle;
\fill[blue!15.0, opacity=0.7] (2.5000, 2.7500, 0.0911) -- (2.5500, 2.7500, 0.0855) -- (2.5500, 2.8000, 0.0794) -- (2.5000, 2.8000, 0.0849) -- cycle;
\fill[blue!15.0, opacity=0.7] (2.5000, 2.8000, 0.0849) -- (2.5500, 2.8000, 0.0794) -- (2.5500, 2.8500, 0.0733) -- (2.5000, 2.8500, 0.0788) -- cycle;
\fill[blue!15.0, opacity=0.7] (2.5000, 2.8500, 0.0788) -- (2.5500, 2.8500, 0.0733) -- (2.5500, 2.9000, 0.0670) -- (2.5000, 2.9000, 0.0725) -- cycle;
\fill[blue!15.0, opacity=0.7] (2.5000, 2.9000, 0.0725) -- (2.5500, 2.9000, 0.0670) -- (2.5500, 2.9500, 0.0608) -- (2.5000, 2.9500, 0.0663) -- cycle;
\fill[blue!15.0, opacity=0.7] (2.5000, 2.9500, 0.0663) -- (2.5500, 2.9500, 0.0608) -- (2.5500, 3.0000, 0.0545) -- (2.5000, 3.0000, 0.0600) -- cycle;
\fill[blue!15.0, opacity=0.7] (2.5500, 0.0000, 0.0545) -- (2.6000, 0.0000, 0.0488) -- (2.6000, 0.0500, 0.0551) -- (2.5500, 0.0500, 0.0608) -- cycle;
\fill[blue!15.0, opacity=0.7] (2.5500, 0.0500, 0.0608) -- (2.6000, 0.0500, 0.0551) -- (2.6000, 0.1000, 0.0614) -- (2.5500, 0.1000, 0.0670) -- cycle;
\fill[blue!15.0, opacity=0.7] (2.5500, 0.1000, 0.0670) -- (2.6000, 0.1000, 0.0614) -- (2.6000, 0.1500, 0.0676) -- (2.5500, 0.1500, 0.0733) -- cycle;
\fill[blue!15.0, opacity=0.7] (2.5500, 0.1500, 0.0733) -- (2.6000, 0.1500, 0.0676) -- (2.6000, 0.2000, 0.0738) -- (2.5500, 0.2000, 0.0794) -- cycle;
\fill[blue!15.0, opacity=0.7] (2.5500, 0.2000, 0.0794) -- (2.6000, 0.2000, 0.0738) -- (2.6000, 0.2500, 0.0799) -- (2.5500, 0.2500, 0.0855) -- cycle;
\fill[blue!15.0, opacity=0.7] (2.5500, 0.2500, 0.0855) -- (2.6000, 0.2500, 0.0799) -- (2.6000, 0.3000, 0.0859) -- (2.5500, 0.3000, 0.0916) -- cycle;
\fill[blue!15.0, opacity=0.7] (2.5500, 0.3000, 0.0916) -- (2.6000, 0.3000, 0.0859) -- (2.6000, 0.3500, 0.0918) -- (2.5500, 0.3500, 0.0975) -- cycle;
\fill[blue!15.0, opacity=0.7] (2.5500, 0.3500, 0.0975) -- (2.6000, 0.3500, 0.0918) -- (2.6000, 0.4000, 0.0976) -- (2.5500, 0.4000, 0.1033) -- cycle;
\fill[blue!15.0, opacity=0.7] (2.5500, 0.4000, 0.1033) -- (2.6000, 0.4000, 0.0976) -- (2.6000, 0.4500, 0.1033) -- (2.5500, 0.4500, 0.1090) -- cycle;
\fill[blue!15.0, opacity=0.7] (2.5500, 0.4500, 0.1090) -- (2.6000, 0.4500, 0.1033) -- (2.6000, 0.5000, 0.1088) -- (2.5500, 0.5000, 0.1145) -- cycle;
\fill[blue!15.0, opacity=0.7] (2.5500, 0.5000, 0.1145) -- (2.6000, 0.5000, 0.1088) -- (2.6000, 0.5500, 0.1142) -- (2.5500, 0.5500, 0.1198) -- cycle;
\fill[blue!15.0, opacity=0.7] (2.5500, 0.5500, 0.1198) -- (2.6000, 0.5500, 0.1142) -- (2.6000, 0.6000, 0.1193) -- (2.5500, 0.6000, 0.1250) -- cycle;
\fill[blue!15.0, opacity=0.7] (2.5500, 0.6000, 0.1250) -- (2.6000, 0.6000, 0.1193) -- (2.6000, 0.6500, 0.1243) -- (2.5500, 0.6500, 0.1300) -- cycle;
\fill[blue!15.0, opacity=0.7] (2.5500, 0.6500, 0.1300) -- (2.6000, 0.6500, 0.1243) -- (2.6000, 0.7000, 0.1291) -- (2.5500, 0.7000, 0.1348) -- cycle;
\fill[blue!15.0, opacity=0.7] (2.5500, 0.7000, 0.1348) -- (2.6000, 0.7000, 0.1291) -- (2.6000, 0.7500, 0.1337) -- (2.5500, 0.7500, 0.1393) -- cycle;
\fill[blue!15.0, opacity=0.7] (2.5500, 0.7500, 0.1393) -- (2.6000, 0.7500, 0.1337) -- (2.6000, 0.8000, 0.1380) -- (2.5500, 0.8000, 0.1437) -- cycle;
\fill[blue!15.0, opacity=0.7] (2.5500, 0.8000, 0.1437) -- (2.6000, 0.8000, 0.1380) -- (2.6000, 0.8500, 0.1421) -- (2.5500, 0.8500, 0.1477) -- cycle;
\fill[blue!15.0, opacity=0.7] (2.5500, 0.8500, 0.1477) -- (2.6000, 0.8500, 0.1421) -- (2.6000, 0.9000, 0.1459) -- (2.5500, 0.9000, 0.1516) -- cycle;
\fill[blue!15.0, opacity=0.7] (2.5500, 0.9000, 0.1516) -- (2.6000, 0.9000, 0.1459) -- (2.6000, 0.9500, 0.1494) -- (2.5500, 0.9500, 0.1551) -- cycle;
\fill[blue!15.0, opacity=0.7] (2.5500, 0.9500, 0.1551) -- (2.6000, 0.9500, 0.1494) -- (2.6000, 1.0000, 0.1527) -- (2.5500, 1.0000, 0.1584) -- cycle;
\fill[blue!15.0, opacity=0.7] (2.5500, 1.0000, 0.1584) -- (2.6000, 1.0000, 0.1527) -- (2.6000, 1.0500, 0.1557) -- (2.5500, 1.0500, 0.1614) -- cycle;
\fill[blue!15.0, opacity=0.7] (2.5500, 1.0500, 0.1614) -- (2.6000, 1.0500, 0.1557) -- (2.6000, 1.1000, 0.1584) -- (2.5500, 1.1000, 0.1641) -- cycle;
\fill[blue!15.0, opacity=0.7] (2.5500, 1.1000, 0.1641) -- (2.6000, 1.1000, 0.1584) -- (2.6000, 1.1500, 0.1608) -- (2.5500, 1.1500, 0.1665) -- cycle;
\fill[blue!15.0, opacity=0.7] (2.5500, 1.1500, 0.1665) -- (2.6000, 1.1500, 0.1608) -- (2.6000, 1.2000, 0.1629) -- (2.5500, 1.2000, 0.1686) -- cycle;
\fill[blue!15.0, opacity=0.7] (2.5500, 1.2000, 0.1686) -- (2.6000, 1.2000, 0.1629) -- (2.6000, 1.2500, 0.1647) -- (2.5500, 1.2500, 0.1704) -- cycle;
\fill[blue!15.0, opacity=0.7] (2.5500, 1.2500, 0.1704) -- (2.6000, 1.2500, 0.1647) -- (2.6000, 1.3000, 0.1662) -- (2.5500, 1.3000, 0.1719) -- cycle;
\fill[blue!15.0, opacity=0.7] (2.5500, 1.3000, 0.1719) -- (2.6000, 1.3000, 0.1662) -- (2.6000, 1.3500, 0.1673) -- (2.5500, 1.3500, 0.1730) -- cycle;
\fill[blue!15.0, opacity=0.7] (2.5500, 1.3500, 0.1730) -- (2.6000, 1.3500, 0.1673) -- (2.6000, 1.4000, 0.1682) -- (2.5500, 1.4000, 0.1738) -- cycle;
\fill[blue!15.0, opacity=0.7] (2.5500, 1.4000, 0.1738) -- (2.6000, 1.4000, 0.1682) -- (2.6000, 1.4500, 0.1686) -- (2.5500, 1.4500, 0.1743) -- cycle;
\fill[blue!15.0, opacity=0.7] (2.5500, 1.4500, 0.1743) -- (2.6000, 1.4500, 0.1686) -- (2.6000, 1.5000, 0.1688) -- (2.5500, 1.5000, 0.1745) -- cycle;
\fill[blue!15.0, opacity=0.7] (2.5500, 1.5000, 0.1745) -- (2.6000, 1.5000, 0.1688) -- (2.6000, 1.5500, 0.1686) -- (2.5500, 1.5500, 0.1743) -- cycle;
\fill[blue!15.0, opacity=0.7] (2.5500, 1.5500, 0.1743) -- (2.6000, 1.5500, 0.1686) -- (2.6000, 1.6000, 0.1682) -- (2.5500, 1.6000, 0.1738) -- cycle;
\fill[blue!15.0, opacity=0.7] (2.5500, 1.6000, 0.1738) -- (2.6000, 1.6000, 0.1682) -- (2.6000, 1.6500, 0.1673) -- (2.5500, 1.6500, 0.1730) -- cycle;
\fill[blue!15.0, opacity=0.7] (2.5500, 1.6500, 0.1730) -- (2.6000, 1.6500, 0.1673) -- (2.6000, 1.7000, 0.1662) -- (2.5500, 1.7000, 0.1719) -- cycle;
\fill[blue!15.0, opacity=0.7] (2.5500, 1.7000, 0.1719) -- (2.6000, 1.7000, 0.1662) -- (2.6000, 1.7500, 0.1647) -- (2.5500, 1.7500, 0.1704) -- cycle;
\fill[blue!15.0, opacity=0.7] (2.5500, 1.7500, 0.1704) -- (2.6000, 1.7500, 0.1647) -- (2.6000, 1.8000, 0.1629) -- (2.5500, 1.8000, 0.1686) -- cycle;
\fill[blue!15.0, opacity=0.7] (2.5500, 1.8000, 0.1686) -- (2.6000, 1.8000, 0.1629) -- (2.6000, 1.8500, 0.1608) -- (2.5500, 1.8500, 0.1665) -- cycle;
\fill[blue!15.0, opacity=0.7] (2.5500, 1.8500, 0.1665) -- (2.6000, 1.8500, 0.1608) -- (2.6000, 1.9000, 0.1584) -- (2.5500, 1.9000, 0.1641) -- cycle;
\fill[blue!15.0, opacity=0.7] (2.5500, 1.9000, 0.1641) -- (2.6000, 1.9000, 0.1584) -- (2.6000, 1.9500, 0.1557) -- (2.5500, 1.9500, 0.1614) -- cycle;
\fill[blue!15.0, opacity=0.7] (2.5500, 1.9500, 0.1614) -- (2.6000, 1.9500, 0.1557) -- (2.6000, 2.0000, 0.1527) -- (2.5500, 2.0000, 0.1584) -- cycle;
\fill[blue!15.0, opacity=0.7] (2.5500, 2.0000, 0.1584) -- (2.6000, 2.0000, 0.1527) -- (2.6000, 2.0500, 0.1494) -- (2.5500, 2.0500, 0.1551) -- cycle;
\fill[blue!15.0, opacity=0.7] (2.5500, 2.0500, 0.1551) -- (2.6000, 2.0500, 0.1494) -- (2.6000, 2.1000, 0.1459) -- (2.5500, 2.1000, 0.1516) -- cycle;
\fill[blue!15.0, opacity=0.7] (2.5500, 2.1000, 0.1516) -- (2.6000, 2.1000, 0.1459) -- (2.6000, 2.1500, 0.1421) -- (2.5500, 2.1500, 0.1477) -- cycle;
\fill[blue!15.0, opacity=0.7] (2.5500, 2.1500, 0.1477) -- (2.6000, 2.1500, 0.1421) -- (2.6000, 2.2000, 0.1380) -- (2.5500, 2.2000, 0.1437) -- cycle;
\fill[blue!15.0, opacity=0.7] (2.5500, 2.2000, 0.1437) -- (2.6000, 2.2000, 0.1380) -- (2.6000, 2.2500, 0.1337) -- (2.5500, 2.2500, 0.1393) -- cycle;
\fill[blue!15.0, opacity=0.7] (2.5500, 2.2500, 0.1393) -- (2.6000, 2.2500, 0.1337) -- (2.6000, 2.3000, 0.1291) -- (2.5500, 2.3000, 0.1348) -- cycle;
\fill[blue!15.0, opacity=0.7] (2.5500, 2.3000, 0.1348) -- (2.6000, 2.3000, 0.1291) -- (2.6000, 2.3500, 0.1243) -- (2.5500, 2.3500, 0.1300) -- cycle;
\fill[blue!15.0, opacity=0.7] (2.5500, 2.3500, 0.1300) -- (2.6000, 2.3500, 0.1243) -- (2.6000, 2.4000, 0.1193) -- (2.5500, 2.4000, 0.1250) -- cycle;
\fill[blue!15.0, opacity=0.7] (2.5500, 2.4000, 0.1250) -- (2.6000, 2.4000, 0.1193) -- (2.6000, 2.4500, 0.1142) -- (2.5500, 2.4500, 0.1198) -- cycle;
\fill[blue!15.0, opacity=0.7] (2.5500, 2.4500, 0.1198) -- (2.6000, 2.4500, 0.1142) -- (2.6000, 2.5000, 0.1088) -- (2.5500, 2.5000, 0.1145) -- cycle;
\fill[blue!15.0, opacity=0.7] (2.5500, 2.5000, 0.1145) -- (2.6000, 2.5000, 0.1088) -- (2.6000, 2.5500, 0.1033) -- (2.5500, 2.5500, 0.1090) -- cycle;
\fill[blue!15.0, opacity=0.7] (2.5500, 2.5500, 0.1090) -- (2.6000, 2.5500, 0.1033) -- (2.6000, 2.6000, 0.0976) -- (2.5500, 2.6000, 0.1033) -- cycle;
\fill[blue!15.0, opacity=0.7] (2.5500, 2.6000, 0.1033) -- (2.6000, 2.6000, 0.0976) -- (2.6000, 2.6500, 0.0918) -- (2.5500, 2.6500, 0.0975) -- cycle;
\fill[blue!15.0, opacity=0.7] (2.5500, 2.6500, 0.0975) -- (2.6000, 2.6500, 0.0918) -- (2.6000, 2.7000, 0.0859) -- (2.5500, 2.7000, 0.0916) -- cycle;
\fill[blue!15.0, opacity=0.7] (2.5500, 2.7000, 0.0916) -- (2.6000, 2.7000, 0.0859) -- (2.6000, 2.7500, 0.0799) -- (2.5500, 2.7500, 0.0855) -- cycle;
\fill[blue!15.0, opacity=0.7] (2.5500, 2.7500, 0.0855) -- (2.6000, 2.7500, 0.0799) -- (2.6000, 2.8000, 0.0738) -- (2.5500, 2.8000, 0.0794) -- cycle;
\fill[blue!15.0, opacity=0.7] (2.5500, 2.8000, 0.0794) -- (2.6000, 2.8000, 0.0738) -- (2.6000, 2.8500, 0.0676) -- (2.5500, 2.8500, 0.0733) -- cycle;
\fill[blue!15.0, opacity=0.7] (2.5500, 2.8500, 0.0733) -- (2.6000, 2.8500, 0.0676) -- (2.6000, 2.9000, 0.0614) -- (2.5500, 2.9000, 0.0670) -- cycle;
\fill[blue!15.0, opacity=0.7] (2.5500, 2.9000, 0.0670) -- (2.6000, 2.9000, 0.0614) -- (2.6000, 2.9500, 0.0551) -- (2.5500, 2.9500, 0.0608) -- cycle;
\fill[blue!15.0, opacity=0.7] (2.5500, 2.9500, 0.0608) -- (2.6000, 2.9500, 0.0551) -- (2.6000, 3.0000, 0.0488) -- (2.5500, 3.0000, 0.0545) -- cycle;
\fill[blue!15.0, opacity=0.7] (2.6000, 0.0000, 0.0488) -- (2.6500, 0.0000, 0.0430) -- (2.6500, 0.0500, 0.0493) -- (2.6000, 0.0500, 0.0551) -- cycle;
\fill[blue!15.0, opacity=0.7] (2.6000, 0.0500, 0.0551) -- (2.6500, 0.0500, 0.0493) -- (2.6500, 0.1000, 0.0555) -- (2.6000, 0.1000, 0.0614) -- cycle;
\fill[blue!15.0, opacity=0.7] (2.6000, 0.1000, 0.0614) -- (2.6500, 0.1000, 0.0555) -- (2.6500, 0.1500, 0.0618) -- (2.6000, 0.1500, 0.0676) -- cycle;
\fill[blue!15.0, opacity=0.7] (2.6000, 0.1500, 0.0676) -- (2.6500, 0.1500, 0.0618) -- (2.6500, 0.2000, 0.0680) -- (2.6000, 0.2000, 0.0738) -- cycle;
\fill[blue!15.0, opacity=0.7] (2.6000, 0.2000, 0.0738) -- (2.6500, 0.2000, 0.0680) -- (2.6500, 0.2500, 0.0741) -- (2.6000, 0.2500, 0.0799) -- cycle;
\fill[blue!15.0, opacity=0.7] (2.6000, 0.2500, 0.0799) -- (2.6500, 0.2500, 0.0741) -- (2.6500, 0.3000, 0.0801) -- (2.6000, 0.3000, 0.0859) -- cycle;
\fill[blue!15.0, opacity=0.7] (2.6000, 0.3000, 0.0859) -- (2.6500, 0.3000, 0.0801) -- (2.6500, 0.3500, 0.0860) -- (2.6000, 0.3500, 0.0918) -- cycle;
\fill[blue!15.0, opacity=0.7] (2.6000, 0.3500, 0.0918) -- (2.6500, 0.3500, 0.0860) -- (2.6500, 0.4000, 0.0918) -- (2.6000, 0.4000, 0.0976) -- cycle;
\fill[blue!15.0, opacity=0.7] (2.6000, 0.4000, 0.0976) -- (2.6500, 0.4000, 0.0918) -- (2.6500, 0.4500, 0.0975) -- (2.6000, 0.4500, 0.1033) -- cycle;
\fill[blue!15.0, opacity=0.7] (2.6000, 0.4500, 0.1033) -- (2.6500, 0.4500, 0.0975) -- (2.6500, 0.5000, 0.1030) -- (2.6000, 0.5000, 0.1088) -- cycle;
\fill[blue!15.0, opacity=0.7] (2.6000, 0.5000, 0.1088) -- (2.6500, 0.5000, 0.1030) -- (2.6500, 0.5500, 0.1084) -- (2.6000, 0.5500, 0.1142) -- cycle;
\fill[blue!15.0, opacity=0.7] (2.6000, 0.5500, 0.1142) -- (2.6500, 0.5500, 0.1084) -- (2.6500, 0.6000, 0.1135) -- (2.6000, 0.6000, 0.1193) -- cycle;
\fill[blue!15.0, opacity=0.7] (2.6000, 0.6000, 0.1193) -- (2.6500, 0.6000, 0.1135) -- (2.6500, 0.6500, 0.1185) -- (2.6000, 0.6500, 0.1243) -- cycle;
\fill[blue!15.0, opacity=0.7] (2.6000, 0.6500, 0.1243) -- (2.6500, 0.6500, 0.1185) -- (2.6500, 0.7000, 0.1233) -- (2.6000, 0.7000, 0.1291) -- cycle;
\fill[blue!15.0, opacity=0.7] (2.6000, 0.7000, 0.1291) -- (2.6500, 0.7000, 0.1233) -- (2.6500, 0.7500, 0.1279) -- (2.6000, 0.7500, 0.1337) -- cycle;
\fill[blue!15.0, opacity=0.7] (2.6000, 0.7500, 0.1337) -- (2.6500, 0.7500, 0.1279) -- (2.6500, 0.8000, 0.1322) -- (2.6000, 0.8000, 0.1380) -- cycle;
\fill[blue!15.0, opacity=0.7] (2.6000, 0.8000, 0.1380) -- (2.6500, 0.8000, 0.1322) -- (2.6500, 0.8500, 0.1363) -- (2.6000, 0.8500, 0.1421) -- cycle;
\fill[blue!15.0, opacity=0.7] (2.6000, 0.8500, 0.1421) -- (2.6500, 0.8500, 0.1363) -- (2.6500, 0.9000, 0.1401) -- (2.6000, 0.9000, 0.1459) -- cycle;
\fill[blue!15.0, opacity=0.7] (2.6000, 0.9000, 0.1459) -- (2.6500, 0.9000, 0.1401) -- (2.6500, 0.9500, 0.1436) -- (2.6000, 0.9500, 0.1494) -- cycle;
\fill[blue!15.0, opacity=0.7] (2.6000, 0.9500, 0.1494) -- (2.6500, 0.9500, 0.1436) -- (2.6500, 1.0000, 0.1469) -- (2.6000, 1.0000, 0.1527) -- cycle;
\fill[blue!15.0, opacity=0.7] (2.6000, 1.0000, 0.1527) -- (2.6500, 1.0000, 0.1469) -- (2.6500, 1.0500, 0.1499) -- (2.6000, 1.0500, 0.1557) -- cycle;
\fill[blue!15.0, opacity=0.7] (2.6000, 1.0500, 0.1557) -- (2.6500, 1.0500, 0.1499) -- (2.6500, 1.1000, 0.1526) -- (2.6000, 1.1000, 0.1584) -- cycle;
\fill[blue!15.0, opacity=0.7] (2.6000, 1.1000, 0.1584) -- (2.6500, 1.1000, 0.1526) -- (2.6500, 1.1500, 0.1550) -- (2.6000, 1.1500, 0.1608) -- cycle;
\fill[blue!15.0, opacity=0.7] (2.6000, 1.1500, 0.1608) -- (2.6500, 1.1500, 0.1550) -- (2.6500, 1.2000, 0.1571) -- (2.6000, 1.2000, 0.1629) -- cycle;
\fill[blue!15.0, opacity=0.7] (2.6000, 1.2000, 0.1629) -- (2.6500, 1.2000, 0.1571) -- (2.6500, 1.2500, 0.1589) -- (2.6000, 1.2500, 0.1647) -- cycle;
\fill[blue!15.0, opacity=0.7] (2.6000, 1.2500, 0.1647) -- (2.6500, 1.2500, 0.1589) -- (2.6500, 1.3000, 0.1604) -- (2.6000, 1.3000, 0.1662) -- cycle;
\fill[blue!15.0, opacity=0.7] (2.6000, 1.3000, 0.1662) -- (2.6500, 1.3000, 0.1604) -- (2.6500, 1.3500, 0.1615) -- (2.6000, 1.3500, 0.1673) -- cycle;
\fill[blue!15.0, opacity=0.7] (2.6000, 1.3500, 0.1673) -- (2.6500, 1.3500, 0.1615) -- (2.6500, 1.4000, 0.1623) -- (2.6000, 1.4000, 0.1682) -- cycle;
\fill[blue!15.0, opacity=0.7] (2.6000, 1.4000, 0.1682) -- (2.6500, 1.4000, 0.1623) -- (2.6500, 1.4500, 0.1628) -- (2.6000, 1.4500, 0.1686) -- cycle;
\fill[blue!15.0, opacity=0.7] (2.6000, 1.4500, 0.1686) -- (2.6500, 1.4500, 0.1628) -- (2.6500, 1.5000, 0.1630) -- (2.6000, 1.5000, 0.1688) -- cycle;
\fill[blue!15.0, opacity=0.7] (2.6000, 1.5000, 0.1688) -- (2.6500, 1.5000, 0.1630) -- (2.6500, 1.5500, 0.1628) -- (2.6000, 1.5500, 0.1686) -- cycle;
\fill[blue!15.0, opacity=0.7] (2.6000, 1.5500, 0.1686) -- (2.6500, 1.5500, 0.1628) -- (2.6500, 1.6000, 0.1623) -- (2.6000, 1.6000, 0.1682) -- cycle;
\fill[blue!15.0, opacity=0.7] (2.6000, 1.6000, 0.1682) -- (2.6500, 1.6000, 0.1623) -- (2.6500, 1.6500, 0.1615) -- (2.6000, 1.6500, 0.1673) -- cycle;
\fill[blue!15.0, opacity=0.7] (2.6000, 1.6500, 0.1673) -- (2.6500, 1.6500, 0.1615) -- (2.6500, 1.7000, 0.1604) -- (2.6000, 1.7000, 0.1662) -- cycle;
\fill[blue!15.0, opacity=0.7] (2.6000, 1.7000, 0.1662) -- (2.6500, 1.7000, 0.1604) -- (2.6500, 1.7500, 0.1589) -- (2.6000, 1.7500, 0.1647) -- cycle;
\fill[blue!15.0, opacity=0.7] (2.6000, 1.7500, 0.1647) -- (2.6500, 1.7500, 0.1589) -- (2.6500, 1.8000, 0.1571) -- (2.6000, 1.8000, 0.1629) -- cycle;
\fill[blue!15.0, opacity=0.7] (2.6000, 1.8000, 0.1629) -- (2.6500, 1.8000, 0.1571) -- (2.6500, 1.8500, 0.1550) -- (2.6000, 1.8500, 0.1608) -- cycle;
\fill[blue!15.0, opacity=0.7] (2.6000, 1.8500, 0.1608) -- (2.6500, 1.8500, 0.1550) -- (2.6500, 1.9000, 0.1526) -- (2.6000, 1.9000, 0.1584) -- cycle;
\fill[blue!15.0, opacity=0.7] (2.6000, 1.9000, 0.1584) -- (2.6500, 1.9000, 0.1526) -- (2.6500, 1.9500, 0.1499) -- (2.6000, 1.9500, 0.1557) -- cycle;
\fill[blue!15.0, opacity=0.7] (2.6000, 1.9500, 0.1557) -- (2.6500, 1.9500, 0.1499) -- (2.6500, 2.0000, 0.1469) -- (2.6000, 2.0000, 0.1527) -- cycle;
\fill[blue!15.0, opacity=0.7] (2.6000, 2.0000, 0.1527) -- (2.6500, 2.0000, 0.1469) -- (2.6500, 2.0500, 0.1436) -- (2.6000, 2.0500, 0.1494) -- cycle;
\fill[blue!15.0, opacity=0.7] (2.6000, 2.0500, 0.1494) -- (2.6500, 2.0500, 0.1436) -- (2.6500, 2.1000, 0.1401) -- (2.6000, 2.1000, 0.1459) -- cycle;
\fill[blue!15.0, opacity=0.7] (2.6000, 2.1000, 0.1459) -- (2.6500, 2.1000, 0.1401) -- (2.6500, 2.1500, 0.1363) -- (2.6000, 2.1500, 0.1421) -- cycle;
\fill[blue!15.0, opacity=0.7] (2.6000, 2.1500, 0.1421) -- (2.6500, 2.1500, 0.1363) -- (2.6500, 2.2000, 0.1322) -- (2.6000, 2.2000, 0.1380) -- cycle;
\fill[blue!15.0, opacity=0.7] (2.6000, 2.2000, 0.1380) -- (2.6500, 2.2000, 0.1322) -- (2.6500, 2.2500, 0.1279) -- (2.6000, 2.2500, 0.1337) -- cycle;
\fill[blue!15.0, opacity=0.7] (2.6000, 2.2500, 0.1337) -- (2.6500, 2.2500, 0.1279) -- (2.6500, 2.3000, 0.1233) -- (2.6000, 2.3000, 0.1291) -- cycle;
\fill[blue!15.0, opacity=0.7] (2.6000, 2.3000, 0.1291) -- (2.6500, 2.3000, 0.1233) -- (2.6500, 2.3500, 0.1185) -- (2.6000, 2.3500, 0.1243) -- cycle;
\fill[blue!15.0, opacity=0.7] (2.6000, 2.3500, 0.1243) -- (2.6500, 2.3500, 0.1185) -- (2.6500, 2.4000, 0.1135) -- (2.6000, 2.4000, 0.1193) -- cycle;
\fill[blue!15.0, opacity=0.7] (2.6000, 2.4000, 0.1193) -- (2.6500, 2.4000, 0.1135) -- (2.6500, 2.4500, 0.1084) -- (2.6000, 2.4500, 0.1142) -- cycle;
\fill[blue!15.0, opacity=0.7] (2.6000, 2.4500, 0.1142) -- (2.6500, 2.4500, 0.1084) -- (2.6500, 2.5000, 0.1030) -- (2.6000, 2.5000, 0.1088) -- cycle;
\fill[blue!15.0, opacity=0.7] (2.6000, 2.5000, 0.1088) -- (2.6500, 2.5000, 0.1030) -- (2.6500, 2.5500, 0.0975) -- (2.6000, 2.5500, 0.1033) -- cycle;
\fill[blue!15.0, opacity=0.7] (2.6000, 2.5500, 0.1033) -- (2.6500, 2.5500, 0.0975) -- (2.6500, 2.6000, 0.0918) -- (2.6000, 2.6000, 0.0976) -- cycle;
\fill[blue!15.0, opacity=0.7] (2.6000, 2.6000, 0.0976) -- (2.6500, 2.6000, 0.0918) -- (2.6500, 2.6500, 0.0860) -- (2.6000, 2.6500, 0.0918) -- cycle;
\fill[blue!15.0, opacity=0.7] (2.6000, 2.6500, 0.0918) -- (2.6500, 2.6500, 0.0860) -- (2.6500, 2.7000, 0.0801) -- (2.6000, 2.7000, 0.0859) -- cycle;
\fill[blue!15.0, opacity=0.7] (2.6000, 2.7000, 0.0859) -- (2.6500, 2.7000, 0.0801) -- (2.6500, 2.7500, 0.0741) -- (2.6000, 2.7500, 0.0799) -- cycle;
\fill[blue!15.0, opacity=0.7] (2.6000, 2.7500, 0.0799) -- (2.6500, 2.7500, 0.0741) -- (2.6500, 2.8000, 0.0680) -- (2.6000, 2.8000, 0.0738) -- cycle;
\fill[blue!15.0, opacity=0.7] (2.6000, 2.8000, 0.0738) -- (2.6500, 2.8000, 0.0680) -- (2.6500, 2.8500, 0.0618) -- (2.6000, 2.8500, 0.0676) -- cycle;
\fill[blue!15.0, opacity=0.7] (2.6000, 2.8500, 0.0676) -- (2.6500, 2.8500, 0.0618) -- (2.6500, 2.9000, 0.0555) -- (2.6000, 2.9000, 0.0614) -- cycle;
\fill[blue!15.0, opacity=0.7] (2.6000, 2.9000, 0.0614) -- (2.6500, 2.9000, 0.0555) -- (2.6500, 2.9500, 0.0493) -- (2.6000, 2.9500, 0.0551) -- cycle;
\fill[blue!15.0, opacity=0.7] (2.6000, 2.9500, 0.0551) -- (2.6500, 2.9500, 0.0493) -- (2.6500, 3.0000, 0.0430) -- (2.6000, 3.0000, 0.0488) -- cycle;
\fill[blue!15.0, opacity=0.7] (2.6500, 0.0000, 0.0430) -- (2.7000, 0.0000, 0.0371) -- (2.7000, 0.0500, 0.0434) -- (2.6500, 0.0500, 0.0493) -- cycle;
\fill[blue!15.0, opacity=0.7] (2.6500, 0.0500, 0.0493) -- (2.7000, 0.0500, 0.0434) -- (2.7000, 0.1000, 0.0496) -- (2.6500, 0.1000, 0.0555) -- cycle;
\fill[blue!15.0, opacity=0.7] (2.6500, 0.1000, 0.0555) -- (2.7000, 0.1000, 0.0496) -- (2.7000, 0.1500, 0.0559) -- (2.6500, 0.1500, 0.0618) -- cycle;
\fill[blue!15.0, opacity=0.7] (2.6500, 0.1500, 0.0618) -- (2.7000, 0.1500, 0.0559) -- (2.7000, 0.2000, 0.0620) -- (2.6500, 0.2000, 0.0680) -- cycle;
\fill[blue!15.0, opacity=0.7] (2.6500, 0.2000, 0.0680) -- (2.7000, 0.2000, 0.0620) -- (2.7000, 0.2500, 0.0681) -- (2.6500, 0.2500, 0.0741) -- cycle;
\fill[blue!15.0, opacity=0.7] (2.6500, 0.2500, 0.0741) -- (2.7000, 0.2500, 0.0681) -- (2.7000, 0.3000, 0.0742) -- (2.6500, 0.3000, 0.0801) -- cycle;
\fill[blue!15.0, opacity=0.7] (2.6500, 0.3000, 0.0801) -- (2.7000, 0.3000, 0.0742) -- (2.7000, 0.3500, 0.0801) -- (2.6500, 0.3500, 0.0860) -- cycle;
\fill[blue!15.0, opacity=0.7] (2.6500, 0.3500, 0.0860) -- (2.7000, 0.3500, 0.0801) -- (2.7000, 0.4000, 0.0859) -- (2.6500, 0.4000, 0.0918) -- cycle;
\fill[blue!15.0, opacity=0.7] (2.6500, 0.4000, 0.0918) -- (2.7000, 0.4000, 0.0859) -- (2.7000, 0.4500, 0.0916) -- (2.6500, 0.4500, 0.0975) -- cycle;
\fill[blue!15.0, opacity=0.7] (2.6500, 0.4500, 0.0975) -- (2.7000, 0.4500, 0.0916) -- (2.7000, 0.5000, 0.0971) -- (2.6500, 0.5000, 0.1030) -- cycle;
\fill[blue!15.0, opacity=0.7] (2.6500, 0.5000, 0.1030) -- (2.7000, 0.5000, 0.0971) -- (2.7000, 0.5500, 0.1024) -- (2.6500, 0.5500, 0.1084) -- cycle;
\fill[blue!15.0, opacity=0.7] (2.6500, 0.5500, 0.1084) -- (2.7000, 0.5500, 0.1024) -- (2.7000, 0.6000, 0.1076) -- (2.6500, 0.6000, 0.1135) -- cycle;
\fill[blue!15.0, opacity=0.7] (2.6500, 0.6000, 0.1135) -- (2.7000, 0.6000, 0.1076) -- (2.7000, 0.6500, 0.1126) -- (2.6500, 0.6500, 0.1185) -- cycle;
\fill[blue!15.0, opacity=0.7] (2.6500, 0.6500, 0.1185) -- (2.7000, 0.6500, 0.1126) -- (2.7000, 0.7000, 0.1174) -- (2.6500, 0.7000, 0.1233) -- cycle;
\fill[blue!15.0, opacity=0.7] (2.6500, 0.7000, 0.1233) -- (2.7000, 0.7000, 0.1174) -- (2.7000, 0.7500, 0.1219) -- (2.6500, 0.7500, 0.1279) -- cycle;
\fill[blue!15.0, opacity=0.7] (2.6500, 0.7500, 0.1279) -- (2.7000, 0.7500, 0.1219) -- (2.7000, 0.8000, 0.1263) -- (2.6500, 0.8000, 0.1322) -- cycle;
\fill[blue!15.0, opacity=0.7] (2.6500, 0.8000, 0.1322) -- (2.7000, 0.8000, 0.1263) -- (2.7000, 0.8500, 0.1303) -- (2.6500, 0.8500, 0.1363) -- cycle;
\fill[blue!15.0, opacity=0.7] (2.6500, 0.8500, 0.1363) -- (2.7000, 0.8500, 0.1303) -- (2.7000, 0.9000, 0.1342) -- (2.6500, 0.9000, 0.1401) -- cycle;
\fill[blue!15.0, opacity=0.7] (2.6500, 0.9000, 0.1401) -- (2.7000, 0.9000, 0.1342) -- (2.7000, 0.9500, 0.1377) -- (2.6500, 0.9500, 0.1436) -- cycle;
\fill[blue!15.0, opacity=0.7] (2.6500, 0.9500, 0.1436) -- (2.7000, 0.9500, 0.1377) -- (2.7000, 1.0000, 0.1410) -- (2.6500, 1.0000, 0.1469) -- cycle;
\fill[blue!15.0, opacity=0.7] (2.6500, 1.0000, 0.1469) -- (2.7000, 1.0000, 0.1410) -- (2.7000, 1.0500, 0.1440) -- (2.6500, 1.0500, 0.1499) -- cycle;
\fill[blue!15.0, opacity=0.7] (2.6500, 1.0500, 0.1499) -- (2.7000, 1.0500, 0.1440) -- (2.7000, 1.1000, 0.1467) -- (2.6500, 1.1000, 0.1526) -- cycle;
\fill[blue!15.0, opacity=0.7] (2.6500, 1.1000, 0.1526) -- (2.7000, 1.1000, 0.1467) -- (2.7000, 1.1500, 0.1491) -- (2.6500, 1.1500, 0.1550) -- cycle;
\fill[blue!15.0, opacity=0.7] (2.6500, 1.1500, 0.1550) -- (2.7000, 1.1500, 0.1491) -- (2.7000, 1.2000, 0.1512) -- (2.6500, 1.2000, 0.1571) -- cycle;
\fill[blue!15.0, opacity=0.7] (2.6500, 1.2000, 0.1571) -- (2.7000, 1.2000, 0.1512) -- (2.7000, 1.2500, 0.1530) -- (2.6500, 1.2500, 0.1589) -- cycle;
\fill[blue!15.0, opacity=0.7] (2.6500, 1.2500, 0.1589) -- (2.7000, 1.2500, 0.1530) -- (2.7000, 1.3000, 0.1545) -- (2.6500, 1.3000, 0.1604) -- cycle;
\fill[blue!15.0, opacity=0.7] (2.6500, 1.3000, 0.1604) -- (2.7000, 1.3000, 0.1545) -- (2.7000, 1.3500, 0.1556) -- (2.6500, 1.3500, 0.1615) -- cycle;
\fill[blue!15.0, opacity=0.7] (2.6500, 1.3500, 0.1615) -- (2.7000, 1.3500, 0.1556) -- (2.7000, 1.4000, 0.1564) -- (2.6500, 1.4000, 0.1623) -- cycle;
\fill[blue!15.0, opacity=0.7] (2.6500, 1.4000, 0.1623) -- (2.7000, 1.4000, 0.1564) -- (2.7000, 1.4500, 0.1569) -- (2.6500, 1.4500, 0.1628) -- cycle;
\fill[blue!15.0, opacity=0.7] (2.6500, 1.4500, 0.1628) -- (2.7000, 1.4500, 0.1569) -- (2.7000, 1.5000, 0.1571) -- (2.6500, 1.5000, 0.1630) -- cycle;
\fill[blue!15.0, opacity=0.7] (2.6500, 1.5000, 0.1630) -- (2.7000, 1.5000, 0.1571) -- (2.7000, 1.5500, 0.1569) -- (2.6500, 1.5500, 0.1628) -- cycle;
\fill[blue!15.0, opacity=0.7] (2.6500, 1.5500, 0.1628) -- (2.7000, 1.5500, 0.1569) -- (2.7000, 1.6000, 0.1564) -- (2.6500, 1.6000, 0.1623) -- cycle;
\fill[blue!15.0, opacity=0.7] (2.6500, 1.6000, 0.1623) -- (2.7000, 1.6000, 0.1564) -- (2.7000, 1.6500, 0.1556) -- (2.6500, 1.6500, 0.1615) -- cycle;
\fill[blue!15.0, opacity=0.7] (2.6500, 1.6500, 0.1615) -- (2.7000, 1.6500, 0.1556) -- (2.7000, 1.7000, 0.1545) -- (2.6500, 1.7000, 0.1604) -- cycle;
\fill[blue!15.0, opacity=0.7] (2.6500, 1.7000, 0.1604) -- (2.7000, 1.7000, 0.1545) -- (2.7000, 1.7500, 0.1530) -- (2.6500, 1.7500, 0.1589) -- cycle;
\fill[blue!15.0, opacity=0.7] (2.6500, 1.7500, 0.1589) -- (2.7000, 1.7500, 0.1530) -- (2.7000, 1.8000, 0.1512) -- (2.6500, 1.8000, 0.1571) -- cycle;
\fill[blue!15.0, opacity=0.7] (2.6500, 1.8000, 0.1571) -- (2.7000, 1.8000, 0.1512) -- (2.7000, 1.8500, 0.1491) -- (2.6500, 1.8500, 0.1550) -- cycle;
\fill[blue!15.0, opacity=0.7] (2.6500, 1.8500, 0.1550) -- (2.7000, 1.8500, 0.1491) -- (2.7000, 1.9000, 0.1467) -- (2.6500, 1.9000, 0.1526) -- cycle;
\fill[blue!15.0, opacity=0.7] (2.6500, 1.9000, 0.1526) -- (2.7000, 1.9000, 0.1467) -- (2.7000, 1.9500, 0.1440) -- (2.6500, 1.9500, 0.1499) -- cycle;
\fill[blue!15.0, opacity=0.7] (2.6500, 1.9500, 0.1499) -- (2.7000, 1.9500, 0.1440) -- (2.7000, 2.0000, 0.1410) -- (2.6500, 2.0000, 0.1469) -- cycle;
\fill[blue!15.0, opacity=0.7] (2.6500, 2.0000, 0.1469) -- (2.7000, 2.0000, 0.1410) -- (2.7000, 2.0500, 0.1377) -- (2.6500, 2.0500, 0.1436) -- cycle;
\fill[blue!15.0, opacity=0.7] (2.6500, 2.0500, 0.1436) -- (2.7000, 2.0500, 0.1377) -- (2.7000, 2.1000, 0.1342) -- (2.6500, 2.1000, 0.1401) -- cycle;
\fill[blue!15.0, opacity=0.7] (2.6500, 2.1000, 0.1401) -- (2.7000, 2.1000, 0.1342) -- (2.7000, 2.1500, 0.1303) -- (2.6500, 2.1500, 0.1363) -- cycle;
\fill[blue!15.0, opacity=0.7] (2.6500, 2.1500, 0.1363) -- (2.7000, 2.1500, 0.1303) -- (2.7000, 2.2000, 0.1263) -- (2.6500, 2.2000, 0.1322) -- cycle;
\fill[blue!15.0, opacity=0.7] (2.6500, 2.2000, 0.1322) -- (2.7000, 2.2000, 0.1263) -- (2.7000, 2.2500, 0.1219) -- (2.6500, 2.2500, 0.1279) -- cycle;
\fill[blue!15.0, opacity=0.7] (2.6500, 2.2500, 0.1279) -- (2.7000, 2.2500, 0.1219) -- (2.7000, 2.3000, 0.1174) -- (2.6500, 2.3000, 0.1233) -- cycle;
\fill[blue!15.0, opacity=0.7] (2.6500, 2.3000, 0.1233) -- (2.7000, 2.3000, 0.1174) -- (2.7000, 2.3500, 0.1126) -- (2.6500, 2.3500, 0.1185) -- cycle;
\fill[blue!15.0, opacity=0.7] (2.6500, 2.3500, 0.1185) -- (2.7000, 2.3500, 0.1126) -- (2.7000, 2.4000, 0.1076) -- (2.6500, 2.4000, 0.1135) -- cycle;
\fill[blue!15.0, opacity=0.7] (2.6500, 2.4000, 0.1135) -- (2.7000, 2.4000, 0.1076) -- (2.7000, 2.4500, 0.1024) -- (2.6500, 2.4500, 0.1084) -- cycle;
\fill[blue!15.0, opacity=0.7] (2.6500, 2.4500, 0.1084) -- (2.7000, 2.4500, 0.1024) -- (2.7000, 2.5000, 0.0971) -- (2.6500, 2.5000, 0.1030) -- cycle;
\fill[blue!15.0, opacity=0.7] (2.6500, 2.5000, 0.1030) -- (2.7000, 2.5000, 0.0971) -- (2.7000, 2.5500, 0.0916) -- (2.6500, 2.5500, 0.0975) -- cycle;
\fill[blue!15.0, opacity=0.7] (2.6500, 2.5500, 0.0975) -- (2.7000, 2.5500, 0.0916) -- (2.7000, 2.6000, 0.0859) -- (2.6500, 2.6000, 0.0918) -- cycle;
\fill[blue!15.0, opacity=0.7] (2.6500, 2.6000, 0.0918) -- (2.7000, 2.6000, 0.0859) -- (2.7000, 2.6500, 0.0801) -- (2.6500, 2.6500, 0.0860) -- cycle;
\fill[blue!15.0, opacity=0.7] (2.6500, 2.6500, 0.0860) -- (2.7000, 2.6500, 0.0801) -- (2.7000, 2.7000, 0.0742) -- (2.6500, 2.7000, 0.0801) -- cycle;
\fill[blue!15.0, opacity=0.7] (2.6500, 2.7000, 0.0801) -- (2.7000, 2.7000, 0.0742) -- (2.7000, 2.7500, 0.0681) -- (2.6500, 2.7500, 0.0741) -- cycle;
\fill[blue!15.0, opacity=0.7] (2.6500, 2.7500, 0.0741) -- (2.7000, 2.7500, 0.0681) -- (2.7000, 2.8000, 0.0620) -- (2.6500, 2.8000, 0.0680) -- cycle;
\fill[blue!15.0, opacity=0.7] (2.6500, 2.8000, 0.0680) -- (2.7000, 2.8000, 0.0620) -- (2.7000, 2.8500, 0.0559) -- (2.6500, 2.8500, 0.0618) -- cycle;
\fill[blue!15.0, opacity=0.7] (2.6500, 2.8500, 0.0618) -- (2.7000, 2.8500, 0.0559) -- (2.7000, 2.9000, 0.0496) -- (2.6500, 2.9000, 0.0555) -- cycle;
\fill[blue!15.0, opacity=0.7] (2.6500, 2.9000, 0.0555) -- (2.7000, 2.9000, 0.0496) -- (2.7000, 2.9500, 0.0434) -- (2.6500, 2.9500, 0.0493) -- cycle;
\fill[blue!15.0, opacity=0.7] (2.6500, 2.9500, 0.0493) -- (2.7000, 2.9500, 0.0434) -- (2.7000, 3.0000, 0.0371) -- (2.6500, 3.0000, 0.0430) -- cycle;
\fill[blue!15.0, opacity=0.7] (2.7000, 0.0000, 0.0371) -- (2.7500, 0.0000, 0.0311) -- (2.7500, 0.0500, 0.0373) -- (2.7000, 0.0500, 0.0434) -- cycle;
\fill[blue!15.0, opacity=0.7] (2.7000, 0.0500, 0.0434) -- (2.7500, 0.0500, 0.0373) -- (2.7500, 0.1000, 0.0436) -- (2.7000, 0.1000, 0.0496) -- cycle;
\fill[blue!15.0, opacity=0.7] (2.7000, 0.1000, 0.0496) -- (2.7500, 0.1000, 0.0436) -- (2.7500, 0.1500, 0.0498) -- (2.7000, 0.1500, 0.0559) -- cycle;
\fill[blue!15.0, opacity=0.7] (2.7000, 0.1500, 0.0559) -- (2.7500, 0.1500, 0.0498) -- (2.7500, 0.2000, 0.0560) -- (2.7000, 0.2000, 0.0620) -- cycle;
\fill[blue!15.0, opacity=0.7] (2.7000, 0.2000, 0.0620) -- (2.7500, 0.2000, 0.0560) -- (2.7500, 0.2500, 0.0621) -- (2.7000, 0.2500, 0.0681) -- cycle;
\fill[blue!15.0, opacity=0.7] (2.7000, 0.2500, 0.0681) -- (2.7500, 0.2500, 0.0621) -- (2.7500, 0.3000, 0.0681) -- (2.7000, 0.3000, 0.0742) -- cycle;
\fill[blue!15.0, opacity=0.7] (2.7000, 0.3000, 0.0742) -- (2.7500, 0.3000, 0.0681) -- (2.7500, 0.3500, 0.0741) -- (2.7000, 0.3500, 0.0801) -- cycle;
\fill[blue!15.0, opacity=0.7] (2.7000, 0.3500, 0.0801) -- (2.7500, 0.3500, 0.0741) -- (2.7500, 0.4000, 0.0799) -- (2.7000, 0.4000, 0.0859) -- cycle;
\fill[blue!15.0, opacity=0.7] (2.7000, 0.4000, 0.0859) -- (2.7500, 0.4000, 0.0799) -- (2.7500, 0.4500, 0.0855) -- (2.7000, 0.4500, 0.0916) -- cycle;
\fill[blue!15.0, opacity=0.7] (2.7000, 0.4500, 0.0916) -- (2.7500, 0.4500, 0.0855) -- (2.7500, 0.5000, 0.0911) -- (2.7000, 0.5000, 0.0971) -- cycle;
\fill[blue!15.0, opacity=0.7] (2.7000, 0.5000, 0.0971) -- (2.7500, 0.5000, 0.0911) -- (2.7500, 0.5500, 0.0964) -- (2.7000, 0.5500, 0.1024) -- cycle;
\fill[blue!15.0, opacity=0.7] (2.7000, 0.5500, 0.1024) -- (2.7500, 0.5500, 0.0964) -- (2.7500, 0.6000, 0.1016) -- (2.7000, 0.6000, 0.1076) -- cycle;
\fill[blue!15.0, opacity=0.7] (2.7000, 0.6000, 0.1076) -- (2.7500, 0.6000, 0.1016) -- (2.7500, 0.6500, 0.1066) -- (2.7000, 0.6500, 0.1126) -- cycle;
\fill[blue!15.0, opacity=0.7] (2.7000, 0.6500, 0.1126) -- (2.7500, 0.6500, 0.1066) -- (2.7500, 0.7000, 0.1114) -- (2.7000, 0.7000, 0.1174) -- cycle;
\fill[blue!15.0, opacity=0.7] (2.7000, 0.7000, 0.1174) -- (2.7500, 0.7000, 0.1114) -- (2.7500, 0.7500, 0.1159) -- (2.7000, 0.7500, 0.1219) -- cycle;
\fill[blue!15.0, opacity=0.7] (2.7000, 0.7500, 0.1219) -- (2.7500, 0.7500, 0.1159) -- (2.7500, 0.8000, 0.1202) -- (2.7000, 0.8000, 0.1263) -- cycle;
\fill[blue!15.0, opacity=0.7] (2.7000, 0.8000, 0.1263) -- (2.7500, 0.8000, 0.1202) -- (2.7500, 0.8500, 0.1243) -- (2.7000, 0.8500, 0.1303) -- cycle;
\fill[blue!15.0, opacity=0.7] (2.7000, 0.8500, 0.1303) -- (2.7500, 0.8500, 0.1243) -- (2.7500, 0.9000, 0.1281) -- (2.7000, 0.9000, 0.1342) -- cycle;
\fill[blue!15.0, opacity=0.7] (2.7000, 0.9000, 0.1342) -- (2.7500, 0.9000, 0.1281) -- (2.7500, 0.9500, 0.1317) -- (2.7000, 0.9500, 0.1377) -- cycle;
\fill[blue!15.0, opacity=0.7] (2.7000, 0.9500, 0.1377) -- (2.7500, 0.9500, 0.1317) -- (2.7500, 1.0000, 0.1350) -- (2.7000, 1.0000, 0.1410) -- cycle;
\fill[blue!15.0, opacity=0.7] (2.7000, 1.0000, 0.1410) -- (2.7500, 1.0000, 0.1350) -- (2.7500, 1.0500, 0.1380) -- (2.7000, 1.0500, 0.1440) -- cycle;
\fill[blue!15.0, opacity=0.7] (2.7000, 1.0500, 0.1440) -- (2.7500, 1.0500, 0.1380) -- (2.7500, 1.1000, 0.1407) -- (2.7000, 1.1000, 0.1467) -- cycle;
\fill[blue!15.0, opacity=0.7] (2.7000, 1.1000, 0.1467) -- (2.7500, 1.1000, 0.1407) -- (2.7500, 1.1500, 0.1431) -- (2.7000, 1.1500, 0.1491) -- cycle;
\fill[blue!15.0, opacity=0.7] (2.7000, 1.1500, 0.1491) -- (2.7500, 1.1500, 0.1431) -- (2.7500, 1.2000, 0.1452) -- (2.7000, 1.2000, 0.1512) -- cycle;
\fill[blue!15.0, opacity=0.7] (2.7000, 1.2000, 0.1512) -- (2.7500, 1.2000, 0.1452) -- (2.7500, 1.2500, 0.1470) -- (2.7000, 1.2500, 0.1530) -- cycle;
\fill[blue!15.0, opacity=0.7] (2.7000, 1.2500, 0.1530) -- (2.7500, 1.2500, 0.1470) -- (2.7500, 1.3000, 0.1484) -- (2.7000, 1.3000, 0.1545) -- cycle;
\fill[blue!15.0, opacity=0.7] (2.7000, 1.3000, 0.1545) -- (2.7500, 1.3000, 0.1484) -- (2.7500, 1.3500, 0.1496) -- (2.7000, 1.3500, 0.1556) -- cycle;
\fill[blue!15.0, opacity=0.7] (2.7000, 1.3500, 0.1556) -- (2.7500, 1.3500, 0.1496) -- (2.7500, 1.4000, 0.1504) -- (2.7000, 1.4000, 0.1564) -- cycle;
\fill[blue!15.0, opacity=0.7] (2.7000, 1.4000, 0.1564) -- (2.7500, 1.4000, 0.1504) -- (2.7500, 1.4500, 0.1509) -- (2.7000, 1.4500, 0.1569) -- cycle;
\fill[blue!15.0, opacity=0.7] (2.7000, 1.4500, 0.1569) -- (2.7500, 1.4500, 0.1509) -- (2.7500, 1.5000, 0.1511) -- (2.7000, 1.5000, 0.1571) -- cycle;
\fill[blue!15.0, opacity=0.7] (2.7000, 1.5000, 0.1571) -- (2.7500, 1.5000, 0.1511) -- (2.7500, 1.5500, 0.1509) -- (2.7000, 1.5500, 0.1569) -- cycle;
\fill[blue!15.0, opacity=0.7] (2.7000, 1.5500, 0.1569) -- (2.7500, 1.5500, 0.1509) -- (2.7500, 1.6000, 0.1504) -- (2.7000, 1.6000, 0.1564) -- cycle;
\fill[blue!15.0, opacity=0.7] (2.7000, 1.6000, 0.1564) -- (2.7500, 1.6000, 0.1504) -- (2.7500, 1.6500, 0.1496) -- (2.7000, 1.6500, 0.1556) -- cycle;
\fill[blue!15.0, opacity=0.7] (2.7000, 1.6500, 0.1556) -- (2.7500, 1.6500, 0.1496) -- (2.7500, 1.7000, 0.1484) -- (2.7000, 1.7000, 0.1545) -- cycle;
\fill[blue!15.0, opacity=0.7] (2.7000, 1.7000, 0.1545) -- (2.7500, 1.7000, 0.1484) -- (2.7500, 1.7500, 0.1470) -- (2.7000, 1.7500, 0.1530) -- cycle;
\fill[blue!15.0, opacity=0.7] (2.7000, 1.7500, 0.1530) -- (2.7500, 1.7500, 0.1470) -- (2.7500, 1.8000, 0.1452) -- (2.7000, 1.8000, 0.1512) -- cycle;
\fill[blue!15.0, opacity=0.7] (2.7000, 1.8000, 0.1512) -- (2.7500, 1.8000, 0.1452) -- (2.7500, 1.8500, 0.1431) -- (2.7000, 1.8500, 0.1491) -- cycle;
\fill[blue!15.0, opacity=0.7] (2.7000, 1.8500, 0.1491) -- (2.7500, 1.8500, 0.1431) -- (2.7500, 1.9000, 0.1407) -- (2.7000, 1.9000, 0.1467) -- cycle;
\fill[blue!15.0, opacity=0.7] (2.7000, 1.9000, 0.1467) -- (2.7500, 1.9000, 0.1407) -- (2.7500, 1.9500, 0.1380) -- (2.7000, 1.9500, 0.1440) -- cycle;
\fill[blue!15.0, opacity=0.7] (2.7000, 1.9500, 0.1440) -- (2.7500, 1.9500, 0.1380) -- (2.7500, 2.0000, 0.1350) -- (2.7000, 2.0000, 0.1410) -- cycle;
\fill[blue!15.0, opacity=0.7] (2.7000, 2.0000, 0.1410) -- (2.7500, 2.0000, 0.1350) -- (2.7500, 2.0500, 0.1317) -- (2.7000, 2.0500, 0.1377) -- cycle;
\fill[blue!15.0, opacity=0.7] (2.7000, 2.0500, 0.1377) -- (2.7500, 2.0500, 0.1317) -- (2.7500, 2.1000, 0.1281) -- (2.7000, 2.1000, 0.1342) -- cycle;
\fill[blue!15.0, opacity=0.7] (2.7000, 2.1000, 0.1342) -- (2.7500, 2.1000, 0.1281) -- (2.7500, 2.1500, 0.1243) -- (2.7000, 2.1500, 0.1303) -- cycle;
\fill[blue!15.0, opacity=0.7] (2.7000, 2.1500, 0.1303) -- (2.7500, 2.1500, 0.1243) -- (2.7500, 2.2000, 0.1202) -- (2.7000, 2.2000, 0.1263) -- cycle;
\fill[blue!15.0, opacity=0.7] (2.7000, 2.2000, 0.1263) -- (2.7500, 2.2000, 0.1202) -- (2.7500, 2.2500, 0.1159) -- (2.7000, 2.2500, 0.1219) -- cycle;
\fill[blue!15.0, opacity=0.7] (2.7000, 2.2500, 0.1219) -- (2.7500, 2.2500, 0.1159) -- (2.7500, 2.3000, 0.1114) -- (2.7000, 2.3000, 0.1174) -- cycle;
\fill[blue!15.0, opacity=0.7] (2.7000, 2.3000, 0.1174) -- (2.7500, 2.3000, 0.1114) -- (2.7500, 2.3500, 0.1066) -- (2.7000, 2.3500, 0.1126) -- cycle;
\fill[blue!15.0, opacity=0.7] (2.7000, 2.3500, 0.1126) -- (2.7500, 2.3500, 0.1066) -- (2.7500, 2.4000, 0.1016) -- (2.7000, 2.4000, 0.1076) -- cycle;
\fill[blue!15.0, opacity=0.7] (2.7000, 2.4000, 0.1076) -- (2.7500, 2.4000, 0.1016) -- (2.7500, 2.4500, 0.0964) -- (2.7000, 2.4500, 0.1024) -- cycle;
\fill[blue!15.0, opacity=0.7] (2.7000, 2.4500, 0.1024) -- (2.7500, 2.4500, 0.0964) -- (2.7500, 2.5000, 0.0911) -- (2.7000, 2.5000, 0.0971) -- cycle;
\fill[blue!15.0, opacity=0.7] (2.7000, 2.5000, 0.0971) -- (2.7500, 2.5000, 0.0911) -- (2.7500, 2.5500, 0.0855) -- (2.7000, 2.5500, 0.0916) -- cycle;
\fill[blue!15.0, opacity=0.7] (2.7000, 2.5500, 0.0916) -- (2.7500, 2.5500, 0.0855) -- (2.7500, 2.6000, 0.0799) -- (2.7000, 2.6000, 0.0859) -- cycle;
\fill[blue!15.0, opacity=0.7] (2.7000, 2.6000, 0.0859) -- (2.7500, 2.6000, 0.0799) -- (2.7500, 2.6500, 0.0741) -- (2.7000, 2.6500, 0.0801) -- cycle;
\fill[blue!15.0, opacity=0.7] (2.7000, 2.6500, 0.0801) -- (2.7500, 2.6500, 0.0741) -- (2.7500, 2.7000, 0.0681) -- (2.7000, 2.7000, 0.0742) -- cycle;
\fill[blue!15.0, opacity=0.7] (2.7000, 2.7000, 0.0742) -- (2.7500, 2.7000, 0.0681) -- (2.7500, 2.7500, 0.0621) -- (2.7000, 2.7500, 0.0681) -- cycle;
\fill[blue!15.0, opacity=0.7] (2.7000, 2.7500, 0.0681) -- (2.7500, 2.7500, 0.0621) -- (2.7500, 2.8000, 0.0560) -- (2.7000, 2.8000, 0.0620) -- cycle;
\fill[blue!15.0, opacity=0.7] (2.7000, 2.8000, 0.0620) -- (2.7500, 2.8000, 0.0560) -- (2.7500, 2.8500, 0.0498) -- (2.7000, 2.8500, 0.0559) -- cycle;
\fill[blue!15.0, opacity=0.7] (2.7000, 2.8500, 0.0559) -- (2.7500, 2.8500, 0.0498) -- (2.7500, 2.9000, 0.0436) -- (2.7000, 2.9000, 0.0496) -- cycle;
\fill[blue!15.0, opacity=0.7] (2.7000, 2.9000, 0.0496) -- (2.7500, 2.9000, 0.0436) -- (2.7500, 2.9500, 0.0373) -- (2.7000, 2.9500, 0.0434) -- cycle;
\fill[blue!15.0, opacity=0.7] (2.7000, 2.9500, 0.0434) -- (2.7500, 2.9500, 0.0373) -- (2.7500, 3.0000, 0.0311) -- (2.7000, 3.0000, 0.0371) -- cycle;
\fill[blue!15.0, opacity=0.7] (2.7500, 0.0000, 0.0311) -- (2.8000, 0.0000, 0.0249) -- (2.8000, 0.0500, 0.0312) -- (2.7500, 0.0500, 0.0373) -- cycle;
\fill[blue!15.0, opacity=0.7] (2.7500, 0.0500, 0.0373) -- (2.8000, 0.0500, 0.0312) -- (2.8000, 0.1000, 0.0375) -- (2.7500, 0.1000, 0.0436) -- cycle;
\fill[blue!15.0, opacity=0.7] (2.7500, 0.1000, 0.0436) -- (2.8000, 0.1000, 0.0375) -- (2.8000, 0.1500, 0.0437) -- (2.7500, 0.1500, 0.0498) -- cycle;
\fill[blue!15.0, opacity=0.7] (2.7500, 0.1500, 0.0498) -- (2.8000, 0.1500, 0.0437) -- (2.8000, 0.2000, 0.0499) -- (2.7500, 0.2000, 0.0560) -- cycle;
\fill[blue!15.0, opacity=0.7] (2.7500, 0.2000, 0.0560) -- (2.8000, 0.2000, 0.0499) -- (2.8000, 0.2500, 0.0560) -- (2.7500, 0.2500, 0.0621) -- cycle;
\fill[blue!15.0, opacity=0.7] (2.7500, 0.2500, 0.0621) -- (2.8000, 0.2500, 0.0560) -- (2.8000, 0.3000, 0.0620) -- (2.7500, 0.3000, 0.0681) -- cycle;
\fill[blue!15.0, opacity=0.7] (2.7500, 0.3000, 0.0681) -- (2.8000, 0.3000, 0.0620) -- (2.8000, 0.3500, 0.0680) -- (2.7500, 0.3500, 0.0741) -- cycle;
\fill[blue!15.0, opacity=0.7] (2.7500, 0.3500, 0.0741) -- (2.8000, 0.3500, 0.0680) -- (2.8000, 0.4000, 0.0738) -- (2.7500, 0.4000, 0.0799) -- cycle;
\fill[blue!15.0, opacity=0.7] (2.7500, 0.4000, 0.0799) -- (2.8000, 0.4000, 0.0738) -- (2.8000, 0.4500, 0.0794) -- (2.7500, 0.4500, 0.0855) -- cycle;
\fill[blue!15.0, opacity=0.7] (2.7500, 0.4500, 0.0855) -- (2.8000, 0.4500, 0.0794) -- (2.8000, 0.5000, 0.0849) -- (2.7500, 0.5000, 0.0911) -- cycle;
\fill[blue!15.0, opacity=0.7] (2.7500, 0.5000, 0.0911) -- (2.8000, 0.5000, 0.0849) -- (2.8000, 0.5500, 0.0903) -- (2.7500, 0.5500, 0.0964) -- cycle;
\fill[blue!15.0, opacity=0.7] (2.7500, 0.5500, 0.0964) -- (2.8000, 0.5500, 0.0903) -- (2.8000, 0.6000, 0.0955) -- (2.7500, 0.6000, 0.1016) -- cycle;
\fill[blue!15.0, opacity=0.7] (2.7500, 0.6000, 0.1016) -- (2.8000, 0.6000, 0.0955) -- (2.8000, 0.6500, 0.1005) -- (2.7500, 0.6500, 0.1066) -- cycle;
\fill[blue!15.0, opacity=0.7] (2.7500, 0.6500, 0.1066) -- (2.8000, 0.6500, 0.1005) -- (2.8000, 0.7000, 0.1052) -- (2.7500, 0.7000, 0.1114) -- cycle;
\fill[blue!15.0, opacity=0.7] (2.7500, 0.7000, 0.1114) -- (2.8000, 0.7000, 0.1052) -- (2.8000, 0.7500, 0.1098) -- (2.7500, 0.7500, 0.1159) -- cycle;
\fill[blue!15.0, opacity=0.7] (2.7500, 0.7500, 0.1159) -- (2.8000, 0.7500, 0.1098) -- (2.8000, 0.8000, 0.1141) -- (2.7500, 0.8000, 0.1202) -- cycle;
\fill[blue!15.0, opacity=0.7] (2.7500, 0.8000, 0.1202) -- (2.8000, 0.8000, 0.1141) -- (2.8000, 0.8500, 0.1182) -- (2.7500, 0.8500, 0.1243) -- cycle;
\fill[blue!15.0, opacity=0.7] (2.7500, 0.8500, 0.1243) -- (2.8000, 0.8500, 0.1182) -- (2.8000, 0.9000, 0.1220) -- (2.7500, 0.9000, 0.1281) -- cycle;
\fill[blue!15.0, opacity=0.7] (2.7500, 0.9000, 0.1281) -- (2.8000, 0.9000, 0.1220) -- (2.8000, 0.9500, 0.1256) -- (2.7500, 0.9500, 0.1317) -- cycle;
\fill[blue!15.0, opacity=0.7] (2.7500, 0.9500, 0.1317) -- (2.8000, 0.9500, 0.1256) -- (2.8000, 1.0000, 0.1289) -- (2.7500, 1.0000, 0.1350) -- cycle;
\fill[blue!15.0, opacity=0.7] (2.7500, 1.0000, 0.1350) -- (2.8000, 1.0000, 0.1289) -- (2.8000, 1.0500, 0.1319) -- (2.7500, 1.0500, 0.1380) -- cycle;
\fill[blue!15.0, opacity=0.7] (2.7500, 1.0500, 0.1380) -- (2.8000, 1.0500, 0.1319) -- (2.8000, 1.1000, 0.1346) -- (2.7500, 1.1000, 0.1407) -- cycle;
\fill[blue!15.0, opacity=0.7] (2.7500, 1.1000, 0.1407) -- (2.8000, 1.1000, 0.1346) -- (2.8000, 1.1500, 0.1370) -- (2.7500, 1.1500, 0.1431) -- cycle;
\fill[blue!15.0, opacity=0.7] (2.7500, 1.1500, 0.1431) -- (2.8000, 1.1500, 0.1370) -- (2.8000, 1.2000, 0.1391) -- (2.7500, 1.2000, 0.1452) -- cycle;
\fill[blue!15.0, opacity=0.7] (2.7500, 1.2000, 0.1452) -- (2.8000, 1.2000, 0.1391) -- (2.8000, 1.2500, 0.1409) -- (2.7500, 1.2500, 0.1470) -- cycle;
\fill[blue!15.0, opacity=0.7] (2.7500, 1.2500, 0.1470) -- (2.8000, 1.2500, 0.1409) -- (2.8000, 1.3000, 0.1423) -- (2.7500, 1.3000, 0.1484) -- cycle;
\fill[blue!15.0, opacity=0.7] (2.7500, 1.3000, 0.1484) -- (2.8000, 1.3000, 0.1423) -- (2.8000, 1.3500, 0.1435) -- (2.7500, 1.3500, 0.1496) -- cycle;
\fill[blue!15.0, opacity=0.7] (2.7500, 1.3500, 0.1496) -- (2.8000, 1.3500, 0.1435) -- (2.8000, 1.4000, 0.1443) -- (2.7500, 1.4000, 0.1504) -- cycle;
\fill[blue!15.0, opacity=0.7] (2.7500, 1.4000, 0.1504) -- (2.8000, 1.4000, 0.1443) -- (2.8000, 1.4500, 0.1448) -- (2.7500, 1.4500, 0.1509) -- cycle;
\fill[blue!15.0, opacity=0.7] (2.7500, 1.4500, 0.1509) -- (2.8000, 1.4500, 0.1448) -- (2.8000, 1.5000, 0.1449) -- (2.7500, 1.5000, 0.1511) -- cycle;
\fill[blue!15.0, opacity=0.7] (2.7500, 1.5000, 0.1511) -- (2.8000, 1.5000, 0.1449) -- (2.8000, 1.5500, 0.1448) -- (2.7500, 1.5500, 0.1509) -- cycle;
\fill[blue!15.0, opacity=0.7] (2.7500, 1.5500, 0.1509) -- (2.8000, 1.5500, 0.1448) -- (2.8000, 1.6000, 0.1443) -- (2.7500, 1.6000, 0.1504) -- cycle;
\fill[blue!15.0, opacity=0.7] (2.7500, 1.6000, 0.1504) -- (2.8000, 1.6000, 0.1443) -- (2.8000, 1.6500, 0.1435) -- (2.7500, 1.6500, 0.1496) -- cycle;
\fill[blue!15.0, opacity=0.7] (2.7500, 1.6500, 0.1496) -- (2.8000, 1.6500, 0.1435) -- (2.8000, 1.7000, 0.1423) -- (2.7500, 1.7000, 0.1484) -- cycle;
\fill[blue!15.0, opacity=0.7] (2.7500, 1.7000, 0.1484) -- (2.8000, 1.7000, 0.1423) -- (2.8000, 1.7500, 0.1409) -- (2.7500, 1.7500, 0.1470) -- cycle;
\fill[blue!15.0, opacity=0.7] (2.7500, 1.7500, 0.1470) -- (2.8000, 1.7500, 0.1409) -- (2.8000, 1.8000, 0.1391) -- (2.7500, 1.8000, 0.1452) -- cycle;
\fill[blue!15.0, opacity=0.7] (2.7500, 1.8000, 0.1452) -- (2.8000, 1.8000, 0.1391) -- (2.8000, 1.8500, 0.1370) -- (2.7500, 1.8500, 0.1431) -- cycle;
\fill[blue!15.0, opacity=0.7] (2.7500, 1.8500, 0.1431) -- (2.8000, 1.8500, 0.1370) -- (2.8000, 1.9000, 0.1346) -- (2.7500, 1.9000, 0.1407) -- cycle;
\fill[blue!15.0, opacity=0.7] (2.7500, 1.9000, 0.1407) -- (2.8000, 1.9000, 0.1346) -- (2.8000, 1.9500, 0.1319) -- (2.7500, 1.9500, 0.1380) -- cycle;
\fill[blue!15.0, opacity=0.7] (2.7500, 1.9500, 0.1380) -- (2.8000, 1.9500, 0.1319) -- (2.8000, 2.0000, 0.1289) -- (2.7500, 2.0000, 0.1350) -- cycle;
\fill[blue!15.0, opacity=0.7] (2.7500, 2.0000, 0.1350) -- (2.8000, 2.0000, 0.1289) -- (2.8000, 2.0500, 0.1256) -- (2.7500, 2.0500, 0.1317) -- cycle;
\fill[blue!15.0, opacity=0.7] (2.7500, 2.0500, 0.1317) -- (2.8000, 2.0500, 0.1256) -- (2.8000, 2.1000, 0.1220) -- (2.7500, 2.1000, 0.1281) -- cycle;
\fill[blue!15.0, opacity=0.7] (2.7500, 2.1000, 0.1281) -- (2.8000, 2.1000, 0.1220) -- (2.8000, 2.1500, 0.1182) -- (2.7500, 2.1500, 0.1243) -- cycle;
\fill[blue!15.0, opacity=0.7] (2.7500, 2.1500, 0.1243) -- (2.8000, 2.1500, 0.1182) -- (2.8000, 2.2000, 0.1141) -- (2.7500, 2.2000, 0.1202) -- cycle;
\fill[blue!15.0, opacity=0.7] (2.7500, 2.2000, 0.1202) -- (2.8000, 2.2000, 0.1141) -- (2.8000, 2.2500, 0.1098) -- (2.7500, 2.2500, 0.1159) -- cycle;
\fill[blue!15.0, opacity=0.7] (2.7500, 2.2500, 0.1159) -- (2.8000, 2.2500, 0.1098) -- (2.8000, 2.3000, 0.1052) -- (2.7500, 2.3000, 0.1114) -- cycle;
\fill[blue!15.0, opacity=0.7] (2.7500, 2.3000, 0.1114) -- (2.8000, 2.3000, 0.1052) -- (2.8000, 2.3500, 0.1005) -- (2.7500, 2.3500, 0.1066) -- cycle;
\fill[blue!15.0, opacity=0.7] (2.7500, 2.3500, 0.1066) -- (2.8000, 2.3500, 0.1005) -- (2.8000, 2.4000, 0.0955) -- (2.7500, 2.4000, 0.1016) -- cycle;
\fill[blue!15.0, opacity=0.7] (2.7500, 2.4000, 0.1016) -- (2.8000, 2.4000, 0.0955) -- (2.8000, 2.4500, 0.0903) -- (2.7500, 2.4500, 0.0964) -- cycle;
\fill[blue!15.0, opacity=0.7] (2.7500, 2.4500, 0.0964) -- (2.8000, 2.4500, 0.0903) -- (2.8000, 2.5000, 0.0849) -- (2.7500, 2.5000, 0.0911) -- cycle;
\fill[blue!15.0, opacity=0.7] (2.7500, 2.5000, 0.0911) -- (2.8000, 2.5000, 0.0849) -- (2.8000, 2.5500, 0.0794) -- (2.7500, 2.5500, 0.0855) -- cycle;
\fill[blue!15.0, opacity=0.7] (2.7500, 2.5500, 0.0855) -- (2.8000, 2.5500, 0.0794) -- (2.8000, 2.6000, 0.0738) -- (2.7500, 2.6000, 0.0799) -- cycle;
\fill[blue!15.0, opacity=0.7] (2.7500, 2.6000, 0.0799) -- (2.8000, 2.6000, 0.0738) -- (2.8000, 2.6500, 0.0680) -- (2.7500, 2.6500, 0.0741) -- cycle;
\fill[blue!15.0, opacity=0.7] (2.7500, 2.6500, 0.0741) -- (2.8000, 2.6500, 0.0680) -- (2.8000, 2.7000, 0.0620) -- (2.7500, 2.7000, 0.0681) -- cycle;
\fill[blue!15.0, opacity=0.7] (2.7500, 2.7000, 0.0681) -- (2.8000, 2.7000, 0.0620) -- (2.8000, 2.7500, 0.0560) -- (2.7500, 2.7500, 0.0621) -- cycle;
\fill[blue!15.0, opacity=0.7] (2.7500, 2.7500, 0.0621) -- (2.8000, 2.7500, 0.0560) -- (2.8000, 2.8000, 0.0499) -- (2.7500, 2.8000, 0.0560) -- cycle;
\fill[blue!15.0, opacity=0.7] (2.7500, 2.8000, 0.0560) -- (2.8000, 2.8000, 0.0499) -- (2.8000, 2.8500, 0.0437) -- (2.7500, 2.8500, 0.0498) -- cycle;
\fill[blue!15.0, opacity=0.7] (2.7500, 2.8500, 0.0498) -- (2.8000, 2.8500, 0.0437) -- (2.8000, 2.9000, 0.0375) -- (2.7500, 2.9000, 0.0436) -- cycle;
\fill[blue!15.0, opacity=0.7] (2.7500, 2.9000, 0.0436) -- (2.8000, 2.9000, 0.0375) -- (2.8000, 2.9500, 0.0312) -- (2.7500, 2.9500, 0.0373) -- cycle;
\fill[blue!15.0, opacity=0.7] (2.7500, 2.9500, 0.0373) -- (2.8000, 2.9500, 0.0312) -- (2.8000, 3.0000, 0.0249) -- (2.7500, 3.0000, 0.0311) -- cycle;
\fill[blue!15.0, opacity=0.7] (2.8000, 0.0000, 0.0249) -- (2.8500, 0.0000, 0.0188) -- (2.8500, 0.0500, 0.0251) -- (2.8000, 0.0500, 0.0312) -- cycle;
\fill[blue!15.0, opacity=0.7] (2.8000, 0.0500, 0.0312) -- (2.8500, 0.0500, 0.0251) -- (2.8500, 0.1000, 0.0313) -- (2.8000, 0.1000, 0.0375) -- cycle;
\fill[blue!15.0, opacity=0.7] (2.8000, 0.1000, 0.0375) -- (2.8500, 0.1000, 0.0313) -- (2.8500, 0.1500, 0.0375) -- (2.8000, 0.1500, 0.0437) -- cycle;
\fill[blue!15.0, opacity=0.7] (2.8000, 0.1500, 0.0437) -- (2.8500, 0.1500, 0.0375) -- (2.8500, 0.2000, 0.0437) -- (2.8000, 0.2000, 0.0499) -- cycle;
\fill[blue!15.0, opacity=0.7] (2.8000, 0.2000, 0.0499) -- (2.8500, 0.2000, 0.0437) -- (2.8500, 0.2500, 0.0498) -- (2.8000, 0.2500, 0.0560) -- cycle;
\fill[blue!15.0, opacity=0.7] (2.8000, 0.2500, 0.0560) -- (2.8500, 0.2500, 0.0498) -- (2.8500, 0.3000, 0.0559) -- (2.8000, 0.3000, 0.0620) -- cycle;
\fill[blue!15.0, opacity=0.7] (2.8000, 0.3000, 0.0620) -- (2.8500, 0.3000, 0.0559) -- (2.8500, 0.3500, 0.0618) -- (2.8000, 0.3500, 0.0680) -- cycle;
\fill[blue!15.0, opacity=0.7] (2.8000, 0.3500, 0.0680) -- (2.8500, 0.3500, 0.0618) -- (2.8500, 0.4000, 0.0676) -- (2.8000, 0.4000, 0.0738) -- cycle;
\fill[blue!15.0, opacity=0.7] (2.8000, 0.4000, 0.0738) -- (2.8500, 0.4000, 0.0676) -- (2.8500, 0.4500, 0.0733) -- (2.8000, 0.4500, 0.0794) -- cycle;
\fill[blue!15.0, opacity=0.7] (2.8000, 0.4500, 0.0794) -- (2.8500, 0.4500, 0.0733) -- (2.8500, 0.5000, 0.0788) -- (2.8000, 0.5000, 0.0849) -- cycle;
\fill[blue!15.0, opacity=0.7] (2.8000, 0.5000, 0.0849) -- (2.8500, 0.5000, 0.0788) -- (2.8500, 0.5500, 0.0841) -- (2.8000, 0.5500, 0.0903) -- cycle;
\fill[blue!15.0, opacity=0.7] (2.8000, 0.5500, 0.0903) -- (2.8500, 0.5500, 0.0841) -- (2.8500, 0.6000, 0.0893) -- (2.8000, 0.6000, 0.0955) -- cycle;
\fill[blue!15.0, opacity=0.7] (2.8000, 0.6000, 0.0955) -- (2.8500, 0.6000, 0.0893) -- (2.8500, 0.6500, 0.0943) -- (2.8000, 0.6500, 0.1005) -- cycle;
\fill[blue!15.0, opacity=0.7] (2.8000, 0.6500, 0.1005) -- (2.8500, 0.6500, 0.0943) -- (2.8500, 0.7000, 0.0991) -- (2.8000, 0.7000, 0.1052) -- cycle;
\fill[blue!15.0, opacity=0.7] (2.8000, 0.7000, 0.1052) -- (2.8500, 0.7000, 0.0991) -- (2.8500, 0.7500, 0.1036) -- (2.8000, 0.7500, 0.1098) -- cycle;
\fill[blue!15.0, opacity=0.7] (2.8000, 0.7500, 0.1098) -- (2.8500, 0.7500, 0.1036) -- (2.8500, 0.8000, 0.1079) -- (2.8000, 0.8000, 0.1141) -- cycle;
\fill[blue!15.0, opacity=0.7] (2.8000, 0.8000, 0.1141) -- (2.8500, 0.8000, 0.1079) -- (2.8500, 0.8500, 0.1120) -- (2.8000, 0.8500, 0.1182) -- cycle;
\fill[blue!15.0, opacity=0.7] (2.8000, 0.8500, 0.1182) -- (2.8500, 0.8500, 0.1120) -- (2.8500, 0.9000, 0.1159) -- (2.8000, 0.9000, 0.1220) -- cycle;
\fill[blue!15.0, opacity=0.7] (2.8000, 0.9000, 0.1220) -- (2.8500, 0.9000, 0.1159) -- (2.8500, 0.9500, 0.1194) -- (2.8000, 0.9500, 0.1256) -- cycle;
\fill[blue!15.0, opacity=0.7] (2.8000, 0.9500, 0.1256) -- (2.8500, 0.9500, 0.1194) -- (2.8500, 1.0000, 0.1227) -- (2.8000, 1.0000, 0.1289) -- cycle;
\fill[blue!15.0, opacity=0.7] (2.8000, 1.0000, 0.1289) -- (2.8500, 1.0000, 0.1227) -- (2.8500, 1.0500, 0.1257) -- (2.8000, 1.0500, 0.1319) -- cycle;
\fill[blue!15.0, opacity=0.7] (2.8000, 1.0500, 0.1319) -- (2.8500, 1.0500, 0.1257) -- (2.8500, 1.1000, 0.1284) -- (2.8000, 1.1000, 0.1346) -- cycle;
\fill[blue!15.0, opacity=0.7] (2.8000, 1.1000, 0.1346) -- (2.8500, 1.1000, 0.1284) -- (2.8500, 1.1500, 0.1308) -- (2.8000, 1.1500, 0.1370) -- cycle;
\fill[blue!15.0, opacity=0.7] (2.8000, 1.1500, 0.1370) -- (2.8500, 1.1500, 0.1308) -- (2.8500, 1.2000, 0.1329) -- (2.8000, 1.2000, 0.1391) -- cycle;
\fill[blue!15.0, opacity=0.7] (2.8000, 1.2000, 0.1391) -- (2.8500, 1.2000, 0.1329) -- (2.8500, 1.2500, 0.1347) -- (2.8000, 1.2500, 0.1409) -- cycle;
\fill[blue!15.0, opacity=0.7] (2.8000, 1.2500, 0.1409) -- (2.8500, 1.2500, 0.1347) -- (2.8500, 1.3000, 0.1361) -- (2.8000, 1.3000, 0.1423) -- cycle;
\fill[blue!15.0, opacity=0.7] (2.8000, 1.3000, 0.1423) -- (2.8500, 1.3000, 0.1361) -- (2.8500, 1.3500, 0.1373) -- (2.8000, 1.3500, 0.1435) -- cycle;
\fill[blue!15.0, opacity=0.7] (2.8000, 1.3500, 0.1435) -- (2.8500, 1.3500, 0.1373) -- (2.8500, 1.4000, 0.1381) -- (2.8000, 1.4000, 0.1443) -- cycle;
\fill[blue!15.0, opacity=0.7] (2.8000, 1.4000, 0.1443) -- (2.8500, 1.4000, 0.1381) -- (2.8500, 1.4500, 0.1386) -- (2.8000, 1.4500, 0.1448) -- cycle;
\fill[blue!15.0, opacity=0.7] (2.8000, 1.4500, 0.1448) -- (2.8500, 1.4500, 0.1386) -- (2.8500, 1.5000, 0.1388) -- (2.8000, 1.5000, 0.1449) -- cycle;
\fill[blue!15.0, opacity=0.7] (2.8000, 1.5000, 0.1449) -- (2.8500, 1.5000, 0.1388) -- (2.8500, 1.5500, 0.1386) -- (2.8000, 1.5500, 0.1448) -- cycle;
\fill[blue!15.0, opacity=0.7] (2.8000, 1.5500, 0.1448) -- (2.8500, 1.5500, 0.1386) -- (2.8500, 1.6000, 0.1381) -- (2.8000, 1.6000, 0.1443) -- cycle;
\fill[blue!15.0, opacity=0.7] (2.8000, 1.6000, 0.1443) -- (2.8500, 1.6000, 0.1381) -- (2.8500, 1.6500, 0.1373) -- (2.8000, 1.6500, 0.1435) -- cycle;
\fill[blue!15.0, opacity=0.7] (2.8000, 1.6500, 0.1435) -- (2.8500, 1.6500, 0.1373) -- (2.8500, 1.7000, 0.1361) -- (2.8000, 1.7000, 0.1423) -- cycle;
\fill[blue!15.0, opacity=0.7] (2.8000, 1.7000, 0.1423) -- (2.8500, 1.7000, 0.1361) -- (2.8500, 1.7500, 0.1347) -- (2.8000, 1.7500, 0.1409) -- cycle;
\fill[blue!15.0, opacity=0.7] (2.8000, 1.7500, 0.1409) -- (2.8500, 1.7500, 0.1347) -- (2.8500, 1.8000, 0.1329) -- (2.8000, 1.8000, 0.1391) -- cycle;
\fill[blue!15.0, opacity=0.7] (2.8000, 1.8000, 0.1391) -- (2.8500, 1.8000, 0.1329) -- (2.8500, 1.8500, 0.1308) -- (2.8000, 1.8500, 0.1370) -- cycle;
\fill[blue!15.0, opacity=0.7] (2.8000, 1.8500, 0.1370) -- (2.8500, 1.8500, 0.1308) -- (2.8500, 1.9000, 0.1284) -- (2.8000, 1.9000, 0.1346) -- cycle;
\fill[blue!15.0, opacity=0.7] (2.8000, 1.9000, 0.1346) -- (2.8500, 1.9000, 0.1284) -- (2.8500, 1.9500, 0.1257) -- (2.8000, 1.9500, 0.1319) -- cycle;
\fill[blue!15.0, opacity=0.7] (2.8000, 1.9500, 0.1319) -- (2.8500, 1.9500, 0.1257) -- (2.8500, 2.0000, 0.1227) -- (2.8000, 2.0000, 0.1289) -- cycle;
\fill[blue!15.0, opacity=0.7] (2.8000, 2.0000, 0.1289) -- (2.8500, 2.0000, 0.1227) -- (2.8500, 2.0500, 0.1194) -- (2.8000, 2.0500, 0.1256) -- cycle;
\fill[blue!15.0, opacity=0.7] (2.8000, 2.0500, 0.1256) -- (2.8500, 2.0500, 0.1194) -- (2.8500, 2.1000, 0.1159) -- (2.8000, 2.1000, 0.1220) -- cycle;
\fill[blue!15.0, opacity=0.7] (2.8000, 2.1000, 0.1220) -- (2.8500, 2.1000, 0.1159) -- (2.8500, 2.1500, 0.1120) -- (2.8000, 2.1500, 0.1182) -- cycle;
\fill[blue!15.0, opacity=0.7] (2.8000, 2.1500, 0.1182) -- (2.8500, 2.1500, 0.1120) -- (2.8500, 2.2000, 0.1079) -- (2.8000, 2.2000, 0.1141) -- cycle;
\fill[blue!15.0, opacity=0.7] (2.8000, 2.2000, 0.1141) -- (2.8500, 2.2000, 0.1079) -- (2.8500, 2.2500, 0.1036) -- (2.8000, 2.2500, 0.1098) -- cycle;
\fill[blue!15.0, opacity=0.7] (2.8000, 2.2500, 0.1098) -- (2.8500, 2.2500, 0.1036) -- (2.8500, 2.3000, 0.0991) -- (2.8000, 2.3000, 0.1052) -- cycle;
\fill[blue!15.0, opacity=0.7] (2.8000, 2.3000, 0.1052) -- (2.8500, 2.3000, 0.0991) -- (2.8500, 2.3500, 0.0943) -- (2.8000, 2.3500, 0.1005) -- cycle;
\fill[blue!15.0, opacity=0.7] (2.8000, 2.3500, 0.1005) -- (2.8500, 2.3500, 0.0943) -- (2.8500, 2.4000, 0.0893) -- (2.8000, 2.4000, 0.0955) -- cycle;
\fill[blue!15.0, opacity=0.7] (2.8000, 2.4000, 0.0955) -- (2.8500, 2.4000, 0.0893) -- (2.8500, 2.4500, 0.0841) -- (2.8000, 2.4500, 0.0903) -- cycle;
\fill[blue!15.0, opacity=0.7] (2.8000, 2.4500, 0.0903) -- (2.8500, 2.4500, 0.0841) -- (2.8500, 2.5000, 0.0788) -- (2.8000, 2.5000, 0.0849) -- cycle;
\fill[blue!15.0, opacity=0.7] (2.8000, 2.5000, 0.0849) -- (2.8500, 2.5000, 0.0788) -- (2.8500, 2.5500, 0.0733) -- (2.8000, 2.5500, 0.0794) -- cycle;
\fill[blue!15.0, opacity=0.7] (2.8000, 2.5500, 0.0794) -- (2.8500, 2.5500, 0.0733) -- (2.8500, 2.6000, 0.0676) -- (2.8000, 2.6000, 0.0738) -- cycle;
\fill[blue!15.0, opacity=0.7] (2.8000, 2.6000, 0.0738) -- (2.8500, 2.6000, 0.0676) -- (2.8500, 2.6500, 0.0618) -- (2.8000, 2.6500, 0.0680) -- cycle;
\fill[blue!15.0, opacity=0.7] (2.8000, 2.6500, 0.0680) -- (2.8500, 2.6500, 0.0618) -- (2.8500, 2.7000, 0.0559) -- (2.8000, 2.7000, 0.0620) -- cycle;
\fill[blue!15.0, opacity=0.7] (2.8000, 2.7000, 0.0620) -- (2.8500, 2.7000, 0.0559) -- (2.8500, 2.7500, 0.0498) -- (2.8000, 2.7500, 0.0560) -- cycle;
\fill[blue!15.0, opacity=0.7] (2.8000, 2.7500, 0.0560) -- (2.8500, 2.7500, 0.0498) -- (2.8500, 2.8000, 0.0437) -- (2.8000, 2.8000, 0.0499) -- cycle;
\fill[blue!15.0, opacity=0.7] (2.8000, 2.8000, 0.0499) -- (2.8500, 2.8000, 0.0437) -- (2.8500, 2.8500, 0.0375) -- (2.8000, 2.8500, 0.0437) -- cycle;
\fill[blue!15.0, opacity=0.7] (2.8000, 2.8500, 0.0437) -- (2.8500, 2.8500, 0.0375) -- (2.8500, 2.9000, 0.0313) -- (2.8000, 2.9000, 0.0375) -- cycle;
\fill[blue!15.0, opacity=0.7] (2.8000, 2.9000, 0.0375) -- (2.8500, 2.9000, 0.0313) -- (2.8500, 2.9500, 0.0251) -- (2.8000, 2.9500, 0.0312) -- cycle;
\fill[blue!15.0, opacity=0.7] (2.8000, 2.9500, 0.0312) -- (2.8500, 2.9500, 0.0251) -- (2.8500, 3.0000, 0.0188) -- (2.8000, 3.0000, 0.0249) -- cycle;
\fill[blue!15.0, opacity=0.7] (2.8500, 0.0000, 0.0188) -- (2.9000, 0.0000, 0.0125) -- (2.9000, 0.0500, 0.0188) -- (2.8500, 0.0500, 0.0251) -- cycle;
\fill[blue!15.0, opacity=0.7] (2.8500, 0.0500, 0.0251) -- (2.9000, 0.0500, 0.0188) -- (2.9000, 0.1000, 0.0251) -- (2.8500, 0.1000, 0.0313) -- cycle;
\fill[blue!15.0, opacity=0.7] (2.8500, 0.1000, 0.0313) -- (2.9000, 0.1000, 0.0251) -- (2.9000, 0.1500, 0.0313) -- (2.8500, 0.1500, 0.0375) -- cycle;
\fill[blue!15.0, opacity=0.7] (2.8500, 0.1500, 0.0375) -- (2.9000, 0.1500, 0.0313) -- (2.9000, 0.2000, 0.0375) -- (2.8500, 0.2000, 0.0437) -- cycle;
\fill[blue!15.0, opacity=0.7] (2.8500, 0.2000, 0.0437) -- (2.9000, 0.2000, 0.0375) -- (2.9000, 0.2500, 0.0436) -- (2.8500, 0.2500, 0.0498) -- cycle;
\fill[blue!15.0, opacity=0.7] (2.8500, 0.2500, 0.0498) -- (2.9000, 0.2500, 0.0436) -- (2.9000, 0.3000, 0.0496) -- (2.8500, 0.3000, 0.0559) -- cycle;
\fill[blue!15.0, opacity=0.7] (2.8500, 0.3000, 0.0559) -- (2.9000, 0.3000, 0.0496) -- (2.9000, 0.3500, 0.0555) -- (2.8500, 0.3500, 0.0618) -- cycle;
\fill[blue!15.0, opacity=0.7] (2.8500, 0.3500, 0.0618) -- (2.9000, 0.3500, 0.0555) -- (2.9000, 0.4000, 0.0614) -- (2.8500, 0.4000, 0.0676) -- cycle;
\fill[blue!15.0, opacity=0.7] (2.8500, 0.4000, 0.0676) -- (2.9000, 0.4000, 0.0614) -- (2.9000, 0.4500, 0.0670) -- (2.8500, 0.4500, 0.0733) -- cycle;
\fill[blue!15.0, opacity=0.7] (2.8500, 0.4500, 0.0733) -- (2.9000, 0.4500, 0.0670) -- (2.9000, 0.5000, 0.0725) -- (2.8500, 0.5000, 0.0788) -- cycle;
\fill[blue!15.0, opacity=0.7] (2.8500, 0.5000, 0.0788) -- (2.9000, 0.5000, 0.0725) -- (2.9000, 0.5500, 0.0779) -- (2.8500, 0.5500, 0.0841) -- cycle;
\fill[blue!15.0, opacity=0.7] (2.8500, 0.5500, 0.0841) -- (2.9000, 0.5500, 0.0779) -- (2.9000, 0.6000, 0.0831) -- (2.8500, 0.6000, 0.0893) -- cycle;
\fill[blue!15.0, opacity=0.7] (2.8500, 0.6000, 0.0893) -- (2.9000, 0.6000, 0.0831) -- (2.9000, 0.6500, 0.0881) -- (2.8500, 0.6500, 0.0943) -- cycle;
\fill[blue!15.0, opacity=0.7] (2.8500, 0.6500, 0.0943) -- (2.9000, 0.6500, 0.0881) -- (2.9000, 0.7000, 0.0928) -- (2.8500, 0.7000, 0.0991) -- cycle;
\fill[blue!15.0, opacity=0.7] (2.8500, 0.7000, 0.0991) -- (2.9000, 0.7000, 0.0928) -- (2.9000, 0.7500, 0.0974) -- (2.8500, 0.7500, 0.1036) -- cycle;
\fill[blue!15.0, opacity=0.7] (2.8500, 0.7500, 0.1036) -- (2.9000, 0.7500, 0.0974) -- (2.9000, 0.8000, 0.1017) -- (2.8500, 0.8000, 0.1079) -- cycle;
\fill[blue!15.0, opacity=0.7] (2.8500, 0.8000, 0.1079) -- (2.9000, 0.8000, 0.1017) -- (2.9000, 0.8500, 0.1058) -- (2.8500, 0.8500, 0.1120) -- cycle;
\fill[blue!15.0, opacity=0.7] (2.8500, 0.8500, 0.1120) -- (2.9000, 0.8500, 0.1058) -- (2.9000, 0.9000, 0.1096) -- (2.8500, 0.9000, 0.1159) -- cycle;
\fill[blue!15.0, opacity=0.7] (2.8500, 0.9000, 0.1159) -- (2.9000, 0.9000, 0.1096) -- (2.9000, 0.9500, 0.1132) -- (2.8500, 0.9500, 0.1194) -- cycle;
\fill[blue!15.0, opacity=0.7] (2.8500, 0.9500, 0.1194) -- (2.9000, 0.9500, 0.1132) -- (2.9000, 1.0000, 0.1165) -- (2.8500, 1.0000, 0.1227) -- cycle;
\fill[blue!15.0, opacity=0.7] (2.8500, 1.0000, 0.1227) -- (2.9000, 1.0000, 0.1165) -- (2.9000, 1.0500, 0.1195) -- (2.8500, 1.0500, 0.1257) -- cycle;
\fill[blue!15.0, opacity=0.7] (2.8500, 1.0500, 0.1257) -- (2.9000, 1.0500, 0.1195) -- (2.9000, 1.1000, 0.1222) -- (2.8500, 1.1000, 0.1284) -- cycle;
\fill[blue!15.0, opacity=0.7] (2.8500, 1.1000, 0.1284) -- (2.9000, 1.1000, 0.1222) -- (2.9000, 1.1500, 0.1246) -- (2.8500, 1.1500, 0.1308) -- cycle;
\fill[blue!15.0, opacity=0.7] (2.8500, 1.1500, 0.1308) -- (2.9000, 1.1500, 0.1246) -- (2.9000, 1.2000, 0.1267) -- (2.8500, 1.2000, 0.1329) -- cycle;
\fill[blue!15.0, opacity=0.7] (2.8500, 1.2000, 0.1329) -- (2.9000, 1.2000, 0.1267) -- (2.9000, 1.2500, 0.1285) -- (2.8500, 1.2500, 0.1347) -- cycle;
\fill[blue!15.0, opacity=0.7] (2.8500, 1.2500, 0.1347) -- (2.9000, 1.2500, 0.1285) -- (2.9000, 1.3000, 0.1299) -- (2.8500, 1.3000, 0.1361) -- cycle;
\fill[blue!15.0, opacity=0.7] (2.8500, 1.3000, 0.1361) -- (2.9000, 1.3000, 0.1299) -- (2.9000, 1.3500, 0.1311) -- (2.8500, 1.3500, 0.1373) -- cycle;
\fill[blue!15.0, opacity=0.7] (2.8500, 1.3500, 0.1373) -- (2.9000, 1.3500, 0.1311) -- (2.9000, 1.4000, 0.1319) -- (2.8500, 1.4000, 0.1381) -- cycle;
\fill[blue!15.0, opacity=0.7] (2.8500, 1.4000, 0.1381) -- (2.9000, 1.4000, 0.1319) -- (2.9000, 1.4500, 0.1324) -- (2.8500, 1.4500, 0.1386) -- cycle;
\fill[blue!15.0, opacity=0.7] (2.8500, 1.4500, 0.1386) -- (2.9000, 1.4500, 0.1324) -- (2.9000, 1.5000, 0.1325) -- (2.8500, 1.5000, 0.1388) -- cycle;
\fill[blue!15.0, opacity=0.7] (2.8500, 1.5000, 0.1388) -- (2.9000, 1.5000, 0.1325) -- (2.9000, 1.5500, 0.1324) -- (2.8500, 1.5500, 0.1386) -- cycle;
\fill[blue!15.0, opacity=0.7] (2.8500, 1.5500, 0.1386) -- (2.9000, 1.5500, 0.1324) -- (2.9000, 1.6000, 0.1319) -- (2.8500, 1.6000, 0.1381) -- cycle;
\fill[blue!15.0, opacity=0.7] (2.8500, 1.6000, 0.1381) -- (2.9000, 1.6000, 0.1319) -- (2.9000, 1.6500, 0.1311) -- (2.8500, 1.6500, 0.1373) -- cycle;
\fill[blue!15.0, opacity=0.7] (2.8500, 1.6500, 0.1373) -- (2.9000, 1.6500, 0.1311) -- (2.9000, 1.7000, 0.1299) -- (2.8500, 1.7000, 0.1361) -- cycle;
\fill[blue!15.0, opacity=0.7] (2.8500, 1.7000, 0.1361) -- (2.9000, 1.7000, 0.1299) -- (2.9000, 1.7500, 0.1285) -- (2.8500, 1.7500, 0.1347) -- cycle;
\fill[blue!15.0, opacity=0.7] (2.8500, 1.7500, 0.1347) -- (2.9000, 1.7500, 0.1285) -- (2.9000, 1.8000, 0.1267) -- (2.8500, 1.8000, 0.1329) -- cycle;
\fill[blue!15.0, opacity=0.7] (2.8500, 1.8000, 0.1329) -- (2.9000, 1.8000, 0.1267) -- (2.9000, 1.8500, 0.1246) -- (2.8500, 1.8500, 0.1308) -- cycle;
\fill[blue!15.0, opacity=0.7] (2.8500, 1.8500, 0.1308) -- (2.9000, 1.8500, 0.1246) -- (2.9000, 1.9000, 0.1222) -- (2.8500, 1.9000, 0.1284) -- cycle;
\fill[blue!15.0, opacity=0.7] (2.8500, 1.9000, 0.1284) -- (2.9000, 1.9000, 0.1222) -- (2.9000, 1.9500, 0.1195) -- (2.8500, 1.9500, 0.1257) -- cycle;
\fill[blue!15.0, opacity=0.7] (2.8500, 1.9500, 0.1257) -- (2.9000, 1.9500, 0.1195) -- (2.9000, 2.0000, 0.1165) -- (2.8500, 2.0000, 0.1227) -- cycle;
\fill[blue!15.0, opacity=0.7] (2.8500, 2.0000, 0.1227) -- (2.9000, 2.0000, 0.1165) -- (2.9000, 2.0500, 0.1132) -- (2.8500, 2.0500, 0.1194) -- cycle;
\fill[blue!15.0, opacity=0.7] (2.8500, 2.0500, 0.1194) -- (2.9000, 2.0500, 0.1132) -- (2.9000, 2.1000, 0.1096) -- (2.8500, 2.1000, 0.1159) -- cycle;
\fill[blue!15.0, opacity=0.7] (2.8500, 2.1000, 0.1159) -- (2.9000, 2.1000, 0.1096) -- (2.9000, 2.1500, 0.1058) -- (2.8500, 2.1500, 0.1120) -- cycle;
\fill[blue!15.0, opacity=0.7] (2.8500, 2.1500, 0.1120) -- (2.9000, 2.1500, 0.1058) -- (2.9000, 2.2000, 0.1017) -- (2.8500, 2.2000, 0.1079) -- cycle;
\fill[blue!15.0, opacity=0.7] (2.8500, 2.2000, 0.1079) -- (2.9000, 2.2000, 0.1017) -- (2.9000, 2.2500, 0.0974) -- (2.8500, 2.2500, 0.1036) -- cycle;
\fill[blue!15.0, opacity=0.7] (2.8500, 2.2500, 0.1036) -- (2.9000, 2.2500, 0.0974) -- (2.9000, 2.3000, 0.0928) -- (2.8500, 2.3000, 0.0991) -- cycle;
\fill[blue!15.0, opacity=0.7] (2.8500, 2.3000, 0.0991) -- (2.9000, 2.3000, 0.0928) -- (2.9000, 2.3500, 0.0881) -- (2.8500, 2.3500, 0.0943) -- cycle;
\fill[blue!15.0, opacity=0.7] (2.8500, 2.3500, 0.0943) -- (2.9000, 2.3500, 0.0881) -- (2.9000, 2.4000, 0.0831) -- (2.8500, 2.4000, 0.0893) -- cycle;
\fill[blue!15.0, opacity=0.7] (2.8500, 2.4000, 0.0893) -- (2.9000, 2.4000, 0.0831) -- (2.9000, 2.4500, 0.0779) -- (2.8500, 2.4500, 0.0841) -- cycle;
\fill[blue!15.0, opacity=0.7] (2.8500, 2.4500, 0.0841) -- (2.9000, 2.4500, 0.0779) -- (2.9000, 2.5000, 0.0725) -- (2.8500, 2.5000, 0.0788) -- cycle;
\fill[blue!15.0, opacity=0.7] (2.8500, 2.5000, 0.0788) -- (2.9000, 2.5000, 0.0725) -- (2.9000, 2.5500, 0.0670) -- (2.8500, 2.5500, 0.0733) -- cycle;
\fill[blue!15.0, opacity=0.7] (2.8500, 2.5500, 0.0733) -- (2.9000, 2.5500, 0.0670) -- (2.9000, 2.6000, 0.0614) -- (2.8500, 2.6000, 0.0676) -- cycle;
\fill[blue!15.0, opacity=0.7] (2.8500, 2.6000, 0.0676) -- (2.9000, 2.6000, 0.0614) -- (2.9000, 2.6500, 0.0555) -- (2.8500, 2.6500, 0.0618) -- cycle;
\fill[blue!15.0, opacity=0.7] (2.8500, 2.6500, 0.0618) -- (2.9000, 2.6500, 0.0555) -- (2.9000, 2.7000, 0.0496) -- (2.8500, 2.7000, 0.0559) -- cycle;
\fill[blue!15.0, opacity=0.7] (2.8500, 2.7000, 0.0559) -- (2.9000, 2.7000, 0.0496) -- (2.9000, 2.7500, 0.0436) -- (2.8500, 2.7500, 0.0498) -- cycle;
\fill[blue!15.0, opacity=0.7] (2.8500, 2.7500, 0.0498) -- (2.9000, 2.7500, 0.0436) -- (2.9000, 2.8000, 0.0375) -- (2.8500, 2.8000, 0.0437) -- cycle;
\fill[blue!15.0, opacity=0.7] (2.8500, 2.8000, 0.0437) -- (2.9000, 2.8000, 0.0375) -- (2.9000, 2.8500, 0.0313) -- (2.8500, 2.8500, 0.0375) -- cycle;
\fill[blue!15.0, opacity=0.7] (2.8500, 2.8500, 0.0375) -- (2.9000, 2.8500, 0.0313) -- (2.9000, 2.9000, 0.0251) -- (2.8500, 2.9000, 0.0313) -- cycle;
\fill[blue!15.0, opacity=0.7] (2.8500, 2.9000, 0.0313) -- (2.9000, 2.9000, 0.0251) -- (2.9000, 2.9500, 0.0188) -- (2.8500, 2.9500, 0.0251) -- cycle;
\fill[blue!15.0, opacity=0.7] (2.8500, 2.9500, 0.0251) -- (2.9000, 2.9500, 0.0188) -- (2.9000, 3.0000, 0.0125) -- (2.8500, 3.0000, 0.0188) -- cycle;
\fill[blue!15.0, opacity=0.7] (2.9000, 0.0000, 0.0125) -- (2.9500, 0.0000, 0.0063) -- (2.9500, 0.0500, 0.0126) -- (2.9000, 0.0500, 0.0188) -- cycle;
\fill[blue!15.0, opacity=0.7] (2.9000, 0.0500, 0.0188) -- (2.9500, 0.0500, 0.0126) -- (2.9500, 0.1000, 0.0188) -- (2.9000, 0.1000, 0.0251) -- cycle;
\fill[blue!15.0, opacity=0.7] (2.9000, 0.1000, 0.0251) -- (2.9500, 0.1000, 0.0188) -- (2.9500, 0.1500, 0.0251) -- (2.9000, 0.1500, 0.0313) -- cycle;
\fill[blue!15.0, opacity=0.7] (2.9000, 0.1500, 0.0313) -- (2.9500, 0.1500, 0.0251) -- (2.9500, 0.2000, 0.0312) -- (2.9000, 0.2000, 0.0375) -- cycle;
\fill[blue!15.0, opacity=0.7] (2.9000, 0.2000, 0.0375) -- (2.9500, 0.2000, 0.0312) -- (2.9500, 0.2500, 0.0373) -- (2.9000, 0.2500, 0.0436) -- cycle;
\fill[blue!15.0, opacity=0.7] (2.9000, 0.2500, 0.0436) -- (2.9500, 0.2500, 0.0373) -- (2.9500, 0.3000, 0.0434) -- (2.9000, 0.3000, 0.0496) -- cycle;
\fill[blue!15.0, opacity=0.7] (2.9000, 0.3000, 0.0496) -- (2.9500, 0.3000, 0.0434) -- (2.9500, 0.3500, 0.0493) -- (2.9000, 0.3500, 0.0555) -- cycle;
\fill[blue!15.0, opacity=0.7] (2.9000, 0.3500, 0.0555) -- (2.9500, 0.3500, 0.0493) -- (2.9500, 0.4000, 0.0551) -- (2.9000, 0.4000, 0.0614) -- cycle;
\fill[blue!15.0, opacity=0.7] (2.9000, 0.4000, 0.0614) -- (2.9500, 0.4000, 0.0551) -- (2.9500, 0.4500, 0.0608) -- (2.9000, 0.4500, 0.0670) -- cycle;
\fill[blue!15.0, opacity=0.7] (2.9000, 0.4500, 0.0670) -- (2.9500, 0.4500, 0.0608) -- (2.9500, 0.5000, 0.0663) -- (2.9000, 0.5000, 0.0725) -- cycle;
\fill[blue!15.0, opacity=0.7] (2.9000, 0.5000, 0.0725) -- (2.9500, 0.5000, 0.0663) -- (2.9500, 0.5500, 0.0716) -- (2.9000, 0.5500, 0.0779) -- cycle;
\fill[blue!15.0, opacity=0.7] (2.9000, 0.5500, 0.0779) -- (2.9500, 0.5500, 0.0716) -- (2.9500, 0.6000, 0.0768) -- (2.9000, 0.6000, 0.0831) -- cycle;
\fill[blue!15.0, opacity=0.7] (2.9000, 0.6000, 0.0831) -- (2.9500, 0.6000, 0.0768) -- (2.9500, 0.6500, 0.0818) -- (2.9000, 0.6500, 0.0881) -- cycle;
\fill[blue!15.0, opacity=0.7] (2.9000, 0.6500, 0.0881) -- (2.9500, 0.6500, 0.0818) -- (2.9500, 0.7000, 0.0866) -- (2.9000, 0.7000, 0.0928) -- cycle;
\fill[blue!15.0, opacity=0.7] (2.9000, 0.7000, 0.0928) -- (2.9500, 0.7000, 0.0866) -- (2.9500, 0.7500, 0.0911) -- (2.9000, 0.7500, 0.0974) -- cycle;
\fill[blue!15.0, opacity=0.7] (2.9000, 0.7500, 0.0974) -- (2.9500, 0.7500, 0.0911) -- (2.9500, 0.8000, 0.0955) -- (2.9000, 0.8000, 0.1017) -- cycle;
\fill[blue!15.0, opacity=0.7] (2.9000, 0.8000, 0.1017) -- (2.9500, 0.8000, 0.0955) -- (2.9500, 0.8500, 0.0995) -- (2.9000, 0.8500, 0.1058) -- cycle;
\fill[blue!15.0, opacity=0.7] (2.9000, 0.8500, 0.1058) -- (2.9500, 0.8500, 0.0995) -- (2.9500, 0.9000, 0.1034) -- (2.9000, 0.9000, 0.1096) -- cycle;
\fill[blue!15.0, opacity=0.7] (2.9000, 0.9000, 0.1096) -- (2.9500, 0.9000, 0.1034) -- (2.9500, 0.9500, 0.1069) -- (2.9000, 0.9500, 0.1132) -- cycle;
\fill[blue!15.0, opacity=0.7] (2.9000, 0.9500, 0.1132) -- (2.9500, 0.9500, 0.1069) -- (2.9500, 1.0000, 0.1102) -- (2.9000, 1.0000, 0.1165) -- cycle;
\fill[blue!15.0, opacity=0.7] (2.9000, 1.0000, 0.1165) -- (2.9500, 1.0000, 0.1102) -- (2.9500, 1.0500, 0.1132) -- (2.9000, 1.0500, 0.1195) -- cycle;
\fill[blue!15.0, opacity=0.7] (2.9000, 1.0500, 0.1195) -- (2.9500, 1.0500, 0.1132) -- (2.9500, 1.1000, 0.1159) -- (2.9000, 1.1000, 0.1222) -- cycle;
\fill[blue!15.0, opacity=0.7] (2.9000, 1.1000, 0.1222) -- (2.9500, 1.1000, 0.1159) -- (2.9500, 1.1500, 0.1183) -- (2.9000, 1.1500, 0.1246) -- cycle;
\fill[blue!15.0, opacity=0.7] (2.9000, 1.1500, 0.1246) -- (2.9500, 1.1500, 0.1183) -- (2.9500, 1.2000, 0.1204) -- (2.9000, 1.2000, 0.1267) -- cycle;
\fill[blue!15.0, opacity=0.7] (2.9000, 1.2000, 0.1267) -- (2.9500, 1.2000, 0.1204) -- (2.9500, 1.2500, 0.1222) -- (2.9000, 1.2500, 0.1285) -- cycle;
\fill[blue!15.0, opacity=0.7] (2.9000, 1.2500, 0.1285) -- (2.9500, 1.2500, 0.1222) -- (2.9500, 1.3000, 0.1237) -- (2.9000, 1.3000, 0.1299) -- cycle;
\fill[blue!15.0, opacity=0.7] (2.9000, 1.3000, 0.1299) -- (2.9500, 1.3000, 0.1237) -- (2.9500, 1.3500, 0.1248) -- (2.9000, 1.3500, 0.1311) -- cycle;
\fill[blue!15.0, opacity=0.7] (2.9000, 1.3500, 0.1311) -- (2.9500, 1.3500, 0.1248) -- (2.9500, 1.4000, 0.1256) -- (2.9000, 1.4000, 0.1319) -- cycle;
\fill[blue!15.0, opacity=0.7] (2.9000, 1.4000, 0.1319) -- (2.9500, 1.4000, 0.1256) -- (2.9500, 1.4500, 0.1261) -- (2.9000, 1.4500, 0.1324) -- cycle;
\fill[blue!15.0, opacity=0.7] (2.9000, 1.4500, 0.1324) -- (2.9500, 1.4500, 0.1261) -- (2.9500, 1.5000, 0.1263) -- (2.9000, 1.5000, 0.1325) -- cycle;
\fill[blue!15.0, opacity=0.7] (2.9000, 1.5000, 0.1325) -- (2.9500, 1.5000, 0.1263) -- (2.9500, 1.5500, 0.1261) -- (2.9000, 1.5500, 0.1324) -- cycle;
\fill[blue!15.0, opacity=0.7] (2.9000, 1.5500, 0.1324) -- (2.9500, 1.5500, 0.1261) -- (2.9500, 1.6000, 0.1256) -- (2.9000, 1.6000, 0.1319) -- cycle;
\fill[blue!15.0, opacity=0.7] (2.9000, 1.6000, 0.1319) -- (2.9500, 1.6000, 0.1256) -- (2.9500, 1.6500, 0.1248) -- (2.9000, 1.6500, 0.1311) -- cycle;
\fill[blue!15.0, opacity=0.7] (2.9000, 1.6500, 0.1311) -- (2.9500, 1.6500, 0.1248) -- (2.9500, 1.7000, 0.1237) -- (2.9000, 1.7000, 0.1299) -- cycle;
\fill[blue!15.0, opacity=0.7] (2.9000, 1.7000, 0.1299) -- (2.9500, 1.7000, 0.1237) -- (2.9500, 1.7500, 0.1222) -- (2.9000, 1.7500, 0.1285) -- cycle;
\fill[blue!15.0, opacity=0.7] (2.9000, 1.7500, 0.1285) -- (2.9500, 1.7500, 0.1222) -- (2.9500, 1.8000, 0.1204) -- (2.9000, 1.8000, 0.1267) -- cycle;
\fill[blue!15.0, opacity=0.7] (2.9000, 1.8000, 0.1267) -- (2.9500, 1.8000, 0.1204) -- (2.9500, 1.8500, 0.1183) -- (2.9000, 1.8500, 0.1246) -- cycle;
\fill[blue!15.0, opacity=0.7] (2.9000, 1.8500, 0.1246) -- (2.9500, 1.8500, 0.1183) -- (2.9500, 1.9000, 0.1159) -- (2.9000, 1.9000, 0.1222) -- cycle;
\fill[blue!15.0, opacity=0.7] (2.9000, 1.9000, 0.1222) -- (2.9500, 1.9000, 0.1159) -- (2.9500, 1.9500, 0.1132) -- (2.9000, 1.9500, 0.1195) -- cycle;
\fill[blue!15.0, opacity=0.7] (2.9000, 1.9500, 0.1195) -- (2.9500, 1.9500, 0.1132) -- (2.9500, 2.0000, 0.1102) -- (2.9000, 2.0000, 0.1165) -- cycle;
\fill[blue!15.0, opacity=0.7] (2.9000, 2.0000, 0.1165) -- (2.9500, 2.0000, 0.1102) -- (2.9500, 2.0500, 0.1069) -- (2.9000, 2.0500, 0.1132) -- cycle;
\fill[blue!15.0, opacity=0.7] (2.9000, 2.0500, 0.1132) -- (2.9500, 2.0500, 0.1069) -- (2.9500, 2.1000, 0.1034) -- (2.9000, 2.1000, 0.1096) -- cycle;
\fill[blue!15.0, opacity=0.7] (2.9000, 2.1000, 0.1096) -- (2.9500, 2.1000, 0.1034) -- (2.9500, 2.1500, 0.0995) -- (2.9000, 2.1500, 0.1058) -- cycle;
\fill[blue!15.0, opacity=0.7] (2.9000, 2.1500, 0.1058) -- (2.9500, 2.1500, 0.0995) -- (2.9500, 2.2000, 0.0955) -- (2.9000, 2.2000, 0.1017) -- cycle;
\fill[blue!15.0, opacity=0.7] (2.9000, 2.2000, 0.1017) -- (2.9500, 2.2000, 0.0955) -- (2.9500, 2.2500, 0.0911) -- (2.9000, 2.2500, 0.0974) -- cycle;
\fill[blue!15.0, opacity=0.7] (2.9000, 2.2500, 0.0974) -- (2.9500, 2.2500, 0.0911) -- (2.9500, 2.3000, 0.0866) -- (2.9000, 2.3000, 0.0928) -- cycle;
\fill[blue!15.0, opacity=0.7] (2.9000, 2.3000, 0.0928) -- (2.9500, 2.3000, 0.0866) -- (2.9500, 2.3500, 0.0818) -- (2.9000, 2.3500, 0.0881) -- cycle;
\fill[blue!15.0, opacity=0.7] (2.9000, 2.3500, 0.0881) -- (2.9500, 2.3500, 0.0818) -- (2.9500, 2.4000, 0.0768) -- (2.9000, 2.4000, 0.0831) -- cycle;
\fill[blue!15.0, opacity=0.7] (2.9000, 2.4000, 0.0831) -- (2.9500, 2.4000, 0.0768) -- (2.9500, 2.4500, 0.0716) -- (2.9000, 2.4500, 0.0779) -- cycle;
\fill[blue!15.0, opacity=0.7] (2.9000, 2.4500, 0.0779) -- (2.9500, 2.4500, 0.0716) -- (2.9500, 2.5000, 0.0663) -- (2.9000, 2.5000, 0.0725) -- cycle;
\fill[blue!15.0, opacity=0.7] (2.9000, 2.5000, 0.0725) -- (2.9500, 2.5000, 0.0663) -- (2.9500, 2.5500, 0.0608) -- (2.9000, 2.5500, 0.0670) -- cycle;
\fill[blue!15.0, opacity=0.7] (2.9000, 2.5500, 0.0670) -- (2.9500, 2.5500, 0.0608) -- (2.9500, 2.6000, 0.0551) -- (2.9000, 2.6000, 0.0614) -- cycle;
\fill[blue!15.0, opacity=0.7] (2.9000, 2.6000, 0.0614) -- (2.9500, 2.6000, 0.0551) -- (2.9500, 2.6500, 0.0493) -- (2.9000, 2.6500, 0.0555) -- cycle;
\fill[blue!15.0, opacity=0.7] (2.9000, 2.6500, 0.0555) -- (2.9500, 2.6500, 0.0493) -- (2.9500, 2.7000, 0.0434) -- (2.9000, 2.7000, 0.0496) -- cycle;
\fill[blue!15.0, opacity=0.7] (2.9000, 2.7000, 0.0496) -- (2.9500, 2.7000, 0.0434) -- (2.9500, 2.7500, 0.0373) -- (2.9000, 2.7500, 0.0436) -- cycle;
\fill[blue!15.0, opacity=0.7] (2.9000, 2.7500, 0.0436) -- (2.9500, 2.7500, 0.0373) -- (2.9500, 2.8000, 0.0312) -- (2.9000, 2.8000, 0.0375) -- cycle;
\fill[blue!15.0, opacity=0.7] (2.9000, 2.8000, 0.0375) -- (2.9500, 2.8000, 0.0312) -- (2.9500, 2.8500, 0.0251) -- (2.9000, 2.8500, 0.0313) -- cycle;
\fill[blue!15.0, opacity=0.7] (2.9000, 2.8500, 0.0313) -- (2.9500, 2.8500, 0.0251) -- (2.9500, 2.9000, 0.0188) -- (2.9000, 2.9000, 0.0251) -- cycle;
\fill[blue!15.0, opacity=0.7] (2.9000, 2.9000, 0.0251) -- (2.9500, 2.9000, 0.0188) -- (2.9500, 2.9500, 0.0126) -- (2.9000, 2.9500, 0.0188) -- cycle;
\fill[blue!15.0, opacity=0.7] (2.9000, 2.9500, 0.0188) -- (2.9500, 2.9500, 0.0126) -- (2.9500, 3.0000, 0.0063) -- (2.9000, 3.0000, 0.0125) -- cycle;
\fill[blue!15.0, opacity=0.7] (2.9500, 0.0000, 0.0063) -- (3.0000, 0.0000, 0.0000) -- (3.0000, 0.0500, 0.0063) -- (2.9500, 0.0500, 0.0126) -- cycle;
\fill[blue!15.0, opacity=0.7] (2.9500, 0.0500, 0.0126) -- (3.0000, 0.0500, 0.0063) -- (3.0000, 0.1000, 0.0125) -- (2.9500, 0.1000, 0.0188) -- cycle;
\fill[blue!15.0, opacity=0.7] (2.9500, 0.1000, 0.0188) -- (3.0000, 0.1000, 0.0125) -- (3.0000, 0.1500, 0.0188) -- (2.9500, 0.1500, 0.0251) -- cycle;
\fill[blue!15.0, opacity=0.7] (2.9500, 0.1500, 0.0251) -- (3.0000, 0.1500, 0.0188) -- (3.0000, 0.2000, 0.0249) -- (2.9500, 0.2000, 0.0312) -- cycle;
\fill[blue!15.0, opacity=0.7] (2.9500, 0.2000, 0.0312) -- (3.0000, 0.2000, 0.0249) -- (3.0000, 0.2500, 0.0311) -- (2.9500, 0.2500, 0.0373) -- cycle;
\fill[blue!15.0, opacity=0.7] (2.9500, 0.2500, 0.0373) -- (3.0000, 0.2500, 0.0311) -- (3.0000, 0.3000, 0.0371) -- (2.9500, 0.3000, 0.0434) -- cycle;
\fill[blue!15.0, opacity=0.7] (2.9500, 0.3000, 0.0434) -- (3.0000, 0.3000, 0.0371) -- (3.0000, 0.3500, 0.0430) -- (2.9500, 0.3500, 0.0493) -- cycle;
\fill[blue!15.0, opacity=0.7] (2.9500, 0.3500, 0.0493) -- (3.0000, 0.3500, 0.0430) -- (3.0000, 0.4000, 0.0488) -- (2.9500, 0.4000, 0.0551) -- cycle;
\fill[blue!15.0, opacity=0.7] (2.9500, 0.4000, 0.0551) -- (3.0000, 0.4000, 0.0488) -- (3.0000, 0.4500, 0.0545) -- (2.9500, 0.4500, 0.0608) -- cycle;
\fill[blue!15.0, opacity=0.7] (2.9500, 0.4500, 0.0608) -- (3.0000, 0.4500, 0.0545) -- (3.0000, 0.5000, 0.0600) -- (2.9500, 0.5000, 0.0663) -- cycle;
\fill[blue!15.0, opacity=0.7] (2.9500, 0.5000, 0.0663) -- (3.0000, 0.5000, 0.0600) -- (3.0000, 0.5500, 0.0654) -- (2.9500, 0.5500, 0.0716) -- cycle;
\fill[blue!15.0, opacity=0.7] (2.9500, 0.5500, 0.0716) -- (3.0000, 0.5500, 0.0654) -- (3.0000, 0.6000, 0.0705) -- (2.9500, 0.6000, 0.0768) -- cycle;
\fill[blue!15.0, opacity=0.7] (2.9500, 0.6000, 0.0768) -- (3.0000, 0.6000, 0.0705) -- (3.0000, 0.6500, 0.0755) -- (2.9500, 0.6500, 0.0818) -- cycle;
\fill[blue!15.0, opacity=0.7] (2.9500, 0.6500, 0.0818) -- (3.0000, 0.6500, 0.0755) -- (3.0000, 0.7000, 0.0803) -- (2.9500, 0.7000, 0.0866) -- cycle;
\fill[blue!15.0, opacity=0.7] (2.9500, 0.7000, 0.0866) -- (3.0000, 0.7000, 0.0803) -- (3.0000, 0.7500, 0.0849) -- (2.9500, 0.7500, 0.0911) -- cycle;
\fill[blue!15.0, opacity=0.7] (2.9500, 0.7500, 0.0911) -- (3.0000, 0.7500, 0.0849) -- (3.0000, 0.8000, 0.0892) -- (2.9500, 0.8000, 0.0955) -- cycle;
\fill[blue!15.0, opacity=0.7] (2.9500, 0.8000, 0.0955) -- (3.0000, 0.8000, 0.0892) -- (3.0000, 0.8500, 0.0933) -- (2.9500, 0.8500, 0.0995) -- cycle;
\fill[blue!15.0, opacity=0.7] (2.9500, 0.8500, 0.0995) -- (3.0000, 0.8500, 0.0933) -- (3.0000, 0.9000, 0.0971) -- (2.9500, 0.9000, 0.1034) -- cycle;
\fill[blue!15.0, opacity=0.7] (2.9500, 0.9000, 0.1034) -- (3.0000, 0.9000, 0.0971) -- (3.0000, 0.9500, 0.1006) -- (2.9500, 0.9500, 0.1069) -- cycle;
\fill[blue!15.0, opacity=0.7] (2.9500, 0.9500, 0.1069) -- (3.0000, 0.9500, 0.1006) -- (3.0000, 1.0000, 0.1039) -- (2.9500, 1.0000, 0.1102) -- cycle;
\fill[blue!15.0, opacity=0.7] (2.9500, 1.0000, 0.1102) -- (3.0000, 1.0000, 0.1039) -- (3.0000, 1.0500, 0.1069) -- (2.9500, 1.0500, 0.1132) -- cycle;
\fill[blue!15.0, opacity=0.7] (2.9500, 1.0500, 0.1132) -- (3.0000, 1.0500, 0.1069) -- (3.0000, 1.1000, 0.1096) -- (2.9500, 1.1000, 0.1159) -- cycle;
\fill[blue!15.0, opacity=0.7] (2.9500, 1.1000, 0.1159) -- (3.0000, 1.1000, 0.1096) -- (3.0000, 1.1500, 0.1120) -- (2.9500, 1.1500, 0.1183) -- cycle;
\fill[blue!15.0, opacity=0.7] (2.9500, 1.1500, 0.1183) -- (3.0000, 1.1500, 0.1120) -- (3.0000, 1.2000, 0.1141) -- (2.9500, 1.2000, 0.1204) -- cycle;
\fill[blue!15.0, opacity=0.7] (2.9500, 1.2000, 0.1204) -- (3.0000, 1.2000, 0.1141) -- (3.0000, 1.2500, 0.1159) -- (2.9500, 1.2500, 0.1222) -- cycle;
\fill[blue!15.0, opacity=0.7] (2.9500, 1.2500, 0.1222) -- (3.0000, 1.2500, 0.1159) -- (3.0000, 1.3000, 0.1174) -- (2.9500, 1.3000, 0.1237) -- cycle;
\fill[blue!15.0, opacity=0.7] (2.9500, 1.3000, 0.1237) -- (3.0000, 1.3000, 0.1174) -- (3.0000, 1.3500, 0.1185) -- (2.9500, 1.3500, 0.1248) -- cycle;
\fill[blue!15.0, opacity=0.7] (2.9500, 1.3500, 0.1248) -- (3.0000, 1.3500, 0.1185) -- (3.0000, 1.4000, 0.1193) -- (2.9500, 1.4000, 0.1256) -- cycle;
\fill[blue!15.0, opacity=0.7] (2.9500, 1.4000, 0.1256) -- (3.0000, 1.4000, 0.1193) -- (3.0000, 1.4500, 0.1198) -- (2.9500, 1.4500, 0.1261) -- cycle;
\fill[blue!15.0, opacity=0.7] (2.9500, 1.4500, 0.1261) -- (3.0000, 1.4500, 0.1198) -- (3.0000, 1.5000, 0.1200) -- (2.9500, 1.5000, 0.1263) -- cycle;
\fill[blue!15.0, opacity=0.7] (2.9500, 1.5000, 0.1263) -- (3.0000, 1.5000, 0.1200) -- (3.0000, 1.5500, 0.1198) -- (2.9500, 1.5500, 0.1261) -- cycle;
\fill[blue!15.0, opacity=0.7] (2.9500, 1.5500, 0.1261) -- (3.0000, 1.5500, 0.1198) -- (3.0000, 1.6000, 0.1193) -- (2.9500, 1.6000, 0.1256) -- cycle;
\fill[blue!15.0, opacity=0.7] (2.9500, 1.6000, 0.1256) -- (3.0000, 1.6000, 0.1193) -- (3.0000, 1.6500, 0.1185) -- (2.9500, 1.6500, 0.1248) -- cycle;
\fill[blue!15.0, opacity=0.7] (2.9500, 1.6500, 0.1248) -- (3.0000, 1.6500, 0.1185) -- (3.0000, 1.7000, 0.1174) -- (2.9500, 1.7000, 0.1237) -- cycle;
\fill[blue!15.0, opacity=0.7] (2.9500, 1.7000, 0.1237) -- (3.0000, 1.7000, 0.1174) -- (3.0000, 1.7500, 0.1159) -- (2.9500, 1.7500, 0.1222) -- cycle;
\fill[blue!15.0, opacity=0.7] (2.9500, 1.7500, 0.1222) -- (3.0000, 1.7500, 0.1159) -- (3.0000, 1.8000, 0.1141) -- (2.9500, 1.8000, 0.1204) -- cycle;
\fill[blue!15.0, opacity=0.7] (2.9500, 1.8000, 0.1204) -- (3.0000, 1.8000, 0.1141) -- (3.0000, 1.8500, 0.1120) -- (2.9500, 1.8500, 0.1183) -- cycle;
\fill[blue!15.0, opacity=0.7] (2.9500, 1.8500, 0.1183) -- (3.0000, 1.8500, 0.1120) -- (3.0000, 1.9000, 0.1096) -- (2.9500, 1.9000, 0.1159) -- cycle;
\fill[blue!15.0, opacity=0.7] (2.9500, 1.9000, 0.1159) -- (3.0000, 1.9000, 0.1096) -- (3.0000, 1.9500, 0.1069) -- (2.9500, 1.9500, 0.1132) -- cycle;
\fill[blue!15.0, opacity=0.7] (2.9500, 1.9500, 0.1132) -- (3.0000, 1.9500, 0.1069) -- (3.0000, 2.0000, 0.1039) -- (2.9500, 2.0000, 0.1102) -- cycle;
\fill[blue!15.0, opacity=0.7] (2.9500, 2.0000, 0.1102) -- (3.0000, 2.0000, 0.1039) -- (3.0000, 2.0500, 0.1006) -- (2.9500, 2.0500, 0.1069) -- cycle;
\fill[blue!15.0, opacity=0.7] (2.9500, 2.0500, 0.1069) -- (3.0000, 2.0500, 0.1006) -- (3.0000, 2.1000, 0.0971) -- (2.9500, 2.1000, 0.1034) -- cycle;
\fill[blue!15.0, opacity=0.7] (2.9500, 2.1000, 0.1034) -- (3.0000, 2.1000, 0.0971) -- (3.0000, 2.1500, 0.0933) -- (2.9500, 2.1500, 0.0995) -- cycle;
\fill[blue!15.0, opacity=0.7] (2.9500, 2.1500, 0.0995) -- (3.0000, 2.1500, 0.0933) -- (3.0000, 2.2000, 0.0892) -- (2.9500, 2.2000, 0.0955) -- cycle;
\fill[blue!15.0, opacity=0.7] (2.9500, 2.2000, 0.0955) -- (3.0000, 2.2000, 0.0892) -- (3.0000, 2.2500, 0.0849) -- (2.9500, 2.2500, 0.0911) -- cycle;
\fill[blue!15.0, opacity=0.7] (2.9500, 2.2500, 0.0911) -- (3.0000, 2.2500, 0.0849) -- (3.0000, 2.3000, 0.0803) -- (2.9500, 2.3000, 0.0866) -- cycle;
\fill[blue!15.0, opacity=0.7] (2.9500, 2.3000, 0.0866) -- (3.0000, 2.3000, 0.0803) -- (3.0000, 2.3500, 0.0755) -- (2.9500, 2.3500, 0.0818) -- cycle;
\fill[blue!15.0, opacity=0.7] (2.9500, 2.3500, 0.0818) -- (3.0000, 2.3500, 0.0755) -- (3.0000, 2.4000, 0.0705) -- (2.9500, 2.4000, 0.0768) -- cycle;
\fill[blue!15.0, opacity=0.7] (2.9500, 2.4000, 0.0768) -- (3.0000, 2.4000, 0.0705) -- (3.0000, 2.4500, 0.0654) -- (2.9500, 2.4500, 0.0716) -- cycle;
\fill[blue!15.0, opacity=0.7] (2.9500, 2.4500, 0.0716) -- (3.0000, 2.4500, 0.0654) -- (3.0000, 2.5000, 0.0600) -- (2.9500, 2.5000, 0.0663) -- cycle;
\fill[blue!15.0, opacity=0.7] (2.9500, 2.5000, 0.0663) -- (3.0000, 2.5000, 0.0600) -- (3.0000, 2.5500, 0.0545) -- (2.9500, 2.5500, 0.0608) -- cycle;
\fill[blue!15.0, opacity=0.7] (2.9500, 2.5500, 0.0608) -- (3.0000, 2.5500, 0.0545) -- (3.0000, 2.6000, 0.0488) -- (2.9500, 2.6000, 0.0551) -- cycle;
\fill[blue!15.0, opacity=0.7] (2.9500, 2.6000, 0.0551) -- (3.0000, 2.6000, 0.0488) -- (3.0000, 2.6500, 0.0430) -- (2.9500, 2.6500, 0.0493) -- cycle;
\fill[blue!15.0, opacity=0.7] (2.9500, 2.6500, 0.0493) -- (3.0000, 2.6500, 0.0430) -- (3.0000, 2.7000, 0.0371) -- (2.9500, 2.7000, 0.0434) -- cycle;
\fill[blue!15.0, opacity=0.7] (2.9500, 2.7000, 0.0434) -- (3.0000, 2.7000, 0.0371) -- (3.0000, 2.7500, 0.0311) -- (2.9500, 2.7500, 0.0373) -- cycle;
\fill[blue!15.0, opacity=0.7] (2.9500, 2.7500, 0.0373) -- (3.0000, 2.7500, 0.0311) -- (3.0000, 2.8000, 0.0249) -- (2.9500, 2.8000, 0.0312) -- cycle;
\fill[blue!15.0, opacity=0.7] (2.9500, 2.8000, 0.0312) -- (3.0000, 2.8000, 0.0249) -- (3.0000, 2.8500, 0.0188) -- (2.9500, 2.8500, 0.0251) -- cycle;
\fill[blue!15.0, opacity=0.7] (2.9500, 2.8500, 0.0251) -- (3.0000, 2.8500, 0.0188) -- (3.0000, 2.9000, 0.0125) -- (2.9500, 2.9000, 0.0188) -- cycle;
\fill[blue!15.0, opacity=0.7] (2.9500, 2.9000, 0.0188) -- (3.0000, 2.9000, 0.0125) -- (3.0000, 2.9500, 0.0063) -- (2.9500, 2.9500, 0.0126) -- cycle;
\fill[blue!15.0, opacity=0.7] (2.9500, 2.9500, 0.0126) -- (3.0000, 2.9500, 0.0063) -- (3.0000, 3.0000, 0.0000) -- (2.9500, 3.0000, 0.0063) -- cycle;
% Slice 1 horizontal patches
\fill[blue!15.0, opacity=0.7] (0.0300, -0.0300, 1.0000) -- (0.0790, -0.0300, 1.0063) -- (0.0790, 0.0210, 1.0126) -- (0.0300, 0.0210, 1.0063) -- cycle;
\fill[blue!15.0, opacity=0.7] (0.0300, 0.0210, 1.0063) -- (0.0790, 0.0210, 1.0126) -- (0.0790, 0.0720, 1.0188) -- (0.0300, 0.0720, 1.0125) -- cycle;
\fill[blue!15.0, opacity=0.7] (0.0300, 0.0720, 1.0125) -- (0.0790, 0.0720, 1.0188) -- (0.0790, 0.1230, 1.0251) -- (0.0300, 0.1230, 1.0188) -- cycle;
\fill[blue!15.0, opacity=0.7] (0.0300, 0.1230, 1.0188) -- (0.0790, 0.1230, 1.0251) -- (0.0790, 0.1740, 1.0312) -- (0.0300, 0.1740, 1.0249) -- cycle;
\fill[blue!15.0, opacity=0.7] (0.0300, 0.1740, 1.0249) -- (0.0790, 0.1740, 1.0312) -- (0.0790, 0.2250, 1.0373) -- (0.0300, 0.2250, 1.0311) -- cycle;
\fill[blue!15.0, opacity=0.7] (0.0300, 0.2250, 1.0311) -- (0.0790, 0.2250, 1.0373) -- (0.0790, 0.2760, 1.0434) -- (0.0300, 0.2760, 1.0371) -- cycle;
\fill[blue!15.0, opacity=0.7] (0.0300, 0.2760, 1.0371) -- (0.0790, 0.2760, 1.0434) -- (0.0790, 0.3270, 1.0493) -- (0.0300, 0.3270, 1.0430) -- cycle;
\fill[blue!15.0, opacity=0.7] (0.0300, 0.3270, 1.0430) -- (0.0790, 0.3270, 1.0493) -- (0.0790, 0.3780, 1.0551) -- (0.0300, 0.3780, 1.0488) -- cycle;
\fill[blue!15.0, opacity=0.7] (0.0300, 0.3780, 1.0488) -- (0.0790, 0.3780, 1.0551) -- (0.0790, 0.4290, 1.0608) -- (0.0300, 0.4290, 1.0545) -- cycle;
\fill[blue!15.0, opacity=0.7] (0.0300, 0.4290, 1.0545) -- (0.0790, 0.4290, 1.0608) -- (0.0790, 0.4800, 1.0663) -- (0.0300, 0.4800, 1.0600) -- cycle;
\fill[blue!15.0, opacity=0.7] (0.0300, 0.4800, 1.0600) -- (0.0790, 0.4800, 1.0663) -- (0.0790, 0.5310, 1.0716) -- (0.0300, 0.5310, 1.0654) -- cycle;
\fill[blue!15.0, opacity=0.7] (0.0300, 0.5310, 1.0654) -- (0.0790, 0.5310, 1.0716) -- (0.0790, 0.5820, 1.0768) -- (0.0300, 0.5820, 1.0705) -- cycle;
\fill[blue!15.1, opacity=0.7] (0.0300, 0.5820, 1.0705) -- (0.0790, 0.5820, 1.0768) -- (0.0790, 0.6330, 1.0818) -- (0.0300, 0.6330, 1.0755) -- cycle;
\fill[blue!15.1, opacity=0.7] (0.0300, 0.6330, 1.0755) -- (0.0790, 0.6330, 1.0818) -- (0.0790, 0.6840, 1.0866) -- (0.0300, 0.6840, 1.0803) -- cycle;
\fill[blue!15.0, opacity=0.7] (0.0300, 0.6840, 1.0803) -- (0.0790, 0.6840, 1.0866) -- (0.0790, 0.7350, 1.0911) -- (0.0300, 0.7350, 1.0849) -- cycle;
\fill[blue!15.0, opacity=0.7] (0.0300, 0.7350, 1.0849) -- (0.0790, 0.7350, 1.0911) -- (0.0790, 0.7860, 1.0955) -- (0.0300, 0.7860, 1.0892) -- cycle;
\fill[blue!15.0, opacity=0.7] (0.0300, 0.7860, 1.0892) -- (0.0790, 0.7860, 1.0955) -- (0.0790, 0.8370, 1.0995) -- (0.0300, 0.8370, 1.0933) -- cycle;
\fill[blue!15.0, opacity=0.7] (0.0300, 0.8370, 1.0933) -- (0.0790, 0.8370, 1.0995) -- (0.0790, 0.8880, 1.1034) -- (0.0300, 0.8880, 1.0971) -- cycle;
\fill[blue!15.0, opacity=0.7] (0.0300, 0.8880, 1.0971) -- (0.0790, 0.8880, 1.1034) -- (0.0790, 0.9390, 1.1069) -- (0.0300, 0.9390, 1.1006) -- cycle;
\fill[blue!15.0, opacity=0.7] (0.0300, 0.9390, 1.1006) -- (0.0790, 0.9390, 1.1069) -- (0.0790, 0.9900, 1.1102) -- (0.0300, 0.9900, 1.1039) -- cycle;
\fill[blue!15.0, opacity=0.7] (0.0300, 0.9900, 1.1039) -- (0.0790, 0.9900, 1.1102) -- (0.0790, 1.0410, 1.1132) -- (0.0300, 1.0410, 1.1069) -- cycle;
\fill[blue!15.0, opacity=0.7] (0.0300, 1.0410, 1.1069) -- (0.0790, 1.0410, 1.1132) -- (0.0790, 1.0920, 1.1159) -- (0.0300, 1.0920, 1.1096) -- cycle;
\fill[blue!15.0, opacity=0.7] (0.0300, 1.0920, 1.1096) -- (0.0790, 1.0920, 1.1159) -- (0.0790, 1.1430, 1.1183) -- (0.0300, 1.1430, 1.1120) -- cycle;
\fill[blue!15.0, opacity=0.7] (0.0300, 1.1430, 1.1120) -- (0.0790, 1.1430, 1.1183) -- (0.0790, 1.1940, 1.1204) -- (0.0300, 1.1940, 1.1141) -- cycle;
\fill[blue!15.0, opacity=0.7] (0.0300, 1.1940, 1.1141) -- (0.0790, 1.1940, 1.1204) -- (0.0790, 1.2450, 1.1222) -- (0.0300, 1.2450, 1.1159) -- cycle;
\fill[blue!15.0, opacity=0.7] (0.0300, 1.2450, 1.1159) -- (0.0790, 1.2450, 1.1222) -- (0.0790, 1.2960, 1.1237) -- (0.0300, 1.2960, 1.1174) -- cycle;
\fill[blue!15.0, opacity=0.7] (0.0300, 1.2960, 1.1174) -- (0.0790, 1.2960, 1.1237) -- (0.0790, 1.3470, 1.1248) -- (0.0300, 1.3470, 1.1185) -- cycle;
\fill[blue!15.0, opacity=0.7] (0.0300, 1.3470, 1.1185) -- (0.0790, 1.3470, 1.1248) -- (0.0790, 1.3980, 1.1256) -- (0.0300, 1.3980, 1.1193) -- cycle;
\fill[blue!15.0, opacity=0.7] (0.0300, 1.3980, 1.1193) -- (0.0790, 1.3980, 1.1256) -- (0.0790, 1.4490, 1.1261) -- (0.0300, 1.4490, 1.1198) -- cycle;
\fill[blue!15.0, opacity=0.7] (0.0300, 1.4490, 1.1198) -- (0.0790, 1.4490, 1.1261) -- (0.0790, 1.5000, 1.1263) -- (0.0300, 1.5000, 1.1200) -- cycle;
\fill[blue!15.0, opacity=0.7] (0.0300, 1.5000, 1.1200) -- (0.0790, 1.5000, 1.1263) -- (0.0790, 1.5510, 1.1261) -- (0.0300, 1.5510, 1.1198) -- cycle;
\fill[blue!15.0, opacity=0.7] (0.0300, 1.5510, 1.1198) -- (0.0790, 1.5510, 1.1261) -- (0.0790, 1.6020, 1.1256) -- (0.0300, 1.6020, 1.1193) -- cycle;
\fill[blue!15.0, opacity=0.7] (0.0300, 1.6020, 1.1193) -- (0.0790, 1.6020, 1.1256) -- (0.0790, 1.6530, 1.1248) -- (0.0300, 1.6530, 1.1185) -- cycle;
\fill[blue!15.0, opacity=0.7] (0.0300, 1.6530, 1.1185) -- (0.0790, 1.6530, 1.1248) -- (0.0790, 1.7040, 1.1237) -- (0.0300, 1.7040, 1.1174) -- cycle;
\fill[blue!15.0, opacity=0.7] (0.0300, 1.7040, 1.1174) -- (0.0790, 1.7040, 1.1237) -- (0.0790, 1.7550, 1.1222) -- (0.0300, 1.7550, 1.1159) -- cycle;
\fill[blue!15.0, opacity=0.7] (0.0300, 1.7550, 1.1159) -- (0.0790, 1.7550, 1.1222) -- (0.0790, 1.8060, 1.1204) -- (0.0300, 1.8060, 1.1141) -- cycle;
\fill[blue!15.0, opacity=0.7] (0.0300, 1.8060, 1.1141) -- (0.0790, 1.8060, 1.1204) -- (0.0790, 1.8570, 1.1183) -- (0.0300, 1.8570, 1.1120) -- cycle;
\fill[blue!15.0, opacity=0.7] (0.0300, 1.8570, 1.1120) -- (0.0790, 1.8570, 1.1183) -- (0.0790, 1.9080, 1.1159) -- (0.0300, 1.9080, 1.1096) -- cycle;
\fill[blue!15.0, opacity=0.7] (0.0300, 1.9080, 1.1096) -- (0.0790, 1.9080, 1.1159) -- (0.0790, 1.9590, 1.1132) -- (0.0300, 1.9590, 1.1069) -- cycle;
\fill[blue!15.0, opacity=0.7] (0.0300, 1.9590, 1.1069) -- (0.0790, 1.9590, 1.1132) -- (0.0790, 2.0100, 1.1102) -- (0.0300, 2.0100, 1.1039) -- cycle;
\fill[blue!15.0, opacity=0.7] (0.0300, 2.0100, 1.1039) -- (0.0790, 2.0100, 1.1102) -- (0.0790, 2.0610, 1.1069) -- (0.0300, 2.0610, 1.1006) -- cycle;
\fill[blue!15.0, opacity=0.7] (0.0300, 2.0610, 1.1006) -- (0.0790, 2.0610, 1.1069) -- (0.0790, 2.1120, 1.1034) -- (0.0300, 2.1120, 1.0971) -- cycle;
\fill[blue!15.0, opacity=0.7] (0.0300, 2.1120, 1.0971) -- (0.0790, 2.1120, 1.1034) -- (0.0790, 2.1630, 1.0995) -- (0.0300, 2.1630, 1.0933) -- cycle;
\fill[blue!15.0, opacity=0.7] (0.0300, 2.1630, 1.0933) -- (0.0790, 2.1630, 1.0995) -- (0.0790, 2.2140, 1.0955) -- (0.0300, 2.2140, 1.0892) -- cycle;
\fill[blue!15.0, opacity=0.7] (0.0300, 2.2140, 1.0892) -- (0.0790, 2.2140, 1.0955) -- (0.0790, 2.2650, 1.0911) -- (0.0300, 2.2650, 1.0849) -- cycle;
\fill[blue!15.0, opacity=0.7] (0.0300, 2.2650, 1.0849) -- (0.0790, 2.2650, 1.0911) -- (0.0790, 2.3160, 1.0866) -- (0.0300, 2.3160, 1.0803) -- cycle;
\fill[blue!15.0, opacity=0.7] (0.0300, 2.3160, 1.0803) -- (0.0790, 2.3160, 1.0866) -- (0.0790, 2.3670, 1.0818) -- (0.0300, 2.3670, 1.0755) -- cycle;
\fill[blue!15.0, opacity=0.7] (0.0300, 2.3670, 1.0755) -- (0.0790, 2.3670, 1.0818) -- (0.0790, 2.4180, 1.0768) -- (0.0300, 2.4180, 1.0705) -- cycle;
\fill[blue!15.0, opacity=0.7] (0.0300, 2.4180, 1.0705) -- (0.0790, 2.4180, 1.0768) -- (0.0790, 2.4690, 1.0716) -- (0.0300, 2.4690, 1.0654) -- cycle;
\fill[blue!15.0, opacity=0.7] (0.0300, 2.4690, 1.0654) -- (0.0790, 2.4690, 1.0716) -- (0.0790, 2.5200, 1.0663) -- (0.0300, 2.5200, 1.0600) -- cycle;
\fill[blue!15.0, opacity=0.7] (0.0300, 2.5200, 1.0600) -- (0.0790, 2.5200, 1.0663) -- (0.0790, 2.5710, 1.0608) -- (0.0300, 2.5710, 1.0545) -- cycle;
\fill[blue!15.0, opacity=0.7] (0.0300, 2.5710, 1.0545) -- (0.0790, 2.5710, 1.0608) -- (0.0790, 2.6220, 1.0551) -- (0.0300, 2.6220, 1.0488) -- cycle;
\fill[blue!15.0, opacity=0.7] (0.0300, 2.6220, 1.0488) -- (0.0790, 2.6220, 1.0551) -- (0.0790, 2.6730, 1.0493) -- (0.0300, 2.6730, 1.0430) -- cycle;
\fill[blue!15.0, opacity=0.7] (0.0300, 2.6730, 1.0430) -- (0.0790, 2.6730, 1.0493) -- (0.0790, 2.7240, 1.0434) -- (0.0300, 2.7240, 1.0371) -- cycle;
\fill[blue!15.0, opacity=0.7] (0.0300, 2.7240, 1.0371) -- (0.0790, 2.7240, 1.0434) -- (0.0790, 2.7750, 1.0373) -- (0.0300, 2.7750, 1.0311) -- cycle;
\fill[blue!15.0, opacity=0.7] (0.0300, 2.7750, 1.0311) -- (0.0790, 2.7750, 1.0373) -- (0.0790, 2.8260, 1.0312) -- (0.0300, 2.8260, 1.0249) -- cycle;
\fill[blue!15.0, opacity=0.7] (0.0300, 2.8260, 1.0249) -- (0.0790, 2.8260, 1.0312) -- (0.0790, 2.8770, 1.0251) -- (0.0300, 2.8770, 1.0188) -- cycle;
\fill[blue!15.0, opacity=0.7] (0.0300, 2.8770, 1.0188) -- (0.0790, 2.8770, 1.0251) -- (0.0790, 2.9280, 1.0188) -- (0.0300, 2.9280, 1.0125) -- cycle;
\fill[blue!15.0, opacity=0.7] (0.0300, 2.9280, 1.0125) -- (0.0790, 2.9280, 1.0188) -- (0.0790, 2.9790, 1.0126) -- (0.0300, 2.9790, 1.0063) -- cycle;
\fill[blue!15.0, opacity=0.7] (0.0300, 2.9790, 1.0063) -- (0.0790, 2.9790, 1.0126) -- (0.0790, 3.0300, 1.0063) -- (0.0300, 3.0300, 1.0000) -- cycle;
\fill[blue!15.0, opacity=0.7] (0.0790, -0.0300, 1.0063) -- (0.1280, -0.0300, 1.0125) -- (0.1280, 0.0210, 1.0188) -- (0.0790, 0.0210, 1.0126) -- cycle;
\fill[blue!15.0, opacity=0.7] (0.0790, 0.0210, 1.0126) -- (0.1280, 0.0210, 1.0188) -- (0.1280, 0.0720, 1.0251) -- (0.0790, 0.0720, 1.0188) -- cycle;
\fill[blue!15.0, opacity=0.7] (0.0790, 0.0720, 1.0188) -- (0.1280, 0.0720, 1.0251) -- (0.1280, 0.1230, 1.0313) -- (0.0790, 0.1230, 1.0251) -- cycle;
\fill[blue!15.0, opacity=0.7] (0.0790, 0.1230, 1.0251) -- (0.1280, 0.1230, 1.0313) -- (0.1280, 0.1740, 1.0375) -- (0.0790, 0.1740, 1.0312) -- cycle;
\fill[blue!15.0, opacity=0.7] (0.0790, 0.1740, 1.0312) -- (0.1280, 0.1740, 1.0375) -- (0.1280, 0.2250, 1.0436) -- (0.0790, 0.2250, 1.0373) -- cycle;
\fill[blue!15.0, opacity=0.7] (0.0790, 0.2250, 1.0373) -- (0.1280, 0.2250, 1.0436) -- (0.1280, 0.2760, 1.0496) -- (0.0790, 0.2760, 1.0434) -- cycle;
\fill[blue!15.0, opacity=0.7] (0.0790, 0.2760, 1.0434) -- (0.1280, 0.2760, 1.0496) -- (0.1280, 0.3270, 1.0555) -- (0.0790, 0.3270, 1.0493) -- cycle;
\fill[blue!15.0, opacity=0.7] (0.0790, 0.3270, 1.0493) -- (0.1280, 0.3270, 1.0555) -- (0.1280, 0.3780, 1.0614) -- (0.0790, 0.3780, 1.0551) -- cycle;
\fill[blue!15.0, opacity=0.7] (0.0790, 0.3780, 1.0551) -- (0.1280, 0.3780, 1.0614) -- (0.1280, 0.4290, 1.0670) -- (0.0790, 0.4290, 1.0608) -- cycle;
\fill[blue!15.0, opacity=0.7] (0.0790, 0.4290, 1.0608) -- (0.1280, 0.4290, 1.0670) -- (0.1280, 0.4800, 1.0725) -- (0.0790, 0.4800, 1.0663) -- cycle;
\fill[blue!15.0, opacity=0.7] (0.0790, 0.4800, 1.0663) -- (0.1280, 0.4800, 1.0725) -- (0.1280, 0.5310, 1.0779) -- (0.0790, 0.5310, 1.0716) -- cycle;
\fill[blue!15.1, opacity=0.7] (0.0790, 0.5310, 1.0716) -- (0.1280, 0.5310, 1.0779) -- (0.1280, 0.5820, 1.0831) -- (0.0790, 0.5820, 1.0768) -- cycle;
\fill[blue!15.1, opacity=0.7] (0.0790, 0.5820, 1.0768) -- (0.1280, 0.5820, 1.0831) -- (0.1280, 0.6330, 1.0881) -- (0.0790, 0.6330, 1.0818) -- cycle;
\fill[blue!15.0, opacity=0.7] (0.0790, 0.6330, 1.0818) -- (0.1280, 0.6330, 1.0881) -- (0.1280, 0.6840, 1.0928) -- (0.0790, 0.6840, 1.0866) -- cycle;
\fill[blue!15.0, opacity=0.7] (0.0790, 0.6840, 1.0866) -- (0.1280, 0.6840, 1.0928) -- (0.1280, 0.7350, 1.0974) -- (0.0790, 0.7350, 1.0911) -- cycle;
\fill[blue!15.0, opacity=0.7] (0.0790, 0.7350, 1.0911) -- (0.1280, 0.7350, 1.0974) -- (0.1280, 0.7860, 1.1017) -- (0.0790, 0.7860, 1.0955) -- cycle;
\fill[blue!15.0, opacity=0.7] (0.0790, 0.7860, 1.0955) -- (0.1280, 0.7860, 1.1017) -- (0.1280, 0.8370, 1.1058) -- (0.0790, 0.8370, 1.0995) -- cycle;
\fill[blue!15.0, opacity=0.7] (0.0790, 0.8370, 1.0995) -- (0.1280, 0.8370, 1.1058) -- (0.1280, 0.8880, 1.1096) -- (0.0790, 0.8880, 1.1034) -- cycle;
\fill[blue!15.0, opacity=0.7] (0.0790, 0.8880, 1.1034) -- (0.1280, 0.8880, 1.1096) -- (0.1280, 0.9390, 1.1132) -- (0.0790, 0.9390, 1.1069) -- cycle;
\fill[blue!15.0, opacity=0.7] (0.0790, 0.9390, 1.1069) -- (0.1280, 0.9390, 1.1132) -- (0.1280, 0.9900, 1.1165) -- (0.0790, 0.9900, 1.1102) -- cycle;
\fill[blue!15.0, opacity=0.7] (0.0790, 0.9900, 1.1102) -- (0.1280, 0.9900, 1.1165) -- (0.1280, 1.0410, 1.1195) -- (0.0790, 1.0410, 1.1132) -- cycle;
\fill[blue!15.0, opacity=0.7] (0.0790, 1.0410, 1.1132) -- (0.1280, 1.0410, 1.1195) -- (0.1280, 1.0920, 1.1222) -- (0.0790, 1.0920, 1.1159) -- cycle;
\fill[blue!15.0, opacity=0.7] (0.0790, 1.0920, 1.1159) -- (0.1280, 1.0920, 1.1222) -- (0.1280, 1.1430, 1.1246) -- (0.0790, 1.1430, 1.1183) -- cycle;
\fill[blue!15.0, opacity=0.7] (0.0790, 1.1430, 1.1183) -- (0.1280, 1.1430, 1.1246) -- (0.1280, 1.1940, 1.1267) -- (0.0790, 1.1940, 1.1204) -- cycle;
\fill[blue!15.0, opacity=0.7] (0.0790, 1.1940, 1.1204) -- (0.1280, 1.1940, 1.1267) -- (0.1280, 1.2450, 1.1285) -- (0.0790, 1.2450, 1.1222) -- cycle;
\fill[blue!15.0, opacity=0.7] (0.0790, 1.2450, 1.1222) -- (0.1280, 1.2450, 1.1285) -- (0.1280, 1.2960, 1.1299) -- (0.0790, 1.2960, 1.1237) -- cycle;
\fill[blue!15.0, opacity=0.7] (0.0790, 1.2960, 1.1237) -- (0.1280, 1.2960, 1.1299) -- (0.1280, 1.3470, 1.1311) -- (0.0790, 1.3470, 1.1248) -- cycle;
\fill[blue!15.0, opacity=0.7] (0.0790, 1.3470, 1.1248) -- (0.1280, 1.3470, 1.1311) -- (0.1280, 1.3980, 1.1319) -- (0.0790, 1.3980, 1.1256) -- cycle;
\fill[blue!15.1, opacity=0.7] (0.0790, 1.3980, 1.1256) -- (0.1280, 1.3980, 1.1319) -- (0.1280, 1.4490, 1.1324) -- (0.0790, 1.4490, 1.1261) -- cycle;
\fill[blue!15.2, opacity=0.7] (0.0790, 1.4490, 1.1261) -- (0.1280, 1.4490, 1.1324) -- (0.1280, 1.5000, 1.1325) -- (0.0790, 1.5000, 1.1263) -- cycle;
\fill[blue!15.4, opacity=0.7] (0.0790, 1.5000, 1.1263) -- (0.1280, 1.5000, 1.1325) -- (0.1280, 1.5510, 1.1324) -- (0.0790, 1.5510, 1.1261) -- cycle;
\fill[blue!15.4, opacity=0.7] (0.0790, 1.5510, 1.1261) -- (0.1280, 1.5510, 1.1324) -- (0.1280, 1.6020, 1.1319) -- (0.0790, 1.6020, 1.1256) -- cycle;
\fill[blue!15.5, opacity=0.7] (0.0790, 1.6020, 1.1256) -- (0.1280, 1.6020, 1.1319) -- (0.1280, 1.6530, 1.1311) -- (0.0790, 1.6530, 1.1248) -- cycle;
\fill[blue!15.4, opacity=0.7] (0.0790, 1.6530, 1.1248) -- (0.1280, 1.6530, 1.1311) -- (0.1280, 1.7040, 1.1299) -- (0.0790, 1.7040, 1.1237) -- cycle;
\fill[blue!15.3, opacity=0.7] (0.0790, 1.7040, 1.1237) -- (0.1280, 1.7040, 1.1299) -- (0.1280, 1.7550, 1.1285) -- (0.0790, 1.7550, 1.1222) -- cycle;
\fill[blue!15.2, opacity=0.7] (0.0790, 1.7550, 1.1222) -- (0.1280, 1.7550, 1.1285) -- (0.1280, 1.8060, 1.1267) -- (0.0790, 1.8060, 1.1204) -- cycle;
\fill[blue!15.1, opacity=0.7] (0.0790, 1.8060, 1.1204) -- (0.1280, 1.8060, 1.1267) -- (0.1280, 1.8570, 1.1246) -- (0.0790, 1.8570, 1.1183) -- cycle;
\fill[blue!15.0, opacity=0.7] (0.0790, 1.8570, 1.1183) -- (0.1280, 1.8570, 1.1246) -- (0.1280, 1.9080, 1.1222) -- (0.0790, 1.9080, 1.1159) -- cycle;
\fill[blue!15.0, opacity=0.7] (0.0790, 1.9080, 1.1159) -- (0.1280, 1.9080, 1.1222) -- (0.1280, 1.9590, 1.1195) -- (0.0790, 1.9590, 1.1132) -- cycle;
\fill[blue!15.0, opacity=0.7] (0.0790, 1.9590, 1.1132) -- (0.1280, 1.9590, 1.1195) -- (0.1280, 2.0100, 1.1165) -- (0.0790, 2.0100, 1.1102) -- cycle;
\fill[blue!15.0, opacity=0.7] (0.0790, 2.0100, 1.1102) -- (0.1280, 2.0100, 1.1165) -- (0.1280, 2.0610, 1.1132) -- (0.0790, 2.0610, 1.1069) -- cycle;
\fill[blue!15.0, opacity=0.7] (0.0790, 2.0610, 1.1069) -- (0.1280, 2.0610, 1.1132) -- (0.1280, 2.1120, 1.1096) -- (0.0790, 2.1120, 1.1034) -- cycle;
\fill[blue!15.0, opacity=0.7] (0.0790, 2.1120, 1.1034) -- (0.1280, 2.1120, 1.1096) -- (0.1280, 2.1630, 1.1058) -- (0.0790, 2.1630, 1.0995) -- cycle;
\fill[blue!15.0, opacity=0.7] (0.0790, 2.1630, 1.0995) -- (0.1280, 2.1630, 1.1058) -- (0.1280, 2.2140, 1.1017) -- (0.0790, 2.2140, 1.0955) -- cycle;
\fill[blue!15.0, opacity=0.7] (0.0790, 2.2140, 1.0955) -- (0.1280, 2.2140, 1.1017) -- (0.1280, 2.2650, 1.0974) -- (0.0790, 2.2650, 1.0911) -- cycle;
\fill[blue!15.0, opacity=0.7] (0.0790, 2.2650, 1.0911) -- (0.1280, 2.2650, 1.0974) -- (0.1280, 2.3160, 1.0928) -- (0.0790, 2.3160, 1.0866) -- cycle;
\fill[blue!15.0, opacity=0.7] (0.0790, 2.3160, 1.0866) -- (0.1280, 2.3160, 1.0928) -- (0.1280, 2.3670, 1.0881) -- (0.0790, 2.3670, 1.0818) -- cycle;
\fill[blue!15.0, opacity=0.7] (0.0790, 2.3670, 1.0818) -- (0.1280, 2.3670, 1.0881) -- (0.1280, 2.4180, 1.0831) -- (0.0790, 2.4180, 1.0768) -- cycle;
\fill[blue!15.0, opacity=0.7] (0.0790, 2.4180, 1.0768) -- (0.1280, 2.4180, 1.0831) -- (0.1280, 2.4690, 1.0779) -- (0.0790, 2.4690, 1.0716) -- cycle;
\fill[blue!15.0, opacity=0.7] (0.0790, 2.4690, 1.0716) -- (0.1280, 2.4690, 1.0779) -- (0.1280, 2.5200, 1.0725) -- (0.0790, 2.5200, 1.0663) -- cycle;
\fill[blue!15.0, opacity=0.7] (0.0790, 2.5200, 1.0663) -- (0.1280, 2.5200, 1.0725) -- (0.1280, 2.5710, 1.0670) -- (0.0790, 2.5710, 1.0608) -- cycle;
\fill[blue!15.0, opacity=0.7] (0.0790, 2.5710, 1.0608) -- (0.1280, 2.5710, 1.0670) -- (0.1280, 2.6220, 1.0614) -- (0.0790, 2.6220, 1.0551) -- cycle;
\fill[blue!15.0, opacity=0.7] (0.0790, 2.6220, 1.0551) -- (0.1280, 2.6220, 1.0614) -- (0.1280, 2.6730, 1.0555) -- (0.0790, 2.6730, 1.0493) -- cycle;
\fill[blue!15.0, opacity=0.7] (0.0790, 2.6730, 1.0493) -- (0.1280, 2.6730, 1.0555) -- (0.1280, 2.7240, 1.0496) -- (0.0790, 2.7240, 1.0434) -- cycle;
\fill[blue!15.0, opacity=0.7] (0.0790, 2.7240, 1.0434) -- (0.1280, 2.7240, 1.0496) -- (0.1280, 2.7750, 1.0436) -- (0.0790, 2.7750, 1.0373) -- cycle;
\fill[blue!15.0, opacity=0.7] (0.0790, 2.7750, 1.0373) -- (0.1280, 2.7750, 1.0436) -- (0.1280, 2.8260, 1.0375) -- (0.0790, 2.8260, 1.0312) -- cycle;
\fill[blue!15.0, opacity=0.7] (0.0790, 2.8260, 1.0312) -- (0.1280, 2.8260, 1.0375) -- (0.1280, 2.8770, 1.0313) -- (0.0790, 2.8770, 1.0251) -- cycle;
\fill[blue!15.0, opacity=0.7] (0.0790, 2.8770, 1.0251) -- (0.1280, 2.8770, 1.0313) -- (0.1280, 2.9280, 1.0251) -- (0.0790, 2.9280, 1.0188) -- cycle;
\fill[blue!15.0, opacity=0.7] (0.0790, 2.9280, 1.0188) -- (0.1280, 2.9280, 1.0251) -- (0.1280, 2.9790, 1.0188) -- (0.0790, 2.9790, 1.0126) -- cycle;
\fill[blue!15.0, opacity=0.7] (0.0790, 2.9790, 1.0126) -- (0.1280, 2.9790, 1.0188) -- (0.1280, 3.0300, 1.0125) -- (0.0790, 3.0300, 1.0063) -- cycle;
\fill[blue!15.0, opacity=0.7] (0.1280, -0.0300, 1.0125) -- (0.1770, -0.0300, 1.0188) -- (0.1770, 0.0210, 1.0251) -- (0.1280, 0.0210, 1.0188) -- cycle;
\fill[blue!15.0, opacity=0.7] (0.1280, 0.0210, 1.0188) -- (0.1770, 0.0210, 1.0251) -- (0.1770, 0.0720, 1.0313) -- (0.1280, 0.0720, 1.0251) -- cycle;
\fill[blue!15.0, opacity=0.7] (0.1280, 0.0720, 1.0251) -- (0.1770, 0.0720, 1.0313) -- (0.1770, 0.1230, 1.0375) -- (0.1280, 0.1230, 1.0313) -- cycle;
\fill[blue!15.0, opacity=0.7] (0.1280, 0.1230, 1.0313) -- (0.1770, 0.1230, 1.0375) -- (0.1770, 0.1740, 1.0437) -- (0.1280, 0.1740, 1.0375) -- cycle;
\fill[blue!15.0, opacity=0.7] (0.1280, 0.1740, 1.0375) -- (0.1770, 0.1740, 1.0437) -- (0.1770, 0.2250, 1.0498) -- (0.1280, 0.2250, 1.0436) -- cycle;
\fill[blue!15.0, opacity=0.7] (0.1280, 0.2250, 1.0436) -- (0.1770, 0.2250, 1.0498) -- (0.1770, 0.2760, 1.0559) -- (0.1280, 0.2760, 1.0496) -- cycle;
\fill[blue!15.0, opacity=0.7] (0.1280, 0.2760, 1.0496) -- (0.1770, 0.2760, 1.0559) -- (0.1770, 0.3270, 1.0618) -- (0.1280, 0.3270, 1.0555) -- cycle;
\fill[blue!15.0, opacity=0.7] (0.1280, 0.3270, 1.0555) -- (0.1770, 0.3270, 1.0618) -- (0.1770, 0.3780, 1.0676) -- (0.1280, 0.3780, 1.0614) -- cycle;
\fill[blue!15.0, opacity=0.7] (0.1280, 0.3780, 1.0614) -- (0.1770, 0.3780, 1.0676) -- (0.1770, 0.4290, 1.0733) -- (0.1280, 0.4290, 1.0670) -- cycle;
\fill[blue!15.1, opacity=0.7] (0.1280, 0.4290, 1.0670) -- (0.1770, 0.4290, 1.0733) -- (0.1770, 0.4800, 1.0788) -- (0.1280, 0.4800, 1.0725) -- cycle;
\fill[blue!15.1, opacity=0.7] (0.1280, 0.4800, 1.0725) -- (0.1770, 0.4800, 1.0788) -- (0.1770, 0.5310, 1.0841) -- (0.1280, 0.5310, 1.0779) -- cycle;
\fill[blue!15.0, opacity=0.7] (0.1280, 0.5310, 1.0779) -- (0.1770, 0.5310, 1.0841) -- (0.1770, 0.5820, 1.0893) -- (0.1280, 0.5820, 1.0831) -- cycle;
\fill[blue!15.0, opacity=0.7] (0.1280, 0.5820, 1.0831) -- (0.1770, 0.5820, 1.0893) -- (0.1770, 0.6330, 1.0943) -- (0.1280, 0.6330, 1.0881) -- cycle;
\fill[blue!15.0, opacity=0.7] (0.1280, 0.6330, 1.0881) -- (0.1770, 0.6330, 1.0943) -- (0.1770, 0.6840, 1.0991) -- (0.1280, 0.6840, 1.0928) -- cycle;
\fill[blue!15.0, opacity=0.7] (0.1280, 0.6840, 1.0928) -- (0.1770, 0.6840, 1.0991) -- (0.1770, 0.7350, 1.1036) -- (0.1280, 0.7350, 1.0974) -- cycle;
\fill[blue!15.0, opacity=0.7] (0.1280, 0.7350, 1.0974) -- (0.1770, 0.7350, 1.1036) -- (0.1770, 0.7860, 1.1079) -- (0.1280, 0.7860, 1.1017) -- cycle;
\fill[blue!15.0, opacity=0.7] (0.1280, 0.7860, 1.1017) -- (0.1770, 0.7860, 1.1079) -- (0.1770, 0.8370, 1.1120) -- (0.1280, 0.8370, 1.1058) -- cycle;
\fill[blue!15.0, opacity=0.7] (0.1280, 0.8370, 1.1058) -- (0.1770, 0.8370, 1.1120) -- (0.1770, 0.8880, 1.1159) -- (0.1280, 0.8880, 1.1096) -- cycle;
\fill[blue!15.0, opacity=0.7] (0.1280, 0.8880, 1.1096) -- (0.1770, 0.8880, 1.1159) -- (0.1770, 0.9390, 1.1194) -- (0.1280, 0.9390, 1.1132) -- cycle;
\fill[blue!15.0, opacity=0.7] (0.1280, 0.9390, 1.1132) -- (0.1770, 0.9390, 1.1194) -- (0.1770, 0.9900, 1.1227) -- (0.1280, 0.9900, 1.1165) -- cycle;
\fill[blue!15.0, opacity=0.7] (0.1280, 0.9900, 1.1165) -- (0.1770, 0.9900, 1.1227) -- (0.1770, 1.0410, 1.1257) -- (0.1280, 1.0410, 1.1195) -- cycle;
\fill[blue!15.0, opacity=0.7] (0.1280, 1.0410, 1.1195) -- (0.1770, 1.0410, 1.1257) -- (0.1770, 1.0920, 1.1284) -- (0.1280, 1.0920, 1.1222) -- cycle;
\fill[blue!15.0, opacity=0.7] (0.1280, 1.0920, 1.1222) -- (0.1770, 1.0920, 1.1284) -- (0.1770, 1.1430, 1.1308) -- (0.1280, 1.1430, 1.1246) -- cycle;
\fill[blue!15.1, opacity=0.7] (0.1280, 1.1430, 1.1246) -- (0.1770, 1.1430, 1.1308) -- (0.1770, 1.1940, 1.1329) -- (0.1280, 1.1940, 1.1267) -- cycle;
\fill[blue!15.3, opacity=0.7] (0.1280, 1.1940, 1.1267) -- (0.1770, 1.1940, 1.1329) -- (0.1770, 1.2450, 1.1347) -- (0.1280, 1.2450, 1.1285) -- cycle;
\fill[blue!16.0, opacity=0.7] (0.1280, 1.2450, 1.1285) -- (0.1770, 1.2450, 1.1347) -- (0.1770, 1.2960, 1.1361) -- (0.1280, 1.2960, 1.1299) -- cycle;
\fill[blue!17.2, opacity=0.7] (0.1280, 1.2960, 1.1299) -- (0.1770, 1.2960, 1.1361) -- (0.1770, 1.3470, 1.1373) -- (0.1280, 1.3470, 1.1311) -- cycle;
\fill[blue!18.7, opacity=0.7] (0.1280, 1.3470, 1.1311) -- (0.1770, 1.3470, 1.1373) -- (0.1770, 1.3980, 1.1381) -- (0.1280, 1.3980, 1.1319) -- cycle;
\fill[blue!20.1, opacity=0.7] (0.1280, 1.3980, 1.1319) -- (0.1770, 1.3980, 1.1381) -- (0.1770, 1.4490, 1.1386) -- (0.1280, 1.4490, 1.1324) -- cycle;
\fill[blue!21.0, opacity=0.7] (0.1280, 1.4490, 1.1324) -- (0.1770, 1.4490, 1.1386) -- (0.1770, 1.5000, 1.1388) -- (0.1280, 1.5000, 1.1325) -- cycle;
\fill[blue!21.4, opacity=0.7] (0.1280, 1.5000, 1.1325) -- (0.1770, 1.5000, 1.1388) -- (0.1770, 1.5510, 1.1386) -- (0.1280, 1.5510, 1.1324) -- cycle;
\fill[blue!21.5, opacity=0.7] (0.1280, 1.5510, 1.1324) -- (0.1770, 1.5510, 1.1386) -- (0.1770, 1.6020, 1.1381) -- (0.1280, 1.6020, 1.1319) -- cycle;
\fill[blue!21.4, opacity=0.7] (0.1280, 1.6020, 1.1319) -- (0.1770, 1.6020, 1.1381) -- (0.1770, 1.6530, 1.1373) -- (0.1280, 1.6530, 1.1311) -- cycle;
\fill[blue!21.0, opacity=0.7] (0.1280, 1.6530, 1.1311) -- (0.1770, 1.6530, 1.1373) -- (0.1770, 1.7040, 1.1361) -- (0.1280, 1.7040, 1.1299) -- cycle;
\fill[blue!20.4, opacity=0.7] (0.1280, 1.7040, 1.1299) -- (0.1770, 1.7040, 1.1361) -- (0.1770, 1.7550, 1.1347) -- (0.1280, 1.7550, 1.1285) -- cycle;
\fill[blue!19.7, opacity=0.7] (0.1280, 1.7550, 1.1285) -- (0.1770, 1.7550, 1.1347) -- (0.1770, 1.8060, 1.1329) -- (0.1280, 1.8060, 1.1267) -- cycle;
\fill[blue!18.7, opacity=0.7] (0.1280, 1.8060, 1.1267) -- (0.1770, 1.8060, 1.1329) -- (0.1770, 1.8570, 1.1308) -- (0.1280, 1.8570, 1.1246) -- cycle;
\fill[blue!17.5, opacity=0.7] (0.1280, 1.8570, 1.1246) -- (0.1770, 1.8570, 1.1308) -- (0.1770, 1.9080, 1.1284) -- (0.1280, 1.9080, 1.1222) -- cycle;
\fill[blue!16.4, opacity=0.7] (0.1280, 1.9080, 1.1222) -- (0.1770, 1.9080, 1.1284) -- (0.1770, 1.9590, 1.1257) -- (0.1280, 1.9590, 1.1195) -- cycle;
\fill[blue!15.6, opacity=0.7] (0.1280, 1.9590, 1.1195) -- (0.1770, 1.9590, 1.1257) -- (0.1770, 2.0100, 1.1227) -- (0.1280, 2.0100, 1.1165) -- cycle;
\fill[blue!15.2, opacity=0.7] (0.1280, 2.0100, 1.1165) -- (0.1770, 2.0100, 1.1227) -- (0.1770, 2.0610, 1.1194) -- (0.1280, 2.0610, 1.1132) -- cycle;
\fill[blue!15.0, opacity=0.7] (0.1280, 2.0610, 1.1132) -- (0.1770, 2.0610, 1.1194) -- (0.1770, 2.1120, 1.1159) -- (0.1280, 2.1120, 1.1096) -- cycle;
\fill[blue!15.0, opacity=0.7] (0.1280, 2.1120, 1.1096) -- (0.1770, 2.1120, 1.1159) -- (0.1770, 2.1630, 1.1120) -- (0.1280, 2.1630, 1.1058) -- cycle;
\fill[blue!15.0, opacity=0.7] (0.1280, 2.1630, 1.1058) -- (0.1770, 2.1630, 1.1120) -- (0.1770, 2.2140, 1.1079) -- (0.1280, 2.2140, 1.1017) -- cycle;
\fill[blue!15.0, opacity=0.7] (0.1280, 2.2140, 1.1017) -- (0.1770, 2.2140, 1.1079) -- (0.1770, 2.2650, 1.1036) -- (0.1280, 2.2650, 1.0974) -- cycle;
\fill[blue!15.0, opacity=0.7] (0.1280, 2.2650, 1.0974) -- (0.1770, 2.2650, 1.1036) -- (0.1770, 2.3160, 1.0991) -- (0.1280, 2.3160, 1.0928) -- cycle;
\fill[blue!15.0, opacity=0.7] (0.1280, 2.3160, 1.0928) -- (0.1770, 2.3160, 1.0991) -- (0.1770, 2.3670, 1.0943) -- (0.1280, 2.3670, 1.0881) -- cycle;
\fill[blue!15.0, opacity=0.7] (0.1280, 2.3670, 1.0881) -- (0.1770, 2.3670, 1.0943) -- (0.1770, 2.4180, 1.0893) -- (0.1280, 2.4180, 1.0831) -- cycle;
\fill[blue!15.0, opacity=0.7] (0.1280, 2.4180, 1.0831) -- (0.1770, 2.4180, 1.0893) -- (0.1770, 2.4690, 1.0841) -- (0.1280, 2.4690, 1.0779) -- cycle;
\fill[blue!15.0, opacity=0.7] (0.1280, 2.4690, 1.0779) -- (0.1770, 2.4690, 1.0841) -- (0.1770, 2.5200, 1.0788) -- (0.1280, 2.5200, 1.0725) -- cycle;
\fill[blue!15.0, opacity=0.7] (0.1280, 2.5200, 1.0725) -- (0.1770, 2.5200, 1.0788) -- (0.1770, 2.5710, 1.0733) -- (0.1280, 2.5710, 1.0670) -- cycle;
\fill[blue!15.0, opacity=0.7] (0.1280, 2.5710, 1.0670) -- (0.1770, 2.5710, 1.0733) -- (0.1770, 2.6220, 1.0676) -- (0.1280, 2.6220, 1.0614) -- cycle;
\fill[blue!15.0, opacity=0.7] (0.1280, 2.6220, 1.0614) -- (0.1770, 2.6220, 1.0676) -- (0.1770, 2.6730, 1.0618) -- (0.1280, 2.6730, 1.0555) -- cycle;
\fill[blue!15.0, opacity=0.7] (0.1280, 2.6730, 1.0555) -- (0.1770, 2.6730, 1.0618) -- (0.1770, 2.7240, 1.0559) -- (0.1280, 2.7240, 1.0496) -- cycle;
\fill[blue!15.0, opacity=0.7] (0.1280, 2.7240, 1.0496) -- (0.1770, 2.7240, 1.0559) -- (0.1770, 2.7750, 1.0498) -- (0.1280, 2.7750, 1.0436) -- cycle;
\fill[blue!15.0, opacity=0.7] (0.1280, 2.7750, 1.0436) -- (0.1770, 2.7750, 1.0498) -- (0.1770, 2.8260, 1.0437) -- (0.1280, 2.8260, 1.0375) -- cycle;
\fill[blue!15.0, opacity=0.7] (0.1280, 2.8260, 1.0375) -- (0.1770, 2.8260, 1.0437) -- (0.1770, 2.8770, 1.0375) -- (0.1280, 2.8770, 1.0313) -- cycle;
\fill[blue!15.0, opacity=0.7] (0.1280, 2.8770, 1.0313) -- (0.1770, 2.8770, 1.0375) -- (0.1770, 2.9280, 1.0313) -- (0.1280, 2.9280, 1.0251) -- cycle;
\fill[blue!15.0, opacity=0.7] (0.1280, 2.9280, 1.0251) -- (0.1770, 2.9280, 1.0313) -- (0.1770, 2.9790, 1.0251) -- (0.1280, 2.9790, 1.0188) -- cycle;
\fill[blue!15.0, opacity=0.7] (0.1280, 2.9790, 1.0188) -- (0.1770, 2.9790, 1.0251) -- (0.1770, 3.0300, 1.0188) -- (0.1280, 3.0300, 1.0125) -- cycle;
\fill[blue!15.0, opacity=0.7] (0.1770, -0.0300, 1.0188) -- (0.2260, -0.0300, 1.0249) -- (0.2260, 0.0210, 1.0312) -- (0.1770, 0.0210, 1.0251) -- cycle;
\fill[blue!15.0, opacity=0.7] (0.1770, 0.0210, 1.0251) -- (0.2260, 0.0210, 1.0312) -- (0.2260, 0.0720, 1.0375) -- (0.1770, 0.0720, 1.0313) -- cycle;
\fill[blue!15.0, opacity=0.7] (0.1770, 0.0720, 1.0313) -- (0.2260, 0.0720, 1.0375) -- (0.2260, 0.1230, 1.0437) -- (0.1770, 0.1230, 1.0375) -- cycle;
\fill[blue!15.0, opacity=0.7] (0.1770, 0.1230, 1.0375) -- (0.2260, 0.1230, 1.0437) -- (0.2260, 0.1740, 1.0499) -- (0.1770, 0.1740, 1.0437) -- cycle;
\fill[blue!15.0, opacity=0.7] (0.1770, 0.1740, 1.0437) -- (0.2260, 0.1740, 1.0499) -- (0.2260, 0.2250, 1.0560) -- (0.1770, 0.2250, 1.0498) -- cycle;
\fill[blue!15.0, opacity=0.7] (0.1770, 0.2250, 1.0498) -- (0.2260, 0.2250, 1.0560) -- (0.2260, 0.2760, 1.0620) -- (0.1770, 0.2760, 1.0559) -- cycle;
\fill[blue!15.0, opacity=0.7] (0.1770, 0.2760, 1.0559) -- (0.2260, 0.2760, 1.0620) -- (0.2260, 0.3270, 1.0680) -- (0.1770, 0.3270, 1.0618) -- cycle;
\fill[blue!15.0, opacity=0.7] (0.1770, 0.3270, 1.0618) -- (0.2260, 0.3270, 1.0680) -- (0.2260, 0.3780, 1.0738) -- (0.1770, 0.3780, 1.0676) -- cycle;
\fill[blue!15.1, opacity=0.7] (0.1770, 0.3780, 1.0676) -- (0.2260, 0.3780, 1.0738) -- (0.2260, 0.4290, 1.0794) -- (0.1770, 0.4290, 1.0733) -- cycle;
\fill[blue!15.1, opacity=0.7] (0.1770, 0.4290, 1.0733) -- (0.2260, 0.4290, 1.0794) -- (0.2260, 0.4800, 1.0849) -- (0.1770, 0.4800, 1.0788) -- cycle;
\fill[blue!15.0, opacity=0.7] (0.1770, 0.4800, 1.0788) -- (0.2260, 0.4800, 1.0849) -- (0.2260, 0.5310, 1.0903) -- (0.1770, 0.5310, 1.0841) -- cycle;
\fill[blue!15.0, opacity=0.7] (0.1770, 0.5310, 1.0841) -- (0.2260, 0.5310, 1.0903) -- (0.2260, 0.5820, 1.0955) -- (0.1770, 0.5820, 1.0893) -- cycle;
\fill[blue!15.0, opacity=0.7] (0.1770, 0.5820, 1.0893) -- (0.2260, 0.5820, 1.0955) -- (0.2260, 0.6330, 1.1005) -- (0.1770, 0.6330, 1.0943) -- cycle;
\fill[blue!15.0, opacity=0.7] (0.1770, 0.6330, 1.0943) -- (0.2260, 0.6330, 1.1005) -- (0.2260, 0.6840, 1.1052) -- (0.1770, 0.6840, 1.0991) -- cycle;
\fill[blue!15.0, opacity=0.7] (0.1770, 0.6840, 1.0991) -- (0.2260, 0.6840, 1.1052) -- (0.2260, 0.7350, 1.1098) -- (0.1770, 0.7350, 1.1036) -- cycle;
\fill[blue!15.0, opacity=0.7] (0.1770, 0.7350, 1.1036) -- (0.2260, 0.7350, 1.1098) -- (0.2260, 0.7860, 1.1141) -- (0.1770, 0.7860, 1.1079) -- cycle;
\fill[blue!15.0, opacity=0.7] (0.1770, 0.7860, 1.1079) -- (0.2260, 0.7860, 1.1141) -- (0.2260, 0.8370, 1.1182) -- (0.1770, 0.8370, 1.1120) -- cycle;
\fill[blue!15.0, opacity=0.7] (0.1770, 0.8370, 1.1120) -- (0.2260, 0.8370, 1.1182) -- (0.2260, 0.8880, 1.1220) -- (0.1770, 0.8880, 1.1159) -- cycle;
\fill[blue!15.0, opacity=0.7] (0.1770, 0.8880, 1.1159) -- (0.2260, 0.8880, 1.1220) -- (0.2260, 0.9390, 1.1256) -- (0.1770, 0.9390, 1.1194) -- cycle;
\fill[blue!15.0, opacity=0.7] (0.1770, 0.9390, 1.1194) -- (0.2260, 0.9390, 1.1256) -- (0.2260, 0.9900, 1.1289) -- (0.1770, 0.9900, 1.1227) -- cycle;
\fill[blue!15.0, opacity=0.7] (0.1770, 0.9900, 1.1227) -- (0.2260, 0.9900, 1.1289) -- (0.2260, 1.0410, 1.1319) -- (0.1770, 1.0410, 1.1257) -- cycle;
\fill[blue!15.2, opacity=0.7] (0.1770, 1.0410, 1.1257) -- (0.2260, 1.0410, 1.1319) -- (0.2260, 1.0920, 1.1346) -- (0.1770, 1.0920, 1.1284) -- cycle;
\fill[blue!16.3, opacity=0.7] (0.1770, 1.0920, 1.1284) -- (0.2260, 1.0920, 1.1346) -- (0.2260, 1.1430, 1.1370) -- (0.1770, 1.1430, 1.1308) -- cycle;
\fill[blue!18.9, opacity=0.7] (0.1770, 1.1430, 1.1308) -- (0.2260, 1.1430, 1.1370) -- (0.2260, 1.1940, 1.1391) -- (0.1770, 1.1940, 1.1329) -- cycle;
\fill[blue!22.0, opacity=0.7] (0.1770, 1.1940, 1.1329) -- (0.2260, 1.1940, 1.1391) -- (0.2260, 1.2450, 1.1409) -- (0.1770, 1.2450, 1.1347) -- cycle;
\fill[blue!24.2, opacity=0.7] (0.1770, 1.2450, 1.1347) -- (0.2260, 1.2450, 1.1409) -- (0.2260, 1.2960, 1.1423) -- (0.1770, 1.2960, 1.1361) -- cycle;
\fill[blue!24.7, opacity=0.7] (0.1770, 1.2960, 1.1361) -- (0.2260, 1.2960, 1.1423) -- (0.2260, 1.3470, 1.1435) -- (0.1770, 1.3470, 1.1373) -- cycle;
\fill[blue!23.9, opacity=0.7] (0.1770, 1.3470, 1.1373) -- (0.2260, 1.3470, 1.1435) -- (0.2260, 1.3980, 1.1443) -- (0.1770, 1.3980, 1.1381) -- cycle;
\fill[blue!22.5, opacity=0.7] (0.1770, 1.3980, 1.1381) -- (0.2260, 1.3980, 1.1443) -- (0.2260, 1.4490, 1.1448) -- (0.1770, 1.4490, 1.1386) -- cycle;
\fill[blue!21.1, opacity=0.7] (0.1770, 1.4490, 1.1386) -- (0.2260, 1.4490, 1.1448) -- (0.2260, 1.5000, 1.1449) -- (0.1770, 1.5000, 1.1388) -- cycle;
\fill[blue!20.0, opacity=0.7] (0.1770, 1.5000, 1.1388) -- (0.2260, 1.5000, 1.1449) -- (0.2260, 1.5510, 1.1448) -- (0.1770, 1.5510, 1.1386) -- cycle;
\fill[blue!19.3, opacity=0.7] (0.1770, 1.5510, 1.1386) -- (0.2260, 1.5510, 1.1448) -- (0.2260, 1.6020, 1.1443) -- (0.1770, 1.6020, 1.1381) -- cycle;
\fill[blue!18.9, opacity=0.7] (0.1770, 1.6020, 1.1381) -- (0.2260, 1.6020, 1.1443) -- (0.2260, 1.6530, 1.1435) -- (0.1770, 1.6530, 1.1373) -- cycle;
\fill[blue!18.8, opacity=0.7] (0.1770, 1.6530, 1.1373) -- (0.2260, 1.6530, 1.1435) -- (0.2260, 1.7040, 1.1423) -- (0.1770, 1.7040, 1.1361) -- cycle;
\fill[blue!18.9, opacity=0.7] (0.1770, 1.7040, 1.1361) -- (0.2260, 1.7040, 1.1423) -- (0.2260, 1.7550, 1.1409) -- (0.1770, 1.7550, 1.1347) -- cycle;
\fill[blue!19.3, opacity=0.7] (0.1770, 1.7550, 1.1347) -- (0.2260, 1.7550, 1.1409) -- (0.2260, 1.8060, 1.1391) -- (0.1770, 1.8060, 1.1329) -- cycle;
\fill[blue!19.8, opacity=0.7] (0.1770, 1.8060, 1.1329) -- (0.2260, 1.8060, 1.1391) -- (0.2260, 1.8570, 1.1370) -- (0.1770, 1.8570, 1.1308) -- cycle;
\fill[blue!20.3, opacity=0.7] (0.1770, 1.8570, 1.1308) -- (0.2260, 1.8570, 1.1370) -- (0.2260, 1.9080, 1.1346) -- (0.1770, 1.9080, 1.1284) -- cycle;
\fill[blue!20.5, opacity=0.7] (0.1770, 1.9080, 1.1284) -- (0.2260, 1.9080, 1.1346) -- (0.2260, 1.9590, 1.1319) -- (0.1770, 1.9590, 1.1257) -- cycle;
\fill[blue!20.0, opacity=0.7] (0.1770, 1.9590, 1.1257) -- (0.2260, 1.9590, 1.1319) -- (0.2260, 2.0100, 1.1289) -- (0.1770, 2.0100, 1.1227) -- cycle;
\fill[blue!18.7, opacity=0.7] (0.1770, 2.0100, 1.1227) -- (0.2260, 2.0100, 1.1289) -- (0.2260, 2.0610, 1.1256) -- (0.1770, 2.0610, 1.1194) -- cycle;
\fill[blue!17.0, opacity=0.7] (0.1770, 2.0610, 1.1194) -- (0.2260, 2.0610, 1.1256) -- (0.2260, 2.1120, 1.1220) -- (0.1770, 2.1120, 1.1159) -- cycle;
\fill[blue!15.7, opacity=0.7] (0.1770, 2.1120, 1.1159) -- (0.2260, 2.1120, 1.1220) -- (0.2260, 2.1630, 1.1182) -- (0.1770, 2.1630, 1.1120) -- cycle;
\fill[blue!15.1, opacity=0.7] (0.1770, 2.1630, 1.1120) -- (0.2260, 2.1630, 1.1182) -- (0.2260, 2.2140, 1.1141) -- (0.1770, 2.2140, 1.1079) -- cycle;
\fill[blue!15.0, opacity=0.7] (0.1770, 2.2140, 1.1079) -- (0.2260, 2.2140, 1.1141) -- (0.2260, 2.2650, 1.1098) -- (0.1770, 2.2650, 1.1036) -- cycle;
\fill[blue!15.0, opacity=0.7] (0.1770, 2.2650, 1.1036) -- (0.2260, 2.2650, 1.1098) -- (0.2260, 2.3160, 1.1052) -- (0.1770, 2.3160, 1.0991) -- cycle;
\fill[blue!15.0, opacity=0.7] (0.1770, 2.3160, 1.0991) -- (0.2260, 2.3160, 1.1052) -- (0.2260, 2.3670, 1.1005) -- (0.1770, 2.3670, 1.0943) -- cycle;
\fill[blue!15.0, opacity=0.7] (0.1770, 2.3670, 1.0943) -- (0.2260, 2.3670, 1.1005) -- (0.2260, 2.4180, 1.0955) -- (0.1770, 2.4180, 1.0893) -- cycle;
\fill[blue!15.0, opacity=0.7] (0.1770, 2.4180, 1.0893) -- (0.2260, 2.4180, 1.0955) -- (0.2260, 2.4690, 1.0903) -- (0.1770, 2.4690, 1.0841) -- cycle;
\fill[blue!15.0, opacity=0.7] (0.1770, 2.4690, 1.0841) -- (0.2260, 2.4690, 1.0903) -- (0.2260, 2.5200, 1.0849) -- (0.1770, 2.5200, 1.0788) -- cycle;
\fill[blue!15.0, opacity=0.7] (0.1770, 2.5200, 1.0788) -- (0.2260, 2.5200, 1.0849) -- (0.2260, 2.5710, 1.0794) -- (0.1770, 2.5710, 1.0733) -- cycle;
\fill[blue!15.0, opacity=0.7] (0.1770, 2.5710, 1.0733) -- (0.2260, 2.5710, 1.0794) -- (0.2260, 2.6220, 1.0738) -- (0.1770, 2.6220, 1.0676) -- cycle;
\fill[blue!15.0, opacity=0.7] (0.1770, 2.6220, 1.0676) -- (0.2260, 2.6220, 1.0738) -- (0.2260, 2.6730, 1.0680) -- (0.1770, 2.6730, 1.0618) -- cycle;
\fill[blue!15.0, opacity=0.7] (0.1770, 2.6730, 1.0618) -- (0.2260, 2.6730, 1.0680) -- (0.2260, 2.7240, 1.0620) -- (0.1770, 2.7240, 1.0559) -- cycle;
\fill[blue!15.0, opacity=0.7] (0.1770, 2.7240, 1.0559) -- (0.2260, 2.7240, 1.0620) -- (0.2260, 2.7750, 1.0560) -- (0.1770, 2.7750, 1.0498) -- cycle;
\fill[blue!15.0, opacity=0.7] (0.1770, 2.7750, 1.0498) -- (0.2260, 2.7750, 1.0560) -- (0.2260, 2.8260, 1.0499) -- (0.1770, 2.8260, 1.0437) -- cycle;
\fill[blue!15.0, opacity=0.7] (0.1770, 2.8260, 1.0437) -- (0.2260, 2.8260, 1.0499) -- (0.2260, 2.8770, 1.0437) -- (0.1770, 2.8770, 1.0375) -- cycle;
\fill[blue!15.0, opacity=0.7] (0.1770, 2.8770, 1.0375) -- (0.2260, 2.8770, 1.0437) -- (0.2260, 2.9280, 1.0375) -- (0.1770, 2.9280, 1.0313) -- cycle;
\fill[blue!15.0, opacity=0.7] (0.1770, 2.9280, 1.0313) -- (0.2260, 2.9280, 1.0375) -- (0.2260, 2.9790, 1.0312) -- (0.1770, 2.9790, 1.0251) -- cycle;
\fill[blue!15.0, opacity=0.7] (0.1770, 2.9790, 1.0251) -- (0.2260, 2.9790, 1.0312) -- (0.2260, 3.0300, 1.0249) -- (0.1770, 3.0300, 1.0188) -- cycle;
\fill[blue!15.0, opacity=0.7] (0.2260, -0.0300, 1.0249) -- (0.2750, -0.0300, 1.0311) -- (0.2750, 0.0210, 1.0373) -- (0.2260, 0.0210, 1.0312) -- cycle;
\fill[blue!15.0, opacity=0.7] (0.2260, 0.0210, 1.0312) -- (0.2750, 0.0210, 1.0373) -- (0.2750, 0.0720, 1.0436) -- (0.2260, 0.0720, 1.0375) -- cycle;
\fill[blue!15.0, opacity=0.7] (0.2260, 0.0720, 1.0375) -- (0.2750, 0.0720, 1.0436) -- (0.2750, 0.1230, 1.0498) -- (0.2260, 0.1230, 1.0437) -- cycle;
\fill[blue!15.0, opacity=0.7] (0.2260, 0.1230, 1.0437) -- (0.2750, 0.1230, 1.0498) -- (0.2750, 0.1740, 1.0560) -- (0.2260, 0.1740, 1.0499) -- cycle;
\fill[blue!15.0, opacity=0.7] (0.2260, 0.1740, 1.0499) -- (0.2750, 0.1740, 1.0560) -- (0.2750, 0.2250, 1.0621) -- (0.2260, 0.2250, 1.0560) -- cycle;
\fill[blue!15.0, opacity=0.7] (0.2260, 0.2250, 1.0560) -- (0.2750, 0.2250, 1.0621) -- (0.2750, 0.2760, 1.0681) -- (0.2260, 0.2760, 1.0620) -- cycle;
\fill[blue!15.0, opacity=0.7] (0.2260, 0.2760, 1.0620) -- (0.2750, 0.2760, 1.0681) -- (0.2750, 0.3270, 1.0741) -- (0.2260, 0.3270, 1.0680) -- cycle;
\fill[blue!15.1, opacity=0.7] (0.2260, 0.3270, 1.0680) -- (0.2750, 0.3270, 1.0741) -- (0.2750, 0.3780, 1.0799) -- (0.2260, 0.3780, 1.0738) -- cycle;
\fill[blue!15.1, opacity=0.7] (0.2260, 0.3780, 1.0738) -- (0.2750, 0.3780, 1.0799) -- (0.2750, 0.4290, 1.0855) -- (0.2260, 0.4290, 1.0794) -- cycle;
\fill[blue!15.0, opacity=0.7] (0.2260, 0.4290, 1.0794) -- (0.2750, 0.4290, 1.0855) -- (0.2750, 0.4800, 1.0911) -- (0.2260, 0.4800, 1.0849) -- cycle;
\fill[blue!15.0, opacity=0.7] (0.2260, 0.4800, 1.0849) -- (0.2750, 0.4800, 1.0911) -- (0.2750, 0.5310, 1.0964) -- (0.2260, 0.5310, 1.0903) -- cycle;
\fill[blue!15.0, opacity=0.7] (0.2260, 0.5310, 1.0903) -- (0.2750, 0.5310, 1.0964) -- (0.2750, 0.5820, 1.1016) -- (0.2260, 0.5820, 1.0955) -- cycle;
\fill[blue!15.0, opacity=0.7] (0.2260, 0.5820, 1.0955) -- (0.2750, 0.5820, 1.1016) -- (0.2750, 0.6330, 1.1066) -- (0.2260, 0.6330, 1.1005) -- cycle;
\fill[blue!15.0, opacity=0.7] (0.2260, 0.6330, 1.1005) -- (0.2750, 0.6330, 1.1066) -- (0.2750, 0.6840, 1.1114) -- (0.2260, 0.6840, 1.1052) -- cycle;
\fill[blue!15.0, opacity=0.7] (0.2260, 0.6840, 1.1052) -- (0.2750, 0.6840, 1.1114) -- (0.2750, 0.7350, 1.1159) -- (0.2260, 0.7350, 1.1098) -- cycle;
\fill[blue!15.0, opacity=0.7] (0.2260, 0.7350, 1.1098) -- (0.2750, 0.7350, 1.1159) -- (0.2750, 0.7860, 1.1202) -- (0.2260, 0.7860, 1.1141) -- cycle;
\fill[blue!15.0, opacity=0.7] (0.2260, 0.7860, 1.1141) -- (0.2750, 0.7860, 1.1202) -- (0.2750, 0.8370, 1.1243) -- (0.2260, 0.8370, 1.1182) -- cycle;
\fill[blue!15.0, opacity=0.7] (0.2260, 0.8370, 1.1182) -- (0.2750, 0.8370, 1.1243) -- (0.2750, 0.8880, 1.1281) -- (0.2260, 0.8880, 1.1220) -- cycle;
\fill[blue!15.0, opacity=0.7] (0.2260, 0.8880, 1.1220) -- (0.2750, 0.8880, 1.1281) -- (0.2750, 0.9390, 1.1317) -- (0.2260, 0.9390, 1.1256) -- cycle;
\fill[blue!15.3, opacity=0.7] (0.2260, 0.9390, 1.1256) -- (0.2750, 0.9390, 1.1317) -- (0.2750, 0.9900, 1.1350) -- (0.2260, 0.9900, 1.1289) -- cycle;
\fill[blue!17.3, opacity=0.7] (0.2260, 0.9900, 1.1289) -- (0.2750, 0.9900, 1.1350) -- (0.2750, 1.0410, 1.1380) -- (0.2260, 1.0410, 1.1319) -- cycle;
\fill[blue!21.6, opacity=0.7] (0.2260, 1.0410, 1.1319) -- (0.2750, 1.0410, 1.1380) -- (0.2750, 1.0920, 1.1407) -- (0.2260, 1.0920, 1.1346) -- cycle;
\fill[blue!25.5, opacity=0.7] (0.2260, 1.0920, 1.1346) -- (0.2750, 1.0920, 1.1407) -- (0.2750, 1.1430, 1.1431) -- (0.2260, 1.1430, 1.1370) -- cycle;
\fill[blue!26.1, opacity=0.7] (0.2260, 1.1430, 1.1370) -- (0.2750, 1.1430, 1.1431) -- (0.2750, 1.1940, 1.1452) -- (0.2260, 1.1940, 1.1391) -- cycle;
\fill[blue!23.8, opacity=0.7] (0.2260, 1.1940, 1.1391) -- (0.2750, 1.1940, 1.1452) -- (0.2750, 1.2450, 1.1470) -- (0.2260, 1.2450, 1.1409) -- cycle;
\fill[blue!20.6, opacity=0.7] (0.2260, 1.2450, 1.1409) -- (0.2750, 1.2450, 1.1470) -- (0.2750, 1.2960, 1.1484) -- (0.2260, 1.2960, 1.1423) -- cycle;
\fill[blue!18.0, opacity=0.7] (0.2260, 1.2960, 1.1423) -- (0.2750, 1.2960, 1.1484) -- (0.2750, 1.3470, 1.1496) -- (0.2260, 1.3470, 1.1435) -- cycle;
\fill[blue!16.5, opacity=0.7] (0.2260, 1.3470, 1.1435) -- (0.2750, 1.3470, 1.1496) -- (0.2750, 1.3980, 1.1504) -- (0.2260, 1.3980, 1.1443) -- cycle;
\fill[blue!15.7, opacity=0.7] (0.2260, 1.3980, 1.1443) -- (0.2750, 1.3980, 1.1504) -- (0.2750, 1.4490, 1.1509) -- (0.2260, 1.4490, 1.1448) -- cycle;
\fill[blue!15.4, opacity=0.7] (0.2260, 1.4490, 1.1448) -- (0.2750, 1.4490, 1.1509) -- (0.2750, 1.5000, 1.1511) -- (0.2260, 1.5000, 1.1449) -- cycle;
\fill[blue!15.2, opacity=0.7] (0.2260, 1.5000, 1.1449) -- (0.2750, 1.5000, 1.1511) -- (0.2750, 1.5510, 1.1509) -- (0.2260, 1.5510, 1.1448) -- cycle;
\fill[blue!15.1, opacity=0.7] (0.2260, 1.5510, 1.1448) -- (0.2750, 1.5510, 1.1509) -- (0.2750, 1.6020, 1.1504) -- (0.2260, 1.6020, 1.1443) -- cycle;
\fill[blue!15.1, opacity=0.7] (0.2260, 1.6020, 1.1443) -- (0.2750, 1.6020, 1.1504) -- (0.2750, 1.6530, 1.1496) -- (0.2260, 1.6530, 1.1435) -- cycle;
\fill[blue!15.1, opacity=0.7] (0.2260, 1.6530, 1.1435) -- (0.2750, 1.6530, 1.1496) -- (0.2750, 1.7040, 1.1484) -- (0.2260, 1.7040, 1.1423) -- cycle;
\fill[blue!15.1, opacity=0.7] (0.2260, 1.7040, 1.1423) -- (0.2750, 1.7040, 1.1484) -- (0.2750, 1.7550, 1.1470) -- (0.2260, 1.7550, 1.1409) -- cycle;
\fill[blue!15.2, opacity=0.7] (0.2260, 1.7550, 1.1409) -- (0.2750, 1.7550, 1.1470) -- (0.2750, 1.8060, 1.1452) -- (0.2260, 1.8060, 1.1391) -- cycle;
\fill[blue!15.3, opacity=0.7] (0.2260, 1.8060, 1.1391) -- (0.2750, 1.8060, 1.1452) -- (0.2750, 1.8570, 1.1431) -- (0.2260, 1.8570, 1.1370) -- cycle;
\fill[blue!15.6, opacity=0.7] (0.2260, 1.8570, 1.1370) -- (0.2750, 1.8570, 1.1431) -- (0.2750, 1.9080, 1.1407) -- (0.2260, 1.9080, 1.1346) -- cycle;
\fill[blue!16.2, opacity=0.7] (0.2260, 1.9080, 1.1346) -- (0.2750, 1.9080, 1.1407) -- (0.2750, 1.9590, 1.1380) -- (0.2260, 1.9590, 1.1319) -- cycle;
\fill[blue!17.1, opacity=0.7] (0.2260, 1.9590, 1.1319) -- (0.2750, 1.9590, 1.1380) -- (0.2750, 2.0100, 1.1350) -- (0.2260, 2.0100, 1.1289) -- cycle;
\fill[blue!18.4, opacity=0.7] (0.2260, 2.0100, 1.1289) -- (0.2750, 2.0100, 1.1350) -- (0.2750, 2.0610, 1.1317) -- (0.2260, 2.0610, 1.1256) -- cycle;
\fill[blue!19.4, opacity=0.7] (0.2260, 2.0610, 1.1256) -- (0.2750, 2.0610, 1.1317) -- (0.2750, 2.1120, 1.1281) -- (0.2260, 2.1120, 1.1220) -- cycle;
\fill[blue!19.4, opacity=0.7] (0.2260, 2.1120, 1.1220) -- (0.2750, 2.1120, 1.1281) -- (0.2750, 2.1630, 1.1243) -- (0.2260, 2.1630, 1.1182) -- cycle;
\fill[blue!17.9, opacity=0.7] (0.2260, 2.1630, 1.1182) -- (0.2750, 2.1630, 1.1243) -- (0.2750, 2.2140, 1.1202) -- (0.2260, 2.2140, 1.1141) -- cycle;
\fill[blue!16.1, opacity=0.7] (0.2260, 2.2140, 1.1141) -- (0.2750, 2.2140, 1.1202) -- (0.2750, 2.2650, 1.1159) -- (0.2260, 2.2650, 1.1098) -- cycle;
\fill[blue!15.2, opacity=0.7] (0.2260, 2.2650, 1.1098) -- (0.2750, 2.2650, 1.1159) -- (0.2750, 2.3160, 1.1114) -- (0.2260, 2.3160, 1.1052) -- cycle;
\fill[blue!15.0, opacity=0.7] (0.2260, 2.3160, 1.1052) -- (0.2750, 2.3160, 1.1114) -- (0.2750, 2.3670, 1.1066) -- (0.2260, 2.3670, 1.1005) -- cycle;
\fill[blue!15.0, opacity=0.7] (0.2260, 2.3670, 1.1005) -- (0.2750, 2.3670, 1.1066) -- (0.2750, 2.4180, 1.1016) -- (0.2260, 2.4180, 1.0955) -- cycle;
\fill[blue!15.0, opacity=0.7] (0.2260, 2.4180, 1.0955) -- (0.2750, 2.4180, 1.1016) -- (0.2750, 2.4690, 1.0964) -- (0.2260, 2.4690, 1.0903) -- cycle;
\fill[blue!15.0, opacity=0.7] (0.2260, 2.4690, 1.0903) -- (0.2750, 2.4690, 1.0964) -- (0.2750, 2.5200, 1.0911) -- (0.2260, 2.5200, 1.0849) -- cycle;
\fill[blue!15.0, opacity=0.7] (0.2260, 2.5200, 1.0849) -- (0.2750, 2.5200, 1.0911) -- (0.2750, 2.5710, 1.0855) -- (0.2260, 2.5710, 1.0794) -- cycle;
\fill[blue!15.0, opacity=0.7] (0.2260, 2.5710, 1.0794) -- (0.2750, 2.5710, 1.0855) -- (0.2750, 2.6220, 1.0799) -- (0.2260, 2.6220, 1.0738) -- cycle;
\fill[blue!15.0, opacity=0.7] (0.2260, 2.6220, 1.0738) -- (0.2750, 2.6220, 1.0799) -- (0.2750, 2.6730, 1.0741) -- (0.2260, 2.6730, 1.0680) -- cycle;
\fill[blue!15.0, opacity=0.7] (0.2260, 2.6730, 1.0680) -- (0.2750, 2.6730, 1.0741) -- (0.2750, 2.7240, 1.0681) -- (0.2260, 2.7240, 1.0620) -- cycle;
\fill[blue!15.0, opacity=0.7] (0.2260, 2.7240, 1.0620) -- (0.2750, 2.7240, 1.0681) -- (0.2750, 2.7750, 1.0621) -- (0.2260, 2.7750, 1.0560) -- cycle;
\fill[blue!15.0, opacity=0.7] (0.2260, 2.7750, 1.0560) -- (0.2750, 2.7750, 1.0621) -- (0.2750, 2.8260, 1.0560) -- (0.2260, 2.8260, 1.0499) -- cycle;
\fill[blue!15.0, opacity=0.7] (0.2260, 2.8260, 1.0499) -- (0.2750, 2.8260, 1.0560) -- (0.2750, 2.8770, 1.0498) -- (0.2260, 2.8770, 1.0437) -- cycle;
\fill[blue!15.0, opacity=0.7] (0.2260, 2.8770, 1.0437) -- (0.2750, 2.8770, 1.0498) -- (0.2750, 2.9280, 1.0436) -- (0.2260, 2.9280, 1.0375) -- cycle;
\fill[blue!15.0, opacity=0.7] (0.2260, 2.9280, 1.0375) -- (0.2750, 2.9280, 1.0436) -- (0.2750, 2.9790, 1.0373) -- (0.2260, 2.9790, 1.0312) -- cycle;
\fill[blue!15.0, opacity=0.7] (0.2260, 2.9790, 1.0312) -- (0.2750, 2.9790, 1.0373) -- (0.2750, 3.0300, 1.0311) -- (0.2260, 3.0300, 1.0249) -- cycle;
\fill[blue!15.0, opacity=0.7] (0.2750, -0.0300, 1.0311) -- (0.3240, -0.0300, 1.0371) -- (0.3240, 0.0210, 1.0434) -- (0.2750, 0.0210, 1.0373) -- cycle;
\fill[blue!15.0, opacity=0.7] (0.2750, 0.0210, 1.0373) -- (0.3240, 0.0210, 1.0434) -- (0.3240, 0.0720, 1.0496) -- (0.2750, 0.0720, 1.0436) -- cycle;
\fill[blue!15.0, opacity=0.7] (0.2750, 0.0720, 1.0436) -- (0.3240, 0.0720, 1.0496) -- (0.3240, 0.1230, 1.0559) -- (0.2750, 0.1230, 1.0498) -- cycle;
\fill[blue!15.0, opacity=0.7] (0.2750, 0.1230, 1.0498) -- (0.3240, 0.1230, 1.0559) -- (0.3240, 0.1740, 1.0620) -- (0.2750, 0.1740, 1.0560) -- cycle;
\fill[blue!15.0, opacity=0.7] (0.2750, 0.1740, 1.0560) -- (0.3240, 0.1740, 1.0620) -- (0.3240, 0.2250, 1.0681) -- (0.2750, 0.2250, 1.0621) -- cycle;
\fill[blue!15.0, opacity=0.7] (0.2750, 0.2250, 1.0621) -- (0.3240, 0.2250, 1.0681) -- (0.3240, 0.2760, 1.0742) -- (0.2750, 0.2760, 1.0681) -- cycle;
\fill[blue!15.1, opacity=0.7] (0.2750, 0.2760, 1.0681) -- (0.3240, 0.2760, 1.0742) -- (0.3240, 0.3270, 1.0801) -- (0.2750, 0.3270, 1.0741) -- cycle;
\fill[blue!15.2, opacity=0.7] (0.2750, 0.3270, 1.0741) -- (0.3240, 0.3270, 1.0801) -- (0.3240, 0.3780, 1.0859) -- (0.2750, 0.3780, 1.0799) -- cycle;
\fill[blue!15.0, opacity=0.7] (0.2750, 0.3780, 1.0799) -- (0.3240, 0.3780, 1.0859) -- (0.3240, 0.4290, 1.0916) -- (0.2750, 0.4290, 1.0855) -- cycle;
\fill[blue!15.0, opacity=0.7] (0.2750, 0.4290, 1.0855) -- (0.3240, 0.4290, 1.0916) -- (0.3240, 0.4800, 1.0971) -- (0.2750, 0.4800, 1.0911) -- cycle;
\fill[blue!15.0, opacity=0.7] (0.2750, 0.4800, 1.0911) -- (0.3240, 0.4800, 1.0971) -- (0.3240, 0.5310, 1.1024) -- (0.2750, 0.5310, 1.0964) -- cycle;
\fill[blue!15.0, opacity=0.7] (0.2750, 0.5310, 1.0964) -- (0.3240, 0.5310, 1.1024) -- (0.3240, 0.5820, 1.1076) -- (0.2750, 0.5820, 1.1016) -- cycle;
\fill[blue!15.0, opacity=0.7] (0.2750, 0.5820, 1.1016) -- (0.3240, 0.5820, 1.1076) -- (0.3240, 0.6330, 1.1126) -- (0.2750, 0.6330, 1.1066) -- cycle;
\fill[blue!15.0, opacity=0.7] (0.2750, 0.6330, 1.1066) -- (0.3240, 0.6330, 1.1126) -- (0.3240, 0.6840, 1.1174) -- (0.2750, 0.6840, 1.1114) -- cycle;
\fill[blue!15.0, opacity=0.7] (0.2750, 0.6840, 1.1114) -- (0.3240, 0.6840, 1.1174) -- (0.3240, 0.7350, 1.1219) -- (0.2750, 0.7350, 1.1159) -- cycle;
\fill[blue!15.0, opacity=0.7] (0.2750, 0.7350, 1.1159) -- (0.3240, 0.7350, 1.1219) -- (0.3240, 0.7860, 1.1263) -- (0.2750, 0.7860, 1.1202) -- cycle;
\fill[blue!15.0, opacity=0.7] (0.2750, 0.7860, 1.1202) -- (0.3240, 0.7860, 1.1263) -- (0.3240, 0.8370, 1.1303) -- (0.2750, 0.8370, 1.1243) -- cycle;
\fill[blue!15.2, opacity=0.7] (0.2750, 0.8370, 1.1243) -- (0.3240, 0.8370, 1.1303) -- (0.3240, 0.8880, 1.1342) -- (0.2750, 0.8880, 1.1281) -- cycle;
\fill[blue!17.0, opacity=0.7] (0.2750, 0.8880, 1.1281) -- (0.3240, 0.8880, 1.1342) -- (0.3240, 0.9390, 1.1377) -- (0.2750, 0.9390, 1.1317) -- cycle;
\fill[blue!22.4, opacity=0.7] (0.2750, 0.9390, 1.1317) -- (0.3240, 0.9390, 1.1377) -- (0.3240, 0.9900, 1.1410) -- (0.2750, 0.9900, 1.1350) -- cycle;
\fill[blue!27.1, opacity=0.7] (0.2750, 0.9900, 1.1350) -- (0.3240, 0.9900, 1.1410) -- (0.3240, 1.0410, 1.1440) -- (0.2750, 1.0410, 1.1380) -- cycle;
\fill[blue!26.6, opacity=0.7] (0.2750, 1.0410, 1.1380) -- (0.3240, 1.0410, 1.1440) -- (0.3240, 1.0920, 1.1467) -- (0.2750, 1.0920, 1.1407) -- cycle;
\fill[blue!22.2, opacity=0.7] (0.2750, 1.0920, 1.1407) -- (0.3240, 1.0920, 1.1467) -- (0.3240, 1.1430, 1.1491) -- (0.2750, 1.1430, 1.1431) -- cycle;
\fill[blue!18.1, opacity=0.7] (0.2750, 1.1430, 1.1431) -- (0.3240, 1.1430, 1.1491) -- (0.3240, 1.1940, 1.1512) -- (0.2750, 1.1940, 1.1452) -- cycle;
\fill[blue!16.0, opacity=0.7] (0.2750, 1.1940, 1.1452) -- (0.3240, 1.1940, 1.1512) -- (0.3240, 1.2450, 1.1530) -- (0.2750, 1.2450, 1.1470) -- cycle;
\fill[blue!15.2, opacity=0.7] (0.2750, 1.2450, 1.1470) -- (0.3240, 1.2450, 1.1530) -- (0.3240, 1.2960, 1.1545) -- (0.2750, 1.2960, 1.1484) -- cycle;
\fill[blue!15.1, opacity=0.7] (0.2750, 1.2960, 1.1484) -- (0.3240, 1.2960, 1.1545) -- (0.3240, 1.3470, 1.1556) -- (0.2750, 1.3470, 1.1496) -- cycle;
\fill[blue!15.0, opacity=0.7] (0.2750, 1.3470, 1.1496) -- (0.3240, 1.3470, 1.1556) -- (0.3240, 1.3980, 1.1564) -- (0.2750, 1.3980, 1.1504) -- cycle;
\fill[blue!15.0, opacity=0.7] (0.2750, 1.3980, 1.1504) -- (0.3240, 1.3980, 1.1564) -- (0.3240, 1.4490, 1.1569) -- (0.2750, 1.4490, 1.1509) -- cycle;
\fill[blue!15.0, opacity=0.7] (0.2750, 1.4490, 1.1509) -- (0.3240, 1.4490, 1.1569) -- (0.3240, 1.5000, 1.1571) -- (0.2750, 1.5000, 1.1511) -- cycle;
\fill[blue!15.0, opacity=0.7] (0.2750, 1.5000, 1.1511) -- (0.3240, 1.5000, 1.1571) -- (0.3240, 1.5510, 1.1569) -- (0.2750, 1.5510, 1.1509) -- cycle;
\fill[blue!15.0, opacity=0.7] (0.2750, 1.5510, 1.1509) -- (0.3240, 1.5510, 1.1569) -- (0.3240, 1.6020, 1.1564) -- (0.2750, 1.6020, 1.1504) -- cycle;
\fill[blue!15.0, opacity=0.7] (0.2750, 1.6020, 1.1504) -- (0.3240, 1.6020, 1.1564) -- (0.3240, 1.6530, 1.1556) -- (0.2750, 1.6530, 1.1496) -- cycle;
\fill[blue!15.0, opacity=0.7] (0.2750, 1.6530, 1.1496) -- (0.3240, 1.6530, 1.1556) -- (0.3240, 1.7040, 1.1545) -- (0.2750, 1.7040, 1.1484) -- cycle;
\fill[blue!15.0, opacity=0.7] (0.2750, 1.7040, 1.1484) -- (0.3240, 1.7040, 1.1545) -- (0.3240, 1.7550, 1.1530) -- (0.2750, 1.7550, 1.1470) -- cycle;
\fill[blue!15.0, opacity=0.7] (0.2750, 1.7550, 1.1470) -- (0.3240, 1.7550, 1.1530) -- (0.3240, 1.8060, 1.1512) -- (0.2750, 1.8060, 1.1452) -- cycle;
\fill[blue!15.0, opacity=0.7] (0.2750, 1.8060, 1.1452) -- (0.3240, 1.8060, 1.1512) -- (0.3240, 1.8570, 1.1491) -- (0.2750, 1.8570, 1.1431) -- cycle;
\fill[blue!15.0, opacity=0.7] (0.2750, 1.8570, 1.1431) -- (0.3240, 1.8570, 1.1491) -- (0.3240, 1.9080, 1.1467) -- (0.2750, 1.9080, 1.1407) -- cycle;
\fill[blue!15.0, opacity=0.7] (0.2750, 1.9080, 1.1407) -- (0.3240, 1.9080, 1.1467) -- (0.3240, 1.9590, 1.1440) -- (0.2750, 1.9590, 1.1380) -- cycle;
\fill[blue!15.0, opacity=0.7] (0.2750, 1.9590, 1.1380) -- (0.3240, 1.9590, 1.1440) -- (0.3240, 2.0100, 1.1410) -- (0.2750, 2.0100, 1.1350) -- cycle;
\fill[blue!15.2, opacity=0.7] (0.2750, 2.0100, 1.1350) -- (0.3240, 2.0100, 1.1410) -- (0.3240, 2.0610, 1.1377) -- (0.2750, 2.0610, 1.1317) -- cycle;
\fill[blue!15.6, opacity=0.7] (0.2750, 2.0610, 1.1317) -- (0.3240, 2.0610, 1.1377) -- (0.3240, 2.1120, 1.1342) -- (0.2750, 2.1120, 1.1281) -- cycle;
\fill[blue!16.7, opacity=0.7] (0.2750, 2.1120, 1.1281) -- (0.3240, 2.1120, 1.1342) -- (0.3240, 2.1630, 1.1303) -- (0.2750, 2.1630, 1.1243) -- cycle;
\fill[blue!18.2, opacity=0.7] (0.2750, 2.1630, 1.1243) -- (0.3240, 2.1630, 1.1303) -- (0.3240, 2.2140, 1.1263) -- (0.2750, 2.2140, 1.1202) -- cycle;
\fill[blue!19.0, opacity=0.7] (0.2750, 2.2140, 1.1202) -- (0.3240, 2.2140, 1.1263) -- (0.3240, 2.2650, 1.1219) -- (0.2750, 2.2650, 1.1159) -- cycle;
\fill[blue!17.9, opacity=0.7] (0.2750, 2.2650, 1.1159) -- (0.3240, 2.2650, 1.1219) -- (0.3240, 2.3160, 1.1174) -- (0.2750, 2.3160, 1.1114) -- cycle;
\fill[blue!16.0, opacity=0.7] (0.2750, 2.3160, 1.1114) -- (0.3240, 2.3160, 1.1174) -- (0.3240, 2.3670, 1.1126) -- (0.2750, 2.3670, 1.1066) -- cycle;
\fill[blue!15.1, opacity=0.7] (0.2750, 2.3670, 1.1066) -- (0.3240, 2.3670, 1.1126) -- (0.3240, 2.4180, 1.1076) -- (0.2750, 2.4180, 1.1016) -- cycle;
\fill[blue!15.0, opacity=0.7] (0.2750, 2.4180, 1.1016) -- (0.3240, 2.4180, 1.1076) -- (0.3240, 2.4690, 1.1024) -- (0.2750, 2.4690, 1.0964) -- cycle;
\fill[blue!15.0, opacity=0.7] (0.2750, 2.4690, 1.0964) -- (0.3240, 2.4690, 1.1024) -- (0.3240, 2.5200, 1.0971) -- (0.2750, 2.5200, 1.0911) -- cycle;
\fill[blue!15.0, opacity=0.7] (0.2750, 2.5200, 1.0911) -- (0.3240, 2.5200, 1.0971) -- (0.3240, 2.5710, 1.0916) -- (0.2750, 2.5710, 1.0855) -- cycle;
\fill[blue!15.0, opacity=0.7] (0.2750, 2.5710, 1.0855) -- (0.3240, 2.5710, 1.0916) -- (0.3240, 2.6220, 1.0859) -- (0.2750, 2.6220, 1.0799) -- cycle;
\fill[blue!15.0, opacity=0.7] (0.2750, 2.6220, 1.0799) -- (0.3240, 2.6220, 1.0859) -- (0.3240, 2.6730, 1.0801) -- (0.2750, 2.6730, 1.0741) -- cycle;
\fill[blue!15.0, opacity=0.7] (0.2750, 2.6730, 1.0741) -- (0.3240, 2.6730, 1.0801) -- (0.3240, 2.7240, 1.0742) -- (0.2750, 2.7240, 1.0681) -- cycle;
\fill[blue!15.0, opacity=0.7] (0.2750, 2.7240, 1.0681) -- (0.3240, 2.7240, 1.0742) -- (0.3240, 2.7750, 1.0681) -- (0.2750, 2.7750, 1.0621) -- cycle;
\fill[blue!15.0, opacity=0.7] (0.2750, 2.7750, 1.0621) -- (0.3240, 2.7750, 1.0681) -- (0.3240, 2.8260, 1.0620) -- (0.2750, 2.8260, 1.0560) -- cycle;
\fill[blue!15.0, opacity=0.7] (0.2750, 2.8260, 1.0560) -- (0.3240, 2.8260, 1.0620) -- (0.3240, 2.8770, 1.0559) -- (0.2750, 2.8770, 1.0498) -- cycle;
\fill[blue!15.0, opacity=0.7] (0.2750, 2.8770, 1.0498) -- (0.3240, 2.8770, 1.0559) -- (0.3240, 2.9280, 1.0496) -- (0.2750, 2.9280, 1.0436) -- cycle;
\fill[blue!15.0, opacity=0.7] (0.2750, 2.9280, 1.0436) -- (0.3240, 2.9280, 1.0496) -- (0.3240, 2.9790, 1.0434) -- (0.2750, 2.9790, 1.0373) -- cycle;
\fill[blue!15.0, opacity=0.7] (0.2750, 2.9790, 1.0373) -- (0.3240, 2.9790, 1.0434) -- (0.3240, 3.0300, 1.0371) -- (0.2750, 3.0300, 1.0311) -- cycle;
\fill[blue!15.0, opacity=0.7] (0.3240, -0.0300, 1.0371) -- (0.3730, -0.0300, 1.0430) -- (0.3730, 0.0210, 1.0493) -- (0.3240, 0.0210, 1.0434) -- cycle;
\fill[blue!15.0, opacity=0.7] (0.3240, 0.0210, 1.0434) -- (0.3730, 0.0210, 1.0493) -- (0.3730, 0.0720, 1.0555) -- (0.3240, 0.0720, 1.0496) -- cycle;
\fill[blue!15.0, opacity=0.7] (0.3240, 0.0720, 1.0496) -- (0.3730, 0.0720, 1.0555) -- (0.3730, 0.1230, 1.0618) -- (0.3240, 0.1230, 1.0559) -- cycle;
\fill[blue!15.0, opacity=0.7] (0.3240, 0.1230, 1.0559) -- (0.3730, 0.1230, 1.0618) -- (0.3730, 0.1740, 1.0680) -- (0.3240, 0.1740, 1.0620) -- cycle;
\fill[blue!15.0, opacity=0.7] (0.3240, 0.1740, 1.0620) -- (0.3730, 0.1740, 1.0680) -- (0.3730, 0.2250, 1.0741) -- (0.3240, 0.2250, 1.0681) -- cycle;
\fill[blue!15.1, opacity=0.7] (0.3240, 0.2250, 1.0681) -- (0.3730, 0.2250, 1.0741) -- (0.3730, 0.2760, 1.0801) -- (0.3240, 0.2760, 1.0742) -- cycle;
\fill[blue!15.2, opacity=0.7] (0.3240, 0.2760, 1.0742) -- (0.3730, 0.2760, 1.0801) -- (0.3730, 0.3270, 1.0860) -- (0.3240, 0.3270, 1.0801) -- cycle;
\fill[blue!15.0, opacity=0.7] (0.3240, 0.3270, 1.0801) -- (0.3730, 0.3270, 1.0860) -- (0.3730, 0.3780, 1.0918) -- (0.3240, 0.3780, 1.0859) -- cycle;
\fill[blue!15.0, opacity=0.7] (0.3240, 0.3780, 1.0859) -- (0.3730, 0.3780, 1.0918) -- (0.3730, 0.4290, 1.0975) -- (0.3240, 0.4290, 1.0916) -- cycle;
\fill[blue!15.0, opacity=0.7] (0.3240, 0.4290, 1.0916) -- (0.3730, 0.4290, 1.0975) -- (0.3730, 0.4800, 1.1030) -- (0.3240, 0.4800, 1.0971) -- cycle;
\fill[blue!15.0, opacity=0.7] (0.3240, 0.4800, 1.0971) -- (0.3730, 0.4800, 1.1030) -- (0.3730, 0.5310, 1.1084) -- (0.3240, 0.5310, 1.1024) -- cycle;
\fill[blue!15.0, opacity=0.7] (0.3240, 0.5310, 1.1024) -- (0.3730, 0.5310, 1.1084) -- (0.3730, 0.5820, 1.1135) -- (0.3240, 0.5820, 1.1076) -- cycle;
\fill[blue!15.0, opacity=0.7] (0.3240, 0.5820, 1.1076) -- (0.3730, 0.5820, 1.1135) -- (0.3730, 0.6330, 1.1185) -- (0.3240, 0.6330, 1.1126) -- cycle;
\fill[blue!15.0, opacity=0.7] (0.3240, 0.6330, 1.1126) -- (0.3730, 0.6330, 1.1185) -- (0.3730, 0.6840, 1.1233) -- (0.3240, 0.6840, 1.1174) -- cycle;
\fill[blue!15.0, opacity=0.7] (0.3240, 0.6840, 1.1174) -- (0.3730, 0.6840, 1.1233) -- (0.3730, 0.7350, 1.1279) -- (0.3240, 0.7350, 1.1219) -- cycle;
\fill[blue!15.0, opacity=0.7] (0.3240, 0.7350, 1.1219) -- (0.3730, 0.7350, 1.1279) -- (0.3730, 0.7860, 1.1322) -- (0.3240, 0.7860, 1.1263) -- cycle;
\fill[blue!15.8, opacity=0.7] (0.3240, 0.7860, 1.1263) -- (0.3730, 0.7860, 1.1322) -- (0.3730, 0.8370, 1.1363) -- (0.3240, 0.8370, 1.1303) -- cycle;
\fill[blue!20.6, opacity=0.7] (0.3240, 0.8370, 1.1303) -- (0.3730, 0.8370, 1.1363) -- (0.3730, 0.8880, 1.1401) -- (0.3240, 0.8880, 1.1342) -- cycle;
\fill[blue!27.4, opacity=0.7] (0.3240, 0.8880, 1.1342) -- (0.3730, 0.8880, 1.1401) -- (0.3730, 0.9390, 1.1436) -- (0.3240, 0.9390, 1.1377) -- cycle;
\fill[blue!28.0, opacity=0.7] (0.3240, 0.9390, 1.1377) -- (0.3730, 0.9390, 1.1436) -- (0.3730, 0.9900, 1.1469) -- (0.3240, 0.9900, 1.1410) -- cycle;
\fill[blue!22.5, opacity=0.7] (0.3240, 0.9900, 1.1410) -- (0.3730, 0.9900, 1.1469) -- (0.3730, 1.0410, 1.1499) -- (0.3240, 1.0410, 1.1440) -- cycle;
\fill[blue!17.5, opacity=0.7] (0.3240, 1.0410, 1.1440) -- (0.3730, 1.0410, 1.1499) -- (0.3730, 1.0920, 1.1526) -- (0.3240, 1.0920, 1.1467) -- cycle;
\fill[blue!15.5, opacity=0.7] (0.3240, 1.0920, 1.1467) -- (0.3730, 1.0920, 1.1526) -- (0.3730, 1.1430, 1.1550) -- (0.3240, 1.1430, 1.1491) -- cycle;
\fill[blue!15.1, opacity=0.7] (0.3240, 1.1430, 1.1491) -- (0.3730, 1.1430, 1.1550) -- (0.3730, 1.1940, 1.1571) -- (0.3240, 1.1940, 1.1512) -- cycle;
\fill[blue!15.0, opacity=0.7] (0.3240, 1.1940, 1.1512) -- (0.3730, 1.1940, 1.1571) -- (0.3730, 1.2450, 1.1589) -- (0.3240, 1.2450, 1.1530) -- cycle;
\fill[blue!15.0, opacity=0.7] (0.3240, 1.2450, 1.1530) -- (0.3730, 1.2450, 1.1589) -- (0.3730, 1.2960, 1.1604) -- (0.3240, 1.2960, 1.1545) -- cycle;
\fill[blue!15.0, opacity=0.7] (0.3240, 1.2960, 1.1545) -- (0.3730, 1.2960, 1.1604) -- (0.3730, 1.3470, 1.1615) -- (0.3240, 1.3470, 1.1556) -- cycle;
\fill[blue!15.0, opacity=0.7] (0.3240, 1.3470, 1.1556) -- (0.3730, 1.3470, 1.1615) -- (0.3730, 1.3980, 1.1623) -- (0.3240, 1.3980, 1.1564) -- cycle;
\fill[blue!15.0, opacity=0.7] (0.3240, 1.3980, 1.1564) -- (0.3730, 1.3980, 1.1623) -- (0.3730, 1.4490, 1.1628) -- (0.3240, 1.4490, 1.1569) -- cycle;
\fill[blue!15.0, opacity=0.7] (0.3240, 1.4490, 1.1569) -- (0.3730, 1.4490, 1.1628) -- (0.3730, 1.5000, 1.1630) -- (0.3240, 1.5000, 1.1571) -- cycle;
\fill[blue!15.0, opacity=0.7] (0.3240, 1.5000, 1.1571) -- (0.3730, 1.5000, 1.1630) -- (0.3730, 1.5510, 1.1628) -- (0.3240, 1.5510, 1.1569) -- cycle;
\fill[blue!15.0, opacity=0.7] (0.3240, 1.5510, 1.1569) -- (0.3730, 1.5510, 1.1628) -- (0.3730, 1.6020, 1.1623) -- (0.3240, 1.6020, 1.1564) -- cycle;
\fill[blue!15.0, opacity=0.7] (0.3240, 1.6020, 1.1564) -- (0.3730, 1.6020, 1.1623) -- (0.3730, 1.6530, 1.1615) -- (0.3240, 1.6530, 1.1556) -- cycle;
\fill[blue!15.0, opacity=0.7] (0.3240, 1.6530, 1.1556) -- (0.3730, 1.6530, 1.1615) -- (0.3730, 1.7040, 1.1604) -- (0.3240, 1.7040, 1.1545) -- cycle;
\fill[blue!15.0, opacity=0.7] (0.3240, 1.7040, 1.1545) -- (0.3730, 1.7040, 1.1604) -- (0.3730, 1.7550, 1.1589) -- (0.3240, 1.7550, 1.1530) -- cycle;
\fill[blue!15.0, opacity=0.7] (0.3240, 1.7550, 1.1530) -- (0.3730, 1.7550, 1.1589) -- (0.3730, 1.8060, 1.1571) -- (0.3240, 1.8060, 1.1512) -- cycle;
\fill[blue!15.0, opacity=0.7] (0.3240, 1.8060, 1.1512) -- (0.3730, 1.8060, 1.1571) -- (0.3730, 1.8570, 1.1550) -- (0.3240, 1.8570, 1.1491) -- cycle;
\fill[blue!15.0, opacity=0.7] (0.3240, 1.8570, 1.1491) -- (0.3730, 1.8570, 1.1550) -- (0.3730, 1.9080, 1.1526) -- (0.3240, 1.9080, 1.1467) -- cycle;
\fill[blue!15.0, opacity=0.7] (0.3240, 1.9080, 1.1467) -- (0.3730, 1.9080, 1.1526) -- (0.3730, 1.9590, 1.1499) -- (0.3240, 1.9590, 1.1440) -- cycle;
\fill[blue!15.0, opacity=0.7] (0.3240, 1.9590, 1.1440) -- (0.3730, 1.9590, 1.1499) -- (0.3730, 2.0100, 1.1469) -- (0.3240, 2.0100, 1.1410) -- cycle;
\fill[blue!15.0, opacity=0.7] (0.3240, 2.0100, 1.1410) -- (0.3730, 2.0100, 1.1469) -- (0.3730, 2.0610, 1.1436) -- (0.3240, 2.0610, 1.1377) -- cycle;
\fill[blue!15.0, opacity=0.7] (0.3240, 2.0610, 1.1377) -- (0.3730, 2.0610, 1.1436) -- (0.3730, 2.1120, 1.1401) -- (0.3240, 2.1120, 1.1342) -- cycle;
\fill[blue!15.0, opacity=0.7] (0.3240, 2.1120, 1.1342) -- (0.3730, 2.1120, 1.1401) -- (0.3730, 2.1630, 1.1363) -- (0.3240, 2.1630, 1.1303) -- cycle;
\fill[blue!15.3, opacity=0.7] (0.3240, 2.1630, 1.1303) -- (0.3730, 2.1630, 1.1363) -- (0.3730, 2.2140, 1.1322) -- (0.3240, 2.2140, 1.1263) -- cycle;
\fill[blue!16.1, opacity=0.7] (0.3240, 2.2140, 1.1263) -- (0.3730, 2.2140, 1.1322) -- (0.3730, 2.2650, 1.1279) -- (0.3240, 2.2650, 1.1219) -- cycle;
\fill[blue!17.8, opacity=0.7] (0.3240, 2.2650, 1.1219) -- (0.3730, 2.2650, 1.1279) -- (0.3730, 2.3160, 1.1233) -- (0.3240, 2.3160, 1.1174) -- cycle;
\fill[blue!18.6, opacity=0.7] (0.3240, 2.3160, 1.1174) -- (0.3730, 2.3160, 1.1233) -- (0.3730, 2.3670, 1.1185) -- (0.3240, 2.3670, 1.1126) -- cycle;
\fill[blue!17.2, opacity=0.7] (0.3240, 2.3670, 1.1126) -- (0.3730, 2.3670, 1.1185) -- (0.3730, 2.4180, 1.1135) -- (0.3240, 2.4180, 1.1076) -- cycle;
\fill[blue!15.5, opacity=0.7] (0.3240, 2.4180, 1.1076) -- (0.3730, 2.4180, 1.1135) -- (0.3730, 2.4690, 1.1084) -- (0.3240, 2.4690, 1.1024) -- cycle;
\fill[blue!15.0, opacity=0.7] (0.3240, 2.4690, 1.1024) -- (0.3730, 2.4690, 1.1084) -- (0.3730, 2.5200, 1.1030) -- (0.3240, 2.5200, 1.0971) -- cycle;
\fill[blue!15.0, opacity=0.7] (0.3240, 2.5200, 1.0971) -- (0.3730, 2.5200, 1.1030) -- (0.3730, 2.5710, 1.0975) -- (0.3240, 2.5710, 1.0916) -- cycle;
\fill[blue!15.0, opacity=0.7] (0.3240, 2.5710, 1.0916) -- (0.3730, 2.5710, 1.0975) -- (0.3730, 2.6220, 1.0918) -- (0.3240, 2.6220, 1.0859) -- cycle;
\fill[blue!15.0, opacity=0.7] (0.3240, 2.6220, 1.0859) -- (0.3730, 2.6220, 1.0918) -- (0.3730, 2.6730, 1.0860) -- (0.3240, 2.6730, 1.0801) -- cycle;
\fill[blue!15.0, opacity=0.7] (0.3240, 2.6730, 1.0801) -- (0.3730, 2.6730, 1.0860) -- (0.3730, 2.7240, 1.0801) -- (0.3240, 2.7240, 1.0742) -- cycle;
\fill[blue!15.0, opacity=0.7] (0.3240, 2.7240, 1.0742) -- (0.3730, 2.7240, 1.0801) -- (0.3730, 2.7750, 1.0741) -- (0.3240, 2.7750, 1.0681) -- cycle;
\fill[blue!15.0, opacity=0.7] (0.3240, 2.7750, 1.0681) -- (0.3730, 2.7750, 1.0741) -- (0.3730, 2.8260, 1.0680) -- (0.3240, 2.8260, 1.0620) -- cycle;
\fill[blue!15.0, opacity=0.7] (0.3240, 2.8260, 1.0620) -- (0.3730, 2.8260, 1.0680) -- (0.3730, 2.8770, 1.0618) -- (0.3240, 2.8770, 1.0559) -- cycle;
\fill[blue!15.0, opacity=0.7] (0.3240, 2.8770, 1.0559) -- (0.3730, 2.8770, 1.0618) -- (0.3730, 2.9280, 1.0555) -- (0.3240, 2.9280, 1.0496) -- cycle;
\fill[blue!15.0, opacity=0.7] (0.3240, 2.9280, 1.0496) -- (0.3730, 2.9280, 1.0555) -- (0.3730, 2.9790, 1.0493) -- (0.3240, 2.9790, 1.0434) -- cycle;
\fill[blue!15.0, opacity=0.7] (0.3240, 2.9790, 1.0434) -- (0.3730, 2.9790, 1.0493) -- (0.3730, 3.0300, 1.0430) -- (0.3240, 3.0300, 1.0371) -- cycle;
\fill[blue!15.0, opacity=0.7] (0.3730, -0.0300, 1.0430) -- (0.4220, -0.0300, 1.0488) -- (0.4220, 0.0210, 1.0551) -- (0.3730, 0.0210, 1.0493) -- cycle;
\fill[blue!15.0, opacity=0.7] (0.3730, 0.0210, 1.0493) -- (0.4220, 0.0210, 1.0551) -- (0.4220, 0.0720, 1.0614) -- (0.3730, 0.0720, 1.0555) -- cycle;
\fill[blue!15.0, opacity=0.7] (0.3730, 0.0720, 1.0555) -- (0.4220, 0.0720, 1.0614) -- (0.4220, 0.1230, 1.0676) -- (0.3730, 0.1230, 1.0618) -- cycle;
\fill[blue!15.0, opacity=0.7] (0.3730, 0.1230, 1.0618) -- (0.4220, 0.1230, 1.0676) -- (0.4220, 0.1740, 1.0738) -- (0.3730, 0.1740, 1.0680) -- cycle;
\fill[blue!15.0, opacity=0.7] (0.3730, 0.1740, 1.0680) -- (0.4220, 0.1740, 1.0738) -- (0.4220, 0.2250, 1.0799) -- (0.3730, 0.2250, 1.0741) -- cycle;
\fill[blue!15.2, opacity=0.7] (0.3730, 0.2250, 1.0741) -- (0.4220, 0.2250, 1.0799) -- (0.4220, 0.2760, 1.0859) -- (0.3730, 0.2760, 1.0801) -- cycle;
\fill[blue!15.1, opacity=0.7] (0.3730, 0.2760, 1.0801) -- (0.4220, 0.2760, 1.0859) -- (0.4220, 0.3270, 1.0918) -- (0.3730, 0.3270, 1.0860) -- cycle;
\fill[blue!15.0, opacity=0.7] (0.3730, 0.3270, 1.0860) -- (0.4220, 0.3270, 1.0918) -- (0.4220, 0.3780, 1.0976) -- (0.3730, 0.3780, 1.0918) -- cycle;
\fill[blue!15.0, opacity=0.7] (0.3730, 0.3780, 1.0918) -- (0.4220, 0.3780, 1.0976) -- (0.4220, 0.4290, 1.1033) -- (0.3730, 0.4290, 1.0975) -- cycle;
\fill[blue!15.0, opacity=0.7] (0.3730, 0.4290, 1.0975) -- (0.4220, 0.4290, 1.1033) -- (0.4220, 0.4800, 1.1088) -- (0.3730, 0.4800, 1.1030) -- cycle;
\fill[blue!15.0, opacity=0.7] (0.3730, 0.4800, 1.1030) -- (0.4220, 0.4800, 1.1088) -- (0.4220, 0.5310, 1.1142) -- (0.3730, 0.5310, 1.1084) -- cycle;
\fill[blue!15.0, opacity=0.7] (0.3730, 0.5310, 1.1084) -- (0.4220, 0.5310, 1.1142) -- (0.4220, 0.5820, 1.1193) -- (0.3730, 0.5820, 1.1135) -- cycle;
\fill[blue!15.0, opacity=0.7] (0.3730, 0.5820, 1.1135) -- (0.4220, 0.5820, 1.1193) -- (0.4220, 0.6330, 1.1243) -- (0.3730, 0.6330, 1.1185) -- cycle;
\fill[blue!15.0, opacity=0.7] (0.3730, 0.6330, 1.1185) -- (0.4220, 0.6330, 1.1243) -- (0.4220, 0.6840, 1.1291) -- (0.3730, 0.6840, 1.1233) -- cycle;
\fill[blue!15.1, opacity=0.7] (0.3730, 0.6840, 1.1233) -- (0.4220, 0.6840, 1.1291) -- (0.4220, 0.7350, 1.1337) -- (0.3730, 0.7350, 1.1279) -- cycle;
\fill[blue!17.0, opacity=0.7] (0.3730, 0.7350, 1.1279) -- (0.4220, 0.7350, 1.1337) -- (0.4220, 0.7860, 1.1380) -- (0.3730, 0.7860, 1.1322) -- cycle;
\fill[blue!24.6, opacity=0.7] (0.3730, 0.7860, 1.1322) -- (0.4220, 0.7860, 1.1380) -- (0.4220, 0.8370, 1.1421) -- (0.3730, 0.8370, 1.1363) -- cycle;
\fill[blue!29.9, opacity=0.7] (0.3730, 0.8370, 1.1363) -- (0.4220, 0.8370, 1.1421) -- (0.4220, 0.8880, 1.1459) -- (0.3730, 0.8880, 1.1401) -- cycle;
\fill[blue!25.5, opacity=0.7] (0.3730, 0.8880, 1.1401) -- (0.4220, 0.8880, 1.1459) -- (0.4220, 0.9390, 1.1494) -- (0.3730, 0.9390, 1.1436) -- cycle;
\fill[blue!18.5, opacity=0.7] (0.3730, 0.9390, 1.1436) -- (0.4220, 0.9390, 1.1494) -- (0.4220, 0.9900, 1.1527) -- (0.3730, 0.9900, 1.1469) -- cycle;
\fill[blue!15.6, opacity=0.7] (0.3730, 0.9900, 1.1469) -- (0.4220, 0.9900, 1.1527) -- (0.4220, 1.0410, 1.1557) -- (0.3730, 1.0410, 1.1499) -- cycle;
\fill[blue!15.1, opacity=0.7] (0.3730, 1.0410, 1.1499) -- (0.4220, 1.0410, 1.1557) -- (0.4220, 1.0920, 1.1584) -- (0.3730, 1.0920, 1.1526) -- cycle;
\fill[blue!15.0, opacity=0.7] (0.3730, 1.0920, 1.1526) -- (0.4220, 1.0920, 1.1584) -- (0.4220, 1.1430, 1.1608) -- (0.3730, 1.1430, 1.1550) -- cycle;
\fill[blue!15.0, opacity=0.7] (0.3730, 1.1430, 1.1550) -- (0.4220, 1.1430, 1.1608) -- (0.4220, 1.1940, 1.1629) -- (0.3730, 1.1940, 1.1571) -- cycle;
\fill[blue!15.0, opacity=0.7] (0.3730, 1.1940, 1.1571) -- (0.4220, 1.1940, 1.1629) -- (0.4220, 1.2450, 1.1647) -- (0.3730, 1.2450, 1.1589) -- cycle;
\fill[blue!15.0, opacity=0.7] (0.3730, 1.2450, 1.1589) -- (0.4220, 1.2450, 1.1647) -- (0.4220, 1.2960, 1.1662) -- (0.3730, 1.2960, 1.1604) -- cycle;
\fill[blue!15.0, opacity=0.7] (0.3730, 1.2960, 1.1604) -- (0.4220, 1.2960, 1.1662) -- (0.4220, 1.3470, 1.1673) -- (0.3730, 1.3470, 1.1615) -- cycle;
\fill[blue!15.0, opacity=0.7] (0.3730, 1.3470, 1.1615) -- (0.4220, 1.3470, 1.1673) -- (0.4220, 1.3980, 1.1682) -- (0.3730, 1.3980, 1.1623) -- cycle;
\fill[blue!15.0, opacity=0.7] (0.3730, 1.3980, 1.1623) -- (0.4220, 1.3980, 1.1682) -- (0.4220, 1.4490, 1.1686) -- (0.3730, 1.4490, 1.1628) -- cycle;
\fill[blue!15.0, opacity=0.7] (0.3730, 1.4490, 1.1628) -- (0.4220, 1.4490, 1.1686) -- (0.4220, 1.5000, 1.1688) -- (0.3730, 1.5000, 1.1630) -- cycle;
\fill[blue!15.0, opacity=0.7] (0.3730, 1.5000, 1.1630) -- (0.4220, 1.5000, 1.1688) -- (0.4220, 1.5510, 1.1686) -- (0.3730, 1.5510, 1.1628) -- cycle;
\fill[blue!15.0, opacity=0.7] (0.3730, 1.5510, 1.1628) -- (0.4220, 1.5510, 1.1686) -- (0.4220, 1.6020, 1.1682) -- (0.3730, 1.6020, 1.1623) -- cycle;
\fill[blue!15.0, opacity=0.7] (0.3730, 1.6020, 1.1623) -- (0.4220, 1.6020, 1.1682) -- (0.4220, 1.6530, 1.1673) -- (0.3730, 1.6530, 1.1615) -- cycle;
\fill[blue!15.0, opacity=0.7] (0.3730, 1.6530, 1.1615) -- (0.4220, 1.6530, 1.1673) -- (0.4220, 1.7040, 1.1662) -- (0.3730, 1.7040, 1.1604) -- cycle;
\fill[blue!15.0, opacity=0.7] (0.3730, 1.7040, 1.1604) -- (0.4220, 1.7040, 1.1662) -- (0.4220, 1.7550, 1.1647) -- (0.3730, 1.7550, 1.1589) -- cycle;
\fill[blue!15.0, opacity=0.7] (0.3730, 1.7550, 1.1589) -- (0.4220, 1.7550, 1.1647) -- (0.4220, 1.8060, 1.1629) -- (0.3730, 1.8060, 1.1571) -- cycle;
\fill[blue!15.0, opacity=0.7] (0.3730, 1.8060, 1.1571) -- (0.4220, 1.8060, 1.1629) -- (0.4220, 1.8570, 1.1608) -- (0.3730, 1.8570, 1.1550) -- cycle;
\fill[blue!15.0, opacity=0.7] (0.3730, 1.8570, 1.1550) -- (0.4220, 1.8570, 1.1608) -- (0.4220, 1.9080, 1.1584) -- (0.3730, 1.9080, 1.1526) -- cycle;
\fill[blue!15.0, opacity=0.7] (0.3730, 1.9080, 1.1526) -- (0.4220, 1.9080, 1.1584) -- (0.4220, 1.9590, 1.1557) -- (0.3730, 1.9590, 1.1499) -- cycle;
\fill[blue!15.0, opacity=0.7] (0.3730, 1.9590, 1.1499) -- (0.4220, 1.9590, 1.1557) -- (0.4220, 2.0100, 1.1527) -- (0.3730, 2.0100, 1.1469) -- cycle;
\fill[blue!15.0, opacity=0.7] (0.3730, 2.0100, 1.1469) -- (0.4220, 2.0100, 1.1527) -- (0.4220, 2.0610, 1.1494) -- (0.3730, 2.0610, 1.1436) -- cycle;
\fill[blue!15.0, opacity=0.7] (0.3730, 2.0610, 1.1436) -- (0.4220, 2.0610, 1.1494) -- (0.4220, 2.1120, 1.1459) -- (0.3730, 2.1120, 1.1401) -- cycle;
\fill[blue!15.0, opacity=0.7] (0.3730, 2.1120, 1.1401) -- (0.4220, 2.1120, 1.1459) -- (0.4220, 2.1630, 1.1421) -- (0.3730, 2.1630, 1.1363) -- cycle;
\fill[blue!15.0, opacity=0.7] (0.3730, 2.1630, 1.1363) -- (0.4220, 2.1630, 1.1421) -- (0.4220, 2.2140, 1.1380) -- (0.3730, 2.2140, 1.1322) -- cycle;
\fill[blue!15.0, opacity=0.7] (0.3730, 2.2140, 1.1322) -- (0.4220, 2.2140, 1.1380) -- (0.4220, 2.2650, 1.1337) -- (0.3730, 2.2650, 1.1279) -- cycle;
\fill[blue!15.2, opacity=0.7] (0.3730, 2.2650, 1.1279) -- (0.4220, 2.2650, 1.1337) -- (0.4220, 2.3160, 1.1291) -- (0.3730, 2.3160, 1.1233) -- cycle;
\fill[blue!16.2, opacity=0.7] (0.3730, 2.3160, 1.1233) -- (0.4220, 2.3160, 1.1291) -- (0.4220, 2.3670, 1.1243) -- (0.3730, 2.3670, 1.1185) -- cycle;
\fill[blue!17.9, opacity=0.7] (0.3730, 2.3670, 1.1185) -- (0.4220, 2.3670, 1.1243) -- (0.4220, 2.4180, 1.1193) -- (0.3730, 2.4180, 1.1135) -- cycle;
\fill[blue!17.9, opacity=0.7] (0.3730, 2.4180, 1.1135) -- (0.4220, 2.4180, 1.1193) -- (0.4220, 2.4690, 1.1142) -- (0.3730, 2.4690, 1.1084) -- cycle;
\fill[blue!16.0, opacity=0.7] (0.3730, 2.4690, 1.1084) -- (0.4220, 2.4690, 1.1142) -- (0.4220, 2.5200, 1.1088) -- (0.3730, 2.5200, 1.1030) -- cycle;
\fill[blue!15.1, opacity=0.7] (0.3730, 2.5200, 1.1030) -- (0.4220, 2.5200, 1.1088) -- (0.4220, 2.5710, 1.1033) -- (0.3730, 2.5710, 1.0975) -- cycle;
\fill[blue!15.0, opacity=0.7] (0.3730, 2.5710, 1.0975) -- (0.4220, 2.5710, 1.1033) -- (0.4220, 2.6220, 1.0976) -- (0.3730, 2.6220, 1.0918) -- cycle;
\fill[blue!15.0, opacity=0.7] (0.3730, 2.6220, 1.0918) -- (0.4220, 2.6220, 1.0976) -- (0.4220, 2.6730, 1.0918) -- (0.3730, 2.6730, 1.0860) -- cycle;
\fill[blue!15.0, opacity=0.7] (0.3730, 2.6730, 1.0860) -- (0.4220, 2.6730, 1.0918) -- (0.4220, 2.7240, 1.0859) -- (0.3730, 2.7240, 1.0801) -- cycle;
\fill[blue!15.0, opacity=0.7] (0.3730, 2.7240, 1.0801) -- (0.4220, 2.7240, 1.0859) -- (0.4220, 2.7750, 1.0799) -- (0.3730, 2.7750, 1.0741) -- cycle;
\fill[blue!15.0, opacity=0.7] (0.3730, 2.7750, 1.0741) -- (0.4220, 2.7750, 1.0799) -- (0.4220, 2.8260, 1.0738) -- (0.3730, 2.8260, 1.0680) -- cycle;
\fill[blue!15.0, opacity=0.7] (0.3730, 2.8260, 1.0680) -- (0.4220, 2.8260, 1.0738) -- (0.4220, 2.8770, 1.0676) -- (0.3730, 2.8770, 1.0618) -- cycle;
\fill[blue!15.0, opacity=0.7] (0.3730, 2.8770, 1.0618) -- (0.4220, 2.8770, 1.0676) -- (0.4220, 2.9280, 1.0614) -- (0.3730, 2.9280, 1.0555) -- cycle;
\fill[blue!15.0, opacity=0.7] (0.3730, 2.9280, 1.0555) -- (0.4220, 2.9280, 1.0614) -- (0.4220, 2.9790, 1.0551) -- (0.3730, 2.9790, 1.0493) -- cycle;
\fill[blue!15.0, opacity=0.7] (0.3730, 2.9790, 1.0493) -- (0.4220, 2.9790, 1.0551) -- (0.4220, 3.0300, 1.0488) -- (0.3730, 3.0300, 1.0430) -- cycle;
\fill[blue!15.0, opacity=0.7] (0.4220, -0.0300, 1.0488) -- (0.4710, -0.0300, 1.0545) -- (0.4710, 0.0210, 1.0608) -- (0.4220, 0.0210, 1.0551) -- cycle;
\fill[blue!15.0, opacity=0.7] (0.4220, 0.0210, 1.0551) -- (0.4710, 0.0210, 1.0608) -- (0.4710, 0.0720, 1.0670) -- (0.4220, 0.0720, 1.0614) -- cycle;
\fill[blue!15.0, opacity=0.7] (0.4220, 0.0720, 1.0614) -- (0.4710, 0.0720, 1.0670) -- (0.4710, 0.1230, 1.0733) -- (0.4220, 0.1230, 1.0676) -- cycle;
\fill[blue!15.0, opacity=0.7] (0.4220, 0.1230, 1.0676) -- (0.4710, 0.1230, 1.0733) -- (0.4710, 0.1740, 1.0794) -- (0.4220, 0.1740, 1.0738) -- cycle;
\fill[blue!15.2, opacity=0.7] (0.4220, 0.1740, 1.0738) -- (0.4710, 0.1740, 1.0794) -- (0.4710, 0.2250, 1.0855) -- (0.4220, 0.2250, 1.0799) -- cycle;
\fill[blue!15.1, opacity=0.7] (0.4220, 0.2250, 1.0799) -- (0.4710, 0.2250, 1.0855) -- (0.4710, 0.2760, 1.0916) -- (0.4220, 0.2760, 1.0859) -- cycle;
\fill[blue!15.0, opacity=0.7] (0.4220, 0.2760, 1.0859) -- (0.4710, 0.2760, 1.0916) -- (0.4710, 0.3270, 1.0975) -- (0.4220, 0.3270, 1.0918) -- cycle;
\fill[blue!15.0, opacity=0.7] (0.4220, 0.3270, 1.0918) -- (0.4710, 0.3270, 1.0975) -- (0.4710, 0.3780, 1.1033) -- (0.4220, 0.3780, 1.0976) -- cycle;
\fill[blue!15.0, opacity=0.7] (0.4220, 0.3780, 1.0976) -- (0.4710, 0.3780, 1.1033) -- (0.4710, 0.4290, 1.1090) -- (0.4220, 0.4290, 1.1033) -- cycle;
\fill[blue!15.0, opacity=0.7] (0.4220, 0.4290, 1.1033) -- (0.4710, 0.4290, 1.1090) -- (0.4710, 0.4800, 1.1145) -- (0.4220, 0.4800, 1.1088) -- cycle;
\fill[blue!15.0, opacity=0.7] (0.4220, 0.4800, 1.1088) -- (0.4710, 0.4800, 1.1145) -- (0.4710, 0.5310, 1.1198) -- (0.4220, 0.5310, 1.1142) -- cycle;
\fill[blue!15.0, opacity=0.7] (0.4220, 0.5310, 1.1142) -- (0.4710, 0.5310, 1.1198) -- (0.4710, 0.5820, 1.1250) -- (0.4220, 0.5820, 1.1193) -- cycle;
\fill[blue!15.0, opacity=0.7] (0.4220, 0.5820, 1.1193) -- (0.4710, 0.5820, 1.1250) -- (0.4710, 0.6330, 1.1300) -- (0.4220, 0.6330, 1.1243) -- cycle;
\fill[blue!15.2, opacity=0.7] (0.4220, 0.6330, 1.1243) -- (0.4710, 0.6330, 1.1300) -- (0.4710, 0.6840, 1.1348) -- (0.4220, 0.6840, 1.1291) -- cycle;
\fill[blue!18.3, opacity=0.7] (0.4220, 0.6840, 1.1291) -- (0.4710, 0.6840, 1.1348) -- (0.4710, 0.7350, 1.1393) -- (0.4220, 0.7350, 1.1337) -- cycle;
\fill[blue!27.8, opacity=0.7] (0.4220, 0.7350, 1.1337) -- (0.4710, 0.7350, 1.1393) -- (0.4710, 0.7860, 1.1437) -- (0.4220, 0.7860, 1.1380) -- cycle;
\fill[blue!30.4, opacity=0.7] (0.4220, 0.7860, 1.1380) -- (0.4710, 0.7860, 1.1437) -- (0.4710, 0.8370, 1.1477) -- (0.4220, 0.8370, 1.1421) -- cycle;
\fill[blue!22.6, opacity=0.7] (0.4220, 0.8370, 1.1421) -- (0.4710, 0.8370, 1.1477) -- (0.4710, 0.8880, 1.1516) -- (0.4220, 0.8880, 1.1459) -- cycle;
\fill[blue!16.5, opacity=0.7] (0.4220, 0.8880, 1.1459) -- (0.4710, 0.8880, 1.1516) -- (0.4710, 0.9390, 1.1551) -- (0.4220, 0.9390, 1.1494) -- cycle;
\fill[blue!15.1, opacity=0.7] (0.4220, 0.9390, 1.1494) -- (0.4710, 0.9390, 1.1551) -- (0.4710, 0.9900, 1.1584) -- (0.4220, 0.9900, 1.1527) -- cycle;
\fill[blue!15.0, opacity=0.7] (0.4220, 0.9900, 1.1527) -- (0.4710, 0.9900, 1.1584) -- (0.4710, 1.0410, 1.1614) -- (0.4220, 1.0410, 1.1557) -- cycle;
\fill[blue!15.0, opacity=0.7] (0.4220, 1.0410, 1.1557) -- (0.4710, 1.0410, 1.1614) -- (0.4710, 1.0920, 1.1641) -- (0.4220, 1.0920, 1.1584) -- cycle;
\fill[blue!15.0, opacity=0.7] (0.4220, 1.0920, 1.1584) -- (0.4710, 1.0920, 1.1641) -- (0.4710, 1.1430, 1.1665) -- (0.4220, 1.1430, 1.1608) -- cycle;
\fill[blue!15.0, opacity=0.7] (0.4220, 1.1430, 1.1608) -- (0.4710, 1.1430, 1.1665) -- (0.4710, 1.1940, 1.1686) -- (0.4220, 1.1940, 1.1629) -- cycle;
\fill[blue!15.0, opacity=0.7] (0.4220, 1.1940, 1.1629) -- (0.4710, 1.1940, 1.1686) -- (0.4710, 1.2450, 1.1704) -- (0.4220, 1.2450, 1.1647) -- cycle;
\fill[blue!15.0, opacity=0.7] (0.4220, 1.2450, 1.1647) -- (0.4710, 1.2450, 1.1704) -- (0.4710, 1.2960, 1.1719) -- (0.4220, 1.2960, 1.1662) -- cycle;
\fill[blue!15.0, opacity=0.7] (0.4220, 1.2960, 1.1662) -- (0.4710, 1.2960, 1.1719) -- (0.4710, 1.3470, 1.1730) -- (0.4220, 1.3470, 1.1673) -- cycle;
\fill[blue!15.0, opacity=0.7] (0.4220, 1.3470, 1.1673) -- (0.4710, 1.3470, 1.1730) -- (0.4710, 1.3980, 1.1738) -- (0.4220, 1.3980, 1.1682) -- cycle;
\fill[blue!15.0, opacity=0.7] (0.4220, 1.3980, 1.1682) -- (0.4710, 1.3980, 1.1738) -- (0.4710, 1.4490, 1.1743) -- (0.4220, 1.4490, 1.1686) -- cycle;
\fill[blue!15.0, opacity=0.7] (0.4220, 1.4490, 1.1686) -- (0.4710, 1.4490, 1.1743) -- (0.4710, 1.5000, 1.1745) -- (0.4220, 1.5000, 1.1688) -- cycle;
\fill[blue!15.0, opacity=0.7] (0.4220, 1.5000, 1.1688) -- (0.4710, 1.5000, 1.1745) -- (0.4710, 1.5510, 1.1743) -- (0.4220, 1.5510, 1.1686) -- cycle;
\fill[blue!15.1, opacity=0.7] (0.4220, 1.5510, 1.1686) -- (0.4710, 1.5510, 1.1743) -- (0.4710, 1.6020, 1.1738) -- (0.4220, 1.6020, 1.1682) -- cycle;
\fill[blue!15.1, opacity=0.7] (0.4220, 1.6020, 1.1682) -- (0.4710, 1.6020, 1.1738) -- (0.4710, 1.6530, 1.1730) -- (0.4220, 1.6530, 1.1673) -- cycle;
\fill[blue!15.1, opacity=0.7] (0.4220, 1.6530, 1.1673) -- (0.4710, 1.6530, 1.1730) -- (0.4710, 1.7040, 1.1719) -- (0.4220, 1.7040, 1.1662) -- cycle;
\fill[blue!15.0, opacity=0.7] (0.4220, 1.7040, 1.1662) -- (0.4710, 1.7040, 1.1719) -- (0.4710, 1.7550, 1.1704) -- (0.4220, 1.7550, 1.1647) -- cycle;
\fill[blue!15.0, opacity=0.7] (0.4220, 1.7550, 1.1647) -- (0.4710, 1.7550, 1.1704) -- (0.4710, 1.8060, 1.1686) -- (0.4220, 1.8060, 1.1629) -- cycle;
\fill[blue!15.0, opacity=0.7] (0.4220, 1.8060, 1.1629) -- (0.4710, 1.8060, 1.1686) -- (0.4710, 1.8570, 1.1665) -- (0.4220, 1.8570, 1.1608) -- cycle;
\fill[blue!15.0, opacity=0.7] (0.4220, 1.8570, 1.1608) -- (0.4710, 1.8570, 1.1665) -- (0.4710, 1.9080, 1.1641) -- (0.4220, 1.9080, 1.1584) -- cycle;
\fill[blue!15.0, opacity=0.7] (0.4220, 1.9080, 1.1584) -- (0.4710, 1.9080, 1.1641) -- (0.4710, 1.9590, 1.1614) -- (0.4220, 1.9590, 1.1557) -- cycle;
\fill[blue!15.0, opacity=0.7] (0.4220, 1.9590, 1.1557) -- (0.4710, 1.9590, 1.1614) -- (0.4710, 2.0100, 1.1584) -- (0.4220, 2.0100, 1.1527) -- cycle;
\fill[blue!15.0, opacity=0.7] (0.4220, 2.0100, 1.1527) -- (0.4710, 2.0100, 1.1584) -- (0.4710, 2.0610, 1.1551) -- (0.4220, 2.0610, 1.1494) -- cycle;
\fill[blue!15.0, opacity=0.7] (0.4220, 2.0610, 1.1494) -- (0.4710, 2.0610, 1.1551) -- (0.4710, 2.1120, 1.1516) -- (0.4220, 2.1120, 1.1459) -- cycle;
\fill[blue!15.0, opacity=0.7] (0.4220, 2.1120, 1.1459) -- (0.4710, 2.1120, 1.1516) -- (0.4710, 2.1630, 1.1477) -- (0.4220, 2.1630, 1.1421) -- cycle;
\fill[blue!15.0, opacity=0.7] (0.4220, 2.1630, 1.1421) -- (0.4710, 2.1630, 1.1477) -- (0.4710, 2.2140, 1.1437) -- (0.4220, 2.2140, 1.1380) -- cycle;
\fill[blue!15.0, opacity=0.7] (0.4220, 2.2140, 1.1380) -- (0.4710, 2.2140, 1.1437) -- (0.4710, 2.2650, 1.1393) -- (0.4220, 2.2650, 1.1337) -- cycle;
\fill[blue!15.0, opacity=0.7] (0.4220, 2.2650, 1.1337) -- (0.4710, 2.2650, 1.1393) -- (0.4710, 2.3160, 1.1348) -- (0.4220, 2.3160, 1.1291) -- cycle;
\fill[blue!15.0, opacity=0.7] (0.4220, 2.3160, 1.1291) -- (0.4710, 2.3160, 1.1348) -- (0.4710, 2.3670, 1.1300) -- (0.4220, 2.3670, 1.1243) -- cycle;
\fill[blue!15.4, opacity=0.7] (0.4220, 2.3670, 1.1243) -- (0.4710, 2.3670, 1.1300) -- (0.4710, 2.4180, 1.1250) -- (0.4220, 2.4180, 1.1193) -- cycle;
\fill[blue!16.8, opacity=0.7] (0.4220, 2.4180, 1.1193) -- (0.4710, 2.4180, 1.1250) -- (0.4710, 2.4690, 1.1198) -- (0.4220, 2.4690, 1.1142) -- cycle;
\fill[blue!18.0, opacity=0.7] (0.4220, 2.4690, 1.1142) -- (0.4710, 2.4690, 1.1198) -- (0.4710, 2.5200, 1.1145) -- (0.4220, 2.5200, 1.1088) -- cycle;
\fill[blue!16.5, opacity=0.7] (0.4220, 2.5200, 1.1088) -- (0.4710, 2.5200, 1.1145) -- (0.4710, 2.5710, 1.1090) -- (0.4220, 2.5710, 1.1033) -- cycle;
\fill[blue!15.2, opacity=0.7] (0.4220, 2.5710, 1.1033) -- (0.4710, 2.5710, 1.1090) -- (0.4710, 2.6220, 1.1033) -- (0.4220, 2.6220, 1.0976) -- cycle;
\fill[blue!15.0, opacity=0.7] (0.4220, 2.6220, 1.0976) -- (0.4710, 2.6220, 1.1033) -- (0.4710, 2.6730, 1.0975) -- (0.4220, 2.6730, 1.0918) -- cycle;
\fill[blue!15.0, opacity=0.7] (0.4220, 2.6730, 1.0918) -- (0.4710, 2.6730, 1.0975) -- (0.4710, 2.7240, 1.0916) -- (0.4220, 2.7240, 1.0859) -- cycle;
\fill[blue!15.0, opacity=0.7] (0.4220, 2.7240, 1.0859) -- (0.4710, 2.7240, 1.0916) -- (0.4710, 2.7750, 1.0855) -- (0.4220, 2.7750, 1.0799) -- cycle;
\fill[blue!15.0, opacity=0.7] (0.4220, 2.7750, 1.0799) -- (0.4710, 2.7750, 1.0855) -- (0.4710, 2.8260, 1.0794) -- (0.4220, 2.8260, 1.0738) -- cycle;
\fill[blue!15.0, opacity=0.7] (0.4220, 2.8260, 1.0738) -- (0.4710, 2.8260, 1.0794) -- (0.4710, 2.8770, 1.0733) -- (0.4220, 2.8770, 1.0676) -- cycle;
\fill[blue!15.0, opacity=0.7] (0.4220, 2.8770, 1.0676) -- (0.4710, 2.8770, 1.0733) -- (0.4710, 2.9280, 1.0670) -- (0.4220, 2.9280, 1.0614) -- cycle;
\fill[blue!15.0, opacity=0.7] (0.4220, 2.9280, 1.0614) -- (0.4710, 2.9280, 1.0670) -- (0.4710, 2.9790, 1.0608) -- (0.4220, 2.9790, 1.0551) -- cycle;
\fill[blue!15.0, opacity=0.7] (0.4220, 2.9790, 1.0551) -- (0.4710, 2.9790, 1.0608) -- (0.4710, 3.0300, 1.0545) -- (0.4220, 3.0300, 1.0488) -- cycle;
\fill[blue!15.0, opacity=0.7] (0.4710, -0.0300, 1.0545) -- (0.5200, -0.0300, 1.0600) -- (0.5200, 0.0210, 1.0663) -- (0.4710, 0.0210, 1.0608) -- cycle;
\fill[blue!15.0, opacity=0.7] (0.4710, 0.0210, 1.0608) -- (0.5200, 0.0210, 1.0663) -- (0.5200, 0.0720, 1.0725) -- (0.4710, 0.0720, 1.0670) -- cycle;
\fill[blue!15.0, opacity=0.7] (0.4710, 0.0720, 1.0670) -- (0.5200, 0.0720, 1.0725) -- (0.5200, 0.1230, 1.0788) -- (0.4710, 0.1230, 1.0733) -- cycle;
\fill[blue!15.1, opacity=0.7] (0.4710, 0.1230, 1.0733) -- (0.5200, 0.1230, 1.0788) -- (0.5200, 0.1740, 1.0849) -- (0.4710, 0.1740, 1.0794) -- cycle;
\fill[blue!15.2, opacity=0.7] (0.4710, 0.1740, 1.0794) -- (0.5200, 0.1740, 1.0849) -- (0.5200, 0.2250, 1.0911) -- (0.4710, 0.2250, 1.0855) -- cycle;
\fill[blue!15.0, opacity=0.7] (0.4710, 0.2250, 1.0855) -- (0.5200, 0.2250, 1.0911) -- (0.5200, 0.2760, 1.0971) -- (0.4710, 0.2760, 1.0916) -- cycle;
\fill[blue!15.0, opacity=0.7] (0.4710, 0.2760, 1.0916) -- (0.5200, 0.2760, 1.0971) -- (0.5200, 0.3270, 1.1030) -- (0.4710, 0.3270, 1.0975) -- cycle;
\fill[blue!15.0, opacity=0.7] (0.4710, 0.3270, 1.0975) -- (0.5200, 0.3270, 1.1030) -- (0.5200, 0.3780, 1.1088) -- (0.4710, 0.3780, 1.1033) -- cycle;
\fill[blue!15.0, opacity=0.7] (0.4710, 0.3780, 1.1033) -- (0.5200, 0.3780, 1.1088) -- (0.5200, 0.4290, 1.1145) -- (0.4710, 0.4290, 1.1090) -- cycle;
\fill[blue!15.0, opacity=0.7] (0.4710, 0.4290, 1.1090) -- (0.5200, 0.4290, 1.1145) -- (0.5200, 0.4800, 1.1200) -- (0.4710, 0.4800, 1.1145) -- cycle;
\fill[blue!15.0, opacity=0.7] (0.4710, 0.4800, 1.1145) -- (0.5200, 0.4800, 1.1200) -- (0.5200, 0.5310, 1.1254) -- (0.4710, 0.5310, 1.1198) -- cycle;
\fill[blue!15.0, opacity=0.7] (0.4710, 0.5310, 1.1198) -- (0.5200, 0.5310, 1.1254) -- (0.5200, 0.5820, 1.1305) -- (0.4710, 0.5820, 1.1250) -- cycle;
\fill[blue!15.2, opacity=0.7] (0.4710, 0.5820, 1.1250) -- (0.5200, 0.5820, 1.1305) -- (0.5200, 0.6330, 1.1355) -- (0.4710, 0.6330, 1.1300) -- cycle;
\fill[blue!19.2, opacity=0.7] (0.4710, 0.6330, 1.1300) -- (0.5200, 0.6330, 1.1355) -- (0.5200, 0.6840, 1.1403) -- (0.4710, 0.6840, 1.1348) -- cycle;
\fill[blue!29.8, opacity=0.7] (0.4710, 0.6840, 1.1348) -- (0.5200, 0.6840, 1.1403) -- (0.5200, 0.7350, 1.1449) -- (0.4710, 0.7350, 1.1393) -- cycle;
\fill[blue!30.2, opacity=0.7] (0.4710, 0.7350, 1.1393) -- (0.5200, 0.7350, 1.1449) -- (0.5200, 0.7860, 1.1492) -- (0.4710, 0.7860, 1.1437) -- cycle;
\fill[blue!20.6, opacity=0.7] (0.4710, 0.7860, 1.1437) -- (0.5200, 0.7860, 1.1492) -- (0.5200, 0.8370, 1.1533) -- (0.4710, 0.8370, 1.1477) -- cycle;
\fill[blue!15.7, opacity=0.7] (0.4710, 0.8370, 1.1477) -- (0.5200, 0.8370, 1.1533) -- (0.5200, 0.8880, 1.1571) -- (0.4710, 0.8880, 1.1516) -- cycle;
\fill[blue!15.0, opacity=0.7] (0.4710, 0.8880, 1.1516) -- (0.5200, 0.8880, 1.1571) -- (0.5200, 0.9390, 1.1606) -- (0.4710, 0.9390, 1.1551) -- cycle;
\fill[blue!15.0, opacity=0.7] (0.4710, 0.9390, 1.1551) -- (0.5200, 0.9390, 1.1606) -- (0.5200, 0.9900, 1.1639) -- (0.4710, 0.9900, 1.1584) -- cycle;
\fill[blue!15.0, opacity=0.7] (0.4710, 0.9900, 1.1584) -- (0.5200, 0.9900, 1.1639) -- (0.5200, 1.0410, 1.1669) -- (0.4710, 1.0410, 1.1614) -- cycle;
\fill[blue!15.0, opacity=0.7] (0.4710, 1.0410, 1.1614) -- (0.5200, 1.0410, 1.1669) -- (0.5200, 1.0920, 1.1696) -- (0.4710, 1.0920, 1.1641) -- cycle;
\fill[blue!15.0, opacity=0.7] (0.4710, 1.0920, 1.1641) -- (0.5200, 1.0920, 1.1696) -- (0.5200, 1.1430, 1.1720) -- (0.4710, 1.1430, 1.1665) -- cycle;
\fill[blue!15.0, opacity=0.7] (0.4710, 1.1430, 1.1665) -- (0.5200, 1.1430, 1.1720) -- (0.5200, 1.1940, 1.1741) -- (0.4710, 1.1940, 1.1686) -- cycle;
\fill[blue!15.0, opacity=0.7] (0.4710, 1.1940, 1.1686) -- (0.5200, 1.1940, 1.1741) -- (0.5200, 1.2450, 1.1759) -- (0.4710, 1.2450, 1.1704) -- cycle;
\fill[blue!15.0, opacity=0.7] (0.4710, 1.2450, 1.1704) -- (0.5200, 1.2450, 1.1759) -- (0.5200, 1.2960, 1.1774) -- (0.4710, 1.2960, 1.1719) -- cycle;
\fill[blue!15.2, opacity=0.7] (0.4710, 1.2960, 1.1719) -- (0.5200, 1.2960, 1.1774) -- (0.5200, 1.3470, 1.1785) -- (0.4710, 1.3470, 1.1730) -- cycle;
\fill[blue!16.0, opacity=0.7] (0.4710, 1.3470, 1.1730) -- (0.5200, 1.3470, 1.1785) -- (0.5200, 1.3980, 1.1793) -- (0.4710, 1.3980, 1.1738) -- cycle;
\fill[blue!18.0, opacity=0.7] (0.4710, 1.3980, 1.1738) -- (0.5200, 1.3980, 1.1793) -- (0.5200, 1.4490, 1.1798) -- (0.4710, 1.4490, 1.1743) -- cycle;
\fill[blue!21.1, opacity=0.7] (0.4710, 1.4490, 1.1743) -- (0.5200, 1.4490, 1.1798) -- (0.5200, 1.5000, 1.1800) -- (0.4710, 1.5000, 1.1745) -- cycle;
\fill[blue!24.5, opacity=0.7] (0.4710, 1.5000, 1.1745) -- (0.5200, 1.5000, 1.1800) -- (0.5200, 1.5510, 1.1798) -- (0.4710, 1.5510, 1.1743) -- cycle;
\fill[blue!27.0, opacity=0.7] (0.4710, 1.5510, 1.1743) -- (0.5200, 1.5510, 1.1798) -- (0.5200, 1.6020, 1.1793) -- (0.4710, 1.6020, 1.1738) -- cycle;
\fill[blue!27.6, opacity=0.7] (0.4710, 1.6020, 1.1738) -- (0.5200, 1.6020, 1.1793) -- (0.5200, 1.6530, 1.1785) -- (0.4710, 1.6530, 1.1730) -- cycle;
\fill[blue!26.2, opacity=0.7] (0.4710, 1.6530, 1.1730) -- (0.5200, 1.6530, 1.1785) -- (0.5200, 1.7040, 1.1774) -- (0.4710, 1.7040, 1.1719) -- cycle;
\fill[blue!23.4, opacity=0.7] (0.4710, 1.7040, 1.1719) -- (0.5200, 1.7040, 1.1774) -- (0.5200, 1.7550, 1.1759) -- (0.4710, 1.7550, 1.1704) -- cycle;
\fill[blue!20.0, opacity=0.7] (0.4710, 1.7550, 1.1704) -- (0.5200, 1.7550, 1.1759) -- (0.5200, 1.8060, 1.1741) -- (0.4710, 1.8060, 1.1686) -- cycle;
\fill[blue!17.3, opacity=0.7] (0.4710, 1.8060, 1.1686) -- (0.5200, 1.8060, 1.1741) -- (0.5200, 1.8570, 1.1720) -- (0.4710, 1.8570, 1.1665) -- cycle;
\fill[blue!15.7, opacity=0.7] (0.4710, 1.8570, 1.1665) -- (0.5200, 1.8570, 1.1720) -- (0.5200, 1.9080, 1.1696) -- (0.4710, 1.9080, 1.1641) -- cycle;
\fill[blue!15.1, opacity=0.7] (0.4710, 1.9080, 1.1641) -- (0.5200, 1.9080, 1.1696) -- (0.5200, 1.9590, 1.1669) -- (0.4710, 1.9590, 1.1614) -- cycle;
\fill[blue!15.0, opacity=0.7] (0.4710, 1.9590, 1.1614) -- (0.5200, 1.9590, 1.1669) -- (0.5200, 2.0100, 1.1639) -- (0.4710, 2.0100, 1.1584) -- cycle;
\fill[blue!15.0, opacity=0.7] (0.4710, 2.0100, 1.1584) -- (0.5200, 2.0100, 1.1639) -- (0.5200, 2.0610, 1.1606) -- (0.4710, 2.0610, 1.1551) -- cycle;
\fill[blue!15.0, opacity=0.7] (0.4710, 2.0610, 1.1551) -- (0.5200, 2.0610, 1.1606) -- (0.5200, 2.1120, 1.1571) -- (0.4710, 2.1120, 1.1516) -- cycle;
\fill[blue!15.0, opacity=0.7] (0.4710, 2.1120, 1.1516) -- (0.5200, 2.1120, 1.1571) -- (0.5200, 2.1630, 1.1533) -- (0.4710, 2.1630, 1.1477) -- cycle;
\fill[blue!15.0, opacity=0.7] (0.4710, 2.1630, 1.1477) -- (0.5200, 2.1630, 1.1533) -- (0.5200, 2.2140, 1.1492) -- (0.4710, 2.2140, 1.1437) -- cycle;
\fill[blue!15.0, opacity=0.7] (0.4710, 2.2140, 1.1437) -- (0.5200, 2.2140, 1.1492) -- (0.5200, 2.2650, 1.1449) -- (0.4710, 2.2650, 1.1393) -- cycle;
\fill[blue!15.0, opacity=0.7] (0.4710, 2.2650, 1.1393) -- (0.5200, 2.2650, 1.1449) -- (0.5200, 2.3160, 1.1403) -- (0.4710, 2.3160, 1.1348) -- cycle;
\fill[blue!15.0, opacity=0.7] (0.4710, 2.3160, 1.1348) -- (0.5200, 2.3160, 1.1403) -- (0.5200, 2.3670, 1.1355) -- (0.4710, 2.3670, 1.1300) -- cycle;
\fill[blue!15.0, opacity=0.7] (0.4710, 2.3670, 1.1300) -- (0.5200, 2.3670, 1.1355) -- (0.5200, 2.4180, 1.1305) -- (0.4710, 2.4180, 1.1250) -- cycle;
\fill[blue!15.1, opacity=0.7] (0.4710, 2.4180, 1.1250) -- (0.5200, 2.4180, 1.1305) -- (0.5200, 2.4690, 1.1254) -- (0.4710, 2.4690, 1.1198) -- cycle;
\fill[blue!16.1, opacity=0.7] (0.4710, 2.4690, 1.1198) -- (0.5200, 2.4690, 1.1254) -- (0.5200, 2.5200, 1.1200) -- (0.4710, 2.5200, 1.1145) -- cycle;
\fill[blue!17.7, opacity=0.7] (0.4710, 2.5200, 1.1145) -- (0.5200, 2.5200, 1.1200) -- (0.5200, 2.5710, 1.1145) -- (0.4710, 2.5710, 1.1090) -- cycle;
\fill[blue!16.8, opacity=0.7] (0.4710, 2.5710, 1.1090) -- (0.5200, 2.5710, 1.1145) -- (0.5200, 2.6220, 1.1088) -- (0.4710, 2.6220, 1.1033) -- cycle;
\fill[blue!15.2, opacity=0.7] (0.4710, 2.6220, 1.1033) -- (0.5200, 2.6220, 1.1088) -- (0.5200, 2.6730, 1.1030) -- (0.4710, 2.6730, 1.0975) -- cycle;
\fill[blue!15.0, opacity=0.7] (0.4710, 2.6730, 1.0975) -- (0.5200, 2.6730, 1.1030) -- (0.5200, 2.7240, 1.0971) -- (0.4710, 2.7240, 1.0916) -- cycle;
\fill[blue!15.0, opacity=0.7] (0.4710, 2.7240, 1.0916) -- (0.5200, 2.7240, 1.0971) -- (0.5200, 2.7750, 1.0911) -- (0.4710, 2.7750, 1.0855) -- cycle;
\fill[blue!15.0, opacity=0.7] (0.4710, 2.7750, 1.0855) -- (0.5200, 2.7750, 1.0911) -- (0.5200, 2.8260, 1.0849) -- (0.4710, 2.8260, 1.0794) -- cycle;
\fill[blue!15.0, opacity=0.7] (0.4710, 2.8260, 1.0794) -- (0.5200, 2.8260, 1.0849) -- (0.5200, 2.8770, 1.0788) -- (0.4710, 2.8770, 1.0733) -- cycle;
\fill[blue!15.0, opacity=0.7] (0.4710, 2.8770, 1.0733) -- (0.5200, 2.8770, 1.0788) -- (0.5200, 2.9280, 1.0725) -- (0.4710, 2.9280, 1.0670) -- cycle;
\fill[blue!15.0, opacity=0.7] (0.4710, 2.9280, 1.0670) -- (0.5200, 2.9280, 1.0725) -- (0.5200, 2.9790, 1.0663) -- (0.4710, 2.9790, 1.0608) -- cycle;
\fill[blue!15.0, opacity=0.7] (0.4710, 2.9790, 1.0608) -- (0.5200, 2.9790, 1.0663) -- (0.5200, 3.0300, 1.0600) -- (0.4710, 3.0300, 1.0545) -- cycle;
\fill[blue!15.0, opacity=0.7] (0.5200, -0.0300, 1.0600) -- (0.5690, -0.0300, 1.0654) -- (0.5690, 0.0210, 1.0716) -- (0.5200, 0.0210, 1.0663) -- cycle;
\fill[blue!15.0, opacity=0.7] (0.5200, 0.0210, 1.0663) -- (0.5690, 0.0210, 1.0716) -- (0.5690, 0.0720, 1.0779) -- (0.5200, 0.0720, 1.0725) -- cycle;
\fill[blue!15.1, opacity=0.7] (0.5200, 0.0720, 1.0725) -- (0.5690, 0.0720, 1.0779) -- (0.5690, 0.1230, 1.0841) -- (0.5200, 0.1230, 1.0788) -- cycle;
\fill[blue!15.2, opacity=0.7] (0.5200, 0.1230, 1.0788) -- (0.5690, 0.1230, 1.0841) -- (0.5690, 0.1740, 1.0903) -- (0.5200, 0.1740, 1.0849) -- cycle;
\fill[blue!15.0, opacity=0.7] (0.5200, 0.1740, 1.0849) -- (0.5690, 0.1740, 1.0903) -- (0.5690, 0.2250, 1.0964) -- (0.5200, 0.2250, 1.0911) -- cycle;
\fill[blue!15.0, opacity=0.7] (0.5200, 0.2250, 1.0911) -- (0.5690, 0.2250, 1.0964) -- (0.5690, 0.2760, 1.1024) -- (0.5200, 0.2760, 1.0971) -- cycle;
\fill[blue!15.0, opacity=0.7] (0.5200, 0.2760, 1.0971) -- (0.5690, 0.2760, 1.1024) -- (0.5690, 0.3270, 1.1084) -- (0.5200, 0.3270, 1.1030) -- cycle;
\fill[blue!15.0, opacity=0.7] (0.5200, 0.3270, 1.1030) -- (0.5690, 0.3270, 1.1084) -- (0.5690, 0.3780, 1.1142) -- (0.5200, 0.3780, 1.1088) -- cycle;
\fill[blue!15.0, opacity=0.7] (0.5200, 0.3780, 1.1088) -- (0.5690, 0.3780, 1.1142) -- (0.5690, 0.4290, 1.1198) -- (0.5200, 0.4290, 1.1145) -- cycle;
\fill[blue!15.0, opacity=0.7] (0.5200, 0.4290, 1.1145) -- (0.5690, 0.4290, 1.1198) -- (0.5690, 0.4800, 1.1254) -- (0.5200, 0.4800, 1.1200) -- cycle;
\fill[blue!15.0, opacity=0.7] (0.5200, 0.4800, 1.1200) -- (0.5690, 0.4800, 1.1254) -- (0.5690, 0.5310, 1.1307) -- (0.5200, 0.5310, 1.1254) -- cycle;
\fill[blue!15.1, opacity=0.7] (0.5200, 0.5310, 1.1254) -- (0.5690, 0.5310, 1.1307) -- (0.5690, 0.5820, 1.1359) -- (0.5200, 0.5820, 1.1305) -- cycle;
\fill[blue!19.1, opacity=0.7] (0.5200, 0.5820, 1.1305) -- (0.5690, 0.5820, 1.1359) -- (0.5690, 0.6330, 1.1409) -- (0.5200, 0.6330, 1.1355) -- cycle;
\fill[blue!30.8, opacity=0.7] (0.5200, 0.6330, 1.1355) -- (0.5690, 0.6330, 1.1409) -- (0.5690, 0.6840, 1.1457) -- (0.5200, 0.6840, 1.1403) -- cycle;
\fill[blue!30.6, opacity=0.7] (0.5200, 0.6840, 1.1403) -- (0.5690, 0.6840, 1.1457) -- (0.5690, 0.7350, 1.1502) -- (0.5200, 0.7350, 1.1449) -- cycle;
\fill[blue!19.8, opacity=0.7] (0.5200, 0.7350, 1.1449) -- (0.5690, 0.7350, 1.1502) -- (0.5690, 0.7860, 1.1545) -- (0.5200, 0.7860, 1.1492) -- cycle;
\fill[blue!15.4, opacity=0.7] (0.5200, 0.7860, 1.1492) -- (0.5690, 0.7860, 1.1545) -- (0.5690, 0.8370, 1.1586) -- (0.5200, 0.8370, 1.1533) -- cycle;
\fill[blue!15.0, opacity=0.7] (0.5200, 0.8370, 1.1533) -- (0.5690, 0.8370, 1.1586) -- (0.5690, 0.8880, 1.1624) -- (0.5200, 0.8880, 1.1571) -- cycle;
\fill[blue!15.0, opacity=0.7] (0.5200, 0.8880, 1.1571) -- (0.5690, 0.8880, 1.1624) -- (0.5690, 0.9390, 1.1660) -- (0.5200, 0.9390, 1.1606) -- cycle;
\fill[blue!15.0, opacity=0.7] (0.5200, 0.9390, 1.1606) -- (0.5690, 0.9390, 1.1660) -- (0.5690, 0.9900, 1.1693) -- (0.5200, 0.9900, 1.1639) -- cycle;
\fill[blue!15.0, opacity=0.7] (0.5200, 0.9900, 1.1639) -- (0.5690, 0.9900, 1.1693) -- (0.5690, 1.0410, 1.1723) -- (0.5200, 1.0410, 1.1669) -- cycle;
\fill[blue!15.0, opacity=0.7] (0.5200, 1.0410, 1.1669) -- (0.5690, 1.0410, 1.1723) -- (0.5690, 1.0920, 1.1750) -- (0.5200, 1.0920, 1.1696) -- cycle;
\fill[blue!15.0, opacity=0.7] (0.5200, 1.0920, 1.1696) -- (0.5690, 1.0920, 1.1750) -- (0.5690, 1.1430, 1.1774) -- (0.5200, 1.1430, 1.1720) -- cycle;
\fill[blue!15.1, opacity=0.7] (0.5200, 1.1430, 1.1720) -- (0.5690, 1.1430, 1.1774) -- (0.5690, 1.1940, 1.1795) -- (0.5200, 1.1940, 1.1741) -- cycle;
\fill[blue!16.2, opacity=0.7] (0.5200, 1.1940, 1.1741) -- (0.5690, 1.1940, 1.1795) -- (0.5690, 1.2450, 1.1813) -- (0.5200, 1.2450, 1.1759) -- cycle;
\fill[blue!21.6, opacity=0.7] (0.5200, 1.2450, 1.1759) -- (0.5690, 1.2450, 1.1813) -- (0.5690, 1.2960, 1.1827) -- (0.5200, 1.2960, 1.1774) -- cycle;
\fill[blue!34.8, opacity=0.7] (0.5200, 1.2960, 1.1774) -- (0.5690, 1.2960, 1.1827) -- (0.5690, 1.3470, 1.1839) -- (0.5200, 1.3470, 1.1785) -- cycle;
\fill[blue!52.7, opacity=0.7] (0.5200, 1.3470, 1.1785) -- (0.5690, 1.3470, 1.1839) -- (0.5690, 1.3980, 1.1847) -- (0.5200, 1.3980, 1.1793) -- cycle;
\fill[blue!68.6, opacity=0.7] (0.5200, 1.3980, 1.1793) -- (0.5690, 1.3980, 1.1847) -- (0.5690, 1.4490, 1.1852) -- (0.5200, 1.4490, 1.1798) -- cycle;
\fill[blue!79.3, opacity=0.7] (0.5200, 1.4490, 1.1798) -- (0.5690, 1.4490, 1.1852) -- (0.5690, 1.5000, 1.1854) -- (0.5200, 1.5000, 1.1800) -- cycle;
\fill[blue!85.2, opacity=0.7] (0.5200, 1.5000, 1.1800) -- (0.5690, 1.5000, 1.1854) -- (0.5690, 1.5510, 1.1852) -- (0.5200, 1.5510, 1.1798) -- cycle;
\fill[blue!87.6, opacity=0.7] (0.5200, 1.5510, 1.1798) -- (0.5690, 1.5510, 1.1852) -- (0.5690, 1.6020, 1.1847) -- (0.5200, 1.6020, 1.1793) -- cycle;
\fill[blue!87.5, opacity=0.7] (0.5200, 1.6020, 1.1793) -- (0.5690, 1.6020, 1.1847) -- (0.5690, 1.6530, 1.1839) -- (0.5200, 1.6530, 1.1785) -- cycle;
\fill[blue!85.1, opacity=0.7] (0.5200, 1.6530, 1.1785) -- (0.5690, 1.6530, 1.1839) -- (0.5690, 1.7040, 1.1827) -- (0.5200, 1.7040, 1.1774) -- cycle;
\fill[blue!80.2, opacity=0.7] (0.5200, 1.7040, 1.1774) -- (0.5690, 1.7040, 1.1827) -- (0.5690, 1.7550, 1.1813) -- (0.5200, 1.7550, 1.1759) -- cycle;
\fill[blue!72.1, opacity=0.7] (0.5200, 1.7550, 1.1759) -- (0.5690, 1.7550, 1.1813) -- (0.5690, 1.8060, 1.1795) -- (0.5200, 1.8060, 1.1741) -- cycle;
\fill[blue!60.1, opacity=0.7] (0.5200, 1.8060, 1.1741) -- (0.5690, 1.8060, 1.1795) -- (0.5690, 1.8570, 1.1774) -- (0.5200, 1.8570, 1.1720) -- cycle;
\fill[blue!44.8, opacity=0.7] (0.5200, 1.8570, 1.1720) -- (0.5690, 1.8570, 1.1774) -- (0.5690, 1.9080, 1.1750) -- (0.5200, 1.9080, 1.1696) -- cycle;
\fill[blue!29.7, opacity=0.7] (0.5200, 1.9080, 1.1696) -- (0.5690, 1.9080, 1.1750) -- (0.5690, 1.9590, 1.1723) -- (0.5200, 1.9590, 1.1669) -- cycle;
\fill[blue!19.6, opacity=0.7] (0.5200, 1.9590, 1.1669) -- (0.5690, 1.9590, 1.1723) -- (0.5690, 2.0100, 1.1693) -- (0.5200, 2.0100, 1.1639) -- cycle;
\fill[blue!15.8, opacity=0.7] (0.5200, 2.0100, 1.1639) -- (0.5690, 2.0100, 1.1693) -- (0.5690, 2.0610, 1.1660) -- (0.5200, 2.0610, 1.1606) -- cycle;
\fill[blue!15.1, opacity=0.7] (0.5200, 2.0610, 1.1606) -- (0.5690, 2.0610, 1.1660) -- (0.5690, 2.1120, 1.1624) -- (0.5200, 2.1120, 1.1571) -- cycle;
\fill[blue!15.0, opacity=0.7] (0.5200, 2.1120, 1.1571) -- (0.5690, 2.1120, 1.1624) -- (0.5690, 2.1630, 1.1586) -- (0.5200, 2.1630, 1.1533) -- cycle;
\fill[blue!15.0, opacity=0.7] (0.5200, 2.1630, 1.1533) -- (0.5690, 2.1630, 1.1586) -- (0.5690, 2.2140, 1.1545) -- (0.5200, 2.2140, 1.1492) -- cycle;
\fill[blue!15.0, opacity=0.7] (0.5200, 2.2140, 1.1492) -- (0.5690, 2.2140, 1.1545) -- (0.5690, 2.2650, 1.1502) -- (0.5200, 2.2650, 1.1449) -- cycle;
\fill[blue!15.0, opacity=0.7] (0.5200, 2.2650, 1.1449) -- (0.5690, 2.2650, 1.1502) -- (0.5690, 2.3160, 1.1457) -- (0.5200, 2.3160, 1.1403) -- cycle;
\fill[blue!15.0, opacity=0.7] (0.5200, 2.3160, 1.1403) -- (0.5690, 2.3160, 1.1457) -- (0.5690, 2.3670, 1.1409) -- (0.5200, 2.3670, 1.1355) -- cycle;
\fill[blue!15.0, opacity=0.7] (0.5200, 2.3670, 1.1355) -- (0.5690, 2.3670, 1.1409) -- (0.5690, 2.4180, 1.1359) -- (0.5200, 2.4180, 1.1305) -- cycle;
\fill[blue!15.0, opacity=0.7] (0.5200, 2.4180, 1.1305) -- (0.5690, 2.4180, 1.1359) -- (0.5690, 2.4690, 1.1307) -- (0.5200, 2.4690, 1.1254) -- cycle;
\fill[blue!15.1, opacity=0.7] (0.5200, 2.4690, 1.1254) -- (0.5690, 2.4690, 1.1307) -- (0.5690, 2.5200, 1.1254) -- (0.5200, 2.5200, 1.1200) -- cycle;
\fill[blue!15.7, opacity=0.7] (0.5200, 2.5200, 1.1200) -- (0.5690, 2.5200, 1.1254) -- (0.5690, 2.5710, 1.1198) -- (0.5200, 2.5710, 1.1145) -- cycle;
\fill[blue!17.3, opacity=0.7] (0.5200, 2.5710, 1.1145) -- (0.5690, 2.5710, 1.1198) -- (0.5690, 2.6220, 1.1142) -- (0.5200, 2.6220, 1.1088) -- cycle;
\fill[blue!16.8, opacity=0.7] (0.5200, 2.6220, 1.1088) -- (0.5690, 2.6220, 1.1142) -- (0.5690, 2.6730, 1.1084) -- (0.5200, 2.6730, 1.1030) -- cycle;
\fill[blue!15.2, opacity=0.7] (0.5200, 2.6730, 1.1030) -- (0.5690, 2.6730, 1.1084) -- (0.5690, 2.7240, 1.1024) -- (0.5200, 2.7240, 1.0971) -- cycle;
\fill[blue!15.0, opacity=0.7] (0.5200, 2.7240, 1.0971) -- (0.5690, 2.7240, 1.1024) -- (0.5690, 2.7750, 1.0964) -- (0.5200, 2.7750, 1.0911) -- cycle;
\fill[blue!15.0, opacity=0.7] (0.5200, 2.7750, 1.0911) -- (0.5690, 2.7750, 1.0964) -- (0.5690, 2.8260, 1.0903) -- (0.5200, 2.8260, 1.0849) -- cycle;
\fill[blue!15.0, opacity=0.7] (0.5200, 2.8260, 1.0849) -- (0.5690, 2.8260, 1.0903) -- (0.5690, 2.8770, 1.0841) -- (0.5200, 2.8770, 1.0788) -- cycle;
\fill[blue!15.0, opacity=0.7] (0.5200, 2.8770, 1.0788) -- (0.5690, 2.8770, 1.0841) -- (0.5690, 2.9280, 1.0779) -- (0.5200, 2.9280, 1.0725) -- cycle;
\fill[blue!15.0, opacity=0.7] (0.5200, 2.9280, 1.0725) -- (0.5690, 2.9280, 1.0779) -- (0.5690, 2.9790, 1.0716) -- (0.5200, 2.9790, 1.0663) -- cycle;
\fill[blue!15.0, opacity=0.7] (0.5200, 2.9790, 1.0663) -- (0.5690, 2.9790, 1.0716) -- (0.5690, 3.0300, 1.0654) -- (0.5200, 3.0300, 1.0600) -- cycle;
\fill[blue!15.0, opacity=0.7] (0.5690, -0.0300, 1.0654) -- (0.6180, -0.0300, 1.0705) -- (0.6180, 0.0210, 1.0768) -- (0.5690, 0.0210, 1.0716) -- cycle;
\fill[blue!15.0, opacity=0.7] (0.5690, 0.0210, 1.0716) -- (0.6180, 0.0210, 1.0768) -- (0.6180, 0.0720, 1.0831) -- (0.5690, 0.0720, 1.0779) -- cycle;
\fill[blue!15.2, opacity=0.7] (0.5690, 0.0720, 1.0779) -- (0.6180, 0.0720, 1.0831) -- (0.6180, 0.1230, 1.0893) -- (0.5690, 0.1230, 1.0841) -- cycle;
\fill[blue!15.1, opacity=0.7] (0.5690, 0.1230, 1.0841) -- (0.6180, 0.1230, 1.0893) -- (0.6180, 0.1740, 1.0955) -- (0.5690, 0.1740, 1.0903) -- cycle;
\fill[blue!15.0, opacity=0.7] (0.5690, 0.1740, 1.0903) -- (0.6180, 0.1740, 1.0955) -- (0.6180, 0.2250, 1.1016) -- (0.5690, 0.2250, 1.0964) -- cycle;
\fill[blue!15.0, opacity=0.7] (0.5690, 0.2250, 1.0964) -- (0.6180, 0.2250, 1.1016) -- (0.6180, 0.2760, 1.1076) -- (0.5690, 0.2760, 1.1024) -- cycle;
\fill[blue!15.0, opacity=0.7] (0.5690, 0.2760, 1.1024) -- (0.6180, 0.2760, 1.1076) -- (0.6180, 0.3270, 1.1135) -- (0.5690, 0.3270, 1.1084) -- cycle;
\fill[blue!15.0, opacity=0.7] (0.5690, 0.3270, 1.1084) -- (0.6180, 0.3270, 1.1135) -- (0.6180, 0.3780, 1.1193) -- (0.5690, 0.3780, 1.1142) -- cycle;
\fill[blue!15.0, opacity=0.7] (0.5690, 0.3780, 1.1142) -- (0.6180, 0.3780, 1.1193) -- (0.6180, 0.4290, 1.1250) -- (0.5690, 0.4290, 1.1198) -- cycle;
\fill[blue!15.0, opacity=0.7] (0.5690, 0.4290, 1.1198) -- (0.6180, 0.4290, 1.1250) -- (0.6180, 0.4800, 1.1305) -- (0.5690, 0.4800, 1.1254) -- cycle;
\fill[blue!15.1, opacity=0.7] (0.5690, 0.4800, 1.1254) -- (0.6180, 0.4800, 1.1305) -- (0.6180, 0.5310, 1.1359) -- (0.5690, 0.5310, 1.1307) -- cycle;
\fill[blue!18.2, opacity=0.7] (0.5690, 0.5310, 1.1307) -- (0.6180, 0.5310, 1.1359) -- (0.6180, 0.5820, 1.1411) -- (0.5690, 0.5820, 1.1359) -- cycle;
\fill[blue!30.8, opacity=0.7] (0.5690, 0.5820, 1.1359) -- (0.6180, 0.5820, 1.1411) -- (0.6180, 0.6330, 1.1461) -- (0.5690, 0.6330, 1.1409) -- cycle;
\fill[blue!32.0, opacity=0.7] (0.5690, 0.6330, 1.1409) -- (0.6180, 0.6330, 1.1461) -- (0.6180, 0.6840, 1.1508) -- (0.5690, 0.6840, 1.1457) -- cycle;
\fill[blue!20.1, opacity=0.7] (0.5690, 0.6840, 1.1457) -- (0.6180, 0.6840, 1.1508) -- (0.6180, 0.7350, 1.1554) -- (0.5690, 0.7350, 1.1502) -- cycle;
\fill[blue!15.4, opacity=0.7] (0.5690, 0.7350, 1.1502) -- (0.6180, 0.7350, 1.1554) -- (0.6180, 0.7860, 1.1597) -- (0.5690, 0.7860, 1.1545) -- cycle;
\fill[blue!15.0, opacity=0.7] (0.5690, 0.7860, 1.1545) -- (0.6180, 0.7860, 1.1597) -- (0.6180, 0.8370, 1.1638) -- (0.5690, 0.8370, 1.1586) -- cycle;
\fill[blue!15.0, opacity=0.7] (0.5690, 0.8370, 1.1586) -- (0.6180, 0.8370, 1.1638) -- (0.6180, 0.8880, 1.1676) -- (0.5690, 0.8880, 1.1624) -- cycle;
\fill[blue!15.0, opacity=0.7] (0.5690, 0.8880, 1.1624) -- (0.6180, 0.8880, 1.1676) -- (0.6180, 0.9390, 1.1712) -- (0.5690, 0.9390, 1.1660) -- cycle;
\fill[blue!15.0, opacity=0.7] (0.5690, 0.9390, 1.1660) -- (0.6180, 0.9390, 1.1712) -- (0.6180, 0.9900, 1.1745) -- (0.5690, 0.9900, 1.1693) -- cycle;
\fill[blue!15.0, opacity=0.7] (0.5690, 0.9900, 1.1693) -- (0.6180, 0.9900, 1.1745) -- (0.6180, 1.0410, 1.1775) -- (0.5690, 1.0410, 1.1723) -- cycle;
\fill[blue!15.1, opacity=0.7] (0.5690, 1.0410, 1.1723) -- (0.6180, 1.0410, 1.1775) -- (0.6180, 1.0920, 1.1802) -- (0.5690, 1.0920, 1.1750) -- cycle;
\fill[blue!16.5, opacity=0.7] (0.5690, 1.0920, 1.1750) -- (0.6180, 1.0920, 1.1802) -- (0.6180, 1.1430, 1.1826) -- (0.5690, 1.1430, 1.1774) -- cycle;
\fill[blue!26.5, opacity=0.7] (0.5690, 1.1430, 1.1774) -- (0.6180, 1.1430, 1.1826) -- (0.6180, 1.1940, 1.1847) -- (0.5690, 1.1940, 1.1795) -- cycle;
\fill[blue!52.1, opacity=0.7] (0.5690, 1.1940, 1.1795) -- (0.6180, 1.1940, 1.1847) -- (0.6180, 1.2450, 1.1864) -- (0.5690, 1.2450, 1.1813) -- cycle;
\fill[blue!79.9, opacity=0.7] (0.5690, 1.2450, 1.1813) -- (0.6180, 1.2450, 1.1864) -- (0.6180, 1.2960, 1.1879) -- (0.5690, 1.2960, 1.1827) -- cycle;
\fill[blue!96.5, opacity=0.7] (0.5690, 1.2960, 1.1827) -- (0.6180, 1.2960, 1.1879) -- (0.6180, 1.3470, 1.1891) -- (0.5690, 1.3470, 1.1839) -- cycle;
\fill[blue!92.0!black, opacity=0.7] (0.5690, 1.3470, 1.1839) -- (0.6180, 1.3470, 1.1891) -- (0.6180, 1.3980, 1.1899) -- (0.5690, 1.3980, 1.1847) -- cycle;
\fill[blue!88.3!black, opacity=0.7] (0.5690, 1.3980, 1.1847) -- (0.6180, 1.3980, 1.1899) -- (0.6180, 1.4490, 1.1904) -- (0.5690, 1.4490, 1.1852) -- cycle;
\fill[blue!91.1!black, opacity=0.7] (0.5690, 1.4490, 1.1852) -- (0.6180, 1.4490, 1.1904) -- (0.6180, 1.5000, 1.1905) -- (0.5690, 1.5000, 1.1854) -- cycle;
\fill[blue!97.2!black, opacity=0.7] (0.5690, 1.5000, 1.1854) -- (0.6180, 1.5000, 1.1905) -- (0.6180, 1.5510, 1.1904) -- (0.5690, 1.5510, 1.1852) -- cycle;
\fill[blue!98.4, opacity=0.7] (0.5690, 1.5510, 1.1852) -- (0.6180, 1.5510, 1.1904) -- (0.6180, 1.6020, 1.1899) -- (0.5690, 1.6020, 1.1847) -- cycle;
\fill[blue!96.3, opacity=0.7] (0.5690, 1.6020, 1.1847) -- (0.6180, 1.6020, 1.1899) -- (0.6180, 1.6530, 1.1891) -- (0.5690, 1.6530, 1.1839) -- cycle;
\fill[blue!95.3, opacity=0.7] (0.5690, 1.6530, 1.1839) -- (0.6180, 1.6530, 1.1891) -- (0.6180, 1.7040, 1.1879) -- (0.5690, 1.7040, 1.1827) -- cycle;
\fill[blue!95.1, opacity=0.7] (0.5690, 1.7040, 1.1827) -- (0.6180, 1.7040, 1.1879) -- (0.6180, 1.7550, 1.1864) -- (0.5690, 1.7550, 1.1813) -- cycle;
\fill[blue!95.2, opacity=0.7] (0.5690, 1.7550, 1.1813) -- (0.6180, 1.7550, 1.1864) -- (0.6180, 1.8060, 1.1847) -- (0.5690, 1.8060, 1.1795) -- cycle;
\fill[blue!94.4, opacity=0.7] (0.5690, 1.8060, 1.1795) -- (0.6180, 1.8060, 1.1847) -- (0.6180, 1.8570, 1.1826) -- (0.5690, 1.8570, 1.1774) -- cycle;
\fill[blue!90.8, opacity=0.7] (0.5690, 1.8570, 1.1774) -- (0.6180, 1.8570, 1.1826) -- (0.6180, 1.9080, 1.1802) -- (0.5690, 1.9080, 1.1750) -- cycle;
\fill[blue!82.1, opacity=0.7] (0.5690, 1.9080, 1.1750) -- (0.6180, 1.9080, 1.1802) -- (0.6180, 1.9590, 1.1775) -- (0.5690, 1.9590, 1.1723) -- cycle;
\fill[blue!65.3, opacity=0.7] (0.5690, 1.9590, 1.1723) -- (0.6180, 1.9590, 1.1775) -- (0.6180, 2.0100, 1.1745) -- (0.5690, 2.0100, 1.1693) -- cycle;
\fill[blue!42.1, opacity=0.7] (0.5690, 2.0100, 1.1693) -- (0.6180, 2.0100, 1.1745) -- (0.6180, 2.0610, 1.1712) -- (0.5690, 2.0610, 1.1660) -- cycle;
\fill[blue!23.2, opacity=0.7] (0.5690, 2.0610, 1.1660) -- (0.6180, 2.0610, 1.1712) -- (0.6180, 2.1120, 1.1676) -- (0.5690, 2.1120, 1.1624) -- cycle;
\fill[blue!16.1, opacity=0.7] (0.5690, 2.1120, 1.1624) -- (0.6180, 2.1120, 1.1676) -- (0.6180, 2.1630, 1.1638) -- (0.5690, 2.1630, 1.1586) -- cycle;
\fill[blue!15.1, opacity=0.7] (0.5690, 2.1630, 1.1586) -- (0.6180, 2.1630, 1.1638) -- (0.6180, 2.2140, 1.1597) -- (0.5690, 2.2140, 1.1545) -- cycle;
\fill[blue!15.0, opacity=0.7] (0.5690, 2.2140, 1.1545) -- (0.6180, 2.2140, 1.1597) -- (0.6180, 2.2650, 1.1554) -- (0.5690, 2.2650, 1.1502) -- cycle;
\fill[blue!15.0, opacity=0.7] (0.5690, 2.2650, 1.1502) -- (0.6180, 2.2650, 1.1554) -- (0.6180, 2.3160, 1.1508) -- (0.5690, 2.3160, 1.1457) -- cycle;
\fill[blue!15.0, opacity=0.7] (0.5690, 2.3160, 1.1457) -- (0.6180, 2.3160, 1.1508) -- (0.6180, 2.3670, 1.1461) -- (0.5690, 2.3670, 1.1409) -- cycle;
\fill[blue!15.0, opacity=0.7] (0.5690, 2.3670, 1.1409) -- (0.6180, 2.3670, 1.1461) -- (0.6180, 2.4180, 1.1411) -- (0.5690, 2.4180, 1.1359) -- cycle;
\fill[blue!15.0, opacity=0.7] (0.5690, 2.4180, 1.1359) -- (0.6180, 2.4180, 1.1411) -- (0.6180, 2.4690, 1.1359) -- (0.5690, 2.4690, 1.1307) -- cycle;
\fill[blue!15.0, opacity=0.7] (0.5690, 2.4690, 1.1307) -- (0.6180, 2.4690, 1.1359) -- (0.6180, 2.5200, 1.1305) -- (0.5690, 2.5200, 1.1254) -- cycle;
\fill[blue!15.0, opacity=0.7] (0.5690, 2.5200, 1.1254) -- (0.6180, 2.5200, 1.1305) -- (0.6180, 2.5710, 1.1250) -- (0.5690, 2.5710, 1.1198) -- cycle;
\fill[blue!15.5, opacity=0.7] (0.5690, 2.5710, 1.1198) -- (0.6180, 2.5710, 1.1250) -- (0.6180, 2.6220, 1.1193) -- (0.5690, 2.6220, 1.1142) -- cycle;
\fill[blue!17.1, opacity=0.7] (0.5690, 2.6220, 1.1142) -- (0.6180, 2.6220, 1.1193) -- (0.6180, 2.6730, 1.1135) -- (0.5690, 2.6730, 1.1084) -- cycle;
\fill[blue!16.7, opacity=0.7] (0.5690, 2.6730, 1.1084) -- (0.6180, 2.6730, 1.1135) -- (0.6180, 2.7240, 1.1076) -- (0.5690, 2.7240, 1.1024) -- cycle;
\fill[blue!15.2, opacity=0.7] (0.5690, 2.7240, 1.1024) -- (0.6180, 2.7240, 1.1076) -- (0.6180, 2.7750, 1.1016) -- (0.5690, 2.7750, 1.0964) -- cycle;
\fill[blue!15.0, opacity=0.7] (0.5690, 2.7750, 1.0964) -- (0.6180, 2.7750, 1.1016) -- (0.6180, 2.8260, 1.0955) -- (0.5690, 2.8260, 1.0903) -- cycle;
\fill[blue!15.0, opacity=0.7] (0.5690, 2.8260, 1.0903) -- (0.6180, 2.8260, 1.0955) -- (0.6180, 2.8770, 1.0893) -- (0.5690, 2.8770, 1.0841) -- cycle;
\fill[blue!15.0, opacity=0.7] (0.5690, 2.8770, 1.0841) -- (0.6180, 2.8770, 1.0893) -- (0.6180, 2.9280, 1.0831) -- (0.5690, 2.9280, 1.0779) -- cycle;
\fill[blue!15.0, opacity=0.7] (0.5690, 2.9280, 1.0779) -- (0.6180, 2.9280, 1.0831) -- (0.6180, 2.9790, 1.0768) -- (0.5690, 2.9790, 1.0716) -- cycle;
\fill[blue!15.0, opacity=0.7] (0.5690, 2.9790, 1.0716) -- (0.6180, 2.9790, 1.0768) -- (0.6180, 3.0300, 1.0705) -- (0.5690, 3.0300, 1.0654) -- cycle;
\fill[blue!15.0, opacity=0.7] (0.6180, -0.0300, 1.0705) -- (0.6670, -0.0300, 1.0755) -- (0.6670, 0.0210, 1.0818) -- (0.6180, 0.0210, 1.0768) -- cycle;
\fill[blue!15.1, opacity=0.7] (0.6180, 0.0210, 1.0768) -- (0.6670, 0.0210, 1.0818) -- (0.6670, 0.0720, 1.0881) -- (0.6180, 0.0720, 1.0831) -- cycle;
\fill[blue!15.3, opacity=0.7] (0.6180, 0.0720, 1.0831) -- (0.6670, 0.0720, 1.0881) -- (0.6670, 0.1230, 1.0943) -- (0.6180, 0.1230, 1.0893) -- cycle;
\fill[blue!15.0, opacity=0.7] (0.6180, 0.1230, 1.0893) -- (0.6670, 0.1230, 1.0943) -- (0.6670, 0.1740, 1.1005) -- (0.6180, 0.1740, 1.0955) -- cycle;
\fill[blue!15.0, opacity=0.7] (0.6180, 0.1740, 1.0955) -- (0.6670, 0.1740, 1.1005) -- (0.6670, 0.2250, 1.1066) -- (0.6180, 0.2250, 1.1016) -- cycle;
\fill[blue!15.0, opacity=0.7] (0.6180, 0.2250, 1.1016) -- (0.6670, 0.2250, 1.1066) -- (0.6670, 0.2760, 1.1126) -- (0.6180, 0.2760, 1.1076) -- cycle;
\fill[blue!15.0, opacity=0.7] (0.6180, 0.2760, 1.1076) -- (0.6670, 0.2760, 1.1126) -- (0.6670, 0.3270, 1.1185) -- (0.6180, 0.3270, 1.1135) -- cycle;
\fill[blue!15.0, opacity=0.7] (0.6180, 0.3270, 1.1135) -- (0.6670, 0.3270, 1.1185) -- (0.6670, 0.3780, 1.1243) -- (0.6180, 0.3780, 1.1193) -- cycle;
\fill[blue!15.0, opacity=0.7] (0.6180, 0.3780, 1.1193) -- (0.6670, 0.3780, 1.1243) -- (0.6670, 0.4290, 1.1300) -- (0.6180, 0.4290, 1.1250) -- cycle;
\fill[blue!15.0, opacity=0.7] (0.6180, 0.4290, 1.1250) -- (0.6670, 0.4290, 1.1300) -- (0.6670, 0.4800, 1.1355) -- (0.6180, 0.4800, 1.1305) -- cycle;
\fill[blue!16.7, opacity=0.7] (0.6180, 0.4800, 1.1305) -- (0.6670, 0.4800, 1.1355) -- (0.6670, 0.5310, 1.1409) -- (0.6180, 0.5310, 1.1359) -- cycle;
\fill[blue!29.1, opacity=0.7] (0.6180, 0.5310, 1.1359) -- (0.6670, 0.5310, 1.1409) -- (0.6670, 0.5820, 1.1461) -- (0.6180, 0.5820, 1.1411) -- cycle;
\fill[blue!34.2, opacity=0.7] (0.6180, 0.5820, 1.1411) -- (0.6670, 0.5820, 1.1461) -- (0.6670, 0.6330, 1.1510) -- (0.6180, 0.6330, 1.1461) -- cycle;
\fill[blue!21.6, opacity=0.7] (0.6180, 0.6330, 1.1461) -- (0.6670, 0.6330, 1.1510) -- (0.6670, 0.6840, 1.1558) -- (0.6180, 0.6840, 1.1508) -- cycle;
\fill[blue!15.5, opacity=0.7] (0.6180, 0.6840, 1.1508) -- (0.6670, 0.6840, 1.1558) -- (0.6670, 0.7350, 1.1604) -- (0.6180, 0.7350, 1.1554) -- cycle;
\fill[blue!15.0, opacity=0.7] (0.6180, 0.7350, 1.1554) -- (0.6670, 0.7350, 1.1604) -- (0.6670, 0.7860, 1.1647) -- (0.6180, 0.7860, 1.1597) -- cycle;
\fill[blue!15.0, opacity=0.7] (0.6180, 0.7860, 1.1597) -- (0.6670, 0.7860, 1.1647) -- (0.6670, 0.8370, 1.1688) -- (0.6180, 0.8370, 1.1638) -- cycle;
\fill[blue!15.0, opacity=0.7] (0.6180, 0.8370, 1.1638) -- (0.6670, 0.8370, 1.1688) -- (0.6670, 0.8880, 1.1726) -- (0.6180, 0.8880, 1.1676) -- cycle;
\fill[blue!15.0, opacity=0.7] (0.6180, 0.8880, 1.1676) -- (0.6670, 0.8880, 1.1726) -- (0.6670, 0.9390, 1.1762) -- (0.6180, 0.9390, 1.1712) -- cycle;
\fill[blue!15.0, opacity=0.7] (0.6180, 0.9390, 1.1712) -- (0.6670, 0.9390, 1.1762) -- (0.6670, 0.9900, 1.1794) -- (0.6180, 0.9900, 1.1745) -- cycle;
\fill[blue!15.5, opacity=0.7] (0.6180, 0.9900, 1.1745) -- (0.6670, 0.9900, 1.1794) -- (0.6670, 1.0410, 1.1824) -- (0.6180, 1.0410, 1.1775) -- cycle;
\fill[blue!22.5, opacity=0.7] (0.6180, 1.0410, 1.1775) -- (0.6670, 1.0410, 1.1824) -- (0.6670, 1.0920, 1.1851) -- (0.6180, 1.0920, 1.1802) -- cycle;
\fill[blue!51.7, opacity=0.7] (0.6180, 1.0920, 1.1802) -- (0.6670, 1.0920, 1.1851) -- (0.6670, 1.1430, 1.1875) -- (0.6180, 1.1430, 1.1826) -- cycle;
\fill[blue!87.4, opacity=0.7] (0.6180, 1.1430, 1.1826) -- (0.6670, 1.1430, 1.1875) -- (0.6670, 1.1940, 1.1896) -- (0.6180, 1.1940, 1.1847) -- cycle;
\fill[blue!88.9!black, opacity=0.7] (0.6180, 1.1940, 1.1847) -- (0.6670, 1.1940, 1.1896) -- (0.6670, 1.2450, 1.1914) -- (0.6180, 1.2450, 1.1864) -- cycle;
\fill[blue!79.8!black, opacity=0.7] (0.6180, 1.2450, 1.1864) -- (0.6670, 1.2450, 1.1914) -- (0.6670, 1.2960, 1.1929) -- (0.6180, 1.2960, 1.1879) -- cycle;
\fill[blue!86.4!black, opacity=0.7] (0.6180, 1.2960, 1.1879) -- (0.6670, 1.2960, 1.1929) -- (0.6670, 1.3470, 1.1940) -- (0.6180, 1.3470, 1.1891) -- cycle;
\fill[blue!97.1, opacity=0.7] (0.6180, 1.3470, 1.1891) -- (0.6670, 1.3470, 1.1940) -- (0.6670, 1.3980, 1.1949) -- (0.6180, 1.3980, 1.1899) -- cycle;
\fill[blue!84.2, opacity=0.7] (0.6180, 1.3980, 1.1899) -- (0.6670, 1.3980, 1.1949) -- (0.6670, 1.4490, 1.1954) -- (0.6180, 1.4490, 1.1904) -- cycle;
\fill[blue!70.1, opacity=0.7] (0.6180, 1.4490, 1.1904) -- (0.6670, 1.4490, 1.1954) -- (0.6670, 1.5000, 1.1955) -- (0.6180, 1.5000, 1.1905) -- cycle;
\fill[blue!58.4, opacity=0.7] (0.6180, 1.5000, 1.1905) -- (0.6670, 1.5000, 1.1955) -- (0.6670, 1.5510, 1.1954) -- (0.6180, 1.5510, 1.1904) -- cycle;
\fill[blue!50.6, opacity=0.7] (0.6180, 1.5510, 1.1904) -- (0.6670, 1.5510, 1.1954) -- (0.6670, 1.6020, 1.1949) -- (0.6180, 1.6020, 1.1899) -- cycle;
\fill[blue!46.8, opacity=0.7] (0.6180, 1.6020, 1.1899) -- (0.6670, 1.6020, 1.1949) -- (0.6670, 1.6530, 1.1940) -- (0.6180, 1.6530, 1.1891) -- cycle;
\fill[blue!46.6, opacity=0.7] (0.6180, 1.6530, 1.1891) -- (0.6670, 1.6530, 1.1940) -- (0.6670, 1.7040, 1.1929) -- (0.6180, 1.7040, 1.1879) -- cycle;
\fill[blue!49.7, opacity=0.7] (0.6180, 1.7040, 1.1879) -- (0.6670, 1.7040, 1.1929) -- (0.6670, 1.7550, 1.1914) -- (0.6180, 1.7550, 1.1864) -- cycle;
\fill[blue!56.2, opacity=0.7] (0.6180, 1.7550, 1.1864) -- (0.6670, 1.7550, 1.1914) -- (0.6670, 1.8060, 1.1896) -- (0.6180, 1.8060, 1.1847) -- cycle;
\fill[blue!65.6, opacity=0.7] (0.6180, 1.8060, 1.1847) -- (0.6670, 1.8060, 1.1896) -- (0.6670, 1.8570, 1.1875) -- (0.6180, 1.8570, 1.1826) -- cycle;
\fill[blue!76.4, opacity=0.7] (0.6180, 1.8570, 1.1826) -- (0.6670, 1.8570, 1.1875) -- (0.6670, 1.9080, 1.1851) -- (0.6180, 1.9080, 1.1802) -- cycle;
\fill[blue!85.4, opacity=0.7] (0.6180, 1.9080, 1.1802) -- (0.6670, 1.9080, 1.1851) -- (0.6670, 1.9590, 1.1824) -- (0.6180, 1.9590, 1.1775) -- cycle;
\fill[blue!89.4, opacity=0.7] (0.6180, 1.9590, 1.1775) -- (0.6670, 1.9590, 1.1824) -- (0.6670, 2.0100, 1.1794) -- (0.6180, 2.0100, 1.1745) -- cycle;
\fill[blue!85.4, opacity=0.7] (0.6180, 2.0100, 1.1745) -- (0.6670, 2.0100, 1.1794) -- (0.6670, 2.0610, 1.1762) -- (0.6180, 2.0610, 1.1712) -- cycle;
\fill[blue!69.3, opacity=0.7] (0.6180, 2.0610, 1.1712) -- (0.6670, 2.0610, 1.1762) -- (0.6670, 2.1120, 1.1726) -- (0.6180, 2.1120, 1.1676) -- cycle;
\fill[blue!42.1, opacity=0.7] (0.6180, 2.1120, 1.1676) -- (0.6670, 2.1120, 1.1726) -- (0.6670, 2.1630, 1.1688) -- (0.6180, 2.1630, 1.1638) -- cycle;
\fill[blue!21.0, opacity=0.7] (0.6180, 2.1630, 1.1638) -- (0.6670, 2.1630, 1.1688) -- (0.6670, 2.2140, 1.1647) -- (0.6180, 2.2140, 1.1597) -- cycle;
\fill[blue!15.4, opacity=0.7] (0.6180, 2.2140, 1.1597) -- (0.6670, 2.2140, 1.1647) -- (0.6670, 2.2650, 1.1604) -- (0.6180, 2.2650, 1.1554) -- cycle;
\fill[blue!15.0, opacity=0.7] (0.6180, 2.2650, 1.1554) -- (0.6670, 2.2650, 1.1604) -- (0.6670, 2.3160, 1.1558) -- (0.6180, 2.3160, 1.1508) -- cycle;
\fill[blue!15.0, opacity=0.7] (0.6180, 2.3160, 1.1508) -- (0.6670, 2.3160, 1.1558) -- (0.6670, 2.3670, 1.1510) -- (0.6180, 2.3670, 1.1461) -- cycle;
\fill[blue!15.0, opacity=0.7] (0.6180, 2.3670, 1.1461) -- (0.6670, 2.3670, 1.1510) -- (0.6670, 2.4180, 1.1461) -- (0.6180, 2.4180, 1.1411) -- cycle;
\fill[blue!15.0, opacity=0.7] (0.6180, 2.4180, 1.1411) -- (0.6670, 2.4180, 1.1461) -- (0.6670, 2.4690, 1.1409) -- (0.6180, 2.4690, 1.1359) -- cycle;
\fill[blue!15.0, opacity=0.7] (0.6180, 2.4690, 1.1359) -- (0.6670, 2.4690, 1.1409) -- (0.6670, 2.5200, 1.1355) -- (0.6180, 2.5200, 1.1305) -- cycle;
\fill[blue!15.0, opacity=0.7] (0.6180, 2.5200, 1.1305) -- (0.6670, 2.5200, 1.1355) -- (0.6670, 2.5710, 1.1300) -- (0.6180, 2.5710, 1.1250) -- cycle;
\fill[blue!15.0, opacity=0.7] (0.6180, 2.5710, 1.1250) -- (0.6670, 2.5710, 1.1300) -- (0.6670, 2.6220, 1.1243) -- (0.6180, 2.6220, 1.1193) -- cycle;
\fill[blue!15.5, opacity=0.7] (0.6180, 2.6220, 1.1193) -- (0.6670, 2.6220, 1.1243) -- (0.6670, 2.6730, 1.1185) -- (0.6180, 2.6730, 1.1135) -- cycle;
\fill[blue!17.0, opacity=0.7] (0.6180, 2.6730, 1.1135) -- (0.6670, 2.6730, 1.1185) -- (0.6670, 2.7240, 1.1126) -- (0.6180, 2.7240, 1.1076) -- cycle;
\fill[blue!16.4, opacity=0.7] (0.6180, 2.7240, 1.1076) -- (0.6670, 2.7240, 1.1126) -- (0.6670, 2.7750, 1.1066) -- (0.6180, 2.7750, 1.1016) -- cycle;
\fill[blue!15.1, opacity=0.7] (0.6180, 2.7750, 1.1016) -- (0.6670, 2.7750, 1.1066) -- (0.6670, 2.8260, 1.1005) -- (0.6180, 2.8260, 1.0955) -- cycle;
\fill[blue!15.0, opacity=0.7] (0.6180, 2.8260, 1.0955) -- (0.6670, 2.8260, 1.1005) -- (0.6670, 2.8770, 1.0943) -- (0.6180, 2.8770, 1.0893) -- cycle;
\fill[blue!15.0, opacity=0.7] (0.6180, 2.8770, 1.0893) -- (0.6670, 2.8770, 1.0943) -- (0.6670, 2.9280, 1.0881) -- (0.6180, 2.9280, 1.0831) -- cycle;
\fill[blue!15.0, opacity=0.7] (0.6180, 2.9280, 1.0831) -- (0.6670, 2.9280, 1.0881) -- (0.6670, 2.9790, 1.0818) -- (0.6180, 2.9790, 1.0768) -- cycle;
\fill[blue!15.0, opacity=0.7] (0.6180, 2.9790, 1.0768) -- (0.6670, 2.9790, 1.0818) -- (0.6670, 3.0300, 1.0755) -- (0.6180, 3.0300, 1.0705) -- cycle;
\fill[blue!15.0, opacity=0.7] (0.6670, -0.0300, 1.0755) -- (0.7160, -0.0300, 1.0803) -- (0.7160, 0.0210, 1.0866) -- (0.6670, 0.0210, 1.0818) -- cycle;
\fill[blue!15.2, opacity=0.7] (0.6670, 0.0210, 1.0818) -- (0.7160, 0.0210, 1.0866) -- (0.7160, 0.0720, 1.0928) -- (0.6670, 0.0720, 1.0881) -- cycle;
\fill[blue!15.1, opacity=0.7] (0.6670, 0.0720, 1.0881) -- (0.7160, 0.0720, 1.0928) -- (0.7160, 0.1230, 1.0991) -- (0.6670, 0.1230, 1.0943) -- cycle;
\fill[blue!15.0, opacity=0.7] (0.6670, 0.1230, 1.0943) -- (0.7160, 0.1230, 1.0991) -- (0.7160, 0.1740, 1.1052) -- (0.6670, 0.1740, 1.1005) -- cycle;
\fill[blue!15.0, opacity=0.7] (0.6670, 0.1740, 1.1005) -- (0.7160, 0.1740, 1.1052) -- (0.7160, 0.2250, 1.1114) -- (0.6670, 0.2250, 1.1066) -- cycle;
\fill[blue!15.0, opacity=0.7] (0.6670, 0.2250, 1.1066) -- (0.7160, 0.2250, 1.1114) -- (0.7160, 0.2760, 1.1174) -- (0.6670, 0.2760, 1.1126) -- cycle;
\fill[blue!15.0, opacity=0.7] (0.6670, 0.2760, 1.1126) -- (0.7160, 0.2760, 1.1174) -- (0.7160, 0.3270, 1.1233) -- (0.6670, 0.3270, 1.1185) -- cycle;
\fill[blue!15.0, opacity=0.7] (0.6670, 0.3270, 1.1185) -- (0.7160, 0.3270, 1.1233) -- (0.7160, 0.3780, 1.1291) -- (0.6670, 0.3780, 1.1243) -- cycle;
\fill[blue!15.0, opacity=0.7] (0.6670, 0.3780, 1.1243) -- (0.7160, 0.3780, 1.1291) -- (0.7160, 0.4290, 1.1348) -- (0.6670, 0.4290, 1.1300) -- cycle;
\fill[blue!15.6, opacity=0.7] (0.6670, 0.4290, 1.1300) -- (0.7160, 0.4290, 1.1348) -- (0.7160, 0.4800, 1.1403) -- (0.6670, 0.4800, 1.1355) -- cycle;
\fill[blue!25.3, opacity=0.7] (0.6670, 0.4800, 1.1355) -- (0.7160, 0.4800, 1.1403) -- (0.7160, 0.5310, 1.1457) -- (0.6670, 0.5310, 1.1409) -- cycle;
\fill[blue!36.2, opacity=0.7] (0.6670, 0.5310, 1.1409) -- (0.7160, 0.5310, 1.1457) -- (0.7160, 0.5820, 1.1508) -- (0.6670, 0.5820, 1.1461) -- cycle;
\fill[blue!24.9, opacity=0.7] (0.6670, 0.5820, 1.1461) -- (0.7160, 0.5820, 1.1508) -- (0.7160, 0.6330, 1.1558) -- (0.6670, 0.6330, 1.1510) -- cycle;
\fill[blue!15.9, opacity=0.7] (0.6670, 0.6330, 1.1510) -- (0.7160, 0.6330, 1.1558) -- (0.7160, 0.6840, 1.1606) -- (0.6670, 0.6840, 1.1558) -- cycle;
\fill[blue!15.0, opacity=0.7] (0.6670, 0.6840, 1.1558) -- (0.7160, 0.6840, 1.1606) -- (0.7160, 0.7350, 1.1651) -- (0.6670, 0.7350, 1.1604) -- cycle;
\fill[blue!15.0, opacity=0.7] (0.6670, 0.7350, 1.1604) -- (0.7160, 0.7350, 1.1651) -- (0.7160, 0.7860, 1.1695) -- (0.6670, 0.7860, 1.1647) -- cycle;
\fill[blue!15.0, opacity=0.7] (0.6670, 0.7860, 1.1647) -- (0.7160, 0.7860, 1.1695) -- (0.7160, 0.8370, 1.1736) -- (0.6670, 0.8370, 1.1688) -- cycle;
\fill[blue!15.0, opacity=0.7] (0.6670, 0.8370, 1.1688) -- (0.7160, 0.8370, 1.1736) -- (0.7160, 0.8880, 1.1774) -- (0.6670, 0.8880, 1.1726) -- cycle;
\fill[blue!15.0, opacity=0.7] (0.6670, 0.8880, 1.1726) -- (0.7160, 0.8880, 1.1774) -- (0.7160, 0.9390, 1.1809) -- (0.6670, 0.9390, 1.1762) -- cycle;
\fill[blue!16.3, opacity=0.7] (0.6670, 0.9390, 1.1762) -- (0.7160, 0.9390, 1.1809) -- (0.7160, 0.9900, 1.1842) -- (0.6670, 0.9900, 1.1794) -- cycle;
\fill[blue!33.2, opacity=0.7] (0.6670, 0.9900, 1.1794) -- (0.7160, 0.9900, 1.1842) -- (0.7160, 1.0410, 1.1872) -- (0.6670, 1.0410, 1.1824) -- cycle;
\fill[blue!76.9, opacity=0.7] (0.6670, 1.0410, 1.1824) -- (0.7160, 1.0410, 1.1872) -- (0.7160, 1.0920, 1.1899) -- (0.6670, 1.0920, 1.1851) -- cycle;
\fill[blue!88.8!black, opacity=0.7] (0.6670, 1.0920, 1.1851) -- (0.7160, 1.0920, 1.1899) -- (0.7160, 1.1430, 1.1923) -- (0.6670, 1.1430, 1.1875) -- cycle;
\fill[blue!75.2!black, opacity=0.7] (0.6670, 1.1430, 1.1875) -- (0.7160, 1.1430, 1.1923) -- (0.7160, 1.1940, 1.1944) -- (0.6670, 1.1940, 1.1896) -- cycle;
\fill[blue!84.6!black, opacity=0.7] (0.6670, 1.1940, 1.1896) -- (0.7160, 1.1940, 1.1944) -- (0.7160, 1.2450, 1.1962) -- (0.6670, 1.2450, 1.1914) -- cycle;
\fill[blue!90.9, opacity=0.7] (0.6670, 1.2450, 1.1914) -- (0.7160, 1.2450, 1.1962) -- (0.7160, 1.2960, 1.1977) -- (0.6670, 1.2960, 1.1929) -- cycle;
\fill[blue!65.5, opacity=0.7] (0.6670, 1.2960, 1.1929) -- (0.7160, 1.2960, 1.1977) -- (0.7160, 1.3470, 1.1988) -- (0.6670, 1.3470, 1.1940) -- cycle;
\fill[blue!42.0, opacity=0.7] (0.6670, 1.3470, 1.1940) -- (0.7160, 1.3470, 1.1988) -- (0.7160, 1.3980, 1.1996) -- (0.6670, 1.3980, 1.1949) -- cycle;
\fill[blue!27.8, opacity=0.7] (0.6670, 1.3980, 1.1949) -- (0.7160, 1.3980, 1.1996) -- (0.7160, 1.4490, 1.2001) -- (0.6670, 1.4490, 1.1954) -- cycle;
\fill[blue!21.1, opacity=0.7] (0.6670, 1.4490, 1.1954) -- (0.7160, 1.4490, 1.2001) -- (0.7160, 1.5000, 1.2003) -- (0.6670, 1.5000, 1.1955) -- cycle;
\fill[blue!18.3, opacity=0.7] (0.6670, 1.5000, 1.1955) -- (0.7160, 1.5000, 1.2003) -- (0.7160, 1.5510, 1.2001) -- (0.6670, 1.5510, 1.1954) -- cycle;
\fill[blue!17.1, opacity=0.7] (0.6670, 1.5510, 1.1954) -- (0.7160, 1.5510, 1.2001) -- (0.7160, 1.6020, 1.1996) -- (0.6670, 1.6020, 1.1949) -- cycle;
\fill[blue!16.6, opacity=0.7] (0.6670, 1.6020, 1.1949) -- (0.7160, 1.6020, 1.1996) -- (0.7160, 1.6530, 1.1988) -- (0.6670, 1.6530, 1.1940) -- cycle;
\fill[blue!16.6, opacity=0.7] (0.6670, 1.6530, 1.1940) -- (0.7160, 1.6530, 1.1988) -- (0.7160, 1.7040, 1.1977) -- (0.6670, 1.7040, 1.1929) -- cycle;
\fill[blue!17.0, opacity=0.7] (0.6670, 1.7040, 1.1929) -- (0.7160, 1.7040, 1.1977) -- (0.7160, 1.7550, 1.1962) -- (0.6670, 1.7550, 1.1914) -- cycle;
\fill[blue!18.1, opacity=0.7] (0.6670, 1.7550, 1.1914) -- (0.7160, 1.7550, 1.1962) -- (0.7160, 1.8060, 1.1944) -- (0.6670, 1.8060, 1.1896) -- cycle;
\fill[blue!20.7, opacity=0.7] (0.6670, 1.8060, 1.1896) -- (0.7160, 1.8060, 1.1944) -- (0.7160, 1.8570, 1.1923) -- (0.6670, 1.8570, 1.1875) -- cycle;
\fill[blue!26.5, opacity=0.7] (0.6670, 1.8570, 1.1875) -- (0.7160, 1.8570, 1.1923) -- (0.7160, 1.9080, 1.1899) -- (0.6670, 1.9080, 1.1851) -- cycle;
\fill[blue!38.2, opacity=0.7] (0.6670, 1.9080, 1.1851) -- (0.7160, 1.9080, 1.1899) -- (0.7160, 1.9590, 1.1872) -- (0.6670, 1.9590, 1.1824) -- cycle;
\fill[blue!56.2, opacity=0.7] (0.6670, 1.9590, 1.1824) -- (0.7160, 1.9590, 1.1872) -- (0.7160, 2.0100, 1.1842) -- (0.6670, 2.0100, 1.1794) -- cycle;
\fill[blue!74.7, opacity=0.7] (0.6670, 2.0100, 1.1794) -- (0.7160, 2.0100, 1.1842) -- (0.7160, 2.0610, 1.1809) -- (0.6670, 2.0610, 1.1762) -- cycle;
\fill[blue!85.1, opacity=0.7] (0.6670, 2.0610, 1.1762) -- (0.7160, 2.0610, 1.1809) -- (0.7160, 2.1120, 1.1774) -- (0.6670, 2.1120, 1.1726) -- cycle;
\fill[blue!81.9, opacity=0.7] (0.6670, 2.1120, 1.1726) -- (0.7160, 2.1120, 1.1774) -- (0.7160, 2.1630, 1.1736) -- (0.6670, 2.1630, 1.1688) -- cycle;
\fill[blue!60.7, opacity=0.7] (0.6670, 2.1630, 1.1688) -- (0.7160, 2.1630, 1.1736) -- (0.7160, 2.2140, 1.1695) -- (0.6670, 2.2140, 1.1647) -- cycle;
\fill[blue!30.4, opacity=0.7] (0.6670, 2.2140, 1.1647) -- (0.7160, 2.2140, 1.1695) -- (0.7160, 2.2650, 1.1651) -- (0.6670, 2.2650, 1.1604) -- cycle;
\fill[blue!16.5, opacity=0.7] (0.6670, 2.2650, 1.1604) -- (0.7160, 2.2650, 1.1651) -- (0.7160, 2.3160, 1.1606) -- (0.6670, 2.3160, 1.1558) -- cycle;
\fill[blue!15.0, opacity=0.7] (0.6670, 2.3160, 1.1558) -- (0.7160, 2.3160, 1.1606) -- (0.7160, 2.3670, 1.1558) -- (0.6670, 2.3670, 1.1510) -- cycle;
\fill[blue!15.0, opacity=0.7] (0.6670, 2.3670, 1.1510) -- (0.7160, 2.3670, 1.1558) -- (0.7160, 2.4180, 1.1508) -- (0.6670, 2.4180, 1.1461) -- cycle;
\fill[blue!15.0, opacity=0.7] (0.6670, 2.4180, 1.1461) -- (0.7160, 2.4180, 1.1508) -- (0.7160, 2.4690, 1.1457) -- (0.6670, 2.4690, 1.1409) -- cycle;
\fill[blue!15.0, opacity=0.7] (0.6670, 2.4690, 1.1409) -- (0.7160, 2.4690, 1.1457) -- (0.7160, 2.5200, 1.1403) -- (0.6670, 2.5200, 1.1355) -- cycle;
\fill[blue!15.0, opacity=0.7] (0.6670, 2.5200, 1.1355) -- (0.7160, 2.5200, 1.1403) -- (0.7160, 2.5710, 1.1348) -- (0.6670, 2.5710, 1.1300) -- cycle;
\fill[blue!15.0, opacity=0.7] (0.6670, 2.5710, 1.1300) -- (0.7160, 2.5710, 1.1348) -- (0.7160, 2.6220, 1.1291) -- (0.6670, 2.6220, 1.1243) -- cycle;
\fill[blue!15.0, opacity=0.7] (0.6670, 2.6220, 1.1243) -- (0.7160, 2.6220, 1.1291) -- (0.7160, 2.6730, 1.1233) -- (0.6670, 2.6730, 1.1185) -- cycle;
\fill[blue!15.6, opacity=0.7] (0.6670, 2.6730, 1.1185) -- (0.7160, 2.6730, 1.1233) -- (0.7160, 2.7240, 1.1174) -- (0.6670, 2.7240, 1.1126) -- cycle;
\fill[blue!17.0, opacity=0.7] (0.6670, 2.7240, 1.1126) -- (0.7160, 2.7240, 1.1174) -- (0.7160, 2.7750, 1.1114) -- (0.6670, 2.7750, 1.1066) -- cycle;
\fill[blue!15.9, opacity=0.7] (0.6670, 2.7750, 1.1066) -- (0.7160, 2.7750, 1.1114) -- (0.7160, 2.8260, 1.1052) -- (0.6670, 2.8260, 1.1005) -- cycle;
\fill[blue!15.0, opacity=0.7] (0.6670, 2.8260, 1.1005) -- (0.7160, 2.8260, 1.1052) -- (0.7160, 2.8770, 1.0991) -- (0.6670, 2.8770, 1.0943) -- cycle;
\fill[blue!15.0, opacity=0.7] (0.6670, 2.8770, 1.0943) -- (0.7160, 2.8770, 1.0991) -- (0.7160, 2.9280, 1.0928) -- (0.6670, 2.9280, 1.0881) -- cycle;
\fill[blue!15.0, opacity=0.7] (0.6670, 2.9280, 1.0881) -- (0.7160, 2.9280, 1.0928) -- (0.7160, 2.9790, 1.0866) -- (0.6670, 2.9790, 1.0818) -- cycle;
\fill[blue!15.0, opacity=0.7] (0.6670, 2.9790, 1.0818) -- (0.7160, 2.9790, 1.0866) -- (0.7160, 3.0300, 1.0803) -- (0.6670, 3.0300, 1.0755) -- cycle;
\fill[blue!15.1, opacity=0.7] (0.7160, -0.0300, 1.0803) -- (0.7650, -0.0300, 1.0849) -- (0.7650, 0.0210, 1.0911) -- (0.7160, 0.0210, 1.0866) -- cycle;
\fill[blue!15.3, opacity=0.7] (0.7160, 0.0210, 1.0866) -- (0.7650, 0.0210, 1.0911) -- (0.7650, 0.0720, 1.0974) -- (0.7160, 0.0720, 1.0928) -- cycle;
\fill[blue!15.0, opacity=0.7] (0.7160, 0.0720, 1.0928) -- (0.7650, 0.0720, 1.0974) -- (0.7650, 0.1230, 1.1036) -- (0.7160, 0.1230, 1.0991) -- cycle;
\fill[blue!15.0, opacity=0.7] (0.7160, 0.1230, 1.0991) -- (0.7650, 0.1230, 1.1036) -- (0.7650, 0.1740, 1.1098) -- (0.7160, 0.1740, 1.1052) -- cycle;
\fill[blue!15.0, opacity=0.7] (0.7160, 0.1740, 1.1052) -- (0.7650, 0.1740, 1.1098) -- (0.7650, 0.2250, 1.1159) -- (0.7160, 0.2250, 1.1114) -- cycle;
\fill[blue!15.0, opacity=0.7] (0.7160, 0.2250, 1.1114) -- (0.7650, 0.2250, 1.1159) -- (0.7650, 0.2760, 1.1219) -- (0.7160, 0.2760, 1.1174) -- cycle;
\fill[blue!15.0, opacity=0.7] (0.7160, 0.2760, 1.1174) -- (0.7650, 0.2760, 1.1219) -- (0.7650, 0.3270, 1.1279) -- (0.7160, 0.3270, 1.1233) -- cycle;
\fill[blue!15.0, opacity=0.7] (0.7160, 0.3270, 1.1233) -- (0.7650, 0.3270, 1.1279) -- (0.7650, 0.3780, 1.1337) -- (0.7160, 0.3780, 1.1291) -- cycle;
\fill[blue!15.1, opacity=0.7] (0.7160, 0.3780, 1.1291) -- (0.7650, 0.3780, 1.1337) -- (0.7650, 0.4290, 1.1393) -- (0.7160, 0.4290, 1.1348) -- cycle;
\fill[blue!20.2, opacity=0.7] (0.7160, 0.4290, 1.1348) -- (0.7650, 0.4290, 1.1393) -- (0.7650, 0.4800, 1.1449) -- (0.7160, 0.4800, 1.1403) -- cycle;
\fill[blue!35.9, opacity=0.7] (0.7160, 0.4800, 1.1403) -- (0.7650, 0.4800, 1.1449) -- (0.7650, 0.5310, 1.1502) -- (0.7160, 0.5310, 1.1457) -- cycle;
\fill[blue!30.3, opacity=0.7] (0.7160, 0.5310, 1.1457) -- (0.7650, 0.5310, 1.1502) -- (0.7650, 0.5820, 1.1554) -- (0.7160, 0.5820, 1.1508) -- cycle;
\fill[blue!17.1, opacity=0.7] (0.7160, 0.5820, 1.1508) -- (0.7650, 0.5820, 1.1554) -- (0.7650, 0.6330, 1.1604) -- (0.7160, 0.6330, 1.1558) -- cycle;
\fill[blue!15.1, opacity=0.7] (0.7160, 0.6330, 1.1558) -- (0.7650, 0.6330, 1.1604) -- (0.7650, 0.6840, 1.1651) -- (0.7160, 0.6840, 1.1606) -- cycle;
\fill[blue!15.0, opacity=0.7] (0.7160, 0.6840, 1.1606) -- (0.7650, 0.6840, 1.1651) -- (0.7650, 0.7350, 1.1697) -- (0.7160, 0.7350, 1.1651) -- cycle;
\fill[blue!15.0, opacity=0.7] (0.7160, 0.7350, 1.1651) -- (0.7650, 0.7350, 1.1697) -- (0.7650, 0.7860, 1.1740) -- (0.7160, 0.7860, 1.1695) -- cycle;
\fill[blue!15.0, opacity=0.7] (0.7160, 0.7860, 1.1695) -- (0.7650, 0.7860, 1.1740) -- (0.7650, 0.8370, 1.1781) -- (0.7160, 0.8370, 1.1736) -- cycle;
\fill[blue!15.0, opacity=0.7] (0.7160, 0.8370, 1.1736) -- (0.7650, 0.8370, 1.1781) -- (0.7650, 0.8880, 1.1819) -- (0.7160, 0.8880, 1.1774) -- cycle;
\fill[blue!17.1, opacity=0.7] (0.7160, 0.8880, 1.1774) -- (0.7650, 0.8880, 1.1819) -- (0.7650, 0.9390, 1.1855) -- (0.7160, 0.9390, 1.1809) -- cycle;
\fill[blue!42.6, opacity=0.7] (0.7160, 0.9390, 1.1809) -- (0.7650, 0.9390, 1.1855) -- (0.7650, 0.9900, 1.1888) -- (0.7160, 0.9900, 1.1842) -- cycle;
\fill[blue!91.7, opacity=0.7] (0.7160, 0.9900, 1.1842) -- (0.7650, 0.9900, 1.1888) -- (0.7650, 1.0410, 1.1918) -- (0.7160, 1.0410, 1.1872) -- cycle;
\fill[blue!75.0!black, opacity=0.7] (0.7160, 1.0410, 1.1872) -- (0.7650, 1.0410, 1.1918) -- (0.7650, 1.0920, 1.1945) -- (0.7160, 1.0920, 1.1899) -- cycle;
\fill[blue!74.7!black, opacity=0.7] (0.7160, 1.0920, 1.1899) -- (0.7650, 1.0920, 1.1945) -- (0.7650, 1.1430, 1.1969) -- (0.7160, 1.1430, 1.1923) -- cycle;
\fill[blue!97.1, opacity=0.7] (0.7160, 1.1430, 1.1923) -- (0.7650, 1.1430, 1.1969) -- (0.7650, 1.1940, 1.1990) -- (0.7160, 1.1940, 1.1944) -- cycle;
\fill[blue!65.1, opacity=0.7] (0.7160, 1.1940, 1.1944) -- (0.7650, 1.1940, 1.1990) -- (0.7650, 1.2450, 1.2008) -- (0.7160, 1.2450, 1.1962) -- cycle;
\fill[blue!34.6, opacity=0.7] (0.7160, 1.2450, 1.1962) -- (0.7650, 1.2450, 1.2008) -- (0.7650, 1.2960, 1.2022) -- (0.7160, 1.2960, 1.1977) -- cycle;
\fill[blue!20.6, opacity=0.7] (0.7160, 1.2960, 1.1977) -- (0.7650, 1.2960, 1.2022) -- (0.7650, 1.3470, 1.2034) -- (0.7160, 1.3470, 1.1988) -- cycle;
\fill[blue!16.5, opacity=0.7] (0.7160, 1.3470, 1.1988) -- (0.7650, 1.3470, 1.2034) -- (0.7650, 1.3980, 1.2042) -- (0.7160, 1.3980, 1.1996) -- cycle;
\fill[blue!15.5, opacity=0.7] (0.7160, 1.3980, 1.1996) -- (0.7650, 1.3980, 1.2042) -- (0.7650, 1.4490, 1.2047) -- (0.7160, 1.4490, 1.2001) -- cycle;
\fill[blue!15.2, opacity=0.7] (0.7160, 1.4490, 1.2001) -- (0.7650, 1.4490, 1.2047) -- (0.7650, 1.5000, 1.2049) -- (0.7160, 1.5000, 1.2003) -- cycle;
\fill[blue!15.1, opacity=0.7] (0.7160, 1.5000, 1.2003) -- (0.7650, 1.5000, 1.2049) -- (0.7650, 1.5510, 1.2047) -- (0.7160, 1.5510, 1.2001) -- cycle;
\fill[blue!15.1, opacity=0.7] (0.7160, 1.5510, 1.2001) -- (0.7650, 1.5510, 1.2047) -- (0.7650, 1.6020, 1.2042) -- (0.7160, 1.6020, 1.1996) -- cycle;
\fill[blue!15.1, opacity=0.7] (0.7160, 1.6020, 1.1996) -- (0.7650, 1.6020, 1.2042) -- (0.7650, 1.6530, 1.2034) -- (0.7160, 1.6530, 1.1988) -- cycle;
\fill[blue!15.1, opacity=0.7] (0.7160, 1.6530, 1.1988) -- (0.7650, 1.6530, 1.2034) -- (0.7650, 1.7040, 1.2022) -- (0.7160, 1.7040, 1.1977) -- cycle;
\fill[blue!15.1, opacity=0.7] (0.7160, 1.7040, 1.1977) -- (0.7650, 1.7040, 1.2022) -- (0.7650, 1.7550, 1.2008) -- (0.7160, 1.7550, 1.1962) -- cycle;
\fill[blue!15.1, opacity=0.7] (0.7160, 1.7550, 1.1962) -- (0.7650, 1.7550, 1.2008) -- (0.7650, 1.8060, 1.1990) -- (0.7160, 1.8060, 1.1944) -- cycle;
\fill[blue!15.1, opacity=0.7] (0.7160, 1.8060, 1.1944) -- (0.7650, 1.8060, 1.1990) -- (0.7650, 1.8570, 1.1969) -- (0.7160, 1.8570, 1.1923) -- cycle;
\fill[blue!15.4, opacity=0.7] (0.7160, 1.8570, 1.1923) -- (0.7650, 1.8570, 1.1969) -- (0.7650, 1.9080, 1.1945) -- (0.7160, 1.9080, 1.1899) -- cycle;
\fill[blue!16.1, opacity=0.7] (0.7160, 1.9080, 1.1899) -- (0.7650, 1.9080, 1.1945) -- (0.7650, 1.9590, 1.1918) -- (0.7160, 1.9590, 1.1872) -- cycle;
\fill[blue!19.2, opacity=0.7] (0.7160, 1.9590, 1.1872) -- (0.7650, 1.9590, 1.1918) -- (0.7650, 2.0100, 1.1888) -- (0.7160, 2.0100, 1.1842) -- cycle;
\fill[blue!29.2, opacity=0.7] (0.7160, 2.0100, 1.1842) -- (0.7650, 2.0100, 1.1888) -- (0.7650, 2.0610, 1.1855) -- (0.7160, 2.0610, 1.1809) -- cycle;
\fill[blue!50.1, opacity=0.7] (0.7160, 2.0610, 1.1809) -- (0.7650, 2.0610, 1.1855) -- (0.7650, 2.1120, 1.1819) -- (0.7160, 2.1120, 1.1774) -- cycle;
\fill[blue!73.2, opacity=0.7] (0.7160, 2.1120, 1.1774) -- (0.7650, 2.1120, 1.1819) -- (0.7650, 2.1630, 1.1781) -- (0.7160, 2.1630, 1.1736) -- cycle;
\fill[blue!82.7, opacity=0.7] (0.7160, 2.1630, 1.1736) -- (0.7650, 2.1630, 1.1781) -- (0.7650, 2.2140, 1.1740) -- (0.7160, 2.2140, 1.1695) -- cycle;
\fill[blue!71.1, opacity=0.7] (0.7160, 2.2140, 1.1695) -- (0.7650, 2.2140, 1.1740) -- (0.7650, 2.2650, 1.1697) -- (0.7160, 2.2650, 1.1651) -- cycle;
\fill[blue!39.6, opacity=0.7] (0.7160, 2.2650, 1.1651) -- (0.7650, 2.2650, 1.1697) -- (0.7650, 2.3160, 1.1651) -- (0.7160, 2.3160, 1.1606) -- cycle;
\fill[blue!18.0, opacity=0.7] (0.7160, 2.3160, 1.1606) -- (0.7650, 2.3160, 1.1651) -- (0.7650, 2.3670, 1.1604) -- (0.7160, 2.3670, 1.1558) -- cycle;
\fill[blue!15.1, opacity=0.7] (0.7160, 2.3670, 1.1558) -- (0.7650, 2.3670, 1.1604) -- (0.7650, 2.4180, 1.1554) -- (0.7160, 2.4180, 1.1508) -- cycle;
\fill[blue!15.0, opacity=0.7] (0.7160, 2.4180, 1.1508) -- (0.7650, 2.4180, 1.1554) -- (0.7650, 2.4690, 1.1502) -- (0.7160, 2.4690, 1.1457) -- cycle;
\fill[blue!15.0, opacity=0.7] (0.7160, 2.4690, 1.1457) -- (0.7650, 2.4690, 1.1502) -- (0.7650, 2.5200, 1.1449) -- (0.7160, 2.5200, 1.1403) -- cycle;
\fill[blue!15.0, opacity=0.7] (0.7160, 2.5200, 1.1403) -- (0.7650, 2.5200, 1.1449) -- (0.7650, 2.5710, 1.1393) -- (0.7160, 2.5710, 1.1348) -- cycle;
\fill[blue!15.0, opacity=0.7] (0.7160, 2.5710, 1.1348) -- (0.7650, 2.5710, 1.1393) -- (0.7650, 2.6220, 1.1337) -- (0.7160, 2.6220, 1.1291) -- cycle;
\fill[blue!15.0, opacity=0.7] (0.7160, 2.6220, 1.1291) -- (0.7650, 2.6220, 1.1337) -- (0.7650, 2.6730, 1.1279) -- (0.7160, 2.6730, 1.1233) -- cycle;
\fill[blue!15.0, opacity=0.7] (0.7160, 2.6730, 1.1233) -- (0.7650, 2.6730, 1.1279) -- (0.7650, 2.7240, 1.1219) -- (0.7160, 2.7240, 1.1174) -- cycle;
\fill[blue!15.8, opacity=0.7] (0.7160, 2.7240, 1.1174) -- (0.7650, 2.7240, 1.1219) -- (0.7650, 2.7750, 1.1159) -- (0.7160, 2.7750, 1.1114) -- cycle;
\fill[blue!16.9, opacity=0.7] (0.7160, 2.7750, 1.1114) -- (0.7650, 2.7750, 1.1159) -- (0.7650, 2.8260, 1.1098) -- (0.7160, 2.8260, 1.1052) -- cycle;
\fill[blue!15.5, opacity=0.7] (0.7160, 2.8260, 1.1052) -- (0.7650, 2.8260, 1.1098) -- (0.7650, 2.8770, 1.1036) -- (0.7160, 2.8770, 1.0991) -- cycle;
\fill[blue!15.0, opacity=0.7] (0.7160, 2.8770, 1.0991) -- (0.7650, 2.8770, 1.1036) -- (0.7650, 2.9280, 1.0974) -- (0.7160, 2.9280, 1.0928) -- cycle;
\fill[blue!15.0, opacity=0.7] (0.7160, 2.9280, 1.0928) -- (0.7650, 2.9280, 1.0974) -- (0.7650, 2.9790, 1.0911) -- (0.7160, 2.9790, 1.0866) -- cycle;
\fill[blue!15.0, opacity=0.7] (0.7160, 2.9790, 1.0866) -- (0.7650, 2.9790, 1.0911) -- (0.7650, 3.0300, 1.0849) -- (0.7160, 3.0300, 1.0803) -- cycle;
\fill[blue!15.2, opacity=0.7] (0.7650, -0.0300, 1.0849) -- (0.8140, -0.0300, 1.0892) -- (0.8140, 0.0210, 1.0955) -- (0.7650, 0.0210, 1.0911) -- cycle;
\fill[blue!15.2, opacity=0.7] (0.7650, 0.0210, 1.0911) -- (0.8140, 0.0210, 1.0955) -- (0.8140, 0.0720, 1.1017) -- (0.7650, 0.0720, 1.0974) -- cycle;
\fill[blue!15.0, opacity=0.7] (0.7650, 0.0720, 1.0974) -- (0.8140, 0.0720, 1.1017) -- (0.8140, 0.1230, 1.1079) -- (0.7650, 0.1230, 1.1036) -- cycle;
\fill[blue!15.0, opacity=0.7] (0.7650, 0.1230, 1.1036) -- (0.8140, 0.1230, 1.1079) -- (0.8140, 0.1740, 1.1141) -- (0.7650, 0.1740, 1.1098) -- cycle;
\fill[blue!15.0, opacity=0.7] (0.7650, 0.1740, 1.1098) -- (0.8140, 0.1740, 1.1141) -- (0.8140, 0.2250, 1.1202) -- (0.7650, 0.2250, 1.1159) -- cycle;
\fill[blue!15.0, opacity=0.7] (0.7650, 0.2250, 1.1159) -- (0.8140, 0.2250, 1.1202) -- (0.8140, 0.2760, 1.1263) -- (0.7650, 0.2760, 1.1219) -- cycle;
\fill[blue!15.0, opacity=0.7] (0.7650, 0.2760, 1.1219) -- (0.8140, 0.2760, 1.1263) -- (0.8140, 0.3270, 1.1322) -- (0.7650, 0.3270, 1.1279) -- cycle;
\fill[blue!15.0, opacity=0.7] (0.7650, 0.3270, 1.1279) -- (0.8140, 0.3270, 1.1322) -- (0.8140, 0.3780, 1.1380) -- (0.7650, 0.3780, 1.1337) -- cycle;
\fill[blue!16.4, opacity=0.7] (0.7650, 0.3780, 1.1337) -- (0.8140, 0.3780, 1.1380) -- (0.8140, 0.4290, 1.1437) -- (0.7650, 0.4290, 1.1393) -- cycle;
\fill[blue!31.0, opacity=0.7] (0.7650, 0.4290, 1.1393) -- (0.8140, 0.4290, 1.1437) -- (0.8140, 0.4800, 1.1492) -- (0.7650, 0.4800, 1.1449) -- cycle;
\fill[blue!36.5, opacity=0.7] (0.7650, 0.4800, 1.1449) -- (0.8140, 0.4800, 1.1492) -- (0.8140, 0.5310, 1.1545) -- (0.7650, 0.5310, 1.1502) -- cycle;
\fill[blue!20.6, opacity=0.7] (0.7650, 0.5310, 1.1502) -- (0.8140, 0.5310, 1.1545) -- (0.8140, 0.5820, 1.1597) -- (0.7650, 0.5820, 1.1554) -- cycle;
\fill[blue!15.2, opacity=0.7] (0.7650, 0.5820, 1.1554) -- (0.8140, 0.5820, 1.1597) -- (0.8140, 0.6330, 1.1647) -- (0.7650, 0.6330, 1.1604) -- cycle;
\fill[blue!15.0, opacity=0.7] (0.7650, 0.6330, 1.1604) -- (0.8140, 0.6330, 1.1647) -- (0.8140, 0.6840, 1.1695) -- (0.7650, 0.6840, 1.1651) -- cycle;
\fill[blue!15.0, opacity=0.7] (0.7650, 0.6840, 1.1651) -- (0.8140, 0.6840, 1.1695) -- (0.8140, 0.7350, 1.1740) -- (0.7650, 0.7350, 1.1697) -- cycle;
\fill[blue!15.0, opacity=0.7] (0.7650, 0.7350, 1.1697) -- (0.8140, 0.7350, 1.1740) -- (0.8140, 0.7860, 1.1784) -- (0.7650, 0.7860, 1.1740) -- cycle;
\fill[blue!15.0, opacity=0.7] (0.7650, 0.7860, 1.1740) -- (0.8140, 0.7860, 1.1784) -- (0.8140, 0.8370, 1.1824) -- (0.7650, 0.8370, 1.1781) -- cycle;
\fill[blue!16.9, opacity=0.7] (0.7650, 0.8370, 1.1781) -- (0.8140, 0.8370, 1.1824) -- (0.8140, 0.8880, 1.1863) -- (0.7650, 0.8880, 1.1819) -- cycle;
\fill[blue!45.6, opacity=0.7] (0.7650, 0.8880, 1.1819) -- (0.8140, 0.8880, 1.1863) -- (0.8140, 0.9390, 1.1898) -- (0.7650, 0.9390, 1.1855) -- cycle;
\fill[blue!97.7, opacity=0.7] (0.7650, 0.9390, 1.1855) -- (0.8140, 0.9390, 1.1898) -- (0.8140, 0.9900, 1.1931) -- (0.7650, 0.9900, 1.1888) -- cycle;
\fill[blue!70.9!black, opacity=0.7] (0.7650, 0.9900, 1.1888) -- (0.8140, 0.9900, 1.1931) -- (0.8140, 1.0410, 1.1961) -- (0.7650, 1.0410, 1.1918) -- cycle;
\fill[blue!76.8!black, opacity=0.7] (0.7650, 1.0410, 1.1918) -- (0.8140, 1.0410, 1.1961) -- (0.8140, 1.0920, 1.1988) -- (0.7650, 1.0920, 1.1945) -- cycle;
\fill[blue!85.8, opacity=0.7] (0.7650, 1.0920, 1.1945) -- (0.8140, 1.0920, 1.1988) -- (0.8140, 1.1430, 1.2012) -- (0.7650, 1.1430, 1.1969) -- cycle;
\fill[blue!44.3, opacity=0.7] (0.7650, 1.1430, 1.1969) -- (0.8140, 1.1430, 1.2012) -- (0.8140, 1.1940, 1.2033) -- (0.7650, 1.1940, 1.1990) -- cycle;
\fill[blue!21.6, opacity=0.7] (0.7650, 1.1940, 1.1990) -- (0.8140, 1.1940, 1.2033) -- (0.8140, 1.2450, 1.2051) -- (0.7650, 1.2450, 1.2008) -- cycle;
\fill[blue!16.2, opacity=0.7] (0.7650, 1.2450, 1.2008) -- (0.8140, 1.2450, 1.2051) -- (0.8140, 1.2960, 1.2066) -- (0.7650, 1.2960, 1.2022) -- cycle;
\fill[blue!15.3, opacity=0.7] (0.7650, 1.2960, 1.2022) -- (0.8140, 1.2960, 1.2066) -- (0.8140, 1.3470, 1.2077) -- (0.7650, 1.3470, 1.2034) -- cycle;
\fill[blue!15.1, opacity=0.7] (0.7650, 1.3470, 1.2034) -- (0.8140, 1.3470, 1.2077) -- (0.8140, 1.3980, 1.2085) -- (0.7650, 1.3980, 1.2042) -- cycle;
\fill[blue!15.1, opacity=0.7] (0.7650, 1.3980, 1.2042) -- (0.8140, 1.3980, 1.2085) -- (0.8140, 1.4490, 1.2090) -- (0.7650, 1.4490, 1.2047) -- cycle;
\fill[blue!15.1, opacity=0.7] (0.7650, 1.4490, 1.2047) -- (0.8140, 1.4490, 1.2090) -- (0.8140, 1.5000, 1.2092) -- (0.7650, 1.5000, 1.2049) -- cycle;
\fill[blue!15.1, opacity=0.7] (0.7650, 1.5000, 1.2049) -- (0.8140, 1.5000, 1.2092) -- (0.8140, 1.5510, 1.2090) -- (0.7650, 1.5510, 1.2047) -- cycle;
\fill[blue!15.1, opacity=0.7] (0.7650, 1.5510, 1.2047) -- (0.8140, 1.5510, 1.2090) -- (0.8140, 1.6020, 1.2085) -- (0.7650, 1.6020, 1.2042) -- cycle;
\fill[blue!15.1, opacity=0.7] (0.7650, 1.6020, 1.2042) -- (0.8140, 1.6020, 1.2085) -- (0.8140, 1.6530, 1.2077) -- (0.7650, 1.6530, 1.2034) -- cycle;
\fill[blue!15.1, opacity=0.7] (0.7650, 1.6530, 1.2034) -- (0.8140, 1.6530, 1.2077) -- (0.8140, 1.7040, 1.2066) -- (0.7650, 1.7040, 1.2022) -- cycle;
\fill[blue!15.1, opacity=0.7] (0.7650, 1.7040, 1.2022) -- (0.8140, 1.7040, 1.2066) -- (0.8140, 1.7550, 1.2051) -- (0.7650, 1.7550, 1.2008) -- cycle;
\fill[blue!15.0, opacity=0.7] (0.7650, 1.7550, 1.2008) -- (0.8140, 1.7550, 1.2051) -- (0.8140, 1.8060, 1.2033) -- (0.7650, 1.8060, 1.1990) -- cycle;
\fill[blue!15.0, opacity=0.7] (0.7650, 1.8060, 1.1990) -- (0.8140, 1.8060, 1.2033) -- (0.8140, 1.8570, 1.2012) -- (0.7650, 1.8570, 1.1969) -- cycle;
\fill[blue!15.0, opacity=0.7] (0.7650, 1.8570, 1.1969) -- (0.8140, 1.8570, 1.2012) -- (0.8140, 1.9080, 1.1988) -- (0.7650, 1.9080, 1.1945) -- cycle;
\fill[blue!15.0, opacity=0.7] (0.7650, 1.9080, 1.1945) -- (0.8140, 1.9080, 1.1988) -- (0.8140, 1.9590, 1.1961) -- (0.7650, 1.9590, 1.1918) -- cycle;
\fill[blue!15.1, opacity=0.7] (0.7650, 1.9590, 1.1918) -- (0.8140, 1.9590, 1.1961) -- (0.8140, 2.0100, 1.1931) -- (0.7650, 2.0100, 1.1888) -- cycle;
\fill[blue!15.6, opacity=0.7] (0.7650, 2.0100, 1.1888) -- (0.8140, 2.0100, 1.1931) -- (0.8140, 2.0610, 1.1898) -- (0.7650, 2.0610, 1.1855) -- cycle;
\fill[blue!18.6, opacity=0.7] (0.7650, 2.0610, 1.1855) -- (0.8140, 2.0610, 1.1898) -- (0.8140, 2.1120, 1.1863) -- (0.7650, 2.1120, 1.1819) -- cycle;
\fill[blue!31.5, opacity=0.7] (0.7650, 2.1120, 1.1819) -- (0.8140, 2.1120, 1.1863) -- (0.8140, 2.1630, 1.1824) -- (0.7650, 2.1630, 1.1781) -- cycle;
\fill[blue!58.0, opacity=0.7] (0.7650, 2.1630, 1.1781) -- (0.8140, 2.1630, 1.1824) -- (0.8140, 2.2140, 1.1784) -- (0.7650, 2.2140, 1.1740) -- cycle;
\fill[blue!78.2, opacity=0.7] (0.7650, 2.2140, 1.1740) -- (0.8140, 2.2140, 1.1784) -- (0.8140, 2.2650, 1.1740) -- (0.7650, 2.2650, 1.1697) -- cycle;
\fill[blue!74.6, opacity=0.7] (0.7650, 2.2650, 1.1697) -- (0.8140, 2.2650, 1.1740) -- (0.8140, 2.3160, 1.1695) -- (0.7650, 2.3160, 1.1651) -- cycle;
\fill[blue!44.6, opacity=0.7] (0.7650, 2.3160, 1.1651) -- (0.8140, 2.3160, 1.1695) -- (0.8140, 2.3670, 1.1647) -- (0.7650, 2.3670, 1.1604) -- cycle;
\fill[blue!18.9, opacity=0.7] (0.7650, 2.3670, 1.1604) -- (0.8140, 2.3670, 1.1647) -- (0.8140, 2.4180, 1.1597) -- (0.7650, 2.4180, 1.1554) -- cycle;
\fill[blue!15.1, opacity=0.7] (0.7650, 2.4180, 1.1554) -- (0.8140, 2.4180, 1.1597) -- (0.8140, 2.4690, 1.1545) -- (0.7650, 2.4690, 1.1502) -- cycle;
\fill[blue!15.0, opacity=0.7] (0.7650, 2.4690, 1.1502) -- (0.8140, 2.4690, 1.1545) -- (0.8140, 2.5200, 1.1492) -- (0.7650, 2.5200, 1.1449) -- cycle;
\fill[blue!15.0, opacity=0.7] (0.7650, 2.5200, 1.1449) -- (0.8140, 2.5200, 1.1492) -- (0.8140, 2.5710, 1.1437) -- (0.7650, 2.5710, 1.1393) -- cycle;
\fill[blue!15.0, opacity=0.7] (0.7650, 2.5710, 1.1393) -- (0.8140, 2.5710, 1.1437) -- (0.8140, 2.6220, 1.1380) -- (0.7650, 2.6220, 1.1337) -- cycle;
\fill[blue!15.0, opacity=0.7] (0.7650, 2.6220, 1.1337) -- (0.8140, 2.6220, 1.1380) -- (0.8140, 2.6730, 1.1322) -- (0.7650, 2.6730, 1.1279) -- cycle;
\fill[blue!15.0, opacity=0.7] (0.7650, 2.6730, 1.1279) -- (0.8140, 2.6730, 1.1322) -- (0.8140, 2.7240, 1.1263) -- (0.7650, 2.7240, 1.1219) -- cycle;
\fill[blue!15.1, opacity=0.7] (0.7650, 2.7240, 1.1219) -- (0.8140, 2.7240, 1.1263) -- (0.8140, 2.7750, 1.1202) -- (0.7650, 2.7750, 1.1159) -- cycle;
\fill[blue!16.2, opacity=0.7] (0.7650, 2.7750, 1.1159) -- (0.8140, 2.7750, 1.1202) -- (0.8140, 2.8260, 1.1141) -- (0.7650, 2.8260, 1.1098) -- cycle;
\fill[blue!16.5, opacity=0.7] (0.7650, 2.8260, 1.1098) -- (0.8140, 2.8260, 1.1141) -- (0.8140, 2.8770, 1.1079) -- (0.7650, 2.8770, 1.1036) -- cycle;
\fill[blue!15.2, opacity=0.7] (0.7650, 2.8770, 1.1036) -- (0.8140, 2.8770, 1.1079) -- (0.8140, 2.9280, 1.1017) -- (0.7650, 2.9280, 1.0974) -- cycle;
\fill[blue!15.0, opacity=0.7] (0.7650, 2.9280, 1.0974) -- (0.8140, 2.9280, 1.1017) -- (0.8140, 2.9790, 1.0955) -- (0.7650, 2.9790, 1.0911) -- cycle;
\fill[blue!15.0, opacity=0.7] (0.7650, 2.9790, 1.0911) -- (0.8140, 2.9790, 1.0955) -- (0.8140, 3.0300, 1.0892) -- (0.7650, 3.0300, 1.0849) -- cycle;
\fill[blue!15.3, opacity=0.7] (0.8140, -0.0300, 1.0892) -- (0.8630, -0.0300, 1.0933) -- (0.8630, 0.0210, 1.0995) -- (0.8140, 0.0210, 1.0955) -- cycle;
\fill[blue!15.1, opacity=0.7] (0.8140, 0.0210, 1.0955) -- (0.8630, 0.0210, 1.0995) -- (0.8630, 0.0720, 1.1058) -- (0.8140, 0.0720, 1.1017) -- cycle;
\fill[blue!15.0, opacity=0.7] (0.8140, 0.0720, 1.1017) -- (0.8630, 0.0720, 1.1058) -- (0.8630, 0.1230, 1.1120) -- (0.8140, 0.1230, 1.1079) -- cycle;
\fill[blue!15.0, opacity=0.7] (0.8140, 0.1230, 1.1079) -- (0.8630, 0.1230, 1.1120) -- (0.8630, 0.1740, 1.1182) -- (0.8140, 0.1740, 1.1141) -- cycle;
\fill[blue!15.0, opacity=0.7] (0.8140, 0.1740, 1.1141) -- (0.8630, 0.1740, 1.1182) -- (0.8630, 0.2250, 1.1243) -- (0.8140, 0.2250, 1.1202) -- cycle;
\fill[blue!15.0, opacity=0.7] (0.8140, 0.2250, 1.1202) -- (0.8630, 0.2250, 1.1243) -- (0.8630, 0.2760, 1.1303) -- (0.8140, 0.2760, 1.1263) -- cycle;
\fill[blue!15.0, opacity=0.7] (0.8140, 0.2760, 1.1263) -- (0.8630, 0.2760, 1.1303) -- (0.8630, 0.3270, 1.1363) -- (0.8140, 0.3270, 1.1322) -- cycle;
\fill[blue!15.1, opacity=0.7] (0.8140, 0.3270, 1.1322) -- (0.8630, 0.3270, 1.1363) -- (0.8630, 0.3780, 1.1421) -- (0.8140, 0.3780, 1.1380) -- cycle;
\fill[blue!22.4, opacity=0.7] (0.8140, 0.3780, 1.1380) -- (0.8630, 0.3780, 1.1421) -- (0.8630, 0.4290, 1.1477) -- (0.8140, 0.4290, 1.1437) -- cycle;
\fill[blue!39.0, opacity=0.7] (0.8140, 0.4290, 1.1437) -- (0.8630, 0.4290, 1.1477) -- (0.8630, 0.4800, 1.1533) -- (0.8140, 0.4800, 1.1492) -- cycle;
\fill[blue!28.2, opacity=0.7] (0.8140, 0.4800, 1.1492) -- (0.8630, 0.4800, 1.1533) -- (0.8630, 0.5310, 1.1586) -- (0.8140, 0.5310, 1.1545) -- cycle;
\fill[blue!16.1, opacity=0.7] (0.8140, 0.5310, 1.1545) -- (0.8630, 0.5310, 1.1586) -- (0.8630, 0.5820, 1.1638) -- (0.8140, 0.5820, 1.1597) -- cycle;
\fill[blue!15.0, opacity=0.7] (0.8140, 0.5820, 1.1597) -- (0.8630, 0.5820, 1.1638) -- (0.8630, 0.6330, 1.1688) -- (0.8140, 0.6330, 1.1647) -- cycle;
\fill[blue!15.0, opacity=0.7] (0.8140, 0.6330, 1.1647) -- (0.8630, 0.6330, 1.1688) -- (0.8630, 0.6840, 1.1736) -- (0.8140, 0.6840, 1.1695) -- cycle;
\fill[blue!15.0, opacity=0.7] (0.8140, 0.6840, 1.1695) -- (0.8630, 0.6840, 1.1736) -- (0.8630, 0.7350, 1.1781) -- (0.8140, 0.7350, 1.1740) -- cycle;
\fill[blue!15.0, opacity=0.7] (0.8140, 0.7350, 1.1740) -- (0.8630, 0.7350, 1.1781) -- (0.8630, 0.7860, 1.1824) -- (0.8140, 0.7860, 1.1784) -- cycle;
\fill[blue!16.0, opacity=0.7] (0.8140, 0.7860, 1.1784) -- (0.8630, 0.7860, 1.1824) -- (0.8630, 0.8370, 1.1865) -- (0.8140, 0.8370, 1.1824) -- cycle;
\fill[blue!40.6, opacity=0.7] (0.8140, 0.8370, 1.1824) -- (0.8630, 0.8370, 1.1865) -- (0.8630, 0.8880, 1.1903) -- (0.8140, 0.8880, 1.1863) -- cycle;
\fill[blue!97.7, opacity=0.7] (0.8140, 0.8880, 1.1863) -- (0.8630, 0.8880, 1.1903) -- (0.8630, 0.9390, 1.1939) -- (0.8140, 0.9390, 1.1898) -- cycle;
\fill[blue!69.4!black, opacity=0.7] (0.8140, 0.9390, 1.1898) -- (0.8630, 0.9390, 1.1939) -- (0.8630, 0.9900, 1.1972) -- (0.8140, 0.9900, 1.1931) -- cycle;
\fill[blue!76.6!black, opacity=0.7] (0.8140, 0.9900, 1.1931) -- (0.8630, 0.9900, 1.1972) -- (0.8630, 1.0410, 1.2002) -- (0.8140, 1.0410, 1.1961) -- cycle;
\fill[blue!79.1, opacity=0.7] (0.8140, 1.0410, 1.1961) -- (0.8630, 1.0410, 1.2002) -- (0.8630, 1.0920, 1.2029) -- (0.8140, 1.0920, 1.1988) -- cycle;
\fill[blue!34.3, opacity=0.7] (0.8140, 1.0920, 1.1988) -- (0.8630, 1.0920, 1.2029) -- (0.8630, 1.1430, 1.2053) -- (0.8140, 1.1430, 1.2012) -- cycle;
\fill[blue!17.9, opacity=0.7] (0.8140, 1.1430, 1.2012) -- (0.8630, 1.1430, 1.2053) -- (0.8630, 1.1940, 1.2074) -- (0.8140, 1.1940, 1.2033) -- cycle;
\fill[blue!15.4, opacity=0.7] (0.8140, 1.1940, 1.2033) -- (0.8630, 1.1940, 1.2074) -- (0.8630, 1.2450, 1.2092) -- (0.8140, 1.2450, 1.2051) -- cycle;
\fill[blue!15.1, opacity=0.7] (0.8140, 1.2450, 1.2051) -- (0.8630, 1.2450, 1.2092) -- (0.8630, 1.2960, 1.2106) -- (0.8140, 1.2960, 1.2066) -- cycle;
\fill[blue!15.1, opacity=0.7] (0.8140, 1.2960, 1.2066) -- (0.8630, 1.2960, 1.2106) -- (0.8630, 1.3470, 1.2118) -- (0.8140, 1.3470, 1.2077) -- cycle;
\fill[blue!15.2, opacity=0.7] (0.8140, 1.3470, 1.2077) -- (0.8630, 1.3470, 1.2118) -- (0.8630, 1.3980, 1.2126) -- (0.8140, 1.3980, 1.2085) -- cycle;
\fill[blue!15.5, opacity=0.7] (0.8140, 1.3980, 1.2085) -- (0.8630, 1.3980, 1.2126) -- (0.8630, 1.4490, 1.2131) -- (0.8140, 1.4490, 1.2090) -- cycle;
\fill[blue!16.3, opacity=0.7] (0.8140, 1.4490, 1.2090) -- (0.8630, 1.4490, 1.2131) -- (0.8630, 1.5000, 1.2133) -- (0.8140, 1.5000, 1.2092) -- cycle;
\fill[blue!17.5, opacity=0.7] (0.8140, 1.5000, 1.2092) -- (0.8630, 1.5000, 1.2133) -- (0.8630, 1.5510, 1.2131) -- (0.8140, 1.5510, 1.2090) -- cycle;
\fill[blue!18.7, opacity=0.7] (0.8140, 1.5510, 1.2090) -- (0.8630, 1.5510, 1.2131) -- (0.8630, 1.6020, 1.2126) -- (0.8140, 1.6020, 1.2085) -- cycle;
\fill[blue!19.0, opacity=0.7] (0.8140, 1.6020, 1.2085) -- (0.8630, 1.6020, 1.2126) -- (0.8630, 1.6530, 1.2118) -- (0.8140, 1.6530, 1.2077) -- cycle;
\fill[blue!18.2, opacity=0.7] (0.8140, 1.6530, 1.2077) -- (0.8630, 1.6530, 1.2118) -- (0.8630, 1.7040, 1.2106) -- (0.8140, 1.7040, 1.2066) -- cycle;
\fill[blue!16.9, opacity=0.7] (0.8140, 1.7040, 1.2066) -- (0.8630, 1.7040, 1.2106) -- (0.8630, 1.7550, 1.2092) -- (0.8140, 1.7550, 1.2051) -- cycle;
\fill[blue!15.8, opacity=0.7] (0.8140, 1.7550, 1.2051) -- (0.8630, 1.7550, 1.2092) -- (0.8630, 1.8060, 1.2074) -- (0.8140, 1.8060, 1.2033) -- cycle;
\fill[blue!15.3, opacity=0.7] (0.8140, 1.8060, 1.2033) -- (0.8630, 1.8060, 1.2074) -- (0.8630, 1.8570, 1.2053) -- (0.8140, 1.8570, 1.2012) -- cycle;
\fill[blue!15.1, opacity=0.7] (0.8140, 1.8570, 1.2012) -- (0.8630, 1.8570, 1.2053) -- (0.8630, 1.9080, 1.2029) -- (0.8140, 1.9080, 1.1988) -- cycle;
\fill[blue!15.0, opacity=0.7] (0.8140, 1.9080, 1.1988) -- (0.8630, 1.9080, 1.2029) -- (0.8630, 1.9590, 1.2002) -- (0.8140, 1.9590, 1.1961) -- cycle;
\fill[blue!15.0, opacity=0.7] (0.8140, 1.9590, 1.1961) -- (0.8630, 1.9590, 1.2002) -- (0.8630, 2.0100, 1.1972) -- (0.8140, 2.0100, 1.1931) -- cycle;
\fill[blue!15.0, opacity=0.7] (0.8140, 2.0100, 1.1931) -- (0.8630, 2.0100, 1.1972) -- (0.8630, 2.0610, 1.1939) -- (0.8140, 2.0610, 1.1898) -- cycle;
\fill[blue!15.1, opacity=0.7] (0.8140, 2.0610, 1.1898) -- (0.8630, 2.0610, 1.1939) -- (0.8630, 2.1120, 1.1903) -- (0.8140, 2.1120, 1.1863) -- cycle;
\fill[blue!16.0, opacity=0.7] (0.8140, 2.1120, 1.1863) -- (0.8630, 2.1120, 1.1903) -- (0.8630, 2.1630, 1.1865) -- (0.8140, 2.1630, 1.1824) -- cycle;
\fill[blue!22.7, opacity=0.7] (0.8140, 2.1630, 1.1824) -- (0.8630, 2.1630, 1.1865) -- (0.8630, 2.2140, 1.1824) -- (0.8140, 2.2140, 1.1784) -- cycle;
\fill[blue!46.3, opacity=0.7] (0.8140, 2.2140, 1.1784) -- (0.8630, 2.2140, 1.1824) -- (0.8630, 2.2650, 1.1781) -- (0.8140, 2.2650, 1.1740) -- cycle;
\fill[blue!72.9, opacity=0.7] (0.8140, 2.2650, 1.1740) -- (0.8630, 2.2650, 1.1781) -- (0.8630, 2.3160, 1.1736) -- (0.8140, 2.3160, 1.1695) -- cycle;
\fill[blue!74.3, opacity=0.7] (0.8140, 2.3160, 1.1695) -- (0.8630, 2.3160, 1.1736) -- (0.8630, 2.3670, 1.1688) -- (0.8140, 2.3670, 1.1647) -- cycle;
\fill[blue!44.8, opacity=0.7] (0.8140, 2.3670, 1.1647) -- (0.8630, 2.3670, 1.1688) -- (0.8630, 2.4180, 1.1638) -- (0.8140, 2.4180, 1.1597) -- cycle;
\fill[blue!18.4, opacity=0.7] (0.8140, 2.4180, 1.1597) -- (0.8630, 2.4180, 1.1638) -- (0.8630, 2.4690, 1.1586) -- (0.8140, 2.4690, 1.1545) -- cycle;
\fill[blue!15.0, opacity=0.7] (0.8140, 2.4690, 1.1545) -- (0.8630, 2.4690, 1.1586) -- (0.8630, 2.5200, 1.1533) -- (0.8140, 2.5200, 1.1492) -- cycle;
\fill[blue!15.0, opacity=0.7] (0.8140, 2.5200, 1.1492) -- (0.8630, 2.5200, 1.1533) -- (0.8630, 2.5710, 1.1477) -- (0.8140, 2.5710, 1.1437) -- cycle;
\fill[blue!15.0, opacity=0.7] (0.8140, 2.5710, 1.1437) -- (0.8630, 2.5710, 1.1477) -- (0.8630, 2.6220, 1.1421) -- (0.8140, 2.6220, 1.1380) -- cycle;
\fill[blue!15.0, opacity=0.7] (0.8140, 2.6220, 1.1380) -- (0.8630, 2.6220, 1.1421) -- (0.8630, 2.6730, 1.1363) -- (0.8140, 2.6730, 1.1322) -- cycle;
\fill[blue!15.0, opacity=0.7] (0.8140, 2.6730, 1.1322) -- (0.8630, 2.6730, 1.1363) -- (0.8630, 2.7240, 1.1303) -- (0.8140, 2.7240, 1.1263) -- cycle;
\fill[blue!15.0, opacity=0.7] (0.8140, 2.7240, 1.1263) -- (0.8630, 2.7240, 1.1303) -- (0.8630, 2.7750, 1.1243) -- (0.8140, 2.7750, 1.1202) -- cycle;
\fill[blue!15.2, opacity=0.7] (0.8140, 2.7750, 1.1202) -- (0.8630, 2.7750, 1.1243) -- (0.8630, 2.8260, 1.1182) -- (0.8140, 2.8260, 1.1141) -- cycle;
\fill[blue!16.5, opacity=0.7] (0.8140, 2.8260, 1.1141) -- (0.8630, 2.8260, 1.1182) -- (0.8630, 2.8770, 1.1120) -- (0.8140, 2.8770, 1.1079) -- cycle;
\fill[blue!15.9, opacity=0.7] (0.8140, 2.8770, 1.1079) -- (0.8630, 2.8770, 1.1120) -- (0.8630, 2.9280, 1.1058) -- (0.8140, 2.9280, 1.1017) -- cycle;
\fill[blue!15.0, opacity=0.7] (0.8140, 2.9280, 1.1017) -- (0.8630, 2.9280, 1.1058) -- (0.8630, 2.9790, 1.0995) -- (0.8140, 2.9790, 1.0955) -- cycle;
\fill[blue!15.0, opacity=0.7] (0.8140, 2.9790, 1.0955) -- (0.8630, 2.9790, 1.0995) -- (0.8630, 3.0300, 1.0933) -- (0.8140, 3.0300, 1.0892) -- cycle;
\fill[blue!15.3, opacity=0.7] (0.8630, -0.0300, 1.0933) -- (0.9120, -0.0300, 1.0971) -- (0.9120, 0.0210, 1.1034) -- (0.8630, 0.0210, 1.0995) -- cycle;
\fill[blue!15.0, opacity=0.7] (0.8630, 0.0210, 1.0995) -- (0.9120, 0.0210, 1.1034) -- (0.9120, 0.0720, 1.1096) -- (0.8630, 0.0720, 1.1058) -- cycle;
\fill[blue!15.0, opacity=0.7] (0.8630, 0.0720, 1.1058) -- (0.9120, 0.0720, 1.1096) -- (0.9120, 0.1230, 1.1159) -- (0.8630, 0.1230, 1.1120) -- cycle;
\fill[blue!15.0, opacity=0.7] (0.8630, 0.1230, 1.1120) -- (0.9120, 0.1230, 1.1159) -- (0.9120, 0.1740, 1.1220) -- (0.8630, 0.1740, 1.1182) -- cycle;
\fill[blue!15.0, opacity=0.7] (0.8630, 0.1740, 1.1182) -- (0.9120, 0.1740, 1.1220) -- (0.9120, 0.2250, 1.1281) -- (0.8630, 0.2250, 1.1243) -- cycle;
\fill[blue!15.0, opacity=0.7] (0.8630, 0.2250, 1.1243) -- (0.9120, 0.2250, 1.1281) -- (0.9120, 0.2760, 1.1342) -- (0.8630, 0.2760, 1.1303) -- cycle;
\fill[blue!15.0, opacity=0.7] (0.8630, 0.2760, 1.1303) -- (0.9120, 0.2760, 1.1342) -- (0.9120, 0.3270, 1.1401) -- (0.8630, 0.3270, 1.1363) -- cycle;
\fill[blue!16.4, opacity=0.7] (0.8630, 0.3270, 1.1363) -- (0.9120, 0.3270, 1.1401) -- (0.9120, 0.3780, 1.1459) -- (0.8630, 0.3780, 1.1421) -- cycle;
\fill[blue!32.6, opacity=0.7] (0.8630, 0.3780, 1.1421) -- (0.9120, 0.3780, 1.1459) -- (0.9120, 0.4290, 1.1516) -- (0.8630, 0.4290, 1.1477) -- cycle;
\fill[blue!38.0, opacity=0.7] (0.8630, 0.4290, 1.1477) -- (0.9120, 0.4290, 1.1516) -- (0.9120, 0.4800, 1.1571) -- (0.8630, 0.4800, 1.1533) -- cycle;
\fill[blue!20.0, opacity=0.7] (0.8630, 0.4800, 1.1533) -- (0.9120, 0.4800, 1.1571) -- (0.9120, 0.5310, 1.1624) -- (0.8630, 0.5310, 1.1586) -- cycle;
\fill[blue!15.1, opacity=0.7] (0.8630, 0.5310, 1.1586) -- (0.9120, 0.5310, 1.1624) -- (0.9120, 0.5820, 1.1676) -- (0.8630, 0.5820, 1.1638) -- cycle;
\fill[blue!15.0, opacity=0.7] (0.8630, 0.5820, 1.1638) -- (0.9120, 0.5820, 1.1676) -- (0.9120, 0.6330, 1.1726) -- (0.8630, 0.6330, 1.1688) -- cycle;
\fill[blue!15.0, opacity=0.7] (0.8630, 0.6330, 1.1688) -- (0.9120, 0.6330, 1.1726) -- (0.9120, 0.6840, 1.1774) -- (0.8630, 0.6840, 1.1736) -- cycle;
\fill[blue!15.0, opacity=0.7] (0.8630, 0.6840, 1.1736) -- (0.9120, 0.6840, 1.1774) -- (0.9120, 0.7350, 1.1819) -- (0.8630, 0.7350, 1.1781) -- cycle;
\fill[blue!15.3, opacity=0.7] (0.8630, 0.7350, 1.1781) -- (0.9120, 0.7350, 1.1819) -- (0.9120, 0.7860, 1.1863) -- (0.8630, 0.7860, 1.1824) -- cycle;
\fill[blue!29.8, opacity=0.7] (0.8630, 0.7860, 1.1824) -- (0.9120, 0.7860, 1.1863) -- (0.9120, 0.8370, 1.1903) -- (0.8630, 0.8370, 1.1865) -- cycle;
\fill[blue!90.9, opacity=0.7] (0.8630, 0.8370, 1.1865) -- (0.9120, 0.8370, 1.1903) -- (0.9120, 0.8880, 1.1942) -- (0.8630, 0.8880, 1.1903) -- cycle;
\fill[blue!69.1!black, opacity=0.7] (0.8630, 0.8880, 1.1903) -- (0.9120, 0.8880, 1.1942) -- (0.9120, 0.9390, 1.1977) -- (0.8630, 0.9390, 1.1939) -- cycle;
\fill[blue!72.6!black, opacity=0.7] (0.8630, 0.9390, 1.1939) -- (0.9120, 0.9390, 1.1977) -- (0.9120, 0.9900, 1.2010) -- (0.8630, 0.9900, 1.1972) -- cycle;
\fill[blue!80.8, opacity=0.7] (0.8630, 0.9900, 1.1972) -- (0.9120, 0.9900, 1.2010) -- (0.9120, 1.0410, 1.2040) -- (0.8630, 1.0410, 1.2002) -- cycle;
\fill[blue!32.0, opacity=0.7] (0.8630, 1.0410, 1.2002) -- (0.9120, 1.0410, 1.2040) -- (0.9120, 1.0920, 1.2067) -- (0.8630, 1.0920, 1.2029) -- cycle;
\fill[blue!17.0, opacity=0.7] (0.8630, 1.0920, 1.2029) -- (0.9120, 1.0920, 1.2067) -- (0.9120, 1.1430, 1.2091) -- (0.8630, 1.1430, 1.2053) -- cycle;
\fill[blue!15.3, opacity=0.7] (0.8630, 1.1430, 1.2053) -- (0.9120, 1.1430, 1.2091) -- (0.9120, 1.1940, 1.2112) -- (0.8630, 1.1940, 1.2074) -- cycle;
\fill[blue!15.2, opacity=0.7] (0.8630, 1.1940, 1.2074) -- (0.9120, 1.1940, 1.2112) -- (0.9120, 1.2450, 1.2130) -- (0.8630, 1.2450, 1.2092) -- cycle;
\fill[blue!15.3, opacity=0.7] (0.8630, 1.2450, 1.2092) -- (0.9120, 1.2450, 1.2130) -- (0.9120, 1.2960, 1.2145) -- (0.8630, 1.2960, 1.2106) -- cycle;
\fill[blue!16.3, opacity=0.7] (0.8630, 1.2960, 1.2106) -- (0.9120, 1.2960, 1.2145) -- (0.9120, 1.3470, 1.2156) -- (0.8630, 1.3470, 1.2118) -- cycle;
\fill[blue!20.9, opacity=0.7] (0.8630, 1.3470, 1.2118) -- (0.9120, 1.3470, 1.2156) -- (0.9120, 1.3980, 1.2164) -- (0.8630, 1.3980, 1.2126) -- cycle;
\fill[blue!34.4, opacity=0.7] (0.8630, 1.3980, 1.2126) -- (0.9120, 1.3980, 1.2164) -- (0.9120, 1.4490, 1.2169) -- (0.8630, 1.4490, 1.2131) -- cycle;
\fill[blue!55.7, opacity=0.7] (0.8630, 1.4490, 1.2131) -- (0.9120, 1.4490, 1.2169) -- (0.9120, 1.5000, 1.2171) -- (0.8630, 1.5000, 1.2133) -- cycle;
\fill[blue!75.0, opacity=0.7] (0.8630, 1.5000, 1.2133) -- (0.9120, 1.5000, 1.2171) -- (0.9120, 1.5510, 1.2169) -- (0.8630, 1.5510, 1.2131) -- cycle;
\fill[blue!86.1, opacity=0.7] (0.8630, 1.5510, 1.2131) -- (0.9120, 1.5510, 1.2169) -- (0.9120, 1.6020, 1.2164) -- (0.8630, 1.6020, 1.2126) -- cycle;
\fill[blue!88.5, opacity=0.7] (0.8630, 1.6020, 1.2126) -- (0.9120, 1.6020, 1.2164) -- (0.9120, 1.6530, 1.2156) -- (0.8630, 1.6530, 1.2118) -- cycle;
\fill[blue!83.0, opacity=0.7] (0.8630, 1.6530, 1.2118) -- (0.9120, 1.6530, 1.2156) -- (0.9120, 1.7040, 1.2145) -- (0.8630, 1.7040, 1.2106) -- cycle;
\fill[blue!69.0, opacity=0.7] (0.8630, 1.7040, 1.2106) -- (0.9120, 1.7040, 1.2145) -- (0.9120, 1.7550, 1.2130) -- (0.8630, 1.7550, 1.2092) -- cycle;
\fill[blue!48.5, opacity=0.7] (0.8630, 1.7550, 1.2092) -- (0.9120, 1.7550, 1.2130) -- (0.9120, 1.8060, 1.2112) -- (0.8630, 1.8060, 1.2074) -- cycle;
\fill[blue!29.2, opacity=0.7] (0.8630, 1.8060, 1.2074) -- (0.9120, 1.8060, 1.2112) -- (0.9120, 1.8570, 1.2091) -- (0.8630, 1.8570, 1.2053) -- cycle;
\fill[blue!18.7, opacity=0.7] (0.8630, 1.8570, 1.2053) -- (0.9120, 1.8570, 1.2091) -- (0.9120, 1.9080, 1.2067) -- (0.8630, 1.9080, 1.2029) -- cycle;
\fill[blue!15.6, opacity=0.7] (0.8630, 1.9080, 1.2029) -- (0.9120, 1.9080, 1.2067) -- (0.9120, 1.9590, 1.2040) -- (0.8630, 1.9590, 1.2002) -- cycle;
\fill[blue!15.1, opacity=0.7] (0.8630, 1.9590, 1.2002) -- (0.9120, 1.9590, 1.2040) -- (0.9120, 2.0100, 1.2010) -- (0.8630, 2.0100, 1.1972) -- cycle;
\fill[blue!15.0, opacity=0.7] (0.8630, 2.0100, 1.1972) -- (0.9120, 2.0100, 1.2010) -- (0.9120, 2.0610, 1.1977) -- (0.8630, 2.0610, 1.1939) -- cycle;
\fill[blue!15.0, opacity=0.7] (0.8630, 2.0610, 1.1939) -- (0.9120, 2.0610, 1.1977) -- (0.9120, 2.1120, 1.1942) -- (0.8630, 2.1120, 1.1903) -- cycle;
\fill[blue!15.0, opacity=0.7] (0.8630, 2.1120, 1.1903) -- (0.9120, 2.1120, 1.1942) -- (0.9120, 2.1630, 1.1903) -- (0.8630, 2.1630, 1.1865) -- cycle;
\fill[blue!15.4, opacity=0.7] (0.8630, 2.1630, 1.1865) -- (0.9120, 2.1630, 1.1903) -- (0.9120, 2.2140, 1.1863) -- (0.8630, 2.2140, 1.1824) -- cycle;
\fill[blue!19.4, opacity=0.7] (0.8630, 2.2140, 1.1824) -- (0.9120, 2.2140, 1.1863) -- (0.9120, 2.2650, 1.1819) -- (0.8630, 2.2650, 1.1781) -- cycle;
\fill[blue!40.3, opacity=0.7] (0.8630, 2.2650, 1.1781) -- (0.9120, 2.2650, 1.1819) -- (0.9120, 2.3160, 1.1774) -- (0.8630, 2.3160, 1.1736) -- cycle;
\fill[blue!69.6, opacity=0.7] (0.8630, 2.3160, 1.1736) -- (0.9120, 2.3160, 1.1774) -- (0.9120, 2.3670, 1.1726) -- (0.8630, 2.3670, 1.1688) -- cycle;
\fill[blue!71.8, opacity=0.7] (0.8630, 2.3670, 1.1688) -- (0.9120, 2.3670, 1.1726) -- (0.9120, 2.4180, 1.1676) -- (0.8630, 2.4180, 1.1638) -- cycle;
\fill[blue!40.3, opacity=0.7] (0.8630, 2.4180, 1.1638) -- (0.9120, 2.4180, 1.1676) -- (0.9120, 2.4690, 1.1624) -- (0.8630, 2.4690, 1.1586) -- cycle;
\fill[blue!16.9, opacity=0.7] (0.8630, 2.4690, 1.1586) -- (0.9120, 2.4690, 1.1624) -- (0.9120, 2.5200, 1.1571) -- (0.8630, 2.5200, 1.1533) -- cycle;
\fill[blue!15.0, opacity=0.7] (0.8630, 2.5200, 1.1533) -- (0.9120, 2.5200, 1.1571) -- (0.9120, 2.5710, 1.1516) -- (0.8630, 2.5710, 1.1477) -- cycle;
\fill[blue!15.0, opacity=0.7] (0.8630, 2.5710, 1.1477) -- (0.9120, 2.5710, 1.1516) -- (0.9120, 2.6220, 1.1459) -- (0.8630, 2.6220, 1.1421) -- cycle;
\fill[blue!15.0, opacity=0.7] (0.8630, 2.6220, 1.1421) -- (0.9120, 2.6220, 1.1459) -- (0.9120, 2.6730, 1.1401) -- (0.8630, 2.6730, 1.1363) -- cycle;
\fill[blue!15.0, opacity=0.7] (0.8630, 2.6730, 1.1363) -- (0.9120, 2.6730, 1.1401) -- (0.9120, 2.7240, 1.1342) -- (0.8630, 2.7240, 1.1303) -- cycle;
\fill[blue!15.0, opacity=0.7] (0.8630, 2.7240, 1.1303) -- (0.9120, 2.7240, 1.1342) -- (0.9120, 2.7750, 1.1281) -- (0.8630, 2.7750, 1.1243) -- cycle;
\fill[blue!15.0, opacity=0.7] (0.8630, 2.7750, 1.1243) -- (0.9120, 2.7750, 1.1281) -- (0.9120, 2.8260, 1.1220) -- (0.8630, 2.8260, 1.1182) -- cycle;
\fill[blue!15.6, opacity=0.7] (0.8630, 2.8260, 1.1182) -- (0.9120, 2.8260, 1.1220) -- (0.9120, 2.8770, 1.1159) -- (0.8630, 2.8770, 1.1120) -- cycle;
\fill[blue!16.6, opacity=0.7] (0.8630, 2.8770, 1.1120) -- (0.9120, 2.8770, 1.1159) -- (0.9120, 2.9280, 1.1096) -- (0.8630, 2.9280, 1.1058) -- cycle;
\fill[blue!15.3, opacity=0.7] (0.8630, 2.9280, 1.1058) -- (0.9120, 2.9280, 1.1096) -- (0.9120, 2.9790, 1.1034) -- (0.8630, 2.9790, 1.0995) -- cycle;
\fill[blue!15.0, opacity=0.7] (0.8630, 2.9790, 1.0995) -- (0.9120, 2.9790, 1.1034) -- (0.9120, 3.0300, 1.0971) -- (0.8630, 3.0300, 1.0933) -- cycle;
\fill[blue!15.2, opacity=0.7] (0.9120, -0.0300, 1.0971) -- (0.9610, -0.0300, 1.1006) -- (0.9610, 0.0210, 1.1069) -- (0.9120, 0.0210, 1.1034) -- cycle;
\fill[blue!15.0, opacity=0.7] (0.9120, 0.0210, 1.1034) -- (0.9610, 0.0210, 1.1069) -- (0.9610, 0.0720, 1.1132) -- (0.9120, 0.0720, 1.1096) -- cycle;
\fill[blue!15.0, opacity=0.7] (0.9120, 0.0720, 1.1096) -- (0.9610, 0.0720, 1.1132) -- (0.9610, 0.1230, 1.1194) -- (0.9120, 0.1230, 1.1159) -- cycle;
\fill[blue!15.0, opacity=0.7] (0.9120, 0.1230, 1.1159) -- (0.9610, 0.1230, 1.1194) -- (0.9610, 0.1740, 1.1256) -- (0.9120, 0.1740, 1.1220) -- cycle;
\fill[blue!15.0, opacity=0.7] (0.9120, 0.1740, 1.1220) -- (0.9610, 0.1740, 1.1256) -- (0.9610, 0.2250, 1.1317) -- (0.9120, 0.2250, 1.1281) -- cycle;
\fill[blue!15.0, opacity=0.7] (0.9120, 0.2250, 1.1281) -- (0.9610, 0.2250, 1.1317) -- (0.9610, 0.2760, 1.1377) -- (0.9120, 0.2760, 1.1342) -- cycle;
\fill[blue!15.1, opacity=0.7] (0.9120, 0.2760, 1.1342) -- (0.9610, 0.2760, 1.1377) -- (0.9610, 0.3270, 1.1436) -- (0.9120, 0.3270, 1.1401) -- cycle;
\fill[blue!21.0, opacity=0.7] (0.9120, 0.3270, 1.1401) -- (0.9610, 0.3270, 1.1436) -- (0.9610, 0.3780, 1.1494) -- (0.9120, 0.3780, 1.1459) -- cycle;
\fill[blue!40.4, opacity=0.7] (0.9120, 0.3780, 1.1459) -- (0.9610, 0.3780, 1.1494) -- (0.9610, 0.4290, 1.1551) -- (0.9120, 0.4290, 1.1516) -- cycle;
\fill[blue!30.6, opacity=0.7] (0.9120, 0.4290, 1.1516) -- (0.9610, 0.4290, 1.1551) -- (0.9610, 0.4800, 1.1606) -- (0.9120, 0.4800, 1.1571) -- cycle;
\fill[blue!16.2, opacity=0.7] (0.9120, 0.4800, 1.1571) -- (0.9610, 0.4800, 1.1606) -- (0.9610, 0.5310, 1.1660) -- (0.9120, 0.5310, 1.1624) -- cycle;
\fill[blue!15.0, opacity=0.7] (0.9120, 0.5310, 1.1624) -- (0.9610, 0.5310, 1.1660) -- (0.9610, 0.5820, 1.1712) -- (0.9120, 0.5820, 1.1676) -- cycle;
\fill[blue!15.0, opacity=0.7] (0.9120, 0.5820, 1.1676) -- (0.9610, 0.5820, 1.1712) -- (0.9610, 0.6330, 1.1762) -- (0.9120, 0.6330, 1.1726) -- cycle;
\fill[blue!15.0, opacity=0.7] (0.9120, 0.6330, 1.1726) -- (0.9610, 0.6330, 1.1762) -- (0.9610, 0.6840, 1.1809) -- (0.9120, 0.6840, 1.1774) -- cycle;
\fill[blue!15.0, opacity=0.7] (0.9120, 0.6840, 1.1774) -- (0.9610, 0.6840, 1.1809) -- (0.9610, 0.7350, 1.1855) -- (0.9120, 0.7350, 1.1819) -- cycle;
\fill[blue!19.8, opacity=0.7] (0.9120, 0.7350, 1.1819) -- (0.9610, 0.7350, 1.1855) -- (0.9610, 0.7860, 1.1898) -- (0.9120, 0.7860, 1.1863) -- cycle;
\fill[blue!72.8, opacity=0.7] (0.9120, 0.7860, 1.1863) -- (0.9610, 0.7860, 1.1898) -- (0.9610, 0.8370, 1.1939) -- (0.9120, 0.8370, 1.1903) -- cycle;
\fill[blue!70.7!black, opacity=0.7] (0.9120, 0.8370, 1.1903) -- (0.9610, 0.8370, 1.1939) -- (0.9610, 0.8880, 1.1977) -- (0.9120, 0.8880, 1.1942) -- cycle;
\fill[blue!68.7!black, opacity=0.7] (0.9120, 0.8880, 1.1942) -- (0.9610, 0.8880, 1.1977) -- (0.9610, 0.9390, 1.2013) -- (0.9120, 0.9390, 1.1977) -- cycle;
\fill[blue!90.8, opacity=0.7] (0.9120, 0.9390, 1.1977) -- (0.9610, 0.9390, 1.2013) -- (0.9610, 0.9900, 1.2046) -- (0.9120, 0.9900, 1.2010) -- cycle;
\fill[blue!36.2, opacity=0.7] (0.9120, 0.9900, 1.2010) -- (0.9610, 0.9900, 1.2046) -- (0.9610, 1.0410, 1.2076) -- (0.9120, 1.0410, 1.2040) -- cycle;
\fill[blue!17.2, opacity=0.7] (0.9120, 1.0410, 1.2040) -- (0.9610, 1.0410, 1.2076) -- (0.9610, 1.0920, 1.2103) -- (0.9120, 1.0920, 1.2067) -- cycle;
\fill[blue!15.3, opacity=0.7] (0.9120, 1.0920, 1.2067) -- (0.9610, 1.0920, 1.2103) -- (0.9610, 1.1430, 1.2127) -- (0.9120, 1.1430, 1.2091) -- cycle;
\fill[blue!15.2, opacity=0.7] (0.9120, 1.1430, 1.2091) -- (0.9610, 1.1430, 1.2127) -- (0.9610, 1.1940, 1.2148) -- (0.9120, 1.1940, 1.2112) -- cycle;
\fill[blue!15.8, opacity=0.7] (0.9120, 1.1940, 1.2112) -- (0.9610, 1.1940, 1.2148) -- (0.9610, 1.2450, 1.2166) -- (0.9120, 1.2450, 1.2130) -- cycle;
\fill[blue!20.8, opacity=0.7] (0.9120, 1.2450, 1.2130) -- (0.9610, 1.2450, 1.2166) -- (0.9610, 1.2960, 1.2180) -- (0.9120, 1.2960, 1.2145) -- cycle;
\fill[blue!45.6, opacity=0.7] (0.9120, 1.2960, 1.2145) -- (0.9610, 1.2960, 1.2180) -- (0.9610, 1.3470, 1.2192) -- (0.9120, 1.3470, 1.2156) -- cycle;
\fill[blue!89.8, opacity=0.7] (0.9120, 1.3470, 1.2156) -- (0.9610, 1.3470, 1.2192) -- (0.9610, 1.3980, 1.2200) -- (0.9120, 1.3980, 1.2164) -- cycle;
\fill[blue!69.4!black, opacity=0.7] (0.9120, 1.3980, 1.2164) -- (0.9610, 1.3980, 1.2200) -- (0.9610, 1.4490, 1.2205) -- (0.9120, 1.4490, 1.2169) -- cycle;
\fill[blue!83.1!black, opacity=0.7] (0.9120, 1.4490, 1.2169) -- (0.9610, 1.4490, 1.2205) -- (0.9610, 1.5000, 1.2206) -- (0.9120, 1.5000, 1.2171) -- cycle;
\fill[blue!95.9, opacity=0.7] (0.9120, 1.5000, 1.2171) -- (0.9610, 1.5000, 1.2206) -- (0.9610, 1.5510, 1.2205) -- (0.9120, 1.5510, 1.2169) -- cycle;
\fill[blue!89.8, opacity=0.7] (0.9120, 1.5510, 1.2169) -- (0.9610, 1.5510, 1.2205) -- (0.9610, 1.6020, 1.2200) -- (0.9120, 1.6020, 1.2164) -- cycle;
\fill[blue!89.3, opacity=0.7] (0.9120, 1.6020, 1.2164) -- (0.9610, 1.6020, 1.2200) -- (0.9610, 1.6530, 1.2192) -- (0.9120, 1.6530, 1.2156) -- cycle;
\fill[blue!93.7, opacity=0.7] (0.9120, 1.6530, 1.2156) -- (0.9610, 1.6530, 1.2192) -- (0.9610, 1.7040, 1.2180) -- (0.9120, 1.7040, 1.2145) -- cycle;
\fill[blue!94.6!black, opacity=0.7] (0.9120, 1.7040, 1.2145) -- (0.9610, 1.7040, 1.2180) -- (0.9610, 1.7550, 1.2166) -- (0.9120, 1.7550, 1.2130) -- cycle;
\fill[blue!71.8!black, opacity=0.7] (0.9120, 1.7550, 1.2130) -- (0.9610, 1.7550, 1.2166) -- (0.9610, 1.8060, 1.2148) -- (0.9120, 1.8060, 1.2112) -- cycle;
\fill[blue!78.5!black, opacity=0.7] (0.9120, 1.8060, 1.2112) -- (0.9610, 1.8060, 1.2148) -- (0.9610, 1.8570, 1.2127) -- (0.9120, 1.8570, 1.2091) -- cycle;
\fill[blue!79.5, opacity=0.7] (0.9120, 1.8570, 1.2091) -- (0.9610, 1.8570, 1.2127) -- (0.9610, 1.9080, 1.2103) -- (0.9120, 1.9080, 1.2067) -- cycle;
\fill[blue!38.8, opacity=0.7] (0.9120, 1.9080, 1.2067) -- (0.9610, 1.9080, 1.2103) -- (0.9610, 1.9590, 1.2076) -- (0.9120, 1.9590, 1.2040) -- cycle;
\fill[blue!18.9, opacity=0.7] (0.9120, 1.9590, 1.2040) -- (0.9610, 1.9590, 1.2076) -- (0.9610, 2.0100, 1.2046) -- (0.9120, 2.0100, 1.2010) -- cycle;
\fill[blue!15.3, opacity=0.7] (0.9120, 2.0100, 1.2010) -- (0.9610, 2.0100, 1.2046) -- (0.9610, 2.0610, 1.2013) -- (0.9120, 2.0610, 1.1977) -- cycle;
\fill[blue!15.0, opacity=0.7] (0.9120, 2.0610, 1.1977) -- (0.9610, 2.0610, 1.2013) -- (0.9610, 2.1120, 1.1977) -- (0.9120, 2.1120, 1.1942) -- cycle;
\fill[blue!15.0, opacity=0.7] (0.9120, 2.1120, 1.1942) -- (0.9610, 2.1120, 1.1977) -- (0.9610, 2.1630, 1.1939) -- (0.9120, 2.1630, 1.1903) -- cycle;
\fill[blue!15.0, opacity=0.7] (0.9120, 2.1630, 1.1903) -- (0.9610, 2.1630, 1.1939) -- (0.9610, 2.2140, 1.1898) -- (0.9120, 2.2140, 1.1863) -- cycle;
\fill[blue!15.2, opacity=0.7] (0.9120, 2.2140, 1.1863) -- (0.9610, 2.2140, 1.1898) -- (0.9610, 2.2650, 1.1855) -- (0.9120, 2.2650, 1.1819) -- cycle;
\fill[blue!18.4, opacity=0.7] (0.9120, 2.2650, 1.1819) -- (0.9610, 2.2650, 1.1855) -- (0.9610, 2.3160, 1.1809) -- (0.9120, 2.3160, 1.1774) -- cycle;
\fill[blue!39.3, opacity=0.7] (0.9120, 2.3160, 1.1774) -- (0.9610, 2.3160, 1.1809) -- (0.9610, 2.3670, 1.1762) -- (0.9120, 2.3670, 1.1726) -- cycle;
\fill[blue!69.0, opacity=0.7] (0.9120, 2.3670, 1.1726) -- (0.9610, 2.3670, 1.1762) -- (0.9610, 2.4180, 1.1712) -- (0.9120, 2.4180, 1.1676) -- cycle;
\fill[blue!67.1, opacity=0.7] (0.9120, 2.4180, 1.1676) -- (0.9610, 2.4180, 1.1712) -- (0.9610, 2.4690, 1.1660) -- (0.9120, 2.4690, 1.1624) -- cycle;
\fill[blue!32.3, opacity=0.7] (0.9120, 2.4690, 1.1624) -- (0.9610, 2.4690, 1.1660) -- (0.9610, 2.5200, 1.1606) -- (0.9120, 2.5200, 1.1571) -- cycle;
\fill[blue!15.7, opacity=0.7] (0.9120, 2.5200, 1.1571) -- (0.9610, 2.5200, 1.1606) -- (0.9610, 2.5710, 1.1551) -- (0.9120, 2.5710, 1.1516) -- cycle;
\fill[blue!15.0, opacity=0.7] (0.9120, 2.5710, 1.1516) -- (0.9610, 2.5710, 1.1551) -- (0.9610, 2.6220, 1.1494) -- (0.9120, 2.6220, 1.1459) -- cycle;
\fill[blue!15.0, opacity=0.7] (0.9120, 2.6220, 1.1459) -- (0.9610, 2.6220, 1.1494) -- (0.9610, 2.6730, 1.1436) -- (0.9120, 2.6730, 1.1401) -- cycle;
\fill[blue!15.0, opacity=0.7] (0.9120, 2.6730, 1.1401) -- (0.9610, 2.6730, 1.1436) -- (0.9610, 2.7240, 1.1377) -- (0.9120, 2.7240, 1.1342) -- cycle;
\fill[blue!15.0, opacity=0.7] (0.9120, 2.7240, 1.1342) -- (0.9610, 2.7240, 1.1377) -- (0.9610, 2.7750, 1.1317) -- (0.9120, 2.7750, 1.1281) -- cycle;
\fill[blue!15.0, opacity=0.7] (0.9120, 2.7750, 1.1281) -- (0.9610, 2.7750, 1.1317) -- (0.9610, 2.8260, 1.1256) -- (0.9120, 2.8260, 1.1220) -- cycle;
\fill[blue!15.1, opacity=0.7] (0.9120, 2.8260, 1.1220) -- (0.9610, 2.8260, 1.1256) -- (0.9610, 2.8770, 1.1194) -- (0.9120, 2.8770, 1.1159) -- cycle;
\fill[blue!16.2, opacity=0.7] (0.9120, 2.8770, 1.1159) -- (0.9610, 2.8770, 1.1194) -- (0.9610, 2.9280, 1.1132) -- (0.9120, 2.9280, 1.1096) -- cycle;
\fill[blue!15.9, opacity=0.7] (0.9120, 2.9280, 1.1096) -- (0.9610, 2.9280, 1.1132) -- (0.9610, 2.9790, 1.1069) -- (0.9120, 2.9790, 1.1034) -- cycle;
\fill[blue!15.0, opacity=0.7] (0.9120, 2.9790, 1.1034) -- (0.9610, 2.9790, 1.1069) -- (0.9610, 3.0300, 1.1006) -- (0.9120, 3.0300, 1.0971) -- cycle;
\fill[blue!15.1, opacity=0.7] (0.9610, -0.0300, 1.1006) -- (1.0100, -0.0300, 1.1039) -- (1.0100, 0.0210, 1.1102) -- (0.9610, 0.0210, 1.1069) -- cycle;
\fill[blue!15.0, opacity=0.7] (0.9610, 0.0210, 1.1069) -- (1.0100, 0.0210, 1.1102) -- (1.0100, 0.0720, 1.1165) -- (0.9610, 0.0720, 1.1132) -- cycle;
\fill[blue!15.0, opacity=0.7] (0.9610, 0.0720, 1.1132) -- (1.0100, 0.0720, 1.1165) -- (1.0100, 0.1230, 1.1227) -- (0.9610, 0.1230, 1.1194) -- cycle;
\fill[blue!15.0, opacity=0.7] (0.9610, 0.1230, 1.1194) -- (1.0100, 0.1230, 1.1227) -- (1.0100, 0.1740, 1.1289) -- (0.9610, 0.1740, 1.1256) -- cycle;
\fill[blue!15.0, opacity=0.7] (0.9610, 0.1740, 1.1256) -- (1.0100, 0.1740, 1.1289) -- (1.0100, 0.2250, 1.1350) -- (0.9610, 0.2250, 1.1317) -- cycle;
\fill[blue!15.0, opacity=0.7] (0.9610, 0.2250, 1.1317) -- (1.0100, 0.2250, 1.1350) -- (1.0100, 0.2760, 1.1410) -- (0.9610, 0.2760, 1.1377) -- cycle;
\fill[blue!15.5, opacity=0.7] (0.9610, 0.2760, 1.1377) -- (1.0100, 0.2760, 1.1410) -- (1.0100, 0.3270, 1.1469) -- (0.9610, 0.3270, 1.1436) -- cycle;
\fill[blue!29.2, opacity=0.7] (0.9610, 0.3270, 1.1436) -- (1.0100, 0.3270, 1.1469) -- (1.0100, 0.3780, 1.1527) -- (0.9610, 0.3780, 1.1494) -- cycle;
\fill[blue!41.9, opacity=0.7] (0.9610, 0.3780, 1.1494) -- (1.0100, 0.3780, 1.1527) -- (1.0100, 0.4290, 1.1584) -- (0.9610, 0.4290, 1.1551) -- cycle;
\fill[blue!22.9, opacity=0.7] (0.9610, 0.4290, 1.1551) -- (1.0100, 0.4290, 1.1584) -- (1.0100, 0.4800, 1.1639) -- (0.9610, 0.4800, 1.1606) -- cycle;
\fill[blue!15.2, opacity=0.7] (0.9610, 0.4800, 1.1606) -- (1.0100, 0.4800, 1.1639) -- (1.0100, 0.5310, 1.1693) -- (0.9610, 0.5310, 1.1660) -- cycle;
\fill[blue!15.0, opacity=0.7] (0.9610, 0.5310, 1.1660) -- (1.0100, 0.5310, 1.1693) -- (1.0100, 0.5820, 1.1745) -- (0.9610, 0.5820, 1.1712) -- cycle;
\fill[blue!15.0, opacity=0.7] (0.9610, 0.5820, 1.1712) -- (1.0100, 0.5820, 1.1745) -- (1.0100, 0.6330, 1.1794) -- (0.9610, 0.6330, 1.1762) -- cycle;
\fill[blue!15.0, opacity=0.7] (0.9610, 0.6330, 1.1762) -- (1.0100, 0.6330, 1.1794) -- (1.0100, 0.6840, 1.1842) -- (0.9610, 0.6840, 1.1809) -- cycle;
\fill[blue!15.7, opacity=0.7] (0.9610, 0.6840, 1.1809) -- (1.0100, 0.6840, 1.1842) -- (1.0100, 0.7350, 1.1888) -- (0.9610, 0.7350, 1.1855) -- cycle;
\fill[blue!43.3, opacity=0.7] (0.9610, 0.7350, 1.1855) -- (1.0100, 0.7350, 1.1888) -- (1.0100, 0.7860, 1.1931) -- (0.9610, 0.7860, 1.1898) -- cycle;
\fill[blue!84.4!black, opacity=0.7] (0.9610, 0.7860, 1.1898) -- (1.0100, 0.7860, 1.1931) -- (1.0100, 0.8370, 1.1972) -- (0.9610, 0.8370, 1.1939) -- cycle;
\fill[blue!68.6!black, opacity=0.7] (0.9610, 0.8370, 1.1939) -- (1.0100, 0.8370, 1.1972) -- (1.0100, 0.8880, 1.2010) -- (0.9610, 0.8880, 1.1977) -- cycle;
\fill[blue!88.3!black, opacity=0.7] (0.9610, 0.8880, 1.1977) -- (1.0100, 0.8880, 1.2010) -- (1.0100, 0.9390, 1.2046) -- (0.9610, 0.9390, 1.2013) -- cycle;
\fill[blue!49.6, opacity=0.7] (0.9610, 0.9390, 1.2013) -- (1.0100, 0.9390, 1.2046) -- (1.0100, 0.9900, 1.2078) -- (0.9610, 0.9900, 1.2046) -- cycle;
\fill[blue!18.7, opacity=0.7] (0.9610, 0.9900, 1.2046) -- (1.0100, 0.9900, 1.2078) -- (1.0100, 1.0410, 1.2108) -- (0.9610, 1.0410, 1.2076) -- cycle;
\fill[blue!15.4, opacity=0.7] (0.9610, 1.0410, 1.2076) -- (1.0100, 1.0410, 1.2108) -- (1.0100, 1.0920, 1.2135) -- (0.9610, 1.0920, 1.2103) -- cycle;
\fill[blue!15.3, opacity=0.7] (0.9610, 1.0920, 1.2103) -- (1.0100, 1.0920, 1.2135) -- (1.0100, 1.1430, 1.2160) -- (0.9610, 1.1430, 1.2127) -- cycle;
\fill[blue!16.3, opacity=0.7] (0.9610, 1.1430, 1.2127) -- (1.0100, 1.1430, 1.2160) -- (1.0100, 1.1940, 1.2180) -- (0.9610, 1.1940, 1.2148) -- cycle;
\fill[blue!28.5, opacity=0.7] (0.9610, 1.1940, 1.2148) -- (1.0100, 1.1940, 1.2180) -- (1.0100, 1.2450, 1.2198) -- (0.9610, 1.2450, 1.2166) -- cycle;
\fill[blue!79.3, opacity=0.7] (0.9610, 1.2450, 1.2166) -- (1.0100, 1.2450, 1.2198) -- (1.0100, 1.2960, 1.2213) -- (0.9610, 1.2960, 1.2180) -- cycle;
\fill[blue!68.6!black, opacity=0.7] (0.9610, 1.2960, 1.2180) -- (1.0100, 1.2960, 1.2213) -- (1.0100, 1.3470, 1.2224) -- (0.9610, 1.3470, 1.2192) -- cycle;
\fill[blue!89.1, opacity=0.7] (0.9610, 1.3470, 1.2192) -- (1.0100, 1.3470, 1.2224) -- (1.0100, 1.3980, 1.2233) -- (0.9610, 1.3980, 1.2200) -- cycle;
\fill[blue!66.3, opacity=0.7] (0.9610, 1.3980, 1.2200) -- (1.0100, 1.3980, 1.2233) -- (1.0100, 1.4490, 1.2238) -- (0.9610, 1.4490, 1.2205) -- cycle;
\fill[blue!57.3, opacity=0.7] (0.9610, 1.4490, 1.2205) -- (1.0100, 1.4490, 1.2238) -- (1.0100, 1.5000, 1.2239) -- (0.9610, 1.5000, 1.2206) -- cycle;
\fill[blue!56.7, opacity=0.7] (0.9610, 1.5000, 1.2206) -- (1.0100, 1.5000, 1.2239) -- (1.0100, 1.5510, 1.2238) -- (0.9610, 1.5510, 1.2205) -- cycle;
\fill[blue!59.5, opacity=0.7] (0.9610, 1.5510, 1.2205) -- (1.0100, 1.5510, 1.2238) -- (1.0100, 1.6020, 1.2233) -- (0.9610, 1.6020, 1.2200) -- cycle;
\fill[blue!62.8, opacity=0.7] (0.9610, 1.6020, 1.2200) -- (1.0100, 1.6020, 1.2233) -- (1.0100, 1.6530, 1.2224) -- (0.9610, 1.6530, 1.2192) -- cycle;
\fill[blue!65.3, opacity=0.7] (0.9610, 1.6530, 1.2192) -- (1.0100, 1.6530, 1.2224) -- (1.0100, 1.7040, 1.2213) -- (0.9610, 1.7040, 1.2180) -- cycle;
\fill[blue!67.9, opacity=0.7] (0.9610, 1.7040, 1.2180) -- (1.0100, 1.7040, 1.2213) -- (1.0100, 1.7550, 1.2198) -- (0.9610, 1.7550, 1.2166) -- cycle;
\fill[blue!73.0, opacity=0.7] (0.9610, 1.7550, 1.2166) -- (1.0100, 1.7550, 1.2198) -- (1.0100, 1.8060, 1.2180) -- (0.9610, 1.8060, 1.2148) -- cycle;
\fill[blue!84.3, opacity=0.7] (0.9610, 1.8060, 1.2148) -- (1.0100, 1.8060, 1.2180) -- (1.0100, 1.8570, 1.2160) -- (0.9610, 1.8570, 1.2127) -- cycle;
\fill[blue!92.5!black, opacity=0.7] (0.9610, 1.8570, 1.2127) -- (1.0100, 1.8570, 1.2160) -- (1.0100, 1.9080, 1.2135) -- (0.9610, 1.9080, 1.2103) -- cycle;
\fill[blue!71.6!black, opacity=0.7] (0.9610, 1.9080, 1.2103) -- (1.0100, 1.9080, 1.2135) -- (1.0100, 1.9590, 1.2108) -- (0.9610, 1.9590, 1.2076) -- cycle;
\fill[blue!73.2, opacity=0.7] (0.9610, 1.9590, 1.2076) -- (1.0100, 1.9590, 1.2108) -- (1.0100, 2.0100, 1.2078) -- (0.9610, 2.0100, 1.2046) -- cycle;
\fill[blue!27.3, opacity=0.7] (0.9610, 2.0100, 1.2046) -- (1.0100, 2.0100, 1.2078) -- (1.0100, 2.0610, 1.2046) -- (0.9610, 2.0610, 1.2013) -- cycle;
\fill[blue!15.9, opacity=0.7] (0.9610, 2.0610, 1.2013) -- (1.0100, 2.0610, 1.2046) -- (1.0100, 2.1120, 1.2010) -- (0.9610, 2.1120, 1.1977) -- cycle;
\fill[blue!15.0, opacity=0.7] (0.9610, 2.1120, 1.1977) -- (1.0100, 2.1120, 1.2010) -- (1.0100, 2.1630, 1.1972) -- (0.9610, 2.1630, 1.1939) -- cycle;
\fill[blue!15.0, opacity=0.7] (0.9610, 2.1630, 1.1939) -- (1.0100, 2.1630, 1.1972) -- (1.0100, 2.2140, 1.1931) -- (0.9610, 2.2140, 1.1898) -- cycle;
\fill[blue!15.0, opacity=0.7] (0.9610, 2.2140, 1.1898) -- (1.0100, 2.2140, 1.1931) -- (1.0100, 2.2650, 1.1888) -- (0.9610, 2.2650, 1.1855) -- cycle;
\fill[blue!15.2, opacity=0.7] (0.9610, 2.2650, 1.1855) -- (1.0100, 2.2650, 1.1888) -- (1.0100, 2.3160, 1.1842) -- (0.9610, 2.3160, 1.1809) -- cycle;
\fill[blue!18.9, opacity=0.7] (0.9610, 2.3160, 1.1809) -- (1.0100, 2.3160, 1.1842) -- (1.0100, 2.3670, 1.1794) -- (0.9610, 2.3670, 1.1762) -- cycle;
\fill[blue!42.8, opacity=0.7] (0.9610, 2.3670, 1.1762) -- (1.0100, 2.3670, 1.1794) -- (1.0100, 2.4180, 1.1745) -- (0.9610, 2.4180, 1.1712) -- cycle;
\fill[blue!70.0, opacity=0.7] (0.9610, 2.4180, 1.1712) -- (1.0100, 2.4180, 1.1745) -- (1.0100, 2.4690, 1.1693) -- (0.9610, 2.4690, 1.1660) -- cycle;
\fill[blue!58.4, opacity=0.7] (0.9610, 2.4690, 1.1660) -- (1.0100, 2.4690, 1.1693) -- (1.0100, 2.5200, 1.1639) -- (0.9610, 2.5200, 1.1606) -- cycle;
\fill[blue!23.2, opacity=0.7] (0.9610, 2.5200, 1.1606) -- (1.0100, 2.5200, 1.1639) -- (1.0100, 2.5710, 1.1584) -- (0.9610, 2.5710, 1.1551) -- cycle;
\fill[blue!15.1, opacity=0.7] (0.9610, 2.5710, 1.1551) -- (1.0100, 2.5710, 1.1584) -- (1.0100, 2.6220, 1.1527) -- (0.9610, 2.6220, 1.1494) -- cycle;
\fill[blue!15.0, opacity=0.7] (0.9610, 2.6220, 1.1494) -- (1.0100, 2.6220, 1.1527) -- (1.0100, 2.6730, 1.1469) -- (0.9610, 2.6730, 1.1436) -- cycle;
\fill[blue!15.0, opacity=0.7] (0.9610, 2.6730, 1.1436) -- (1.0100, 2.6730, 1.1469) -- (1.0100, 2.7240, 1.1410) -- (0.9610, 2.7240, 1.1377) -- cycle;
\fill[blue!15.0, opacity=0.7] (0.9610, 2.7240, 1.1377) -- (1.0100, 2.7240, 1.1410) -- (1.0100, 2.7750, 1.1350) -- (0.9610, 2.7750, 1.1317) -- cycle;
\fill[blue!15.0, opacity=0.7] (0.9610, 2.7750, 1.1317) -- (1.0100, 2.7750, 1.1350) -- (1.0100, 2.8260, 1.1289) -- (0.9610, 2.8260, 1.1256) -- cycle;
\fill[blue!15.0, opacity=0.7] (0.9610, 2.8260, 1.1256) -- (1.0100, 2.8260, 1.1289) -- (1.0100, 2.8770, 1.1227) -- (0.9610, 2.8770, 1.1194) -- cycle;
\fill[blue!15.4, opacity=0.7] (0.9610, 2.8770, 1.1194) -- (1.0100, 2.8770, 1.1227) -- (1.0100, 2.9280, 1.1165) -- (0.9610, 2.9280, 1.1132) -- cycle;
\fill[blue!16.4, opacity=0.7] (0.9610, 2.9280, 1.1132) -- (1.0100, 2.9280, 1.1165) -- (1.0100, 2.9790, 1.1102) -- (0.9610, 2.9790, 1.1069) -- cycle;
\fill[blue!15.2, opacity=0.7] (0.9610, 2.9790, 1.1069) -- (1.0100, 2.9790, 1.1102) -- (1.0100, 3.0300, 1.1039) -- (0.9610, 3.0300, 1.1006) -- cycle;
\fill[blue!15.0, opacity=0.7] (1.0100, -0.0300, 1.1039) -- (1.0590, -0.0300, 1.1069) -- (1.0590, 0.0210, 1.1132) -- (1.0100, 0.0210, 1.1102) -- cycle;
\fill[blue!15.0, opacity=0.7] (1.0100, 0.0210, 1.1102) -- (1.0590, 0.0210, 1.1132) -- (1.0590, 0.0720, 1.1195) -- (1.0100, 0.0720, 1.1165) -- cycle;
\fill[blue!15.0, opacity=0.7] (1.0100, 0.0720, 1.1165) -- (1.0590, 0.0720, 1.1195) -- (1.0590, 0.1230, 1.1257) -- (1.0100, 0.1230, 1.1227) -- cycle;
\fill[blue!15.0, opacity=0.7] (1.0100, 0.1230, 1.1227) -- (1.0590, 0.1230, 1.1257) -- (1.0590, 0.1740, 1.1319) -- (1.0100, 0.1740, 1.1289) -- cycle;
\fill[blue!15.0, opacity=0.7] (1.0100, 0.1740, 1.1289) -- (1.0590, 0.1740, 1.1319) -- (1.0590, 0.2250, 1.1380) -- (1.0100, 0.2250, 1.1350) -- cycle;
\fill[blue!15.0, opacity=0.7] (1.0100, 0.2250, 1.1350) -- (1.0590, 0.2250, 1.1380) -- (1.0590, 0.2760, 1.1440) -- (1.0100, 0.2760, 1.1410) -- cycle;
\fill[blue!17.2, opacity=0.7] (1.0100, 0.2760, 1.1410) -- (1.0590, 0.2760, 1.1440) -- (1.0590, 0.3270, 1.1499) -- (1.0100, 0.3270, 1.1469) -- cycle;
\fill[blue!37.7, opacity=0.7] (1.0100, 0.3270, 1.1469) -- (1.0590, 0.3270, 1.1499) -- (1.0590, 0.3780, 1.1557) -- (1.0100, 0.3780, 1.1527) -- cycle;
\fill[blue!38.2, opacity=0.7] (1.0100, 0.3780, 1.1527) -- (1.0590, 0.3780, 1.1557) -- (1.0590, 0.4290, 1.1614) -- (1.0100, 0.4290, 1.1584) -- cycle;
\fill[blue!18.2, opacity=0.7] (1.0100, 0.4290, 1.1584) -- (1.0590, 0.4290, 1.1614) -- (1.0590, 0.4800, 1.1669) -- (1.0100, 0.4800, 1.1639) -- cycle;
\fill[blue!15.0, opacity=0.7] (1.0100, 0.4800, 1.1639) -- (1.0590, 0.4800, 1.1669) -- (1.0590, 0.5310, 1.1723) -- (1.0100, 0.5310, 1.1693) -- cycle;
\fill[blue!15.0, opacity=0.7] (1.0100, 0.5310, 1.1693) -- (1.0590, 0.5310, 1.1723) -- (1.0590, 0.5820, 1.1775) -- (1.0100, 0.5820, 1.1745) -- cycle;
\fill[blue!15.0, opacity=0.7] (1.0100, 0.5820, 1.1745) -- (1.0590, 0.5820, 1.1775) -- (1.0590, 0.6330, 1.1824) -- (1.0100, 0.6330, 1.1794) -- cycle;
\fill[blue!15.0, opacity=0.7] (1.0100, 0.6330, 1.1794) -- (1.0590, 0.6330, 1.1824) -- (1.0590, 0.6840, 1.1872) -- (1.0100, 0.6840, 1.1842) -- cycle;
\fill[blue!20.8, opacity=0.7] (1.0100, 0.6840, 1.1842) -- (1.0590, 0.6840, 1.1872) -- (1.0590, 0.7350, 1.1918) -- (1.0100, 0.7350, 1.1888) -- cycle;
\fill[blue!82.3, opacity=0.7] (1.0100, 0.7350, 1.1888) -- (1.0590, 0.7350, 1.1918) -- (1.0590, 0.7860, 1.1961) -- (1.0100, 0.7860, 1.1931) -- cycle;
\fill[blue!68.3!black, opacity=0.7] (1.0100, 0.7860, 1.1931) -- (1.0590, 0.7860, 1.1961) -- (1.0590, 0.8370, 1.2002) -- (1.0100, 0.8370, 1.1972) -- cycle;
\fill[blue!69.1!black, opacity=0.7] (1.0100, 0.8370, 1.1972) -- (1.0590, 0.8370, 1.2002) -- (1.0590, 0.8880, 1.2040) -- (1.0100, 0.8880, 1.2010) -- cycle;
\fill[blue!76.3, opacity=0.7] (1.0100, 0.8880, 1.2010) -- (1.0590, 0.8880, 1.2040) -- (1.0590, 0.9390, 1.2076) -- (1.0100, 0.9390, 1.2046) -- cycle;
\fill[blue!24.3, opacity=0.7] (1.0100, 0.9390, 1.2046) -- (1.0590, 0.9390, 1.2076) -- (1.0590, 0.9900, 1.2108) -- (1.0100, 0.9900, 1.2078) -- cycle;
\fill[blue!15.7, opacity=0.7] (1.0100, 0.9900, 1.2078) -- (1.0590, 0.9900, 1.2108) -- (1.0590, 1.0410, 1.2138) -- (1.0100, 1.0410, 1.2108) -- cycle;
\fill[blue!15.3, opacity=0.7] (1.0100, 1.0410, 1.2108) -- (1.0590, 1.0410, 1.2138) -- (1.0590, 1.0920, 1.2165) -- (1.0100, 1.0920, 1.2135) -- cycle;
\fill[blue!16.3, opacity=0.7] (1.0100, 1.0920, 1.2135) -- (1.0590, 1.0920, 1.2165) -- (1.0590, 1.1430, 1.2190) -- (1.0100, 1.1430, 1.2160) -- cycle;
\fill[blue!31.9, opacity=0.7] (1.0100, 1.1430, 1.2160) -- (1.0590, 1.1430, 1.2190) -- (1.0590, 1.1940, 1.2210) -- (1.0100, 1.1940, 1.2180) -- cycle;
\fill[blue!95.5, opacity=0.7] (1.0100, 1.1940, 1.2180) -- (1.0590, 1.1940, 1.2210) -- (1.0590, 1.2450, 1.2228) -- (1.0100, 1.2450, 1.2198) -- cycle;
\fill[blue!95.7!black, opacity=0.7] (1.0100, 1.2450, 1.2198) -- (1.0590, 1.2450, 1.2228) -- (1.0590, 1.2960, 1.2243) -- (1.0100, 1.2960, 1.2213) -- cycle;
\fill[blue!63.4, opacity=0.7] (1.0100, 1.2960, 1.2213) -- (1.0590, 1.2960, 1.2243) -- (1.0590, 1.3470, 1.2254) -- (1.0100, 1.3470, 1.2224) -- cycle;
\fill[blue!48.9, opacity=0.7] (1.0100, 1.3470, 1.2224) -- (1.0590, 1.3470, 1.2254) -- (1.0590, 1.3980, 1.2263) -- (1.0100, 1.3980, 1.2233) -- cycle;
\fill[blue!52.4, opacity=0.7] (1.0100, 1.3980, 1.2233) -- (1.0590, 1.3980, 1.2263) -- (1.0590, 1.4490, 1.2268) -- (1.0100, 1.4490, 1.2238) -- cycle;
\fill[blue!66.3, opacity=0.7] (1.0100, 1.4490, 1.2238) -- (1.0590, 1.4490, 1.2268) -- (1.0590, 1.5000, 1.2269) -- (1.0100, 1.5000, 1.2239) -- cycle;
\fill[blue!83.5, opacity=0.7] (1.0100, 1.5000, 1.2239) -- (1.0590, 1.5000, 1.2269) -- (1.0590, 1.5510, 1.2268) -- (1.0100, 1.5510, 1.2238) -- cycle;
\fill[blue!96.1, opacity=0.7] (1.0100, 1.5510, 1.2238) -- (1.0590, 1.5510, 1.2268) -- (1.0590, 1.6020, 1.2263) -- (1.0100, 1.6020, 1.2233) -- cycle;
\fill[blue!95.3!black, opacity=0.7] (1.0100, 1.6020, 1.2233) -- (1.0590, 1.6020, 1.2263) -- (1.0590, 1.6530, 1.2254) -- (1.0100, 1.6530, 1.2224) -- cycle;
\fill[blue!95.0!black, opacity=0.7] (1.0100, 1.6530, 1.2224) -- (1.0590, 1.6530, 1.2254) -- (1.0590, 1.7040, 1.2243) -- (1.0100, 1.7040, 1.2213) -- cycle;
\fill[blue!97.3, opacity=0.7] (1.0100, 1.7040, 1.2213) -- (1.0590, 1.7040, 1.2243) -- (1.0590, 1.7550, 1.2228) -- (1.0100, 1.7550, 1.2198) -- cycle;
\fill[blue!88.5, opacity=0.7] (1.0100, 1.7550, 1.2198) -- (1.0590, 1.7550, 1.2228) -- (1.0590, 1.8060, 1.2210) -- (1.0100, 1.8060, 1.2180) -- cycle;
\fill[blue!79.3, opacity=0.7] (1.0100, 1.8060, 1.2180) -- (1.0590, 1.8060, 1.2210) -- (1.0590, 1.8570, 1.2190) -- (1.0100, 1.8570, 1.2160) -- cycle;
\fill[blue!76.6, opacity=0.7] (1.0100, 1.8570, 1.2160) -- (1.0590, 1.8570, 1.2190) -- (1.0590, 1.9080, 1.2165) -- (1.0100, 1.9080, 1.2135) -- cycle;
\fill[blue!87.4, opacity=0.7] (1.0100, 1.9080, 1.2135) -- (1.0590, 1.9080, 1.2165) -- (1.0590, 1.9590, 1.2138) -- (1.0100, 1.9590, 1.2108) -- cycle;
\fill[blue!76.2!black, opacity=0.7] (1.0100, 1.9590, 1.2108) -- (1.0590, 1.9590, 1.2138) -- (1.0590, 2.0100, 1.2108) -- (1.0100, 2.0100, 1.2078) -- cycle;
\fill[blue!95.2, opacity=0.7] (1.0100, 2.0100, 1.2078) -- (1.0590, 2.0100, 1.2108) -- (1.0590, 2.0610, 1.2076) -- (1.0100, 2.0610, 1.2046) -- cycle;
\fill[blue!35.9, opacity=0.7] (1.0100, 2.0610, 1.2046) -- (1.0590, 2.0610, 1.2076) -- (1.0590, 2.1120, 1.2040) -- (1.0100, 2.1120, 1.2010) -- cycle;
\fill[blue!16.3, opacity=0.7] (1.0100, 2.1120, 1.2010) -- (1.0590, 2.1120, 1.2040) -- (1.0590, 2.1630, 1.2002) -- (1.0100, 2.1630, 1.1972) -- cycle;
\fill[blue!15.0, opacity=0.7] (1.0100, 2.1630, 1.1972) -- (1.0590, 2.1630, 1.2002) -- (1.0590, 2.2140, 1.1961) -- (1.0100, 2.2140, 1.1931) -- cycle;
\fill[blue!15.0, opacity=0.7] (1.0100, 2.2140, 1.1931) -- (1.0590, 2.2140, 1.1961) -- (1.0590, 2.2650, 1.1918) -- (1.0100, 2.2650, 1.1888) -- cycle;
\fill[blue!15.0, opacity=0.7] (1.0100, 2.2650, 1.1888) -- (1.0590, 2.2650, 1.1918) -- (1.0590, 2.3160, 1.1872) -- (1.0100, 2.3160, 1.1842) -- cycle;
\fill[blue!15.3, opacity=0.7] (1.0100, 2.3160, 1.1842) -- (1.0590, 2.3160, 1.1872) -- (1.0590, 2.3670, 1.1824) -- (1.0100, 2.3670, 1.1794) -- cycle;
\fill[blue!21.2, opacity=0.7] (1.0100, 2.3670, 1.1794) -- (1.0590, 2.3670, 1.1824) -- (1.0590, 2.4180, 1.1775) -- (1.0100, 2.4180, 1.1745) -- cycle;
\fill[blue!50.4, opacity=0.7] (1.0100, 2.4180, 1.1745) -- (1.0590, 2.4180, 1.1775) -- (1.0590, 2.4690, 1.1723) -- (1.0100, 2.4690, 1.1693) -- cycle;
\fill[blue!69.4, opacity=0.7] (1.0100, 2.4690, 1.1693) -- (1.0590, 2.4690, 1.1723) -- (1.0590, 2.5200, 1.1669) -- (1.0100, 2.5200, 1.1639) -- cycle;
\fill[blue!44.1, opacity=0.7] (1.0100, 2.5200, 1.1639) -- (1.0590, 2.5200, 1.1669) -- (1.0590, 2.5710, 1.1614) -- (1.0100, 2.5710, 1.1584) -- cycle;
\fill[blue!17.1, opacity=0.7] (1.0100, 2.5710, 1.1584) -- (1.0590, 2.5710, 1.1614) -- (1.0590, 2.6220, 1.1557) -- (1.0100, 2.6220, 1.1527) -- cycle;
\fill[blue!15.0, opacity=0.7] (1.0100, 2.6220, 1.1527) -- (1.0590, 2.6220, 1.1557) -- (1.0590, 2.6730, 1.1499) -- (1.0100, 2.6730, 1.1469) -- cycle;
\fill[blue!15.0, opacity=0.7] (1.0100, 2.6730, 1.1469) -- (1.0590, 2.6730, 1.1499) -- (1.0590, 2.7240, 1.1440) -- (1.0100, 2.7240, 1.1410) -- cycle;
\fill[blue!15.0, opacity=0.7] (1.0100, 2.7240, 1.1410) -- (1.0590, 2.7240, 1.1440) -- (1.0590, 2.7750, 1.1380) -- (1.0100, 2.7750, 1.1350) -- cycle;
\fill[blue!15.0, opacity=0.7] (1.0100, 2.7750, 1.1350) -- (1.0590, 2.7750, 1.1380) -- (1.0590, 2.8260, 1.1319) -- (1.0100, 2.8260, 1.1289) -- cycle;
\fill[blue!15.0, opacity=0.7] (1.0100, 2.8260, 1.1289) -- (1.0590, 2.8260, 1.1319) -- (1.0590, 2.8770, 1.1257) -- (1.0100, 2.8770, 1.1227) -- cycle;
\fill[blue!15.1, opacity=0.7] (1.0100, 2.8770, 1.1227) -- (1.0590, 2.8770, 1.1257) -- (1.0590, 2.9280, 1.1195) -- (1.0100, 2.9280, 1.1165) -- cycle;
\fill[blue!16.1, opacity=0.7] (1.0100, 2.9280, 1.1165) -- (1.0590, 2.9280, 1.1195) -- (1.0590, 2.9790, 1.1132) -- (1.0100, 2.9790, 1.1102) -- cycle;
\fill[blue!15.7, opacity=0.7] (1.0100, 2.9790, 1.1102) -- (1.0590, 2.9790, 1.1132) -- (1.0590, 3.0300, 1.1069) -- (1.0100, 3.0300, 1.1039) -- cycle;
\fill[blue!15.0, opacity=0.7] (1.0590, -0.0300, 1.1069) -- (1.1080, -0.0300, 1.1096) -- (1.1080, 0.0210, 1.1159) -- (1.0590, 0.0210, 1.1132) -- cycle;
\fill[blue!15.0, opacity=0.7] (1.0590, 0.0210, 1.1132) -- (1.1080, 0.0210, 1.1159) -- (1.1080, 0.0720, 1.1222) -- (1.0590, 0.0720, 1.1195) -- cycle;
\fill[blue!15.0, opacity=0.7] (1.0590, 0.0720, 1.1195) -- (1.1080, 0.0720, 1.1222) -- (1.1080, 0.1230, 1.1284) -- (1.0590, 0.1230, 1.1257) -- cycle;
\fill[blue!15.0, opacity=0.7] (1.0590, 0.1230, 1.1257) -- (1.1080, 0.1230, 1.1284) -- (1.1080, 0.1740, 1.1346) -- (1.0590, 0.1740, 1.1319) -- cycle;
\fill[blue!15.0, opacity=0.7] (1.0590, 0.1740, 1.1319) -- (1.1080, 0.1740, 1.1346) -- (1.1080, 0.2250, 1.1407) -- (1.0590, 0.2250, 1.1380) -- cycle;
\fill[blue!15.0, opacity=0.7] (1.0590, 0.2250, 1.1380) -- (1.1080, 0.2250, 1.1407) -- (1.1080, 0.2760, 1.1467) -- (1.0590, 0.2760, 1.1440) -- cycle;
\fill[blue!20.9, opacity=0.7] (1.0590, 0.2760, 1.1440) -- (1.1080, 0.2760, 1.1467) -- (1.1080, 0.3270, 1.1526) -- (1.0590, 0.3270, 1.1499) -- cycle;
\fill[blue!43.4, opacity=0.7] (1.0590, 0.3270, 1.1499) -- (1.1080, 0.3270, 1.1526) -- (1.1080, 0.3780, 1.1584) -- (1.0590, 0.3780, 1.1557) -- cycle;
\fill[blue!32.3, opacity=0.7] (1.0590, 0.3780, 1.1557) -- (1.1080, 0.3780, 1.1584) -- (1.1080, 0.4290, 1.1641) -- (1.0590, 0.4290, 1.1614) -- cycle;
\fill[blue!16.2, opacity=0.7] (1.0590, 0.4290, 1.1614) -- (1.1080, 0.4290, 1.1641) -- (1.1080, 0.4800, 1.1696) -- (1.0590, 0.4800, 1.1669) -- cycle;
\fill[blue!15.0, opacity=0.7] (1.0590, 0.4800, 1.1669) -- (1.1080, 0.4800, 1.1696) -- (1.1080, 0.5310, 1.1750) -- (1.0590, 0.5310, 1.1723) -- cycle;
\fill[blue!15.0, opacity=0.7] (1.0590, 0.5310, 1.1723) -- (1.1080, 0.5310, 1.1750) -- (1.1080, 0.5820, 1.1802) -- (1.0590, 0.5820, 1.1775) -- cycle;
\fill[blue!15.0, opacity=0.7] (1.0590, 0.5820, 1.1775) -- (1.1080, 0.5820, 1.1802) -- (1.1080, 0.6330, 1.1851) -- (1.0590, 0.6330, 1.1824) -- cycle;
\fill[blue!15.3, opacity=0.7] (1.0590, 0.6330, 1.1824) -- (1.1080, 0.6330, 1.1851) -- (1.1080, 0.6840, 1.1899) -- (1.0590, 0.6840, 1.1872) -- cycle;
\fill[blue!38.5, opacity=0.7] (1.0590, 0.6840, 1.1872) -- (1.1080, 0.6840, 1.1899) -- (1.1080, 0.7350, 1.1945) -- (1.0590, 0.7350, 1.1918) -- cycle;
\fill[blue!82.8!black, opacity=0.7] (1.0590, 0.7350, 1.1918) -- (1.1080, 0.7350, 1.1945) -- (1.1080, 0.7860, 1.1988) -- (1.0590, 0.7860, 1.1961) -- cycle;
\fill[blue!70.3!black, opacity=0.7] (1.0590, 0.7860, 1.1961) -- (1.1080, 0.7860, 1.1988) -- (1.1080, 0.8370, 1.2029) -- (1.0590, 0.8370, 1.2002) -- cycle;
\fill[blue!87.5!black, opacity=0.7] (1.0590, 0.8370, 1.2002) -- (1.1080, 0.8370, 1.2029) -- (1.1080, 0.8880, 1.2067) -- (1.0590, 0.8880, 1.2040) -- cycle;
\fill[blue!43.7, opacity=0.7] (1.0590, 0.8880, 1.2040) -- (1.1080, 0.8880, 1.2067) -- (1.1080, 0.9390, 1.2103) -- (1.0590, 0.9390, 1.2076) -- cycle;
\fill[blue!17.3, opacity=0.7] (1.0590, 0.9390, 1.2076) -- (1.1080, 0.9390, 1.2103) -- (1.1080, 0.9900, 1.2135) -- (1.0590, 0.9900, 1.2108) -- cycle;
\fill[blue!15.4, opacity=0.7] (1.0590, 0.9900, 1.2108) -- (1.1080, 0.9900, 1.2135) -- (1.1080, 1.0410, 1.2165) -- (1.0590, 1.0410, 1.2138) -- cycle;
\fill[blue!15.8, opacity=0.7] (1.0590, 1.0410, 1.2138) -- (1.1080, 1.0410, 1.2165) -- (1.1080, 1.0920, 1.2193) -- (1.0590, 1.0920, 1.2165) -- cycle;
\fill[blue!26.9, opacity=0.7] (1.0590, 1.0920, 1.2165) -- (1.1080, 1.0920, 1.2193) -- (1.1080, 1.1430, 1.2217) -- (1.0590, 1.1430, 1.2190) -- cycle;
\fill[blue!93.3, opacity=0.7] (1.0590, 1.1430, 1.2190) -- (1.1080, 1.1430, 1.2217) -- (1.1080, 1.1940, 1.2238) -- (1.0590, 1.1940, 1.2210) -- cycle;
\fill[blue!95.9, opacity=0.7] (1.0590, 1.1940, 1.2210) -- (1.1080, 1.1940, 1.2238) -- (1.1080, 1.2450, 1.2255) -- (1.0590, 1.2450, 1.2228) -- cycle;
\fill[blue!52.6, opacity=0.7] (1.0590, 1.2450, 1.2228) -- (1.1080, 1.2450, 1.2255) -- (1.1080, 1.2960, 1.2270) -- (1.0590, 1.2960, 1.2243) -- cycle;
\fill[blue!44.1, opacity=0.7] (1.0590, 1.2960, 1.2243) -- (1.1080, 1.2960, 1.2270) -- (1.1080, 1.3470, 1.2281) -- (1.0590, 1.3470, 1.2254) -- cycle;
\fill[blue!60.0, opacity=0.7] (1.0590, 1.3470, 1.2254) -- (1.1080, 1.3470, 1.2281) -- (1.1080, 1.3980, 1.2290) -- (1.0590, 1.3980, 1.2263) -- cycle;
\fill[blue!92.4, opacity=0.7] (1.0590, 1.3980, 1.2263) -- (1.1080, 1.3980, 1.2290) -- (1.1080, 1.4490, 1.2295) -- (1.0590, 1.4490, 1.2268) -- cycle;
\fill[blue!69.0!black, opacity=0.7] (1.0590, 1.4490, 1.2268) -- (1.1080, 1.4490, 1.2295) -- (1.1080, 1.5000, 1.2296) -- (1.0590, 1.5000, 1.2269) -- cycle;
\fill[blue!90.0!black, opacity=0.7] (1.0590, 1.5000, 1.2269) -- (1.1080, 1.5000, 1.2296) -- (1.1080, 1.5510, 1.2295) -- (1.0590, 1.5510, 1.2268) -- cycle;
\fill[blue!88.2, opacity=0.7] (1.0590, 1.5510, 1.2268) -- (1.1080, 1.5510, 1.2295) -- (1.1080, 1.6020, 1.2290) -- (1.0590, 1.6020, 1.2263) -- cycle;
\fill[blue!79.3, opacity=0.7] (1.0590, 1.6020, 1.2263) -- (1.1080, 1.6020, 1.2290) -- (1.1080, 1.6530, 1.2281) -- (1.0590, 1.6530, 1.2254) -- cycle;
\fill[blue!80.1, opacity=0.7] (1.0590, 1.6530, 1.2254) -- (1.1080, 1.6530, 1.2281) -- (1.1080, 1.7040, 1.2270) -- (1.0590, 1.7040, 1.2243) -- cycle;
\fill[blue!90.0, opacity=0.7] (1.0590, 1.7040, 1.2243) -- (1.1080, 1.7040, 1.2270) -- (1.1080, 1.7550, 1.2255) -- (1.0590, 1.7550, 1.2228) -- cycle;
\fill[blue!87.1!black, opacity=0.7] (1.0590, 1.7550, 1.2228) -- (1.1080, 1.7550, 1.2255) -- (1.1080, 1.8060, 1.2238) -- (1.0590, 1.8060, 1.2210) -- cycle;
\fill[blue!68.5!black, opacity=0.7] (1.0590, 1.8060, 1.2210) -- (1.1080, 1.8060, 1.2238) -- (1.1080, 1.8570, 1.2217) -- (1.0590, 1.8570, 1.2190) -- cycle;
\fill[blue!97.0!black, opacity=0.7] (1.0590, 1.8570, 1.2190) -- (1.1080, 1.8570, 1.2217) -- (1.1080, 1.9080, 1.2193) -- (1.0590, 1.9080, 1.2165) -- cycle;
\fill[blue!85.0, opacity=0.7] (1.0590, 1.9080, 1.2165) -- (1.1080, 1.9080, 1.2193) -- (1.1080, 1.9590, 1.2165) -- (1.0590, 1.9590, 1.2138) -- cycle;
\fill[blue!83.8, opacity=0.7] (1.0590, 1.9590, 1.2138) -- (1.1080, 1.9590, 1.2165) -- (1.1080, 2.0100, 1.2135) -- (1.0590, 2.0100, 1.2108) -- cycle;
\fill[blue!89.0!black, opacity=0.7] (1.0590, 2.0100, 1.2108) -- (1.1080, 2.0100, 1.2135) -- (1.1080, 2.0610, 1.2103) -- (1.0590, 2.0610, 1.2076) -- cycle;
\fill[blue!95.7!black, opacity=0.7] (1.0590, 2.0610, 1.2076) -- (1.1080, 2.0610, 1.2103) -- (1.1080, 2.1120, 1.2067) -- (1.0590, 2.1120, 1.2040) -- cycle;
\fill[blue!37.2, opacity=0.7] (1.0590, 2.1120, 1.2040) -- (1.1080, 2.1120, 1.2067) -- (1.1080, 2.1630, 1.2029) -- (1.0590, 2.1630, 1.2002) -- cycle;
\fill[blue!16.0, opacity=0.7] (1.0590, 2.1630, 1.2002) -- (1.1080, 2.1630, 1.2029) -- (1.1080, 2.2140, 1.1988) -- (1.0590, 2.2140, 1.1961) -- cycle;
\fill[blue!15.0, opacity=0.7] (1.0590, 2.2140, 1.1961) -- (1.1080, 2.2140, 1.1988) -- (1.1080, 2.2650, 1.1945) -- (1.0590, 2.2650, 1.1918) -- cycle;
\fill[blue!15.0, opacity=0.7] (1.0590, 2.2650, 1.1918) -- (1.1080, 2.2650, 1.1945) -- (1.1080, 2.3160, 1.1899) -- (1.0590, 2.3160, 1.1872) -- cycle;
\fill[blue!15.0, opacity=0.7] (1.0590, 2.3160, 1.1872) -- (1.1080, 2.3160, 1.1899) -- (1.1080, 2.3670, 1.1851) -- (1.0590, 2.3670, 1.1824) -- cycle;
\fill[blue!15.7, opacity=0.7] (1.0590, 2.3670, 1.1824) -- (1.1080, 2.3670, 1.1851) -- (1.1080, 2.4180, 1.1802) -- (1.0590, 2.4180, 1.1775) -- cycle;
\fill[blue!27.3, opacity=0.7] (1.0590, 2.4180, 1.1775) -- (1.1080, 2.4180, 1.1802) -- (1.1080, 2.4690, 1.1750) -- (1.0590, 2.4690, 1.1723) -- cycle;
\fill[blue!60.1, opacity=0.7] (1.0590, 2.4690, 1.1723) -- (1.1080, 2.4690, 1.1750) -- (1.1080, 2.5200, 1.1696) -- (1.0590, 2.5200, 1.1669) -- cycle;
\fill[blue!62.5, opacity=0.7] (1.0590, 2.5200, 1.1669) -- (1.1080, 2.5200, 1.1696) -- (1.1080, 2.5710, 1.1641) -- (1.0590, 2.5710, 1.1614) -- cycle;
\fill[blue!27.3, opacity=0.7] (1.0590, 2.5710, 1.1614) -- (1.1080, 2.5710, 1.1641) -- (1.1080, 2.6220, 1.1584) -- (1.0590, 2.6220, 1.1557) -- cycle;
\fill[blue!15.2, opacity=0.7] (1.0590, 2.6220, 1.1557) -- (1.1080, 2.6220, 1.1584) -- (1.1080, 2.6730, 1.1526) -- (1.0590, 2.6730, 1.1499) -- cycle;
\fill[blue!15.0, opacity=0.7] (1.0590, 2.6730, 1.1499) -- (1.1080, 2.6730, 1.1526) -- (1.1080, 2.7240, 1.1467) -- (1.0590, 2.7240, 1.1440) -- cycle;
\fill[blue!15.0, opacity=0.7] (1.0590, 2.7240, 1.1440) -- (1.1080, 2.7240, 1.1467) -- (1.1080, 2.7750, 1.1407) -- (1.0590, 2.7750, 1.1380) -- cycle;
\fill[blue!15.0, opacity=0.7] (1.0590, 2.7750, 1.1380) -- (1.1080, 2.7750, 1.1407) -- (1.1080, 2.8260, 1.1346) -- (1.0590, 2.8260, 1.1319) -- cycle;
\fill[blue!15.0, opacity=0.7] (1.0590, 2.8260, 1.1319) -- (1.1080, 2.8260, 1.1346) -- (1.1080, 2.8770, 1.1284) -- (1.0590, 2.8770, 1.1257) -- cycle;
\fill[blue!15.0, opacity=0.7] (1.0590, 2.8770, 1.1257) -- (1.1080, 2.8770, 1.1284) -- (1.1080, 2.9280, 1.1222) -- (1.0590, 2.9280, 1.1195) -- cycle;
\fill[blue!15.5, opacity=0.7] (1.0590, 2.9280, 1.1195) -- (1.1080, 2.9280, 1.1222) -- (1.1080, 2.9790, 1.1159) -- (1.0590, 2.9790, 1.1132) -- cycle;
\fill[blue!16.2, opacity=0.7] (1.0590, 2.9790, 1.1132) -- (1.1080, 2.9790, 1.1159) -- (1.1080, 3.0300, 1.1096) -- (1.0590, 3.0300, 1.1069) -- cycle;
\fill[blue!15.0, opacity=0.7] (1.1080, -0.0300, 1.1096) -- (1.1570, -0.0300, 1.1120) -- (1.1570, 0.0210, 1.1183) -- (1.1080, 0.0210, 1.1159) -- cycle;
\fill[blue!15.0, opacity=0.7] (1.1080, 0.0210, 1.1159) -- (1.1570, 0.0210, 1.1183) -- (1.1570, 0.0720, 1.1246) -- (1.1080, 0.0720, 1.1222) -- cycle;
\fill[blue!15.0, opacity=0.7] (1.1080, 0.0720, 1.1222) -- (1.1570, 0.0720, 1.1246) -- (1.1570, 0.1230, 1.1308) -- (1.1080, 0.1230, 1.1284) -- cycle;
\fill[blue!15.0, opacity=0.7] (1.1080, 0.1230, 1.1284) -- (1.1570, 0.1230, 1.1308) -- (1.1570, 0.1740, 1.1370) -- (1.1080, 0.1740, 1.1346) -- cycle;
\fill[blue!15.0, opacity=0.7] (1.1080, 0.1740, 1.1346) -- (1.1570, 0.1740, 1.1370) -- (1.1570, 0.2250, 1.1431) -- (1.1080, 0.2250, 1.1407) -- cycle;
\fill[blue!15.2, opacity=0.7] (1.1080, 0.2250, 1.1407) -- (1.1570, 0.2250, 1.1431) -- (1.1570, 0.2760, 1.1491) -- (1.1080, 0.2760, 1.1467) -- cycle;
\fill[blue!26.3, opacity=0.7] (1.1080, 0.2760, 1.1467) -- (1.1570, 0.2760, 1.1491) -- (1.1570, 0.3270, 1.1550) -- (1.1080, 0.3270, 1.1526) -- cycle;
\fill[blue!45.8, opacity=0.7] (1.1080, 0.3270, 1.1526) -- (1.1570, 0.3270, 1.1550) -- (1.1570, 0.3780, 1.1608) -- (1.1080, 0.3780, 1.1584) -- cycle;
\fill[blue!26.8, opacity=0.7] (1.1080, 0.3780, 1.1584) -- (1.1570, 0.3780, 1.1608) -- (1.1570, 0.4290, 1.1665) -- (1.1080, 0.4290, 1.1641) -- cycle;
\fill[blue!15.4, opacity=0.7] (1.1080, 0.4290, 1.1641) -- (1.1570, 0.4290, 1.1665) -- (1.1570, 0.4800, 1.1720) -- (1.1080, 0.4800, 1.1696) -- cycle;
\fill[blue!15.0, opacity=0.7] (1.1080, 0.4800, 1.1696) -- (1.1570, 0.4800, 1.1720) -- (1.1570, 0.5310, 1.1774) -- (1.1080, 0.5310, 1.1750) -- cycle;
\fill[blue!15.0, opacity=0.7] (1.1080, 0.5310, 1.1750) -- (1.1570, 0.5310, 1.1774) -- (1.1570, 0.5820, 1.1826) -- (1.1080, 0.5820, 1.1802) -- cycle;
\fill[blue!15.0, opacity=0.7] (1.1080, 0.5820, 1.1802) -- (1.1570, 0.5820, 1.1826) -- (1.1570, 0.6330, 1.1875) -- (1.1080, 0.6330, 1.1851) -- cycle;
\fill[blue!16.9, opacity=0.7] (1.1080, 0.6330, 1.1851) -- (1.1570, 0.6330, 1.1875) -- (1.1570, 0.6840, 1.1923) -- (1.1080, 0.6840, 1.1899) -- cycle;
\fill[blue!66.7, opacity=0.7] (1.1080, 0.6840, 1.1899) -- (1.1570, 0.6840, 1.1923) -- (1.1570, 0.7350, 1.1969) -- (1.1080, 0.7350, 1.1945) -- cycle;
\fill[blue!68.4!black, opacity=0.7] (1.1080, 0.7350, 1.1945) -- (1.1570, 0.7350, 1.1969) -- (1.1570, 0.7860, 1.2012) -- (1.1080, 0.7860, 1.1988) -- cycle;
\fill[blue!68.9!black, opacity=0.7] (1.1080, 0.7860, 1.1988) -- (1.1570, 0.7860, 1.2012) -- (1.1570, 0.8370, 1.2053) -- (1.1080, 0.8370, 1.2029) -- cycle;
\fill[blue!85.6, opacity=0.7] (1.1080, 0.8370, 1.2029) -- (1.1570, 0.8370, 1.2053) -- (1.1570, 0.8880, 1.2091) -- (1.1080, 0.8880, 1.2067) -- cycle;
\fill[blue!25.9, opacity=0.7] (1.1080, 0.8880, 1.2067) -- (1.1570, 0.8880, 1.2091) -- (1.1570, 0.9390, 1.2127) -- (1.1080, 0.9390, 1.2103) -- cycle;
\fill[blue!15.8, opacity=0.7] (1.1080, 0.9390, 1.2103) -- (1.1570, 0.9390, 1.2127) -- (1.1570, 0.9900, 1.2160) -- (1.1080, 0.9900, 1.2135) -- cycle;
\fill[blue!15.5, opacity=0.7] (1.1080, 0.9900, 1.2135) -- (1.1570, 0.9900, 1.2160) -- (1.1570, 1.0410, 1.2190) -- (1.1080, 1.0410, 1.2165) -- cycle;
\fill[blue!19.6, opacity=0.7] (1.1080, 1.0410, 1.2165) -- (1.1570, 1.0410, 1.2190) -- (1.1570, 1.0920, 1.2217) -- (1.1080, 1.0920, 1.2193) -- cycle;
\fill[blue!71.0, opacity=0.7] (1.1080, 1.0920, 1.2193) -- (1.1570, 1.0920, 1.2217) -- (1.1570, 1.1430, 1.2241) -- (1.1080, 1.1430, 1.2217) -- cycle;
\fill[blue!90.0!black, opacity=0.7] (1.1080, 1.1430, 1.2217) -- (1.1570, 1.1430, 1.2241) -- (1.1570, 1.1940, 1.2262) -- (1.1080, 1.1940, 1.2238) -- cycle;
\fill[blue!51.2, opacity=0.7] (1.1080, 1.1940, 1.2238) -- (1.1570, 1.1940, 1.2262) -- (1.1570, 1.2450, 1.2279) -- (1.1080, 1.2450, 1.2255) -- cycle;
\fill[blue!40.6, opacity=0.7] (1.1080, 1.2450, 1.2255) -- (1.1570, 1.2450, 1.2279) -- (1.1570, 1.2960, 1.2294) -- (1.1080, 1.2960, 1.2270) -- cycle;
\fill[blue!63.6, opacity=0.7] (1.1080, 1.2960, 1.2270) -- (1.1570, 1.2960, 1.2294) -- (1.1570, 1.3470, 1.2306) -- (1.1080, 1.3470, 1.2281) -- cycle;
\fill[blue!82.8!black, opacity=0.7] (1.1080, 1.3470, 1.2281) -- (1.1570, 1.3470, 1.2306) -- (1.1570, 1.3980, 1.2314) -- (1.1080, 1.3980, 1.2290) -- cycle;
\fill[blue!99.4, opacity=0.7] (1.1080, 1.3980, 1.2290) -- (1.1570, 1.3980, 1.2314) -- (1.1570, 1.4490, 1.2319) -- (1.1080, 1.4490, 1.2295) -- cycle;
\fill[blue!61.1, opacity=0.7] (1.1080, 1.4490, 1.2295) -- (1.1570, 1.4490, 1.2319) -- (1.1570, 1.5000, 1.2320) -- (1.1080, 1.5000, 1.2296) -- cycle;
\fill[blue!39.1, opacity=0.7] (1.1080, 1.5000, 1.2296) -- (1.1570, 1.5000, 1.2320) -- (1.1570, 1.5510, 1.2319) -- (1.1080, 1.5510, 1.2295) -- cycle;
\fill[blue!30.3, opacity=0.7] (1.1080, 1.5510, 1.2295) -- (1.1570, 1.5510, 1.2319) -- (1.1570, 1.6020, 1.2314) -- (1.1080, 1.6020, 1.2290) -- cycle;
\fill[blue!27.0, opacity=0.7] (1.1080, 1.6020, 1.2290) -- (1.1570, 1.6020, 1.2314) -- (1.1570, 1.6530, 1.2306) -- (1.1080, 1.6530, 1.2281) -- cycle;
\fill[blue!26.5, opacity=0.7] (1.1080, 1.6530, 1.2281) -- (1.1570, 1.6530, 1.2306) -- (1.1570, 1.7040, 1.2294) -- (1.1080, 1.7040, 1.2270) -- cycle;
\fill[blue!29.0, opacity=0.7] (1.1080, 1.7040, 1.2270) -- (1.1570, 1.7040, 1.2294) -- (1.1570, 1.7550, 1.2279) -- (1.1080, 1.7550, 1.2255) -- cycle;
\fill[blue!37.5, opacity=0.7] (1.1080, 1.7550, 1.2255) -- (1.1570, 1.7550, 1.2279) -- (1.1570, 1.8060, 1.2262) -- (1.1080, 1.8060, 1.2238) -- cycle;
\fill[blue!59.3, opacity=0.7] (1.1080, 1.8060, 1.2238) -- (1.1570, 1.8060, 1.2262) -- (1.1570, 1.8570, 1.2241) -- (1.1080, 1.8570, 1.2217) -- cycle;
\fill[blue!95.1, opacity=0.7] (1.1080, 1.8570, 1.2217) -- (1.1570, 1.8570, 1.2241) -- (1.1570, 1.9080, 1.2217) -- (1.1080, 1.9080, 1.2193) -- cycle;
\fill[blue!68.9!black, opacity=0.7] (1.1080, 1.9080, 1.2193) -- (1.1570, 1.9080, 1.2217) -- (1.1570, 1.9590, 1.2190) -- (1.1080, 1.9590, 1.2165) -- cycle;
\fill[blue!95.2, opacity=0.7] (1.1080, 1.9590, 1.2165) -- (1.1570, 1.9590, 1.2190) -- (1.1570, 2.0100, 1.2160) -- (1.1080, 2.0100, 1.2135) -- cycle;
\fill[blue!86.0, opacity=0.7] (1.1080, 2.0100, 1.2135) -- (1.1570, 2.0100, 1.2160) -- (1.1570, 2.0610, 1.2127) -- (1.1080, 2.0610, 1.2103) -- cycle;
\fill[blue!88.3!black, opacity=0.7] (1.1080, 2.0610, 1.2103) -- (1.1570, 2.0610, 1.2127) -- (1.1570, 2.1120, 1.2091) -- (1.1080, 2.1120, 1.2067) -- cycle;
\fill[blue!97.8, opacity=0.7] (1.1080, 2.1120, 1.2067) -- (1.1570, 2.1120, 1.2091) -- (1.1570, 2.1630, 1.2053) -- (1.1080, 2.1630, 1.2029) -- cycle;
\fill[blue!30.2, opacity=0.7] (1.1080, 2.1630, 1.2029) -- (1.1570, 2.1630, 1.2053) -- (1.1570, 2.2140, 1.2012) -- (1.1080, 2.2140, 1.1988) -- cycle;
\fill[blue!15.4, opacity=0.7] (1.1080, 2.2140, 1.1988) -- (1.1570, 2.2140, 1.2012) -- (1.1570, 2.2650, 1.1969) -- (1.1080, 2.2650, 1.1945) -- cycle;
\fill[blue!15.0, opacity=0.7] (1.1080, 2.2650, 1.1945) -- (1.1570, 2.2650, 1.1969) -- (1.1570, 2.3160, 1.1923) -- (1.1080, 2.3160, 1.1899) -- cycle;
\fill[blue!15.0, opacity=0.7] (1.1080, 2.3160, 1.1899) -- (1.1570, 2.3160, 1.1923) -- (1.1570, 2.3670, 1.1875) -- (1.1080, 2.3670, 1.1851) -- cycle;
\fill[blue!15.1, opacity=0.7] (1.1080, 2.3670, 1.1851) -- (1.1570, 2.3670, 1.1875) -- (1.1570, 2.4180, 1.1826) -- (1.1080, 2.4180, 1.1802) -- cycle;
\fill[blue!17.2, opacity=0.7] (1.1080, 2.4180, 1.1802) -- (1.1570, 2.4180, 1.1826) -- (1.1570, 2.4690, 1.1774) -- (1.1080, 2.4690, 1.1750) -- cycle;
\fill[blue!39.6, opacity=0.7] (1.1080, 2.4690, 1.1750) -- (1.1570, 2.4690, 1.1774) -- (1.1570, 2.5200, 1.1720) -- (1.1080, 2.5200, 1.1696) -- cycle;
\fill[blue!65.7, opacity=0.7] (1.1080, 2.5200, 1.1696) -- (1.1570, 2.5200, 1.1720) -- (1.1570, 2.5710, 1.1665) -- (1.1080, 2.5710, 1.1641) -- cycle;
\fill[blue!45.4, opacity=0.7] (1.1080, 2.5710, 1.1641) -- (1.1570, 2.5710, 1.1665) -- (1.1570, 2.6220, 1.1608) -- (1.1080, 2.6220, 1.1584) -- cycle;
\fill[blue!17.2, opacity=0.7] (1.1080, 2.6220, 1.1584) -- (1.1570, 2.6220, 1.1608) -- (1.1570, 2.6730, 1.1550) -- (1.1080, 2.6730, 1.1526) -- cycle;
\fill[blue!15.0, opacity=0.7] (1.1080, 2.6730, 1.1526) -- (1.1570, 2.6730, 1.1550) -- (1.1570, 2.7240, 1.1491) -- (1.1080, 2.7240, 1.1467) -- cycle;
\fill[blue!15.0, opacity=0.7] (1.1080, 2.7240, 1.1467) -- (1.1570, 2.7240, 1.1491) -- (1.1570, 2.7750, 1.1431) -- (1.1080, 2.7750, 1.1407) -- cycle;
\fill[blue!15.0, opacity=0.7] (1.1080, 2.7750, 1.1407) -- (1.1570, 2.7750, 1.1431) -- (1.1570, 2.8260, 1.1370) -- (1.1080, 2.8260, 1.1346) -- cycle;
\fill[blue!15.0, opacity=0.7] (1.1080, 2.8260, 1.1346) -- (1.1570, 2.8260, 1.1370) -- (1.1570, 2.8770, 1.1308) -- (1.1080, 2.8770, 1.1284) -- cycle;
\fill[blue!15.0, opacity=0.7] (1.1080, 2.8770, 1.1284) -- (1.1570, 2.8770, 1.1308) -- (1.1570, 2.9280, 1.1246) -- (1.1080, 2.9280, 1.1222) -- cycle;
\fill[blue!15.2, opacity=0.7] (1.1080, 2.9280, 1.1222) -- (1.1570, 2.9280, 1.1246) -- (1.1570, 2.9790, 1.1183) -- (1.1080, 2.9790, 1.1159) -- cycle;
\fill[blue!16.1, opacity=0.7] (1.1080, 2.9790, 1.1159) -- (1.1570, 2.9790, 1.1183) -- (1.1570, 3.0300, 1.1120) -- (1.1080, 3.0300, 1.1096) -- cycle;
\fill[blue!15.0, opacity=0.7] (1.1570, -0.0300, 1.1120) -- (1.2060, -0.0300, 1.1141) -- (1.2060, 0.0210, 1.1204) -- (1.1570, 0.0210, 1.1183) -- cycle;
\fill[blue!15.0, opacity=0.7] (1.1570, 0.0210, 1.1183) -- (1.2060, 0.0210, 1.1204) -- (1.2060, 0.0720, 1.1267) -- (1.1570, 0.0720, 1.1246) -- cycle;
\fill[blue!15.0, opacity=0.7] (1.1570, 0.0720, 1.1246) -- (1.2060, 0.0720, 1.1267) -- (1.2060, 0.1230, 1.1329) -- (1.1570, 0.1230, 1.1308) -- cycle;
\fill[blue!15.0, opacity=0.7] (1.1570, 0.1230, 1.1308) -- (1.2060, 0.1230, 1.1329) -- (1.2060, 0.1740, 1.1391) -- (1.1570, 0.1740, 1.1370) -- cycle;
\fill[blue!15.0, opacity=0.7] (1.1570, 0.1740, 1.1370) -- (1.2060, 0.1740, 1.1391) -- (1.2060, 0.2250, 1.1452) -- (1.1570, 0.2250, 1.1431) -- cycle;
\fill[blue!15.5, opacity=0.7] (1.1570, 0.2250, 1.1431) -- (1.2060, 0.2250, 1.1452) -- (1.2060, 0.2760, 1.1512) -- (1.1570, 0.2760, 1.1491) -- cycle;
\fill[blue!32.1, opacity=0.7] (1.1570, 0.2760, 1.1491) -- (1.2060, 0.2760, 1.1512) -- (1.2060, 0.3270, 1.1571) -- (1.1570, 0.3270, 1.1550) -- cycle;
\fill[blue!45.9, opacity=0.7] (1.1570, 0.3270, 1.1550) -- (1.2060, 0.3270, 1.1571) -- (1.2060, 0.3780, 1.1629) -- (1.1570, 0.3780, 1.1608) -- cycle;
\fill[blue!22.9, opacity=0.7] (1.1570, 0.3780, 1.1608) -- (1.2060, 0.3780, 1.1629) -- (1.2060, 0.4290, 1.1686) -- (1.1570, 0.4290, 1.1665) -- cycle;
\fill[blue!15.2, opacity=0.7] (1.1570, 0.4290, 1.1665) -- (1.2060, 0.4290, 1.1686) -- (1.2060, 0.4800, 1.1741) -- (1.1570, 0.4800, 1.1720) -- cycle;
\fill[blue!15.0, opacity=0.7] (1.1570, 0.4800, 1.1720) -- (1.2060, 0.4800, 1.1741) -- (1.2060, 0.5310, 1.1795) -- (1.1570, 0.5310, 1.1774) -- cycle;
\fill[blue!15.0, opacity=0.7] (1.1570, 0.5310, 1.1774) -- (1.2060, 0.5310, 1.1795) -- (1.2060, 0.5820, 1.1847) -- (1.1570, 0.5820, 1.1826) -- cycle;
\fill[blue!15.0, opacity=0.7] (1.1570, 0.5820, 1.1826) -- (1.2060, 0.5820, 1.1847) -- (1.2060, 0.6330, 1.1896) -- (1.1570, 0.6330, 1.1875) -- cycle;
\fill[blue!21.7, opacity=0.7] (1.1570, 0.6330, 1.1875) -- (1.2060, 0.6330, 1.1896) -- (1.2060, 0.6840, 1.1944) -- (1.1570, 0.6840, 1.1923) -- cycle;
\fill[blue!91.1, opacity=0.7] (1.1570, 0.6840, 1.1923) -- (1.2060, 0.6840, 1.1944) -- (1.2060, 0.7350, 1.1990) -- (1.1570, 0.7350, 1.1969) -- cycle;
\fill[blue!71.5!black, opacity=0.7] (1.1570, 0.7350, 1.1969) -- (1.2060, 0.7350, 1.1990) -- (1.2060, 0.7860, 1.2033) -- (1.1570, 0.7860, 1.2012) -- cycle;
\fill[blue!69.2!black, opacity=0.7] (1.1570, 0.7860, 1.2012) -- (1.2060, 0.7860, 1.2033) -- (1.2060, 0.8370, 1.2074) -- (1.1570, 0.8370, 1.2053) -- cycle;
\fill[blue!62.9, opacity=0.7] (1.1570, 0.8370, 1.2053) -- (1.2060, 0.8370, 1.2074) -- (1.2060, 0.8880, 1.2112) -- (1.1570, 0.8880, 1.2091) -- cycle;
\fill[blue!19.4, opacity=0.7] (1.1570, 0.8880, 1.2091) -- (1.2060, 0.8880, 1.2112) -- (1.2060, 0.9390, 1.2148) -- (1.1570, 0.9390, 1.2127) -- cycle;
\fill[blue!15.5, opacity=0.7] (1.1570, 0.9390, 1.2127) -- (1.2060, 0.9390, 1.2148) -- (1.2060, 0.9900, 1.2180) -- (1.1570, 0.9900, 1.2160) -- cycle;
\fill[blue!16.2, opacity=0.7] (1.1570, 0.9900, 1.2160) -- (1.2060, 0.9900, 1.2180) -- (1.2060, 1.0410, 1.2210) -- (1.1570, 1.0410, 1.2190) -- cycle;
\fill[blue!35.9, opacity=0.7] (1.1570, 1.0410, 1.2190) -- (1.2060, 1.0410, 1.2210) -- (1.2060, 1.0920, 1.2238) -- (1.1570, 1.0920, 1.2217) -- cycle;
\fill[blue!70.5!black, opacity=0.7] (1.1570, 1.0920, 1.2217) -- (1.2060, 1.0920, 1.2238) -- (1.2060, 1.1430, 1.2262) -- (1.1570, 1.1430, 1.2241) -- cycle;
\fill[blue!62.2, opacity=0.7] (1.1570, 1.1430, 1.2241) -- (1.2060, 1.1430, 1.2262) -- (1.2060, 1.1940, 1.2283) -- (1.1570, 1.1940, 1.2262) -- cycle;
\fill[blue!36.9, opacity=0.7] (1.1570, 1.1940, 1.2262) -- (1.2060, 1.1940, 1.2283) -- (1.2060, 1.2450, 1.2300) -- (1.1570, 1.2450, 1.2279) -- cycle;
\fill[blue!54.5, opacity=0.7] (1.1570, 1.2450, 1.2279) -- (1.2060, 1.2450, 1.2300) -- (1.2060, 1.2960, 1.2315) -- (1.1570, 1.2960, 1.2294) -- cycle;
\fill[blue!83.8!black, opacity=0.7] (1.1570, 1.2960, 1.2294) -- (1.2060, 1.2960, 1.2315) -- (1.2060, 1.3470, 1.2326) -- (1.1570, 1.3470, 1.2306) -- cycle;
\fill[blue!88.8, opacity=0.7] (1.1570, 1.3470, 1.2306) -- (1.2060, 1.3470, 1.2326) -- (1.2060, 1.3980, 1.2335) -- (1.1570, 1.3980, 1.2314) -- cycle;
\fill[blue!43.7, opacity=0.7] (1.1570, 1.3980, 1.2314) -- (1.2060, 1.3980, 1.2335) -- (1.2060, 1.4490, 1.2340) -- (1.1570, 1.4490, 1.2319) -- cycle;
\fill[blue!28.8, opacity=0.7] (1.1570, 1.4490, 1.2319) -- (1.2060, 1.4490, 1.2340) -- (1.2060, 1.5000, 1.2341) -- (1.1570, 1.5000, 1.2320) -- cycle;
\fill[blue!25.9, opacity=0.7] (1.1570, 1.5000, 1.2320) -- (1.2060, 1.5000, 1.2341) -- (1.2060, 1.5510, 1.2340) -- (1.1570, 1.5510, 1.2319) -- cycle;
\fill[blue!25.8, opacity=0.7] (1.1570, 1.5510, 1.2319) -- (1.2060, 1.5510, 1.2340) -- (1.2060, 1.6020, 1.2335) -- (1.1570, 1.6020, 1.2314) -- cycle;
\fill[blue!25.0, opacity=0.7] (1.1570, 1.6020, 1.2314) -- (1.2060, 1.6020, 1.2335) -- (1.2060, 1.6530, 1.2326) -- (1.1570, 1.6530, 1.2306) -- cycle;
\fill[blue!22.8, opacity=0.7] (1.1570, 1.6530, 1.2306) -- (1.2060, 1.6530, 1.2326) -- (1.2060, 1.7040, 1.2315) -- (1.1570, 1.7040, 1.2294) -- cycle;
\fill[blue!20.7, opacity=0.7] (1.1570, 1.7040, 1.2294) -- (1.2060, 1.7040, 1.2315) -- (1.2060, 1.7550, 1.2300) -- (1.1570, 1.7550, 1.2279) -- cycle;
\fill[blue!20.0, opacity=0.7] (1.1570, 1.7550, 1.2279) -- (1.2060, 1.7550, 1.2300) -- (1.2060, 1.8060, 1.2283) -- (1.1570, 1.8060, 1.2262) -- cycle;
\fill[blue!22.3, opacity=0.7] (1.1570, 1.8060, 1.2262) -- (1.2060, 1.8060, 1.2283) -- (1.2060, 1.8570, 1.2262) -- (1.1570, 1.8570, 1.2241) -- cycle;
\fill[blue!34.0, opacity=0.7] (1.1570, 1.8570, 1.2241) -- (1.2060, 1.8570, 1.2262) -- (1.2060, 1.9080, 1.2238) -- (1.1570, 1.9080, 1.2217) -- cycle;
\fill[blue!70.7, opacity=0.7] (1.1570, 1.9080, 1.2217) -- (1.2060, 1.9080, 1.2238) -- (1.2060, 1.9590, 1.2210) -- (1.1570, 1.9590, 1.2190) -- cycle;
\fill[blue!72.6!black, opacity=0.7] (1.1570, 1.9590, 1.2190) -- (1.2060, 1.9590, 1.2210) -- (1.2060, 2.0100, 1.2180) -- (1.1570, 2.0100, 1.2160) -- cycle;
\fill[blue!99.8!black, opacity=0.7] (1.1570, 2.0100, 1.2160) -- (1.2060, 2.0100, 1.2180) -- (1.2060, 2.0610, 1.2148) -- (1.1570, 2.0610, 1.2127) -- cycle;
\fill[blue!89.4, opacity=0.7] (1.1570, 2.0610, 1.2127) -- (1.2060, 2.0610, 1.2148) -- (1.2060, 2.1120, 1.2112) -- (1.1570, 2.1120, 1.2091) -- cycle;
\fill[blue!76.5!black, opacity=0.7] (1.1570, 2.1120, 1.2091) -- (1.2060, 2.1120, 1.2112) -- (1.2060, 2.1630, 1.2074) -- (1.1570, 2.1630, 1.2053) -- cycle;
\fill[blue!81.1, opacity=0.7] (1.1570, 2.1630, 1.2053) -- (1.2060, 2.1630, 1.2074) -- (1.2060, 2.2140, 1.2033) -- (1.1570, 2.2140, 1.2012) -- cycle;
\fill[blue!20.9, opacity=0.7] (1.1570, 2.2140, 1.2012) -- (1.2060, 2.2140, 1.2033) -- (1.2060, 2.2650, 1.1990) -- (1.1570, 2.2650, 1.1969) -- cycle;
\fill[blue!15.1, opacity=0.7] (1.1570, 2.2650, 1.1969) -- (1.2060, 2.2650, 1.1990) -- (1.2060, 2.3160, 1.1944) -- (1.1570, 2.3160, 1.1923) -- cycle;
\fill[blue!15.0, opacity=0.7] (1.1570, 2.3160, 1.1923) -- (1.2060, 2.3160, 1.1944) -- (1.2060, 2.3670, 1.1896) -- (1.1570, 2.3670, 1.1875) -- cycle;
\fill[blue!15.0, opacity=0.7] (1.1570, 2.3670, 1.1875) -- (1.2060, 2.3670, 1.1896) -- (1.2060, 2.4180, 1.1847) -- (1.1570, 2.4180, 1.1826) -- cycle;
\fill[blue!15.3, opacity=0.7] (1.1570, 2.4180, 1.1826) -- (1.2060, 2.4180, 1.1847) -- (1.2060, 2.4690, 1.1795) -- (1.1570, 2.4690, 1.1774) -- cycle;
\fill[blue!23.3, opacity=0.7] (1.1570, 2.4690, 1.1774) -- (1.2060, 2.4690, 1.1795) -- (1.2060, 2.5200, 1.1741) -- (1.1570, 2.5200, 1.1720) -- cycle;
\fill[blue!55.5, opacity=0.7] (1.1570, 2.5200, 1.1720) -- (1.2060, 2.5200, 1.1741) -- (1.2060, 2.5710, 1.1686) -- (1.1570, 2.5710, 1.1665) -- cycle;
\fill[blue!59.2, opacity=0.7] (1.1570, 2.5710, 1.1665) -- (1.2060, 2.5710, 1.1686) -- (1.2060, 2.6220, 1.1629) -- (1.1570, 2.6220, 1.1608) -- cycle;
\fill[blue!24.6, opacity=0.7] (1.1570, 2.6220, 1.1608) -- (1.2060, 2.6220, 1.1629) -- (1.2060, 2.6730, 1.1571) -- (1.1570, 2.6730, 1.1550) -- cycle;
\fill[blue!15.1, opacity=0.7] (1.1570, 2.6730, 1.1550) -- (1.2060, 2.6730, 1.1571) -- (1.2060, 2.7240, 1.1512) -- (1.1570, 2.7240, 1.1491) -- cycle;
\fill[blue!15.0, opacity=0.7] (1.1570, 2.7240, 1.1491) -- (1.2060, 2.7240, 1.1512) -- (1.2060, 2.7750, 1.1452) -- (1.1570, 2.7750, 1.1431) -- cycle;
\fill[blue!15.0, opacity=0.7] (1.1570, 2.7750, 1.1431) -- (1.2060, 2.7750, 1.1452) -- (1.2060, 2.8260, 1.1391) -- (1.1570, 2.8260, 1.1370) -- cycle;
\fill[blue!15.0, opacity=0.7] (1.1570, 2.8260, 1.1370) -- (1.2060, 2.8260, 1.1391) -- (1.2060, 2.8770, 1.1329) -- (1.1570, 2.8770, 1.1308) -- cycle;
\fill[blue!15.0, opacity=0.7] (1.1570, 2.8770, 1.1308) -- (1.2060, 2.8770, 1.1329) -- (1.2060, 2.9280, 1.1267) -- (1.1570, 2.9280, 1.1246) -- cycle;
\fill[blue!15.0, opacity=0.7] (1.1570, 2.9280, 1.1246) -- (1.2060, 2.9280, 1.1267) -- (1.2060, 2.9790, 1.1204) -- (1.1570, 2.9790, 1.1183) -- cycle;
\fill[blue!15.8, opacity=0.7] (1.1570, 2.9790, 1.1183) -- (1.2060, 2.9790, 1.1204) -- (1.2060, 3.0300, 1.1141) -- (1.1570, 3.0300, 1.1120) -- cycle;
\fill[blue!15.0, opacity=0.7] (1.2060, -0.0300, 1.1141) -- (1.2550, -0.0300, 1.1159) -- (1.2550, 0.0210, 1.1222) -- (1.2060, 0.0210, 1.1204) -- cycle;
\fill[blue!15.0, opacity=0.7] (1.2060, 0.0210, 1.1204) -- (1.2550, 0.0210, 1.1222) -- (1.2550, 0.0720, 1.1285) -- (1.2060, 0.0720, 1.1267) -- cycle;
\fill[blue!15.0, opacity=0.7] (1.2060, 0.0720, 1.1267) -- (1.2550, 0.0720, 1.1285) -- (1.2550, 0.1230, 1.1347) -- (1.2060, 0.1230, 1.1329) -- cycle;
\fill[blue!15.0, opacity=0.7] (1.2060, 0.1230, 1.1329) -- (1.2550, 0.1230, 1.1347) -- (1.2550, 0.1740, 1.1409) -- (1.2060, 0.1740, 1.1391) -- cycle;
\fill[blue!15.0, opacity=0.7] (1.2060, 0.1740, 1.1391) -- (1.2550, 0.1740, 1.1409) -- (1.2550, 0.2250, 1.1470) -- (1.2060, 0.2250, 1.1452) -- cycle;
\fill[blue!16.2, opacity=0.7] (1.2060, 0.2250, 1.1452) -- (1.2550, 0.2250, 1.1470) -- (1.2550, 0.2760, 1.1530) -- (1.2060, 0.2760, 1.1512) -- cycle;
\fill[blue!37.0, opacity=0.7] (1.2060, 0.2760, 1.1512) -- (1.2550, 0.2760, 1.1530) -- (1.2550, 0.3270, 1.1589) -- (1.2060, 0.3270, 1.1571) -- cycle;
\fill[blue!44.9, opacity=0.7] (1.2060, 0.3270, 1.1571) -- (1.2550, 0.3270, 1.1589) -- (1.2550, 0.3780, 1.1647) -- (1.2060, 0.3780, 1.1629) -- cycle;
\fill[blue!20.5, opacity=0.7] (1.2060, 0.3780, 1.1629) -- (1.2550, 0.3780, 1.1647) -- (1.2550, 0.4290, 1.1704) -- (1.2060, 0.4290, 1.1686) -- cycle;
\fill[blue!15.1, opacity=0.7] (1.2060, 0.4290, 1.1686) -- (1.2550, 0.4290, 1.1704) -- (1.2550, 0.4800, 1.1759) -- (1.2060, 0.4800, 1.1741) -- cycle;
\fill[blue!15.0, opacity=0.7] (1.2060, 0.4800, 1.1741) -- (1.2550, 0.4800, 1.1759) -- (1.2550, 0.5310, 1.1813) -- (1.2060, 0.5310, 1.1795) -- cycle;
\fill[blue!15.0, opacity=0.7] (1.2060, 0.5310, 1.1795) -- (1.2550, 0.5310, 1.1813) -- (1.2550, 0.5820, 1.1864) -- (1.2060, 0.5820, 1.1847) -- cycle;
\fill[blue!15.1, opacity=0.7] (1.2060, 0.5820, 1.1847) -- (1.2550, 0.5820, 1.1864) -- (1.2550, 0.6330, 1.1914) -- (1.2060, 0.6330, 1.1896) -- cycle;
\fill[blue!30.2, opacity=0.7] (1.2060, 0.6330, 1.1896) -- (1.2550, 0.6330, 1.1914) -- (1.2550, 0.6840, 1.1962) -- (1.2060, 0.6840, 1.1944) -- cycle;
\fill[blue!87.4!black, opacity=0.7] (1.2060, 0.6840, 1.1944) -- (1.2550, 0.6840, 1.1962) -- (1.2550, 0.7350, 1.2008) -- (1.2060, 0.7350, 1.1990) -- cycle;
\fill[blue!75.4!black, opacity=0.7] (1.2060, 0.7350, 1.1990) -- (1.2550, 0.7350, 1.2008) -- (1.2550, 0.7860, 1.2051) -- (1.2060, 0.7860, 1.2033) -- cycle;
\fill[blue!76.5!black, opacity=0.7] (1.2060, 0.7860, 1.2033) -- (1.2550, 0.7860, 1.2051) -- (1.2550, 0.8370, 1.2092) -- (1.2060, 0.8370, 1.2074) -- cycle;
\fill[blue!46.2, opacity=0.7] (1.2060, 0.8370, 1.2074) -- (1.2550, 0.8370, 1.2092) -- (1.2550, 0.8880, 1.2130) -- (1.2060, 0.8880, 1.2112) -- cycle;
\fill[blue!17.2, opacity=0.7] (1.2060, 0.8880, 1.2112) -- (1.2550, 0.8880, 1.2130) -- (1.2550, 0.9390, 1.2166) -- (1.2060, 0.9390, 1.2148) -- cycle;
\fill[blue!15.6, opacity=0.7] (1.2060, 0.9390, 1.2148) -- (1.2550, 0.9390, 1.2166) -- (1.2550, 0.9900, 1.2198) -- (1.2060, 0.9900, 1.2180) -- cycle;
\fill[blue!18.5, opacity=0.7] (1.2060, 0.9900, 1.2180) -- (1.2550, 0.9900, 1.2198) -- (1.2550, 1.0410, 1.2228) -- (1.2060, 1.0410, 1.2210) -- cycle;
\fill[blue!69.5, opacity=0.7] (1.2060, 1.0410, 1.2210) -- (1.2550, 1.0410, 1.2228) -- (1.2550, 1.0920, 1.2255) -- (1.2060, 1.0920, 1.2238) -- cycle;
\fill[blue!96.5, opacity=0.7] (1.2060, 1.0920, 1.2238) -- (1.2550, 1.0920, 1.2255) -- (1.2550, 1.1430, 1.2279) -- (1.2060, 1.1430, 1.2262) -- cycle;
\fill[blue!39.8, opacity=0.7] (1.2060, 1.1430, 1.2262) -- (1.2550, 1.1430, 1.2279) -- (1.2550, 1.1940, 1.2300) -- (1.2060, 1.1940, 1.2283) -- cycle;
\fill[blue!38.9, opacity=0.7] (1.2060, 1.1940, 1.2283) -- (1.2550, 1.1940, 1.2300) -- (1.2550, 1.2450, 1.2318) -- (1.2060, 1.2450, 1.2300) -- cycle;
\fill[blue!87.3, opacity=0.7] (1.2060, 1.2450, 1.2300) -- (1.2550, 1.2450, 1.2318) -- (1.2550, 1.2960, 1.2333) -- (1.2060, 1.2960, 1.2315) -- cycle;
\fill[blue!99.8!black, opacity=0.7] (1.2060, 1.2960, 1.2315) -- (1.2550, 1.2960, 1.2333) -- (1.2550, 1.3470, 1.2344) -- (1.2060, 1.3470, 1.2326) -- cycle;
\fill[blue!45.3, opacity=0.7] (1.2060, 1.3470, 1.2326) -- (1.2550, 1.3470, 1.2344) -- (1.2550, 1.3980, 1.2353) -- (1.2060, 1.3980, 1.2335) -- cycle;
\fill[blue!30.8, opacity=0.7] (1.2060, 1.3980, 1.2335) -- (1.2550, 1.3980, 1.2353) -- (1.2550, 1.4490, 1.2357) -- (1.2060, 1.4490, 1.2340) -- cycle;
\fill[blue!37.1, opacity=0.7] (1.2060, 1.4490, 1.2340) -- (1.2550, 1.4490, 1.2357) -- (1.2550, 1.5000, 1.2359) -- (1.2060, 1.5000, 1.2341) -- cycle;
\fill[blue!56.3, opacity=0.7] (1.2060, 1.5000, 1.2341) -- (1.2550, 1.5000, 1.2359) -- (1.2550, 1.5510, 1.2357) -- (1.2060, 1.5510, 1.2340) -- cycle;
\fill[blue!74.2, opacity=0.7] (1.2060, 1.5510, 1.2340) -- (1.2550, 1.5510, 1.2357) -- (1.2550, 1.6020, 1.2353) -- (1.2060, 1.6020, 1.2335) -- cycle;
\fill[blue!76.8, opacity=0.7] (1.2060, 1.6020, 1.2335) -- (1.2550, 1.6020, 1.2353) -- (1.2550, 1.6530, 1.2344) -- (1.2060, 1.6530, 1.2326) -- cycle;
\fill[blue!62.8, opacity=0.7] (1.2060, 1.6530, 1.2326) -- (1.2550, 1.6530, 1.2344) -- (1.2550, 1.7040, 1.2333) -- (1.2060, 1.7040, 1.2315) -- cycle;
\fill[blue!40.3, opacity=0.7] (1.2060, 1.7040, 1.2315) -- (1.2550, 1.7040, 1.2333) -- (1.2550, 1.7550, 1.2318) -- (1.2060, 1.7550, 1.2300) -- cycle;
\fill[blue!24.7, opacity=0.7] (1.2060, 1.7550, 1.2300) -- (1.2550, 1.7550, 1.2318) -- (1.2550, 1.8060, 1.2300) -- (1.2060, 1.8060, 1.2283) -- cycle;
\fill[blue!19.2, opacity=0.7] (1.2060, 1.8060, 1.2283) -- (1.2550, 1.8060, 1.2300) -- (1.2550, 1.8570, 1.2279) -- (1.2060, 1.8570, 1.2262) -- cycle;
\fill[blue!19.0, opacity=0.7] (1.2060, 1.8570, 1.2262) -- (1.2550, 1.8570, 1.2279) -- (1.2550, 1.9080, 1.2255) -- (1.2060, 1.9080, 1.2238) -- cycle;
\fill[blue!26.3, opacity=0.7] (1.2060, 1.9080, 1.2238) -- (1.2550, 1.9080, 1.2255) -- (1.2550, 1.9590, 1.2228) -- (1.2060, 1.9590, 1.2210) -- cycle;
\fill[blue!61.4, opacity=0.7] (1.2060, 1.9590, 1.2210) -- (1.2550, 1.9590, 1.2228) -- (1.2550, 2.0100, 1.2198) -- (1.2060, 2.0100, 1.2180) -- cycle;
\fill[blue!74.2!black, opacity=0.7] (1.2060, 2.0100, 1.2180) -- (1.2550, 2.0100, 1.2198) -- (1.2550, 2.0610, 1.2166) -- (1.2060, 2.0610, 1.2148) -- cycle;
\fill[blue!99.6, opacity=0.7] (1.2060, 2.0610, 1.2148) -- (1.2550, 2.0610, 1.2166) -- (1.2550, 2.1120, 1.2130) -- (1.2060, 2.1120, 1.2112) -- cycle;
\fill[blue!94.5, opacity=0.7] (1.2060, 2.1120, 1.2112) -- (1.2550, 2.1120, 1.2130) -- (1.2550, 2.1630, 1.2092) -- (1.2060, 2.1630, 1.2074) -- cycle;
\fill[blue!69.0!black, opacity=0.7] (1.2060, 2.1630, 1.2074) -- (1.2550, 2.1630, 1.2092) -- (1.2550, 2.2140, 1.2051) -- (1.2060, 2.2140, 1.2033) -- cycle;
\fill[blue!49.8, opacity=0.7] (1.2060, 2.2140, 1.2033) -- (1.2550, 2.2140, 1.2051) -- (1.2550, 2.2650, 1.2008) -- (1.2060, 2.2650, 1.1990) -- cycle;
\fill[blue!16.1, opacity=0.7] (1.2060, 2.2650, 1.1990) -- (1.2550, 2.2650, 1.2008) -- (1.2550, 2.3160, 1.1962) -- (1.2060, 2.3160, 1.1944) -- cycle;
\fill[blue!15.0, opacity=0.7] (1.2060, 2.3160, 1.1944) -- (1.2550, 2.3160, 1.1962) -- (1.2550, 2.3670, 1.1914) -- (1.2060, 2.3670, 1.1896) -- cycle;
\fill[blue!15.0, opacity=0.7] (1.2060, 2.3670, 1.1896) -- (1.2550, 2.3670, 1.1914) -- (1.2550, 2.4180, 1.1864) -- (1.2060, 2.4180, 1.1847) -- cycle;
\fill[blue!15.0, opacity=0.7] (1.2060, 2.4180, 1.1847) -- (1.2550, 2.4180, 1.1864) -- (1.2550, 2.4690, 1.1813) -- (1.2060, 2.4690, 1.1795) -- cycle;
\fill[blue!16.9, opacity=0.7] (1.2060, 2.4690, 1.1795) -- (1.2550, 2.4690, 1.1813) -- (1.2550, 2.5200, 1.1759) -- (1.2060, 2.5200, 1.1741) -- cycle;
\fill[blue!39.3, opacity=0.7] (1.2060, 2.5200, 1.1741) -- (1.2550, 2.5200, 1.1759) -- (1.2550, 2.5710, 1.1704) -- (1.2060, 2.5710, 1.1686) -- cycle;
\fill[blue!62.8, opacity=0.7] (1.2060, 2.5710, 1.1686) -- (1.2550, 2.5710, 1.1704) -- (1.2550, 2.6220, 1.1647) -- (1.2060, 2.6220, 1.1629) -- cycle;
\fill[blue!37.3, opacity=0.7] (1.2060, 2.6220, 1.1629) -- (1.2550, 2.6220, 1.1647) -- (1.2550, 2.6730, 1.1589) -- (1.2060, 2.6730, 1.1571) -- cycle;
\fill[blue!15.8, opacity=0.7] (1.2060, 2.6730, 1.1571) -- (1.2550, 2.6730, 1.1589) -- (1.2550, 2.7240, 1.1530) -- (1.2060, 2.7240, 1.1512) -- cycle;
\fill[blue!15.0, opacity=0.7] (1.2060, 2.7240, 1.1512) -- (1.2550, 2.7240, 1.1530) -- (1.2550, 2.7750, 1.1470) -- (1.2060, 2.7750, 1.1452) -- cycle;
\fill[blue!15.0, opacity=0.7] (1.2060, 2.7750, 1.1452) -- (1.2550, 2.7750, 1.1470) -- (1.2550, 2.8260, 1.1409) -- (1.2060, 2.8260, 1.1391) -- cycle;
\fill[blue!15.0, opacity=0.7] (1.2060, 2.8260, 1.1391) -- (1.2550, 2.8260, 1.1409) -- (1.2550, 2.8770, 1.1347) -- (1.2060, 2.8770, 1.1329) -- cycle;
\fill[blue!15.0, opacity=0.7] (1.2060, 2.8770, 1.1329) -- (1.2550, 2.8770, 1.1347) -- (1.2550, 2.9280, 1.1285) -- (1.2060, 2.9280, 1.1267) -- cycle;
\fill[blue!15.0, opacity=0.7] (1.2060, 2.9280, 1.1267) -- (1.2550, 2.9280, 1.1285) -- (1.2550, 2.9790, 1.1222) -- (1.2060, 2.9790, 1.1204) -- cycle;
\fill[blue!15.4, opacity=0.7] (1.2060, 2.9790, 1.1204) -- (1.2550, 2.9790, 1.1222) -- (1.2550, 3.0300, 1.1159) -- (1.2060, 3.0300, 1.1141) -- cycle;
\fill[blue!15.0, opacity=0.7] (1.2550, -0.0300, 1.1159) -- (1.3040, -0.0300, 1.1174) -- (1.3040, 0.0210, 1.1237) -- (1.2550, 0.0210, 1.1222) -- cycle;
\fill[blue!15.0, opacity=0.7] (1.2550, 0.0210, 1.1222) -- (1.3040, 0.0210, 1.1237) -- (1.3040, 0.0720, 1.1299) -- (1.2550, 0.0720, 1.1285) -- cycle;
\fill[blue!15.0, opacity=0.7] (1.2550, 0.0720, 1.1285) -- (1.3040, 0.0720, 1.1299) -- (1.3040, 0.1230, 1.1361) -- (1.2550, 0.1230, 1.1347) -- cycle;
\fill[blue!15.0, opacity=0.7] (1.2550, 0.1230, 1.1347) -- (1.3040, 0.1230, 1.1361) -- (1.3040, 0.1740, 1.1423) -- (1.2550, 0.1740, 1.1409) -- cycle;
\fill[blue!15.0, opacity=0.7] (1.2550, 0.1740, 1.1409) -- (1.3040, 0.1740, 1.1423) -- (1.3040, 0.2250, 1.1484) -- (1.2550, 0.2250, 1.1470) -- cycle;
\fill[blue!16.9, opacity=0.7] (1.2550, 0.2250, 1.1470) -- (1.3040, 0.2250, 1.1484) -- (1.3040, 0.2760, 1.1545) -- (1.2550, 0.2760, 1.1530) -- cycle;
\fill[blue!40.8, opacity=0.7] (1.2550, 0.2760, 1.1530) -- (1.3040, 0.2760, 1.1545) -- (1.3040, 0.3270, 1.1604) -- (1.2550, 0.3270, 1.1589) -- cycle;
\fill[blue!43.9, opacity=0.7] (1.2550, 0.3270, 1.1589) -- (1.3040, 0.3270, 1.1604) -- (1.3040, 0.3780, 1.1662) -- (1.2550, 0.3780, 1.1647) -- cycle;
\fill[blue!19.2, opacity=0.7] (1.2550, 0.3780, 1.1647) -- (1.3040, 0.3780, 1.1662) -- (1.3040, 0.4290, 1.1719) -- (1.2550, 0.4290, 1.1704) -- cycle;
\fill[blue!15.1, opacity=0.7] (1.2550, 0.4290, 1.1704) -- (1.3040, 0.4290, 1.1719) -- (1.3040, 0.4800, 1.1774) -- (1.2550, 0.4800, 1.1759) -- cycle;
\fill[blue!15.0, opacity=0.7] (1.2550, 0.4800, 1.1759) -- (1.3040, 0.4800, 1.1774) -- (1.3040, 0.5310, 1.1827) -- (1.2550, 0.5310, 1.1813) -- cycle;
\fill[blue!15.0, opacity=0.7] (1.2550, 0.5310, 1.1813) -- (1.3040, 0.5310, 1.1827) -- (1.3040, 0.5820, 1.1879) -- (1.2550, 0.5820, 1.1864) -- cycle;
\fill[blue!15.3, opacity=0.7] (1.2550, 0.5820, 1.1864) -- (1.3040, 0.5820, 1.1879) -- (1.3040, 0.6330, 1.1929) -- (1.2550, 0.6330, 1.1914) -- cycle;
\fill[blue!40.1, opacity=0.7] (1.2550, 0.6330, 1.1914) -- (1.3040, 0.6330, 1.1929) -- (1.3040, 0.6840, 1.1977) -- (1.2550, 0.6840, 1.1962) -- cycle;
\fill[blue!73.1!black, opacity=0.7] (1.2550, 0.6840, 1.1962) -- (1.3040, 0.6840, 1.1977) -- (1.3040, 0.7350, 1.2022) -- (1.2550, 0.7350, 1.2008) -- cycle;
\fill[blue!77.3!black, opacity=0.7] (1.2550, 0.7350, 1.2008) -- (1.3040, 0.7350, 1.2022) -- (1.3040, 0.7860, 1.2066) -- (1.2550, 0.7860, 1.2051) -- cycle;
\fill[blue!87.7!black, opacity=0.7] (1.2550, 0.7860, 1.2051) -- (1.3040, 0.7860, 1.2066) -- (1.3040, 0.8370, 1.2106) -- (1.2550, 0.8370, 1.2092) -- cycle;
\fill[blue!36.9, opacity=0.7] (1.2550, 0.8370, 1.2092) -- (1.3040, 0.8370, 1.2106) -- (1.3040, 0.8880, 1.2145) -- (1.2550, 0.8880, 1.2130) -- cycle;
\fill[blue!16.5, opacity=0.7] (1.2550, 0.8880, 1.2130) -- (1.3040, 0.8880, 1.2145) -- (1.3040, 0.9390, 1.2180) -- (1.2550, 0.9390, 1.2166) -- cycle;
\fill[blue!15.7, opacity=0.7] (1.2550, 0.9390, 1.2166) -- (1.3040, 0.9390, 1.2180) -- (1.3040, 0.9900, 1.2213) -- (1.2550, 0.9900, 1.2198) -- cycle;
\fill[blue!23.5, opacity=0.7] (1.2550, 0.9900, 1.2198) -- (1.3040, 0.9900, 1.2213) -- (1.3040, 1.0410, 1.2243) -- (1.2550, 1.0410, 1.2228) -- cycle;
\fill[blue!99.3, opacity=0.7] (1.2550, 1.0410, 1.2228) -- (1.3040, 1.0410, 1.2243) -- (1.3040, 1.0920, 1.2270) -- (1.2550, 1.0920, 1.2255) -- cycle;
\fill[blue!68.8, opacity=0.7] (1.2550, 1.0920, 1.2255) -- (1.3040, 1.0920, 1.2270) -- (1.3040, 1.1430, 1.2294) -- (1.2550, 1.1430, 1.2279) -- cycle;
\fill[blue!32.3, opacity=0.7] (1.2550, 1.1430, 1.2279) -- (1.3040, 1.1430, 1.2294) -- (1.3040, 1.1940, 1.2315) -- (1.2550, 1.1940, 1.2300) -- cycle;
\fill[blue!48.8, opacity=0.7] (1.2550, 1.1940, 1.2300) -- (1.3040, 1.1940, 1.2315) -- (1.3040, 1.2450, 1.2333) -- (1.2550, 1.2450, 1.2318) -- cycle;
\fill[blue!73.0!black, opacity=0.7] (1.2550, 1.2450, 1.2318) -- (1.3040, 1.2450, 1.2333) -- (1.3040, 1.2960, 1.2348) -- (1.2550, 1.2960, 1.2333) -- cycle;
\fill[blue!65.8, opacity=0.7] (1.2550, 1.2960, 1.2333) -- (1.3040, 1.2960, 1.2348) -- (1.3040, 1.3470, 1.2359) -- (1.2550, 1.3470, 1.2344) -- cycle;
\fill[blue!35.2, opacity=0.7] (1.2550, 1.3470, 1.2344) -- (1.3040, 1.3470, 1.2359) -- (1.3040, 1.3980, 1.2367) -- (1.2550, 1.3980, 1.2353) -- cycle;
\fill[blue!49.3, opacity=0.7] (1.2550, 1.3980, 1.2353) -- (1.3040, 1.3980, 1.2367) -- (1.3040, 1.4490, 1.2372) -- (1.2550, 1.4490, 1.2357) -- cycle;
\fill[blue!100.0, opacity=0.7] (1.2550, 1.4490, 1.2357) -- (1.3040, 1.4490, 1.2372) -- (1.3040, 1.5000, 1.2374) -- (1.2550, 1.5000, 1.2359) -- cycle;
\fill[blue!86.3!black, opacity=0.7] (1.2550, 1.5000, 1.2359) -- (1.3040, 1.5000, 1.2374) -- (1.3040, 1.5510, 1.2372) -- (1.2550, 1.5510, 1.2357) -- cycle;
\fill[blue!80.6, opacity=0.7] (1.2550, 1.5510, 1.2357) -- (1.3040, 1.5510, 1.2372) -- (1.3040, 1.6020, 1.2367) -- (1.2550, 1.6020, 1.2353) -- cycle;
\fill[blue!77.5, opacity=0.7] (1.2550, 1.6020, 1.2353) -- (1.3040, 1.6020, 1.2367) -- (1.3040, 1.6530, 1.2359) -- (1.2550, 1.6530, 1.2344) -- cycle;
\fill[blue!96.1, opacity=0.7] (1.2550, 1.6530, 1.2344) -- (1.3040, 1.6530, 1.2359) -- (1.3040, 1.7040, 1.2348) -- (1.2550, 1.7040, 1.2333) -- cycle;
\fill[blue!68.9!black, opacity=0.7] (1.2550, 1.7040, 1.2333) -- (1.3040, 1.7040, 1.2348) -- (1.3040, 1.7550, 1.2333) -- (1.2550, 1.7550, 1.2318) -- cycle;
\fill[blue!76.6, opacity=0.7] (1.2550, 1.7550, 1.2318) -- (1.3040, 1.7550, 1.2333) -- (1.3040, 1.8060, 1.2315) -- (1.2550, 1.8060, 1.2300) -- cycle;
\fill[blue!32.0, opacity=0.7] (1.2550, 1.8060, 1.2300) -- (1.3040, 1.8060, 1.2315) -- (1.3040, 1.8570, 1.2294) -- (1.2550, 1.8570, 1.2279) -- cycle;
\fill[blue!19.1, opacity=0.7] (1.2550, 1.8570, 1.2279) -- (1.3040, 1.8570, 1.2294) -- (1.3040, 1.9080, 1.2270) -- (1.2550, 1.9080, 1.2255) -- cycle;
\fill[blue!18.1, opacity=0.7] (1.2550, 1.9080, 1.2255) -- (1.3040, 1.9080, 1.2270) -- (1.3040, 1.9590, 1.2243) -- (1.2550, 1.9590, 1.2228) -- cycle;
\fill[blue!25.5, opacity=0.7] (1.2550, 1.9590, 1.2228) -- (1.3040, 1.9590, 1.2243) -- (1.3040, 2.0100, 1.2213) -- (1.2550, 2.0100, 1.2198) -- cycle;
\fill[blue!67.5, opacity=0.7] (1.2550, 2.0100, 1.2198) -- (1.3040, 2.0100, 1.2213) -- (1.3040, 2.0610, 1.2180) -- (1.2550, 2.0610, 1.2166) -- cycle;
\fill[blue!68.5!black, opacity=0.7] (1.2550, 2.0610, 1.2166) -- (1.3040, 2.0610, 1.2180) -- (1.3040, 2.1120, 1.2145) -- (1.2550, 2.1120, 1.2130) -- cycle;
\fill[blue!96.4, opacity=0.7] (1.2550, 2.1120, 1.2130) -- (1.3040, 2.1120, 1.2145) -- (1.3040, 2.1630, 1.2106) -- (1.2550, 2.1630, 1.2092) -- cycle;
\fill[blue!88.6!black, opacity=0.7] (1.2550, 2.1630, 1.2092) -- (1.3040, 2.1630, 1.2106) -- (1.3040, 2.2140, 1.2066) -- (1.2550, 2.2140, 1.2051) -- cycle;
\fill[blue!93.8, opacity=0.7] (1.2550, 2.2140, 1.2051) -- (1.3040, 2.2140, 1.2066) -- (1.3040, 2.2650, 1.2022) -- (1.2550, 2.2650, 1.2008) -- cycle;
\fill[blue!23.1, opacity=0.7] (1.2550, 2.2650, 1.2008) -- (1.3040, 2.2650, 1.2022) -- (1.3040, 2.3160, 1.1977) -- (1.2550, 2.3160, 1.1962) -- cycle;
\fill[blue!15.1, opacity=0.7] (1.2550, 2.3160, 1.1962) -- (1.3040, 2.3160, 1.1977) -- (1.3040, 2.3670, 1.1929) -- (1.2550, 2.3670, 1.1914) -- cycle;
\fill[blue!15.0, opacity=0.7] (1.2550, 2.3670, 1.1914) -- (1.3040, 2.3670, 1.1929) -- (1.3040, 2.4180, 1.1879) -- (1.2550, 2.4180, 1.1864) -- cycle;
\fill[blue!15.0, opacity=0.7] (1.2550, 2.4180, 1.1864) -- (1.3040, 2.4180, 1.1879) -- (1.3040, 2.4690, 1.1827) -- (1.2550, 2.4690, 1.1813) -- cycle;
\fill[blue!15.4, opacity=0.7] (1.2550, 2.4690, 1.1813) -- (1.3040, 2.4690, 1.1827) -- (1.3040, 2.5200, 1.1774) -- (1.2550, 2.5200, 1.1759) -- cycle;
\fill[blue!26.0, opacity=0.7] (1.2550, 2.5200, 1.1759) -- (1.3040, 2.5200, 1.1774) -- (1.3040, 2.5710, 1.1719) -- (1.2550, 2.5710, 1.1704) -- cycle;
\fill[blue!57.6, opacity=0.7] (1.2550, 2.5710, 1.1704) -- (1.3040, 2.5710, 1.1719) -- (1.3040, 2.6220, 1.1662) -- (1.2550, 2.6220, 1.1647) -- cycle;
\fill[blue!49.3, opacity=0.7] (1.2550, 2.6220, 1.1647) -- (1.3040, 2.6220, 1.1662) -- (1.3040, 2.6730, 1.1604) -- (1.2550, 2.6730, 1.1589) -- cycle;
\fill[blue!18.3, opacity=0.7] (1.2550, 2.6730, 1.1589) -- (1.3040, 2.6730, 1.1604) -- (1.3040, 2.7240, 1.1545) -- (1.2550, 2.7240, 1.1530) -- cycle;
\fill[blue!15.0, opacity=0.7] (1.2550, 2.7240, 1.1530) -- (1.3040, 2.7240, 1.1545) -- (1.3040, 2.7750, 1.1484) -- (1.2550, 2.7750, 1.1470) -- cycle;
\fill[blue!15.0, opacity=0.7] (1.2550, 2.7750, 1.1470) -- (1.3040, 2.7750, 1.1484) -- (1.3040, 2.8260, 1.1423) -- (1.2550, 2.8260, 1.1409) -- cycle;
\fill[blue!15.0, opacity=0.7] (1.2550, 2.8260, 1.1409) -- (1.3040, 2.8260, 1.1423) -- (1.3040, 2.8770, 1.1361) -- (1.2550, 2.8770, 1.1347) -- cycle;
\fill[blue!15.0, opacity=0.7] (1.2550, 2.8770, 1.1347) -- (1.3040, 2.8770, 1.1361) -- (1.3040, 2.9280, 1.1299) -- (1.2550, 2.9280, 1.1285) -- cycle;
\fill[blue!15.0, opacity=0.7] (1.2550, 2.9280, 1.1285) -- (1.3040, 2.9280, 1.1299) -- (1.3040, 2.9790, 1.1237) -- (1.2550, 2.9790, 1.1222) -- cycle;
\fill[blue!15.2, opacity=0.7] (1.2550, 2.9790, 1.1222) -- (1.3040, 2.9790, 1.1237) -- (1.3040, 3.0300, 1.1174) -- (1.2550, 3.0300, 1.1159) -- cycle;
\fill[blue!15.0, opacity=0.7] (1.3040, -0.0300, 1.1174) -- (1.3530, -0.0300, 1.1185) -- (1.3530, 0.0210, 1.1248) -- (1.3040, 0.0210, 1.1237) -- cycle;
\fill[blue!15.0, opacity=0.7] (1.3040, 0.0210, 1.1237) -- (1.3530, 0.0210, 1.1248) -- (1.3530, 0.0720, 1.1311) -- (1.3040, 0.0720, 1.1299) -- cycle;
\fill[blue!15.0, opacity=0.7] (1.3040, 0.0720, 1.1299) -- (1.3530, 0.0720, 1.1311) -- (1.3530, 0.1230, 1.1373) -- (1.3040, 0.1230, 1.1361) -- cycle;
\fill[blue!15.0, opacity=0.7] (1.3040, 0.1230, 1.1361) -- (1.3530, 0.1230, 1.1373) -- (1.3530, 0.1740, 1.1435) -- (1.3040, 0.1740, 1.1423) -- cycle;
\fill[blue!15.0, opacity=0.7] (1.3040, 0.1740, 1.1423) -- (1.3530, 0.1740, 1.1435) -- (1.3530, 0.2250, 1.1496) -- (1.3040, 0.2250, 1.1484) -- cycle;
\fill[blue!17.6, opacity=0.7] (1.3040, 0.2250, 1.1484) -- (1.3530, 0.2250, 1.1496) -- (1.3530, 0.2760, 1.1556) -- (1.3040, 0.2760, 1.1545) -- cycle;
\fill[blue!43.4, opacity=0.7] (1.3040, 0.2760, 1.1545) -- (1.3530, 0.2760, 1.1556) -- (1.3530, 0.3270, 1.1615) -- (1.3040, 0.3270, 1.1604) -- cycle;
\fill[blue!43.6, opacity=0.7] (1.3040, 0.3270, 1.1604) -- (1.3530, 0.3270, 1.1615) -- (1.3530, 0.3780, 1.1673) -- (1.3040, 0.3780, 1.1662) -- cycle;
\fill[blue!18.6, opacity=0.7] (1.3040, 0.3780, 1.1662) -- (1.3530, 0.3780, 1.1673) -- (1.3530, 0.4290, 1.1730) -- (1.3040, 0.4290, 1.1719) -- cycle;
\fill[blue!15.0, opacity=0.7] (1.3040, 0.4290, 1.1719) -- (1.3530, 0.4290, 1.1730) -- (1.3530, 0.4800, 1.1785) -- (1.3040, 0.4800, 1.1774) -- cycle;
\fill[blue!15.0, opacity=0.7] (1.3040, 0.4800, 1.1774) -- (1.3530, 0.4800, 1.1785) -- (1.3530, 0.5310, 1.1839) -- (1.3040, 0.5310, 1.1827) -- cycle;
\fill[blue!15.0, opacity=0.7] (1.3040, 0.5310, 1.1827) -- (1.3530, 0.5310, 1.1839) -- (1.3530, 0.5820, 1.1891) -- (1.3040, 0.5820, 1.1879) -- cycle;
\fill[blue!15.5, opacity=0.7] (1.3040, 0.5820, 1.1879) -- (1.3530, 0.5820, 1.1891) -- (1.3530, 0.6330, 1.1940) -- (1.3040, 0.6330, 1.1929) -- cycle;
\fill[blue!48.1, opacity=0.7] (1.3040, 0.6330, 1.1929) -- (1.3530, 0.6330, 1.1940) -- (1.3530, 0.6840, 1.1988) -- (1.3040, 0.6840, 1.1977) -- cycle;
\fill[blue!69.0!black, opacity=0.7] (1.3040, 0.6840, 1.1977) -- (1.3530, 0.6840, 1.1988) -- (1.3530, 0.7350, 1.2034) -- (1.3040, 0.7350, 1.2022) -- cycle;
\fill[blue!78.6!black, opacity=0.7] (1.3040, 0.7350, 1.2022) -- (1.3530, 0.7350, 1.2034) -- (1.3530, 0.7860, 1.2077) -- (1.3040, 0.7860, 1.2066) -- cycle;
\fill[blue!95.5!black, opacity=0.7] (1.3040, 0.7860, 1.2066) -- (1.3530, 0.7860, 1.2077) -- (1.3530, 0.8370, 1.2118) -- (1.3040, 0.8370, 1.2106) -- cycle;
\fill[blue!33.1, opacity=0.7] (1.3040, 0.8370, 1.2106) -- (1.3530, 0.8370, 1.2118) -- (1.3530, 0.8880, 1.2156) -- (1.3040, 0.8880, 1.2145) -- cycle;
\fill[blue!16.4, opacity=0.7] (1.3040, 0.8880, 1.2145) -- (1.3530, 0.8880, 1.2156) -- (1.3530, 0.9390, 1.2192) -- (1.3040, 0.9390, 1.2180) -- cycle;
\fill[blue!16.0, opacity=0.7] (1.3040, 0.9390, 1.2180) -- (1.3530, 0.9390, 1.2192) -- (1.3530, 0.9900, 1.2224) -- (1.3040, 0.9900, 1.2213) -- cycle;
\fill[blue!29.8, opacity=0.7] (1.3040, 0.9900, 1.2213) -- (1.3530, 0.9900, 1.2224) -- (1.3530, 1.0410, 1.2254) -- (1.3040, 1.0410, 1.2243) -- cycle;
\fill[blue!72.3!black, opacity=0.7] (1.3040, 1.0410, 1.2243) -- (1.3530, 1.0410, 1.2254) -- (1.3530, 1.0920, 1.2281) -- (1.3040, 1.0920, 1.2270) -- cycle;
\fill[blue!52.3, opacity=0.7] (1.3040, 1.0920, 1.2270) -- (1.3530, 1.0920, 1.2281) -- (1.3530, 1.1430, 1.2306) -- (1.3040, 1.1430, 1.2294) -- cycle;
\fill[blue!29.8, opacity=0.7] (1.3040, 1.1430, 1.2294) -- (1.3530, 1.1430, 1.2306) -- (1.3530, 1.1940, 1.2326) -- (1.3040, 1.1940, 1.2315) -- cycle;
\fill[blue!59.2, opacity=0.7] (1.3040, 1.1940, 1.2315) -- (1.3530, 1.1940, 1.2326) -- (1.3530, 1.2450, 1.2344) -- (1.3040, 1.2450, 1.2333) -- cycle;
\fill[blue!72.9!black, opacity=0.7] (1.3040, 1.2450, 1.2333) -- (1.3530, 1.2450, 1.2344) -- (1.3530, 1.2960, 1.2359) -- (1.3040, 1.2960, 1.2348) -- cycle;
\fill[blue!50.4, opacity=0.7] (1.3040, 1.2960, 1.2348) -- (1.3530, 1.2960, 1.2359) -- (1.3530, 1.3470, 1.2370) -- (1.3040, 1.3470, 1.2359) -- cycle;
\fill[blue!44.0, opacity=0.7] (1.3040, 1.3470, 1.2359) -- (1.3530, 1.3470, 1.2370) -- (1.3530, 1.3980, 1.2379) -- (1.3040, 1.3980, 1.2367) -- cycle;
\fill[blue!93.3!black, opacity=0.7] (1.3040, 1.3980, 1.2367) -- (1.3530, 1.3980, 1.2379) -- (1.3530, 1.4490, 1.2384) -- (1.3040, 1.4490, 1.2372) -- cycle;
\fill[blue!68.1, opacity=0.7] (1.3040, 1.4490, 1.2372) -- (1.3530, 1.4490, 1.2384) -- (1.3530, 1.5000, 1.2385) -- (1.3040, 1.5000, 1.2374) -- cycle;
\fill[blue!24.7, opacity=0.7] (1.3040, 1.5000, 1.2374) -- (1.3530, 1.5000, 1.2385) -- (1.3530, 1.5510, 1.2384) -- (1.3040, 1.5510, 1.2372) -- cycle;
\fill[blue!19.3, opacity=0.7] (1.3040, 1.5510, 1.2372) -- (1.3530, 1.5510, 1.2384) -- (1.3530, 1.6020, 1.2379) -- (1.3040, 1.6020, 1.2367) -- cycle;
\fill[blue!19.9, opacity=0.7] (1.3040, 1.6020, 1.2367) -- (1.3530, 1.6020, 1.2379) -- (1.3530, 1.6530, 1.2370) -- (1.3040, 1.6530, 1.2359) -- cycle;
\fill[blue!24.6, opacity=0.7] (1.3040, 1.6530, 1.2359) -- (1.3530, 1.6530, 1.2370) -- (1.3530, 1.7040, 1.2359) -- (1.3040, 1.7040, 1.2348) -- cycle;
\fill[blue!44.3, opacity=0.7] (1.3040, 1.7040, 1.2348) -- (1.3530, 1.7040, 1.2359) -- (1.3530, 1.7550, 1.2344) -- (1.3040, 1.7550, 1.2333) -- cycle;
\fill[blue!97.2, opacity=0.7] (1.3040, 1.7550, 1.2333) -- (1.3530, 1.7550, 1.2344) -- (1.3530, 1.8060, 1.2326) -- (1.3040, 1.8060, 1.2315) -- cycle;
\fill[blue!93.2, opacity=0.7] (1.3040, 1.8060, 1.2315) -- (1.3530, 1.8060, 1.2326) -- (1.3530, 1.8570, 1.2306) -- (1.3040, 1.8570, 1.2294) -- cycle;
\fill[blue!32.2, opacity=0.7] (1.3040, 1.8570, 1.2294) -- (1.3530, 1.8570, 1.2306) -- (1.3530, 1.9080, 1.2281) -- (1.3040, 1.9080, 1.2270) -- cycle;
\fill[blue!18.1, opacity=0.7] (1.3040, 1.9080, 1.2270) -- (1.3530, 1.9080, 1.2281) -- (1.3530, 1.9590, 1.2254) -- (1.3040, 1.9590, 1.2243) -- cycle;
\fill[blue!17.9, opacity=0.7] (1.3040, 1.9590, 1.2243) -- (1.3530, 1.9590, 1.2254) -- (1.3530, 2.0100, 1.2224) -- (1.3040, 2.0100, 1.2213) -- cycle;
\fill[blue!30.6, opacity=0.7] (1.3040, 2.0100, 1.2213) -- (1.3530, 2.0100, 1.2224) -- (1.3530, 2.0610, 1.2192) -- (1.3040, 2.0610, 1.2180) -- cycle;
\fill[blue!88.2, opacity=0.7] (1.3040, 2.0610, 1.2180) -- (1.3530, 2.0610, 1.2192) -- (1.3530, 2.1120, 1.2156) -- (1.3040, 2.1120, 1.2145) -- cycle;
\fill[blue!77.0!black, opacity=0.7] (1.3040, 2.1120, 1.2145) -- (1.3530, 2.1120, 1.2156) -- (1.3530, 2.1630, 1.2118) -- (1.3040, 2.1630, 1.2106) -- cycle;
\fill[blue!97.3, opacity=0.7] (1.3040, 2.1630, 1.2106) -- (1.3530, 2.1630, 1.2118) -- (1.3530, 2.2140, 1.2077) -- (1.3040, 2.2140, 1.2066) -- cycle;
\fill[blue!69.2!black, opacity=0.7] (1.3040, 2.2140, 1.2066) -- (1.3530, 2.2140, 1.2077) -- (1.3530, 2.2650, 1.2034) -- (1.3040, 2.2650, 1.2022) -- cycle;
\fill[blue!45.3, opacity=0.7] (1.3040, 2.2650, 1.2022) -- (1.3530, 2.2650, 1.2034) -- (1.3530, 2.3160, 1.1988) -- (1.3040, 2.3160, 1.1977) -- cycle;
\fill[blue!15.6, opacity=0.7] (1.3040, 2.3160, 1.1977) -- (1.3530, 2.3160, 1.1988) -- (1.3530, 2.3670, 1.1940) -- (1.3040, 2.3670, 1.1929) -- cycle;
\fill[blue!15.0, opacity=0.7] (1.3040, 2.3670, 1.1929) -- (1.3530, 2.3670, 1.1940) -- (1.3530, 2.4180, 1.1891) -- (1.3040, 2.4180, 1.1879) -- cycle;
\fill[blue!15.0, opacity=0.7] (1.3040, 2.4180, 1.1879) -- (1.3530, 2.4180, 1.1891) -- (1.3530, 2.4690, 1.1839) -- (1.3040, 2.4690, 1.1827) -- cycle;
\fill[blue!15.1, opacity=0.7] (1.3040, 2.4690, 1.1827) -- (1.3530, 2.4690, 1.1839) -- (1.3530, 2.5200, 1.1785) -- (1.3040, 2.5200, 1.1774) -- cycle;
\fill[blue!19.1, opacity=0.7] (1.3040, 2.5200, 1.1774) -- (1.3530, 2.5200, 1.1785) -- (1.3530, 2.5710, 1.1730) -- (1.3040, 2.5710, 1.1719) -- cycle;
\fill[blue!47.6, opacity=0.7] (1.3040, 2.5710, 1.1719) -- (1.3530, 2.5710, 1.1730) -- (1.3530, 2.6220, 1.1673) -- (1.3040, 2.6220, 1.1662) -- cycle;
\fill[blue!56.3, opacity=0.7] (1.3040, 2.6220, 1.1662) -- (1.3530, 2.6220, 1.1673) -- (1.3530, 2.6730, 1.1615) -- (1.3040, 2.6730, 1.1604) -- cycle;
\fill[blue!23.6, opacity=0.7] (1.3040, 2.6730, 1.1604) -- (1.3530, 2.6730, 1.1615) -- (1.3530, 2.7240, 1.1556) -- (1.3040, 2.7240, 1.1545) -- cycle;
\fill[blue!15.1, opacity=0.7] (1.3040, 2.7240, 1.1545) -- (1.3530, 2.7240, 1.1556) -- (1.3530, 2.7750, 1.1496) -- (1.3040, 2.7750, 1.1484) -- cycle;
\fill[blue!15.0, opacity=0.7] (1.3040, 2.7750, 1.1484) -- (1.3530, 2.7750, 1.1496) -- (1.3530, 2.8260, 1.1435) -- (1.3040, 2.8260, 1.1423) -- cycle;
\fill[blue!15.0, opacity=0.7] (1.3040, 2.8260, 1.1423) -- (1.3530, 2.8260, 1.1435) -- (1.3530, 2.8770, 1.1373) -- (1.3040, 2.8770, 1.1361) -- cycle;
\fill[blue!15.0, opacity=0.7] (1.3040, 2.8770, 1.1361) -- (1.3530, 2.8770, 1.1373) -- (1.3530, 2.9280, 1.1311) -- (1.3040, 2.9280, 1.1299) -- cycle;
\fill[blue!15.0, opacity=0.7] (1.3040, 2.9280, 1.1299) -- (1.3530, 2.9280, 1.1311) -- (1.3530, 2.9790, 1.1248) -- (1.3040, 2.9790, 1.1237) -- cycle;
\fill[blue!15.1, opacity=0.7] (1.3040, 2.9790, 1.1237) -- (1.3530, 2.9790, 1.1248) -- (1.3530, 3.0300, 1.1185) -- (1.3040, 3.0300, 1.1174) -- cycle;
\fill[blue!15.0, opacity=0.7] (1.3530, -0.0300, 1.1185) -- (1.4020, -0.0300, 1.1193) -- (1.4020, 0.0210, 1.1256) -- (1.3530, 0.0210, 1.1248) -- cycle;
\fill[blue!15.0, opacity=0.7] (1.3530, 0.0210, 1.1248) -- (1.4020, 0.0210, 1.1256) -- (1.4020, 0.0720, 1.1319) -- (1.3530, 0.0720, 1.1311) -- cycle;
\fill[blue!15.0, opacity=0.7] (1.3530, 0.0720, 1.1311) -- (1.4020, 0.0720, 1.1319) -- (1.4020, 0.1230, 1.1381) -- (1.3530, 0.1230, 1.1373) -- cycle;
\fill[blue!15.0, opacity=0.7] (1.3530, 0.1230, 1.1373) -- (1.4020, 0.1230, 1.1381) -- (1.4020, 0.1740, 1.1443) -- (1.3530, 0.1740, 1.1435) -- cycle;
\fill[blue!15.0, opacity=0.7] (1.3530, 0.1740, 1.1435) -- (1.4020, 0.1740, 1.1443) -- (1.4020, 0.2250, 1.1504) -- (1.3530, 0.2250, 1.1496) -- cycle;
\fill[blue!18.0, opacity=0.7] (1.3530, 0.2250, 1.1496) -- (1.4020, 0.2250, 1.1504) -- (1.4020, 0.2760, 1.1564) -- (1.3530, 0.2760, 1.1556) -- cycle;
\fill[blue!44.9, opacity=0.7] (1.3530, 0.2760, 1.1556) -- (1.4020, 0.2760, 1.1564) -- (1.4020, 0.3270, 1.1623) -- (1.3530, 0.3270, 1.1615) -- cycle;
\fill[blue!44.3, opacity=0.7] (1.3530, 0.3270, 1.1615) -- (1.4020, 0.3270, 1.1623) -- (1.4020, 0.3780, 1.1682) -- (1.3530, 0.3780, 1.1673) -- cycle;
\fill[blue!18.6, opacity=0.7] (1.3530, 0.3780, 1.1673) -- (1.4020, 0.3780, 1.1682) -- (1.4020, 0.4290, 1.1738) -- (1.3530, 0.4290, 1.1730) -- cycle;
\fill[blue!15.0, opacity=0.7] (1.3530, 0.4290, 1.1730) -- (1.4020, 0.4290, 1.1738) -- (1.4020, 0.4800, 1.1793) -- (1.3530, 0.4800, 1.1785) -- cycle;
\fill[blue!15.0, opacity=0.7] (1.3530, 0.4800, 1.1785) -- (1.4020, 0.4800, 1.1793) -- (1.4020, 0.5310, 1.1847) -- (1.3530, 0.5310, 1.1839) -- cycle;
\fill[blue!15.0, opacity=0.7] (1.3530, 0.5310, 1.1839) -- (1.4020, 0.5310, 1.1847) -- (1.4020, 0.5820, 1.1899) -- (1.3530, 0.5820, 1.1891) -- cycle;
\fill[blue!15.6, opacity=0.7] (1.3530, 0.5820, 1.1891) -- (1.4020, 0.5820, 1.1899) -- (1.4020, 0.6330, 1.1949) -- (1.3530, 0.6330, 1.1940) -- cycle;
\fill[blue!51.6, opacity=0.7] (1.3530, 0.6330, 1.1940) -- (1.4020, 0.6330, 1.1949) -- (1.4020, 0.6840, 1.1996) -- (1.3530, 0.6840, 1.1988) -- cycle;
\fill[blue!68.4!black, opacity=0.7] (1.3530, 0.6840, 1.1988) -- (1.4020, 0.6840, 1.1996) -- (1.4020, 0.7350, 1.2042) -- (1.3530, 0.7350, 1.2034) -- cycle;
\fill[blue!80.9!black, opacity=0.7] (1.3530, 0.7350, 1.2034) -- (1.4020, 0.7350, 1.2042) -- (1.4020, 0.7860, 1.2085) -- (1.3530, 0.7860, 1.2077) -- cycle;
\fill[blue!94.8!black, opacity=0.7] (1.3530, 0.7860, 1.2077) -- (1.4020, 0.7860, 1.2085) -- (1.4020, 0.8370, 1.2126) -- (1.3530, 0.8370, 1.2118) -- cycle;
\fill[blue!33.1, opacity=0.7] (1.3530, 0.8370, 1.2118) -- (1.4020, 0.8370, 1.2126) -- (1.4020, 0.8880, 1.2164) -- (1.3530, 0.8880, 1.2156) -- cycle;
\fill[blue!16.5, opacity=0.7] (1.3530, 0.8880, 1.2156) -- (1.4020, 0.8880, 1.2164) -- (1.4020, 0.9390, 1.2200) -- (1.3530, 0.9390, 1.2192) -- cycle;
\fill[blue!16.3, opacity=0.7] (1.3530, 0.9390, 1.2192) -- (1.4020, 0.9390, 1.2200) -- (1.4020, 0.9900, 1.2233) -- (1.3530, 0.9900, 1.2224) -- cycle;
\fill[blue!33.7, opacity=0.7] (1.3530, 0.9900, 1.2224) -- (1.4020, 0.9900, 1.2233) -- (1.4020, 1.0410, 1.2263) -- (1.3530, 1.0410, 1.2254) -- cycle;
\fill[blue!68.4!black, opacity=0.7] (1.3530, 1.0410, 1.2254) -- (1.4020, 1.0410, 1.2263) -- (1.4020, 1.0920, 1.2290) -- (1.3530, 1.0920, 1.2281) -- cycle;
\fill[blue!45.6, opacity=0.7] (1.3530, 1.0920, 1.2281) -- (1.4020, 1.0920, 1.2290) -- (1.4020, 1.1430, 1.2314) -- (1.3530, 1.1430, 1.2306) -- cycle;
\fill[blue!28.0, opacity=0.7] (1.3530, 1.1430, 1.2306) -- (1.4020, 1.1430, 1.2314) -- (1.4020, 1.1940, 1.2335) -- (1.3530, 1.1940, 1.2326) -- cycle;
\fill[blue!61.4, opacity=0.7] (1.3530, 1.1940, 1.2326) -- (1.4020, 1.1940, 1.2335) -- (1.4020, 1.2450, 1.2353) -- (1.3530, 1.2450, 1.2344) -- cycle;
\fill[blue!79.1!black, opacity=0.7] (1.3530, 1.2450, 1.2344) -- (1.4020, 1.2450, 1.2353) -- (1.4020, 1.2960, 1.2367) -- (1.3530, 1.2960, 1.2359) -- cycle;
\fill[blue!49.5, opacity=0.7] (1.3530, 1.2960, 1.2359) -- (1.4020, 1.2960, 1.2367) -- (1.4020, 1.3470, 1.2379) -- (1.3530, 1.3470, 1.2370) -- cycle;
\fill[blue!61.9, opacity=0.7] (1.3530, 1.3470, 1.2370) -- (1.4020, 1.3470, 1.2379) -- (1.4020, 1.3980, 1.2387) -- (1.3530, 1.3980, 1.2379) -- cycle;
\fill[blue!99.4, opacity=0.7] (1.3530, 1.3980, 1.2379) -- (1.4020, 1.3980, 1.2387) -- (1.4020, 1.4490, 1.2392) -- (1.3530, 1.4490, 1.2384) -- cycle;
\fill[blue!19.9, opacity=0.7] (1.3530, 1.4490, 1.2384) -- (1.4020, 1.4490, 1.2392) -- (1.4020, 1.5000, 1.2393) -- (1.3530, 1.5000, 1.2385) -- cycle;
\fill[blue!16.1, opacity=0.7] (1.3530, 1.5000, 1.2385) -- (1.4020, 1.5000, 1.2393) -- (1.4020, 1.5510, 1.2392) -- (1.3530, 1.5510, 1.2384) -- cycle;
\fill[blue!17.6, opacity=0.7] (1.3530, 1.5510, 1.2384) -- (1.4020, 1.5510, 1.2392) -- (1.4020, 1.6020, 1.2387) -- (1.3530, 1.6020, 1.2379) -- cycle;
\fill[blue!18.9, opacity=0.7] (1.3530, 1.6020, 1.2379) -- (1.4020, 1.6020, 1.2387) -- (1.4020, 1.6530, 1.2379) -- (1.3530, 1.6530, 1.2370) -- cycle;
\fill[blue!18.7, opacity=0.7] (1.3530, 1.6530, 1.2370) -- (1.4020, 1.6530, 1.2379) -- (1.4020, 1.7040, 1.2367) -- (1.3530, 1.7040, 1.2359) -- cycle;
\fill[blue!20.3, opacity=0.7] (1.3530, 1.7040, 1.2359) -- (1.4020, 1.7040, 1.2367) -- (1.4020, 1.7550, 1.2353) -- (1.3530, 1.7550, 1.2344) -- cycle;
\fill[blue!34.5, opacity=0.7] (1.3530, 1.7550, 1.2344) -- (1.4020, 1.7550, 1.2353) -- (1.4020, 1.8060, 1.2335) -- (1.3530, 1.8060, 1.2326) -- cycle;
\fill[blue!95.4, opacity=0.7] (1.3530, 1.8060, 1.2326) -- (1.4020, 1.8060, 1.2335) -- (1.4020, 1.8570, 1.2314) -- (1.3530, 1.8570, 1.2306) -- cycle;
\fill[blue!81.6, opacity=0.7] (1.3530, 1.8570, 1.2306) -- (1.4020, 1.8570, 1.2314) -- (1.4020, 1.9080, 1.2290) -- (1.3530, 1.9080, 1.2281) -- cycle;
\fill[blue!24.2, opacity=0.7] (1.3530, 1.9080, 1.2281) -- (1.4020, 1.9080, 1.2290) -- (1.4020, 1.9590, 1.2263) -- (1.3530, 1.9590, 1.2254) -- cycle;
\fill[blue!17.0, opacity=0.7] (1.3530, 1.9590, 1.2254) -- (1.4020, 1.9590, 1.2263) -- (1.4020, 2.0100, 1.2233) -- (1.3530, 2.0100, 1.2224) -- cycle;
\fill[blue!19.4, opacity=0.7] (1.3530, 2.0100, 1.2224) -- (1.4020, 2.0100, 1.2233) -- (1.4020, 2.0610, 1.2200) -- (1.3530, 2.0610, 1.2192) -- cycle;
\fill[blue!49.4, opacity=0.7] (1.3530, 2.0610, 1.2192) -- (1.4020, 2.0610, 1.2200) -- (1.4020, 2.1120, 1.2164) -- (1.3530, 2.1120, 1.2156) -- cycle;
\fill[blue!74.2!black, opacity=0.7] (1.3530, 2.1120, 1.2156) -- (1.4020, 2.1120, 1.2164) -- (1.4020, 2.1630, 1.2126) -- (1.3530, 2.1630, 1.2118) -- cycle;
\fill[blue!98.9!black, opacity=0.7] (1.3530, 2.1630, 1.2118) -- (1.4020, 2.1630, 1.2126) -- (1.4020, 2.2140, 1.2085) -- (1.3530, 2.2140, 1.2077) -- cycle;
\fill[blue!76.9!black, opacity=0.7] (1.3530, 2.2140, 1.2077) -- (1.4020, 2.2140, 1.2085) -- (1.4020, 2.2650, 1.2042) -- (1.3530, 2.2650, 1.2034) -- cycle;
\fill[blue!76.7, opacity=0.7] (1.3530, 2.2650, 1.2034) -- (1.4020, 2.2650, 1.2042) -- (1.4020, 2.3160, 1.1996) -- (1.3530, 2.3160, 1.1988) -- cycle;
\fill[blue!17.9, opacity=0.7] (1.3530, 2.3160, 1.1988) -- (1.4020, 2.3160, 1.1996) -- (1.4020, 2.3670, 1.1949) -- (1.3530, 2.3670, 1.1940) -- cycle;
\fill[blue!15.0, opacity=0.7] (1.3530, 2.3670, 1.1940) -- (1.4020, 2.3670, 1.1949) -- (1.4020, 2.4180, 1.1899) -- (1.3530, 2.4180, 1.1891) -- cycle;
\fill[blue!15.0, opacity=0.7] (1.3530, 2.4180, 1.1891) -- (1.4020, 2.4180, 1.1899) -- (1.4020, 2.4690, 1.1847) -- (1.3530, 2.4690, 1.1839) -- cycle;
\fill[blue!15.0, opacity=0.7] (1.3530, 2.4690, 1.1839) -- (1.4020, 2.4690, 1.1847) -- (1.4020, 2.5200, 1.1793) -- (1.3530, 2.5200, 1.1785) -- cycle;
\fill[blue!16.4, opacity=0.7] (1.3530, 2.5200, 1.1785) -- (1.4020, 2.5200, 1.1793) -- (1.4020, 2.5710, 1.1738) -- (1.3530, 2.5710, 1.1730) -- cycle;
\fill[blue!37.1, opacity=0.7] (1.3530, 2.5710, 1.1730) -- (1.4020, 2.5710, 1.1738) -- (1.4020, 2.6220, 1.1682) -- (1.3530, 2.6220, 1.1673) -- cycle;
\fill[blue!58.2, opacity=0.7] (1.3530, 2.6220, 1.1673) -- (1.4020, 2.6220, 1.1682) -- (1.4020, 2.6730, 1.1623) -- (1.3530, 2.6730, 1.1615) -- cycle;
\fill[blue!30.4, opacity=0.7] (1.3530, 2.6730, 1.1615) -- (1.4020, 2.6730, 1.1623) -- (1.4020, 2.7240, 1.1564) -- (1.3530, 2.7240, 1.1556) -- cycle;
\fill[blue!15.3, opacity=0.7] (1.3530, 2.7240, 1.1556) -- (1.4020, 2.7240, 1.1564) -- (1.4020, 2.7750, 1.1504) -- (1.3530, 2.7750, 1.1496) -- cycle;
\fill[blue!15.0, opacity=0.7] (1.3530, 2.7750, 1.1496) -- (1.4020, 2.7750, 1.1504) -- (1.4020, 2.8260, 1.1443) -- (1.3530, 2.8260, 1.1435) -- cycle;
\fill[blue!15.0, opacity=0.7] (1.3530, 2.8260, 1.1435) -- (1.4020, 2.8260, 1.1443) -- (1.4020, 2.8770, 1.1381) -- (1.3530, 2.8770, 1.1373) -- cycle;
\fill[blue!15.0, opacity=0.7] (1.3530, 2.8770, 1.1373) -- (1.4020, 2.8770, 1.1381) -- (1.4020, 2.9280, 1.1319) -- (1.3530, 2.9280, 1.1311) -- cycle;
\fill[blue!15.0, opacity=0.7] (1.3530, 2.9280, 1.1311) -- (1.4020, 2.9280, 1.1319) -- (1.4020, 2.9790, 1.1256) -- (1.3530, 2.9790, 1.1248) -- cycle;
\fill[blue!15.0, opacity=0.7] (1.3530, 2.9790, 1.1248) -- (1.4020, 2.9790, 1.1256) -- (1.4020, 3.0300, 1.1193) -- (1.3530, 3.0300, 1.1185) -- cycle;
\fill[blue!15.0, opacity=0.7] (1.4020, -0.0300, 1.1193) -- (1.4510, -0.0300, 1.1198) -- (1.4510, 0.0210, 1.1261) -- (1.4020, 0.0210, 1.1256) -- cycle;
\fill[blue!15.0, opacity=0.7] (1.4020, 0.0210, 1.1256) -- (1.4510, 0.0210, 1.1261) -- (1.4510, 0.0720, 1.1324) -- (1.4020, 0.0720, 1.1319) -- cycle;
\fill[blue!15.0, opacity=0.7] (1.4020, 0.0720, 1.1319) -- (1.4510, 0.0720, 1.1324) -- (1.4510, 0.1230, 1.1386) -- (1.4020, 0.1230, 1.1381) -- cycle;
\fill[blue!15.0, opacity=0.7] (1.4020, 0.1230, 1.1381) -- (1.4510, 0.1230, 1.1386) -- (1.4510, 0.1740, 1.1448) -- (1.4020, 0.1740, 1.1443) -- cycle;
\fill[blue!15.0, opacity=0.7] (1.4020, 0.1740, 1.1443) -- (1.4510, 0.1740, 1.1448) -- (1.4510, 0.2250, 1.1509) -- (1.4020, 0.2250, 1.1504) -- cycle;
\fill[blue!17.8, opacity=0.7] (1.4020, 0.2250, 1.1504) -- (1.4510, 0.2250, 1.1509) -- (1.4510, 0.2760, 1.1569) -- (1.4020, 0.2760, 1.1564) -- cycle;
\fill[blue!45.2, opacity=0.7] (1.4020, 0.2760, 1.1564) -- (1.4510, 0.2760, 1.1569) -- (1.4510, 0.3270, 1.1628) -- (1.4020, 0.3270, 1.1623) -- cycle;
\fill[blue!46.2, opacity=0.7] (1.4020, 0.3270, 1.1623) -- (1.4510, 0.3270, 1.1628) -- (1.4510, 0.3780, 1.1686) -- (1.4020, 0.3780, 1.1682) -- cycle;
\fill[blue!19.3, opacity=0.7] (1.4020, 0.3780, 1.1682) -- (1.4510, 0.3780, 1.1686) -- (1.4510, 0.4290, 1.1743) -- (1.4020, 0.4290, 1.1738) -- cycle;
\fill[blue!15.1, opacity=0.7] (1.4020, 0.4290, 1.1738) -- (1.4510, 0.4290, 1.1743) -- (1.4510, 0.4800, 1.1798) -- (1.4020, 0.4800, 1.1793) -- cycle;
\fill[blue!15.0, opacity=0.7] (1.4020, 0.4800, 1.1793) -- (1.4510, 0.4800, 1.1798) -- (1.4510, 0.5310, 1.1852) -- (1.4020, 0.5310, 1.1847) -- cycle;
\fill[blue!15.0, opacity=0.7] (1.4020, 0.5310, 1.1847) -- (1.4510, 0.5310, 1.1852) -- (1.4510, 0.5820, 1.1904) -- (1.4020, 0.5820, 1.1899) -- cycle;
\fill[blue!15.6, opacity=0.7] (1.4020, 0.5820, 1.1899) -- (1.4510, 0.5820, 1.1904) -- (1.4510, 0.6330, 1.1954) -- (1.4020, 0.6330, 1.1949) -- cycle;
\fill[blue!50.0, opacity=0.7] (1.4020, 0.6330, 1.1949) -- (1.4510, 0.6330, 1.1954) -- (1.4510, 0.6840, 1.2001) -- (1.4020, 0.6840, 1.1996) -- cycle;
\fill[blue!68.4!black, opacity=0.7] (1.4020, 0.6840, 1.1996) -- (1.4510, 0.6840, 1.2001) -- (1.4510, 0.7350, 1.2047) -- (1.4020, 0.7350, 1.2042) -- cycle;
\fill[blue!84.8!black, opacity=0.7] (1.4020, 0.7350, 1.2042) -- (1.4510, 0.7350, 1.2047) -- (1.4510, 0.7860, 1.2090) -- (1.4020, 0.7860, 1.2085) -- cycle;
\fill[blue!85.5!black, opacity=0.7] (1.4020, 0.7860, 1.2085) -- (1.4510, 0.7860, 1.2090) -- (1.4510, 0.8370, 1.2131) -- (1.4020, 0.8370, 1.2126) -- cycle;
\fill[blue!37.0, opacity=0.7] (1.4020, 0.8370, 1.2126) -- (1.4510, 0.8370, 1.2131) -- (1.4510, 0.8880, 1.2169) -- (1.4020, 0.8880, 1.2164) -- cycle;
\fill[blue!16.8, opacity=0.7] (1.4020, 0.8880, 1.2164) -- (1.4510, 0.8880, 1.2169) -- (1.4510, 0.9390, 1.2205) -- (1.4020, 0.9390, 1.2200) -- cycle;
\fill[blue!16.4, opacity=0.7] (1.4020, 0.9390, 1.2200) -- (1.4510, 0.9390, 1.2205) -- (1.4510, 0.9900, 1.2238) -- (1.4020, 0.9900, 1.2233) -- cycle;
\fill[blue!32.4, opacity=0.7] (1.4020, 0.9900, 1.2233) -- (1.4510, 0.9900, 1.2238) -- (1.4510, 1.0410, 1.2268) -- (1.4020, 1.0410, 1.2263) -- cycle;
\fill[blue!69.2!black, opacity=0.7] (1.4020, 1.0410, 1.2263) -- (1.4510, 1.0410, 1.2268) -- (1.4510, 1.0920, 1.2295) -- (1.4020, 1.0920, 1.2290) -- cycle;
\fill[blue!46.2, opacity=0.7] (1.4020, 1.0920, 1.2290) -- (1.4510, 1.0920, 1.2295) -- (1.4510, 1.1430, 1.2319) -- (1.4020, 1.1430, 1.2314) -- cycle;
\fill[blue!25.9, opacity=0.7] (1.4020, 1.1430, 1.2314) -- (1.4510, 1.1430, 1.2319) -- (1.4510, 1.1940, 1.2340) -- (1.4020, 1.1940, 1.2335) -- cycle;
\fill[blue!51.2, opacity=0.7] (1.4020, 1.1940, 1.2335) -- (1.4510, 1.1940, 1.2340) -- (1.4510, 1.2450, 1.2357) -- (1.4020, 1.2450, 1.2353) -- cycle;
\fill[blue!69.0!black, opacity=0.7] (1.4020, 1.2450, 1.2353) -- (1.4510, 1.2450, 1.2357) -- (1.4510, 1.2960, 1.2372) -- (1.4020, 1.2960, 1.2367) -- cycle;
\fill[blue!57.8, opacity=0.7] (1.4020, 1.2960, 1.2367) -- (1.4510, 1.2960, 1.2372) -- (1.4510, 1.3470, 1.2384) -- (1.4020, 1.3470, 1.2379) -- cycle;
\fill[blue!68.0, opacity=0.7] (1.4020, 1.3470, 1.2379) -- (1.4510, 1.3470, 1.2384) -- (1.4510, 1.3980, 1.2392) -- (1.4020, 1.3980, 1.2387) -- cycle;
\fill[blue!78.5, opacity=0.7] (1.4020, 1.3980, 1.2387) -- (1.4510, 1.3980, 1.2392) -- (1.4510, 1.4490, 1.2397) -- (1.4020, 1.4490, 1.2392) -- cycle;
\fill[blue!15.7, opacity=0.7] (1.4020, 1.4490, 1.2392) -- (1.4510, 1.4490, 1.2397) -- (1.4510, 1.5000, 1.2398) -- (1.4020, 1.5000, 1.2393) -- cycle;
\fill[blue!24.1, opacity=0.7] (1.4020, 1.5000, 1.2393) -- (1.4510, 1.5000, 1.2398) -- (1.4510, 1.5510, 1.2397) -- (1.4020, 1.5510, 1.2392) -- cycle;
\fill[blue!73.5, opacity=0.7] (1.4020, 1.5510, 1.2392) -- (1.4510, 1.5510, 1.2397) -- (1.4510, 1.6020, 1.2392) -- (1.4020, 1.6020, 1.2387) -- cycle;
\fill[blue!70.3, opacity=0.7] (1.4020, 1.6020, 1.2387) -- (1.4510, 1.6020, 1.2392) -- (1.4510, 1.6530, 1.2384) -- (1.4020, 1.6530, 1.2379) -- cycle;
\fill[blue!39.2, opacity=0.7] (1.4020, 1.6530, 1.2379) -- (1.4510, 1.6530, 1.2384) -- (1.4510, 1.7040, 1.2372) -- (1.4020, 1.7040, 1.2367) -- cycle;
\fill[blue!22.6, opacity=0.7] (1.4020, 1.7040, 1.2367) -- (1.4510, 1.7040, 1.2372) -- (1.4510, 1.7550, 1.2357) -- (1.4020, 1.7550, 1.2353) -- cycle;
\fill[blue!21.6, opacity=0.7] (1.4020, 1.7550, 1.2353) -- (1.4510, 1.7550, 1.2357) -- (1.4510, 1.8060, 1.2340) -- (1.4020, 1.8060, 1.2335) -- cycle;
\fill[blue!43.8, opacity=0.7] (1.4020, 1.8060, 1.2335) -- (1.4510, 1.8060, 1.2340) -- (1.4510, 1.8570, 1.2319) -- (1.4020, 1.8570, 1.2314) -- cycle;
\fill[blue!69.9!black, opacity=0.7] (1.4020, 1.8570, 1.2314) -- (1.4510, 1.8570, 1.2319) -- (1.4510, 1.9080, 1.2295) -- (1.4020, 1.9080, 1.2290) -- cycle;
\fill[blue!46.2, opacity=0.7] (1.4020, 1.9080, 1.2290) -- (1.4510, 1.9080, 1.2295) -- (1.4510, 1.9590, 1.2268) -- (1.4020, 1.9590, 1.2263) -- cycle;
\fill[blue!18.0, opacity=0.7] (1.4020, 1.9590, 1.2263) -- (1.4510, 1.9590, 1.2268) -- (1.4510, 2.0100, 1.2238) -- (1.4020, 2.0100, 1.2233) -- cycle;
\fill[blue!16.9, opacity=0.7] (1.4020, 2.0100, 1.2233) -- (1.4510, 2.0100, 1.2238) -- (1.4510, 2.0610, 1.2205) -- (1.4020, 2.0610, 1.2200) -- cycle;
\fill[blue!28.1, opacity=0.7] (1.4020, 2.0610, 1.2200) -- (1.4510, 2.0610, 1.2205) -- (1.4510, 2.1120, 1.2169) -- (1.4020, 2.1120, 1.2164) -- cycle;
\fill[blue!90.0, opacity=0.7] (1.4020, 2.1120, 1.2164) -- (1.4510, 2.1120, 1.2169) -- (1.4510, 2.1630, 1.2131) -- (1.4020, 2.1630, 1.2126) -- cycle;
\fill[blue!79.4!black, opacity=0.7] (1.4020, 2.1630, 1.2126) -- (1.4510, 2.1630, 1.2131) -- (1.4510, 2.2140, 1.2090) -- (1.4020, 2.2140, 1.2085) -- cycle;
\fill[blue!89.9!black, opacity=0.7] (1.4020, 2.2140, 1.2085) -- (1.4510, 2.2140, 1.2090) -- (1.4510, 2.2650, 1.2047) -- (1.4020, 2.2650, 1.2042) -- cycle;
\fill[blue!99.3, opacity=0.7] (1.4020, 2.2650, 1.2042) -- (1.4510, 2.2650, 1.2047) -- (1.4510, 2.3160, 1.2001) -- (1.4020, 2.3160, 1.1996) -- cycle;
\fill[blue!23.9, opacity=0.7] (1.4020, 2.3160, 1.1996) -- (1.4510, 2.3160, 1.2001) -- (1.4510, 2.3670, 1.1954) -- (1.4020, 2.3670, 1.1949) -- cycle;
\fill[blue!15.1, opacity=0.7] (1.4020, 2.3670, 1.1949) -- (1.4510, 2.3670, 1.1954) -- (1.4510, 2.4180, 1.1904) -- (1.4020, 2.4180, 1.1899) -- cycle;
\fill[blue!15.0, opacity=0.7] (1.4020, 2.4180, 1.1899) -- (1.4510, 2.4180, 1.1904) -- (1.4510, 2.4690, 1.1852) -- (1.4020, 2.4690, 1.1847) -- cycle;
\fill[blue!15.0, opacity=0.7] (1.4020, 2.4690, 1.1847) -- (1.4510, 2.4690, 1.1852) -- (1.4510, 2.5200, 1.1798) -- (1.4020, 2.5200, 1.1793) -- cycle;
\fill[blue!15.5, opacity=0.7] (1.4020, 2.5200, 1.1793) -- (1.4510, 2.5200, 1.1798) -- (1.4510, 2.5710, 1.1743) -- (1.4020, 2.5710, 1.1738) -- cycle;
\fill[blue!29.0, opacity=0.7] (1.4020, 2.5710, 1.1738) -- (1.4510, 2.5710, 1.1743) -- (1.4510, 2.6220, 1.1686) -- (1.4020, 2.6220, 1.1682) -- cycle;
\fill[blue!56.4, opacity=0.7] (1.4020, 2.6220, 1.1682) -- (1.4510, 2.6220, 1.1686) -- (1.4510, 2.6730, 1.1628) -- (1.4020, 2.6730, 1.1623) -- cycle;
\fill[blue!36.8, opacity=0.7] (1.4020, 2.6730, 1.1623) -- (1.4510, 2.6730, 1.1628) -- (1.4510, 2.7240, 1.1569) -- (1.4020, 2.7240, 1.1564) -- cycle;
\fill[blue!15.7, opacity=0.7] (1.4020, 2.7240, 1.1564) -- (1.4510, 2.7240, 1.1569) -- (1.4510, 2.7750, 1.1509) -- (1.4020, 2.7750, 1.1504) -- cycle;
\fill[blue!15.0, opacity=0.7] (1.4020, 2.7750, 1.1504) -- (1.4510, 2.7750, 1.1509) -- (1.4510, 2.8260, 1.1448) -- (1.4020, 2.8260, 1.1443) -- cycle;
\fill[blue!15.0, opacity=0.7] (1.4020, 2.8260, 1.1443) -- (1.4510, 2.8260, 1.1448) -- (1.4510, 2.8770, 1.1386) -- (1.4020, 2.8770, 1.1381) -- cycle;
\fill[blue!15.0, opacity=0.7] (1.4020, 2.8770, 1.1381) -- (1.4510, 2.8770, 1.1386) -- (1.4510, 2.9280, 1.1324) -- (1.4020, 2.9280, 1.1319) -- cycle;
\fill[blue!15.0, opacity=0.7] (1.4020, 2.9280, 1.1319) -- (1.4510, 2.9280, 1.1324) -- (1.4510, 2.9790, 1.1261) -- (1.4020, 2.9790, 1.1256) -- cycle;
\fill[blue!15.0, opacity=0.7] (1.4020, 2.9790, 1.1256) -- (1.4510, 2.9790, 1.1261) -- (1.4510, 3.0300, 1.1198) -- (1.4020, 3.0300, 1.1193) -- cycle;
\fill[blue!15.0, opacity=0.7] (1.4510, -0.0300, 1.1198) -- (1.5000, -0.0300, 1.1200) -- (1.5000, 0.0210, 1.1263) -- (1.4510, 0.0210, 1.1261) -- cycle;
\fill[blue!15.0, opacity=0.7] (1.4510, 0.0210, 1.1261) -- (1.5000, 0.0210, 1.1263) -- (1.5000, 0.0720, 1.1325) -- (1.4510, 0.0720, 1.1324) -- cycle;
\fill[blue!15.0, opacity=0.7] (1.4510, 0.0720, 1.1324) -- (1.5000, 0.0720, 1.1325) -- (1.5000, 0.1230, 1.1388) -- (1.4510, 0.1230, 1.1386) -- cycle;
\fill[blue!15.0, opacity=0.7] (1.4510, 0.1230, 1.1386) -- (1.5000, 0.1230, 1.1388) -- (1.5000, 0.1740, 1.1449) -- (1.4510, 0.1740, 1.1448) -- cycle;
\fill[blue!15.0, opacity=0.7] (1.4510, 0.1740, 1.1448) -- (1.5000, 0.1740, 1.1449) -- (1.5000, 0.2250, 1.1511) -- (1.4510, 0.2250, 1.1509) -- cycle;
\fill[blue!17.2, opacity=0.7] (1.4510, 0.2250, 1.1509) -- (1.5000, 0.2250, 1.1511) -- (1.5000, 0.2760, 1.1571) -- (1.4510, 0.2760, 1.1569) -- cycle;
\fill[blue!44.2, opacity=0.7] (1.4510, 0.2760, 1.1569) -- (1.5000, 0.2760, 1.1571) -- (1.5000, 0.3270, 1.1630) -- (1.4510, 0.3270, 1.1628) -- cycle;
\fill[blue!49.2, opacity=0.7] (1.4510, 0.3270, 1.1628) -- (1.5000, 0.3270, 1.1630) -- (1.5000, 0.3780, 1.1688) -- (1.4510, 0.3780, 1.1686) -- cycle;
\fill[blue!20.8, opacity=0.7] (1.4510, 0.3780, 1.1686) -- (1.5000, 0.3780, 1.1688) -- (1.5000, 0.4290, 1.1745) -- (1.4510, 0.4290, 1.1743) -- cycle;
\fill[blue!15.1, opacity=0.7] (1.4510, 0.4290, 1.1743) -- (1.5000, 0.4290, 1.1745) -- (1.5000, 0.4800, 1.1800) -- (1.4510, 0.4800, 1.1798) -- cycle;
\fill[blue!15.0, opacity=0.7] (1.4510, 0.4800, 1.1798) -- (1.5000, 0.4800, 1.1800) -- (1.5000, 0.5310, 1.1854) -- (1.4510, 0.5310, 1.1852) -- cycle;
\fill[blue!15.0, opacity=0.7] (1.4510, 0.5310, 1.1852) -- (1.5000, 0.5310, 1.1854) -- (1.5000, 0.5820, 1.1905) -- (1.4510, 0.5820, 1.1904) -- cycle;
\fill[blue!15.4, opacity=0.7] (1.4510, 0.5820, 1.1904) -- (1.5000, 0.5820, 1.1905) -- (1.5000, 0.6330, 1.1955) -- (1.4510, 0.6330, 1.1954) -- cycle;
\fill[blue!43.3, opacity=0.7] (1.4510, 0.6330, 1.1954) -- (1.5000, 0.6330, 1.1955) -- (1.5000, 0.6840, 1.2003) -- (1.4510, 0.6840, 1.2001) -- cycle;
\fill[blue!69.4!black, opacity=0.7] (1.4510, 0.6840, 1.2001) -- (1.5000, 0.6840, 1.2003) -- (1.5000, 0.7350, 1.2049) -- (1.4510, 0.7350, 1.2047) -- cycle;
\fill[blue!89.9!black, opacity=0.7] (1.4510, 0.7350, 1.2047) -- (1.5000, 0.7350, 1.2049) -- (1.5000, 0.7860, 1.2092) -- (1.4510, 0.7860, 1.2090) -- cycle;
\fill[blue!73.2!black, opacity=0.7] (1.4510, 0.7860, 1.2090) -- (1.5000, 0.7860, 1.2092) -- (1.5000, 0.8370, 1.2133) -- (1.4510, 0.8370, 1.2131) -- cycle;
\fill[blue!46.6, opacity=0.7] (1.4510, 0.8370, 1.2131) -- (1.5000, 0.8370, 1.2133) -- (1.5000, 0.8880, 1.2171) -- (1.4510, 0.8880, 1.2169) -- cycle;
\fill[blue!17.8, opacity=0.7] (1.4510, 0.8880, 1.2169) -- (1.5000, 0.8880, 1.2171) -- (1.5000, 0.9390, 1.2206) -- (1.4510, 0.9390, 1.2205) -- cycle;
\fill[blue!16.3, opacity=0.7] (1.4510, 0.9390, 1.2205) -- (1.5000, 0.9390, 1.2206) -- (1.5000, 0.9900, 1.2239) -- (1.4510, 0.9900, 1.2238) -- cycle;
\fill[blue!27.0, opacity=0.7] (1.4510, 0.9900, 1.2238) -- (1.5000, 0.9900, 1.2239) -- (1.5000, 1.0410, 1.2269) -- (1.4510, 1.0410, 1.2268) -- cycle;
\fill[blue!84.3!black, opacity=0.7] (1.4510, 1.0410, 1.2268) -- (1.5000, 1.0410, 1.2269) -- (1.5000, 1.0920, 1.2296) -- (1.4510, 1.0920, 1.2295) -- cycle;
\fill[blue!56.2, opacity=0.7] (1.4510, 1.0920, 1.2295) -- (1.5000, 1.0920, 1.2296) -- (1.5000, 1.1430, 1.2320) -- (1.4510, 1.1430, 1.2319) -- cycle;
\fill[blue!24.6, opacity=0.7] (1.4510, 1.1430, 1.2319) -- (1.5000, 1.1430, 1.2320) -- (1.5000, 1.1940, 1.2341) -- (1.4510, 1.1940, 1.2340) -- cycle;
\fill[blue!34.5, opacity=0.7] (1.4510, 1.1940, 1.2340) -- (1.5000, 1.1940, 1.2341) -- (1.5000, 1.2450, 1.2359) -- (1.4510, 1.2450, 1.2357) -- cycle;
\fill[blue!99.7, opacity=0.7] (1.4510, 1.2450, 1.2357) -- (1.5000, 1.2450, 1.2359) -- (1.5000, 1.2960, 1.2374) -- (1.4510, 1.2960, 1.2372) -- cycle;
\fill[blue!84.5, opacity=0.7] (1.4510, 1.2960, 1.2372) -- (1.5000, 1.2960, 1.2374) -- (1.5000, 1.3470, 1.2385) -- (1.4510, 1.3470, 1.2384) -- cycle;
\fill[blue!59.8, opacity=0.7] (1.4510, 1.3470, 1.2384) -- (1.5000, 1.3470, 1.2385) -- (1.5000, 1.3980, 1.2393) -- (1.4510, 1.3980, 1.2392) -- cycle;
\fill[blue!68.4!black, opacity=0.7] (1.4510, 1.3980, 1.2392) -- (1.5000, 1.3980, 1.2393) -- (1.5000, 1.4490, 1.2398) -- (1.4510, 1.4490, 1.2397) -- cycle;
\fill[blue!15.8, opacity=0.7] (1.4510, 1.4490, 1.2397) -- (1.5000, 1.4490, 1.2398) -- (1.5000, 1.5000, 1.2400) -- (1.4510, 1.5000, 1.2398) -- cycle;
\fill[blue!97.8, opacity=0.7] (1.4510, 1.5000, 1.2398) -- (1.5000, 1.5000, 1.2400) -- (1.5000, 1.5510, 1.2398) -- (1.4510, 1.5510, 1.2397) -- cycle;
\fill[blue!75.6, opacity=0.7] (1.4510, 1.5510, 1.2397) -- (1.5000, 1.5510, 1.2398) -- (1.5000, 1.6020, 1.2393) -- (1.4510, 1.6020, 1.2392) -- cycle;
\fill[blue!91.1, opacity=0.7] (1.4510, 1.6020, 1.2392) -- (1.5000, 1.6020, 1.2393) -- (1.5000, 1.6530, 1.2385) -- (1.4510, 1.6530, 1.2384) -- cycle;
\fill[blue!74.9!black, opacity=0.7] (1.4510, 1.6530, 1.2384) -- (1.5000, 1.6530, 1.2385) -- (1.5000, 1.7040, 1.2374) -- (1.4510, 1.7040, 1.2372) -- cycle;
\fill[blue!50.2, opacity=0.7] (1.4510, 1.7040, 1.2372) -- (1.5000, 1.7040, 1.2374) -- (1.5000, 1.7550, 1.2359) -- (1.4510, 1.7550, 1.2357) -- cycle;
\fill[blue!23.5, opacity=0.7] (1.4510, 1.7550, 1.2357) -- (1.5000, 1.7550, 1.2359) -- (1.5000, 1.8060, 1.2341) -- (1.4510, 1.8060, 1.2340) -- cycle;
\fill[blue!27.2, opacity=0.7] (1.4510, 1.8060, 1.2340) -- (1.5000, 1.8060, 1.2341) -- (1.5000, 1.8570, 1.2320) -- (1.4510, 1.8570, 1.2319) -- cycle;
\fill[blue!82.1, opacity=0.7] (1.4510, 1.8570, 1.2319) -- (1.5000, 1.8570, 1.2320) -- (1.5000, 1.9080, 1.2296) -- (1.4510, 1.9080, 1.2295) -- cycle;
\fill[blue!82.4, opacity=0.7] (1.4510, 1.9080, 1.2295) -- (1.5000, 1.9080, 1.2296) -- (1.5000, 1.9590, 1.2269) -- (1.4510, 1.9590, 1.2268) -- cycle;
\fill[blue!21.3, opacity=0.7] (1.4510, 1.9590, 1.2268) -- (1.5000, 1.9590, 1.2269) -- (1.5000, 2.0100, 1.2239) -- (1.4510, 2.0100, 1.2238) -- cycle;
\fill[blue!16.4, opacity=0.7] (1.4510, 2.0100, 1.2238) -- (1.5000, 2.0100, 1.2239) -- (1.5000, 2.0610, 1.2206) -- (1.4510, 2.0610, 1.2205) -- cycle;
\fill[blue!20.4, opacity=0.7] (1.4510, 2.0610, 1.2205) -- (1.5000, 2.0610, 1.2206) -- (1.5000, 2.1120, 1.2171) -- (1.4510, 2.1120, 1.2169) -- cycle;
\fill[blue!64.6, opacity=0.7] (1.4510, 2.1120, 1.2169) -- (1.5000, 2.1120, 1.2171) -- (1.5000, 2.1630, 1.2133) -- (1.4510, 2.1630, 1.2131) -- cycle;
\fill[blue!68.6!black, opacity=0.7] (1.4510, 2.1630, 1.2131) -- (1.5000, 2.1630, 1.2133) -- (1.5000, 2.2140, 1.2092) -- (1.4510, 2.2140, 1.2090) -- cycle;
\fill[blue!93.3!black, opacity=0.7] (1.4510, 2.2140, 1.2090) -- (1.5000, 2.2140, 1.2092) -- (1.5000, 2.2650, 1.2049) -- (1.4510, 2.2650, 1.2047) -- cycle;
\fill[blue!76.3!black, opacity=0.7] (1.4510, 2.2650, 1.2047) -- (1.5000, 2.2650, 1.2049) -- (1.5000, 2.3160, 1.2003) -- (1.4510, 2.3160, 1.2001) -- cycle;
\fill[blue!33.4, opacity=0.7] (1.4510, 2.3160, 1.2001) -- (1.5000, 2.3160, 1.2003) -- (1.5000, 2.3670, 1.1955) -- (1.4510, 2.3670, 1.1954) -- cycle;
\fill[blue!15.2, opacity=0.7] (1.4510, 2.3670, 1.1954) -- (1.5000, 2.3670, 1.1955) -- (1.5000, 2.4180, 1.1905) -- (1.4510, 2.4180, 1.1904) -- cycle;
\fill[blue!15.0, opacity=0.7] (1.4510, 2.4180, 1.1904) -- (1.5000, 2.4180, 1.1905) -- (1.5000, 2.4690, 1.1854) -- (1.4510, 2.4690, 1.1852) -- cycle;
\fill[blue!15.0, opacity=0.7] (1.4510, 2.4690, 1.1852) -- (1.5000, 2.4690, 1.1854) -- (1.5000, 2.5200, 1.1800) -- (1.4510, 2.5200, 1.1798) -- cycle;
\fill[blue!15.2, opacity=0.7] (1.4510, 2.5200, 1.1798) -- (1.5000, 2.5200, 1.1800) -- (1.5000, 2.5710, 1.1745) -- (1.4510, 2.5710, 1.1743) -- cycle;
\fill[blue!23.8, opacity=0.7] (1.4510, 2.5710, 1.1743) -- (1.5000, 2.5710, 1.1745) -- (1.5000, 2.6220, 1.1688) -- (1.4510, 2.6220, 1.1686) -- cycle;
\fill[blue!53.0, opacity=0.7] (1.4510, 2.6220, 1.1686) -- (1.5000, 2.6220, 1.1688) -- (1.5000, 2.6730, 1.1630) -- (1.4510, 2.6730, 1.1628) -- cycle;
\fill[blue!41.5, opacity=0.7] (1.4510, 2.6730, 1.1628) -- (1.5000, 2.6730, 1.1630) -- (1.5000, 2.7240, 1.1571) -- (1.4510, 2.7240, 1.1569) -- cycle;
\fill[blue!16.4, opacity=0.7] (1.4510, 2.7240, 1.1569) -- (1.5000, 2.7240, 1.1571) -- (1.5000, 2.7750, 1.1511) -- (1.4510, 2.7750, 1.1509) -- cycle;
\fill[blue!15.0, opacity=0.7] (1.4510, 2.7750, 1.1509) -- (1.5000, 2.7750, 1.1511) -- (1.5000, 2.8260, 1.1449) -- (1.4510, 2.8260, 1.1448) -- cycle;
\fill[blue!15.0, opacity=0.7] (1.4510, 2.8260, 1.1448) -- (1.5000, 2.8260, 1.1449) -- (1.5000, 2.8770, 1.1388) -- (1.4510, 2.8770, 1.1386) -- cycle;
\fill[blue!15.0, opacity=0.7] (1.4510, 2.8770, 1.1386) -- (1.5000, 2.8770, 1.1388) -- (1.5000, 2.9280, 1.1325) -- (1.4510, 2.9280, 1.1324) -- cycle;
\fill[blue!15.0, opacity=0.7] (1.4510, 2.9280, 1.1324) -- (1.5000, 2.9280, 1.1325) -- (1.5000, 2.9790, 1.1263) -- (1.4510, 2.9790, 1.1261) -- cycle;
\fill[blue!15.0, opacity=0.7] (1.4510, 2.9790, 1.1261) -- (1.5000, 2.9790, 1.1263) -- (1.5000, 3.0300, 1.1200) -- (1.4510, 3.0300, 1.1198) -- cycle;
\fill[blue!15.0, opacity=0.7] (1.5000, -0.0300, 1.1200) -- (1.5490, -0.0300, 1.1198) -- (1.5490, 0.0210, 1.1261) -- (1.5000, 0.0210, 1.1263) -- cycle;
\fill[blue!15.0, opacity=0.7] (1.5000, 0.0210, 1.1263) -- (1.5490, 0.0210, 1.1261) -- (1.5490, 0.0720, 1.1324) -- (1.5000, 0.0720, 1.1325) -- cycle;
\fill[blue!15.0, opacity=0.7] (1.5000, 0.0720, 1.1325) -- (1.5490, 0.0720, 1.1324) -- (1.5490, 0.1230, 1.1386) -- (1.5000, 0.1230, 1.1388) -- cycle;
\fill[blue!15.0, opacity=0.7] (1.5000, 0.1230, 1.1388) -- (1.5490, 0.1230, 1.1386) -- (1.5490, 0.1740, 1.1448) -- (1.5000, 0.1740, 1.1449) -- cycle;
\fill[blue!15.0, opacity=0.7] (1.5000, 0.1740, 1.1449) -- (1.5490, 0.1740, 1.1448) -- (1.5490, 0.2250, 1.1509) -- (1.5000, 0.2250, 1.1511) -- cycle;
\fill[blue!16.4, opacity=0.7] (1.5000, 0.2250, 1.1511) -- (1.5490, 0.2250, 1.1509) -- (1.5490, 0.2760, 1.1569) -- (1.5000, 0.2760, 1.1571) -- cycle;
\fill[blue!41.5, opacity=0.7] (1.5000, 0.2760, 1.1571) -- (1.5490, 0.2760, 1.1569) -- (1.5490, 0.3270, 1.1628) -- (1.5000, 0.3270, 1.1630) -- cycle;
\fill[blue!53.0, opacity=0.7] (1.5000, 0.3270, 1.1630) -- (1.5490, 0.3270, 1.1628) -- (1.5490, 0.3780, 1.1686) -- (1.5000, 0.3780, 1.1688) -- cycle;
\fill[blue!23.8, opacity=0.7] (1.5000, 0.3780, 1.1688) -- (1.5490, 0.3780, 1.1686) -- (1.5490, 0.4290, 1.1743) -- (1.5000, 0.4290, 1.1745) -- cycle;
\fill[blue!15.2, opacity=0.7] (1.5000, 0.4290, 1.1745) -- (1.5490, 0.4290, 1.1743) -- (1.5490, 0.4800, 1.1798) -- (1.5000, 0.4800, 1.1800) -- cycle;
\fill[blue!15.0, opacity=0.7] (1.5000, 0.4800, 1.1800) -- (1.5490, 0.4800, 1.1798) -- (1.5490, 0.5310, 1.1852) -- (1.5000, 0.5310, 1.1854) -- cycle;
\fill[blue!15.0, opacity=0.7] (1.5000, 0.5310, 1.1854) -- (1.5490, 0.5310, 1.1852) -- (1.5490, 0.5820, 1.1904) -- (1.5000, 0.5820, 1.1905) -- cycle;
\fill[blue!15.2, opacity=0.7] (1.5000, 0.5820, 1.1905) -- (1.5490, 0.5820, 1.1904) -- (1.5490, 0.6330, 1.1954) -- (1.5000, 0.6330, 1.1955) -- cycle;
\fill[blue!33.4, opacity=0.7] (1.5000, 0.6330, 1.1955) -- (1.5490, 0.6330, 1.1954) -- (1.5490, 0.6840, 1.2001) -- (1.5000, 0.6840, 1.2003) -- cycle;
\fill[blue!76.3!black, opacity=0.7] (1.5000, 0.6840, 1.2003) -- (1.5490, 0.6840, 1.2001) -- (1.5490, 0.7350, 1.2047) -- (1.5000, 0.7350, 1.2049) -- cycle;
\fill[blue!93.3!black, opacity=0.7] (1.5000, 0.7350, 1.2049) -- (1.5490, 0.7350, 1.2047) -- (1.5490, 0.7860, 1.2090) -- (1.5000, 0.7860, 1.2092) -- cycle;
\fill[blue!68.6!black, opacity=0.7] (1.5000, 0.7860, 1.2092) -- (1.5490, 0.7860, 1.2090) -- (1.5490, 0.8370, 1.2131) -- (1.5000, 0.8370, 1.2133) -- cycle;
\fill[blue!64.6, opacity=0.7] (1.5000, 0.8370, 1.2133) -- (1.5490, 0.8370, 1.2131) -- (1.5490, 0.8880, 1.2169) -- (1.5000, 0.8880, 1.2171) -- cycle;
\fill[blue!20.4, opacity=0.7] (1.5000, 0.8880, 1.2171) -- (1.5490, 0.8880, 1.2169) -- (1.5490, 0.9390, 1.2205) -- (1.5000, 0.9390, 1.2206) -- cycle;
\fill[blue!16.4, opacity=0.7] (1.5000, 0.9390, 1.2206) -- (1.5490, 0.9390, 1.2205) -- (1.5490, 0.9900, 1.2238) -- (1.5000, 0.9900, 1.2239) -- cycle;
\fill[blue!21.3, opacity=0.7] (1.5000, 0.9900, 1.2239) -- (1.5490, 0.9900, 1.2238) -- (1.5490, 1.0410, 1.2268) -- (1.5000, 1.0410, 1.2269) -- cycle;
\fill[blue!82.4, opacity=0.7] (1.5000, 1.0410, 1.2269) -- (1.5490, 1.0410, 1.2268) -- (1.5490, 1.0920, 1.2295) -- (1.5000, 1.0920, 1.2296) -- cycle;
\fill[blue!82.1, opacity=0.7] (1.5000, 1.0920, 1.2296) -- (1.5490, 1.0920, 1.2295) -- (1.5490, 1.1430, 1.2319) -- (1.5000, 1.1430, 1.2320) -- cycle;
\fill[blue!27.2, opacity=0.7] (1.5000, 1.1430, 1.2320) -- (1.5490, 1.1430, 1.2319) -- (1.5490, 1.1940, 1.2340) -- (1.5000, 1.1940, 1.2341) -- cycle;
\fill[blue!23.5, opacity=0.7] (1.5000, 1.1940, 1.2341) -- (1.5490, 1.1940, 1.2340) -- (1.5490, 1.2450, 1.2357) -- (1.5000, 1.2450, 1.2359) -- cycle;
\fill[blue!50.2, opacity=0.7] (1.5000, 1.2450, 1.2359) -- (1.5490, 1.2450, 1.2357) -- (1.5490, 1.2960, 1.2372) -- (1.5000, 1.2960, 1.2374) -- cycle;
\fill[blue!74.9!black, opacity=0.7] (1.5000, 1.2960, 1.2374) -- (1.5490, 1.2960, 1.2372) -- (1.5490, 1.3470, 1.2384) -- (1.5000, 1.3470, 1.2385) -- cycle;
\fill[blue!91.1, opacity=0.7] (1.5000, 1.3470, 1.2385) -- (1.5490, 1.3470, 1.2384) -- (1.5490, 1.3980, 1.2392) -- (1.5000, 1.3980, 1.2393) -- cycle;
\fill[blue!75.6, opacity=0.7] (1.5000, 1.3980, 1.2393) -- (1.5490, 1.3980, 1.2392) -- (1.5490, 1.4490, 1.2397) -- (1.5000, 1.4490, 1.2398) -- cycle;
\fill[blue!97.8, opacity=0.7] (1.5000, 1.4490, 1.2398) -- (1.5490, 1.4490, 1.2397) -- (1.5490, 1.5000, 1.2398) -- (1.5000, 1.5000, 1.2400) -- cycle;
\fill[blue!15.8, opacity=0.7] (1.5000, 1.5000, 1.2400) -- (1.5490, 1.5000, 1.2398) -- (1.5490, 1.5510, 1.2397) -- (1.5000, 1.5510, 1.2398) -- cycle;
\fill[blue!68.4!black, opacity=0.7] (1.5000, 1.5510, 1.2398) -- (1.5490, 1.5510, 1.2397) -- (1.5490, 1.6020, 1.2392) -- (1.5000, 1.6020, 1.2393) -- cycle;
\fill[blue!59.8, opacity=0.7] (1.5000, 1.6020, 1.2393) -- (1.5490, 1.6020, 1.2392) -- (1.5490, 1.6530, 1.2384) -- (1.5000, 1.6530, 1.2385) -- cycle;
\fill[blue!84.5, opacity=0.7] (1.5000, 1.6530, 1.2385) -- (1.5490, 1.6530, 1.2384) -- (1.5490, 1.7040, 1.2372) -- (1.5000, 1.7040, 1.2374) -- cycle;
\fill[blue!99.7, opacity=0.7] (1.5000, 1.7040, 1.2374) -- (1.5490, 1.7040, 1.2372) -- (1.5490, 1.7550, 1.2357) -- (1.5000, 1.7550, 1.2359) -- cycle;
\fill[blue!34.5, opacity=0.7] (1.5000, 1.7550, 1.2359) -- (1.5490, 1.7550, 1.2357) -- (1.5490, 1.8060, 1.2340) -- (1.5000, 1.8060, 1.2341) -- cycle;
\fill[blue!24.6, opacity=0.7] (1.5000, 1.8060, 1.2341) -- (1.5490, 1.8060, 1.2340) -- (1.5490, 1.8570, 1.2319) -- (1.5000, 1.8570, 1.2320) -- cycle;
\fill[blue!56.2, opacity=0.7] (1.5000, 1.8570, 1.2320) -- (1.5490, 1.8570, 1.2319) -- (1.5490, 1.9080, 1.2295) -- (1.5000, 1.9080, 1.2296) -- cycle;
\fill[blue!84.3!black, opacity=0.7] (1.5000, 1.9080, 1.2296) -- (1.5490, 1.9080, 1.2295) -- (1.5490, 1.9590, 1.2268) -- (1.5000, 1.9590, 1.2269) -- cycle;
\fill[blue!27.0, opacity=0.7] (1.5000, 1.9590, 1.2269) -- (1.5490, 1.9590, 1.2268) -- (1.5490, 2.0100, 1.2238) -- (1.5000, 2.0100, 1.2239) -- cycle;
\fill[blue!16.3, opacity=0.7] (1.5000, 2.0100, 1.2239) -- (1.5490, 2.0100, 1.2238) -- (1.5490, 2.0610, 1.2205) -- (1.5000, 2.0610, 1.2206) -- cycle;
\fill[blue!17.8, opacity=0.7] (1.5000, 2.0610, 1.2206) -- (1.5490, 2.0610, 1.2205) -- (1.5490, 2.1120, 1.2169) -- (1.5000, 2.1120, 1.2171) -- cycle;
\fill[blue!46.6, opacity=0.7] (1.5000, 2.1120, 1.2171) -- (1.5490, 2.1120, 1.2169) -- (1.5490, 2.1630, 1.2131) -- (1.5000, 2.1630, 1.2133) -- cycle;
\fill[blue!73.2!black, opacity=0.7] (1.5000, 2.1630, 1.2133) -- (1.5490, 2.1630, 1.2131) -- (1.5490, 2.2140, 1.2090) -- (1.5000, 2.2140, 1.2092) -- cycle;
\fill[blue!89.9!black, opacity=0.7] (1.5000, 2.2140, 1.2092) -- (1.5490, 2.2140, 1.2090) -- (1.5490, 2.2650, 1.2047) -- (1.5000, 2.2650, 1.2049) -- cycle;
\fill[blue!69.4!black, opacity=0.7] (1.5000, 2.2650, 1.2049) -- (1.5490, 2.2650, 1.2047) -- (1.5490, 2.3160, 1.2001) -- (1.5000, 2.3160, 1.2003) -- cycle;
\fill[blue!43.3, opacity=0.7] (1.5000, 2.3160, 1.2003) -- (1.5490, 2.3160, 1.2001) -- (1.5490, 2.3670, 1.1954) -- (1.5000, 2.3670, 1.1955) -- cycle;
\fill[blue!15.4, opacity=0.7] (1.5000, 2.3670, 1.1955) -- (1.5490, 2.3670, 1.1954) -- (1.5490, 2.4180, 1.1904) -- (1.5000, 2.4180, 1.1905) -- cycle;
\fill[blue!15.0, opacity=0.7] (1.5000, 2.4180, 1.1905) -- (1.5490, 2.4180, 1.1904) -- (1.5490, 2.4690, 1.1852) -- (1.5000, 2.4690, 1.1854) -- cycle;
\fill[blue!15.0, opacity=0.7] (1.5000, 2.4690, 1.1854) -- (1.5490, 2.4690, 1.1852) -- (1.5490, 2.5200, 1.1798) -- (1.5000, 2.5200, 1.1800) -- cycle;
\fill[blue!15.1, opacity=0.7] (1.5000, 2.5200, 1.1800) -- (1.5490, 2.5200, 1.1798) -- (1.5490, 2.5710, 1.1743) -- (1.5000, 2.5710, 1.1745) -- cycle;
\fill[blue!20.8, opacity=0.7] (1.5000, 2.5710, 1.1745) -- (1.5490, 2.5710, 1.1743) -- (1.5490, 2.6220, 1.1686) -- (1.5000, 2.6220, 1.1688) -- cycle;
\fill[blue!49.2, opacity=0.7] (1.5000, 2.6220, 1.1688) -- (1.5490, 2.6220, 1.1686) -- (1.5490, 2.6730, 1.1628) -- (1.5000, 2.6730, 1.1630) -- cycle;
\fill[blue!44.2, opacity=0.7] (1.5000, 2.6730, 1.1630) -- (1.5490, 2.6730, 1.1628) -- (1.5490, 2.7240, 1.1569) -- (1.5000, 2.7240, 1.1571) -- cycle;
\fill[blue!17.2, opacity=0.7] (1.5000, 2.7240, 1.1571) -- (1.5490, 2.7240, 1.1569) -- (1.5490, 2.7750, 1.1509) -- (1.5000, 2.7750, 1.1511) -- cycle;
\fill[blue!15.0, opacity=0.7] (1.5000, 2.7750, 1.1511) -- (1.5490, 2.7750, 1.1509) -- (1.5490, 2.8260, 1.1448) -- (1.5000, 2.8260, 1.1449) -- cycle;
\fill[blue!15.0, opacity=0.7] (1.5000, 2.8260, 1.1449) -- (1.5490, 2.8260, 1.1448) -- (1.5490, 2.8770, 1.1386) -- (1.5000, 2.8770, 1.1388) -- cycle;
\fill[blue!15.0, opacity=0.7] (1.5000, 2.8770, 1.1388) -- (1.5490, 2.8770, 1.1386) -- (1.5490, 2.9280, 1.1324) -- (1.5000, 2.9280, 1.1325) -- cycle;
\fill[blue!15.0, opacity=0.7] (1.5000, 2.9280, 1.1325) -- (1.5490, 2.9280, 1.1324) -- (1.5490, 2.9790, 1.1261) -- (1.5000, 2.9790, 1.1263) -- cycle;
\fill[blue!15.0, opacity=0.7] (1.5000, 2.9790, 1.1263) -- (1.5490, 2.9790, 1.1261) -- (1.5490, 3.0300, 1.1198) -- (1.5000, 3.0300, 1.1200) -- cycle;
\fill[blue!15.0, opacity=0.7] (1.5490, -0.0300, 1.1198) -- (1.5980, -0.0300, 1.1193) -- (1.5980, 0.0210, 1.1256) -- (1.5490, 0.0210, 1.1261) -- cycle;
\fill[blue!15.0, opacity=0.7] (1.5490, 0.0210, 1.1261) -- (1.5980, 0.0210, 1.1256) -- (1.5980, 0.0720, 1.1319) -- (1.5490, 0.0720, 1.1324) -- cycle;
\fill[blue!15.0, opacity=0.7] (1.5490, 0.0720, 1.1324) -- (1.5980, 0.0720, 1.1319) -- (1.5980, 0.1230, 1.1381) -- (1.5490, 0.1230, 1.1386) -- cycle;
\fill[blue!15.0, opacity=0.7] (1.5490, 0.1230, 1.1386) -- (1.5980, 0.1230, 1.1381) -- (1.5980, 0.1740, 1.1443) -- (1.5490, 0.1740, 1.1448) -- cycle;
\fill[blue!15.0, opacity=0.7] (1.5490, 0.1740, 1.1448) -- (1.5980, 0.1740, 1.1443) -- (1.5980, 0.2250, 1.1504) -- (1.5490, 0.2250, 1.1509) -- cycle;
\fill[blue!15.7, opacity=0.7] (1.5490, 0.2250, 1.1509) -- (1.5980, 0.2250, 1.1504) -- (1.5980, 0.2760, 1.1564) -- (1.5490, 0.2760, 1.1569) -- cycle;
\fill[blue!36.8, opacity=0.7] (1.5490, 0.2760, 1.1569) -- (1.5980, 0.2760, 1.1564) -- (1.5980, 0.3270, 1.1623) -- (1.5490, 0.3270, 1.1628) -- cycle;
\fill[blue!56.4, opacity=0.7] (1.5490, 0.3270, 1.1628) -- (1.5980, 0.3270, 1.1623) -- (1.5980, 0.3780, 1.1682) -- (1.5490, 0.3780, 1.1686) -- cycle;
\fill[blue!29.0, opacity=0.7] (1.5490, 0.3780, 1.1686) -- (1.5980, 0.3780, 1.1682) -- (1.5980, 0.4290, 1.1738) -- (1.5490, 0.4290, 1.1743) -- cycle;
\fill[blue!15.5, opacity=0.7] (1.5490, 0.4290, 1.1743) -- (1.5980, 0.4290, 1.1738) -- (1.5980, 0.4800, 1.1793) -- (1.5490, 0.4800, 1.1798) -- cycle;
\fill[blue!15.0, opacity=0.7] (1.5490, 0.4800, 1.1798) -- (1.5980, 0.4800, 1.1793) -- (1.5980, 0.5310, 1.1847) -- (1.5490, 0.5310, 1.1852) -- cycle;
\fill[blue!15.0, opacity=0.7] (1.5490, 0.5310, 1.1852) -- (1.5980, 0.5310, 1.1847) -- (1.5980, 0.5820, 1.1899) -- (1.5490, 0.5820, 1.1904) -- cycle;
\fill[blue!15.1, opacity=0.7] (1.5490, 0.5820, 1.1904) -- (1.5980, 0.5820, 1.1899) -- (1.5980, 0.6330, 1.1949) -- (1.5490, 0.6330, 1.1954) -- cycle;
\fill[blue!23.9, opacity=0.7] (1.5490, 0.6330, 1.1954) -- (1.5980, 0.6330, 1.1949) -- (1.5980, 0.6840, 1.1996) -- (1.5490, 0.6840, 1.2001) -- cycle;
\fill[blue!99.3, opacity=0.7] (1.5490, 0.6840, 1.2001) -- (1.5980, 0.6840, 1.1996) -- (1.5980, 0.7350, 1.2042) -- (1.5490, 0.7350, 1.2047) -- cycle;
\fill[blue!89.9!black, opacity=0.7] (1.5490, 0.7350, 1.2047) -- (1.5980, 0.7350, 1.2042) -- (1.5980, 0.7860, 1.2085) -- (1.5490, 0.7860, 1.2090) -- cycle;
\fill[blue!79.4!black, opacity=0.7] (1.5490, 0.7860, 1.2090) -- (1.5980, 0.7860, 1.2085) -- (1.5980, 0.8370, 1.2126) -- (1.5490, 0.8370, 1.2131) -- cycle;
\fill[blue!90.0, opacity=0.7] (1.5490, 0.8370, 1.2131) -- (1.5980, 0.8370, 1.2126) -- (1.5980, 0.8880, 1.2164) -- (1.5490, 0.8880, 1.2169) -- cycle;
\fill[blue!28.1, opacity=0.7] (1.5490, 0.8880, 1.2169) -- (1.5980, 0.8880, 1.2164) -- (1.5980, 0.9390, 1.2200) -- (1.5490, 0.9390, 1.2205) -- cycle;
\fill[blue!16.9, opacity=0.7] (1.5490, 0.9390, 1.2205) -- (1.5980, 0.9390, 1.2200) -- (1.5980, 0.9900, 1.2233) -- (1.5490, 0.9900, 1.2238) -- cycle;
\fill[blue!18.0, opacity=0.7] (1.5490, 0.9900, 1.2238) -- (1.5980, 0.9900, 1.2233) -- (1.5980, 1.0410, 1.2263) -- (1.5490, 1.0410, 1.2268) -- cycle;
\fill[blue!46.2, opacity=0.7] (1.5490, 1.0410, 1.2268) -- (1.5980, 1.0410, 1.2263) -- (1.5980, 1.0920, 1.2290) -- (1.5490, 1.0920, 1.2295) -- cycle;
\fill[blue!69.9!black, opacity=0.7] (1.5490, 1.0920, 1.2295) -- (1.5980, 1.0920, 1.2290) -- (1.5980, 1.1430, 1.2314) -- (1.5490, 1.1430, 1.2319) -- cycle;
\fill[blue!43.8, opacity=0.7] (1.5490, 1.1430, 1.2319) -- (1.5980, 1.1430, 1.2314) -- (1.5980, 1.1940, 1.2335) -- (1.5490, 1.1940, 1.2340) -- cycle;
\fill[blue!21.6, opacity=0.7] (1.5490, 1.1940, 1.2340) -- (1.5980, 1.1940, 1.2335) -- (1.5980, 1.2450, 1.2353) -- (1.5490, 1.2450, 1.2357) -- cycle;
\fill[blue!22.6, opacity=0.7] (1.5490, 1.2450, 1.2357) -- (1.5980, 1.2450, 1.2353) -- (1.5980, 1.2960, 1.2367) -- (1.5490, 1.2960, 1.2372) -- cycle;
\fill[blue!39.2, opacity=0.7] (1.5490, 1.2960, 1.2372) -- (1.5980, 1.2960, 1.2367) -- (1.5980, 1.3470, 1.2379) -- (1.5490, 1.3470, 1.2384) -- cycle;
\fill[blue!70.3, opacity=0.7] (1.5490, 1.3470, 1.2384) -- (1.5980, 1.3470, 1.2379) -- (1.5980, 1.3980, 1.2387) -- (1.5490, 1.3980, 1.2392) -- cycle;
\fill[blue!73.5, opacity=0.7] (1.5490, 1.3980, 1.2392) -- (1.5980, 1.3980, 1.2387) -- (1.5980, 1.4490, 1.2392) -- (1.5490, 1.4490, 1.2397) -- cycle;
\fill[blue!24.1, opacity=0.7] (1.5490, 1.4490, 1.2397) -- (1.5980, 1.4490, 1.2392) -- (1.5980, 1.5000, 1.2393) -- (1.5490, 1.5000, 1.2398) -- cycle;
\fill[blue!15.7, opacity=0.7] (1.5490, 1.5000, 1.2398) -- (1.5980, 1.5000, 1.2393) -- (1.5980, 1.5510, 1.2392) -- (1.5490, 1.5510, 1.2397) -- cycle;
\fill[blue!78.5, opacity=0.7] (1.5490, 1.5510, 1.2397) -- (1.5980, 1.5510, 1.2392) -- (1.5980, 1.6020, 1.2387) -- (1.5490, 1.6020, 1.2392) -- cycle;
\fill[blue!68.0, opacity=0.7] (1.5490, 1.6020, 1.2392) -- (1.5980, 1.6020, 1.2387) -- (1.5980, 1.6530, 1.2379) -- (1.5490, 1.6530, 1.2384) -- cycle;
\fill[blue!57.8, opacity=0.7] (1.5490, 1.6530, 1.2384) -- (1.5980, 1.6530, 1.2379) -- (1.5980, 1.7040, 1.2367) -- (1.5490, 1.7040, 1.2372) -- cycle;
\fill[blue!69.0!black, opacity=0.7] (1.5490, 1.7040, 1.2372) -- (1.5980, 1.7040, 1.2367) -- (1.5980, 1.7550, 1.2353) -- (1.5490, 1.7550, 1.2357) -- cycle;
\fill[blue!51.2, opacity=0.7] (1.5490, 1.7550, 1.2357) -- (1.5980, 1.7550, 1.2353) -- (1.5980, 1.8060, 1.2335) -- (1.5490, 1.8060, 1.2340) -- cycle;
\fill[blue!25.9, opacity=0.7] (1.5490, 1.8060, 1.2340) -- (1.5980, 1.8060, 1.2335) -- (1.5980, 1.8570, 1.2314) -- (1.5490, 1.8570, 1.2319) -- cycle;
\fill[blue!46.2, opacity=0.7] (1.5490, 1.8570, 1.2319) -- (1.5980, 1.8570, 1.2314) -- (1.5980, 1.9080, 1.2290) -- (1.5490, 1.9080, 1.2295) -- cycle;
\fill[blue!69.2!black, opacity=0.7] (1.5490, 1.9080, 1.2295) -- (1.5980, 1.9080, 1.2290) -- (1.5980, 1.9590, 1.2263) -- (1.5490, 1.9590, 1.2268) -- cycle;
\fill[blue!32.4, opacity=0.7] (1.5490, 1.9590, 1.2268) -- (1.5980, 1.9590, 1.2263) -- (1.5980, 2.0100, 1.2233) -- (1.5490, 2.0100, 1.2238) -- cycle;
\fill[blue!16.4, opacity=0.7] (1.5490, 2.0100, 1.2238) -- (1.5980, 2.0100, 1.2233) -- (1.5980, 2.0610, 1.2200) -- (1.5490, 2.0610, 1.2205) -- cycle;
\fill[blue!16.8, opacity=0.7] (1.5490, 2.0610, 1.2205) -- (1.5980, 2.0610, 1.2200) -- (1.5980, 2.1120, 1.2164) -- (1.5490, 2.1120, 1.2169) -- cycle;
\fill[blue!37.0, opacity=0.7] (1.5490, 2.1120, 1.2169) -- (1.5980, 2.1120, 1.2164) -- (1.5980, 2.1630, 1.2126) -- (1.5490, 2.1630, 1.2131) -- cycle;
\fill[blue!85.5!black, opacity=0.7] (1.5490, 2.1630, 1.2131) -- (1.5980, 2.1630, 1.2126) -- (1.5980, 2.2140, 1.2085) -- (1.5490, 2.2140, 1.2090) -- cycle;
\fill[blue!84.8!black, opacity=0.7] (1.5490, 2.2140, 1.2090) -- (1.5980, 2.2140, 1.2085) -- (1.5980, 2.2650, 1.2042) -- (1.5490, 2.2650, 1.2047) -- cycle;
\fill[blue!68.4!black, opacity=0.7] (1.5490, 2.2650, 1.2047) -- (1.5980, 2.2650, 1.2042) -- (1.5980, 2.3160, 1.1996) -- (1.5490, 2.3160, 1.2001) -- cycle;
\fill[blue!50.0, opacity=0.7] (1.5490, 2.3160, 1.2001) -- (1.5980, 2.3160, 1.1996) -- (1.5980, 2.3670, 1.1949) -- (1.5490, 2.3670, 1.1954) -- cycle;
\fill[blue!15.6, opacity=0.7] (1.5490, 2.3670, 1.1954) -- (1.5980, 2.3670, 1.1949) -- (1.5980, 2.4180, 1.1899) -- (1.5490, 2.4180, 1.1904) -- cycle;
\fill[blue!15.0, opacity=0.7] (1.5490, 2.4180, 1.1904) -- (1.5980, 2.4180, 1.1899) -- (1.5980, 2.4690, 1.1847) -- (1.5490, 2.4690, 1.1852) -- cycle;
\fill[blue!15.0, opacity=0.7] (1.5490, 2.4690, 1.1852) -- (1.5980, 2.4690, 1.1847) -- (1.5980, 2.5200, 1.1793) -- (1.5490, 2.5200, 1.1798) -- cycle;
\fill[blue!15.1, opacity=0.7] (1.5490, 2.5200, 1.1798) -- (1.5980, 2.5200, 1.1793) -- (1.5980, 2.5710, 1.1738) -- (1.5490, 2.5710, 1.1743) -- cycle;
\fill[blue!19.3, opacity=0.7] (1.5490, 2.5710, 1.1743) -- (1.5980, 2.5710, 1.1738) -- (1.5980, 2.6220, 1.1682) -- (1.5490, 2.6220, 1.1686) -- cycle;
\fill[blue!46.2, opacity=0.7] (1.5490, 2.6220, 1.1686) -- (1.5980, 2.6220, 1.1682) -- (1.5980, 2.6730, 1.1623) -- (1.5490, 2.6730, 1.1628) -- cycle;
\fill[blue!45.2, opacity=0.7] (1.5490, 2.6730, 1.1628) -- (1.5980, 2.6730, 1.1623) -- (1.5980, 2.7240, 1.1564) -- (1.5490, 2.7240, 1.1569) -- cycle;
\fill[blue!17.8, opacity=0.7] (1.5490, 2.7240, 1.1569) -- (1.5980, 2.7240, 1.1564) -- (1.5980, 2.7750, 1.1504) -- (1.5490, 2.7750, 1.1509) -- cycle;
\fill[blue!15.0, opacity=0.7] (1.5490, 2.7750, 1.1509) -- (1.5980, 2.7750, 1.1504) -- (1.5980, 2.8260, 1.1443) -- (1.5490, 2.8260, 1.1448) -- cycle;
\fill[blue!15.0, opacity=0.7] (1.5490, 2.8260, 1.1448) -- (1.5980, 2.8260, 1.1443) -- (1.5980, 2.8770, 1.1381) -- (1.5490, 2.8770, 1.1386) -- cycle;
\fill[blue!15.0, opacity=0.7] (1.5490, 2.8770, 1.1386) -- (1.5980, 2.8770, 1.1381) -- (1.5980, 2.9280, 1.1319) -- (1.5490, 2.9280, 1.1324) -- cycle;
\fill[blue!15.0, opacity=0.7] (1.5490, 2.9280, 1.1324) -- (1.5980, 2.9280, 1.1319) -- (1.5980, 2.9790, 1.1256) -- (1.5490, 2.9790, 1.1261) -- cycle;
\fill[blue!15.0, opacity=0.7] (1.5490, 2.9790, 1.1261) -- (1.5980, 2.9790, 1.1256) -- (1.5980, 3.0300, 1.1193) -- (1.5490, 3.0300, 1.1198) -- cycle;
\fill[blue!15.0, opacity=0.7] (1.5980, -0.0300, 1.1193) -- (1.6470, -0.0300, 1.1185) -- (1.6470, 0.0210, 1.1248) -- (1.5980, 0.0210, 1.1256) -- cycle;
\fill[blue!15.0, opacity=0.7] (1.5980, 0.0210, 1.1256) -- (1.6470, 0.0210, 1.1248) -- (1.6470, 0.0720, 1.1311) -- (1.5980, 0.0720, 1.1319) -- cycle;
\fill[blue!15.0, opacity=0.7] (1.5980, 0.0720, 1.1319) -- (1.6470, 0.0720, 1.1311) -- (1.6470, 0.1230, 1.1373) -- (1.5980, 0.1230, 1.1381) -- cycle;
\fill[blue!15.0, opacity=0.7] (1.5980, 0.1230, 1.1381) -- (1.6470, 0.1230, 1.1373) -- (1.6470, 0.1740, 1.1435) -- (1.5980, 0.1740, 1.1443) -- cycle;
\fill[blue!15.0, opacity=0.7] (1.5980, 0.1740, 1.1443) -- (1.6470, 0.1740, 1.1435) -- (1.6470, 0.2250, 1.1496) -- (1.5980, 0.2250, 1.1504) -- cycle;
\fill[blue!15.3, opacity=0.7] (1.5980, 0.2250, 1.1504) -- (1.6470, 0.2250, 1.1496) -- (1.6470, 0.2760, 1.1556) -- (1.5980, 0.2760, 1.1564) -- cycle;
\fill[blue!30.4, opacity=0.7] (1.5980, 0.2760, 1.1564) -- (1.6470, 0.2760, 1.1556) -- (1.6470, 0.3270, 1.1615) -- (1.5980, 0.3270, 1.1623) -- cycle;
\fill[blue!58.2, opacity=0.7] (1.5980, 0.3270, 1.1623) -- (1.6470, 0.3270, 1.1615) -- (1.6470, 0.3780, 1.1673) -- (1.5980, 0.3780, 1.1682) -- cycle;
\fill[blue!37.1, opacity=0.7] (1.5980, 0.3780, 1.1682) -- (1.6470, 0.3780, 1.1673) -- (1.6470, 0.4290, 1.1730) -- (1.5980, 0.4290, 1.1738) -- cycle;
\fill[blue!16.4, opacity=0.7] (1.5980, 0.4290, 1.1738) -- (1.6470, 0.4290, 1.1730) -- (1.6470, 0.4800, 1.1785) -- (1.5980, 0.4800, 1.1793) -- cycle;
\fill[blue!15.0, opacity=0.7] (1.5980, 0.4800, 1.1793) -- (1.6470, 0.4800, 1.1785) -- (1.6470, 0.5310, 1.1839) -- (1.5980, 0.5310, 1.1847) -- cycle;
\fill[blue!15.0, opacity=0.7] (1.5980, 0.5310, 1.1847) -- (1.6470, 0.5310, 1.1839) -- (1.6470, 0.5820, 1.1891) -- (1.5980, 0.5820, 1.1899) -- cycle;
\fill[blue!15.0, opacity=0.7] (1.5980, 0.5820, 1.1899) -- (1.6470, 0.5820, 1.1891) -- (1.6470, 0.6330, 1.1940) -- (1.5980, 0.6330, 1.1949) -- cycle;
\fill[blue!17.9, opacity=0.7] (1.5980, 0.6330, 1.1949) -- (1.6470, 0.6330, 1.1940) -- (1.6470, 0.6840, 1.1988) -- (1.5980, 0.6840, 1.1996) -- cycle;
\fill[blue!76.7, opacity=0.7] (1.5980, 0.6840, 1.1996) -- (1.6470, 0.6840, 1.1988) -- (1.6470, 0.7350, 1.2034) -- (1.5980, 0.7350, 1.2042) -- cycle;
\fill[blue!76.9!black, opacity=0.7] (1.5980, 0.7350, 1.2042) -- (1.6470, 0.7350, 1.2034) -- (1.6470, 0.7860, 1.2077) -- (1.5980, 0.7860, 1.2085) -- cycle;
\fill[blue!98.9!black, opacity=0.7] (1.5980, 0.7860, 1.2085) -- (1.6470, 0.7860, 1.2077) -- (1.6470, 0.8370, 1.2118) -- (1.5980, 0.8370, 1.2126) -- cycle;
\fill[blue!74.2!black, opacity=0.7] (1.5980, 0.8370, 1.2126) -- (1.6470, 0.8370, 1.2118) -- (1.6470, 0.8880, 1.2156) -- (1.5980, 0.8880, 1.2164) -- cycle;
\fill[blue!49.4, opacity=0.7] (1.5980, 0.8880, 1.2164) -- (1.6470, 0.8880, 1.2156) -- (1.6470, 0.9390, 1.2192) -- (1.5980, 0.9390, 1.2200) -- cycle;
\fill[blue!19.4, opacity=0.7] (1.5980, 0.9390, 1.2200) -- (1.6470, 0.9390, 1.2192) -- (1.6470, 0.9900, 1.2224) -- (1.5980, 0.9900, 1.2233) -- cycle;
\fill[blue!17.0, opacity=0.7] (1.5980, 0.9900, 1.2233) -- (1.6470, 0.9900, 1.2224) -- (1.6470, 1.0410, 1.2254) -- (1.5980, 1.0410, 1.2263) -- cycle;
\fill[blue!24.2, opacity=0.7] (1.5980, 1.0410, 1.2263) -- (1.6470, 1.0410, 1.2254) -- (1.6470, 1.0920, 1.2281) -- (1.5980, 1.0920, 1.2290) -- cycle;
\fill[blue!81.6, opacity=0.7] (1.5980, 1.0920, 1.2290) -- (1.6470, 1.0920, 1.2281) -- (1.6470, 1.1430, 1.2306) -- (1.5980, 1.1430, 1.2314) -- cycle;
\fill[blue!95.4, opacity=0.7] (1.5980, 1.1430, 1.2314) -- (1.6470, 1.1430, 1.2306) -- (1.6470, 1.1940, 1.2326) -- (1.5980, 1.1940, 1.2335) -- cycle;
\fill[blue!34.5, opacity=0.7] (1.5980, 1.1940, 1.2335) -- (1.6470, 1.1940, 1.2326) -- (1.6470, 1.2450, 1.2344) -- (1.5980, 1.2450, 1.2353) -- cycle;
\fill[blue!20.3, opacity=0.7] (1.5980, 1.2450, 1.2353) -- (1.6470, 1.2450, 1.2344) -- (1.6470, 1.2960, 1.2359) -- (1.5980, 1.2960, 1.2367) -- cycle;
\fill[blue!18.7, opacity=0.7] (1.5980, 1.2960, 1.2367) -- (1.6470, 1.2960, 1.2359) -- (1.6470, 1.3470, 1.2370) -- (1.5980, 1.3470, 1.2379) -- cycle;
\fill[blue!18.9, opacity=0.7] (1.5980, 1.3470, 1.2379) -- (1.6470, 1.3470, 1.2370) -- (1.6470, 1.3980, 1.2379) -- (1.5980, 1.3980, 1.2387) -- cycle;
\fill[blue!17.6, opacity=0.7] (1.5980, 1.3980, 1.2387) -- (1.6470, 1.3980, 1.2379) -- (1.6470, 1.4490, 1.2384) -- (1.5980, 1.4490, 1.2392) -- cycle;
\fill[blue!16.1, opacity=0.7] (1.5980, 1.4490, 1.2392) -- (1.6470, 1.4490, 1.2384) -- (1.6470, 1.5000, 1.2385) -- (1.5980, 1.5000, 1.2393) -- cycle;
\fill[blue!19.9, opacity=0.7] (1.5980, 1.5000, 1.2393) -- (1.6470, 1.5000, 1.2385) -- (1.6470, 1.5510, 1.2384) -- (1.5980, 1.5510, 1.2392) -- cycle;
\fill[blue!99.4, opacity=0.7] (1.5980, 1.5510, 1.2392) -- (1.6470, 1.5510, 1.2384) -- (1.6470, 1.6020, 1.2379) -- (1.5980, 1.6020, 1.2387) -- cycle;
\fill[blue!61.9, opacity=0.7] (1.5980, 1.6020, 1.2387) -- (1.6470, 1.6020, 1.2379) -- (1.6470, 1.6530, 1.2370) -- (1.5980, 1.6530, 1.2379) -- cycle;
\fill[blue!49.5, opacity=0.7] (1.5980, 1.6530, 1.2379) -- (1.6470, 1.6530, 1.2370) -- (1.6470, 1.7040, 1.2359) -- (1.5980, 1.7040, 1.2367) -- cycle;
\fill[blue!79.1!black, opacity=0.7] (1.5980, 1.7040, 1.2367) -- (1.6470, 1.7040, 1.2359) -- (1.6470, 1.7550, 1.2344) -- (1.5980, 1.7550, 1.2353) -- cycle;
\fill[blue!61.4, opacity=0.7] (1.5980, 1.7550, 1.2353) -- (1.6470, 1.7550, 1.2344) -- (1.6470, 1.8060, 1.2326) -- (1.5980, 1.8060, 1.2335) -- cycle;
\fill[blue!28.0, opacity=0.7] (1.5980, 1.8060, 1.2335) -- (1.6470, 1.8060, 1.2326) -- (1.6470, 1.8570, 1.2306) -- (1.5980, 1.8570, 1.2314) -- cycle;
\fill[blue!45.6, opacity=0.7] (1.5980, 1.8570, 1.2314) -- (1.6470, 1.8570, 1.2306) -- (1.6470, 1.9080, 1.2281) -- (1.5980, 1.9080, 1.2290) -- cycle;
\fill[blue!68.4!black, opacity=0.7] (1.5980, 1.9080, 1.2290) -- (1.6470, 1.9080, 1.2281) -- (1.6470, 1.9590, 1.2254) -- (1.5980, 1.9590, 1.2263) -- cycle;
\fill[blue!33.7, opacity=0.7] (1.5980, 1.9590, 1.2263) -- (1.6470, 1.9590, 1.2254) -- (1.6470, 2.0100, 1.2224) -- (1.5980, 2.0100, 1.2233) -- cycle;
\fill[blue!16.3, opacity=0.7] (1.5980, 2.0100, 1.2233) -- (1.6470, 2.0100, 1.2224) -- (1.6470, 2.0610, 1.2192) -- (1.5980, 2.0610, 1.2200) -- cycle;
\fill[blue!16.5, opacity=0.7] (1.5980, 2.0610, 1.2200) -- (1.6470, 2.0610, 1.2192) -- (1.6470, 2.1120, 1.2156) -- (1.5980, 2.1120, 1.2164) -- cycle;
\fill[blue!33.1, opacity=0.7] (1.5980, 2.1120, 1.2164) -- (1.6470, 2.1120, 1.2156) -- (1.6470, 2.1630, 1.2118) -- (1.5980, 2.1630, 1.2126) -- cycle;
\fill[blue!94.8!black, opacity=0.7] (1.5980, 2.1630, 1.2126) -- (1.6470, 2.1630, 1.2118) -- (1.6470, 2.2140, 1.2077) -- (1.5980, 2.2140, 1.2085) -- cycle;
\fill[blue!80.9!black, opacity=0.7] (1.5980, 2.2140, 1.2085) -- (1.6470, 2.2140, 1.2077) -- (1.6470, 2.2650, 1.2034) -- (1.5980, 2.2650, 1.2042) -- cycle;
\fill[blue!68.4!black, opacity=0.7] (1.5980, 2.2650, 1.2042) -- (1.6470, 2.2650, 1.2034) -- (1.6470, 2.3160, 1.1988) -- (1.5980, 2.3160, 1.1996) -- cycle;
\fill[blue!51.6, opacity=0.7] (1.5980, 2.3160, 1.1996) -- (1.6470, 2.3160, 1.1988) -- (1.6470, 2.3670, 1.1940) -- (1.5980, 2.3670, 1.1949) -- cycle;
\fill[blue!15.6, opacity=0.7] (1.5980, 2.3670, 1.1949) -- (1.6470, 2.3670, 1.1940) -- (1.6470, 2.4180, 1.1891) -- (1.5980, 2.4180, 1.1899) -- cycle;
\fill[blue!15.0, opacity=0.7] (1.5980, 2.4180, 1.1899) -- (1.6470, 2.4180, 1.1891) -- (1.6470, 2.4690, 1.1839) -- (1.5980, 2.4690, 1.1847) -- cycle;
\fill[blue!15.0, opacity=0.7] (1.5980, 2.4690, 1.1847) -- (1.6470, 2.4690, 1.1839) -- (1.6470, 2.5200, 1.1785) -- (1.5980, 2.5200, 1.1793) -- cycle;
\fill[blue!15.0, opacity=0.7] (1.5980, 2.5200, 1.1793) -- (1.6470, 2.5200, 1.1785) -- (1.6470, 2.5710, 1.1730) -- (1.5980, 2.5710, 1.1738) -- cycle;
\fill[blue!18.6, opacity=0.7] (1.5980, 2.5710, 1.1738) -- (1.6470, 2.5710, 1.1730) -- (1.6470, 2.6220, 1.1673) -- (1.5980, 2.6220, 1.1682) -- cycle;
\fill[blue!44.3, opacity=0.7] (1.5980, 2.6220, 1.1682) -- (1.6470, 2.6220, 1.1673) -- (1.6470, 2.6730, 1.1615) -- (1.5980, 2.6730, 1.1623) -- cycle;
\fill[blue!44.9, opacity=0.7] (1.5980, 2.6730, 1.1623) -- (1.6470, 2.6730, 1.1615) -- (1.6470, 2.7240, 1.1556) -- (1.5980, 2.7240, 1.1564) -- cycle;
\fill[blue!18.0, opacity=0.7] (1.5980, 2.7240, 1.1564) -- (1.6470, 2.7240, 1.1556) -- (1.6470, 2.7750, 1.1496) -- (1.5980, 2.7750, 1.1504) -- cycle;
\fill[blue!15.0, opacity=0.7] (1.5980, 2.7750, 1.1504) -- (1.6470, 2.7750, 1.1496) -- (1.6470, 2.8260, 1.1435) -- (1.5980, 2.8260, 1.1443) -- cycle;
\fill[blue!15.0, opacity=0.7] (1.5980, 2.8260, 1.1443) -- (1.6470, 2.8260, 1.1435) -- (1.6470, 2.8770, 1.1373) -- (1.5980, 2.8770, 1.1381) -- cycle;
\fill[blue!15.0, opacity=0.7] (1.5980, 2.8770, 1.1381) -- (1.6470, 2.8770, 1.1373) -- (1.6470, 2.9280, 1.1311) -- (1.5980, 2.9280, 1.1319) -- cycle;
\fill[blue!15.0, opacity=0.7] (1.5980, 2.9280, 1.1319) -- (1.6470, 2.9280, 1.1311) -- (1.6470, 2.9790, 1.1248) -- (1.5980, 2.9790, 1.1256) -- cycle;
\fill[blue!15.0, opacity=0.7] (1.5980, 2.9790, 1.1256) -- (1.6470, 2.9790, 1.1248) -- (1.6470, 3.0300, 1.1185) -- (1.5980, 3.0300, 1.1193) -- cycle;
\fill[blue!15.1, opacity=0.7] (1.6470, -0.0300, 1.1185) -- (1.6960, -0.0300, 1.1174) -- (1.6960, 0.0210, 1.1237) -- (1.6470, 0.0210, 1.1248) -- cycle;
\fill[blue!15.0, opacity=0.7] (1.6470, 0.0210, 1.1248) -- (1.6960, 0.0210, 1.1237) -- (1.6960, 0.0720, 1.1299) -- (1.6470, 0.0720, 1.1311) -- cycle;
\fill[blue!15.0, opacity=0.7] (1.6470, 0.0720, 1.1311) -- (1.6960, 0.0720, 1.1299) -- (1.6960, 0.1230, 1.1361) -- (1.6470, 0.1230, 1.1373) -- cycle;
\fill[blue!15.0, opacity=0.7] (1.6470, 0.1230, 1.1373) -- (1.6960, 0.1230, 1.1361) -- (1.6960, 0.1740, 1.1423) -- (1.6470, 0.1740, 1.1435) -- cycle;
\fill[blue!15.0, opacity=0.7] (1.6470, 0.1740, 1.1435) -- (1.6960, 0.1740, 1.1423) -- (1.6960, 0.2250, 1.1484) -- (1.6470, 0.2250, 1.1496) -- cycle;
\fill[blue!15.1, opacity=0.7] (1.6470, 0.2250, 1.1496) -- (1.6960, 0.2250, 1.1484) -- (1.6960, 0.2760, 1.1545) -- (1.6470, 0.2760, 1.1556) -- cycle;
\fill[blue!23.6, opacity=0.7] (1.6470, 0.2760, 1.1556) -- (1.6960, 0.2760, 1.1545) -- (1.6960, 0.3270, 1.1604) -- (1.6470, 0.3270, 1.1615) -- cycle;
\fill[blue!56.3, opacity=0.7] (1.6470, 0.3270, 1.1615) -- (1.6960, 0.3270, 1.1604) -- (1.6960, 0.3780, 1.1662) -- (1.6470, 0.3780, 1.1673) -- cycle;
\fill[blue!47.6, opacity=0.7] (1.6470, 0.3780, 1.1673) -- (1.6960, 0.3780, 1.1662) -- (1.6960, 0.4290, 1.1719) -- (1.6470, 0.4290, 1.1730) -- cycle;
\fill[blue!19.1, opacity=0.7] (1.6470, 0.4290, 1.1730) -- (1.6960, 0.4290, 1.1719) -- (1.6960, 0.4800, 1.1774) -- (1.6470, 0.4800, 1.1785) -- cycle;
\fill[blue!15.1, opacity=0.7] (1.6470, 0.4800, 1.1785) -- (1.6960, 0.4800, 1.1774) -- (1.6960, 0.5310, 1.1827) -- (1.6470, 0.5310, 1.1839) -- cycle;
\fill[blue!15.0, opacity=0.7] (1.6470, 0.5310, 1.1839) -- (1.6960, 0.5310, 1.1827) -- (1.6960, 0.5820, 1.1879) -- (1.6470, 0.5820, 1.1891) -- cycle;
\fill[blue!15.0, opacity=0.7] (1.6470, 0.5820, 1.1891) -- (1.6960, 0.5820, 1.1879) -- (1.6960, 0.6330, 1.1929) -- (1.6470, 0.6330, 1.1940) -- cycle;
\fill[blue!15.6, opacity=0.7] (1.6470, 0.6330, 1.1940) -- (1.6960, 0.6330, 1.1929) -- (1.6960, 0.6840, 1.1977) -- (1.6470, 0.6840, 1.1988) -- cycle;
\fill[blue!45.3, opacity=0.7] (1.6470, 0.6840, 1.1988) -- (1.6960, 0.6840, 1.1977) -- (1.6960, 0.7350, 1.2022) -- (1.6470, 0.7350, 1.2034) -- cycle;
\fill[blue!69.2!black, opacity=0.7] (1.6470, 0.7350, 1.2034) -- (1.6960, 0.7350, 1.2022) -- (1.6960, 0.7860, 1.2066) -- (1.6470, 0.7860, 1.2077) -- cycle;
\fill[blue!97.3, opacity=0.7] (1.6470, 0.7860, 1.2077) -- (1.6960, 0.7860, 1.2066) -- (1.6960, 0.8370, 1.2106) -- (1.6470, 0.8370, 1.2118) -- cycle;
\fill[blue!77.0!black, opacity=0.7] (1.6470, 0.8370, 1.2118) -- (1.6960, 0.8370, 1.2106) -- (1.6960, 0.8880, 1.2145) -- (1.6470, 0.8880, 1.2156) -- cycle;
\fill[blue!88.2, opacity=0.7] (1.6470, 0.8880, 1.2156) -- (1.6960, 0.8880, 1.2145) -- (1.6960, 0.9390, 1.2180) -- (1.6470, 0.9390, 1.2192) -- cycle;
\fill[blue!30.6, opacity=0.7] (1.6470, 0.9390, 1.2192) -- (1.6960, 0.9390, 1.2180) -- (1.6960, 0.9900, 1.2213) -- (1.6470, 0.9900, 1.2224) -- cycle;
\fill[blue!17.9, opacity=0.7] (1.6470, 0.9900, 1.2224) -- (1.6960, 0.9900, 1.2213) -- (1.6960, 1.0410, 1.2243) -- (1.6470, 1.0410, 1.2254) -- cycle;
\fill[blue!18.1, opacity=0.7] (1.6470, 1.0410, 1.2254) -- (1.6960, 1.0410, 1.2243) -- (1.6960, 1.0920, 1.2270) -- (1.6470, 1.0920, 1.2281) -- cycle;
\fill[blue!32.2, opacity=0.7] (1.6470, 1.0920, 1.2281) -- (1.6960, 1.0920, 1.2270) -- (1.6960, 1.1430, 1.2294) -- (1.6470, 1.1430, 1.2306) -- cycle;
\fill[blue!93.2, opacity=0.7] (1.6470, 1.1430, 1.2306) -- (1.6960, 1.1430, 1.2294) -- (1.6960, 1.1940, 1.2315) -- (1.6470, 1.1940, 1.2326) -- cycle;
\fill[blue!97.2, opacity=0.7] (1.6470, 1.1940, 1.2326) -- (1.6960, 1.1940, 1.2315) -- (1.6960, 1.2450, 1.2333) -- (1.6470, 1.2450, 1.2344) -- cycle;
\fill[blue!44.3, opacity=0.7] (1.6470, 1.2450, 1.2344) -- (1.6960, 1.2450, 1.2333) -- (1.6960, 1.2960, 1.2348) -- (1.6470, 1.2960, 1.2359) -- cycle;
\fill[blue!24.6, opacity=0.7] (1.6470, 1.2960, 1.2359) -- (1.6960, 1.2960, 1.2348) -- (1.6960, 1.3470, 1.2359) -- (1.6470, 1.3470, 1.2370) -- cycle;
\fill[blue!19.9, opacity=0.7] (1.6470, 1.3470, 1.2370) -- (1.6960, 1.3470, 1.2359) -- (1.6960, 1.3980, 1.2367) -- (1.6470, 1.3980, 1.2379) -- cycle;
\fill[blue!19.3, opacity=0.7] (1.6470, 1.3980, 1.2379) -- (1.6960, 1.3980, 1.2367) -- (1.6960, 1.4490, 1.2372) -- (1.6470, 1.4490, 1.2384) -- cycle;
\fill[blue!24.7, opacity=0.7] (1.6470, 1.4490, 1.2384) -- (1.6960, 1.4490, 1.2372) -- (1.6960, 1.5000, 1.2374) -- (1.6470, 1.5000, 1.2385) -- cycle;
\fill[blue!68.1, opacity=0.7] (1.6470, 1.5000, 1.2385) -- (1.6960, 1.5000, 1.2374) -- (1.6960, 1.5510, 1.2372) -- (1.6470, 1.5510, 1.2384) -- cycle;
\fill[blue!93.3!black, opacity=0.7] (1.6470, 1.5510, 1.2384) -- (1.6960, 1.5510, 1.2372) -- (1.6960, 1.6020, 1.2367) -- (1.6470, 1.6020, 1.2379) -- cycle;
\fill[blue!44.0, opacity=0.7] (1.6470, 1.6020, 1.2379) -- (1.6960, 1.6020, 1.2367) -- (1.6960, 1.6530, 1.2359) -- (1.6470, 1.6530, 1.2370) -- cycle;
\fill[blue!50.4, opacity=0.7] (1.6470, 1.6530, 1.2370) -- (1.6960, 1.6530, 1.2359) -- (1.6960, 1.7040, 1.2348) -- (1.6470, 1.7040, 1.2359) -- cycle;
\fill[blue!72.9!black, opacity=0.7] (1.6470, 1.7040, 1.2359) -- (1.6960, 1.7040, 1.2348) -- (1.6960, 1.7550, 1.2333) -- (1.6470, 1.7550, 1.2344) -- cycle;
\fill[blue!59.2, opacity=0.7] (1.6470, 1.7550, 1.2344) -- (1.6960, 1.7550, 1.2333) -- (1.6960, 1.8060, 1.2315) -- (1.6470, 1.8060, 1.2326) -- cycle;
\fill[blue!29.8, opacity=0.7] (1.6470, 1.8060, 1.2326) -- (1.6960, 1.8060, 1.2315) -- (1.6960, 1.8570, 1.2294) -- (1.6470, 1.8570, 1.2306) -- cycle;
\fill[blue!52.3, opacity=0.7] (1.6470, 1.8570, 1.2306) -- (1.6960, 1.8570, 1.2294) -- (1.6960, 1.9080, 1.2270) -- (1.6470, 1.9080, 1.2281) -- cycle;
\fill[blue!72.3!black, opacity=0.7] (1.6470, 1.9080, 1.2281) -- (1.6960, 1.9080, 1.2270) -- (1.6960, 1.9590, 1.2243) -- (1.6470, 1.9590, 1.2254) -- cycle;
\fill[blue!29.8, opacity=0.7] (1.6470, 1.9590, 1.2254) -- (1.6960, 1.9590, 1.2243) -- (1.6960, 2.0100, 1.2213) -- (1.6470, 2.0100, 1.2224) -- cycle;
\fill[blue!16.0, opacity=0.7] (1.6470, 2.0100, 1.2224) -- (1.6960, 2.0100, 1.2213) -- (1.6960, 2.0610, 1.2180) -- (1.6470, 2.0610, 1.2192) -- cycle;
\fill[blue!16.4, opacity=0.7] (1.6470, 2.0610, 1.2192) -- (1.6960, 2.0610, 1.2180) -- (1.6960, 2.1120, 1.2145) -- (1.6470, 2.1120, 1.2156) -- cycle;
\fill[blue!33.1, opacity=0.7] (1.6470, 2.1120, 1.2156) -- (1.6960, 2.1120, 1.2145) -- (1.6960, 2.1630, 1.2106) -- (1.6470, 2.1630, 1.2118) -- cycle;
\fill[blue!95.5!black, opacity=0.7] (1.6470, 2.1630, 1.2118) -- (1.6960, 2.1630, 1.2106) -- (1.6960, 2.2140, 1.2066) -- (1.6470, 2.2140, 1.2077) -- cycle;
\fill[blue!78.6!black, opacity=0.7] (1.6470, 2.2140, 1.2077) -- (1.6960, 2.2140, 1.2066) -- (1.6960, 2.2650, 1.2022) -- (1.6470, 2.2650, 1.2034) -- cycle;
\fill[blue!69.0!black, opacity=0.7] (1.6470, 2.2650, 1.2034) -- (1.6960, 2.2650, 1.2022) -- (1.6960, 2.3160, 1.1977) -- (1.6470, 2.3160, 1.1988) -- cycle;
\fill[blue!48.1, opacity=0.7] (1.6470, 2.3160, 1.1988) -- (1.6960, 2.3160, 1.1977) -- (1.6960, 2.3670, 1.1929) -- (1.6470, 2.3670, 1.1940) -- cycle;
\fill[blue!15.5, opacity=0.7] (1.6470, 2.3670, 1.1940) -- (1.6960, 2.3670, 1.1929) -- (1.6960, 2.4180, 1.1879) -- (1.6470, 2.4180, 1.1891) -- cycle;
\fill[blue!15.0, opacity=0.7] (1.6470, 2.4180, 1.1891) -- (1.6960, 2.4180, 1.1879) -- (1.6960, 2.4690, 1.1827) -- (1.6470, 2.4690, 1.1839) -- cycle;
\fill[blue!15.0, opacity=0.7] (1.6470, 2.4690, 1.1839) -- (1.6960, 2.4690, 1.1827) -- (1.6960, 2.5200, 1.1774) -- (1.6470, 2.5200, 1.1785) -- cycle;
\fill[blue!15.0, opacity=0.7] (1.6470, 2.5200, 1.1785) -- (1.6960, 2.5200, 1.1774) -- (1.6960, 2.5710, 1.1719) -- (1.6470, 2.5710, 1.1730) -- cycle;
\fill[blue!18.6, opacity=0.7] (1.6470, 2.5710, 1.1730) -- (1.6960, 2.5710, 1.1719) -- (1.6960, 2.6220, 1.1662) -- (1.6470, 2.6220, 1.1673) -- cycle;
\fill[blue!43.6, opacity=0.7] (1.6470, 2.6220, 1.1673) -- (1.6960, 2.6220, 1.1662) -- (1.6960, 2.6730, 1.1604) -- (1.6470, 2.6730, 1.1615) -- cycle;
\fill[blue!43.4, opacity=0.7] (1.6470, 2.6730, 1.1615) -- (1.6960, 2.6730, 1.1604) -- (1.6960, 2.7240, 1.1545) -- (1.6470, 2.7240, 1.1556) -- cycle;
\fill[blue!17.6, opacity=0.7] (1.6470, 2.7240, 1.1556) -- (1.6960, 2.7240, 1.1545) -- (1.6960, 2.7750, 1.1484) -- (1.6470, 2.7750, 1.1496) -- cycle;
\fill[blue!15.0, opacity=0.7] (1.6470, 2.7750, 1.1496) -- (1.6960, 2.7750, 1.1484) -- (1.6960, 2.8260, 1.1423) -- (1.6470, 2.8260, 1.1435) -- cycle;
\fill[blue!15.0, opacity=0.7] (1.6470, 2.8260, 1.1435) -- (1.6960, 2.8260, 1.1423) -- (1.6960, 2.8770, 1.1361) -- (1.6470, 2.8770, 1.1373) -- cycle;
\fill[blue!15.0, opacity=0.7] (1.6470, 2.8770, 1.1373) -- (1.6960, 2.8770, 1.1361) -- (1.6960, 2.9280, 1.1299) -- (1.6470, 2.9280, 1.1311) -- cycle;
\fill[blue!15.0, opacity=0.7] (1.6470, 2.9280, 1.1311) -- (1.6960, 2.9280, 1.1299) -- (1.6960, 2.9790, 1.1237) -- (1.6470, 2.9790, 1.1248) -- cycle;
\fill[blue!15.0, opacity=0.7] (1.6470, 2.9790, 1.1248) -- (1.6960, 2.9790, 1.1237) -- (1.6960, 3.0300, 1.1174) -- (1.6470, 3.0300, 1.1185) -- cycle;
\fill[blue!15.2, opacity=0.7] (1.6960, -0.0300, 1.1174) -- (1.7450, -0.0300, 1.1159) -- (1.7450, 0.0210, 1.1222) -- (1.6960, 0.0210, 1.1237) -- cycle;
\fill[blue!15.0, opacity=0.7] (1.6960, 0.0210, 1.1237) -- (1.7450, 0.0210, 1.1222) -- (1.7450, 0.0720, 1.1285) -- (1.6960, 0.0720, 1.1299) -- cycle;
\fill[blue!15.0, opacity=0.7] (1.6960, 0.0720, 1.1299) -- (1.7450, 0.0720, 1.1285) -- (1.7450, 0.1230, 1.1347) -- (1.6960, 0.1230, 1.1361) -- cycle;
\fill[blue!15.0, opacity=0.7] (1.6960, 0.1230, 1.1361) -- (1.7450, 0.1230, 1.1347) -- (1.7450, 0.1740, 1.1409) -- (1.6960, 0.1740, 1.1423) -- cycle;
\fill[blue!15.0, opacity=0.7] (1.6960, 0.1740, 1.1423) -- (1.7450, 0.1740, 1.1409) -- (1.7450, 0.2250, 1.1470) -- (1.6960, 0.2250, 1.1484) -- cycle;
\fill[blue!15.0, opacity=0.7] (1.6960, 0.2250, 1.1484) -- (1.7450, 0.2250, 1.1470) -- (1.7450, 0.2760, 1.1530) -- (1.6960, 0.2760, 1.1545) -- cycle;
\fill[blue!18.3, opacity=0.7] (1.6960, 0.2760, 1.1545) -- (1.7450, 0.2760, 1.1530) -- (1.7450, 0.3270, 1.1589) -- (1.6960, 0.3270, 1.1604) -- cycle;
\fill[blue!49.3, opacity=0.7] (1.6960, 0.3270, 1.1604) -- (1.7450, 0.3270, 1.1589) -- (1.7450, 0.3780, 1.1647) -- (1.6960, 0.3780, 1.1662) -- cycle;
\fill[blue!57.6, opacity=0.7] (1.6960, 0.3780, 1.1662) -- (1.7450, 0.3780, 1.1647) -- (1.7450, 0.4290, 1.1704) -- (1.6960, 0.4290, 1.1719) -- cycle;
\fill[blue!26.0, opacity=0.7] (1.6960, 0.4290, 1.1719) -- (1.7450, 0.4290, 1.1704) -- (1.7450, 0.4800, 1.1759) -- (1.6960, 0.4800, 1.1774) -- cycle;
\fill[blue!15.4, opacity=0.7] (1.6960, 0.4800, 1.1774) -- (1.7450, 0.4800, 1.1759) -- (1.7450, 0.5310, 1.1813) -- (1.6960, 0.5310, 1.1827) -- cycle;
\fill[blue!15.0, opacity=0.7] (1.6960, 0.5310, 1.1827) -- (1.7450, 0.5310, 1.1813) -- (1.7450, 0.5820, 1.1864) -- (1.6960, 0.5820, 1.1879) -- cycle;
\fill[blue!15.0, opacity=0.7] (1.6960, 0.5820, 1.1879) -- (1.7450, 0.5820, 1.1864) -- (1.7450, 0.6330, 1.1914) -- (1.6960, 0.6330, 1.1929) -- cycle;
\fill[blue!15.1, opacity=0.7] (1.6960, 0.6330, 1.1929) -- (1.7450, 0.6330, 1.1914) -- (1.7450, 0.6840, 1.1962) -- (1.6960, 0.6840, 1.1977) -- cycle;
\fill[blue!23.1, opacity=0.7] (1.6960, 0.6840, 1.1977) -- (1.7450, 0.6840, 1.1962) -- (1.7450, 0.7350, 1.2008) -- (1.6960, 0.7350, 1.2022) -- cycle;
\fill[blue!93.8, opacity=0.7] (1.6960, 0.7350, 1.2022) -- (1.7450, 0.7350, 1.2008) -- (1.7450, 0.7860, 1.2051) -- (1.6960, 0.7860, 1.2066) -- cycle;
\fill[blue!88.6!black, opacity=0.7] (1.6960, 0.7860, 1.2066) -- (1.7450, 0.7860, 1.2051) -- (1.7450, 0.8370, 1.2092) -- (1.6960, 0.8370, 1.2106) -- cycle;
\fill[blue!96.4, opacity=0.7] (1.6960, 0.8370, 1.2106) -- (1.7450, 0.8370, 1.2092) -- (1.7450, 0.8880, 1.2130) -- (1.6960, 0.8880, 1.2145) -- cycle;
\fill[blue!68.5!black, opacity=0.7] (1.6960, 0.8880, 1.2145) -- (1.7450, 0.8880, 1.2130) -- (1.7450, 0.9390, 1.2166) -- (1.6960, 0.9390, 1.2180) -- cycle;
\fill[blue!67.5, opacity=0.7] (1.6960, 0.9390, 1.2180) -- (1.7450, 0.9390, 1.2166) -- (1.7450, 0.9900, 1.2198) -- (1.6960, 0.9900, 1.2213) -- cycle;
\fill[blue!25.5, opacity=0.7] (1.6960, 0.9900, 1.2213) -- (1.7450, 0.9900, 1.2198) -- (1.7450, 1.0410, 1.2228) -- (1.6960, 1.0410, 1.2243) -- cycle;
\fill[blue!18.1, opacity=0.7] (1.6960, 1.0410, 1.2243) -- (1.7450, 1.0410, 1.2228) -- (1.7450, 1.0920, 1.2255) -- (1.6960, 1.0920, 1.2270) -- cycle;
\fill[blue!19.1, opacity=0.7] (1.6960, 1.0920, 1.2270) -- (1.7450, 1.0920, 1.2255) -- (1.7450, 1.1430, 1.2279) -- (1.6960, 1.1430, 1.2294) -- cycle;
\fill[blue!32.0, opacity=0.7] (1.6960, 1.1430, 1.2294) -- (1.7450, 1.1430, 1.2279) -- (1.7450, 1.1940, 1.2300) -- (1.6960, 1.1940, 1.2315) -- cycle;
\fill[blue!76.6, opacity=0.7] (1.6960, 1.1940, 1.2315) -- (1.7450, 1.1940, 1.2300) -- (1.7450, 1.2450, 1.2318) -- (1.6960, 1.2450, 1.2333) -- cycle;
\fill[blue!68.9!black, opacity=0.7] (1.6960, 1.2450, 1.2333) -- (1.7450, 1.2450, 1.2318) -- (1.7450, 1.2960, 1.2333) -- (1.6960, 1.2960, 1.2348) -- cycle;
\fill[blue!96.1, opacity=0.7] (1.6960, 1.2960, 1.2348) -- (1.7450, 1.2960, 1.2333) -- (1.7450, 1.3470, 1.2344) -- (1.6960, 1.3470, 1.2359) -- cycle;
\fill[blue!77.5, opacity=0.7] (1.6960, 1.3470, 1.2359) -- (1.7450, 1.3470, 1.2344) -- (1.7450, 1.3980, 1.2353) -- (1.6960, 1.3980, 1.2367) -- cycle;
\fill[blue!80.6, opacity=0.7] (1.6960, 1.3980, 1.2367) -- (1.7450, 1.3980, 1.2353) -- (1.7450, 1.4490, 1.2357) -- (1.6960, 1.4490, 1.2372) -- cycle;
\fill[blue!86.3!black, opacity=0.7] (1.6960, 1.4490, 1.2372) -- (1.7450, 1.4490, 1.2357) -- (1.7450, 1.5000, 1.2359) -- (1.6960, 1.5000, 1.2374) -- cycle;
\fill[blue!100.0, opacity=0.7] (1.6960, 1.5000, 1.2374) -- (1.7450, 1.5000, 1.2359) -- (1.7450, 1.5510, 1.2357) -- (1.6960, 1.5510, 1.2372) -- cycle;
\fill[blue!49.3, opacity=0.7] (1.6960, 1.5510, 1.2372) -- (1.7450, 1.5510, 1.2357) -- (1.7450, 1.6020, 1.2353) -- (1.6960, 1.6020, 1.2367) -- cycle;
\fill[blue!35.2, opacity=0.7] (1.6960, 1.6020, 1.2367) -- (1.7450, 1.6020, 1.2353) -- (1.7450, 1.6530, 1.2344) -- (1.6960, 1.6530, 1.2359) -- cycle;
\fill[blue!65.8, opacity=0.7] (1.6960, 1.6530, 1.2359) -- (1.7450, 1.6530, 1.2344) -- (1.7450, 1.7040, 1.2333) -- (1.6960, 1.7040, 1.2348) -- cycle;
\fill[blue!73.0!black, opacity=0.7] (1.6960, 1.7040, 1.2348) -- (1.7450, 1.7040, 1.2333) -- (1.7450, 1.7550, 1.2318) -- (1.6960, 1.7550, 1.2333) -- cycle;
\fill[blue!48.8, opacity=0.7] (1.6960, 1.7550, 1.2333) -- (1.7450, 1.7550, 1.2318) -- (1.7450, 1.8060, 1.2300) -- (1.6960, 1.8060, 1.2315) -- cycle;
\fill[blue!32.3, opacity=0.7] (1.6960, 1.8060, 1.2315) -- (1.7450, 1.8060, 1.2300) -- (1.7450, 1.8570, 1.2279) -- (1.6960, 1.8570, 1.2294) -- cycle;
\fill[blue!68.8, opacity=0.7] (1.6960, 1.8570, 1.2294) -- (1.7450, 1.8570, 1.2279) -- (1.7450, 1.9080, 1.2255) -- (1.6960, 1.9080, 1.2270) -- cycle;
\fill[blue!99.3, opacity=0.7] (1.6960, 1.9080, 1.2270) -- (1.7450, 1.9080, 1.2255) -- (1.7450, 1.9590, 1.2228) -- (1.6960, 1.9590, 1.2243) -- cycle;
\fill[blue!23.5, opacity=0.7] (1.6960, 1.9590, 1.2243) -- (1.7450, 1.9590, 1.2228) -- (1.7450, 2.0100, 1.2198) -- (1.6960, 2.0100, 1.2213) -- cycle;
\fill[blue!15.7, opacity=0.7] (1.6960, 2.0100, 1.2213) -- (1.7450, 2.0100, 1.2198) -- (1.7450, 2.0610, 1.2166) -- (1.6960, 2.0610, 1.2180) -- cycle;
\fill[blue!16.5, opacity=0.7] (1.6960, 2.0610, 1.2180) -- (1.7450, 2.0610, 1.2166) -- (1.7450, 2.1120, 1.2130) -- (1.6960, 2.1120, 1.2145) -- cycle;
\fill[blue!36.9, opacity=0.7] (1.6960, 2.1120, 1.2145) -- (1.7450, 2.1120, 1.2130) -- (1.7450, 2.1630, 1.2092) -- (1.6960, 2.1630, 1.2106) -- cycle;
\fill[blue!87.7!black, opacity=0.7] (1.6960, 2.1630, 1.2106) -- (1.7450, 2.1630, 1.2092) -- (1.7450, 2.2140, 1.2051) -- (1.6960, 2.2140, 1.2066) -- cycle;
\fill[blue!77.3!black, opacity=0.7] (1.6960, 2.2140, 1.2066) -- (1.7450, 2.2140, 1.2051) -- (1.7450, 2.2650, 1.2008) -- (1.6960, 2.2650, 1.2022) -- cycle;
\fill[blue!73.1!black, opacity=0.7] (1.6960, 2.2650, 1.2022) -- (1.7450, 2.2650, 1.2008) -- (1.7450, 2.3160, 1.1962) -- (1.6960, 2.3160, 1.1977) -- cycle;
\fill[blue!40.1, opacity=0.7] (1.6960, 2.3160, 1.1977) -- (1.7450, 2.3160, 1.1962) -- (1.7450, 2.3670, 1.1914) -- (1.6960, 2.3670, 1.1929) -- cycle;
\fill[blue!15.3, opacity=0.7] (1.6960, 2.3670, 1.1929) -- (1.7450, 2.3670, 1.1914) -- (1.7450, 2.4180, 1.1864) -- (1.6960, 2.4180, 1.1879) -- cycle;
\fill[blue!15.0, opacity=0.7] (1.6960, 2.4180, 1.1879) -- (1.7450, 2.4180, 1.1864) -- (1.7450, 2.4690, 1.1813) -- (1.6960, 2.4690, 1.1827) -- cycle;
\fill[blue!15.0, opacity=0.7] (1.6960, 2.4690, 1.1827) -- (1.7450, 2.4690, 1.1813) -- (1.7450, 2.5200, 1.1759) -- (1.6960, 2.5200, 1.1774) -- cycle;
\fill[blue!15.1, opacity=0.7] (1.6960, 2.5200, 1.1774) -- (1.7450, 2.5200, 1.1759) -- (1.7450, 2.5710, 1.1704) -- (1.6960, 2.5710, 1.1719) -- cycle;
\fill[blue!19.2, opacity=0.7] (1.6960, 2.5710, 1.1719) -- (1.7450, 2.5710, 1.1704) -- (1.7450, 2.6220, 1.1647) -- (1.6960, 2.6220, 1.1662) -- cycle;
\fill[blue!43.9, opacity=0.7] (1.6960, 2.6220, 1.1662) -- (1.7450, 2.6220, 1.1647) -- (1.7450, 2.6730, 1.1589) -- (1.6960, 2.6730, 1.1604) -- cycle;
\fill[blue!40.8, opacity=0.7] (1.6960, 2.6730, 1.1604) -- (1.7450, 2.6730, 1.1589) -- (1.7450, 2.7240, 1.1530) -- (1.6960, 2.7240, 1.1545) -- cycle;
\fill[blue!16.9, opacity=0.7] (1.6960, 2.7240, 1.1545) -- (1.7450, 2.7240, 1.1530) -- (1.7450, 2.7750, 1.1470) -- (1.6960, 2.7750, 1.1484) -- cycle;
\fill[blue!15.0, opacity=0.7] (1.6960, 2.7750, 1.1484) -- (1.7450, 2.7750, 1.1470) -- (1.7450, 2.8260, 1.1409) -- (1.6960, 2.8260, 1.1423) -- cycle;
\fill[blue!15.0, opacity=0.7] (1.6960, 2.8260, 1.1423) -- (1.7450, 2.8260, 1.1409) -- (1.7450, 2.8770, 1.1347) -- (1.6960, 2.8770, 1.1361) -- cycle;
\fill[blue!15.0, opacity=0.7] (1.6960, 2.8770, 1.1361) -- (1.7450, 2.8770, 1.1347) -- (1.7450, 2.9280, 1.1285) -- (1.6960, 2.9280, 1.1299) -- cycle;
\fill[blue!15.0, opacity=0.7] (1.6960, 2.9280, 1.1299) -- (1.7450, 2.9280, 1.1285) -- (1.7450, 2.9790, 1.1222) -- (1.6960, 2.9790, 1.1237) -- cycle;
\fill[blue!15.0, opacity=0.7] (1.6960, 2.9790, 1.1237) -- (1.7450, 2.9790, 1.1222) -- (1.7450, 3.0300, 1.1159) -- (1.6960, 3.0300, 1.1174) -- cycle;
\fill[blue!15.4, opacity=0.7] (1.7450, -0.0300, 1.1159) -- (1.7940, -0.0300, 1.1141) -- (1.7940, 0.0210, 1.1204) -- (1.7450, 0.0210, 1.1222) -- cycle;
\fill[blue!15.0, opacity=0.7] (1.7450, 0.0210, 1.1222) -- (1.7940, 0.0210, 1.1204) -- (1.7940, 0.0720, 1.1267) -- (1.7450, 0.0720, 1.1285) -- cycle;
\fill[blue!15.0, opacity=0.7] (1.7450, 0.0720, 1.1285) -- (1.7940, 0.0720, 1.1267) -- (1.7940, 0.1230, 1.1329) -- (1.7450, 0.1230, 1.1347) -- cycle;
\fill[blue!15.0, opacity=0.7] (1.7450, 0.1230, 1.1347) -- (1.7940, 0.1230, 1.1329) -- (1.7940, 0.1740, 1.1391) -- (1.7450, 0.1740, 1.1409) -- cycle;
\fill[blue!15.0, opacity=0.7] (1.7450, 0.1740, 1.1409) -- (1.7940, 0.1740, 1.1391) -- (1.7940, 0.2250, 1.1452) -- (1.7450, 0.2250, 1.1470) -- cycle;
\fill[blue!15.0, opacity=0.7] (1.7450, 0.2250, 1.1470) -- (1.7940, 0.2250, 1.1452) -- (1.7940, 0.2760, 1.1512) -- (1.7450, 0.2760, 1.1530) -- cycle;
\fill[blue!15.8, opacity=0.7] (1.7450, 0.2760, 1.1530) -- (1.7940, 0.2760, 1.1512) -- (1.7940, 0.3270, 1.1571) -- (1.7450, 0.3270, 1.1589) -- cycle;
\fill[blue!37.3, opacity=0.7] (1.7450, 0.3270, 1.1589) -- (1.7940, 0.3270, 1.1571) -- (1.7940, 0.3780, 1.1629) -- (1.7450, 0.3780, 1.1647) -- cycle;
\fill[blue!62.8, opacity=0.7] (1.7450, 0.3780, 1.1647) -- (1.7940, 0.3780, 1.1629) -- (1.7940, 0.4290, 1.1686) -- (1.7450, 0.4290, 1.1704) -- cycle;
\fill[blue!39.3, opacity=0.7] (1.7450, 0.4290, 1.1704) -- (1.7940, 0.4290, 1.1686) -- (1.7940, 0.4800, 1.1741) -- (1.7450, 0.4800, 1.1759) -- cycle;
\fill[blue!16.9, opacity=0.7] (1.7450, 0.4800, 1.1759) -- (1.7940, 0.4800, 1.1741) -- (1.7940, 0.5310, 1.1795) -- (1.7450, 0.5310, 1.1813) -- cycle;
\fill[blue!15.0, opacity=0.7] (1.7450, 0.5310, 1.1813) -- (1.7940, 0.5310, 1.1795) -- (1.7940, 0.5820, 1.1847) -- (1.7450, 0.5820, 1.1864) -- cycle;
\fill[blue!15.0, opacity=0.7] (1.7450, 0.5820, 1.1864) -- (1.7940, 0.5820, 1.1847) -- (1.7940, 0.6330, 1.1896) -- (1.7450, 0.6330, 1.1914) -- cycle;
\fill[blue!15.0, opacity=0.7] (1.7450, 0.6330, 1.1914) -- (1.7940, 0.6330, 1.1896) -- (1.7940, 0.6840, 1.1944) -- (1.7450, 0.6840, 1.1962) -- cycle;
\fill[blue!16.1, opacity=0.7] (1.7450, 0.6840, 1.1962) -- (1.7940, 0.6840, 1.1944) -- (1.7940, 0.7350, 1.1990) -- (1.7450, 0.7350, 1.2008) -- cycle;
\fill[blue!49.8, opacity=0.7] (1.7450, 0.7350, 1.2008) -- (1.7940, 0.7350, 1.1990) -- (1.7940, 0.7860, 1.2033) -- (1.7450, 0.7860, 1.2051) -- cycle;
\fill[blue!69.0!black, opacity=0.7] (1.7450, 0.7860, 1.2051) -- (1.7940, 0.7860, 1.2033) -- (1.7940, 0.8370, 1.2074) -- (1.7450, 0.8370, 1.2092) -- cycle;
\fill[blue!94.5, opacity=0.7] (1.7450, 0.8370, 1.2092) -- (1.7940, 0.8370, 1.2074) -- (1.7940, 0.8880, 1.2112) -- (1.7450, 0.8880, 1.2130) -- cycle;
\fill[blue!99.6, opacity=0.7] (1.7450, 0.8880, 1.2130) -- (1.7940, 0.8880, 1.2112) -- (1.7940, 0.9390, 1.2148) -- (1.7450, 0.9390, 1.2166) -- cycle;
\fill[blue!74.2!black, opacity=0.7] (1.7450, 0.9390, 1.2166) -- (1.7940, 0.9390, 1.2148) -- (1.7940, 0.9900, 1.2180) -- (1.7450, 0.9900, 1.2198) -- cycle;
\fill[blue!61.4, opacity=0.7] (1.7450, 0.9900, 1.2198) -- (1.7940, 0.9900, 1.2180) -- (1.7940, 1.0410, 1.2210) -- (1.7450, 1.0410, 1.2228) -- cycle;
\fill[blue!26.3, opacity=0.7] (1.7450, 1.0410, 1.2228) -- (1.7940, 1.0410, 1.2210) -- (1.7940, 1.0920, 1.2238) -- (1.7450, 1.0920, 1.2255) -- cycle;
\fill[blue!19.0, opacity=0.7] (1.7450, 1.0920, 1.2255) -- (1.7940, 1.0920, 1.2238) -- (1.7940, 1.1430, 1.2262) -- (1.7450, 1.1430, 1.2279) -- cycle;
\fill[blue!19.2, opacity=0.7] (1.7450, 1.1430, 1.2279) -- (1.7940, 1.1430, 1.2262) -- (1.7940, 1.1940, 1.2283) -- (1.7450, 1.1940, 1.2300) -- cycle;
\fill[blue!24.7, opacity=0.7] (1.7450, 1.1940, 1.2300) -- (1.7940, 1.1940, 1.2283) -- (1.7940, 1.2450, 1.2300) -- (1.7450, 1.2450, 1.2318) -- cycle;
\fill[blue!40.3, opacity=0.7] (1.7450, 1.2450, 1.2318) -- (1.7940, 1.2450, 1.2300) -- (1.7940, 1.2960, 1.2315) -- (1.7450, 1.2960, 1.2333) -- cycle;
\fill[blue!62.8, opacity=0.7] (1.7450, 1.2960, 1.2333) -- (1.7940, 1.2960, 1.2315) -- (1.7940, 1.3470, 1.2326) -- (1.7450, 1.3470, 1.2344) -- cycle;
\fill[blue!76.8, opacity=0.7] (1.7450, 1.3470, 1.2344) -- (1.7940, 1.3470, 1.2326) -- (1.7940, 1.3980, 1.2335) -- (1.7450, 1.3980, 1.2353) -- cycle;
\fill[blue!74.2, opacity=0.7] (1.7450, 1.3980, 1.2353) -- (1.7940, 1.3980, 1.2335) -- (1.7940, 1.4490, 1.2340) -- (1.7450, 1.4490, 1.2357) -- cycle;
\fill[blue!56.3, opacity=0.7] (1.7450, 1.4490, 1.2357) -- (1.7940, 1.4490, 1.2340) -- (1.7940, 1.5000, 1.2341) -- (1.7450, 1.5000, 1.2359) -- cycle;
\fill[blue!37.1, opacity=0.7] (1.7450, 1.5000, 1.2359) -- (1.7940, 1.5000, 1.2341) -- (1.7940, 1.5510, 1.2340) -- (1.7450, 1.5510, 1.2357) -- cycle;
\fill[blue!30.8, opacity=0.7] (1.7450, 1.5510, 1.2357) -- (1.7940, 1.5510, 1.2340) -- (1.7940, 1.6020, 1.2335) -- (1.7450, 1.6020, 1.2353) -- cycle;
\fill[blue!45.3, opacity=0.7] (1.7450, 1.6020, 1.2353) -- (1.7940, 1.6020, 1.2335) -- (1.7940, 1.6530, 1.2326) -- (1.7450, 1.6530, 1.2344) -- cycle;
\fill[blue!99.8!black, opacity=0.7] (1.7450, 1.6530, 1.2344) -- (1.7940, 1.6530, 1.2326) -- (1.7940, 1.7040, 1.2315) -- (1.7450, 1.7040, 1.2333) -- cycle;
\fill[blue!87.3, opacity=0.7] (1.7450, 1.7040, 1.2333) -- (1.7940, 1.7040, 1.2315) -- (1.7940, 1.7550, 1.2300) -- (1.7450, 1.7550, 1.2318) -- cycle;
\fill[blue!38.9, opacity=0.7] (1.7450, 1.7550, 1.2318) -- (1.7940, 1.7550, 1.2300) -- (1.7940, 1.8060, 1.2283) -- (1.7450, 1.8060, 1.2300) -- cycle;
\fill[blue!39.8, opacity=0.7] (1.7450, 1.8060, 1.2300) -- (1.7940, 1.8060, 1.2283) -- (1.7940, 1.8570, 1.2262) -- (1.7450, 1.8570, 1.2279) -- cycle;
\fill[blue!96.5, opacity=0.7] (1.7450, 1.8570, 1.2279) -- (1.7940, 1.8570, 1.2262) -- (1.7940, 1.9080, 1.2238) -- (1.7450, 1.9080, 1.2255) -- cycle;
\fill[blue!69.5, opacity=0.7] (1.7450, 1.9080, 1.2255) -- (1.7940, 1.9080, 1.2238) -- (1.7940, 1.9590, 1.2210) -- (1.7450, 1.9590, 1.2228) -- cycle;
\fill[blue!18.5, opacity=0.7] (1.7450, 1.9590, 1.2228) -- (1.7940, 1.9590, 1.2210) -- (1.7940, 2.0100, 1.2180) -- (1.7450, 2.0100, 1.2198) -- cycle;
\fill[blue!15.6, opacity=0.7] (1.7450, 2.0100, 1.2198) -- (1.7940, 2.0100, 1.2180) -- (1.7940, 2.0610, 1.2148) -- (1.7450, 2.0610, 1.2166) -- cycle;
\fill[blue!17.2, opacity=0.7] (1.7450, 2.0610, 1.2166) -- (1.7940, 2.0610, 1.2148) -- (1.7940, 2.1120, 1.2112) -- (1.7450, 2.1120, 1.2130) -- cycle;
\fill[blue!46.2, opacity=0.7] (1.7450, 2.1120, 1.2130) -- (1.7940, 2.1120, 1.2112) -- (1.7940, 2.1630, 1.2074) -- (1.7450, 2.1630, 1.2092) -- cycle;
\fill[blue!76.5!black, opacity=0.7] (1.7450, 2.1630, 1.2092) -- (1.7940, 2.1630, 1.2074) -- (1.7940, 2.2140, 1.2033) -- (1.7450, 2.2140, 1.2051) -- cycle;
\fill[blue!75.4!black, opacity=0.7] (1.7450, 2.2140, 1.2051) -- (1.7940, 2.2140, 1.2033) -- (1.7940, 2.2650, 1.1990) -- (1.7450, 2.2650, 1.2008) -- cycle;
\fill[blue!87.4!black, opacity=0.7] (1.7450, 2.2650, 1.2008) -- (1.7940, 2.2650, 1.1990) -- (1.7940, 2.3160, 1.1944) -- (1.7450, 2.3160, 1.1962) -- cycle;
\fill[blue!30.2, opacity=0.7] (1.7450, 2.3160, 1.1962) -- (1.7940, 2.3160, 1.1944) -- (1.7940, 2.3670, 1.1896) -- (1.7450, 2.3670, 1.1914) -- cycle;
\fill[blue!15.1, opacity=0.7] (1.7450, 2.3670, 1.1914) -- (1.7940, 2.3670, 1.1896) -- (1.7940, 2.4180, 1.1847) -- (1.7450, 2.4180, 1.1864) -- cycle;
\fill[blue!15.0, opacity=0.7] (1.7450, 2.4180, 1.1864) -- (1.7940, 2.4180, 1.1847) -- (1.7940, 2.4690, 1.1795) -- (1.7450, 2.4690, 1.1813) -- cycle;
\fill[blue!15.0, opacity=0.7] (1.7450, 2.4690, 1.1813) -- (1.7940, 2.4690, 1.1795) -- (1.7940, 2.5200, 1.1741) -- (1.7450, 2.5200, 1.1759) -- cycle;
\fill[blue!15.1, opacity=0.7] (1.7450, 2.5200, 1.1759) -- (1.7940, 2.5200, 1.1741) -- (1.7940, 2.5710, 1.1686) -- (1.7450, 2.5710, 1.1704) -- cycle;
\fill[blue!20.5, opacity=0.7] (1.7450, 2.5710, 1.1704) -- (1.7940, 2.5710, 1.1686) -- (1.7940, 2.6220, 1.1629) -- (1.7450, 2.6220, 1.1647) -- cycle;
\fill[blue!44.9, opacity=0.7] (1.7450, 2.6220, 1.1647) -- (1.7940, 2.6220, 1.1629) -- (1.7940, 2.6730, 1.1571) -- (1.7450, 2.6730, 1.1589) -- cycle;
\fill[blue!37.0, opacity=0.7] (1.7450, 2.6730, 1.1589) -- (1.7940, 2.6730, 1.1571) -- (1.7940, 2.7240, 1.1512) -- (1.7450, 2.7240, 1.1530) -- cycle;
\fill[blue!16.2, opacity=0.7] (1.7450, 2.7240, 1.1530) -- (1.7940, 2.7240, 1.1512) -- (1.7940, 2.7750, 1.1452) -- (1.7450, 2.7750, 1.1470) -- cycle;
\fill[blue!15.0, opacity=0.7] (1.7450, 2.7750, 1.1470) -- (1.7940, 2.7750, 1.1452) -- (1.7940, 2.8260, 1.1391) -- (1.7450, 2.8260, 1.1409) -- cycle;
\fill[blue!15.0, opacity=0.7] (1.7450, 2.8260, 1.1409) -- (1.7940, 2.8260, 1.1391) -- (1.7940, 2.8770, 1.1329) -- (1.7450, 2.8770, 1.1347) -- cycle;
\fill[blue!15.0, opacity=0.7] (1.7450, 2.8770, 1.1347) -- (1.7940, 2.8770, 1.1329) -- (1.7940, 2.9280, 1.1267) -- (1.7450, 2.9280, 1.1285) -- cycle;
\fill[blue!15.0, opacity=0.7] (1.7450, 2.9280, 1.1285) -- (1.7940, 2.9280, 1.1267) -- (1.7940, 2.9790, 1.1204) -- (1.7450, 2.9790, 1.1222) -- cycle;
\fill[blue!15.0, opacity=0.7] (1.7450, 2.9790, 1.1222) -- (1.7940, 2.9790, 1.1204) -- (1.7940, 3.0300, 1.1141) -- (1.7450, 3.0300, 1.1159) -- cycle;
\fill[blue!15.8, opacity=0.7] (1.7940, -0.0300, 1.1141) -- (1.8430, -0.0300, 1.1120) -- (1.8430, 0.0210, 1.1183) -- (1.7940, 0.0210, 1.1204) -- cycle;
\fill[blue!15.0, opacity=0.7] (1.7940, 0.0210, 1.1204) -- (1.8430, 0.0210, 1.1183) -- (1.8430, 0.0720, 1.1246) -- (1.7940, 0.0720, 1.1267) -- cycle;
\fill[blue!15.0, opacity=0.7] (1.7940, 0.0720, 1.1267) -- (1.8430, 0.0720, 1.1246) -- (1.8430, 0.1230, 1.1308) -- (1.7940, 0.1230, 1.1329) -- cycle;
\fill[blue!15.0, opacity=0.7] (1.7940, 0.1230, 1.1329) -- (1.8430, 0.1230, 1.1308) -- (1.8430, 0.1740, 1.1370) -- (1.7940, 0.1740, 1.1391) -- cycle;
\fill[blue!15.0, opacity=0.7] (1.7940, 0.1740, 1.1391) -- (1.8430, 0.1740, 1.1370) -- (1.8430, 0.2250, 1.1431) -- (1.7940, 0.2250, 1.1452) -- cycle;
\fill[blue!15.0, opacity=0.7] (1.7940, 0.2250, 1.1452) -- (1.8430, 0.2250, 1.1431) -- (1.8430, 0.2760, 1.1491) -- (1.7940, 0.2760, 1.1512) -- cycle;
\fill[blue!15.1, opacity=0.7] (1.7940, 0.2760, 1.1512) -- (1.8430, 0.2760, 1.1491) -- (1.8430, 0.3270, 1.1550) -- (1.7940, 0.3270, 1.1571) -- cycle;
\fill[blue!24.6, opacity=0.7] (1.7940, 0.3270, 1.1571) -- (1.8430, 0.3270, 1.1550) -- (1.8430, 0.3780, 1.1608) -- (1.7940, 0.3780, 1.1629) -- cycle;
\fill[blue!59.2, opacity=0.7] (1.7940, 0.3780, 1.1629) -- (1.8430, 0.3780, 1.1608) -- (1.8430, 0.4290, 1.1665) -- (1.7940, 0.4290, 1.1686) -- cycle;
\fill[blue!55.5, opacity=0.7] (1.7940, 0.4290, 1.1686) -- (1.8430, 0.4290, 1.1665) -- (1.8430, 0.4800, 1.1720) -- (1.7940, 0.4800, 1.1741) -- cycle;
\fill[blue!23.3, opacity=0.7] (1.7940, 0.4800, 1.1741) -- (1.8430, 0.4800, 1.1720) -- (1.8430, 0.5310, 1.1774) -- (1.7940, 0.5310, 1.1795) -- cycle;
\fill[blue!15.3, opacity=0.7] (1.7940, 0.5310, 1.1795) -- (1.8430, 0.5310, 1.1774) -- (1.8430, 0.5820, 1.1826) -- (1.7940, 0.5820, 1.1847) -- cycle;
\fill[blue!15.0, opacity=0.7] (1.7940, 0.5820, 1.1847) -- (1.8430, 0.5820, 1.1826) -- (1.8430, 0.6330, 1.1875) -- (1.7940, 0.6330, 1.1896) -- cycle;
\fill[blue!15.0, opacity=0.7] (1.7940, 0.6330, 1.1896) -- (1.8430, 0.6330, 1.1875) -- (1.8430, 0.6840, 1.1923) -- (1.7940, 0.6840, 1.1944) -- cycle;
\fill[blue!15.1, opacity=0.7] (1.7940, 0.6840, 1.1944) -- (1.8430, 0.6840, 1.1923) -- (1.8430, 0.7350, 1.1969) -- (1.7940, 0.7350, 1.1990) -- cycle;
\fill[blue!20.9, opacity=0.7] (1.7940, 0.7350, 1.1990) -- (1.8430, 0.7350, 1.1969) -- (1.8430, 0.7860, 1.2012) -- (1.7940, 0.7860, 1.2033) -- cycle;
\fill[blue!81.1, opacity=0.7] (1.7940, 0.7860, 1.2033) -- (1.8430, 0.7860, 1.2012) -- (1.8430, 0.8370, 1.2053) -- (1.7940, 0.8370, 1.2074) -- cycle;
\fill[blue!76.5!black, opacity=0.7] (1.7940, 0.8370, 1.2074) -- (1.8430, 0.8370, 1.2053) -- (1.8430, 0.8880, 1.2091) -- (1.7940, 0.8880, 1.2112) -- cycle;
\fill[blue!89.4, opacity=0.7] (1.7940, 0.8880, 1.2112) -- (1.8430, 0.8880, 1.2091) -- (1.8430, 0.9390, 1.2127) -- (1.7940, 0.9390, 1.2148) -- cycle;
\fill[blue!99.8!black, opacity=0.7] (1.7940, 0.9390, 1.2148) -- (1.8430, 0.9390, 1.2127) -- (1.8430, 0.9900, 1.2160) -- (1.7940, 0.9900, 1.2180) -- cycle;
\fill[blue!72.6!black, opacity=0.7] (1.7940, 0.9900, 1.2180) -- (1.8430, 0.9900, 1.2160) -- (1.8430, 1.0410, 1.2190) -- (1.7940, 1.0410, 1.2210) -- cycle;
\fill[blue!70.7, opacity=0.7] (1.7940, 1.0410, 1.2210) -- (1.8430, 1.0410, 1.2190) -- (1.8430, 1.0920, 1.2217) -- (1.7940, 1.0920, 1.2238) -- cycle;
\fill[blue!34.0, opacity=0.7] (1.7940, 1.0920, 1.2238) -- (1.8430, 1.0920, 1.2217) -- (1.8430, 1.1430, 1.2241) -- (1.7940, 1.1430, 1.2262) -- cycle;
\fill[blue!22.3, opacity=0.7] (1.7940, 1.1430, 1.2262) -- (1.8430, 1.1430, 1.2241) -- (1.8430, 1.1940, 1.2262) -- (1.7940, 1.1940, 1.2283) -- cycle;
\fill[blue!20.0, opacity=0.7] (1.7940, 1.1940, 1.2283) -- (1.8430, 1.1940, 1.2262) -- (1.8430, 1.2450, 1.2279) -- (1.7940, 1.2450, 1.2300) -- cycle;
\fill[blue!20.7, opacity=0.7] (1.7940, 1.2450, 1.2300) -- (1.8430, 1.2450, 1.2279) -- (1.8430, 1.2960, 1.2294) -- (1.7940, 1.2960, 1.2315) -- cycle;
\fill[blue!22.8, opacity=0.7] (1.7940, 1.2960, 1.2315) -- (1.8430, 1.2960, 1.2294) -- (1.8430, 1.3470, 1.2306) -- (1.7940, 1.3470, 1.2326) -- cycle;
\fill[blue!25.0, opacity=0.7] (1.7940, 1.3470, 1.2326) -- (1.8430, 1.3470, 1.2306) -- (1.8430, 1.3980, 1.2314) -- (1.7940, 1.3980, 1.2335) -- cycle;
\fill[blue!25.8, opacity=0.7] (1.7940, 1.3980, 1.2335) -- (1.8430, 1.3980, 1.2314) -- (1.8430, 1.4490, 1.2319) -- (1.7940, 1.4490, 1.2340) -- cycle;
\fill[blue!25.9, opacity=0.7] (1.7940, 1.4490, 1.2340) -- (1.8430, 1.4490, 1.2319) -- (1.8430, 1.5000, 1.2320) -- (1.7940, 1.5000, 1.2341) -- cycle;
\fill[blue!28.8, opacity=0.7] (1.7940, 1.5000, 1.2341) -- (1.8430, 1.5000, 1.2320) -- (1.8430, 1.5510, 1.2319) -- (1.7940, 1.5510, 1.2340) -- cycle;
\fill[blue!43.7, opacity=0.7] (1.7940, 1.5510, 1.2340) -- (1.8430, 1.5510, 1.2319) -- (1.8430, 1.6020, 1.2314) -- (1.7940, 1.6020, 1.2335) -- cycle;
\fill[blue!88.8, opacity=0.7] (1.7940, 1.6020, 1.2335) -- (1.8430, 1.6020, 1.2314) -- (1.8430, 1.6530, 1.2306) -- (1.7940, 1.6530, 1.2326) -- cycle;
\fill[blue!83.8!black, opacity=0.7] (1.7940, 1.6530, 1.2326) -- (1.8430, 1.6530, 1.2306) -- (1.8430, 1.7040, 1.2294) -- (1.7940, 1.7040, 1.2315) -- cycle;
\fill[blue!54.5, opacity=0.7] (1.7940, 1.7040, 1.2315) -- (1.8430, 1.7040, 1.2294) -- (1.8430, 1.7550, 1.2279) -- (1.7940, 1.7550, 1.2300) -- cycle;
\fill[blue!36.9, opacity=0.7] (1.7940, 1.7550, 1.2300) -- (1.8430, 1.7550, 1.2279) -- (1.8430, 1.8060, 1.2262) -- (1.7940, 1.8060, 1.2283) -- cycle;
\fill[blue!62.2, opacity=0.7] (1.7940, 1.8060, 1.2283) -- (1.8430, 1.8060, 1.2262) -- (1.8430, 1.8570, 1.2241) -- (1.7940, 1.8570, 1.2262) -- cycle;
\fill[blue!70.5!black, opacity=0.7] (1.7940, 1.8570, 1.2262) -- (1.8430, 1.8570, 1.2241) -- (1.8430, 1.9080, 1.2217) -- (1.7940, 1.9080, 1.2238) -- cycle;
\fill[blue!35.9, opacity=0.7] (1.7940, 1.9080, 1.2238) -- (1.8430, 1.9080, 1.2217) -- (1.8430, 1.9590, 1.2190) -- (1.7940, 1.9590, 1.2210) -- cycle;
\fill[blue!16.2, opacity=0.7] (1.7940, 1.9590, 1.2210) -- (1.8430, 1.9590, 1.2190) -- (1.8430, 2.0100, 1.2160) -- (1.7940, 2.0100, 1.2180) -- cycle;
\fill[blue!15.5, opacity=0.7] (1.7940, 2.0100, 1.2180) -- (1.8430, 2.0100, 1.2160) -- (1.8430, 2.0610, 1.2127) -- (1.7940, 2.0610, 1.2148) -- cycle;
\fill[blue!19.4, opacity=0.7] (1.7940, 2.0610, 1.2148) -- (1.8430, 2.0610, 1.2127) -- (1.8430, 2.1120, 1.2091) -- (1.7940, 2.1120, 1.2112) -- cycle;
\fill[blue!62.9, opacity=0.7] (1.7940, 2.1120, 1.2112) -- (1.8430, 2.1120, 1.2091) -- (1.8430, 2.1630, 1.2053) -- (1.7940, 2.1630, 1.2074) -- cycle;
\fill[blue!69.2!black, opacity=0.7] (1.7940, 2.1630, 1.2074) -- (1.8430, 2.1630, 1.2053) -- (1.8430, 2.2140, 1.2012) -- (1.7940, 2.2140, 1.2033) -- cycle;
\fill[blue!71.5!black, opacity=0.7] (1.7940, 2.2140, 1.2033) -- (1.8430, 2.2140, 1.2012) -- (1.8430, 2.2650, 1.1969) -- (1.7940, 2.2650, 1.1990) -- cycle;
\fill[blue!91.1, opacity=0.7] (1.7940, 2.2650, 1.1990) -- (1.8430, 2.2650, 1.1969) -- (1.8430, 2.3160, 1.1923) -- (1.7940, 2.3160, 1.1944) -- cycle;
\fill[blue!21.7, opacity=0.7] (1.7940, 2.3160, 1.1944) -- (1.8430, 2.3160, 1.1923) -- (1.8430, 2.3670, 1.1875) -- (1.7940, 2.3670, 1.1896) -- cycle;
\fill[blue!15.0, opacity=0.7] (1.7940, 2.3670, 1.1896) -- (1.8430, 2.3670, 1.1875) -- (1.8430, 2.4180, 1.1826) -- (1.7940, 2.4180, 1.1847) -- cycle;
\fill[blue!15.0, opacity=0.7] (1.7940, 2.4180, 1.1847) -- (1.8430, 2.4180, 1.1826) -- (1.8430, 2.4690, 1.1774) -- (1.7940, 2.4690, 1.1795) -- cycle;
\fill[blue!15.0, opacity=0.7] (1.7940, 2.4690, 1.1795) -- (1.8430, 2.4690, 1.1774) -- (1.8430, 2.5200, 1.1720) -- (1.7940, 2.5200, 1.1741) -- cycle;
\fill[blue!15.2, opacity=0.7] (1.7940, 2.5200, 1.1741) -- (1.8430, 2.5200, 1.1720) -- (1.8430, 2.5710, 1.1665) -- (1.7940, 2.5710, 1.1686) -- cycle;
\fill[blue!22.9, opacity=0.7] (1.7940, 2.5710, 1.1686) -- (1.8430, 2.5710, 1.1665) -- (1.8430, 2.6220, 1.1608) -- (1.7940, 2.6220, 1.1629) -- cycle;
\fill[blue!45.9, opacity=0.7] (1.7940, 2.6220, 1.1629) -- (1.8430, 2.6220, 1.1608) -- (1.8430, 2.6730, 1.1550) -- (1.7940, 2.6730, 1.1571) -- cycle;
\fill[blue!32.1, opacity=0.7] (1.7940, 2.6730, 1.1571) -- (1.8430, 2.6730, 1.1550) -- (1.8430, 2.7240, 1.1491) -- (1.7940, 2.7240, 1.1512) -- cycle;
\fill[blue!15.5, opacity=0.7] (1.7940, 2.7240, 1.1512) -- (1.8430, 2.7240, 1.1491) -- (1.8430, 2.7750, 1.1431) -- (1.7940, 2.7750, 1.1452) -- cycle;
\fill[blue!15.0, opacity=0.7] (1.7940, 2.7750, 1.1452) -- (1.8430, 2.7750, 1.1431) -- (1.8430, 2.8260, 1.1370) -- (1.7940, 2.8260, 1.1391) -- cycle;
\fill[blue!15.0, opacity=0.7] (1.7940, 2.8260, 1.1391) -- (1.8430, 2.8260, 1.1370) -- (1.8430, 2.8770, 1.1308) -- (1.7940, 2.8770, 1.1329) -- cycle;
\fill[blue!15.0, opacity=0.7] (1.7940, 2.8770, 1.1329) -- (1.8430, 2.8770, 1.1308) -- (1.8430, 2.9280, 1.1246) -- (1.7940, 2.9280, 1.1267) -- cycle;
\fill[blue!15.0, opacity=0.7] (1.7940, 2.9280, 1.1267) -- (1.8430, 2.9280, 1.1246) -- (1.8430, 2.9790, 1.1183) -- (1.7940, 2.9790, 1.1204) -- cycle;
\fill[blue!15.0, opacity=0.7] (1.7940, 2.9790, 1.1204) -- (1.8430, 2.9790, 1.1183) -- (1.8430, 3.0300, 1.1120) -- (1.7940, 3.0300, 1.1141) -- cycle;
\fill[blue!16.1, opacity=0.7] (1.8430, -0.0300, 1.1120) -- (1.8920, -0.0300, 1.1096) -- (1.8920, 0.0210, 1.1159) -- (1.8430, 0.0210, 1.1183) -- cycle;
\fill[blue!15.2, opacity=0.7] (1.8430, 0.0210, 1.1183) -- (1.8920, 0.0210, 1.1159) -- (1.8920, 0.0720, 1.1222) -- (1.8430, 0.0720, 1.1246) -- cycle;
\fill[blue!15.0, opacity=0.7] (1.8430, 0.0720, 1.1246) -- (1.8920, 0.0720, 1.1222) -- (1.8920, 0.1230, 1.1284) -- (1.8430, 0.1230, 1.1308) -- cycle;
\fill[blue!15.0, opacity=0.7] (1.8430, 0.1230, 1.1308) -- (1.8920, 0.1230, 1.1284) -- (1.8920, 0.1740, 1.1346) -- (1.8430, 0.1740, 1.1370) -- cycle;
\fill[blue!15.0, opacity=0.7] (1.8430, 0.1740, 1.1370) -- (1.8920, 0.1740, 1.1346) -- (1.8920, 0.2250, 1.1407) -- (1.8430, 0.2250, 1.1431) -- cycle;
\fill[blue!15.0, opacity=0.7] (1.8430, 0.2250, 1.1431) -- (1.8920, 0.2250, 1.1407) -- (1.8920, 0.2760, 1.1467) -- (1.8430, 0.2760, 1.1491) -- cycle;
\fill[blue!15.0, opacity=0.7] (1.8430, 0.2760, 1.1491) -- (1.8920, 0.2760, 1.1467) -- (1.8920, 0.3270, 1.1526) -- (1.8430, 0.3270, 1.1550) -- cycle;
\fill[blue!17.2, opacity=0.7] (1.8430, 0.3270, 1.1550) -- (1.8920, 0.3270, 1.1526) -- (1.8920, 0.3780, 1.1584) -- (1.8430, 0.3780, 1.1608) -- cycle;
\fill[blue!45.4, opacity=0.7] (1.8430, 0.3780, 1.1608) -- (1.8920, 0.3780, 1.1584) -- (1.8920, 0.4290, 1.1641) -- (1.8430, 0.4290, 1.1665) -- cycle;
\fill[blue!65.7, opacity=0.7] (1.8430, 0.4290, 1.1665) -- (1.8920, 0.4290, 1.1641) -- (1.8920, 0.4800, 1.1696) -- (1.8430, 0.4800, 1.1720) -- cycle;
\fill[blue!39.6, opacity=0.7] (1.8430, 0.4800, 1.1720) -- (1.8920, 0.4800, 1.1696) -- (1.8920, 0.5310, 1.1750) -- (1.8430, 0.5310, 1.1774) -- cycle;
\fill[blue!17.2, opacity=0.7] (1.8430, 0.5310, 1.1774) -- (1.8920, 0.5310, 1.1750) -- (1.8920, 0.5820, 1.1802) -- (1.8430, 0.5820, 1.1826) -- cycle;
\fill[blue!15.1, opacity=0.7] (1.8430, 0.5820, 1.1826) -- (1.8920, 0.5820, 1.1802) -- (1.8920, 0.6330, 1.1851) -- (1.8430, 0.6330, 1.1875) -- cycle;
\fill[blue!15.0, opacity=0.7] (1.8430, 0.6330, 1.1875) -- (1.8920, 0.6330, 1.1851) -- (1.8920, 0.6840, 1.1899) -- (1.8430, 0.6840, 1.1923) -- cycle;
\fill[blue!15.0, opacity=0.7] (1.8430, 0.6840, 1.1923) -- (1.8920, 0.6840, 1.1899) -- (1.8920, 0.7350, 1.1945) -- (1.8430, 0.7350, 1.1969) -- cycle;
\fill[blue!15.4, opacity=0.7] (1.8430, 0.7350, 1.1969) -- (1.8920, 0.7350, 1.1945) -- (1.8920, 0.7860, 1.1988) -- (1.8430, 0.7860, 1.2012) -- cycle;
\fill[blue!30.2, opacity=0.7] (1.8430, 0.7860, 1.2012) -- (1.8920, 0.7860, 1.1988) -- (1.8920, 0.8370, 1.2029) -- (1.8430, 0.8370, 1.2053) -- cycle;
\fill[blue!97.8, opacity=0.7] (1.8430, 0.8370, 1.2053) -- (1.8920, 0.8370, 1.2029) -- (1.8920, 0.8880, 1.2067) -- (1.8430, 0.8880, 1.2091) -- cycle;
\fill[blue!88.3!black, opacity=0.7] (1.8430, 0.8880, 1.2091) -- (1.8920, 0.8880, 1.2067) -- (1.8920, 0.9390, 1.2103) -- (1.8430, 0.9390, 1.2127) -- cycle;
\fill[blue!86.0, opacity=0.7] (1.8430, 0.9390, 1.2127) -- (1.8920, 0.9390, 1.2103) -- (1.8920, 0.9900, 1.2135) -- (1.8430, 0.9900, 1.2160) -- cycle;
\fill[blue!95.2, opacity=0.7] (1.8430, 0.9900, 1.2160) -- (1.8920, 0.9900, 1.2135) -- (1.8920, 1.0410, 1.2165) -- (1.8430, 1.0410, 1.2190) -- cycle;
\fill[blue!68.9!black, opacity=0.7] (1.8430, 1.0410, 1.2190) -- (1.8920, 1.0410, 1.2165) -- (1.8920, 1.0920, 1.2193) -- (1.8430, 1.0920, 1.2217) -- cycle;
\fill[blue!95.1, opacity=0.7] (1.8430, 1.0920, 1.2217) -- (1.8920, 1.0920, 1.2193) -- (1.8920, 1.1430, 1.2217) -- (1.8430, 1.1430, 1.2241) -- cycle;
\fill[blue!59.3, opacity=0.7] (1.8430, 1.1430, 1.2241) -- (1.8920, 1.1430, 1.2217) -- (1.8920, 1.1940, 1.2238) -- (1.8430, 1.1940, 1.2262) -- cycle;
\fill[blue!37.5, opacity=0.7] (1.8430, 1.1940, 1.2262) -- (1.8920, 1.1940, 1.2238) -- (1.8920, 1.2450, 1.2255) -- (1.8430, 1.2450, 1.2279) -- cycle;
\fill[blue!29.0, opacity=0.7] (1.8430, 1.2450, 1.2279) -- (1.8920, 1.2450, 1.2255) -- (1.8920, 1.2960, 1.2270) -- (1.8430, 1.2960, 1.2294) -- cycle;
\fill[blue!26.5, opacity=0.7] (1.8430, 1.2960, 1.2294) -- (1.8920, 1.2960, 1.2270) -- (1.8920, 1.3470, 1.2281) -- (1.8430, 1.3470, 1.2306) -- cycle;
\fill[blue!27.0, opacity=0.7] (1.8430, 1.3470, 1.2306) -- (1.8920, 1.3470, 1.2281) -- (1.8920, 1.3980, 1.2290) -- (1.8430, 1.3980, 1.2314) -- cycle;
\fill[blue!30.3, opacity=0.7] (1.8430, 1.3980, 1.2314) -- (1.8920, 1.3980, 1.2290) -- (1.8920, 1.4490, 1.2295) -- (1.8430, 1.4490, 1.2319) -- cycle;
\fill[blue!39.1, opacity=0.7] (1.8430, 1.4490, 1.2319) -- (1.8920, 1.4490, 1.2295) -- (1.8920, 1.5000, 1.2296) -- (1.8430, 1.5000, 1.2320) -- cycle;
\fill[blue!61.1, opacity=0.7] (1.8430, 1.5000, 1.2320) -- (1.8920, 1.5000, 1.2296) -- (1.8920, 1.5510, 1.2295) -- (1.8430, 1.5510, 1.2319) -- cycle;
\fill[blue!99.4, opacity=0.7] (1.8430, 1.5510, 1.2319) -- (1.8920, 1.5510, 1.2295) -- (1.8920, 1.6020, 1.2290) -- (1.8430, 1.6020, 1.2314) -- cycle;
\fill[blue!82.8!black, opacity=0.7] (1.8430, 1.6020, 1.2314) -- (1.8920, 1.6020, 1.2290) -- (1.8920, 1.6530, 1.2281) -- (1.8430, 1.6530, 1.2306) -- cycle;
\fill[blue!63.6, opacity=0.7] (1.8430, 1.6530, 1.2306) -- (1.8920, 1.6530, 1.2281) -- (1.8920, 1.7040, 1.2270) -- (1.8430, 1.7040, 1.2294) -- cycle;
\fill[blue!40.6, opacity=0.7] (1.8430, 1.7040, 1.2294) -- (1.8920, 1.7040, 1.2270) -- (1.8920, 1.7550, 1.2255) -- (1.8430, 1.7550, 1.2279) -- cycle;
\fill[blue!51.2, opacity=0.7] (1.8430, 1.7550, 1.2279) -- (1.8920, 1.7550, 1.2255) -- (1.8920, 1.8060, 1.2238) -- (1.8430, 1.8060, 1.2262) -- cycle;
\fill[blue!90.0!black, opacity=0.7] (1.8430, 1.8060, 1.2262) -- (1.8920, 1.8060, 1.2238) -- (1.8920, 1.8570, 1.2217) -- (1.8430, 1.8570, 1.2241) -- cycle;
\fill[blue!71.0, opacity=0.7] (1.8430, 1.8570, 1.2241) -- (1.8920, 1.8570, 1.2217) -- (1.8920, 1.9080, 1.2193) -- (1.8430, 1.9080, 1.2217) -- cycle;
\fill[blue!19.6, opacity=0.7] (1.8430, 1.9080, 1.2217) -- (1.8920, 1.9080, 1.2193) -- (1.8920, 1.9590, 1.2165) -- (1.8430, 1.9590, 1.2190) -- cycle;
\fill[blue!15.5, opacity=0.7] (1.8430, 1.9590, 1.2190) -- (1.8920, 1.9590, 1.2165) -- (1.8920, 2.0100, 1.2135) -- (1.8430, 2.0100, 1.2160) -- cycle;
\fill[blue!15.8, opacity=0.7] (1.8430, 2.0100, 1.2160) -- (1.8920, 2.0100, 1.2135) -- (1.8920, 2.0610, 1.2103) -- (1.8430, 2.0610, 1.2127) -- cycle;
\fill[blue!25.9, opacity=0.7] (1.8430, 2.0610, 1.2127) -- (1.8920, 2.0610, 1.2103) -- (1.8920, 2.1120, 1.2067) -- (1.8430, 2.1120, 1.2091) -- cycle;
\fill[blue!85.6, opacity=0.7] (1.8430, 2.1120, 1.2091) -- (1.8920, 2.1120, 1.2067) -- (1.8920, 2.1630, 1.2029) -- (1.8430, 2.1630, 1.2053) -- cycle;
\fill[blue!68.9!black, opacity=0.7] (1.8430, 2.1630, 1.2053) -- (1.8920, 2.1630, 1.2029) -- (1.8920, 2.2140, 1.1988) -- (1.8430, 2.2140, 1.2012) -- cycle;
\fill[blue!68.4!black, opacity=0.7] (1.8430, 2.2140, 1.2012) -- (1.8920, 2.2140, 1.1988) -- (1.8920, 2.2650, 1.1945) -- (1.8430, 2.2650, 1.1969) -- cycle;
\fill[blue!66.7, opacity=0.7] (1.8430, 2.2650, 1.1969) -- (1.8920, 2.2650, 1.1945) -- (1.8920, 2.3160, 1.1899) -- (1.8430, 2.3160, 1.1923) -- cycle;
\fill[blue!16.9, opacity=0.7] (1.8430, 2.3160, 1.1923) -- (1.8920, 2.3160, 1.1899) -- (1.8920, 2.3670, 1.1851) -- (1.8430, 2.3670, 1.1875) -- cycle;
\fill[blue!15.0, opacity=0.7] (1.8430, 2.3670, 1.1875) -- (1.8920, 2.3670, 1.1851) -- (1.8920, 2.4180, 1.1802) -- (1.8430, 2.4180, 1.1826) -- cycle;
\fill[blue!15.0, opacity=0.7] (1.8430, 2.4180, 1.1826) -- (1.8920, 2.4180, 1.1802) -- (1.8920, 2.4690, 1.1750) -- (1.8430, 2.4690, 1.1774) -- cycle;
\fill[blue!15.0, opacity=0.7] (1.8430, 2.4690, 1.1774) -- (1.8920, 2.4690, 1.1750) -- (1.8920, 2.5200, 1.1696) -- (1.8430, 2.5200, 1.1720) -- cycle;
\fill[blue!15.4, opacity=0.7] (1.8430, 2.5200, 1.1720) -- (1.8920, 2.5200, 1.1696) -- (1.8920, 2.5710, 1.1641) -- (1.8430, 2.5710, 1.1665) -- cycle;
\fill[blue!26.8, opacity=0.7] (1.8430, 2.5710, 1.1665) -- (1.8920, 2.5710, 1.1641) -- (1.8920, 2.6220, 1.1584) -- (1.8430, 2.6220, 1.1608) -- cycle;
\fill[blue!45.8, opacity=0.7] (1.8430, 2.6220, 1.1608) -- (1.8920, 2.6220, 1.1584) -- (1.8920, 2.6730, 1.1526) -- (1.8430, 2.6730, 1.1550) -- cycle;
\fill[blue!26.3, opacity=0.7] (1.8430, 2.6730, 1.1550) -- (1.8920, 2.6730, 1.1526) -- (1.8920, 2.7240, 1.1467) -- (1.8430, 2.7240, 1.1491) -- cycle;
\fill[blue!15.2, opacity=0.7] (1.8430, 2.7240, 1.1491) -- (1.8920, 2.7240, 1.1467) -- (1.8920, 2.7750, 1.1407) -- (1.8430, 2.7750, 1.1431) -- cycle;
\fill[blue!15.0, opacity=0.7] (1.8430, 2.7750, 1.1431) -- (1.8920, 2.7750, 1.1407) -- (1.8920, 2.8260, 1.1346) -- (1.8430, 2.8260, 1.1370) -- cycle;
\fill[blue!15.0, opacity=0.7] (1.8430, 2.8260, 1.1370) -- (1.8920, 2.8260, 1.1346) -- (1.8920, 2.8770, 1.1284) -- (1.8430, 2.8770, 1.1308) -- cycle;
\fill[blue!15.0, opacity=0.7] (1.8430, 2.8770, 1.1308) -- (1.8920, 2.8770, 1.1284) -- (1.8920, 2.9280, 1.1222) -- (1.8430, 2.9280, 1.1246) -- cycle;
\fill[blue!15.0, opacity=0.7] (1.8430, 2.9280, 1.1246) -- (1.8920, 2.9280, 1.1222) -- (1.8920, 2.9790, 1.1159) -- (1.8430, 2.9790, 1.1183) -- cycle;
\fill[blue!15.0, opacity=0.7] (1.8430, 2.9790, 1.1183) -- (1.8920, 2.9790, 1.1159) -- (1.8920, 3.0300, 1.1096) -- (1.8430, 3.0300, 1.1120) -- cycle;
\fill[blue!16.2, opacity=0.7] (1.8920, -0.0300, 1.1096) -- (1.9410, -0.0300, 1.1069) -- (1.9410, 0.0210, 1.1132) -- (1.8920, 0.0210, 1.1159) -- cycle;
\fill[blue!15.5, opacity=0.7] (1.8920, 0.0210, 1.1159) -- (1.9410, 0.0210, 1.1132) -- (1.9410, 0.0720, 1.1195) -- (1.8920, 0.0720, 1.1222) -- cycle;
\fill[blue!15.0, opacity=0.7] (1.8920, 0.0720, 1.1222) -- (1.9410, 0.0720, 1.1195) -- (1.9410, 0.1230, 1.1257) -- (1.8920, 0.1230, 1.1284) -- cycle;
\fill[blue!15.0, opacity=0.7] (1.8920, 0.1230, 1.1284) -- (1.9410, 0.1230, 1.1257) -- (1.9410, 0.1740, 1.1319) -- (1.8920, 0.1740, 1.1346) -- cycle;
\fill[blue!15.0, opacity=0.7] (1.8920, 0.1740, 1.1346) -- (1.9410, 0.1740, 1.1319) -- (1.9410, 0.2250, 1.1380) -- (1.8920, 0.2250, 1.1407) -- cycle;
\fill[blue!15.0, opacity=0.7] (1.8920, 0.2250, 1.1407) -- (1.9410, 0.2250, 1.1380) -- (1.9410, 0.2760, 1.1440) -- (1.8920, 0.2760, 1.1467) -- cycle;
\fill[blue!15.0, opacity=0.7] (1.8920, 0.2760, 1.1467) -- (1.9410, 0.2760, 1.1440) -- (1.9410, 0.3270, 1.1499) -- (1.8920, 0.3270, 1.1526) -- cycle;
\fill[blue!15.2, opacity=0.7] (1.8920, 0.3270, 1.1526) -- (1.9410, 0.3270, 1.1499) -- (1.9410, 0.3780, 1.1557) -- (1.8920, 0.3780, 1.1584) -- cycle;
\fill[blue!27.3, opacity=0.7] (1.8920, 0.3780, 1.1584) -- (1.9410, 0.3780, 1.1557) -- (1.9410, 0.4290, 1.1614) -- (1.8920, 0.4290, 1.1641) -- cycle;
\fill[blue!62.5, opacity=0.7] (1.8920, 0.4290, 1.1641) -- (1.9410, 0.4290, 1.1614) -- (1.9410, 0.4800, 1.1669) -- (1.8920, 0.4800, 1.1696) -- cycle;
\fill[blue!60.1, opacity=0.7] (1.8920, 0.4800, 1.1696) -- (1.9410, 0.4800, 1.1669) -- (1.9410, 0.5310, 1.1723) -- (1.8920, 0.5310, 1.1750) -- cycle;
\fill[blue!27.3, opacity=0.7] (1.8920, 0.5310, 1.1750) -- (1.9410, 0.5310, 1.1723) -- (1.9410, 0.5820, 1.1775) -- (1.8920, 0.5820, 1.1802) -- cycle;
\fill[blue!15.7, opacity=0.7] (1.8920, 0.5820, 1.1802) -- (1.9410, 0.5820, 1.1775) -- (1.9410, 0.6330, 1.1824) -- (1.8920, 0.6330, 1.1851) -- cycle;
\fill[blue!15.0, opacity=0.7] (1.8920, 0.6330, 1.1851) -- (1.9410, 0.6330, 1.1824) -- (1.9410, 0.6840, 1.1872) -- (1.8920, 0.6840, 1.1899) -- cycle;
\fill[blue!15.0, opacity=0.7] (1.8920, 0.6840, 1.1899) -- (1.9410, 0.6840, 1.1872) -- (1.9410, 0.7350, 1.1918) -- (1.8920, 0.7350, 1.1945) -- cycle;
\fill[blue!15.0, opacity=0.7] (1.8920, 0.7350, 1.1945) -- (1.9410, 0.7350, 1.1918) -- (1.9410, 0.7860, 1.1961) -- (1.8920, 0.7860, 1.1988) -- cycle;
\fill[blue!16.0, opacity=0.7] (1.8920, 0.7860, 1.1988) -- (1.9410, 0.7860, 1.1961) -- (1.9410, 0.8370, 1.2002) -- (1.8920, 0.8370, 1.2029) -- cycle;
\fill[blue!37.2, opacity=0.7] (1.8920, 0.8370, 1.2029) -- (1.9410, 0.8370, 1.2002) -- (1.9410, 0.8880, 1.2040) -- (1.8920, 0.8880, 1.2067) -- cycle;
\fill[blue!95.7!black, opacity=0.7] (1.8920, 0.8880, 1.2067) -- (1.9410, 0.8880, 1.2040) -- (1.9410, 0.9390, 1.2076) -- (1.8920, 0.9390, 1.2103) -- cycle;
\fill[blue!89.0!black, opacity=0.7] (1.8920, 0.9390, 1.2103) -- (1.9410, 0.9390, 1.2076) -- (1.9410, 0.9900, 1.2108) -- (1.8920, 0.9900, 1.2135) -- cycle;
\fill[blue!83.8, opacity=0.7] (1.8920, 0.9900, 1.2135) -- (1.9410, 0.9900, 1.2108) -- (1.9410, 1.0410, 1.2138) -- (1.8920, 1.0410, 1.2165) -- cycle;
\fill[blue!85.0, opacity=0.7] (1.8920, 1.0410, 1.2165) -- (1.9410, 1.0410, 1.2138) -- (1.9410, 1.0920, 1.2165) -- (1.8920, 1.0920, 1.2193) -- cycle;
\fill[blue!97.0!black, opacity=0.7] (1.8920, 1.0920, 1.2193) -- (1.9410, 1.0920, 1.2165) -- (1.9410, 1.1430, 1.2190) -- (1.8920, 1.1430, 1.2217) -- cycle;
\fill[blue!68.5!black, opacity=0.7] (1.8920, 1.1430, 1.2217) -- (1.9410, 1.1430, 1.2190) -- (1.9410, 1.1940, 1.2210) -- (1.8920, 1.1940, 1.2238) -- cycle;
\fill[blue!87.1!black, opacity=0.7] (1.8920, 1.1940, 1.2238) -- (1.9410, 1.1940, 1.2210) -- (1.9410, 1.2450, 1.2228) -- (1.8920, 1.2450, 1.2255) -- cycle;
\fill[blue!90.0, opacity=0.7] (1.8920, 1.2450, 1.2255) -- (1.9410, 1.2450, 1.2228) -- (1.9410, 1.2960, 1.2243) -- (1.8920, 1.2960, 1.2270) -- cycle;
\fill[blue!80.1, opacity=0.7] (1.8920, 1.2960, 1.2270) -- (1.9410, 1.2960, 1.2243) -- (1.9410, 1.3470, 1.2254) -- (1.8920, 1.3470, 1.2281) -- cycle;
\fill[blue!79.3, opacity=0.7] (1.8920, 1.3470, 1.2281) -- (1.9410, 1.3470, 1.2254) -- (1.9410, 1.3980, 1.2263) -- (1.8920, 1.3980, 1.2290) -- cycle;
\fill[blue!88.2, opacity=0.7] (1.8920, 1.3980, 1.2290) -- (1.9410, 1.3980, 1.2263) -- (1.9410, 1.4490, 1.2268) -- (1.8920, 1.4490, 1.2295) -- cycle;
\fill[blue!90.0!black, opacity=0.7] (1.8920, 1.4490, 1.2295) -- (1.9410, 1.4490, 1.2268) -- (1.9410, 1.5000, 1.2269) -- (1.8920, 1.5000, 1.2296) -- cycle;
\fill[blue!69.0!black, opacity=0.7] (1.8920, 1.5000, 1.2296) -- (1.9410, 1.5000, 1.2269) -- (1.9410, 1.5510, 1.2268) -- (1.8920, 1.5510, 1.2295) -- cycle;
\fill[blue!92.4, opacity=0.7] (1.8920, 1.5510, 1.2295) -- (1.9410, 1.5510, 1.2268) -- (1.9410, 1.6020, 1.2263) -- (1.8920, 1.6020, 1.2290) -- cycle;
\fill[blue!60.0, opacity=0.7] (1.8920, 1.6020, 1.2290) -- (1.9410, 1.6020, 1.2263) -- (1.9410, 1.6530, 1.2254) -- (1.8920, 1.6530, 1.2281) -- cycle;
\fill[blue!44.1, opacity=0.7] (1.8920, 1.6530, 1.2281) -- (1.9410, 1.6530, 1.2254) -- (1.9410, 1.7040, 1.2243) -- (1.8920, 1.7040, 1.2270) -- cycle;
\fill[blue!52.6, opacity=0.7] (1.8920, 1.7040, 1.2270) -- (1.9410, 1.7040, 1.2243) -- (1.9410, 1.7550, 1.2228) -- (1.8920, 1.7550, 1.2255) -- cycle;
\fill[blue!95.9, opacity=0.7] (1.8920, 1.7550, 1.2255) -- (1.9410, 1.7550, 1.2228) -- (1.9410, 1.8060, 1.2210) -- (1.8920, 1.8060, 1.2238) -- cycle;
\fill[blue!93.3, opacity=0.7] (1.8920, 1.8060, 1.2238) -- (1.9410, 1.8060, 1.2210) -- (1.9410, 1.8570, 1.2190) -- (1.8920, 1.8570, 1.2217) -- cycle;
\fill[blue!26.9, opacity=0.7] (1.8920, 1.8570, 1.2217) -- (1.9410, 1.8570, 1.2190) -- (1.9410, 1.9080, 1.2165) -- (1.8920, 1.9080, 1.2193) -- cycle;
\fill[blue!15.8, opacity=0.7] (1.8920, 1.9080, 1.2193) -- (1.9410, 1.9080, 1.2165) -- (1.9410, 1.9590, 1.2138) -- (1.8920, 1.9590, 1.2165) -- cycle;
\fill[blue!15.4, opacity=0.7] (1.8920, 1.9590, 1.2165) -- (1.9410, 1.9590, 1.2138) -- (1.9410, 2.0100, 1.2108) -- (1.8920, 2.0100, 1.2135) -- cycle;
\fill[blue!17.3, opacity=0.7] (1.8920, 2.0100, 1.2135) -- (1.9410, 2.0100, 1.2108) -- (1.9410, 2.0610, 1.2076) -- (1.8920, 2.0610, 1.2103) -- cycle;
\fill[blue!43.7, opacity=0.7] (1.8920, 2.0610, 1.2103) -- (1.9410, 2.0610, 1.2076) -- (1.9410, 2.1120, 1.2040) -- (1.8920, 2.1120, 1.2067) -- cycle;
\fill[blue!87.5!black, opacity=0.7] (1.8920, 2.1120, 1.2067) -- (1.9410, 2.1120, 1.2040) -- (1.9410, 2.1630, 1.2002) -- (1.8920, 2.1630, 1.2029) -- cycle;
\fill[blue!70.3!black, opacity=0.7] (1.8920, 2.1630, 1.2029) -- (1.9410, 2.1630, 1.2002) -- (1.9410, 2.2140, 1.1961) -- (1.8920, 2.2140, 1.1988) -- cycle;
\fill[blue!82.8!black, opacity=0.7] (1.8920, 2.2140, 1.1988) -- (1.9410, 2.2140, 1.1961) -- (1.9410, 2.2650, 1.1918) -- (1.8920, 2.2650, 1.1945) -- cycle;
\fill[blue!38.5, opacity=0.7] (1.8920, 2.2650, 1.1945) -- (1.9410, 2.2650, 1.1918) -- (1.9410, 2.3160, 1.1872) -- (1.8920, 2.3160, 1.1899) -- cycle;
\fill[blue!15.3, opacity=0.7] (1.8920, 2.3160, 1.1899) -- (1.9410, 2.3160, 1.1872) -- (1.9410, 2.3670, 1.1824) -- (1.8920, 2.3670, 1.1851) -- cycle;
\fill[blue!15.0, opacity=0.7] (1.8920, 2.3670, 1.1851) -- (1.9410, 2.3670, 1.1824) -- (1.9410, 2.4180, 1.1775) -- (1.8920, 2.4180, 1.1802) -- cycle;
\fill[blue!15.0, opacity=0.7] (1.8920, 2.4180, 1.1802) -- (1.9410, 2.4180, 1.1775) -- (1.9410, 2.4690, 1.1723) -- (1.8920, 2.4690, 1.1750) -- cycle;
\fill[blue!15.0, opacity=0.7] (1.8920, 2.4690, 1.1750) -- (1.9410, 2.4690, 1.1723) -- (1.9410, 2.5200, 1.1669) -- (1.8920, 2.5200, 1.1696) -- cycle;
\fill[blue!16.2, opacity=0.7] (1.8920, 2.5200, 1.1696) -- (1.9410, 2.5200, 1.1669) -- (1.9410, 2.5710, 1.1614) -- (1.8920, 2.5710, 1.1641) -- cycle;
\fill[blue!32.3, opacity=0.7] (1.8920, 2.5710, 1.1641) -- (1.9410, 2.5710, 1.1614) -- (1.9410, 2.6220, 1.1557) -- (1.8920, 2.6220, 1.1584) -- cycle;
\fill[blue!43.4, opacity=0.7] (1.8920, 2.6220, 1.1584) -- (1.9410, 2.6220, 1.1557) -- (1.9410, 2.6730, 1.1499) -- (1.8920, 2.6730, 1.1526) -- cycle;
\fill[blue!20.9, opacity=0.7] (1.8920, 2.6730, 1.1526) -- (1.9410, 2.6730, 1.1499) -- (1.9410, 2.7240, 1.1440) -- (1.8920, 2.7240, 1.1467) -- cycle;
\fill[blue!15.0, opacity=0.7] (1.8920, 2.7240, 1.1467) -- (1.9410, 2.7240, 1.1440) -- (1.9410, 2.7750, 1.1380) -- (1.8920, 2.7750, 1.1407) -- cycle;
\fill[blue!15.0, opacity=0.7] (1.8920, 2.7750, 1.1407) -- (1.9410, 2.7750, 1.1380) -- (1.9410, 2.8260, 1.1319) -- (1.8920, 2.8260, 1.1346) -- cycle;
\fill[blue!15.0, opacity=0.7] (1.8920, 2.8260, 1.1346) -- (1.9410, 2.8260, 1.1319) -- (1.9410, 2.8770, 1.1257) -- (1.8920, 2.8770, 1.1284) -- cycle;
\fill[blue!15.0, opacity=0.7] (1.8920, 2.8770, 1.1284) -- (1.9410, 2.8770, 1.1257) -- (1.9410, 2.9280, 1.1195) -- (1.8920, 2.9280, 1.1222) -- cycle;
\fill[blue!15.0, opacity=0.7] (1.8920, 2.9280, 1.1222) -- (1.9410, 2.9280, 1.1195) -- (1.9410, 2.9790, 1.1132) -- (1.8920, 2.9790, 1.1159) -- cycle;
\fill[blue!15.0, opacity=0.7] (1.8920, 2.9790, 1.1159) -- (1.9410, 2.9790, 1.1132) -- (1.9410, 3.0300, 1.1069) -- (1.8920, 3.0300, 1.1096) -- cycle;
\fill[blue!15.7, opacity=0.7] (1.9410, -0.0300, 1.1069) -- (1.9900, -0.0300, 1.1039) -- (1.9900, 0.0210, 1.1102) -- (1.9410, 0.0210, 1.1132) -- cycle;
\fill[blue!16.1, opacity=0.7] (1.9410, 0.0210, 1.1132) -- (1.9900, 0.0210, 1.1102) -- (1.9900, 0.0720, 1.1165) -- (1.9410, 0.0720, 1.1195) -- cycle;
\fill[blue!15.1, opacity=0.7] (1.9410, 0.0720, 1.1195) -- (1.9900, 0.0720, 1.1165) -- (1.9900, 0.1230, 1.1227) -- (1.9410, 0.1230, 1.1257) -- cycle;
\fill[blue!15.0, opacity=0.7] (1.9410, 0.1230, 1.1257) -- (1.9900, 0.1230, 1.1227) -- (1.9900, 0.1740, 1.1289) -- (1.9410, 0.1740, 1.1319) -- cycle;
\fill[blue!15.0, opacity=0.7] (1.9410, 0.1740, 1.1319) -- (1.9900, 0.1740, 1.1289) -- (1.9900, 0.2250, 1.1350) -- (1.9410, 0.2250, 1.1380) -- cycle;
\fill[blue!15.0, opacity=0.7] (1.9410, 0.2250, 1.1380) -- (1.9900, 0.2250, 1.1350) -- (1.9900, 0.2760, 1.1410) -- (1.9410, 0.2760, 1.1440) -- cycle;
\fill[blue!15.0, opacity=0.7] (1.9410, 0.2760, 1.1440) -- (1.9900, 0.2760, 1.1410) -- (1.9900, 0.3270, 1.1469) -- (1.9410, 0.3270, 1.1499) -- cycle;
\fill[blue!15.0, opacity=0.7] (1.9410, 0.3270, 1.1499) -- (1.9900, 0.3270, 1.1469) -- (1.9900, 0.3780, 1.1527) -- (1.9410, 0.3780, 1.1557) -- cycle;
\fill[blue!17.1, opacity=0.7] (1.9410, 0.3780, 1.1557) -- (1.9900, 0.3780, 1.1527) -- (1.9900, 0.4290, 1.1584) -- (1.9410, 0.4290, 1.1614) -- cycle;
\fill[blue!44.1, opacity=0.7] (1.9410, 0.4290, 1.1614) -- (1.9900, 0.4290, 1.1584) -- (1.9900, 0.4800, 1.1639) -- (1.9410, 0.4800, 1.1669) -- cycle;
\fill[blue!69.4, opacity=0.7] (1.9410, 0.4800, 1.1669) -- (1.9900, 0.4800, 1.1639) -- (1.9900, 0.5310, 1.1693) -- (1.9410, 0.5310, 1.1723) -- cycle;
\fill[blue!50.4, opacity=0.7] (1.9410, 0.5310, 1.1723) -- (1.9900, 0.5310, 1.1693) -- (1.9900, 0.5820, 1.1745) -- (1.9410, 0.5820, 1.1775) -- cycle;
\fill[blue!21.2, opacity=0.7] (1.9410, 0.5820, 1.1775) -- (1.9900, 0.5820, 1.1745) -- (1.9900, 0.6330, 1.1794) -- (1.9410, 0.6330, 1.1824) -- cycle;
\fill[blue!15.3, opacity=0.7] (1.9410, 0.6330, 1.1824) -- (1.9900, 0.6330, 1.1794) -- (1.9900, 0.6840, 1.1842) -- (1.9410, 0.6840, 1.1872) -- cycle;
\fill[blue!15.0, opacity=0.7] (1.9410, 0.6840, 1.1872) -- (1.9900, 0.6840, 1.1842) -- (1.9900, 0.7350, 1.1888) -- (1.9410, 0.7350, 1.1918) -- cycle;
\fill[blue!15.0, opacity=0.7] (1.9410, 0.7350, 1.1918) -- (1.9900, 0.7350, 1.1888) -- (1.9900, 0.7860, 1.1931) -- (1.9410, 0.7860, 1.1961) -- cycle;
\fill[blue!15.0, opacity=0.7] (1.9410, 0.7860, 1.1961) -- (1.9900, 0.7860, 1.1931) -- (1.9900, 0.8370, 1.1972) -- (1.9410, 0.8370, 1.2002) -- cycle;
\fill[blue!16.3, opacity=0.7] (1.9410, 0.8370, 1.2002) -- (1.9900, 0.8370, 1.1972) -- (1.9900, 0.8880, 1.2010) -- (1.9410, 0.8880, 1.2040) -- cycle;
\fill[blue!35.9, opacity=0.7] (1.9410, 0.8880, 1.2040) -- (1.9900, 0.8880, 1.2010) -- (1.9900, 0.9390, 1.2046) -- (1.9410, 0.9390, 1.2076) -- cycle;
\fill[blue!95.2, opacity=0.7] (1.9410, 0.9390, 1.2076) -- (1.9900, 0.9390, 1.2046) -- (1.9900, 0.9900, 1.2078) -- (1.9410, 0.9900, 1.2108) -- cycle;
\fill[blue!76.2!black, opacity=0.7] (1.9410, 0.9900, 1.2108) -- (1.9900, 0.9900, 1.2078) -- (1.9900, 1.0410, 1.2108) -- (1.9410, 1.0410, 1.2138) -- cycle;
\fill[blue!87.4, opacity=0.7] (1.9410, 1.0410, 1.2138) -- (1.9900, 1.0410, 1.2108) -- (1.9900, 1.0920, 1.2135) -- (1.9410, 1.0920, 1.2165) -- cycle;
\fill[blue!76.6, opacity=0.7] (1.9410, 1.0920, 1.2165) -- (1.9900, 1.0920, 1.2135) -- (1.9900, 1.1430, 1.2160) -- (1.9410, 1.1430, 1.2190) -- cycle;
\fill[blue!79.3, opacity=0.7] (1.9410, 1.1430, 1.2190) -- (1.9900, 1.1430, 1.2160) -- (1.9900, 1.1940, 1.2180) -- (1.9410, 1.1940, 1.2210) -- cycle;
\fill[blue!88.5, opacity=0.7] (1.9410, 1.1940, 1.2210) -- (1.9900, 1.1940, 1.2180) -- (1.9900, 1.2450, 1.2198) -- (1.9410, 1.2450, 1.2228) -- cycle;
\fill[blue!97.3, opacity=0.7] (1.9410, 1.2450, 1.2228) -- (1.9900, 1.2450, 1.2198) -- (1.9900, 1.2960, 1.2213) -- (1.9410, 1.2960, 1.2243) -- cycle;
\fill[blue!95.0!black, opacity=0.7] (1.9410, 1.2960, 1.2243) -- (1.9900, 1.2960, 1.2213) -- (1.9900, 1.3470, 1.2224) -- (1.9410, 1.3470, 1.2254) -- cycle;
\fill[blue!95.3!black, opacity=0.7] (1.9410, 1.3470, 1.2254) -- (1.9900, 1.3470, 1.2224) -- (1.9900, 1.3980, 1.2233) -- (1.9410, 1.3980, 1.2263) -- cycle;
\fill[blue!96.1, opacity=0.7] (1.9410, 1.3980, 1.2263) -- (1.9900, 1.3980, 1.2233) -- (1.9900, 1.4490, 1.2238) -- (1.9410, 1.4490, 1.2268) -- cycle;
\fill[blue!83.5, opacity=0.7] (1.9410, 1.4490, 1.2268) -- (1.9900, 1.4490, 1.2238) -- (1.9900, 1.5000, 1.2239) -- (1.9410, 1.5000, 1.2269) -- cycle;
\fill[blue!66.3, opacity=0.7] (1.9410, 1.5000, 1.2269) -- (1.9900, 1.5000, 1.2239) -- (1.9900, 1.5510, 1.2238) -- (1.9410, 1.5510, 1.2268) -- cycle;
\fill[blue!52.4, opacity=0.7] (1.9410, 1.5510, 1.2268) -- (1.9900, 1.5510, 1.2238) -- (1.9900, 1.6020, 1.2233) -- (1.9410, 1.6020, 1.2263) -- cycle;
\fill[blue!48.9, opacity=0.7] (1.9410, 1.6020, 1.2263) -- (1.9900, 1.6020, 1.2233) -- (1.9900, 1.6530, 1.2224) -- (1.9410, 1.6530, 1.2254) -- cycle;
\fill[blue!63.4, opacity=0.7] (1.9410, 1.6530, 1.2254) -- (1.9900, 1.6530, 1.2224) -- (1.9900, 1.7040, 1.2213) -- (1.9410, 1.7040, 1.2243) -- cycle;
\fill[blue!95.7!black, opacity=0.7] (1.9410, 1.7040, 1.2243) -- (1.9900, 1.7040, 1.2213) -- (1.9900, 1.7550, 1.2198) -- (1.9410, 1.7550, 1.2228) -- cycle;
\fill[blue!95.5, opacity=0.7] (1.9410, 1.7550, 1.2228) -- (1.9900, 1.7550, 1.2198) -- (1.9900, 1.8060, 1.2180) -- (1.9410, 1.8060, 1.2210) -- cycle;
\fill[blue!31.9, opacity=0.7] (1.9410, 1.8060, 1.2210) -- (1.9900, 1.8060, 1.2180) -- (1.9900, 1.8570, 1.2160) -- (1.9410, 1.8570, 1.2190) -- cycle;
\fill[blue!16.3, opacity=0.7] (1.9410, 1.8570, 1.2190) -- (1.9900, 1.8570, 1.2160) -- (1.9900, 1.9080, 1.2135) -- (1.9410, 1.9080, 1.2165) -- cycle;
\fill[blue!15.3, opacity=0.7] (1.9410, 1.9080, 1.2165) -- (1.9900, 1.9080, 1.2135) -- (1.9900, 1.9590, 1.2108) -- (1.9410, 1.9590, 1.2138) -- cycle;
\fill[blue!15.7, opacity=0.7] (1.9410, 1.9590, 1.2138) -- (1.9900, 1.9590, 1.2108) -- (1.9900, 2.0100, 1.2078) -- (1.9410, 2.0100, 1.2108) -- cycle;
\fill[blue!24.3, opacity=0.7] (1.9410, 2.0100, 1.2108) -- (1.9900, 2.0100, 1.2078) -- (1.9900, 2.0610, 1.2046) -- (1.9410, 2.0610, 1.2076) -- cycle;
\fill[blue!76.3, opacity=0.7] (1.9410, 2.0610, 1.2076) -- (1.9900, 2.0610, 1.2046) -- (1.9900, 2.1120, 1.2010) -- (1.9410, 2.1120, 1.2040) -- cycle;
\fill[blue!69.1!black, opacity=0.7] (1.9410, 2.1120, 1.2040) -- (1.9900, 2.1120, 1.2010) -- (1.9900, 2.1630, 1.1972) -- (1.9410, 2.1630, 1.2002) -- cycle;
\fill[blue!68.3!black, opacity=0.7] (1.9410, 2.1630, 1.2002) -- (1.9900, 2.1630, 1.1972) -- (1.9900, 2.2140, 1.1931) -- (1.9410, 2.2140, 1.1961) -- cycle;
\fill[blue!82.3, opacity=0.7] (1.9410, 2.2140, 1.1961) -- (1.9900, 2.2140, 1.1931) -- (1.9900, 2.2650, 1.1888) -- (1.9410, 2.2650, 1.1918) -- cycle;
\fill[blue!20.8, opacity=0.7] (1.9410, 2.2650, 1.1918) -- (1.9900, 2.2650, 1.1888) -- (1.9900, 2.3160, 1.1842) -- (1.9410, 2.3160, 1.1872) -- cycle;
\fill[blue!15.0, opacity=0.7] (1.9410, 2.3160, 1.1872) -- (1.9900, 2.3160, 1.1842) -- (1.9900, 2.3670, 1.1794) -- (1.9410, 2.3670, 1.1824) -- cycle;
\fill[blue!15.0, opacity=0.7] (1.9410, 2.3670, 1.1824) -- (1.9900, 2.3670, 1.1794) -- (1.9900, 2.4180, 1.1745) -- (1.9410, 2.4180, 1.1775) -- cycle;
\fill[blue!15.0, opacity=0.7] (1.9410, 2.4180, 1.1775) -- (1.9900, 2.4180, 1.1745) -- (1.9900, 2.4690, 1.1693) -- (1.9410, 2.4690, 1.1723) -- cycle;
\fill[blue!15.0, opacity=0.7] (1.9410, 2.4690, 1.1723) -- (1.9900, 2.4690, 1.1693) -- (1.9900, 2.5200, 1.1639) -- (1.9410, 2.5200, 1.1669) -- cycle;
\fill[blue!18.2, opacity=0.7] (1.9410, 2.5200, 1.1669) -- (1.9900, 2.5200, 1.1639) -- (1.9900, 2.5710, 1.1584) -- (1.9410, 2.5710, 1.1614) -- cycle;
\fill[blue!38.2, opacity=0.7] (1.9410, 2.5710, 1.1614) -- (1.9900, 2.5710, 1.1584) -- (1.9900, 2.6220, 1.1527) -- (1.9410, 2.6220, 1.1557) -- cycle;
\fill[blue!37.7, opacity=0.7] (1.9410, 2.6220, 1.1557) -- (1.9900, 2.6220, 1.1527) -- (1.9900, 2.6730, 1.1469) -- (1.9410, 2.6730, 1.1499) -- cycle;
\fill[blue!17.2, opacity=0.7] (1.9410, 2.6730, 1.1499) -- (1.9900, 2.6730, 1.1469) -- (1.9900, 2.7240, 1.1410) -- (1.9410, 2.7240, 1.1440) -- cycle;
\fill[blue!15.0, opacity=0.7] (1.9410, 2.7240, 1.1440) -- (1.9900, 2.7240, 1.1410) -- (1.9900, 2.7750, 1.1350) -- (1.9410, 2.7750, 1.1380) -- cycle;
\fill[blue!15.0, opacity=0.7] (1.9410, 2.7750, 1.1380) -- (1.9900, 2.7750, 1.1350) -- (1.9900, 2.8260, 1.1289) -- (1.9410, 2.8260, 1.1319) -- cycle;
\fill[blue!15.0, opacity=0.7] (1.9410, 2.8260, 1.1319) -- (1.9900, 2.8260, 1.1289) -- (1.9900, 2.8770, 1.1227) -- (1.9410, 2.8770, 1.1257) -- cycle;
\fill[blue!15.0, opacity=0.7] (1.9410, 2.8770, 1.1257) -- (1.9900, 2.8770, 1.1227) -- (1.9900, 2.9280, 1.1165) -- (1.9410, 2.9280, 1.1195) -- cycle;
\fill[blue!15.0, opacity=0.7] (1.9410, 2.9280, 1.1195) -- (1.9900, 2.9280, 1.1165) -- (1.9900, 2.9790, 1.1102) -- (1.9410, 2.9790, 1.1132) -- cycle;
\fill[blue!15.0, opacity=0.7] (1.9410, 2.9790, 1.1132) -- (1.9900, 2.9790, 1.1102) -- (1.9900, 3.0300, 1.1039) -- (1.9410, 3.0300, 1.1069) -- cycle;
\fill[blue!15.2, opacity=0.7] (1.9900, -0.0300, 1.1039) -- (2.0390, -0.0300, 1.1006) -- (2.0390, 0.0210, 1.1069) -- (1.9900, 0.0210, 1.1102) -- cycle;
\fill[blue!16.4, opacity=0.7] (1.9900, 0.0210, 1.1102) -- (2.0390, 0.0210, 1.1069) -- (2.0390, 0.0720, 1.1132) -- (1.9900, 0.0720, 1.1165) -- cycle;
\fill[blue!15.4, opacity=0.7] (1.9900, 0.0720, 1.1165) -- (2.0390, 0.0720, 1.1132) -- (2.0390, 0.1230, 1.1194) -- (1.9900, 0.1230, 1.1227) -- cycle;
\fill[blue!15.0, opacity=0.7] (1.9900, 0.1230, 1.1227) -- (2.0390, 0.1230, 1.1194) -- (2.0390, 0.1740, 1.1256) -- (1.9900, 0.1740, 1.1289) -- cycle;
\fill[blue!15.0, opacity=0.7] (1.9900, 0.1740, 1.1289) -- (2.0390, 0.1740, 1.1256) -- (2.0390, 0.2250, 1.1317) -- (1.9900, 0.2250, 1.1350) -- cycle;
\fill[blue!15.0, opacity=0.7] (1.9900, 0.2250, 1.1350) -- (2.0390, 0.2250, 1.1317) -- (2.0390, 0.2760, 1.1377) -- (1.9900, 0.2760, 1.1410) -- cycle;
\fill[blue!15.0, opacity=0.7] (1.9900, 0.2760, 1.1410) -- (2.0390, 0.2760, 1.1377) -- (2.0390, 0.3270, 1.1436) -- (1.9900, 0.3270, 1.1469) -- cycle;
\fill[blue!15.0, opacity=0.7] (1.9900, 0.3270, 1.1469) -- (2.0390, 0.3270, 1.1436) -- (2.0390, 0.3780, 1.1494) -- (1.9900, 0.3780, 1.1527) -- cycle;
\fill[blue!15.1, opacity=0.7] (1.9900, 0.3780, 1.1527) -- (2.0390, 0.3780, 1.1494) -- (2.0390, 0.4290, 1.1551) -- (1.9900, 0.4290, 1.1584) -- cycle;
\fill[blue!23.2, opacity=0.7] (1.9900, 0.4290, 1.1584) -- (2.0390, 0.4290, 1.1551) -- (2.0390, 0.4800, 1.1606) -- (1.9900, 0.4800, 1.1639) -- cycle;
\fill[blue!58.4, opacity=0.7] (1.9900, 0.4800, 1.1639) -- (2.0390, 0.4800, 1.1606) -- (2.0390, 0.5310, 1.1660) -- (1.9900, 0.5310, 1.1693) -- cycle;
\fill[blue!70.0, opacity=0.7] (1.9900, 0.5310, 1.1693) -- (2.0390, 0.5310, 1.1660) -- (2.0390, 0.5820, 1.1712) -- (1.9900, 0.5820, 1.1745) -- cycle;
\fill[blue!42.8, opacity=0.7] (1.9900, 0.5820, 1.1745) -- (2.0390, 0.5820, 1.1712) -- (2.0390, 0.6330, 1.1762) -- (1.9900, 0.6330, 1.1794) -- cycle;
\fill[blue!18.9, opacity=0.7] (1.9900, 0.6330, 1.1794) -- (2.0390, 0.6330, 1.1762) -- (2.0390, 0.6840, 1.1809) -- (1.9900, 0.6840, 1.1842) -- cycle;
\fill[blue!15.2, opacity=0.7] (1.9900, 0.6840, 1.1842) -- (2.0390, 0.6840, 1.1809) -- (2.0390, 0.7350, 1.1855) -- (1.9900, 0.7350, 1.1888) -- cycle;
\fill[blue!15.0, opacity=0.7] (1.9900, 0.7350, 1.1888) -- (2.0390, 0.7350, 1.1855) -- (2.0390, 0.7860, 1.1898) -- (1.9900, 0.7860, 1.1931) -- cycle;
\fill[blue!15.0, opacity=0.7] (1.9900, 0.7860, 1.1931) -- (2.0390, 0.7860, 1.1898) -- (2.0390, 0.8370, 1.1939) -- (1.9900, 0.8370, 1.1972) -- cycle;
\fill[blue!15.0, opacity=0.7] (1.9900, 0.8370, 1.1972) -- (2.0390, 0.8370, 1.1939) -- (2.0390, 0.8880, 1.1977) -- (1.9900, 0.8880, 1.2010) -- cycle;
\fill[blue!15.9, opacity=0.7] (1.9900, 0.8880, 1.2010) -- (2.0390, 0.8880, 1.1977) -- (2.0390, 0.9390, 1.2013) -- (1.9900, 0.9390, 1.2046) -- cycle;
\fill[blue!27.3, opacity=0.7] (1.9900, 0.9390, 1.2046) -- (2.0390, 0.9390, 1.2013) -- (2.0390, 0.9900, 1.2046) -- (1.9900, 0.9900, 1.2078) -- cycle;
\fill[blue!73.2, opacity=0.7] (1.9900, 0.9900, 1.2078) -- (2.0390, 0.9900, 1.2046) -- (2.0390, 1.0410, 1.2076) -- (1.9900, 1.0410, 1.2108) -- cycle;
\fill[blue!71.6!black, opacity=0.7] (1.9900, 1.0410, 1.2108) -- (2.0390, 1.0410, 1.2076) -- (2.0390, 1.0920, 1.2103) -- (1.9900, 1.0920, 1.2135) -- cycle;
\fill[blue!92.5!black, opacity=0.7] (1.9900, 1.0920, 1.2135) -- (2.0390, 1.0920, 1.2103) -- (2.0390, 1.1430, 1.2127) -- (1.9900, 1.1430, 1.2160) -- cycle;
\fill[blue!84.3, opacity=0.7] (1.9900, 1.1430, 1.2160) -- (2.0390, 1.1430, 1.2127) -- (2.0390, 1.1940, 1.2148) -- (1.9900, 1.1940, 1.2180) -- cycle;
\fill[blue!73.0, opacity=0.7] (1.9900, 1.1940, 1.2180) -- (2.0390, 1.1940, 1.2148) -- (2.0390, 1.2450, 1.2166) -- (1.9900, 1.2450, 1.2198) -- cycle;
\fill[blue!67.9, opacity=0.7] (1.9900, 1.2450, 1.2198) -- (2.0390, 1.2450, 1.2166) -- (2.0390, 1.2960, 1.2180) -- (1.9900, 1.2960, 1.2213) -- cycle;
\fill[blue!65.3, opacity=0.7] (1.9900, 1.2960, 1.2213) -- (2.0390, 1.2960, 1.2180) -- (2.0390, 1.3470, 1.2192) -- (1.9900, 1.3470, 1.2224) -- cycle;
\fill[blue!62.8, opacity=0.7] (1.9900, 1.3470, 1.2224) -- (2.0390, 1.3470, 1.2192) -- (2.0390, 1.3980, 1.2200) -- (1.9900, 1.3980, 1.2233) -- cycle;
\fill[blue!59.5, opacity=0.7] (1.9900, 1.3980, 1.2233) -- (2.0390, 1.3980, 1.2200) -- (2.0390, 1.4490, 1.2205) -- (1.9900, 1.4490, 1.2238) -- cycle;
\fill[blue!56.7, opacity=0.7] (1.9900, 1.4490, 1.2238) -- (2.0390, 1.4490, 1.2205) -- (2.0390, 1.5000, 1.2206) -- (1.9900, 1.5000, 1.2239) -- cycle;
\fill[blue!57.3, opacity=0.7] (1.9900, 1.5000, 1.2239) -- (2.0390, 1.5000, 1.2206) -- (2.0390, 1.5510, 1.2205) -- (1.9900, 1.5510, 1.2238) -- cycle;
\fill[blue!66.3, opacity=0.7] (1.9900, 1.5510, 1.2238) -- (2.0390, 1.5510, 1.2205) -- (2.0390, 1.6020, 1.2200) -- (1.9900, 1.6020, 1.2233) -- cycle;
\fill[blue!89.1, opacity=0.7] (1.9900, 1.6020, 1.2233) -- (2.0390, 1.6020, 1.2200) -- (2.0390, 1.6530, 1.2192) -- (1.9900, 1.6530, 1.2224) -- cycle;
\fill[blue!68.6!black, opacity=0.7] (1.9900, 1.6530, 1.2224) -- (2.0390, 1.6530, 1.2192) -- (2.0390, 1.7040, 1.2180) -- (1.9900, 1.7040, 1.2213) -- cycle;
\fill[blue!79.3, opacity=0.7] (1.9900, 1.7040, 1.2213) -- (2.0390, 1.7040, 1.2180) -- (2.0390, 1.7550, 1.2166) -- (1.9900, 1.7550, 1.2198) -- cycle;
\fill[blue!28.5, opacity=0.7] (1.9900, 1.7550, 1.2198) -- (2.0390, 1.7550, 1.2166) -- (2.0390, 1.8060, 1.2148) -- (1.9900, 1.8060, 1.2180) -- cycle;
\fill[blue!16.3, opacity=0.7] (1.9900, 1.8060, 1.2180) -- (2.0390, 1.8060, 1.2148) -- (2.0390, 1.8570, 1.2127) -- (1.9900, 1.8570, 1.2160) -- cycle;
\fill[blue!15.3, opacity=0.7] (1.9900, 1.8570, 1.2160) -- (2.0390, 1.8570, 1.2127) -- (2.0390, 1.9080, 1.2103) -- (1.9900, 1.9080, 1.2135) -- cycle;
\fill[blue!15.4, opacity=0.7] (1.9900, 1.9080, 1.2135) -- (2.0390, 1.9080, 1.2103) -- (2.0390, 1.9590, 1.2076) -- (1.9900, 1.9590, 1.2108) -- cycle;
\fill[blue!18.7, opacity=0.7] (1.9900, 1.9590, 1.2108) -- (2.0390, 1.9590, 1.2076) -- (2.0390, 2.0100, 1.2046) -- (1.9900, 2.0100, 1.2078) -- cycle;
\fill[blue!49.6, opacity=0.7] (1.9900, 2.0100, 1.2078) -- (2.0390, 2.0100, 1.2046) -- (2.0390, 2.0610, 1.2013) -- (1.9900, 2.0610, 1.2046) -- cycle;
\fill[blue!88.3!black, opacity=0.7] (1.9900, 2.0610, 1.2046) -- (2.0390, 2.0610, 1.2013) -- (2.0390, 2.1120, 1.1977) -- (1.9900, 2.1120, 1.2010) -- cycle;
\fill[blue!68.6!black, opacity=0.7] (1.9900, 2.1120, 1.2010) -- (2.0390, 2.1120, 1.1977) -- (2.0390, 2.1630, 1.1939) -- (1.9900, 2.1630, 1.1972) -- cycle;
\fill[blue!84.4!black, opacity=0.7] (1.9900, 2.1630, 1.1972) -- (2.0390, 2.1630, 1.1939) -- (2.0390, 2.2140, 1.1898) -- (1.9900, 2.2140, 1.1931) -- cycle;
\fill[blue!43.3, opacity=0.7] (1.9900, 2.2140, 1.1931) -- (2.0390, 2.2140, 1.1898) -- (2.0390, 2.2650, 1.1855) -- (1.9900, 2.2650, 1.1888) -- cycle;
\fill[blue!15.7, opacity=0.7] (1.9900, 2.2650, 1.1888) -- (2.0390, 2.2650, 1.1855) -- (2.0390, 2.3160, 1.1809) -- (1.9900, 2.3160, 1.1842) -- cycle;
\fill[blue!15.0, opacity=0.7] (1.9900, 2.3160, 1.1842) -- (2.0390, 2.3160, 1.1809) -- (2.0390, 2.3670, 1.1762) -- (1.9900, 2.3670, 1.1794) -- cycle;
\fill[blue!15.0, opacity=0.7] (1.9900, 2.3670, 1.1794) -- (2.0390, 2.3670, 1.1762) -- (2.0390, 2.4180, 1.1712) -- (1.9900, 2.4180, 1.1745) -- cycle;
\fill[blue!15.0, opacity=0.7] (1.9900, 2.4180, 1.1745) -- (2.0390, 2.4180, 1.1712) -- (2.0390, 2.4690, 1.1660) -- (1.9900, 2.4690, 1.1693) -- cycle;
\fill[blue!15.2, opacity=0.7] (1.9900, 2.4690, 1.1693) -- (2.0390, 2.4690, 1.1660) -- (2.0390, 2.5200, 1.1606) -- (1.9900, 2.5200, 1.1639) -- cycle;
\fill[blue!22.9, opacity=0.7] (1.9900, 2.5200, 1.1639) -- (2.0390, 2.5200, 1.1606) -- (2.0390, 2.5710, 1.1551) -- (1.9900, 2.5710, 1.1584) -- cycle;
\fill[blue!41.9, opacity=0.7] (1.9900, 2.5710, 1.1584) -- (2.0390, 2.5710, 1.1551) -- (2.0390, 2.6220, 1.1494) -- (1.9900, 2.6220, 1.1527) -- cycle;
\fill[blue!29.2, opacity=0.7] (1.9900, 2.6220, 1.1527) -- (2.0390, 2.6220, 1.1494) -- (2.0390, 2.6730, 1.1436) -- (1.9900, 2.6730, 1.1469) -- cycle;
\fill[blue!15.5, opacity=0.7] (1.9900, 2.6730, 1.1469) -- (2.0390, 2.6730, 1.1436) -- (2.0390, 2.7240, 1.1377) -- (1.9900, 2.7240, 1.1410) -- cycle;
\fill[blue!15.0, opacity=0.7] (1.9900, 2.7240, 1.1410) -- (2.0390, 2.7240, 1.1377) -- (2.0390, 2.7750, 1.1317) -- (1.9900, 2.7750, 1.1350) -- cycle;
\fill[blue!15.0, opacity=0.7] (1.9900, 2.7750, 1.1350) -- (2.0390, 2.7750, 1.1317) -- (2.0390, 2.8260, 1.1256) -- (1.9900, 2.8260, 1.1289) -- cycle;
\fill[blue!15.0, opacity=0.7] (1.9900, 2.8260, 1.1289) -- (2.0390, 2.8260, 1.1256) -- (2.0390, 2.8770, 1.1194) -- (1.9900, 2.8770, 1.1227) -- cycle;
\fill[blue!15.0, opacity=0.7] (1.9900, 2.8770, 1.1227) -- (2.0390, 2.8770, 1.1194) -- (2.0390, 2.9280, 1.1132) -- (1.9900, 2.9280, 1.1165) -- cycle;
\fill[blue!15.0, opacity=0.7] (1.9900, 2.9280, 1.1165) -- (2.0390, 2.9280, 1.1132) -- (2.0390, 2.9790, 1.1069) -- (1.9900, 2.9790, 1.1102) -- cycle;
\fill[blue!15.1, opacity=0.7] (1.9900, 2.9790, 1.1102) -- (2.0390, 2.9790, 1.1069) -- (2.0390, 3.0300, 1.1006) -- (1.9900, 3.0300, 1.1039) -- cycle;
\fill[blue!15.0, opacity=0.7] (2.0390, -0.0300, 1.1006) -- (2.0880, -0.0300, 1.0971) -- (2.0880, 0.0210, 1.1034) -- (2.0390, 0.0210, 1.1069) -- cycle;
\fill[blue!15.9, opacity=0.7] (2.0390, 0.0210, 1.1069) -- (2.0880, 0.0210, 1.1034) -- (2.0880, 0.0720, 1.1096) -- (2.0390, 0.0720, 1.1132) -- cycle;
\fill[blue!16.2, opacity=0.7] (2.0390, 0.0720, 1.1132) -- (2.0880, 0.0720, 1.1096) -- (2.0880, 0.1230, 1.1159) -- (2.0390, 0.1230, 1.1194) -- cycle;
\fill[blue!15.1, opacity=0.7] (2.0390, 0.1230, 1.1194) -- (2.0880, 0.1230, 1.1159) -- (2.0880, 0.1740, 1.1220) -- (2.0390, 0.1740, 1.1256) -- cycle;
\fill[blue!15.0, opacity=0.7] (2.0390, 0.1740, 1.1256) -- (2.0880, 0.1740, 1.1220) -- (2.0880, 0.2250, 1.1281) -- (2.0390, 0.2250, 1.1317) -- cycle;
\fill[blue!15.0, opacity=0.7] (2.0390, 0.2250, 1.1317) -- (2.0880, 0.2250, 1.1281) -- (2.0880, 0.2760, 1.1342) -- (2.0390, 0.2760, 1.1377) -- cycle;
\fill[blue!15.0, opacity=0.7] (2.0390, 0.2760, 1.1377) -- (2.0880, 0.2760, 1.1342) -- (2.0880, 0.3270, 1.1401) -- (2.0390, 0.3270, 1.1436) -- cycle;
\fill[blue!15.0, opacity=0.7] (2.0390, 0.3270, 1.1436) -- (2.0880, 0.3270, 1.1401) -- (2.0880, 0.3780, 1.1459) -- (2.0390, 0.3780, 1.1494) -- cycle;
\fill[blue!15.0, opacity=0.7] (2.0390, 0.3780, 1.1494) -- (2.0880, 0.3780, 1.1459) -- (2.0880, 0.4290, 1.1516) -- (2.0390, 0.4290, 1.1551) -- cycle;
\fill[blue!15.7, opacity=0.7] (2.0390, 0.4290, 1.1551) -- (2.0880, 0.4290, 1.1516) -- (2.0880, 0.4800, 1.1571) -- (2.0390, 0.4800, 1.1606) -- cycle;
\fill[blue!32.3, opacity=0.7] (2.0390, 0.4800, 1.1606) -- (2.0880, 0.4800, 1.1571) -- (2.0880, 0.5310, 1.1624) -- (2.0390, 0.5310, 1.1660) -- cycle;
\fill[blue!67.1, opacity=0.7] (2.0390, 0.5310, 1.1660) -- (2.0880, 0.5310, 1.1624) -- (2.0880, 0.5820, 1.1676) -- (2.0390, 0.5820, 1.1712) -- cycle;
\fill[blue!69.0, opacity=0.7] (2.0390, 0.5820, 1.1712) -- (2.0880, 0.5820, 1.1676) -- (2.0880, 0.6330, 1.1726) -- (2.0390, 0.6330, 1.1762) -- cycle;
\fill[blue!39.3, opacity=0.7] (2.0390, 0.6330, 1.1762) -- (2.0880, 0.6330, 1.1726) -- (2.0880, 0.6840, 1.1774) -- (2.0390, 0.6840, 1.1809) -- cycle;
\fill[blue!18.4, opacity=0.7] (2.0390, 0.6840, 1.1809) -- (2.0880, 0.6840, 1.1774) -- (2.0880, 0.7350, 1.1819) -- (2.0390, 0.7350, 1.1855) -- cycle;
\fill[blue!15.2, opacity=0.7] (2.0390, 0.7350, 1.1855) -- (2.0880, 0.7350, 1.1819) -- (2.0880, 0.7860, 1.1863) -- (2.0390, 0.7860, 1.1898) -- cycle;
\fill[blue!15.0, opacity=0.7] (2.0390, 0.7860, 1.1898) -- (2.0880, 0.7860, 1.1863) -- (2.0880, 0.8370, 1.1903) -- (2.0390, 0.8370, 1.1939) -- cycle;
\fill[blue!15.0, opacity=0.7] (2.0390, 0.8370, 1.1939) -- (2.0880, 0.8370, 1.1903) -- (2.0880, 0.8880, 1.1942) -- (2.0390, 0.8880, 1.1977) -- cycle;
\fill[blue!15.0, opacity=0.7] (2.0390, 0.8880, 1.1977) -- (2.0880, 0.8880, 1.1942) -- (2.0880, 0.9390, 1.1977) -- (2.0390, 0.9390, 1.2013) -- cycle;
\fill[blue!15.3, opacity=0.7] (2.0390, 0.9390, 1.2013) -- (2.0880, 0.9390, 1.1977) -- (2.0880, 0.9900, 1.2010) -- (2.0390, 0.9900, 1.2046) -- cycle;
\fill[blue!18.9, opacity=0.7] (2.0390, 0.9900, 1.2046) -- (2.0880, 0.9900, 1.2010) -- (2.0880, 1.0410, 1.2040) -- (2.0390, 1.0410, 1.2076) -- cycle;
\fill[blue!38.8, opacity=0.7] (2.0390, 1.0410, 1.2076) -- (2.0880, 1.0410, 1.2040) -- (2.0880, 1.0920, 1.2067) -- (2.0390, 1.0920, 1.2103) -- cycle;
\fill[blue!79.5, opacity=0.7] (2.0390, 1.0920, 1.2103) -- (2.0880, 1.0920, 1.2067) -- (2.0880, 1.1430, 1.2091) -- (2.0390, 1.1430, 1.2127) -- cycle;
\fill[blue!78.5!black, opacity=0.7] (2.0390, 1.1430, 1.2127) -- (2.0880, 1.1430, 1.2091) -- (2.0880, 1.1940, 1.2112) -- (2.0390, 1.1940, 1.2148) -- cycle;
\fill[blue!71.8!black, opacity=0.7] (2.0390, 1.1940, 1.2148) -- (2.0880, 1.1940, 1.2112) -- (2.0880, 1.2450, 1.2130) -- (2.0390, 1.2450, 1.2166) -- cycle;
\fill[blue!94.6!black, opacity=0.7] (2.0390, 1.2450, 1.2166) -- (2.0880, 1.2450, 1.2130) -- (2.0880, 1.2960, 1.2145) -- (2.0390, 1.2960, 1.2180) -- cycle;
\fill[blue!93.7, opacity=0.7] (2.0390, 1.2960, 1.2180) -- (2.0880, 1.2960, 1.2145) -- (2.0880, 1.3470, 1.2156) -- (2.0390, 1.3470, 1.2192) -- cycle;
\fill[blue!89.3, opacity=0.7] (2.0390, 1.3470, 1.2192) -- (2.0880, 1.3470, 1.2156) -- (2.0880, 1.3980, 1.2164) -- (2.0390, 1.3980, 1.2200) -- cycle;
\fill[blue!89.8, opacity=0.7] (2.0390, 1.3980, 1.2200) -- (2.0880, 1.3980, 1.2164) -- (2.0880, 1.4490, 1.2169) -- (2.0390, 1.4490, 1.2205) -- cycle;
\fill[blue!95.9, opacity=0.7] (2.0390, 1.4490, 1.2205) -- (2.0880, 1.4490, 1.2169) -- (2.0880, 1.5000, 1.2171) -- (2.0390, 1.5000, 1.2206) -- cycle;
\fill[blue!83.1!black, opacity=0.7] (2.0390, 1.5000, 1.2206) -- (2.0880, 1.5000, 1.2171) -- (2.0880, 1.5510, 1.2169) -- (2.0390, 1.5510, 1.2205) -- cycle;
\fill[blue!69.4!black, opacity=0.7] (2.0390, 1.5510, 1.2205) -- (2.0880, 1.5510, 1.2169) -- (2.0880, 1.6020, 1.2164) -- (2.0390, 1.6020, 1.2200) -- cycle;
\fill[blue!89.8, opacity=0.7] (2.0390, 1.6020, 1.2200) -- (2.0880, 1.6020, 1.2164) -- (2.0880, 1.6530, 1.2156) -- (2.0390, 1.6530, 1.2192) -- cycle;
\fill[blue!45.6, opacity=0.7] (2.0390, 1.6530, 1.2192) -- (2.0880, 1.6530, 1.2156) -- (2.0880, 1.7040, 1.2145) -- (2.0390, 1.7040, 1.2180) -- cycle;
\fill[blue!20.8, opacity=0.7] (2.0390, 1.7040, 1.2180) -- (2.0880, 1.7040, 1.2145) -- (2.0880, 1.7550, 1.2130) -- (2.0390, 1.7550, 1.2166) -- cycle;
\fill[blue!15.8, opacity=0.7] (2.0390, 1.7550, 1.2166) -- (2.0880, 1.7550, 1.2130) -- (2.0880, 1.8060, 1.2112) -- (2.0390, 1.8060, 1.2148) -- cycle;
\fill[blue!15.2, opacity=0.7] (2.0390, 1.8060, 1.2148) -- (2.0880, 1.8060, 1.2112) -- (2.0880, 1.8570, 1.2091) -- (2.0390, 1.8570, 1.2127) -- cycle;
\fill[blue!15.3, opacity=0.7] (2.0390, 1.8570, 1.2127) -- (2.0880, 1.8570, 1.2091) -- (2.0880, 1.9080, 1.2067) -- (2.0390, 1.9080, 1.2103) -- cycle;
\fill[blue!17.2, opacity=0.7] (2.0390, 1.9080, 1.2103) -- (2.0880, 1.9080, 1.2067) -- (2.0880, 1.9590, 1.2040) -- (2.0390, 1.9590, 1.2076) -- cycle;
\fill[blue!36.2, opacity=0.7] (2.0390, 1.9590, 1.2076) -- (2.0880, 1.9590, 1.2040) -- (2.0880, 2.0100, 1.2010) -- (2.0390, 2.0100, 1.2046) -- cycle;
\fill[blue!90.8, opacity=0.7] (2.0390, 2.0100, 1.2046) -- (2.0880, 2.0100, 1.2010) -- (2.0880, 2.0610, 1.1977) -- (2.0390, 2.0610, 1.2013) -- cycle;
\fill[blue!68.7!black, opacity=0.7] (2.0390, 2.0610, 1.2013) -- (2.0880, 2.0610, 1.1977) -- (2.0880, 2.1120, 1.1942) -- (2.0390, 2.1120, 1.1977) -- cycle;
\fill[blue!70.7!black, opacity=0.7] (2.0390, 2.1120, 1.1977) -- (2.0880, 2.1120, 1.1942) -- (2.0880, 2.1630, 1.1903) -- (2.0390, 2.1630, 1.1939) -- cycle;
\fill[blue!72.8, opacity=0.7] (2.0390, 2.1630, 1.1939) -- (2.0880, 2.1630, 1.1903) -- (2.0880, 2.2140, 1.1863) -- (2.0390, 2.2140, 1.1898) -- cycle;
\fill[blue!19.8, opacity=0.7] (2.0390, 2.2140, 1.1898) -- (2.0880, 2.2140, 1.1863) -- (2.0880, 2.2650, 1.1819) -- (2.0390, 2.2650, 1.1855) -- cycle;
\fill[blue!15.0, opacity=0.7] (2.0390, 2.2650, 1.1855) -- (2.0880, 2.2650, 1.1819) -- (2.0880, 2.3160, 1.1774) -- (2.0390, 2.3160, 1.1809) -- cycle;
\fill[blue!15.0, opacity=0.7] (2.0390, 2.3160, 1.1809) -- (2.0880, 2.3160, 1.1774) -- (2.0880, 2.3670, 1.1726) -- (2.0390, 2.3670, 1.1762) -- cycle;
\fill[blue!15.0, opacity=0.7] (2.0390, 2.3670, 1.1762) -- (2.0880, 2.3670, 1.1726) -- (2.0880, 2.4180, 1.1676) -- (2.0390, 2.4180, 1.1712) -- cycle;
\fill[blue!15.0, opacity=0.7] (2.0390, 2.4180, 1.1712) -- (2.0880, 2.4180, 1.1676) -- (2.0880, 2.4690, 1.1624) -- (2.0390, 2.4690, 1.1660) -- cycle;
\fill[blue!16.2, opacity=0.7] (2.0390, 2.4690, 1.1660) -- (2.0880, 2.4690, 1.1624) -- (2.0880, 2.5200, 1.1571) -- (2.0390, 2.5200, 1.1606) -- cycle;
\fill[blue!30.6, opacity=0.7] (2.0390, 2.5200, 1.1606) -- (2.0880, 2.5200, 1.1571) -- (2.0880, 2.5710, 1.1516) -- (2.0390, 2.5710, 1.1551) -- cycle;
\fill[blue!40.4, opacity=0.7] (2.0390, 2.5710, 1.1551) -- (2.0880, 2.5710, 1.1516) -- (2.0880, 2.6220, 1.1459) -- (2.0390, 2.6220, 1.1494) -- cycle;
\fill[blue!21.0, opacity=0.7] (2.0390, 2.6220, 1.1494) -- (2.0880, 2.6220, 1.1459) -- (2.0880, 2.6730, 1.1401) -- (2.0390, 2.6730, 1.1436) -- cycle;
\fill[blue!15.1, opacity=0.7] (2.0390, 2.6730, 1.1436) -- (2.0880, 2.6730, 1.1401) -- (2.0880, 2.7240, 1.1342) -- (2.0390, 2.7240, 1.1377) -- cycle;
\fill[blue!15.0, opacity=0.7] (2.0390, 2.7240, 1.1377) -- (2.0880, 2.7240, 1.1342) -- (2.0880, 2.7750, 1.1281) -- (2.0390, 2.7750, 1.1317) -- cycle;
\fill[blue!15.0, opacity=0.7] (2.0390, 2.7750, 1.1317) -- (2.0880, 2.7750, 1.1281) -- (2.0880, 2.8260, 1.1220) -- (2.0390, 2.8260, 1.1256) -- cycle;
\fill[blue!15.0, opacity=0.7] (2.0390, 2.8260, 1.1256) -- (2.0880, 2.8260, 1.1220) -- (2.0880, 2.8770, 1.1159) -- (2.0390, 2.8770, 1.1194) -- cycle;
\fill[blue!15.0, opacity=0.7] (2.0390, 2.8770, 1.1194) -- (2.0880, 2.8770, 1.1159) -- (2.0880, 2.9280, 1.1096) -- (2.0390, 2.9280, 1.1132) -- cycle;
\fill[blue!15.0, opacity=0.7] (2.0390, 2.9280, 1.1132) -- (2.0880, 2.9280, 1.1096) -- (2.0880, 2.9790, 1.1034) -- (2.0390, 2.9790, 1.1069) -- cycle;
\fill[blue!15.2, opacity=0.7] (2.0390, 2.9790, 1.1069) -- (2.0880, 2.9790, 1.1034) -- (2.0880, 3.0300, 1.0971) -- (2.0390, 3.0300, 1.1006) -- cycle;
\fill[blue!15.0, opacity=0.7] (2.0880, -0.0300, 1.0971) -- (2.1370, -0.0300, 1.0933) -- (2.1370, 0.0210, 1.0995) -- (2.0880, 0.0210, 1.1034) -- cycle;
\fill[blue!15.3, opacity=0.7] (2.0880, 0.0210, 1.1034) -- (2.1370, 0.0210, 1.0995) -- (2.1370, 0.0720, 1.1058) -- (2.0880, 0.0720, 1.1096) -- cycle;
\fill[blue!16.6, opacity=0.7] (2.0880, 0.0720, 1.1096) -- (2.1370, 0.0720, 1.1058) -- (2.1370, 0.1230, 1.1120) -- (2.0880, 0.1230, 1.1159) -- cycle;
\fill[blue!15.6, opacity=0.7] (2.0880, 0.1230, 1.1159) -- (2.1370, 0.1230, 1.1120) -- (2.1370, 0.1740, 1.1182) -- (2.0880, 0.1740, 1.1220) -- cycle;
\fill[blue!15.0, opacity=0.7] (2.0880, 0.1740, 1.1220) -- (2.1370, 0.1740, 1.1182) -- (2.1370, 0.2250, 1.1243) -- (2.0880, 0.2250, 1.1281) -- cycle;
\fill[blue!15.0, opacity=0.7] (2.0880, 0.2250, 1.1281) -- (2.1370, 0.2250, 1.1243) -- (2.1370, 0.2760, 1.1303) -- (2.0880, 0.2760, 1.1342) -- cycle;
\fill[blue!15.0, opacity=0.7] (2.0880, 0.2760, 1.1342) -- (2.1370, 0.2760, 1.1303) -- (2.1370, 0.3270, 1.1363) -- (2.0880, 0.3270, 1.1401) -- cycle;
\fill[blue!15.0, opacity=0.7] (2.0880, 0.3270, 1.1401) -- (2.1370, 0.3270, 1.1363) -- (2.1370, 0.3780, 1.1421) -- (2.0880, 0.3780, 1.1459) -- cycle;
\fill[blue!15.0, opacity=0.7] (2.0880, 0.3780, 1.1459) -- (2.1370, 0.3780, 1.1421) -- (2.1370, 0.4290, 1.1477) -- (2.0880, 0.4290, 1.1516) -- cycle;
\fill[blue!15.0, opacity=0.7] (2.0880, 0.4290, 1.1516) -- (2.1370, 0.4290, 1.1477) -- (2.1370, 0.4800, 1.1533) -- (2.0880, 0.4800, 1.1571) -- cycle;
\fill[blue!16.9, opacity=0.7] (2.0880, 0.4800, 1.1571) -- (2.1370, 0.4800, 1.1533) -- (2.1370, 0.5310, 1.1586) -- (2.0880, 0.5310, 1.1624) -- cycle;
\fill[blue!40.3, opacity=0.7] (2.0880, 0.5310, 1.1624) -- (2.1370, 0.5310, 1.1586) -- (2.1370, 0.5820, 1.1638) -- (2.0880, 0.5820, 1.1676) -- cycle;
\fill[blue!71.8, opacity=0.7] (2.0880, 0.5820, 1.1676) -- (2.1370, 0.5820, 1.1638) -- (2.1370, 0.6330, 1.1688) -- (2.0880, 0.6330, 1.1726) -- cycle;
\fill[blue!69.6, opacity=0.7] (2.0880, 0.6330, 1.1726) -- (2.1370, 0.6330, 1.1688) -- (2.1370, 0.6840, 1.1736) -- (2.0880, 0.6840, 1.1774) -- cycle;
\fill[blue!40.3, opacity=0.7] (2.0880, 0.6840, 1.1774) -- (2.1370, 0.6840, 1.1736) -- (2.1370, 0.7350, 1.1781) -- (2.0880, 0.7350, 1.1819) -- cycle;
\fill[blue!19.4, opacity=0.7] (2.0880, 0.7350, 1.1819) -- (2.1370, 0.7350, 1.1781) -- (2.1370, 0.7860, 1.1824) -- (2.0880, 0.7860, 1.1863) -- cycle;
\fill[blue!15.4, opacity=0.7] (2.0880, 0.7860, 1.1863) -- (2.1370, 0.7860, 1.1824) -- (2.1370, 0.8370, 1.1865) -- (2.0880, 0.8370, 1.1903) -- cycle;
\fill[blue!15.0, opacity=0.7] (2.0880, 0.8370, 1.1903) -- (2.1370, 0.8370, 1.1865) -- (2.1370, 0.8880, 1.1903) -- (2.0880, 0.8880, 1.1942) -- cycle;
\fill[blue!15.0, opacity=0.7] (2.0880, 0.8880, 1.1942) -- (2.1370, 0.8880, 1.1903) -- (2.1370, 0.9390, 1.1939) -- (2.0880, 0.9390, 1.1977) -- cycle;
\fill[blue!15.0, opacity=0.7] (2.0880, 0.9390, 1.1977) -- (2.1370, 0.9390, 1.1939) -- (2.1370, 0.9900, 1.1972) -- (2.0880, 0.9900, 1.2010) -- cycle;
\fill[blue!15.1, opacity=0.7] (2.0880, 0.9900, 1.2010) -- (2.1370, 0.9900, 1.1972) -- (2.1370, 1.0410, 1.2002) -- (2.0880, 1.0410, 1.2040) -- cycle;
\fill[blue!15.6, opacity=0.7] (2.0880, 1.0410, 1.2040) -- (2.1370, 1.0410, 1.2002) -- (2.1370, 1.0920, 1.2029) -- (2.0880, 1.0920, 1.2067) -- cycle;
\fill[blue!18.7, opacity=0.7] (2.0880, 1.0920, 1.2067) -- (2.1370, 1.0920, 1.2029) -- (2.1370, 1.1430, 1.2053) -- (2.0880, 1.1430, 1.2091) -- cycle;
\fill[blue!29.2, opacity=0.7] (2.0880, 1.1430, 1.2091) -- (2.1370, 1.1430, 1.2053) -- (2.1370, 1.1940, 1.2074) -- (2.0880, 1.1940, 1.2112) -- cycle;
\fill[blue!48.5, opacity=0.7] (2.0880, 1.1940, 1.2112) -- (2.1370, 1.1940, 1.2074) -- (2.1370, 1.2450, 1.2092) -- (2.0880, 1.2450, 1.2130) -- cycle;
\fill[blue!69.0, opacity=0.7] (2.0880, 1.2450, 1.2130) -- (2.1370, 1.2450, 1.2092) -- (2.1370, 1.2960, 1.2106) -- (2.0880, 1.2960, 1.2145) -- cycle;
\fill[blue!83.0, opacity=0.7] (2.0880, 1.2960, 1.2145) -- (2.1370, 1.2960, 1.2106) -- (2.1370, 1.3470, 1.2118) -- (2.0880, 1.3470, 1.2156) -- cycle;
\fill[blue!88.5, opacity=0.7] (2.0880, 1.3470, 1.2156) -- (2.1370, 1.3470, 1.2118) -- (2.1370, 1.3980, 1.2126) -- (2.0880, 1.3980, 1.2164) -- cycle;
\fill[blue!86.1, opacity=0.7] (2.0880, 1.3980, 1.2164) -- (2.1370, 1.3980, 1.2126) -- (2.1370, 1.4490, 1.2131) -- (2.0880, 1.4490, 1.2169) -- cycle;
\fill[blue!75.0, opacity=0.7] (2.0880, 1.4490, 1.2169) -- (2.1370, 1.4490, 1.2131) -- (2.1370, 1.5000, 1.2133) -- (2.0880, 1.5000, 1.2171) -- cycle;
\fill[blue!55.7, opacity=0.7] (2.0880, 1.5000, 1.2171) -- (2.1370, 1.5000, 1.2133) -- (2.1370, 1.5510, 1.2131) -- (2.0880, 1.5510, 1.2169) -- cycle;
\fill[blue!34.4, opacity=0.7] (2.0880, 1.5510, 1.2169) -- (2.1370, 1.5510, 1.2131) -- (2.1370, 1.6020, 1.2126) -- (2.0880, 1.6020, 1.2164) -- cycle;
\fill[blue!20.9, opacity=0.7] (2.0880, 1.6020, 1.2164) -- (2.1370, 1.6020, 1.2126) -- (2.1370, 1.6530, 1.2118) -- (2.0880, 1.6530, 1.2156) -- cycle;
\fill[blue!16.3, opacity=0.7] (2.0880, 1.6530, 1.2156) -- (2.1370, 1.6530, 1.2118) -- (2.1370, 1.7040, 1.2106) -- (2.0880, 1.7040, 1.2145) -- cycle;
\fill[blue!15.3, opacity=0.7] (2.0880, 1.7040, 1.2145) -- (2.1370, 1.7040, 1.2106) -- (2.1370, 1.7550, 1.2092) -- (2.0880, 1.7550, 1.2130) -- cycle;
\fill[blue!15.2, opacity=0.7] (2.0880, 1.7550, 1.2130) -- (2.1370, 1.7550, 1.2092) -- (2.1370, 1.8060, 1.2074) -- (2.0880, 1.8060, 1.2112) -- cycle;
\fill[blue!15.3, opacity=0.7] (2.0880, 1.8060, 1.2112) -- (2.1370, 1.8060, 1.2074) -- (2.1370, 1.8570, 1.2053) -- (2.0880, 1.8570, 1.2091) -- cycle;
\fill[blue!17.0, opacity=0.7] (2.0880, 1.8570, 1.2091) -- (2.1370, 1.8570, 1.2053) -- (2.1370, 1.9080, 1.2029) -- (2.0880, 1.9080, 1.2067) -- cycle;
\fill[blue!32.0, opacity=0.7] (2.0880, 1.9080, 1.2067) -- (2.1370, 1.9080, 1.2029) -- (2.1370, 1.9590, 1.2002) -- (2.0880, 1.9590, 1.2040) -- cycle;
\fill[blue!80.8, opacity=0.7] (2.0880, 1.9590, 1.2040) -- (2.1370, 1.9590, 1.2002) -- (2.1370, 2.0100, 1.1972) -- (2.0880, 2.0100, 1.2010) -- cycle;
\fill[blue!72.6!black, opacity=0.7] (2.0880, 2.0100, 1.2010) -- (2.1370, 2.0100, 1.1972) -- (2.1370, 2.0610, 1.1939) -- (2.0880, 2.0610, 1.1977) -- cycle;
\fill[blue!69.1!black, opacity=0.7] (2.0880, 2.0610, 1.1977) -- (2.1370, 2.0610, 1.1939) -- (2.1370, 2.1120, 1.1903) -- (2.0880, 2.1120, 1.1942) -- cycle;
\fill[blue!90.9, opacity=0.7] (2.0880, 2.1120, 1.1942) -- (2.1370, 2.1120, 1.1903) -- (2.1370, 2.1630, 1.1865) -- (2.0880, 2.1630, 1.1903) -- cycle;
\fill[blue!29.8, opacity=0.7] (2.0880, 2.1630, 1.1903) -- (2.1370, 2.1630, 1.1865) -- (2.1370, 2.2140, 1.1824) -- (2.0880, 2.2140, 1.1863) -- cycle;
\fill[blue!15.3, opacity=0.7] (2.0880, 2.2140, 1.1863) -- (2.1370, 2.2140, 1.1824) -- (2.1370, 2.2650, 1.1781) -- (2.0880, 2.2650, 1.1819) -- cycle;
\fill[blue!15.0, opacity=0.7] (2.0880, 2.2650, 1.1819) -- (2.1370, 2.2650, 1.1781) -- (2.1370, 2.3160, 1.1736) -- (2.0880, 2.3160, 1.1774) -- cycle;
\fill[blue!15.0, opacity=0.7] (2.0880, 2.3160, 1.1774) -- (2.1370, 2.3160, 1.1736) -- (2.1370, 2.3670, 1.1688) -- (2.0880, 2.3670, 1.1726) -- cycle;
\fill[blue!15.0, opacity=0.7] (2.0880, 2.3670, 1.1726) -- (2.1370, 2.3670, 1.1688) -- (2.1370, 2.4180, 1.1638) -- (2.0880, 2.4180, 1.1676) -- cycle;
\fill[blue!15.1, opacity=0.7] (2.0880, 2.4180, 1.1676) -- (2.1370, 2.4180, 1.1638) -- (2.1370, 2.4690, 1.1586) -- (2.0880, 2.4690, 1.1624) -- cycle;
\fill[blue!20.0, opacity=0.7] (2.0880, 2.4690, 1.1624) -- (2.1370, 2.4690, 1.1586) -- (2.1370, 2.5200, 1.1533) -- (2.0880, 2.5200, 1.1571) -- cycle;
\fill[blue!38.0, opacity=0.7] (2.0880, 2.5200, 1.1571) -- (2.1370, 2.5200, 1.1533) -- (2.1370, 2.5710, 1.1477) -- (2.0880, 2.5710, 1.1516) -- cycle;
\fill[blue!32.6, opacity=0.7] (2.0880, 2.5710, 1.1516) -- (2.1370, 2.5710, 1.1477) -- (2.1370, 2.6220, 1.1421) -- (2.0880, 2.6220, 1.1459) -- cycle;
\fill[blue!16.4, opacity=0.7] (2.0880, 2.6220, 1.1459) -- (2.1370, 2.6220, 1.1421) -- (2.1370, 2.6730, 1.1363) -- (2.0880, 2.6730, 1.1401) -- cycle;
\fill[blue!15.0, opacity=0.7] (2.0880, 2.6730, 1.1401) -- (2.1370, 2.6730, 1.1363) -- (2.1370, 2.7240, 1.1303) -- (2.0880, 2.7240, 1.1342) -- cycle;
\fill[blue!15.0, opacity=0.7] (2.0880, 2.7240, 1.1342) -- (2.1370, 2.7240, 1.1303) -- (2.1370, 2.7750, 1.1243) -- (2.0880, 2.7750, 1.1281) -- cycle;
\fill[blue!15.0, opacity=0.7] (2.0880, 2.7750, 1.1281) -- (2.1370, 2.7750, 1.1243) -- (2.1370, 2.8260, 1.1182) -- (2.0880, 2.8260, 1.1220) -- cycle;
\fill[blue!15.0, opacity=0.7] (2.0880, 2.8260, 1.1220) -- (2.1370, 2.8260, 1.1182) -- (2.1370, 2.8770, 1.1120) -- (2.0880, 2.8770, 1.1159) -- cycle;
\fill[blue!15.0, opacity=0.7] (2.0880, 2.8770, 1.1159) -- (2.1370, 2.8770, 1.1120) -- (2.1370, 2.9280, 1.1058) -- (2.0880, 2.9280, 1.1096) -- cycle;
\fill[blue!15.0, opacity=0.7] (2.0880, 2.9280, 1.1096) -- (2.1370, 2.9280, 1.1058) -- (2.1370, 2.9790, 1.0995) -- (2.0880, 2.9790, 1.1034) -- cycle;
\fill[blue!15.3, opacity=0.7] (2.0880, 2.9790, 1.1034) -- (2.1370, 2.9790, 1.0995) -- (2.1370, 3.0300, 1.0933) -- (2.0880, 3.0300, 1.0971) -- cycle;
\fill[blue!15.0, opacity=0.7] (2.1370, -0.0300, 1.0933) -- (2.1860, -0.0300, 1.0892) -- (2.1860, 0.0210, 1.0955) -- (2.1370, 0.0210, 1.0995) -- cycle;
\fill[blue!15.0, opacity=0.7] (2.1370, 0.0210, 1.0995) -- (2.1860, 0.0210, 1.0955) -- (2.1860, 0.0720, 1.1017) -- (2.1370, 0.0720, 1.1058) -- cycle;
\fill[blue!15.9, opacity=0.7] (2.1370, 0.0720, 1.1058) -- (2.1860, 0.0720, 1.1017) -- (2.1860, 0.1230, 1.1079) -- (2.1370, 0.1230, 1.1120) -- cycle;
\fill[blue!16.5, opacity=0.7] (2.1370, 0.1230, 1.1120) -- (2.1860, 0.1230, 1.1079) -- (2.1860, 0.1740, 1.1141) -- (2.1370, 0.1740, 1.1182) -- cycle;
\fill[blue!15.2, opacity=0.7] (2.1370, 0.1740, 1.1182) -- (2.1860, 0.1740, 1.1141) -- (2.1860, 0.2250, 1.1202) -- (2.1370, 0.2250, 1.1243) -- cycle;
\fill[blue!15.0, opacity=0.7] (2.1370, 0.2250, 1.1243) -- (2.1860, 0.2250, 1.1202) -- (2.1860, 0.2760, 1.1263) -- (2.1370, 0.2760, 1.1303) -- cycle;
\fill[blue!15.0, opacity=0.7] (2.1370, 0.2760, 1.1303) -- (2.1860, 0.2760, 1.1263) -- (2.1860, 0.3270, 1.1322) -- (2.1370, 0.3270, 1.1363) -- cycle;
\fill[blue!15.0, opacity=0.7] (2.1370, 0.3270, 1.1363) -- (2.1860, 0.3270, 1.1322) -- (2.1860, 0.3780, 1.1380) -- (2.1370, 0.3780, 1.1421) -- cycle;
\fill[blue!15.0, opacity=0.7] (2.1370, 0.3780, 1.1421) -- (2.1860, 0.3780, 1.1380) -- (2.1860, 0.4290, 1.1437) -- (2.1370, 0.4290, 1.1477) -- cycle;
\fill[blue!15.0, opacity=0.7] (2.1370, 0.4290, 1.1477) -- (2.1860, 0.4290, 1.1437) -- (2.1860, 0.4800, 1.1492) -- (2.1370, 0.4800, 1.1533) -- cycle;
\fill[blue!15.0, opacity=0.7] (2.1370, 0.4800, 1.1533) -- (2.1860, 0.4800, 1.1492) -- (2.1860, 0.5310, 1.1545) -- (2.1370, 0.5310, 1.1586) -- cycle;
\fill[blue!18.4, opacity=0.7] (2.1370, 0.5310, 1.1586) -- (2.1860, 0.5310, 1.1545) -- (2.1860, 0.5820, 1.1597) -- (2.1370, 0.5820, 1.1638) -- cycle;
\fill[blue!44.8, opacity=0.7] (2.1370, 0.5820, 1.1638) -- (2.1860, 0.5820, 1.1597) -- (2.1860, 0.6330, 1.1647) -- (2.1370, 0.6330, 1.1688) -- cycle;
\fill[blue!74.3, opacity=0.7] (2.1370, 0.6330, 1.1688) -- (2.1860, 0.6330, 1.1647) -- (2.1860, 0.6840, 1.1695) -- (2.1370, 0.6840, 1.1736) -- cycle;
\fill[blue!72.9, opacity=0.7] (2.1370, 0.6840, 1.1736) -- (2.1860, 0.6840, 1.1695) -- (2.1860, 0.7350, 1.1740) -- (2.1370, 0.7350, 1.1781) -- cycle;
\fill[blue!46.3, opacity=0.7] (2.1370, 0.7350, 1.1781) -- (2.1860, 0.7350, 1.1740) -- (2.1860, 0.7860, 1.1784) -- (2.1370, 0.7860, 1.1824) -- cycle;
\fill[blue!22.7, opacity=0.7] (2.1370, 0.7860, 1.1824) -- (2.1860, 0.7860, 1.1784) -- (2.1860, 0.8370, 1.1824) -- (2.1370, 0.8370, 1.1865) -- cycle;
\fill[blue!16.0, opacity=0.7] (2.1370, 0.8370, 1.1865) -- (2.1860, 0.8370, 1.1824) -- (2.1860, 0.8880, 1.1863) -- (2.1370, 0.8880, 1.1903) -- cycle;
\fill[blue!15.1, opacity=0.7] (2.1370, 0.8880, 1.1903) -- (2.1860, 0.8880, 1.1863) -- (2.1860, 0.9390, 1.1898) -- (2.1370, 0.9390, 1.1939) -- cycle;
\fill[blue!15.0, opacity=0.7] (2.1370, 0.9390, 1.1939) -- (2.1860, 0.9390, 1.1898) -- (2.1860, 0.9900, 1.1931) -- (2.1370, 0.9900, 1.1972) -- cycle;
\fill[blue!15.0, opacity=0.7] (2.1370, 0.9900, 1.1972) -- (2.1860, 0.9900, 1.1931) -- (2.1860, 1.0410, 1.1961) -- (2.1370, 1.0410, 1.2002) -- cycle;
\fill[blue!15.0, opacity=0.7] (2.1370, 1.0410, 1.2002) -- (2.1860, 1.0410, 1.1961) -- (2.1860, 1.0920, 1.1988) -- (2.1370, 1.0920, 1.2029) -- cycle;
\fill[blue!15.1, opacity=0.7] (2.1370, 1.0920, 1.2029) -- (2.1860, 1.0920, 1.1988) -- (2.1860, 1.1430, 1.2012) -- (2.1370, 1.1430, 1.2053) -- cycle;
\fill[blue!15.3, opacity=0.7] (2.1370, 1.1430, 1.2053) -- (2.1860, 1.1430, 1.2012) -- (2.1860, 1.1940, 1.2033) -- (2.1370, 1.1940, 1.2074) -- cycle;
\fill[blue!15.8, opacity=0.7] (2.1370, 1.1940, 1.2074) -- (2.1860, 1.1940, 1.2033) -- (2.1860, 1.2450, 1.2051) -- (2.1370, 1.2450, 1.2092) -- cycle;
\fill[blue!16.9, opacity=0.7] (2.1370, 1.2450, 1.2092) -- (2.1860, 1.2450, 1.2051) -- (2.1860, 1.2960, 1.2066) -- (2.1370, 1.2960, 1.2106) -- cycle;
\fill[blue!18.2, opacity=0.7] (2.1370, 1.2960, 1.2106) -- (2.1860, 1.2960, 1.2066) -- (2.1860, 1.3470, 1.2077) -- (2.1370, 1.3470, 1.2118) -- cycle;
\fill[blue!19.0, opacity=0.7] (2.1370, 1.3470, 1.2118) -- (2.1860, 1.3470, 1.2077) -- (2.1860, 1.3980, 1.2085) -- (2.1370, 1.3980, 1.2126) -- cycle;
\fill[blue!18.7, opacity=0.7] (2.1370, 1.3980, 1.2126) -- (2.1860, 1.3980, 1.2085) -- (2.1860, 1.4490, 1.2090) -- (2.1370, 1.4490, 1.2131) -- cycle;
\fill[blue!17.5, opacity=0.7] (2.1370, 1.4490, 1.2131) -- (2.1860, 1.4490, 1.2090) -- (2.1860, 1.5000, 1.2092) -- (2.1370, 1.5000, 1.2133) -- cycle;
\fill[blue!16.3, opacity=0.7] (2.1370, 1.5000, 1.2133) -- (2.1860, 1.5000, 1.2092) -- (2.1860, 1.5510, 1.2090) -- (2.1370, 1.5510, 1.2131) -- cycle;
\fill[blue!15.5, opacity=0.7] (2.1370, 1.5510, 1.2131) -- (2.1860, 1.5510, 1.2090) -- (2.1860, 1.6020, 1.2085) -- (2.1370, 1.6020, 1.2126) -- cycle;
\fill[blue!15.2, opacity=0.7] (2.1370, 1.6020, 1.2126) -- (2.1860, 1.6020, 1.2085) -- (2.1860, 1.6530, 1.2077) -- (2.1370, 1.6530, 1.2118) -- cycle;
\fill[blue!15.1, opacity=0.7] (2.1370, 1.6530, 1.2118) -- (2.1860, 1.6530, 1.2077) -- (2.1860, 1.7040, 1.2066) -- (2.1370, 1.7040, 1.2106) -- cycle;
\fill[blue!15.1, opacity=0.7] (2.1370, 1.7040, 1.2106) -- (2.1860, 1.7040, 1.2066) -- (2.1860, 1.7550, 1.2051) -- (2.1370, 1.7550, 1.2092) -- cycle;
\fill[blue!15.4, opacity=0.7] (2.1370, 1.7550, 1.2092) -- (2.1860, 1.7550, 1.2051) -- (2.1860, 1.8060, 1.2033) -- (2.1370, 1.8060, 1.2074) -- cycle;
\fill[blue!17.9, opacity=0.7] (2.1370, 1.8060, 1.2074) -- (2.1860, 1.8060, 1.2033) -- (2.1860, 1.8570, 1.2012) -- (2.1370, 1.8570, 1.2053) -- cycle;
\fill[blue!34.3, opacity=0.7] (2.1370, 1.8570, 1.2053) -- (2.1860, 1.8570, 1.2012) -- (2.1860, 1.9080, 1.1988) -- (2.1370, 1.9080, 1.2029) -- cycle;
\fill[blue!79.1, opacity=0.7] (2.1370, 1.9080, 1.2029) -- (2.1860, 1.9080, 1.1988) -- (2.1860, 1.9590, 1.1961) -- (2.1370, 1.9590, 1.2002) -- cycle;
\fill[blue!76.6!black, opacity=0.7] (2.1370, 1.9590, 1.2002) -- (2.1860, 1.9590, 1.1961) -- (2.1860, 2.0100, 1.1931) -- (2.1370, 2.0100, 1.1972) -- cycle;
\fill[blue!69.4!black, opacity=0.7] (2.1370, 2.0100, 1.1972) -- (2.1860, 2.0100, 1.1931) -- (2.1860, 2.0610, 1.1898) -- (2.1370, 2.0610, 1.1939) -- cycle;
\fill[blue!97.7, opacity=0.7] (2.1370, 2.0610, 1.1939) -- (2.1860, 2.0610, 1.1898) -- (2.1860, 2.1120, 1.1863) -- (2.1370, 2.1120, 1.1903) -- cycle;
\fill[blue!40.6, opacity=0.7] (2.1370, 2.1120, 1.1903) -- (2.1860, 2.1120, 1.1863) -- (2.1860, 2.1630, 1.1824) -- (2.1370, 2.1630, 1.1865) -- cycle;
\fill[blue!16.0, opacity=0.7] (2.1370, 2.1630, 1.1865) -- (2.1860, 2.1630, 1.1824) -- (2.1860, 2.2140, 1.1784) -- (2.1370, 2.2140, 1.1824) -- cycle;
\fill[blue!15.0, opacity=0.7] (2.1370, 2.2140, 1.1824) -- (2.1860, 2.2140, 1.1784) -- (2.1860, 2.2650, 1.1740) -- (2.1370, 2.2650, 1.1781) -- cycle;
\fill[blue!15.0, opacity=0.7] (2.1370, 2.2650, 1.1781) -- (2.1860, 2.2650, 1.1740) -- (2.1860, 2.3160, 1.1695) -- (2.1370, 2.3160, 1.1736) -- cycle;
\fill[blue!15.0, opacity=0.7] (2.1370, 2.3160, 1.1736) -- (2.1860, 2.3160, 1.1695) -- (2.1860, 2.3670, 1.1647) -- (2.1370, 2.3670, 1.1688) -- cycle;
\fill[blue!15.0, opacity=0.7] (2.1370, 2.3670, 1.1688) -- (2.1860, 2.3670, 1.1647) -- (2.1860, 2.4180, 1.1597) -- (2.1370, 2.4180, 1.1638) -- cycle;
\fill[blue!16.1, opacity=0.7] (2.1370, 2.4180, 1.1638) -- (2.1860, 2.4180, 1.1597) -- (2.1860, 2.4690, 1.1545) -- (2.1370, 2.4690, 1.1586) -- cycle;
\fill[blue!28.2, opacity=0.7] (2.1370, 2.4690, 1.1586) -- (2.1860, 2.4690, 1.1545) -- (2.1860, 2.5200, 1.1492) -- (2.1370, 2.5200, 1.1533) -- cycle;
\fill[blue!39.0, opacity=0.7] (2.1370, 2.5200, 1.1533) -- (2.1860, 2.5200, 1.1492) -- (2.1860, 2.5710, 1.1437) -- (2.1370, 2.5710, 1.1477) -- cycle;
\fill[blue!22.4, opacity=0.7] (2.1370, 2.5710, 1.1477) -- (2.1860, 2.5710, 1.1437) -- (2.1860, 2.6220, 1.1380) -- (2.1370, 2.6220, 1.1421) -- cycle;
\fill[blue!15.1, opacity=0.7] (2.1370, 2.6220, 1.1421) -- (2.1860, 2.6220, 1.1380) -- (2.1860, 2.6730, 1.1322) -- (2.1370, 2.6730, 1.1363) -- cycle;
\fill[blue!15.0, opacity=0.7] (2.1370, 2.6730, 1.1363) -- (2.1860, 2.6730, 1.1322) -- (2.1860, 2.7240, 1.1263) -- (2.1370, 2.7240, 1.1303) -- cycle;
\fill[blue!15.0, opacity=0.7] (2.1370, 2.7240, 1.1303) -- (2.1860, 2.7240, 1.1263) -- (2.1860, 2.7750, 1.1202) -- (2.1370, 2.7750, 1.1243) -- cycle;
\fill[blue!15.0, opacity=0.7] (2.1370, 2.7750, 1.1243) -- (2.1860, 2.7750, 1.1202) -- (2.1860, 2.8260, 1.1141) -- (2.1370, 2.8260, 1.1182) -- cycle;
\fill[blue!15.0, opacity=0.7] (2.1370, 2.8260, 1.1182) -- (2.1860, 2.8260, 1.1141) -- (2.1860, 2.8770, 1.1079) -- (2.1370, 2.8770, 1.1120) -- cycle;
\fill[blue!15.0, opacity=0.7] (2.1370, 2.8770, 1.1120) -- (2.1860, 2.8770, 1.1079) -- (2.1860, 2.9280, 1.1017) -- (2.1370, 2.9280, 1.1058) -- cycle;
\fill[blue!15.1, opacity=0.7] (2.1370, 2.9280, 1.1058) -- (2.1860, 2.9280, 1.1017) -- (2.1860, 2.9790, 1.0955) -- (2.1370, 2.9790, 1.0995) -- cycle;
\fill[blue!15.3, opacity=0.7] (2.1370, 2.9790, 1.0995) -- (2.1860, 2.9790, 1.0955) -- (2.1860, 3.0300, 1.0892) -- (2.1370, 3.0300, 1.0933) -- cycle;
\fill[blue!15.0, opacity=0.7] (2.1860, -0.0300, 1.0892) -- (2.2350, -0.0300, 1.0849) -- (2.2350, 0.0210, 1.0911) -- (2.1860, 0.0210, 1.0955) -- cycle;
\fill[blue!15.0, opacity=0.7] (2.1860, 0.0210, 1.0955) -- (2.2350, 0.0210, 1.0911) -- (2.2350, 0.0720, 1.0974) -- (2.1860, 0.0720, 1.1017) -- cycle;
\fill[blue!15.2, opacity=0.7] (2.1860, 0.0720, 1.1017) -- (2.2350, 0.0720, 1.0974) -- (2.2350, 0.1230, 1.1036) -- (2.1860, 0.1230, 1.1079) -- cycle;
\fill[blue!16.5, opacity=0.7] (2.1860, 0.1230, 1.1079) -- (2.2350, 0.1230, 1.1036) -- (2.2350, 0.1740, 1.1098) -- (2.1860, 0.1740, 1.1141) -- cycle;
\fill[blue!16.2, opacity=0.7] (2.1860, 0.1740, 1.1141) -- (2.2350, 0.1740, 1.1098) -- (2.2350, 0.2250, 1.1159) -- (2.1860, 0.2250, 1.1202) -- cycle;
\fill[blue!15.1, opacity=0.7] (2.1860, 0.2250, 1.1202) -- (2.2350, 0.2250, 1.1159) -- (2.2350, 0.2760, 1.1219) -- (2.1860, 0.2760, 1.1263) -- cycle;
\fill[blue!15.0, opacity=0.7] (2.1860, 0.2760, 1.1263) -- (2.2350, 0.2760, 1.1219) -- (2.2350, 0.3270, 1.1279) -- (2.1860, 0.3270, 1.1322) -- cycle;
\fill[blue!15.0, opacity=0.7] (2.1860, 0.3270, 1.1322) -- (2.2350, 0.3270, 1.1279) -- (2.2350, 0.3780, 1.1337) -- (2.1860, 0.3780, 1.1380) -- cycle;
\fill[blue!15.0, opacity=0.7] (2.1860, 0.3780, 1.1380) -- (2.2350, 0.3780, 1.1337) -- (2.2350, 0.4290, 1.1393) -- (2.1860, 0.4290, 1.1437) -- cycle;
\fill[blue!15.0, opacity=0.7] (2.1860, 0.4290, 1.1437) -- (2.2350, 0.4290, 1.1393) -- (2.2350, 0.4800, 1.1449) -- (2.1860, 0.4800, 1.1492) -- cycle;
\fill[blue!15.0, opacity=0.7] (2.1860, 0.4800, 1.1492) -- (2.2350, 0.4800, 1.1449) -- (2.2350, 0.5310, 1.1502) -- (2.1860, 0.5310, 1.1545) -- cycle;
\fill[blue!15.1, opacity=0.7] (2.1860, 0.5310, 1.1545) -- (2.2350, 0.5310, 1.1502) -- (2.2350, 0.5820, 1.1554) -- (2.1860, 0.5820, 1.1597) -- cycle;
\fill[blue!18.9, opacity=0.7] (2.1860, 0.5820, 1.1597) -- (2.2350, 0.5820, 1.1554) -- (2.2350, 0.6330, 1.1604) -- (2.1860, 0.6330, 1.1647) -- cycle;
\fill[blue!44.6, opacity=0.7] (2.1860, 0.6330, 1.1647) -- (2.2350, 0.6330, 1.1604) -- (2.2350, 0.6840, 1.1651) -- (2.1860, 0.6840, 1.1695) -- cycle;
\fill[blue!74.6, opacity=0.7] (2.1860, 0.6840, 1.1695) -- (2.2350, 0.6840, 1.1651) -- (2.2350, 0.7350, 1.1697) -- (2.1860, 0.7350, 1.1740) -- cycle;
\fill[blue!78.2, opacity=0.7] (2.1860, 0.7350, 1.1740) -- (2.2350, 0.7350, 1.1697) -- (2.2350, 0.7860, 1.1740) -- (2.1860, 0.7860, 1.1784) -- cycle;
\fill[blue!58.0, opacity=0.7] (2.1860, 0.7860, 1.1784) -- (2.2350, 0.7860, 1.1740) -- (2.2350, 0.8370, 1.1781) -- (2.1860, 0.8370, 1.1824) -- cycle;
\fill[blue!31.5, opacity=0.7] (2.1860, 0.8370, 1.1824) -- (2.2350, 0.8370, 1.1781) -- (2.2350, 0.8880, 1.1819) -- (2.1860, 0.8880, 1.1863) -- cycle;
\fill[blue!18.6, opacity=0.7] (2.1860, 0.8880, 1.1863) -- (2.2350, 0.8880, 1.1819) -- (2.2350, 0.9390, 1.1855) -- (2.1860, 0.9390, 1.1898) -- cycle;
\fill[blue!15.6, opacity=0.7] (2.1860, 0.9390, 1.1898) -- (2.2350, 0.9390, 1.1855) -- (2.2350, 0.9900, 1.1888) -- (2.1860, 0.9900, 1.1931) -- cycle;
\fill[blue!15.1, opacity=0.7] (2.1860, 0.9900, 1.1931) -- (2.2350, 0.9900, 1.1888) -- (2.2350, 1.0410, 1.1918) -- (2.1860, 1.0410, 1.1961) -- cycle;
\fill[blue!15.0, opacity=0.7] (2.1860, 1.0410, 1.1961) -- (2.2350, 1.0410, 1.1918) -- (2.2350, 1.0920, 1.1945) -- (2.1860, 1.0920, 1.1988) -- cycle;
\fill[blue!15.0, opacity=0.7] (2.1860, 1.0920, 1.1988) -- (2.2350, 1.0920, 1.1945) -- (2.2350, 1.1430, 1.1969) -- (2.1860, 1.1430, 1.2012) -- cycle;
\fill[blue!15.0, opacity=0.7] (2.1860, 1.1430, 1.2012) -- (2.2350, 1.1430, 1.1969) -- (2.2350, 1.1940, 1.1990) -- (2.1860, 1.1940, 1.2033) -- cycle;
\fill[blue!15.0, opacity=0.7] (2.1860, 1.1940, 1.2033) -- (2.2350, 1.1940, 1.1990) -- (2.2350, 1.2450, 1.2008) -- (2.1860, 1.2450, 1.2051) -- cycle;
\fill[blue!15.1, opacity=0.7] (2.1860, 1.2450, 1.2051) -- (2.2350, 1.2450, 1.2008) -- (2.2350, 1.2960, 1.2022) -- (2.1860, 1.2960, 1.2066) -- cycle;
\fill[blue!15.1, opacity=0.7] (2.1860, 1.2960, 1.2066) -- (2.2350, 1.2960, 1.2022) -- (2.2350, 1.3470, 1.2034) -- (2.1860, 1.3470, 1.2077) -- cycle;
\fill[blue!15.1, opacity=0.7] (2.1860, 1.3470, 1.2077) -- (2.2350, 1.3470, 1.2034) -- (2.2350, 1.3980, 1.2042) -- (2.1860, 1.3980, 1.2085) -- cycle;
\fill[blue!15.1, opacity=0.7] (2.1860, 1.3980, 1.2085) -- (2.2350, 1.3980, 1.2042) -- (2.2350, 1.4490, 1.2047) -- (2.1860, 1.4490, 1.2090) -- cycle;
\fill[blue!15.1, opacity=0.7] (2.1860, 1.4490, 1.2090) -- (2.2350, 1.4490, 1.2047) -- (2.2350, 1.5000, 1.2049) -- (2.1860, 1.5000, 1.2092) -- cycle;
\fill[blue!15.1, opacity=0.7] (2.1860, 1.5000, 1.2092) -- (2.2350, 1.5000, 1.2049) -- (2.2350, 1.5510, 1.2047) -- (2.1860, 1.5510, 1.2090) -- cycle;
\fill[blue!15.1, opacity=0.7] (2.1860, 1.5510, 1.2090) -- (2.2350, 1.5510, 1.2047) -- (2.2350, 1.6020, 1.2042) -- (2.1860, 1.6020, 1.2085) -- cycle;
\fill[blue!15.1, opacity=0.7] (2.1860, 1.6020, 1.2085) -- (2.2350, 1.6020, 1.2042) -- (2.2350, 1.6530, 1.2034) -- (2.1860, 1.6530, 1.2077) -- cycle;
\fill[blue!15.3, opacity=0.7] (2.1860, 1.6530, 1.2077) -- (2.2350, 1.6530, 1.2034) -- (2.2350, 1.7040, 1.2022) -- (2.1860, 1.7040, 1.2066) -- cycle;
\fill[blue!16.2, opacity=0.7] (2.1860, 1.7040, 1.2066) -- (2.2350, 1.7040, 1.2022) -- (2.2350, 1.7550, 1.2008) -- (2.1860, 1.7550, 1.2051) -- cycle;
\fill[blue!21.6, opacity=0.7] (2.1860, 1.7550, 1.2051) -- (2.2350, 1.7550, 1.2008) -- (2.2350, 1.8060, 1.1990) -- (2.1860, 1.8060, 1.2033) -- cycle;
\fill[blue!44.3, opacity=0.7] (2.1860, 1.8060, 1.2033) -- (2.2350, 1.8060, 1.1990) -- (2.2350, 1.8570, 1.1969) -- (2.1860, 1.8570, 1.2012) -- cycle;
\fill[blue!85.8, opacity=0.7] (2.1860, 1.8570, 1.2012) -- (2.2350, 1.8570, 1.1969) -- (2.2350, 1.9080, 1.1945) -- (2.1860, 1.9080, 1.1988) -- cycle;
\fill[blue!76.8!black, opacity=0.7] (2.1860, 1.9080, 1.1988) -- (2.2350, 1.9080, 1.1945) -- (2.2350, 1.9590, 1.1918) -- (2.1860, 1.9590, 1.1961) -- cycle;
\fill[blue!70.9!black, opacity=0.7] (2.1860, 1.9590, 1.1961) -- (2.2350, 1.9590, 1.1918) -- (2.2350, 2.0100, 1.1888) -- (2.1860, 2.0100, 1.1931) -- cycle;
\fill[blue!97.7, opacity=0.7] (2.1860, 2.0100, 1.1931) -- (2.2350, 2.0100, 1.1888) -- (2.2350, 2.0610, 1.1855) -- (2.1860, 2.0610, 1.1898) -- cycle;
\fill[blue!45.6, opacity=0.7] (2.1860, 2.0610, 1.1898) -- (2.2350, 2.0610, 1.1855) -- (2.2350, 2.1120, 1.1819) -- (2.1860, 2.1120, 1.1863) -- cycle;
\fill[blue!16.9, opacity=0.7] (2.1860, 2.1120, 1.1863) -- (2.2350, 2.1120, 1.1819) -- (2.2350, 2.1630, 1.1781) -- (2.1860, 2.1630, 1.1824) -- cycle;
\fill[blue!15.0, opacity=0.7] (2.1860, 2.1630, 1.1824) -- (2.2350, 2.1630, 1.1781) -- (2.2350, 2.2140, 1.1740) -- (2.1860, 2.2140, 1.1784) -- cycle;
\fill[blue!15.0, opacity=0.7] (2.1860, 2.2140, 1.1784) -- (2.2350, 2.2140, 1.1740) -- (2.2350, 2.2650, 1.1697) -- (2.1860, 2.2650, 1.1740) -- cycle;
\fill[blue!15.0, opacity=0.7] (2.1860, 2.2650, 1.1740) -- (2.2350, 2.2650, 1.1697) -- (2.2350, 2.3160, 1.1651) -- (2.1860, 2.3160, 1.1695) -- cycle;
\fill[blue!15.0, opacity=0.7] (2.1860, 2.3160, 1.1695) -- (2.2350, 2.3160, 1.1651) -- (2.2350, 2.3670, 1.1604) -- (2.1860, 2.3670, 1.1647) -- cycle;
\fill[blue!15.2, opacity=0.7] (2.1860, 2.3670, 1.1647) -- (2.2350, 2.3670, 1.1604) -- (2.2350, 2.4180, 1.1554) -- (2.1860, 2.4180, 1.1597) -- cycle;
\fill[blue!20.6, opacity=0.7] (2.1860, 2.4180, 1.1597) -- (2.2350, 2.4180, 1.1554) -- (2.2350, 2.4690, 1.1502) -- (2.1860, 2.4690, 1.1545) -- cycle;
\fill[blue!36.5, opacity=0.7] (2.1860, 2.4690, 1.1545) -- (2.2350, 2.4690, 1.1502) -- (2.2350, 2.5200, 1.1449) -- (2.1860, 2.5200, 1.1492) -- cycle;
\fill[blue!31.0, opacity=0.7] (2.1860, 2.5200, 1.1492) -- (2.2350, 2.5200, 1.1449) -- (2.2350, 2.5710, 1.1393) -- (2.1860, 2.5710, 1.1437) -- cycle;
\fill[blue!16.4, opacity=0.7] (2.1860, 2.5710, 1.1437) -- (2.2350, 2.5710, 1.1393) -- (2.2350, 2.6220, 1.1337) -- (2.1860, 2.6220, 1.1380) -- cycle;
\fill[blue!15.0, opacity=0.7] (2.1860, 2.6220, 1.1380) -- (2.2350, 2.6220, 1.1337) -- (2.2350, 2.6730, 1.1279) -- (2.1860, 2.6730, 1.1322) -- cycle;
\fill[blue!15.0, opacity=0.7] (2.1860, 2.6730, 1.1322) -- (2.2350, 2.6730, 1.1279) -- (2.2350, 2.7240, 1.1219) -- (2.1860, 2.7240, 1.1263) -- cycle;
\fill[blue!15.0, opacity=0.7] (2.1860, 2.7240, 1.1263) -- (2.2350, 2.7240, 1.1219) -- (2.2350, 2.7750, 1.1159) -- (2.1860, 2.7750, 1.1202) -- cycle;
\fill[blue!15.0, opacity=0.7] (2.1860, 2.7750, 1.1202) -- (2.2350, 2.7750, 1.1159) -- (2.2350, 2.8260, 1.1098) -- (2.1860, 2.8260, 1.1141) -- cycle;
\fill[blue!15.0, opacity=0.7] (2.1860, 2.8260, 1.1141) -- (2.2350, 2.8260, 1.1098) -- (2.2350, 2.8770, 1.1036) -- (2.1860, 2.8770, 1.1079) -- cycle;
\fill[blue!15.0, opacity=0.7] (2.1860, 2.8770, 1.1079) -- (2.2350, 2.8770, 1.1036) -- (2.2350, 2.9280, 1.0974) -- (2.1860, 2.9280, 1.1017) -- cycle;
\fill[blue!15.2, opacity=0.7] (2.1860, 2.9280, 1.1017) -- (2.2350, 2.9280, 1.0974) -- (2.2350, 2.9790, 1.0911) -- (2.1860, 2.9790, 1.0955) -- cycle;
\fill[blue!15.2, opacity=0.7] (2.1860, 2.9790, 1.0955) -- (2.2350, 2.9790, 1.0911) -- (2.2350, 3.0300, 1.0849) -- (2.1860, 3.0300, 1.0892) -- cycle;
\fill[blue!15.0, opacity=0.7] (2.2350, -0.0300, 1.0849) -- (2.2840, -0.0300, 1.0803) -- (2.2840, 0.0210, 1.0866) -- (2.2350, 0.0210, 1.0911) -- cycle;
\fill[blue!15.0, opacity=0.7] (2.2350, 0.0210, 1.0911) -- (2.2840, 0.0210, 1.0866) -- (2.2840, 0.0720, 1.0928) -- (2.2350, 0.0720, 1.0974) -- cycle;
\fill[blue!15.0, opacity=0.7] (2.2350, 0.0720, 1.0974) -- (2.2840, 0.0720, 1.0928) -- (2.2840, 0.1230, 1.0991) -- (2.2350, 0.1230, 1.1036) -- cycle;
\fill[blue!15.5, opacity=0.7] (2.2350, 0.1230, 1.1036) -- (2.2840, 0.1230, 1.0991) -- (2.2840, 0.1740, 1.1052) -- (2.2350, 0.1740, 1.1098) -- cycle;
\fill[blue!16.9, opacity=0.7] (2.2350, 0.1740, 1.1098) -- (2.2840, 0.1740, 1.1052) -- (2.2840, 0.2250, 1.1114) -- (2.2350, 0.2250, 1.1159) -- cycle;
\fill[blue!15.8, opacity=0.7] (2.2350, 0.2250, 1.1159) -- (2.2840, 0.2250, 1.1114) -- (2.2840, 0.2760, 1.1174) -- (2.2350, 0.2760, 1.1219) -- cycle;
\fill[blue!15.0, opacity=0.7] (2.2350, 0.2760, 1.1219) -- (2.2840, 0.2760, 1.1174) -- (2.2840, 0.3270, 1.1233) -- (2.2350, 0.3270, 1.1279) -- cycle;
\fill[blue!15.0, opacity=0.7] (2.2350, 0.3270, 1.1279) -- (2.2840, 0.3270, 1.1233) -- (2.2840, 0.3780, 1.1291) -- (2.2350, 0.3780, 1.1337) -- cycle;
\fill[blue!15.0, opacity=0.7] (2.2350, 0.3780, 1.1337) -- (2.2840, 0.3780, 1.1291) -- (2.2840, 0.4290, 1.1348) -- (2.2350, 0.4290, 1.1393) -- cycle;
\fill[blue!15.0, opacity=0.7] (2.2350, 0.4290, 1.1393) -- (2.2840, 0.4290, 1.1348) -- (2.2840, 0.4800, 1.1403) -- (2.2350, 0.4800, 1.1449) -- cycle;
\fill[blue!15.0, opacity=0.7] (2.2350, 0.4800, 1.1449) -- (2.2840, 0.4800, 1.1403) -- (2.2840, 0.5310, 1.1457) -- (2.2350, 0.5310, 1.1502) -- cycle;
\fill[blue!15.0, opacity=0.7] (2.2350, 0.5310, 1.1502) -- (2.2840, 0.5310, 1.1457) -- (2.2840, 0.5820, 1.1508) -- (2.2350, 0.5820, 1.1554) -- cycle;
\fill[blue!15.1, opacity=0.7] (2.2350, 0.5820, 1.1554) -- (2.2840, 0.5820, 1.1508) -- (2.2840, 0.6330, 1.1558) -- (2.2350, 0.6330, 1.1604) -- cycle;
\fill[blue!18.0, opacity=0.7] (2.2350, 0.6330, 1.1604) -- (2.2840, 0.6330, 1.1558) -- (2.2840, 0.6840, 1.1606) -- (2.2350, 0.6840, 1.1651) -- cycle;
\fill[blue!39.6, opacity=0.7] (2.2350, 0.6840, 1.1651) -- (2.2840, 0.6840, 1.1606) -- (2.2840, 0.7350, 1.1651) -- (2.2350, 0.7350, 1.1697) -- cycle;
\fill[blue!71.1, opacity=0.7] (2.2350, 0.7350, 1.1697) -- (2.2840, 0.7350, 1.1651) -- (2.2840, 0.7860, 1.1695) -- (2.2350, 0.7860, 1.1740) -- cycle;
\fill[blue!82.7, opacity=0.7] (2.2350, 0.7860, 1.1740) -- (2.2840, 0.7860, 1.1695) -- (2.2840, 0.8370, 1.1736) -- (2.2350, 0.8370, 1.1781) -- cycle;
\fill[blue!73.2, opacity=0.7] (2.2350, 0.8370, 1.1781) -- (2.2840, 0.8370, 1.1736) -- (2.2840, 0.8880, 1.1774) -- (2.2350, 0.8880, 1.1819) -- cycle;
\fill[blue!50.1, opacity=0.7] (2.2350, 0.8880, 1.1819) -- (2.2840, 0.8880, 1.1774) -- (2.2840, 0.9390, 1.1809) -- (2.2350, 0.9390, 1.1855) -- cycle;
\fill[blue!29.2, opacity=0.7] (2.2350, 0.9390, 1.1855) -- (2.2840, 0.9390, 1.1809) -- (2.2840, 0.9900, 1.1842) -- (2.2350, 0.9900, 1.1888) -- cycle;
\fill[blue!19.2, opacity=0.7] (2.2350, 0.9900, 1.1888) -- (2.2840, 0.9900, 1.1842) -- (2.2840, 1.0410, 1.1872) -- (2.2350, 1.0410, 1.1918) -- cycle;
\fill[blue!16.1, opacity=0.7] (2.2350, 1.0410, 1.1918) -- (2.2840, 1.0410, 1.1872) -- (2.2840, 1.0920, 1.1899) -- (2.2350, 1.0920, 1.1945) -- cycle;
\fill[blue!15.4, opacity=0.7] (2.2350, 1.0920, 1.1945) -- (2.2840, 1.0920, 1.1899) -- (2.2840, 1.1430, 1.1923) -- (2.2350, 1.1430, 1.1969) -- cycle;
\fill[blue!15.1, opacity=0.7] (2.2350, 1.1430, 1.1969) -- (2.2840, 1.1430, 1.1923) -- (2.2840, 1.1940, 1.1944) -- (2.2350, 1.1940, 1.1990) -- cycle;
\fill[blue!15.1, opacity=0.7] (2.2350, 1.1940, 1.1990) -- (2.2840, 1.1940, 1.1944) -- (2.2840, 1.2450, 1.1962) -- (2.2350, 1.2450, 1.2008) -- cycle;
\fill[blue!15.1, opacity=0.7] (2.2350, 1.2450, 1.2008) -- (2.2840, 1.2450, 1.1962) -- (2.2840, 1.2960, 1.1977) -- (2.2350, 1.2960, 1.2022) -- cycle;
\fill[blue!15.1, opacity=0.7] (2.2350, 1.2960, 1.2022) -- (2.2840, 1.2960, 1.1977) -- (2.2840, 1.3470, 1.1988) -- (2.2350, 1.3470, 1.2034) -- cycle;
\fill[blue!15.1, opacity=0.7] (2.2350, 1.3470, 1.2034) -- (2.2840, 1.3470, 1.1988) -- (2.2840, 1.3980, 1.1996) -- (2.2350, 1.3980, 1.2042) -- cycle;
\fill[blue!15.1, opacity=0.7] (2.2350, 1.3980, 1.2042) -- (2.2840, 1.3980, 1.1996) -- (2.2840, 1.4490, 1.2001) -- (2.2350, 1.4490, 1.2047) -- cycle;
\fill[blue!15.1, opacity=0.7] (2.2350, 1.4490, 1.2047) -- (2.2840, 1.4490, 1.2001) -- (2.2840, 1.5000, 1.2003) -- (2.2350, 1.5000, 1.2049) -- cycle;
\fill[blue!15.2, opacity=0.7] (2.2350, 1.5000, 1.2049) -- (2.2840, 1.5000, 1.2003) -- (2.2840, 1.5510, 1.2001) -- (2.2350, 1.5510, 1.2047) -- cycle;
\fill[blue!15.5, opacity=0.7] (2.2350, 1.5510, 1.2047) -- (2.2840, 1.5510, 1.2001) -- (2.2840, 1.6020, 1.1996) -- (2.2350, 1.6020, 1.2042) -- cycle;
\fill[blue!16.5, opacity=0.7] (2.2350, 1.6020, 1.2042) -- (2.2840, 1.6020, 1.1996) -- (2.2840, 1.6530, 1.1988) -- (2.2350, 1.6530, 1.2034) -- cycle;
\fill[blue!20.6, opacity=0.7] (2.2350, 1.6530, 1.2034) -- (2.2840, 1.6530, 1.1988) -- (2.2840, 1.7040, 1.1977) -- (2.2350, 1.7040, 1.2022) -- cycle;
\fill[blue!34.6, opacity=0.7] (2.2350, 1.7040, 1.2022) -- (2.2840, 1.7040, 1.1977) -- (2.2840, 1.7550, 1.1962) -- (2.2350, 1.7550, 1.2008) -- cycle;
\fill[blue!65.1, opacity=0.7] (2.2350, 1.7550, 1.2008) -- (2.2840, 1.7550, 1.1962) -- (2.2840, 1.8060, 1.1944) -- (2.2350, 1.8060, 1.1990) -- cycle;
\fill[blue!97.1, opacity=0.7] (2.2350, 1.8060, 1.1990) -- (2.2840, 1.8060, 1.1944) -- (2.2840, 1.8570, 1.1923) -- (2.2350, 1.8570, 1.1969) -- cycle;
\fill[blue!74.7!black, opacity=0.7] (2.2350, 1.8570, 1.1969) -- (2.2840, 1.8570, 1.1923) -- (2.2840, 1.9080, 1.1899) -- (2.2350, 1.9080, 1.1945) -- cycle;
\fill[blue!75.0!black, opacity=0.7] (2.2350, 1.9080, 1.1945) -- (2.2840, 1.9080, 1.1899) -- (2.2840, 1.9590, 1.1872) -- (2.2350, 1.9590, 1.1918) -- cycle;
\fill[blue!91.7, opacity=0.7] (2.2350, 1.9590, 1.1918) -- (2.2840, 1.9590, 1.1872) -- (2.2840, 2.0100, 1.1842) -- (2.2350, 2.0100, 1.1888) -- cycle;
\fill[blue!42.6, opacity=0.7] (2.2350, 2.0100, 1.1888) -- (2.2840, 2.0100, 1.1842) -- (2.2840, 2.0610, 1.1809) -- (2.2350, 2.0610, 1.1855) -- cycle;
\fill[blue!17.1, opacity=0.7] (2.2350, 2.0610, 1.1855) -- (2.2840, 2.0610, 1.1809) -- (2.2840, 2.1120, 1.1774) -- (2.2350, 2.1120, 1.1819) -- cycle;
\fill[blue!15.0, opacity=0.7] (2.2350, 2.1120, 1.1819) -- (2.2840, 2.1120, 1.1774) -- (2.2840, 2.1630, 1.1736) -- (2.2350, 2.1630, 1.1781) -- cycle;
\fill[blue!15.0, opacity=0.7] (2.2350, 2.1630, 1.1781) -- (2.2840, 2.1630, 1.1736) -- (2.2840, 2.2140, 1.1695) -- (2.2350, 2.2140, 1.1740) -- cycle;
\fill[blue!15.0, opacity=0.7] (2.2350, 2.2140, 1.1740) -- (2.2840, 2.2140, 1.1695) -- (2.2840, 2.2650, 1.1651) -- (2.2350, 2.2650, 1.1697) -- cycle;
\fill[blue!15.0, opacity=0.7] (2.2350, 2.2650, 1.1697) -- (2.2840, 2.2650, 1.1651) -- (2.2840, 2.3160, 1.1606) -- (2.2350, 2.3160, 1.1651) -- cycle;
\fill[blue!15.1, opacity=0.7] (2.2350, 2.3160, 1.1651) -- (2.2840, 2.3160, 1.1606) -- (2.2840, 2.3670, 1.1558) -- (2.2350, 2.3670, 1.1604) -- cycle;
\fill[blue!17.1, opacity=0.7] (2.2350, 2.3670, 1.1604) -- (2.2840, 2.3670, 1.1558) -- (2.2840, 2.4180, 1.1508) -- (2.2350, 2.4180, 1.1554) -- cycle;
\fill[blue!30.3, opacity=0.7] (2.2350, 2.4180, 1.1554) -- (2.2840, 2.4180, 1.1508) -- (2.2840, 2.4690, 1.1457) -- (2.2350, 2.4690, 1.1502) -- cycle;
\fill[blue!35.9, opacity=0.7] (2.2350, 2.4690, 1.1502) -- (2.2840, 2.4690, 1.1457) -- (2.2840, 2.5200, 1.1403) -- (2.2350, 2.5200, 1.1449) -- cycle;
\fill[blue!20.2, opacity=0.7] (2.2350, 2.5200, 1.1449) -- (2.2840, 2.5200, 1.1403) -- (2.2840, 2.5710, 1.1348) -- (2.2350, 2.5710, 1.1393) -- cycle;
\fill[blue!15.1, opacity=0.7] (2.2350, 2.5710, 1.1393) -- (2.2840, 2.5710, 1.1348) -- (2.2840, 2.6220, 1.1291) -- (2.2350, 2.6220, 1.1337) -- cycle;
\fill[blue!15.0, opacity=0.7] (2.2350, 2.6220, 1.1337) -- (2.2840, 2.6220, 1.1291) -- (2.2840, 2.6730, 1.1233) -- (2.2350, 2.6730, 1.1279) -- cycle;
\fill[blue!15.0, opacity=0.7] (2.2350, 2.6730, 1.1279) -- (2.2840, 2.6730, 1.1233) -- (2.2840, 2.7240, 1.1174) -- (2.2350, 2.7240, 1.1219) -- cycle;
\fill[blue!15.0, opacity=0.7] (2.2350, 2.7240, 1.1219) -- (2.2840, 2.7240, 1.1174) -- (2.2840, 2.7750, 1.1114) -- (2.2350, 2.7750, 1.1159) -- cycle;
\fill[blue!15.0, opacity=0.7] (2.2350, 2.7750, 1.1159) -- (2.2840, 2.7750, 1.1114) -- (2.2840, 2.8260, 1.1052) -- (2.2350, 2.8260, 1.1098) -- cycle;
\fill[blue!15.0, opacity=0.7] (2.2350, 2.8260, 1.1098) -- (2.2840, 2.8260, 1.1052) -- (2.2840, 2.8770, 1.0991) -- (2.2350, 2.8770, 1.1036) -- cycle;
\fill[blue!15.0, opacity=0.7] (2.2350, 2.8770, 1.1036) -- (2.2840, 2.8770, 1.0991) -- (2.2840, 2.9280, 1.0928) -- (2.2350, 2.9280, 1.0974) -- cycle;
\fill[blue!15.3, opacity=0.7] (2.2350, 2.9280, 1.0974) -- (2.2840, 2.9280, 1.0928) -- (2.2840, 2.9790, 1.0866) -- (2.2350, 2.9790, 1.0911) -- cycle;
\fill[blue!15.1, opacity=0.7] (2.2350, 2.9790, 1.0911) -- (2.2840, 2.9790, 1.0866) -- (2.2840, 3.0300, 1.0803) -- (2.2350, 3.0300, 1.0849) -- cycle;
\fill[blue!15.0, opacity=0.7] (2.2840, -0.0300, 1.0803) -- (2.3330, -0.0300, 1.0755) -- (2.3330, 0.0210, 1.0818) -- (2.2840, 0.0210, 1.0866) -- cycle;
\fill[blue!15.0, opacity=0.7] (2.2840, 0.0210, 1.0866) -- (2.3330, 0.0210, 1.0818) -- (2.3330, 0.0720, 1.0881) -- (2.2840, 0.0720, 1.0928) -- cycle;
\fill[blue!15.0, opacity=0.7] (2.2840, 0.0720, 1.0928) -- (2.3330, 0.0720, 1.0881) -- (2.3330, 0.1230, 1.0943) -- (2.2840, 0.1230, 1.0991) -- cycle;
\fill[blue!15.0, opacity=0.7] (2.2840, 0.1230, 1.0991) -- (2.3330, 0.1230, 1.0943) -- (2.3330, 0.1740, 1.1005) -- (2.2840, 0.1740, 1.1052) -- cycle;
\fill[blue!15.9, opacity=0.7] (2.2840, 0.1740, 1.1052) -- (2.3330, 0.1740, 1.1005) -- (2.3330, 0.2250, 1.1066) -- (2.2840, 0.2250, 1.1114) -- cycle;
\fill[blue!17.0, opacity=0.7] (2.2840, 0.2250, 1.1114) -- (2.3330, 0.2250, 1.1066) -- (2.3330, 0.2760, 1.1126) -- (2.2840, 0.2760, 1.1174) -- cycle;
\fill[blue!15.6, opacity=0.7] (2.2840, 0.2760, 1.1174) -- (2.3330, 0.2760, 1.1126) -- (2.3330, 0.3270, 1.1185) -- (2.2840, 0.3270, 1.1233) -- cycle;
\fill[blue!15.0, opacity=0.7] (2.2840, 0.3270, 1.1233) -- (2.3330, 0.3270, 1.1185) -- (2.3330, 0.3780, 1.1243) -- (2.2840, 0.3780, 1.1291) -- cycle;
\fill[blue!15.0, opacity=0.7] (2.2840, 0.3780, 1.1291) -- (2.3330, 0.3780, 1.1243) -- (2.3330, 0.4290, 1.1300) -- (2.2840, 0.4290, 1.1348) -- cycle;
\fill[blue!15.0, opacity=0.7] (2.2840, 0.4290, 1.1348) -- (2.3330, 0.4290, 1.1300) -- (2.3330, 0.4800, 1.1355) -- (2.2840, 0.4800, 1.1403) -- cycle;
\fill[blue!15.0, opacity=0.7] (2.2840, 0.4800, 1.1403) -- (2.3330, 0.4800, 1.1355) -- (2.3330, 0.5310, 1.1409) -- (2.2840, 0.5310, 1.1457) -- cycle;
\fill[blue!15.0, opacity=0.7] (2.2840, 0.5310, 1.1457) -- (2.3330, 0.5310, 1.1409) -- (2.3330, 0.5820, 1.1461) -- (2.2840, 0.5820, 1.1508) -- cycle;
\fill[blue!15.0, opacity=0.7] (2.2840, 0.5820, 1.1508) -- (2.3330, 0.5820, 1.1461) -- (2.3330, 0.6330, 1.1510) -- (2.2840, 0.6330, 1.1558) -- cycle;
\fill[blue!15.0, opacity=0.7] (2.2840, 0.6330, 1.1558) -- (2.3330, 0.6330, 1.1510) -- (2.3330, 0.6840, 1.1558) -- (2.2840, 0.6840, 1.1606) -- cycle;
\fill[blue!16.5, opacity=0.7] (2.2840, 0.6840, 1.1606) -- (2.3330, 0.6840, 1.1558) -- (2.3330, 0.7350, 1.1604) -- (2.2840, 0.7350, 1.1651) -- cycle;
\fill[blue!30.4, opacity=0.7] (2.2840, 0.7350, 1.1651) -- (2.3330, 0.7350, 1.1604) -- (2.3330, 0.7860, 1.1647) -- (2.2840, 0.7860, 1.1695) -- cycle;
\fill[blue!60.7, opacity=0.7] (2.2840, 0.7860, 1.1695) -- (2.3330, 0.7860, 1.1647) -- (2.3330, 0.8370, 1.1688) -- (2.2840, 0.8370, 1.1736) -- cycle;
\fill[blue!81.9, opacity=0.7] (2.2840, 0.8370, 1.1736) -- (2.3330, 0.8370, 1.1688) -- (2.3330, 0.8880, 1.1726) -- (2.2840, 0.8880, 1.1774) -- cycle;
\fill[blue!85.1, opacity=0.7] (2.2840, 0.8880, 1.1774) -- (2.3330, 0.8880, 1.1726) -- (2.3330, 0.9390, 1.1762) -- (2.2840, 0.9390, 1.1809) -- cycle;
\fill[blue!74.7, opacity=0.7] (2.2840, 0.9390, 1.1809) -- (2.3330, 0.9390, 1.1762) -- (2.3330, 0.9900, 1.1794) -- (2.2840, 0.9900, 1.1842) -- cycle;
\fill[blue!56.2, opacity=0.7] (2.2840, 0.9900, 1.1842) -- (2.3330, 0.9900, 1.1794) -- (2.3330, 1.0410, 1.1824) -- (2.2840, 1.0410, 1.1872) -- cycle;
\fill[blue!38.2, opacity=0.7] (2.2840, 1.0410, 1.1872) -- (2.3330, 1.0410, 1.1824) -- (2.3330, 1.0920, 1.1851) -- (2.2840, 1.0920, 1.1899) -- cycle;
\fill[blue!26.5, opacity=0.7] (2.2840, 1.0920, 1.1899) -- (2.3330, 1.0920, 1.1851) -- (2.3330, 1.1430, 1.1875) -- (2.2840, 1.1430, 1.1923) -- cycle;
\fill[blue!20.7, opacity=0.7] (2.2840, 1.1430, 1.1923) -- (2.3330, 1.1430, 1.1875) -- (2.3330, 1.1940, 1.1896) -- (2.2840, 1.1940, 1.1944) -- cycle;
\fill[blue!18.1, opacity=0.7] (2.2840, 1.1940, 1.1944) -- (2.3330, 1.1940, 1.1896) -- (2.3330, 1.2450, 1.1914) -- (2.2840, 1.2450, 1.1962) -- cycle;
\fill[blue!17.0, opacity=0.7] (2.2840, 1.2450, 1.1962) -- (2.3330, 1.2450, 1.1914) -- (2.3330, 1.2960, 1.1929) -- (2.2840, 1.2960, 1.1977) -- cycle;
\fill[blue!16.6, opacity=0.7] (2.2840, 1.2960, 1.1977) -- (2.3330, 1.2960, 1.1929) -- (2.3330, 1.3470, 1.1940) -- (2.2840, 1.3470, 1.1988) -- cycle;
\fill[blue!16.6, opacity=0.7] (2.2840, 1.3470, 1.1988) -- (2.3330, 1.3470, 1.1940) -- (2.3330, 1.3980, 1.1949) -- (2.2840, 1.3980, 1.1996) -- cycle;
\fill[blue!17.1, opacity=0.7] (2.2840, 1.3980, 1.1996) -- (2.3330, 1.3980, 1.1949) -- (2.3330, 1.4490, 1.1954) -- (2.2840, 1.4490, 1.2001) -- cycle;
\fill[blue!18.3, opacity=0.7] (2.2840, 1.4490, 1.2001) -- (2.3330, 1.4490, 1.1954) -- (2.3330, 1.5000, 1.1955) -- (2.2840, 1.5000, 1.2003) -- cycle;
\fill[blue!21.1, opacity=0.7] (2.2840, 1.5000, 1.2003) -- (2.3330, 1.5000, 1.1955) -- (2.3330, 1.5510, 1.1954) -- (2.2840, 1.5510, 1.2001) -- cycle;
\fill[blue!27.8, opacity=0.7] (2.2840, 1.5510, 1.2001) -- (2.3330, 1.5510, 1.1954) -- (2.3330, 1.6020, 1.1949) -- (2.2840, 1.6020, 1.1996) -- cycle;
\fill[blue!42.0, opacity=0.7] (2.2840, 1.6020, 1.1996) -- (2.3330, 1.6020, 1.1949) -- (2.3330, 1.6530, 1.1940) -- (2.2840, 1.6530, 1.1988) -- cycle;
\fill[blue!65.5, opacity=0.7] (2.2840, 1.6530, 1.1988) -- (2.3330, 1.6530, 1.1940) -- (2.3330, 1.7040, 1.1929) -- (2.2840, 1.7040, 1.1977) -- cycle;
\fill[blue!90.9, opacity=0.7] (2.2840, 1.7040, 1.1977) -- (2.3330, 1.7040, 1.1929) -- (2.3330, 1.7550, 1.1914) -- (2.2840, 1.7550, 1.1962) -- cycle;
\fill[blue!84.6!black, opacity=0.7] (2.2840, 1.7550, 1.1962) -- (2.3330, 1.7550, 1.1914) -- (2.3330, 1.8060, 1.1896) -- (2.2840, 1.8060, 1.1944) -- cycle;
\fill[blue!75.2!black, opacity=0.7] (2.2840, 1.8060, 1.1944) -- (2.3330, 1.8060, 1.1896) -- (2.3330, 1.8570, 1.1875) -- (2.2840, 1.8570, 1.1923) -- cycle;
\fill[blue!88.8!black, opacity=0.7] (2.2840, 1.8570, 1.1923) -- (2.3330, 1.8570, 1.1875) -- (2.3330, 1.9080, 1.1851) -- (2.2840, 1.9080, 1.1899) -- cycle;
\fill[blue!76.9, opacity=0.7] (2.2840, 1.9080, 1.1899) -- (2.3330, 1.9080, 1.1851) -- (2.3330, 1.9590, 1.1824) -- (2.2840, 1.9590, 1.1872) -- cycle;
\fill[blue!33.2, opacity=0.7] (2.2840, 1.9590, 1.1872) -- (2.3330, 1.9590, 1.1824) -- (2.3330, 2.0100, 1.1794) -- (2.2840, 2.0100, 1.1842) -- cycle;
\fill[blue!16.3, opacity=0.7] (2.2840, 2.0100, 1.1842) -- (2.3330, 2.0100, 1.1794) -- (2.3330, 2.0610, 1.1762) -- (2.2840, 2.0610, 1.1809) -- cycle;
\fill[blue!15.0, opacity=0.7] (2.2840, 2.0610, 1.1809) -- (2.3330, 2.0610, 1.1762) -- (2.3330, 2.1120, 1.1726) -- (2.2840, 2.1120, 1.1774) -- cycle;
\fill[blue!15.0, opacity=0.7] (2.2840, 2.1120, 1.1774) -- (2.3330, 2.1120, 1.1726) -- (2.3330, 2.1630, 1.1688) -- (2.2840, 2.1630, 1.1736) -- cycle;
\fill[blue!15.0, opacity=0.7] (2.2840, 2.1630, 1.1736) -- (2.3330, 2.1630, 1.1688) -- (2.3330, 2.2140, 1.1647) -- (2.2840, 2.2140, 1.1695) -- cycle;
\fill[blue!15.0, opacity=0.7] (2.2840, 2.2140, 1.1695) -- (2.3330, 2.2140, 1.1647) -- (2.3330, 2.2650, 1.1604) -- (2.2840, 2.2650, 1.1651) -- cycle;
\fill[blue!15.0, opacity=0.7] (2.2840, 2.2650, 1.1651) -- (2.3330, 2.2650, 1.1604) -- (2.3330, 2.3160, 1.1558) -- (2.2840, 2.3160, 1.1606) -- cycle;
\fill[blue!15.9, opacity=0.7] (2.2840, 2.3160, 1.1606) -- (2.3330, 2.3160, 1.1558) -- (2.3330, 2.3670, 1.1510) -- (2.2840, 2.3670, 1.1558) -- cycle;
\fill[blue!24.9, opacity=0.7] (2.2840, 2.3670, 1.1558) -- (2.3330, 2.3670, 1.1510) -- (2.3330, 2.4180, 1.1461) -- (2.2840, 2.4180, 1.1508) -- cycle;
\fill[blue!36.2, opacity=0.7] (2.2840, 2.4180, 1.1508) -- (2.3330, 2.4180, 1.1461) -- (2.3330, 2.4690, 1.1409) -- (2.2840, 2.4690, 1.1457) -- cycle;
\fill[blue!25.3, opacity=0.7] (2.2840, 2.4690, 1.1457) -- (2.3330, 2.4690, 1.1409) -- (2.3330, 2.5200, 1.1355) -- (2.2840, 2.5200, 1.1403) -- cycle;
\fill[blue!15.6, opacity=0.7] (2.2840, 2.5200, 1.1403) -- (2.3330, 2.5200, 1.1355) -- (2.3330, 2.5710, 1.1300) -- (2.2840, 2.5710, 1.1348) -- cycle;
\fill[blue!15.0, opacity=0.7] (2.2840, 2.5710, 1.1348) -- (2.3330, 2.5710, 1.1300) -- (2.3330, 2.6220, 1.1243) -- (2.2840, 2.6220, 1.1291) -- cycle;
\fill[blue!15.0, opacity=0.7] (2.2840, 2.6220, 1.1291) -- (2.3330, 2.6220, 1.1243) -- (2.3330, 2.6730, 1.1185) -- (2.2840, 2.6730, 1.1233) -- cycle;
\fill[blue!15.0, opacity=0.7] (2.2840, 2.6730, 1.1233) -- (2.3330, 2.6730, 1.1185) -- (2.3330, 2.7240, 1.1126) -- (2.2840, 2.7240, 1.1174) -- cycle;
\fill[blue!15.0, opacity=0.7] (2.2840, 2.7240, 1.1174) -- (2.3330, 2.7240, 1.1126) -- (2.3330, 2.7750, 1.1066) -- (2.2840, 2.7750, 1.1114) -- cycle;
\fill[blue!15.0, opacity=0.7] (2.2840, 2.7750, 1.1114) -- (2.3330, 2.7750, 1.1066) -- (2.3330, 2.8260, 1.1005) -- (2.2840, 2.8260, 1.1052) -- cycle;
\fill[blue!15.0, opacity=0.7] (2.2840, 2.8260, 1.1052) -- (2.3330, 2.8260, 1.1005) -- (2.3330, 2.8770, 1.0943) -- (2.2840, 2.8770, 1.0991) -- cycle;
\fill[blue!15.1, opacity=0.7] (2.2840, 2.8770, 1.0991) -- (2.3330, 2.8770, 1.0943) -- (2.3330, 2.9280, 1.0881) -- (2.2840, 2.9280, 1.0928) -- cycle;
\fill[blue!15.2, opacity=0.7] (2.2840, 2.9280, 1.0928) -- (2.3330, 2.9280, 1.0881) -- (2.3330, 2.9790, 1.0818) -- (2.2840, 2.9790, 1.0866) -- cycle;
\fill[blue!15.0, opacity=0.7] (2.2840, 2.9790, 1.0866) -- (2.3330, 2.9790, 1.0818) -- (2.3330, 3.0300, 1.0755) -- (2.2840, 3.0300, 1.0803) -- cycle;
\fill[blue!15.0, opacity=0.7] (2.3330, -0.0300, 1.0755) -- (2.3820, -0.0300, 1.0705) -- (2.3820, 0.0210, 1.0768) -- (2.3330, 0.0210, 1.0818) -- cycle;
\fill[blue!15.0, opacity=0.7] (2.3330, 0.0210, 1.0818) -- (2.3820, 0.0210, 1.0768) -- (2.3820, 0.0720, 1.0831) -- (2.3330, 0.0720, 1.0881) -- cycle;
\fill[blue!15.0, opacity=0.7] (2.3330, 0.0720, 1.0881) -- (2.3820, 0.0720, 1.0831) -- (2.3820, 0.1230, 1.0893) -- (2.3330, 0.1230, 1.0943) -- cycle;
\fill[blue!15.0, opacity=0.7] (2.3330, 0.1230, 1.0943) -- (2.3820, 0.1230, 1.0893) -- (2.3820, 0.1740, 1.0955) -- (2.3330, 0.1740, 1.1005) -- cycle;
\fill[blue!15.1, opacity=0.7] (2.3330, 0.1740, 1.1005) -- (2.3820, 0.1740, 1.0955) -- (2.3820, 0.2250, 1.1016) -- (2.3330, 0.2250, 1.1066) -- cycle;
\fill[blue!16.4, opacity=0.7] (2.3330, 0.2250, 1.1066) -- (2.3820, 0.2250, 1.1016) -- (2.3820, 0.2760, 1.1076) -- (2.3330, 0.2760, 1.1126) -- cycle;
\fill[blue!17.0, opacity=0.7] (2.3330, 0.2760, 1.1126) -- (2.3820, 0.2760, 1.1076) -- (2.3820, 0.3270, 1.1135) -- (2.3330, 0.3270, 1.1185) -- cycle;
\fill[blue!15.5, opacity=0.7] (2.3330, 0.3270, 1.1185) -- (2.3820, 0.3270, 1.1135) -- (2.3820, 0.3780, 1.1193) -- (2.3330, 0.3780, 1.1243) -- cycle;
\fill[blue!15.0, opacity=0.7] (2.3330, 0.3780, 1.1243) -- (2.3820, 0.3780, 1.1193) -- (2.3820, 0.4290, 1.1250) -- (2.3330, 0.4290, 1.1300) -- cycle;
\fill[blue!15.0, opacity=0.7] (2.3330, 0.4290, 1.1300) -- (2.3820, 0.4290, 1.1250) -- (2.3820, 0.4800, 1.1305) -- (2.3330, 0.4800, 1.1355) -- cycle;
\fill[blue!15.0, opacity=0.7] (2.3330, 0.4800, 1.1355) -- (2.3820, 0.4800, 1.1305) -- (2.3820, 0.5310, 1.1359) -- (2.3330, 0.5310, 1.1409) -- cycle;
\fill[blue!15.0, opacity=0.7] (2.3330, 0.5310, 1.1409) -- (2.3820, 0.5310, 1.1359) -- (2.3820, 0.5820, 1.1411) -- (2.3330, 0.5820, 1.1461) -- cycle;
\fill[blue!15.0, opacity=0.7] (2.3330, 0.5820, 1.1461) -- (2.3820, 0.5820, 1.1411) -- (2.3820, 0.6330, 1.1461) -- (2.3330, 0.6330, 1.1510) -- cycle;
\fill[blue!15.0, opacity=0.7] (2.3330, 0.6330, 1.1510) -- (2.3820, 0.6330, 1.1461) -- (2.3820, 0.6840, 1.1508) -- (2.3330, 0.6840, 1.1558) -- cycle;
\fill[blue!15.0, opacity=0.7] (2.3330, 0.6840, 1.1558) -- (2.3820, 0.6840, 1.1508) -- (2.3820, 0.7350, 1.1554) -- (2.3330, 0.7350, 1.1604) -- cycle;
\fill[blue!15.4, opacity=0.7] (2.3330, 0.7350, 1.1604) -- (2.3820, 0.7350, 1.1554) -- (2.3820, 0.7860, 1.1597) -- (2.3330, 0.7860, 1.1647) -- cycle;
\fill[blue!21.0, opacity=0.7] (2.3330, 0.7860, 1.1647) -- (2.3820, 0.7860, 1.1597) -- (2.3820, 0.8370, 1.1638) -- (2.3330, 0.8370, 1.1688) -- cycle;
\fill[blue!42.1, opacity=0.7] (2.3330, 0.8370, 1.1688) -- (2.3820, 0.8370, 1.1638) -- (2.3820, 0.8880, 1.1676) -- (2.3330, 0.8880, 1.1726) -- cycle;
\fill[blue!69.3, opacity=0.7] (2.3330, 0.8880, 1.1726) -- (2.3820, 0.8880, 1.1676) -- (2.3820, 0.9390, 1.1712) -- (2.3330, 0.9390, 1.1762) -- cycle;
\fill[blue!85.4, opacity=0.7] (2.3330, 0.9390, 1.1762) -- (2.3820, 0.9390, 1.1712) -- (2.3820, 0.9900, 1.1745) -- (2.3330, 0.9900, 1.1794) -- cycle;
\fill[blue!89.4, opacity=0.7] (2.3330, 0.9900, 1.1794) -- (2.3820, 0.9900, 1.1745) -- (2.3820, 1.0410, 1.1775) -- (2.3330, 1.0410, 1.1824) -- cycle;
\fill[blue!85.4, opacity=0.7] (2.3330, 1.0410, 1.1824) -- (2.3820, 1.0410, 1.1775) -- (2.3820, 1.0920, 1.1802) -- (2.3330, 1.0920, 1.1851) -- cycle;
\fill[blue!76.4, opacity=0.7] (2.3330, 1.0920, 1.1851) -- (2.3820, 1.0920, 1.1802) -- (2.3820, 1.1430, 1.1826) -- (2.3330, 1.1430, 1.1875) -- cycle;
\fill[blue!65.6, opacity=0.7] (2.3330, 1.1430, 1.1875) -- (2.3820, 1.1430, 1.1826) -- (2.3820, 1.1940, 1.1847) -- (2.3330, 1.1940, 1.1896) -- cycle;
\fill[blue!56.2, opacity=0.7] (2.3330, 1.1940, 1.1896) -- (2.3820, 1.1940, 1.1847) -- (2.3820, 1.2450, 1.1864) -- (2.3330, 1.2450, 1.1914) -- cycle;
\fill[blue!49.7, opacity=0.7] (2.3330, 1.2450, 1.1914) -- (2.3820, 1.2450, 1.1864) -- (2.3820, 1.2960, 1.1879) -- (2.3330, 1.2960, 1.1929) -- cycle;
\fill[blue!46.6, opacity=0.7] (2.3330, 1.2960, 1.1929) -- (2.3820, 1.2960, 1.1879) -- (2.3820, 1.3470, 1.1891) -- (2.3330, 1.3470, 1.1940) -- cycle;
\fill[blue!46.8, opacity=0.7] (2.3330, 1.3470, 1.1940) -- (2.3820, 1.3470, 1.1891) -- (2.3820, 1.3980, 1.1899) -- (2.3330, 1.3980, 1.1949) -- cycle;
\fill[blue!50.6, opacity=0.7] (2.3330, 1.3980, 1.1949) -- (2.3820, 1.3980, 1.1899) -- (2.3820, 1.4490, 1.1904) -- (2.3330, 1.4490, 1.1954) -- cycle;
\fill[blue!58.4, opacity=0.7] (2.3330, 1.4490, 1.1954) -- (2.3820, 1.4490, 1.1904) -- (2.3820, 1.5000, 1.1905) -- (2.3330, 1.5000, 1.1955) -- cycle;
\fill[blue!70.1, opacity=0.7] (2.3330, 1.5000, 1.1955) -- (2.3820, 1.5000, 1.1905) -- (2.3820, 1.5510, 1.1904) -- (2.3330, 1.5510, 1.1954) -- cycle;
\fill[blue!84.2, opacity=0.7] (2.3330, 1.5510, 1.1954) -- (2.3820, 1.5510, 1.1904) -- (2.3820, 1.6020, 1.1899) -- (2.3330, 1.6020, 1.1949) -- cycle;
\fill[blue!97.1, opacity=0.7] (2.3330, 1.6020, 1.1949) -- (2.3820, 1.6020, 1.1899) -- (2.3820, 1.6530, 1.1891) -- (2.3330, 1.6530, 1.1940) -- cycle;
\fill[blue!86.4!black, opacity=0.7] (2.3330, 1.6530, 1.1940) -- (2.3820, 1.6530, 1.1891) -- (2.3820, 1.7040, 1.1879) -- (2.3330, 1.7040, 1.1929) -- cycle;
\fill[blue!79.8!black, opacity=0.7] (2.3330, 1.7040, 1.1929) -- (2.3820, 1.7040, 1.1879) -- (2.3820, 1.7550, 1.1864) -- (2.3330, 1.7550, 1.1914) -- cycle;
\fill[blue!88.9!black, opacity=0.7] (2.3330, 1.7550, 1.1914) -- (2.3820, 1.7550, 1.1864) -- (2.3820, 1.8060, 1.1847) -- (2.3330, 1.8060, 1.1896) -- cycle;
\fill[blue!87.4, opacity=0.7] (2.3330, 1.8060, 1.1896) -- (2.3820, 1.8060, 1.1847) -- (2.3820, 1.8570, 1.1826) -- (2.3330, 1.8570, 1.1875) -- cycle;
\fill[blue!51.7, opacity=0.7] (2.3330, 1.8570, 1.1875) -- (2.3820, 1.8570, 1.1826) -- (2.3820, 1.9080, 1.1802) -- (2.3330, 1.9080, 1.1851) -- cycle;
\fill[blue!22.5, opacity=0.7] (2.3330, 1.9080, 1.1851) -- (2.3820, 1.9080, 1.1802) -- (2.3820, 1.9590, 1.1775) -- (2.3330, 1.9590, 1.1824) -- cycle;
\fill[blue!15.5, opacity=0.7] (2.3330, 1.9590, 1.1824) -- (2.3820, 1.9590, 1.1775) -- (2.3820, 2.0100, 1.1745) -- (2.3330, 2.0100, 1.1794) -- cycle;
\fill[blue!15.0, opacity=0.7] (2.3330, 2.0100, 1.1794) -- (2.3820, 2.0100, 1.1745) -- (2.3820, 2.0610, 1.1712) -- (2.3330, 2.0610, 1.1762) -- cycle;
\fill[blue!15.0, opacity=0.7] (2.3330, 2.0610, 1.1762) -- (2.3820, 2.0610, 1.1712) -- (2.3820, 2.1120, 1.1676) -- (2.3330, 2.1120, 1.1726) -- cycle;
\fill[blue!15.0, opacity=0.7] (2.3330, 2.1120, 1.1726) -- (2.3820, 2.1120, 1.1676) -- (2.3820, 2.1630, 1.1638) -- (2.3330, 2.1630, 1.1688) -- cycle;
\fill[blue!15.0, opacity=0.7] (2.3330, 2.1630, 1.1688) -- (2.3820, 2.1630, 1.1638) -- (2.3820, 2.2140, 1.1597) -- (2.3330, 2.2140, 1.1647) -- cycle;
\fill[blue!15.0, opacity=0.7] (2.3330, 2.2140, 1.1647) -- (2.3820, 2.2140, 1.1597) -- (2.3820, 2.2650, 1.1554) -- (2.3330, 2.2650, 1.1604) -- cycle;
\fill[blue!15.5, opacity=0.7] (2.3330, 2.2650, 1.1604) -- (2.3820, 2.2650, 1.1554) -- (2.3820, 2.3160, 1.1508) -- (2.3330, 2.3160, 1.1558) -- cycle;
\fill[blue!21.6, opacity=0.7] (2.3330, 2.3160, 1.1558) -- (2.3820, 2.3160, 1.1508) -- (2.3820, 2.3670, 1.1461) -- (2.3330, 2.3670, 1.1510) -- cycle;
\fill[blue!34.2, opacity=0.7] (2.3330, 2.3670, 1.1510) -- (2.3820, 2.3670, 1.1461) -- (2.3820, 2.4180, 1.1411) -- (2.3330, 2.4180, 1.1461) -- cycle;
\fill[blue!29.1, opacity=0.7] (2.3330, 2.4180, 1.1461) -- (2.3820, 2.4180, 1.1411) -- (2.3820, 2.4690, 1.1359) -- (2.3330, 2.4690, 1.1409) -- cycle;
\fill[blue!16.7, opacity=0.7] (2.3330, 2.4690, 1.1409) -- (2.3820, 2.4690, 1.1359) -- (2.3820, 2.5200, 1.1305) -- (2.3330, 2.5200, 1.1355) -- cycle;
\fill[blue!15.0, opacity=0.7] (2.3330, 2.5200, 1.1355) -- (2.3820, 2.5200, 1.1305) -- (2.3820, 2.5710, 1.1250) -- (2.3330, 2.5710, 1.1300) -- cycle;
\fill[blue!15.0, opacity=0.7] (2.3330, 2.5710, 1.1300) -- (2.3820, 2.5710, 1.1250) -- (2.3820, 2.6220, 1.1193) -- (2.3330, 2.6220, 1.1243) -- cycle;
\fill[blue!15.0, opacity=0.7] (2.3330, 2.6220, 1.1243) -- (2.3820, 2.6220, 1.1193) -- (2.3820, 2.6730, 1.1135) -- (2.3330, 2.6730, 1.1185) -- cycle;
\fill[blue!15.0, opacity=0.7] (2.3330, 2.6730, 1.1185) -- (2.3820, 2.6730, 1.1135) -- (2.3820, 2.7240, 1.1076) -- (2.3330, 2.7240, 1.1126) -- cycle;
\fill[blue!15.0, opacity=0.7] (2.3330, 2.7240, 1.1126) -- (2.3820, 2.7240, 1.1076) -- (2.3820, 2.7750, 1.1016) -- (2.3330, 2.7750, 1.1066) -- cycle;
\fill[blue!15.0, opacity=0.7] (2.3330, 2.7750, 1.1066) -- (2.3820, 2.7750, 1.1016) -- (2.3820, 2.8260, 1.0955) -- (2.3330, 2.8260, 1.1005) -- cycle;
\fill[blue!15.0, opacity=0.7] (2.3330, 2.8260, 1.1005) -- (2.3820, 2.8260, 1.0955) -- (2.3820, 2.8770, 1.0893) -- (2.3330, 2.8770, 1.0943) -- cycle;
\fill[blue!15.3, opacity=0.7] (2.3330, 2.8770, 1.0943) -- (2.3820, 2.8770, 1.0893) -- (2.3820, 2.9280, 1.0831) -- (2.3330, 2.9280, 1.0881) -- cycle;
\fill[blue!15.1, opacity=0.7] (2.3330, 2.9280, 1.0881) -- (2.3820, 2.9280, 1.0831) -- (2.3820, 2.9790, 1.0768) -- (2.3330, 2.9790, 1.0818) -- cycle;
\fill[blue!15.0, opacity=0.7] (2.3330, 2.9790, 1.0818) -- (2.3820, 2.9790, 1.0768) -- (2.3820, 3.0300, 1.0705) -- (2.3330, 3.0300, 1.0755) -- cycle;
\fill[blue!15.0, opacity=0.7] (2.3820, -0.0300, 1.0705) -- (2.4310, -0.0300, 1.0654) -- (2.4310, 0.0210, 1.0716) -- (2.3820, 0.0210, 1.0768) -- cycle;
\fill[blue!15.0, opacity=0.7] (2.3820, 0.0210, 1.0768) -- (2.4310, 0.0210, 1.0716) -- (2.4310, 0.0720, 1.0779) -- (2.3820, 0.0720, 1.0831) -- cycle;
\fill[blue!15.0, opacity=0.7] (2.3820, 0.0720, 1.0831) -- (2.4310, 0.0720, 1.0779) -- (2.4310, 0.1230, 1.0841) -- (2.3820, 0.1230, 1.0893) -- cycle;
\fill[blue!15.0, opacity=0.7] (2.3820, 0.1230, 1.0893) -- (2.4310, 0.1230, 1.0841) -- (2.4310, 0.1740, 1.0903) -- (2.3820, 0.1740, 1.0955) -- cycle;
\fill[blue!15.0, opacity=0.7] (2.3820, 0.1740, 1.0955) -- (2.4310, 0.1740, 1.0903) -- (2.4310, 0.2250, 1.0964) -- (2.3820, 0.2250, 1.1016) -- cycle;
\fill[blue!15.2, opacity=0.7] (2.3820, 0.2250, 1.1016) -- (2.4310, 0.2250, 1.0964) -- (2.4310, 0.2760, 1.1024) -- (2.3820, 0.2760, 1.1076) -- cycle;
\fill[blue!16.7, opacity=0.7] (2.3820, 0.2760, 1.1076) -- (2.4310, 0.2760, 1.1024) -- (2.4310, 0.3270, 1.1084) -- (2.3820, 0.3270, 1.1135) -- cycle;
\fill[blue!17.1, opacity=0.7] (2.3820, 0.3270, 1.1135) -- (2.4310, 0.3270, 1.1084) -- (2.4310, 0.3780, 1.1142) -- (2.3820, 0.3780, 1.1193) -- cycle;
\fill[blue!15.5, opacity=0.7] (2.3820, 0.3780, 1.1193) -- (2.4310, 0.3780, 1.1142) -- (2.4310, 0.4290, 1.1198) -- (2.3820, 0.4290, 1.1250) -- cycle;
\fill[blue!15.0, opacity=0.7] (2.3820, 0.4290, 1.1250) -- (2.4310, 0.4290, 1.1198) -- (2.4310, 0.4800, 1.1254) -- (2.3820, 0.4800, 1.1305) -- cycle;
\fill[blue!15.0, opacity=0.7] (2.3820, 0.4800, 1.1305) -- (2.4310, 0.4800, 1.1254) -- (2.4310, 0.5310, 1.1307) -- (2.3820, 0.5310, 1.1359) -- cycle;
\fill[blue!15.0, opacity=0.7] (2.3820, 0.5310, 1.1359) -- (2.4310, 0.5310, 1.1307) -- (2.4310, 0.5820, 1.1359) -- (2.3820, 0.5820, 1.1411) -- cycle;
\fill[blue!15.0, opacity=0.7] (2.3820, 0.5820, 1.1411) -- (2.4310, 0.5820, 1.1359) -- (2.4310, 0.6330, 1.1409) -- (2.3820, 0.6330, 1.1461) -- cycle;
\fill[blue!15.0, opacity=0.7] (2.3820, 0.6330, 1.1461) -- (2.4310, 0.6330, 1.1409) -- (2.4310, 0.6840, 1.1457) -- (2.3820, 0.6840, 1.1508) -- cycle;
\fill[blue!15.0, opacity=0.7] (2.3820, 0.6840, 1.1508) -- (2.4310, 0.6840, 1.1457) -- (2.4310, 0.7350, 1.1502) -- (2.3820, 0.7350, 1.1554) -- cycle;
\fill[blue!15.0, opacity=0.7] (2.3820, 0.7350, 1.1554) -- (2.4310, 0.7350, 1.1502) -- (2.4310, 0.7860, 1.1545) -- (2.3820, 0.7860, 1.1597) -- cycle;
\fill[blue!15.1, opacity=0.7] (2.3820, 0.7860, 1.1597) -- (2.4310, 0.7860, 1.1545) -- (2.4310, 0.8370, 1.1586) -- (2.3820, 0.8370, 1.1638) -- cycle;
\fill[blue!16.1, opacity=0.7] (2.3820, 0.8370, 1.1638) -- (2.4310, 0.8370, 1.1586) -- (2.4310, 0.8880, 1.1624) -- (2.3820, 0.8880, 1.1676) -- cycle;
\fill[blue!23.2, opacity=0.7] (2.3820, 0.8880, 1.1676) -- (2.4310, 0.8880, 1.1624) -- (2.4310, 0.9390, 1.1660) -- (2.3820, 0.9390, 1.1712) -- cycle;
\fill[blue!42.1, opacity=0.7] (2.3820, 0.9390, 1.1712) -- (2.4310, 0.9390, 1.1660) -- (2.4310, 0.9900, 1.1693) -- (2.3820, 0.9900, 1.1745) -- cycle;
\fill[blue!65.3, opacity=0.7] (2.3820, 0.9900, 1.1745) -- (2.4310, 0.9900, 1.1693) -- (2.4310, 1.0410, 1.1723) -- (2.3820, 1.0410, 1.1775) -- cycle;
\fill[blue!82.1, opacity=0.7] (2.3820, 1.0410, 1.1775) -- (2.4310, 1.0410, 1.1723) -- (2.4310, 1.0920, 1.1750) -- (2.3820, 1.0920, 1.1802) -- cycle;
\fill[blue!90.8, opacity=0.7] (2.3820, 1.0920, 1.1802) -- (2.4310, 1.0920, 1.1750) -- (2.4310, 1.1430, 1.1774) -- (2.3820, 1.1430, 1.1826) -- cycle;
\fill[blue!94.4, opacity=0.7] (2.3820, 1.1430, 1.1826) -- (2.4310, 1.1430, 1.1774) -- (2.4310, 1.1940, 1.1795) -- (2.3820, 1.1940, 1.1847) -- cycle;
\fill[blue!95.2, opacity=0.7] (2.3820, 1.1940, 1.1847) -- (2.4310, 1.1940, 1.1795) -- (2.4310, 1.2450, 1.1813) -- (2.3820, 1.2450, 1.1864) -- cycle;
\fill[blue!95.1, opacity=0.7] (2.3820, 1.2450, 1.1864) -- (2.4310, 1.2450, 1.1813) -- (2.4310, 1.2960, 1.1827) -- (2.3820, 1.2960, 1.1879) -- cycle;
\fill[blue!95.3, opacity=0.7] (2.3820, 1.2960, 1.1879) -- (2.4310, 1.2960, 1.1827) -- (2.4310, 1.3470, 1.1839) -- (2.3820, 1.3470, 1.1891) -- cycle;
\fill[blue!96.3, opacity=0.7] (2.3820, 1.3470, 1.1891) -- (2.4310, 1.3470, 1.1839) -- (2.4310, 1.3980, 1.1847) -- (2.3820, 1.3980, 1.1899) -- cycle;
\fill[blue!98.4, opacity=0.7] (2.3820, 1.3980, 1.1899) -- (2.4310, 1.3980, 1.1847) -- (2.4310, 1.4490, 1.1852) -- (2.3820, 1.4490, 1.1904) -- cycle;
\fill[blue!97.2!black, opacity=0.7] (2.3820, 1.4490, 1.1904) -- (2.4310, 1.4490, 1.1852) -- (2.4310, 1.5000, 1.1854) -- (2.3820, 1.5000, 1.1905) -- cycle;
\fill[blue!91.1!black, opacity=0.7] (2.3820, 1.5000, 1.1905) -- (2.4310, 1.5000, 1.1854) -- (2.4310, 1.5510, 1.1852) -- (2.3820, 1.5510, 1.1904) -- cycle;
\fill[blue!88.3!black, opacity=0.7] (2.3820, 1.5510, 1.1904) -- (2.4310, 1.5510, 1.1852) -- (2.4310, 1.6020, 1.1847) -- (2.3820, 1.6020, 1.1899) -- cycle;
\fill[blue!92.0!black, opacity=0.7] (2.3820, 1.6020, 1.1899) -- (2.4310, 1.6020, 1.1847) -- (2.4310, 1.6530, 1.1839) -- (2.3820, 1.6530, 1.1891) -- cycle;
\fill[blue!96.5, opacity=0.7] (2.3820, 1.6530, 1.1891) -- (2.4310, 1.6530, 1.1839) -- (2.4310, 1.7040, 1.1827) -- (2.3820, 1.7040, 1.1879) -- cycle;
\fill[blue!79.9, opacity=0.7] (2.3820, 1.7040, 1.1879) -- (2.4310, 1.7040, 1.1827) -- (2.4310, 1.7550, 1.1813) -- (2.3820, 1.7550, 1.1864) -- cycle;
\fill[blue!52.1, opacity=0.7] (2.3820, 1.7550, 1.1864) -- (2.4310, 1.7550, 1.1813) -- (2.4310, 1.8060, 1.1795) -- (2.3820, 1.8060, 1.1847) -- cycle;
\fill[blue!26.5, opacity=0.7] (2.3820, 1.8060, 1.1847) -- (2.4310, 1.8060, 1.1795) -- (2.4310, 1.8570, 1.1774) -- (2.3820, 1.8570, 1.1826) -- cycle;
\fill[blue!16.5, opacity=0.7] (2.3820, 1.8570, 1.1826) -- (2.4310, 1.8570, 1.1774) -- (2.4310, 1.9080, 1.1750) -- (2.3820, 1.9080, 1.1802) -- cycle;
\fill[blue!15.1, opacity=0.7] (2.3820, 1.9080, 1.1802) -- (2.4310, 1.9080, 1.1750) -- (2.4310, 1.9590, 1.1723) -- (2.3820, 1.9590, 1.1775) -- cycle;
\fill[blue!15.0, opacity=0.7] (2.3820, 1.9590, 1.1775) -- (2.4310, 1.9590, 1.1723) -- (2.4310, 2.0100, 1.1693) -- (2.3820, 2.0100, 1.1745) -- cycle;
\fill[blue!15.0, opacity=0.7] (2.3820, 2.0100, 1.1745) -- (2.4310, 2.0100, 1.1693) -- (2.4310, 2.0610, 1.1660) -- (2.3820, 2.0610, 1.1712) -- cycle;
\fill[blue!15.0, opacity=0.7] (2.3820, 2.0610, 1.1712) -- (2.4310, 2.0610, 1.1660) -- (2.4310, 2.1120, 1.1624) -- (2.3820, 2.1120, 1.1676) -- cycle;
\fill[blue!15.0, opacity=0.7] (2.3820, 2.1120, 1.1676) -- (2.4310, 2.1120, 1.1624) -- (2.4310, 2.1630, 1.1586) -- (2.3820, 2.1630, 1.1638) -- cycle;
\fill[blue!15.0, opacity=0.7] (2.3820, 2.1630, 1.1638) -- (2.4310, 2.1630, 1.1586) -- (2.4310, 2.2140, 1.1545) -- (2.3820, 2.2140, 1.1597) -- cycle;
\fill[blue!15.4, opacity=0.7] (2.3820, 2.2140, 1.1597) -- (2.4310, 2.2140, 1.1545) -- (2.4310, 2.2650, 1.1502) -- (2.3820, 2.2650, 1.1554) -- cycle;
\fill[blue!20.1, opacity=0.7] (2.3820, 2.2650, 1.1554) -- (2.4310, 2.2650, 1.1502) -- (2.4310, 2.3160, 1.1457) -- (2.3820, 2.3160, 1.1508) -- cycle;
\fill[blue!32.0, opacity=0.7] (2.3820, 2.3160, 1.1508) -- (2.4310, 2.3160, 1.1457) -- (2.4310, 2.3670, 1.1409) -- (2.3820, 2.3670, 1.1461) -- cycle;
\fill[blue!30.8, opacity=0.7] (2.3820, 2.3670, 1.1461) -- (2.4310, 2.3670, 1.1409) -- (2.4310, 2.4180, 1.1359) -- (2.3820, 2.4180, 1.1411) -- cycle;
\fill[blue!18.2, opacity=0.7] (2.3820, 2.4180, 1.1411) -- (2.4310, 2.4180, 1.1359) -- (2.4310, 2.4690, 1.1307) -- (2.3820, 2.4690, 1.1359) -- cycle;
\fill[blue!15.1, opacity=0.7] (2.3820, 2.4690, 1.1359) -- (2.4310, 2.4690, 1.1307) -- (2.4310, 2.5200, 1.1254) -- (2.3820, 2.5200, 1.1305) -- cycle;
\fill[blue!15.0, opacity=0.7] (2.3820, 2.5200, 1.1305) -- (2.4310, 2.5200, 1.1254) -- (2.4310, 2.5710, 1.1198) -- (2.3820, 2.5710, 1.1250) -- cycle;
\fill[blue!15.0, opacity=0.7] (2.3820, 2.5710, 1.1250) -- (2.4310, 2.5710, 1.1198) -- (2.4310, 2.6220, 1.1142) -- (2.3820, 2.6220, 1.1193) -- cycle;
\fill[blue!15.0, opacity=0.7] (2.3820, 2.6220, 1.1193) -- (2.4310, 2.6220, 1.1142) -- (2.4310, 2.6730, 1.1084) -- (2.3820, 2.6730, 1.1135) -- cycle;
\fill[blue!15.0, opacity=0.7] (2.3820, 2.6730, 1.1135) -- (2.4310, 2.6730, 1.1084) -- (2.4310, 2.7240, 1.1024) -- (2.3820, 2.7240, 1.1076) -- cycle;
\fill[blue!15.0, opacity=0.7] (2.3820, 2.7240, 1.1076) -- (2.4310, 2.7240, 1.1024) -- (2.4310, 2.7750, 1.0964) -- (2.3820, 2.7750, 1.1016) -- cycle;
\fill[blue!15.0, opacity=0.7] (2.3820, 2.7750, 1.1016) -- (2.4310, 2.7750, 1.0964) -- (2.4310, 2.8260, 1.0903) -- (2.3820, 2.8260, 1.0955) -- cycle;
\fill[blue!15.1, opacity=0.7] (2.3820, 2.8260, 1.0955) -- (2.4310, 2.8260, 1.0903) -- (2.4310, 2.8770, 1.0841) -- (2.3820, 2.8770, 1.0893) -- cycle;
\fill[blue!15.2, opacity=0.7] (2.3820, 2.8770, 1.0893) -- (2.4310, 2.8770, 1.0841) -- (2.4310, 2.9280, 1.0779) -- (2.3820, 2.9280, 1.0831) -- cycle;
\fill[blue!15.0, opacity=0.7] (2.3820, 2.9280, 1.0831) -- (2.4310, 2.9280, 1.0779) -- (2.4310, 2.9790, 1.0716) -- (2.3820, 2.9790, 1.0768) -- cycle;
\fill[blue!15.0, opacity=0.7] (2.3820, 2.9790, 1.0768) -- (2.4310, 2.9790, 1.0716) -- (2.4310, 3.0300, 1.0654) -- (2.3820, 3.0300, 1.0705) -- cycle;
\fill[blue!15.0, opacity=0.7] (2.4310, -0.0300, 1.0654) -- (2.4800, -0.0300, 1.0600) -- (2.4800, 0.0210, 1.0663) -- (2.4310, 0.0210, 1.0716) -- cycle;
\fill[blue!15.0, opacity=0.7] (2.4310, 0.0210, 1.0716) -- (2.4800, 0.0210, 1.0663) -- (2.4800, 0.0720, 1.0725) -- (2.4310, 0.0720, 1.0779) -- cycle;
\fill[blue!15.0, opacity=0.7] (2.4310, 0.0720, 1.0779) -- (2.4800, 0.0720, 1.0725) -- (2.4800, 0.1230, 1.0788) -- (2.4310, 0.1230, 1.0841) -- cycle;
\fill[blue!15.0, opacity=0.7] (2.4310, 0.1230, 1.0841) -- (2.4800, 0.1230, 1.0788) -- (2.4800, 0.1740, 1.0849) -- (2.4310, 0.1740, 1.0903) -- cycle;
\fill[blue!15.0, opacity=0.7] (2.4310, 0.1740, 1.0903) -- (2.4800, 0.1740, 1.0849) -- (2.4800, 0.2250, 1.0911) -- (2.4310, 0.2250, 1.0964) -- cycle;
\fill[blue!15.0, opacity=0.7] (2.4310, 0.2250, 1.0964) -- (2.4800, 0.2250, 1.0911) -- (2.4800, 0.2760, 1.0971) -- (2.4310, 0.2760, 1.1024) -- cycle;
\fill[blue!15.2, opacity=0.7] (2.4310, 0.2760, 1.1024) -- (2.4800, 0.2760, 1.0971) -- (2.4800, 0.3270, 1.1030) -- (2.4310, 0.3270, 1.1084) -- cycle;
\fill[blue!16.8, opacity=0.7] (2.4310, 0.3270, 1.1084) -- (2.4800, 0.3270, 1.1030) -- (2.4800, 0.3780, 1.1088) -- (2.4310, 0.3780, 1.1142) -- cycle;
\fill[blue!17.3, opacity=0.7] (2.4310, 0.3780, 1.1142) -- (2.4800, 0.3780, 1.1088) -- (2.4800, 0.4290, 1.1145) -- (2.4310, 0.4290, 1.1198) -- cycle;
\fill[blue!15.7, opacity=0.7] (2.4310, 0.4290, 1.1198) -- (2.4800, 0.4290, 1.1145) -- (2.4800, 0.4800, 1.1200) -- (2.4310, 0.4800, 1.1254) -- cycle;
\fill[blue!15.1, opacity=0.7] (2.4310, 0.4800, 1.1254) -- (2.4800, 0.4800, 1.1200) -- (2.4800, 0.5310, 1.1254) -- (2.4310, 0.5310, 1.1307) -- cycle;
\fill[blue!15.0, opacity=0.7] (2.4310, 0.5310, 1.1307) -- (2.4800, 0.5310, 1.1254) -- (2.4800, 0.5820, 1.1305) -- (2.4310, 0.5820, 1.1359) -- cycle;
\fill[blue!15.0, opacity=0.7] (2.4310, 0.5820, 1.1359) -- (2.4800, 0.5820, 1.1305) -- (2.4800, 0.6330, 1.1355) -- (2.4310, 0.6330, 1.1409) -- cycle;
\fill[blue!15.0, opacity=0.7] (2.4310, 0.6330, 1.1409) -- (2.4800, 0.6330, 1.1355) -- (2.4800, 0.6840, 1.1403) -- (2.4310, 0.6840, 1.1457) -- cycle;
\fill[blue!15.0, opacity=0.7] (2.4310, 0.6840, 1.1457) -- (2.4800, 0.6840, 1.1403) -- (2.4800, 0.7350, 1.1449) -- (2.4310, 0.7350, 1.1502) -- cycle;
\fill[blue!15.0, opacity=0.7] (2.4310, 0.7350, 1.1502) -- (2.4800, 0.7350, 1.1449) -- (2.4800, 0.7860, 1.1492) -- (2.4310, 0.7860, 1.1545) -- cycle;
\fill[blue!15.0, opacity=0.7] (2.4310, 0.7860, 1.1545) -- (2.4800, 0.7860, 1.1492) -- (2.4800, 0.8370, 1.1533) -- (2.4310, 0.8370, 1.1586) -- cycle;
\fill[blue!15.0, opacity=0.7] (2.4310, 0.8370, 1.1586) -- (2.4800, 0.8370, 1.1533) -- (2.4800, 0.8880, 1.1571) -- (2.4310, 0.8880, 1.1624) -- cycle;
\fill[blue!15.1, opacity=0.7] (2.4310, 0.8880, 1.1624) -- (2.4800, 0.8880, 1.1571) -- (2.4800, 0.9390, 1.1606) -- (2.4310, 0.9390, 1.1660) -- cycle;
\fill[blue!15.8, opacity=0.7] (2.4310, 0.9390, 1.1660) -- (2.4800, 0.9390, 1.1606) -- (2.4800, 0.9900, 1.1639) -- (2.4310, 0.9900, 1.1693) -- cycle;
\fill[blue!19.6, opacity=0.7] (2.4310, 0.9900, 1.1693) -- (2.4800, 0.9900, 1.1639) -- (2.4800, 1.0410, 1.1669) -- (2.4310, 1.0410, 1.1723) -- cycle;
\fill[blue!29.7, opacity=0.7] (2.4310, 1.0410, 1.1723) -- (2.4800, 1.0410, 1.1669) -- (2.4800, 1.0920, 1.1696) -- (2.4310, 1.0920, 1.1750) -- cycle;
\fill[blue!44.8, opacity=0.7] (2.4310, 1.0920, 1.1750) -- (2.4800, 1.0920, 1.1696) -- (2.4800, 1.1430, 1.1720) -- (2.4310, 1.1430, 1.1774) -- cycle;
\fill[blue!60.1, opacity=0.7] (2.4310, 1.1430, 1.1774) -- (2.4800, 1.1430, 1.1720) -- (2.4800, 1.1940, 1.1741) -- (2.4310, 1.1940, 1.1795) -- cycle;
\fill[blue!72.1, opacity=0.7] (2.4310, 1.1940, 1.1795) -- (2.4800, 1.1940, 1.1741) -- (2.4800, 1.2450, 1.1759) -- (2.4310, 1.2450, 1.1813) -- cycle;
\fill[blue!80.2, opacity=0.7] (2.4310, 1.2450, 1.1813) -- (2.4800, 1.2450, 1.1759) -- (2.4800, 1.2960, 1.1774) -- (2.4310, 1.2960, 1.1827) -- cycle;
\fill[blue!85.1, opacity=0.7] (2.4310, 1.2960, 1.1827) -- (2.4800, 1.2960, 1.1774) -- (2.4800, 1.3470, 1.1785) -- (2.4310, 1.3470, 1.1839) -- cycle;
\fill[blue!87.5, opacity=0.7] (2.4310, 1.3470, 1.1839) -- (2.4800, 1.3470, 1.1785) -- (2.4800, 1.3980, 1.1793) -- (2.4310, 1.3980, 1.1847) -- cycle;
\fill[blue!87.6, opacity=0.7] (2.4310, 1.3980, 1.1847) -- (2.4800, 1.3980, 1.1793) -- (2.4800, 1.4490, 1.1798) -- (2.4310, 1.4490, 1.1852) -- cycle;
\fill[blue!85.2, opacity=0.7] (2.4310, 1.4490, 1.1852) -- (2.4800, 1.4490, 1.1798) -- (2.4800, 1.5000, 1.1800) -- (2.4310, 1.5000, 1.1854) -- cycle;
\fill[blue!79.3, opacity=0.7] (2.4310, 1.5000, 1.1854) -- (2.4800, 1.5000, 1.1800) -- (2.4800, 1.5510, 1.1798) -- (2.4310, 1.5510, 1.1852) -- cycle;
\fill[blue!68.6, opacity=0.7] (2.4310, 1.5510, 1.1852) -- (2.4800, 1.5510, 1.1798) -- (2.4800, 1.6020, 1.1793) -- (2.4310, 1.6020, 1.1847) -- cycle;
\fill[blue!52.7, opacity=0.7] (2.4310, 1.6020, 1.1847) -- (2.4800, 1.6020, 1.1793) -- (2.4800, 1.6530, 1.1785) -- (2.4310, 1.6530, 1.1839) -- cycle;
\fill[blue!34.8, opacity=0.7] (2.4310, 1.6530, 1.1839) -- (2.4800, 1.6530, 1.1785) -- (2.4800, 1.7040, 1.1774) -- (2.4310, 1.7040, 1.1827) -- cycle;
\fill[blue!21.6, opacity=0.7] (2.4310, 1.7040, 1.1827) -- (2.4800, 1.7040, 1.1774) -- (2.4800, 1.7550, 1.1759) -- (2.4310, 1.7550, 1.1813) -- cycle;
\fill[blue!16.2, opacity=0.7] (2.4310, 1.7550, 1.1813) -- (2.4800, 1.7550, 1.1759) -- (2.4800, 1.8060, 1.1741) -- (2.4310, 1.8060, 1.1795) -- cycle;
\fill[blue!15.1, opacity=0.7] (2.4310, 1.8060, 1.1795) -- (2.4800, 1.8060, 1.1741) -- (2.4800, 1.8570, 1.1720) -- (2.4310, 1.8570, 1.1774) -- cycle;
\fill[blue!15.0, opacity=0.7] (2.4310, 1.8570, 1.1774) -- (2.4800, 1.8570, 1.1720) -- (2.4800, 1.9080, 1.1696) -- (2.4310, 1.9080, 1.1750) -- cycle;
\fill[blue!15.0, opacity=0.7] (2.4310, 1.9080, 1.1750) -- (2.4800, 1.9080, 1.1696) -- (2.4800, 1.9590, 1.1669) -- (2.4310, 1.9590, 1.1723) -- cycle;
\fill[blue!15.0, opacity=0.7] (2.4310, 1.9590, 1.1723) -- (2.4800, 1.9590, 1.1669) -- (2.4800, 2.0100, 1.1639) -- (2.4310, 2.0100, 1.1693) -- cycle;
\fill[blue!15.0, opacity=0.7] (2.4310, 2.0100, 1.1693) -- (2.4800, 2.0100, 1.1639) -- (2.4800, 2.0610, 1.1606) -- (2.4310, 2.0610, 1.1660) -- cycle;
\fill[blue!15.0, opacity=0.7] (2.4310, 2.0610, 1.1660) -- (2.4800, 2.0610, 1.1606) -- (2.4800, 2.1120, 1.1571) -- (2.4310, 2.1120, 1.1624) -- cycle;
\fill[blue!15.0, opacity=0.7] (2.4310, 2.1120, 1.1624) -- (2.4800, 2.1120, 1.1571) -- (2.4800, 2.1630, 1.1533) -- (2.4310, 2.1630, 1.1586) -- cycle;
\fill[blue!15.4, opacity=0.7] (2.4310, 2.1630, 1.1586) -- (2.4800, 2.1630, 1.1533) -- (2.4800, 2.2140, 1.1492) -- (2.4310, 2.2140, 1.1545) -- cycle;
\fill[blue!19.8, opacity=0.7] (2.4310, 2.2140, 1.1545) -- (2.4800, 2.2140, 1.1492) -- (2.4800, 2.2650, 1.1449) -- (2.4310, 2.2650, 1.1502) -- cycle;
\fill[blue!30.6, opacity=0.7] (2.4310, 2.2650, 1.1502) -- (2.4800, 2.2650, 1.1449) -- (2.4800, 2.3160, 1.1403) -- (2.4310, 2.3160, 1.1457) -- cycle;
\fill[blue!30.8, opacity=0.7] (2.4310, 2.3160, 1.1457) -- (2.4800, 2.3160, 1.1403) -- (2.4800, 2.3670, 1.1355) -- (2.4310, 2.3670, 1.1409) -- cycle;
\fill[blue!19.1, opacity=0.7] (2.4310, 2.3670, 1.1409) -- (2.4800, 2.3670, 1.1355) -- (2.4800, 2.4180, 1.1305) -- (2.4310, 2.4180, 1.1359) -- cycle;
\fill[blue!15.1, opacity=0.7] (2.4310, 2.4180, 1.1359) -- (2.4800, 2.4180, 1.1305) -- (2.4800, 2.4690, 1.1254) -- (2.4310, 2.4690, 1.1307) -- cycle;
\fill[blue!15.0, opacity=0.7] (2.4310, 2.4690, 1.1307) -- (2.4800, 2.4690, 1.1254) -- (2.4800, 2.5200, 1.1200) -- (2.4310, 2.5200, 1.1254) -- cycle;
\fill[blue!15.0, opacity=0.7] (2.4310, 2.5200, 1.1254) -- (2.4800, 2.5200, 1.1200) -- (2.4800, 2.5710, 1.1145) -- (2.4310, 2.5710, 1.1198) -- cycle;
\fill[blue!15.0, opacity=0.7] (2.4310, 2.5710, 1.1198) -- (2.4800, 2.5710, 1.1145) -- (2.4800, 2.6220, 1.1088) -- (2.4310, 2.6220, 1.1142) -- cycle;
\fill[blue!15.0, opacity=0.7] (2.4310, 2.6220, 1.1142) -- (2.4800, 2.6220, 1.1088) -- (2.4800, 2.6730, 1.1030) -- (2.4310, 2.6730, 1.1084) -- cycle;
\fill[blue!15.0, opacity=0.7] (2.4310, 2.6730, 1.1084) -- (2.4800, 2.6730, 1.1030) -- (2.4800, 2.7240, 1.0971) -- (2.4310, 2.7240, 1.1024) -- cycle;
\fill[blue!15.0, opacity=0.7] (2.4310, 2.7240, 1.1024) -- (2.4800, 2.7240, 1.0971) -- (2.4800, 2.7750, 1.0911) -- (2.4310, 2.7750, 1.0964) -- cycle;
\fill[blue!15.0, opacity=0.7] (2.4310, 2.7750, 1.0964) -- (2.4800, 2.7750, 1.0911) -- (2.4800, 2.8260, 1.0849) -- (2.4310, 2.8260, 1.0903) -- cycle;
\fill[blue!15.2, opacity=0.7] (2.4310, 2.8260, 1.0903) -- (2.4800, 2.8260, 1.0849) -- (2.4800, 2.8770, 1.0788) -- (2.4310, 2.8770, 1.0841) -- cycle;
\fill[blue!15.1, opacity=0.7] (2.4310, 2.8770, 1.0841) -- (2.4800, 2.8770, 1.0788) -- (2.4800, 2.9280, 1.0725) -- (2.4310, 2.9280, 1.0779) -- cycle;
\fill[blue!15.0, opacity=0.7] (2.4310, 2.9280, 1.0779) -- (2.4800, 2.9280, 1.0725) -- (2.4800, 2.9790, 1.0663) -- (2.4310, 2.9790, 1.0716) -- cycle;
\fill[blue!15.0, opacity=0.7] (2.4310, 2.9790, 1.0716) -- (2.4800, 2.9790, 1.0663) -- (2.4800, 3.0300, 1.0600) -- (2.4310, 3.0300, 1.0654) -- cycle;
\fill[blue!15.0, opacity=0.7] (2.4800, -0.0300, 1.0600) -- (2.5290, -0.0300, 1.0545) -- (2.5290, 0.0210, 1.0608) -- (2.4800, 0.0210, 1.0663) -- cycle;
\fill[blue!15.0, opacity=0.7] (2.4800, 0.0210, 1.0663) -- (2.5290, 0.0210, 1.0608) -- (2.5290, 0.0720, 1.0670) -- (2.4800, 0.0720, 1.0725) -- cycle;
\fill[blue!15.0, opacity=0.7] (2.4800, 0.0720, 1.0725) -- (2.5290, 0.0720, 1.0670) -- (2.5290, 0.1230, 1.0733) -- (2.4800, 0.1230, 1.0788) -- cycle;
\fill[blue!15.0, opacity=0.7] (2.4800, 0.1230, 1.0788) -- (2.5290, 0.1230, 1.0733) -- (2.5290, 0.1740, 1.0794) -- (2.4800, 0.1740, 1.0849) -- cycle;
\fill[blue!15.0, opacity=0.7] (2.4800, 0.1740, 1.0849) -- (2.5290, 0.1740, 1.0794) -- (2.5290, 0.2250, 1.0855) -- (2.4800, 0.2250, 1.0911) -- cycle;
\fill[blue!15.0, opacity=0.7] (2.4800, 0.2250, 1.0911) -- (2.5290, 0.2250, 1.0855) -- (2.5290, 0.2760, 1.0916) -- (2.4800, 0.2760, 1.0971) -- cycle;
\fill[blue!15.0, opacity=0.7] (2.4800, 0.2760, 1.0971) -- (2.5290, 0.2760, 1.0916) -- (2.5290, 0.3270, 1.0975) -- (2.4800, 0.3270, 1.1030) -- cycle;
\fill[blue!15.2, opacity=0.7] (2.4800, 0.3270, 1.1030) -- (2.5290, 0.3270, 1.0975) -- (2.5290, 0.3780, 1.1033) -- (2.4800, 0.3780, 1.1088) -- cycle;
\fill[blue!16.8, opacity=0.7] (2.4800, 0.3780, 1.1088) -- (2.5290, 0.3780, 1.1033) -- (2.5290, 0.4290, 1.1090) -- (2.4800, 0.4290, 1.1145) -- cycle;
\fill[blue!17.7, opacity=0.7] (2.4800, 0.4290, 1.1145) -- (2.5290, 0.4290, 1.1090) -- (2.5290, 0.4800, 1.1145) -- (2.4800, 0.4800, 1.1200) -- cycle;
\fill[blue!16.1, opacity=0.7] (2.4800, 0.4800, 1.1200) -- (2.5290, 0.4800, 1.1145) -- (2.5290, 0.5310, 1.1198) -- (2.4800, 0.5310, 1.1254) -- cycle;
\fill[blue!15.1, opacity=0.7] (2.4800, 0.5310, 1.1254) -- (2.5290, 0.5310, 1.1198) -- (2.5290, 0.5820, 1.1250) -- (2.4800, 0.5820, 1.1305) -- cycle;
\fill[blue!15.0, opacity=0.7] (2.4800, 0.5820, 1.1305) -- (2.5290, 0.5820, 1.1250) -- (2.5290, 0.6330, 1.1300) -- (2.4800, 0.6330, 1.1355) -- cycle;
\fill[blue!15.0, opacity=0.7] (2.4800, 0.6330, 1.1355) -- (2.5290, 0.6330, 1.1300) -- (2.5290, 0.6840, 1.1348) -- (2.4800, 0.6840, 1.1403) -- cycle;
\fill[blue!15.0, opacity=0.7] (2.4800, 0.6840, 1.1403) -- (2.5290, 0.6840, 1.1348) -- (2.5290, 0.7350, 1.1393) -- (2.4800, 0.7350, 1.1449) -- cycle;
\fill[blue!15.0, opacity=0.7] (2.4800, 0.7350, 1.1449) -- (2.5290, 0.7350, 1.1393) -- (2.5290, 0.7860, 1.1437) -- (2.4800, 0.7860, 1.1492) -- cycle;
\fill[blue!15.0, opacity=0.7] (2.4800, 0.7860, 1.1492) -- (2.5290, 0.7860, 1.1437) -- (2.5290, 0.8370, 1.1477) -- (2.4800, 0.8370, 1.1533) -- cycle;
\fill[blue!15.0, opacity=0.7] (2.4800, 0.8370, 1.1533) -- (2.5290, 0.8370, 1.1477) -- (2.5290, 0.8880, 1.1516) -- (2.4800, 0.8880, 1.1571) -- cycle;
\fill[blue!15.0, opacity=0.7] (2.4800, 0.8880, 1.1571) -- (2.5290, 0.8880, 1.1516) -- (2.5290, 0.9390, 1.1551) -- (2.4800, 0.9390, 1.1606) -- cycle;
\fill[blue!15.0, opacity=0.7] (2.4800, 0.9390, 1.1606) -- (2.5290, 0.9390, 1.1551) -- (2.5290, 0.9900, 1.1584) -- (2.4800, 0.9900, 1.1639) -- cycle;
\fill[blue!15.0, opacity=0.7] (2.4800, 0.9900, 1.1639) -- (2.5290, 0.9900, 1.1584) -- (2.5290, 1.0410, 1.1614) -- (2.4800, 1.0410, 1.1669) -- cycle;
\fill[blue!15.1, opacity=0.7] (2.4800, 1.0410, 1.1669) -- (2.5290, 1.0410, 1.1614) -- (2.5290, 1.0920, 1.1641) -- (2.4800, 1.0920, 1.1696) -- cycle;
\fill[blue!15.7, opacity=0.7] (2.4800, 1.0920, 1.1696) -- (2.5290, 1.0920, 1.1641) -- (2.5290, 1.1430, 1.1665) -- (2.4800, 1.1430, 1.1720) -- cycle;
\fill[blue!17.3, opacity=0.7] (2.4800, 1.1430, 1.1720) -- (2.5290, 1.1430, 1.1665) -- (2.5290, 1.1940, 1.1686) -- (2.4800, 1.1940, 1.1741) -- cycle;
\fill[blue!20.0, opacity=0.7] (2.4800, 1.1940, 1.1741) -- (2.5290, 1.1940, 1.1686) -- (2.5290, 1.2450, 1.1704) -- (2.4800, 1.2450, 1.1759) -- cycle;
\fill[blue!23.4, opacity=0.7] (2.4800, 1.2450, 1.1759) -- (2.5290, 1.2450, 1.1704) -- (2.5290, 1.2960, 1.1719) -- (2.4800, 1.2960, 1.1774) -- cycle;
\fill[blue!26.2, opacity=0.7] (2.4800, 1.2960, 1.1774) -- (2.5290, 1.2960, 1.1719) -- (2.5290, 1.3470, 1.1730) -- (2.4800, 1.3470, 1.1785) -- cycle;
\fill[blue!27.6, opacity=0.7] (2.4800, 1.3470, 1.1785) -- (2.5290, 1.3470, 1.1730) -- (2.5290, 1.3980, 1.1738) -- (2.4800, 1.3980, 1.1793) -- cycle;
\fill[blue!27.0, opacity=0.7] (2.4800, 1.3980, 1.1793) -- (2.5290, 1.3980, 1.1738) -- (2.5290, 1.4490, 1.1743) -- (2.4800, 1.4490, 1.1798) -- cycle;
\fill[blue!24.5, opacity=0.7] (2.4800, 1.4490, 1.1798) -- (2.5290, 1.4490, 1.1743) -- (2.5290, 1.5000, 1.1745) -- (2.4800, 1.5000, 1.1800) -- cycle;
\fill[blue!21.1, opacity=0.7] (2.4800, 1.5000, 1.1800) -- (2.5290, 1.5000, 1.1745) -- (2.5290, 1.5510, 1.1743) -- (2.4800, 1.5510, 1.1798) -- cycle;
\fill[blue!18.0, opacity=0.7] (2.4800, 1.5510, 1.1798) -- (2.5290, 1.5510, 1.1743) -- (2.5290, 1.6020, 1.1738) -- (2.4800, 1.6020, 1.1793) -- cycle;
\fill[blue!16.0, opacity=0.7] (2.4800, 1.6020, 1.1793) -- (2.5290, 1.6020, 1.1738) -- (2.5290, 1.6530, 1.1730) -- (2.4800, 1.6530, 1.1785) -- cycle;
\fill[blue!15.2, opacity=0.7] (2.4800, 1.6530, 1.1785) -- (2.5290, 1.6530, 1.1730) -- (2.5290, 1.7040, 1.1719) -- (2.4800, 1.7040, 1.1774) -- cycle;
\fill[blue!15.0, opacity=0.7] (2.4800, 1.7040, 1.1774) -- (2.5290, 1.7040, 1.1719) -- (2.5290, 1.7550, 1.1704) -- (2.4800, 1.7550, 1.1759) -- cycle;
\fill[blue!15.0, opacity=0.7] (2.4800, 1.7550, 1.1759) -- (2.5290, 1.7550, 1.1704) -- (2.5290, 1.8060, 1.1686) -- (2.4800, 1.8060, 1.1741) -- cycle;
\fill[blue!15.0, opacity=0.7] (2.4800, 1.8060, 1.1741) -- (2.5290, 1.8060, 1.1686) -- (2.5290, 1.8570, 1.1665) -- (2.4800, 1.8570, 1.1720) -- cycle;
\fill[blue!15.0, opacity=0.7] (2.4800, 1.8570, 1.1720) -- (2.5290, 1.8570, 1.1665) -- (2.5290, 1.9080, 1.1641) -- (2.4800, 1.9080, 1.1696) -- cycle;
\fill[blue!15.0, opacity=0.7] (2.4800, 1.9080, 1.1696) -- (2.5290, 1.9080, 1.1641) -- (2.5290, 1.9590, 1.1614) -- (2.4800, 1.9590, 1.1669) -- cycle;
\fill[blue!15.0, opacity=0.7] (2.4800, 1.9590, 1.1669) -- (2.5290, 1.9590, 1.1614) -- (2.5290, 2.0100, 1.1584) -- (2.4800, 2.0100, 1.1639) -- cycle;
\fill[blue!15.0, opacity=0.7] (2.4800, 2.0100, 1.1639) -- (2.5290, 2.0100, 1.1584) -- (2.5290, 2.0610, 1.1551) -- (2.4800, 2.0610, 1.1606) -- cycle;
\fill[blue!15.0, opacity=0.7] (2.4800, 2.0610, 1.1606) -- (2.5290, 2.0610, 1.1551) -- (2.5290, 2.1120, 1.1516) -- (2.4800, 2.1120, 1.1571) -- cycle;
\fill[blue!15.7, opacity=0.7] (2.4800, 2.1120, 1.1571) -- (2.5290, 2.1120, 1.1516) -- (2.5290, 2.1630, 1.1477) -- (2.4800, 2.1630, 1.1533) -- cycle;
\fill[blue!20.6, opacity=0.7] (2.4800, 2.1630, 1.1533) -- (2.5290, 2.1630, 1.1477) -- (2.5290, 2.2140, 1.1437) -- (2.4800, 2.2140, 1.1492) -- cycle;
\fill[blue!30.2, opacity=0.7] (2.4800, 2.2140, 1.1492) -- (2.5290, 2.2140, 1.1437) -- (2.5290, 2.2650, 1.1393) -- (2.4800, 2.2650, 1.1449) -- cycle;
\fill[blue!29.8, opacity=0.7] (2.4800, 2.2650, 1.1449) -- (2.5290, 2.2650, 1.1393) -- (2.5290, 2.3160, 1.1348) -- (2.4800, 2.3160, 1.1403) -- cycle;
\fill[blue!19.2, opacity=0.7] (2.4800, 2.3160, 1.1403) -- (2.5290, 2.3160, 1.1348) -- (2.5290, 2.3670, 1.1300) -- (2.4800, 2.3670, 1.1355) -- cycle;
\fill[blue!15.2, opacity=0.7] (2.4800, 2.3670, 1.1355) -- (2.5290, 2.3670, 1.1300) -- (2.5290, 2.4180, 1.1250) -- (2.4800, 2.4180, 1.1305) -- cycle;
\fill[blue!15.0, opacity=0.7] (2.4800, 2.4180, 1.1305) -- (2.5290, 2.4180, 1.1250) -- (2.5290, 2.4690, 1.1198) -- (2.4800, 2.4690, 1.1254) -- cycle;
\fill[blue!15.0, opacity=0.7] (2.4800, 2.4690, 1.1254) -- (2.5290, 2.4690, 1.1198) -- (2.5290, 2.5200, 1.1145) -- (2.4800, 2.5200, 1.1200) -- cycle;
\fill[blue!15.0, opacity=0.7] (2.4800, 2.5200, 1.1200) -- (2.5290, 2.5200, 1.1145) -- (2.5290, 2.5710, 1.1090) -- (2.4800, 2.5710, 1.1145) -- cycle;
\fill[blue!15.0, opacity=0.7] (2.4800, 2.5710, 1.1145) -- (2.5290, 2.5710, 1.1090) -- (2.5290, 2.6220, 1.1033) -- (2.4800, 2.6220, 1.1088) -- cycle;
\fill[blue!15.0, opacity=0.7] (2.4800, 2.6220, 1.1088) -- (2.5290, 2.6220, 1.1033) -- (2.5290, 2.6730, 1.0975) -- (2.4800, 2.6730, 1.1030) -- cycle;
\fill[blue!15.0, opacity=0.7] (2.4800, 2.6730, 1.1030) -- (2.5290, 2.6730, 1.0975) -- (2.5290, 2.7240, 1.0916) -- (2.4800, 2.7240, 1.0971) -- cycle;
\fill[blue!15.0, opacity=0.7] (2.4800, 2.7240, 1.0971) -- (2.5290, 2.7240, 1.0916) -- (2.5290, 2.7750, 1.0855) -- (2.4800, 2.7750, 1.0911) -- cycle;
\fill[blue!15.2, opacity=0.7] (2.4800, 2.7750, 1.0911) -- (2.5290, 2.7750, 1.0855) -- (2.5290, 2.8260, 1.0794) -- (2.4800, 2.8260, 1.0849) -- cycle;
\fill[blue!15.1, opacity=0.7] (2.4800, 2.8260, 1.0849) -- (2.5290, 2.8260, 1.0794) -- (2.5290, 2.8770, 1.0733) -- (2.4800, 2.8770, 1.0788) -- cycle;
\fill[blue!15.0, opacity=0.7] (2.4800, 2.8770, 1.0788) -- (2.5290, 2.8770, 1.0733) -- (2.5290, 2.9280, 1.0670) -- (2.4800, 2.9280, 1.0725) -- cycle;
\fill[blue!15.0, opacity=0.7] (2.4800, 2.9280, 1.0725) -- (2.5290, 2.9280, 1.0670) -- (2.5290, 2.9790, 1.0608) -- (2.4800, 2.9790, 1.0663) -- cycle;
\fill[blue!15.0, opacity=0.7] (2.4800, 2.9790, 1.0663) -- (2.5290, 2.9790, 1.0608) -- (2.5290, 3.0300, 1.0545) -- (2.4800, 3.0300, 1.0600) -- cycle;
\fill[blue!15.0, opacity=0.7] (2.5290, -0.0300, 1.0545) -- (2.5780, -0.0300, 1.0488) -- (2.5780, 0.0210, 1.0551) -- (2.5290, 0.0210, 1.0608) -- cycle;
\fill[blue!15.0, opacity=0.7] (2.5290, 0.0210, 1.0608) -- (2.5780, 0.0210, 1.0551) -- (2.5780, 0.0720, 1.0614) -- (2.5290, 0.0720, 1.0670) -- cycle;
\fill[blue!15.0, opacity=0.7] (2.5290, 0.0720, 1.0670) -- (2.5780, 0.0720, 1.0614) -- (2.5780, 0.1230, 1.0676) -- (2.5290, 0.1230, 1.0733) -- cycle;
\fill[blue!15.0, opacity=0.7] (2.5290, 0.1230, 1.0733) -- (2.5780, 0.1230, 1.0676) -- (2.5780, 0.1740, 1.0738) -- (2.5290, 0.1740, 1.0794) -- cycle;
\fill[blue!15.0, opacity=0.7] (2.5290, 0.1740, 1.0794) -- (2.5780, 0.1740, 1.0738) -- (2.5780, 0.2250, 1.0799) -- (2.5290, 0.2250, 1.0855) -- cycle;
\fill[blue!15.0, opacity=0.7] (2.5290, 0.2250, 1.0855) -- (2.5780, 0.2250, 1.0799) -- (2.5780, 0.2760, 1.0859) -- (2.5290, 0.2760, 1.0916) -- cycle;
\fill[blue!15.0, opacity=0.7] (2.5290, 0.2760, 1.0916) -- (2.5780, 0.2760, 1.0859) -- (2.5780, 0.3270, 1.0918) -- (2.5290, 0.3270, 1.0975) -- cycle;
\fill[blue!15.0, opacity=0.7] (2.5290, 0.3270, 1.0975) -- (2.5780, 0.3270, 1.0918) -- (2.5780, 0.3780, 1.0976) -- (2.5290, 0.3780, 1.1033) -- cycle;
\fill[blue!15.2, opacity=0.7] (2.5290, 0.3780, 1.1033) -- (2.5780, 0.3780, 1.0976) -- (2.5780, 0.4290, 1.1033) -- (2.5290, 0.4290, 1.1090) -- cycle;
\fill[blue!16.5, opacity=0.7] (2.5290, 0.4290, 1.1090) -- (2.5780, 0.4290, 1.1033) -- (2.5780, 0.4800, 1.1088) -- (2.5290, 0.4800, 1.1145) -- cycle;
\fill[blue!18.0, opacity=0.7] (2.5290, 0.4800, 1.1145) -- (2.5780, 0.4800, 1.1088) -- (2.5780, 0.5310, 1.1142) -- (2.5290, 0.5310, 1.1198) -- cycle;
\fill[blue!16.8, opacity=0.7] (2.5290, 0.5310, 1.1198) -- (2.5780, 0.5310, 1.1142) -- (2.5780, 0.5820, 1.1193) -- (2.5290, 0.5820, 1.1250) -- cycle;
\fill[blue!15.4, opacity=0.7] (2.5290, 0.5820, 1.1250) -- (2.5780, 0.5820, 1.1193) -- (2.5780, 0.6330, 1.1243) -- (2.5290, 0.6330, 1.1300) -- cycle;
\fill[blue!15.0, opacity=0.7] (2.5290, 0.6330, 1.1300) -- (2.5780, 0.6330, 1.1243) -- (2.5780, 0.6840, 1.1291) -- (2.5290, 0.6840, 1.1348) -- cycle;
\fill[blue!15.0, opacity=0.7] (2.5290, 0.6840, 1.1348) -- (2.5780, 0.6840, 1.1291) -- (2.5780, 0.7350, 1.1337) -- (2.5290, 0.7350, 1.1393) -- cycle;
\fill[blue!15.0, opacity=0.7] (2.5290, 0.7350, 1.1393) -- (2.5780, 0.7350, 1.1337) -- (2.5780, 0.7860, 1.1380) -- (2.5290, 0.7860, 1.1437) -- cycle;
\fill[blue!15.0, opacity=0.7] (2.5290, 0.7860, 1.1437) -- (2.5780, 0.7860, 1.1380) -- (2.5780, 0.8370, 1.1421) -- (2.5290, 0.8370, 1.1477) -- cycle;
\fill[blue!15.0, opacity=0.7] (2.5290, 0.8370, 1.1477) -- (2.5780, 0.8370, 1.1421) -- (2.5780, 0.8880, 1.1459) -- (2.5290, 0.8880, 1.1516) -- cycle;
\fill[blue!15.0, opacity=0.7] (2.5290, 0.8880, 1.1516) -- (2.5780, 0.8880, 1.1459) -- (2.5780, 0.9390, 1.1494) -- (2.5290, 0.9390, 1.1551) -- cycle;
\fill[blue!15.0, opacity=0.7] (2.5290, 0.9390, 1.1551) -- (2.5780, 0.9390, 1.1494) -- (2.5780, 0.9900, 1.1527) -- (2.5290, 0.9900, 1.1584) -- cycle;
\fill[blue!15.0, opacity=0.7] (2.5290, 0.9900, 1.1584) -- (2.5780, 0.9900, 1.1527) -- (2.5780, 1.0410, 1.1557) -- (2.5290, 1.0410, 1.1614) -- cycle;
\fill[blue!15.0, opacity=0.7] (2.5290, 1.0410, 1.1614) -- (2.5780, 1.0410, 1.1557) -- (2.5780, 1.0920, 1.1584) -- (2.5290, 1.0920, 1.1641) -- cycle;
\fill[blue!15.0, opacity=0.7] (2.5290, 1.0920, 1.1641) -- (2.5780, 1.0920, 1.1584) -- (2.5780, 1.1430, 1.1608) -- (2.5290, 1.1430, 1.1665) -- cycle;
\fill[blue!15.0, opacity=0.7] (2.5290, 1.1430, 1.1665) -- (2.5780, 1.1430, 1.1608) -- (2.5780, 1.1940, 1.1629) -- (2.5290, 1.1940, 1.1686) -- cycle;
\fill[blue!15.0, opacity=0.7] (2.5290, 1.1940, 1.1686) -- (2.5780, 1.1940, 1.1629) -- (2.5780, 1.2450, 1.1647) -- (2.5290, 1.2450, 1.1704) -- cycle;
\fill[blue!15.0, opacity=0.7] (2.5290, 1.2450, 1.1704) -- (2.5780, 1.2450, 1.1647) -- (2.5780, 1.2960, 1.1662) -- (2.5290, 1.2960, 1.1719) -- cycle;
\fill[blue!15.1, opacity=0.7] (2.5290, 1.2960, 1.1719) -- (2.5780, 1.2960, 1.1662) -- (2.5780, 1.3470, 1.1673) -- (2.5290, 1.3470, 1.1730) -- cycle;
\fill[blue!15.1, opacity=0.7] (2.5290, 1.3470, 1.1730) -- (2.5780, 1.3470, 1.1673) -- (2.5780, 1.3980, 1.1682) -- (2.5290, 1.3980, 1.1738) -- cycle;
\fill[blue!15.1, opacity=0.7] (2.5290, 1.3980, 1.1738) -- (2.5780, 1.3980, 1.1682) -- (2.5780, 1.4490, 1.1686) -- (2.5290, 1.4490, 1.1743) -- cycle;
\fill[blue!15.0, opacity=0.7] (2.5290, 1.4490, 1.1743) -- (2.5780, 1.4490, 1.1686) -- (2.5780, 1.5000, 1.1688) -- (2.5290, 1.5000, 1.1745) -- cycle;
\fill[blue!15.0, opacity=0.7] (2.5290, 1.5000, 1.1745) -- (2.5780, 1.5000, 1.1688) -- (2.5780, 1.5510, 1.1686) -- (2.5290, 1.5510, 1.1743) -- cycle;
\fill[blue!15.0, opacity=0.7] (2.5290, 1.5510, 1.1743) -- (2.5780, 1.5510, 1.1686) -- (2.5780, 1.6020, 1.1682) -- (2.5290, 1.6020, 1.1738) -- cycle;
\fill[blue!15.0, opacity=0.7] (2.5290, 1.6020, 1.1738) -- (2.5780, 1.6020, 1.1682) -- (2.5780, 1.6530, 1.1673) -- (2.5290, 1.6530, 1.1730) -- cycle;
\fill[blue!15.0, opacity=0.7] (2.5290, 1.6530, 1.1730) -- (2.5780, 1.6530, 1.1673) -- (2.5780, 1.7040, 1.1662) -- (2.5290, 1.7040, 1.1719) -- cycle;
\fill[blue!15.0, opacity=0.7] (2.5290, 1.7040, 1.1719) -- (2.5780, 1.7040, 1.1662) -- (2.5780, 1.7550, 1.1647) -- (2.5290, 1.7550, 1.1704) -- cycle;
\fill[blue!15.0, opacity=0.7] (2.5290, 1.7550, 1.1704) -- (2.5780, 1.7550, 1.1647) -- (2.5780, 1.8060, 1.1629) -- (2.5290, 1.8060, 1.1686) -- cycle;
\fill[blue!15.0, opacity=0.7] (2.5290, 1.8060, 1.1686) -- (2.5780, 1.8060, 1.1629) -- (2.5780, 1.8570, 1.1608) -- (2.5290, 1.8570, 1.1665) -- cycle;
\fill[blue!15.0, opacity=0.7] (2.5290, 1.8570, 1.1665) -- (2.5780, 1.8570, 1.1608) -- (2.5780, 1.9080, 1.1584) -- (2.5290, 1.9080, 1.1641) -- cycle;
\fill[blue!15.0, opacity=0.7] (2.5290, 1.9080, 1.1641) -- (2.5780, 1.9080, 1.1584) -- (2.5780, 1.9590, 1.1557) -- (2.5290, 1.9590, 1.1614) -- cycle;
\fill[blue!15.0, opacity=0.7] (2.5290, 1.9590, 1.1614) -- (2.5780, 1.9590, 1.1557) -- (2.5780, 2.0100, 1.1527) -- (2.5290, 2.0100, 1.1584) -- cycle;
\fill[blue!15.1, opacity=0.7] (2.5290, 2.0100, 1.1584) -- (2.5780, 2.0100, 1.1527) -- (2.5780, 2.0610, 1.1494) -- (2.5290, 2.0610, 1.1551) -- cycle;
\fill[blue!16.5, opacity=0.7] (2.5290, 2.0610, 1.1551) -- (2.5780, 2.0610, 1.1494) -- (2.5780, 2.1120, 1.1459) -- (2.5290, 2.1120, 1.1516) -- cycle;
\fill[blue!22.6, opacity=0.7] (2.5290, 2.1120, 1.1516) -- (2.5780, 2.1120, 1.1459) -- (2.5780, 2.1630, 1.1421) -- (2.5290, 2.1630, 1.1477) -- cycle;
\fill[blue!30.4, opacity=0.7] (2.5290, 2.1630, 1.1477) -- (2.5780, 2.1630, 1.1421) -- (2.5780, 2.2140, 1.1380) -- (2.5290, 2.2140, 1.1437) -- cycle;
\fill[blue!27.8, opacity=0.7] (2.5290, 2.2140, 1.1437) -- (2.5780, 2.2140, 1.1380) -- (2.5780, 2.2650, 1.1337) -- (2.5290, 2.2650, 1.1393) -- cycle;
\fill[blue!18.3, opacity=0.7] (2.5290, 2.2650, 1.1393) -- (2.5780, 2.2650, 1.1337) -- (2.5780, 2.3160, 1.1291) -- (2.5290, 2.3160, 1.1348) -- cycle;
\fill[blue!15.2, opacity=0.7] (2.5290, 2.3160, 1.1348) -- (2.5780, 2.3160, 1.1291) -- (2.5780, 2.3670, 1.1243) -- (2.5290, 2.3670, 1.1300) -- cycle;
\fill[blue!15.0, opacity=0.7] (2.5290, 2.3670, 1.1300) -- (2.5780, 2.3670, 1.1243) -- (2.5780, 2.4180, 1.1193) -- (2.5290, 2.4180, 1.1250) -- cycle;
\fill[blue!15.0, opacity=0.7] (2.5290, 2.4180, 1.1250) -- (2.5780, 2.4180, 1.1193) -- (2.5780, 2.4690, 1.1142) -- (2.5290, 2.4690, 1.1198) -- cycle;
\fill[blue!15.0, opacity=0.7] (2.5290, 2.4690, 1.1198) -- (2.5780, 2.4690, 1.1142) -- (2.5780, 2.5200, 1.1088) -- (2.5290, 2.5200, 1.1145) -- cycle;
\fill[blue!15.0, opacity=0.7] (2.5290, 2.5200, 1.1145) -- (2.5780, 2.5200, 1.1088) -- (2.5780, 2.5710, 1.1033) -- (2.5290, 2.5710, 1.1090) -- cycle;
\fill[blue!15.0, opacity=0.7] (2.5290, 2.5710, 1.1090) -- (2.5780, 2.5710, 1.1033) -- (2.5780, 2.6220, 1.0976) -- (2.5290, 2.6220, 1.1033) -- cycle;
\fill[blue!15.0, opacity=0.7] (2.5290, 2.6220, 1.1033) -- (2.5780, 2.6220, 1.0976) -- (2.5780, 2.6730, 1.0918) -- (2.5290, 2.6730, 1.0975) -- cycle;
\fill[blue!15.0, opacity=0.7] (2.5290, 2.6730, 1.0975) -- (2.5780, 2.6730, 1.0918) -- (2.5780, 2.7240, 1.0859) -- (2.5290, 2.7240, 1.0916) -- cycle;
\fill[blue!15.1, opacity=0.7] (2.5290, 2.7240, 1.0916) -- (2.5780, 2.7240, 1.0859) -- (2.5780, 2.7750, 1.0799) -- (2.5290, 2.7750, 1.0855) -- cycle;
\fill[blue!15.2, opacity=0.7] (2.5290, 2.7750, 1.0855) -- (2.5780, 2.7750, 1.0799) -- (2.5780, 2.8260, 1.0738) -- (2.5290, 2.8260, 1.0794) -- cycle;
\fill[blue!15.0, opacity=0.7] (2.5290, 2.8260, 1.0794) -- (2.5780, 2.8260, 1.0738) -- (2.5780, 2.8770, 1.0676) -- (2.5290, 2.8770, 1.0733) -- cycle;
\fill[blue!15.0, opacity=0.7] (2.5290, 2.8770, 1.0733) -- (2.5780, 2.8770, 1.0676) -- (2.5780, 2.9280, 1.0614) -- (2.5290, 2.9280, 1.0670) -- cycle;
\fill[blue!15.0, opacity=0.7] (2.5290, 2.9280, 1.0670) -- (2.5780, 2.9280, 1.0614) -- (2.5780, 2.9790, 1.0551) -- (2.5290, 2.9790, 1.0608) -- cycle;
\fill[blue!15.0, opacity=0.7] (2.5290, 2.9790, 1.0608) -- (2.5780, 2.9790, 1.0551) -- (2.5780, 3.0300, 1.0488) -- (2.5290, 3.0300, 1.0545) -- cycle;
\fill[blue!15.0, opacity=0.7] (2.5780, -0.0300, 1.0488) -- (2.6270, -0.0300, 1.0430) -- (2.6270, 0.0210, 1.0493) -- (2.5780, 0.0210, 1.0551) -- cycle;
\fill[blue!15.0, opacity=0.7] (2.5780, 0.0210, 1.0551) -- (2.6270, 0.0210, 1.0493) -- (2.6270, 0.0720, 1.0555) -- (2.5780, 0.0720, 1.0614) -- cycle;
\fill[blue!15.0, opacity=0.7] (2.5780, 0.0720, 1.0614) -- (2.6270, 0.0720, 1.0555) -- (2.6270, 0.1230, 1.0618) -- (2.5780, 0.1230, 1.0676) -- cycle;
\fill[blue!15.0, opacity=0.7] (2.5780, 0.1230, 1.0676) -- (2.6270, 0.1230, 1.0618) -- (2.6270, 0.1740, 1.0680) -- (2.5780, 0.1740, 1.0738) -- cycle;
\fill[blue!15.0, opacity=0.7] (2.5780, 0.1740, 1.0738) -- (2.6270, 0.1740, 1.0680) -- (2.6270, 0.2250, 1.0741) -- (2.5780, 0.2250, 1.0799) -- cycle;
\fill[blue!15.0, opacity=0.7] (2.5780, 0.2250, 1.0799) -- (2.6270, 0.2250, 1.0741) -- (2.6270, 0.2760, 1.0801) -- (2.5780, 0.2760, 1.0859) -- cycle;
\fill[blue!15.0, opacity=0.7] (2.5780, 0.2760, 1.0859) -- (2.6270, 0.2760, 1.0801) -- (2.6270, 0.3270, 1.0860) -- (2.5780, 0.3270, 1.0918) -- cycle;
\fill[blue!15.0, opacity=0.7] (2.5780, 0.3270, 1.0918) -- (2.6270, 0.3270, 1.0860) -- (2.6270, 0.3780, 1.0918) -- (2.5780, 0.3780, 1.0976) -- cycle;
\fill[blue!15.0, opacity=0.7] (2.5780, 0.3780, 1.0976) -- (2.6270, 0.3780, 1.0918) -- (2.6270, 0.4290, 1.0975) -- (2.5780, 0.4290, 1.1033) -- cycle;
\fill[blue!15.1, opacity=0.7] (2.5780, 0.4290, 1.1033) -- (2.6270, 0.4290, 1.0975) -- (2.6270, 0.4800, 1.1030) -- (2.5780, 0.4800, 1.1088) -- cycle;
\fill[blue!16.0, opacity=0.7] (2.5780, 0.4800, 1.1088) -- (2.6270, 0.4800, 1.1030) -- (2.6270, 0.5310, 1.1084) -- (2.5780, 0.5310, 1.1142) -- cycle;
\fill[blue!17.9, opacity=0.7] (2.5780, 0.5310, 1.1142) -- (2.6270, 0.5310, 1.1084) -- (2.6270, 0.5820, 1.1135) -- (2.5780, 0.5820, 1.1193) -- cycle;
\fill[blue!17.9, opacity=0.7] (2.5780, 0.5820, 1.1193) -- (2.6270, 0.5820, 1.1135) -- (2.6270, 0.6330, 1.1185) -- (2.5780, 0.6330, 1.1243) -- cycle;
\fill[blue!16.2, opacity=0.7] (2.5780, 0.6330, 1.1243) -- (2.6270, 0.6330, 1.1185) -- (2.6270, 0.6840, 1.1233) -- (2.5780, 0.6840, 1.1291) -- cycle;
\fill[blue!15.2, opacity=0.7] (2.5780, 0.6840, 1.1291) -- (2.6270, 0.6840, 1.1233) -- (2.6270, 0.7350, 1.1279) -- (2.5780, 0.7350, 1.1337) -- cycle;
\fill[blue!15.0, opacity=0.7] (2.5780, 0.7350, 1.1337) -- (2.6270, 0.7350, 1.1279) -- (2.6270, 0.7860, 1.1322) -- (2.5780, 0.7860, 1.1380) -- cycle;
\fill[blue!15.0, opacity=0.7] (2.5780, 0.7860, 1.1380) -- (2.6270, 0.7860, 1.1322) -- (2.6270, 0.8370, 1.1363) -- (2.5780, 0.8370, 1.1421) -- cycle;
\fill[blue!15.0, opacity=0.7] (2.5780, 0.8370, 1.1421) -- (2.6270, 0.8370, 1.1363) -- (2.6270, 0.8880, 1.1401) -- (2.5780, 0.8880, 1.1459) -- cycle;
\fill[blue!15.0, opacity=0.7] (2.5780, 0.8880, 1.1459) -- (2.6270, 0.8880, 1.1401) -- (2.6270, 0.9390, 1.1436) -- (2.5780, 0.9390, 1.1494) -- cycle;
\fill[blue!15.0, opacity=0.7] (2.5780, 0.9390, 1.1494) -- (2.6270, 0.9390, 1.1436) -- (2.6270, 0.9900, 1.1469) -- (2.5780, 0.9900, 1.1527) -- cycle;
\fill[blue!15.0, opacity=0.7] (2.5780, 0.9900, 1.1527) -- (2.6270, 0.9900, 1.1469) -- (2.6270, 1.0410, 1.1499) -- (2.5780, 1.0410, 1.1557) -- cycle;
\fill[blue!15.0, opacity=0.7] (2.5780, 1.0410, 1.1557) -- (2.6270, 1.0410, 1.1499) -- (2.6270, 1.0920, 1.1526) -- (2.5780, 1.0920, 1.1584) -- cycle;
\fill[blue!15.0, opacity=0.7] (2.5780, 1.0920, 1.1584) -- (2.6270, 1.0920, 1.1526) -- (2.6270, 1.1430, 1.1550) -- (2.5780, 1.1430, 1.1608) -- cycle;
\fill[blue!15.0, opacity=0.7] (2.5780, 1.1430, 1.1608) -- (2.6270, 1.1430, 1.1550) -- (2.6270, 1.1940, 1.1571) -- (2.5780, 1.1940, 1.1629) -- cycle;
\fill[blue!15.0, opacity=0.7] (2.5780, 1.1940, 1.1629) -- (2.6270, 1.1940, 1.1571) -- (2.6270, 1.2450, 1.1589) -- (2.5780, 1.2450, 1.1647) -- cycle;
\fill[blue!15.0, opacity=0.7] (2.5780, 1.2450, 1.1647) -- (2.6270, 1.2450, 1.1589) -- (2.6270, 1.2960, 1.1604) -- (2.5780, 1.2960, 1.1662) -- cycle;
\fill[blue!15.0, opacity=0.7] (2.5780, 1.2960, 1.1662) -- (2.6270, 1.2960, 1.1604) -- (2.6270, 1.3470, 1.1615) -- (2.5780, 1.3470, 1.1673) -- cycle;
\fill[blue!15.0, opacity=0.7] (2.5780, 1.3470, 1.1673) -- (2.6270, 1.3470, 1.1615) -- (2.6270, 1.3980, 1.1623) -- (2.5780, 1.3980, 1.1682) -- cycle;
\fill[blue!15.0, opacity=0.7] (2.5780, 1.3980, 1.1682) -- (2.6270, 1.3980, 1.1623) -- (2.6270, 1.4490, 1.1628) -- (2.5780, 1.4490, 1.1686) -- cycle;
\fill[blue!15.0, opacity=0.7] (2.5780, 1.4490, 1.1686) -- (2.6270, 1.4490, 1.1628) -- (2.6270, 1.5000, 1.1630) -- (2.5780, 1.5000, 1.1688) -- cycle;
\fill[blue!15.0, opacity=0.7] (2.5780, 1.5000, 1.1688) -- (2.6270, 1.5000, 1.1630) -- (2.6270, 1.5510, 1.1628) -- (2.5780, 1.5510, 1.1686) -- cycle;
\fill[blue!15.0, opacity=0.7] (2.5780, 1.5510, 1.1686) -- (2.6270, 1.5510, 1.1628) -- (2.6270, 1.6020, 1.1623) -- (2.5780, 1.6020, 1.1682) -- cycle;
\fill[blue!15.0, opacity=0.7] (2.5780, 1.6020, 1.1682) -- (2.6270, 1.6020, 1.1623) -- (2.6270, 1.6530, 1.1615) -- (2.5780, 1.6530, 1.1673) -- cycle;
\fill[blue!15.0, opacity=0.7] (2.5780, 1.6530, 1.1673) -- (2.6270, 1.6530, 1.1615) -- (2.6270, 1.7040, 1.1604) -- (2.5780, 1.7040, 1.1662) -- cycle;
\fill[blue!15.0, opacity=0.7] (2.5780, 1.7040, 1.1662) -- (2.6270, 1.7040, 1.1604) -- (2.6270, 1.7550, 1.1589) -- (2.5780, 1.7550, 1.1647) -- cycle;
\fill[blue!15.0, opacity=0.7] (2.5780, 1.7550, 1.1647) -- (2.6270, 1.7550, 1.1589) -- (2.6270, 1.8060, 1.1571) -- (2.5780, 1.8060, 1.1629) -- cycle;
\fill[blue!15.0, opacity=0.7] (2.5780, 1.8060, 1.1629) -- (2.6270, 1.8060, 1.1571) -- (2.6270, 1.8570, 1.1550) -- (2.5780, 1.8570, 1.1608) -- cycle;
\fill[blue!15.0, opacity=0.7] (2.5780, 1.8570, 1.1608) -- (2.6270, 1.8570, 1.1550) -- (2.6270, 1.9080, 1.1526) -- (2.5780, 1.9080, 1.1584) -- cycle;
\fill[blue!15.1, opacity=0.7] (2.5780, 1.9080, 1.1584) -- (2.6270, 1.9080, 1.1526) -- (2.6270, 1.9590, 1.1499) -- (2.5780, 1.9590, 1.1557) -- cycle;
\fill[blue!15.6, opacity=0.7] (2.5780, 1.9590, 1.1557) -- (2.6270, 1.9590, 1.1499) -- (2.6270, 2.0100, 1.1469) -- (2.5780, 2.0100, 1.1527) -- cycle;
\fill[blue!18.5, opacity=0.7] (2.5780, 2.0100, 1.1527) -- (2.6270, 2.0100, 1.1469) -- (2.6270, 2.0610, 1.1436) -- (2.5780, 2.0610, 1.1494) -- cycle;
\fill[blue!25.5, opacity=0.7] (2.5780, 2.0610, 1.1494) -- (2.6270, 2.0610, 1.1436) -- (2.6270, 2.1120, 1.1401) -- (2.5780, 2.1120, 1.1459) -- cycle;
\fill[blue!29.9, opacity=0.7] (2.5780, 2.1120, 1.1459) -- (2.6270, 2.1120, 1.1401) -- (2.6270, 2.1630, 1.1363) -- (2.5780, 2.1630, 1.1421) -- cycle;
\fill[blue!24.6, opacity=0.7] (2.5780, 2.1630, 1.1421) -- (2.6270, 2.1630, 1.1363) -- (2.6270, 2.2140, 1.1322) -- (2.5780, 2.2140, 1.1380) -- cycle;
\fill[blue!17.0, opacity=0.7] (2.5780, 2.2140, 1.1380) -- (2.6270, 2.2140, 1.1322) -- (2.6270, 2.2650, 1.1279) -- (2.5780, 2.2650, 1.1337) -- cycle;
\fill[blue!15.1, opacity=0.7] (2.5780, 2.2650, 1.1337) -- (2.6270, 2.2650, 1.1279) -- (2.6270, 2.3160, 1.1233) -- (2.5780, 2.3160, 1.1291) -- cycle;
\fill[blue!15.0, opacity=0.7] (2.5780, 2.3160, 1.1291) -- (2.6270, 2.3160, 1.1233) -- (2.6270, 2.3670, 1.1185) -- (2.5780, 2.3670, 1.1243) -- cycle;
\fill[blue!15.0, opacity=0.7] (2.5780, 2.3670, 1.1243) -- (2.6270, 2.3670, 1.1185) -- (2.6270, 2.4180, 1.1135) -- (2.5780, 2.4180, 1.1193) -- cycle;
\fill[blue!15.0, opacity=0.7] (2.5780, 2.4180, 1.1193) -- (2.6270, 2.4180, 1.1135) -- (2.6270, 2.4690, 1.1084) -- (2.5780, 2.4690, 1.1142) -- cycle;
\fill[blue!15.0, opacity=0.7] (2.5780, 2.4690, 1.1142) -- (2.6270, 2.4690, 1.1084) -- (2.6270, 2.5200, 1.1030) -- (2.5780, 2.5200, 1.1088) -- cycle;
\fill[blue!15.0, opacity=0.7] (2.5780, 2.5200, 1.1088) -- (2.6270, 2.5200, 1.1030) -- (2.6270, 2.5710, 1.0975) -- (2.5780, 2.5710, 1.1033) -- cycle;
\fill[blue!15.0, opacity=0.7] (2.5780, 2.5710, 1.1033) -- (2.6270, 2.5710, 1.0975) -- (2.6270, 2.6220, 1.0918) -- (2.5780, 2.6220, 1.0976) -- cycle;
\fill[blue!15.0, opacity=0.7] (2.5780, 2.6220, 1.0976) -- (2.6270, 2.6220, 1.0918) -- (2.6270, 2.6730, 1.0860) -- (2.5780, 2.6730, 1.0918) -- cycle;
\fill[blue!15.1, opacity=0.7] (2.5780, 2.6730, 1.0918) -- (2.6270, 2.6730, 1.0860) -- (2.6270, 2.7240, 1.0801) -- (2.5780, 2.7240, 1.0859) -- cycle;
\fill[blue!15.2, opacity=0.7] (2.5780, 2.7240, 1.0859) -- (2.6270, 2.7240, 1.0801) -- (2.6270, 2.7750, 1.0741) -- (2.5780, 2.7750, 1.0799) -- cycle;
\fill[blue!15.0, opacity=0.7] (2.5780, 2.7750, 1.0799) -- (2.6270, 2.7750, 1.0741) -- (2.6270, 2.8260, 1.0680) -- (2.5780, 2.8260, 1.0738) -- cycle;
\fill[blue!15.0, opacity=0.7] (2.5780, 2.8260, 1.0738) -- (2.6270, 2.8260, 1.0680) -- (2.6270, 2.8770, 1.0618) -- (2.5780, 2.8770, 1.0676) -- cycle;
\fill[blue!15.0, opacity=0.7] (2.5780, 2.8770, 1.0676) -- (2.6270, 2.8770, 1.0618) -- (2.6270, 2.9280, 1.0555) -- (2.5780, 2.9280, 1.0614) -- cycle;
\fill[blue!15.0, opacity=0.7] (2.5780, 2.9280, 1.0614) -- (2.6270, 2.9280, 1.0555) -- (2.6270, 2.9790, 1.0493) -- (2.5780, 2.9790, 1.0551) -- cycle;
\fill[blue!15.0, opacity=0.7] (2.5780, 2.9790, 1.0551) -- (2.6270, 2.9790, 1.0493) -- (2.6270, 3.0300, 1.0430) -- (2.5780, 3.0300, 1.0488) -- cycle;
\fill[blue!15.0, opacity=0.7] (2.6270, -0.0300, 1.0430) -- (2.6760, -0.0300, 1.0371) -- (2.6760, 0.0210, 1.0434) -- (2.6270, 0.0210, 1.0493) -- cycle;
\fill[blue!15.0, opacity=0.7] (2.6270, 0.0210, 1.0493) -- (2.6760, 0.0210, 1.0434) -- (2.6760, 0.0720, 1.0496) -- (2.6270, 0.0720, 1.0555) -- cycle;
\fill[blue!15.0, opacity=0.7] (2.6270, 0.0720, 1.0555) -- (2.6760, 0.0720, 1.0496) -- (2.6760, 0.1230, 1.0559) -- (2.6270, 0.1230, 1.0618) -- cycle;
\fill[blue!15.0, opacity=0.7] (2.6270, 0.1230, 1.0618) -- (2.6760, 0.1230, 1.0559) -- (2.6760, 0.1740, 1.0620) -- (2.6270, 0.1740, 1.0680) -- cycle;
\fill[blue!15.0, opacity=0.7] (2.6270, 0.1740, 1.0680) -- (2.6760, 0.1740, 1.0620) -- (2.6760, 0.2250, 1.0681) -- (2.6270, 0.2250, 1.0741) -- cycle;
\fill[blue!15.0, opacity=0.7] (2.6270, 0.2250, 1.0741) -- (2.6760, 0.2250, 1.0681) -- (2.6760, 0.2760, 1.0742) -- (2.6270, 0.2760, 1.0801) -- cycle;
\fill[blue!15.0, opacity=0.7] (2.6270, 0.2760, 1.0801) -- (2.6760, 0.2760, 1.0742) -- (2.6760, 0.3270, 1.0801) -- (2.6270, 0.3270, 1.0860) -- cycle;
\fill[blue!15.0, opacity=0.7] (2.6270, 0.3270, 1.0860) -- (2.6760, 0.3270, 1.0801) -- (2.6760, 0.3780, 1.0859) -- (2.6270, 0.3780, 1.0918) -- cycle;
\fill[blue!15.0, opacity=0.7] (2.6270, 0.3780, 1.0918) -- (2.6760, 0.3780, 1.0859) -- (2.6760, 0.4290, 1.0916) -- (2.6270, 0.4290, 1.0975) -- cycle;
\fill[blue!15.0, opacity=0.7] (2.6270, 0.4290, 1.0975) -- (2.6760, 0.4290, 1.0916) -- (2.6760, 0.4800, 1.0971) -- (2.6270, 0.4800, 1.1030) -- cycle;
\fill[blue!15.0, opacity=0.7] (2.6270, 0.4800, 1.1030) -- (2.6760, 0.4800, 1.0971) -- (2.6760, 0.5310, 1.1024) -- (2.6270, 0.5310, 1.1084) -- cycle;
\fill[blue!15.5, opacity=0.7] (2.6270, 0.5310, 1.1084) -- (2.6760, 0.5310, 1.1024) -- (2.6760, 0.5820, 1.1076) -- (2.6270, 0.5820, 1.1135) -- cycle;
\fill[blue!17.2, opacity=0.7] (2.6270, 0.5820, 1.1135) -- (2.6760, 0.5820, 1.1076) -- (2.6760, 0.6330, 1.1126) -- (2.6270, 0.6330, 1.1185) -- cycle;
\fill[blue!18.6, opacity=0.7] (2.6270, 0.6330, 1.1185) -- (2.6760, 0.6330, 1.1126) -- (2.6760, 0.6840, 1.1174) -- (2.6270, 0.6840, 1.1233) -- cycle;
\fill[blue!17.8, opacity=0.7] (2.6270, 0.6840, 1.1233) -- (2.6760, 0.6840, 1.1174) -- (2.6760, 0.7350, 1.1219) -- (2.6270, 0.7350, 1.1279) -- cycle;
\fill[blue!16.1, opacity=0.7] (2.6270, 0.7350, 1.1279) -- (2.6760, 0.7350, 1.1219) -- (2.6760, 0.7860, 1.1263) -- (2.6270, 0.7860, 1.1322) -- cycle;
\fill[blue!15.3, opacity=0.7] (2.6270, 0.7860, 1.1322) -- (2.6760, 0.7860, 1.1263) -- (2.6760, 0.8370, 1.1303) -- (2.6270, 0.8370, 1.1363) -- cycle;
\fill[blue!15.0, opacity=0.7] (2.6270, 0.8370, 1.1363) -- (2.6760, 0.8370, 1.1303) -- (2.6760, 0.8880, 1.1342) -- (2.6270, 0.8880, 1.1401) -- cycle;
\fill[blue!15.0, opacity=0.7] (2.6270, 0.8880, 1.1401) -- (2.6760, 0.8880, 1.1342) -- (2.6760, 0.9390, 1.1377) -- (2.6270, 0.9390, 1.1436) -- cycle;
\fill[blue!15.0, opacity=0.7] (2.6270, 0.9390, 1.1436) -- (2.6760, 0.9390, 1.1377) -- (2.6760, 0.9900, 1.1410) -- (2.6270, 0.9900, 1.1469) -- cycle;
\fill[blue!15.0, opacity=0.7] (2.6270, 0.9900, 1.1469) -- (2.6760, 0.9900, 1.1410) -- (2.6760, 1.0410, 1.1440) -- (2.6270, 1.0410, 1.1499) -- cycle;
\fill[blue!15.0, opacity=0.7] (2.6270, 1.0410, 1.1499) -- (2.6760, 1.0410, 1.1440) -- (2.6760, 1.0920, 1.1467) -- (2.6270, 1.0920, 1.1526) -- cycle;
\fill[blue!15.0, opacity=0.7] (2.6270, 1.0920, 1.1526) -- (2.6760, 1.0920, 1.1467) -- (2.6760, 1.1430, 1.1491) -- (2.6270, 1.1430, 1.1550) -- cycle;
\fill[blue!15.0, opacity=0.7] (2.6270, 1.1430, 1.1550) -- (2.6760, 1.1430, 1.1491) -- (2.6760, 1.1940, 1.1512) -- (2.6270, 1.1940, 1.1571) -- cycle;
\fill[blue!15.0, opacity=0.7] (2.6270, 1.1940, 1.1571) -- (2.6760, 1.1940, 1.1512) -- (2.6760, 1.2450, 1.1530) -- (2.6270, 1.2450, 1.1589) -- cycle;
\fill[blue!15.0, opacity=0.7] (2.6270, 1.2450, 1.1589) -- (2.6760, 1.2450, 1.1530) -- (2.6760, 1.2960, 1.1545) -- (2.6270, 1.2960, 1.1604) -- cycle;
\fill[blue!15.0, opacity=0.7] (2.6270, 1.2960, 1.1604) -- (2.6760, 1.2960, 1.1545) -- (2.6760, 1.3470, 1.1556) -- (2.6270, 1.3470, 1.1615) -- cycle;
\fill[blue!15.0, opacity=0.7] (2.6270, 1.3470, 1.1615) -- (2.6760, 1.3470, 1.1556) -- (2.6760, 1.3980, 1.1564) -- (2.6270, 1.3980, 1.1623) -- cycle;
\fill[blue!15.0, opacity=0.7] (2.6270, 1.3980, 1.1623) -- (2.6760, 1.3980, 1.1564) -- (2.6760, 1.4490, 1.1569) -- (2.6270, 1.4490, 1.1628) -- cycle;
\fill[blue!15.0, opacity=0.7] (2.6270, 1.4490, 1.1628) -- (2.6760, 1.4490, 1.1569) -- (2.6760, 1.5000, 1.1571) -- (2.6270, 1.5000, 1.1630) -- cycle;
\fill[blue!15.0, opacity=0.7] (2.6270, 1.5000, 1.1630) -- (2.6760, 1.5000, 1.1571) -- (2.6760, 1.5510, 1.1569) -- (2.6270, 1.5510, 1.1628) -- cycle;
\fill[blue!15.0, opacity=0.7] (2.6270, 1.5510, 1.1628) -- (2.6760, 1.5510, 1.1569) -- (2.6760, 1.6020, 1.1564) -- (2.6270, 1.6020, 1.1623) -- cycle;
\fill[blue!15.0, opacity=0.7] (2.6270, 1.6020, 1.1623) -- (2.6760, 1.6020, 1.1564) -- (2.6760, 1.6530, 1.1556) -- (2.6270, 1.6530, 1.1615) -- cycle;
\fill[blue!15.0, opacity=0.7] (2.6270, 1.6530, 1.1615) -- (2.6760, 1.6530, 1.1556) -- (2.6760, 1.7040, 1.1545) -- (2.6270, 1.7040, 1.1604) -- cycle;
\fill[blue!15.0, opacity=0.7] (2.6270, 1.7040, 1.1604) -- (2.6760, 1.7040, 1.1545) -- (2.6760, 1.7550, 1.1530) -- (2.6270, 1.7550, 1.1589) -- cycle;
\fill[blue!15.0, opacity=0.7] (2.6270, 1.7550, 1.1589) -- (2.6760, 1.7550, 1.1530) -- (2.6760, 1.8060, 1.1512) -- (2.6270, 1.8060, 1.1571) -- cycle;
\fill[blue!15.1, opacity=0.7] (2.6270, 1.8060, 1.1571) -- (2.6760, 1.8060, 1.1512) -- (2.6760, 1.8570, 1.1491) -- (2.6270, 1.8570, 1.1550) -- cycle;
\fill[blue!15.5, opacity=0.7] (2.6270, 1.8570, 1.1550) -- (2.6760, 1.8570, 1.1491) -- (2.6760, 1.9080, 1.1467) -- (2.6270, 1.9080, 1.1526) -- cycle;
\fill[blue!17.5, opacity=0.7] (2.6270, 1.9080, 1.1526) -- (2.6760, 1.9080, 1.1467) -- (2.6760, 1.9590, 1.1440) -- (2.6270, 1.9590, 1.1499) -- cycle;
\fill[blue!22.5, opacity=0.7] (2.6270, 1.9590, 1.1499) -- (2.6760, 1.9590, 1.1440) -- (2.6760, 2.0100, 1.1410) -- (2.6270, 2.0100, 1.1469) -- cycle;
\fill[blue!28.0, opacity=0.7] (2.6270, 2.0100, 1.1469) -- (2.6760, 2.0100, 1.1410) -- (2.6760, 2.0610, 1.1377) -- (2.6270, 2.0610, 1.1436) -- cycle;
\fill[blue!27.4, opacity=0.7] (2.6270, 2.0610, 1.1436) -- (2.6760, 2.0610, 1.1377) -- (2.6760, 2.1120, 1.1342) -- (2.6270, 2.1120, 1.1401) -- cycle;
\fill[blue!20.6, opacity=0.7] (2.6270, 2.1120, 1.1401) -- (2.6760, 2.1120, 1.1342) -- (2.6760, 2.1630, 1.1303) -- (2.6270, 2.1630, 1.1363) -- cycle;
\fill[blue!15.8, opacity=0.7] (2.6270, 2.1630, 1.1363) -- (2.6760, 2.1630, 1.1303) -- (2.6760, 2.2140, 1.1263) -- (2.6270, 2.2140, 1.1322) -- cycle;
\fill[blue!15.0, opacity=0.7] (2.6270, 2.2140, 1.1322) -- (2.6760, 2.2140, 1.1263) -- (2.6760, 2.2650, 1.1219) -- (2.6270, 2.2650, 1.1279) -- cycle;
\fill[blue!15.0, opacity=0.7] (2.6270, 2.2650, 1.1279) -- (2.6760, 2.2650, 1.1219) -- (2.6760, 2.3160, 1.1174) -- (2.6270, 2.3160, 1.1233) -- cycle;
\fill[blue!15.0, opacity=0.7] (2.6270, 2.3160, 1.1233) -- (2.6760, 2.3160, 1.1174) -- (2.6760, 2.3670, 1.1126) -- (2.6270, 2.3670, 1.1185) -- cycle;
\fill[blue!15.0, opacity=0.7] (2.6270, 2.3670, 1.1185) -- (2.6760, 2.3670, 1.1126) -- (2.6760, 2.4180, 1.1076) -- (2.6270, 2.4180, 1.1135) -- cycle;
\fill[blue!15.0, opacity=0.7] (2.6270, 2.4180, 1.1135) -- (2.6760, 2.4180, 1.1076) -- (2.6760, 2.4690, 1.1024) -- (2.6270, 2.4690, 1.1084) -- cycle;
\fill[blue!15.0, opacity=0.7] (2.6270, 2.4690, 1.1084) -- (2.6760, 2.4690, 1.1024) -- (2.6760, 2.5200, 1.0971) -- (2.6270, 2.5200, 1.1030) -- cycle;
\fill[blue!15.0, opacity=0.7] (2.6270, 2.5200, 1.1030) -- (2.6760, 2.5200, 1.0971) -- (2.6760, 2.5710, 1.0916) -- (2.6270, 2.5710, 1.0975) -- cycle;
\fill[blue!15.0, opacity=0.7] (2.6270, 2.5710, 1.0975) -- (2.6760, 2.5710, 1.0916) -- (2.6760, 2.6220, 1.0859) -- (2.6270, 2.6220, 1.0918) -- cycle;
\fill[blue!15.0, opacity=0.7] (2.6270, 2.6220, 1.0918) -- (2.6760, 2.6220, 1.0859) -- (2.6760, 2.6730, 1.0801) -- (2.6270, 2.6730, 1.0860) -- cycle;
\fill[blue!15.2, opacity=0.7] (2.6270, 2.6730, 1.0860) -- (2.6760, 2.6730, 1.0801) -- (2.6760, 2.7240, 1.0742) -- (2.6270, 2.7240, 1.0801) -- cycle;
\fill[blue!15.1, opacity=0.7] (2.6270, 2.7240, 1.0801) -- (2.6760, 2.7240, 1.0742) -- (2.6760, 2.7750, 1.0681) -- (2.6270, 2.7750, 1.0741) -- cycle;
\fill[blue!15.0, opacity=0.7] (2.6270, 2.7750, 1.0741) -- (2.6760, 2.7750, 1.0681) -- (2.6760, 2.8260, 1.0620) -- (2.6270, 2.8260, 1.0680) -- cycle;
\fill[blue!15.0, opacity=0.7] (2.6270, 2.8260, 1.0680) -- (2.6760, 2.8260, 1.0620) -- (2.6760, 2.8770, 1.0559) -- (2.6270, 2.8770, 1.0618) -- cycle;
\fill[blue!15.0, opacity=0.7] (2.6270, 2.8770, 1.0618) -- (2.6760, 2.8770, 1.0559) -- (2.6760, 2.9280, 1.0496) -- (2.6270, 2.9280, 1.0555) -- cycle;
\fill[blue!15.0, opacity=0.7] (2.6270, 2.9280, 1.0555) -- (2.6760, 2.9280, 1.0496) -- (2.6760, 2.9790, 1.0434) -- (2.6270, 2.9790, 1.0493) -- cycle;
\fill[blue!15.0, opacity=0.7] (2.6270, 2.9790, 1.0493) -- (2.6760, 2.9790, 1.0434) -- (2.6760, 3.0300, 1.0371) -- (2.6270, 3.0300, 1.0430) -- cycle;
\fill[blue!15.0, opacity=0.7] (2.6760, -0.0300, 1.0371) -- (2.7250, -0.0300, 1.0311) -- (2.7250, 0.0210, 1.0373) -- (2.6760, 0.0210, 1.0434) -- cycle;
\fill[blue!15.0, opacity=0.7] (2.6760, 0.0210, 1.0434) -- (2.7250, 0.0210, 1.0373) -- (2.7250, 0.0720, 1.0436) -- (2.6760, 0.0720, 1.0496) -- cycle;
\fill[blue!15.0, opacity=0.7] (2.6760, 0.0720, 1.0496) -- (2.7250, 0.0720, 1.0436) -- (2.7250, 0.1230, 1.0498) -- (2.6760, 0.1230, 1.0559) -- cycle;
\fill[blue!15.0, opacity=0.7] (2.6760, 0.1230, 1.0559) -- (2.7250, 0.1230, 1.0498) -- (2.7250, 0.1740, 1.0560) -- (2.6760, 0.1740, 1.0620) -- cycle;
\fill[blue!15.0, opacity=0.7] (2.6760, 0.1740, 1.0620) -- (2.7250, 0.1740, 1.0560) -- (2.7250, 0.2250, 1.0621) -- (2.6760, 0.2250, 1.0681) -- cycle;
\fill[blue!15.0, opacity=0.7] (2.6760, 0.2250, 1.0681) -- (2.7250, 0.2250, 1.0621) -- (2.7250, 0.2760, 1.0681) -- (2.6760, 0.2760, 1.0742) -- cycle;
\fill[blue!15.0, opacity=0.7] (2.6760, 0.2760, 1.0742) -- (2.7250, 0.2760, 1.0681) -- (2.7250, 0.3270, 1.0741) -- (2.6760, 0.3270, 1.0801) -- cycle;
\fill[blue!15.0, opacity=0.7] (2.6760, 0.3270, 1.0801) -- (2.7250, 0.3270, 1.0741) -- (2.7250, 0.3780, 1.0799) -- (2.6760, 0.3780, 1.0859) -- cycle;
\fill[blue!15.0, opacity=0.7] (2.6760, 0.3780, 1.0859) -- (2.7250, 0.3780, 1.0799) -- (2.7250, 0.4290, 1.0855) -- (2.6760, 0.4290, 1.0916) -- cycle;
\fill[blue!15.0, opacity=0.7] (2.6760, 0.4290, 1.0916) -- (2.7250, 0.4290, 1.0855) -- (2.7250, 0.4800, 1.0911) -- (2.6760, 0.4800, 1.0971) -- cycle;
\fill[blue!15.0, opacity=0.7] (2.6760, 0.4800, 1.0971) -- (2.7250, 0.4800, 1.0911) -- (2.7250, 0.5310, 1.0964) -- (2.6760, 0.5310, 1.1024) -- cycle;
\fill[blue!15.0, opacity=0.7] (2.6760, 0.5310, 1.1024) -- (2.7250, 0.5310, 1.0964) -- (2.7250, 0.5820, 1.1016) -- (2.6760, 0.5820, 1.1076) -- cycle;
\fill[blue!15.1, opacity=0.7] (2.6760, 0.5820, 1.1076) -- (2.7250, 0.5820, 1.1016) -- (2.7250, 0.6330, 1.1066) -- (2.6760, 0.6330, 1.1126) -- cycle;
\fill[blue!16.0, opacity=0.7] (2.6760, 0.6330, 1.1126) -- (2.7250, 0.6330, 1.1066) -- (2.7250, 0.6840, 1.1114) -- (2.6760, 0.6840, 1.1174) -- cycle;
\fill[blue!17.9, opacity=0.7] (2.6760, 0.6840, 1.1174) -- (2.7250, 0.6840, 1.1114) -- (2.7250, 0.7350, 1.1159) -- (2.6760, 0.7350, 1.1219) -- cycle;
\fill[blue!19.0, opacity=0.7] (2.6760, 0.7350, 1.1219) -- (2.7250, 0.7350, 1.1159) -- (2.7250, 0.7860, 1.1202) -- (2.6760, 0.7860, 1.1263) -- cycle;
\fill[blue!18.2, opacity=0.7] (2.6760, 0.7860, 1.1263) -- (2.7250, 0.7860, 1.1202) -- (2.7250, 0.8370, 1.1243) -- (2.6760, 0.8370, 1.1303) -- cycle;
\fill[blue!16.7, opacity=0.7] (2.6760, 0.8370, 1.1303) -- (2.7250, 0.8370, 1.1243) -- (2.7250, 0.8880, 1.1281) -- (2.6760, 0.8880, 1.1342) -- cycle;
\fill[blue!15.6, opacity=0.7] (2.6760, 0.8880, 1.1342) -- (2.7250, 0.8880, 1.1281) -- (2.7250, 0.9390, 1.1317) -- (2.6760, 0.9390, 1.1377) -- cycle;
\fill[blue!15.2, opacity=0.7] (2.6760, 0.9390, 1.1377) -- (2.7250, 0.9390, 1.1317) -- (2.7250, 0.9900, 1.1350) -- (2.6760, 0.9900, 1.1410) -- cycle;
\fill[blue!15.0, opacity=0.7] (2.6760, 0.9900, 1.1410) -- (2.7250, 0.9900, 1.1350) -- (2.7250, 1.0410, 1.1380) -- (2.6760, 1.0410, 1.1440) -- cycle;
\fill[blue!15.0, opacity=0.7] (2.6760, 1.0410, 1.1440) -- (2.7250, 1.0410, 1.1380) -- (2.7250, 1.0920, 1.1407) -- (2.6760, 1.0920, 1.1467) -- cycle;
\fill[blue!15.0, opacity=0.7] (2.6760, 1.0920, 1.1467) -- (2.7250, 1.0920, 1.1407) -- (2.7250, 1.1430, 1.1431) -- (2.6760, 1.1430, 1.1491) -- cycle;
\fill[blue!15.0, opacity=0.7] (2.6760, 1.1430, 1.1491) -- (2.7250, 1.1430, 1.1431) -- (2.7250, 1.1940, 1.1452) -- (2.6760, 1.1940, 1.1512) -- cycle;
\fill[blue!15.0, opacity=0.7] (2.6760, 1.1940, 1.1512) -- (2.7250, 1.1940, 1.1452) -- (2.7250, 1.2450, 1.1470) -- (2.6760, 1.2450, 1.1530) -- cycle;
\fill[blue!15.0, opacity=0.7] (2.6760, 1.2450, 1.1530) -- (2.7250, 1.2450, 1.1470) -- (2.7250, 1.2960, 1.1484) -- (2.6760, 1.2960, 1.1545) -- cycle;
\fill[blue!15.0, opacity=0.7] (2.6760, 1.2960, 1.1545) -- (2.7250, 1.2960, 1.1484) -- (2.7250, 1.3470, 1.1496) -- (2.6760, 1.3470, 1.1556) -- cycle;
\fill[blue!15.0, opacity=0.7] (2.6760, 1.3470, 1.1556) -- (2.7250, 1.3470, 1.1496) -- (2.7250, 1.3980, 1.1504) -- (2.6760, 1.3980, 1.1564) -- cycle;
\fill[blue!15.0, opacity=0.7] (2.6760, 1.3980, 1.1564) -- (2.7250, 1.3980, 1.1504) -- (2.7250, 1.4490, 1.1509) -- (2.6760, 1.4490, 1.1569) -- cycle;
\fill[blue!15.0, opacity=0.7] (2.6760, 1.4490, 1.1569) -- (2.7250, 1.4490, 1.1509) -- (2.7250, 1.5000, 1.1511) -- (2.6760, 1.5000, 1.1571) -- cycle;
\fill[blue!15.0, opacity=0.7] (2.6760, 1.5000, 1.1571) -- (2.7250, 1.5000, 1.1511) -- (2.7250, 1.5510, 1.1509) -- (2.6760, 1.5510, 1.1569) -- cycle;
\fill[blue!15.0, opacity=0.7] (2.6760, 1.5510, 1.1569) -- (2.7250, 1.5510, 1.1509) -- (2.7250, 1.6020, 1.1504) -- (2.6760, 1.6020, 1.1564) -- cycle;
\fill[blue!15.0, opacity=0.7] (2.6760, 1.6020, 1.1564) -- (2.7250, 1.6020, 1.1504) -- (2.7250, 1.6530, 1.1496) -- (2.6760, 1.6530, 1.1556) -- cycle;
\fill[blue!15.1, opacity=0.7] (2.6760, 1.6530, 1.1556) -- (2.7250, 1.6530, 1.1496) -- (2.7250, 1.7040, 1.1484) -- (2.6760, 1.7040, 1.1545) -- cycle;
\fill[blue!15.2, opacity=0.7] (2.6760, 1.7040, 1.1545) -- (2.7250, 1.7040, 1.1484) -- (2.7250, 1.7550, 1.1470) -- (2.6760, 1.7550, 1.1530) -- cycle;
\fill[blue!16.0, opacity=0.7] (2.6760, 1.7550, 1.1530) -- (2.7250, 1.7550, 1.1470) -- (2.7250, 1.8060, 1.1452) -- (2.6760, 1.8060, 1.1512) -- cycle;
\fill[blue!18.1, opacity=0.7] (2.6760, 1.8060, 1.1512) -- (2.7250, 1.8060, 1.1452) -- (2.7250, 1.8570, 1.1431) -- (2.6760, 1.8570, 1.1491) -- cycle;
\fill[blue!22.2, opacity=0.7] (2.6760, 1.8570, 1.1491) -- (2.7250, 1.8570, 1.1431) -- (2.7250, 1.9080, 1.1407) -- (2.6760, 1.9080, 1.1467) -- cycle;
\fill[blue!26.6, opacity=0.7] (2.6760, 1.9080, 1.1467) -- (2.7250, 1.9080, 1.1407) -- (2.7250, 1.9590, 1.1380) -- (2.6760, 1.9590, 1.1440) -- cycle;
\fill[blue!27.1, opacity=0.7] (2.6760, 1.9590, 1.1440) -- (2.7250, 1.9590, 1.1380) -- (2.7250, 2.0100, 1.1350) -- (2.6760, 2.0100, 1.1410) -- cycle;
\fill[blue!22.4, opacity=0.7] (2.6760, 2.0100, 1.1410) -- (2.7250, 2.0100, 1.1350) -- (2.7250, 2.0610, 1.1317) -- (2.6760, 2.0610, 1.1377) -- cycle;
\fill[blue!17.0, opacity=0.7] (2.6760, 2.0610, 1.1377) -- (2.7250, 2.0610, 1.1317) -- (2.7250, 2.1120, 1.1281) -- (2.6760, 2.1120, 1.1342) -- cycle;
\fill[blue!15.2, opacity=0.7] (2.6760, 2.1120, 1.1342) -- (2.7250, 2.1120, 1.1281) -- (2.7250, 2.1630, 1.1243) -- (2.6760, 2.1630, 1.1303) -- cycle;
\fill[blue!15.0, opacity=0.7] (2.6760, 2.1630, 1.1303) -- (2.7250, 2.1630, 1.1243) -- (2.7250, 2.2140, 1.1202) -- (2.6760, 2.2140, 1.1263) -- cycle;
\fill[blue!15.0, opacity=0.7] (2.6760, 2.2140, 1.1263) -- (2.7250, 2.2140, 1.1202) -- (2.7250, 2.2650, 1.1159) -- (2.6760, 2.2650, 1.1219) -- cycle;
\fill[blue!15.0, opacity=0.7] (2.6760, 2.2650, 1.1219) -- (2.7250, 2.2650, 1.1159) -- (2.7250, 2.3160, 1.1114) -- (2.6760, 2.3160, 1.1174) -- cycle;
\fill[blue!15.0, opacity=0.7] (2.6760, 2.3160, 1.1174) -- (2.7250, 2.3160, 1.1114) -- (2.7250, 2.3670, 1.1066) -- (2.6760, 2.3670, 1.1126) -- cycle;
\fill[blue!15.0, opacity=0.7] (2.6760, 2.3670, 1.1126) -- (2.7250, 2.3670, 1.1066) -- (2.7250, 2.4180, 1.1016) -- (2.6760, 2.4180, 1.1076) -- cycle;
\fill[blue!15.0, opacity=0.7] (2.6760, 2.4180, 1.1076) -- (2.7250, 2.4180, 1.1016) -- (2.7250, 2.4690, 1.0964) -- (2.6760, 2.4690, 1.1024) -- cycle;
\fill[blue!15.0, opacity=0.7] (2.6760, 2.4690, 1.1024) -- (2.7250, 2.4690, 1.0964) -- (2.7250, 2.5200, 1.0911) -- (2.6760, 2.5200, 1.0971) -- cycle;
\fill[blue!15.0, opacity=0.7] (2.6760, 2.5200, 1.0971) -- (2.7250, 2.5200, 1.0911) -- (2.7250, 2.5710, 1.0855) -- (2.6760, 2.5710, 1.0916) -- cycle;
\fill[blue!15.0, opacity=0.7] (2.6760, 2.5710, 1.0916) -- (2.7250, 2.5710, 1.0855) -- (2.7250, 2.6220, 1.0799) -- (2.6760, 2.6220, 1.0859) -- cycle;
\fill[blue!15.2, opacity=0.7] (2.6760, 2.6220, 1.0859) -- (2.7250, 2.6220, 1.0799) -- (2.7250, 2.6730, 1.0741) -- (2.6760, 2.6730, 1.0801) -- cycle;
\fill[blue!15.1, opacity=0.7] (2.6760, 2.6730, 1.0801) -- (2.7250, 2.6730, 1.0741) -- (2.7250, 2.7240, 1.0681) -- (2.6760, 2.7240, 1.0742) -- cycle;
\fill[blue!15.0, opacity=0.7] (2.6760, 2.7240, 1.0742) -- (2.7250, 2.7240, 1.0681) -- (2.7250, 2.7750, 1.0621) -- (2.6760, 2.7750, 1.0681) -- cycle;
\fill[blue!15.0, opacity=0.7] (2.6760, 2.7750, 1.0681) -- (2.7250, 2.7750, 1.0621) -- (2.7250, 2.8260, 1.0560) -- (2.6760, 2.8260, 1.0620) -- cycle;
\fill[blue!15.0, opacity=0.7] (2.6760, 2.8260, 1.0620) -- (2.7250, 2.8260, 1.0560) -- (2.7250, 2.8770, 1.0498) -- (2.6760, 2.8770, 1.0559) -- cycle;
\fill[blue!15.0, opacity=0.7] (2.6760, 2.8770, 1.0559) -- (2.7250, 2.8770, 1.0498) -- (2.7250, 2.9280, 1.0436) -- (2.6760, 2.9280, 1.0496) -- cycle;
\fill[blue!15.0, opacity=0.7] (2.6760, 2.9280, 1.0496) -- (2.7250, 2.9280, 1.0436) -- (2.7250, 2.9790, 1.0373) -- (2.6760, 2.9790, 1.0434) -- cycle;
\fill[blue!15.0, opacity=0.7] (2.6760, 2.9790, 1.0434) -- (2.7250, 2.9790, 1.0373) -- (2.7250, 3.0300, 1.0311) -- (2.6760, 3.0300, 1.0371) -- cycle;
\fill[blue!15.0, opacity=0.7] (2.7250, -0.0300, 1.0311) -- (2.7740, -0.0300, 1.0249) -- (2.7740, 0.0210, 1.0312) -- (2.7250, 0.0210, 1.0373) -- cycle;
\fill[blue!15.0, opacity=0.7] (2.7250, 0.0210, 1.0373) -- (2.7740, 0.0210, 1.0312) -- (2.7740, 0.0720, 1.0375) -- (2.7250, 0.0720, 1.0436) -- cycle;
\fill[blue!15.0, opacity=0.7] (2.7250, 0.0720, 1.0436) -- (2.7740, 0.0720, 1.0375) -- (2.7740, 0.1230, 1.0437) -- (2.7250, 0.1230, 1.0498) -- cycle;
\fill[blue!15.0, opacity=0.7] (2.7250, 0.1230, 1.0498) -- (2.7740, 0.1230, 1.0437) -- (2.7740, 0.1740, 1.0499) -- (2.7250, 0.1740, 1.0560) -- cycle;
\fill[blue!15.0, opacity=0.7] (2.7250, 0.1740, 1.0560) -- (2.7740, 0.1740, 1.0499) -- (2.7740, 0.2250, 1.0560) -- (2.7250, 0.2250, 1.0621) -- cycle;
\fill[blue!15.0, opacity=0.7] (2.7250, 0.2250, 1.0621) -- (2.7740, 0.2250, 1.0560) -- (2.7740, 0.2760, 1.0620) -- (2.7250, 0.2760, 1.0681) -- cycle;
\fill[blue!15.0, opacity=0.7] (2.7250, 0.2760, 1.0681) -- (2.7740, 0.2760, 1.0620) -- (2.7740, 0.3270, 1.0680) -- (2.7250, 0.3270, 1.0741) -- cycle;
\fill[blue!15.0, opacity=0.7] (2.7250, 0.3270, 1.0741) -- (2.7740, 0.3270, 1.0680) -- (2.7740, 0.3780, 1.0738) -- (2.7250, 0.3780, 1.0799) -- cycle;
\fill[blue!15.0, opacity=0.7] (2.7250, 0.3780, 1.0799) -- (2.7740, 0.3780, 1.0738) -- (2.7740, 0.4290, 1.0794) -- (2.7250, 0.4290, 1.0855) -- cycle;
\fill[blue!15.0, opacity=0.7] (2.7250, 0.4290, 1.0855) -- (2.7740, 0.4290, 1.0794) -- (2.7740, 0.4800, 1.0849) -- (2.7250, 0.4800, 1.0911) -- cycle;
\fill[blue!15.0, opacity=0.7] (2.7250, 0.4800, 1.0911) -- (2.7740, 0.4800, 1.0849) -- (2.7740, 0.5310, 1.0903) -- (2.7250, 0.5310, 1.0964) -- cycle;
\fill[blue!15.0, opacity=0.7] (2.7250, 0.5310, 1.0964) -- (2.7740, 0.5310, 1.0903) -- (2.7740, 0.5820, 1.0955) -- (2.7250, 0.5820, 1.1016) -- cycle;
\fill[blue!15.0, opacity=0.7] (2.7250, 0.5820, 1.1016) -- (2.7740, 0.5820, 1.0955) -- (2.7740, 0.6330, 1.1005) -- (2.7250, 0.6330, 1.1066) -- cycle;
\fill[blue!15.0, opacity=0.7] (2.7250, 0.6330, 1.1066) -- (2.7740, 0.6330, 1.1005) -- (2.7740, 0.6840, 1.1052) -- (2.7250, 0.6840, 1.1114) -- cycle;
\fill[blue!15.2, opacity=0.7] (2.7250, 0.6840, 1.1114) -- (2.7740, 0.6840, 1.1052) -- (2.7740, 0.7350, 1.1098) -- (2.7250, 0.7350, 1.1159) -- cycle;
\fill[blue!16.1, opacity=0.7] (2.7250, 0.7350, 1.1159) -- (2.7740, 0.7350, 1.1098) -- (2.7740, 0.7860, 1.1141) -- (2.7250, 0.7860, 1.1202) -- cycle;
\fill[blue!17.9, opacity=0.7] (2.7250, 0.7860, 1.1202) -- (2.7740, 0.7860, 1.1141) -- (2.7740, 0.8370, 1.1182) -- (2.7250, 0.8370, 1.1243) -- cycle;
\fill[blue!19.4, opacity=0.7] (2.7250, 0.8370, 1.1243) -- (2.7740, 0.8370, 1.1182) -- (2.7740, 0.8880, 1.1220) -- (2.7250, 0.8880, 1.1281) -- cycle;
\fill[blue!19.4, opacity=0.7] (2.7250, 0.8880, 1.1281) -- (2.7740, 0.8880, 1.1220) -- (2.7740, 0.9390, 1.1256) -- (2.7250, 0.9390, 1.1317) -- cycle;
\fill[blue!18.4, opacity=0.7] (2.7250, 0.9390, 1.1317) -- (2.7740, 0.9390, 1.1256) -- (2.7740, 0.9900, 1.1289) -- (2.7250, 0.9900, 1.1350) -- cycle;
\fill[blue!17.1, opacity=0.7] (2.7250, 0.9900, 1.1350) -- (2.7740, 0.9900, 1.1289) -- (2.7740, 1.0410, 1.1319) -- (2.7250, 1.0410, 1.1380) -- cycle;
\fill[blue!16.2, opacity=0.7] (2.7250, 1.0410, 1.1380) -- (2.7740, 1.0410, 1.1319) -- (2.7740, 1.0920, 1.1346) -- (2.7250, 1.0920, 1.1407) -- cycle;
\fill[blue!15.6, opacity=0.7] (2.7250, 1.0920, 1.1407) -- (2.7740, 1.0920, 1.1346) -- (2.7740, 1.1430, 1.1370) -- (2.7250, 1.1430, 1.1431) -- cycle;
\fill[blue!15.3, opacity=0.7] (2.7250, 1.1430, 1.1431) -- (2.7740, 1.1430, 1.1370) -- (2.7740, 1.1940, 1.1391) -- (2.7250, 1.1940, 1.1452) -- cycle;
\fill[blue!15.2, opacity=0.7] (2.7250, 1.1940, 1.1452) -- (2.7740, 1.1940, 1.1391) -- (2.7740, 1.2450, 1.1409) -- (2.7250, 1.2450, 1.1470) -- cycle;
\fill[blue!15.1, opacity=0.7] (2.7250, 1.2450, 1.1470) -- (2.7740, 1.2450, 1.1409) -- (2.7740, 1.2960, 1.1423) -- (2.7250, 1.2960, 1.1484) -- cycle;
\fill[blue!15.1, opacity=0.7] (2.7250, 1.2960, 1.1484) -- (2.7740, 1.2960, 1.1423) -- (2.7740, 1.3470, 1.1435) -- (2.7250, 1.3470, 1.1496) -- cycle;
\fill[blue!15.1, opacity=0.7] (2.7250, 1.3470, 1.1496) -- (2.7740, 1.3470, 1.1435) -- (2.7740, 1.3980, 1.1443) -- (2.7250, 1.3980, 1.1504) -- cycle;
\fill[blue!15.1, opacity=0.7] (2.7250, 1.3980, 1.1504) -- (2.7740, 1.3980, 1.1443) -- (2.7740, 1.4490, 1.1448) -- (2.7250, 1.4490, 1.1509) -- cycle;
\fill[blue!15.2, opacity=0.7] (2.7250, 1.4490, 1.1509) -- (2.7740, 1.4490, 1.1448) -- (2.7740, 1.5000, 1.1449) -- (2.7250, 1.5000, 1.1511) -- cycle;
\fill[blue!15.4, opacity=0.7] (2.7250, 1.5000, 1.1511) -- (2.7740, 1.5000, 1.1449) -- (2.7740, 1.5510, 1.1448) -- (2.7250, 1.5510, 1.1509) -- cycle;
\fill[blue!15.7, opacity=0.7] (2.7250, 1.5510, 1.1509) -- (2.7740, 1.5510, 1.1448) -- (2.7740, 1.6020, 1.1443) -- (2.7250, 1.6020, 1.1504) -- cycle;
\fill[blue!16.5, opacity=0.7] (2.7250, 1.6020, 1.1504) -- (2.7740, 1.6020, 1.1443) -- (2.7740, 1.6530, 1.1435) -- (2.7250, 1.6530, 1.1496) -- cycle;
\fill[blue!18.0, opacity=0.7] (2.7250, 1.6530, 1.1496) -- (2.7740, 1.6530, 1.1435) -- (2.7740, 1.7040, 1.1423) -- (2.7250, 1.7040, 1.1484) -- cycle;
\fill[blue!20.6, opacity=0.7] (2.7250, 1.7040, 1.1484) -- (2.7740, 1.7040, 1.1423) -- (2.7740, 1.7550, 1.1409) -- (2.7250, 1.7550, 1.1470) -- cycle;
\fill[blue!23.8, opacity=0.7] (2.7250, 1.7550, 1.1470) -- (2.7740, 1.7550, 1.1409) -- (2.7740, 1.8060, 1.1391) -- (2.7250, 1.8060, 1.1452) -- cycle;
\fill[blue!26.1, opacity=0.7] (2.7250, 1.8060, 1.1452) -- (2.7740, 1.8060, 1.1391) -- (2.7740, 1.8570, 1.1370) -- (2.7250, 1.8570, 1.1431) -- cycle;
\fill[blue!25.5, opacity=0.7] (2.7250, 1.8570, 1.1431) -- (2.7740, 1.8570, 1.1370) -- (2.7740, 1.9080, 1.1346) -- (2.7250, 1.9080, 1.1407) -- cycle;
\fill[blue!21.6, opacity=0.7] (2.7250, 1.9080, 1.1407) -- (2.7740, 1.9080, 1.1346) -- (2.7740, 1.9590, 1.1319) -- (2.7250, 1.9590, 1.1380) -- cycle;
\fill[blue!17.3, opacity=0.7] (2.7250, 1.9590, 1.1380) -- (2.7740, 1.9590, 1.1319) -- (2.7740, 2.0100, 1.1289) -- (2.7250, 2.0100, 1.1350) -- cycle;
\fill[blue!15.3, opacity=0.7] (2.7250, 2.0100, 1.1350) -- (2.7740, 2.0100, 1.1289) -- (2.7740, 2.0610, 1.1256) -- (2.7250, 2.0610, 1.1317) -- cycle;
\fill[blue!15.0, opacity=0.7] (2.7250, 2.0610, 1.1317) -- (2.7740, 2.0610, 1.1256) -- (2.7740, 2.1120, 1.1220) -- (2.7250, 2.1120, 1.1281) -- cycle;
\fill[blue!15.0, opacity=0.7] (2.7250, 2.1120, 1.1281) -- (2.7740, 2.1120, 1.1220) -- (2.7740, 2.1630, 1.1182) -- (2.7250, 2.1630, 1.1243) -- cycle;
\fill[blue!15.0, opacity=0.7] (2.7250, 2.1630, 1.1243) -- (2.7740, 2.1630, 1.1182) -- (2.7740, 2.2140, 1.1141) -- (2.7250, 2.2140, 1.1202) -- cycle;
\fill[blue!15.0, opacity=0.7] (2.7250, 2.2140, 1.1202) -- (2.7740, 2.2140, 1.1141) -- (2.7740, 2.2650, 1.1098) -- (2.7250, 2.2650, 1.1159) -- cycle;
\fill[blue!15.0, opacity=0.7] (2.7250, 2.2650, 1.1159) -- (2.7740, 2.2650, 1.1098) -- (2.7740, 2.3160, 1.1052) -- (2.7250, 2.3160, 1.1114) -- cycle;
\fill[blue!15.0, opacity=0.7] (2.7250, 2.3160, 1.1114) -- (2.7740, 2.3160, 1.1052) -- (2.7740, 2.3670, 1.1005) -- (2.7250, 2.3670, 1.1066) -- cycle;
\fill[blue!15.0, opacity=0.7] (2.7250, 2.3670, 1.1066) -- (2.7740, 2.3670, 1.1005) -- (2.7740, 2.4180, 1.0955) -- (2.7250, 2.4180, 1.1016) -- cycle;
\fill[blue!15.0, opacity=0.7] (2.7250, 2.4180, 1.1016) -- (2.7740, 2.4180, 1.0955) -- (2.7740, 2.4690, 1.0903) -- (2.7250, 2.4690, 1.0964) -- cycle;
\fill[blue!15.0, opacity=0.7] (2.7250, 2.4690, 1.0964) -- (2.7740, 2.4690, 1.0903) -- (2.7740, 2.5200, 1.0849) -- (2.7250, 2.5200, 1.0911) -- cycle;
\fill[blue!15.0, opacity=0.7] (2.7250, 2.5200, 1.0911) -- (2.7740, 2.5200, 1.0849) -- (2.7740, 2.5710, 1.0794) -- (2.7250, 2.5710, 1.0855) -- cycle;
\fill[blue!15.1, opacity=0.7] (2.7250, 2.5710, 1.0855) -- (2.7740, 2.5710, 1.0794) -- (2.7740, 2.6220, 1.0738) -- (2.7250, 2.6220, 1.0799) -- cycle;
\fill[blue!15.1, opacity=0.7] (2.7250, 2.6220, 1.0799) -- (2.7740, 2.6220, 1.0738) -- (2.7740, 2.6730, 1.0680) -- (2.7250, 2.6730, 1.0741) -- cycle;
\fill[blue!15.0, opacity=0.7] (2.7250, 2.6730, 1.0741) -- (2.7740, 2.6730, 1.0680) -- (2.7740, 2.7240, 1.0620) -- (2.7250, 2.7240, 1.0681) -- cycle;
\fill[blue!15.0, opacity=0.7] (2.7250, 2.7240, 1.0681) -- (2.7740, 2.7240, 1.0620) -- (2.7740, 2.7750, 1.0560) -- (2.7250, 2.7750, 1.0621) -- cycle;
\fill[blue!15.0, opacity=0.7] (2.7250, 2.7750, 1.0621) -- (2.7740, 2.7750, 1.0560) -- (2.7740, 2.8260, 1.0499) -- (2.7250, 2.8260, 1.0560) -- cycle;
\fill[blue!15.0, opacity=0.7] (2.7250, 2.8260, 1.0560) -- (2.7740, 2.8260, 1.0499) -- (2.7740, 2.8770, 1.0437) -- (2.7250, 2.8770, 1.0498) -- cycle;
\fill[blue!15.0, opacity=0.7] (2.7250, 2.8770, 1.0498) -- (2.7740, 2.8770, 1.0437) -- (2.7740, 2.9280, 1.0375) -- (2.7250, 2.9280, 1.0436) -- cycle;
\fill[blue!15.0, opacity=0.7] (2.7250, 2.9280, 1.0436) -- (2.7740, 2.9280, 1.0375) -- (2.7740, 2.9790, 1.0312) -- (2.7250, 2.9790, 1.0373) -- cycle;
\fill[blue!15.0, opacity=0.7] (2.7250, 2.9790, 1.0373) -- (2.7740, 2.9790, 1.0312) -- (2.7740, 3.0300, 1.0249) -- (2.7250, 3.0300, 1.0311) -- cycle;
\fill[blue!15.0, opacity=0.7] (2.7740, -0.0300, 1.0249) -- (2.8230, -0.0300, 1.0188) -- (2.8230, 0.0210, 1.0251) -- (2.7740, 0.0210, 1.0312) -- cycle;
\fill[blue!15.0, opacity=0.7] (2.7740, 0.0210, 1.0312) -- (2.8230, 0.0210, 1.0251) -- (2.8230, 0.0720, 1.0313) -- (2.7740, 0.0720, 1.0375) -- cycle;
\fill[blue!15.0, opacity=0.7] (2.7740, 0.0720, 1.0375) -- (2.8230, 0.0720, 1.0313) -- (2.8230, 0.1230, 1.0375) -- (2.7740, 0.1230, 1.0437) -- cycle;
\fill[blue!15.0, opacity=0.7] (2.7740, 0.1230, 1.0437) -- (2.8230, 0.1230, 1.0375) -- (2.8230, 0.1740, 1.0437) -- (2.7740, 0.1740, 1.0499) -- cycle;
\fill[blue!15.0, opacity=0.7] (2.7740, 0.1740, 1.0499) -- (2.8230, 0.1740, 1.0437) -- (2.8230, 0.2250, 1.0498) -- (2.7740, 0.2250, 1.0560) -- cycle;
\fill[blue!15.0, opacity=0.7] (2.7740, 0.2250, 1.0560) -- (2.8230, 0.2250, 1.0498) -- (2.8230, 0.2760, 1.0559) -- (2.7740, 0.2760, 1.0620) -- cycle;
\fill[blue!15.0, opacity=0.7] (2.7740, 0.2760, 1.0620) -- (2.8230, 0.2760, 1.0559) -- (2.8230, 0.3270, 1.0618) -- (2.7740, 0.3270, 1.0680) -- cycle;
\fill[blue!15.0, opacity=0.7] (2.7740, 0.3270, 1.0680) -- (2.8230, 0.3270, 1.0618) -- (2.8230, 0.3780, 1.0676) -- (2.7740, 0.3780, 1.0738) -- cycle;
\fill[blue!15.0, opacity=0.7] (2.7740, 0.3780, 1.0738) -- (2.8230, 0.3780, 1.0676) -- (2.8230, 0.4290, 1.0733) -- (2.7740, 0.4290, 1.0794) -- cycle;
\fill[blue!15.0, opacity=0.7] (2.7740, 0.4290, 1.0794) -- (2.8230, 0.4290, 1.0733) -- (2.8230, 0.4800, 1.0788) -- (2.7740, 0.4800, 1.0849) -- cycle;
\fill[blue!15.0, opacity=0.7] (2.7740, 0.4800, 1.0849) -- (2.8230, 0.4800, 1.0788) -- (2.8230, 0.5310, 1.0841) -- (2.7740, 0.5310, 1.0903) -- cycle;
\fill[blue!15.0, opacity=0.7] (2.7740, 0.5310, 1.0903) -- (2.8230, 0.5310, 1.0841) -- (2.8230, 0.5820, 1.0893) -- (2.7740, 0.5820, 1.0955) -- cycle;
\fill[blue!15.0, opacity=0.7] (2.7740, 0.5820, 1.0955) -- (2.8230, 0.5820, 1.0893) -- (2.8230, 0.6330, 1.0943) -- (2.7740, 0.6330, 1.1005) -- cycle;
\fill[blue!15.0, opacity=0.7] (2.7740, 0.6330, 1.1005) -- (2.8230, 0.6330, 1.0943) -- (2.8230, 0.6840, 1.0991) -- (2.7740, 0.6840, 1.1052) -- cycle;
\fill[blue!15.0, opacity=0.7] (2.7740, 0.6840, 1.1052) -- (2.8230, 0.6840, 1.0991) -- (2.8230, 0.7350, 1.1036) -- (2.7740, 0.7350, 1.1098) -- cycle;
\fill[blue!15.0, opacity=0.7] (2.7740, 0.7350, 1.1098) -- (2.8230, 0.7350, 1.1036) -- (2.8230, 0.7860, 1.1079) -- (2.7740, 0.7860, 1.1141) -- cycle;
\fill[blue!15.1, opacity=0.7] (2.7740, 0.7860, 1.1141) -- (2.8230, 0.7860, 1.1079) -- (2.8230, 0.8370, 1.1120) -- (2.7740, 0.8370, 1.1182) -- cycle;
\fill[blue!15.7, opacity=0.7] (2.7740, 0.8370, 1.1182) -- (2.8230, 0.8370, 1.1120) -- (2.8230, 0.8880, 1.1159) -- (2.7740, 0.8880, 1.1220) -- cycle;
\fill[blue!17.0, opacity=0.7] (2.7740, 0.8880, 1.1220) -- (2.8230, 0.8880, 1.1159) -- (2.8230, 0.9390, 1.1194) -- (2.7740, 0.9390, 1.1256) -- cycle;
\fill[blue!18.7, opacity=0.7] (2.7740, 0.9390, 1.1256) -- (2.8230, 0.9390, 1.1194) -- (2.8230, 0.9900, 1.1227) -- (2.7740, 0.9900, 1.1289) -- cycle;
\fill[blue!20.0, opacity=0.7] (2.7740, 0.9900, 1.1289) -- (2.8230, 0.9900, 1.1227) -- (2.8230, 1.0410, 1.1257) -- (2.7740, 1.0410, 1.1319) -- cycle;
\fill[blue!20.5, opacity=0.7] (2.7740, 1.0410, 1.1319) -- (2.8230, 1.0410, 1.1257) -- (2.8230, 1.0920, 1.1284) -- (2.7740, 1.0920, 1.1346) -- cycle;
\fill[blue!20.3, opacity=0.7] (2.7740, 1.0920, 1.1346) -- (2.8230, 1.0920, 1.1284) -- (2.8230, 1.1430, 1.1308) -- (2.7740, 1.1430, 1.1370) -- cycle;
\fill[blue!19.8, opacity=0.7] (2.7740, 1.1430, 1.1370) -- (2.8230, 1.1430, 1.1308) -- (2.8230, 1.1940, 1.1329) -- (2.7740, 1.1940, 1.1391) -- cycle;
\fill[blue!19.3, opacity=0.7] (2.7740, 1.1940, 1.1391) -- (2.8230, 1.1940, 1.1329) -- (2.8230, 1.2450, 1.1347) -- (2.7740, 1.2450, 1.1409) -- cycle;
\fill[blue!18.9, opacity=0.7] (2.7740, 1.2450, 1.1409) -- (2.8230, 1.2450, 1.1347) -- (2.8230, 1.2960, 1.1361) -- (2.7740, 1.2960, 1.1423) -- cycle;
\fill[blue!18.8, opacity=0.7] (2.7740, 1.2960, 1.1423) -- (2.8230, 1.2960, 1.1361) -- (2.8230, 1.3470, 1.1373) -- (2.7740, 1.3470, 1.1435) -- cycle;
\fill[blue!18.9, opacity=0.7] (2.7740, 1.3470, 1.1435) -- (2.8230, 1.3470, 1.1373) -- (2.8230, 1.3980, 1.1381) -- (2.7740, 1.3980, 1.1443) -- cycle;
\fill[blue!19.3, opacity=0.7] (2.7740, 1.3980, 1.1443) -- (2.8230, 1.3980, 1.1381) -- (2.8230, 1.4490, 1.1386) -- (2.7740, 1.4490, 1.1448) -- cycle;
\fill[blue!20.0, opacity=0.7] (2.7740, 1.4490, 1.1448) -- (2.8230, 1.4490, 1.1386) -- (2.8230, 1.5000, 1.1388) -- (2.7740, 1.5000, 1.1449) -- cycle;
\fill[blue!21.1, opacity=0.7] (2.7740, 1.5000, 1.1449) -- (2.8230, 1.5000, 1.1388) -- (2.8230, 1.5510, 1.1386) -- (2.7740, 1.5510, 1.1448) -- cycle;
\fill[blue!22.5, opacity=0.7] (2.7740, 1.5510, 1.1448) -- (2.8230, 1.5510, 1.1386) -- (2.8230, 1.6020, 1.1381) -- (2.7740, 1.6020, 1.1443) -- cycle;
\fill[blue!23.9, opacity=0.7] (2.7740, 1.6020, 1.1443) -- (2.8230, 1.6020, 1.1381) -- (2.8230, 1.6530, 1.1373) -- (2.7740, 1.6530, 1.1435) -- cycle;
\fill[blue!24.7, opacity=0.7] (2.7740, 1.6530, 1.1435) -- (2.8230, 1.6530, 1.1373) -- (2.8230, 1.7040, 1.1361) -- (2.7740, 1.7040, 1.1423) -- cycle;
\fill[blue!24.2, opacity=0.7] (2.7740, 1.7040, 1.1423) -- (2.8230, 1.7040, 1.1361) -- (2.8230, 1.7550, 1.1347) -- (2.7740, 1.7550, 1.1409) -- cycle;
\fill[blue!22.0, opacity=0.7] (2.7740, 1.7550, 1.1409) -- (2.8230, 1.7550, 1.1347) -- (2.8230, 1.8060, 1.1329) -- (2.7740, 1.8060, 1.1391) -- cycle;
\fill[blue!18.9, opacity=0.7] (2.7740, 1.8060, 1.1391) -- (2.8230, 1.8060, 1.1329) -- (2.8230, 1.8570, 1.1308) -- (2.7740, 1.8570, 1.1370) -- cycle;
\fill[blue!16.3, opacity=0.7] (2.7740, 1.8570, 1.1370) -- (2.8230, 1.8570, 1.1308) -- (2.8230, 1.9080, 1.1284) -- (2.7740, 1.9080, 1.1346) -- cycle;
\fill[blue!15.2, opacity=0.7] (2.7740, 1.9080, 1.1346) -- (2.8230, 1.9080, 1.1284) -- (2.8230, 1.9590, 1.1257) -- (2.7740, 1.9590, 1.1319) -- cycle;
\fill[blue!15.0, opacity=0.7] (2.7740, 1.9590, 1.1319) -- (2.8230, 1.9590, 1.1257) -- (2.8230, 2.0100, 1.1227) -- (2.7740, 2.0100, 1.1289) -- cycle;
\fill[blue!15.0, opacity=0.7] (2.7740, 2.0100, 1.1289) -- (2.8230, 2.0100, 1.1227) -- (2.8230, 2.0610, 1.1194) -- (2.7740, 2.0610, 1.1256) -- cycle;
\fill[blue!15.0, opacity=0.7] (2.7740, 2.0610, 1.1256) -- (2.8230, 2.0610, 1.1194) -- (2.8230, 2.1120, 1.1159) -- (2.7740, 2.1120, 1.1220) -- cycle;
\fill[blue!15.0, opacity=0.7] (2.7740, 2.1120, 1.1220) -- (2.8230, 2.1120, 1.1159) -- (2.8230, 2.1630, 1.1120) -- (2.7740, 2.1630, 1.1182) -- cycle;
\fill[blue!15.0, opacity=0.7] (2.7740, 2.1630, 1.1182) -- (2.8230, 2.1630, 1.1120) -- (2.8230, 2.2140, 1.1079) -- (2.7740, 2.2140, 1.1141) -- cycle;
\fill[blue!15.0, opacity=0.7] (2.7740, 2.2140, 1.1141) -- (2.8230, 2.2140, 1.1079) -- (2.8230, 2.2650, 1.1036) -- (2.7740, 2.2650, 1.1098) -- cycle;
\fill[blue!15.0, opacity=0.7] (2.7740, 2.2650, 1.1098) -- (2.8230, 2.2650, 1.1036) -- (2.8230, 2.3160, 1.0991) -- (2.7740, 2.3160, 1.1052) -- cycle;
\fill[blue!15.0, opacity=0.7] (2.7740, 2.3160, 1.1052) -- (2.8230, 2.3160, 1.0991) -- (2.8230, 2.3670, 1.0943) -- (2.7740, 2.3670, 1.1005) -- cycle;
\fill[blue!15.0, opacity=0.7] (2.7740, 2.3670, 1.1005) -- (2.8230, 2.3670, 1.0943) -- (2.8230, 2.4180, 1.0893) -- (2.7740, 2.4180, 1.0955) -- cycle;
\fill[blue!15.0, opacity=0.7] (2.7740, 2.4180, 1.0955) -- (2.8230, 2.4180, 1.0893) -- (2.8230, 2.4690, 1.0841) -- (2.7740, 2.4690, 1.0903) -- cycle;
\fill[blue!15.0, opacity=0.7] (2.7740, 2.4690, 1.0903) -- (2.8230, 2.4690, 1.0841) -- (2.8230, 2.5200, 1.0788) -- (2.7740, 2.5200, 1.0849) -- cycle;
\fill[blue!15.1, opacity=0.7] (2.7740, 2.5200, 1.0849) -- (2.8230, 2.5200, 1.0788) -- (2.8230, 2.5710, 1.0733) -- (2.7740, 2.5710, 1.0794) -- cycle;
\fill[blue!15.1, opacity=0.7] (2.7740, 2.5710, 1.0794) -- (2.8230, 2.5710, 1.0733) -- (2.8230, 2.6220, 1.0676) -- (2.7740, 2.6220, 1.0738) -- cycle;
\fill[blue!15.0, opacity=0.7] (2.7740, 2.6220, 1.0738) -- (2.8230, 2.6220, 1.0676) -- (2.8230, 2.6730, 1.0618) -- (2.7740, 2.6730, 1.0680) -- cycle;
\fill[blue!15.0, opacity=0.7] (2.7740, 2.6730, 1.0680) -- (2.8230, 2.6730, 1.0618) -- (2.8230, 2.7240, 1.0559) -- (2.7740, 2.7240, 1.0620) -- cycle;
\fill[blue!15.0, opacity=0.7] (2.7740, 2.7240, 1.0620) -- (2.8230, 2.7240, 1.0559) -- (2.8230, 2.7750, 1.0498) -- (2.7740, 2.7750, 1.0560) -- cycle;
\fill[blue!15.0, opacity=0.7] (2.7740, 2.7750, 1.0560) -- (2.8230, 2.7750, 1.0498) -- (2.8230, 2.8260, 1.0437) -- (2.7740, 2.8260, 1.0499) -- cycle;
\fill[blue!15.0, opacity=0.7] (2.7740, 2.8260, 1.0499) -- (2.8230, 2.8260, 1.0437) -- (2.8230, 2.8770, 1.0375) -- (2.7740, 2.8770, 1.0437) -- cycle;
\fill[blue!15.0, opacity=0.7] (2.7740, 2.8770, 1.0437) -- (2.8230, 2.8770, 1.0375) -- (2.8230, 2.9280, 1.0313) -- (2.7740, 2.9280, 1.0375) -- cycle;
\fill[blue!15.0, opacity=0.7] (2.7740, 2.9280, 1.0375) -- (2.8230, 2.9280, 1.0313) -- (2.8230, 2.9790, 1.0251) -- (2.7740, 2.9790, 1.0312) -- cycle;
\fill[blue!15.0, opacity=0.7] (2.7740, 2.9790, 1.0312) -- (2.8230, 2.9790, 1.0251) -- (2.8230, 3.0300, 1.0188) -- (2.7740, 3.0300, 1.0249) -- cycle;
\fill[blue!15.0, opacity=0.7] (2.8230, -0.0300, 1.0188) -- (2.8720, -0.0300, 1.0125) -- (2.8720, 0.0210, 1.0188) -- (2.8230, 0.0210, 1.0251) -- cycle;
\fill[blue!15.0, opacity=0.7] (2.8230, 0.0210, 1.0251) -- (2.8720, 0.0210, 1.0188) -- (2.8720, 0.0720, 1.0251) -- (2.8230, 0.0720, 1.0313) -- cycle;
\fill[blue!15.0, opacity=0.7] (2.8230, 0.0720, 1.0313) -- (2.8720, 0.0720, 1.0251) -- (2.8720, 0.1230, 1.0313) -- (2.8230, 0.1230, 1.0375) -- cycle;
\fill[blue!15.0, opacity=0.7] (2.8230, 0.1230, 1.0375) -- (2.8720, 0.1230, 1.0313) -- (2.8720, 0.1740, 1.0375) -- (2.8230, 0.1740, 1.0437) -- cycle;
\fill[blue!15.0, opacity=0.7] (2.8230, 0.1740, 1.0437) -- (2.8720, 0.1740, 1.0375) -- (2.8720, 0.2250, 1.0436) -- (2.8230, 0.2250, 1.0498) -- cycle;
\fill[blue!15.0, opacity=0.7] (2.8230, 0.2250, 1.0498) -- (2.8720, 0.2250, 1.0436) -- (2.8720, 0.2760, 1.0496) -- (2.8230, 0.2760, 1.0559) -- cycle;
\fill[blue!15.0, opacity=0.7] (2.8230, 0.2760, 1.0559) -- (2.8720, 0.2760, 1.0496) -- (2.8720, 0.3270, 1.0555) -- (2.8230, 0.3270, 1.0618) -- cycle;
\fill[blue!15.0, opacity=0.7] (2.8230, 0.3270, 1.0618) -- (2.8720, 0.3270, 1.0555) -- (2.8720, 0.3780, 1.0614) -- (2.8230, 0.3780, 1.0676) -- cycle;
\fill[blue!15.0, opacity=0.7] (2.8230, 0.3780, 1.0676) -- (2.8720, 0.3780, 1.0614) -- (2.8720, 0.4290, 1.0670) -- (2.8230, 0.4290, 1.0733) -- cycle;
\fill[blue!15.0, opacity=0.7] (2.8230, 0.4290, 1.0733) -- (2.8720, 0.4290, 1.0670) -- (2.8720, 0.4800, 1.0725) -- (2.8230, 0.4800, 1.0788) -- cycle;
\fill[blue!15.0, opacity=0.7] (2.8230, 0.4800, 1.0788) -- (2.8720, 0.4800, 1.0725) -- (2.8720, 0.5310, 1.0779) -- (2.8230, 0.5310, 1.0841) -- cycle;
\fill[blue!15.0, opacity=0.7] (2.8230, 0.5310, 1.0841) -- (2.8720, 0.5310, 1.0779) -- (2.8720, 0.5820, 1.0831) -- (2.8230, 0.5820, 1.0893) -- cycle;
\fill[blue!15.0, opacity=0.7] (2.8230, 0.5820, 1.0893) -- (2.8720, 0.5820, 1.0831) -- (2.8720, 0.6330, 1.0881) -- (2.8230, 0.6330, 1.0943) -- cycle;
\fill[blue!15.0, opacity=0.7] (2.8230, 0.6330, 1.0943) -- (2.8720, 0.6330, 1.0881) -- (2.8720, 0.6840, 1.0928) -- (2.8230, 0.6840, 1.0991) -- cycle;
\fill[blue!15.0, opacity=0.7] (2.8230, 0.6840, 1.0991) -- (2.8720, 0.6840, 1.0928) -- (2.8720, 0.7350, 1.0974) -- (2.8230, 0.7350, 1.1036) -- cycle;
\fill[blue!15.0, opacity=0.7] (2.8230, 0.7350, 1.1036) -- (2.8720, 0.7350, 1.0974) -- (2.8720, 0.7860, 1.1017) -- (2.8230, 0.7860, 1.1079) -- cycle;
\fill[blue!15.0, opacity=0.7] (2.8230, 0.7860, 1.1079) -- (2.8720, 0.7860, 1.1017) -- (2.8720, 0.8370, 1.1058) -- (2.8230, 0.8370, 1.1120) -- cycle;
\fill[blue!15.0, opacity=0.7] (2.8230, 0.8370, 1.1120) -- (2.8720, 0.8370, 1.1058) -- (2.8720, 0.8880, 1.1096) -- (2.8230, 0.8880, 1.1159) -- cycle;
\fill[blue!15.0, opacity=0.7] (2.8230, 0.8880, 1.1159) -- (2.8720, 0.8880, 1.1096) -- (2.8720, 0.9390, 1.1132) -- (2.8230, 0.9390, 1.1194) -- cycle;
\fill[blue!15.2, opacity=0.7] (2.8230, 0.9390, 1.1194) -- (2.8720, 0.9390, 1.1132) -- (2.8720, 0.9900, 1.1165) -- (2.8230, 0.9900, 1.1227) -- cycle;
\fill[blue!15.6, opacity=0.7] (2.8230, 0.9900, 1.1227) -- (2.8720, 0.9900, 1.1165) -- (2.8720, 1.0410, 1.1195) -- (2.8230, 1.0410, 1.1257) -- cycle;
\fill[blue!16.4, opacity=0.7] (2.8230, 1.0410, 1.1257) -- (2.8720, 1.0410, 1.1195) -- (2.8720, 1.0920, 1.1222) -- (2.8230, 1.0920, 1.1284) -- cycle;
\fill[blue!17.5, opacity=0.7] (2.8230, 1.0920, 1.1284) -- (2.8720, 1.0920, 1.1222) -- (2.8720, 1.1430, 1.1246) -- (2.8230, 1.1430, 1.1308) -- cycle;
\fill[blue!18.7, opacity=0.7] (2.8230, 1.1430, 1.1308) -- (2.8720, 1.1430, 1.1246) -- (2.8720, 1.1940, 1.1267) -- (2.8230, 1.1940, 1.1329) -- cycle;
\fill[blue!19.7, opacity=0.7] (2.8230, 1.1940, 1.1329) -- (2.8720, 1.1940, 1.1267) -- (2.8720, 1.2450, 1.1285) -- (2.8230, 1.2450, 1.1347) -- cycle;
\fill[blue!20.4, opacity=0.7] (2.8230, 1.2450, 1.1347) -- (2.8720, 1.2450, 1.1285) -- (2.8720, 1.2960, 1.1299) -- (2.8230, 1.2960, 1.1361) -- cycle;
\fill[blue!21.0, opacity=0.7] (2.8230, 1.2960, 1.1361) -- (2.8720, 1.2960, 1.1299) -- (2.8720, 1.3470, 1.1311) -- (2.8230, 1.3470, 1.1373) -- cycle;
\fill[blue!21.4, opacity=0.7] (2.8230, 1.3470, 1.1373) -- (2.8720, 1.3470, 1.1311) -- (2.8720, 1.3980, 1.1319) -- (2.8230, 1.3980, 1.1381) -- cycle;
\fill[blue!21.5, opacity=0.7] (2.8230, 1.3980, 1.1381) -- (2.8720, 1.3980, 1.1319) -- (2.8720, 1.4490, 1.1324) -- (2.8230, 1.4490, 1.1386) -- cycle;
\fill[blue!21.4, opacity=0.7] (2.8230, 1.4490, 1.1386) -- (2.8720, 1.4490, 1.1324) -- (2.8720, 1.5000, 1.1325) -- (2.8230, 1.5000, 1.1388) -- cycle;
\fill[blue!21.0, opacity=0.7] (2.8230, 1.5000, 1.1388) -- (2.8720, 1.5000, 1.1325) -- (2.8720, 1.5510, 1.1324) -- (2.8230, 1.5510, 1.1386) -- cycle;
\fill[blue!20.1, opacity=0.7] (2.8230, 1.5510, 1.1386) -- (2.8720, 1.5510, 1.1324) -- (2.8720, 1.6020, 1.1319) -- (2.8230, 1.6020, 1.1381) -- cycle;
\fill[blue!18.7, opacity=0.7] (2.8230, 1.6020, 1.1381) -- (2.8720, 1.6020, 1.1319) -- (2.8720, 1.6530, 1.1311) -- (2.8230, 1.6530, 1.1373) -- cycle;
\fill[blue!17.2, opacity=0.7] (2.8230, 1.6530, 1.1373) -- (2.8720, 1.6530, 1.1311) -- (2.8720, 1.7040, 1.1299) -- (2.8230, 1.7040, 1.1361) -- cycle;
\fill[blue!16.0, opacity=0.7] (2.8230, 1.7040, 1.1361) -- (2.8720, 1.7040, 1.1299) -- (2.8720, 1.7550, 1.1285) -- (2.8230, 1.7550, 1.1347) -- cycle;
\fill[blue!15.3, opacity=0.7] (2.8230, 1.7550, 1.1347) -- (2.8720, 1.7550, 1.1285) -- (2.8720, 1.8060, 1.1267) -- (2.8230, 1.8060, 1.1329) -- cycle;
\fill[blue!15.1, opacity=0.7] (2.8230, 1.8060, 1.1329) -- (2.8720, 1.8060, 1.1267) -- (2.8720, 1.8570, 1.1246) -- (2.8230, 1.8570, 1.1308) -- cycle;
\fill[blue!15.0, opacity=0.7] (2.8230, 1.8570, 1.1308) -- (2.8720, 1.8570, 1.1246) -- (2.8720, 1.9080, 1.1222) -- (2.8230, 1.9080, 1.1284) -- cycle;
\fill[blue!15.0, opacity=0.7] (2.8230, 1.9080, 1.1284) -- (2.8720, 1.9080, 1.1222) -- (2.8720, 1.9590, 1.1195) -- (2.8230, 1.9590, 1.1257) -- cycle;
\fill[blue!15.0, opacity=0.7] (2.8230, 1.9590, 1.1257) -- (2.8720, 1.9590, 1.1195) -- (2.8720, 2.0100, 1.1165) -- (2.8230, 2.0100, 1.1227) -- cycle;
\fill[blue!15.0, opacity=0.7] (2.8230, 2.0100, 1.1227) -- (2.8720, 2.0100, 1.1165) -- (2.8720, 2.0610, 1.1132) -- (2.8230, 2.0610, 1.1194) -- cycle;
\fill[blue!15.0, opacity=0.7] (2.8230, 2.0610, 1.1194) -- (2.8720, 2.0610, 1.1132) -- (2.8720, 2.1120, 1.1096) -- (2.8230, 2.1120, 1.1159) -- cycle;
\fill[blue!15.0, opacity=0.7] (2.8230, 2.1120, 1.1159) -- (2.8720, 2.1120, 1.1096) -- (2.8720, 2.1630, 1.1058) -- (2.8230, 2.1630, 1.1120) -- cycle;
\fill[blue!15.0, opacity=0.7] (2.8230, 2.1630, 1.1120) -- (2.8720, 2.1630, 1.1058) -- (2.8720, 2.2140, 1.1017) -- (2.8230, 2.2140, 1.1079) -- cycle;
\fill[blue!15.0, opacity=0.7] (2.8230, 2.2140, 1.1079) -- (2.8720, 2.2140, 1.1017) -- (2.8720, 2.2650, 1.0974) -- (2.8230, 2.2650, 1.1036) -- cycle;
\fill[blue!15.0, opacity=0.7] (2.8230, 2.2650, 1.1036) -- (2.8720, 2.2650, 1.0974) -- (2.8720, 2.3160, 1.0928) -- (2.8230, 2.3160, 1.0991) -- cycle;
\fill[blue!15.0, opacity=0.7] (2.8230, 2.3160, 1.0991) -- (2.8720, 2.3160, 1.0928) -- (2.8720, 2.3670, 1.0881) -- (2.8230, 2.3670, 1.0943) -- cycle;
\fill[blue!15.0, opacity=0.7] (2.8230, 2.3670, 1.0943) -- (2.8720, 2.3670, 1.0881) -- (2.8720, 2.4180, 1.0831) -- (2.8230, 2.4180, 1.0893) -- cycle;
\fill[blue!15.0, opacity=0.7] (2.8230, 2.4180, 1.0893) -- (2.8720, 2.4180, 1.0831) -- (2.8720, 2.4690, 1.0779) -- (2.8230, 2.4690, 1.0841) -- cycle;
\fill[blue!15.1, opacity=0.7] (2.8230, 2.4690, 1.0841) -- (2.8720, 2.4690, 1.0779) -- (2.8720, 2.5200, 1.0725) -- (2.8230, 2.5200, 1.0788) -- cycle;
\fill[blue!15.1, opacity=0.7] (2.8230, 2.5200, 1.0788) -- (2.8720, 2.5200, 1.0725) -- (2.8720, 2.5710, 1.0670) -- (2.8230, 2.5710, 1.0733) -- cycle;
\fill[blue!15.0, opacity=0.7] (2.8230, 2.5710, 1.0733) -- (2.8720, 2.5710, 1.0670) -- (2.8720, 2.6220, 1.0614) -- (2.8230, 2.6220, 1.0676) -- cycle;
\fill[blue!15.0, opacity=0.7] (2.8230, 2.6220, 1.0676) -- (2.8720, 2.6220, 1.0614) -- (2.8720, 2.6730, 1.0555) -- (2.8230, 2.6730, 1.0618) -- cycle;
\fill[blue!15.0, opacity=0.7] (2.8230, 2.6730, 1.0618) -- (2.8720, 2.6730, 1.0555) -- (2.8720, 2.7240, 1.0496) -- (2.8230, 2.7240, 1.0559) -- cycle;
\fill[blue!15.0, opacity=0.7] (2.8230, 2.7240, 1.0559) -- (2.8720, 2.7240, 1.0496) -- (2.8720, 2.7750, 1.0436) -- (2.8230, 2.7750, 1.0498) -- cycle;
\fill[blue!15.0, opacity=0.7] (2.8230, 2.7750, 1.0498) -- (2.8720, 2.7750, 1.0436) -- (2.8720, 2.8260, 1.0375) -- (2.8230, 2.8260, 1.0437) -- cycle;
\fill[blue!15.0, opacity=0.7] (2.8230, 2.8260, 1.0437) -- (2.8720, 2.8260, 1.0375) -- (2.8720, 2.8770, 1.0313) -- (2.8230, 2.8770, 1.0375) -- cycle;
\fill[blue!15.0, opacity=0.7] (2.8230, 2.8770, 1.0375) -- (2.8720, 2.8770, 1.0313) -- (2.8720, 2.9280, 1.0251) -- (2.8230, 2.9280, 1.0313) -- cycle;
\fill[blue!15.0, opacity=0.7] (2.8230, 2.9280, 1.0313) -- (2.8720, 2.9280, 1.0251) -- (2.8720, 2.9790, 1.0188) -- (2.8230, 2.9790, 1.0251) -- cycle;
\fill[blue!15.0, opacity=0.7] (2.8230, 2.9790, 1.0251) -- (2.8720, 2.9790, 1.0188) -- (2.8720, 3.0300, 1.0125) -- (2.8230, 3.0300, 1.0188) -- cycle;
\fill[blue!15.0, opacity=0.7] (2.8720, -0.0300, 1.0125) -- (2.9210, -0.0300, 1.0063) -- (2.9210, 0.0210, 1.0126) -- (2.8720, 0.0210, 1.0188) -- cycle;
\fill[blue!15.0, opacity=0.7] (2.8720, 0.0210, 1.0188) -- (2.9210, 0.0210, 1.0126) -- (2.9210, 0.0720, 1.0188) -- (2.8720, 0.0720, 1.0251) -- cycle;
\fill[blue!15.0, opacity=0.7] (2.8720, 0.0720, 1.0251) -- (2.9210, 0.0720, 1.0188) -- (2.9210, 0.1230, 1.0251) -- (2.8720, 0.1230, 1.0313) -- cycle;
\fill[blue!15.0, opacity=0.7] (2.8720, 0.1230, 1.0313) -- (2.9210, 0.1230, 1.0251) -- (2.9210, 0.1740, 1.0312) -- (2.8720, 0.1740, 1.0375) -- cycle;
\fill[blue!15.0, opacity=0.7] (2.8720, 0.1740, 1.0375) -- (2.9210, 0.1740, 1.0312) -- (2.9210, 0.2250, 1.0373) -- (2.8720, 0.2250, 1.0436) -- cycle;
\fill[blue!15.0, opacity=0.7] (2.8720, 0.2250, 1.0436) -- (2.9210, 0.2250, 1.0373) -- (2.9210, 0.2760, 1.0434) -- (2.8720, 0.2760, 1.0496) -- cycle;
\fill[blue!15.0, opacity=0.7] (2.8720, 0.2760, 1.0496) -- (2.9210, 0.2760, 1.0434) -- (2.9210, 0.3270, 1.0493) -- (2.8720, 0.3270, 1.0555) -- cycle;
\fill[blue!15.0, opacity=0.7] (2.8720, 0.3270, 1.0555) -- (2.9210, 0.3270, 1.0493) -- (2.9210, 0.3780, 1.0551) -- (2.8720, 0.3780, 1.0614) -- cycle;
\fill[blue!15.0, opacity=0.7] (2.8720, 0.3780, 1.0614) -- (2.9210, 0.3780, 1.0551) -- (2.9210, 0.4290, 1.0608) -- (2.8720, 0.4290, 1.0670) -- cycle;
\fill[blue!15.0, opacity=0.7] (2.8720, 0.4290, 1.0670) -- (2.9210, 0.4290, 1.0608) -- (2.9210, 0.4800, 1.0663) -- (2.8720, 0.4800, 1.0725) -- cycle;
\fill[blue!15.0, opacity=0.7] (2.8720, 0.4800, 1.0725) -- (2.9210, 0.4800, 1.0663) -- (2.9210, 0.5310, 1.0716) -- (2.8720, 0.5310, 1.0779) -- cycle;
\fill[blue!15.0, opacity=0.7] (2.8720, 0.5310, 1.0779) -- (2.9210, 0.5310, 1.0716) -- (2.9210, 0.5820, 1.0768) -- (2.8720, 0.5820, 1.0831) -- cycle;
\fill[blue!15.0, opacity=0.7] (2.8720, 0.5820, 1.0831) -- (2.9210, 0.5820, 1.0768) -- (2.9210, 0.6330, 1.0818) -- (2.8720, 0.6330, 1.0881) -- cycle;
\fill[blue!15.0, opacity=0.7] (2.8720, 0.6330, 1.0881) -- (2.9210, 0.6330, 1.0818) -- (2.9210, 0.6840, 1.0866) -- (2.8720, 0.6840, 1.0928) -- cycle;
\fill[blue!15.0, opacity=0.7] (2.8720, 0.6840, 1.0928) -- (2.9210, 0.6840, 1.0866) -- (2.9210, 0.7350, 1.0911) -- (2.8720, 0.7350, 1.0974) -- cycle;
\fill[blue!15.0, opacity=0.7] (2.8720, 0.7350, 1.0974) -- (2.9210, 0.7350, 1.0911) -- (2.9210, 0.7860, 1.0955) -- (2.8720, 0.7860, 1.1017) -- cycle;
\fill[blue!15.0, opacity=0.7] (2.8720, 0.7860, 1.1017) -- (2.9210, 0.7860, 1.0955) -- (2.9210, 0.8370, 1.0995) -- (2.8720, 0.8370, 1.1058) -- cycle;
\fill[blue!15.0, opacity=0.7] (2.8720, 0.8370, 1.1058) -- (2.9210, 0.8370, 1.0995) -- (2.9210, 0.8880, 1.1034) -- (2.8720, 0.8880, 1.1096) -- cycle;
\fill[blue!15.0, opacity=0.7] (2.8720, 0.8880, 1.1096) -- (2.9210, 0.8880, 1.1034) -- (2.9210, 0.9390, 1.1069) -- (2.8720, 0.9390, 1.1132) -- cycle;
\fill[blue!15.0, opacity=0.7] (2.8720, 0.9390, 1.1132) -- (2.9210, 0.9390, 1.1069) -- (2.9210, 0.9900, 1.1102) -- (2.8720, 0.9900, 1.1165) -- cycle;
\fill[blue!15.0, opacity=0.7] (2.8720, 0.9900, 1.1165) -- (2.9210, 0.9900, 1.1102) -- (2.9210, 1.0410, 1.1132) -- (2.8720, 1.0410, 1.1195) -- cycle;
\fill[blue!15.0, opacity=0.7] (2.8720, 1.0410, 1.1195) -- (2.9210, 1.0410, 1.1132) -- (2.9210, 1.0920, 1.1159) -- (2.8720, 1.0920, 1.1222) -- cycle;
\fill[blue!15.0, opacity=0.7] (2.8720, 1.0920, 1.1222) -- (2.9210, 1.0920, 1.1159) -- (2.9210, 1.1430, 1.1183) -- (2.8720, 1.1430, 1.1246) -- cycle;
\fill[blue!15.1, opacity=0.7] (2.8720, 1.1430, 1.1246) -- (2.9210, 1.1430, 1.1183) -- (2.9210, 1.1940, 1.1204) -- (2.8720, 1.1940, 1.1267) -- cycle;
\fill[blue!15.2, opacity=0.7] (2.8720, 1.1940, 1.1267) -- (2.9210, 1.1940, 1.1204) -- (2.9210, 1.2450, 1.1222) -- (2.8720, 1.2450, 1.1285) -- cycle;
\fill[blue!15.3, opacity=0.7] (2.8720, 1.2450, 1.1285) -- (2.9210, 1.2450, 1.1222) -- (2.9210, 1.2960, 1.1237) -- (2.8720, 1.2960, 1.1299) -- cycle;
\fill[blue!15.4, opacity=0.7] (2.8720, 1.2960, 1.1299) -- (2.9210, 1.2960, 1.1237) -- (2.9210, 1.3470, 1.1248) -- (2.8720, 1.3470, 1.1311) -- cycle;
\fill[blue!15.5, opacity=0.7] (2.8720, 1.3470, 1.1311) -- (2.9210, 1.3470, 1.1248) -- (2.9210, 1.3980, 1.1256) -- (2.8720, 1.3980, 1.1319) -- cycle;
\fill[blue!15.4, opacity=0.7] (2.8720, 1.3980, 1.1319) -- (2.9210, 1.3980, 1.1256) -- (2.9210, 1.4490, 1.1261) -- (2.8720, 1.4490, 1.1324) -- cycle;
\fill[blue!15.4, opacity=0.7] (2.8720, 1.4490, 1.1324) -- (2.9210, 1.4490, 1.1261) -- (2.9210, 1.5000, 1.1263) -- (2.8720, 1.5000, 1.1325) -- cycle;
\fill[blue!15.2, opacity=0.7] (2.8720, 1.5000, 1.1325) -- (2.9210, 1.5000, 1.1263) -- (2.9210, 1.5510, 1.1261) -- (2.8720, 1.5510, 1.1324) -- cycle;
\fill[blue!15.1, opacity=0.7] (2.8720, 1.5510, 1.1324) -- (2.9210, 1.5510, 1.1261) -- (2.9210, 1.6020, 1.1256) -- (2.8720, 1.6020, 1.1319) -- cycle;
\fill[blue!15.0, opacity=0.7] (2.8720, 1.6020, 1.1319) -- (2.9210, 1.6020, 1.1256) -- (2.9210, 1.6530, 1.1248) -- (2.8720, 1.6530, 1.1311) -- cycle;
\fill[blue!15.0, opacity=0.7] (2.8720, 1.6530, 1.1311) -- (2.9210, 1.6530, 1.1248) -- (2.9210, 1.7040, 1.1237) -- (2.8720, 1.7040, 1.1299) -- cycle;
\fill[blue!15.0, opacity=0.7] (2.8720, 1.7040, 1.1299) -- (2.9210, 1.7040, 1.1237) -- (2.9210, 1.7550, 1.1222) -- (2.8720, 1.7550, 1.1285) -- cycle;
\fill[blue!15.0, opacity=0.7] (2.8720, 1.7550, 1.1285) -- (2.9210, 1.7550, 1.1222) -- (2.9210, 1.8060, 1.1204) -- (2.8720, 1.8060, 1.1267) -- cycle;
\fill[blue!15.0, opacity=0.7] (2.8720, 1.8060, 1.1267) -- (2.9210, 1.8060, 1.1204) -- (2.9210, 1.8570, 1.1183) -- (2.8720, 1.8570, 1.1246) -- cycle;
\fill[blue!15.0, opacity=0.7] (2.8720, 1.8570, 1.1246) -- (2.9210, 1.8570, 1.1183) -- (2.9210, 1.9080, 1.1159) -- (2.8720, 1.9080, 1.1222) -- cycle;
\fill[blue!15.0, opacity=0.7] (2.8720, 1.9080, 1.1222) -- (2.9210, 1.9080, 1.1159) -- (2.9210, 1.9590, 1.1132) -- (2.8720, 1.9590, 1.1195) -- cycle;
\fill[blue!15.0, opacity=0.7] (2.8720, 1.9590, 1.1195) -- (2.9210, 1.9590, 1.1132) -- (2.9210, 2.0100, 1.1102) -- (2.8720, 2.0100, 1.1165) -- cycle;
\fill[blue!15.0, opacity=0.7] (2.8720, 2.0100, 1.1165) -- (2.9210, 2.0100, 1.1102) -- (2.9210, 2.0610, 1.1069) -- (2.8720, 2.0610, 1.1132) -- cycle;
\fill[blue!15.0, opacity=0.7] (2.8720, 2.0610, 1.1132) -- (2.9210, 2.0610, 1.1069) -- (2.9210, 2.1120, 1.1034) -- (2.8720, 2.1120, 1.1096) -- cycle;
\fill[blue!15.0, opacity=0.7] (2.8720, 2.1120, 1.1096) -- (2.9210, 2.1120, 1.1034) -- (2.9210, 2.1630, 1.0995) -- (2.8720, 2.1630, 1.1058) -- cycle;
\fill[blue!15.0, opacity=0.7] (2.8720, 2.1630, 1.1058) -- (2.9210, 2.1630, 1.0995) -- (2.9210, 2.2140, 1.0955) -- (2.8720, 2.2140, 1.1017) -- cycle;
\fill[blue!15.0, opacity=0.7] (2.8720, 2.2140, 1.1017) -- (2.9210, 2.2140, 1.0955) -- (2.9210, 2.2650, 1.0911) -- (2.8720, 2.2650, 1.0974) -- cycle;
\fill[blue!15.0, opacity=0.7] (2.8720, 2.2650, 1.0974) -- (2.9210, 2.2650, 1.0911) -- (2.9210, 2.3160, 1.0866) -- (2.8720, 2.3160, 1.0928) -- cycle;
\fill[blue!15.0, opacity=0.7] (2.8720, 2.3160, 1.0928) -- (2.9210, 2.3160, 1.0866) -- (2.9210, 2.3670, 1.0818) -- (2.8720, 2.3670, 1.0881) -- cycle;
\fill[blue!15.1, opacity=0.7] (2.8720, 2.3670, 1.0881) -- (2.9210, 2.3670, 1.0818) -- (2.9210, 2.4180, 1.0768) -- (2.8720, 2.4180, 1.0831) -- cycle;
\fill[blue!15.1, opacity=0.7] (2.8720, 2.4180, 1.0831) -- (2.9210, 2.4180, 1.0768) -- (2.9210, 2.4690, 1.0716) -- (2.8720, 2.4690, 1.0779) -- cycle;
\fill[blue!15.0, opacity=0.7] (2.8720, 2.4690, 1.0779) -- (2.9210, 2.4690, 1.0716) -- (2.9210, 2.5200, 1.0663) -- (2.8720, 2.5200, 1.0725) -- cycle;
\fill[blue!15.0, opacity=0.7] (2.8720, 2.5200, 1.0725) -- (2.9210, 2.5200, 1.0663) -- (2.9210, 2.5710, 1.0608) -- (2.8720, 2.5710, 1.0670) -- cycle;
\fill[blue!15.0, opacity=0.7] (2.8720, 2.5710, 1.0670) -- (2.9210, 2.5710, 1.0608) -- (2.9210, 2.6220, 1.0551) -- (2.8720, 2.6220, 1.0614) -- cycle;
\fill[blue!15.0, opacity=0.7] (2.8720, 2.6220, 1.0614) -- (2.9210, 2.6220, 1.0551) -- (2.9210, 2.6730, 1.0493) -- (2.8720, 2.6730, 1.0555) -- cycle;
\fill[blue!15.0, opacity=0.7] (2.8720, 2.6730, 1.0555) -- (2.9210, 2.6730, 1.0493) -- (2.9210, 2.7240, 1.0434) -- (2.8720, 2.7240, 1.0496) -- cycle;
\fill[blue!15.0, opacity=0.7] (2.8720, 2.7240, 1.0496) -- (2.9210, 2.7240, 1.0434) -- (2.9210, 2.7750, 1.0373) -- (2.8720, 2.7750, 1.0436) -- cycle;
\fill[blue!15.0, opacity=0.7] (2.8720, 2.7750, 1.0436) -- (2.9210, 2.7750, 1.0373) -- (2.9210, 2.8260, 1.0312) -- (2.8720, 2.8260, 1.0375) -- cycle;
\fill[blue!15.0, opacity=0.7] (2.8720, 2.8260, 1.0375) -- (2.9210, 2.8260, 1.0312) -- (2.9210, 2.8770, 1.0251) -- (2.8720, 2.8770, 1.0313) -- cycle;
\fill[blue!15.0, opacity=0.7] (2.8720, 2.8770, 1.0313) -- (2.9210, 2.8770, 1.0251) -- (2.9210, 2.9280, 1.0188) -- (2.8720, 2.9280, 1.0251) -- cycle;
\fill[blue!15.0, opacity=0.7] (2.8720, 2.9280, 1.0251) -- (2.9210, 2.9280, 1.0188) -- (2.9210, 2.9790, 1.0126) -- (2.8720, 2.9790, 1.0188) -- cycle;
\fill[blue!15.0, opacity=0.7] (2.8720, 2.9790, 1.0188) -- (2.9210, 2.9790, 1.0126) -- (2.9210, 3.0300, 1.0063) -- (2.8720, 3.0300, 1.0125) -- cycle;
\fill[blue!15.0, opacity=0.7] (2.9210, -0.0300, 1.0063) -- (2.9700, -0.0300, 1.0000) -- (2.9700, 0.0210, 1.0063) -- (2.9210, 0.0210, 1.0126) -- cycle;
\fill[blue!15.0, opacity=0.7] (2.9210, 0.0210, 1.0126) -- (2.9700, 0.0210, 1.0063) -- (2.9700, 0.0720, 1.0125) -- (2.9210, 0.0720, 1.0188) -- cycle;
\fill[blue!15.0, opacity=0.7] (2.9210, 0.0720, 1.0188) -- (2.9700, 0.0720, 1.0125) -- (2.9700, 0.1230, 1.0188) -- (2.9210, 0.1230, 1.0251) -- cycle;
\fill[blue!15.0, opacity=0.7] (2.9210, 0.1230, 1.0251) -- (2.9700, 0.1230, 1.0188) -- (2.9700, 0.1740, 1.0249) -- (2.9210, 0.1740, 1.0312) -- cycle;
\fill[blue!15.0, opacity=0.7] (2.9210, 0.1740, 1.0312) -- (2.9700, 0.1740, 1.0249) -- (2.9700, 0.2250, 1.0311) -- (2.9210, 0.2250, 1.0373) -- cycle;
\fill[blue!15.0, opacity=0.7] (2.9210, 0.2250, 1.0373) -- (2.9700, 0.2250, 1.0311) -- (2.9700, 0.2760, 1.0371) -- (2.9210, 0.2760, 1.0434) -- cycle;
\fill[blue!15.0, opacity=0.7] (2.9210, 0.2760, 1.0434) -- (2.9700, 0.2760, 1.0371) -- (2.9700, 0.3270, 1.0430) -- (2.9210, 0.3270, 1.0493) -- cycle;
\fill[blue!15.0, opacity=0.7] (2.9210, 0.3270, 1.0493) -- (2.9700, 0.3270, 1.0430) -- (2.9700, 0.3780, 1.0488) -- (2.9210, 0.3780, 1.0551) -- cycle;
\fill[blue!15.0, opacity=0.7] (2.9210, 0.3780, 1.0551) -- (2.9700, 0.3780, 1.0488) -- (2.9700, 0.4290, 1.0545) -- (2.9210, 0.4290, 1.0608) -- cycle;
\fill[blue!15.0, opacity=0.7] (2.9210, 0.4290, 1.0608) -- (2.9700, 0.4290, 1.0545) -- (2.9700, 0.4800, 1.0600) -- (2.9210, 0.4800, 1.0663) -- cycle;
\fill[blue!15.0, opacity=0.7] (2.9210, 0.4800, 1.0663) -- (2.9700, 0.4800, 1.0600) -- (2.9700, 0.5310, 1.0654) -- (2.9210, 0.5310, 1.0716) -- cycle;
\fill[blue!15.0, opacity=0.7] (2.9210, 0.5310, 1.0716) -- (2.9700, 0.5310, 1.0654) -- (2.9700, 0.5820, 1.0705) -- (2.9210, 0.5820, 1.0768) -- cycle;
\fill[blue!15.0, opacity=0.7] (2.9210, 0.5820, 1.0768) -- (2.9700, 0.5820, 1.0705) -- (2.9700, 0.6330, 1.0755) -- (2.9210, 0.6330, 1.0818) -- cycle;
\fill[blue!15.0, opacity=0.7] (2.9210, 0.6330, 1.0818) -- (2.9700, 0.6330, 1.0755) -- (2.9700, 0.6840, 1.0803) -- (2.9210, 0.6840, 1.0866) -- cycle;
\fill[blue!15.0, opacity=0.7] (2.9210, 0.6840, 1.0866) -- (2.9700, 0.6840, 1.0803) -- (2.9700, 0.7350, 1.0849) -- (2.9210, 0.7350, 1.0911) -- cycle;
\fill[blue!15.0, opacity=0.7] (2.9210, 0.7350, 1.0911) -- (2.9700, 0.7350, 1.0849) -- (2.9700, 0.7860, 1.0892) -- (2.9210, 0.7860, 1.0955) -- cycle;
\fill[blue!15.0, opacity=0.7] (2.9210, 0.7860, 1.0955) -- (2.9700, 0.7860, 1.0892) -- (2.9700, 0.8370, 1.0933) -- (2.9210, 0.8370, 1.0995) -- cycle;
\fill[blue!15.0, opacity=0.7] (2.9210, 0.8370, 1.0995) -- (2.9700, 0.8370, 1.0933) -- (2.9700, 0.8880, 1.0971) -- (2.9210, 0.8880, 1.1034) -- cycle;
\fill[blue!15.0, opacity=0.7] (2.9210, 0.8880, 1.1034) -- (2.9700, 0.8880, 1.0971) -- (2.9700, 0.9390, 1.1006) -- (2.9210, 0.9390, 1.1069) -- cycle;
\fill[blue!15.0, opacity=0.7] (2.9210, 0.9390, 1.1069) -- (2.9700, 0.9390, 1.1006) -- (2.9700, 0.9900, 1.1039) -- (2.9210, 0.9900, 1.1102) -- cycle;
\fill[blue!15.0, opacity=0.7] (2.9210, 0.9900, 1.1102) -- (2.9700, 0.9900, 1.1039) -- (2.9700, 1.0410, 1.1069) -- (2.9210, 1.0410, 1.1132) -- cycle;
\fill[blue!15.0, opacity=0.7] (2.9210, 1.0410, 1.1132) -- (2.9700, 1.0410, 1.1069) -- (2.9700, 1.0920, 1.1096) -- (2.9210, 1.0920, 1.1159) -- cycle;
\fill[blue!15.0, opacity=0.7] (2.9210, 1.0920, 1.1159) -- (2.9700, 1.0920, 1.1096) -- (2.9700, 1.1430, 1.1120) -- (2.9210, 1.1430, 1.1183) -- cycle;
\fill[blue!15.0, opacity=0.7] (2.9210, 1.1430, 1.1183) -- (2.9700, 1.1430, 1.1120) -- (2.9700, 1.1940, 1.1141) -- (2.9210, 1.1940, 1.1204) -- cycle;
\fill[blue!15.0, opacity=0.7] (2.9210, 1.1940, 1.1204) -- (2.9700, 1.1940, 1.1141) -- (2.9700, 1.2450, 1.1159) -- (2.9210, 1.2450, 1.1222) -- cycle;
\fill[blue!15.0, opacity=0.7] (2.9210, 1.2450, 1.1222) -- (2.9700, 1.2450, 1.1159) -- (2.9700, 1.2960, 1.1174) -- (2.9210, 1.2960, 1.1237) -- cycle;
\fill[blue!15.0, opacity=0.7] (2.9210, 1.2960, 1.1237) -- (2.9700, 1.2960, 1.1174) -- (2.9700, 1.3470, 1.1185) -- (2.9210, 1.3470, 1.1248) -- cycle;
\fill[blue!15.0, opacity=0.7] (2.9210, 1.3470, 1.1248) -- (2.9700, 1.3470, 1.1185) -- (2.9700, 1.3980, 1.1193) -- (2.9210, 1.3980, 1.1256) -- cycle;
\fill[blue!15.0, opacity=0.7] (2.9210, 1.3980, 1.1256) -- (2.9700, 1.3980, 1.1193) -- (2.9700, 1.4490, 1.1198) -- (2.9210, 1.4490, 1.1261) -- cycle;
\fill[blue!15.0, opacity=0.7] (2.9210, 1.4490, 1.1261) -- (2.9700, 1.4490, 1.1198) -- (2.9700, 1.5000, 1.1200) -- (2.9210, 1.5000, 1.1263) -- cycle;
\fill[blue!15.0, opacity=0.7] (2.9210, 1.5000, 1.1263) -- (2.9700, 1.5000, 1.1200) -- (2.9700, 1.5510, 1.1198) -- (2.9210, 1.5510, 1.1261) -- cycle;
\fill[blue!15.0, opacity=0.7] (2.9210, 1.5510, 1.1261) -- (2.9700, 1.5510, 1.1198) -- (2.9700, 1.6020, 1.1193) -- (2.9210, 1.6020, 1.1256) -- cycle;
\fill[blue!15.0, opacity=0.7] (2.9210, 1.6020, 1.1256) -- (2.9700, 1.6020, 1.1193) -- (2.9700, 1.6530, 1.1185) -- (2.9210, 1.6530, 1.1248) -- cycle;
\fill[blue!15.0, opacity=0.7] (2.9210, 1.6530, 1.1248) -- (2.9700, 1.6530, 1.1185) -- (2.9700, 1.7040, 1.1174) -- (2.9210, 1.7040, 1.1237) -- cycle;
\fill[blue!15.0, opacity=0.7] (2.9210, 1.7040, 1.1237) -- (2.9700, 1.7040, 1.1174) -- (2.9700, 1.7550, 1.1159) -- (2.9210, 1.7550, 1.1222) -- cycle;
\fill[blue!15.0, opacity=0.7] (2.9210, 1.7550, 1.1222) -- (2.9700, 1.7550, 1.1159) -- (2.9700, 1.8060, 1.1141) -- (2.9210, 1.8060, 1.1204) -- cycle;
\fill[blue!15.0, opacity=0.7] (2.9210, 1.8060, 1.1204) -- (2.9700, 1.8060, 1.1141) -- (2.9700, 1.8570, 1.1120) -- (2.9210, 1.8570, 1.1183) -- cycle;
\fill[blue!15.0, opacity=0.7] (2.9210, 1.8570, 1.1183) -- (2.9700, 1.8570, 1.1120) -- (2.9700, 1.9080, 1.1096) -- (2.9210, 1.9080, 1.1159) -- cycle;
\fill[blue!15.0, opacity=0.7] (2.9210, 1.9080, 1.1159) -- (2.9700, 1.9080, 1.1096) -- (2.9700, 1.9590, 1.1069) -- (2.9210, 1.9590, 1.1132) -- cycle;
\fill[blue!15.0, opacity=0.7] (2.9210, 1.9590, 1.1132) -- (2.9700, 1.9590, 1.1069) -- (2.9700, 2.0100, 1.1039) -- (2.9210, 2.0100, 1.1102) -- cycle;
\fill[blue!15.0, opacity=0.7] (2.9210, 2.0100, 1.1102) -- (2.9700, 2.0100, 1.1039) -- (2.9700, 2.0610, 1.1006) -- (2.9210, 2.0610, 1.1069) -- cycle;
\fill[blue!15.0, opacity=0.7] (2.9210, 2.0610, 1.1069) -- (2.9700, 2.0610, 1.1006) -- (2.9700, 2.1120, 1.0971) -- (2.9210, 2.1120, 1.1034) -- cycle;
\fill[blue!15.0, opacity=0.7] (2.9210, 2.1120, 1.1034) -- (2.9700, 2.1120, 1.0971) -- (2.9700, 2.1630, 1.0933) -- (2.9210, 2.1630, 1.0995) -- cycle;
\fill[blue!15.0, opacity=0.7] (2.9210, 2.1630, 1.0995) -- (2.9700, 2.1630, 1.0933) -- (2.9700, 2.2140, 1.0892) -- (2.9210, 2.2140, 1.0955) -- cycle;
\fill[blue!15.0, opacity=0.7] (2.9210, 2.2140, 1.0955) -- (2.9700, 2.2140, 1.0892) -- (2.9700, 2.2650, 1.0849) -- (2.9210, 2.2650, 1.0911) -- cycle;
\fill[blue!15.0, opacity=0.7] (2.9210, 2.2650, 1.0911) -- (2.9700, 2.2650, 1.0849) -- (2.9700, 2.3160, 1.0803) -- (2.9210, 2.3160, 1.0866) -- cycle;
\fill[blue!15.1, opacity=0.7] (2.9210, 2.3160, 1.0866) -- (2.9700, 2.3160, 1.0803) -- (2.9700, 2.3670, 1.0755) -- (2.9210, 2.3670, 1.0818) -- cycle;
\fill[blue!15.1, opacity=0.7] (2.9210, 2.3670, 1.0818) -- (2.9700, 2.3670, 1.0755) -- (2.9700, 2.4180, 1.0705) -- (2.9210, 2.4180, 1.0768) -- cycle;
\fill[blue!15.0, opacity=0.7] (2.9210, 2.4180, 1.0768) -- (2.9700, 2.4180, 1.0705) -- (2.9700, 2.4690, 1.0654) -- (2.9210, 2.4690, 1.0716) -- cycle;
\fill[blue!15.0, opacity=0.7] (2.9210, 2.4690, 1.0716) -- (2.9700, 2.4690, 1.0654) -- (2.9700, 2.5200, 1.0600) -- (2.9210, 2.5200, 1.0663) -- cycle;
\fill[blue!15.0, opacity=0.7] (2.9210, 2.5200, 1.0663) -- (2.9700, 2.5200, 1.0600) -- (2.9700, 2.5710, 1.0545) -- (2.9210, 2.5710, 1.0608) -- cycle;
\fill[blue!15.0, opacity=0.7] (2.9210, 2.5710, 1.0608) -- (2.9700, 2.5710, 1.0545) -- (2.9700, 2.6220, 1.0488) -- (2.9210, 2.6220, 1.0551) -- cycle;
\fill[blue!15.0, opacity=0.7] (2.9210, 2.6220, 1.0551) -- (2.9700, 2.6220, 1.0488) -- (2.9700, 2.6730, 1.0430) -- (2.9210, 2.6730, 1.0493) -- cycle;
\fill[blue!15.0, opacity=0.7] (2.9210, 2.6730, 1.0493) -- (2.9700, 2.6730, 1.0430) -- (2.9700, 2.7240, 1.0371) -- (2.9210, 2.7240, 1.0434) -- cycle;
\fill[blue!15.0, opacity=0.7] (2.9210, 2.7240, 1.0434) -- (2.9700, 2.7240, 1.0371) -- (2.9700, 2.7750, 1.0311) -- (2.9210, 2.7750, 1.0373) -- cycle;
\fill[blue!15.0, opacity=0.7] (2.9210, 2.7750, 1.0373) -- (2.9700, 2.7750, 1.0311) -- (2.9700, 2.8260, 1.0249) -- (2.9210, 2.8260, 1.0312) -- cycle;
\fill[blue!15.0, opacity=0.7] (2.9210, 2.8260, 1.0312) -- (2.9700, 2.8260, 1.0249) -- (2.9700, 2.8770, 1.0188) -- (2.9210, 2.8770, 1.0251) -- cycle;
\fill[blue!15.0, opacity=0.7] (2.9210, 2.8770, 1.0251) -- (2.9700, 2.8770, 1.0188) -- (2.9700, 2.9280, 1.0125) -- (2.9210, 2.9280, 1.0188) -- cycle;
\fill[blue!15.0, opacity=0.7] (2.9210, 2.9280, 1.0188) -- (2.9700, 2.9280, 1.0125) -- (2.9700, 2.9790, 1.0063) -- (2.9210, 2.9790, 1.0126) -- cycle;
\fill[blue!15.0, opacity=0.7] (2.9210, 2.9790, 1.0126) -- (2.9700, 2.9790, 1.0063) -- (2.9700, 3.0300, 1.0000) -- (2.9210, 3.0300, 1.0063) -- cycle;
% Slice 2 horizontal patches
\fill[blue!16.5, opacity=0.7] (0.0900, -0.0900, 2.0000) -- (0.1370, -0.0900, 2.0063) -- (0.1370, -0.0370, 2.0126) -- (0.0900, -0.0370, 2.0063) -- cycle;
\fill[blue!15.8, opacity=0.7] (0.0900, -0.0370, 2.0063) -- (0.1370, -0.0370, 2.0126) -- (0.1370, 0.0160, 2.0188) -- (0.0900, 0.0160, 2.0125) -- cycle;
\fill[blue!15.1, opacity=0.7] (0.0900, 0.0160, 2.0125) -- (0.1370, 0.0160, 2.0188) -- (0.1370, 0.0690, 2.0251) -- (0.0900, 0.0690, 2.0188) -- cycle;
\fill[blue!15.0, opacity=0.7] (0.0900, 0.0690, 2.0188) -- (0.1370, 0.0690, 2.0251) -- (0.1370, 0.1220, 2.0312) -- (0.0900, 0.1220, 2.0249) -- cycle;
\fill[blue!15.0, opacity=0.7] (0.0900, 0.1220, 2.0249) -- (0.1370, 0.1220, 2.0312) -- (0.1370, 0.1750, 2.0373) -- (0.0900, 0.1750, 2.0311) -- cycle;
\fill[blue!15.0, opacity=0.7] (0.0900, 0.1750, 2.0311) -- (0.1370, 0.1750, 2.0373) -- (0.1370, 0.2280, 2.0434) -- (0.0900, 0.2280, 2.0371) -- cycle;
\fill[blue!15.0, opacity=0.7] (0.0900, 0.2280, 2.0371) -- (0.1370, 0.2280, 2.0434) -- (0.1370, 0.2810, 2.0493) -- (0.0900, 0.2810, 2.0430) -- cycle;
\fill[blue!15.0, opacity=0.7] (0.0900, 0.2810, 2.0430) -- (0.1370, 0.2810, 2.0493) -- (0.1370, 0.3340, 2.0551) -- (0.0900, 0.3340, 2.0488) -- cycle;
\fill[blue!15.0, opacity=0.7] (0.0900, 0.3340, 2.0488) -- (0.1370, 0.3340, 2.0551) -- (0.1370, 0.3870, 2.0608) -- (0.0900, 0.3870, 2.0545) -- cycle;
\fill[blue!15.4, opacity=0.7] (0.0900, 0.3870, 2.0545) -- (0.1370, 0.3870, 2.0608) -- (0.1370, 0.4400, 2.0663) -- (0.0900, 0.4400, 2.0600) -- cycle;
\fill[blue!18.2, opacity=0.7] (0.0900, 0.4400, 2.0600) -- (0.1370, 0.4400, 2.0663) -- (0.1370, 0.4930, 2.0716) -- (0.0900, 0.4930, 2.0654) -- cycle;
\fill[blue!25.9, opacity=0.7] (0.0900, 0.4930, 2.0654) -- (0.1370, 0.4930, 2.0716) -- (0.1370, 0.5460, 2.0768) -- (0.0900, 0.5460, 2.0705) -- cycle;
\fill[blue!33.1, opacity=0.7] (0.0900, 0.5460, 2.0705) -- (0.1370, 0.5460, 2.0768) -- (0.1370, 0.5990, 2.0818) -- (0.0900, 0.5990, 2.0755) -- cycle;
\fill[blue!33.1, opacity=0.7] (0.0900, 0.5990, 2.0755) -- (0.1370, 0.5990, 2.0818) -- (0.1370, 0.6520, 2.0866) -- (0.0900, 0.6520, 2.0803) -- cycle;
\fill[blue!26.7, opacity=0.7] (0.0900, 0.6520, 2.0803) -- (0.1370, 0.6520, 2.0866) -- (0.1370, 0.7050, 2.0911) -- (0.0900, 0.7050, 2.0849) -- cycle;
\fill[blue!19.9, opacity=0.7] (0.0900, 0.7050, 2.0849) -- (0.1370, 0.7050, 2.0911) -- (0.1370, 0.7580, 2.0955) -- (0.0900, 0.7580, 2.0892) -- cycle;
\fill[blue!16.4, opacity=0.7] (0.0900, 0.7580, 2.0892) -- (0.1370, 0.7580, 2.0955) -- (0.1370, 0.8110, 2.0995) -- (0.0900, 0.8110, 2.0933) -- cycle;
\fill[blue!15.3, opacity=0.7] (0.0900, 0.8110, 2.0933) -- (0.1370, 0.8110, 2.0995) -- (0.1370, 0.8640, 2.1034) -- (0.0900, 0.8640, 2.0971) -- cycle;
\fill[blue!15.1, opacity=0.7] (0.0900, 0.8640, 2.0971) -- (0.1370, 0.8640, 2.1034) -- (0.1370, 0.9170, 2.1069) -- (0.0900, 0.9170, 2.1006) -- cycle;
\fill[blue!15.0, opacity=0.7] (0.0900, 0.9170, 2.1006) -- (0.1370, 0.9170, 2.1069) -- (0.1370, 0.9700, 2.1102) -- (0.0900, 0.9700, 2.1039) -- cycle;
\fill[blue!15.0, opacity=0.7] (0.0900, 0.9700, 2.1039) -- (0.1370, 0.9700, 2.1102) -- (0.1370, 1.0230, 2.1132) -- (0.0900, 1.0230, 2.1069) -- cycle;
\fill[blue!15.0, opacity=0.7] (0.0900, 1.0230, 2.1069) -- (0.1370, 1.0230, 2.1132) -- (0.1370, 1.0760, 2.1159) -- (0.0900, 1.0760, 2.1096) -- cycle;
\fill[blue!15.0, opacity=0.7] (0.0900, 1.0760, 2.1096) -- (0.1370, 1.0760, 2.1159) -- (0.1370, 1.1290, 2.1183) -- (0.0900, 1.1290, 2.1120) -- cycle;
\fill[blue!15.0, opacity=0.7] (0.0900, 1.1290, 2.1120) -- (0.1370, 1.1290, 2.1183) -- (0.1370, 1.1820, 2.1204) -- (0.0900, 1.1820, 2.1141) -- cycle;
\fill[blue!15.1, opacity=0.7] (0.0900, 1.1820, 2.1141) -- (0.1370, 1.1820, 2.1204) -- (0.1370, 1.2350, 2.1222) -- (0.0900, 1.2350, 2.1159) -- cycle;
\fill[blue!15.2, opacity=0.7] (0.0900, 1.2350, 2.1159) -- (0.1370, 1.2350, 2.1222) -- (0.1370, 1.2880, 2.1237) -- (0.0900, 1.2880, 2.1174) -- cycle;
\fill[blue!15.4, opacity=0.7] (0.0900, 1.2880, 2.1174) -- (0.1370, 1.2880, 2.1237) -- (0.1370, 1.3410, 2.1248) -- (0.0900, 1.3410, 2.1185) -- cycle;
\fill[blue!15.7, opacity=0.7] (0.0900, 1.3410, 2.1185) -- (0.1370, 1.3410, 2.1248) -- (0.1370, 1.3940, 2.1256) -- (0.0900, 1.3940, 2.1193) -- cycle;
\fill[blue!16.3, opacity=0.7] (0.0900, 1.3940, 2.1193) -- (0.1370, 1.3940, 2.1256) -- (0.1370, 1.4470, 2.1261) -- (0.0900, 1.4470, 2.1198) -- cycle;
\fill[blue!16.9, opacity=0.7] (0.0900, 1.4470, 2.1198) -- (0.1370, 1.4470, 2.1261) -- (0.1370, 1.5000, 2.1263) -- (0.0900, 1.5000, 2.1200) -- cycle;
\fill[blue!17.4, opacity=0.7] (0.0900, 1.5000, 2.1200) -- (0.1370, 1.5000, 2.1263) -- (0.1370, 1.5530, 2.1261) -- (0.0900, 1.5530, 2.1198) -- cycle;
\fill[blue!17.8, opacity=0.7] (0.0900, 1.5530, 2.1198) -- (0.1370, 1.5530, 2.1261) -- (0.1370, 1.6060, 2.1256) -- (0.0900, 1.6060, 2.1193) -- cycle;
\fill[blue!18.0, opacity=0.7] (0.0900, 1.6060, 2.1193) -- (0.1370, 1.6060, 2.1256) -- (0.1370, 1.6590, 2.1248) -- (0.0900, 1.6590, 2.1185) -- cycle;
\fill[blue!17.7, opacity=0.7] (0.0900, 1.6590, 2.1185) -- (0.1370, 1.6590, 2.1248) -- (0.1370, 1.7120, 2.1237) -- (0.0900, 1.7120, 2.1174) -- cycle;
\fill[blue!17.3, opacity=0.7] (0.0900, 1.7120, 2.1174) -- (0.1370, 1.7120, 2.1237) -- (0.1370, 1.7650, 2.1222) -- (0.0900, 1.7650, 2.1159) -- cycle;
\fill[blue!16.7, opacity=0.7] (0.0900, 1.7650, 2.1159) -- (0.1370, 1.7650, 2.1222) -- (0.1370, 1.8180, 2.1204) -- (0.0900, 1.8180, 2.1141) -- cycle;
\fill[blue!16.1, opacity=0.7] (0.0900, 1.8180, 2.1141) -- (0.1370, 1.8180, 2.1204) -- (0.1370, 1.8710, 2.1183) -- (0.0900, 1.8710, 2.1120) -- cycle;
\fill[blue!15.6, opacity=0.7] (0.0900, 1.8710, 2.1120) -- (0.1370, 1.8710, 2.1183) -- (0.1370, 1.9240, 2.1159) -- (0.0900, 1.9240, 2.1096) -- cycle;
\fill[blue!15.3, opacity=0.7] (0.0900, 1.9240, 2.1096) -- (0.1370, 1.9240, 2.1159) -- (0.1370, 1.9770, 2.1132) -- (0.0900, 1.9770, 2.1069) -- cycle;
\fill[blue!15.1, opacity=0.7] (0.0900, 1.9770, 2.1069) -- (0.1370, 1.9770, 2.1132) -- (0.1370, 2.0300, 2.1102) -- (0.0900, 2.0300, 2.1039) -- cycle;
\fill[blue!15.0, opacity=0.7] (0.0900, 2.0300, 2.1039) -- (0.1370, 2.0300, 2.1102) -- (0.1370, 2.0830, 2.1069) -- (0.0900, 2.0830, 2.1006) -- cycle;
\fill[blue!15.0, opacity=0.7] (0.0900, 2.0830, 2.1006) -- (0.1370, 2.0830, 2.1069) -- (0.1370, 2.1360, 2.1034) -- (0.0900, 2.1360, 2.0971) -- cycle;
\fill[blue!15.0, opacity=0.7] (0.0900, 2.1360, 2.0971) -- (0.1370, 2.1360, 2.1034) -- (0.1370, 2.1890, 2.0995) -- (0.0900, 2.1890, 2.0933) -- cycle;
\fill[blue!15.0, opacity=0.7] (0.0900, 2.1890, 2.0933) -- (0.1370, 2.1890, 2.0995) -- (0.1370, 2.2420, 2.0955) -- (0.0900, 2.2420, 2.0892) -- cycle;
\fill[blue!15.0, opacity=0.7] (0.0900, 2.2420, 2.0892) -- (0.1370, 2.2420, 2.0955) -- (0.1370, 2.2950, 2.0911) -- (0.0900, 2.2950, 2.0849) -- cycle;
\fill[blue!15.0, opacity=0.7] (0.0900, 2.2950, 2.0849) -- (0.1370, 2.2950, 2.0911) -- (0.1370, 2.3480, 2.0866) -- (0.0900, 2.3480, 2.0803) -- cycle;
\fill[blue!15.0, opacity=0.7] (0.0900, 2.3480, 2.0803) -- (0.1370, 2.3480, 2.0866) -- (0.1370, 2.4010, 2.0818) -- (0.0900, 2.4010, 2.0755) -- cycle;
\fill[blue!15.0, opacity=0.7] (0.0900, 2.4010, 2.0755) -- (0.1370, 2.4010, 2.0818) -- (0.1370, 2.4540, 2.0768) -- (0.0900, 2.4540, 2.0705) -- cycle;
\fill[blue!15.2, opacity=0.7] (0.0900, 2.4540, 2.0705) -- (0.1370, 2.4540, 2.0768) -- (0.1370, 2.5070, 2.0716) -- (0.0900, 2.5070, 2.0654) -- cycle;
\fill[blue!15.8, opacity=0.7] (0.0900, 2.5070, 2.0654) -- (0.1370, 2.5070, 2.0716) -- (0.1370, 2.5600, 2.0663) -- (0.0900, 2.5600, 2.0600) -- cycle;
\fill[blue!17.7, opacity=0.7] (0.0900, 2.5600, 2.0600) -- (0.1370, 2.5600, 2.0663) -- (0.1370, 2.6130, 2.0608) -- (0.0900, 2.6130, 2.0545) -- cycle;
\fill[blue!20.8, opacity=0.7] (0.0900, 2.6130, 2.0545) -- (0.1370, 2.6130, 2.0608) -- (0.1370, 2.6660, 2.0551) -- (0.0900, 2.6660, 2.0488) -- cycle;
\fill[blue!22.9, opacity=0.7] (0.0900, 2.6660, 2.0488) -- (0.1370, 2.6660, 2.0551) -- (0.1370, 2.7190, 2.0493) -- (0.0900, 2.7190, 2.0430) -- cycle;
\fill[blue!21.2, opacity=0.7] (0.0900, 2.7190, 2.0430) -- (0.1370, 2.7190, 2.0493) -- (0.1370, 2.7720, 2.0434) -- (0.0900, 2.7720, 2.0371) -- cycle;
\fill[blue!17.5, opacity=0.7] (0.0900, 2.7720, 2.0371) -- (0.1370, 2.7720, 2.0434) -- (0.1370, 2.8250, 2.0373) -- (0.0900, 2.8250, 2.0311) -- cycle;
\fill[blue!15.4, opacity=0.7] (0.0900, 2.8250, 2.0311) -- (0.1370, 2.8250, 2.0373) -- (0.1370, 2.8780, 2.0312) -- (0.0900, 2.8780, 2.0249) -- cycle;
\fill[blue!15.0, opacity=0.7] (0.0900, 2.8780, 2.0249) -- (0.1370, 2.8780, 2.0312) -- (0.1370, 2.9310, 2.0251) -- (0.0900, 2.9310, 2.0188) -- cycle;
\fill[blue!15.0, opacity=0.7] (0.0900, 2.9310, 2.0188) -- (0.1370, 2.9310, 2.0251) -- (0.1370, 2.9840, 2.0188) -- (0.0900, 2.9840, 2.0125) -- cycle;
\fill[blue!15.0, opacity=0.7] (0.0900, 2.9840, 2.0125) -- (0.1370, 2.9840, 2.0188) -- (0.1370, 3.0370, 2.0126) -- (0.0900, 3.0370, 2.0063) -- cycle;
\fill[blue!15.0, opacity=0.7] (0.0900, 3.0370, 2.0063) -- (0.1370, 3.0370, 2.0126) -- (0.1370, 3.0900, 2.0063) -- (0.0900, 3.0900, 2.0000) -- cycle;
\fill[blue!15.9, opacity=0.7] (0.1370, -0.0900, 2.0063) -- (0.1840, -0.0900, 2.0125) -- (0.1840, -0.0370, 2.0188) -- (0.1370, -0.0370, 2.0126) -- cycle;
\fill[blue!15.1, opacity=0.7] (0.1370, -0.0370, 2.0126) -- (0.1840, -0.0370, 2.0188) -- (0.1840, 0.0160, 2.0251) -- (0.1370, 0.0160, 2.0188) -- cycle;
\fill[blue!15.0, opacity=0.7] (0.1370, 0.0160, 2.0188) -- (0.1840, 0.0160, 2.0251) -- (0.1840, 0.0690, 2.0313) -- (0.1370, 0.0690, 2.0251) -- cycle;
\fill[blue!15.0, opacity=0.7] (0.1370, 0.0690, 2.0251) -- (0.1840, 0.0690, 2.0313) -- (0.1840, 0.1220, 2.0375) -- (0.1370, 0.1220, 2.0312) -- cycle;
\fill[blue!15.0, opacity=0.7] (0.1370, 0.1220, 2.0312) -- (0.1840, 0.1220, 2.0375) -- (0.1840, 0.1750, 2.0436) -- (0.1370, 0.1750, 2.0373) -- cycle;
\fill[blue!15.0, opacity=0.7] (0.1370, 0.1750, 2.0373) -- (0.1840, 0.1750, 2.0436) -- (0.1840, 0.2280, 2.0496) -- (0.1370, 0.2280, 2.0434) -- cycle;
\fill[blue!15.0, opacity=0.7] (0.1370, 0.2280, 2.0434) -- (0.1840, 0.2280, 2.0496) -- (0.1840, 0.2810, 2.0555) -- (0.1370, 0.2810, 2.0493) -- cycle;
\fill[blue!15.0, opacity=0.7] (0.1370, 0.2810, 2.0493) -- (0.1840, 0.2810, 2.0555) -- (0.1840, 0.3340, 2.0614) -- (0.1370, 0.3340, 2.0551) -- cycle;
\fill[blue!15.6, opacity=0.7] (0.1370, 0.3340, 2.0551) -- (0.1840, 0.3340, 2.0614) -- (0.1840, 0.3870, 2.0670) -- (0.1370, 0.3870, 2.0608) -- cycle;
\fill[blue!19.9, opacity=0.7] (0.1370, 0.3870, 2.0608) -- (0.1840, 0.3870, 2.0670) -- (0.1840, 0.4400, 2.0725) -- (0.1370, 0.4400, 2.0663) -- cycle;
\fill[blue!29.1, opacity=0.7] (0.1370, 0.4400, 2.0663) -- (0.1840, 0.4400, 2.0725) -- (0.1840, 0.4930, 2.0779) -- (0.1370, 0.4930, 2.0716) -- cycle;
\fill[blue!34.7, opacity=0.7] (0.1370, 0.4930, 2.0716) -- (0.1840, 0.4930, 2.0779) -- (0.1840, 0.5460, 2.0831) -- (0.1370, 0.5460, 2.0768) -- cycle;
\fill[blue!31.3, opacity=0.7] (0.1370, 0.5460, 2.0768) -- (0.1840, 0.5460, 2.0831) -- (0.1840, 0.5990, 2.0881) -- (0.1370, 0.5990, 2.0818) -- cycle;
\fill[blue!23.1, opacity=0.7] (0.1370, 0.5990, 2.0818) -- (0.1840, 0.5990, 2.0881) -- (0.1840, 0.6520, 2.0928) -- (0.1370, 0.6520, 2.0866) -- cycle;
\fill[blue!17.5, opacity=0.7] (0.1370, 0.6520, 2.0866) -- (0.1840, 0.6520, 2.0928) -- (0.1840, 0.7050, 2.0974) -- (0.1370, 0.7050, 2.0911) -- cycle;
\fill[blue!15.5, opacity=0.7] (0.1370, 0.7050, 2.0911) -- (0.1840, 0.7050, 2.0974) -- (0.1840, 0.7580, 2.1017) -- (0.1370, 0.7580, 2.0955) -- cycle;
\fill[blue!15.1, opacity=0.7] (0.1370, 0.7580, 2.0955) -- (0.1840, 0.7580, 2.1017) -- (0.1840, 0.8110, 2.1058) -- (0.1370, 0.8110, 2.0995) -- cycle;
\fill[blue!15.0, opacity=0.7] (0.1370, 0.8110, 2.0995) -- (0.1840, 0.8110, 2.1058) -- (0.1840, 0.8640, 2.1096) -- (0.1370, 0.8640, 2.1034) -- cycle;
\fill[blue!15.0, opacity=0.7] (0.1370, 0.8640, 2.1034) -- (0.1840, 0.8640, 2.1096) -- (0.1840, 0.9170, 2.1132) -- (0.1370, 0.9170, 2.1069) -- cycle;
\fill[blue!15.0, opacity=0.7] (0.1370, 0.9170, 2.1069) -- (0.1840, 0.9170, 2.1132) -- (0.1840, 0.9700, 2.1165) -- (0.1370, 0.9700, 2.1102) -- cycle;
\fill[blue!15.0, opacity=0.7] (0.1370, 0.9700, 2.1102) -- (0.1840, 0.9700, 2.1165) -- (0.1840, 1.0230, 2.1195) -- (0.1370, 1.0230, 2.1132) -- cycle;
\fill[blue!15.1, opacity=0.7] (0.1370, 1.0230, 2.1132) -- (0.1840, 1.0230, 2.1195) -- (0.1840, 1.0760, 2.1222) -- (0.1370, 1.0760, 2.1159) -- cycle;
\fill[blue!15.3, opacity=0.7] (0.1370, 1.0760, 2.1159) -- (0.1840, 1.0760, 2.1222) -- (0.1840, 1.1290, 2.1246) -- (0.1370, 1.1290, 2.1183) -- cycle;
\fill[blue!15.9, opacity=0.7] (0.1370, 1.1290, 2.1183) -- (0.1840, 1.1290, 2.1246) -- (0.1840, 1.1820, 2.1267) -- (0.1370, 1.1820, 2.1204) -- cycle;
\fill[blue!17.4, opacity=0.7] (0.1370, 1.1820, 2.1204) -- (0.1840, 1.1820, 2.1267) -- (0.1840, 1.2350, 2.1285) -- (0.1370, 1.2350, 2.1222) -- cycle;
\fill[blue!20.2, opacity=0.7] (0.1370, 1.2350, 2.1222) -- (0.1840, 1.2350, 2.1285) -- (0.1840, 1.2880, 2.1299) -- (0.1370, 1.2880, 2.1237) -- cycle;
\fill[blue!24.6, opacity=0.7] (0.1370, 1.2880, 2.1237) -- (0.1840, 1.2880, 2.1299) -- (0.1840, 1.3410, 2.1311) -- (0.1370, 1.3410, 2.1248) -- cycle;
\fill[blue!29.9, opacity=0.7] (0.1370, 1.3410, 2.1248) -- (0.1840, 1.3410, 2.1311) -- (0.1840, 1.3940, 2.1319) -- (0.1370, 1.3940, 2.1256) -- cycle;
\fill[blue!35.2, opacity=0.7] (0.1370, 1.3940, 2.1256) -- (0.1840, 1.3940, 2.1319) -- (0.1840, 1.4470, 2.1324) -- (0.1370, 1.4470, 2.1261) -- cycle;
\fill[blue!39.7, opacity=0.7] (0.1370, 1.4470, 2.1261) -- (0.1840, 1.4470, 2.1324) -- (0.1840, 1.5000, 2.1325) -- (0.1370, 1.5000, 2.1263) -- cycle;
\fill[blue!43.0, opacity=0.7] (0.1370, 1.5000, 2.1263) -- (0.1840, 1.5000, 2.1325) -- (0.1840, 1.5530, 2.1324) -- (0.1370, 1.5530, 2.1261) -- cycle;
\fill[blue!44.8, opacity=0.7] (0.1370, 1.5530, 2.1261) -- (0.1840, 1.5530, 2.1324) -- (0.1840, 1.6060, 2.1319) -- (0.1370, 1.6060, 2.1256) -- cycle;
\fill[blue!45.2, opacity=0.7] (0.1370, 1.6060, 2.1256) -- (0.1840, 1.6060, 2.1319) -- (0.1840, 1.6590, 2.1311) -- (0.1370, 1.6590, 2.1248) -- cycle;
\fill[blue!44.1, opacity=0.7] (0.1370, 1.6590, 2.1248) -- (0.1840, 1.6590, 2.1311) -- (0.1840, 1.7120, 2.1299) -- (0.1370, 1.7120, 2.1237) -- cycle;
\fill[blue!41.7, opacity=0.7] (0.1370, 1.7120, 2.1237) -- (0.1840, 1.7120, 2.1299) -- (0.1840, 1.7650, 2.1285) -- (0.1370, 1.7650, 2.1222) -- cycle;
\fill[blue!38.0, opacity=0.7] (0.1370, 1.7650, 2.1222) -- (0.1840, 1.7650, 2.1285) -- (0.1840, 1.8180, 2.1267) -- (0.1370, 1.8180, 2.1204) -- cycle;
\fill[blue!33.3, opacity=0.7] (0.1370, 1.8180, 2.1204) -- (0.1840, 1.8180, 2.1267) -- (0.1840, 1.8710, 2.1246) -- (0.1370, 1.8710, 2.1183) -- cycle;
\fill[blue!28.1, opacity=0.7] (0.1370, 1.8710, 2.1183) -- (0.1840, 1.8710, 2.1246) -- (0.1840, 1.9240, 2.1222) -- (0.1370, 1.9240, 2.1159) -- cycle;
\fill[blue!23.2, opacity=0.7] (0.1370, 1.9240, 2.1159) -- (0.1840, 1.9240, 2.1222) -- (0.1840, 1.9770, 2.1195) -- (0.1370, 1.9770, 2.1132) -- cycle;
\fill[blue!19.3, opacity=0.7] (0.1370, 1.9770, 2.1132) -- (0.1840, 1.9770, 2.1195) -- (0.1840, 2.0300, 2.1165) -- (0.1370, 2.0300, 2.1102) -- cycle;
\fill[blue!16.9, opacity=0.7] (0.1370, 2.0300, 2.1102) -- (0.1840, 2.0300, 2.1165) -- (0.1840, 2.0830, 2.1132) -- (0.1370, 2.0830, 2.1069) -- cycle;
\fill[blue!15.7, opacity=0.7] (0.1370, 2.0830, 2.1069) -- (0.1840, 2.0830, 2.1132) -- (0.1840, 2.1360, 2.1096) -- (0.1370, 2.1360, 2.1034) -- cycle;
\fill[blue!15.2, opacity=0.7] (0.1370, 2.1360, 2.1034) -- (0.1840, 2.1360, 2.1096) -- (0.1840, 2.1890, 2.1058) -- (0.1370, 2.1890, 2.0995) -- cycle;
\fill[blue!15.0, opacity=0.7] (0.1370, 2.1890, 2.0995) -- (0.1840, 2.1890, 2.1058) -- (0.1840, 2.2420, 2.1017) -- (0.1370, 2.2420, 2.0955) -- cycle;
\fill[blue!15.0, opacity=0.7] (0.1370, 2.2420, 2.0955) -- (0.1840, 2.2420, 2.1017) -- (0.1840, 2.2950, 2.0974) -- (0.1370, 2.2950, 2.0911) -- cycle;
\fill[blue!15.0, opacity=0.7] (0.1370, 2.2950, 2.0911) -- (0.1840, 2.2950, 2.0974) -- (0.1840, 2.3480, 2.0928) -- (0.1370, 2.3480, 2.0866) -- cycle;
\fill[blue!15.0, opacity=0.7] (0.1370, 2.3480, 2.0866) -- (0.1840, 2.3480, 2.0928) -- (0.1840, 2.4010, 2.0881) -- (0.1370, 2.4010, 2.0818) -- cycle;
\fill[blue!15.0, opacity=0.7] (0.1370, 2.4010, 2.0818) -- (0.1840, 2.4010, 2.0881) -- (0.1840, 2.4540, 2.0831) -- (0.1370, 2.4540, 2.0768) -- cycle;
\fill[blue!15.0, opacity=0.7] (0.1370, 2.4540, 2.0768) -- (0.1840, 2.4540, 2.0831) -- (0.1840, 2.5070, 2.0779) -- (0.1370, 2.5070, 2.0716) -- cycle;
\fill[blue!15.0, opacity=0.7] (0.1370, 2.5070, 2.0716) -- (0.1840, 2.5070, 2.0779) -- (0.1840, 2.5600, 2.0725) -- (0.1370, 2.5600, 2.0663) -- cycle;
\fill[blue!15.2, opacity=0.7] (0.1370, 2.5600, 2.0663) -- (0.1840, 2.5600, 2.0725) -- (0.1840, 2.6130, 2.0670) -- (0.1370, 2.6130, 2.0608) -- cycle;
\fill[blue!16.2, opacity=0.7] (0.1370, 2.6130, 2.0608) -- (0.1840, 2.6130, 2.0670) -- (0.1840, 2.6660, 2.0614) -- (0.1370, 2.6660, 2.0551) -- cycle;
\fill[blue!18.8, opacity=0.7] (0.1370, 2.6660, 2.0551) -- (0.1840, 2.6660, 2.0614) -- (0.1840, 2.7190, 2.0555) -- (0.1370, 2.7190, 2.0493) -- cycle;
\fill[blue!21.9, opacity=0.7] (0.1370, 2.7190, 2.0493) -- (0.1840, 2.7190, 2.0555) -- (0.1840, 2.7720, 2.0496) -- (0.1370, 2.7720, 2.0434) -- cycle;
\fill[blue!22.0, opacity=0.7] (0.1370, 2.7720, 2.0434) -- (0.1840, 2.7720, 2.0496) -- (0.1840, 2.8250, 2.0436) -- (0.1370, 2.8250, 2.0373) -- cycle;
\fill[blue!18.5, opacity=0.7] (0.1370, 2.8250, 2.0373) -- (0.1840, 2.8250, 2.0436) -- (0.1840, 2.8780, 2.0375) -- (0.1370, 2.8780, 2.0312) -- cycle;
\fill[blue!15.7, opacity=0.7] (0.1370, 2.8780, 2.0312) -- (0.1840, 2.8780, 2.0375) -- (0.1840, 2.9310, 2.0313) -- (0.1370, 2.9310, 2.0251) -- cycle;
\fill[blue!15.0, opacity=0.7] (0.1370, 2.9310, 2.0251) -- (0.1840, 2.9310, 2.0313) -- (0.1840, 2.9840, 2.0251) -- (0.1370, 2.9840, 2.0188) -- cycle;
\fill[blue!15.0, opacity=0.7] (0.1370, 2.9840, 2.0188) -- (0.1840, 2.9840, 2.0251) -- (0.1840, 3.0370, 2.0188) -- (0.1370, 3.0370, 2.0126) -- cycle;
\fill[blue!15.0, opacity=0.7] (0.1370, 3.0370, 2.0126) -- (0.1840, 3.0370, 2.0188) -- (0.1840, 3.0900, 2.0125) -- (0.1370, 3.0900, 2.0063) -- cycle;
\fill[blue!15.2, opacity=0.7] (0.1840, -0.0900, 2.0125) -- (0.2310, -0.0900, 2.0188) -- (0.2310, -0.0370, 2.0251) -- (0.1840, -0.0370, 2.0188) -- cycle;
\fill[blue!15.0, opacity=0.7] (0.1840, -0.0370, 2.0188) -- (0.2310, -0.0370, 2.0251) -- (0.2310, 0.0160, 2.0313) -- (0.1840, 0.0160, 2.0251) -- cycle;
\fill[blue!15.0, opacity=0.7] (0.1840, 0.0160, 2.0251) -- (0.2310, 0.0160, 2.0313) -- (0.2310, 0.0690, 2.0375) -- (0.1840, 0.0690, 2.0313) -- cycle;
\fill[blue!15.0, opacity=0.7] (0.1840, 0.0690, 2.0313) -- (0.2310, 0.0690, 2.0375) -- (0.2310, 0.1220, 2.0437) -- (0.1840, 0.1220, 2.0375) -- cycle;
\fill[blue!15.0, opacity=0.7] (0.1840, 0.1220, 2.0375) -- (0.2310, 0.1220, 2.0437) -- (0.2310, 0.1750, 2.0498) -- (0.1840, 0.1750, 2.0436) -- cycle;
\fill[blue!15.0, opacity=0.7] (0.1840, 0.1750, 2.0436) -- (0.2310, 0.1750, 2.0498) -- (0.2310, 0.2280, 2.0559) -- (0.1840, 0.2280, 2.0496) -- cycle;
\fill[blue!15.0, opacity=0.7] (0.1840, 0.2280, 2.0496) -- (0.2310, 0.2280, 2.0559) -- (0.2310, 0.2810, 2.0618) -- (0.1840, 0.2810, 2.0555) -- cycle;
\fill[blue!15.9, opacity=0.7] (0.1840, 0.2810, 2.0555) -- (0.2310, 0.2810, 2.0618) -- (0.2310, 0.3340, 2.0676) -- (0.1840, 0.3340, 2.0614) -- cycle;
\fill[blue!21.2, opacity=0.7] (0.1840, 0.3340, 2.0614) -- (0.2310, 0.3340, 2.0676) -- (0.2310, 0.3870, 2.0733) -- (0.1840, 0.3870, 2.0670) -- cycle;
\fill[blue!31.3, opacity=0.7] (0.1840, 0.3870, 2.0670) -- (0.2310, 0.3870, 2.0733) -- (0.2310, 0.4400, 2.0788) -- (0.1840, 0.4400, 2.0725) -- cycle;
\fill[blue!35.3, opacity=0.7] (0.1840, 0.4400, 2.0725) -- (0.2310, 0.4400, 2.0788) -- (0.2310, 0.4930, 2.0841) -- (0.1840, 0.4930, 2.0779) -- cycle;
\fill[blue!29.3, opacity=0.7] (0.1840, 0.4930, 2.0779) -- (0.2310, 0.4930, 2.0841) -- (0.2310, 0.5460, 2.0893) -- (0.1840, 0.5460, 2.0831) -- cycle;
\fill[blue!20.6, opacity=0.7] (0.1840, 0.5460, 2.0831) -- (0.2310, 0.5460, 2.0893) -- (0.2310, 0.5990, 2.0943) -- (0.1840, 0.5990, 2.0881) -- cycle;
\fill[blue!16.3, opacity=0.7] (0.1840, 0.5990, 2.0881) -- (0.2310, 0.5990, 2.0943) -- (0.2310, 0.6520, 2.0991) -- (0.1840, 0.6520, 2.0928) -- cycle;
\fill[blue!15.2, opacity=0.7] (0.1840, 0.6520, 2.0928) -- (0.2310, 0.6520, 2.0991) -- (0.2310, 0.7050, 2.1036) -- (0.1840, 0.7050, 2.0974) -- cycle;
\fill[blue!15.0, opacity=0.7] (0.1840, 0.7050, 2.0974) -- (0.2310, 0.7050, 2.1036) -- (0.2310, 0.7580, 2.1079) -- (0.1840, 0.7580, 2.1017) -- cycle;
\fill[blue!15.0, opacity=0.7] (0.1840, 0.7580, 2.1017) -- (0.2310, 0.7580, 2.1079) -- (0.2310, 0.8110, 2.1120) -- (0.1840, 0.8110, 2.1058) -- cycle;
\fill[blue!15.0, opacity=0.7] (0.1840, 0.8110, 2.1058) -- (0.2310, 0.8110, 2.1120) -- (0.2310, 0.8640, 2.1159) -- (0.1840, 0.8640, 2.1096) -- cycle;
\fill[blue!15.0, opacity=0.7] (0.1840, 0.8640, 2.1096) -- (0.2310, 0.8640, 2.1159) -- (0.2310, 0.9170, 2.1194) -- (0.1840, 0.9170, 2.1132) -- cycle;
\fill[blue!15.1, opacity=0.7] (0.1840, 0.9170, 2.1132) -- (0.2310, 0.9170, 2.1194) -- (0.2310, 0.9700, 2.1227) -- (0.1840, 0.9700, 2.1165) -- cycle;
\fill[blue!15.6, opacity=0.7] (0.1840, 0.9700, 2.1165) -- (0.2310, 0.9700, 2.1227) -- (0.2310, 1.0230, 2.1257) -- (0.1840, 1.0230, 2.1195) -- cycle;
\fill[blue!17.3, opacity=0.7] (0.1840, 1.0230, 2.1195) -- (0.2310, 1.0230, 2.1257) -- (0.2310, 1.0760, 2.1284) -- (0.1840, 1.0760, 2.1222) -- cycle;
\fill[blue!21.8, opacity=0.7] (0.1840, 1.0760, 2.1222) -- (0.2310, 1.0760, 2.1284) -- (0.2310, 1.1290, 2.1308) -- (0.1840, 1.1290, 2.1246) -- cycle;
\fill[blue!30.3, opacity=0.7] (0.1840, 1.1290, 2.1246) -- (0.2310, 1.1290, 2.1308) -- (0.2310, 1.1820, 2.1329) -- (0.1840, 1.1820, 2.1267) -- cycle;
\fill[blue!41.6, opacity=0.7] (0.1840, 1.1820, 2.1267) -- (0.2310, 1.1820, 2.1329) -- (0.2310, 1.2350, 2.1347) -- (0.1840, 1.2350, 2.1285) -- cycle;
\fill[blue!52.6, opacity=0.7] (0.1840, 1.2350, 2.1285) -- (0.2310, 1.2350, 2.1347) -- (0.2310, 1.2880, 2.1361) -- (0.1840, 1.2880, 2.1299) -- cycle;
\fill[blue!61.4, opacity=0.7] (0.1840, 1.2880, 2.1299) -- (0.2310, 1.2880, 2.1361) -- (0.2310, 1.3410, 2.1373) -- (0.1840, 1.3410, 2.1311) -- cycle;
\fill[blue!67.3, opacity=0.7] (0.1840, 1.3410, 2.1311) -- (0.2310, 1.3410, 2.1373) -- (0.2310, 1.3940, 2.1381) -- (0.1840, 1.3940, 2.1319) -- cycle;
\fill[blue!71.0, opacity=0.7] (0.1840, 1.3940, 2.1319) -- (0.2310, 1.3940, 2.1381) -- (0.2310, 1.4470, 2.1386) -- (0.1840, 1.4470, 2.1324) -- cycle;
\fill[blue!72.9, opacity=0.7] (0.1840, 1.4470, 2.1324) -- (0.2310, 1.4470, 2.1386) -- (0.2310, 1.5000, 2.1388) -- (0.1840, 1.5000, 2.1325) -- cycle;
\fill[blue!73.8, opacity=0.7] (0.1840, 1.5000, 2.1325) -- (0.2310, 1.5000, 2.1388) -- (0.2310, 1.5530, 2.1386) -- (0.1840, 1.5530, 2.1324) -- cycle;
\fill[blue!74.0, opacity=0.7] (0.1840, 1.5530, 2.1324) -- (0.2310, 1.5530, 2.1386) -- (0.2310, 1.6060, 2.1381) -- (0.1840, 1.6060, 2.1319) -- cycle;
\fill[blue!73.7, opacity=0.7] (0.1840, 1.6060, 2.1319) -- (0.2310, 1.6060, 2.1381) -- (0.2310, 1.6590, 2.1373) -- (0.1840, 1.6590, 2.1311) -- cycle;
\fill[blue!73.0, opacity=0.7] (0.1840, 1.6590, 2.1311) -- (0.2310, 1.6590, 2.1373) -- (0.2310, 1.7120, 2.1361) -- (0.1840, 1.7120, 2.1299) -- cycle;
\fill[blue!71.8, opacity=0.7] (0.1840, 1.7120, 2.1299) -- (0.2310, 1.7120, 2.1361) -- (0.2310, 1.7650, 2.1347) -- (0.1840, 1.7650, 2.1285) -- cycle;
\fill[blue!70.0, opacity=0.7] (0.1840, 1.7650, 2.1285) -- (0.2310, 1.7650, 2.1347) -- (0.2310, 1.8180, 2.1329) -- (0.1840, 1.8180, 2.1267) -- cycle;
\fill[blue!67.1, opacity=0.7] (0.1840, 1.8180, 2.1267) -- (0.2310, 1.8180, 2.1329) -- (0.2310, 1.8710, 2.1308) -- (0.1840, 1.8710, 2.1246) -- cycle;
\fill[blue!62.8, opacity=0.7] (0.1840, 1.8710, 2.1246) -- (0.2310, 1.8710, 2.1308) -- (0.2310, 1.9240, 2.1284) -- (0.1840, 1.9240, 2.1222) -- cycle;
\fill[blue!56.4, opacity=0.7] (0.1840, 1.9240, 2.1222) -- (0.2310, 1.9240, 2.1284) -- (0.2310, 1.9770, 2.1257) -- (0.1840, 1.9770, 2.1195) -- cycle;
\fill[blue!47.8, opacity=0.7] (0.1840, 1.9770, 2.1195) -- (0.2310, 1.9770, 2.1257) -- (0.2310, 2.0300, 2.1227) -- (0.1840, 2.0300, 2.1165) -- cycle;
\fill[blue!37.6, opacity=0.7] (0.1840, 2.0300, 2.1165) -- (0.2310, 2.0300, 2.1227) -- (0.2310, 2.0830, 2.1194) -- (0.1840, 2.0830, 2.1132) -- cycle;
\fill[blue!27.7, opacity=0.7] (0.1840, 2.0830, 2.1132) -- (0.2310, 2.0830, 2.1194) -- (0.2310, 2.1360, 2.1159) -- (0.1840, 2.1360, 2.1096) -- cycle;
\fill[blue!20.5, opacity=0.7] (0.1840, 2.1360, 2.1096) -- (0.2310, 2.1360, 2.1159) -- (0.2310, 2.1890, 2.1120) -- (0.1840, 2.1890, 2.1058) -- cycle;
\fill[blue!16.7, opacity=0.7] (0.1840, 2.1890, 2.1058) -- (0.2310, 2.1890, 2.1120) -- (0.2310, 2.2420, 2.1079) -- (0.1840, 2.2420, 2.1017) -- cycle;
\fill[blue!15.4, opacity=0.7] (0.1840, 2.2420, 2.1017) -- (0.2310, 2.2420, 2.1079) -- (0.2310, 2.2950, 2.1036) -- (0.1840, 2.2950, 2.0974) -- cycle;
\fill[blue!15.1, opacity=0.7] (0.1840, 2.2950, 2.0974) -- (0.2310, 2.2950, 2.1036) -- (0.2310, 2.3480, 2.0991) -- (0.1840, 2.3480, 2.0928) -- cycle;
\fill[blue!15.0, opacity=0.7] (0.1840, 2.3480, 2.0928) -- (0.2310, 2.3480, 2.0991) -- (0.2310, 2.4010, 2.0943) -- (0.1840, 2.4010, 2.0881) -- cycle;
\fill[blue!15.0, opacity=0.7] (0.1840, 2.4010, 2.0881) -- (0.2310, 2.4010, 2.0943) -- (0.2310, 2.4540, 2.0893) -- (0.1840, 2.4540, 2.0831) -- cycle;
\fill[blue!15.0, opacity=0.7] (0.1840, 2.4540, 2.0831) -- (0.2310, 2.4540, 2.0893) -- (0.2310, 2.5070, 2.0841) -- (0.1840, 2.5070, 2.0779) -- cycle;
\fill[blue!15.0, opacity=0.7] (0.1840, 2.5070, 2.0779) -- (0.2310, 2.5070, 2.0841) -- (0.2310, 2.5600, 2.0788) -- (0.1840, 2.5600, 2.0725) -- cycle;
\fill[blue!15.0, opacity=0.7] (0.1840, 2.5600, 2.0725) -- (0.2310, 2.5600, 2.0788) -- (0.2310, 2.6130, 2.0733) -- (0.1840, 2.6130, 2.0670) -- cycle;
\fill[blue!15.1, opacity=0.7] (0.1840, 2.6130, 2.0670) -- (0.2310, 2.6130, 2.0733) -- (0.2310, 2.6660, 2.0676) -- (0.1840, 2.6660, 2.0614) -- cycle;
\fill[blue!15.5, opacity=0.7] (0.1840, 2.6660, 2.0614) -- (0.2310, 2.6660, 2.0676) -- (0.2310, 2.7190, 2.0618) -- (0.1840, 2.7190, 2.0555) -- cycle;
\fill[blue!17.3, opacity=0.7] (0.1840, 2.7190, 2.0555) -- (0.2310, 2.7190, 2.0618) -- (0.2310, 2.7720, 2.0559) -- (0.1840, 2.7720, 2.0496) -- cycle;
\fill[blue!20.7, opacity=0.7] (0.1840, 2.7720, 2.0496) -- (0.2310, 2.7720, 2.0559) -- (0.2310, 2.8250, 2.0498) -- (0.1840, 2.8250, 2.0436) -- cycle;
\fill[blue!22.2, opacity=0.7] (0.1840, 2.8250, 2.0436) -- (0.2310, 2.8250, 2.0498) -- (0.2310, 2.8780, 2.0437) -- (0.1840, 2.8780, 2.0375) -- cycle;
\fill[blue!19.4, opacity=0.7] (0.1840, 2.8780, 2.0375) -- (0.2310, 2.8780, 2.0437) -- (0.2310, 2.9310, 2.0375) -- (0.1840, 2.9310, 2.0313) -- cycle;
\fill[blue!16.0, opacity=0.7] (0.1840, 2.9310, 2.0313) -- (0.2310, 2.9310, 2.0375) -- (0.2310, 2.9840, 2.0313) -- (0.1840, 2.9840, 2.0251) -- cycle;
\fill[blue!15.1, opacity=0.7] (0.1840, 2.9840, 2.0251) -- (0.2310, 2.9840, 2.0313) -- (0.2310, 3.0370, 2.0251) -- (0.1840, 3.0370, 2.0188) -- cycle;
\fill[blue!15.0, opacity=0.7] (0.1840, 3.0370, 2.0188) -- (0.2310, 3.0370, 2.0251) -- (0.2310, 3.0900, 2.0188) -- (0.1840, 3.0900, 2.0125) -- cycle;
\fill[blue!15.0, opacity=0.7] (0.2310, -0.0900, 2.0188) -- (0.2780, -0.0900, 2.0249) -- (0.2780, -0.0370, 2.0312) -- (0.2310, -0.0370, 2.0251) -- cycle;
\fill[blue!15.0, opacity=0.7] (0.2310, -0.0370, 2.0251) -- (0.2780, -0.0370, 2.0312) -- (0.2780, 0.0160, 2.0375) -- (0.2310, 0.0160, 2.0313) -- cycle;
\fill[blue!15.0, opacity=0.7] (0.2310, 0.0160, 2.0313) -- (0.2780, 0.0160, 2.0375) -- (0.2780, 0.0690, 2.0437) -- (0.2310, 0.0690, 2.0375) -- cycle;
\fill[blue!15.0, opacity=0.7] (0.2310, 0.0690, 2.0375) -- (0.2780, 0.0690, 2.0437) -- (0.2780, 0.1220, 2.0499) -- (0.2310, 0.1220, 2.0437) -- cycle;
\fill[blue!15.0, opacity=0.7] (0.2310, 0.1220, 2.0437) -- (0.2780, 0.1220, 2.0499) -- (0.2780, 0.1750, 2.0560) -- (0.2310, 0.1750, 2.0498) -- cycle;
\fill[blue!15.0, opacity=0.7] (0.2310, 0.1750, 2.0498) -- (0.2780, 0.1750, 2.0560) -- (0.2780, 0.2280, 2.0620) -- (0.2310, 0.2280, 2.0559) -- cycle;
\fill[blue!15.9, opacity=0.7] (0.2310, 0.2280, 2.0559) -- (0.2780, 0.2280, 2.0620) -- (0.2780, 0.2810, 2.0680) -- (0.2310, 0.2810, 2.0618) -- cycle;
\fill[blue!22.0, opacity=0.7] (0.2310, 0.2810, 2.0618) -- (0.2780, 0.2810, 2.0680) -- (0.2780, 0.3340, 2.0738) -- (0.2310, 0.3340, 2.0676) -- cycle;
\fill[blue!32.7, opacity=0.7] (0.2310, 0.3340, 2.0676) -- (0.2780, 0.3340, 2.0738) -- (0.2780, 0.3870, 2.0794) -- (0.2310, 0.3870, 2.0733) -- cycle;
\fill[blue!35.5, opacity=0.7] (0.2310, 0.3870, 2.0733) -- (0.2780, 0.3870, 2.0794) -- (0.2780, 0.4400, 2.0849) -- (0.2310, 0.4400, 2.0788) -- cycle;
\fill[blue!27.8, opacity=0.7] (0.2310, 0.4400, 2.0788) -- (0.2780, 0.4400, 2.0849) -- (0.2780, 0.4930, 2.0903) -- (0.2310, 0.4930, 2.0841) -- cycle;
\fill[blue!19.2, opacity=0.7] (0.2310, 0.4930, 2.0841) -- (0.2780, 0.4930, 2.0903) -- (0.2780, 0.5460, 2.0955) -- (0.2310, 0.5460, 2.0893) -- cycle;
\fill[blue!15.8, opacity=0.7] (0.2310, 0.5460, 2.0893) -- (0.2780, 0.5460, 2.0955) -- (0.2780, 0.5990, 2.1005) -- (0.2310, 0.5990, 2.0943) -- cycle;
\fill[blue!15.1, opacity=0.7] (0.2310, 0.5990, 2.0943) -- (0.2780, 0.5990, 2.1005) -- (0.2780, 0.6520, 2.1052) -- (0.2310, 0.6520, 2.0991) -- cycle;
\fill[blue!15.0, opacity=0.7] (0.2310, 0.6520, 2.0991) -- (0.2780, 0.6520, 2.1052) -- (0.2780, 0.7050, 2.1098) -- (0.2310, 0.7050, 2.1036) -- cycle;
\fill[blue!15.0, opacity=0.7] (0.2310, 0.7050, 2.1036) -- (0.2780, 0.7050, 2.1098) -- (0.2780, 0.7580, 2.1141) -- (0.2310, 0.7580, 2.1079) -- cycle;
\fill[blue!15.0, opacity=0.7] (0.2310, 0.7580, 2.1079) -- (0.2780, 0.7580, 2.1141) -- (0.2780, 0.8110, 2.1182) -- (0.2310, 0.8110, 2.1120) -- cycle;
\fill[blue!15.1, opacity=0.7] (0.2310, 0.8110, 2.1120) -- (0.2780, 0.8110, 2.1182) -- (0.2780, 0.8640, 2.1220) -- (0.2310, 0.8640, 2.1159) -- cycle;
\fill[blue!15.6, opacity=0.7] (0.2310, 0.8640, 2.1159) -- (0.2780, 0.8640, 2.1220) -- (0.2780, 0.9170, 2.1256) -- (0.2310, 0.9170, 2.1194) -- cycle;
\fill[blue!18.1, opacity=0.7] (0.2310, 0.9170, 2.1194) -- (0.2780, 0.9170, 2.1256) -- (0.2780, 0.9700, 2.1289) -- (0.2310, 0.9700, 2.1227) -- cycle;
\fill[blue!25.7, opacity=0.7] (0.2310, 0.9700, 2.1227) -- (0.2780, 0.9700, 2.1289) -- (0.2780, 1.0230, 2.1319) -- (0.2310, 1.0230, 2.1257) -- cycle;
\fill[blue!39.6, opacity=0.7] (0.2310, 1.0230, 2.1257) -- (0.2780, 1.0230, 2.1319) -- (0.2780, 1.0760, 2.1346) -- (0.2310, 1.0760, 2.1284) -- cycle;
\fill[blue!55.5, opacity=0.7] (0.2310, 1.0760, 2.1284) -- (0.2780, 1.0760, 2.1346) -- (0.2780, 1.1290, 2.1370) -- (0.2310, 1.1290, 2.1308) -- cycle;
\fill[blue!67.8, opacity=0.7] (0.2310, 1.1290, 2.1308) -- (0.2780, 1.1290, 2.1370) -- (0.2780, 1.1820, 2.1391) -- (0.2310, 1.1820, 2.1329) -- cycle;
\fill[blue!74.8, opacity=0.7] (0.2310, 1.1820, 2.1329) -- (0.2780, 1.1820, 2.1391) -- (0.2780, 1.2350, 2.1409) -- (0.2310, 1.2350, 2.1347) -- cycle;
\fill[blue!77.9, opacity=0.7] (0.2310, 1.2350, 2.1347) -- (0.2780, 1.2350, 2.1409) -- (0.2780, 1.2880, 2.1423) -- (0.2310, 1.2880, 2.1361) -- cycle;
\fill[blue!78.5, opacity=0.7] (0.2310, 1.2880, 2.1361) -- (0.2780, 1.2880, 2.1423) -- (0.2780, 1.3410, 2.1435) -- (0.2310, 1.3410, 2.1373) -- cycle;
\fill[blue!77.6, opacity=0.7] (0.2310, 1.3410, 2.1373) -- (0.2780, 1.3410, 2.1435) -- (0.2780, 1.3940, 2.1443) -- (0.2310, 1.3940, 2.1381) -- cycle;
\fill[blue!75.6, opacity=0.7] (0.2310, 1.3940, 2.1381) -- (0.2780, 1.3940, 2.1443) -- (0.2780, 1.4470, 2.1448) -- (0.2310, 1.4470, 2.1386) -- cycle;
\fill[blue!73.2, opacity=0.7] (0.2310, 1.4470, 2.1386) -- (0.2780, 1.4470, 2.1448) -- (0.2780, 1.5000, 2.1449) -- (0.2310, 1.5000, 2.1388) -- cycle;
\fill[blue!70.9, opacity=0.7] (0.2310, 1.5000, 2.1388) -- (0.2780, 1.5000, 2.1449) -- (0.2780, 1.5530, 2.1448) -- (0.2310, 1.5530, 2.1386) -- cycle;
\fill[blue!69.0, opacity=0.7] (0.2310, 1.5530, 2.1386) -- (0.2780, 1.5530, 2.1448) -- (0.2780, 1.6060, 2.1443) -- (0.2310, 1.6060, 2.1381) -- cycle;
\fill[blue!67.8, opacity=0.7] (0.2310, 1.6060, 2.1381) -- (0.2780, 1.6060, 2.1443) -- (0.2780, 1.6590, 2.1435) -- (0.2310, 1.6590, 2.1373) -- cycle;
\fill[blue!67.5, opacity=0.7] (0.2310, 1.6590, 2.1373) -- (0.2780, 1.6590, 2.1435) -- (0.2780, 1.7120, 2.1423) -- (0.2310, 1.7120, 2.1361) -- cycle;
\fill[blue!67.9, opacity=0.7] (0.2310, 1.7120, 2.1361) -- (0.2780, 1.7120, 2.1423) -- (0.2780, 1.7650, 2.1409) -- (0.2310, 1.7650, 2.1347) -- cycle;
\fill[blue!69.0, opacity=0.7] (0.2310, 1.7650, 2.1347) -- (0.2780, 1.7650, 2.1409) -- (0.2780, 1.8180, 2.1391) -- (0.2310, 1.8180, 2.1329) -- cycle;
\fill[blue!70.3, opacity=0.7] (0.2310, 1.8180, 2.1329) -- (0.2780, 1.8180, 2.1391) -- (0.2780, 1.8710, 2.1370) -- (0.2310, 1.8710, 2.1308) -- cycle;
\fill[blue!71.5, opacity=0.7] (0.2310, 1.8710, 2.1308) -- (0.2780, 1.8710, 2.1370) -- (0.2780, 1.9240, 2.1346) -- (0.2310, 1.9240, 2.1284) -- cycle;
\fill[blue!71.9, opacity=0.7] (0.2310, 1.9240, 2.1284) -- (0.2780, 1.9240, 2.1346) -- (0.2780, 1.9770, 2.1319) -- (0.2310, 1.9770, 2.1257) -- cycle;
\fill[blue!70.8, opacity=0.7] (0.2310, 1.9770, 2.1257) -- (0.2780, 1.9770, 2.1319) -- (0.2780, 2.0300, 2.1289) -- (0.2310, 2.0300, 2.1227) -- cycle;
\fill[blue!67.3, opacity=0.7] (0.2310, 2.0300, 2.1227) -- (0.2780, 2.0300, 2.1289) -- (0.2780, 2.0830, 2.1256) -- (0.2310, 2.0830, 2.1194) -- cycle;
\fill[blue!60.2, opacity=0.7] (0.2310, 2.0830, 2.1194) -- (0.2780, 2.0830, 2.1256) -- (0.2780, 2.1360, 2.1220) -- (0.2310, 2.1360, 2.1159) -- cycle;
\fill[blue!49.0, opacity=0.7] (0.2310, 2.1360, 2.1159) -- (0.2780, 2.1360, 2.1220) -- (0.2780, 2.1890, 2.1182) -- (0.2310, 2.1890, 2.1120) -- cycle;
\fill[blue!35.4, opacity=0.7] (0.2310, 2.1890, 2.1120) -- (0.2780, 2.1890, 2.1182) -- (0.2780, 2.2420, 2.1141) -- (0.2310, 2.2420, 2.1079) -- cycle;
\fill[blue!23.8, opacity=0.7] (0.2310, 2.2420, 2.1079) -- (0.2780, 2.2420, 2.1141) -- (0.2780, 2.2950, 2.1098) -- (0.2310, 2.2950, 2.1036) -- cycle;
\fill[blue!17.5, opacity=0.7] (0.2310, 2.2950, 2.1036) -- (0.2780, 2.2950, 2.1098) -- (0.2780, 2.3480, 2.1052) -- (0.2310, 2.3480, 2.0991) -- cycle;
\fill[blue!15.5, opacity=0.7] (0.2310, 2.3480, 2.0991) -- (0.2780, 2.3480, 2.1052) -- (0.2780, 2.4010, 2.1005) -- (0.2310, 2.4010, 2.0943) -- cycle;
\fill[blue!15.1, opacity=0.7] (0.2310, 2.4010, 2.0943) -- (0.2780, 2.4010, 2.1005) -- (0.2780, 2.4540, 2.0955) -- (0.2310, 2.4540, 2.0893) -- cycle;
\fill[blue!15.0, opacity=0.7] (0.2310, 2.4540, 2.0893) -- (0.2780, 2.4540, 2.0955) -- (0.2780, 2.5070, 2.0903) -- (0.2310, 2.5070, 2.0841) -- cycle;
\fill[blue!15.0, opacity=0.7] (0.2310, 2.5070, 2.0841) -- (0.2780, 2.5070, 2.0903) -- (0.2780, 2.5600, 2.0849) -- (0.2310, 2.5600, 2.0788) -- cycle;
\fill[blue!15.0, opacity=0.7] (0.2310, 2.5600, 2.0788) -- (0.2780, 2.5600, 2.0849) -- (0.2780, 2.6130, 2.0794) -- (0.2310, 2.6130, 2.0733) -- cycle;
\fill[blue!15.0, opacity=0.7] (0.2310, 2.6130, 2.0733) -- (0.2780, 2.6130, 2.0794) -- (0.2780, 2.6660, 2.0738) -- (0.2310, 2.6660, 2.0676) -- cycle;
\fill[blue!15.0, opacity=0.7] (0.2310, 2.6660, 2.0676) -- (0.2780, 2.6660, 2.0738) -- (0.2780, 2.7190, 2.0680) -- (0.2310, 2.7190, 2.0618) -- cycle;
\fill[blue!15.2, opacity=0.7] (0.2310, 2.7190, 2.0618) -- (0.2780, 2.7190, 2.0680) -- (0.2780, 2.7720, 2.0620) -- (0.2310, 2.7720, 2.0559) -- cycle;
\fill[blue!16.4, opacity=0.7] (0.2310, 2.7720, 2.0559) -- (0.2780, 2.7720, 2.0620) -- (0.2780, 2.8250, 2.0560) -- (0.2310, 2.8250, 2.0498) -- cycle;
\fill[blue!19.6, opacity=0.7] (0.2310, 2.8250, 2.0498) -- (0.2780, 2.8250, 2.0560) -- (0.2780, 2.8780, 2.0499) -- (0.2310, 2.8780, 2.0437) -- cycle;
\fill[blue!22.0, opacity=0.7] (0.2310, 2.8780, 2.0437) -- (0.2780, 2.8780, 2.0499) -- (0.2780, 2.9310, 2.0437) -- (0.2310, 2.9310, 2.0375) -- cycle;
\fill[blue!19.8, opacity=0.7] (0.2310, 2.9310, 2.0375) -- (0.2780, 2.9310, 2.0437) -- (0.2780, 2.9840, 2.0375) -- (0.2310, 2.9840, 2.0313) -- cycle;
\fill[blue!16.2, opacity=0.7] (0.2310, 2.9840, 2.0313) -- (0.2780, 2.9840, 2.0375) -- (0.2780, 3.0370, 2.0312) -- (0.2310, 3.0370, 2.0251) -- cycle;
\fill[blue!15.1, opacity=0.7] (0.2310, 3.0370, 2.0251) -- (0.2780, 3.0370, 2.0312) -- (0.2780, 3.0900, 2.0249) -- (0.2310, 3.0900, 2.0188) -- cycle;
\fill[blue!15.0, opacity=0.7] (0.2780, -0.0900, 2.0249) -- (0.3250, -0.0900, 2.0311) -- (0.3250, -0.0370, 2.0373) -- (0.2780, -0.0370, 2.0312) -- cycle;
\fill[blue!15.0, opacity=0.7] (0.2780, -0.0370, 2.0312) -- (0.3250, -0.0370, 2.0373) -- (0.3250, 0.0160, 2.0436) -- (0.2780, 0.0160, 2.0375) -- cycle;
\fill[blue!15.0, opacity=0.7] (0.2780, 0.0160, 2.0375) -- (0.3250, 0.0160, 2.0436) -- (0.3250, 0.0690, 2.0498) -- (0.2780, 0.0690, 2.0437) -- cycle;
\fill[blue!15.0, opacity=0.7] (0.2780, 0.0690, 2.0437) -- (0.3250, 0.0690, 2.0498) -- (0.3250, 0.1220, 2.0560) -- (0.2780, 0.1220, 2.0499) -- cycle;
\fill[blue!15.0, opacity=0.7] (0.2780, 0.1220, 2.0499) -- (0.3250, 0.1220, 2.0560) -- (0.3250, 0.1750, 2.0621) -- (0.2780, 0.1750, 2.0560) -- cycle;
\fill[blue!15.8, opacity=0.7] (0.2780, 0.1750, 2.0560) -- (0.3250, 0.1750, 2.0621) -- (0.3250, 0.2280, 2.0681) -- (0.2780, 0.2280, 2.0620) -- cycle;
\fill[blue!21.9, opacity=0.7] (0.2780, 0.2280, 2.0620) -- (0.3250, 0.2280, 2.0681) -- (0.3250, 0.2810, 2.0741) -- (0.2780, 0.2810, 2.0680) -- cycle;
\fill[blue!33.3, opacity=0.7] (0.2780, 0.2810, 2.0680) -- (0.3250, 0.2810, 2.0741) -- (0.3250, 0.3340, 2.0799) -- (0.2780, 0.3340, 2.0738) -- cycle;
\fill[blue!35.9, opacity=0.7] (0.2780, 0.3340, 2.0738) -- (0.3250, 0.3340, 2.0799) -- (0.3250, 0.3870, 2.0855) -- (0.2780, 0.3870, 2.0794) -- cycle;
\fill[blue!27.1, opacity=0.7] (0.2780, 0.3870, 2.0794) -- (0.3250, 0.3870, 2.0855) -- (0.3250, 0.4400, 2.0911) -- (0.2780, 0.4400, 2.0849) -- cycle;
\fill[blue!18.4, opacity=0.7] (0.2780, 0.4400, 2.0849) -- (0.3250, 0.4400, 2.0911) -- (0.3250, 0.4930, 2.0964) -- (0.2780, 0.4930, 2.0903) -- cycle;
\fill[blue!15.5, opacity=0.7] (0.2780, 0.4930, 2.0903) -- (0.3250, 0.4930, 2.0964) -- (0.3250, 0.5460, 2.1016) -- (0.2780, 0.5460, 2.0955) -- cycle;
\fill[blue!15.1, opacity=0.7] (0.2780, 0.5460, 2.0955) -- (0.3250, 0.5460, 2.1016) -- (0.3250, 0.5990, 2.1066) -- (0.2780, 0.5990, 2.1005) -- cycle;
\fill[blue!15.0, opacity=0.7] (0.2780, 0.5990, 2.1005) -- (0.3250, 0.5990, 2.1066) -- (0.3250, 0.6520, 2.1114) -- (0.2780, 0.6520, 2.1052) -- cycle;
\fill[blue!15.0, opacity=0.7] (0.2780, 0.6520, 2.1052) -- (0.3250, 0.6520, 2.1114) -- (0.3250, 0.7050, 2.1159) -- (0.2780, 0.7050, 2.1098) -- cycle;
\fill[blue!15.1, opacity=0.7] (0.2780, 0.7050, 2.1098) -- (0.3250, 0.7050, 2.1159) -- (0.3250, 0.7580, 2.1202) -- (0.2780, 0.7580, 2.1141) -- cycle;
\fill[blue!15.4, opacity=0.7] (0.2780, 0.7580, 2.1141) -- (0.3250, 0.7580, 2.1202) -- (0.3250, 0.8110, 2.1243) -- (0.2780, 0.8110, 2.1182) -- cycle;
\fill[blue!17.4, opacity=0.7] (0.2780, 0.8110, 2.1182) -- (0.3250, 0.8110, 2.1243) -- (0.3250, 0.8640, 2.1281) -- (0.2780, 0.8640, 2.1220) -- cycle;
\fill[blue!25.4, opacity=0.7] (0.2780, 0.8640, 2.1220) -- (0.3250, 0.8640, 2.1281) -- (0.3250, 0.9170, 2.1317) -- (0.2780, 0.9170, 2.1256) -- cycle;
\fill[blue!42.5, opacity=0.7] (0.2780, 0.9170, 2.1256) -- (0.3250, 0.9170, 2.1317) -- (0.3250, 0.9700, 2.1350) -- (0.2780, 0.9700, 2.1289) -- cycle;
\fill[blue!61.6, opacity=0.7] (0.2780, 0.9700, 2.1289) -- (0.3250, 0.9700, 2.1350) -- (0.3250, 1.0230, 2.1380) -- (0.2780, 1.0230, 2.1319) -- cycle;
\fill[blue!74.1, opacity=0.7] (0.2780, 1.0230, 2.1319) -- (0.3250, 1.0230, 2.1380) -- (0.3250, 1.0760, 2.1407) -- (0.2780, 1.0760, 2.1346) -- cycle;
\fill[blue!79.4, opacity=0.7] (0.2780, 1.0760, 2.1346) -- (0.3250, 1.0760, 2.1407) -- (0.3250, 1.1290, 2.1431) -- (0.2780, 1.1290, 2.1370) -- cycle;
\fill[blue!80.0, opacity=0.7] (0.2780, 1.1290, 2.1370) -- (0.3250, 1.1290, 2.1431) -- (0.3250, 1.1820, 2.1452) -- (0.2780, 1.1820, 2.1391) -- cycle;
\fill[blue!77.4, opacity=0.7] (0.2780, 1.1820, 2.1391) -- (0.3250, 1.1820, 2.1452) -- (0.3250, 1.2350, 2.1470) -- (0.2780, 1.2350, 2.1409) -- cycle;
\fill[blue!72.1, opacity=0.7] (0.2780, 1.2350, 2.1409) -- (0.3250, 1.2350, 2.1470) -- (0.3250, 1.2880, 2.1484) -- (0.2780, 1.2880, 2.1423) -- cycle;
\fill[blue!64.8, opacity=0.7] (0.2780, 1.2880, 2.1423) -- (0.3250, 1.2880, 2.1484) -- (0.3250, 1.3410, 2.1496) -- (0.2780, 1.3410, 2.1435) -- cycle;
\fill[blue!56.8, opacity=0.7] (0.2780, 1.3410, 2.1435) -- (0.3250, 1.3410, 2.1496) -- (0.3250, 1.3940, 2.1504) -- (0.2780, 1.3940, 2.1443) -- cycle;
\fill[blue!49.4, opacity=0.7] (0.2780, 1.3940, 2.1443) -- (0.3250, 1.3940, 2.1504) -- (0.3250, 1.4470, 2.1509) -- (0.2780, 1.4470, 2.1448) -- cycle;
\fill[blue!43.5, opacity=0.7] (0.2780, 1.4470, 2.1448) -- (0.3250, 1.4470, 2.1509) -- (0.3250, 1.5000, 2.1511) -- (0.2780, 1.5000, 2.1449) -- cycle;
\fill[blue!39.2, opacity=0.7] (0.2780, 1.5000, 2.1449) -- (0.3250, 1.5000, 2.1511) -- (0.3250, 1.5530, 2.1509) -- (0.2780, 1.5530, 2.1448) -- cycle;
\fill[blue!36.5, opacity=0.7] (0.2780, 1.5530, 2.1448) -- (0.3250, 1.5530, 2.1509) -- (0.3250, 1.6060, 2.1504) -- (0.2780, 1.6060, 2.1443) -- cycle;
\fill[blue!35.2, opacity=0.7] (0.2780, 1.6060, 2.1443) -- (0.3250, 1.6060, 2.1504) -- (0.3250, 1.6590, 2.1496) -- (0.2780, 1.6590, 2.1435) -- cycle;
\fill[blue!35.1, opacity=0.7] (0.2780, 1.6590, 2.1435) -- (0.3250, 1.6590, 2.1496) -- (0.3250, 1.7120, 2.1484) -- (0.2780, 1.7120, 2.1423) -- cycle;
\fill[blue!36.2, opacity=0.7] (0.2780, 1.7120, 2.1423) -- (0.3250, 1.7120, 2.1484) -- (0.3250, 1.7650, 2.1470) -- (0.2780, 1.7650, 2.1409) -- cycle;
\fill[blue!38.7, opacity=0.7] (0.2780, 1.7650, 2.1409) -- (0.3250, 1.7650, 2.1470) -- (0.3250, 1.8180, 2.1452) -- (0.2780, 1.8180, 2.1391) -- cycle;
\fill[blue!42.6, opacity=0.7] (0.2780, 1.8180, 2.1391) -- (0.3250, 1.8180, 2.1452) -- (0.3250, 1.8710, 2.1431) -- (0.2780, 1.8710, 2.1370) -- cycle;
\fill[blue!47.8, opacity=0.7] (0.2780, 1.8710, 2.1370) -- (0.3250, 1.8710, 2.1431) -- (0.3250, 1.9240, 2.1407) -- (0.2780, 1.9240, 2.1346) -- cycle;
\fill[blue!54.1, opacity=0.7] (0.2780, 1.9240, 2.1346) -- (0.3250, 1.9240, 2.1407) -- (0.3250, 1.9770, 2.1380) -- (0.2780, 1.9770, 2.1319) -- cycle;
\fill[blue!60.6, opacity=0.7] (0.2780, 1.9770, 2.1319) -- (0.3250, 1.9770, 2.1380) -- (0.3250, 2.0300, 2.1350) -- (0.2780, 2.0300, 2.1289) -- cycle;
\fill[blue!66.1, opacity=0.7] (0.2780, 2.0300, 2.1289) -- (0.3250, 2.0300, 2.1350) -- (0.3250, 2.0830, 2.1317) -- (0.2780, 2.0830, 2.1256) -- cycle;
\fill[blue!69.4, opacity=0.7] (0.2780, 2.0830, 2.1256) -- (0.3250, 2.0830, 2.1317) -- (0.3250, 2.1360, 2.1281) -- (0.2780, 2.1360, 2.1220) -- cycle;
\fill[blue!69.2, opacity=0.7] (0.2780, 2.1360, 2.1220) -- (0.3250, 2.1360, 2.1281) -- (0.3250, 2.1890, 2.1243) -- (0.2780, 2.1890, 2.1182) -- cycle;
\fill[blue!64.4, opacity=0.7] (0.2780, 2.1890, 2.1182) -- (0.3250, 2.1890, 2.1243) -- (0.3250, 2.2420, 2.1202) -- (0.2780, 2.2420, 2.1141) -- cycle;
\fill[blue!53.5, opacity=0.7] (0.2780, 2.2420, 2.1141) -- (0.3250, 2.2420, 2.1202) -- (0.3250, 2.2950, 2.1159) -- (0.2780, 2.2950, 2.1098) -- cycle;
\fill[blue!37.9, opacity=0.7] (0.2780, 2.2950, 2.1098) -- (0.3250, 2.2950, 2.1159) -- (0.3250, 2.3480, 2.1114) -- (0.2780, 2.3480, 2.1052) -- cycle;
\fill[blue!24.0, opacity=0.7] (0.2780, 2.3480, 2.1052) -- (0.3250, 2.3480, 2.1114) -- (0.3250, 2.4010, 2.1066) -- (0.2780, 2.4010, 2.1005) -- cycle;
\fill[blue!17.1, opacity=0.7] (0.2780, 2.4010, 2.1005) -- (0.3250, 2.4010, 2.1066) -- (0.3250, 2.4540, 2.1016) -- (0.2780, 2.4540, 2.0955) -- cycle;
\fill[blue!15.3, opacity=0.7] (0.2780, 2.4540, 2.0955) -- (0.3250, 2.4540, 2.1016) -- (0.3250, 2.5070, 2.0964) -- (0.2780, 2.5070, 2.0903) -- cycle;
\fill[blue!15.0, opacity=0.7] (0.2780, 2.5070, 2.0903) -- (0.3250, 2.5070, 2.0964) -- (0.3250, 2.5600, 2.0911) -- (0.2780, 2.5600, 2.0849) -- cycle;
\fill[blue!15.0, opacity=0.7] (0.2780, 2.5600, 2.0849) -- (0.3250, 2.5600, 2.0911) -- (0.3250, 2.6130, 2.0855) -- (0.2780, 2.6130, 2.0794) -- cycle;
\fill[blue!15.0, opacity=0.7] (0.2780, 2.6130, 2.0794) -- (0.3250, 2.6130, 2.0855) -- (0.3250, 2.6660, 2.0799) -- (0.2780, 2.6660, 2.0738) -- cycle;
\fill[blue!15.0, opacity=0.7] (0.2780, 2.6660, 2.0738) -- (0.3250, 2.6660, 2.0799) -- (0.3250, 2.7190, 2.0741) -- (0.2780, 2.7190, 2.0680) -- cycle;
\fill[blue!15.0, opacity=0.7] (0.2780, 2.7190, 2.0680) -- (0.3250, 2.7190, 2.0741) -- (0.3250, 2.7720, 2.0681) -- (0.2780, 2.7720, 2.0620) -- cycle;
\fill[blue!15.1, opacity=0.7] (0.2780, 2.7720, 2.0620) -- (0.3250, 2.7720, 2.0681) -- (0.3250, 2.8250, 2.0621) -- (0.2780, 2.8250, 2.0560) -- cycle;
\fill[blue!16.0, opacity=0.7] (0.2780, 2.8250, 2.0560) -- (0.3250, 2.8250, 2.0621) -- (0.3250, 2.8780, 2.0560) -- (0.2780, 2.8780, 2.0499) -- cycle;
\fill[blue!18.9, opacity=0.7] (0.2780, 2.8780, 2.0499) -- (0.3250, 2.8780, 2.0560) -- (0.3250, 2.9310, 2.0498) -- (0.2780, 2.9310, 2.0437) -- cycle;
\fill[blue!21.7, opacity=0.7] (0.2780, 2.9310, 2.0437) -- (0.3250, 2.9310, 2.0498) -- (0.3250, 2.9840, 2.0436) -- (0.2780, 2.9840, 2.0375) -- cycle;
\fill[blue!19.9, opacity=0.7] (0.2780, 2.9840, 2.0375) -- (0.3250, 2.9840, 2.0436) -- (0.3250, 3.0370, 2.0373) -- (0.2780, 3.0370, 2.0312) -- cycle;
\fill[blue!16.2, opacity=0.7] (0.2780, 3.0370, 2.0312) -- (0.3250, 3.0370, 2.0373) -- (0.3250, 3.0900, 2.0311) -- (0.2780, 3.0900, 2.0249) -- cycle;
\fill[blue!15.0, opacity=0.7] (0.3250, -0.0900, 2.0311) -- (0.3720, -0.0900, 2.0371) -- (0.3720, -0.0370, 2.0434) -- (0.3250, -0.0370, 2.0373) -- cycle;
\fill[blue!15.0, opacity=0.7] (0.3250, -0.0370, 2.0373) -- (0.3720, -0.0370, 2.0434) -- (0.3720, 0.0160, 2.0496) -- (0.3250, 0.0160, 2.0436) -- cycle;
\fill[blue!15.0, opacity=0.7] (0.3250, 0.0160, 2.0436) -- (0.3720, 0.0160, 2.0496) -- (0.3720, 0.0690, 2.0559) -- (0.3250, 0.0690, 2.0498) -- cycle;
\fill[blue!15.0, opacity=0.7] (0.3250, 0.0690, 2.0498) -- (0.3720, 0.0690, 2.0559) -- (0.3720, 0.1220, 2.0620) -- (0.3250, 0.1220, 2.0560) -- cycle;
\fill[blue!15.6, opacity=0.7] (0.3250, 0.1220, 2.0560) -- (0.3720, 0.1220, 2.0620) -- (0.3720, 0.1750, 2.0681) -- (0.3250, 0.1750, 2.0621) -- cycle;
\fill[blue!21.0, opacity=0.7] (0.3250, 0.1750, 2.0621) -- (0.3720, 0.1750, 2.0681) -- (0.3720, 0.2280, 2.0742) -- (0.3250, 0.2280, 2.0681) -- cycle;
\fill[blue!33.1, opacity=0.7] (0.3250, 0.2280, 2.0681) -- (0.3720, 0.2280, 2.0742) -- (0.3720, 0.2810, 2.0801) -- (0.3250, 0.2810, 2.0741) -- cycle;
\fill[blue!36.6, opacity=0.7] (0.3250, 0.2810, 2.0741) -- (0.3720, 0.2810, 2.0801) -- (0.3720, 0.3340, 2.0859) -- (0.3250, 0.3340, 2.0799) -- cycle;
\fill[blue!27.4, opacity=0.7] (0.3250, 0.3340, 2.0799) -- (0.3720, 0.3340, 2.0859) -- (0.3720, 0.3870, 2.0916) -- (0.3250, 0.3870, 2.0855) -- cycle;
\fill[blue!18.3, opacity=0.7] (0.3250, 0.3870, 2.0855) -- (0.3720, 0.3870, 2.0916) -- (0.3720, 0.4400, 2.0971) -- (0.3250, 0.4400, 2.0911) -- cycle;
\fill[blue!15.5, opacity=0.7] (0.3250, 0.4400, 2.0911) -- (0.3720, 0.4400, 2.0971) -- (0.3720, 0.4930, 2.1024) -- (0.3250, 0.4930, 2.0964) -- cycle;
\fill[blue!15.1, opacity=0.7] (0.3250, 0.4930, 2.0964) -- (0.3720, 0.4930, 2.1024) -- (0.3720, 0.5460, 2.1076) -- (0.3250, 0.5460, 2.1016) -- cycle;
\fill[blue!15.0, opacity=0.7] (0.3250, 0.5460, 2.1016) -- (0.3720, 0.5460, 2.1076) -- (0.3720, 0.5990, 2.1126) -- (0.3250, 0.5990, 2.1066) -- cycle;
\fill[blue!15.0, opacity=0.7] (0.3250, 0.5990, 2.1066) -- (0.3720, 0.5990, 2.1126) -- (0.3720, 0.6520, 2.1174) -- (0.3250, 0.6520, 2.1114) -- cycle;
\fill[blue!15.1, opacity=0.7] (0.3250, 0.6520, 2.1114) -- (0.3720, 0.6520, 2.1174) -- (0.3720, 0.7050, 2.1219) -- (0.3250, 0.7050, 2.1159) -- cycle;
\fill[blue!16.0, opacity=0.7] (0.3250, 0.7050, 2.1159) -- (0.3720, 0.7050, 2.1219) -- (0.3720, 0.7580, 2.1263) -- (0.3250, 0.7580, 2.1202) -- cycle;
\fill[blue!21.3, opacity=0.7] (0.3250, 0.7580, 2.1202) -- (0.3720, 0.7580, 2.1263) -- (0.3720, 0.8110, 2.1303) -- (0.3250, 0.8110, 2.1243) -- cycle;
\fill[blue!37.6, opacity=0.7] (0.3250, 0.8110, 2.1243) -- (0.3720, 0.8110, 2.1303) -- (0.3720, 0.8640, 2.1342) -- (0.3250, 0.8640, 2.1281) -- cycle;
\fill[blue!60.2, opacity=0.7] (0.3250, 0.8640, 2.1281) -- (0.3720, 0.8640, 2.1342) -- (0.3720, 0.9170, 2.1377) -- (0.3250, 0.9170, 2.1317) -- cycle;
\fill[blue!75.4, opacity=0.7] (0.3250, 0.9170, 2.1317) -- (0.3720, 0.9170, 2.1377) -- (0.3720, 0.9700, 2.1410) -- (0.3250, 0.9700, 2.1350) -- cycle;
\fill[blue!80.9, opacity=0.7] (0.3250, 0.9700, 2.1350) -- (0.3720, 0.9700, 2.1410) -- (0.3720, 1.0230, 2.1440) -- (0.3250, 1.0230, 2.1380) -- cycle;
\fill[blue!80.5, opacity=0.7] (0.3250, 1.0230, 2.1380) -- (0.3720, 1.0230, 2.1440) -- (0.3720, 1.0760, 2.1467) -- (0.3250, 1.0760, 2.1407) -- cycle;
\fill[blue!75.1, opacity=0.7] (0.3250, 1.0760, 2.1407) -- (0.3720, 1.0760, 2.1467) -- (0.3720, 1.1290, 2.1491) -- (0.3250, 1.1290, 2.1431) -- cycle;
\fill[blue!65.0, opacity=0.7] (0.3250, 1.1290, 2.1431) -- (0.3720, 1.1290, 2.1491) -- (0.3720, 1.1820, 2.1512) -- (0.3250, 1.1820, 2.1452) -- cycle;
\fill[blue!52.2, opacity=0.7] (0.3250, 1.1820, 2.1452) -- (0.3720, 1.1820, 2.1512) -- (0.3720, 1.2350, 2.1530) -- (0.3250, 1.2350, 2.1470) -- cycle;
\fill[blue!40.1, opacity=0.7] (0.3250, 1.2350, 2.1470) -- (0.3720, 1.2350, 2.1530) -- (0.3720, 1.2880, 2.1545) -- (0.3250, 1.2880, 2.1484) -- cycle;
\fill[blue!31.1, opacity=0.7] (0.3250, 1.2880, 2.1484) -- (0.3720, 1.2880, 2.1545) -- (0.3720, 1.3410, 2.1556) -- (0.3250, 1.3410, 2.1496) -- cycle;
\fill[blue!25.3, opacity=0.7] (0.3250, 1.3410, 2.1496) -- (0.3720, 1.3410, 2.1556) -- (0.3720, 1.3940, 2.1564) -- (0.3250, 1.3940, 2.1504) -- cycle;
\fill[blue!21.9, opacity=0.7] (0.3250, 1.3940, 2.1504) -- (0.3720, 1.3940, 2.1564) -- (0.3720, 1.4470, 2.1569) -- (0.3250, 1.4470, 2.1509) -- cycle;
\fill[blue!19.9, opacity=0.7] (0.3250, 1.4470, 2.1509) -- (0.3720, 1.4470, 2.1569) -- (0.3720, 1.5000, 2.1571) -- (0.3250, 1.5000, 2.1511) -- cycle;
\fill[blue!18.8, opacity=0.7] (0.3250, 1.5000, 2.1511) -- (0.3720, 1.5000, 2.1571) -- (0.3720, 1.5530, 2.1569) -- (0.3250, 1.5530, 2.1509) -- cycle;
\fill[blue!18.1, opacity=0.7] (0.3250, 1.5530, 2.1509) -- (0.3720, 1.5530, 2.1569) -- (0.3720, 1.6060, 2.1564) -- (0.3250, 1.6060, 2.1504) -- cycle;
\fill[blue!17.9, opacity=0.7] (0.3250, 1.6060, 2.1504) -- (0.3720, 1.6060, 2.1564) -- (0.3720, 1.6590, 2.1556) -- (0.3250, 1.6590, 2.1496) -- cycle;
\fill[blue!17.8, opacity=0.7] (0.3250, 1.6590, 2.1496) -- (0.3720, 1.6590, 2.1556) -- (0.3720, 1.7120, 2.1545) -- (0.3250, 1.7120, 2.1484) -- cycle;
\fill[blue!18.0, opacity=0.7] (0.3250, 1.7120, 2.1484) -- (0.3720, 1.7120, 2.1545) -- (0.3720, 1.7650, 2.1530) -- (0.3250, 1.7650, 2.1470) -- cycle;
\fill[blue!18.6, opacity=0.7] (0.3250, 1.7650, 2.1470) -- (0.3720, 1.7650, 2.1530) -- (0.3720, 1.8180, 2.1512) -- (0.3250, 1.8180, 2.1452) -- cycle;
\fill[blue!19.6, opacity=0.7] (0.3250, 1.8180, 2.1452) -- (0.3720, 1.8180, 2.1512) -- (0.3720, 1.8710, 2.1491) -- (0.3250, 1.8710, 2.1431) -- cycle;
\fill[blue!21.4, opacity=0.7] (0.3250, 1.8710, 2.1431) -- (0.3720, 1.8710, 2.1491) -- (0.3720, 1.9240, 2.1467) -- (0.3250, 1.9240, 2.1407) -- cycle;
\fill[blue!24.6, opacity=0.7] (0.3250, 1.9240, 2.1407) -- (0.3720, 1.9240, 2.1467) -- (0.3720, 1.9770, 2.1440) -- (0.3250, 1.9770, 2.1380) -- cycle;
\fill[blue!29.9, opacity=0.7] (0.3250, 1.9770, 2.1380) -- (0.3720, 1.9770, 2.1440) -- (0.3720, 2.0300, 2.1410) -- (0.3250, 2.0300, 2.1350) -- cycle;
\fill[blue!37.8, opacity=0.7] (0.3250, 2.0300, 2.1350) -- (0.3720, 2.0300, 2.1410) -- (0.3720, 2.0830, 2.1377) -- (0.3250, 2.0830, 2.1317) -- cycle;
\fill[blue!47.9, opacity=0.7] (0.3250, 2.0830, 2.1317) -- (0.3720, 2.0830, 2.1377) -- (0.3720, 2.1360, 2.1342) -- (0.3250, 2.1360, 2.1281) -- cycle;
\fill[blue!58.1, opacity=0.7] (0.3250, 2.1360, 2.1281) -- (0.3720, 2.1360, 2.1342) -- (0.3720, 2.1890, 2.1303) -- (0.3250, 2.1890, 2.1243) -- cycle;
\fill[blue!65.7, opacity=0.7] (0.3250, 2.1890, 2.1243) -- (0.3720, 2.1890, 2.1303) -- (0.3720, 2.2420, 2.1263) -- (0.3250, 2.2420, 2.1202) -- cycle;
\fill[blue!68.2, opacity=0.7] (0.3250, 2.2420, 2.1202) -- (0.3720, 2.2420, 2.1263) -- (0.3720, 2.2950, 2.1219) -- (0.3250, 2.2950, 2.1159) -- cycle;
\fill[blue!64.2, opacity=0.7] (0.3250, 2.2950, 2.1159) -- (0.3720, 2.2950, 2.1219) -- (0.3720, 2.3480, 2.1174) -- (0.3250, 2.3480, 2.1114) -- cycle;
\fill[blue!52.1, opacity=0.7] (0.3250, 2.3480, 2.1114) -- (0.3720, 2.3480, 2.1174) -- (0.3720, 2.4010, 2.1126) -- (0.3250, 2.4010, 2.1066) -- cycle;
\fill[blue!34.5, opacity=0.7] (0.3250, 2.4010, 2.1066) -- (0.3720, 2.4010, 2.1126) -- (0.3720, 2.4540, 2.1076) -- (0.3250, 2.4540, 2.1016) -- cycle;
\fill[blue!20.9, opacity=0.7] (0.3250, 2.4540, 2.1016) -- (0.3720, 2.4540, 2.1076) -- (0.3720, 2.5070, 2.1024) -- (0.3250, 2.5070, 2.0964) -- cycle;
\fill[blue!15.9, opacity=0.7] (0.3250, 2.5070, 2.0964) -- (0.3720, 2.5070, 2.1024) -- (0.3720, 2.5600, 2.0971) -- (0.3250, 2.5600, 2.0911) -- cycle;
\fill[blue!15.1, opacity=0.7] (0.3250, 2.5600, 2.0911) -- (0.3720, 2.5600, 2.0971) -- (0.3720, 2.6130, 2.0916) -- (0.3250, 2.6130, 2.0855) -- cycle;
\fill[blue!15.0, opacity=0.7] (0.3250, 2.6130, 2.0855) -- (0.3720, 2.6130, 2.0916) -- (0.3720, 2.6660, 2.0859) -- (0.3250, 2.6660, 2.0799) -- cycle;
\fill[blue!15.0, opacity=0.7] (0.3250, 2.6660, 2.0799) -- (0.3720, 2.6660, 2.0859) -- (0.3720, 2.7190, 2.0801) -- (0.3250, 2.7190, 2.0741) -- cycle;
\fill[blue!15.0, opacity=0.7] (0.3250, 2.7190, 2.0741) -- (0.3720, 2.7190, 2.0801) -- (0.3720, 2.7720, 2.0742) -- (0.3250, 2.7720, 2.0681) -- cycle;
\fill[blue!15.0, opacity=0.7] (0.3250, 2.7720, 2.0681) -- (0.3720, 2.7720, 2.0742) -- (0.3720, 2.8250, 2.0681) -- (0.3250, 2.8250, 2.0621) -- cycle;
\fill[blue!15.1, opacity=0.7] (0.3250, 2.8250, 2.0621) -- (0.3720, 2.8250, 2.0681) -- (0.3720, 2.8780, 2.0620) -- (0.3250, 2.8780, 2.0560) -- cycle;
\fill[blue!15.8, opacity=0.7] (0.3250, 2.8780, 2.0560) -- (0.3720, 2.8780, 2.0620) -- (0.3720, 2.9310, 2.0559) -- (0.3250, 2.9310, 2.0498) -- cycle;
\fill[blue!18.5, opacity=0.7] (0.3250, 2.9310, 2.0498) -- (0.3720, 2.9310, 2.0559) -- (0.3720, 2.9840, 2.0496) -- (0.3250, 2.9840, 2.0436) -- cycle;
\fill[blue!21.5, opacity=0.7] (0.3250, 2.9840, 2.0436) -- (0.3720, 2.9840, 2.0496) -- (0.3720, 3.0370, 2.0434) -- (0.3250, 3.0370, 2.0373) -- cycle;
\fill[blue!19.7, opacity=0.7] (0.3250, 3.0370, 2.0373) -- (0.3720, 3.0370, 2.0434) -- (0.3720, 3.0900, 2.0371) -- (0.3250, 3.0900, 2.0311) -- cycle;
\fill[blue!15.0, opacity=0.7] (0.3720, -0.0900, 2.0371) -- (0.4190, -0.0900, 2.0430) -- (0.4190, -0.0370, 2.0493) -- (0.3720, -0.0370, 2.0434) -- cycle;
\fill[blue!15.0, opacity=0.7] (0.3720, -0.0370, 2.0434) -- (0.4190, -0.0370, 2.0493) -- (0.4190, 0.0160, 2.0555) -- (0.3720, 0.0160, 2.0496) -- cycle;
\fill[blue!15.0, opacity=0.7] (0.3720, 0.0160, 2.0496) -- (0.4190, 0.0160, 2.0555) -- (0.4190, 0.0690, 2.0618) -- (0.3720, 0.0690, 2.0559) -- cycle;
\fill[blue!15.3, opacity=0.7] (0.3720, 0.0690, 2.0559) -- (0.4190, 0.0690, 2.0618) -- (0.4190, 0.1220, 2.0680) -- (0.3720, 0.1220, 2.0620) -- cycle;
\fill[blue!19.5, opacity=0.7] (0.3720, 0.1220, 2.0620) -- (0.4190, 0.1220, 2.0680) -- (0.4190, 0.1750, 2.0741) -- (0.3720, 0.1750, 2.0681) -- cycle;
\fill[blue!31.9, opacity=0.7] (0.3720, 0.1750, 2.0681) -- (0.4190, 0.1750, 2.0741) -- (0.4190, 0.2280, 2.0801) -- (0.3720, 0.2280, 2.0742) -- cycle;
\fill[blue!37.5, opacity=0.7] (0.3720, 0.2280, 2.0742) -- (0.4190, 0.2280, 2.0801) -- (0.4190, 0.2810, 2.0860) -- (0.3720, 0.2810, 2.0801) -- cycle;
\fill[blue!28.7, opacity=0.7] (0.3720, 0.2810, 2.0801) -- (0.4190, 0.2810, 2.0860) -- (0.4190, 0.3340, 2.0918) -- (0.3720, 0.3340, 2.0859) -- cycle;
\fill[blue!18.6, opacity=0.7] (0.3720, 0.3340, 2.0859) -- (0.4190, 0.3340, 2.0918) -- (0.4190, 0.3870, 2.0975) -- (0.3720, 0.3870, 2.0916) -- cycle;
\fill[blue!15.5, opacity=0.7] (0.3720, 0.3870, 2.0916) -- (0.4190, 0.3870, 2.0975) -- (0.4190, 0.4400, 2.1030) -- (0.3720, 0.4400, 2.0971) -- cycle;
\fill[blue!15.1, opacity=0.7] (0.3720, 0.4400, 2.0971) -- (0.4190, 0.4400, 2.1030) -- (0.4190, 0.4930, 2.1084) -- (0.3720, 0.4930, 2.1024) -- cycle;
\fill[blue!15.0, opacity=0.7] (0.3720, 0.4930, 2.1024) -- (0.4190, 0.4930, 2.1084) -- (0.4190, 0.5460, 2.1135) -- (0.3720, 0.5460, 2.1076) -- cycle;
\fill[blue!15.0, opacity=0.7] (0.3720, 0.5460, 2.1076) -- (0.4190, 0.5460, 2.1135) -- (0.4190, 0.5990, 2.1185) -- (0.3720, 0.5990, 2.1126) -- cycle;
\fill[blue!15.2, opacity=0.7] (0.3720, 0.5990, 2.1126) -- (0.4190, 0.5990, 2.1185) -- (0.4190, 0.6520, 2.1233) -- (0.3720, 0.6520, 2.1174) -- cycle;
\fill[blue!17.0, opacity=0.7] (0.3720, 0.6520, 2.1174) -- (0.4190, 0.6520, 2.1233) -- (0.4190, 0.7050, 2.1279) -- (0.3720, 0.7050, 2.1219) -- cycle;
\fill[blue!27.0, opacity=0.7] (0.3720, 0.7050, 2.1219) -- (0.4190, 0.7050, 2.1279) -- (0.4190, 0.7580, 2.1322) -- (0.3720, 0.7580, 2.1263) -- cycle;
\fill[blue!50.3, opacity=0.7] (0.3720, 0.7580, 2.1263) -- (0.4190, 0.7580, 2.1322) -- (0.4190, 0.8110, 2.1363) -- (0.3720, 0.8110, 2.1303) -- cycle;
\fill[blue!72.1, opacity=0.7] (0.3720, 0.8110, 2.1303) -- (0.4190, 0.8110, 2.1363) -- (0.4190, 0.8640, 2.1401) -- (0.3720, 0.8640, 2.1342) -- cycle;
\fill[blue!81.2, opacity=0.7] (0.3720, 0.8640, 2.1342) -- (0.4190, 0.8640, 2.1401) -- (0.4190, 0.9170, 2.1436) -- (0.3720, 0.9170, 2.1377) -- cycle;
\fill[blue!81.7, opacity=0.7] (0.3720, 0.9170, 2.1377) -- (0.4190, 0.9170, 2.1436) -- (0.4190, 0.9700, 2.1469) -- (0.3720, 0.9700, 2.1410) -- cycle;
\fill[blue!75.6, opacity=0.7] (0.3720, 0.9700, 2.1410) -- (0.4190, 0.9700, 2.1469) -- (0.4190, 1.0230, 2.1499) -- (0.3720, 1.0230, 2.1440) -- cycle;
\fill[blue!62.4, opacity=0.7] (0.3720, 1.0230, 2.1440) -- (0.4190, 1.0230, 2.1499) -- (0.4190, 1.0760, 2.1526) -- (0.3720, 1.0760, 2.1467) -- cycle;
\fill[blue!45.7, opacity=0.7] (0.3720, 1.0760, 2.1467) -- (0.4190, 1.0760, 2.1526) -- (0.4190, 1.1290, 2.1550) -- (0.3720, 1.1290, 2.1491) -- cycle;
\fill[blue!31.9, opacity=0.7] (0.3720, 1.1290, 2.1491) -- (0.4190, 1.1290, 2.1550) -- (0.4190, 1.1820, 2.1571) -- (0.3720, 1.1820, 2.1512) -- cycle;
\fill[blue!23.6, opacity=0.7] (0.3720, 1.1820, 2.1512) -- (0.4190, 1.1820, 2.1571) -- (0.4190, 1.2350, 2.1589) -- (0.3720, 1.2350, 2.1530) -- cycle;
\fill[blue!19.4, opacity=0.7] (0.3720, 1.2350, 2.1530) -- (0.4190, 1.2350, 2.1589) -- (0.4190, 1.2880, 2.1604) -- (0.3720, 1.2880, 2.1545) -- cycle;
\fill[blue!17.5, opacity=0.7] (0.3720, 1.2880, 2.1545) -- (0.4190, 1.2880, 2.1604) -- (0.4190, 1.3410, 2.1615) -- (0.3720, 1.3410, 2.1556) -- cycle;
\fill[blue!16.6, opacity=0.7] (0.3720, 1.3410, 2.1556) -- (0.4190, 1.3410, 2.1615) -- (0.4190, 1.3940, 2.1623) -- (0.3720, 1.3940, 2.1564) -- cycle;
\fill[blue!16.2, opacity=0.7] (0.3720, 1.3940, 2.1564) -- (0.4190, 1.3940, 2.1623) -- (0.4190, 1.4470, 2.1628) -- (0.3720, 1.4470, 2.1569) -- cycle;
\fill[blue!16.0, opacity=0.7] (0.3720, 1.4470, 2.1569) -- (0.4190, 1.4470, 2.1628) -- (0.4190, 1.5000, 2.1630) -- (0.3720, 1.5000, 2.1571) -- cycle;
\fill[blue!15.8, opacity=0.7] (0.3720, 1.5000, 2.1571) -- (0.4190, 1.5000, 2.1630) -- (0.4190, 1.5530, 2.1628) -- (0.3720, 1.5530, 2.1569) -- cycle;
\fill[blue!15.8, opacity=0.7] (0.3720, 1.5530, 2.1569) -- (0.4190, 1.5530, 2.1628) -- (0.4190, 1.6060, 2.1623) -- (0.3720, 1.6060, 2.1564) -- cycle;
\fill[blue!15.7, opacity=0.7] (0.3720, 1.6060, 2.1564) -- (0.4190, 1.6060, 2.1623) -- (0.4190, 1.6590, 2.1615) -- (0.3720, 1.6590, 2.1556) -- cycle;
\fill[blue!15.7, opacity=0.7] (0.3720, 1.6590, 2.1556) -- (0.4190, 1.6590, 2.1615) -- (0.4190, 1.7120, 2.1604) -- (0.3720, 1.7120, 2.1545) -- cycle;
\fill[blue!15.7, opacity=0.7] (0.3720, 1.7120, 2.1545) -- (0.4190, 1.7120, 2.1604) -- (0.4190, 1.7650, 2.1589) -- (0.3720, 1.7650, 2.1530) -- cycle;
\fill[blue!15.7, opacity=0.7] (0.3720, 1.7650, 2.1530) -- (0.4190, 1.7650, 2.1589) -- (0.4190, 1.8180, 2.1571) -- (0.3720, 1.8180, 2.1512) -- cycle;
\fill[blue!15.7, opacity=0.7] (0.3720, 1.8180, 2.1512) -- (0.4190, 1.8180, 2.1571) -- (0.4190, 1.8710, 2.1550) -- (0.3720, 1.8710, 2.1491) -- cycle;
\fill[blue!15.9, opacity=0.7] (0.3720, 1.8710, 2.1491) -- (0.4190, 1.8710, 2.1550) -- (0.4190, 1.9240, 2.1526) -- (0.3720, 1.9240, 2.1467) -- cycle;
\fill[blue!16.2, opacity=0.7] (0.3720, 1.9240, 2.1467) -- (0.4190, 1.9240, 2.1526) -- (0.4190, 1.9770, 2.1499) -- (0.3720, 1.9770, 2.1440) -- cycle;
\fill[blue!17.0, opacity=0.7] (0.3720, 1.9770, 2.1440) -- (0.4190, 1.9770, 2.1499) -- (0.4190, 2.0300, 2.1469) -- (0.3720, 2.0300, 2.1410) -- cycle;
\fill[blue!18.6, opacity=0.7] (0.3720, 2.0300, 2.1410) -- (0.4190, 2.0300, 2.1469) -- (0.4190, 2.0830, 2.1436) -- (0.3720, 2.0830, 2.1377) -- cycle;
\fill[blue!22.2, opacity=0.7] (0.3720, 2.0830, 2.1377) -- (0.4190, 2.0830, 2.1436) -- (0.4190, 2.1360, 2.1401) -- (0.3720, 2.1360, 2.1342) -- cycle;
\fill[blue!29.2, opacity=0.7] (0.3720, 2.1360, 2.1342) -- (0.4190, 2.1360, 2.1401) -- (0.4190, 2.1890, 2.1363) -- (0.3720, 2.1890, 2.1303) -- cycle;
\fill[blue!40.5, opacity=0.7] (0.3720, 2.1890, 2.1303) -- (0.4190, 2.1890, 2.1363) -- (0.4190, 2.2420, 2.1322) -- (0.3720, 2.2420, 2.1263) -- cycle;
\fill[blue!53.7, opacity=0.7] (0.3720, 2.2420, 2.1263) -- (0.4190, 2.2420, 2.1322) -- (0.4190, 2.2950, 2.1279) -- (0.3720, 2.2950, 2.1219) -- cycle;
\fill[blue!63.8, opacity=0.7] (0.3720, 2.2950, 2.1219) -- (0.4190, 2.2950, 2.1279) -- (0.4190, 2.3480, 2.1233) -- (0.3720, 2.3480, 2.1174) -- cycle;
\fill[blue!66.9, opacity=0.7] (0.3720, 2.3480, 2.1174) -- (0.4190, 2.3480, 2.1233) -- (0.4190, 2.4010, 2.1185) -- (0.3720, 2.4010, 2.1126) -- cycle;
\fill[blue!60.9, opacity=0.7] (0.3720, 2.4010, 2.1126) -- (0.4190, 2.4010, 2.1185) -- (0.4190, 2.4540, 2.1135) -- (0.3720, 2.4540, 2.1076) -- cycle;
\fill[blue!45.1, opacity=0.7] (0.3720, 2.4540, 2.1076) -- (0.4190, 2.4540, 2.1135) -- (0.4190, 2.5070, 2.1084) -- (0.3720, 2.5070, 2.1024) -- cycle;
\fill[blue!26.6, opacity=0.7] (0.3720, 2.5070, 2.1024) -- (0.4190, 2.5070, 2.1084) -- (0.4190, 2.5600, 2.1030) -- (0.3720, 2.5600, 2.0971) -- cycle;
\fill[blue!17.2, opacity=0.7] (0.3720, 2.5600, 2.0971) -- (0.4190, 2.5600, 2.1030) -- (0.4190, 2.6130, 2.0975) -- (0.3720, 2.6130, 2.0916) -- cycle;
\fill[blue!15.2, opacity=0.7] (0.3720, 2.6130, 2.0916) -- (0.4190, 2.6130, 2.0975) -- (0.4190, 2.6660, 2.0918) -- (0.3720, 2.6660, 2.0859) -- cycle;
\fill[blue!15.0, opacity=0.7] (0.3720, 2.6660, 2.0859) -- (0.4190, 2.6660, 2.0918) -- (0.4190, 2.7190, 2.0860) -- (0.3720, 2.7190, 2.0801) -- cycle;
\fill[blue!15.0, opacity=0.7] (0.3720, 2.7190, 2.0801) -- (0.4190, 2.7190, 2.0860) -- (0.4190, 2.7720, 2.0801) -- (0.3720, 2.7720, 2.0742) -- cycle;
\fill[blue!15.0, opacity=0.7] (0.3720, 2.7720, 2.0742) -- (0.4190, 2.7720, 2.0801) -- (0.4190, 2.8250, 2.0741) -- (0.3720, 2.8250, 2.0681) -- cycle;
\fill[blue!15.0, opacity=0.7] (0.3720, 2.8250, 2.0681) -- (0.4190, 2.8250, 2.0741) -- (0.4190, 2.8780, 2.0680) -- (0.3720, 2.8780, 2.0620) -- cycle;
\fill[blue!15.1, opacity=0.7] (0.3720, 2.8780, 2.0620) -- (0.4190, 2.8780, 2.0680) -- (0.4190, 2.9310, 2.0618) -- (0.3720, 2.9310, 2.0559) -- cycle;
\fill[blue!15.7, opacity=0.7] (0.3720, 2.9310, 2.0559) -- (0.4190, 2.9310, 2.0618) -- (0.4190, 2.9840, 2.0555) -- (0.3720, 2.9840, 2.0496) -- cycle;
\fill[blue!18.4, opacity=0.7] (0.3720, 2.9840, 2.0496) -- (0.4190, 2.9840, 2.0555) -- (0.4190, 3.0370, 2.0493) -- (0.3720, 3.0370, 2.0434) -- cycle;
\fill[blue!21.3, opacity=0.7] (0.3720, 3.0370, 2.0434) -- (0.4190, 3.0370, 2.0493) -- (0.4190, 3.0900, 2.0430) -- (0.3720, 3.0900, 2.0371) -- cycle;
\fill[blue!15.0, opacity=0.7] (0.4190, -0.0900, 2.0430) -- (0.4660, -0.0900, 2.0488) -- (0.4660, -0.0370, 2.0551) -- (0.4190, -0.0370, 2.0493) -- cycle;
\fill[blue!15.0, opacity=0.7] (0.4190, -0.0370, 2.0493) -- (0.4660, -0.0370, 2.0551) -- (0.4660, 0.0160, 2.0614) -- (0.4190, 0.0160, 2.0555) -- cycle;
\fill[blue!15.1, opacity=0.7] (0.4190, 0.0160, 2.0555) -- (0.4660, 0.0160, 2.0614) -- (0.4660, 0.0690, 2.0676) -- (0.4190, 0.0690, 2.0618) -- cycle;
\fill[blue!17.7, opacity=0.7] (0.4190, 0.0690, 2.0618) -- (0.4660, 0.0690, 2.0676) -- (0.4660, 0.1220, 2.0738) -- (0.4190, 0.1220, 2.0680) -- cycle;
\fill[blue!29.4, opacity=0.7] (0.4190, 0.1220, 2.0680) -- (0.4660, 0.1220, 2.0738) -- (0.4660, 0.1750, 2.0799) -- (0.4190, 0.1750, 2.0741) -- cycle;
\fill[blue!38.2, opacity=0.7] (0.4190, 0.1750, 2.0741) -- (0.4660, 0.1750, 2.0799) -- (0.4660, 0.2280, 2.0859) -- (0.4190, 0.2280, 2.0801) -- cycle;
\fill[blue!31.1, opacity=0.7] (0.4190, 0.2280, 2.0801) -- (0.4660, 0.2280, 2.0859) -- (0.4660, 0.2810, 2.0918) -- (0.4190, 0.2810, 2.0860) -- cycle;
\fill[blue!19.6, opacity=0.7] (0.4190, 0.2810, 2.0860) -- (0.4660, 0.2810, 2.0918) -- (0.4660, 0.3340, 2.0976) -- (0.4190, 0.3340, 2.0918) -- cycle;
\fill[blue!15.6, opacity=0.7] (0.4190, 0.3340, 2.0918) -- (0.4660, 0.3340, 2.0976) -- (0.4660, 0.3870, 2.1033) -- (0.4190, 0.3870, 2.0975) -- cycle;
\fill[blue!15.1, opacity=0.7] (0.4190, 0.3870, 2.0975) -- (0.4660, 0.3870, 2.1033) -- (0.4660, 0.4400, 2.1088) -- (0.4190, 0.4400, 2.1030) -- cycle;
\fill[blue!15.0, opacity=0.7] (0.4190, 0.4400, 2.1030) -- (0.4660, 0.4400, 2.1088) -- (0.4660, 0.4930, 2.1142) -- (0.4190, 0.4930, 2.1084) -- cycle;
\fill[blue!15.0, opacity=0.7] (0.4190, 0.4930, 2.1084) -- (0.4660, 0.4930, 2.1142) -- (0.4660, 0.5460, 2.1193) -- (0.4190, 0.5460, 2.1135) -- cycle;
\fill[blue!15.3, opacity=0.7] (0.4190, 0.5460, 2.1135) -- (0.4660, 0.5460, 2.1193) -- (0.4660, 0.5990, 2.1243) -- (0.4190, 0.5990, 2.1185) -- cycle;
\fill[blue!18.1, opacity=0.7] (0.4190, 0.5990, 2.1185) -- (0.4660, 0.5990, 2.1243) -- (0.4660, 0.6520, 2.1291) -- (0.4190, 0.6520, 2.1233) -- cycle;
\fill[blue!32.7, opacity=0.7] (0.4190, 0.6520, 2.1233) -- (0.4660, 0.6520, 2.1291) -- (0.4660, 0.7050, 2.1337) -- (0.4190, 0.7050, 2.1279) -- cycle;
\fill[blue!60.1, opacity=0.7] (0.4190, 0.7050, 2.1279) -- (0.4660, 0.7050, 2.1337) -- (0.4660, 0.7580, 2.1380) -- (0.4190, 0.7580, 2.1322) -- cycle;
\fill[blue!78.4, opacity=0.7] (0.4190, 0.7580, 2.1322) -- (0.4660, 0.7580, 2.1380) -- (0.4660, 0.8110, 2.1421) -- (0.4190, 0.8110, 2.1363) -- cycle;
\fill[blue!83.0, opacity=0.7] (0.4190, 0.8110, 2.1363) -- (0.4660, 0.8110, 2.1421) -- (0.4660, 0.8640, 2.1459) -- (0.4190, 0.8640, 2.1401) -- cycle;
\fill[blue!79.4, opacity=0.7] (0.4190, 0.8640, 2.1401) -- (0.4660, 0.8640, 2.1459) -- (0.4660, 0.9170, 2.1494) -- (0.4190, 0.9170, 2.1436) -- cycle;
\fill[blue!66.6, opacity=0.7] (0.4190, 0.9170, 2.1436) -- (0.4660, 0.9170, 2.1494) -- (0.4660, 0.9700, 2.1527) -- (0.4190, 0.9700, 2.1469) -- cycle;
\fill[blue!47.1, opacity=0.7] (0.4190, 0.9700, 2.1469) -- (0.4660, 0.9700, 2.1527) -- (0.4660, 1.0230, 2.1557) -- (0.4190, 1.0230, 2.1499) -- cycle;
\fill[blue!30.6, opacity=0.7] (0.4190, 1.0230, 2.1499) -- (0.4660, 1.0230, 2.1557) -- (0.4660, 1.0760, 2.1584) -- (0.4190, 1.0760, 2.1526) -- cycle;
\fill[blue!21.6, opacity=0.7] (0.4190, 1.0760, 2.1526) -- (0.4660, 1.0760, 2.1584) -- (0.4660, 1.1290, 2.1608) -- (0.4190, 1.1290, 2.1550) -- cycle;
\fill[blue!18.0, opacity=0.7] (0.4190, 1.1290, 2.1550) -- (0.4660, 1.1290, 2.1608) -- (0.4660, 1.1820, 2.1629) -- (0.4190, 1.1820, 2.1571) -- cycle;
\fill[blue!16.6, opacity=0.7] (0.4190, 1.1820, 2.1571) -- (0.4660, 1.1820, 2.1629) -- (0.4660, 1.2350, 2.1647) -- (0.4190, 1.2350, 2.1589) -- cycle;
\fill[blue!16.1, opacity=0.7] (0.4190, 1.2350, 2.1589) -- (0.4660, 1.2350, 2.1647) -- (0.4660, 1.2880, 2.1662) -- (0.4190, 1.2880, 2.1604) -- cycle;
\fill[blue!16.0, opacity=0.7] (0.4190, 1.2880, 2.1604) -- (0.4660, 1.2880, 2.1662) -- (0.4660, 1.3410, 2.1673) -- (0.4190, 1.3410, 2.1615) -- cycle;
\fill[blue!16.0, opacity=0.7] (0.4190, 1.3410, 2.1615) -- (0.4660, 1.3410, 2.1673) -- (0.4660, 1.3940, 2.1682) -- (0.4190, 1.3940, 2.1623) -- cycle;
\fill[blue!16.2, opacity=0.7] (0.4190, 1.3940, 2.1623) -- (0.4660, 1.3940, 2.1682) -- (0.4660, 1.4470, 2.1686) -- (0.4190, 1.4470, 2.1628) -- cycle;
\fill[blue!16.4, opacity=0.7] (0.4190, 1.4470, 2.1628) -- (0.4660, 1.4470, 2.1686) -- (0.4660, 1.5000, 2.1688) -- (0.4190, 1.5000, 2.1630) -- cycle;
\fill[blue!16.6, opacity=0.7] (0.4190, 1.5000, 2.1630) -- (0.4660, 1.5000, 2.1688) -- (0.4660, 1.5530, 2.1686) -- (0.4190, 1.5530, 2.1628) -- cycle;
\fill[blue!16.8, opacity=0.7] (0.4190, 1.5530, 2.1628) -- (0.4660, 1.5530, 2.1686) -- (0.4660, 1.6060, 2.1682) -- (0.4190, 1.6060, 2.1623) -- cycle;
\fill[blue!16.8, opacity=0.7] (0.4190, 1.6060, 2.1623) -- (0.4660, 1.6060, 2.1682) -- (0.4660, 1.6590, 2.1673) -- (0.4190, 1.6590, 2.1615) -- cycle;
\fill[blue!16.6, opacity=0.7] (0.4190, 1.6590, 2.1615) -- (0.4660, 1.6590, 2.1673) -- (0.4660, 1.7120, 2.1662) -- (0.4190, 1.7120, 2.1604) -- cycle;
\fill[blue!16.4, opacity=0.7] (0.4190, 1.7120, 2.1604) -- (0.4660, 1.7120, 2.1662) -- (0.4660, 1.7650, 2.1647) -- (0.4190, 1.7650, 2.1589) -- cycle;
\fill[blue!16.1, opacity=0.7] (0.4190, 1.7650, 2.1589) -- (0.4660, 1.7650, 2.1647) -- (0.4660, 1.8180, 2.1629) -- (0.4190, 1.8180, 2.1571) -- cycle;
\fill[blue!15.8, opacity=0.7] (0.4190, 1.8180, 2.1571) -- (0.4660, 1.8180, 2.1629) -- (0.4660, 1.8710, 2.1608) -- (0.4190, 1.8710, 2.1550) -- cycle;
\fill[blue!15.6, opacity=0.7] (0.4190, 1.8710, 2.1550) -- (0.4660, 1.8710, 2.1608) -- (0.4660, 1.9240, 2.1584) -- (0.4190, 1.9240, 2.1526) -- cycle;
\fill[blue!15.5, opacity=0.7] (0.4190, 1.9240, 2.1526) -- (0.4660, 1.9240, 2.1584) -- (0.4660, 1.9770, 2.1557) -- (0.4190, 1.9770, 2.1499) -- cycle;
\fill[blue!15.5, opacity=0.7] (0.4190, 1.9770, 2.1499) -- (0.4660, 1.9770, 2.1557) -- (0.4660, 2.0300, 2.1527) -- (0.4190, 2.0300, 2.1469) -- cycle;
\fill[blue!15.6, opacity=0.7] (0.4190, 2.0300, 2.1469) -- (0.4660, 2.0300, 2.1527) -- (0.4660, 2.0830, 2.1494) -- (0.4190, 2.0830, 2.1436) -- cycle;
\fill[blue!16.0, opacity=0.7] (0.4190, 2.0830, 2.1436) -- (0.4660, 2.0830, 2.1494) -- (0.4660, 2.1360, 2.1459) -- (0.4190, 2.1360, 2.1401) -- cycle;
\fill[blue!17.0, opacity=0.7] (0.4190, 2.1360, 2.1401) -- (0.4660, 2.1360, 2.1459) -- (0.4660, 2.1890, 2.1421) -- (0.4190, 2.1890, 2.1363) -- cycle;
\fill[blue!19.8, opacity=0.7] (0.4190, 2.1890, 2.1363) -- (0.4660, 2.1890, 2.1421) -- (0.4660, 2.2420, 2.1380) -- (0.4190, 2.2420, 2.1322) -- cycle;
\fill[blue!26.7, opacity=0.7] (0.4190, 2.2420, 2.1322) -- (0.4660, 2.2420, 2.1380) -- (0.4660, 2.2950, 2.1337) -- (0.4190, 2.2950, 2.1279) -- cycle;
\fill[blue!39.4, opacity=0.7] (0.4190, 2.2950, 2.1279) -- (0.4660, 2.2950, 2.1337) -- (0.4660, 2.3480, 2.1291) -- (0.4190, 2.3480, 2.1233) -- cycle;
\fill[blue!54.4, opacity=0.7] (0.4190, 2.3480, 2.1233) -- (0.4660, 2.3480, 2.1291) -- (0.4660, 2.4010, 2.1243) -- (0.4190, 2.4010, 2.1185) -- cycle;
\fill[blue!64.3, opacity=0.7] (0.4190, 2.4010, 2.1185) -- (0.4660, 2.4010, 2.1243) -- (0.4660, 2.4540, 2.1193) -- (0.4190, 2.4540, 2.1135) -- cycle;
\fill[blue!64.5, opacity=0.7] (0.4190, 2.4540, 2.1135) -- (0.4660, 2.4540, 2.1193) -- (0.4660, 2.5070, 2.1142) -- (0.4190, 2.5070, 2.1084) -- cycle;
\fill[blue!52.7, opacity=0.7] (0.4190, 2.5070, 2.1084) -- (0.4660, 2.5070, 2.1142) -- (0.4660, 2.5600, 2.1088) -- (0.4190, 2.5600, 2.1030) -- cycle;
\fill[blue!32.6, opacity=0.7] (0.4190, 2.5600, 2.1030) -- (0.4660, 2.5600, 2.1088) -- (0.4660, 2.6130, 2.1033) -- (0.4190, 2.6130, 2.0975) -- cycle;
\fill[blue!18.8, opacity=0.7] (0.4190, 2.6130, 2.0975) -- (0.4660, 2.6130, 2.1033) -- (0.4660, 2.6660, 2.0976) -- (0.4190, 2.6660, 2.0918) -- cycle;
\fill[blue!15.4, opacity=0.7] (0.4190, 2.6660, 2.0918) -- (0.4660, 2.6660, 2.0976) -- (0.4660, 2.7190, 2.0918) -- (0.4190, 2.7190, 2.0860) -- cycle;
\fill[blue!15.0, opacity=0.7] (0.4190, 2.7190, 2.0860) -- (0.4660, 2.7190, 2.0918) -- (0.4660, 2.7720, 2.0859) -- (0.4190, 2.7720, 2.0801) -- cycle;
\fill[blue!15.0, opacity=0.7] (0.4190, 2.7720, 2.0801) -- (0.4660, 2.7720, 2.0859) -- (0.4660, 2.8250, 2.0799) -- (0.4190, 2.8250, 2.0741) -- cycle;
\fill[blue!15.0, opacity=0.7] (0.4190, 2.8250, 2.0741) -- (0.4660, 2.8250, 2.0799) -- (0.4660, 2.8780, 2.0738) -- (0.4190, 2.8780, 2.0680) -- cycle;
\fill[blue!15.0, opacity=0.7] (0.4190, 2.8780, 2.0680) -- (0.4660, 2.8780, 2.0738) -- (0.4660, 2.9310, 2.0676) -- (0.4190, 2.9310, 2.0618) -- cycle;
\fill[blue!15.1, opacity=0.7] (0.4190, 2.9310, 2.0618) -- (0.4660, 2.9310, 2.0676) -- (0.4660, 2.9840, 2.0614) -- (0.4190, 2.9840, 2.0555) -- cycle;
\fill[blue!15.7, opacity=0.7] (0.4190, 2.9840, 2.0555) -- (0.4660, 2.9840, 2.0614) -- (0.4660, 3.0370, 2.0551) -- (0.4190, 3.0370, 2.0493) -- cycle;
\fill[blue!18.5, opacity=0.7] (0.4190, 3.0370, 2.0493) -- (0.4660, 3.0370, 2.0551) -- (0.4660, 3.0900, 2.0488) -- (0.4190, 3.0900, 2.0430) -- cycle;
\fill[blue!15.0, opacity=0.7] (0.4660, -0.0900, 2.0488) -- (0.5130, -0.0900, 2.0545) -- (0.5130, -0.0370, 2.0608) -- (0.4660, -0.0370, 2.0551) -- cycle;
\fill[blue!15.0, opacity=0.7] (0.4660, -0.0370, 2.0551) -- (0.5130, -0.0370, 2.0608) -- (0.5130, 0.0160, 2.0670) -- (0.4660, 0.0160, 2.0614) -- cycle;
\fill[blue!16.3, opacity=0.7] (0.4660, 0.0160, 2.0614) -- (0.5130, 0.0160, 2.0670) -- (0.5130, 0.0690, 2.0733) -- (0.4660, 0.0690, 2.0676) -- cycle;
\fill[blue!25.6, opacity=0.7] (0.4660, 0.0690, 2.0676) -- (0.5130, 0.0690, 2.0733) -- (0.5130, 0.1220, 2.0794) -- (0.4660, 0.1220, 2.0738) -- cycle;
\fill[blue!37.9, opacity=0.7] (0.4660, 0.1220, 2.0738) -- (0.5130, 0.1220, 2.0794) -- (0.5130, 0.1750, 2.0855) -- (0.4660, 0.1750, 2.0799) -- cycle;
\fill[blue!34.3, opacity=0.7] (0.4660, 0.1750, 2.0799) -- (0.5130, 0.1750, 2.0855) -- (0.5130, 0.2280, 2.0916) -- (0.4660, 0.2280, 2.0859) -- cycle;
\fill[blue!21.6, opacity=0.7] (0.4660, 0.2280, 2.0859) -- (0.5130, 0.2280, 2.0916) -- (0.5130, 0.2810, 2.0975) -- (0.4660, 0.2810, 2.0918) -- cycle;
\fill[blue!15.9, opacity=0.7] (0.4660, 0.2810, 2.0918) -- (0.5130, 0.2810, 2.0975) -- (0.5130, 0.3340, 2.1033) -- (0.4660, 0.3340, 2.0976) -- cycle;
\fill[blue!15.1, opacity=0.7] (0.4660, 0.3340, 2.0976) -- (0.5130, 0.3340, 2.1033) -- (0.5130, 0.3870, 2.1090) -- (0.4660, 0.3870, 2.1033) -- cycle;
\fill[blue!15.0, opacity=0.7] (0.4660, 0.3870, 2.1033) -- (0.5130, 0.3870, 2.1090) -- (0.5130, 0.4400, 2.1145) -- (0.4660, 0.4400, 2.1088) -- cycle;
\fill[blue!15.1, opacity=0.7] (0.4660, 0.4400, 2.1088) -- (0.5130, 0.4400, 2.1145) -- (0.5130, 0.4930, 2.1198) -- (0.4660, 0.4930, 2.1142) -- cycle;
\fill[blue!15.4, opacity=0.7] (0.4660, 0.4930, 2.1142) -- (0.5130, 0.4930, 2.1198) -- (0.5130, 0.5460, 2.1250) -- (0.4660, 0.5460, 2.1193) -- cycle;
\fill[blue!18.8, opacity=0.7] (0.4660, 0.5460, 2.1193) -- (0.5130, 0.5460, 2.1250) -- (0.5130, 0.5990, 2.1300) -- (0.4660, 0.5990, 2.1243) -- cycle;
\fill[blue!36.6, opacity=0.7] (0.4660, 0.5990, 2.1243) -- (0.5130, 0.5990, 2.1300) -- (0.5130, 0.6520, 2.1348) -- (0.4660, 0.6520, 2.1291) -- cycle;
\fill[blue!66.0, opacity=0.7] (0.4660, 0.6520, 2.1291) -- (0.5130, 0.6520, 2.1348) -- (0.5130, 0.7050, 2.1393) -- (0.4660, 0.7050, 2.1337) -- cycle;
\fill[blue!81.5, opacity=0.7] (0.4660, 0.7050, 2.1337) -- (0.5130, 0.7050, 2.1393) -- (0.5130, 0.7580, 2.1437) -- (0.4660, 0.7580, 2.1380) -- cycle;
\fill[blue!83.2, opacity=0.7] (0.4660, 0.7580, 2.1380) -- (0.5130, 0.7580, 2.1437) -- (0.5130, 0.8110, 2.1477) -- (0.4660, 0.8110, 2.1421) -- cycle;
\fill[blue!75.7, opacity=0.7] (0.4660, 0.8110, 2.1421) -- (0.5130, 0.8110, 2.1477) -- (0.5130, 0.8640, 2.1516) -- (0.4660, 0.8640, 2.1459) -- cycle;
\fill[blue!56.7, opacity=0.7] (0.4660, 0.8640, 2.1459) -- (0.5130, 0.8640, 2.1516) -- (0.5130, 0.9170, 2.1551) -- (0.4660, 0.9170, 2.1494) -- cycle;
\fill[blue!35.3, opacity=0.7] (0.4660, 0.9170, 2.1494) -- (0.5130, 0.9170, 2.1551) -- (0.5130, 0.9700, 2.1584) -- (0.4660, 0.9700, 2.1527) -- cycle;
\fill[blue!22.8, opacity=0.7] (0.4660, 0.9700, 2.1527) -- (0.5130, 0.9700, 2.1584) -- (0.5130, 1.0230, 2.1614) -- (0.4660, 1.0230, 2.1557) -- cycle;
\fill[blue!18.0, opacity=0.7] (0.4660, 1.0230, 2.1557) -- (0.5130, 1.0230, 2.1614) -- (0.5130, 1.0760, 2.1641) -- (0.4660, 1.0760, 2.1584) -- cycle;
\fill[blue!16.5, opacity=0.7] (0.4660, 1.0760, 2.1584) -- (0.5130, 1.0760, 2.1641) -- (0.5130, 1.1290, 2.1665) -- (0.4660, 1.1290, 2.1608) -- cycle;
\fill[blue!16.2, opacity=0.7] (0.4660, 1.1290, 2.1608) -- (0.5130, 1.1290, 2.1665) -- (0.5130, 1.1820, 2.1686) -- (0.4660, 1.1820, 2.1629) -- cycle;
\fill[blue!16.3, opacity=0.7] (0.4660, 1.1820, 2.1629) -- (0.5130, 1.1820, 2.1686) -- (0.5130, 1.2350, 2.1704) -- (0.4660, 1.2350, 2.1647) -- cycle;
\fill[blue!16.9, opacity=0.7] (0.4660, 1.2350, 2.1647) -- (0.5130, 1.2350, 2.1704) -- (0.5130, 1.2880, 2.1719) -- (0.4660, 1.2880, 2.1662) -- cycle;
\fill[blue!18.1, opacity=0.7] (0.4660, 1.2880, 2.1662) -- (0.5130, 1.2880, 2.1719) -- (0.5130, 1.3410, 2.1730) -- (0.4660, 1.3410, 2.1673) -- cycle;
\fill[blue!20.2, opacity=0.7] (0.4660, 1.3410, 2.1673) -- (0.5130, 1.3410, 2.1730) -- (0.5130, 1.3940, 2.1738) -- (0.4660, 1.3940, 2.1682) -- cycle;
\fill[blue!23.1, opacity=0.7] (0.4660, 1.3940, 2.1682) -- (0.5130, 1.3940, 2.1738) -- (0.5130, 1.4470, 2.1743) -- (0.4660, 1.4470, 2.1686) -- cycle;
\fill[blue!26.5, opacity=0.7] (0.4660, 1.4470, 2.1686) -- (0.5130, 1.4470, 2.1743) -- (0.5130, 1.5000, 2.1745) -- (0.4660, 1.5000, 2.1688) -- cycle;
\fill[blue!29.6, opacity=0.7] (0.4660, 1.5000, 2.1688) -- (0.5130, 1.5000, 2.1745) -- (0.5130, 1.5530, 2.1743) -- (0.4660, 1.5530, 2.1686) -- cycle;
\fill[blue!31.7, opacity=0.7] (0.4660, 1.5530, 2.1686) -- (0.5130, 1.5530, 2.1743) -- (0.5130, 1.6060, 2.1738) -- (0.4660, 1.6060, 2.1682) -- cycle;
\fill[blue!32.2, opacity=0.7] (0.4660, 1.6060, 2.1682) -- (0.5130, 1.6060, 2.1738) -- (0.5130, 1.6590, 2.1730) -- (0.4660, 1.6590, 2.1673) -- cycle;
\fill[blue!31.1, opacity=0.7] (0.4660, 1.6590, 2.1673) -- (0.5130, 1.6590, 2.1730) -- (0.5130, 1.7120, 2.1719) -- (0.4660, 1.7120, 2.1662) -- cycle;
\fill[blue!28.6, opacity=0.7] (0.4660, 1.7120, 2.1662) -- (0.5130, 1.7120, 2.1719) -- (0.5130, 1.7650, 2.1704) -- (0.4660, 1.7650, 2.1647) -- cycle;
\fill[blue!25.2, opacity=0.7] (0.4660, 1.7650, 2.1647) -- (0.5130, 1.7650, 2.1704) -- (0.5130, 1.8180, 2.1686) -- (0.4660, 1.8180, 2.1629) -- cycle;
\fill[blue!21.8, opacity=0.7] (0.4660, 1.8180, 2.1629) -- (0.5130, 1.8180, 2.1686) -- (0.5130, 1.8710, 2.1665) -- (0.4660, 1.8710, 2.1608) -- cycle;
\fill[blue!19.0, opacity=0.7] (0.4660, 1.8710, 2.1608) -- (0.5130, 1.8710, 2.1665) -- (0.5130, 1.9240, 2.1641) -- (0.4660, 1.9240, 2.1584) -- cycle;
\fill[blue!17.2, opacity=0.7] (0.4660, 1.9240, 2.1584) -- (0.5130, 1.9240, 2.1641) -- (0.5130, 1.9770, 2.1614) -- (0.4660, 1.9770, 2.1557) -- cycle;
\fill[blue!16.1, opacity=0.7] (0.4660, 1.9770, 2.1557) -- (0.5130, 1.9770, 2.1614) -- (0.5130, 2.0300, 2.1584) -- (0.4660, 2.0300, 2.1527) -- cycle;
\fill[blue!15.6, opacity=0.7] (0.4660, 2.0300, 2.1527) -- (0.5130, 2.0300, 2.1584) -- (0.5130, 2.0830, 2.1551) -- (0.4660, 2.0830, 2.1494) -- cycle;
\fill[blue!15.5, opacity=0.7] (0.4660, 2.0830, 2.1494) -- (0.5130, 2.0830, 2.1551) -- (0.5130, 2.1360, 2.1516) -- (0.4660, 2.1360, 2.1459) -- cycle;
\fill[blue!15.5, opacity=0.7] (0.4660, 2.1360, 2.1459) -- (0.5130, 2.1360, 2.1516) -- (0.5130, 2.1890, 2.1477) -- (0.4660, 2.1890, 2.1421) -- cycle;
\fill[blue!15.7, opacity=0.7] (0.4660, 2.1890, 2.1421) -- (0.5130, 2.1890, 2.1477) -- (0.5130, 2.2420, 2.1437) -- (0.4660, 2.2420, 2.1380) -- cycle;
\fill[blue!16.7, opacity=0.7] (0.4660, 2.2420, 2.1380) -- (0.5130, 2.2420, 2.1437) -- (0.5130, 2.2950, 2.1393) -- (0.4660, 2.2950, 2.1337) -- cycle;
\fill[blue!19.9, opacity=0.7] (0.4660, 2.2950, 2.1337) -- (0.5130, 2.2950, 2.1393) -- (0.5130, 2.3480, 2.1348) -- (0.4660, 2.3480, 2.1291) -- cycle;
\fill[blue!28.6, opacity=0.7] (0.4660, 2.3480, 2.1291) -- (0.5130, 2.3480, 2.1348) -- (0.5130, 2.4010, 2.1300) -- (0.4660, 2.4010, 2.1243) -- cycle;
\fill[blue!44.1, opacity=0.7] (0.4660, 2.4010, 2.1243) -- (0.5130, 2.4010, 2.1300) -- (0.5130, 2.4540, 2.1250) -- (0.4660, 2.4540, 2.1193) -- cycle;
\fill[blue!59.2, opacity=0.7] (0.4660, 2.4540, 2.1193) -- (0.5130, 2.4540, 2.1250) -- (0.5130, 2.5070, 2.1198) -- (0.4660, 2.5070, 2.1142) -- cycle;
\fill[blue!64.7, opacity=0.7] (0.4660, 2.5070, 2.1142) -- (0.5130, 2.5070, 2.1198) -- (0.5130, 2.5600, 2.1145) -- (0.4660, 2.5600, 2.1088) -- cycle;
\fill[blue!56.9, opacity=0.7] (0.4660, 2.5600, 2.1088) -- (0.5130, 2.5600, 2.1145) -- (0.5130, 2.6130, 2.1090) -- (0.4660, 2.6130, 2.1033) -- cycle;
\fill[blue!37.1, opacity=0.7] (0.4660, 2.6130, 2.1033) -- (0.5130, 2.6130, 2.1090) -- (0.5130, 2.6660, 2.1033) -- (0.4660, 2.6660, 2.0976) -- cycle;
\fill[blue!20.2, opacity=0.7] (0.4660, 2.6660, 2.0976) -- (0.5130, 2.6660, 2.1033) -- (0.5130, 2.7190, 2.0975) -- (0.4660, 2.7190, 2.0918) -- cycle;
\fill[blue!15.5, opacity=0.7] (0.4660, 2.7190, 2.0918) -- (0.5130, 2.7190, 2.0975) -- (0.5130, 2.7720, 2.0916) -- (0.4660, 2.7720, 2.0859) -- cycle;
\fill[blue!15.0, opacity=0.7] (0.4660, 2.7720, 2.0859) -- (0.5130, 2.7720, 2.0916) -- (0.5130, 2.8250, 2.0855) -- (0.4660, 2.8250, 2.0799) -- cycle;
\fill[blue!15.0, opacity=0.7] (0.4660, 2.8250, 2.0799) -- (0.5130, 2.8250, 2.0855) -- (0.5130, 2.8780, 2.0794) -- (0.4660, 2.8780, 2.0738) -- cycle;
\fill[blue!15.0, opacity=0.7] (0.4660, 2.8780, 2.0738) -- (0.5130, 2.8780, 2.0794) -- (0.5130, 2.9310, 2.0733) -- (0.4660, 2.9310, 2.0676) -- cycle;
\fill[blue!15.0, opacity=0.7] (0.4660, 2.9310, 2.0676) -- (0.5130, 2.9310, 2.0733) -- (0.5130, 2.9840, 2.0670) -- (0.4660, 2.9840, 2.0614) -- cycle;
\fill[blue!15.1, opacity=0.7] (0.4660, 2.9840, 2.0614) -- (0.5130, 2.9840, 2.0670) -- (0.5130, 3.0370, 2.0608) -- (0.4660, 3.0370, 2.0551) -- cycle;
\fill[blue!15.9, opacity=0.7] (0.4660, 3.0370, 2.0551) -- (0.5130, 3.0370, 2.0608) -- (0.5130, 3.0900, 2.0545) -- (0.4660, 3.0900, 2.0488) -- cycle;
\fill[blue!15.0, opacity=0.7] (0.5130, -0.0900, 2.0545) -- (0.5600, -0.0900, 2.0600) -- (0.5600, -0.0370, 2.0663) -- (0.5130, -0.0370, 2.0608) -- cycle;
\fill[blue!15.4, opacity=0.7] (0.5130, -0.0370, 2.0608) -- (0.5600, -0.0370, 2.0663) -- (0.5600, 0.0160, 2.0725) -- (0.5130, 0.0160, 2.0670) -- cycle;
\fill[blue!21.3, opacity=0.7] (0.5130, 0.0160, 2.0670) -- (0.5600, 0.0160, 2.0725) -- (0.5600, 0.0690, 2.0788) -- (0.5130, 0.0690, 2.0733) -- cycle;
\fill[blue!35.7, opacity=0.7] (0.5130, 0.0690, 2.0733) -- (0.5600, 0.0690, 2.0788) -- (0.5600, 0.1220, 2.0849) -- (0.5130, 0.1220, 2.0794) -- cycle;
\fill[blue!37.6, opacity=0.7] (0.5130, 0.1220, 2.0794) -- (0.5600, 0.1220, 2.0849) -- (0.5600, 0.1750, 2.0911) -- (0.5130, 0.1750, 2.0855) -- cycle;
\fill[blue!24.9, opacity=0.7] (0.5130, 0.1750, 2.0855) -- (0.5600, 0.1750, 2.0911) -- (0.5600, 0.2280, 2.0971) -- (0.5130, 0.2280, 2.0916) -- cycle;
\fill[blue!16.6, opacity=0.7] (0.5130, 0.2280, 2.0916) -- (0.5600, 0.2280, 2.0971) -- (0.5600, 0.2810, 2.1030) -- (0.5130, 0.2810, 2.0975) -- cycle;
\fill[blue!15.2, opacity=0.7] (0.5130, 0.2810, 2.0975) -- (0.5600, 0.2810, 2.1030) -- (0.5600, 0.3340, 2.1088) -- (0.5130, 0.3340, 2.1033) -- cycle;
\fill[blue!15.0, opacity=0.7] (0.5130, 0.3340, 2.1033) -- (0.5600, 0.3340, 2.1088) -- (0.5600, 0.3870, 2.1145) -- (0.5130, 0.3870, 2.1090) -- cycle;
\fill[blue!15.0, opacity=0.7] (0.5130, 0.3870, 2.1090) -- (0.5600, 0.3870, 2.1145) -- (0.5600, 0.4400, 2.1200) -- (0.5130, 0.4400, 2.1145) -- cycle;
\fill[blue!15.3, opacity=0.7] (0.5130, 0.4400, 2.1145) -- (0.5600, 0.4400, 2.1200) -- (0.5600, 0.4930, 2.1254) -- (0.5130, 0.4930, 2.1198) -- cycle;
\fill[blue!18.7, opacity=0.7] (0.5130, 0.4930, 2.1198) -- (0.5600, 0.4930, 2.1254) -- (0.5600, 0.5460, 2.1305) -- (0.5130, 0.5460, 2.1250) -- cycle;
\fill[blue!37.7, opacity=0.7] (0.5130, 0.5460, 2.1250) -- (0.5600, 0.5460, 2.1305) -- (0.5600, 0.5990, 2.1355) -- (0.5130, 0.5990, 2.1300) -- cycle;
\fill[blue!68.7, opacity=0.7] (0.5130, 0.5990, 2.1300) -- (0.5600, 0.5990, 2.1355) -- (0.5600, 0.6520, 2.1403) -- (0.5130, 0.6520, 2.1348) -- cycle;
\fill[blue!82.9, opacity=0.7] (0.5130, 0.6520, 2.1348) -- (0.5600, 0.6520, 2.1403) -- (0.5600, 0.7050, 2.1449) -- (0.5130, 0.7050, 2.1393) -- cycle;
\fill[blue!83.2, opacity=0.7] (0.5130, 0.7050, 2.1393) -- (0.5600, 0.7050, 2.1449) -- (0.5600, 0.7580, 2.1492) -- (0.5130, 0.7580, 2.1437) -- cycle;
\fill[blue!72.2, opacity=0.7] (0.5130, 0.7580, 2.1437) -- (0.5600, 0.7580, 2.1492) -- (0.5600, 0.8110, 2.1533) -- (0.5130, 0.8110, 2.1477) -- cycle;
\fill[blue!48.9, opacity=0.7] (0.5130, 0.8110, 2.1477) -- (0.5600, 0.8110, 2.1533) -- (0.5600, 0.8640, 2.1571) -- (0.5130, 0.8640, 2.1516) -- cycle;
\fill[blue!28.4, opacity=0.7] (0.5130, 0.8640, 2.1516) -- (0.5600, 0.8640, 2.1571) -- (0.5600, 0.9170, 2.1606) -- (0.5130, 0.9170, 2.1551) -- cycle;
\fill[blue!19.5, opacity=0.7] (0.5130, 0.9170, 2.1551) -- (0.5600, 0.9170, 2.1606) -- (0.5600, 0.9700, 2.1639) -- (0.5130, 0.9700, 2.1584) -- cycle;
\fill[blue!16.9, opacity=0.7] (0.5130, 0.9700, 2.1584) -- (0.5600, 0.9700, 2.1639) -- (0.5600, 1.0230, 2.1669) -- (0.5130, 1.0230, 2.1614) -- cycle;
\fill[blue!16.3, opacity=0.7] (0.5130, 1.0230, 2.1614) -- (0.5600, 1.0230, 2.1669) -- (0.5600, 1.0760, 2.1696) -- (0.5130, 1.0760, 2.1641) -- cycle;
\fill[blue!16.6, opacity=0.7] (0.5130, 1.0760, 2.1641) -- (0.5600, 1.0760, 2.1696) -- (0.5600, 1.1290, 2.1720) -- (0.5130, 1.1290, 2.1665) -- cycle;
\fill[blue!17.9, opacity=0.7] (0.5130, 1.1290, 2.1665) -- (0.5600, 1.1290, 2.1720) -- (0.5600, 1.1820, 2.1741) -- (0.5130, 1.1820, 2.1686) -- cycle;
\fill[blue!21.3, opacity=0.7] (0.5130, 1.1820, 2.1686) -- (0.5600, 1.1820, 2.1741) -- (0.5600, 1.2350, 2.1759) -- (0.5130, 1.2350, 2.1704) -- cycle;
\fill[blue!28.4, opacity=0.7] (0.5130, 1.2350, 2.1704) -- (0.5600, 1.2350, 2.1759) -- (0.5600, 1.2880, 2.1774) -- (0.5130, 1.2880, 2.1719) -- cycle;
\fill[blue!39.6, opacity=0.7] (0.5130, 1.2880, 2.1719) -- (0.5600, 1.2880, 2.1774) -- (0.5600, 1.3410, 2.1785) -- (0.5130, 1.3410, 2.1730) -- cycle;
\fill[blue!52.8, opacity=0.7] (0.5130, 1.3410, 2.1730) -- (0.5600, 1.3410, 2.1785) -- (0.5600, 1.3940, 2.1793) -- (0.5130, 1.3940, 2.1738) -- cycle;
\fill[blue!64.7, opacity=0.7] (0.5130, 1.3940, 2.1738) -- (0.5600, 1.3940, 2.1793) -- (0.5600, 1.4470, 2.1798) -- (0.5130, 1.4470, 2.1743) -- cycle;
\fill[blue!73.3, opacity=0.7] (0.5130, 1.4470, 2.1743) -- (0.5600, 1.4470, 2.1798) -- (0.5600, 1.5000, 2.1800) -- (0.5130, 1.5000, 2.1745) -- cycle;
\fill[blue!78.4, opacity=0.7] (0.5130, 1.5000, 2.1745) -- (0.5600, 1.5000, 2.1800) -- (0.5600, 1.5530, 2.1798) -- (0.5130, 1.5530, 2.1743) -- cycle;
\fill[blue!80.8, opacity=0.7] (0.5130, 1.5530, 2.1743) -- (0.5600, 1.5530, 2.1798) -- (0.5600, 1.6060, 2.1793) -- (0.5130, 1.6060, 2.1738) -- cycle;
\fill[blue!81.3, opacity=0.7] (0.5130, 1.6060, 2.1738) -- (0.5600, 1.6060, 2.1793) -- (0.5600, 1.6590, 2.1785) -- (0.5130, 1.6590, 2.1730) -- cycle;
\fill[blue!80.1, opacity=0.7] (0.5130, 1.6590, 2.1730) -- (0.5600, 1.6590, 2.1785) -- (0.5600, 1.7120, 2.1774) -- (0.5130, 1.7120, 2.1719) -- cycle;
\fill[blue!76.9, opacity=0.7] (0.5130, 1.7120, 2.1719) -- (0.5600, 1.7120, 2.1774) -- (0.5600, 1.7650, 2.1759) -- (0.5130, 1.7650, 2.1704) -- cycle;
\fill[blue!70.9, opacity=0.7] (0.5130, 1.7650, 2.1704) -- (0.5600, 1.7650, 2.1759) -- (0.5600, 1.8180, 2.1741) -- (0.5130, 1.8180, 2.1686) -- cycle;
\fill[blue!61.5, opacity=0.7] (0.5130, 1.8180, 2.1686) -- (0.5600, 1.8180, 2.1741) -- (0.5600, 1.8710, 2.1720) -- (0.5130, 1.8710, 2.1665) -- cycle;
\fill[blue!49.2, opacity=0.7] (0.5130, 1.8710, 2.1665) -- (0.5600, 1.8710, 2.1720) -- (0.5600, 1.9240, 2.1696) -- (0.5130, 1.9240, 2.1641) -- cycle;
\fill[blue!36.3, opacity=0.7] (0.5130, 1.9240, 2.1641) -- (0.5600, 1.9240, 2.1696) -- (0.5600, 1.9770, 2.1669) -- (0.5130, 1.9770, 2.1614) -- cycle;
\fill[blue!25.9, opacity=0.7] (0.5130, 1.9770, 2.1614) -- (0.5600, 1.9770, 2.1669) -- (0.5600, 2.0300, 2.1639) -- (0.5130, 2.0300, 2.1584) -- cycle;
\fill[blue!19.6, opacity=0.7] (0.5130, 2.0300, 2.1584) -- (0.5600, 2.0300, 2.1639) -- (0.5600, 2.0830, 2.1606) -- (0.5130, 2.0830, 2.1551) -- cycle;
\fill[blue!16.8, opacity=0.7] (0.5130, 2.0830, 2.1551) -- (0.5600, 2.0830, 2.1606) -- (0.5600, 2.1360, 2.1571) -- (0.5130, 2.1360, 2.1516) -- cycle;
\fill[blue!15.7, opacity=0.7] (0.5130, 2.1360, 2.1516) -- (0.5600, 2.1360, 2.1571) -- (0.5600, 2.1890, 2.1533) -- (0.5130, 2.1890, 2.1477) -- cycle;
\fill[blue!15.4, opacity=0.7] (0.5130, 2.1890, 2.1477) -- (0.5600, 2.1890, 2.1533) -- (0.5600, 2.2420, 2.1492) -- (0.5130, 2.2420, 2.1437) -- cycle;
\fill[blue!15.4, opacity=0.7] (0.5130, 2.2420, 2.1437) -- (0.5600, 2.2420, 2.1492) -- (0.5600, 2.2950, 2.1449) -- (0.5130, 2.2950, 2.1393) -- cycle;
\fill[blue!15.7, opacity=0.7] (0.5130, 2.2950, 2.1393) -- (0.5600, 2.2950, 2.1449) -- (0.5600, 2.3480, 2.1403) -- (0.5130, 2.3480, 2.1348) -- cycle;
\fill[blue!17.2, opacity=0.7] (0.5130, 2.3480, 2.1348) -- (0.5600, 2.3480, 2.1403) -- (0.5600, 2.4010, 2.1355) -- (0.5130, 2.4010, 2.1300) -- cycle;
\fill[blue!22.4, opacity=0.7] (0.5130, 2.4010, 2.1300) -- (0.5600, 2.4010, 2.1355) -- (0.5600, 2.4540, 2.1305) -- (0.5130, 2.4540, 2.1250) -- cycle;
\fill[blue!35.8, opacity=0.7] (0.5130, 2.4540, 2.1250) -- (0.5600, 2.4540, 2.1305) -- (0.5600, 2.5070, 2.1254) -- (0.5130, 2.5070, 2.1198) -- cycle;
\fill[blue!53.5, opacity=0.7] (0.5130, 2.5070, 2.1198) -- (0.5600, 2.5070, 2.1254) -- (0.5600, 2.5600, 2.1200) -- (0.5130, 2.5600, 2.1145) -- cycle;
\fill[blue!63.4, opacity=0.7] (0.5130, 2.5600, 2.1145) -- (0.5600, 2.5600, 2.1200) -- (0.5600, 2.6130, 2.1145) -- (0.5130, 2.6130, 2.1090) -- cycle;
\fill[blue!58.8, opacity=0.7] (0.5130, 2.6130, 2.1090) -- (0.5600, 2.6130, 2.1145) -- (0.5600, 2.6660, 2.1088) -- (0.5130, 2.6660, 2.1033) -- cycle;
\fill[blue!39.5, opacity=0.7] (0.5130, 2.6660, 2.1033) -- (0.5600, 2.6660, 2.1088) -- (0.5600, 2.7190, 2.1030) -- (0.5130, 2.7190, 2.0975) -- cycle;
\fill[blue!20.9, opacity=0.7] (0.5130, 2.7190, 2.0975) -- (0.5600, 2.7190, 2.1030) -- (0.5600, 2.7720, 2.0971) -- (0.5130, 2.7720, 2.0916) -- cycle;
\fill[blue!15.5, opacity=0.7] (0.5130, 2.7720, 2.0916) -- (0.5600, 2.7720, 2.0971) -- (0.5600, 2.8250, 2.0911) -- (0.5130, 2.8250, 2.0855) -- cycle;
\fill[blue!15.0, opacity=0.7] (0.5130, 2.8250, 2.0855) -- (0.5600, 2.8250, 2.0911) -- (0.5600, 2.8780, 2.0849) -- (0.5130, 2.8780, 2.0794) -- cycle;
\fill[blue!15.0, opacity=0.7] (0.5130, 2.8780, 2.0794) -- (0.5600, 2.8780, 2.0849) -- (0.5600, 2.9310, 2.0788) -- (0.5130, 2.9310, 2.0733) -- cycle;
\fill[blue!15.0, opacity=0.7] (0.5130, 2.9310, 2.0733) -- (0.5600, 2.9310, 2.0788) -- (0.5600, 2.9840, 2.0725) -- (0.5130, 2.9840, 2.0670) -- cycle;
\fill[blue!15.0, opacity=0.7] (0.5130, 2.9840, 2.0670) -- (0.5600, 2.9840, 2.0725) -- (0.5600, 3.0370, 2.0663) -- (0.5130, 3.0370, 2.0608) -- cycle;
\fill[blue!15.1, opacity=0.7] (0.5130, 3.0370, 2.0608) -- (0.5600, 3.0370, 2.0663) -- (0.5600, 3.0900, 2.0600) -- (0.5130, 3.0900, 2.0545) -- cycle;
\fill[blue!15.1, opacity=0.7] (0.5600, -0.0900, 2.0600) -- (0.6070, -0.0900, 2.0654) -- (0.6070, -0.0370, 2.0716) -- (0.5600, -0.0370, 2.0663) -- cycle;
\fill[blue!17.6, opacity=0.7] (0.5600, -0.0370, 2.0663) -- (0.6070, -0.0370, 2.0716) -- (0.6070, 0.0160, 2.0779) -- (0.5600, 0.0160, 2.0725) -- cycle;
\fill[blue!30.8, opacity=0.7] (0.5600, 0.0160, 2.0725) -- (0.6070, 0.0160, 2.0779) -- (0.6070, 0.0690, 2.0841) -- (0.5600, 0.0690, 2.0788) -- cycle;
\fill[blue!39.8, opacity=0.7] (0.5600, 0.0690, 2.0788) -- (0.6070, 0.0690, 2.0841) -- (0.6070, 0.1220, 2.0903) -- (0.5600, 0.1220, 2.0849) -- cycle;
\fill[blue!30.0, opacity=0.7] (0.5600, 0.1220, 2.0849) -- (0.6070, 0.1220, 2.0903) -- (0.6070, 0.1750, 2.0964) -- (0.5600, 0.1750, 2.0911) -- cycle;
\fill[blue!18.2, opacity=0.7] (0.5600, 0.1750, 2.0911) -- (0.6070, 0.1750, 2.0964) -- (0.6070, 0.2280, 2.1024) -- (0.5600, 0.2280, 2.0971) -- cycle;
\fill[blue!15.3, opacity=0.7] (0.5600, 0.2280, 2.0971) -- (0.6070, 0.2280, 2.1024) -- (0.6070, 0.2810, 2.1084) -- (0.5600, 0.2810, 2.1030) -- cycle;
\fill[blue!15.0, opacity=0.7] (0.5600, 0.2810, 2.1030) -- (0.6070, 0.2810, 2.1084) -- (0.6070, 0.3340, 2.1142) -- (0.5600, 0.3340, 2.1088) -- cycle;
\fill[blue!15.0, opacity=0.7] (0.5600, 0.3340, 2.1088) -- (0.6070, 0.3340, 2.1142) -- (0.6070, 0.3870, 2.1198) -- (0.5600, 0.3870, 2.1145) -- cycle;
\fill[blue!15.2, opacity=0.7] (0.5600, 0.3870, 2.1145) -- (0.6070, 0.3870, 2.1198) -- (0.6070, 0.4400, 2.1254) -- (0.5600, 0.4400, 2.1200) -- cycle;
\fill[blue!17.8, opacity=0.7] (0.5600, 0.4400, 2.1200) -- (0.6070, 0.4400, 2.1254) -- (0.6070, 0.4930, 2.1307) -- (0.5600, 0.4930, 2.1254) -- cycle;
\fill[blue!35.7, opacity=0.7] (0.5600, 0.4930, 2.1254) -- (0.6070, 0.4930, 2.1307) -- (0.6070, 0.5460, 2.1359) -- (0.5600, 0.5460, 2.1305) -- cycle;
\fill[blue!68.6, opacity=0.7] (0.5600, 0.5460, 2.1305) -- (0.6070, 0.5460, 2.1359) -- (0.6070, 0.5990, 2.1409) -- (0.5600, 0.5990, 2.1355) -- cycle;
\fill[blue!83.5, opacity=0.7] (0.5600, 0.5990, 2.1355) -- (0.6070, 0.5990, 2.1409) -- (0.6070, 0.6520, 2.1457) -- (0.5600, 0.6520, 2.1403) -- cycle;
\fill[blue!83.4, opacity=0.7] (0.5600, 0.6520, 2.1403) -- (0.6070, 0.6520, 2.1457) -- (0.6070, 0.7050, 2.1502) -- (0.5600, 0.7050, 2.1449) -- cycle;
\fill[blue!70.4, opacity=0.7] (0.5600, 0.7050, 2.1449) -- (0.6070, 0.7050, 2.1502) -- (0.6070, 0.7580, 2.1545) -- (0.5600, 0.7580, 2.1492) -- cycle;
\fill[blue!44.4, opacity=0.7] (0.5600, 0.7580, 2.1492) -- (0.6070, 0.7580, 2.1545) -- (0.6070, 0.8110, 2.1586) -- (0.5600, 0.8110, 2.1533) -- cycle;
\fill[blue!25.0, opacity=0.7] (0.5600, 0.8110, 2.1533) -- (0.6070, 0.8110, 2.1586) -- (0.6070, 0.8640, 2.1624) -- (0.5600, 0.8640, 2.1571) -- cycle;
\fill[blue!18.2, opacity=0.7] (0.5600, 0.8640, 2.1571) -- (0.6070, 0.8640, 2.1624) -- (0.6070, 0.9170, 2.1660) -- (0.5600, 0.9170, 2.1606) -- cycle;
\fill[blue!16.6, opacity=0.7] (0.5600, 0.9170, 2.1606) -- (0.6070, 0.9170, 2.1660) -- (0.6070, 0.9700, 2.1693) -- (0.5600, 0.9700, 2.1639) -- cycle;
\fill[blue!16.6, opacity=0.7] (0.5600, 0.9700, 2.1639) -- (0.6070, 0.9700, 2.1693) -- (0.6070, 1.0230, 2.1723) -- (0.5600, 1.0230, 2.1669) -- cycle;
\fill[blue!18.0, opacity=0.7] (0.5600, 1.0230, 2.1669) -- (0.6070, 1.0230, 2.1723) -- (0.6070, 1.0760, 2.1750) -- (0.5600, 1.0760, 2.1696) -- cycle;
\fill[blue!22.8, opacity=0.7] (0.5600, 1.0760, 2.1696) -- (0.6070, 1.0760, 2.1750) -- (0.6070, 1.1290, 2.1774) -- (0.5600, 1.1290, 2.1720) -- cycle;
\fill[blue!34.7, opacity=0.7] (0.5600, 1.1290, 2.1720) -- (0.6070, 1.1290, 2.1774) -- (0.6070, 1.1820, 2.1795) -- (0.5600, 1.1820, 2.1741) -- cycle;
\fill[blue!54.5, opacity=0.7] (0.5600, 1.1820, 2.1741) -- (0.6070, 1.1820, 2.1795) -- (0.6070, 1.2350, 2.1813) -- (0.5600, 1.2350, 2.1759) -- cycle;
\fill[blue!74.2, opacity=0.7] (0.5600, 1.2350, 2.1759) -- (0.6070, 1.2350, 2.1813) -- (0.6070, 1.2880, 2.1827) -- (0.5600, 1.2880, 2.1774) -- cycle;
\fill[blue!85.4, opacity=0.7] (0.5600, 1.2880, 2.1774) -- (0.6070, 1.2880, 2.1827) -- (0.6070, 1.3410, 2.1839) -- (0.5600, 1.3410, 2.1785) -- cycle;
\fill[blue!87.8, opacity=0.7] (0.5600, 1.3410, 2.1785) -- (0.6070, 1.3410, 2.1839) -- (0.6070, 1.3940, 2.1847) -- (0.5600, 1.3940, 2.1793) -- cycle;
\fill[blue!85.9, opacity=0.7] (0.5600, 1.3940, 2.1793) -- (0.6070, 1.3940, 2.1847) -- (0.6070, 1.4470, 2.1852) -- (0.5600, 1.4470, 2.1798) -- cycle;
\fill[blue!83.3, opacity=0.7] (0.5600, 1.4470, 2.1798) -- (0.6070, 1.4470, 2.1852) -- (0.6070, 1.5000, 2.1854) -- (0.5600, 1.5000, 2.1800) -- cycle;
\fill[blue!81.3, opacity=0.7] (0.5600, 1.5000, 2.1800) -- (0.6070, 1.5000, 2.1854) -- (0.6070, 1.5530, 2.1852) -- (0.5600, 1.5530, 2.1798) -- cycle;
\fill[blue!80.4, opacity=0.7] (0.5600, 1.5530, 2.1798) -- (0.6070, 1.5530, 2.1852) -- (0.6070, 1.6060, 2.1847) -- (0.5600, 1.6060, 2.1793) -- cycle;
\fill[blue!80.4, opacity=0.7] (0.5600, 1.6060, 2.1793) -- (0.6070, 1.6060, 2.1847) -- (0.6070, 1.6590, 2.1839) -- (0.5600, 1.6590, 2.1785) -- cycle;
\fill[blue!81.3, opacity=0.7] (0.5600, 1.6590, 2.1785) -- (0.6070, 1.6590, 2.1839) -- (0.6070, 1.7120, 2.1827) -- (0.5600, 1.7120, 2.1774) -- cycle;
\fill[blue!83.0, opacity=0.7] (0.5600, 1.7120, 2.1774) -- (0.6070, 1.7120, 2.1827) -- (0.6070, 1.7650, 2.1813) -- (0.5600, 1.7650, 2.1759) -- cycle;
\fill[blue!85.2, opacity=0.7] (0.5600, 1.7650, 2.1759) -- (0.6070, 1.7650, 2.1813) -- (0.6070, 1.8180, 2.1795) -- (0.5600, 1.8180, 2.1741) -- cycle;
\fill[blue!87.3, opacity=0.7] (0.5600, 1.8180, 2.1741) -- (0.6070, 1.8180, 2.1795) -- (0.6070, 1.8710, 2.1774) -- (0.5600, 1.8710, 2.1720) -- cycle;
\fill[blue!87.6, opacity=0.7] (0.5600, 1.8710, 2.1720) -- (0.6070, 1.8710, 2.1774) -- (0.6070, 1.9240, 2.1750) -- (0.5600, 1.9240, 2.1696) -- cycle;
\fill[blue!82.8, opacity=0.7] (0.5600, 1.9240, 2.1696) -- (0.6070, 1.9240, 2.1750) -- (0.6070, 1.9770, 2.1723) -- (0.5600, 1.9770, 2.1669) -- cycle;
\fill[blue!69.9, opacity=0.7] (0.5600, 1.9770, 2.1669) -- (0.6070, 1.9770, 2.1723) -- (0.6070, 2.0300, 2.1693) -- (0.5600, 2.0300, 2.1639) -- cycle;
\fill[blue!50.2, opacity=0.7] (0.5600, 2.0300, 2.1639) -- (0.6070, 2.0300, 2.1693) -- (0.6070, 2.0830, 2.1660) -- (0.5600, 2.0830, 2.1606) -- cycle;
\fill[blue!31.6, opacity=0.7] (0.5600, 2.0830, 2.1606) -- (0.6070, 2.0830, 2.1660) -- (0.6070, 2.1360, 2.1624) -- (0.5600, 2.1360, 2.1571) -- cycle;
\fill[blue!20.8, opacity=0.7] (0.5600, 2.1360, 2.1571) -- (0.6070, 2.1360, 2.1624) -- (0.6070, 2.1890, 2.1586) -- (0.5600, 2.1890, 2.1533) -- cycle;
\fill[blue!16.8, opacity=0.7] (0.5600, 2.1890, 2.1533) -- (0.6070, 2.1890, 2.1586) -- (0.6070, 2.2420, 2.1545) -- (0.5600, 2.2420, 2.1492) -- cycle;
\fill[blue!15.6, opacity=0.7] (0.5600, 2.2420, 2.1492) -- (0.6070, 2.2420, 2.1545) -- (0.6070, 2.2950, 2.1502) -- (0.5600, 2.2950, 2.1449) -- cycle;
\fill[blue!15.4, opacity=0.7] (0.5600, 2.2950, 2.1449) -- (0.6070, 2.2950, 2.1502) -- (0.6070, 2.3480, 2.1457) -- (0.5600, 2.3480, 2.1403) -- cycle;
\fill[blue!15.4, opacity=0.7] (0.5600, 2.3480, 2.1403) -- (0.6070, 2.3480, 2.1457) -- (0.6070, 2.4010, 2.1409) -- (0.5600, 2.4010, 2.1355) -- cycle;
\fill[blue!16.1, opacity=0.7] (0.5600, 2.4010, 2.1355) -- (0.6070, 2.4010, 2.1409) -- (0.6070, 2.4540, 2.1359) -- (0.5600, 2.4540, 2.1305) -- cycle;
\fill[blue!19.4, opacity=0.7] (0.5600, 2.4540, 2.1305) -- (0.6070, 2.4540, 2.1359) -- (0.6070, 2.5070, 2.1307) -- (0.5600, 2.5070, 2.1254) -- cycle;
\fill[blue!30.3, opacity=0.7] (0.5600, 2.5070, 2.1254) -- (0.6070, 2.5070, 2.1307) -- (0.6070, 2.5600, 2.1254) -- (0.5600, 2.5600, 2.1200) -- cycle;
\fill[blue!49.0, opacity=0.7] (0.5600, 2.5600, 2.1200) -- (0.6070, 2.5600, 2.1254) -- (0.6070, 2.6130, 2.1198) -- (0.5600, 2.6130, 2.1145) -- cycle;
\fill[blue!61.9, opacity=0.7] (0.5600, 2.6130, 2.1145) -- (0.6070, 2.6130, 2.1198) -- (0.6070, 2.6660, 2.1142) -- (0.5600, 2.6660, 2.1088) -- cycle;
\fill[blue!59.0, opacity=0.7] (0.5600, 2.6660, 2.1088) -- (0.6070, 2.6660, 2.1142) -- (0.6070, 2.7190, 2.1084) -- (0.5600, 2.7190, 2.1030) -- cycle;
\fill[blue!39.6, opacity=0.7] (0.5600, 2.7190, 2.1030) -- (0.6070, 2.7190, 2.1084) -- (0.6070, 2.7720, 2.1024) -- (0.5600, 2.7720, 2.0971) -- cycle;
\fill[blue!20.5, opacity=0.7] (0.5600, 2.7720, 2.0971) -- (0.6070, 2.7720, 2.1024) -- (0.6070, 2.8250, 2.0964) -- (0.5600, 2.8250, 2.0911) -- cycle;
\fill[blue!15.4, opacity=0.7] (0.5600, 2.8250, 2.0911) -- (0.6070, 2.8250, 2.0964) -- (0.6070, 2.8780, 2.0903) -- (0.5600, 2.8780, 2.0849) -- cycle;
\fill[blue!15.0, opacity=0.7] (0.5600, 2.8780, 2.0849) -- (0.6070, 2.8780, 2.0903) -- (0.6070, 2.9310, 2.0841) -- (0.5600, 2.9310, 2.0788) -- cycle;
\fill[blue!15.0, opacity=0.7] (0.5600, 2.9310, 2.0788) -- (0.6070, 2.9310, 2.0841) -- (0.6070, 2.9840, 2.0779) -- (0.5600, 2.9840, 2.0725) -- cycle;
\fill[blue!15.0, opacity=0.7] (0.5600, 2.9840, 2.0725) -- (0.6070, 2.9840, 2.0779) -- (0.6070, 3.0370, 2.0716) -- (0.5600, 3.0370, 2.0663) -- cycle;
\fill[blue!15.0, opacity=0.7] (0.5600, 3.0370, 2.0663) -- (0.6070, 3.0370, 2.0716) -- (0.6070, 3.0900, 2.0654) -- (0.5600, 3.0900, 2.0600) -- cycle;
\fill[blue!15.7, opacity=0.7] (0.6070, -0.0900, 2.0654) -- (0.6540, -0.0900, 2.0705) -- (0.6540, -0.0370, 2.0768) -- (0.6070, -0.0370, 2.0716) -- cycle;
\fill[blue!24.1, opacity=0.7] (0.6070, -0.0370, 2.0716) -- (0.6540, -0.0370, 2.0768) -- (0.6540, 0.0160, 2.0831) -- (0.6070, 0.0160, 2.0779) -- cycle;
\fill[blue!38.8, opacity=0.7] (0.6070, 0.0160, 2.0779) -- (0.6540, 0.0160, 2.0831) -- (0.6540, 0.0690, 2.0893) -- (0.6070, 0.0690, 2.0841) -- cycle;
\fill[blue!36.0, opacity=0.7] (0.6070, 0.0690, 2.0841) -- (0.6540, 0.0690, 2.0893) -- (0.6540, 0.1220, 2.0955) -- (0.6070, 0.1220, 2.0903) -- cycle;
\fill[blue!21.6, opacity=0.7] (0.6070, 0.1220, 2.0903) -- (0.6540, 0.1220, 2.0955) -- (0.6540, 0.1750, 2.1016) -- (0.6070, 0.1750, 2.0964) -- cycle;
\fill[blue!15.8, opacity=0.7] (0.6070, 0.1750, 2.0964) -- (0.6540, 0.1750, 2.1016) -- (0.6540, 0.2280, 2.1076) -- (0.6070, 0.2280, 2.1024) -- cycle;
\fill[blue!15.1, opacity=0.7] (0.6070, 0.2280, 2.1024) -- (0.6540, 0.2280, 2.1076) -- (0.6540, 0.2810, 2.1135) -- (0.6070, 0.2810, 2.1084) -- cycle;
\fill[blue!15.0, opacity=0.7] (0.6070, 0.2810, 2.1084) -- (0.6540, 0.2810, 2.1135) -- (0.6540, 0.3340, 2.1193) -- (0.6070, 0.3340, 2.1142) -- cycle;
\fill[blue!15.1, opacity=0.7] (0.6070, 0.3340, 2.1142) -- (0.6540, 0.3340, 2.1193) -- (0.6540, 0.3870, 2.1250) -- (0.6070, 0.3870, 2.1198) -- cycle;
\fill[blue!16.7, opacity=0.7] (0.6070, 0.3870, 2.1198) -- (0.6540, 0.3870, 2.1250) -- (0.6540, 0.4400, 2.1305) -- (0.6070, 0.4400, 2.1254) -- cycle;
\fill[blue!31.1, opacity=0.7] (0.6070, 0.4400, 2.1254) -- (0.6540, 0.4400, 2.1305) -- (0.6540, 0.4930, 2.1359) -- (0.6070, 0.4930, 2.1307) -- cycle;
\fill[blue!65.4, opacity=0.7] (0.6070, 0.4930, 2.1307) -- (0.6540, 0.4930, 2.1359) -- (0.6540, 0.5460, 2.1411) -- (0.6070, 0.5460, 2.1359) -- cycle;
\fill[blue!83.5, opacity=0.7] (0.6070, 0.5460, 2.1359) -- (0.6540, 0.5460, 2.1411) -- (0.6540, 0.5990, 2.1461) -- (0.6070, 0.5990, 2.1409) -- cycle;
\fill[blue!84.1, opacity=0.7] (0.6070, 0.5990, 2.1409) -- (0.6540, 0.5990, 2.1461) -- (0.6540, 0.6520, 2.1508) -- (0.6070, 0.6520, 2.1457) -- cycle;
\fill[blue!71.1, opacity=0.7] (0.6070, 0.6520, 2.1457) -- (0.6540, 0.6520, 2.1508) -- (0.6540, 0.7050, 2.1554) -- (0.6070, 0.7050, 2.1502) -- cycle;
\fill[blue!43.3, opacity=0.7] (0.6070, 0.7050, 2.1502) -- (0.6540, 0.7050, 2.1554) -- (0.6540, 0.7580, 2.1597) -- (0.6070, 0.7580, 2.1545) -- cycle;
\fill[blue!23.8, opacity=0.7] (0.6070, 0.7580, 2.1545) -- (0.6540, 0.7580, 2.1597) -- (0.6540, 0.8110, 2.1638) -- (0.6070, 0.8110, 2.1586) -- cycle;
\fill[blue!17.7, opacity=0.7] (0.6070, 0.8110, 2.1586) -- (0.6540, 0.8110, 2.1638) -- (0.6540, 0.8640, 2.1676) -- (0.6070, 0.8640, 2.1624) -- cycle;
\fill[blue!16.6, opacity=0.7] (0.6070, 0.8640, 2.1624) -- (0.6540, 0.8640, 2.1676) -- (0.6540, 0.9170, 2.1712) -- (0.6070, 0.9170, 2.1660) -- cycle;
\fill[blue!17.2, opacity=0.7] (0.6070, 0.9170, 2.1660) -- (0.6540, 0.9170, 2.1712) -- (0.6540, 0.9700, 2.1745) -- (0.6070, 0.9700, 2.1693) -- cycle;
\fill[blue!20.7, opacity=0.7] (0.6070, 0.9700, 2.1693) -- (0.6540, 0.9700, 2.1745) -- (0.6540, 1.0230, 2.1775) -- (0.6070, 1.0230, 2.1723) -- cycle;
\fill[blue!32.6, opacity=0.7] (0.6070, 1.0230, 2.1723) -- (0.6540, 1.0230, 2.1775) -- (0.6540, 1.0760, 2.1802) -- (0.6070, 1.0760, 2.1750) -- cycle;
\fill[blue!56.7, opacity=0.7] (0.6070, 1.0760, 2.1750) -- (0.6540, 1.0760, 2.1802) -- (0.6540, 1.1290, 2.1826) -- (0.6070, 1.1290, 2.1774) -- cycle;
\fill[blue!80.4, opacity=0.7] (0.6070, 1.1290, 2.1774) -- (0.6540, 1.1290, 2.1826) -- (0.6540, 1.1820, 2.1847) -- (0.6070, 1.1820, 2.1795) -- cycle;
\fill[blue!87.8, opacity=0.7] (0.6070, 1.1820, 2.1795) -- (0.6540, 1.1820, 2.1847) -- (0.6540, 1.2350, 2.1864) -- (0.6070, 1.2350, 2.1813) -- cycle;
\fill[blue!83.1, opacity=0.7] (0.6070, 1.2350, 2.1813) -- (0.6540, 1.2350, 2.1864) -- (0.6540, 1.2880, 2.1879) -- (0.6070, 1.2880, 2.1827) -- cycle;
\fill[blue!76.3, opacity=0.7] (0.6070, 1.2880, 2.1827) -- (0.6540, 1.2880, 2.1879) -- (0.6540, 1.3410, 2.1891) -- (0.6070, 1.3410, 2.1839) -- cycle;
\fill[blue!72.3, opacity=0.7] (0.6070, 1.3410, 2.1839) -- (0.6540, 1.3410, 2.1891) -- (0.6540, 1.3940, 2.1899) -- (0.6070, 1.3940, 2.1847) -- cycle;
\fill[blue!71.2, opacity=0.7] (0.6070, 1.3940, 2.1847) -- (0.6540, 1.3940, 2.1899) -- (0.6540, 1.4470, 2.1904) -- (0.6070, 1.4470, 2.1852) -- cycle;
\fill[blue!72.0, opacity=0.7] (0.6070, 1.4470, 2.1852) -- (0.6540, 1.4470, 2.1904) -- (0.6540, 1.5000, 2.1905) -- (0.6070, 1.5000, 2.1854) -- cycle;
\fill[blue!73.7, opacity=0.7] (0.6070, 1.5000, 2.1854) -- (0.6540, 1.5000, 2.1905) -- (0.6540, 1.5530, 2.1904) -- (0.6070, 1.5530, 2.1852) -- cycle;
\fill[blue!75.3, opacity=0.7] (0.6070, 1.5530, 2.1852) -- (0.6540, 1.5530, 2.1904) -- (0.6540, 1.6060, 2.1899) -- (0.6070, 1.6060, 2.1847) -- cycle;
\fill[blue!76.4, opacity=0.7] (0.6070, 1.6060, 2.1847) -- (0.6540, 1.6060, 2.1899) -- (0.6540, 1.6590, 2.1891) -- (0.6070, 1.6590, 2.1839) -- cycle;
\fill[blue!77.0, opacity=0.7] (0.6070, 1.6590, 2.1839) -- (0.6540, 1.6590, 2.1891) -- (0.6540, 1.7120, 2.1879) -- (0.6070, 1.7120, 2.1827) -- cycle;
\fill[blue!77.1, opacity=0.7] (0.6070, 1.7120, 2.1827) -- (0.6540, 1.7120, 2.1879) -- (0.6540, 1.7650, 2.1864) -- (0.6070, 1.7650, 2.1813) -- cycle;
\fill[blue!77.0, opacity=0.7] (0.6070, 1.7650, 2.1813) -- (0.6540, 1.7650, 2.1864) -- (0.6540, 1.8180, 2.1847) -- (0.6070, 1.8180, 2.1795) -- cycle;
\fill[blue!77.4, opacity=0.7] (0.6070, 1.8180, 2.1795) -- (0.6540, 1.8180, 2.1847) -- (0.6540, 1.8710, 2.1826) -- (0.6070, 1.8710, 2.1774) -- cycle;
\fill[blue!79.0, opacity=0.7] (0.6070, 1.8710, 2.1774) -- (0.6540, 1.8710, 2.1826) -- (0.6540, 1.9240, 2.1802) -- (0.6070, 1.9240, 2.1750) -- cycle;
\fill[blue!82.4, opacity=0.7] (0.6070, 1.9240, 2.1750) -- (0.6540, 1.9240, 2.1802) -- (0.6540, 1.9770, 2.1775) -- (0.6070, 1.9770, 2.1723) -- cycle;
\fill[blue!86.5, opacity=0.7] (0.6070, 1.9770, 2.1723) -- (0.6540, 1.9770, 2.1775) -- (0.6540, 2.0300, 2.1745) -- (0.6070, 2.0300, 2.1693) -- cycle;
\fill[blue!87.2, opacity=0.7] (0.6070, 2.0300, 2.1693) -- (0.6540, 2.0300, 2.1745) -- (0.6540, 2.0830, 2.1712) -- (0.6070, 2.0830, 2.1660) -- cycle;
\fill[blue!76.7, opacity=0.7] (0.6070, 2.0830, 2.1660) -- (0.6540, 2.0830, 2.1712) -- (0.6540, 2.1360, 2.1676) -- (0.6070, 2.1360, 2.1624) -- cycle;
\fill[blue!53.3, opacity=0.7] (0.6070, 2.1360, 2.1624) -- (0.6540, 2.1360, 2.1676) -- (0.6540, 2.1890, 2.1638) -- (0.6070, 2.1890, 2.1586) -- cycle;
\fill[blue!30.4, opacity=0.7] (0.6070, 2.1890, 2.1586) -- (0.6540, 2.1890, 2.1638) -- (0.6540, 2.2420, 2.1597) -- (0.6070, 2.2420, 2.1545) -- cycle;
\fill[blue!19.2, opacity=0.7] (0.6070, 2.2420, 2.1545) -- (0.6540, 2.2420, 2.1597) -- (0.6540, 2.2950, 2.1554) -- (0.6070, 2.2950, 2.1502) -- cycle;
\fill[blue!16.1, opacity=0.7] (0.6070, 2.2950, 2.1502) -- (0.6540, 2.2950, 2.1554) -- (0.6540, 2.3480, 2.1508) -- (0.6070, 2.3480, 2.1457) -- cycle;
\fill[blue!15.4, opacity=0.7] (0.6070, 2.3480, 2.1457) -- (0.6540, 2.3480, 2.1508) -- (0.6540, 2.4010, 2.1461) -- (0.6070, 2.4010, 2.1409) -- cycle;
\fill[blue!15.3, opacity=0.7] (0.6070, 2.4010, 2.1409) -- (0.6540, 2.4010, 2.1461) -- (0.6540, 2.4540, 2.1411) -- (0.6070, 2.4540, 2.1359) -- cycle;
\fill[blue!15.7, opacity=0.7] (0.6070, 2.4540, 2.1359) -- (0.6540, 2.4540, 2.1411) -- (0.6540, 2.5070, 2.1359) -- (0.6070, 2.5070, 2.1307) -- cycle;
\fill[blue!18.0, opacity=0.7] (0.6070, 2.5070, 2.1307) -- (0.6540, 2.5070, 2.1359) -- (0.6540, 2.5600, 2.1305) -- (0.6070, 2.5600, 2.1254) -- cycle;
\fill[blue!27.4, opacity=0.7] (0.6070, 2.5600, 2.1254) -- (0.6540, 2.5600, 2.1305) -- (0.6540, 2.6130, 2.1250) -- (0.6070, 2.6130, 2.1198) -- cycle;
\fill[blue!46.4, opacity=0.7] (0.6070, 2.6130, 2.1198) -- (0.6540, 2.6130, 2.1250) -- (0.6540, 2.6660, 2.1193) -- (0.6070, 2.6660, 2.1142) -- cycle;
\fill[blue!60.8, opacity=0.7] (0.6070, 2.6660, 2.1142) -- (0.6540, 2.6660, 2.1193) -- (0.6540, 2.7190, 2.1135) -- (0.6070, 2.7190, 2.1084) -- cycle;
\fill[blue!58.0, opacity=0.7] (0.6070, 2.7190, 2.1084) -- (0.6540, 2.7190, 2.1135) -- (0.6540, 2.7720, 2.1076) -- (0.6070, 2.7720, 2.1024) -- cycle;
\fill[blue!37.6, opacity=0.7] (0.6070, 2.7720, 2.1024) -- (0.6540, 2.7720, 2.1076) -- (0.6540, 2.8250, 2.1016) -- (0.6070, 2.8250, 2.0964) -- cycle;
\fill[blue!19.3, opacity=0.7] (0.6070, 2.8250, 2.0964) -- (0.6540, 2.8250, 2.1016) -- (0.6540, 2.8780, 2.0955) -- (0.6070, 2.8780, 2.0903) -- cycle;
\fill[blue!15.3, opacity=0.7] (0.6070, 2.8780, 2.0903) -- (0.6540, 2.8780, 2.0955) -- (0.6540, 2.9310, 2.0893) -- (0.6070, 2.9310, 2.0841) -- cycle;
\fill[blue!15.0, opacity=0.7] (0.6070, 2.9310, 2.0841) -- (0.6540, 2.9310, 2.0893) -- (0.6540, 2.9840, 2.0831) -- (0.6070, 2.9840, 2.0779) -- cycle;
\fill[blue!15.0, opacity=0.7] (0.6070, 2.9840, 2.0779) -- (0.6540, 2.9840, 2.0831) -- (0.6540, 3.0370, 2.0768) -- (0.6070, 3.0370, 2.0716) -- cycle;
\fill[blue!15.0, opacity=0.7] (0.6070, 3.0370, 2.0716) -- (0.6540, 3.0370, 2.0768) -- (0.6540, 3.0900, 2.0705) -- (0.6070, 3.0900, 2.0654) -- cycle;
\fill[blue!18.4, opacity=0.7] (0.6540, -0.0900, 2.0705) -- (0.7010, -0.0900, 2.0755) -- (0.7010, -0.0370, 2.0818) -- (0.6540, -0.0370, 2.0768) -- cycle;
\fill[blue!33.3, opacity=0.7] (0.6540, -0.0370, 2.0768) -- (0.7010, -0.0370, 2.0818) -- (0.7010, 0.0160, 2.0881) -- (0.6540, 0.0160, 2.0831) -- cycle;
\fill[blue!40.5, opacity=0.7] (0.6540, 0.0160, 2.0831) -- (0.7010, 0.0160, 2.0881) -- (0.7010, 0.0690, 2.0943) -- (0.6540, 0.0690, 2.0893) -- cycle;
\fill[blue!27.8, opacity=0.7] (0.6540, 0.0690, 2.0893) -- (0.7010, 0.0690, 2.0943) -- (0.7010, 0.1220, 2.1005) -- (0.6540, 0.1220, 2.0955) -- cycle;
\fill[blue!17.1, opacity=0.7] (0.6540, 0.1220, 2.0955) -- (0.7010, 0.1220, 2.1005) -- (0.7010, 0.1750, 2.1066) -- (0.6540, 0.1750, 2.1016) -- cycle;
\fill[blue!15.2, opacity=0.7] (0.6540, 0.1750, 2.1016) -- (0.7010, 0.1750, 2.1066) -- (0.7010, 0.2280, 2.1126) -- (0.6540, 0.2280, 2.1076) -- cycle;
\fill[blue!15.0, opacity=0.7] (0.6540, 0.2280, 2.1076) -- (0.7010, 0.2280, 2.1126) -- (0.7010, 0.2810, 2.1185) -- (0.6540, 0.2810, 2.1135) -- cycle;
\fill[blue!15.1, opacity=0.7] (0.6540, 0.2810, 2.1135) -- (0.7010, 0.2810, 2.1185) -- (0.7010, 0.3340, 2.1243) -- (0.6540, 0.3340, 2.1193) -- cycle;
\fill[blue!15.8, opacity=0.7] (0.6540, 0.3340, 2.1193) -- (0.7010, 0.3340, 2.1243) -- (0.7010, 0.3870, 2.1300) -- (0.6540, 0.3870, 2.1250) -- cycle;
\fill[blue!25.0, opacity=0.7] (0.6540, 0.3870, 2.1250) -- (0.7010, 0.3870, 2.1300) -- (0.7010, 0.4400, 2.1355) -- (0.6540, 0.4400, 2.1305) -- cycle;
\fill[blue!58.4, opacity=0.7] (0.6540, 0.4400, 2.1305) -- (0.7010, 0.4400, 2.1355) -- (0.7010, 0.4930, 2.1409) -- (0.6540, 0.4930, 2.1359) -- cycle;
\fill[blue!82.4, opacity=0.7] (0.6540, 0.4930, 2.1359) -- (0.7010, 0.4930, 2.1409) -- (0.7010, 0.5460, 2.1461) -- (0.6540, 0.5460, 2.1411) -- cycle;
\fill[blue!85.1, opacity=0.7] (0.6540, 0.5460, 2.1411) -- (0.7010, 0.5460, 2.1461) -- (0.7010, 0.5990, 2.1510) -- (0.6540, 0.5990, 2.1461) -- cycle;
\fill[blue!74.1, opacity=0.7] (0.6540, 0.5990, 2.1461) -- (0.7010, 0.5990, 2.1510) -- (0.7010, 0.6520, 2.1558) -- (0.6540, 0.6520, 2.1508) -- cycle;
\fill[blue!45.5, opacity=0.7] (0.6540, 0.6520, 2.1508) -- (0.7010, 0.6520, 2.1558) -- (0.7010, 0.7050, 2.1604) -- (0.6540, 0.7050, 2.1554) -- cycle;
\fill[blue!24.1, opacity=0.7] (0.6540, 0.7050, 2.1554) -- (0.7010, 0.7050, 2.1604) -- (0.7010, 0.7580, 2.1647) -- (0.6540, 0.7580, 2.1597) -- cycle;
\fill[blue!17.7, opacity=0.7] (0.6540, 0.7580, 2.1597) -- (0.7010, 0.7580, 2.1647) -- (0.7010, 0.8110, 2.1688) -- (0.6540, 0.8110, 2.1638) -- cycle;
\fill[blue!16.7, opacity=0.7] (0.6540, 0.8110, 2.1638) -- (0.7010, 0.8110, 2.1688) -- (0.7010, 0.8640, 2.1726) -- (0.6540, 0.8640, 2.1676) -- cycle;
\fill[blue!17.9, opacity=0.7] (0.6540, 0.8640, 2.1676) -- (0.7010, 0.8640, 2.1726) -- (0.7010, 0.9170, 2.1762) -- (0.6540, 0.9170, 2.1712) -- cycle;
\fill[blue!24.4, opacity=0.7] (0.6540, 0.9170, 2.1712) -- (0.7010, 0.9170, 2.1762) -- (0.7010, 0.9700, 2.1794) -- (0.6540, 0.9700, 2.1745) -- cycle;
\fill[blue!45.1, opacity=0.7] (0.6540, 0.9700, 2.1745) -- (0.7010, 0.9700, 2.1794) -- (0.7010, 1.0230, 2.1824) -- (0.6540, 1.0230, 2.1775) -- cycle;
\fill[blue!75.6, opacity=0.7] (0.6540, 1.0230, 2.1775) -- (0.7010, 1.0230, 2.1824) -- (0.7010, 1.0760, 2.1851) -- (0.6540, 1.0760, 2.1802) -- cycle;
\fill[blue!87.8, opacity=0.7] (0.6540, 1.0760, 2.1802) -- (0.7010, 1.0760, 2.1851) -- (0.7010, 1.1290, 2.1875) -- (0.6540, 1.1290, 2.1826) -- cycle;
\fill[blue!80.5, opacity=0.7] (0.6540, 1.1290, 2.1826) -- (0.7010, 1.1290, 2.1875) -- (0.7010, 1.1820, 2.1896) -- (0.6540, 1.1820, 2.1847) -- cycle;
\fill[blue!71.4, opacity=0.7] (0.6540, 1.1820, 2.1847) -- (0.7010, 1.1820, 2.1896) -- (0.7010, 1.2350, 2.1914) -- (0.6540, 1.2350, 2.1864) -- cycle;
\fill[blue!68.2, opacity=0.7] (0.6540, 1.2350, 2.1864) -- (0.7010, 1.2350, 2.1914) -- (0.7010, 1.2880, 2.1929) -- (0.6540, 1.2880, 2.1879) -- cycle;
\fill[blue!70.6, opacity=0.7] (0.6540, 1.2880, 2.1879) -- (0.7010, 1.2880, 2.1929) -- (0.7010, 1.3410, 2.1940) -- (0.6540, 1.3410, 2.1891) -- cycle;
\fill[blue!76.0, opacity=0.7] (0.6540, 1.3410, 2.1891) -- (0.7010, 1.3410, 2.1940) -- (0.7010, 1.3940, 2.1949) -- (0.6540, 1.3940, 2.1899) -- cycle;
\fill[blue!81.6, opacity=0.7] (0.6540, 1.3940, 2.1899) -- (0.7010, 1.3940, 2.1949) -- (0.7010, 1.4470, 2.1954) -- (0.6540, 1.4470, 2.1904) -- cycle;
\fill[blue!85.6, opacity=0.7] (0.6540, 1.4470, 2.1904) -- (0.7010, 1.4470, 2.1954) -- (0.7010, 1.5000, 2.1955) -- (0.6540, 1.5000, 2.1905) -- cycle;
\fill[blue!87.5, opacity=0.7] (0.6540, 1.5000, 2.1905) -- (0.7010, 1.5000, 2.1955) -- (0.7010, 1.5530, 2.1954) -- (0.6540, 1.5530, 2.1904) -- cycle;
\fill[blue!87.9, opacity=0.7] (0.6540, 1.5530, 2.1904) -- (0.7010, 1.5530, 2.1954) -- (0.7010, 1.6060, 2.1949) -- (0.6540, 1.6060, 2.1899) -- cycle;
\fill[blue!87.7, opacity=0.7] (0.6540, 1.6060, 2.1899) -- (0.7010, 1.6060, 2.1949) -- (0.7010, 1.6590, 2.1940) -- (0.6540, 1.6590, 2.1891) -- cycle;
\fill[blue!87.7, opacity=0.7] (0.6540, 1.6590, 2.1891) -- (0.7010, 1.6590, 2.1940) -- (0.7010, 1.7120, 2.1929) -- (0.6540, 1.7120, 2.1879) -- cycle;
\fill[blue!87.8, opacity=0.7] (0.6540, 1.7120, 2.1879) -- (0.7010, 1.7120, 2.1929) -- (0.7010, 1.7650, 2.1914) -- (0.6540, 1.7650, 2.1864) -- cycle;
\fill[blue!87.7, opacity=0.7] (0.6540, 1.7650, 2.1864) -- (0.7010, 1.7650, 2.1914) -- (0.7010, 1.8180, 2.1896) -- (0.6540, 1.8180, 2.1847) -- cycle;
\fill[blue!86.5, opacity=0.7] (0.6540, 1.8180, 2.1847) -- (0.7010, 1.8180, 2.1896) -- (0.7010, 1.8710, 2.1875) -- (0.6540, 1.8710, 2.1826) -- cycle;
\fill[blue!84.1, opacity=0.7] (0.6540, 1.8710, 2.1826) -- (0.7010, 1.8710, 2.1875) -- (0.7010, 1.9240, 2.1851) -- (0.6540, 1.9240, 2.1802) -- cycle;
\fill[blue!81.2, opacity=0.7] (0.6540, 1.9240, 2.1802) -- (0.7010, 1.9240, 2.1851) -- (0.7010, 1.9770, 2.1824) -- (0.6540, 1.9770, 2.1775) -- cycle;
\fill[blue!79.7, opacity=0.7] (0.6540, 1.9770, 2.1775) -- (0.7010, 1.9770, 2.1824) -- (0.7010, 2.0300, 2.1794) -- (0.6540, 2.0300, 2.1745) -- cycle;
\fill[blue!81.2, opacity=0.7] (0.6540, 2.0300, 2.1745) -- (0.7010, 2.0300, 2.1794) -- (0.7010, 2.0830, 2.1762) -- (0.6540, 2.0830, 2.1712) -- cycle;
\fill[blue!85.8, opacity=0.7] (0.6540, 2.0830, 2.1712) -- (0.7010, 2.0830, 2.1762) -- (0.7010, 2.1360, 2.1726) -- (0.6540, 2.1360, 2.1676) -- cycle;
\fill[blue!87.2, opacity=0.7] (0.6540, 2.1360, 2.1676) -- (0.7010, 2.1360, 2.1726) -- (0.7010, 2.1890, 2.1688) -- (0.6540, 2.1890, 2.1638) -- cycle;
\fill[blue!73.0, opacity=0.7] (0.6540, 2.1890, 2.1638) -- (0.7010, 2.1890, 2.1688) -- (0.7010, 2.2420, 2.1647) -- (0.6540, 2.2420, 2.1597) -- cycle;
\fill[blue!44.2, opacity=0.7] (0.6540, 2.2420, 2.1597) -- (0.7010, 2.2420, 2.1647) -- (0.7010, 2.2950, 2.1604) -- (0.6540, 2.2950, 2.1554) -- cycle;
\fill[blue!23.4, opacity=0.7] (0.6540, 2.2950, 2.1554) -- (0.7010, 2.2950, 2.1604) -- (0.7010, 2.3480, 2.1558) -- (0.6540, 2.3480, 2.1508) -- cycle;
\fill[blue!16.7, opacity=0.7] (0.6540, 2.3480, 2.1508) -- (0.7010, 2.3480, 2.1558) -- (0.7010, 2.4010, 2.1510) -- (0.6540, 2.4010, 2.1461) -- cycle;
\fill[blue!15.5, opacity=0.7] (0.6540, 2.4010, 2.1461) -- (0.7010, 2.4010, 2.1510) -- (0.7010, 2.4540, 2.1461) -- (0.6540, 2.4540, 2.1411) -- cycle;
\fill[blue!15.3, opacity=0.7] (0.6540, 2.4540, 2.1411) -- (0.7010, 2.4540, 2.1461) -- (0.7010, 2.5070, 2.1409) -- (0.6540, 2.5070, 2.1359) -- cycle;
\fill[blue!15.6, opacity=0.7] (0.6540, 2.5070, 2.1359) -- (0.7010, 2.5070, 2.1409) -- (0.7010, 2.5600, 2.1355) -- (0.6540, 2.5600, 2.1305) -- cycle;
\fill[blue!17.5, opacity=0.7] (0.6540, 2.5600, 2.1305) -- (0.7010, 2.5600, 2.1355) -- (0.7010, 2.6130, 2.1300) -- (0.6540, 2.6130, 2.1250) -- cycle;
\fill[blue!26.3, opacity=0.7] (0.6540, 2.6130, 2.1250) -- (0.7010, 2.6130, 2.1300) -- (0.7010, 2.6660, 2.1243) -- (0.6540, 2.6660, 2.1193) -- cycle;
\fill[blue!45.8, opacity=0.7] (0.6540, 2.6660, 2.1193) -- (0.7010, 2.6660, 2.1243) -- (0.7010, 2.7190, 2.1185) -- (0.6540, 2.7190, 2.1135) -- cycle;
\fill[blue!60.3, opacity=0.7] (0.6540, 2.7190, 2.1135) -- (0.7010, 2.7190, 2.1185) -- (0.7010, 2.7720, 2.1126) -- (0.6540, 2.7720, 2.1076) -- cycle;
\fill[blue!55.8, opacity=0.7] (0.6540, 2.7720, 2.1076) -- (0.7010, 2.7720, 2.1126) -- (0.7010, 2.8250, 2.1066) -- (0.6540, 2.8250, 2.1016) -- cycle;
\fill[blue!33.7, opacity=0.7] (0.6540, 2.8250, 2.1016) -- (0.7010, 2.8250, 2.1066) -- (0.7010, 2.8780, 2.1005) -- (0.6540, 2.8780, 2.0955) -- cycle;
\fill[blue!17.7, opacity=0.7] (0.6540, 2.8780, 2.0955) -- (0.7010, 2.8780, 2.1005) -- (0.7010, 2.9310, 2.0943) -- (0.6540, 2.9310, 2.0893) -- cycle;
\fill[blue!15.1, opacity=0.7] (0.6540, 2.9310, 2.0893) -- (0.7010, 2.9310, 2.0943) -- (0.7010, 2.9840, 2.0881) -- (0.6540, 2.9840, 2.0831) -- cycle;
\fill[blue!15.0, opacity=0.7] (0.6540, 2.9840, 2.0831) -- (0.7010, 2.9840, 2.0881) -- (0.7010, 3.0370, 2.0818) -- (0.6540, 3.0370, 2.0768) -- cycle;
\fill[blue!15.0, opacity=0.7] (0.6540, 3.0370, 2.0768) -- (0.7010, 3.0370, 2.0818) -- (0.7010, 3.0900, 2.0755) -- (0.6540, 3.0900, 2.0705) -- cycle;
\fill[blue!24.6, opacity=0.7] (0.7010, -0.0900, 2.0755) -- (0.7480, -0.0900, 2.0803) -- (0.7480, -0.0370, 2.0866) -- (0.7010, -0.0370, 2.0818) -- cycle;
\fill[blue!40.1, opacity=0.7] (0.7010, -0.0370, 2.0818) -- (0.7480, -0.0370, 2.0866) -- (0.7480, 0.0160, 2.0928) -- (0.7010, 0.0160, 2.0881) -- cycle;
\fill[blue!35.9, opacity=0.7] (0.7010, 0.0160, 2.0881) -- (0.7480, 0.0160, 2.0928) -- (0.7480, 0.0690, 2.0991) -- (0.7010, 0.0690, 2.0943) -- cycle;
\fill[blue!20.8, opacity=0.7] (0.7010, 0.0690, 2.0943) -- (0.7480, 0.0690, 2.0991) -- (0.7480, 0.1220, 2.1052) -- (0.7010, 0.1220, 2.1005) -- cycle;
\fill[blue!15.6, opacity=0.7] (0.7010, 0.1220, 2.1005) -- (0.7480, 0.1220, 2.1052) -- (0.7480, 0.1750, 2.1114) -- (0.7010, 0.1750, 2.1066) -- cycle;
\fill[blue!15.1, opacity=0.7] (0.7010, 0.1750, 2.1066) -- (0.7480, 0.1750, 2.1114) -- (0.7480, 0.2280, 2.1174) -- (0.7010, 0.2280, 2.1126) -- cycle;
\fill[blue!15.0, opacity=0.7] (0.7010, 0.2280, 2.1126) -- (0.7480, 0.2280, 2.1174) -- (0.7480, 0.2810, 2.1233) -- (0.7010, 0.2810, 2.1185) -- cycle;
\fill[blue!15.3, opacity=0.7] (0.7010, 0.2810, 2.1185) -- (0.7480, 0.2810, 2.1233) -- (0.7480, 0.3340, 2.1291) -- (0.7010, 0.3340, 2.1243) -- cycle;
\fill[blue!19.7, opacity=0.7] (0.7010, 0.3340, 2.1243) -- (0.7480, 0.3340, 2.1291) -- (0.7480, 0.3870, 2.1348) -- (0.7010, 0.3870, 2.1300) -- cycle;
\fill[blue!47.0, opacity=0.7] (0.7010, 0.3870, 2.1300) -- (0.7480, 0.3870, 2.1348) -- (0.7480, 0.4400, 2.1403) -- (0.7010, 0.4400, 2.1355) -- cycle;
\fill[blue!79.2, opacity=0.7] (0.7010, 0.4400, 2.1355) -- (0.7480, 0.4400, 2.1403) -- (0.7480, 0.4930, 2.1457) -- (0.7010, 0.4930, 2.1409) -- cycle;
\fill[blue!85.9, opacity=0.7] (0.7010, 0.4930, 2.1409) -- (0.7480, 0.4930, 2.1457) -- (0.7480, 0.5460, 2.1508) -- (0.7010, 0.5460, 2.1461) -- cycle;
\fill[blue!78.7, opacity=0.7] (0.7010, 0.5460, 2.1461) -- (0.7480, 0.5460, 2.1508) -- (0.7480, 0.5990, 2.1558) -- (0.7010, 0.5990, 2.1510) -- cycle;
\fill[blue!51.2, opacity=0.7] (0.7010, 0.5990, 2.1510) -- (0.7480, 0.5990, 2.1558) -- (0.7480, 0.6520, 2.1606) -- (0.7010, 0.6520, 2.1558) -- cycle;
\fill[blue!25.9, opacity=0.7] (0.7010, 0.6520, 2.1558) -- (0.7480, 0.6520, 2.1606) -- (0.7480, 0.7050, 2.1651) -- (0.7010, 0.7050, 2.1604) -- cycle;
\fill[blue!18.0, opacity=0.7] (0.7010, 0.7050, 2.1604) -- (0.7480, 0.7050, 2.1651) -- (0.7480, 0.7580, 2.1695) -- (0.7010, 0.7580, 2.1647) -- cycle;
\fill[blue!16.8, opacity=0.7] (0.7010, 0.7580, 2.1647) -- (0.7480, 0.7580, 2.1695) -- (0.7480, 0.8110, 2.1736) -- (0.7010, 0.8110, 2.1688) -- cycle;
\fill[blue!18.4, opacity=0.7] (0.7010, 0.8110, 2.1688) -- (0.7480, 0.8110, 2.1736) -- (0.7480, 0.8640, 2.1774) -- (0.7010, 0.8640, 2.1726) -- cycle;
\fill[blue!27.6, opacity=0.7] (0.7010, 0.8640, 2.1726) -- (0.7480, 0.8640, 2.1774) -- (0.7480, 0.9170, 2.1809) -- (0.7010, 0.9170, 2.1762) -- cycle;
\fill[blue!55.4, opacity=0.7] (0.7010, 0.9170, 2.1762) -- (0.7480, 0.9170, 2.1809) -- (0.7480, 0.9700, 2.1842) -- (0.7010, 0.9700, 2.1794) -- cycle;
\fill[blue!84.7, opacity=0.7] (0.7010, 0.9700, 2.1794) -- (0.7480, 0.9700, 2.1842) -- (0.7480, 1.0230, 2.1872) -- (0.7010, 1.0230, 2.1824) -- cycle;
\fill[blue!83.9, opacity=0.7] (0.7010, 1.0230, 2.1824) -- (0.7480, 1.0230, 2.1872) -- (0.7480, 1.0760, 2.1899) -- (0.7010, 1.0760, 2.1851) -- cycle;
\fill[blue!71.3, opacity=0.7] (0.7010, 1.0760, 2.1851) -- (0.7480, 1.0760, 2.1899) -- (0.7480, 1.1290, 2.1923) -- (0.7010, 1.1290, 2.1875) -- cycle;
\fill[blue!66.1, opacity=0.7] (0.7010, 1.1290, 2.1875) -- (0.7480, 1.1290, 2.1923) -- (0.7480, 1.1820, 2.1944) -- (0.7010, 1.1820, 2.1896) -- cycle;
\fill[blue!70.0, opacity=0.7] (0.7010, 1.1820, 2.1896) -- (0.7480, 1.1820, 2.1944) -- (0.7480, 1.2350, 2.1962) -- (0.7010, 1.2350, 2.1914) -- cycle;
\fill[blue!79.0, opacity=0.7] (0.7010, 1.2350, 2.1914) -- (0.7480, 1.2350, 2.1962) -- (0.7480, 1.2880, 2.1977) -- (0.7010, 1.2880, 2.1929) -- cycle;
\fill[blue!86.5, opacity=0.7] (0.7010, 1.2880, 2.1929) -- (0.7480, 1.2880, 2.1977) -- (0.7480, 1.3410, 2.1988) -- (0.7010, 1.3410, 2.1940) -- cycle;
\fill[blue!87.2, opacity=0.7] (0.7010, 1.3410, 2.1940) -- (0.7480, 1.3410, 2.1988) -- (0.7480, 1.3940, 2.1996) -- (0.7010, 1.3940, 2.1949) -- cycle;
\fill[blue!81.5, opacity=0.7] (0.7010, 1.3940, 2.1949) -- (0.7480, 1.3940, 2.1996) -- (0.7480, 1.4470, 2.2001) -- (0.7010, 1.4470, 2.1954) -- cycle;
\fill[blue!73.2, opacity=0.7] (0.7010, 1.4470, 2.1954) -- (0.7480, 1.4470, 2.2001) -- (0.7480, 1.5000, 2.2003) -- (0.7010, 1.5000, 2.1955) -- cycle;
\fill[blue!65.8, opacity=0.7] (0.7010, 1.5000, 2.1955) -- (0.7480, 1.5000, 2.2003) -- (0.7480, 1.5530, 2.2001) -- (0.7010, 1.5530, 2.1954) -- cycle;
\fill[blue!60.6, opacity=0.7] (0.7010, 1.5530, 2.1954) -- (0.7480, 1.5530, 2.2001) -- (0.7480, 1.6060, 2.1996) -- (0.7010, 1.6060, 2.1949) -- cycle;
\fill[blue!57.9, opacity=0.7] (0.7010, 1.6060, 2.1949) -- (0.7480, 1.6060, 2.1996) -- (0.7480, 1.6590, 2.1988) -- (0.7010, 1.6590, 2.1940) -- cycle;
\fill[blue!57.8, opacity=0.7] (0.7010, 1.6590, 2.1940) -- (0.7480, 1.6590, 2.1988) -- (0.7480, 1.7120, 2.1977) -- (0.7010, 1.7120, 2.1929) -- cycle;
\fill[blue!60.2, opacity=0.7] (0.7010, 1.7120, 2.1929) -- (0.7480, 1.7120, 2.1977) -- (0.7480, 1.7650, 2.1962) -- (0.7010, 1.7650, 2.1914) -- cycle;
\fill[blue!65.1, opacity=0.7] (0.7010, 1.7650, 2.1914) -- (0.7480, 1.7650, 2.1962) -- (0.7480, 1.8180, 2.1944) -- (0.7010, 1.8180, 2.1896) -- cycle;
\fill[blue!72.4, opacity=0.7] (0.7010, 1.8180, 2.1896) -- (0.7480, 1.8180, 2.1944) -- (0.7480, 1.8710, 2.1923) -- (0.7010, 1.8710, 2.1875) -- cycle;
\fill[blue!80.4, opacity=0.7] (0.7010, 1.8710, 2.1875) -- (0.7480, 1.8710, 2.1923) -- (0.7480, 1.9240, 2.1899) -- (0.7010, 1.9240, 2.1851) -- cycle;
\fill[blue!86.4, opacity=0.7] (0.7010, 1.9240, 2.1851) -- (0.7480, 1.9240, 2.1899) -- (0.7480, 1.9770, 2.1872) -- (0.7010, 1.9770, 2.1824) -- cycle;
\fill[blue!87.7, opacity=0.7] (0.7010, 1.9770, 2.1824) -- (0.7480, 1.9770, 2.1872) -- (0.7480, 2.0300, 2.1842) -- (0.7010, 2.0300, 2.1794) -- cycle;
\fill[blue!84.5, opacity=0.7] (0.7010, 2.0300, 2.1794) -- (0.7480, 2.0300, 2.1842) -- (0.7480, 2.0830, 2.1809) -- (0.7010, 2.0830, 2.1762) -- cycle;
\fill[blue!81.3, opacity=0.7] (0.7010, 2.0830, 2.1762) -- (0.7480, 2.0830, 2.1809) -- (0.7480, 2.1360, 2.1774) -- (0.7010, 2.1360, 2.1726) -- cycle;
\fill[blue!82.5, opacity=0.7] (0.7010, 2.1360, 2.1726) -- (0.7480, 2.1360, 2.1774) -- (0.7480, 2.1890, 2.1736) -- (0.7010, 2.1890, 2.1688) -- cycle;
\fill[blue!87.2, opacity=0.7] (0.7010, 2.1890, 2.1688) -- (0.7480, 2.1890, 2.1736) -- (0.7480, 2.2420, 2.1695) -- (0.7010, 2.2420, 2.1647) -- cycle;
\fill[blue!83.3, opacity=0.7] (0.7010, 2.2420, 2.1647) -- (0.7480, 2.2420, 2.1695) -- (0.7480, 2.2950, 2.1651) -- (0.7010, 2.2950, 2.1604) -- cycle;
\fill[blue!56.8, opacity=0.7] (0.7010, 2.2950, 2.1604) -- (0.7480, 2.2950, 2.1651) -- (0.7480, 2.3480, 2.1606) -- (0.7010, 2.3480, 2.1558) -- cycle;
\fill[blue!28.2, opacity=0.7] (0.7010, 2.3480, 2.1558) -- (0.7480, 2.3480, 2.1606) -- (0.7480, 2.4010, 2.1558) -- (0.7010, 2.4010, 2.1510) -- cycle;
\fill[blue!17.5, opacity=0.7] (0.7010, 2.4010, 2.1510) -- (0.7480, 2.4010, 2.1558) -- (0.7480, 2.4540, 2.1508) -- (0.7010, 2.4540, 2.1461) -- cycle;
\fill[blue!15.5, opacity=0.7] (0.7010, 2.4540, 2.1461) -- (0.7480, 2.4540, 2.1508) -- (0.7480, 2.5070, 2.1457) -- (0.7010, 2.5070, 2.1409) -- cycle;
\fill[blue!15.3, opacity=0.7] (0.7010, 2.5070, 2.1409) -- (0.7480, 2.5070, 2.1457) -- (0.7480, 2.5600, 2.1403) -- (0.7010, 2.5600, 2.1355) -- cycle;
\fill[blue!15.5, opacity=0.7] (0.7010, 2.5600, 2.1355) -- (0.7480, 2.5600, 2.1403) -- (0.7480, 2.6130, 2.1348) -- (0.7010, 2.6130, 2.1300) -- cycle;
\fill[blue!17.4, opacity=0.7] (0.7010, 2.6130, 2.1300) -- (0.7480, 2.6130, 2.1348) -- (0.7480, 2.6660, 2.1291) -- (0.7010, 2.6660, 2.1243) -- cycle;
\fill[blue!26.9, opacity=0.7] (0.7010, 2.6660, 2.1243) -- (0.7480, 2.6660, 2.1291) -- (0.7480, 2.7190, 2.1233) -- (0.7010, 2.7190, 2.1185) -- cycle;
\fill[blue!47.2, opacity=0.7] (0.7010, 2.7190, 2.1185) -- (0.7480, 2.7190, 2.1233) -- (0.7480, 2.7720, 2.1174) -- (0.7010, 2.7720, 2.1126) -- cycle;
\fill[blue!60.2, opacity=0.7] (0.7010, 2.7720, 2.1126) -- (0.7480, 2.7720, 2.1174) -- (0.7480, 2.8250, 2.1114) -- (0.7010, 2.8250, 2.1066) -- cycle;
\fill[blue!51.9, opacity=0.7] (0.7010, 2.8250, 2.1066) -- (0.7480, 2.8250, 2.1114) -- (0.7480, 2.8780, 2.1052) -- (0.7010, 2.8780, 2.1005) -- cycle;
\fill[blue!28.2, opacity=0.7] (0.7010, 2.8780, 2.1005) -- (0.7480, 2.8780, 2.1052) -- (0.7480, 2.9310, 2.0991) -- (0.7010, 2.9310, 2.0943) -- cycle;
\fill[blue!16.3, opacity=0.7] (0.7010, 2.9310, 2.0943) -- (0.7480, 2.9310, 2.0991) -- (0.7480, 2.9840, 2.0928) -- (0.7010, 2.9840, 2.0881) -- cycle;
\fill[blue!15.0, opacity=0.7] (0.7010, 2.9840, 2.0881) -- (0.7480, 2.9840, 2.0928) -- (0.7480, 3.0370, 2.0866) -- (0.7010, 3.0370, 2.0818) -- cycle;
\fill[blue!15.0, opacity=0.7] (0.7010, 3.0370, 2.0818) -- (0.7480, 3.0370, 2.0866) -- (0.7480, 3.0900, 2.0803) -- (0.7010, 3.0900, 2.0755) -- cycle;
\fill[blue!33.0, opacity=0.7] (0.7480, -0.0900, 2.0803) -- (0.7950, -0.0900, 2.0849) -- (0.7950, -0.0370, 2.0911) -- (0.7480, -0.0370, 2.0866) -- cycle;
\fill[blue!41.6, opacity=0.7] (0.7480, -0.0370, 2.0866) -- (0.7950, -0.0370, 2.0911) -- (0.7950, 0.0160, 2.0974) -- (0.7480, 0.0160, 2.0928) -- cycle;
\fill[blue!28.5, opacity=0.7] (0.7480, 0.0160, 2.0928) -- (0.7950, 0.0160, 2.0974) -- (0.7950, 0.0690, 2.1036) -- (0.7480, 0.0690, 2.0991) -- cycle;
\fill[blue!17.1, opacity=0.7] (0.7480, 0.0690, 2.0991) -- (0.7950, 0.0690, 2.1036) -- (0.7950, 0.1220, 2.1098) -- (0.7480, 0.1220, 2.1052) -- cycle;
\fill[blue!15.2, opacity=0.7] (0.7480, 0.1220, 2.1052) -- (0.7950, 0.1220, 2.1098) -- (0.7950, 0.1750, 2.1159) -- (0.7480, 0.1750, 2.1114) -- cycle;
\fill[blue!15.0, opacity=0.7] (0.7480, 0.1750, 2.1114) -- (0.7950, 0.1750, 2.1159) -- (0.7950, 0.2280, 2.1219) -- (0.7480, 0.2280, 2.1174) -- cycle;
\fill[blue!15.1, opacity=0.7] (0.7480, 0.2280, 2.1174) -- (0.7950, 0.2280, 2.1219) -- (0.7950, 0.2810, 2.1279) -- (0.7480, 0.2810, 2.1233) -- cycle;
\fill[blue!16.6, opacity=0.7] (0.7480, 0.2810, 2.1233) -- (0.7950, 0.2810, 2.1279) -- (0.7950, 0.3340, 2.1337) -- (0.7480, 0.3340, 2.1291) -- cycle;
\fill[blue!33.2, opacity=0.7] (0.7480, 0.3340, 2.1291) -- (0.7950, 0.3340, 2.1337) -- (0.7950, 0.3870, 2.1393) -- (0.7480, 0.3870, 2.1348) -- cycle;
\fill[blue!71.4, opacity=0.7] (0.7480, 0.3870, 2.1348) -- (0.7950, 0.3870, 2.1393) -- (0.7950, 0.4400, 2.1449) -- (0.7480, 0.4400, 2.1403) -- cycle;
\fill[blue!85.8, opacity=0.7] (0.7480, 0.4400, 2.1403) -- (0.7950, 0.4400, 2.1449) -- (0.7950, 0.4930, 2.1502) -- (0.7480, 0.4930, 2.1457) -- cycle;
\fill[blue!83.2, opacity=0.7] (0.7480, 0.4930, 2.1457) -- (0.7950, 0.4930, 2.1502) -- (0.7950, 0.5460, 2.1554) -- (0.7480, 0.5460, 2.1508) -- cycle;
\fill[blue!60.7, opacity=0.7] (0.7480, 0.5460, 2.1508) -- (0.7950, 0.5460, 2.1554) -- (0.7950, 0.5990, 2.1604) -- (0.7480, 0.5990, 2.1558) -- cycle;
\fill[blue!30.3, opacity=0.7] (0.7480, 0.5990, 2.1558) -- (0.7950, 0.5990, 2.1604) -- (0.7950, 0.6520, 2.1651) -- (0.7480, 0.6520, 2.1606) -- cycle;
\fill[blue!18.8, opacity=0.7] (0.7480, 0.6520, 2.1606) -- (0.7950, 0.6520, 2.1651) -- (0.7950, 0.7050, 2.1697) -- (0.7480, 0.7050, 2.1651) -- cycle;
\fill[blue!16.9, opacity=0.7] (0.7480, 0.7050, 2.1651) -- (0.7950, 0.7050, 2.1697) -- (0.7950, 0.7580, 2.1740) -- (0.7480, 0.7580, 2.1695) -- cycle;
\fill[blue!18.4, opacity=0.7] (0.7480, 0.7580, 2.1695) -- (0.7950, 0.7580, 2.1740) -- (0.7950, 0.8110, 2.1781) -- (0.7480, 0.8110, 2.1736) -- cycle;
\fill[blue!28.6, opacity=0.7] (0.7480, 0.8110, 2.1736) -- (0.7950, 0.8110, 2.1781) -- (0.7950, 0.8640, 2.1819) -- (0.7480, 0.8640, 2.1774) -- cycle;
\fill[blue!60.5, opacity=0.7] (0.7480, 0.8640, 2.1774) -- (0.7950, 0.8640, 2.1819) -- (0.7950, 0.9170, 2.1855) -- (0.7480, 0.9170, 2.1809) -- cycle;
\fill[blue!87.3, opacity=0.7] (0.7480, 0.9170, 2.1809) -- (0.7950, 0.9170, 2.1855) -- (0.7950, 0.9700, 2.1888) -- (0.7480, 0.9700, 2.1842) -- cycle;
\fill[blue!78.7, opacity=0.7] (0.7480, 0.9700, 2.1842) -- (0.7950, 0.9700, 2.1888) -- (0.7950, 1.0230, 2.1918) -- (0.7480, 1.0230, 2.1872) -- cycle;
\fill[blue!66.0, opacity=0.7] (0.7480, 1.0230, 2.1872) -- (0.7950, 1.0230, 2.1918) -- (0.7950, 1.0760, 2.1945) -- (0.7480, 1.0760, 2.1899) -- cycle;
\fill[blue!65.8, opacity=0.7] (0.7480, 1.0760, 2.1899) -- (0.7950, 1.0760, 2.1945) -- (0.7950, 1.1290, 2.1969) -- (0.7480, 1.1290, 2.1923) -- cycle;
\fill[blue!76.0, opacity=0.7] (0.7480, 1.1290, 2.1923) -- (0.7950, 1.1290, 2.1969) -- (0.7950, 1.1820, 2.1990) -- (0.7480, 1.1820, 2.1944) -- cycle;
\fill[blue!86.6, opacity=0.7] (0.7480, 1.1820, 2.1944) -- (0.7950, 1.1820, 2.1990) -- (0.7950, 1.2350, 2.2008) -- (0.7480, 1.2350, 2.1962) -- cycle;
\fill[blue!85.3, opacity=0.7] (0.7480, 1.2350, 2.1962) -- (0.7950, 1.2350, 2.2008) -- (0.7950, 1.2880, 2.2022) -- (0.7480, 1.2880, 2.1977) -- cycle;
\fill[blue!72.2, opacity=0.7] (0.7480, 1.2880, 2.1977) -- (0.7950, 1.2880, 2.2022) -- (0.7950, 1.3410, 2.2034) -- (0.7480, 1.3410, 2.1988) -- cycle;
\fill[blue!57.1, opacity=0.7] (0.7480, 1.3410, 2.1988) -- (0.7950, 1.3410, 2.2034) -- (0.7950, 1.3940, 2.2042) -- (0.7480, 1.3940, 2.1996) -- cycle;
\fill[blue!45.9, opacity=0.7] (0.7480, 1.3940, 2.1996) -- (0.7950, 1.3940, 2.2042) -- (0.7950, 1.4470, 2.2047) -- (0.7480, 1.4470, 2.2001) -- cycle;
\fill[blue!39.1, opacity=0.7] (0.7480, 1.4470, 2.2001) -- (0.7950, 1.4470, 2.2047) -- (0.7950, 1.5000, 2.2049) -- (0.7480, 1.5000, 2.2003) -- cycle;
\fill[blue!35.1, opacity=0.7] (0.7480, 1.5000, 2.2003) -- (0.7950, 1.5000, 2.2049) -- (0.7950, 1.5530, 2.2047) -- (0.7480, 1.5530, 2.2001) -- cycle;
\fill[blue!32.9, opacity=0.7] (0.7480, 1.5530, 2.2001) -- (0.7950, 1.5530, 2.2047) -- (0.7950, 1.6060, 2.2042) -- (0.7480, 1.6060, 2.1996) -- cycle;
\fill[blue!31.7, opacity=0.7] (0.7480, 1.6060, 2.1996) -- (0.7950, 1.6060, 2.2042) -- (0.7950, 1.6590, 2.2034) -- (0.7480, 1.6590, 2.1988) -- cycle;
\fill[blue!31.2, opacity=0.7] (0.7480, 1.6590, 2.1988) -- (0.7950, 1.6590, 2.2034) -- (0.7950, 1.7120, 2.2022) -- (0.7480, 1.7120, 2.1977) -- cycle;
\fill[blue!31.6, opacity=0.7] (0.7480, 1.7120, 2.1977) -- (0.7950, 1.7120, 2.2022) -- (0.7950, 1.7650, 2.2008) -- (0.7480, 1.7650, 2.1962) -- cycle;
\fill[blue!33.0, opacity=0.7] (0.7480, 1.7650, 2.1962) -- (0.7950, 1.7650, 2.2008) -- (0.7950, 1.8180, 2.1990) -- (0.7480, 1.8180, 2.1944) -- cycle;
\fill[blue!36.4, opacity=0.7] (0.7480, 1.8180, 2.1944) -- (0.7950, 1.8180, 2.1990) -- (0.7950, 1.8710, 2.1969) -- (0.7480, 1.8710, 2.1923) -- cycle;
\fill[blue!42.9, opacity=0.7] (0.7480, 1.8710, 2.1923) -- (0.7950, 1.8710, 2.1969) -- (0.7950, 1.9240, 2.1945) -- (0.7480, 1.9240, 2.1899) -- cycle;
\fill[blue!53.8, opacity=0.7] (0.7480, 1.9240, 2.1899) -- (0.7950, 1.9240, 2.1945) -- (0.7950, 1.9770, 2.1918) -- (0.7480, 1.9770, 2.1872) -- cycle;
\fill[blue!68.7, opacity=0.7] (0.7480, 1.9770, 2.1872) -- (0.7950, 1.9770, 2.1918) -- (0.7950, 2.0300, 2.1888) -- (0.7480, 2.0300, 2.1842) -- cycle;
\fill[blue!82.5, opacity=0.7] (0.7480, 2.0300, 2.1842) -- (0.7950, 2.0300, 2.1888) -- (0.7950, 2.0830, 2.1855) -- (0.7480, 2.0830, 2.1809) -- cycle;
\fill[blue!87.9, opacity=0.7] (0.7480, 2.0830, 2.1809) -- (0.7950, 2.0830, 2.1855) -- (0.7950, 2.1360, 2.1819) -- (0.7480, 2.1360, 2.1774) -- cycle;
\fill[blue!84.9, opacity=0.7] (0.7480, 2.1360, 2.1774) -- (0.7950, 2.1360, 2.1819) -- (0.7950, 2.1890, 2.1781) -- (0.7480, 2.1890, 2.1736) -- cycle;
\fill[blue!82.2, opacity=0.7] (0.7480, 2.1890, 2.1736) -- (0.7950, 2.1890, 2.1781) -- (0.7950, 2.2420, 2.1740) -- (0.7480, 2.2420, 2.1695) -- cycle;
\fill[blue!85.4, opacity=0.7] (0.7480, 2.2420, 2.1695) -- (0.7950, 2.2420, 2.1740) -- (0.7950, 2.2950, 2.1697) -- (0.7480, 2.2950, 2.1651) -- cycle;
\fill[blue!86.8, opacity=0.7] (0.7480, 2.2950, 2.1651) -- (0.7950, 2.2950, 2.1697) -- (0.7950, 2.3480, 2.1651) -- (0.7480, 2.3480, 2.1606) -- cycle;
\fill[blue!64.8, opacity=0.7] (0.7480, 2.3480, 2.1606) -- (0.7950, 2.3480, 2.1651) -- (0.7950, 2.4010, 2.1604) -- (0.7480, 2.4010, 2.1558) -- cycle;
\fill[blue!31.5, opacity=0.7] (0.7480, 2.4010, 2.1558) -- (0.7950, 2.4010, 2.1604) -- (0.7950, 2.4540, 2.1554) -- (0.7480, 2.4540, 2.1508) -- cycle;
\fill[blue!17.8, opacity=0.7] (0.7480, 2.4540, 2.1508) -- (0.7950, 2.4540, 2.1554) -- (0.7950, 2.5070, 2.1502) -- (0.7480, 2.5070, 2.1457) -- cycle;
\fill[blue!15.5, opacity=0.7] (0.7480, 2.5070, 2.1457) -- (0.7950, 2.5070, 2.1502) -- (0.7950, 2.5600, 2.1449) -- (0.7480, 2.5600, 2.1403) -- cycle;
\fill[blue!15.2, opacity=0.7] (0.7480, 2.5600, 2.1403) -- (0.7950, 2.5600, 2.1449) -- (0.7950, 2.6130, 2.1393) -- (0.7480, 2.6130, 2.1348) -- cycle;
\fill[blue!15.5, opacity=0.7] (0.7480, 2.6130, 2.1348) -- (0.7950, 2.6130, 2.1393) -- (0.7950, 2.6660, 2.1337) -- (0.7480, 2.6660, 2.1291) -- cycle;
\fill[blue!17.8, opacity=0.7] (0.7480, 2.6660, 2.1291) -- (0.7950, 2.6660, 2.1337) -- (0.7950, 2.7190, 2.1279) -- (0.7480, 2.7190, 2.1233) -- cycle;
\fill[blue!29.1, opacity=0.7] (0.7480, 2.7190, 2.1233) -- (0.7950, 2.7190, 2.1279) -- (0.7950, 2.7720, 2.1219) -- (0.7480, 2.7720, 2.1174) -- cycle;
\fill[blue!50.3, opacity=0.7] (0.7480, 2.7720, 2.1174) -- (0.7950, 2.7720, 2.1219) -- (0.7950, 2.8250, 2.1159) -- (0.7480, 2.8250, 2.1114) -- cycle;
\fill[blue!59.5, opacity=0.7] (0.7480, 2.8250, 2.1114) -- (0.7950, 2.8250, 2.1159) -- (0.7950, 2.8780, 2.1098) -- (0.7480, 2.8780, 2.1052) -- cycle;
\fill[blue!45.6, opacity=0.7] (0.7480, 2.8780, 2.1052) -- (0.7950, 2.8780, 2.1098) -- (0.7950, 2.9310, 2.1036) -- (0.7480, 2.9310, 2.0991) -- cycle;
\fill[blue!22.5, opacity=0.7] (0.7480, 2.9310, 2.0991) -- (0.7950, 2.9310, 2.1036) -- (0.7950, 2.9840, 2.0974) -- (0.7480, 2.9840, 2.0928) -- cycle;
\fill[blue!15.5, opacity=0.7] (0.7480, 2.9840, 2.0928) -- (0.7950, 2.9840, 2.0974) -- (0.7950, 3.0370, 2.0911) -- (0.7480, 3.0370, 2.0866) -- cycle;
\fill[blue!15.0, opacity=0.7] (0.7480, 3.0370, 2.0866) -- (0.7950, 3.0370, 2.0911) -- (0.7950, 3.0900, 2.0849) -- (0.7480, 3.0900, 2.0803) -- cycle;
\fill[blue!39.8, opacity=0.7] (0.7950, -0.0900, 2.0849) -- (0.8420, -0.0900, 2.0892) -- (0.8420, -0.0370, 2.0955) -- (0.7950, -0.0370, 2.0911) -- cycle;
\fill[blue!38.3, opacity=0.7] (0.7950, -0.0370, 2.0911) -- (0.8420, -0.0370, 2.0955) -- (0.8420, 0.0160, 2.1017) -- (0.7950, 0.0160, 2.0974) -- cycle;
\fill[blue!22.0, opacity=0.7] (0.7950, 0.0160, 2.0974) -- (0.8420, 0.0160, 2.1017) -- (0.8420, 0.0690, 2.1079) -- (0.7950, 0.0690, 2.1036) -- cycle;
\fill[blue!15.7, opacity=0.7] (0.7950, 0.0690, 2.1036) -- (0.8420, 0.0690, 2.1079) -- (0.8420, 0.1220, 2.1141) -- (0.7950, 0.1220, 2.1098) -- cycle;
\fill[blue!15.1, opacity=0.7] (0.7950, 0.1220, 2.1098) -- (0.8420, 0.1220, 2.1141) -- (0.8420, 0.1750, 2.1202) -- (0.7950, 0.1750, 2.1159) -- cycle;
\fill[blue!15.1, opacity=0.7] (0.7950, 0.1750, 2.1159) -- (0.8420, 0.1750, 2.1202) -- (0.8420, 0.2280, 2.1263) -- (0.7950, 0.2280, 2.1219) -- cycle;
\fill[blue!15.4, opacity=0.7] (0.7950, 0.2280, 2.1219) -- (0.8420, 0.2280, 2.1263) -- (0.8420, 0.2810, 2.1322) -- (0.7950, 0.2810, 2.1279) -- cycle;
\fill[blue!22.1, opacity=0.7] (0.7950, 0.2810, 2.1279) -- (0.8420, 0.2810, 2.1322) -- (0.8420, 0.3340, 2.1380) -- (0.7950, 0.3340, 2.1337) -- cycle;
\fill[blue!56.1, opacity=0.7] (0.7950, 0.3340, 2.1337) -- (0.8420, 0.3340, 2.1380) -- (0.8420, 0.3870, 2.1437) -- (0.7950, 0.3870, 2.1393) -- cycle;
\fill[blue!83.6, opacity=0.7] (0.7950, 0.3870, 2.1393) -- (0.8420, 0.3870, 2.1437) -- (0.8420, 0.4400, 2.1492) -- (0.7950, 0.4400, 2.1449) -- cycle;
\fill[blue!86.0, opacity=0.7] (0.7950, 0.4400, 2.1449) -- (0.8420, 0.4400, 2.1492) -- (0.8420, 0.4930, 2.1545) -- (0.7950, 0.4930, 2.1502) -- cycle;
\fill[blue!72.2, opacity=0.7] (0.7950, 0.4930, 2.1502) -- (0.8420, 0.4930, 2.1545) -- (0.8420, 0.5460, 2.1597) -- (0.7950, 0.5460, 2.1554) -- cycle;
\fill[blue!38.9, opacity=0.7] (0.7950, 0.5460, 2.1554) -- (0.8420, 0.5460, 2.1597) -- (0.8420, 0.5990, 2.1647) -- (0.7950, 0.5990, 2.1604) -- cycle;
\fill[blue!20.7, opacity=0.7] (0.7950, 0.5990, 2.1604) -- (0.8420, 0.5990, 2.1647) -- (0.8420, 0.6520, 2.1695) -- (0.7950, 0.6520, 2.1651) -- cycle;
\fill[blue!17.2, opacity=0.7] (0.7950, 0.6520, 2.1651) -- (0.8420, 0.6520, 2.1695) -- (0.8420, 0.7050, 2.1740) -- (0.7950, 0.7050, 2.1697) -- cycle;
\fill[blue!18.0, opacity=0.7] (0.7950, 0.7050, 2.1697) -- (0.8420, 0.7050, 2.1740) -- (0.8420, 0.7580, 2.1784) -- (0.7950, 0.7580, 2.1740) -- cycle;
\fill[blue!27.0, opacity=0.7] (0.7950, 0.7580, 2.1740) -- (0.8420, 0.7580, 2.1784) -- (0.8420, 0.8110, 2.1824) -- (0.7950, 0.8110, 2.1781) -- cycle;
\fill[blue!59.6, opacity=0.7] (0.7950, 0.8110, 2.1781) -- (0.8420, 0.8110, 2.1824) -- (0.8420, 0.8640, 2.1863) -- (0.7950, 0.8640, 2.1819) -- cycle;
\fill[blue!87.7, opacity=0.7] (0.7950, 0.8640, 2.1819) -- (0.8420, 0.8640, 2.1863) -- (0.8420, 0.9170, 2.1898) -- (0.7950, 0.9170, 2.1855) -- cycle;
\fill[blue!75.7, opacity=0.7] (0.7950, 0.9170, 2.1855) -- (0.8420, 0.9170, 2.1898) -- (0.8420, 0.9700, 2.1931) -- (0.7950, 0.9700, 2.1888) -- cycle;
\fill[blue!63.2, opacity=0.7] (0.7950, 0.9700, 2.1888) -- (0.8420, 0.9700, 2.1931) -- (0.8420, 1.0230, 2.1961) -- (0.7950, 1.0230, 2.1918) -- cycle;
\fill[blue!66.9, opacity=0.7] (0.7950, 1.0230, 2.1918) -- (0.8420, 1.0230, 2.1961) -- (0.8420, 1.0760, 2.1988) -- (0.7950, 1.0760, 2.1945) -- cycle;
\fill[blue!81.1, opacity=0.7] (0.7950, 1.0760, 2.1945) -- (0.8420, 1.0760, 2.1988) -- (0.8420, 1.1290, 2.2012) -- (0.7950, 1.1290, 2.1969) -- cycle;
\fill[blue!87.5, opacity=0.7] (0.7950, 1.1290, 2.1969) -- (0.8420, 1.1290, 2.2012) -- (0.8420, 1.1820, 2.2033) -- (0.7950, 1.1820, 2.1990) -- cycle;
\fill[blue!74.1, opacity=0.7] (0.7950, 1.1820, 2.1990) -- (0.8420, 1.1820, 2.2033) -- (0.8420, 1.2350, 2.2051) -- (0.7950, 1.2350, 2.2008) -- cycle;
\fill[blue!54.4, opacity=0.7] (0.7950, 1.2350, 2.2008) -- (0.8420, 1.2350, 2.2051) -- (0.8420, 1.2880, 2.2066) -- (0.7950, 1.2880, 2.2022) -- cycle;
\fill[blue!41.1, opacity=0.7] (0.7950, 1.2880, 2.2022) -- (0.8420, 1.2880, 2.2066) -- (0.8420, 1.3410, 2.2077) -- (0.7950, 1.3410, 2.2034) -- cycle;
\fill[blue!34.8, opacity=0.7] (0.7950, 1.3410, 2.2034) -- (0.8420, 1.3410, 2.2077) -- (0.8420, 1.3940, 2.2085) -- (0.7950, 1.3940, 2.2042) -- cycle;
\fill[blue!32.5, opacity=0.7] (0.7950, 1.3940, 2.2042) -- (0.8420, 1.3940, 2.2085) -- (0.8420, 1.4470, 2.2090) -- (0.7950, 1.4470, 2.2047) -- cycle;
\fill[blue!32.4, opacity=0.7] (0.7950, 1.4470, 2.2047) -- (0.8420, 1.4470, 2.2090) -- (0.8420, 1.5000, 2.2092) -- (0.7950, 1.5000, 2.2049) -- cycle;
\fill[blue!33.1, opacity=0.7] (0.7950, 1.5000, 2.2049) -- (0.8420, 1.5000, 2.2092) -- (0.8420, 1.5530, 2.2090) -- (0.7950, 1.5530, 2.2047) -- cycle;
\fill[blue!33.7, opacity=0.7] (0.7950, 1.5530, 2.2047) -- (0.8420, 1.5530, 2.2090) -- (0.8420, 1.6060, 2.2085) -- (0.7950, 1.6060, 2.2042) -- cycle;
\fill[blue!33.5, opacity=0.7] (0.7950, 1.6060, 2.2042) -- (0.8420, 1.6060, 2.2085) -- (0.8420, 1.6590, 2.2077) -- (0.7950, 1.6590, 2.2034) -- cycle;
\fill[blue!32.3, opacity=0.7] (0.7950, 1.6590, 2.2034) -- (0.8420, 1.6590, 2.2077) -- (0.8420, 1.7120, 2.2066) -- (0.7950, 1.7120, 2.2022) -- cycle;
\fill[blue!30.4, opacity=0.7] (0.7950, 1.7120, 2.2022) -- (0.8420, 1.7120, 2.2066) -- (0.8420, 1.7650, 2.2051) -- (0.7950, 1.7650, 2.2008) -- cycle;
\fill[blue!28.4, opacity=0.7] (0.7950, 1.7650, 2.2008) -- (0.8420, 1.7650, 2.2051) -- (0.8420, 1.8180, 2.2033) -- (0.7950, 1.8180, 2.1990) -- cycle;
\fill[blue!27.0, opacity=0.7] (0.7950, 1.8180, 2.1990) -- (0.8420, 1.8180, 2.2033) -- (0.8420, 1.8710, 2.2012) -- (0.7950, 1.8710, 2.1969) -- cycle;
\fill[blue!26.9, opacity=0.7] (0.7950, 1.8710, 2.1969) -- (0.8420, 1.8710, 2.2012) -- (0.8420, 1.9240, 2.1988) -- (0.7950, 1.9240, 2.1945) -- cycle;
\fill[blue!28.9, opacity=0.7] (0.7950, 1.9240, 2.1945) -- (0.8420, 1.9240, 2.1988) -- (0.8420, 1.9770, 2.1961) -- (0.7950, 1.9770, 2.1918) -- cycle;
\fill[blue!34.9, opacity=0.7] (0.7950, 1.9770, 2.1918) -- (0.8420, 1.9770, 2.1961) -- (0.8420, 2.0300, 2.1931) -- (0.7950, 2.0300, 2.1888) -- cycle;
\fill[blue!47.4, opacity=0.7] (0.7950, 2.0300, 2.1888) -- (0.8420, 2.0300, 2.1931) -- (0.8420, 2.0830, 2.1898) -- (0.7950, 2.0830, 2.1855) -- cycle;
\fill[blue!66.8, opacity=0.7] (0.7950, 2.0830, 2.1855) -- (0.8420, 2.0830, 2.1898) -- (0.8420, 2.1360, 2.1863) -- (0.7950, 2.1360, 2.1819) -- cycle;
\fill[blue!83.9, opacity=0.7] (0.7950, 2.1360, 2.1819) -- (0.8420, 2.1360, 2.1863) -- (0.8420, 2.1890, 2.1824) -- (0.7950, 2.1890, 2.1781) -- cycle;
\fill[blue!87.5, opacity=0.7] (0.7950, 2.1890, 2.1781) -- (0.8420, 2.1890, 2.1824) -- (0.8420, 2.2420, 2.1784) -- (0.7950, 2.2420, 2.1740) -- cycle;
\fill[blue!83.6, opacity=0.7] (0.7950, 2.2420, 2.1740) -- (0.8420, 2.2420, 2.1784) -- (0.8420, 2.2950, 2.1740) -- (0.7950, 2.2950, 2.1697) -- cycle;
\fill[blue!84.6, opacity=0.7] (0.7950, 2.2950, 2.1697) -- (0.8420, 2.2950, 2.1740) -- (0.8420, 2.3480, 2.1695) -- (0.7950, 2.3480, 2.1651) -- cycle;
\fill[blue!87.6, opacity=0.7] (0.7950, 2.3480, 2.1651) -- (0.8420, 2.3480, 2.1695) -- (0.8420, 2.4010, 2.1647) -- (0.7950, 2.4010, 2.1604) -- cycle;
\fill[blue!67.7, opacity=0.7] (0.7950, 2.4010, 2.1604) -- (0.8420, 2.4010, 2.1647) -- (0.8420, 2.4540, 2.1597) -- (0.7950, 2.4540, 2.1554) -- cycle;
\fill[blue!32.0, opacity=0.7] (0.7950, 2.4540, 2.1554) -- (0.8420, 2.4540, 2.1597) -- (0.8420, 2.5070, 2.1545) -- (0.7950, 2.5070, 2.1502) -- cycle;
\fill[blue!17.6, opacity=0.7] (0.7950, 2.5070, 2.1502) -- (0.8420, 2.5070, 2.1545) -- (0.8420, 2.5600, 2.1492) -- (0.7950, 2.5600, 2.1449) -- cycle;
\fill[blue!15.4, opacity=0.7] (0.7950, 2.5600, 2.1449) -- (0.8420, 2.5600, 2.1492) -- (0.8420, 2.6130, 2.1437) -- (0.7950, 2.6130, 2.1393) -- cycle;
\fill[blue!15.2, opacity=0.7] (0.7950, 2.6130, 2.1393) -- (0.8420, 2.6130, 2.1437) -- (0.8420, 2.6660, 2.1380) -- (0.7950, 2.6660, 2.1337) -- cycle;
\fill[blue!15.6, opacity=0.7] (0.7950, 2.6660, 2.1337) -- (0.8420, 2.6660, 2.1380) -- (0.8420, 2.7190, 2.1322) -- (0.7950, 2.7190, 2.1279) -- cycle;
\fill[blue!18.8, opacity=0.7] (0.7950, 2.7190, 2.1279) -- (0.8420, 2.7190, 2.1322) -- (0.8420, 2.7720, 2.1263) -- (0.7950, 2.7720, 2.1219) -- cycle;
\fill[blue!33.3, opacity=0.7] (0.7950, 2.7720, 2.1219) -- (0.8420, 2.7720, 2.1263) -- (0.8420, 2.8250, 2.1202) -- (0.7950, 2.8250, 2.1159) -- cycle;
\fill[blue!54.1, opacity=0.7] (0.7950, 2.8250, 2.1159) -- (0.8420, 2.8250, 2.1202) -- (0.8420, 2.8780, 2.1141) -- (0.7950, 2.8780, 2.1098) -- cycle;
\fill[blue!57.1, opacity=0.7] (0.7950, 2.8780, 2.1098) -- (0.8420, 2.8780, 2.1141) -- (0.8420, 2.9310, 2.1079) -- (0.7950, 2.9310, 2.1036) -- cycle;
\fill[blue!36.7, opacity=0.7] (0.7950, 2.9310, 2.1036) -- (0.8420, 2.9310, 2.1079) -- (0.8420, 2.9840, 2.1017) -- (0.7950, 2.9840, 2.0974) -- cycle;
\fill[blue!18.1, opacity=0.7] (0.7950, 2.9840, 2.0974) -- (0.8420, 2.9840, 2.1017) -- (0.8420, 3.0370, 2.0955) -- (0.7950, 3.0370, 2.0911) -- cycle;
\fill[blue!15.1, opacity=0.7] (0.7950, 3.0370, 2.0911) -- (0.8420, 3.0370, 2.0955) -- (0.8420, 3.0900, 2.0892) -- (0.7950, 3.0900, 2.0849) -- cycle;
\fill[blue!42.9, opacity=0.7] (0.8420, -0.0900, 2.0892) -- (0.8890, -0.0900, 2.0933) -- (0.8890, -0.0370, 2.0995) -- (0.8420, -0.0370, 2.0955) -- cycle;
\fill[blue!32.3, opacity=0.7] (0.8420, -0.0370, 2.0955) -- (0.8890, -0.0370, 2.0995) -- (0.8890, 0.0160, 2.1058) -- (0.8420, 0.0160, 2.1017) -- cycle;
\fill[blue!18.1, opacity=0.7] (0.8420, 0.0160, 2.1017) -- (0.8890, 0.0160, 2.1058) -- (0.8890, 0.0690, 2.1120) -- (0.8420, 0.0690, 2.1079) -- cycle;
\fill[blue!15.2, opacity=0.7] (0.8420, 0.0690, 2.1079) -- (0.8890, 0.0690, 2.1120) -- (0.8890, 0.1220, 2.1182) -- (0.8420, 0.1220, 2.1141) -- cycle;
\fill[blue!15.0, opacity=0.7] (0.8420, 0.1220, 2.1141) -- (0.8890, 0.1220, 2.1182) -- (0.8890, 0.1750, 2.1243) -- (0.8420, 0.1750, 2.1202) -- cycle;
\fill[blue!15.1, opacity=0.7] (0.8420, 0.1750, 2.1202) -- (0.8890, 0.1750, 2.1243) -- (0.8890, 0.2280, 2.1303) -- (0.8420, 0.2280, 2.1263) -- cycle;
\fill[blue!16.8, opacity=0.7] (0.8420, 0.2280, 2.1263) -- (0.8890, 0.2280, 2.1303) -- (0.8890, 0.2810, 2.1363) -- (0.8420, 0.2810, 2.1322) -- cycle;
\fill[blue!35.9, opacity=0.7] (0.8420, 0.2810, 2.1322) -- (0.8890, 0.2810, 2.1363) -- (0.8890, 0.3340, 2.1421) -- (0.8420, 0.3340, 2.1380) -- cycle;
\fill[blue!75.4, opacity=0.7] (0.8420, 0.3340, 2.1380) -- (0.8890, 0.3340, 2.1421) -- (0.8890, 0.3870, 2.1477) -- (0.8420, 0.3870, 2.1437) -- cycle;
\fill[blue!86.6, opacity=0.7] (0.8420, 0.3870, 2.1437) -- (0.8890, 0.3870, 2.1477) -- (0.8890, 0.4400, 2.1533) -- (0.8420, 0.4400, 2.1492) -- cycle;
\fill[blue!81.8, opacity=0.7] (0.8420, 0.4400, 2.1492) -- (0.8890, 0.4400, 2.1533) -- (0.8890, 0.4930, 2.1586) -- (0.8420, 0.4930, 2.1545) -- cycle;
\fill[blue!53.2, opacity=0.7] (0.8420, 0.4930, 2.1545) -- (0.8890, 0.4930, 2.1586) -- (0.8890, 0.5460, 2.1638) -- (0.8420, 0.5460, 2.1597) -- cycle;
\fill[blue!25.2, opacity=0.7] (0.8420, 0.5460, 2.1597) -- (0.8890, 0.5460, 2.1638) -- (0.8890, 0.5990, 2.1688) -- (0.8420, 0.5990, 2.1647) -- cycle;
\fill[blue!17.8, opacity=0.7] (0.8420, 0.5990, 2.1647) -- (0.8890, 0.5990, 2.1688) -- (0.8890, 0.6520, 2.1736) -- (0.8420, 0.6520, 2.1695) -- cycle;
\fill[blue!17.5, opacity=0.7] (0.8420, 0.6520, 2.1695) -- (0.8890, 0.6520, 2.1736) -- (0.8890, 0.7050, 2.1781) -- (0.8420, 0.7050, 2.1740) -- cycle;
\fill[blue!23.7, opacity=0.7] (0.8420, 0.7050, 2.1740) -- (0.8890, 0.7050, 2.1781) -- (0.8890, 0.7580, 2.1824) -- (0.8420, 0.7580, 2.1784) -- cycle;
\fill[blue!52.8, opacity=0.7] (0.8420, 0.7580, 2.1784) -- (0.8890, 0.7580, 2.1824) -- (0.8890, 0.8110, 2.1865) -- (0.8420, 0.8110, 2.1824) -- cycle;
\fill[blue!87.0, opacity=0.7] (0.8420, 0.8110, 2.1824) -- (0.8890, 0.8110, 2.1865) -- (0.8890, 0.8640, 2.1903) -- (0.8420, 0.8640, 2.1863) -- cycle;
\fill[blue!75.7, opacity=0.7] (0.8420, 0.8640, 2.1863) -- (0.8890, 0.8640, 2.1903) -- (0.8890, 0.9170, 2.1939) -- (0.8420, 0.9170, 2.1898) -- cycle;
\fill[blue!61.7, opacity=0.7] (0.8420, 0.9170, 2.1898) -- (0.8890, 0.9170, 2.1939) -- (0.8890, 0.9700, 2.1972) -- (0.8420, 0.9700, 2.1931) -- cycle;
\fill[blue!66.8, opacity=0.7] (0.8420, 0.9700, 2.1931) -- (0.8890, 0.9700, 2.1972) -- (0.8890, 1.0230, 2.2002) -- (0.8420, 1.0230, 2.1961) -- cycle;
\fill[blue!83.3, opacity=0.7] (0.8420, 1.0230, 2.1961) -- (0.8890, 1.0230, 2.2002) -- (0.8890, 1.0760, 2.2029) -- (0.8420, 1.0760, 2.1988) -- cycle;
\fill[blue!85.2, opacity=0.7] (0.8420, 1.0760, 2.1988) -- (0.8890, 1.0760, 2.2029) -- (0.8890, 1.1290, 2.2053) -- (0.8420, 1.1290, 2.2012) -- cycle;
\fill[blue!64.3, opacity=0.7] (0.8420, 1.1290, 2.2012) -- (0.8890, 1.1290, 2.2053) -- (0.8890, 1.1820, 2.2074) -- (0.8420, 1.1820, 2.2033) -- cycle;
\fill[blue!44.8, opacity=0.7] (0.8420, 1.1820, 2.2033) -- (0.8890, 1.1820, 2.2074) -- (0.8890, 1.2350, 2.2092) -- (0.8420, 1.2350, 2.2051) -- cycle;
\fill[blue!36.1, opacity=0.7] (0.8420, 1.2350, 2.2051) -- (0.8890, 1.2350, 2.2092) -- (0.8890, 1.2880, 2.2106) -- (0.8420, 1.2880, 2.2066) -- cycle;
\fill[blue!35.0, opacity=0.7] (0.8420, 1.2880, 2.2066) -- (0.8890, 1.2880, 2.2106) -- (0.8890, 1.3410, 2.2118) -- (0.8420, 1.3410, 2.2077) -- cycle;
\fill[blue!38.9, opacity=0.7] (0.8420, 1.3410, 2.2077) -- (0.8890, 1.3410, 2.2118) -- (0.8890, 1.3940, 2.2126) -- (0.8420, 1.3940, 2.2085) -- cycle;
\fill[blue!46.3, opacity=0.7] (0.8420, 1.3940, 2.2085) -- (0.8890, 1.3940, 2.2126) -- (0.8890, 1.4470, 2.2131) -- (0.8420, 1.4470, 2.2090) -- cycle;
\fill[blue!55.1, opacity=0.7] (0.8420, 1.4470, 2.2090) -- (0.8890, 1.4470, 2.2131) -- (0.8890, 1.5000, 2.2133) -- (0.8420, 1.5000, 2.2092) -- cycle;
\fill[blue!62.6, opacity=0.7] (0.8420, 1.5000, 2.2092) -- (0.8890, 1.5000, 2.2133) -- (0.8890, 1.5530, 2.2131) -- (0.8420, 1.5530, 2.2090) -- cycle;
\fill[blue!67.1, opacity=0.7] (0.8420, 1.5530, 2.2090) -- (0.8890, 1.5530, 2.2131) -- (0.8890, 1.6060, 2.2126) -- (0.8420, 1.6060, 2.2085) -- cycle;
\fill[blue!68.1, opacity=0.7] (0.8420, 1.6060, 2.2085) -- (0.8890, 1.6060, 2.2126) -- (0.8890, 1.6590, 2.2118) -- (0.8420, 1.6590, 2.2077) -- cycle;
\fill[blue!65.5, opacity=0.7] (0.8420, 1.6590, 2.2077) -- (0.8890, 1.6590, 2.2118) -- (0.8890, 1.7120, 2.2106) -- (0.8420, 1.7120, 2.2066) -- cycle;
\fill[blue!59.3, opacity=0.7] (0.8420, 1.7120, 2.2066) -- (0.8890, 1.7120, 2.2106) -- (0.8890, 1.7650, 2.2092) -- (0.8420, 1.7650, 2.2051) -- cycle;
\fill[blue!50.3, opacity=0.7] (0.8420, 1.7650, 2.2051) -- (0.8890, 1.7650, 2.2092) -- (0.8890, 1.8180, 2.2074) -- (0.8420, 1.8180, 2.2033) -- cycle;
\fill[blue!40.4, opacity=0.7] (0.8420, 1.8180, 2.2033) -- (0.8890, 1.8180, 2.2074) -- (0.8890, 1.8710, 2.2053) -- (0.8420, 1.8710, 2.2012) -- cycle;
\fill[blue!32.2, opacity=0.7] (0.8420, 1.8710, 2.2012) -- (0.8890, 1.8710, 2.2053) -- (0.8890, 1.9240, 2.2029) -- (0.8420, 1.9240, 2.1988) -- cycle;
\fill[blue!27.1, opacity=0.7] (0.8420, 1.9240, 2.1988) -- (0.8890, 1.9240, 2.2029) -- (0.8890, 1.9770, 2.2002) -- (0.8420, 1.9770, 2.1961) -- cycle;
\fill[blue!25.3, opacity=0.7] (0.8420, 1.9770, 2.1961) -- (0.8890, 1.9770, 2.2002) -- (0.8890, 2.0300, 2.1972) -- (0.8420, 2.0300, 2.1931) -- cycle;
\fill[blue!27.1, opacity=0.7] (0.8420, 2.0300, 2.1931) -- (0.8890, 2.0300, 2.1972) -- (0.8890, 2.0830, 2.1939) -- (0.8420, 2.0830, 2.1898) -- cycle;
\fill[blue!34.7, opacity=0.7] (0.8420, 2.0830, 2.1898) -- (0.8890, 2.0830, 2.1939) -- (0.8890, 2.1360, 2.1903) -- (0.8420, 2.1360, 2.1863) -- cycle;
\fill[blue!52.2, opacity=0.7] (0.8420, 2.1360, 2.1863) -- (0.8890, 2.1360, 2.1903) -- (0.8890, 2.1890, 2.1865) -- (0.8420, 2.1890, 2.1824) -- cycle;
\fill[blue!75.9, opacity=0.7] (0.8420, 2.1890, 2.1824) -- (0.8890, 2.1890, 2.1865) -- (0.8890, 2.2420, 2.1824) -- (0.8420, 2.2420, 2.1784) -- cycle;
\fill[blue!87.7, opacity=0.7] (0.8420, 2.2420, 2.1784) -- (0.8890, 2.2420, 2.1824) -- (0.8890, 2.2950, 2.1781) -- (0.8420, 2.2950, 2.1740) -- cycle;
\fill[blue!85.0, opacity=0.7] (0.8420, 2.2950, 2.1740) -- (0.8890, 2.2950, 2.1781) -- (0.8890, 2.3480, 2.1736) -- (0.8420, 2.3480, 2.1695) -- cycle;
\fill[blue!84.6, opacity=0.7] (0.8420, 2.3480, 2.1695) -- (0.8890, 2.3480, 2.1736) -- (0.8890, 2.4010, 2.1688) -- (0.8420, 2.4010, 2.1647) -- cycle;
\fill[blue!87.6, opacity=0.7] (0.8420, 2.4010, 2.1647) -- (0.8890, 2.4010, 2.1688) -- (0.8890, 2.4540, 2.1638) -- (0.8420, 2.4540, 2.1597) -- cycle;
\fill[blue!66.1, opacity=0.7] (0.8420, 2.4540, 2.1597) -- (0.8890, 2.4540, 2.1638) -- (0.8890, 2.5070, 2.1586) -- (0.8420, 2.5070, 2.1545) -- cycle;
\fill[blue!29.4, opacity=0.7] (0.8420, 2.5070, 2.1545) -- (0.8890, 2.5070, 2.1586) -- (0.8890, 2.5600, 2.1533) -- (0.8420, 2.5600, 2.1492) -- cycle;
\fill[blue!16.9, opacity=0.7] (0.8420, 2.5600, 2.1492) -- (0.8890, 2.5600, 2.1533) -- (0.8890, 2.6130, 2.1477) -- (0.8420, 2.6130, 2.1437) -- cycle;
\fill[blue!15.3, opacity=0.7] (0.8420, 2.6130, 2.1437) -- (0.8890, 2.6130, 2.1477) -- (0.8890, 2.6660, 2.1421) -- (0.8420, 2.6660, 2.1380) -- cycle;
\fill[blue!15.2, opacity=0.7] (0.8420, 2.6660, 2.1380) -- (0.8890, 2.6660, 2.1421) -- (0.8890, 2.7190, 2.1363) -- (0.8420, 2.7190, 2.1322) -- cycle;
\fill[blue!15.9, opacity=0.7] (0.8420, 2.7190, 2.1322) -- (0.8890, 2.7190, 2.1363) -- (0.8890, 2.7720, 2.1303) -- (0.8420, 2.7720, 2.1263) -- cycle;
\fill[blue!21.1, opacity=0.7] (0.8420, 2.7720, 2.1263) -- (0.8890, 2.7720, 2.1303) -- (0.8890, 2.8250, 2.1243) -- (0.8420, 2.8250, 2.1202) -- cycle;
\fill[blue!39.8, opacity=0.7] (0.8420, 2.8250, 2.1202) -- (0.8890, 2.8250, 2.1243) -- (0.8890, 2.8780, 2.1182) -- (0.8420, 2.8780, 2.1141) -- cycle;
\fill[blue!57.2, opacity=0.7] (0.8420, 2.8780, 2.1141) -- (0.8890, 2.8780, 2.1182) -- (0.8890, 2.9310, 2.1120) -- (0.8420, 2.9310, 2.1079) -- cycle;
\fill[blue!51.0, opacity=0.7] (0.8420, 2.9310, 2.1079) -- (0.8890, 2.9310, 2.1120) -- (0.8890, 2.9840, 2.1058) -- (0.8420, 2.9840, 2.1017) -- cycle;
\fill[blue!26.8, opacity=0.7] (0.8420, 2.9840, 2.1017) -- (0.8890, 2.9840, 2.1058) -- (0.8890, 3.0370, 2.0995) -- (0.8420, 3.0370, 2.0955) -- cycle;
\fill[blue!15.9, opacity=0.7] (0.8420, 3.0370, 2.0955) -- (0.8890, 3.0370, 2.0995) -- (0.8890, 3.0900, 2.0933) -- (0.8420, 3.0900, 2.0892) -- cycle;
\fill[blue!42.2, opacity=0.7] (0.8890, -0.0900, 2.0933) -- (0.9360, -0.0900, 2.0971) -- (0.9360, -0.0370, 2.1034) -- (0.8890, -0.0370, 2.0995) -- cycle;
\fill[blue!26.1, opacity=0.7] (0.8890, -0.0370, 2.0995) -- (0.9360, -0.0370, 2.1034) -- (0.9360, 0.0160, 2.1096) -- (0.8890, 0.0160, 2.1058) -- cycle;
\fill[blue!16.3, opacity=0.7] (0.8890, 0.0160, 2.1058) -- (0.9360, 0.0160, 2.1096) -- (0.9360, 0.0690, 2.1159) -- (0.8890, 0.0690, 2.1120) -- cycle;
\fill[blue!15.1, opacity=0.7] (0.8890, 0.0690, 2.1120) -- (0.9360, 0.0690, 2.1159) -- (0.9360, 0.1220, 2.1220) -- (0.8890, 0.1220, 2.1182) -- cycle;
\fill[blue!15.1, opacity=0.7] (0.8890, 0.1220, 2.1182) -- (0.9360, 0.1220, 2.1220) -- (0.9360, 0.1750, 2.1281) -- (0.8890, 0.1750, 2.1243) -- cycle;
\fill[blue!15.3, opacity=0.7] (0.8890, 0.1750, 2.1243) -- (0.9360, 0.1750, 2.1281) -- (0.9360, 0.2280, 2.1342) -- (0.8890, 0.2280, 2.1303) -- cycle;
\fill[blue!21.2, opacity=0.7] (0.8890, 0.2280, 2.1303) -- (0.9360, 0.2280, 2.1342) -- (0.9360, 0.2810, 2.1401) -- (0.8890, 0.2810, 2.1363) -- cycle;
\fill[blue!55.9, opacity=0.7] (0.8890, 0.2810, 2.1363) -- (0.9360, 0.2810, 2.1401) -- (0.9360, 0.3340, 2.1459) -- (0.8890, 0.3340, 2.1421) -- cycle;
\fill[blue!84.4, opacity=0.7] (0.8890, 0.3340, 2.1421) -- (0.9360, 0.3340, 2.1459) -- (0.9360, 0.3870, 2.1516) -- (0.8890, 0.3870, 2.1477) -- cycle;
\fill[blue!86.4, opacity=0.7] (0.8890, 0.3870, 2.1477) -- (0.9360, 0.3870, 2.1516) -- (0.9360, 0.4400, 2.1571) -- (0.8890, 0.4400, 2.1533) -- cycle;
\fill[blue!70.8, opacity=0.7] (0.8890, 0.4400, 2.1533) -- (0.9360, 0.4400, 2.1571) -- (0.9360, 0.4930, 2.1624) -- (0.8890, 0.4930, 2.1586) -- cycle;
\fill[blue!35.6, opacity=0.7] (0.8890, 0.4930, 2.1586) -- (0.9360, 0.4930, 2.1624) -- (0.9360, 0.5460, 2.1676) -- (0.8890, 0.5460, 2.1638) -- cycle;
\fill[blue!19.7, opacity=0.7] (0.8890, 0.5460, 2.1638) -- (0.9360, 0.5460, 2.1676) -- (0.9360, 0.5990, 2.1726) -- (0.8890, 0.5990, 2.1688) -- cycle;
\fill[blue!17.3, opacity=0.7] (0.8890, 0.5990, 2.1688) -- (0.9360, 0.5990, 2.1726) -- (0.9360, 0.6520, 2.1774) -- (0.8890, 0.6520, 2.1736) -- cycle;
\fill[blue!20.4, opacity=0.7] (0.8890, 0.6520, 2.1736) -- (0.9360, 0.6520, 2.1774) -- (0.9360, 0.7050, 2.1819) -- (0.8890, 0.7050, 2.1781) -- cycle;
\fill[blue!41.3, opacity=0.7] (0.8890, 0.7050, 2.1781) -- (0.9360, 0.7050, 2.1819) -- (0.9360, 0.7580, 2.1863) -- (0.8890, 0.7580, 2.1824) -- cycle;
\fill[blue!82.9, opacity=0.7] (0.8890, 0.7580, 2.1824) -- (0.9360, 0.7580, 2.1863) -- (0.9360, 0.8110, 2.1903) -- (0.8890, 0.8110, 2.1865) -- cycle;
\fill[blue!79.0, opacity=0.7] (0.8890, 0.8110, 2.1865) -- (0.9360, 0.8110, 2.1903) -- (0.9360, 0.8640, 2.1942) -- (0.8890, 0.8640, 2.1903) -- cycle;
\fill[blue!61.1, opacity=0.7] (0.8890, 0.8640, 2.1903) -- (0.9360, 0.8640, 2.1942) -- (0.9360, 0.9170, 2.1977) -- (0.8890, 0.9170, 2.1939) -- cycle;
\fill[blue!64.5, opacity=0.7] (0.8890, 0.9170, 2.1939) -- (0.9360, 0.9170, 2.1977) -- (0.9360, 0.9700, 2.2010) -- (0.8890, 0.9700, 2.1972) -- cycle;
\fill[blue!82.8, opacity=0.7] (0.8890, 0.9700, 2.1972) -- (0.9360, 0.9700, 2.2010) -- (0.9360, 1.0230, 2.2040) -- (0.8890, 1.0230, 2.2002) -- cycle;
\fill[blue!84.2, opacity=0.7] (0.8890, 1.0230, 2.2002) -- (0.9360, 1.0230, 2.2040) -- (0.9360, 1.0760, 2.2067) -- (0.8890, 1.0760, 2.2029) -- cycle;
\fill[blue!59.9, opacity=0.7] (0.8890, 1.0760, 2.2029) -- (0.9360, 1.0760, 2.2067) -- (0.9360, 1.1290, 2.2091) -- (0.8890, 1.1290, 2.2053) -- cycle;
\fill[blue!41.5, opacity=0.7] (0.8890, 1.1290, 2.2053) -- (0.9360, 1.1290, 2.2091) -- (0.9360, 1.1820, 2.2112) -- (0.8890, 1.1820, 2.2074) -- cycle;
\fill[blue!36.6, opacity=0.7] (0.8890, 1.1820, 2.2074) -- (0.9360, 1.1820, 2.2112) -- (0.9360, 1.2350, 2.2130) -- (0.8890, 1.2350, 2.2092) -- cycle;
\fill[blue!41.5, opacity=0.7] (0.8890, 1.2350, 2.2092) -- (0.9360, 1.2350, 2.2130) -- (0.9360, 1.2880, 2.2145) -- (0.8890, 1.2880, 2.2106) -- cycle;
\fill[blue!55.1, opacity=0.7] (0.8890, 1.2880, 2.2106) -- (0.9360, 1.2880, 2.2145) -- (0.9360, 1.3410, 2.2156) -- (0.8890, 1.3410, 2.2118) -- cycle;
\fill[blue!72.9, opacity=0.7] (0.8890, 1.3410, 2.2118) -- (0.9360, 1.3410, 2.2156) -- (0.9360, 1.3940, 2.2164) -- (0.8890, 1.3940, 2.2126) -- cycle;
\fill[blue!85.2, opacity=0.7] (0.8890, 1.3940, 2.2126) -- (0.9360, 1.3940, 2.2164) -- (0.9360, 1.4470, 2.2169) -- (0.8890, 1.4470, 2.2131) -- cycle;
\fill[blue!87.7, opacity=0.7] (0.8890, 1.4470, 2.2131) -- (0.9360, 1.4470, 2.2169) -- (0.9360, 1.5000, 2.2171) -- (0.8890, 1.5000, 2.2133) -- cycle;
\fill[blue!84.4, opacity=0.7] (0.8890, 1.5000, 2.2133) -- (0.9360, 1.5000, 2.2171) -- (0.9360, 1.5530, 2.2169) -- (0.8890, 1.5530, 2.2131) -- cycle;
\fill[blue!81.0, opacity=0.7] (0.8890, 1.5530, 2.2131) -- (0.9360, 1.5530, 2.2169) -- (0.9360, 1.6060, 2.2164) -- (0.8890, 1.6060, 2.2126) -- cycle;
\fill[blue!80.0, opacity=0.7] (0.8890, 1.6060, 2.2126) -- (0.9360, 1.6060, 2.2164) -- (0.9360, 1.6590, 2.2156) -- (0.8890, 1.6590, 2.2118) -- cycle;
\fill[blue!82.1, opacity=0.7] (0.8890, 1.6590, 2.2118) -- (0.9360, 1.6590, 2.2156) -- (0.9360, 1.7120, 2.2145) -- (0.8890, 1.7120, 2.2106) -- cycle;
\fill[blue!85.8, opacity=0.7] (0.8890, 1.7120, 2.2106) -- (0.9360, 1.7120, 2.2145) -- (0.9360, 1.7650, 2.2130) -- (0.8890, 1.7650, 2.2092) -- cycle;
\fill[blue!87.8, opacity=0.7] (0.8890, 1.7650, 2.2092) -- (0.9360, 1.7650, 2.2130) -- (0.9360, 1.8180, 2.2112) -- (0.8890, 1.8180, 2.2074) -- cycle;
\fill[blue!82.5, opacity=0.7] (0.8890, 1.8180, 2.2074) -- (0.9360, 1.8180, 2.2112) -- (0.9360, 1.8710, 2.2091) -- (0.8890, 1.8710, 2.2053) -- cycle;
\fill[blue!67.2, opacity=0.7] (0.8890, 1.8710, 2.2053) -- (0.9360, 1.8710, 2.2091) -- (0.9360, 1.9240, 2.2067) -- (0.8890, 1.9240, 2.2029) -- cycle;
\fill[blue!47.6, opacity=0.7] (0.8890, 1.9240, 2.2029) -- (0.9360, 1.9240, 2.2067) -- (0.9360, 1.9770, 2.2040) -- (0.8890, 1.9770, 2.2002) -- cycle;
\fill[blue!33.0, opacity=0.7] (0.8890, 1.9770, 2.2002) -- (0.9360, 1.9770, 2.2040) -- (0.9360, 2.0300, 2.2010) -- (0.8890, 2.0300, 2.1972) -- cycle;
\fill[blue!25.8, opacity=0.7] (0.8890, 2.0300, 2.1972) -- (0.9360, 2.0300, 2.2010) -- (0.9360, 2.0830, 2.1977) -- (0.8890, 2.0830, 2.1939) -- cycle;
\fill[blue!24.5, opacity=0.7] (0.8890, 2.0830, 2.1939) -- (0.9360, 2.0830, 2.1977) -- (0.9360, 2.1360, 2.1942) -- (0.8890, 2.1360, 2.1903) -- cycle;
\fill[blue!28.8, opacity=0.7] (0.8890, 2.1360, 2.1903) -- (0.9360, 2.1360, 2.1942) -- (0.9360, 2.1890, 2.1903) -- (0.8890, 2.1890, 2.1865) -- cycle;
\fill[blue!43.2, opacity=0.7] (0.8890, 2.1890, 2.1865) -- (0.9360, 2.1890, 2.1903) -- (0.9360, 2.2420, 2.1863) -- (0.8890, 2.2420, 2.1824) -- cycle;
\fill[blue!69.2, opacity=0.7] (0.8890, 2.2420, 2.1824) -- (0.9360, 2.2420, 2.1863) -- (0.9360, 2.2950, 2.1819) -- (0.8890, 2.2950, 2.1781) -- cycle;
\fill[blue!86.9, opacity=0.7] (0.8890, 2.2950, 2.1781) -- (0.9360, 2.2950, 2.1819) -- (0.9360, 2.3480, 2.1774) -- (0.8890, 2.3480, 2.1736) -- cycle;
\fill[blue!85.7, opacity=0.7] (0.8890, 2.3480, 2.1736) -- (0.9360, 2.3480, 2.1774) -- (0.9360, 2.4010, 2.1726) -- (0.8890, 2.4010, 2.1688) -- cycle;
\fill[blue!85.2, opacity=0.7] (0.8890, 2.4010, 2.1688) -- (0.9360, 2.4010, 2.1726) -- (0.9360, 2.4540, 2.1676) -- (0.8890, 2.4540, 2.1638) -- cycle;
\fill[blue!86.9, opacity=0.7] (0.8890, 2.4540, 2.1638) -- (0.9360, 2.4540, 2.1676) -- (0.9360, 2.5070, 2.1624) -- (0.8890, 2.5070, 2.1586) -- cycle;
\fill[blue!59.7, opacity=0.7] (0.8890, 2.5070, 2.1586) -- (0.9360, 2.5070, 2.1624) -- (0.9360, 2.5600, 2.1571) -- (0.8890, 2.5600, 2.1533) -- cycle;
\fill[blue!24.9, opacity=0.7] (0.8890, 2.5600, 2.1533) -- (0.9360, 2.5600, 2.1571) -- (0.9360, 2.6130, 2.1516) -- (0.8890, 2.6130, 2.1477) -- cycle;
\fill[blue!16.1, opacity=0.7] (0.8890, 2.6130, 2.1477) -- (0.9360, 2.6130, 2.1516) -- (0.9360, 2.6660, 2.1459) -- (0.8890, 2.6660, 2.1421) -- cycle;
\fill[blue!15.2, opacity=0.7] (0.8890, 2.6660, 2.1421) -- (0.9360, 2.6660, 2.1459) -- (0.9360, 2.7190, 2.1401) -- (0.8890, 2.7190, 2.1363) -- cycle;
\fill[blue!15.3, opacity=0.7] (0.8890, 2.7190, 2.1363) -- (0.9360, 2.7190, 2.1401) -- (0.9360, 2.7720, 2.1342) -- (0.8890, 2.7720, 2.1303) -- cycle;
\fill[blue!16.6, opacity=0.7] (0.8890, 2.7720, 2.1303) -- (0.9360, 2.7720, 2.1342) -- (0.9360, 2.8250, 2.1281) -- (0.8890, 2.8250, 2.1243) -- cycle;
\fill[blue!25.8, opacity=0.7] (0.8890, 2.8250, 2.1243) -- (0.9360, 2.8250, 2.1281) -- (0.9360, 2.8780, 2.1220) -- (0.8890, 2.8780, 2.1182) -- cycle;
\fill[blue!47.7, opacity=0.7] (0.8890, 2.8780, 2.1182) -- (0.9360, 2.8780, 2.1220) -- (0.9360, 2.9310, 2.1159) -- (0.8890, 2.9310, 2.1120) -- cycle;
\fill[blue!57.3, opacity=0.7] (0.8890, 2.9310, 2.1120) -- (0.9360, 2.9310, 2.1159) -- (0.9360, 2.9840, 2.1096) -- (0.8890, 2.9840, 2.1058) -- cycle;
\fill[blue!40.5, opacity=0.7] (0.8890, 2.9840, 2.1058) -- (0.9360, 2.9840, 2.1096) -- (0.9360, 3.0370, 2.1034) -- (0.8890, 3.0370, 2.0995) -- cycle;
\fill[blue!19.3, opacity=0.7] (0.8890, 3.0370, 2.0995) -- (0.9360, 3.0370, 2.1034) -- (0.9360, 3.0900, 2.0971) -- (0.8890, 3.0900, 2.0933) -- cycle;
\fill[blue!38.9, opacity=0.7] (0.9360, -0.0900, 2.0971) -- (0.9830, -0.0900, 2.1006) -- (0.9830, -0.0370, 2.1069) -- (0.9360, -0.0370, 2.1034) -- cycle;
\fill[blue!21.5, opacity=0.7] (0.9360, -0.0370, 2.1034) -- (0.9830, -0.0370, 2.1069) -- (0.9830, 0.0160, 2.1132) -- (0.9360, 0.0160, 2.1096) -- cycle;
\fill[blue!15.6, opacity=0.7] (0.9360, 0.0160, 2.1096) -- (0.9830, 0.0160, 2.1132) -- (0.9830, 0.0690, 2.1194) -- (0.9360, 0.0690, 2.1159) -- cycle;
\fill[blue!15.1, opacity=0.7] (0.9360, 0.0690, 2.1159) -- (0.9830, 0.0690, 2.1194) -- (0.9830, 0.1220, 2.1256) -- (0.9360, 0.1220, 2.1220) -- cycle;
\fill[blue!15.1, opacity=0.7] (0.9360, 0.1220, 2.1220) -- (0.9830, 0.1220, 2.1256) -- (0.9830, 0.1750, 2.1317) -- (0.9360, 0.1750, 2.1281) -- cycle;
\fill[blue!16.1, opacity=0.7] (0.9360, 0.1750, 2.1281) -- (0.9830, 0.1750, 2.1317) -- (0.9830, 0.2280, 2.1377) -- (0.9360, 0.2280, 2.1342) -- cycle;
\fill[blue!31.1, opacity=0.7] (0.9360, 0.2280, 2.1342) -- (0.9830, 0.2280, 2.1377) -- (0.9830, 0.2810, 2.1436) -- (0.9360, 0.2810, 2.1401) -- cycle;
\fill[blue!72.9, opacity=0.7] (0.9360, 0.2810, 2.1401) -- (0.9830, 0.2810, 2.1436) -- (0.9830, 0.3340, 2.1494) -- (0.9360, 0.3340, 2.1459) -- cycle;
\fill[blue!86.9, opacity=0.7] (0.9360, 0.3340, 2.1459) -- (0.9830, 0.3340, 2.1494) -- (0.9830, 0.3870, 2.1551) -- (0.9360, 0.3870, 2.1516) -- cycle;
\fill[blue!83.4, opacity=0.7] (0.9360, 0.3870, 2.1516) -- (0.9830, 0.3870, 2.1551) -- (0.9830, 0.4400, 2.1606) -- (0.9360, 0.4400, 2.1571) -- cycle;
\fill[blue!54.7, opacity=0.7] (0.9360, 0.4400, 2.1571) -- (0.9830, 0.4400, 2.1606) -- (0.9830, 0.4930, 2.1660) -- (0.9360, 0.4930, 2.1624) -- cycle;
\fill[blue!25.2, opacity=0.7] (0.9360, 0.4930, 2.1624) -- (0.9830, 0.4930, 2.1660) -- (0.9830, 0.5460, 2.1712) -- (0.9360, 0.5460, 2.1676) -- cycle;
\fill[blue!17.9, opacity=0.7] (0.9360, 0.5460, 2.1676) -- (0.9830, 0.5460, 2.1712) -- (0.9830, 0.5990, 2.1762) -- (0.9360, 0.5990, 2.1726) -- cycle;
\fill[blue!18.3, opacity=0.7] (0.9360, 0.5990, 2.1726) -- (0.9830, 0.5990, 2.1762) -- (0.9830, 0.6520, 2.1809) -- (0.9360, 0.6520, 2.1774) -- cycle;
\fill[blue!29.4, opacity=0.7] (0.9360, 0.6520, 2.1774) -- (0.9830, 0.6520, 2.1809) -- (0.9830, 0.7050, 2.1855) -- (0.9360, 0.7050, 2.1819) -- cycle;
\fill[blue!70.3, opacity=0.7] (0.9360, 0.7050, 2.1819) -- (0.9830, 0.7050, 2.1855) -- (0.9830, 0.7580, 2.1898) -- (0.9360, 0.7580, 2.1863) -- cycle;
\fill[blue!85.0, opacity=0.7] (0.9360, 0.7580, 2.1863) -- (0.9830, 0.7580, 2.1898) -- (0.9830, 0.8110, 2.1939) -- (0.9360, 0.8110, 2.1903) -- cycle;
\fill[blue!63.0, opacity=0.7] (0.9360, 0.8110, 2.1903) -- (0.9830, 0.8110, 2.1939) -- (0.9830, 0.8640, 2.1977) -- (0.9360, 0.8640, 2.1942) -- cycle;
\fill[blue!60.5, opacity=0.7] (0.9360, 0.8640, 2.1942) -- (0.9830, 0.8640, 2.1977) -- (0.9830, 0.9170, 2.2013) -- (0.9360, 0.9170, 2.1977) -- cycle;
\fill[blue!79.1, opacity=0.7] (0.9360, 0.9170, 2.1977) -- (0.9830, 0.9170, 2.2013) -- (0.9830, 0.9700, 2.2046) -- (0.9360, 0.9700, 2.2010) -- cycle;
\fill[blue!85.8, opacity=0.7] (0.9360, 0.9700, 2.2010) -- (0.9830, 0.9700, 2.2046) -- (0.9830, 1.0230, 2.2076) -- (0.9360, 1.0230, 2.2040) -- cycle;
\fill[blue!60.9, opacity=0.7] (0.9360, 1.0230, 2.2040) -- (0.9830, 1.0230, 2.2076) -- (0.9830, 1.0760, 2.2103) -- (0.9360, 1.0760, 2.2067) -- cycle;
\fill[blue!41.6, opacity=0.7] (0.9360, 1.0760, 2.2067) -- (0.9830, 1.0760, 2.2103) -- (0.9830, 1.1290, 2.2127) -- (0.9360, 1.1290, 2.2091) -- cycle;
\fill[blue!38.9, opacity=0.7] (0.9360, 1.1290, 2.2091) -- (0.9830, 1.1290, 2.2127) -- (0.9830, 1.1820, 2.2148) -- (0.9360, 1.1820, 2.2112) -- cycle;
\fill[blue!50.1, opacity=0.7] (0.9360, 1.1820, 2.2112) -- (0.9830, 1.1820, 2.2148) -- (0.9830, 1.2350, 2.2166) -- (0.9360, 1.2350, 2.2130) -- cycle;
\fill[blue!72.6, opacity=0.7] (0.9360, 1.2350, 2.2130) -- (0.9830, 1.2350, 2.2166) -- (0.9830, 1.2880, 2.2180) -- (0.9360, 1.2880, 2.2145) -- cycle;
\fill[blue!87.7, opacity=0.7] (0.9360, 1.2880, 2.2145) -- (0.9830, 1.2880, 2.2180) -- (0.9830, 1.3410, 2.2192) -- (0.9360, 1.3410, 2.2156) -- cycle;
\fill[blue!79.5, opacity=0.7] (0.9360, 1.3410, 2.2156) -- (0.9830, 1.3410, 2.2192) -- (0.9830, 1.3940, 2.2200) -- (0.9360, 1.3940, 2.2164) -- cycle;
\fill[blue!61.6, opacity=0.7] (0.9360, 1.3940, 2.2164) -- (0.9830, 1.3940, 2.2200) -- (0.9830, 1.4470, 2.2205) -- (0.9360, 1.4470, 2.2169) -- cycle;
\fill[blue!48.0, opacity=0.7] (0.9360, 1.4470, 2.2169) -- (0.9830, 1.4470, 2.2205) -- (0.9830, 1.5000, 2.2206) -- (0.9360, 1.5000, 2.2171) -- cycle;
\fill[blue!40.7, opacity=0.7] (0.9360, 1.5000, 2.2171) -- (0.9830, 1.5000, 2.2206) -- (0.9830, 1.5530, 2.2205) -- (0.9360, 1.5530, 2.2169) -- cycle;
\fill[blue!37.7, opacity=0.7] (0.9360, 1.5530, 2.2169) -- (0.9830, 1.5530, 2.2205) -- (0.9830, 1.6060, 2.2200) -- (0.9360, 1.6060, 2.2164) -- cycle;
\fill[blue!37.5, opacity=0.7] (0.9360, 1.6060, 2.2164) -- (0.9830, 1.6060, 2.2200) -- (0.9830, 1.6590, 2.2192) -- (0.9360, 1.6590, 2.2156) -- cycle;
\fill[blue!39.6, opacity=0.7] (0.9360, 1.6590, 2.2156) -- (0.9830, 1.6590, 2.2192) -- (0.9830, 1.7120, 2.2180) -- (0.9360, 1.7120, 2.2145) -- cycle;
\fill[blue!44.5, opacity=0.7] (0.9360, 1.7120, 2.2145) -- (0.9830, 1.7120, 2.2180) -- (0.9830, 1.7650, 2.2166) -- (0.9360, 1.7650, 2.2130) -- cycle;
\fill[blue!53.5, opacity=0.7] (0.9360, 1.7650, 2.2130) -- (0.9830, 1.7650, 2.2166) -- (0.9830, 1.8180, 2.2148) -- (0.9360, 1.8180, 2.2112) -- cycle;
\fill[blue!67.7, opacity=0.7] (0.9360, 1.8180, 2.2112) -- (0.9830, 1.8180, 2.2148) -- (0.9830, 1.8710, 2.2127) -- (0.9360, 1.8710, 2.2091) -- cycle;
\fill[blue!83.2, opacity=0.7] (0.9360, 1.8710, 2.2091) -- (0.9830, 1.8710, 2.2127) -- (0.9830, 1.9240, 2.2103) -- (0.9360, 1.9240, 2.2067) -- cycle;
\fill[blue!86.6, opacity=0.7] (0.9360, 1.9240, 2.2067) -- (0.9830, 1.9240, 2.2103) -- (0.9830, 1.9770, 2.2076) -- (0.9360, 1.9770, 2.2040) -- cycle;
\fill[blue!67.7, opacity=0.7] (0.9360, 1.9770, 2.2040) -- (0.9830, 1.9770, 2.2076) -- (0.9830, 2.0300, 2.2046) -- (0.9360, 2.0300, 2.2010) -- cycle;
\fill[blue!42.4, opacity=0.7] (0.9360, 2.0300, 2.2010) -- (0.9830, 2.0300, 2.2046) -- (0.9830, 2.0830, 2.2013) -- (0.9360, 2.0830, 2.1977) -- cycle;
\fill[blue!28.0, opacity=0.7] (0.9360, 2.0830, 2.1977) -- (0.9830, 2.0830, 2.2013) -- (0.9830, 2.1360, 2.1977) -- (0.9360, 2.1360, 2.1942) -- cycle;
\fill[blue!23.7, opacity=0.7] (0.9360, 2.1360, 2.1942) -- (0.9830, 2.1360, 2.1977) -- (0.9830, 2.1890, 2.1939) -- (0.9360, 2.1890, 2.1903) -- cycle;
\fill[blue!26.1, opacity=0.7] (0.9360, 2.1890, 2.1903) -- (0.9830, 2.1890, 2.1939) -- (0.9830, 2.2420, 2.1898) -- (0.9360, 2.2420, 2.1863) -- cycle;
\fill[blue!38.9, opacity=0.7] (0.9360, 2.2420, 2.1863) -- (0.9830, 2.2420, 2.1898) -- (0.9830, 2.2950, 2.1855) -- (0.9360, 2.2950, 2.1819) -- cycle;
\fill[blue!66.3, opacity=0.7] (0.9360, 2.2950, 2.1819) -- (0.9830, 2.2950, 2.1855) -- (0.9830, 2.3480, 2.1809) -- (0.9360, 2.3480, 2.1774) -- cycle;
\fill[blue!86.7, opacity=0.7] (0.9360, 2.3480, 2.1774) -- (0.9830, 2.3480, 2.1809) -- (0.9830, 2.4010, 2.1762) -- (0.9360, 2.4010, 2.1726) -- cycle;
\fill[blue!85.8, opacity=0.7] (0.9360, 2.4010, 2.1726) -- (0.9830, 2.4010, 2.1762) -- (0.9830, 2.4540, 2.1712) -- (0.9360, 2.4540, 2.1676) -- cycle;
\fill[blue!86.2, opacity=0.7] (0.9360, 2.4540, 2.1676) -- (0.9830, 2.4540, 2.1712) -- (0.9830, 2.5070, 2.1660) -- (0.9360, 2.5070, 2.1624) -- cycle;
\fill[blue!84.3, opacity=0.7] (0.9360, 2.5070, 2.1624) -- (0.9830, 2.5070, 2.1660) -- (0.9830, 2.5600, 2.1606) -- (0.9360, 2.5600, 2.1571) -- cycle;
\fill[blue!48.6, opacity=0.7] (0.9360, 2.5600, 2.1571) -- (0.9830, 2.5600, 2.1606) -- (0.9830, 2.6130, 2.1551) -- (0.9360, 2.6130, 2.1516) -- cycle;
\fill[blue!20.3, opacity=0.7] (0.9360, 2.6130, 2.1516) -- (0.9830, 2.6130, 2.1551) -- (0.9830, 2.6660, 2.1494) -- (0.9360, 2.6660, 2.1459) -- cycle;
\fill[blue!15.6, opacity=0.7] (0.9360, 2.6660, 2.1459) -- (0.9830, 2.6660, 2.1494) -- (0.9830, 2.7190, 2.1436) -- (0.9360, 2.7190, 2.1401) -- cycle;
\fill[blue!15.2, opacity=0.7] (0.9360, 2.7190, 2.1401) -- (0.9830, 2.7190, 2.1436) -- (0.9830, 2.7720, 2.1377) -- (0.9360, 2.7720, 2.1342) -- cycle;
\fill[blue!15.5, opacity=0.7] (0.9360, 2.7720, 2.1342) -- (0.9830, 2.7720, 2.1377) -- (0.9830, 2.8250, 2.1317) -- (0.9360, 2.8250, 2.1281) -- cycle;
\fill[blue!18.5, opacity=0.7] (0.9360, 2.8250, 2.1281) -- (0.9830, 2.8250, 2.1317) -- (0.9830, 2.8780, 2.1256) -- (0.9360, 2.8780, 2.1220) -- cycle;
\fill[blue!34.0, opacity=0.7] (0.9360, 2.8780, 2.1220) -- (0.9830, 2.8780, 2.1256) -- (0.9830, 2.9310, 2.1194) -- (0.9360, 2.9310, 2.1159) -- cycle;
\fill[blue!54.4, opacity=0.7] (0.9360, 2.9310, 2.1159) -- (0.9830, 2.9310, 2.1194) -- (0.9830, 2.9840, 2.1132) -- (0.9360, 2.9840, 2.1096) -- cycle;
\fill[blue!51.7, opacity=0.7] (0.9360, 2.9840, 2.1096) -- (0.9830, 2.9840, 2.1132) -- (0.9830, 3.0370, 2.1069) -- (0.9360, 3.0370, 2.1034) -- cycle;
\fill[blue!27.7, opacity=0.7] (0.9360, 3.0370, 2.1034) -- (0.9830, 3.0370, 2.1069) -- (0.9830, 3.0900, 2.1006) -- (0.9360, 3.0900, 2.0971) -- cycle;
\fill[blue!34.4, opacity=0.7] (0.9830, -0.0900, 2.1006) -- (1.0300, -0.0900, 2.1039) -- (1.0300, -0.0370, 2.1102) -- (0.9830, -0.0370, 2.1069) -- cycle;
\fill[blue!18.6, opacity=0.7] (0.9830, -0.0370, 2.1069) -- (1.0300, -0.0370, 2.1102) -- (1.0300, 0.0160, 2.1165) -- (0.9830, 0.0160, 2.1132) -- cycle;
\fill[blue!15.3, opacity=0.7] (0.9830, 0.0160, 2.1132) -- (1.0300, 0.0160, 2.1165) -- (1.0300, 0.0690, 2.1227) -- (0.9830, 0.0690, 2.1194) -- cycle;
\fill[blue!15.1, opacity=0.7] (0.9830, 0.0690, 2.1194) -- (1.0300, 0.0690, 2.1227) -- (1.0300, 0.1220, 2.1289) -- (0.9830, 0.1220, 2.1256) -- cycle;
\fill[blue!15.2, opacity=0.7] (0.9830, 0.1220, 2.1256) -- (1.0300, 0.1220, 2.1289) -- (1.0300, 0.1750, 2.1350) -- (0.9830, 0.1750, 2.1317) -- cycle;
\fill[blue!18.1, opacity=0.7] (0.9830, 0.1750, 2.1317) -- (1.0300, 0.1750, 2.1350) -- (1.0300, 0.2280, 2.1410) -- (0.9830, 0.2280, 2.1377) -- cycle;
\fill[blue!45.9, opacity=0.7] (0.9830, 0.2280, 2.1377) -- (1.0300, 0.2280, 2.1410) -- (1.0300, 0.2810, 2.1469) -- (0.9830, 0.2810, 2.1436) -- cycle;
\fill[blue!82.5, opacity=0.7] (0.9830, 0.2810, 2.1436) -- (1.0300, 0.2810, 2.1469) -- (1.0300, 0.3340, 2.1527) -- (0.9830, 0.3340, 2.1494) -- cycle;
\fill[blue!87.2, opacity=0.7] (0.9830, 0.3340, 2.1494) -- (1.0300, 0.3340, 2.1527) -- (1.0300, 0.3870, 2.1584) -- (0.9830, 0.3870, 2.1551) -- cycle;
\fill[blue!76.2, opacity=0.7] (0.9830, 0.3870, 2.1551) -- (1.0300, 0.3870, 2.1584) -- (1.0300, 0.4400, 2.1639) -- (0.9830, 0.4400, 2.1606) -- cycle;
\fill[blue!39.9, opacity=0.7] (0.9830, 0.4400, 2.1606) -- (1.0300, 0.4400, 2.1639) -- (1.0300, 0.4930, 2.1693) -- (0.9830, 0.4930, 2.1660) -- cycle;
\fill[blue!20.5, opacity=0.7] (0.9830, 0.4930, 2.1660) -- (1.0300, 0.4930, 2.1693) -- (1.0300, 0.5460, 2.1745) -- (0.9830, 0.5460, 2.1712) -- cycle;
\fill[blue!17.6, opacity=0.7] (0.9830, 0.5460, 2.1712) -- (1.0300, 0.5460, 2.1745) -- (1.0300, 0.5990, 2.1794) -- (0.9830, 0.5990, 2.1762) -- cycle;
\fill[blue!21.6, opacity=0.7] (0.9830, 0.5990, 2.1762) -- (1.0300, 0.5990, 2.1794) -- (1.0300, 0.6520, 2.1842) -- (0.9830, 0.6520, 2.1809) -- cycle;
\fill[blue!48.7, opacity=0.7] (0.9830, 0.6520, 2.1809) -- (1.0300, 0.6520, 2.1842) -- (1.0300, 0.7050, 2.1888) -- (0.9830, 0.7050, 2.1855) -- cycle;
\fill[blue!87.4, opacity=0.7] (0.9830, 0.7050, 2.1855) -- (1.0300, 0.7050, 2.1888) -- (1.0300, 0.7580, 2.1931) -- (0.9830, 0.7580, 2.1898) -- cycle;
\fill[blue!69.9, opacity=0.7] (0.9830, 0.7580, 2.1898) -- (1.0300, 0.7580, 2.1931) -- (1.0300, 0.8110, 2.1972) -- (0.9830, 0.8110, 2.1939) -- cycle;
\fill[blue!57.2, opacity=0.7] (0.9830, 0.8110, 2.1939) -- (1.0300, 0.8110, 2.1972) -- (1.0300, 0.8640, 2.2010) -- (0.9830, 0.8640, 2.1977) -- cycle;
\fill[blue!71.2, opacity=0.7] (0.9830, 0.8640, 2.1977) -- (1.0300, 0.8640, 2.2010) -- (1.0300, 0.9170, 2.2046) -- (0.9830, 0.9170, 2.2013) -- cycle;
\fill[blue!87.8, opacity=0.7] (0.9830, 0.9170, 2.2013) -- (1.0300, 0.9170, 2.2046) -- (1.0300, 0.9700, 2.2078) -- (0.9830, 0.9700, 2.2046) -- cycle;
\fill[blue!67.1, opacity=0.7] (0.9830, 0.9700, 2.2046) -- (1.0300, 0.9700, 2.2078) -- (1.0300, 1.0230, 2.2108) -- (0.9830, 1.0230, 2.2076) -- cycle;
\fill[blue!44.0, opacity=0.7] (0.9830, 1.0230, 2.2076) -- (1.0300, 1.0230, 2.2108) -- (1.0300, 1.0760, 2.2135) -- (0.9830, 1.0760, 2.2103) -- cycle;
\fill[blue!40.6, opacity=0.7] (0.9830, 1.0760, 2.2103) -- (1.0300, 1.0760, 2.2135) -- (1.0300, 1.1290, 2.2160) -- (0.9830, 1.1290, 2.2127) -- cycle;
\fill[blue!55.7, opacity=0.7] (0.9830, 1.1290, 2.2127) -- (1.0300, 1.1290, 2.2160) -- (1.0300, 1.1820, 2.2180) -- (0.9830, 1.1820, 2.2148) -- cycle;
\fill[blue!82.0, opacity=0.7] (0.9830, 1.1820, 2.2148) -- (1.0300, 1.1820, 2.2180) -- (1.0300, 1.2350, 2.2198) -- (0.9830, 1.2350, 2.2166) -- cycle;
\fill[blue!83.3, opacity=0.7] (0.9830, 1.2350, 2.2166) -- (1.0300, 1.2350, 2.2198) -- (1.0300, 1.2880, 2.2213) -- (0.9830, 1.2880, 2.2180) -- cycle;
\fill[blue!57.3, opacity=0.7] (0.9830, 1.2880, 2.2180) -- (1.0300, 1.2880, 2.2213) -- (1.0300, 1.3410, 2.2224) -- (0.9830, 1.3410, 2.2192) -- cycle;
\fill[blue!37.4, opacity=0.7] (0.9830, 1.3410, 2.2192) -- (1.0300, 1.3410, 2.2224) -- (1.0300, 1.3940, 2.2233) -- (0.9830, 1.3940, 2.2200) -- cycle;
\fill[blue!29.0, opacity=0.7] (0.9830, 1.3940, 2.2200) -- (1.0300, 1.3940, 2.2233) -- (1.0300, 1.4470, 2.2238) -- (0.9830, 1.4470, 2.2205) -- cycle;
\fill[blue!26.3, opacity=0.7] (0.9830, 1.4470, 2.2205) -- (1.0300, 1.4470, 2.2238) -- (1.0300, 1.5000, 2.2239) -- (0.9830, 1.5000, 2.2206) -- cycle;
\fill[blue!26.1, opacity=0.7] (0.9830, 1.5000, 2.2206) -- (1.0300, 1.5000, 2.2239) -- (1.0300, 1.5530, 2.2238) -- (0.9830, 1.5530, 2.2205) -- cycle;
\fill[blue!26.9, opacity=0.7] (0.9830, 1.5530, 2.2205) -- (1.0300, 1.5530, 2.2238) -- (1.0300, 1.6060, 2.2233) -- (0.9830, 1.6060, 2.2200) -- cycle;
\fill[blue!27.9, opacity=0.7] (0.9830, 1.6060, 2.2200) -- (1.0300, 1.6060, 2.2233) -- (1.0300, 1.6590, 2.2224) -- (0.9830, 1.6590, 2.2192) -- cycle;
\fill[blue!28.7, opacity=0.7] (0.9830, 1.6590, 2.2192) -- (1.0300, 1.6590, 2.2224) -- (1.0300, 1.7120, 2.2213) -- (0.9830, 1.7120, 2.2180) -- cycle;
\fill[blue!29.5, opacity=0.7] (0.9830, 1.7120, 2.2180) -- (1.0300, 1.7120, 2.2213) -- (1.0300, 1.7650, 2.2198) -- (0.9830, 1.7650, 2.2166) -- cycle;
\fill[blue!31.2, opacity=0.7] (0.9830, 1.7650, 2.2166) -- (1.0300, 1.7650, 2.2198) -- (1.0300, 1.8180, 2.2180) -- (0.9830, 1.8180, 2.2148) -- cycle;
\fill[blue!35.4, opacity=0.7] (0.9830, 1.8180, 2.2148) -- (1.0300, 1.8180, 2.2180) -- (1.0300, 1.8710, 2.2160) -- (0.9830, 1.8710, 2.2127) -- cycle;
\fill[blue!45.1, opacity=0.7] (0.9830, 1.8710, 2.2127) -- (1.0300, 1.8710, 2.2160) -- (1.0300, 1.9240, 2.2135) -- (0.9830, 1.9240, 2.2103) -- cycle;
\fill[blue!63.8, opacity=0.7] (0.9830, 1.9240, 2.2103) -- (1.0300, 1.9240, 2.2135) -- (1.0300, 1.9770, 2.2108) -- (0.9830, 1.9770, 2.2076) -- cycle;
\fill[blue!84.9, opacity=0.7] (0.9830, 1.9770, 2.2076) -- (1.0300, 1.9770, 2.2108) -- (1.0300, 2.0300, 2.2078) -- (0.9830, 2.0300, 2.2046) -- cycle;
\fill[blue!81.0, opacity=0.7] (0.9830, 2.0300, 2.2046) -- (1.0300, 2.0300, 2.2078) -- (1.0300, 2.0830, 2.2046) -- (0.9830, 2.0830, 2.2013) -- cycle;
\fill[blue!51.1, opacity=0.7] (0.9830, 2.0830, 2.2013) -- (1.0300, 2.0830, 2.2046) -- (1.0300, 2.1360, 2.2010) -- (0.9830, 2.1360, 2.1977) -- cycle;
\fill[blue!29.8, opacity=0.7] (0.9830, 2.1360, 2.1977) -- (1.0300, 2.1360, 2.2010) -- (1.0300, 2.1890, 2.1972) -- (0.9830, 2.1890, 2.1939) -- cycle;
\fill[blue!23.3, opacity=0.7] (0.9830, 2.1890, 2.1939) -- (1.0300, 2.1890, 2.1972) -- (1.0300, 2.2420, 2.1931) -- (0.9830, 2.2420, 2.1898) -- cycle;
\fill[blue!25.1, opacity=0.7] (0.9830, 2.2420, 2.1898) -- (1.0300, 2.2420, 2.1931) -- (1.0300, 2.2950, 2.1888) -- (0.9830, 2.2950, 2.1855) -- cycle;
\fill[blue!38.3, opacity=0.7] (0.9830, 2.2950, 2.1855) -- (1.0300, 2.2950, 2.1888) -- (1.0300, 2.3480, 2.1842) -- (0.9830, 2.3480, 2.1809) -- cycle;
\fill[blue!67.8, opacity=0.7] (0.9830, 2.3480, 2.1809) -- (1.0300, 2.3480, 2.1842) -- (1.0300, 2.4010, 2.1794) -- (0.9830, 2.4010, 2.1762) -- cycle;
\fill[blue!87.3, opacity=0.7] (0.9830, 2.4010, 2.1762) -- (1.0300, 2.4010, 2.1794) -- (1.0300, 2.4540, 2.1745) -- (0.9830, 2.4540, 2.1712) -- cycle;
\fill[blue!85.6, opacity=0.7] (0.9830, 2.4540, 2.1712) -- (1.0300, 2.4540, 2.1745) -- (1.0300, 2.5070, 2.1693) -- (0.9830, 2.5070, 2.1660) -- cycle;
\fill[blue!87.5, opacity=0.7] (0.9830, 2.5070, 2.1660) -- (1.0300, 2.5070, 2.1693) -- (1.0300, 2.5600, 2.1639) -- (0.9830, 2.5600, 2.1606) -- cycle;
\fill[blue!76.6, opacity=0.7] (0.9830, 2.5600, 2.1606) -- (1.0300, 2.5600, 2.1639) -- (1.0300, 2.6130, 2.1584) -- (0.9830, 2.6130, 2.1551) -- cycle;
\fill[blue!34.9, opacity=0.7] (0.9830, 2.6130, 2.1551) -- (1.0300, 2.6130, 2.1584) -- (1.0300, 2.6660, 2.1527) -- (0.9830, 2.6660, 2.1494) -- cycle;
\fill[blue!17.2, opacity=0.7] (0.9830, 2.6660, 2.1494) -- (1.0300, 2.6660, 2.1527) -- (1.0300, 2.7190, 2.1469) -- (0.9830, 2.7190, 2.1436) -- cycle;
\fill[blue!15.3, opacity=0.7] (0.9830, 2.7190, 2.1436) -- (1.0300, 2.7190, 2.1469) -- (1.0300, 2.7720, 2.1410) -- (0.9830, 2.7720, 2.1377) -- cycle;
\fill[blue!15.2, opacity=0.7] (0.9830, 2.7720, 2.1377) -- (1.0300, 2.7720, 2.1410) -- (1.0300, 2.8250, 2.1350) -- (0.9830, 2.8250, 2.1317) -- cycle;
\fill[blue!16.0, opacity=0.7] (0.9830, 2.8250, 2.1317) -- (1.0300, 2.8250, 2.1350) -- (1.0300, 2.8780, 2.1289) -- (0.9830, 2.8780, 2.1256) -- cycle;
\fill[blue!23.3, opacity=0.7] (0.9830, 2.8780, 2.1256) -- (1.0300, 2.8780, 2.1289) -- (1.0300, 2.9310, 2.1227) -- (0.9830, 2.9310, 2.1194) -- cycle;
\fill[blue!44.8, opacity=0.7] (0.9830, 2.9310, 2.1194) -- (1.0300, 2.9310, 2.1227) -- (1.0300, 2.9840, 2.1165) -- (0.9830, 2.9840, 2.1132) -- cycle;
\fill[blue!56.1, opacity=0.7] (0.9830, 2.9840, 2.1132) -- (1.0300, 2.9840, 2.1165) -- (1.0300, 3.0370, 2.1102) -- (0.9830, 3.0370, 2.1069) -- cycle;
\fill[blue!39.4, opacity=0.7] (0.9830, 3.0370, 2.1069) -- (1.0300, 3.0370, 2.1102) -- (1.0300, 3.0900, 2.1039) -- (0.9830, 3.0900, 2.1006) -- cycle;
\fill[blue!29.9, opacity=0.7] (1.0300, -0.0900, 2.1039) -- (1.0770, -0.0900, 2.1069) -- (1.0770, -0.0370, 2.1132) -- (1.0300, -0.0370, 2.1102) -- cycle;
\fill[blue!17.0, opacity=0.7] (1.0300, -0.0370, 2.1102) -- (1.0770, -0.0370, 2.1132) -- (1.0770, 0.0160, 2.1195) -- (1.0300, 0.0160, 2.1165) -- cycle;
\fill[blue!15.2, opacity=0.7] (1.0300, 0.0160, 2.1165) -- (1.0770, 0.0160, 2.1195) -- (1.0770, 0.0690, 2.1257) -- (1.0300, 0.0690, 2.1227) -- cycle;
\fill[blue!15.1, opacity=0.7] (1.0300, 0.0690, 2.1227) -- (1.0770, 0.0690, 2.1257) -- (1.0770, 0.1220, 2.1319) -- (1.0300, 0.1220, 2.1289) -- cycle;
\fill[blue!15.4, opacity=0.7] (1.0300, 0.1220, 2.1289) -- (1.0770, 0.1220, 2.1319) -- (1.0770, 0.1750, 2.1380) -- (1.0300, 0.1750, 2.1350) -- cycle;
\fill[blue!22.4, opacity=0.7] (1.0300, 0.1750, 2.1350) -- (1.0770, 0.1750, 2.1380) -- (1.0770, 0.2280, 2.1440) -- (1.0300, 0.2280, 2.1410) -- cycle;
\fill[blue!61.1, opacity=0.7] (1.0300, 0.2280, 2.1410) -- (1.0770, 0.2280, 2.1440) -- (1.0770, 0.2810, 2.1499) -- (1.0300, 0.2810, 2.1469) -- cycle;
\fill[blue!86.3, opacity=0.7] (1.0300, 0.2810, 2.1469) -- (1.0770, 0.2810, 2.1499) -- (1.0770, 0.3340, 2.1557) -- (1.0300, 0.3340, 2.1527) -- cycle;
\fill[blue!86.4, opacity=0.7] (1.0300, 0.3340, 2.1527) -- (1.0770, 0.3340, 2.1557) -- (1.0770, 0.3870, 2.1614) -- (1.0300, 0.3870, 2.1584) -- cycle;
\fill[blue!65.5, opacity=0.7] (1.0300, 0.3870, 2.1584) -- (1.0770, 0.3870, 2.1614) -- (1.0770, 0.4400, 2.1669) -- (1.0300, 0.4400, 2.1639) -- cycle;
\fill[blue!29.9, opacity=0.7] (1.0300, 0.4400, 2.1639) -- (1.0770, 0.4400, 2.1669) -- (1.0770, 0.4930, 2.1723) -- (1.0300, 0.4930, 2.1693) -- cycle;
\fill[blue!18.6, opacity=0.7] (1.0300, 0.4930, 2.1693) -- (1.0770, 0.4930, 2.1723) -- (1.0770, 0.5460, 2.1775) -- (1.0300, 0.5460, 2.1745) -- cycle;
\fill[blue!18.3, opacity=0.7] (1.0300, 0.5460, 2.1745) -- (1.0770, 0.5460, 2.1775) -- (1.0770, 0.5990, 2.1824) -- (1.0300, 0.5990, 2.1794) -- cycle;
\fill[blue!29.3, opacity=0.7] (1.0300, 0.5990, 2.1794) -- (1.0770, 0.5990, 2.1824) -- (1.0770, 0.6520, 2.1872) -- (1.0300, 0.6520, 2.1842) -- cycle;
\fill[blue!72.5, opacity=0.7] (1.0300, 0.6520, 2.1842) -- (1.0770, 0.6520, 2.1872) -- (1.0770, 0.7050, 2.1918) -- (1.0300, 0.7050, 2.1888) -- cycle;
\fill[blue!82.3, opacity=0.7] (1.0300, 0.7050, 2.1888) -- (1.0770, 0.7050, 2.1918) -- (1.0770, 0.7580, 2.1961) -- (1.0300, 0.7580, 2.1931) -- cycle;
\fill[blue!58.5, opacity=0.7] (1.0300, 0.7580, 2.1931) -- (1.0770, 0.7580, 2.1961) -- (1.0770, 0.8110, 2.2002) -- (1.0300, 0.8110, 2.1972) -- cycle;
\fill[blue!61.1, opacity=0.7] (1.0300, 0.8110, 2.1972) -- (1.0770, 0.8110, 2.2002) -- (1.0770, 0.8640, 2.2040) -- (1.0300, 0.8640, 2.2010) -- cycle;
\fill[blue!84.1, opacity=0.7] (1.0300, 0.8640, 2.2010) -- (1.0770, 0.8640, 2.2040) -- (1.0770, 0.9170, 2.2076) -- (1.0300, 0.9170, 2.2046) -- cycle;
\fill[blue!78.0, opacity=0.7] (1.0300, 0.9170, 2.2046) -- (1.0770, 0.9170, 2.2076) -- (1.0770, 0.9700, 2.2108) -- (1.0300, 0.9700, 2.2078) -- cycle;
\fill[blue!49.6, opacity=0.7] (1.0300, 0.9700, 2.2078) -- (1.0770, 0.9700, 2.2108) -- (1.0770, 1.0230, 2.2138) -- (1.0300, 1.0230, 2.2108) -- cycle;
\fill[blue!41.4, opacity=0.7] (1.0300, 1.0230, 2.2108) -- (1.0770, 1.0230, 2.2138) -- (1.0770, 1.0760, 2.2165) -- (1.0300, 1.0760, 2.2135) -- cycle;
\fill[blue!55.7, opacity=0.7] (1.0300, 1.0760, 2.2135) -- (1.0770, 1.0760, 2.2165) -- (1.0770, 1.1290, 2.2190) -- (1.0300, 1.1290, 2.2160) -- cycle;
\fill[blue!84.1, opacity=0.7] (1.0300, 1.1290, 2.2160) -- (1.0770, 1.1290, 2.2190) -- (1.0770, 1.1820, 2.2210) -- (1.0300, 1.1820, 2.2180) -- cycle;
\fill[blue!76.9, opacity=0.7] (1.0300, 1.1820, 2.2180) -- (1.0770, 1.1820, 2.2210) -- (1.0770, 1.2350, 2.2228) -- (1.0300, 1.2350, 2.2198) -- cycle;
\fill[blue!44.2, opacity=0.7] (1.0300, 1.2350, 2.2198) -- (1.0770, 1.2350, 2.2228) -- (1.0770, 1.2880, 2.2243) -- (1.0300, 1.2880, 2.2213) -- cycle;
\fill[blue!28.1, opacity=0.7] (1.0300, 1.2880, 2.2213) -- (1.0770, 1.2880, 2.2243) -- (1.0770, 1.3410, 2.2254) -- (1.0300, 1.3410, 2.2224) -- cycle;
\fill[blue!24.0, opacity=0.7] (1.0300, 1.3410, 2.2224) -- (1.0770, 1.3410, 2.2254) -- (1.0770, 1.3940, 2.2263) -- (1.0300, 1.3940, 2.2233) -- cycle;
\fill[blue!24.9, opacity=0.7] (1.0300, 1.3940, 2.2233) -- (1.0770, 1.3940, 2.2263) -- (1.0770, 1.4470, 2.2268) -- (1.0300, 1.4470, 2.2238) -- cycle;
\fill[blue!29.0, opacity=0.7] (1.0300, 1.4470, 2.2238) -- (1.0770, 1.4470, 2.2268) -- (1.0770, 1.5000, 2.2269) -- (1.0300, 1.5000, 2.2239) -- cycle;
\fill[blue!35.0, opacity=0.7] (1.0300, 1.5000, 2.2239) -- (1.0770, 1.5000, 2.2269) -- (1.0770, 1.5530, 2.2268) -- (1.0300, 1.5530, 2.2238) -- cycle;
\fill[blue!40.9, opacity=0.7] (1.0300, 1.5530, 2.2238) -- (1.0770, 1.5530, 2.2268) -- (1.0770, 1.6060, 2.2263) -- (1.0300, 1.6060, 2.2233) -- cycle;
\fill[blue!44.3, opacity=0.7] (1.0300, 1.6060, 2.2233) -- (1.0770, 1.6060, 2.2263) -- (1.0770, 1.6590, 2.2254) -- (1.0300, 1.6590, 2.2224) -- cycle;
\fill[blue!44.4, opacity=0.7] (1.0300, 1.6590, 2.2224) -- (1.0770, 1.6590, 2.2254) -- (1.0770, 1.7120, 2.2243) -- (1.0300, 1.7120, 2.2213) -- cycle;
\fill[blue!41.5, opacity=0.7] (1.0300, 1.7120, 2.2213) -- (1.0770, 1.7120, 2.2243) -- (1.0770, 1.7650, 2.2228) -- (1.0300, 1.7650, 2.2198) -- cycle;
\fill[blue!37.1, opacity=0.7] (1.0300, 1.7650, 2.2198) -- (1.0770, 1.7650, 2.2228) -- (1.0770, 1.8180, 2.2210) -- (1.0300, 1.8180, 2.2180) -- cycle;
\fill[blue!33.4, opacity=0.7] (1.0300, 1.8180, 2.2180) -- (1.0770, 1.8180, 2.2210) -- (1.0770, 1.8710, 2.2190) -- (1.0300, 1.8710, 2.2160) -- cycle;
\fill[blue!32.4, opacity=0.7] (1.0300, 1.8710, 2.2160) -- (1.0770, 1.8710, 2.2190) -- (1.0770, 1.9240, 2.2165) -- (1.0300, 1.9240, 2.2135) -- cycle;
\fill[blue!36.6, opacity=0.7] (1.0300, 1.9240, 2.2135) -- (1.0770, 1.9240, 2.2165) -- (1.0770, 1.9770, 2.2138) -- (1.0300, 1.9770, 2.2108) -- cycle;
\fill[blue!50.9, opacity=0.7] (1.0300, 1.9770, 2.2108) -- (1.0770, 1.9770, 2.2138) -- (1.0770, 2.0300, 2.2108) -- (1.0300, 2.0300, 2.2078) -- cycle;
\fill[blue!77.0, opacity=0.7] (1.0300, 2.0300, 2.2078) -- (1.0770, 2.0300, 2.2108) -- (1.0770, 2.0830, 2.2076) -- (1.0300, 2.0830, 2.2046) -- cycle;
\fill[blue!85.7, opacity=0.7] (1.0300, 2.0830, 2.2046) -- (1.0770, 2.0830, 2.2076) -- (1.0770, 2.1360, 2.2040) -- (1.0300, 2.1360, 2.2010) -- cycle;
\fill[blue!54.9, opacity=0.7] (1.0300, 2.1360, 2.2010) -- (1.0770, 2.1360, 2.2040) -- (1.0770, 2.1890, 2.2002) -- (1.0300, 2.1890, 2.1972) -- cycle;
\fill[blue!29.7, opacity=0.7] (1.0300, 2.1890, 2.1972) -- (1.0770, 2.1890, 2.2002) -- (1.0770, 2.2420, 2.1961) -- (1.0300, 2.2420, 2.1931) -- cycle;
\fill[blue!22.7, opacity=0.7] (1.0300, 2.2420, 2.1931) -- (1.0770, 2.2420, 2.1961) -- (1.0770, 2.2950, 2.1918) -- (1.0300, 2.2950, 2.1888) -- cycle;
\fill[blue!25.1, opacity=0.7] (1.0300, 2.2950, 2.1888) -- (1.0770, 2.2950, 2.1918) -- (1.0770, 2.3480, 2.1872) -- (1.0300, 2.3480, 2.1842) -- cycle;
\fill[blue!41.1, opacity=0.7] (1.0300, 2.3480, 2.1842) -- (1.0770, 2.3480, 2.1872) -- (1.0770, 2.4010, 2.1824) -- (1.0300, 2.4010, 2.1794) -- cycle;
\fill[blue!73.4, opacity=0.7] (1.0300, 2.4010, 2.1794) -- (1.0770, 2.4010, 2.1824) -- (1.0770, 2.4540, 2.1775) -- (1.0300, 2.4540, 2.1745) -- cycle;
\fill[blue!87.9, opacity=0.7] (1.0300, 2.4540, 2.1745) -- (1.0770, 2.4540, 2.1775) -- (1.0770, 2.5070, 2.1723) -- (1.0300, 2.5070, 2.1693) -- cycle;
\fill[blue!85.8, opacity=0.7] (1.0300, 2.5070, 2.1693) -- (1.0770, 2.5070, 2.1723) -- (1.0770, 2.5600, 2.1669) -- (1.0300, 2.5600, 2.1639) -- cycle;
\fill[blue!87.5, opacity=0.7] (1.0300, 2.5600, 2.1639) -- (1.0770, 2.5600, 2.1669) -- (1.0770, 2.6130, 2.1614) -- (1.0300, 2.6130, 2.1584) -- cycle;
\fill[blue!60.6, opacity=0.7] (1.0300, 2.6130, 2.1584) -- (1.0770, 2.6130, 2.1614) -- (1.0770, 2.6660, 2.1557) -- (1.0300, 2.6660, 2.1527) -- cycle;
\fill[blue!23.4, opacity=0.7] (1.0300, 2.6660, 2.1527) -- (1.0770, 2.6660, 2.1557) -- (1.0770, 2.7190, 2.1499) -- (1.0300, 2.7190, 2.1469) -- cycle;
\fill[blue!15.7, opacity=0.7] (1.0300, 2.7190, 2.1469) -- (1.0770, 2.7190, 2.1499) -- (1.0770, 2.7720, 2.1440) -- (1.0300, 2.7720, 2.1410) -- cycle;
\fill[blue!15.2, opacity=0.7] (1.0300, 2.7720, 2.1410) -- (1.0770, 2.7720, 2.1440) -- (1.0770, 2.8250, 2.1380) -- (1.0300, 2.8250, 2.1350) -- cycle;
\fill[blue!15.4, opacity=0.7] (1.0300, 2.8250, 2.1350) -- (1.0770, 2.8250, 2.1380) -- (1.0770, 2.8780, 2.1319) -- (1.0300, 2.8780, 2.1289) -- cycle;
\fill[blue!18.0, opacity=0.7] (1.0300, 2.8780, 2.1289) -- (1.0770, 2.8780, 2.1319) -- (1.0770, 2.9310, 2.1257) -- (1.0300, 2.9310, 2.1227) -- cycle;
\fill[blue!33.1, opacity=0.7] (1.0300, 2.9310, 2.1227) -- (1.0770, 2.9310, 2.1257) -- (1.0770, 2.9840, 2.1195) -- (1.0300, 2.9840, 2.1165) -- cycle;
\fill[blue!53.5, opacity=0.7] (1.0300, 2.9840, 2.1165) -- (1.0770, 2.9840, 2.1195) -- (1.0770, 3.0370, 2.1132) -- (1.0300, 3.0370, 2.1102) -- cycle;
\fill[blue!49.2, opacity=0.7] (1.0300, 3.0370, 2.1102) -- (1.0770, 3.0370, 2.1132) -- (1.0770, 3.0900, 2.1069) -- (1.0300, 3.0900, 2.1039) -- cycle;
\fill[blue!26.1, opacity=0.7] (1.0770, -0.0900, 2.1069) -- (1.1240, -0.0900, 2.1096) -- (1.1240, -0.0370, 2.1159) -- (1.0770, -0.0370, 2.1132) -- cycle;
\fill[blue!16.1, opacity=0.7] (1.0770, -0.0370, 2.1132) -- (1.1240, -0.0370, 2.1159) -- (1.1240, 0.0160, 2.1222) -- (1.0770, 0.0160, 2.1195) -- cycle;
\fill[blue!15.1, opacity=0.7] (1.0770, 0.0160, 2.1195) -- (1.1240, 0.0160, 2.1222) -- (1.1240, 0.0690, 2.1284) -- (1.0770, 0.0690, 2.1257) -- cycle;
\fill[blue!15.1, opacity=0.7] (1.0770, 0.0690, 2.1257) -- (1.1240, 0.0690, 2.1284) -- (1.1240, 0.1220, 2.1346) -- (1.0770, 0.1220, 2.1319) -- cycle;
\fill[blue!15.8, opacity=0.7] (1.0770, 0.1220, 2.1319) -- (1.1240, 0.1220, 2.1346) -- (1.1240, 0.1750, 2.1407) -- (1.0770, 0.1750, 2.1380) -- cycle;
\fill[blue!29.3, opacity=0.7] (1.0770, 0.1750, 2.1380) -- (1.1240, 0.1750, 2.1407) -- (1.1240, 0.2280, 2.1467) -- (1.0770, 0.2280, 2.1440) -- cycle;
\fill[blue!72.9, opacity=0.7] (1.0770, 0.2280, 2.1440) -- (1.1240, 0.2280, 2.1467) -- (1.1240, 0.2810, 2.1526) -- (1.0770, 0.2810, 2.1499) -- cycle;
\fill[blue!87.4, opacity=0.7] (1.0770, 0.2810, 2.1499) -- (1.1240, 0.2810, 2.1526) -- (1.1240, 0.3340, 2.1584) -- (1.0770, 0.3340, 2.1557) -- cycle;
\fill[blue!84.3, opacity=0.7] (1.0770, 0.3340, 2.1557) -- (1.1240, 0.3340, 2.1584) -- (1.1240, 0.3870, 2.1641) -- (1.0770, 0.3870, 2.1614) -- cycle;
\fill[blue!54.1, opacity=0.7] (1.0770, 0.3870, 2.1614) -- (1.1240, 0.3870, 2.1641) -- (1.1240, 0.4400, 2.1696) -- (1.0770, 0.4400, 2.1669) -- cycle;
\fill[blue!24.3, opacity=0.7] (1.0770, 0.4400, 2.1669) -- (1.1240, 0.4400, 2.1696) -- (1.1240, 0.4930, 2.1750) -- (1.0770, 0.4930, 2.1723) -- cycle;
\fill[blue!18.0, opacity=0.7] (1.0770, 0.4930, 2.1723) -- (1.1240, 0.4930, 2.1750) -- (1.1240, 0.5460, 2.1802) -- (1.0770, 0.5460, 2.1775) -- cycle;
\fill[blue!20.3, opacity=0.7] (1.0770, 0.5460, 2.1775) -- (1.1240, 0.5460, 2.1802) -- (1.1240, 0.5990, 2.1851) -- (1.0770, 0.5990, 2.1824) -- cycle;
\fill[blue!42.9, opacity=0.7] (1.0770, 0.5990, 2.1824) -- (1.1240, 0.5990, 2.1851) -- (1.1240, 0.6520, 2.1899) -- (1.0770, 0.6520, 2.1872) -- cycle;
\fill[blue!86.5, opacity=0.7] (1.0770, 0.6520, 2.1872) -- (1.1240, 0.6520, 2.1899) -- (1.1240, 0.7050, 2.1945) -- (1.0770, 0.7050, 2.1918) -- cycle;
\fill[blue!69.4, opacity=0.7] (1.0770, 0.7050, 2.1918) -- (1.1240, 0.7050, 2.1945) -- (1.1240, 0.7580, 2.1988) -- (1.0770, 0.7580, 2.1961) -- cycle;
\fill[blue!54.8, opacity=0.7] (1.0770, 0.7580, 2.1961) -- (1.1240, 0.7580, 2.1988) -- (1.1240, 0.8110, 2.2029) -- (1.0770, 0.8110, 2.2002) -- cycle;
\fill[blue!70.9, opacity=0.7] (1.0770, 0.8110, 2.2002) -- (1.1240, 0.8110, 2.2029) -- (1.1240, 0.8640, 2.2067) -- (1.0770, 0.8640, 2.2040) -- cycle;
\fill[blue!87.5, opacity=0.7] (1.0770, 0.8640, 2.2040) -- (1.1240, 0.8640, 2.2067) -- (1.1240, 0.9170, 2.2103) -- (1.0770, 0.9170, 2.2076) -- cycle;
\fill[blue!61.5, opacity=0.7] (1.0770, 0.9170, 2.2076) -- (1.1240, 0.9170, 2.2103) -- (1.1240, 0.9700, 2.2135) -- (1.0770, 0.9700, 2.2108) -- cycle;
\fill[blue!43.1, opacity=0.7] (1.0770, 0.9700, 2.2108) -- (1.1240, 0.9700, 2.2135) -- (1.1240, 1.0230, 2.2165) -- (1.0770, 1.0230, 2.2138) -- cycle;
\fill[blue!50.8, opacity=0.7] (1.0770, 1.0230, 2.2138) -- (1.1240, 1.0230, 2.2165) -- (1.1240, 1.0760, 2.2193) -- (1.0770, 1.0760, 2.2165) -- cycle;
\fill[blue!80.7, opacity=0.7] (1.0770, 1.0760, 2.2165) -- (1.1240, 1.0760, 2.2193) -- (1.1240, 1.1290, 2.2217) -- (1.0770, 1.1290, 2.2190) -- cycle;
\fill[blue!77.9, opacity=0.7] (1.0770, 1.1290, 2.2190) -- (1.1240, 1.1290, 2.2217) -- (1.1240, 1.1820, 2.2238) -- (1.0770, 1.1820, 2.2210) -- cycle;
\fill[blue!40.7, opacity=0.7] (1.0770, 1.1820, 2.2210) -- (1.1240, 1.1820, 2.2238) -- (1.1240, 1.2350, 2.2255) -- (1.0770, 1.2350, 2.2228) -- cycle;
\fill[blue!25.0, opacity=0.7] (1.0770, 1.2350, 2.2228) -- (1.1240, 1.2350, 2.2255) -- (1.1240, 1.2880, 2.2270) -- (1.0770, 1.2880, 2.2243) -- cycle;
\fill[blue!22.7, opacity=0.7] (1.0770, 1.2880, 2.2243) -- (1.1240, 1.2880, 2.2270) -- (1.1240, 1.3410, 2.2281) -- (1.0770, 1.3410, 2.2254) -- cycle;
\fill[blue!27.1, opacity=0.7] (1.0770, 1.3410, 2.2254) -- (1.1240, 1.3410, 2.2281) -- (1.1240, 1.3940, 2.2290) -- (1.0770, 1.3940, 2.2263) -- cycle;
\fill[blue!39.0, opacity=0.7] (1.0770, 1.3940, 2.2263) -- (1.1240, 1.3940, 2.2290) -- (1.1240, 1.4470, 2.2295) -- (1.0770, 1.4470, 2.2268) -- cycle;
\fill[blue!56.5, opacity=0.7] (1.0770, 1.4470, 2.2268) -- (1.1240, 1.4470, 2.2295) -- (1.1240, 1.5000, 2.2296) -- (1.0770, 1.5000, 2.2269) -- cycle;
\fill[blue!71.7, opacity=0.7] (1.0770, 1.5000, 2.2269) -- (1.1240, 1.5000, 2.2296) -- (1.1240, 1.5530, 2.2295) -- (1.0770, 1.5530, 2.2268) -- cycle;
\fill[blue!80.2, opacity=0.7] (1.0770, 1.5530, 2.2268) -- (1.1240, 1.5530, 2.2295) -- (1.1240, 1.6060, 2.2290) -- (1.0770, 1.6060, 2.2263) -- cycle;
\fill[blue!83.2, opacity=0.7] (1.0770, 1.6060, 2.2263) -- (1.1240, 1.6060, 2.2290) -- (1.1240, 1.6590, 2.2281) -- (1.0770, 1.6590, 2.2254) -- cycle;
\fill[blue!83.0, opacity=0.7] (1.0770, 1.6590, 2.2254) -- (1.1240, 1.6590, 2.2281) -- (1.1240, 1.7120, 2.2270) -- (1.0770, 1.7120, 2.2243) -- cycle;
\fill[blue!79.4, opacity=0.7] (1.0770, 1.7120, 2.2243) -- (1.1240, 1.7120, 2.2270) -- (1.1240, 1.7650, 2.2255) -- (1.0770, 1.7650, 2.2228) -- cycle;
\fill[blue!70.8, opacity=0.7] (1.0770, 1.7650, 2.2228) -- (1.1240, 1.7650, 2.2255) -- (1.1240, 1.8180, 2.2238) -- (1.0770, 1.8180, 2.2210) -- cycle;
\fill[blue!57.4, opacity=0.7] (1.0770, 1.8180, 2.2210) -- (1.1240, 1.8180, 2.2238) -- (1.1240, 1.8710, 2.2217) -- (1.0770, 1.8710, 2.2190) -- cycle;
\fill[blue!43.9, opacity=0.7] (1.0770, 1.8710, 2.2190) -- (1.1240, 1.8710, 2.2217) -- (1.1240, 1.9240, 2.2193) -- (1.0770, 1.9240, 2.2165) -- cycle;
\fill[blue!35.6, opacity=0.7] (1.0770, 1.9240, 2.2165) -- (1.1240, 1.9240, 2.2193) -- (1.1240, 1.9770, 2.2165) -- (1.0770, 1.9770, 2.2138) -- cycle;
\fill[blue!35.1, opacity=0.7] (1.0770, 1.9770, 2.2138) -- (1.1240, 1.9770, 2.2165) -- (1.1240, 2.0300, 2.2135) -- (1.0770, 2.0300, 2.2108) -- cycle;
\fill[blue!46.1, opacity=0.7] (1.0770, 2.0300, 2.2108) -- (1.1240, 2.0300, 2.2135) -- (1.1240, 2.0830, 2.2103) -- (1.0770, 2.0830, 2.2076) -- cycle;
\fill[blue!73.3, opacity=0.7] (1.0770, 2.0830, 2.2076) -- (1.1240, 2.0830, 2.2103) -- (1.1240, 2.1360, 2.2067) -- (1.0770, 2.1360, 2.2040) -- cycle;
\fill[blue!86.2, opacity=0.7] (1.0770, 2.1360, 2.2040) -- (1.1240, 2.1360, 2.2067) -- (1.1240, 2.1890, 2.2029) -- (1.0770, 2.1890, 2.2002) -- cycle;
\fill[blue!52.5, opacity=0.7] (1.0770, 2.1890, 2.2002) -- (1.1240, 2.1890, 2.2029) -- (1.1240, 2.2420, 2.1988) -- (1.0770, 2.2420, 2.1961) -- cycle;
\fill[blue!27.5, opacity=0.7] (1.0770, 2.2420, 2.1961) -- (1.1240, 2.2420, 2.1988) -- (1.1240, 2.2950, 2.1945) -- (1.0770, 2.2950, 2.1918) -- cycle;
\fill[blue!22.0, opacity=0.7] (1.0770, 2.2950, 2.1918) -- (1.1240, 2.2950, 2.1945) -- (1.1240, 2.3480, 2.1899) -- (1.0770, 2.3480, 2.1872) -- cycle;
\fill[blue!26.6, opacity=0.7] (1.0770, 2.3480, 2.1872) -- (1.1240, 2.3480, 2.1899) -- (1.1240, 2.4010, 2.1851) -- (1.0770, 2.4010, 2.1824) -- cycle;
\fill[blue!48.5, opacity=0.7] (1.0770, 2.4010, 2.1824) -- (1.1240, 2.4010, 2.1851) -- (1.1240, 2.4540, 2.1802) -- (1.0770, 2.4540, 2.1775) -- cycle;
\fill[blue!81.1, opacity=0.7] (1.0770, 2.4540, 2.1775) -- (1.1240, 2.4540, 2.1802) -- (1.1240, 2.5070, 2.1750) -- (1.0770, 2.5070, 2.1723) -- cycle;
\fill[blue!87.3, opacity=0.7] (1.0770, 2.5070, 2.1723) -- (1.1240, 2.5070, 2.1750) -- (1.1240, 2.5600, 2.1696) -- (1.0770, 2.5600, 2.1669) -- cycle;
\fill[blue!87.0, opacity=0.7] (1.0770, 2.5600, 2.1669) -- (1.1240, 2.5600, 2.1696) -- (1.1240, 2.6130, 2.1641) -- (1.0770, 2.6130, 2.1614) -- cycle;
\fill[blue!81.1, opacity=0.7] (1.0770, 2.6130, 2.1614) -- (1.1240, 2.6130, 2.1641) -- (1.1240, 2.6660, 2.1584) -- (1.0770, 2.6660, 2.1557) -- cycle;
\fill[blue!39.0, opacity=0.7] (1.0770, 2.6660, 2.1557) -- (1.1240, 2.6660, 2.1584) -- (1.1240, 2.7190, 2.1526) -- (1.0770, 2.7190, 2.1499) -- cycle;
\fill[blue!17.4, opacity=0.7] (1.0770, 2.7190, 2.1499) -- (1.1240, 2.7190, 2.1526) -- (1.1240, 2.7720, 2.1467) -- (1.0770, 2.7720, 2.1440) -- cycle;
\fill[blue!15.3, opacity=0.7] (1.0770, 2.7720, 2.1440) -- (1.1240, 2.7720, 2.1467) -- (1.1240, 2.8250, 2.1407) -- (1.0770, 2.8250, 2.1380) -- cycle;
\fill[blue!15.2, opacity=0.7] (1.0770, 2.8250, 2.1380) -- (1.1240, 2.8250, 2.1407) -- (1.1240, 2.8780, 2.1346) -- (1.0770, 2.8780, 2.1319) -- cycle;
\fill[blue!16.0, opacity=0.7] (1.0770, 2.8780, 2.1319) -- (1.1240, 2.8780, 2.1346) -- (1.1240, 2.9310, 2.1284) -- (1.0770, 2.9310, 2.1257) -- cycle;
\fill[blue!24.0, opacity=0.7] (1.0770, 2.9310, 2.1257) -- (1.1240, 2.9310, 2.1284) -- (1.1240, 2.9840, 2.1222) -- (1.0770, 2.9840, 2.1195) -- cycle;
\fill[blue!46.1, opacity=0.7] (1.0770, 2.9840, 2.1195) -- (1.1240, 2.9840, 2.1222) -- (1.1240, 3.0370, 2.1159) -- (1.0770, 3.0370, 2.1132) -- cycle;
\fill[blue!54.1, opacity=0.7] (1.0770, 3.0370, 2.1132) -- (1.1240, 3.0370, 2.1159) -- (1.1240, 3.0900, 2.1096) -- (1.0770, 3.0900, 2.1069) -- cycle;
\fill[blue!23.3, opacity=0.7] (1.1240, -0.0900, 2.1096) -- (1.1710, -0.0900, 2.1120) -- (1.1710, -0.0370, 2.1183) -- (1.1240, -0.0370, 2.1159) -- cycle;
\fill[blue!15.7, opacity=0.7] (1.1240, -0.0370, 2.1159) -- (1.1710, -0.0370, 2.1183) -- (1.1710, 0.0160, 2.1246) -- (1.1240, 0.0160, 2.1222) -- cycle;
\fill[blue!15.1, opacity=0.7] (1.1240, 0.0160, 2.1222) -- (1.1710, 0.0160, 2.1246) -- (1.1710, 0.0690, 2.1308) -- (1.1240, 0.0690, 2.1284) -- cycle;
\fill[blue!15.1, opacity=0.7] (1.1240, 0.0690, 2.1284) -- (1.1710, 0.0690, 2.1308) -- (1.1710, 0.1220, 2.1370) -- (1.1240, 0.1220, 2.1346) -- cycle;
\fill[blue!16.7, opacity=0.7] (1.1240, 0.1220, 2.1346) -- (1.1710, 0.1220, 2.1370) -- (1.1710, 0.1750, 2.1431) -- (1.1240, 0.1750, 2.1407) -- cycle;
\fill[blue!37.9, opacity=0.7] (1.1240, 0.1750, 2.1407) -- (1.1710, 0.1750, 2.1431) -- (1.1710, 0.2280, 2.1491) -- (1.1240, 0.2280, 2.1467) -- cycle;
\fill[blue!80.2, opacity=0.7] (1.1240, 0.2280, 2.1467) -- (1.1710, 0.2280, 2.1491) -- (1.1710, 0.2810, 2.1550) -- (1.1240, 0.2810, 2.1526) -- cycle;
\fill[blue!87.7, opacity=0.7] (1.1240, 0.2810, 2.1526) -- (1.1710, 0.2810, 2.1550) -- (1.1710, 0.3340, 2.1608) -- (1.1240, 0.3340, 2.1584) -- cycle;
\fill[blue!80.7, opacity=0.7] (1.1240, 0.3340, 2.1584) -- (1.1710, 0.3340, 2.1608) -- (1.1710, 0.3870, 2.1665) -- (1.1240, 0.3870, 2.1641) -- cycle;
\fill[blue!44.5, opacity=0.7] (1.1240, 0.3870, 2.1641) -- (1.1710, 0.3870, 2.1665) -- (1.1710, 0.4400, 2.1720) -- (1.1240, 0.4400, 2.1696) -- cycle;
\fill[blue!21.4, opacity=0.7] (1.1240, 0.4400, 2.1696) -- (1.1710, 0.4400, 2.1720) -- (1.1710, 0.4930, 2.1774) -- (1.1240, 0.4930, 2.1750) -- cycle;
\fill[blue!18.1, opacity=0.7] (1.1240, 0.4930, 2.1750) -- (1.1710, 0.4930, 2.1774) -- (1.1710, 0.5460, 2.1826) -- (1.1240, 0.5460, 2.1802) -- cycle;
\fill[blue!23.8, opacity=0.7] (1.1240, 0.5460, 2.1802) -- (1.1710, 0.5460, 2.1826) -- (1.1710, 0.5990, 2.1875) -- (1.1240, 0.5990, 2.1851) -- cycle;
\fill[blue!59.7, opacity=0.7] (1.1240, 0.5990, 2.1851) -- (1.1710, 0.5990, 2.1875) -- (1.1710, 0.6520, 2.1923) -- (1.1240, 0.6520, 2.1899) -- cycle;
\fill[blue!86.3, opacity=0.7] (1.1240, 0.6520, 2.1899) -- (1.1710, 0.6520, 2.1923) -- (1.1710, 0.7050, 2.1969) -- (1.1240, 0.7050, 2.1945) -- cycle;
\fill[blue!59.3, opacity=0.7] (1.1240, 0.7050, 2.1945) -- (1.1710, 0.7050, 2.1969) -- (1.1710, 0.7580, 2.2012) -- (1.1240, 0.7580, 2.1988) -- cycle;
\fill[blue!56.6, opacity=0.7] (1.1240, 0.7580, 2.1988) -- (1.1710, 0.7580, 2.2012) -- (1.1710, 0.8110, 2.2053) -- (1.1240, 0.8110, 2.2029) -- cycle;
\fill[blue!81.1, opacity=0.7] (1.1240, 0.8110, 2.2029) -- (1.1710, 0.8110, 2.2053) -- (1.1710, 0.8640, 2.2091) -- (1.1240, 0.8640, 2.2067) -- cycle;
\fill[blue!79.8, opacity=0.7] (1.1240, 0.8640, 2.2067) -- (1.1710, 0.8640, 2.2091) -- (1.1710, 0.9170, 2.2127) -- (1.1240, 0.9170, 2.2103) -- cycle;
\fill[blue!50.5, opacity=0.7] (1.1240, 0.9170, 2.2103) -- (1.1710, 0.9170, 2.2127) -- (1.1710, 0.9700, 2.2160) -- (1.1240, 0.9700, 2.2135) -- cycle;
\fill[blue!45.5, opacity=0.7] (1.1240, 0.9700, 2.2135) -- (1.1710, 0.9700, 2.2160) -- (1.1710, 1.0230, 2.2190) -- (1.1240, 1.0230, 2.2165) -- cycle;
\fill[blue!69.8, opacity=0.7] (1.1240, 1.0230, 2.2165) -- (1.1710, 1.0230, 2.2190) -- (1.1710, 1.0760, 2.2217) -- (1.1240, 1.0760, 2.2193) -- cycle;
\fill[blue!85.4, opacity=0.7] (1.1240, 1.0760, 2.2193) -- (1.1710, 1.0760, 2.2217) -- (1.1710, 1.1290, 2.2241) -- (1.1240, 1.1290, 2.2217) -- cycle;
\fill[blue!45.8, opacity=0.7] (1.1240, 1.1290, 2.2217) -- (1.1710, 1.1290, 2.2241) -- (1.1710, 1.1820, 2.2262) -- (1.1240, 1.1820, 2.2238) -- cycle;
\fill[blue!24.6, opacity=0.7] (1.1240, 1.1820, 2.2238) -- (1.1710, 1.1820, 2.2262) -- (1.1710, 1.2350, 2.2279) -- (1.1240, 1.2350, 2.2255) -- cycle;
\fill[blue!21.8, opacity=0.7] (1.1240, 1.2350, 2.2255) -- (1.1710, 1.2350, 2.2279) -- (1.1710, 1.2880, 2.2294) -- (1.1240, 1.2880, 2.2270) -- cycle;
\fill[blue!28.1, opacity=0.7] (1.1240, 1.2880, 2.2270) -- (1.1710, 1.2880, 2.2294) -- (1.1710, 1.3410, 2.2306) -- (1.1240, 1.3410, 2.2281) -- cycle;
\fill[blue!48.1, opacity=0.7] (1.1240, 1.3410, 2.2281) -- (1.1710, 1.3410, 2.2306) -- (1.1710, 1.3940, 2.2314) -- (1.1240, 1.3940, 2.2290) -- cycle;
\fill[blue!74.7, opacity=0.7] (1.1240, 1.3940, 2.2290) -- (1.1710, 1.3940, 2.2314) -- (1.1710, 1.4470, 2.2319) -- (1.1240, 1.4470, 2.2295) -- cycle;
\fill[blue!87.2, opacity=0.7] (1.1240, 1.4470, 2.2295) -- (1.1710, 1.4470, 2.2319) -- (1.1710, 1.5000, 2.2320) -- (1.1240, 1.5000, 2.2296) -- cycle;
\fill[blue!86.6, opacity=0.7] (1.1240, 1.5000, 2.2296) -- (1.1710, 1.5000, 2.2320) -- (1.1710, 1.5530, 2.2319) -- (1.1240, 1.5530, 2.2295) -- cycle;
\fill[blue!83.2, opacity=0.7] (1.1240, 1.5530, 2.2295) -- (1.1710, 1.5530, 2.2319) -- (1.1710, 1.6060, 2.2314) -- (1.1240, 1.6060, 2.2290) -- cycle;
\fill[blue!80.8, opacity=0.7] (1.1240, 1.6060, 2.2290) -- (1.1710, 1.6060, 2.2314) -- (1.1710, 1.6590, 2.2306) -- (1.1240, 1.6590, 2.2281) -- cycle;
\fill[blue!80.4, opacity=0.7] (1.1240, 1.6590, 2.2281) -- (1.1710, 1.6590, 2.2306) -- (1.1710, 1.7120, 2.2294) -- (1.1240, 1.7120, 2.2270) -- cycle;
\fill[blue!82.4, opacity=0.7] (1.1240, 1.7120, 2.2270) -- (1.1710, 1.7120, 2.2294) -- (1.1710, 1.7650, 2.2279) -- (1.1240, 1.7650, 2.2255) -- cycle;
\fill[blue!86.3, opacity=0.7] (1.1240, 1.7650, 2.2255) -- (1.1710, 1.7650, 2.2279) -- (1.1710, 1.8180, 2.2262) -- (1.1240, 1.8180, 2.2238) -- cycle;
\fill[blue!87.4, opacity=0.7] (1.1240, 1.8180, 2.2238) -- (1.1710, 1.8180, 2.2262) -- (1.1710, 1.8710, 2.2241) -- (1.1240, 1.8710, 2.2217) -- cycle;
\fill[blue!77.1, opacity=0.7] (1.1240, 1.8710, 2.2217) -- (1.1710, 1.8710, 2.2241) -- (1.1710, 1.9240, 2.2217) -- (1.1240, 1.9240, 2.2193) -- cycle;
\fill[blue!56.6, opacity=0.7] (1.1240, 1.9240, 2.2193) -- (1.1710, 1.9240, 2.2217) -- (1.1710, 1.9770, 2.2190) -- (1.1240, 1.9770, 2.2165) -- cycle;
\fill[blue!40.4, opacity=0.7] (1.1240, 1.9770, 2.2165) -- (1.1710, 1.9770, 2.2190) -- (1.1710, 2.0300, 2.2160) -- (1.1240, 2.0300, 2.2135) -- cycle;
\fill[blue!36.1, opacity=0.7] (1.1240, 2.0300, 2.2135) -- (1.1710, 2.0300, 2.2160) -- (1.1710, 2.0830, 2.2127) -- (1.1240, 2.0830, 2.2103) -- cycle;
\fill[blue!46.3, opacity=0.7] (1.1240, 2.0830, 2.2103) -- (1.1710, 2.0830, 2.2127) -- (1.1710, 2.1360, 2.2091) -- (1.1240, 2.1360, 2.2067) -- cycle;
\fill[blue!75.6, opacity=0.7] (1.1240, 2.1360, 2.2067) -- (1.1710, 2.1360, 2.2091) -- (1.1710, 2.1890, 2.2053) -- (1.1240, 2.1890, 2.2029) -- cycle;
\fill[blue!83.1, opacity=0.7] (1.1240, 2.1890, 2.2029) -- (1.1710, 2.1890, 2.2053) -- (1.1710, 2.2420, 2.2012) -- (1.1240, 2.2420, 2.1988) -- cycle;
\fill[blue!44.4, opacity=0.7] (1.1240, 2.2420, 2.1988) -- (1.1710, 2.2420, 2.2012) -- (1.1710, 2.2950, 2.1969) -- (1.1240, 2.2950, 2.1945) -- cycle;
\fill[blue!24.3, opacity=0.7] (1.1240, 2.2950, 2.1945) -- (1.1710, 2.2950, 2.1969) -- (1.1710, 2.3480, 2.1923) -- (1.1240, 2.3480, 2.1899) -- cycle;
\fill[blue!21.9, opacity=0.7] (1.1240, 2.3480, 2.1899) -- (1.1710, 2.3480, 2.1923) -- (1.1710, 2.4010, 2.1875) -- (1.1240, 2.4010, 2.1851) -- cycle;
\fill[blue!30.9, opacity=0.7] (1.1240, 2.4010, 2.1851) -- (1.1710, 2.4010, 2.1875) -- (1.1710, 2.4540, 2.1826) -- (1.1240, 2.4540, 2.1802) -- cycle;
\fill[blue!61.3, opacity=0.7] (1.1240, 2.4540, 2.1802) -- (1.1710, 2.4540, 2.1826) -- (1.1710, 2.5070, 2.1774) -- (1.1240, 2.5070, 2.1750) -- cycle;
\fill[blue!86.8, opacity=0.7] (1.1240, 2.5070, 2.1750) -- (1.1710, 2.5070, 2.1774) -- (1.1710, 2.5600, 2.1720) -- (1.1240, 2.5600, 2.1696) -- cycle;
\fill[blue!86.5, opacity=0.7] (1.1240, 2.5600, 2.1696) -- (1.1710, 2.5600, 2.1720) -- (1.1710, 2.6130, 2.1665) -- (1.1240, 2.6130, 2.1641) -- cycle;
\fill[blue!87.6, opacity=0.7] (1.1240, 2.6130, 2.1641) -- (1.1710, 2.6130, 2.1665) -- (1.1710, 2.6660, 2.1608) -- (1.1240, 2.6660, 2.1584) -- cycle;
\fill[blue!61.0, opacity=0.7] (1.1240, 2.6660, 2.1584) -- (1.1710, 2.6660, 2.1608) -- (1.1710, 2.7190, 2.1550) -- (1.1240, 2.7190, 2.1526) -- cycle;
\fill[blue!22.7, opacity=0.7] (1.1240, 2.7190, 2.1526) -- (1.1710, 2.7190, 2.1550) -- (1.1710, 2.7720, 2.1491) -- (1.1240, 2.7720, 2.1467) -- cycle;
\fill[blue!15.6, opacity=0.7] (1.1240, 2.7720, 2.1467) -- (1.1710, 2.7720, 2.1491) -- (1.1710, 2.8250, 2.1431) -- (1.1240, 2.8250, 2.1407) -- cycle;
\fill[blue!15.2, opacity=0.7] (1.1240, 2.8250, 2.1407) -- (1.1710, 2.8250, 2.1431) -- (1.1710, 2.8780, 2.1370) -- (1.1240, 2.8780, 2.1346) -- cycle;
\fill[blue!15.4, opacity=0.7] (1.1240, 2.8780, 2.1346) -- (1.1710, 2.8780, 2.1370) -- (1.1710, 2.9310, 2.1308) -- (1.1240, 2.9310, 2.1284) -- cycle;
\fill[blue!18.9, opacity=0.7] (1.1240, 2.9310, 2.1284) -- (1.1710, 2.9310, 2.1308) -- (1.1710, 2.9840, 2.1246) -- (1.1240, 2.9840, 2.1222) -- cycle;
\fill[blue!36.7, opacity=0.7] (1.1240, 2.9840, 2.1222) -- (1.1710, 2.9840, 2.1246) -- (1.1710, 3.0370, 2.1183) -- (1.1240, 3.0370, 2.1159) -- cycle;
\fill[blue!54.0, opacity=0.7] (1.1240, 3.0370, 2.1159) -- (1.1710, 3.0370, 2.1183) -- (1.1710, 3.0900, 2.1120) -- (1.1240, 3.0900, 2.1096) -- cycle;
\fill[blue!21.3, opacity=0.7] (1.1710, -0.0900, 2.1120) -- (1.2180, -0.0900, 2.1141) -- (1.2180, -0.0370, 2.1204) -- (1.1710, -0.0370, 2.1183) -- cycle;
\fill[blue!15.5, opacity=0.7] (1.1710, -0.0370, 2.1183) -- (1.2180, -0.0370, 2.1204) -- (1.2180, 0.0160, 2.1267) -- (1.1710, 0.0160, 2.1246) -- cycle;
\fill[blue!15.1, opacity=0.7] (1.1710, 0.0160, 2.1246) -- (1.2180, 0.0160, 2.1267) -- (1.2180, 0.0690, 2.1329) -- (1.1710, 0.0690, 2.1308) -- cycle;
\fill[blue!15.2, opacity=0.7] (1.1710, 0.0690, 2.1308) -- (1.2180, 0.0690, 2.1329) -- (1.2180, 0.1220, 2.1391) -- (1.1710, 0.1220, 2.1370) -- cycle;
\fill[blue!17.9, opacity=0.7] (1.1710, 0.1220, 2.1370) -- (1.2180, 0.1220, 2.1391) -- (1.2180, 0.1750, 2.1452) -- (1.1710, 0.1750, 2.1431) -- cycle;
\fill[blue!46.7, opacity=0.7] (1.1710, 0.1750, 2.1431) -- (1.2180, 0.1750, 2.1452) -- (1.2180, 0.2280, 2.1512) -- (1.1710, 0.2280, 2.1491) -- cycle;
\fill[blue!84.2, opacity=0.7] (1.1710, 0.2280, 2.1491) -- (1.2180, 0.2280, 2.1512) -- (1.2180, 0.2810, 2.1571) -- (1.1710, 0.2810, 2.1550) -- cycle;
\fill[blue!87.7, opacity=0.7] (1.1710, 0.2810, 2.1550) -- (1.2180, 0.2810, 2.1571) -- (1.2180, 0.3340, 2.1629) -- (1.1710, 0.3340, 2.1608) -- cycle;
\fill[blue!76.2, opacity=0.7] (1.1710, 0.3340, 2.1608) -- (1.2180, 0.3340, 2.1629) -- (1.2180, 0.3870, 2.1686) -- (1.1710, 0.3870, 2.1665) -- cycle;
\fill[blue!37.7, opacity=0.7] (1.1710, 0.3870, 2.1665) -- (1.2180, 0.3870, 2.1686) -- (1.2180, 0.4400, 2.1741) -- (1.1710, 0.4400, 2.1720) -- cycle;
\fill[blue!20.0, opacity=0.7] (1.1710, 0.4400, 2.1720) -- (1.2180, 0.4400, 2.1741) -- (1.2180, 0.4930, 2.1795) -- (1.1710, 0.4930, 2.1774) -- cycle;
\fill[blue!18.6, opacity=0.7] (1.1710, 0.4930, 2.1774) -- (1.2180, 0.4930, 2.1795) -- (1.2180, 0.5460, 2.1847) -- (1.1710, 0.5460, 2.1826) -- cycle;
\fill[blue!29.1, opacity=0.7] (1.1710, 0.5460, 2.1826) -- (1.2180, 0.5460, 2.1847) -- (1.2180, 0.5990, 2.1896) -- (1.1710, 0.5990, 2.1875) -- cycle;
\fill[blue!74.3, opacity=0.7] (1.1710, 0.5990, 2.1875) -- (1.2180, 0.5990, 2.1896) -- (1.2180, 0.6520, 2.1944) -- (1.1710, 0.6520, 2.1923) -- cycle;
\fill[blue!78.9, opacity=0.7] (1.1710, 0.6520, 2.1923) -- (1.2180, 0.6520, 2.1944) -- (1.2180, 0.7050, 2.1990) -- (1.1710, 0.7050, 2.1969) -- cycle;
\fill[blue!53.8, opacity=0.7] (1.1710, 0.7050, 2.1969) -- (1.2180, 0.7050, 2.1990) -- (1.2180, 0.7580, 2.2033) -- (1.1710, 0.7580, 2.2012) -- cycle;
\fill[blue!61.3, opacity=0.7] (1.1710, 0.7580, 2.2012) -- (1.2180, 0.7580, 2.2033) -- (1.2180, 0.8110, 2.2074) -- (1.1710, 0.8110, 2.2053) -- cycle;
\fill[blue!86.9, opacity=0.7] (1.1710, 0.8110, 2.2053) -- (1.2180, 0.8110, 2.2074) -- (1.2180, 0.8640, 2.2112) -- (1.1710, 0.8640, 2.2091) -- cycle;
\fill[blue!69.2, opacity=0.7] (1.1710, 0.8640, 2.2091) -- (1.2180, 0.8640, 2.2112) -- (1.2180, 0.9170, 2.2148) -- (1.1710, 0.9170, 2.2127) -- cycle;
\fill[blue!46.3, opacity=0.7] (1.1710, 0.9170, 2.2127) -- (1.2180, 0.9170, 2.2148) -- (1.2180, 0.9700, 2.2180) -- (1.1710, 0.9700, 2.2160) -- cycle;
\fill[blue!54.5, opacity=0.7] (1.1710, 0.9700, 2.2160) -- (1.2180, 0.9700, 2.2180) -- (1.2180, 1.0230, 2.2210) -- (1.1710, 1.0230, 2.2190) -- cycle;
\fill[blue!85.8, opacity=0.7] (1.1710, 1.0230, 2.2190) -- (1.2180, 1.0230, 2.2210) -- (1.2180, 1.0760, 2.2238) -- (1.1710, 1.0760, 2.2217) -- cycle;
\fill[blue!62.8, opacity=0.7] (1.1710, 1.0760, 2.2217) -- (1.2180, 1.0760, 2.2238) -- (1.2180, 1.1290, 2.2262) -- (1.1710, 1.1290, 2.2241) -- cycle;
\fill[blue!27.7, opacity=0.7] (1.1710, 1.1290, 2.2241) -- (1.2180, 1.1290, 2.2262) -- (1.2180, 1.1820, 2.2283) -- (1.1710, 1.1820, 2.2262) -- cycle;
\fill[blue!20.9, opacity=0.7] (1.1710, 1.1820, 2.2262) -- (1.2180, 1.1820, 2.2283) -- (1.2180, 1.2350, 2.2300) -- (1.1710, 1.2350, 2.2279) -- cycle;
\fill[blue!25.5, opacity=0.7] (1.1710, 1.2350, 2.2279) -- (1.2180, 1.2350, 2.2300) -- (1.2180, 1.2880, 2.2315) -- (1.1710, 1.2880, 2.2294) -- cycle;
\fill[blue!47.8, opacity=0.7] (1.1710, 1.2880, 2.2294) -- (1.2180, 1.2880, 2.2315) -- (1.2180, 1.3410, 2.2326) -- (1.1710, 1.3410, 2.2306) -- cycle;
\fill[blue!79.9, opacity=0.7] (1.1710, 1.3410, 2.2306) -- (1.2180, 1.3410, 2.2326) -- (1.2180, 1.3940, 2.2335) -- (1.1710, 1.3940, 2.2314) -- cycle;
\fill[blue!87.5, opacity=0.7] (1.1710, 1.3940, 2.2314) -- (1.2180, 1.3940, 2.2335) -- (1.2180, 1.4470, 2.2340) -- (1.1710, 1.4470, 2.2319) -- cycle;
\fill[blue!82.2, opacity=0.7] (1.1710, 1.4470, 2.2319) -- (1.2180, 1.4470, 2.2340) -- (1.2180, 1.5000, 2.2341) -- (1.1710, 1.5000, 2.2320) -- cycle;
\fill[blue!79.9, opacity=0.7] (1.1710, 1.5000, 2.2320) -- (1.2180, 1.5000, 2.2341) -- (1.2180, 1.5530, 2.2340) -- (1.1710, 1.5530, 2.2319) -- cycle;
\fill[blue!79.8, opacity=0.7] (1.1710, 1.5530, 2.2319) -- (1.2180, 1.5530, 2.2340) -- (1.2180, 1.6060, 2.2335) -- (1.1710, 1.6060, 2.2314) -- cycle;
\fill[blue!78.9, opacity=0.7] (1.1710, 1.6060, 2.2314) -- (1.2180, 1.6060, 2.2335) -- (1.2180, 1.6590, 2.2326) -- (1.1710, 1.6590, 2.2306) -- cycle;
\fill[blue!76.1, opacity=0.7] (1.1710, 1.6590, 2.2306) -- (1.2180, 1.6590, 2.2326) -- (1.2180, 1.7120, 2.2315) -- (1.1710, 1.7120, 2.2294) -- cycle;
\fill[blue!72.3, opacity=0.7] (1.1710, 1.7120, 2.2294) -- (1.2180, 1.7120, 2.2315) -- (1.2180, 1.7650, 2.2300) -- (1.1710, 1.7650, 2.2279) -- cycle;
\fill[blue!70.8, opacity=0.7] (1.1710, 1.7650, 2.2279) -- (1.2180, 1.7650, 2.2300) -- (1.2180, 1.8180, 2.2283) -- (1.1710, 1.8180, 2.2262) -- cycle;
\fill[blue!75.4, opacity=0.7] (1.1710, 1.8180, 2.2262) -- (1.2180, 1.8180, 2.2283) -- (1.2180, 1.8710, 2.2262) -- (1.1710, 1.8710, 2.2241) -- cycle;
\fill[blue!85.1, opacity=0.7] (1.1710, 1.8710, 2.2241) -- (1.2180, 1.8710, 2.2262) -- (1.2180, 1.9240, 2.2238) -- (1.1710, 1.9240, 2.2217) -- cycle;
\fill[blue!85.5, opacity=0.7] (1.1710, 1.9240, 2.2217) -- (1.2180, 1.9240, 2.2238) -- (1.2180, 1.9770, 2.2210) -- (1.1710, 1.9770, 2.2190) -- cycle;
\fill[blue!64.6, opacity=0.7] (1.1710, 1.9770, 2.2190) -- (1.2180, 1.9770, 2.2210) -- (1.2180, 2.0300, 2.2180) -- (1.1710, 2.0300, 2.2160) -- cycle;
\fill[blue!43.2, opacity=0.7] (1.1710, 2.0300, 2.2160) -- (1.2180, 2.0300, 2.2180) -- (1.2180, 2.0830, 2.2148) -- (1.1710, 2.0830, 2.2127) -- cycle;
\fill[blue!37.6, opacity=0.7] (1.1710, 2.0830, 2.2127) -- (1.2180, 2.0830, 2.2148) -- (1.2180, 2.1360, 2.2112) -- (1.1710, 2.1360, 2.2091) -- cycle;
\fill[blue!50.8, opacity=0.7] (1.1710, 2.1360, 2.2091) -- (1.2180, 2.1360, 2.2112) -- (1.2180, 2.1890, 2.2074) -- (1.1710, 2.1890, 2.2053) -- cycle;
\fill[blue!82.7, opacity=0.7] (1.1710, 2.1890, 2.2053) -- (1.2180, 2.1890, 2.2074) -- (1.2180, 2.2420, 2.2033) -- (1.1710, 2.2420, 2.2012) -- cycle;
\fill[blue!72.8, opacity=0.7] (1.1710, 2.2420, 2.2012) -- (1.2180, 2.2420, 2.2033) -- (1.2180, 2.2950, 2.1990) -- (1.1710, 2.2950, 2.1969) -- cycle;
\fill[blue!33.7, opacity=0.7] (1.1710, 2.2950, 2.1969) -- (1.2180, 2.2950, 2.1990) -- (1.2180, 2.3480, 2.1944) -- (1.1710, 2.3480, 2.1923) -- cycle;
\fill[blue!21.8, opacity=0.7] (1.1710, 2.3480, 2.1923) -- (1.2180, 2.3480, 2.1944) -- (1.2180, 2.4010, 2.1896) -- (1.1710, 2.4010, 2.1875) -- cycle;
\fill[blue!23.4, opacity=0.7] (1.1710, 2.4010, 2.1875) -- (1.2180, 2.4010, 2.1896) -- (1.2180, 2.4540, 2.1847) -- (1.1710, 2.4540, 2.1826) -- cycle;
\fill[blue!41.0, opacity=0.7] (1.1710, 2.4540, 2.1826) -- (1.2180, 2.4540, 2.1847) -- (1.2180, 2.5070, 2.1795) -- (1.1710, 2.5070, 2.1774) -- cycle;
\fill[blue!76.9, opacity=0.7] (1.1710, 2.5070, 2.1774) -- (1.2180, 2.5070, 2.1795) -- (1.2180, 2.5600, 2.1741) -- (1.1710, 2.5600, 2.1720) -- cycle;
\fill[blue!87.7, opacity=0.7] (1.1710, 2.5600, 2.1720) -- (1.2180, 2.5600, 2.1741) -- (1.2180, 2.6130, 2.1686) -- (1.1710, 2.6130, 2.1665) -- cycle;
\fill[blue!87.4, opacity=0.7] (1.1710, 2.6130, 2.1665) -- (1.2180, 2.6130, 2.1686) -- (1.2180, 2.6660, 2.1629) -- (1.1710, 2.6660, 2.1608) -- cycle;
\fill[blue!78.4, opacity=0.7] (1.1710, 2.6660, 2.1608) -- (1.2180, 2.6660, 2.1629) -- (1.2180, 2.7190, 2.1571) -- (1.1710, 2.7190, 2.1550) -- cycle;
\fill[blue!34.0, opacity=0.7] (1.1710, 2.7190, 2.1550) -- (1.2180, 2.7190, 2.1571) -- (1.2180, 2.7720, 2.1512) -- (1.1710, 2.7720, 2.1491) -- cycle;
\fill[blue!16.6, opacity=0.7] (1.1710, 2.7720, 2.1491) -- (1.2180, 2.7720, 2.1512) -- (1.2180, 2.8250, 2.1452) -- (1.1710, 2.8250, 2.1431) -- cycle;
\fill[blue!15.2, opacity=0.7] (1.1710, 2.8250, 2.1431) -- (1.2180, 2.8250, 2.1452) -- (1.2180, 2.8780, 2.1391) -- (1.1710, 2.8780, 2.1370) -- cycle;
\fill[blue!15.2, opacity=0.7] (1.1710, 2.8780, 2.1370) -- (1.2180, 2.8780, 2.1391) -- (1.2180, 2.9310, 2.1329) -- (1.1710, 2.9310, 2.1308) -- cycle;
\fill[blue!16.6, opacity=0.7] (1.1710, 2.9310, 2.1308) -- (1.2180, 2.9310, 2.1329) -- (1.2180, 2.9840, 2.1267) -- (1.1710, 2.9840, 2.1246) -- cycle;
\fill[blue!28.3, opacity=0.7] (1.1710, 2.9840, 2.1246) -- (1.2180, 2.9840, 2.1267) -- (1.2180, 3.0370, 2.1204) -- (1.1710, 3.0370, 2.1183) -- cycle;
\fill[blue!50.1, opacity=0.7] (1.1710, 3.0370, 2.1183) -- (1.2180, 3.0370, 2.1204) -- (1.2180, 3.0900, 2.1141) -- (1.1710, 3.0900, 2.1120) -- cycle;
\fill[blue!20.1, opacity=0.7] (1.2180, -0.0900, 2.1141) -- (1.2650, -0.0900, 2.1159) -- (1.2650, -0.0370, 2.1222) -- (1.2180, -0.0370, 2.1204) -- cycle;
\fill[blue!15.4, opacity=0.7] (1.2180, -0.0370, 2.1204) -- (1.2650, -0.0370, 2.1222) -- (1.2650, 0.0160, 2.1285) -- (1.2180, 0.0160, 2.1267) -- cycle;
\fill[blue!15.1, opacity=0.7] (1.2180, 0.0160, 2.1267) -- (1.2650, 0.0160, 2.1285) -- (1.2650, 0.0690, 2.1347) -- (1.2180, 0.0690, 2.1329) -- cycle;
\fill[blue!15.2, opacity=0.7] (1.2180, 0.0690, 2.1329) -- (1.2650, 0.0690, 2.1347) -- (1.2650, 0.1220, 2.1409) -- (1.2180, 0.1220, 2.1391) -- cycle;
\fill[blue!19.4, opacity=0.7] (1.2180, 0.1220, 2.1391) -- (1.2650, 0.1220, 2.1409) -- (1.2650, 0.1750, 2.1470) -- (1.2180, 0.1750, 2.1452) -- cycle;
\fill[blue!54.2, opacity=0.7] (1.2180, 0.1750, 2.1452) -- (1.2650, 0.1750, 2.1470) -- (1.2650, 0.2280, 2.1530) -- (1.2180, 0.2280, 2.1512) -- cycle;
\fill[blue!86.1, opacity=0.7] (1.2180, 0.2280, 2.1512) -- (1.2650, 0.2280, 2.1530) -- (1.2650, 0.2810, 2.1589) -- (1.2180, 0.2810, 2.1571) -- cycle;
\fill[blue!87.6, opacity=0.7] (1.2180, 0.2810, 2.1571) -- (1.2650, 0.2810, 2.1589) -- (1.2650, 0.3340, 2.1647) -- (1.2180, 0.3340, 2.1629) -- cycle;
\fill[blue!71.9, opacity=0.7] (1.2180, 0.3340, 2.1629) -- (1.2650, 0.3340, 2.1647) -- (1.2650, 0.3870, 2.1704) -- (1.2180, 0.3870, 2.1686) -- cycle;
\fill[blue!33.3, opacity=0.7] (1.2180, 0.3870, 2.1686) -- (1.2650, 0.3870, 2.1704) -- (1.2650, 0.4400, 2.1759) -- (1.2180, 0.4400, 2.1741) -- cycle;
\fill[blue!19.4, opacity=0.7] (1.2180, 0.4400, 2.1741) -- (1.2650, 0.4400, 2.1759) -- (1.2650, 0.4930, 2.1813) -- (1.2180, 0.4930, 2.1795) -- cycle;
\fill[blue!19.3, opacity=0.7] (1.2180, 0.4930, 2.1795) -- (1.2650, 0.4930, 2.1813) -- (1.2650, 0.5460, 2.1864) -- (1.2180, 0.5460, 2.1847) -- cycle;
\fill[blue!35.5, opacity=0.7] (1.2180, 0.5460, 2.1847) -- (1.2650, 0.5460, 2.1864) -- (1.2650, 0.5990, 2.1914) -- (1.2180, 0.5990, 2.1896) -- cycle;
\fill[blue!83.2, opacity=0.7] (1.2180, 0.5990, 2.1896) -- (1.2650, 0.5990, 2.1914) -- (1.2650, 0.6520, 2.1962) -- (1.2180, 0.6520, 2.1944) -- cycle;
\fill[blue!70.9, opacity=0.7] (1.2180, 0.6520, 2.1944) -- (1.2650, 0.6520, 2.1962) -- (1.2650, 0.7050, 2.2008) -- (1.2180, 0.7050, 2.1990) -- cycle;
\fill[blue!51.3, opacity=0.7] (1.2180, 0.7050, 2.1990) -- (1.2650, 0.7050, 2.2008) -- (1.2650, 0.7580, 2.2051) -- (1.2180, 0.7580, 2.2033) -- cycle;
\fill[blue!66.8, opacity=0.7] (1.2180, 0.7580, 2.2033) -- (1.2650, 0.7580, 2.2051) -- (1.2650, 0.8110, 2.2092) -- (1.2180, 0.8110, 2.2074) -- cycle;
\fill[blue!87.7, opacity=0.7] (1.2180, 0.8110, 2.2074) -- (1.2650, 0.8110, 2.2092) -- (1.2650, 0.8640, 2.2130) -- (1.2180, 0.8640, 2.2112) -- cycle;
\fill[blue!61.4, opacity=0.7] (1.2180, 0.8640, 2.2112) -- (1.2650, 0.8640, 2.2130) -- (1.2650, 0.9170, 2.2166) -- (1.2180, 0.9170, 2.2148) -- cycle;
\fill[blue!46.8, opacity=0.7] (1.2180, 0.9170, 2.2148) -- (1.2650, 0.9170, 2.2166) -- (1.2650, 0.9700, 2.2198) -- (1.2180, 0.9700, 2.2180) -- cycle;
\fill[blue!66.6, opacity=0.7] (1.2180, 0.9700, 2.2180) -- (1.2650, 0.9700, 2.2198) -- (1.2650, 1.0230, 2.2228) -- (1.2180, 1.0230, 2.2210) -- cycle;
\fill[blue!85.7, opacity=0.7] (1.2180, 1.0230, 2.2210) -- (1.2650, 1.0230, 2.2228) -- (1.2650, 1.0760, 2.2255) -- (1.2180, 1.0760, 2.2238) -- cycle;
\fill[blue!41.1, opacity=0.7] (1.2180, 1.0760, 2.2238) -- (1.2650, 1.0760, 2.2255) -- (1.2650, 1.1290, 2.2279) -- (1.2180, 1.1290, 2.2262) -- cycle;
\fill[blue!21.6, opacity=0.7] (1.2180, 1.1290, 2.2262) -- (1.2650, 1.1290, 2.2279) -- (1.2650, 1.1820, 2.2300) -- (1.2180, 1.1820, 2.2283) -- cycle;
\fill[blue!21.4, opacity=0.7] (1.2180, 1.1820, 2.2283) -- (1.2650, 1.1820, 2.2300) -- (1.2650, 1.2350, 2.2318) -- (1.2180, 1.2350, 2.2300) -- cycle;
\fill[blue!36.6, opacity=0.7] (1.2180, 1.2350, 2.2300) -- (1.2650, 1.2350, 2.2318) -- (1.2650, 1.2880, 2.2333) -- (1.2180, 1.2880, 2.2315) -- cycle;
\fill[blue!74.3, opacity=0.7] (1.2180, 1.2880, 2.2315) -- (1.2650, 1.2880, 2.2333) -- (1.2650, 1.3410, 2.2344) -- (1.2180, 1.3410, 2.2326) -- cycle;
\fill[blue!87.6, opacity=0.7] (1.2180, 1.3410, 2.2326) -- (1.2650, 1.3410, 2.2344) -- (1.2650, 1.3940, 2.2353) -- (1.2180, 1.3940, 2.2335) -- cycle;
\fill[blue!83.5, opacity=0.7] (1.2180, 1.3940, 2.2335) -- (1.2650, 1.3940, 2.2353) -- (1.2650, 1.4470, 2.2357) -- (1.2180, 1.4470, 2.2340) -- cycle;
\fill[blue!86.1, opacity=0.7] (1.2180, 1.4470, 2.2340) -- (1.2650, 1.4470, 2.2357) -- (1.2650, 1.5000, 2.2359) -- (1.2180, 1.5000, 2.2341) -- cycle;
\fill[blue!87.6, opacity=0.7] (1.2180, 1.5000, 2.2341) -- (1.2650, 1.5000, 2.2359) -- (1.2650, 1.5530, 2.2357) -- (1.2180, 1.5530, 2.2340) -- cycle;
\fill[blue!84.7, opacity=0.7] (1.2180, 1.5530, 2.2340) -- (1.2650, 1.5530, 2.2357) -- (1.2650, 1.6060, 2.2353) -- (1.2180, 1.6060, 2.2335) -- cycle;
\fill[blue!84.0, opacity=0.7] (1.2180, 1.6060, 2.2335) -- (1.2650, 1.6060, 2.2353) -- (1.2650, 1.6590, 2.2344) -- (1.2180, 1.6590, 2.2326) -- cycle;
\fill[blue!86.9, opacity=0.7] (1.2180, 1.6590, 2.2326) -- (1.2650, 1.6590, 2.2344) -- (1.2650, 1.7120, 2.2333) -- (1.2180, 1.7120, 2.2315) -- cycle;
\fill[blue!86.9, opacity=0.7] (1.2180, 1.7120, 2.2315) -- (1.2650, 1.7120, 2.2333) -- (1.2650, 1.7650, 2.2318) -- (1.2180, 1.7650, 2.2300) -- cycle;
\fill[blue!78.5, opacity=0.7] (1.2180, 1.7650, 2.2300) -- (1.2650, 1.7650, 2.2318) -- (1.2650, 1.8180, 2.2300) -- (1.2180, 1.8180, 2.2283) -- cycle;
\fill[blue!68.7, opacity=0.7] (1.2180, 1.8180, 2.2283) -- (1.2650, 1.8180, 2.2300) -- (1.2650, 1.8710, 2.2279) -- (1.2180, 1.8710, 2.2262) -- cycle;
\fill[blue!68.2, opacity=0.7] (1.2180, 1.8710, 2.2262) -- (1.2650, 1.8710, 2.2279) -- (1.2650, 1.9240, 2.2255) -- (1.2180, 1.9240, 2.2238) -- cycle;
\fill[blue!80.2, opacity=0.7] (1.2180, 1.9240, 2.2238) -- (1.2650, 1.9240, 2.2255) -- (1.2650, 1.9770, 2.2228) -- (1.2180, 1.9770, 2.2210) -- cycle;
\fill[blue!87.1, opacity=0.7] (1.2180, 1.9770, 2.2210) -- (1.2650, 1.9770, 2.2228) -- (1.2650, 2.0300, 2.2198) -- (1.2180, 2.0300, 2.2180) -- cycle;
\fill[blue!65.5, opacity=0.7] (1.2180, 2.0300, 2.2180) -- (1.2650, 2.0300, 2.2198) -- (1.2650, 2.0830, 2.2166) -- (1.2180, 2.0830, 2.2148) -- cycle;
\fill[blue!42.9, opacity=0.7] (1.2180, 2.0830, 2.2148) -- (1.2650, 2.0830, 2.2166) -- (1.2650, 2.1360, 2.2130) -- (1.2180, 2.1360, 2.2112) -- cycle;
\fill[blue!40.0, opacity=0.7] (1.2180, 2.1360, 2.2112) -- (1.2650, 2.1360, 2.2130) -- (1.2650, 2.1890, 2.2092) -- (1.2180, 2.1890, 2.2074) -- cycle;
\fill[blue!61.0, opacity=0.7] (1.2180, 2.1890, 2.2074) -- (1.2650, 2.1890, 2.2092) -- (1.2650, 2.2420, 2.2051) -- (1.2180, 2.2420, 2.2033) -- cycle;
\fill[blue!87.9, opacity=0.7] (1.2180, 2.2420, 2.2033) -- (1.2650, 2.2420, 2.2051) -- (1.2650, 2.2950, 2.2008) -- (1.2180, 2.2950, 2.1990) -- cycle;
\fill[blue!53.4, opacity=0.7] (1.2180, 2.2950, 2.1990) -- (1.2650, 2.2950, 2.2008) -- (1.2650, 2.3480, 2.1962) -- (1.2180, 2.3480, 2.1944) -- cycle;
\fill[blue!25.3, opacity=0.7] (1.2180, 2.3480, 2.1944) -- (1.2650, 2.3480, 2.1962) -- (1.2650, 2.4010, 2.1914) -- (1.2180, 2.4010, 2.1896) -- cycle;
\fill[blue!21.0, opacity=0.7] (1.2180, 2.4010, 2.1896) -- (1.2650, 2.4010, 2.1914) -- (1.2650, 2.4540, 2.1864) -- (1.2180, 2.4540, 2.1847) -- cycle;
\fill[blue!28.7, opacity=0.7] (1.2180, 2.4540, 2.1847) -- (1.2650, 2.4540, 2.1864) -- (1.2650, 2.5070, 2.1813) -- (1.2180, 2.5070, 2.1795) -- cycle;
\fill[blue!59.5, opacity=0.7] (1.2180, 2.5070, 2.1795) -- (1.2650, 2.5070, 2.1813) -- (1.2650, 2.5600, 2.1759) -- (1.2180, 2.5600, 2.1741) -- cycle;
\fill[blue!86.7, opacity=0.7] (1.2180, 2.5600, 2.1741) -- (1.2650, 2.5600, 2.1759) -- (1.2650, 2.6130, 2.1704) -- (1.2180, 2.6130, 2.1686) -- cycle;
\fill[blue!86.9, opacity=0.7] (1.2180, 2.6130, 2.1686) -- (1.2650, 2.6130, 2.1704) -- (1.2650, 2.6660, 2.1647) -- (1.2180, 2.6660, 2.1629) -- cycle;
\fill[blue!86.2, opacity=0.7] (1.2180, 2.6660, 2.1629) -- (1.2650, 2.6660, 2.1647) -- (1.2650, 2.7190, 2.1589) -- (1.2180, 2.7190, 2.1571) -- cycle;
\fill[blue!50.2, opacity=0.7] (1.2180, 2.7190, 2.1571) -- (1.2650, 2.7190, 2.1589) -- (1.2650, 2.7720, 2.1530) -- (1.2180, 2.7720, 2.1512) -- cycle;
\fill[blue!19.0, opacity=0.7] (1.2180, 2.7720, 2.1512) -- (1.2650, 2.7720, 2.1530) -- (1.2650, 2.8250, 2.1470) -- (1.2180, 2.8250, 2.1452) -- cycle;
\fill[blue!15.3, opacity=0.7] (1.2180, 2.8250, 2.1452) -- (1.2650, 2.8250, 2.1470) -- (1.2650, 2.8780, 2.1409) -- (1.2180, 2.8780, 2.1391) -- cycle;
\fill[blue!15.1, opacity=0.7] (1.2180, 2.8780, 2.1391) -- (1.2650, 2.8780, 2.1409) -- (1.2650, 2.9310, 2.1347) -- (1.2180, 2.9310, 2.1329) -- cycle;
\fill[blue!15.7, opacity=0.7] (1.2180, 2.9310, 2.1329) -- (1.2650, 2.9310, 2.1347) -- (1.2650, 2.9840, 2.1285) -- (1.2180, 2.9840, 2.1267) -- cycle;
\fill[blue!22.4, opacity=0.7] (1.2180, 2.9840, 2.1267) -- (1.2650, 2.9840, 2.1285) -- (1.2650, 3.0370, 2.1222) -- (1.2180, 3.0370, 2.1204) -- cycle;
\fill[blue!44.1, opacity=0.7] (1.2180, 3.0370, 2.1204) -- (1.2650, 3.0370, 2.1222) -- (1.2650, 3.0900, 2.1159) -- (1.2180, 3.0900, 2.1141) -- cycle;
\fill[blue!19.4, opacity=0.7] (1.2650, -0.0900, 2.1159) -- (1.3120, -0.0900, 2.1174) -- (1.3120, -0.0370, 2.1237) -- (1.2650, -0.0370, 2.1222) -- cycle;
\fill[blue!15.3, opacity=0.7] (1.2650, -0.0370, 2.1222) -- (1.3120, -0.0370, 2.1237) -- (1.3120, 0.0160, 2.1299) -- (1.2650, 0.0160, 2.1285) -- cycle;
\fill[blue!15.1, opacity=0.7] (1.2650, 0.0160, 2.1285) -- (1.3120, 0.0160, 2.1299) -- (1.3120, 0.0690, 2.1361) -- (1.2650, 0.0690, 2.1347) -- cycle;
\fill[blue!15.3, opacity=0.7] (1.2650, 0.0690, 2.1347) -- (1.3120, 0.0690, 2.1361) -- (1.3120, 0.1220, 2.1423) -- (1.2650, 0.1220, 2.1409) -- cycle;
\fill[blue!21.0, opacity=0.7] (1.2650, 0.1220, 2.1409) -- (1.3120, 0.1220, 2.1423) -- (1.3120, 0.1750, 2.1484) -- (1.2650, 0.1750, 2.1470) -- cycle;
\fill[blue!59.7, opacity=0.7] (1.2650, 0.1750, 2.1470) -- (1.3120, 0.1750, 2.1484) -- (1.3120, 0.2280, 2.1545) -- (1.2650, 0.2280, 2.1530) -- cycle;
\fill[blue!87.0, opacity=0.7] (1.2650, 0.2280, 2.1530) -- (1.3120, 0.2280, 2.1545) -- (1.3120, 0.2810, 2.1604) -- (1.2650, 0.2810, 2.1589) -- cycle;
\fill[blue!87.5, opacity=0.7] (1.2650, 0.2810, 2.1589) -- (1.3120, 0.2810, 2.1604) -- (1.3120, 0.3340, 2.1662) -- (1.2650, 0.3340, 2.1647) -- cycle;
\fill[blue!68.6, opacity=0.7] (1.2650, 0.3340, 2.1647) -- (1.3120, 0.3340, 2.1662) -- (1.3120, 0.3870, 2.1719) -- (1.2650, 0.3870, 2.1704) -- cycle;
\fill[blue!30.8, opacity=0.7] (1.2650, 0.3870, 2.1704) -- (1.3120, 0.3870, 2.1719) -- (1.3120, 0.4400, 2.1774) -- (1.2650, 0.4400, 2.1759) -- cycle;
\fill[blue!19.1, opacity=0.7] (1.2650, 0.4400, 2.1759) -- (1.3120, 0.4400, 2.1774) -- (1.3120, 0.4930, 2.1827) -- (1.2650, 0.4930, 2.1813) -- cycle;
\fill[blue!20.2, opacity=0.7] (1.2650, 0.4930, 2.1813) -- (1.3120, 0.4930, 2.1827) -- (1.3120, 0.5460, 2.1879) -- (1.2650, 0.5460, 2.1864) -- cycle;
\fill[blue!41.6, opacity=0.7] (1.2650, 0.5460, 2.1864) -- (1.3120, 0.5460, 2.1879) -- (1.3120, 0.5990, 2.1929) -- (1.2650, 0.5990, 2.1914) -- cycle;
\fill[blue!86.9, opacity=0.7] (1.2650, 0.5990, 2.1914) -- (1.3120, 0.5990, 2.1929) -- (1.3120, 0.6520, 2.1977) -- (1.2650, 0.6520, 2.1962) -- cycle;
\fill[blue!64.9, opacity=0.7] (1.2650, 0.6520, 2.1962) -- (1.3120, 0.6520, 2.1977) -- (1.3120, 0.7050, 2.2022) -- (1.2650, 0.7050, 2.2008) -- cycle;
\fill[blue!50.4, opacity=0.7] (1.2650, 0.7050, 2.2008) -- (1.3120, 0.7050, 2.2022) -- (1.3120, 0.7580, 2.2066) -- (1.2650, 0.7580, 2.2051) -- cycle;
\fill[blue!71.0, opacity=0.7] (1.2650, 0.7580, 2.2051) -- (1.3120, 0.7580, 2.2066) -- (1.3120, 0.8110, 2.2106) -- (1.2650, 0.8110, 2.2092) -- cycle;
\fill[blue!86.1, opacity=0.7] (1.2650, 0.8110, 2.2092) -- (1.3120, 0.8110, 2.2106) -- (1.3120, 0.8640, 2.2145) -- (1.2650, 0.8640, 2.2130) -- cycle;
\fill[blue!57.2, opacity=0.7] (1.2650, 0.8640, 2.2130) -- (1.3120, 0.8640, 2.2145) -- (1.3120, 0.9170, 2.2180) -- (1.2650, 0.9170, 2.2166) -- cycle;
\fill[blue!49.7, opacity=0.7] (1.2650, 0.9170, 2.2166) -- (1.3120, 0.9170, 2.2180) -- (1.3120, 0.9700, 2.2213) -- (1.2650, 0.9700, 2.2198) -- cycle;
\fill[blue!77.0, opacity=0.7] (1.2650, 0.9700, 2.2198) -- (1.3120, 0.9700, 2.2213) -- (1.3120, 1.0230, 2.2243) -- (1.2650, 1.0230, 2.2228) -- cycle;
\fill[blue!74.8, opacity=0.7] (1.2650, 1.0230, 2.2228) -- (1.3120, 1.0230, 2.2243) -- (1.3120, 1.0760, 2.2270) -- (1.2650, 1.0760, 2.2255) -- cycle;
\fill[blue!29.8, opacity=0.7] (1.2650, 1.0760, 2.2255) -- (1.3120, 1.0760, 2.2270) -- (1.3120, 1.1290, 2.2294) -- (1.2650, 1.1290, 2.2279) -- cycle;
\fill[blue!19.7, opacity=0.7] (1.2650, 1.1290, 2.2279) -- (1.3120, 1.1290, 2.2294) -- (1.3120, 1.1820, 2.2315) -- (1.2650, 1.1820, 2.2300) -- cycle;
\fill[blue!23.9, opacity=0.7] (1.2650, 1.1820, 2.2300) -- (1.3120, 1.1820, 2.2315) -- (1.3120, 1.2350, 2.2333) -- (1.2650, 1.2350, 2.2318) -- cycle;
\fill[blue!52.7, opacity=0.7] (1.2650, 1.2350, 2.2318) -- (1.3120, 1.2350, 2.2333) -- (1.3120, 1.2880, 2.2348) -- (1.2650, 1.2880, 2.2333) -- cycle;
\fill[blue!86.5, opacity=0.7] (1.2650, 1.2880, 2.2333) -- (1.3120, 1.2880, 2.2348) -- (1.3120, 1.3410, 2.2359) -- (1.2650, 1.3410, 2.2344) -- cycle;
\fill[blue!85.5, opacity=0.7] (1.2650, 1.3410, 2.2344) -- (1.3120, 1.3410, 2.2359) -- (1.3120, 1.3940, 2.2367) -- (1.2650, 1.3940, 2.2353) -- cycle;
\fill[blue!87.8, opacity=0.7] (1.2650, 1.3940, 2.2353) -- (1.3120, 1.3940, 2.2367) -- (1.3120, 1.4470, 2.2372) -- (1.2650, 1.4470, 2.2357) -- cycle;
\fill[blue!74.3, opacity=0.7] (1.2650, 1.4470, 2.2357) -- (1.3120, 1.4470, 2.2372) -- (1.3120, 1.5000, 2.2374) -- (1.2650, 1.5000, 2.2359) -- cycle;
\fill[blue!46.9, opacity=0.7] (1.2650, 1.5000, 2.2359) -- (1.3120, 1.5000, 2.2374) -- (1.3120, 1.5530, 2.2372) -- (1.2650, 1.5530, 2.2357) -- cycle;
\fill[blue!33.9, opacity=0.7] (1.2650, 1.5530, 2.2357) -- (1.3120, 1.5530, 2.2372) -- (1.3120, 1.6060, 2.2367) -- (1.2650, 1.6060, 2.2353) -- cycle;
\fill[blue!32.7, opacity=0.7] (1.2650, 1.6060, 2.2353) -- (1.3120, 1.6060, 2.2367) -- (1.3120, 1.6590, 2.2359) -- (1.2650, 1.6590, 2.2344) -- cycle;
\fill[blue!40.8, opacity=0.7] (1.2650, 1.6590, 2.2344) -- (1.3120, 1.6590, 2.2359) -- (1.3120, 1.7120, 2.2348) -- (1.2650, 1.7120, 2.2333) -- cycle;
\fill[blue!60.8, opacity=0.7] (1.2650, 1.7120, 2.2333) -- (1.3120, 1.7120, 2.2348) -- (1.3120, 1.7650, 2.2333) -- (1.2650, 1.7650, 2.2318) -- cycle;
\fill[blue!84.0, opacity=0.7] (1.2650, 1.7650, 2.2318) -- (1.3120, 1.7650, 2.2333) -- (1.3120, 1.8180, 2.2315) -- (1.2650, 1.8180, 2.2300) -- cycle;
\fill[blue!84.2, opacity=0.7] (1.2650, 1.8180, 2.2300) -- (1.3120, 1.8180, 2.2315) -- (1.3120, 1.8710, 2.2294) -- (1.2650, 1.8710, 2.2279) -- cycle;
\fill[blue!68.5, opacity=0.7] (1.2650, 1.8710, 2.2279) -- (1.3120, 1.8710, 2.2294) -- (1.3120, 1.9240, 2.2270) -- (1.2650, 1.9240, 2.2255) -- cycle;
\fill[blue!64.9, opacity=0.7] (1.2650, 1.9240, 2.2255) -- (1.3120, 1.9240, 2.2270) -- (1.3120, 1.9770, 2.2243) -- (1.2650, 1.9770, 2.2228) -- cycle;
\fill[blue!79.4, opacity=0.7] (1.2650, 1.9770, 2.2228) -- (1.3120, 1.9770, 2.2243) -- (1.3120, 2.0300, 2.2213) -- (1.2650, 2.0300, 2.2198) -- cycle;
\fill[blue!86.1, opacity=0.7] (1.2650, 2.0300, 2.2198) -- (1.3120, 2.0300, 2.2213) -- (1.3120, 2.0830, 2.2180) -- (1.2650, 2.0830, 2.2166) -- cycle;
\fill[blue!59.9, opacity=0.7] (1.2650, 2.0830, 2.2166) -- (1.3120, 2.0830, 2.2180) -- (1.3120, 2.1360, 2.2145) -- (1.2650, 2.1360, 2.2130) -- cycle;
\fill[blue!41.0, opacity=0.7] (1.2650, 2.1360, 2.2130) -- (1.3120, 2.1360, 2.2145) -- (1.3120, 2.1890, 2.2106) -- (1.2650, 2.1890, 2.2092) -- cycle;
\fill[blue!46.2, opacity=0.7] (1.2650, 2.1890, 2.2092) -- (1.3120, 2.1890, 2.2106) -- (1.3120, 2.2420, 2.2066) -- (1.2650, 2.2420, 2.2051) -- cycle;
\fill[blue!77.7, opacity=0.7] (1.2650, 2.2420, 2.2051) -- (1.3120, 2.2420, 2.2066) -- (1.3120, 2.2950, 2.2022) -- (1.2650, 2.2950, 2.2008) -- cycle;
\fill[blue!76.5, opacity=0.7] (1.2650, 2.2950, 2.2008) -- (1.3120, 2.2950, 2.2022) -- (1.3120, 2.3480, 2.1977) -- (1.2650, 2.3480, 2.1962) -- cycle;
\fill[blue!33.7, opacity=0.7] (1.2650, 2.3480, 2.1962) -- (1.3120, 2.3480, 2.1977) -- (1.3120, 2.4010, 2.1929) -- (1.2650, 2.4010, 2.1914) -- cycle;
\fill[blue!21.1, opacity=0.7] (1.2650, 2.4010, 2.1914) -- (1.3120, 2.4010, 2.1929) -- (1.3120, 2.4540, 2.1879) -- (1.2650, 2.4540, 2.1864) -- cycle;
\fill[blue!23.1, opacity=0.7] (1.2650, 2.4540, 2.1864) -- (1.3120, 2.4540, 2.1879) -- (1.3120, 2.5070, 2.1827) -- (1.2650, 2.5070, 2.1813) -- cycle;
\fill[blue!43.1, opacity=0.7] (1.2650, 2.5070, 2.1813) -- (1.3120, 2.5070, 2.1827) -- (1.3120, 2.5600, 2.1774) -- (1.2650, 2.5600, 2.1759) -- cycle;
\fill[blue!79.9, opacity=0.7] (1.2650, 2.5600, 2.1759) -- (1.3120, 2.5600, 2.1774) -- (1.3120, 2.6130, 2.1719) -- (1.2650, 2.6130, 2.1704) -- cycle;
\fill[blue!87.5, opacity=0.7] (1.2650, 2.6130, 2.1704) -- (1.3120, 2.6130, 2.1719) -- (1.3120, 2.6660, 2.1662) -- (1.2650, 2.6660, 2.1647) -- cycle;
\fill[blue!87.8, opacity=0.7] (1.2650, 2.6660, 2.1647) -- (1.3120, 2.6660, 2.1662) -- (1.3120, 2.7190, 2.1604) -- (1.2650, 2.7190, 2.1589) -- cycle;
\fill[blue!66.0, opacity=0.7] (1.2650, 2.7190, 2.1589) -- (1.3120, 2.7190, 2.1604) -- (1.3120, 2.7720, 2.1545) -- (1.2650, 2.7720, 2.1530) -- cycle;
\fill[blue!23.9, opacity=0.7] (1.2650, 2.7720, 2.1530) -- (1.3120, 2.7720, 2.1545) -- (1.3120, 2.8250, 2.1484) -- (1.2650, 2.8250, 2.1470) -- cycle;
\fill[blue!15.6, opacity=0.7] (1.2650, 2.8250, 2.1470) -- (1.3120, 2.8250, 2.1484) -- (1.3120, 2.8780, 2.1423) -- (1.2650, 2.8780, 2.1409) -- cycle;
\fill[blue!15.1, opacity=0.7] (1.2650, 2.8780, 2.1409) -- (1.3120, 2.8780, 2.1423) -- (1.3120, 2.9310, 2.1361) -- (1.2650, 2.9310, 2.1347) -- cycle;
\fill[blue!15.4, opacity=0.7] (1.2650, 2.9310, 2.1347) -- (1.3120, 2.9310, 2.1361) -- (1.3120, 2.9840, 2.1299) -- (1.2650, 2.9840, 2.1285) -- cycle;
\fill[blue!19.0, opacity=0.7] (1.2650, 2.9840, 2.1285) -- (1.3120, 2.9840, 2.1299) -- (1.3120, 3.0370, 2.1237) -- (1.2650, 3.0370, 2.1222) -- cycle;
\fill[blue!37.5, opacity=0.7] (1.2650, 3.0370, 2.1222) -- (1.3120, 3.0370, 2.1237) -- (1.3120, 3.0900, 2.1174) -- (1.2650, 3.0900, 2.1159) -- cycle;
\fill[blue!19.1, opacity=0.7] (1.3120, -0.0900, 2.1174) -- (1.3590, -0.0900, 2.1185) -- (1.3590, -0.0370, 2.1248) -- (1.3120, -0.0370, 2.1237) -- cycle;
\fill[blue!15.3, opacity=0.7] (1.3120, -0.0370, 2.1237) -- (1.3590, -0.0370, 2.1248) -- (1.3590, 0.0160, 2.1311) -- (1.3120, 0.0160, 2.1299) -- cycle;
\fill[blue!15.1, opacity=0.7] (1.3120, 0.0160, 2.1299) -- (1.3590, 0.0160, 2.1311) -- (1.3590, 0.0690, 2.1373) -- (1.3120, 0.0690, 2.1361) -- cycle;
\fill[blue!15.4, opacity=0.7] (1.3120, 0.0690, 2.1361) -- (1.3590, 0.0690, 2.1373) -- (1.3590, 0.1220, 2.1435) -- (1.3120, 0.1220, 2.1423) -- cycle;
\fill[blue!22.2, opacity=0.7] (1.3120, 0.1220, 2.1423) -- (1.3590, 0.1220, 2.1435) -- (1.3590, 0.1750, 2.1496) -- (1.3120, 0.1750, 2.1484) -- cycle;
\fill[blue!63.2, opacity=0.7] (1.3120, 0.1750, 2.1484) -- (1.3590, 0.1750, 2.1496) -- (1.3590, 0.2280, 2.1556) -- (1.3120, 0.2280, 2.1545) -- cycle;
\fill[blue!87.4, opacity=0.7] (1.3120, 0.2280, 2.1545) -- (1.3590, 0.2280, 2.1556) -- (1.3590, 0.2810, 2.1615) -- (1.3120, 0.2810, 2.1604) -- cycle;
\fill[blue!87.4, opacity=0.7] (1.3120, 0.2810, 2.1604) -- (1.3590, 0.2810, 2.1615) -- (1.3590, 0.3340, 2.1673) -- (1.3120, 0.3340, 2.1662) -- cycle;
\fill[blue!66.9, opacity=0.7] (1.3120, 0.3340, 2.1662) -- (1.3590, 0.3340, 2.1673) -- (1.3590, 0.3870, 2.1730) -- (1.3120, 0.3870, 2.1719) -- cycle;
\fill[blue!29.7, opacity=0.7] (1.3120, 0.3870, 2.1719) -- (1.3590, 0.3870, 2.1730) -- (1.3590, 0.4400, 2.1785) -- (1.3120, 0.4400, 2.1774) -- cycle;
\fill[blue!19.1, opacity=0.7] (1.3120, 0.4400, 2.1774) -- (1.3590, 0.4400, 2.1785) -- (1.3590, 0.4930, 2.1839) -- (1.3120, 0.4930, 2.1827) -- cycle;
\fill[blue!21.0, opacity=0.7] (1.3120, 0.4930, 2.1827) -- (1.3590, 0.4930, 2.1839) -- (1.3590, 0.5460, 2.1891) -- (1.3120, 0.5460, 2.1879) -- cycle;
\fill[blue!46.0, opacity=0.7] (1.3120, 0.5460, 2.1879) -- (1.3590, 0.5460, 2.1891) -- (1.3590, 0.5990, 2.1940) -- (1.3120, 0.5990, 2.1929) -- cycle;
\fill[blue!87.8, opacity=0.7] (1.3120, 0.5990, 2.1929) -- (1.3590, 0.5990, 2.1940) -- (1.3590, 0.6520, 2.1988) -- (1.3120, 0.6520, 2.1977) -- cycle;
\fill[blue!61.0, opacity=0.7] (1.3120, 0.6520, 2.1977) -- (1.3590, 0.6520, 2.1988) -- (1.3590, 0.7050, 2.2034) -- (1.3120, 0.7050, 2.2022) -- cycle;
\fill[blue!49.8, opacity=0.7] (1.3120, 0.7050, 2.2022) -- (1.3590, 0.7050, 2.2034) -- (1.3590, 0.7580, 2.2077) -- (1.3120, 0.7580, 2.2066) -- cycle;
\fill[blue!73.2, opacity=0.7] (1.3120, 0.7580, 2.2066) -- (1.3590, 0.7580, 2.2077) -- (1.3590, 0.8110, 2.2118) -- (1.3120, 0.8110, 2.2106) -- cycle;
\fill[blue!84.7, opacity=0.7] (1.3120, 0.8110, 2.2106) -- (1.3590, 0.8110, 2.2118) -- (1.3590, 0.8640, 2.2156) -- (1.3120, 0.8640, 2.2145) -- cycle;
\fill[blue!55.8, opacity=0.7] (1.3120, 0.8640, 2.2145) -- (1.3590, 0.8640, 2.2156) -- (1.3590, 0.9170, 2.2192) -- (1.3120, 0.9170, 2.2180) -- cycle;
\fill[blue!52.9, opacity=0.7] (1.3120, 0.9170, 2.2180) -- (1.3590, 0.9170, 2.2192) -- (1.3590, 0.9700, 2.2224) -- (1.3120, 0.9700, 2.2213) -- cycle;
\fill[blue!82.9, opacity=0.7] (1.3120, 0.9700, 2.2213) -- (1.3590, 0.9700, 2.2224) -- (1.3590, 1.0230, 2.2254) -- (1.3120, 1.0230, 2.2243) -- cycle;
\fill[blue!64.3, opacity=0.7] (1.3120, 1.0230, 2.2243) -- (1.3590, 1.0230, 2.2254) -- (1.3590, 1.0760, 2.2281) -- (1.3120, 1.0760, 2.2270) -- cycle;
\fill[blue!24.9, opacity=0.7] (1.3120, 1.0760, 2.2270) -- (1.3590, 1.0760, 2.2281) -- (1.3590, 1.1290, 2.2306) -- (1.3120, 1.1290, 2.2294) -- cycle;
\fill[blue!19.1, opacity=0.7] (1.3120, 1.1290, 2.2294) -- (1.3590, 1.1290, 2.2306) -- (1.3590, 1.1820, 2.2326) -- (1.3120, 1.1820, 2.2315) -- cycle;
\fill[blue!26.8, opacity=0.7] (1.3120, 1.1820, 2.2315) -- (1.3590, 1.1820, 2.2326) -- (1.3590, 1.2350, 2.2344) -- (1.3120, 1.2350, 2.2333) -- cycle;
\fill[blue!64.8, opacity=0.7] (1.3120, 1.2350, 2.2333) -- (1.3590, 1.2350, 2.2344) -- (1.3590, 1.2880, 2.2359) -- (1.3120, 1.2880, 2.2348) -- cycle;
\fill[blue!87.9, opacity=0.7] (1.3120, 1.2880, 2.2348) -- (1.3590, 1.2880, 2.2359) -- (1.3590, 1.3410, 2.2370) -- (1.3120, 1.3410, 2.2359) -- cycle;
\fill[blue!87.5, opacity=0.7] (1.3120, 1.3410, 2.2359) -- (1.3590, 1.3410, 2.2370) -- (1.3590, 1.3940, 2.2379) -- (1.3120, 1.3940, 2.2367) -- cycle;
\fill[blue!72.6, opacity=0.7] (1.3120, 1.3940, 2.2367) -- (1.3590, 1.3940, 2.2379) -- (1.3590, 1.4470, 2.2384) -- (1.3120, 1.4470, 2.2372) -- cycle;
\fill[blue!29.6, opacity=0.7] (1.3120, 1.4470, 2.2372) -- (1.3590, 1.4470, 2.2384) -- (1.3590, 1.5000, 2.2385) -- (1.3120, 1.5000, 2.2374) -- cycle;
\fill[blue!17.9, opacity=0.7] (1.3120, 1.5000, 2.2374) -- (1.3590, 1.5000, 2.2385) -- (1.3590, 1.5530, 2.2384) -- (1.3120, 1.5530, 2.2372) -- cycle;
\fill[blue!16.5, opacity=0.7] (1.3120, 1.5530, 2.2372) -- (1.3590, 1.5530, 2.2384) -- (1.3590, 1.6060, 2.2379) -- (1.3120, 1.6060, 2.2367) -- cycle;
\fill[blue!16.6, opacity=0.7] (1.3120, 1.6060, 2.2367) -- (1.3590, 1.6060, 2.2379) -- (1.3590, 1.6590, 2.2370) -- (1.3120, 1.6590, 2.2359) -- cycle;
\fill[blue!17.9, opacity=0.7] (1.3120, 1.6590, 2.2359) -- (1.3590, 1.6590, 2.2370) -- (1.3590, 1.7120, 2.2359) -- (1.3120, 1.7120, 2.2348) -- cycle;
\fill[blue!22.8, opacity=0.7] (1.3120, 1.7120, 2.2348) -- (1.3590, 1.7120, 2.2359) -- (1.3590, 1.7650, 2.2344) -- (1.3120, 1.7650, 2.2333) -- cycle;
\fill[blue!41.5, opacity=0.7] (1.3120, 1.7650, 2.2333) -- (1.3590, 1.7650, 2.2344) -- (1.3590, 1.8180, 2.2326) -- (1.3120, 1.8180, 2.2315) -- cycle;
\fill[blue!78.0, opacity=0.7] (1.3120, 1.8180, 2.2315) -- (1.3590, 1.8180, 2.2326) -- (1.3590, 1.8710, 2.2306) -- (1.3120, 1.8710, 2.2294) -- cycle;
\fill[blue!84.3, opacity=0.7] (1.3120, 1.8710, 2.2294) -- (1.3590, 1.8710, 2.2306) -- (1.3590, 1.9240, 2.2281) -- (1.3120, 1.9240, 2.2270) -- cycle;
\fill[blue!65.2, opacity=0.7] (1.3120, 1.9240, 2.2270) -- (1.3590, 1.9240, 2.2281) -- (1.3590, 1.9770, 2.2254) -- (1.3120, 1.9770, 2.2243) -- cycle;
\fill[blue!64.4, opacity=0.7] (1.3120, 1.9770, 2.2243) -- (1.3590, 1.9770, 2.2254) -- (1.3590, 2.0300, 2.2224) -- (1.3120, 2.0300, 2.2213) -- cycle;
\fill[blue!83.4, opacity=0.7] (1.3120, 2.0300, 2.2213) -- (1.3590, 2.0300, 2.2224) -- (1.3590, 2.0830, 2.2192) -- (1.3120, 2.0830, 2.2180) -- cycle;
\fill[blue!80.2, opacity=0.7] (1.3120, 2.0830, 2.2180) -- (1.3590, 2.0830, 2.2192) -- (1.3590, 2.1360, 2.2156) -- (1.3120, 2.1360, 2.2145) -- cycle;
\fill[blue!50.5, opacity=0.7] (1.3120, 2.1360, 2.2145) -- (1.3590, 2.1360, 2.2156) -- (1.3590, 2.1890, 2.2118) -- (1.3120, 2.1890, 2.2106) -- cycle;
\fill[blue!41.5, opacity=0.7] (1.3120, 2.1890, 2.2106) -- (1.3590, 2.1890, 2.2118) -- (1.3590, 2.2420, 2.2077) -- (1.3120, 2.2420, 2.2066) -- cycle;
\fill[blue!61.3, opacity=0.7] (1.3120, 2.2420, 2.2066) -- (1.3590, 2.2420, 2.2077) -- (1.3590, 2.2950, 2.2034) -- (1.3120, 2.2950, 2.2022) -- cycle;
\fill[blue!87.6, opacity=0.7] (1.3120, 2.2950, 2.2022) -- (1.3590, 2.2950, 2.2034) -- (1.3590, 2.3480, 2.1988) -- (1.3120, 2.3480, 2.1977) -- cycle;
\fill[blue!47.8, opacity=0.7] (1.3120, 2.3480, 2.1977) -- (1.3590, 2.3480, 2.1988) -- (1.3590, 2.4010, 2.1940) -- (1.3120, 2.4010, 2.1929) -- cycle;
\fill[blue!23.0, opacity=0.7] (1.3120, 2.4010, 2.1929) -- (1.3590, 2.4010, 2.1940) -- (1.3590, 2.4540, 2.1891) -- (1.3120, 2.4540, 2.1879) -- cycle;
\fill[blue!20.8, opacity=0.7] (1.3120, 2.4540, 2.1879) -- (1.3590, 2.4540, 2.1891) -- (1.3590, 2.5070, 2.1839) -- (1.3120, 2.5070, 2.1827) -- cycle;
\fill[blue!32.2, opacity=0.7] (1.3120, 2.5070, 2.1827) -- (1.3590, 2.5070, 2.1839) -- (1.3590, 2.5600, 2.1785) -- (1.3120, 2.5600, 2.1774) -- cycle;
\fill[blue!68.4, opacity=0.7] (1.3120, 2.5600, 2.1774) -- (1.3590, 2.5600, 2.1785) -- (1.3590, 2.6130, 2.1730) -- (1.3120, 2.6130, 2.1719) -- cycle;
\fill[blue!87.8, opacity=0.7] (1.3120, 2.6130, 2.1719) -- (1.3590, 2.6130, 2.1730) -- (1.3590, 2.6660, 2.1673) -- (1.3120, 2.6660, 2.1662) -- cycle;
\fill[blue!87.6, opacity=0.7] (1.3120, 2.6660, 2.1662) -- (1.3590, 2.6660, 2.1673) -- (1.3590, 2.7190, 2.1615) -- (1.3120, 2.7190, 2.1604) -- cycle;
\fill[blue!77.1, opacity=0.7] (1.3120, 2.7190, 2.1604) -- (1.3590, 2.7190, 2.1615) -- (1.3590, 2.7720, 2.1556) -- (1.3120, 2.7720, 2.1545) -- cycle;
\fill[blue!31.5, opacity=0.7] (1.3120, 2.7720, 2.1545) -- (1.3590, 2.7720, 2.1556) -- (1.3590, 2.8250, 2.1496) -- (1.3120, 2.8250, 2.1484) -- cycle;
\fill[blue!16.1, opacity=0.7] (1.3120, 2.8250, 2.1484) -- (1.3590, 2.8250, 2.1496) -- (1.3590, 2.8780, 2.1435) -- (1.3120, 2.8780, 2.1423) -- cycle;
\fill[blue!15.1, opacity=0.7] (1.3120, 2.8780, 2.1423) -- (1.3590, 2.8780, 2.1435) -- (1.3590, 2.9310, 2.1373) -- (1.3120, 2.9310, 2.1361) -- cycle;
\fill[blue!15.2, opacity=0.7] (1.3120, 2.9310, 2.1361) -- (1.3590, 2.9310, 2.1373) -- (1.3590, 2.9840, 2.1311) -- (1.3120, 2.9840, 2.1299) -- cycle;
\fill[blue!17.2, opacity=0.7] (1.3120, 2.9840, 2.1299) -- (1.3590, 2.9840, 2.1311) -- (1.3590, 3.0370, 2.1248) -- (1.3120, 3.0370, 2.1237) -- cycle;
\fill[blue!31.6, opacity=0.7] (1.3120, 3.0370, 2.1237) -- (1.3590, 3.0370, 2.1248) -- (1.3590, 3.0900, 2.1185) -- (1.3120, 3.0900, 2.1174) -- cycle;
\fill[blue!19.1, opacity=0.7] (1.3590, -0.0900, 2.1185) -- (1.4060, -0.0900, 2.1193) -- (1.4060, -0.0370, 2.1256) -- (1.3590, -0.0370, 2.1248) -- cycle;
\fill[blue!15.3, opacity=0.7] (1.3590, -0.0370, 2.1248) -- (1.4060, -0.0370, 2.1256) -- (1.4060, 0.0160, 2.1319) -- (1.3590, 0.0160, 2.1311) -- cycle;
\fill[blue!15.1, opacity=0.7] (1.3590, 0.0160, 2.1311) -- (1.4060, 0.0160, 2.1319) -- (1.4060, 0.0690, 2.1381) -- (1.3590, 0.0690, 2.1373) -- cycle;
\fill[blue!15.4, opacity=0.7] (1.3590, 0.0690, 2.1373) -- (1.4060, 0.0690, 2.1381) -- (1.4060, 0.1220, 2.1443) -- (1.3590, 0.1220, 2.1435) -- cycle;
\fill[blue!22.7, opacity=0.7] (1.3590, 0.1220, 2.1435) -- (1.4060, 0.1220, 2.1443) -- (1.4060, 0.1750, 2.1504) -- (1.3590, 0.1750, 2.1496) -- cycle;
\fill[blue!64.6, opacity=0.7] (1.3590, 0.1750, 2.1496) -- (1.4060, 0.1750, 2.1504) -- (1.4060, 0.2280, 2.1564) -- (1.3590, 0.2280, 2.1556) -- cycle;
\fill[blue!87.6, opacity=0.7] (1.3590, 0.2280, 2.1556) -- (1.4060, 0.2280, 2.1564) -- (1.4060, 0.2810, 2.1623) -- (1.3590, 0.2810, 2.1615) -- cycle;
\fill[blue!87.5, opacity=0.7] (1.3590, 0.2810, 2.1615) -- (1.4060, 0.2810, 2.1623) -- (1.4060, 0.3340, 2.1682) -- (1.3590, 0.3340, 2.1673) -- cycle;
\fill[blue!67.1, opacity=0.7] (1.3590, 0.3340, 2.1673) -- (1.4060, 0.3340, 2.1682) -- (1.4060, 0.3870, 2.1738) -- (1.3590, 0.3870, 2.1730) -- cycle;
\fill[blue!29.8, opacity=0.7] (1.3590, 0.3870, 2.1730) -- (1.4060, 0.3870, 2.1738) -- (1.4060, 0.4400, 2.1793) -- (1.3590, 0.4400, 2.1785) -- cycle;
\fill[blue!19.3, opacity=0.7] (1.3590, 0.4400, 2.1785) -- (1.4060, 0.4400, 2.1793) -- (1.4060, 0.4930, 2.1847) -- (1.3590, 0.4930, 2.1839) -- cycle;
\fill[blue!21.5, opacity=0.7] (1.3590, 0.4930, 2.1839) -- (1.4060, 0.4930, 2.1847) -- (1.4060, 0.5460, 2.1899) -- (1.3590, 0.5460, 2.1891) -- cycle;
\fill[blue!48.0, opacity=0.7] (1.3590, 0.5460, 2.1891) -- (1.4060, 0.5460, 2.1899) -- (1.4060, 0.5990, 2.1949) -- (1.3590, 0.5990, 2.1940) -- cycle;
\fill[blue!87.9, opacity=0.7] (1.3590, 0.5990, 2.1940) -- (1.4060, 0.5990, 2.1949) -- (1.4060, 0.6520, 2.1996) -- (1.3590, 0.6520, 2.1988) -- cycle;
\fill[blue!59.2, opacity=0.7] (1.3590, 0.6520, 2.1988) -- (1.4060, 0.6520, 2.1996) -- (1.4060, 0.7050, 2.2042) -- (1.3590, 0.7050, 2.2034) -- cycle;
\fill[blue!48.9, opacity=0.7] (1.3590, 0.7050, 2.2034) -- (1.4060, 0.7050, 2.2042) -- (1.4060, 0.7580, 2.2085) -- (1.3590, 0.7580, 2.2077) -- cycle;
\fill[blue!73.0, opacity=0.7] (1.3590, 0.7580, 2.2077) -- (1.4060, 0.7580, 2.2085) -- (1.4060, 0.8110, 2.2126) -- (1.3590, 0.8110, 2.2118) -- cycle;
\fill[blue!84.7, opacity=0.7] (1.3590, 0.8110, 2.2118) -- (1.4060, 0.8110, 2.2126) -- (1.4060, 0.8640, 2.2164) -- (1.3590, 0.8640, 2.2156) -- cycle;
\fill[blue!56.6, opacity=0.7] (1.3590, 0.8640, 2.2156) -- (1.4060, 0.8640, 2.2164) -- (1.4060, 0.9170, 2.2200) -- (1.3590, 0.9170, 2.2192) -- cycle;
\fill[blue!55.1, opacity=0.7] (1.3590, 0.9170, 2.2192) -- (1.4060, 0.9170, 2.2200) -- (1.4060, 0.9700, 2.2233) -- (1.3590, 0.9700, 2.2224) -- cycle;
\fill[blue!84.9, opacity=0.7] (1.3590, 0.9700, 2.2224) -- (1.4060, 0.9700, 2.2233) -- (1.4060, 1.0230, 2.2263) -- (1.3590, 1.0230, 2.2254) -- cycle;
\fill[blue!59.5, opacity=0.7] (1.3590, 1.0230, 2.2254) -- (1.4060, 1.0230, 2.2263) -- (1.4060, 1.0760, 2.2290) -- (1.3590, 1.0760, 2.2281) -- cycle;
\fill[blue!23.1, opacity=0.7] (1.3590, 1.0760, 2.2281) -- (1.4060, 1.0760, 2.2290) -- (1.4060, 1.1290, 2.2314) -- (1.3590, 1.1290, 2.2306) -- cycle;
\fill[blue!18.7, opacity=0.7] (1.3590, 1.1290, 2.2306) -- (1.4060, 1.1290, 2.2314) -- (1.4060, 1.1820, 2.2335) -- (1.3590, 1.1820, 2.2326) -- cycle;
\fill[blue!27.5, opacity=0.7] (1.3590, 1.1820, 2.2326) -- (1.4060, 1.1820, 2.2335) -- (1.4060, 1.2350, 2.2353) -- (1.3590, 1.2350, 2.2344) -- cycle;
\fill[blue!67.9, opacity=0.7] (1.3590, 1.2350, 2.2344) -- (1.4060, 1.2350, 2.2353) -- (1.4060, 1.2880, 2.2367) -- (1.3590, 1.2880, 2.2359) -- cycle;
\fill[blue!87.8, opacity=0.7] (1.3590, 1.2880, 2.2359) -- (1.4060, 1.2880, 2.2367) -- (1.4060, 1.3410, 2.2379) -- (1.3590, 1.3410, 2.2370) -- cycle;
\fill[blue!87.1, opacity=0.7] (1.3590, 1.3410, 2.2370) -- (1.4060, 1.3410, 2.2379) -- (1.4060, 1.3940, 2.2387) -- (1.3590, 1.3940, 2.2379) -- cycle;
\fill[blue!42.8, opacity=0.7] (1.3590, 1.3940, 2.2379) -- (1.4060, 1.3940, 2.2387) -- (1.4060, 1.4470, 2.2392) -- (1.3590, 1.4470, 2.2384) -- cycle;
\fill[blue!16.6, opacity=0.7] (1.3590, 1.4470, 2.2384) -- (1.4060, 1.4470, 2.2392) -- (1.4060, 1.5000, 2.2393) -- (1.3590, 1.5000, 2.2385) -- cycle;
\fill[blue!15.5, opacity=0.7] (1.3590, 1.5000, 2.2385) -- (1.4060, 1.5000, 2.2393) -- (1.4060, 1.5530, 2.2392) -- (1.3590, 1.5530, 2.2384) -- cycle;
\fill[blue!16.0, opacity=0.7] (1.3590, 1.5530, 2.2384) -- (1.4060, 1.5530, 2.2392) -- (1.4060, 1.6060, 2.2387) -- (1.3590, 1.6060, 2.2379) -- cycle;
\fill[blue!16.4, opacity=0.7] (1.3590, 1.6060, 2.2379) -- (1.4060, 1.6060, 2.2387) -- (1.4060, 1.6590, 2.2379) -- (1.3590, 1.6590, 2.2370) -- cycle;
\fill[blue!16.3, opacity=0.7] (1.3590, 1.6590, 2.2370) -- (1.4060, 1.6590, 2.2379) -- (1.4060, 1.7120, 2.2367) -- (1.3590, 1.7120, 2.2359) -- cycle;
\fill[blue!16.7, opacity=0.7] (1.3590, 1.7120, 2.2359) -- (1.4060, 1.7120, 2.2367) -- (1.4060, 1.7650, 2.2353) -- (1.3590, 1.7650, 2.2344) -- cycle;
\fill[blue!20.3, opacity=0.7] (1.3590, 1.7650, 2.2344) -- (1.4060, 1.7650, 2.2353) -- (1.4060, 1.8180, 2.2335) -- (1.3590, 1.8180, 2.2326) -- cycle;
\fill[blue!40.5, opacity=0.7] (1.3590, 1.8180, 2.2326) -- (1.4060, 1.8180, 2.2335) -- (1.4060, 1.8710, 2.2314) -- (1.3590, 1.8710, 2.2306) -- cycle;
\fill[blue!82.5, opacity=0.7] (1.3590, 1.8710, 2.2306) -- (1.4060, 1.8710, 2.2314) -- (1.4060, 1.9240, 2.2290) -- (1.3590, 1.9240, 2.2281) -- cycle;
\fill[blue!78.0, opacity=0.7] (1.3590, 1.9240, 2.2281) -- (1.4060, 1.9240, 2.2290) -- (1.4060, 1.9770, 2.2263) -- (1.3590, 1.9770, 2.2254) -- cycle;
\fill[blue!60.0, opacity=0.7] (1.3590, 1.9770, 2.2254) -- (1.4060, 1.9770, 2.2263) -- (1.4060, 2.0300, 2.2233) -- (1.3590, 2.0300, 2.2224) -- cycle;
\fill[blue!69.3, opacity=0.7] (1.3590, 2.0300, 2.2224) -- (1.4060, 2.0300, 2.2233) -- (1.4060, 2.0830, 2.2200) -- (1.3590, 2.0830, 2.2192) -- cycle;
\fill[blue!87.8, opacity=0.7] (1.3590, 2.0830, 2.2192) -- (1.4060, 2.0830, 2.2200) -- (1.4060, 2.1360, 2.2164) -- (1.3590, 2.1360, 2.2156) -- cycle;
\fill[blue!65.5, opacity=0.7] (1.3590, 2.1360, 2.2156) -- (1.4060, 2.1360, 2.2164) -- (1.4060, 2.1890, 2.2126) -- (1.3590, 2.1890, 2.2118) -- cycle;
\fill[blue!43.4, opacity=0.7] (1.3590, 2.1890, 2.2118) -- (1.4060, 2.1890, 2.2126) -- (1.4060, 2.2420, 2.2085) -- (1.3590, 2.2420, 2.2077) -- cycle;
\fill[blue!50.6, opacity=0.7] (1.3590, 2.2420, 2.2077) -- (1.4060, 2.2420, 2.2085) -- (1.4060, 2.2950, 2.2042) -- (1.3590, 2.2950, 2.2034) -- cycle;
\fill[blue!84.0, opacity=0.7] (1.3590, 2.2950, 2.2034) -- (1.4060, 2.2950, 2.2042) -- (1.4060, 2.3480, 2.1996) -- (1.3590, 2.3480, 2.1988) -- cycle;
\fill[blue!64.3, opacity=0.7] (1.3590, 2.3480, 2.1988) -- (1.4060, 2.3480, 2.1996) -- (1.4060, 2.4010, 2.1949) -- (1.3590, 2.4010, 2.1940) -- cycle;
\fill[blue!26.7, opacity=0.7] (1.3590, 2.4010, 2.1940) -- (1.4060, 2.4010, 2.1949) -- (1.4060, 2.4540, 2.1899) -- (1.3590, 2.4540, 2.1891) -- cycle;
\fill[blue!20.1, opacity=0.7] (1.3590, 2.4540, 2.1891) -- (1.4060, 2.4540, 2.1899) -- (1.4060, 2.5070, 2.1847) -- (1.3590, 2.5070, 2.1839) -- cycle;
\fill[blue!26.0, opacity=0.7] (1.3590, 2.5070, 2.1839) -- (1.4060, 2.5070, 2.1847) -- (1.4060, 2.5600, 2.1793) -- (1.3590, 2.5600, 2.1785) -- cycle;
\fill[blue!56.0, opacity=0.7] (1.3590, 2.5600, 2.1785) -- (1.4060, 2.5600, 2.1793) -- (1.4060, 2.6130, 2.1738) -- (1.3590, 2.6130, 2.1730) -- cycle;
\fill[blue!86.1, opacity=0.7] (1.3590, 2.6130, 2.1730) -- (1.4060, 2.6130, 2.1738) -- (1.4060, 2.6660, 2.1682) -- (1.3590, 2.6660, 2.1673) -- cycle;
\fill[blue!87.5, opacity=0.7] (1.3590, 2.6660, 2.1673) -- (1.4060, 2.6660, 2.1682) -- (1.4060, 2.7190, 2.1623) -- (1.3590, 2.7190, 2.1615) -- cycle;
\fill[blue!83.3, opacity=0.7] (1.3590, 2.7190, 2.1615) -- (1.4060, 2.7190, 2.1623) -- (1.4060, 2.7720, 2.1564) -- (1.3590, 2.7720, 2.1556) -- cycle;
\fill[blue!40.6, opacity=0.7] (1.3590, 2.7720, 2.1556) -- (1.4060, 2.7720, 2.1564) -- (1.4060, 2.8250, 2.1504) -- (1.3590, 2.8250, 2.1496) -- cycle;
\fill[blue!17.1, opacity=0.7] (1.3590, 2.8250, 2.1496) -- (1.4060, 2.8250, 2.1504) -- (1.4060, 2.8780, 2.1443) -- (1.3590, 2.8780, 2.1435) -- cycle;
\fill[blue!15.2, opacity=0.7] (1.3590, 2.8780, 2.1435) -- (1.4060, 2.8780, 2.1443) -- (1.4060, 2.9310, 2.1381) -- (1.3590, 2.9310, 2.1373) -- cycle;
\fill[blue!15.2, opacity=0.7] (1.3590, 2.9310, 2.1373) -- (1.4060, 2.9310, 2.1381) -- (1.4060, 2.9840, 2.1319) -- (1.3590, 2.9840, 2.1311) -- cycle;
\fill[blue!16.3, opacity=0.7] (1.3590, 2.9840, 2.1311) -- (1.4060, 2.9840, 2.1319) -- (1.4060, 3.0370, 2.1256) -- (1.3590, 3.0370, 2.1248) -- cycle;
\fill[blue!27.0, opacity=0.7] (1.3590, 3.0370, 2.1248) -- (1.4060, 3.0370, 2.1256) -- (1.4060, 3.0900, 2.1193) -- (1.3590, 3.0900, 2.1185) -- cycle;
\fill[blue!19.5, opacity=0.7] (1.4060, -0.0900, 2.1193) -- (1.4530, -0.0900, 2.1198) -- (1.4530, -0.0370, 2.1261) -- (1.4060, -0.0370, 2.1256) -- cycle;
\fill[blue!15.3, opacity=0.7] (1.4060, -0.0370, 2.1256) -- (1.4530, -0.0370, 2.1261) -- (1.4530, 0.0160, 2.1324) -- (1.4060, 0.0160, 2.1319) -- cycle;
\fill[blue!15.1, opacity=0.7] (1.4060, 0.0160, 2.1319) -- (1.4530, 0.0160, 2.1324) -- (1.4530, 0.0690, 2.1386) -- (1.4060, 0.0690, 2.1381) -- cycle;
\fill[blue!15.4, opacity=0.7] (1.4060, 0.0690, 2.1381) -- (1.4530, 0.0690, 2.1386) -- (1.4530, 0.1220, 2.1448) -- (1.4060, 0.1220, 2.1443) -- cycle;
\fill[blue!22.4, opacity=0.7] (1.4060, 0.1220, 2.1443) -- (1.4530, 0.1220, 2.1448) -- (1.4530, 0.1750, 2.1509) -- (1.4060, 0.1750, 2.1504) -- cycle;
\fill[blue!64.0, opacity=0.7] (1.4060, 0.1750, 2.1504) -- (1.4530, 0.1750, 2.1509) -- (1.4530, 0.2280, 2.1569) -- (1.4060, 0.2280, 2.1564) -- cycle;
\fill[blue!87.6, opacity=0.7] (1.4060, 0.2280, 2.1564) -- (1.4530, 0.2280, 2.1569) -- (1.4530, 0.2810, 2.1628) -- (1.4060, 0.2810, 2.1623) -- cycle;
\fill[blue!87.7, opacity=0.7] (1.4060, 0.2810, 2.1623) -- (1.4530, 0.2810, 2.1628) -- (1.4530, 0.3340, 2.1686) -- (1.4060, 0.3340, 2.1682) -- cycle;
\fill[blue!69.0, opacity=0.7] (1.4060, 0.3340, 2.1682) -- (1.4530, 0.3340, 2.1686) -- (1.4530, 0.3870, 2.1743) -- (1.4060, 0.3870, 2.1738) -- cycle;
\fill[blue!31.1, opacity=0.7] (1.4060, 0.3870, 2.1738) -- (1.4530, 0.3870, 2.1743) -- (1.4530, 0.4400, 2.1798) -- (1.4060, 0.4400, 2.1793) -- cycle;
\fill[blue!19.6, opacity=0.7] (1.4060, 0.4400, 2.1793) -- (1.4530, 0.4400, 2.1798) -- (1.4530, 0.4930, 2.1852) -- (1.4060, 0.4930, 2.1847) -- cycle;
\fill[blue!21.5, opacity=0.7] (1.4060, 0.4930, 2.1847) -- (1.4530, 0.4930, 2.1852) -- (1.4530, 0.5460, 2.1904) -- (1.4060, 0.5460, 2.1899) -- cycle;
\fill[blue!47.1, opacity=0.7] (1.4060, 0.5460, 2.1899) -- (1.4530, 0.5460, 2.1904) -- (1.4530, 0.5990, 2.1954) -- (1.4060, 0.5990, 2.1949) -- cycle;
\fill[blue!87.9, opacity=0.7] (1.4060, 0.5990, 2.1949) -- (1.4530, 0.5990, 2.1954) -- (1.4530, 0.6520, 2.2001) -- (1.4060, 0.6520, 2.1996) -- cycle;
\fill[blue!59.3, opacity=0.7] (1.4060, 0.6520, 2.1996) -- (1.4530, 0.6520, 2.2001) -- (1.4530, 0.7050, 2.2047) -- (1.4060, 0.7050, 2.2042) -- cycle;
\fill[blue!47.4, opacity=0.7] (1.4060, 0.7050, 2.2042) -- (1.4530, 0.7050, 2.2047) -- (1.4530, 0.7580, 2.2090) -- (1.4060, 0.7580, 2.2085) -- cycle;
\fill[blue!70.3, opacity=0.7] (1.4060, 0.7580, 2.2085) -- (1.4530, 0.7580, 2.2090) -- (1.4530, 0.8110, 2.2131) -- (1.4060, 0.8110, 2.2126) -- cycle;
\fill[blue!86.1, opacity=0.7] (1.4060, 0.8110, 2.2126) -- (1.4530, 0.8110, 2.2131) -- (1.4530, 0.8640, 2.2169) -- (1.4060, 0.8640, 2.2164) -- cycle;
\fill[blue!59.2, opacity=0.7] (1.4060, 0.8640, 2.2164) -- (1.4530, 0.8640, 2.2169) -- (1.4530, 0.9170, 2.2205) -- (1.4060, 0.9170, 2.2200) -- cycle;
\fill[blue!55.8, opacity=0.7] (1.4060, 0.9170, 2.2200) -- (1.4530, 0.9170, 2.2205) -- (1.4530, 0.9700, 2.2238) -- (1.4060, 0.9700, 2.2233) -- cycle;
\fill[blue!84.4, opacity=0.7] (1.4060, 0.9700, 2.2233) -- (1.4530, 0.9700, 2.2238) -- (1.4530, 1.0230, 2.2268) -- (1.4060, 1.0230, 2.2263) -- cycle;
\fill[blue!61.3, opacity=0.7] (1.4060, 1.0230, 2.2263) -- (1.4530, 1.0230, 2.2268) -- (1.4530, 1.0760, 2.2295) -- (1.4060, 1.0760, 2.2290) -- cycle;
\fill[blue!23.3, opacity=0.7] (1.4060, 1.0760, 2.2290) -- (1.4530, 1.0760, 2.2295) -- (1.4530, 1.1290, 2.2319) -- (1.4060, 1.1290, 2.2314) -- cycle;
\fill[blue!18.2, opacity=0.7] (1.4060, 1.1290, 2.2314) -- (1.4530, 1.1290, 2.2319) -- (1.4530, 1.1820, 2.2340) -- (1.4060, 1.1820, 2.2335) -- cycle;
\fill[blue!24.6, opacity=0.7] (1.4060, 1.1820, 2.2335) -- (1.4530, 1.1820, 2.2340) -- (1.4530, 1.2350, 2.2357) -- (1.4060, 1.2350, 2.2353) -- cycle;
\fill[blue!61.0, opacity=0.7] (1.4060, 1.2350, 2.2353) -- (1.4530, 1.2350, 2.2357) -- (1.4530, 1.2880, 2.2372) -- (1.4060, 1.2880, 2.2367) -- cycle;
\fill[blue!87.5, opacity=0.7] (1.4060, 1.2880, 2.2367) -- (1.4530, 1.2880, 2.2372) -- (1.4530, 1.3410, 2.2384) -- (1.4060, 1.3410, 2.2379) -- cycle;
\fill[blue!86.0, opacity=0.7] (1.4060, 1.3410, 2.2379) -- (1.4530, 1.3410, 2.2384) -- (1.4530, 1.3940, 2.2392) -- (1.4060, 1.3940, 2.2387) -- cycle;
\fill[blue!33.1, opacity=0.7] (1.4060, 1.3940, 2.2387) -- (1.4530, 1.3940, 2.2392) -- (1.4530, 1.4470, 2.2397) -- (1.4060, 1.4470, 2.2392) -- cycle;
\fill[blue!15.4, opacity=0.7] (1.4060, 1.4470, 2.2392) -- (1.4530, 1.4470, 2.2397) -- (1.4530, 1.5000, 2.2398) -- (1.4060, 1.5000, 2.2393) -- cycle;
\fill[blue!17.7, opacity=0.7] (1.4060, 1.5000, 2.2393) -- (1.4530, 1.5000, 2.2398) -- (1.4530, 1.5530, 2.2397) -- (1.4060, 1.5530, 2.2392) -- cycle;
\fill[blue!31.3, opacity=0.7] (1.4060, 1.5530, 2.2392) -- (1.4530, 1.5530, 2.2397) -- (1.4530, 1.6060, 2.2392) -- (1.4060, 1.6060, 2.2387) -- cycle;
\fill[blue!30.2, opacity=0.7] (1.4060, 1.6060, 2.2387) -- (1.4530, 1.6060, 2.2392) -- (1.4530, 1.6590, 2.2384) -- (1.4060, 1.6590, 2.2379) -- cycle;
\fill[blue!21.5, opacity=0.7] (1.4060, 1.6590, 2.2379) -- (1.4530, 1.6590, 2.2384) -- (1.4530, 1.7120, 2.2372) -- (1.4060, 1.7120, 2.2367) -- cycle;
\fill[blue!17.3, opacity=0.7] (1.4060, 1.7120, 2.2367) -- (1.4530, 1.7120, 2.2372) -- (1.4530, 1.7650, 2.2357) -- (1.4060, 1.7650, 2.2353) -- cycle;
\fill[blue!17.1, opacity=0.7] (1.4060, 1.7650, 2.2353) -- (1.4530, 1.7650, 2.2357) -- (1.4530, 1.8180, 2.2340) -- (1.4060, 1.8180, 2.2335) -- cycle;
\fill[blue!22.6, opacity=0.7] (1.4060, 1.8180, 2.2335) -- (1.4530, 1.8180, 2.2340) -- (1.4530, 1.8710, 2.2319) -- (1.4060, 1.8710, 2.2314) -- cycle;
\fill[blue!55.2, opacity=0.7] (1.4060, 1.8710, 2.2314) -- (1.4530, 1.8710, 2.2319) -- (1.4530, 1.9240, 2.2295) -- (1.4060, 1.9240, 2.2290) -- cycle;
\fill[blue!87.7, opacity=0.7] (1.4060, 1.9240, 2.2290) -- (1.4530, 1.9240, 2.2295) -- (1.4530, 1.9770, 2.2268) -- (1.4060, 1.9770, 2.2263) -- cycle;
\fill[blue!64.8, opacity=0.7] (1.4060, 1.9770, 2.2263) -- (1.4530, 1.9770, 2.2268) -- (1.4530, 2.0300, 2.2238) -- (1.4060, 2.0300, 2.2233) -- cycle;
\fill[blue!59.6, opacity=0.7] (1.4060, 2.0300, 2.2233) -- (1.4530, 2.0300, 2.2238) -- (1.4530, 2.0830, 2.2205) -- (1.4060, 2.0830, 2.2200) -- cycle;
\fill[blue!81.7, opacity=0.7] (1.4060, 2.0830, 2.2200) -- (1.4530, 2.0830, 2.2205) -- (1.4530, 2.1360, 2.2169) -- (1.4060, 2.1360, 2.2164) -- cycle;
\fill[blue!79.4, opacity=0.7] (1.4060, 2.1360, 2.2164) -- (1.4530, 2.1360, 2.2169) -- (1.4530, 2.1890, 2.2131) -- (1.4060, 2.1890, 2.2126) -- cycle;
\fill[blue!49.5, opacity=0.7] (1.4060, 2.1890, 2.2126) -- (1.4530, 2.1890, 2.2131) -- (1.4530, 2.2420, 2.2090) -- (1.4060, 2.2420, 2.2085) -- cycle;
\fill[blue!45.8, opacity=0.7] (1.4060, 2.2420, 2.2085) -- (1.4530, 2.2420, 2.2090) -- (1.4530, 2.2950, 2.2047) -- (1.4060, 2.2950, 2.2042) -- cycle;
\fill[blue!74.7, opacity=0.7] (1.4060, 2.2950, 2.2042) -- (1.4530, 2.2950, 2.2047) -- (1.4530, 2.3480, 2.2001) -- (1.4060, 2.3480, 2.1996) -- cycle;
\fill[blue!77.6, opacity=0.7] (1.4060, 2.3480, 2.1996) -- (1.4530, 2.3480, 2.2001) -- (1.4530, 2.4010, 2.1954) -- (1.4060, 2.4010, 2.1949) -- cycle;
\fill[blue!32.1, opacity=0.7] (1.4060, 2.4010, 2.1949) -- (1.4530, 2.4010, 2.1954) -- (1.4530, 2.4540, 2.1904) -- (1.4060, 2.4540, 2.1899) -- cycle;
\fill[blue!20.1, opacity=0.7] (1.4060, 2.4540, 2.1899) -- (1.4530, 2.4540, 2.1904) -- (1.4530, 2.5070, 2.1852) -- (1.4060, 2.5070, 2.1847) -- cycle;
\fill[blue!22.8, opacity=0.7] (1.4060, 2.5070, 2.1847) -- (1.4530, 2.5070, 2.1852) -- (1.4530, 2.5600, 2.1798) -- (1.4060, 2.5600, 2.1793) -- cycle;
\fill[blue!45.7, opacity=0.7] (1.4060, 2.5600, 2.1793) -- (1.4530, 2.5600, 2.1798) -- (1.4530, 2.6130, 2.1743) -- (1.4060, 2.6130, 2.1738) -- cycle;
\fill[blue!82.4, opacity=0.7] (1.4060, 2.6130, 2.1738) -- (1.4530, 2.6130, 2.1743) -- (1.4530, 2.6660, 2.1686) -- (1.4060, 2.6660, 2.1682) -- cycle;
\fill[blue!87.6, opacity=0.7] (1.4060, 2.6660, 2.1682) -- (1.4530, 2.6660, 2.1686) -- (1.4530, 2.7190, 2.1628) -- (1.4060, 2.7190, 2.1623) -- cycle;
\fill[blue!86.0, opacity=0.7] (1.4060, 2.7190, 2.1623) -- (1.4530, 2.7190, 2.1628) -- (1.4530, 2.7720, 2.1569) -- (1.4060, 2.7720, 2.1564) -- cycle;
\fill[blue!49.4, opacity=0.7] (1.4060, 2.7720, 2.1564) -- (1.4530, 2.7720, 2.1569) -- (1.4530, 2.8250, 2.1509) -- (1.4060, 2.8250, 2.1504) -- cycle;
\fill[blue!18.4, opacity=0.7] (1.4060, 2.8250, 2.1504) -- (1.4530, 2.8250, 2.1509) -- (1.4530, 2.8780, 2.1448) -- (1.4060, 2.8780, 2.1443) -- cycle;
\fill[blue!15.2, opacity=0.7] (1.4060, 2.8780, 2.1443) -- (1.4530, 2.8780, 2.1448) -- (1.4530, 2.9310, 2.1386) -- (1.4060, 2.9310, 2.1381) -- cycle;
\fill[blue!15.1, opacity=0.7] (1.4060, 2.9310, 2.1381) -- (1.4530, 2.9310, 2.1386) -- (1.4530, 2.9840, 2.1324) -- (1.4060, 2.9840, 2.1319) -- cycle;
\fill[blue!15.8, opacity=0.7] (1.4060, 2.9840, 2.1319) -- (1.4530, 2.9840, 2.1324) -- (1.4530, 3.0370, 2.1261) -- (1.4060, 3.0370, 2.1256) -- cycle;
\fill[blue!23.7, opacity=0.7] (1.4060, 3.0370, 2.1256) -- (1.4530, 3.0370, 2.1261) -- (1.4530, 3.0900, 2.1198) -- (1.4060, 3.0900, 2.1193) -- cycle;
\fill[blue!20.2, opacity=0.7] (1.4530, -0.0900, 2.1198) -- (1.5000, -0.0900, 2.1200) -- (1.5000, -0.0370, 2.1263) -- (1.4530, -0.0370, 2.1261) -- cycle;
\fill[blue!15.4, opacity=0.7] (1.4530, -0.0370, 2.1261) -- (1.5000, -0.0370, 2.1263) -- (1.5000, 0.0160, 2.1325) -- (1.4530, 0.0160, 2.1324) -- cycle;
\fill[blue!15.1, opacity=0.7] (1.4530, 0.0160, 2.1324) -- (1.5000, 0.0160, 2.1325) -- (1.5000, 0.0690, 2.1388) -- (1.4530, 0.0690, 2.1386) -- cycle;
\fill[blue!15.4, opacity=0.7] (1.4530, 0.0690, 2.1386) -- (1.5000, 0.0690, 2.1388) -- (1.5000, 0.1220, 2.1449) -- (1.4530, 0.1220, 2.1448) -- cycle;
\fill[blue!21.4, opacity=0.7] (1.4530, 0.1220, 2.1448) -- (1.5000, 0.1220, 2.1449) -- (1.5000, 0.1750, 2.1511) -- (1.4530, 0.1750, 2.1509) -- cycle;
\fill[blue!61.4, opacity=0.7] (1.4530, 0.1750, 2.1509) -- (1.5000, 0.1750, 2.1511) -- (1.5000, 0.2280, 2.1571) -- (1.4530, 0.2280, 2.1569) -- cycle;
\fill[blue!87.5, opacity=0.7] (1.4530, 0.2280, 2.1569) -- (1.5000, 0.2280, 2.1571) -- (1.5000, 0.2810, 2.1630) -- (1.4530, 0.2810, 2.1628) -- cycle;
\fill[blue!87.8, opacity=0.7] (1.4530, 0.2810, 2.1628) -- (1.5000, 0.2810, 2.1630) -- (1.5000, 0.3340, 2.1688) -- (1.4530, 0.3340, 2.1686) -- cycle;
\fill[blue!72.7, opacity=0.7] (1.4530, 0.3340, 2.1686) -- (1.5000, 0.3340, 2.1688) -- (1.5000, 0.3870, 2.1745) -- (1.4530, 0.3870, 2.1743) -- cycle;
\fill[blue!33.8, opacity=0.7] (1.4530, 0.3870, 2.1743) -- (1.5000, 0.3870, 2.1745) -- (1.5000, 0.4400, 2.1800) -- (1.4530, 0.4400, 2.1798) -- cycle;
\fill[blue!20.1, opacity=0.7] (1.4530, 0.4400, 2.1798) -- (1.5000, 0.4400, 2.1800) -- (1.5000, 0.4930, 2.1854) -- (1.4530, 0.4930, 2.1852) -- cycle;
\fill[blue!21.1, opacity=0.7] (1.4530, 0.4930, 2.1852) -- (1.5000, 0.4930, 2.1854) -- (1.5000, 0.5460, 2.1905) -- (1.4530, 0.5460, 2.1904) -- cycle;
\fill[blue!43.6, opacity=0.7] (1.4530, 0.5460, 2.1904) -- (1.5000, 0.5460, 2.1905) -- (1.5000, 0.5990, 2.1955) -- (1.4530, 0.5990, 2.1954) -- cycle;
\fill[blue!87.4, opacity=0.7] (1.4530, 0.5990, 2.1954) -- (1.5000, 0.5990, 2.1955) -- (1.5000, 0.6520, 2.2003) -- (1.4530, 0.6520, 2.2001) -- cycle;
\fill[blue!61.6, opacity=0.7] (1.4530, 0.6520, 2.2001) -- (1.5000, 0.6520, 2.2003) -- (1.5000, 0.7050, 2.2049) -- (1.4530, 0.7050, 2.2047) -- cycle;
\fill[blue!45.8, opacity=0.7] (1.4530, 0.7050, 2.2047) -- (1.5000, 0.7050, 2.2049) -- (1.5000, 0.7580, 2.2092) -- (1.4530, 0.7580, 2.2090) -- cycle;
\fill[blue!64.9, opacity=0.7] (1.4530, 0.7580, 2.2090) -- (1.5000, 0.7580, 2.2092) -- (1.5000, 0.8110, 2.2133) -- (1.4530, 0.8110, 2.2131) -- cycle;
\fill[blue!87.7, opacity=0.7] (1.4530, 0.8110, 2.2131) -- (1.5000, 0.8110, 2.2133) -- (1.5000, 0.8640, 2.2171) -- (1.4530, 0.8640, 2.2169) -- cycle;
\fill[blue!64.0, opacity=0.7] (1.4530, 0.8640, 2.2169) -- (1.5000, 0.8640, 2.2171) -- (1.5000, 0.9170, 2.2206) -- (1.4530, 0.9170, 2.2205) -- cycle;
\fill[blue!55.5, opacity=0.7] (1.4530, 0.9170, 2.2205) -- (1.5000, 0.9170, 2.2206) -- (1.5000, 0.9700, 2.2239) -- (1.4530, 0.9700, 2.2238) -- cycle;
\fill[blue!80.8, opacity=0.7] (1.4530, 0.9700, 2.2238) -- (1.5000, 0.9700, 2.2239) -- (1.5000, 1.0230, 2.2269) -- (1.4530, 1.0230, 2.2268) -- cycle;
\fill[blue!69.9, opacity=0.7] (1.4530, 1.0230, 2.2268) -- (1.5000, 1.0230, 2.2269) -- (1.5000, 1.0760, 2.2296) -- (1.4530, 1.0760, 2.2295) -- cycle;
\fill[blue!26.0, opacity=0.7] (1.4530, 1.0760, 2.2295) -- (1.5000, 1.0760, 2.2296) -- (1.5000, 1.1290, 2.2320) -- (1.4530, 1.1290, 2.2319) -- cycle;
\fill[blue!17.8, opacity=0.7] (1.4530, 1.1290, 2.2319) -- (1.5000, 1.1290, 2.2320) -- (1.5000, 1.1820, 2.2341) -- (1.4530, 1.1820, 2.2340) -- cycle;
\fill[blue!20.3, opacity=0.7] (1.4530, 1.1820, 2.2340) -- (1.5000, 1.1820, 2.2341) -- (1.5000, 1.2350, 2.2359) -- (1.4530, 1.2350, 2.2357) -- cycle;
\fill[blue!43.0, opacity=0.7] (1.4530, 1.2350, 2.2357) -- (1.5000, 1.2350, 2.2359) -- (1.5000, 1.2880, 2.2374) -- (1.4530, 1.2880, 2.2372) -- cycle;
\fill[blue!81.5, opacity=0.7] (1.4530, 1.2880, 2.2372) -- (1.5000, 1.2880, 2.2374) -- (1.5000, 1.3410, 2.2385) -- (1.4530, 1.3410, 2.2384) -- cycle;
\fill[blue!87.3, opacity=0.7] (1.4530, 1.3410, 2.2384) -- (1.5000, 1.3410, 2.2385) -- (1.5000, 1.3940, 2.2393) -- (1.4530, 1.3940, 2.2392) -- cycle;
\fill[blue!59.3, opacity=0.7] (1.4530, 1.3940, 2.2392) -- (1.5000, 1.3940, 2.2393) -- (1.5000, 1.4470, 2.2398) -- (1.4530, 1.4470, 2.2397) -- cycle;
\fill[blue!15.4, opacity=0.7] (1.4530, 1.4470, 2.2397) -- (1.5000, 1.4470, 2.2398) -- (1.5000, 1.5000, 2.2400) -- (1.4530, 1.5000, 2.2398) -- cycle;
\fill[blue!75.6, opacity=0.7] (1.4530, 1.5000, 2.2398) -- (1.5000, 1.5000, 2.2400) -- (1.5000, 1.5530, 2.2398) -- (1.4530, 1.5530, 2.2397) -- cycle;
\fill[blue!84.3, opacity=0.7] (1.4530, 1.5530, 2.2397) -- (1.5000, 1.5530, 2.2398) -- (1.5000, 1.6060, 2.2393) -- (1.4530, 1.6060, 2.2392) -- cycle;
\fill[blue!78.9, opacity=0.7] (1.4530, 1.6060, 2.2392) -- (1.5000, 1.6060, 2.2393) -- (1.5000, 1.6590, 2.2385) -- (1.4530, 1.6590, 2.2384) -- cycle;
\fill[blue!51.6, opacity=0.7] (1.4530, 1.6590, 2.2384) -- (1.5000, 1.6590, 2.2385) -- (1.5000, 1.7120, 2.2374) -- (1.4530, 1.7120, 2.2372) -- cycle;
\fill[blue!24.3, opacity=0.7] (1.4530, 1.7120, 2.2372) -- (1.5000, 1.7120, 2.2374) -- (1.5000, 1.7650, 2.2359) -- (1.4530, 1.7650, 2.2357) -- cycle;
\fill[blue!17.6, opacity=0.7] (1.4530, 1.7650, 2.2357) -- (1.5000, 1.7650, 2.2359) -- (1.5000, 1.8180, 2.2341) -- (1.4530, 1.8180, 2.2340) -- cycle;
\fill[blue!18.5, opacity=0.7] (1.4530, 1.8180, 2.2340) -- (1.5000, 1.8180, 2.2341) -- (1.5000, 1.8710, 2.2320) -- (1.4530, 1.8710, 2.2319) -- cycle;
\fill[blue!34.5, opacity=0.7] (1.4530, 1.8710, 2.2319) -- (1.5000, 1.8710, 2.2320) -- (1.5000, 1.9240, 2.2296) -- (1.4530, 1.9240, 2.2295) -- cycle;
\fill[blue!82.3, opacity=0.7] (1.4530, 1.9240, 2.2295) -- (1.5000, 1.9240, 2.2296) -- (1.5000, 1.9770, 2.2269) -- (1.4530, 1.9770, 2.2268) -- cycle;
\fill[blue!73.6, opacity=0.7] (1.4530, 1.9770, 2.2268) -- (1.5000, 1.9770, 2.2269) -- (1.5000, 2.0300, 2.2239) -- (1.4530, 2.0300, 2.2238) -- cycle;
\fill[blue!55.8, opacity=0.7] (1.4530, 2.0300, 2.2238) -- (1.5000, 2.0300, 2.2239) -- (1.5000, 2.0830, 2.2206) -- (1.4530, 2.0830, 2.2205) -- cycle;
\fill[blue!71.8, opacity=0.7] (1.4530, 2.0830, 2.2205) -- (1.5000, 2.0830, 2.2206) -- (1.5000, 2.1360, 2.2171) -- (1.4530, 2.1360, 2.2169) -- cycle;
\fill[blue!86.7, opacity=0.7] (1.4530, 2.1360, 2.2169) -- (1.5000, 2.1360, 2.2171) -- (1.5000, 2.1890, 2.2133) -- (1.4530, 2.1890, 2.2131) -- cycle;
\fill[blue!57.4, opacity=0.7] (1.4530, 2.1890, 2.2131) -- (1.5000, 2.1890, 2.2133) -- (1.5000, 2.2420, 2.2092) -- (1.4530, 2.2420, 2.2090) -- cycle;
\fill[blue!44.9, opacity=0.7] (1.4530, 2.2420, 2.2090) -- (1.5000, 2.2420, 2.2092) -- (1.5000, 2.2950, 2.2049) -- (1.4530, 2.2950, 2.2047) -- cycle;
\fill[blue!66.7, opacity=0.7] (1.4530, 2.2950, 2.2047) -- (1.5000, 2.2950, 2.2049) -- (1.5000, 2.3480, 2.2003) -- (1.4530, 2.3480, 2.2001) -- cycle;
\fill[blue!84.8, opacity=0.7] (1.4530, 2.3480, 2.2001) -- (1.5000, 2.3480, 2.2003) -- (1.5000, 2.4010, 2.1955) -- (1.4530, 2.4010, 2.1954) -- cycle;
\fill[blue!38.2, opacity=0.7] (1.4530, 2.4010, 2.1954) -- (1.5000, 2.4010, 2.1955) -- (1.5000, 2.4540, 2.1905) -- (1.4530, 2.4540, 2.1904) -- cycle;
\fill[blue!20.6, opacity=0.7] (1.4530, 2.4540, 2.1904) -- (1.5000, 2.4540, 2.1905) -- (1.5000, 2.5070, 2.1854) -- (1.4530, 2.5070, 2.1852) -- cycle;
\fill[blue!21.0, opacity=0.7] (1.4530, 2.5070, 2.1852) -- (1.5000, 2.5070, 2.1854) -- (1.5000, 2.5600, 2.1800) -- (1.4530, 2.5600, 2.1798) -- cycle;
\fill[blue!38.5, opacity=0.7] (1.4530, 2.5600, 2.1798) -- (1.5000, 2.5600, 2.1800) -- (1.5000, 2.6130, 2.1745) -- (1.4530, 2.6130, 2.1743) -- cycle;
\fill[blue!77.5, opacity=0.7] (1.4530, 2.6130, 2.1743) -- (1.5000, 2.6130, 2.1745) -- (1.5000, 2.6660, 2.1688) -- (1.4530, 2.6660, 2.1686) -- cycle;
\fill[blue!87.8, opacity=0.7] (1.4530, 2.6660, 2.1686) -- (1.5000, 2.6660, 2.1688) -- (1.5000, 2.7190, 2.1630) -- (1.4530, 2.7190, 2.1628) -- cycle;
\fill[blue!87.1, opacity=0.7] (1.4530, 2.7190, 2.1628) -- (1.5000, 2.7190, 2.1630) -- (1.5000, 2.7720, 2.1571) -- (1.4530, 2.7720, 2.1569) -- cycle;
\fill[blue!56.5, opacity=0.7] (1.4530, 2.7720, 2.1569) -- (1.5000, 2.7720, 2.1571) -- (1.5000, 2.8250, 2.1511) -- (1.4530, 2.8250, 2.1509) -- cycle;
\fill[blue!20.0, opacity=0.7] (1.4530, 2.8250, 2.1509) -- (1.5000, 2.8250, 2.1511) -- (1.5000, 2.8780, 2.1449) -- (1.4530, 2.8780, 2.1448) -- cycle;
\fill[blue!15.3, opacity=0.7] (1.4530, 2.8780, 2.1448) -- (1.5000, 2.8780, 2.1449) -- (1.5000, 2.9310, 2.1388) -- (1.4530, 2.9310, 2.1386) -- cycle;
\fill[blue!15.1, opacity=0.7] (1.4530, 2.9310, 2.1386) -- (1.5000, 2.9310, 2.1388) -- (1.5000, 2.9840, 2.1325) -- (1.4530, 2.9840, 2.1324) -- cycle;
\fill[blue!15.5, opacity=0.7] (1.4530, 2.9840, 2.1324) -- (1.5000, 2.9840, 2.1325) -- (1.5000, 3.0370, 2.1263) -- (1.4530, 3.0370, 2.1261) -- cycle;
\fill[blue!21.6, opacity=0.7] (1.4530, 3.0370, 2.1261) -- (1.5000, 3.0370, 2.1263) -- (1.5000, 3.0900, 2.1200) -- (1.4530, 3.0900, 2.1198) -- cycle;
\fill[blue!21.6, opacity=0.7] (1.5000, -0.0900, 2.1200) -- (1.5470, -0.0900, 2.1198) -- (1.5470, -0.0370, 2.1261) -- (1.5000, -0.0370, 2.1263) -- cycle;
\fill[blue!15.5, opacity=0.7] (1.5000, -0.0370, 2.1263) -- (1.5470, -0.0370, 2.1261) -- (1.5470, 0.0160, 2.1324) -- (1.5000, 0.0160, 2.1325) -- cycle;
\fill[blue!15.1, opacity=0.7] (1.5000, 0.0160, 2.1325) -- (1.5470, 0.0160, 2.1324) -- (1.5470, 0.0690, 2.1386) -- (1.5000, 0.0690, 2.1388) -- cycle;
\fill[blue!15.3, opacity=0.7] (1.5000, 0.0690, 2.1388) -- (1.5470, 0.0690, 2.1386) -- (1.5470, 0.1220, 2.1448) -- (1.5000, 0.1220, 2.1449) -- cycle;
\fill[blue!20.0, opacity=0.7] (1.5000, 0.1220, 2.1449) -- (1.5470, 0.1220, 2.1448) -- (1.5470, 0.1750, 2.1509) -- (1.5000, 0.1750, 2.1511) -- cycle;
\fill[blue!56.5, opacity=0.7] (1.5000, 0.1750, 2.1511) -- (1.5470, 0.1750, 2.1509) -- (1.5470, 0.2280, 2.1569) -- (1.5000, 0.2280, 2.1571) -- cycle;
\fill[blue!87.1, opacity=0.7] (1.5000, 0.2280, 2.1571) -- (1.5470, 0.2280, 2.1569) -- (1.5470, 0.2810, 2.1628) -- (1.5000, 0.2810, 2.1630) -- cycle;
\fill[blue!87.8, opacity=0.7] (1.5000, 0.2810, 2.1630) -- (1.5470, 0.2810, 2.1628) -- (1.5470, 0.3340, 2.1686) -- (1.5000, 0.3340, 2.1688) -- cycle;
\fill[blue!77.5, opacity=0.7] (1.5000, 0.3340, 2.1688) -- (1.5470, 0.3340, 2.1686) -- (1.5470, 0.3870, 2.1743) -- (1.5000, 0.3870, 2.1745) -- cycle;
\fill[blue!38.5, opacity=0.7] (1.5000, 0.3870, 2.1745) -- (1.5470, 0.3870, 2.1743) -- (1.5470, 0.4400, 2.1798) -- (1.5000, 0.4400, 2.1800) -- cycle;
\fill[blue!21.0, opacity=0.7] (1.5000, 0.4400, 2.1800) -- (1.5470, 0.4400, 2.1798) -- (1.5470, 0.4930, 2.1852) -- (1.5000, 0.4930, 2.1854) -- cycle;
\fill[blue!20.6, opacity=0.7] (1.5000, 0.4930, 2.1854) -- (1.5470, 0.4930, 2.1852) -- (1.5470, 0.5460, 2.1904) -- (1.5000, 0.5460, 2.1905) -- cycle;
\fill[blue!38.2, opacity=0.7] (1.5000, 0.5460, 2.1905) -- (1.5470, 0.5460, 2.1904) -- (1.5470, 0.5990, 2.1954) -- (1.5000, 0.5990, 2.1955) -- cycle;
\fill[blue!84.8, opacity=0.7] (1.5000, 0.5990, 2.1955) -- (1.5470, 0.5990, 2.1954) -- (1.5470, 0.6520, 2.2001) -- (1.5000, 0.6520, 2.2003) -- cycle;
\fill[blue!66.7, opacity=0.7] (1.5000, 0.6520, 2.2003) -- (1.5470, 0.6520, 2.2001) -- (1.5470, 0.7050, 2.2047) -- (1.5000, 0.7050, 2.2049) -- cycle;
\fill[blue!44.9, opacity=0.7] (1.5000, 0.7050, 2.2049) -- (1.5470, 0.7050, 2.2047) -- (1.5470, 0.7580, 2.2090) -- (1.5000, 0.7580, 2.2092) -- cycle;
\fill[blue!57.4, opacity=0.7] (1.5000, 0.7580, 2.2092) -- (1.5470, 0.7580, 2.2090) -- (1.5470, 0.8110, 2.2131) -- (1.5000, 0.8110, 2.2133) -- cycle;
\fill[blue!86.7, opacity=0.7] (1.5000, 0.8110, 2.2133) -- (1.5470, 0.8110, 2.2131) -- (1.5470, 0.8640, 2.2169) -- (1.5000, 0.8640, 2.2171) -- cycle;
\fill[blue!71.8, opacity=0.7] (1.5000, 0.8640, 2.2171) -- (1.5470, 0.8640, 2.2169) -- (1.5470, 0.9170, 2.2205) -- (1.5000, 0.9170, 2.2206) -- cycle;
\fill[blue!55.8, opacity=0.7] (1.5000, 0.9170, 2.2206) -- (1.5470, 0.9170, 2.2205) -- (1.5470, 0.9700, 2.2238) -- (1.5000, 0.9700, 2.2239) -- cycle;
\fill[blue!73.6, opacity=0.7] (1.5000, 0.9700, 2.2239) -- (1.5470, 0.9700, 2.2238) -- (1.5470, 1.0230, 2.2268) -- (1.5000, 1.0230, 2.2269) -- cycle;
\fill[blue!82.3, opacity=0.7] (1.5000, 1.0230, 2.2269) -- (1.5470, 1.0230, 2.2268) -- (1.5470, 1.0760, 2.2295) -- (1.5000, 1.0760, 2.2296) -- cycle;
\fill[blue!34.5, opacity=0.7] (1.5000, 1.0760, 2.2296) -- (1.5470, 1.0760, 2.2295) -- (1.5470, 1.1290, 2.2319) -- (1.5000, 1.1290, 2.2320) -- cycle;
\fill[blue!18.5, opacity=0.7] (1.5000, 1.1290, 2.2320) -- (1.5470, 1.1290, 2.2319) -- (1.5470, 1.1820, 2.2340) -- (1.5000, 1.1820, 2.2341) -- cycle;
\fill[blue!17.6, opacity=0.7] (1.5000, 1.1820, 2.2341) -- (1.5470, 1.1820, 2.2340) -- (1.5470, 1.2350, 2.2357) -- (1.5000, 1.2350, 2.2359) -- cycle;
\fill[blue!24.3, opacity=0.7] (1.5000, 1.2350, 2.2359) -- (1.5470, 1.2350, 2.2357) -- (1.5470, 1.2880, 2.2372) -- (1.5000, 1.2880, 2.2374) -- cycle;
\fill[blue!51.6, opacity=0.7] (1.5000, 1.2880, 2.2374) -- (1.5470, 1.2880, 2.2372) -- (1.5470, 1.3410, 2.2384) -- (1.5000, 1.3410, 2.2385) -- cycle;
\fill[blue!78.9, opacity=0.7] (1.5000, 1.3410, 2.2385) -- (1.5470, 1.3410, 2.2384) -- (1.5470, 1.3940, 2.2392) -- (1.5000, 1.3940, 2.2393) -- cycle;
\fill[blue!84.3, opacity=0.7] (1.5000, 1.3940, 2.2393) -- (1.5470, 1.3940, 2.2392) -- (1.5470, 1.4470, 2.2397) -- (1.5000, 1.4470, 2.2398) -- cycle;
\fill[blue!75.6, opacity=0.7] (1.5000, 1.4470, 2.2398) -- (1.5470, 1.4470, 2.2397) -- (1.5470, 1.5000, 2.2398) -- (1.5000, 1.5000, 2.2400) -- cycle;
\fill[blue!15.4, opacity=0.7] (1.5000, 1.5000, 2.2400) -- (1.5470, 1.5000, 2.2398) -- (1.5470, 1.5530, 2.2397) -- (1.5000, 1.5530, 2.2398) -- cycle;
\fill[blue!59.3, opacity=0.7] (1.5000, 1.5530, 2.2398) -- (1.5470, 1.5530, 2.2397) -- (1.5470, 1.6060, 2.2392) -- (1.5000, 1.6060, 2.2393) -- cycle;
\fill[blue!87.3, opacity=0.7] (1.5000, 1.6060, 2.2393) -- (1.5470, 1.6060, 2.2392) -- (1.5470, 1.6590, 2.2384) -- (1.5000, 1.6590, 2.2385) -- cycle;
\fill[blue!81.5, opacity=0.7] (1.5000, 1.6590, 2.2385) -- (1.5470, 1.6590, 2.2384) -- (1.5470, 1.7120, 2.2372) -- (1.5000, 1.7120, 2.2374) -- cycle;
\fill[blue!43.0, opacity=0.7] (1.5000, 1.7120, 2.2374) -- (1.5470, 1.7120, 2.2372) -- (1.5470, 1.7650, 2.2357) -- (1.5000, 1.7650, 2.2359) -- cycle;
\fill[blue!20.3, opacity=0.7] (1.5000, 1.7650, 2.2359) -- (1.5470, 1.7650, 2.2357) -- (1.5470, 1.8180, 2.2340) -- (1.5000, 1.8180, 2.2341) -- cycle;
\fill[blue!17.8, opacity=0.7] (1.5000, 1.8180, 2.2341) -- (1.5470, 1.8180, 2.2340) -- (1.5470, 1.8710, 2.2319) -- (1.5000, 1.8710, 2.2320) -- cycle;
\fill[blue!26.0, opacity=0.7] (1.5000, 1.8710, 2.2320) -- (1.5470, 1.8710, 2.2319) -- (1.5470, 1.9240, 2.2295) -- (1.5000, 1.9240, 2.2296) -- cycle;
\fill[blue!69.9, opacity=0.7] (1.5000, 1.9240, 2.2296) -- (1.5470, 1.9240, 2.2295) -- (1.5470, 1.9770, 2.2268) -- (1.5000, 1.9770, 2.2269) -- cycle;
\fill[blue!80.8, opacity=0.7] (1.5000, 1.9770, 2.2269) -- (1.5470, 1.9770, 2.2268) -- (1.5470, 2.0300, 2.2238) -- (1.5000, 2.0300, 2.2239) -- cycle;
\fill[blue!55.5, opacity=0.7] (1.5000, 2.0300, 2.2239) -- (1.5470, 2.0300, 2.2238) -- (1.5470, 2.0830, 2.2205) -- (1.5000, 2.0830, 2.2206) -- cycle;
\fill[blue!64.0, opacity=0.7] (1.5000, 2.0830, 2.2206) -- (1.5470, 2.0830, 2.2205) -- (1.5470, 2.1360, 2.2169) -- (1.5000, 2.1360, 2.2171) -- cycle;
\fill[blue!87.7, opacity=0.7] (1.5000, 2.1360, 2.2171) -- (1.5470, 2.1360, 2.2169) -- (1.5470, 2.1890, 2.2131) -- (1.5000, 2.1890, 2.2133) -- cycle;
\fill[blue!64.9, opacity=0.7] (1.5000, 2.1890, 2.2133) -- (1.5470, 2.1890, 2.2131) -- (1.5470, 2.2420, 2.2090) -- (1.5000, 2.2420, 2.2092) -- cycle;
\fill[blue!45.8, opacity=0.7] (1.5000, 2.2420, 2.2092) -- (1.5470, 2.2420, 2.2090) -- (1.5470, 2.2950, 2.2047) -- (1.5000, 2.2950, 2.2049) -- cycle;
\fill[blue!61.6, opacity=0.7] (1.5000, 2.2950, 2.2049) -- (1.5470, 2.2950, 2.2047) -- (1.5470, 2.3480, 2.2001) -- (1.5000, 2.3480, 2.2003) -- cycle;
\fill[blue!87.4, opacity=0.7] (1.5000, 2.3480, 2.2003) -- (1.5470, 2.3480, 2.2001) -- (1.5470, 2.4010, 2.1954) -- (1.5000, 2.4010, 2.1955) -- cycle;
\fill[blue!43.6, opacity=0.7] (1.5000, 2.4010, 2.1955) -- (1.5470, 2.4010, 2.1954) -- (1.5470, 2.4540, 2.1904) -- (1.5000, 2.4540, 2.1905) -- cycle;
\fill[blue!21.1, opacity=0.7] (1.5000, 2.4540, 2.1905) -- (1.5470, 2.4540, 2.1904) -- (1.5470, 2.5070, 2.1852) -- (1.5000, 2.5070, 2.1854) -- cycle;
\fill[blue!20.1, opacity=0.7] (1.5000, 2.5070, 2.1854) -- (1.5470, 2.5070, 2.1852) -- (1.5470, 2.5600, 2.1798) -- (1.5000, 2.5600, 2.1800) -- cycle;
\fill[blue!33.8, opacity=0.7] (1.5000, 2.5600, 2.1800) -- (1.5470, 2.5600, 2.1798) -- (1.5470, 2.6130, 2.1743) -- (1.5000, 2.6130, 2.1745) -- cycle;
\fill[blue!72.7, opacity=0.7] (1.5000, 2.6130, 2.1745) -- (1.5470, 2.6130, 2.1743) -- (1.5470, 2.6660, 2.1686) -- (1.5000, 2.6660, 2.1688) -- cycle;
\fill[blue!87.8, opacity=0.7] (1.5000, 2.6660, 2.1688) -- (1.5470, 2.6660, 2.1686) -- (1.5470, 2.7190, 2.1628) -- (1.5000, 2.7190, 2.1630) -- cycle;
\fill[blue!87.5, opacity=0.7] (1.5000, 2.7190, 2.1630) -- (1.5470, 2.7190, 2.1628) -- (1.5470, 2.7720, 2.1569) -- (1.5000, 2.7720, 2.1571) -- cycle;
\fill[blue!61.4, opacity=0.7] (1.5000, 2.7720, 2.1571) -- (1.5470, 2.7720, 2.1569) -- (1.5470, 2.8250, 2.1509) -- (1.5000, 2.8250, 2.1511) -- cycle;
\fill[blue!21.4, opacity=0.7] (1.5000, 2.8250, 2.1511) -- (1.5470, 2.8250, 2.1509) -- (1.5470, 2.8780, 2.1448) -- (1.5000, 2.8780, 2.1449) -- cycle;
\fill[blue!15.4, opacity=0.7] (1.5000, 2.8780, 2.1449) -- (1.5470, 2.8780, 2.1448) -- (1.5470, 2.9310, 2.1386) -- (1.5000, 2.9310, 2.1388) -- cycle;
\fill[blue!15.1, opacity=0.7] (1.5000, 2.9310, 2.1388) -- (1.5470, 2.9310, 2.1386) -- (1.5470, 2.9840, 2.1324) -- (1.5000, 2.9840, 2.1325) -- cycle;
\fill[blue!15.4, opacity=0.7] (1.5000, 2.9840, 2.1325) -- (1.5470, 2.9840, 2.1324) -- (1.5470, 3.0370, 2.1261) -- (1.5000, 3.0370, 2.1263) -- cycle;
\fill[blue!20.2, opacity=0.7] (1.5000, 3.0370, 2.1263) -- (1.5470, 3.0370, 2.1261) -- (1.5470, 3.0900, 2.1198) -- (1.5000, 3.0900, 2.1200) -- cycle;
\fill[blue!23.7, opacity=0.7] (1.5470, -0.0900, 2.1198) -- (1.5940, -0.0900, 2.1193) -- (1.5940, -0.0370, 2.1256) -- (1.5470, -0.0370, 2.1261) -- cycle;
\fill[blue!15.8, opacity=0.7] (1.5470, -0.0370, 2.1261) -- (1.5940, -0.0370, 2.1256) -- (1.5940, 0.0160, 2.1319) -- (1.5470, 0.0160, 2.1324) -- cycle;
\fill[blue!15.1, opacity=0.7] (1.5470, 0.0160, 2.1324) -- (1.5940, 0.0160, 2.1319) -- (1.5940, 0.0690, 2.1381) -- (1.5470, 0.0690, 2.1386) -- cycle;
\fill[blue!15.2, opacity=0.7] (1.5470, 0.0690, 2.1386) -- (1.5940, 0.0690, 2.1381) -- (1.5940, 0.1220, 2.1443) -- (1.5470, 0.1220, 2.1448) -- cycle;
\fill[blue!18.4, opacity=0.7] (1.5470, 0.1220, 2.1448) -- (1.5940, 0.1220, 2.1443) -- (1.5940, 0.1750, 2.1504) -- (1.5470, 0.1750, 2.1509) -- cycle;
\fill[blue!49.4, opacity=0.7] (1.5470, 0.1750, 2.1509) -- (1.5940, 0.1750, 2.1504) -- (1.5940, 0.2280, 2.1564) -- (1.5470, 0.2280, 2.1569) -- cycle;
\fill[blue!86.0, opacity=0.7] (1.5470, 0.2280, 2.1569) -- (1.5940, 0.2280, 2.1564) -- (1.5940, 0.2810, 2.1623) -- (1.5470, 0.2810, 2.1628) -- cycle;
\fill[blue!87.6, opacity=0.7] (1.5470, 0.2810, 2.1628) -- (1.5940, 0.2810, 2.1623) -- (1.5940, 0.3340, 2.1682) -- (1.5470, 0.3340, 2.1686) -- cycle;
\fill[blue!82.4, opacity=0.7] (1.5470, 0.3340, 2.1686) -- (1.5940, 0.3340, 2.1682) -- (1.5940, 0.3870, 2.1738) -- (1.5470, 0.3870, 2.1743) -- cycle;
\fill[blue!45.7, opacity=0.7] (1.5470, 0.3870, 2.1743) -- (1.5940, 0.3870, 2.1738) -- (1.5940, 0.4400, 2.1793) -- (1.5470, 0.4400, 2.1798) -- cycle;
\fill[blue!22.8, opacity=0.7] (1.5470, 0.4400, 2.1798) -- (1.5940, 0.4400, 2.1793) -- (1.5940, 0.4930, 2.1847) -- (1.5470, 0.4930, 2.1852) -- cycle;
\fill[blue!20.1, opacity=0.7] (1.5470, 0.4930, 2.1852) -- (1.5940, 0.4930, 2.1847) -- (1.5940, 0.5460, 2.1899) -- (1.5470, 0.5460, 2.1904) -- cycle;
\fill[blue!32.1, opacity=0.7] (1.5470, 0.5460, 2.1904) -- (1.5940, 0.5460, 2.1899) -- (1.5940, 0.5990, 2.1949) -- (1.5470, 0.5990, 2.1954) -- cycle;
\fill[blue!77.6, opacity=0.7] (1.5470, 0.5990, 2.1954) -- (1.5940, 0.5990, 2.1949) -- (1.5940, 0.6520, 2.1996) -- (1.5470, 0.6520, 2.2001) -- cycle;
\fill[blue!74.7, opacity=0.7] (1.5470, 0.6520, 2.2001) -- (1.5940, 0.6520, 2.1996) -- (1.5940, 0.7050, 2.2042) -- (1.5470, 0.7050, 2.2047) -- cycle;
\fill[blue!45.8, opacity=0.7] (1.5470, 0.7050, 2.2047) -- (1.5940, 0.7050, 2.2042) -- (1.5940, 0.7580, 2.2085) -- (1.5470, 0.7580, 2.2090) -- cycle;
\fill[blue!49.5, opacity=0.7] (1.5470, 0.7580, 2.2090) -- (1.5940, 0.7580, 2.2085) -- (1.5940, 0.8110, 2.2126) -- (1.5470, 0.8110, 2.2131) -- cycle;
\fill[blue!79.4, opacity=0.7] (1.5470, 0.8110, 2.2131) -- (1.5940, 0.8110, 2.2126) -- (1.5940, 0.8640, 2.2164) -- (1.5470, 0.8640, 2.2169) -- cycle;
\fill[blue!81.7, opacity=0.7] (1.5470, 0.8640, 2.2169) -- (1.5940, 0.8640, 2.2164) -- (1.5940, 0.9170, 2.2200) -- (1.5470, 0.9170, 2.2205) -- cycle;
\fill[blue!59.6, opacity=0.7] (1.5470, 0.9170, 2.2205) -- (1.5940, 0.9170, 2.2200) -- (1.5940, 0.9700, 2.2233) -- (1.5470, 0.9700, 2.2238) -- cycle;
\fill[blue!64.8, opacity=0.7] (1.5470, 0.9700, 2.2238) -- (1.5940, 0.9700, 2.2233) -- (1.5940, 1.0230, 2.2263) -- (1.5470, 1.0230, 2.2268) -- cycle;
\fill[blue!87.7, opacity=0.7] (1.5470, 1.0230, 2.2268) -- (1.5940, 1.0230, 2.2263) -- (1.5940, 1.0760, 2.2290) -- (1.5470, 1.0760, 2.2295) -- cycle;
\fill[blue!55.2, opacity=0.7] (1.5470, 1.0760, 2.2295) -- (1.5940, 1.0760, 2.2290) -- (1.5940, 1.1290, 2.2314) -- (1.5470, 1.1290, 2.2319) -- cycle;
\fill[blue!22.6, opacity=0.7] (1.5470, 1.1290, 2.2319) -- (1.5940, 1.1290, 2.2314) -- (1.5940, 1.1820, 2.2335) -- (1.5470, 1.1820, 2.2340) -- cycle;
\fill[blue!17.1, opacity=0.7] (1.5470, 1.1820, 2.2340) -- (1.5940, 1.1820, 2.2335) -- (1.5940, 1.2350, 2.2353) -- (1.5470, 1.2350, 2.2357) -- cycle;
\fill[blue!17.3, opacity=0.7] (1.5470, 1.2350, 2.2357) -- (1.5940, 1.2350, 2.2353) -- (1.5940, 1.2880, 2.2367) -- (1.5470, 1.2880, 2.2372) -- cycle;
\fill[blue!21.5, opacity=0.7] (1.5470, 1.2880, 2.2372) -- (1.5940, 1.2880, 2.2367) -- (1.5940, 1.3410, 2.2379) -- (1.5470, 1.3410, 2.2384) -- cycle;
\fill[blue!30.2, opacity=0.7] (1.5470, 1.3410, 2.2384) -- (1.5940, 1.3410, 2.2379) -- (1.5940, 1.3940, 2.2387) -- (1.5470, 1.3940, 2.2392) -- cycle;
\fill[blue!31.3, opacity=0.7] (1.5470, 1.3940, 2.2392) -- (1.5940, 1.3940, 2.2387) -- (1.5940, 1.4470, 2.2392) -- (1.5470, 1.4470, 2.2397) -- cycle;
\fill[blue!17.7, opacity=0.7] (1.5470, 1.4470, 2.2397) -- (1.5940, 1.4470, 2.2392) -- (1.5940, 1.5000, 2.2393) -- (1.5470, 1.5000, 2.2398) -- cycle;
\fill[blue!15.4, opacity=0.7] (1.5470, 1.5000, 2.2398) -- (1.5940, 1.5000, 2.2393) -- (1.5940, 1.5530, 2.2392) -- (1.5470, 1.5530, 2.2397) -- cycle;
\fill[blue!33.1, opacity=0.7] (1.5470, 1.5530, 2.2397) -- (1.5940, 1.5530, 2.2392) -- (1.5940, 1.6060, 2.2387) -- (1.5470, 1.6060, 2.2392) -- cycle;
\fill[blue!86.0, opacity=0.7] (1.5470, 1.6060, 2.2392) -- (1.5940, 1.6060, 2.2387) -- (1.5940, 1.6590, 2.2379) -- (1.5470, 1.6590, 2.2384) -- cycle;
\fill[blue!87.5, opacity=0.7] (1.5470, 1.6590, 2.2384) -- (1.5940, 1.6590, 2.2379) -- (1.5940, 1.7120, 2.2367) -- (1.5470, 1.7120, 2.2372) -- cycle;
\fill[blue!61.0, opacity=0.7] (1.5470, 1.7120, 2.2372) -- (1.5940, 1.7120, 2.2367) -- (1.5940, 1.7650, 2.2353) -- (1.5470, 1.7650, 2.2357) -- cycle;
\fill[blue!24.6, opacity=0.7] (1.5470, 1.7650, 2.2357) -- (1.5940, 1.7650, 2.2353) -- (1.5940, 1.8180, 2.2335) -- (1.5470, 1.8180, 2.2340) -- cycle;
\fill[blue!18.2, opacity=0.7] (1.5470, 1.8180, 2.2340) -- (1.5940, 1.8180, 2.2335) -- (1.5940, 1.8710, 2.2314) -- (1.5470, 1.8710, 2.2319) -- cycle;
\fill[blue!23.3, opacity=0.7] (1.5470, 1.8710, 2.2319) -- (1.5940, 1.8710, 2.2314) -- (1.5940, 1.9240, 2.2290) -- (1.5470, 1.9240, 2.2295) -- cycle;
\fill[blue!61.3, opacity=0.7] (1.5470, 1.9240, 2.2295) -- (1.5940, 1.9240, 2.2290) -- (1.5940, 1.9770, 2.2263) -- (1.5470, 1.9770, 2.2268) -- cycle;
\fill[blue!84.4, opacity=0.7] (1.5470, 1.9770, 2.2268) -- (1.5940, 1.9770, 2.2263) -- (1.5940, 2.0300, 2.2233) -- (1.5470, 2.0300, 2.2238) -- cycle;
\fill[blue!55.8, opacity=0.7] (1.5470, 2.0300, 2.2238) -- (1.5940, 2.0300, 2.2233) -- (1.5940, 2.0830, 2.2200) -- (1.5470, 2.0830, 2.2205) -- cycle;
\fill[blue!59.2, opacity=0.7] (1.5470, 2.0830, 2.2205) -- (1.5940, 2.0830, 2.2200) -- (1.5940, 2.1360, 2.2164) -- (1.5470, 2.1360, 2.2169) -- cycle;
\fill[blue!86.1, opacity=0.7] (1.5470, 2.1360, 2.2169) -- (1.5940, 2.1360, 2.2164) -- (1.5940, 2.1890, 2.2126) -- (1.5470, 2.1890, 2.2131) -- cycle;
\fill[blue!70.3, opacity=0.7] (1.5470, 2.1890, 2.2131) -- (1.5940, 2.1890, 2.2126) -- (1.5940, 2.2420, 2.2085) -- (1.5470, 2.2420, 2.2090) -- cycle;
\fill[blue!47.4, opacity=0.7] (1.5470, 2.2420, 2.2090) -- (1.5940, 2.2420, 2.2085) -- (1.5940, 2.2950, 2.2042) -- (1.5470, 2.2950, 2.2047) -- cycle;
\fill[blue!59.3, opacity=0.7] (1.5470, 2.2950, 2.2047) -- (1.5940, 2.2950, 2.2042) -- (1.5940, 2.3480, 2.1996) -- (1.5470, 2.3480, 2.2001) -- cycle;
\fill[blue!87.9, opacity=0.7] (1.5470, 2.3480, 2.2001) -- (1.5940, 2.3480, 2.1996) -- (1.5940, 2.4010, 2.1949) -- (1.5470, 2.4010, 2.1954) -- cycle;
\fill[blue!47.1, opacity=0.7] (1.5470, 2.4010, 2.1954) -- (1.5940, 2.4010, 2.1949) -- (1.5940, 2.4540, 2.1899) -- (1.5470, 2.4540, 2.1904) -- cycle;
\fill[blue!21.5, opacity=0.7] (1.5470, 2.4540, 2.1904) -- (1.5940, 2.4540, 2.1899) -- (1.5940, 2.5070, 2.1847) -- (1.5470, 2.5070, 2.1852) -- cycle;
\fill[blue!19.6, opacity=0.7] (1.5470, 2.5070, 2.1852) -- (1.5940, 2.5070, 2.1847) -- (1.5940, 2.5600, 2.1793) -- (1.5470, 2.5600, 2.1798) -- cycle;
\fill[blue!31.1, opacity=0.7] (1.5470, 2.5600, 2.1798) -- (1.5940, 2.5600, 2.1793) -- (1.5940, 2.6130, 2.1738) -- (1.5470, 2.6130, 2.1743) -- cycle;
\fill[blue!69.0, opacity=0.7] (1.5470, 2.6130, 2.1743) -- (1.5940, 2.6130, 2.1738) -- (1.5940, 2.6660, 2.1682) -- (1.5470, 2.6660, 2.1686) -- cycle;
\fill[blue!87.7, opacity=0.7] (1.5470, 2.6660, 2.1686) -- (1.5940, 2.6660, 2.1682) -- (1.5940, 2.7190, 2.1623) -- (1.5470, 2.7190, 2.1628) -- cycle;
\fill[blue!87.6, opacity=0.7] (1.5470, 2.7190, 2.1628) -- (1.5940, 2.7190, 2.1623) -- (1.5940, 2.7720, 2.1564) -- (1.5470, 2.7720, 2.1569) -- cycle;
\fill[blue!64.0, opacity=0.7] (1.5470, 2.7720, 2.1569) -- (1.5940, 2.7720, 2.1564) -- (1.5940, 2.8250, 2.1504) -- (1.5470, 2.8250, 2.1509) -- cycle;
\fill[blue!22.4, opacity=0.7] (1.5470, 2.8250, 2.1509) -- (1.5940, 2.8250, 2.1504) -- (1.5940, 2.8780, 2.1443) -- (1.5470, 2.8780, 2.1448) -- cycle;
\fill[blue!15.4, opacity=0.7] (1.5470, 2.8780, 2.1448) -- (1.5940, 2.8780, 2.1443) -- (1.5940, 2.9310, 2.1381) -- (1.5470, 2.9310, 2.1386) -- cycle;
\fill[blue!15.1, opacity=0.7] (1.5470, 2.9310, 2.1386) -- (1.5940, 2.9310, 2.1381) -- (1.5940, 2.9840, 2.1319) -- (1.5470, 2.9840, 2.1324) -- cycle;
\fill[blue!15.3, opacity=0.7] (1.5470, 2.9840, 2.1324) -- (1.5940, 2.9840, 2.1319) -- (1.5940, 3.0370, 2.1256) -- (1.5470, 3.0370, 2.1261) -- cycle;
\fill[blue!19.5, opacity=0.7] (1.5470, 3.0370, 2.1261) -- (1.5940, 3.0370, 2.1256) -- (1.5940, 3.0900, 2.1193) -- (1.5470, 3.0900, 2.1198) -- cycle;
\fill[blue!27.0, opacity=0.7] (1.5940, -0.0900, 2.1193) -- (1.6410, -0.0900, 2.1185) -- (1.6410, -0.0370, 2.1248) -- (1.5940, -0.0370, 2.1256) -- cycle;
\fill[blue!16.3, opacity=0.7] (1.5940, -0.0370, 2.1256) -- (1.6410, -0.0370, 2.1248) -- (1.6410, 0.0160, 2.1311) -- (1.5940, 0.0160, 2.1319) -- cycle;
\fill[blue!15.2, opacity=0.7] (1.5940, 0.0160, 2.1319) -- (1.6410, 0.0160, 2.1311) -- (1.6410, 0.0690, 2.1373) -- (1.5940, 0.0690, 2.1381) -- cycle;
\fill[blue!15.2, opacity=0.7] (1.5940, 0.0690, 2.1381) -- (1.6410, 0.0690, 2.1373) -- (1.6410, 0.1220, 2.1435) -- (1.5940, 0.1220, 2.1443) -- cycle;
\fill[blue!17.1, opacity=0.7] (1.5940, 0.1220, 2.1443) -- (1.6410, 0.1220, 2.1435) -- (1.6410, 0.1750, 2.1496) -- (1.5940, 0.1750, 2.1504) -- cycle;
\fill[blue!40.6, opacity=0.7] (1.5940, 0.1750, 2.1504) -- (1.6410, 0.1750, 2.1496) -- (1.6410, 0.2280, 2.1556) -- (1.5940, 0.2280, 2.1564) -- cycle;
\fill[blue!83.3, opacity=0.7] (1.5940, 0.2280, 2.1564) -- (1.6410, 0.2280, 2.1556) -- (1.6410, 0.2810, 2.1615) -- (1.5940, 0.2810, 2.1623) -- cycle;
\fill[blue!87.5, opacity=0.7] (1.5940, 0.2810, 2.1623) -- (1.6410, 0.2810, 2.1615) -- (1.6410, 0.3340, 2.1673) -- (1.5940, 0.3340, 2.1682) -- cycle;
\fill[blue!86.1, opacity=0.7] (1.5940, 0.3340, 2.1682) -- (1.6410, 0.3340, 2.1673) -- (1.6410, 0.3870, 2.1730) -- (1.5940, 0.3870, 2.1738) -- cycle;
\fill[blue!56.0, opacity=0.7] (1.5940, 0.3870, 2.1738) -- (1.6410, 0.3870, 2.1730) -- (1.6410, 0.4400, 2.1785) -- (1.5940, 0.4400, 2.1793) -- cycle;
\fill[blue!26.0, opacity=0.7] (1.5940, 0.4400, 2.1793) -- (1.6410, 0.4400, 2.1785) -- (1.6410, 0.4930, 2.1839) -- (1.5940, 0.4930, 2.1847) -- cycle;
\fill[blue!20.1, opacity=0.7] (1.5940, 0.4930, 2.1847) -- (1.6410, 0.4930, 2.1839) -- (1.6410, 0.5460, 2.1891) -- (1.5940, 0.5460, 2.1899) -- cycle;
\fill[blue!26.7, opacity=0.7] (1.5940, 0.5460, 2.1899) -- (1.6410, 0.5460, 2.1891) -- (1.6410, 0.5990, 2.1940) -- (1.5940, 0.5990, 2.1949) -- cycle;
\fill[blue!64.3, opacity=0.7] (1.5940, 0.5990, 2.1949) -- (1.6410, 0.5990, 2.1940) -- (1.6410, 0.6520, 2.1988) -- (1.5940, 0.6520, 2.1996) -- cycle;
\fill[blue!84.0, opacity=0.7] (1.5940, 0.6520, 2.1996) -- (1.6410, 0.6520, 2.1988) -- (1.6410, 0.7050, 2.2034) -- (1.5940, 0.7050, 2.2042) -- cycle;
\fill[blue!50.6, opacity=0.7] (1.5940, 0.7050, 2.2042) -- (1.6410, 0.7050, 2.2034) -- (1.6410, 0.7580, 2.2077) -- (1.5940, 0.7580, 2.2085) -- cycle;
\fill[blue!43.4, opacity=0.7] (1.5940, 0.7580, 2.2085) -- (1.6410, 0.7580, 2.2077) -- (1.6410, 0.8110, 2.2118) -- (1.5940, 0.8110, 2.2126) -- cycle;
\fill[blue!65.5, opacity=0.7] (1.5940, 0.8110, 2.2126) -- (1.6410, 0.8110, 2.2118) -- (1.6410, 0.8640, 2.2156) -- (1.5940, 0.8640, 2.2164) -- cycle;
\fill[blue!87.8, opacity=0.7] (1.5940, 0.8640, 2.2164) -- (1.6410, 0.8640, 2.2156) -- (1.6410, 0.9170, 2.2192) -- (1.5940, 0.9170, 2.2200) -- cycle;
\fill[blue!69.3, opacity=0.7] (1.5940, 0.9170, 2.2200) -- (1.6410, 0.9170, 2.2192) -- (1.6410, 0.9700, 2.2224) -- (1.5940, 0.9700, 2.2233) -- cycle;
\fill[blue!60.0, opacity=0.7] (1.5940, 0.9700, 2.2233) -- (1.6410, 0.9700, 2.2224) -- (1.6410, 1.0230, 2.2254) -- (1.5940, 1.0230, 2.2263) -- cycle;
\fill[blue!78.0, opacity=0.7] (1.5940, 1.0230, 2.2263) -- (1.6410, 1.0230, 2.2254) -- (1.6410, 1.0760, 2.2281) -- (1.5940, 1.0760, 2.2290) -- cycle;
\fill[blue!82.5, opacity=0.7] (1.5940, 1.0760, 2.2290) -- (1.6410, 1.0760, 2.2281) -- (1.6410, 1.1290, 2.2306) -- (1.5940, 1.1290, 2.2314) -- cycle;
\fill[blue!40.5, opacity=0.7] (1.5940, 1.1290, 2.2314) -- (1.6410, 1.1290, 2.2306) -- (1.6410, 1.1820, 2.2326) -- (1.5940, 1.1820, 2.2335) -- cycle;
\fill[blue!20.3, opacity=0.7] (1.5940, 1.1820, 2.2335) -- (1.6410, 1.1820, 2.2326) -- (1.6410, 1.2350, 2.2344) -- (1.5940, 1.2350, 2.2353) -- cycle;
\fill[blue!16.7, opacity=0.7] (1.5940, 1.2350, 2.2353) -- (1.6410, 1.2350, 2.2344) -- (1.6410, 1.2880, 2.2359) -- (1.5940, 1.2880, 2.2367) -- cycle;
\fill[blue!16.3, opacity=0.7] (1.5940, 1.2880, 2.2367) -- (1.6410, 1.2880, 2.2359) -- (1.6410, 1.3410, 2.2370) -- (1.5940, 1.3410, 2.2379) -- cycle;
\fill[blue!16.4, opacity=0.7] (1.5940, 1.3410, 2.2379) -- (1.6410, 1.3410, 2.2370) -- (1.6410, 1.3940, 2.2379) -- (1.5940, 1.3940, 2.2387) -- cycle;
\fill[blue!16.0, opacity=0.7] (1.5940, 1.3940, 2.2387) -- (1.6410, 1.3940, 2.2379) -- (1.6410, 1.4470, 2.2384) -- (1.5940, 1.4470, 2.2392) -- cycle;
\fill[blue!15.5, opacity=0.7] (1.5940, 1.4470, 2.2392) -- (1.6410, 1.4470, 2.2384) -- (1.6410, 1.5000, 2.2385) -- (1.5940, 1.5000, 2.2393) -- cycle;
\fill[blue!16.6, opacity=0.7] (1.5940, 1.5000, 2.2393) -- (1.6410, 1.5000, 2.2385) -- (1.6410, 1.5530, 2.2384) -- (1.5940, 1.5530, 2.2392) -- cycle;
\fill[blue!42.8, opacity=0.7] (1.5940, 1.5530, 2.2392) -- (1.6410, 1.5530, 2.2384) -- (1.6410, 1.6060, 2.2379) -- (1.5940, 1.6060, 2.2387) -- cycle;
\fill[blue!87.1, opacity=0.7] (1.5940, 1.6060, 2.2387) -- (1.6410, 1.6060, 2.2379) -- (1.6410, 1.6590, 2.2370) -- (1.5940, 1.6590, 2.2379) -- cycle;
\fill[blue!87.8, opacity=0.7] (1.5940, 1.6590, 2.2379) -- (1.6410, 1.6590, 2.2370) -- (1.6410, 1.7120, 2.2359) -- (1.5940, 1.7120, 2.2367) -- cycle;
\fill[blue!67.9, opacity=0.7] (1.5940, 1.7120, 2.2367) -- (1.6410, 1.7120, 2.2359) -- (1.6410, 1.7650, 2.2344) -- (1.5940, 1.7650, 2.2353) -- cycle;
\fill[blue!27.5, opacity=0.7] (1.5940, 1.7650, 2.2353) -- (1.6410, 1.7650, 2.2344) -- (1.6410, 1.8180, 2.2326) -- (1.5940, 1.8180, 2.2335) -- cycle;
\fill[blue!18.7, opacity=0.7] (1.5940, 1.8180, 2.2335) -- (1.6410, 1.8180, 2.2326) -- (1.6410, 1.8710, 2.2306) -- (1.5940, 1.8710, 2.2314) -- cycle;
\fill[blue!23.1, opacity=0.7] (1.5940, 1.8710, 2.2314) -- (1.6410, 1.8710, 2.2306) -- (1.6410, 1.9240, 2.2281) -- (1.5940, 1.9240, 2.2290) -- cycle;
\fill[blue!59.5, opacity=0.7] (1.5940, 1.9240, 2.2290) -- (1.6410, 1.9240, 2.2281) -- (1.6410, 1.9770, 2.2254) -- (1.5940, 1.9770, 2.2263) -- cycle;
\fill[blue!84.9, opacity=0.7] (1.5940, 1.9770, 2.2263) -- (1.6410, 1.9770, 2.2254) -- (1.6410, 2.0300, 2.2224) -- (1.5940, 2.0300, 2.2233) -- cycle;
\fill[blue!55.1, opacity=0.7] (1.5940, 2.0300, 2.2233) -- (1.6410, 2.0300, 2.2224) -- (1.6410, 2.0830, 2.2192) -- (1.5940, 2.0830, 2.2200) -- cycle;
\fill[blue!56.6, opacity=0.7] (1.5940, 2.0830, 2.2200) -- (1.6410, 2.0830, 2.2192) -- (1.6410, 2.1360, 2.2156) -- (1.5940, 2.1360, 2.2164) -- cycle;
\fill[blue!84.7, opacity=0.7] (1.5940, 2.1360, 2.2164) -- (1.6410, 2.1360, 2.2156) -- (1.6410, 2.1890, 2.2118) -- (1.5940, 2.1890, 2.2126) -- cycle;
\fill[blue!73.0, opacity=0.7] (1.5940, 2.1890, 2.2126) -- (1.6410, 2.1890, 2.2118) -- (1.6410, 2.2420, 2.2077) -- (1.5940, 2.2420, 2.2085) -- cycle;
\fill[blue!48.9, opacity=0.7] (1.5940, 2.2420, 2.2085) -- (1.6410, 2.2420, 2.2077) -- (1.6410, 2.2950, 2.2034) -- (1.5940, 2.2950, 2.2042) -- cycle;
\fill[blue!59.2, opacity=0.7] (1.5940, 2.2950, 2.2042) -- (1.6410, 2.2950, 2.2034) -- (1.6410, 2.3480, 2.1988) -- (1.5940, 2.3480, 2.1996) -- cycle;
\fill[blue!87.9, opacity=0.7] (1.5940, 2.3480, 2.1996) -- (1.6410, 2.3480, 2.1988) -- (1.6410, 2.4010, 2.1940) -- (1.5940, 2.4010, 2.1949) -- cycle;
\fill[blue!48.0, opacity=0.7] (1.5940, 2.4010, 2.1949) -- (1.6410, 2.4010, 2.1940) -- (1.6410, 2.4540, 2.1891) -- (1.5940, 2.4540, 2.1899) -- cycle;
\fill[blue!21.5, opacity=0.7] (1.5940, 2.4540, 2.1899) -- (1.6410, 2.4540, 2.1891) -- (1.6410, 2.5070, 2.1839) -- (1.5940, 2.5070, 2.1847) -- cycle;
\fill[blue!19.3, opacity=0.7] (1.5940, 2.5070, 2.1847) -- (1.6410, 2.5070, 2.1839) -- (1.6410, 2.5600, 2.1785) -- (1.5940, 2.5600, 2.1793) -- cycle;
\fill[blue!29.8, opacity=0.7] (1.5940, 2.5600, 2.1793) -- (1.6410, 2.5600, 2.1785) -- (1.6410, 2.6130, 2.1730) -- (1.5940, 2.6130, 2.1738) -- cycle;
\fill[blue!67.1, opacity=0.7] (1.5940, 2.6130, 2.1738) -- (1.6410, 2.6130, 2.1730) -- (1.6410, 2.6660, 2.1673) -- (1.5940, 2.6660, 2.1682) -- cycle;
\fill[blue!87.5, opacity=0.7] (1.5940, 2.6660, 2.1682) -- (1.6410, 2.6660, 2.1673) -- (1.6410, 2.7190, 2.1615) -- (1.5940, 2.7190, 2.1623) -- cycle;
\fill[blue!87.6, opacity=0.7] (1.5940, 2.7190, 2.1623) -- (1.6410, 2.7190, 2.1615) -- (1.6410, 2.7720, 2.1556) -- (1.5940, 2.7720, 2.1564) -- cycle;
\fill[blue!64.6, opacity=0.7] (1.5940, 2.7720, 2.1564) -- (1.6410, 2.7720, 2.1556) -- (1.6410, 2.8250, 2.1496) -- (1.5940, 2.8250, 2.1504) -- cycle;
\fill[blue!22.7, opacity=0.7] (1.5940, 2.8250, 2.1504) -- (1.6410, 2.8250, 2.1496) -- (1.6410, 2.8780, 2.1435) -- (1.5940, 2.8780, 2.1443) -- cycle;
\fill[blue!15.4, opacity=0.7] (1.5940, 2.8780, 2.1443) -- (1.6410, 2.8780, 2.1435) -- (1.6410, 2.9310, 2.1373) -- (1.5940, 2.9310, 2.1381) -- cycle;
\fill[blue!15.1, opacity=0.7] (1.5940, 2.9310, 2.1381) -- (1.6410, 2.9310, 2.1373) -- (1.6410, 2.9840, 2.1311) -- (1.5940, 2.9840, 2.1319) -- cycle;
\fill[blue!15.3, opacity=0.7] (1.5940, 2.9840, 2.1319) -- (1.6410, 2.9840, 2.1311) -- (1.6410, 3.0370, 2.1248) -- (1.5940, 3.0370, 2.1256) -- cycle;
\fill[blue!19.1, opacity=0.7] (1.5940, 3.0370, 2.1256) -- (1.6410, 3.0370, 2.1248) -- (1.6410, 3.0900, 2.1185) -- (1.5940, 3.0900, 2.1193) -- cycle;
\fill[blue!31.6, opacity=0.7] (1.6410, -0.0900, 2.1185) -- (1.6880, -0.0900, 2.1174) -- (1.6880, -0.0370, 2.1237) -- (1.6410, -0.0370, 2.1248) -- cycle;
\fill[blue!17.2, opacity=0.7] (1.6410, -0.0370, 2.1248) -- (1.6880, -0.0370, 2.1237) -- (1.6880, 0.0160, 2.1299) -- (1.6410, 0.0160, 2.1311) -- cycle;
\fill[blue!15.2, opacity=0.7] (1.6410, 0.0160, 2.1311) -- (1.6880, 0.0160, 2.1299) -- (1.6880, 0.0690, 2.1361) -- (1.6410, 0.0690, 2.1373) -- cycle;
\fill[blue!15.1, opacity=0.7] (1.6410, 0.0690, 2.1373) -- (1.6880, 0.0690, 2.1361) -- (1.6880, 0.1220, 2.1423) -- (1.6410, 0.1220, 2.1435) -- cycle;
\fill[blue!16.1, opacity=0.7] (1.6410, 0.1220, 2.1435) -- (1.6880, 0.1220, 2.1423) -- (1.6880, 0.1750, 2.1484) -- (1.6410, 0.1750, 2.1496) -- cycle;
\fill[blue!31.5, opacity=0.7] (1.6410, 0.1750, 2.1496) -- (1.6880, 0.1750, 2.1484) -- (1.6880, 0.2280, 2.1545) -- (1.6410, 0.2280, 2.1556) -- cycle;
\fill[blue!77.1, opacity=0.7] (1.6410, 0.2280, 2.1556) -- (1.6880, 0.2280, 2.1545) -- (1.6880, 0.2810, 2.1604) -- (1.6410, 0.2810, 2.1615) -- cycle;
\fill[blue!87.6, opacity=0.7] (1.6410, 0.2810, 2.1615) -- (1.6880, 0.2810, 2.1604) -- (1.6880, 0.3340, 2.1662) -- (1.6410, 0.3340, 2.1673) -- cycle;
\fill[blue!87.8, opacity=0.7] (1.6410, 0.3340, 2.1673) -- (1.6880, 0.3340, 2.1662) -- (1.6880, 0.3870, 2.1719) -- (1.6410, 0.3870, 2.1730) -- cycle;
\fill[blue!68.4, opacity=0.7] (1.6410, 0.3870, 2.1730) -- (1.6880, 0.3870, 2.1719) -- (1.6880, 0.4400, 2.1774) -- (1.6410, 0.4400, 2.1785) -- cycle;
\fill[blue!32.2, opacity=0.7] (1.6410, 0.4400, 2.1785) -- (1.6880, 0.4400, 2.1774) -- (1.6880, 0.4930, 2.1827) -- (1.6410, 0.4930, 2.1839) -- cycle;
\fill[blue!20.8, opacity=0.7] (1.6410, 0.4930, 2.1839) -- (1.6880, 0.4930, 2.1827) -- (1.6880, 0.5460, 2.1879) -- (1.6410, 0.5460, 2.1891) -- cycle;
\fill[blue!23.0, opacity=0.7] (1.6410, 0.5460, 2.1891) -- (1.6880, 0.5460, 2.1879) -- (1.6880, 0.5990, 2.1929) -- (1.6410, 0.5990, 2.1940) -- cycle;
\fill[blue!47.8, opacity=0.7] (1.6410, 0.5990, 2.1940) -- (1.6880, 0.5990, 2.1929) -- (1.6880, 0.6520, 2.1977) -- (1.6410, 0.6520, 2.1988) -- cycle;
\fill[blue!87.6, opacity=0.7] (1.6410, 0.6520, 2.1988) -- (1.6880, 0.6520, 2.1977) -- (1.6880, 0.7050, 2.2022) -- (1.6410, 0.7050, 2.2034) -- cycle;
\fill[blue!61.3, opacity=0.7] (1.6410, 0.7050, 2.2034) -- (1.6880, 0.7050, 2.2022) -- (1.6880, 0.7580, 2.2066) -- (1.6410, 0.7580, 2.2077) -- cycle;
\fill[blue!41.5, opacity=0.7] (1.6410, 0.7580, 2.2077) -- (1.6880, 0.7580, 2.2066) -- (1.6880, 0.8110, 2.2106) -- (1.6410, 0.8110, 2.2118) -- cycle;
\fill[blue!50.5, opacity=0.7] (1.6410, 0.8110, 2.2118) -- (1.6880, 0.8110, 2.2106) -- (1.6880, 0.8640, 2.2145) -- (1.6410, 0.8640, 2.2156) -- cycle;
\fill[blue!80.2, opacity=0.7] (1.6410, 0.8640, 2.2156) -- (1.6880, 0.8640, 2.2145) -- (1.6880, 0.9170, 2.2180) -- (1.6410, 0.9170, 2.2192) -- cycle;
\fill[blue!83.4, opacity=0.7] (1.6410, 0.9170, 2.2192) -- (1.6880, 0.9170, 2.2180) -- (1.6880, 0.9700, 2.2213) -- (1.6410, 0.9700, 2.2224) -- cycle;
\fill[blue!64.4, opacity=0.7] (1.6410, 0.9700, 2.2224) -- (1.6880, 0.9700, 2.2213) -- (1.6880, 1.0230, 2.2243) -- (1.6410, 1.0230, 2.2254) -- cycle;
\fill[blue!65.2, opacity=0.7] (1.6410, 1.0230, 2.2254) -- (1.6880, 1.0230, 2.2243) -- (1.6880, 1.0760, 2.2270) -- (1.6410, 1.0760, 2.2281) -- cycle;
\fill[blue!84.3, opacity=0.7] (1.6410, 1.0760, 2.2281) -- (1.6880, 1.0760, 2.2270) -- (1.6880, 1.1290, 2.2294) -- (1.6410, 1.1290, 2.2306) -- cycle;
\fill[blue!78.0, opacity=0.7] (1.6410, 1.1290, 2.2306) -- (1.6880, 1.1290, 2.2294) -- (1.6880, 1.1820, 2.2315) -- (1.6410, 1.1820, 2.2326) -- cycle;
\fill[blue!41.5, opacity=0.7] (1.6410, 1.1820, 2.2326) -- (1.6880, 1.1820, 2.2315) -- (1.6880, 1.2350, 2.2333) -- (1.6410, 1.2350, 2.2344) -- cycle;
\fill[blue!22.8, opacity=0.7] (1.6410, 1.2350, 2.2344) -- (1.6880, 1.2350, 2.2333) -- (1.6880, 1.2880, 2.2348) -- (1.6410, 1.2880, 2.2359) -- cycle;
\fill[blue!17.9, opacity=0.7] (1.6410, 1.2880, 2.2359) -- (1.6880, 1.2880, 2.2348) -- (1.6880, 1.3410, 2.2359) -- (1.6410, 1.3410, 2.2370) -- cycle;
\fill[blue!16.6, opacity=0.7] (1.6410, 1.3410, 2.2370) -- (1.6880, 1.3410, 2.2359) -- (1.6880, 1.3940, 2.2367) -- (1.6410, 1.3940, 2.2379) -- cycle;
\fill[blue!16.5, opacity=0.7] (1.6410, 1.3940, 2.2379) -- (1.6880, 1.3940, 2.2367) -- (1.6880, 1.4470, 2.2372) -- (1.6410, 1.4470, 2.2384) -- cycle;
\fill[blue!17.9, opacity=0.7] (1.6410, 1.4470, 2.2384) -- (1.6880, 1.4470, 2.2372) -- (1.6880, 1.5000, 2.2374) -- (1.6410, 1.5000, 2.2385) -- cycle;
\fill[blue!29.6, opacity=0.7] (1.6410, 1.5000, 2.2385) -- (1.6880, 1.5000, 2.2374) -- (1.6880, 1.5530, 2.2372) -- (1.6410, 1.5530, 2.2384) -- cycle;
\fill[blue!72.6, opacity=0.7] (1.6410, 1.5530, 2.2384) -- (1.6880, 1.5530, 2.2372) -- (1.6880, 1.6060, 2.2367) -- (1.6410, 1.6060, 2.2379) -- cycle;
\fill[blue!87.5, opacity=0.7] (1.6410, 1.6060, 2.2379) -- (1.6880, 1.6060, 2.2367) -- (1.6880, 1.6590, 2.2359) -- (1.6410, 1.6590, 2.2370) -- cycle;
\fill[blue!87.9, opacity=0.7] (1.6410, 1.6590, 2.2370) -- (1.6880, 1.6590, 2.2359) -- (1.6880, 1.7120, 2.2348) -- (1.6410, 1.7120, 2.2359) -- cycle;
\fill[blue!64.8, opacity=0.7] (1.6410, 1.7120, 2.2359) -- (1.6880, 1.7120, 2.2348) -- (1.6880, 1.7650, 2.2333) -- (1.6410, 1.7650, 2.2344) -- cycle;
\fill[blue!26.8, opacity=0.7] (1.6410, 1.7650, 2.2344) -- (1.6880, 1.7650, 2.2333) -- (1.6880, 1.8180, 2.2315) -- (1.6410, 1.8180, 2.2326) -- cycle;
\fill[blue!19.1, opacity=0.7] (1.6410, 1.8180, 2.2326) -- (1.6880, 1.8180, 2.2315) -- (1.6880, 1.8710, 2.2294) -- (1.6410, 1.8710, 2.2306) -- cycle;
\fill[blue!24.9, opacity=0.7] (1.6410, 1.8710, 2.2306) -- (1.6880, 1.8710, 2.2294) -- (1.6880, 1.9240, 2.2270) -- (1.6410, 1.9240, 2.2281) -- cycle;
\fill[blue!64.3, opacity=0.7] (1.6410, 1.9240, 2.2281) -- (1.6880, 1.9240, 2.2270) -- (1.6880, 1.9770, 2.2243) -- (1.6410, 1.9770, 2.2254) -- cycle;
\fill[blue!82.9, opacity=0.7] (1.6410, 1.9770, 2.2254) -- (1.6880, 1.9770, 2.2243) -- (1.6880, 2.0300, 2.2213) -- (1.6410, 2.0300, 2.2224) -- cycle;
\fill[blue!52.9, opacity=0.7] (1.6410, 2.0300, 2.2224) -- (1.6880, 2.0300, 2.2213) -- (1.6880, 2.0830, 2.2180) -- (1.6410, 2.0830, 2.2192) -- cycle;
\fill[blue!55.8, opacity=0.7] (1.6410, 2.0830, 2.2192) -- (1.6880, 2.0830, 2.2180) -- (1.6880, 2.1360, 2.2145) -- (1.6410, 2.1360, 2.2156) -- cycle;
\fill[blue!84.7, opacity=0.7] (1.6410, 2.1360, 2.2156) -- (1.6880, 2.1360, 2.2145) -- (1.6880, 2.1890, 2.2106) -- (1.6410, 2.1890, 2.2118) -- cycle;
\fill[blue!73.2, opacity=0.7] (1.6410, 2.1890, 2.2118) -- (1.6880, 2.1890, 2.2106) -- (1.6880, 2.2420, 2.2066) -- (1.6410, 2.2420, 2.2077) -- cycle;
\fill[blue!49.8, opacity=0.7] (1.6410, 2.2420, 2.2077) -- (1.6880, 2.2420, 2.2066) -- (1.6880, 2.2950, 2.2022) -- (1.6410, 2.2950, 2.2034) -- cycle;
\fill[blue!61.0, opacity=0.7] (1.6410, 2.2950, 2.2034) -- (1.6880, 2.2950, 2.2022) -- (1.6880, 2.3480, 2.1977) -- (1.6410, 2.3480, 2.1988) -- cycle;
\fill[blue!87.8, opacity=0.7] (1.6410, 2.3480, 2.1988) -- (1.6880, 2.3480, 2.1977) -- (1.6880, 2.4010, 2.1929) -- (1.6410, 2.4010, 2.1940) -- cycle;
\fill[blue!46.0, opacity=0.7] (1.6410, 2.4010, 2.1940) -- (1.6880, 2.4010, 2.1929) -- (1.6880, 2.4540, 2.1879) -- (1.6410, 2.4540, 2.1891) -- cycle;
\fill[blue!21.0, opacity=0.7] (1.6410, 2.4540, 2.1891) -- (1.6880, 2.4540, 2.1879) -- (1.6880, 2.5070, 2.1827) -- (1.6410, 2.5070, 2.1839) -- cycle;
\fill[blue!19.1, opacity=0.7] (1.6410, 2.5070, 2.1839) -- (1.6880, 2.5070, 2.1827) -- (1.6880, 2.5600, 2.1774) -- (1.6410, 2.5600, 2.1785) -- cycle;
\fill[blue!29.7, opacity=0.7] (1.6410, 2.5600, 2.1785) -- (1.6880, 2.5600, 2.1774) -- (1.6880, 2.6130, 2.1719) -- (1.6410, 2.6130, 2.1730) -- cycle;
\fill[blue!66.9, opacity=0.7] (1.6410, 2.6130, 2.1730) -- (1.6880, 2.6130, 2.1719) -- (1.6880, 2.6660, 2.1662) -- (1.6410, 2.6660, 2.1673) -- cycle;
\fill[blue!87.4, opacity=0.7] (1.6410, 2.6660, 2.1673) -- (1.6880, 2.6660, 2.1662) -- (1.6880, 2.7190, 2.1604) -- (1.6410, 2.7190, 2.1615) -- cycle;
\fill[blue!87.4, opacity=0.7] (1.6410, 2.7190, 2.1615) -- (1.6880, 2.7190, 2.1604) -- (1.6880, 2.7720, 2.1545) -- (1.6410, 2.7720, 2.1556) -- cycle;
\fill[blue!63.2, opacity=0.7] (1.6410, 2.7720, 2.1556) -- (1.6880, 2.7720, 2.1545) -- (1.6880, 2.8250, 2.1484) -- (1.6410, 2.8250, 2.1496) -- cycle;
\fill[blue!22.2, opacity=0.7] (1.6410, 2.8250, 2.1496) -- (1.6880, 2.8250, 2.1484) -- (1.6880, 2.8780, 2.1423) -- (1.6410, 2.8780, 2.1435) -- cycle;
\fill[blue!15.4, opacity=0.7] (1.6410, 2.8780, 2.1435) -- (1.6880, 2.8780, 2.1423) -- (1.6880, 2.9310, 2.1361) -- (1.6410, 2.9310, 2.1373) -- cycle;
\fill[blue!15.1, opacity=0.7] (1.6410, 2.9310, 2.1373) -- (1.6880, 2.9310, 2.1361) -- (1.6880, 2.9840, 2.1299) -- (1.6410, 2.9840, 2.1311) -- cycle;
\fill[blue!15.3, opacity=0.7] (1.6410, 2.9840, 2.1311) -- (1.6880, 2.9840, 2.1299) -- (1.6880, 3.0370, 2.1237) -- (1.6410, 3.0370, 2.1248) -- cycle;
\fill[blue!19.1, opacity=0.7] (1.6410, 3.0370, 2.1248) -- (1.6880, 3.0370, 2.1237) -- (1.6880, 3.0900, 2.1174) -- (1.6410, 3.0900, 2.1185) -- cycle;
\fill[blue!37.5, opacity=0.7] (1.6880, -0.0900, 2.1174) -- (1.7350, -0.0900, 2.1159) -- (1.7350, -0.0370, 2.1222) -- (1.6880, -0.0370, 2.1237) -- cycle;
\fill[blue!19.0, opacity=0.7] (1.6880, -0.0370, 2.1237) -- (1.7350, -0.0370, 2.1222) -- (1.7350, 0.0160, 2.1285) -- (1.6880, 0.0160, 2.1299) -- cycle;
\fill[blue!15.4, opacity=0.7] (1.6880, 0.0160, 2.1299) -- (1.7350, 0.0160, 2.1285) -- (1.7350, 0.0690, 2.1347) -- (1.6880, 0.0690, 2.1361) -- cycle;
\fill[blue!15.1, opacity=0.7] (1.6880, 0.0690, 2.1361) -- (1.7350, 0.0690, 2.1347) -- (1.7350, 0.1220, 2.1409) -- (1.6880, 0.1220, 2.1423) -- cycle;
\fill[blue!15.6, opacity=0.7] (1.6880, 0.1220, 2.1423) -- (1.7350, 0.1220, 2.1409) -- (1.7350, 0.1750, 2.1470) -- (1.6880, 0.1750, 2.1484) -- cycle;
\fill[blue!23.9, opacity=0.7] (1.6880, 0.1750, 2.1484) -- (1.7350, 0.1750, 2.1470) -- (1.7350, 0.2280, 2.1530) -- (1.6880, 0.2280, 2.1545) -- cycle;
\fill[blue!66.0, opacity=0.7] (1.6880, 0.2280, 2.1545) -- (1.7350, 0.2280, 2.1530) -- (1.7350, 0.2810, 2.1589) -- (1.6880, 0.2810, 2.1604) -- cycle;
\fill[blue!87.8, opacity=0.7] (1.6880, 0.2810, 2.1604) -- (1.7350, 0.2810, 2.1589) -- (1.7350, 0.3340, 2.1647) -- (1.6880, 0.3340, 2.1662) -- cycle;
\fill[blue!87.5, opacity=0.7] (1.6880, 0.3340, 2.1662) -- (1.7350, 0.3340, 2.1647) -- (1.7350, 0.3870, 2.1704) -- (1.6880, 0.3870, 2.1719) -- cycle;
\fill[blue!79.9, opacity=0.7] (1.6880, 0.3870, 2.1719) -- (1.7350, 0.3870, 2.1704) -- (1.7350, 0.4400, 2.1759) -- (1.6880, 0.4400, 2.1774) -- cycle;
\fill[blue!43.1, opacity=0.7] (1.6880, 0.4400, 2.1774) -- (1.7350, 0.4400, 2.1759) -- (1.7350, 0.4930, 2.1813) -- (1.6880, 0.4930, 2.1827) -- cycle;
\fill[blue!23.1, opacity=0.7] (1.6880, 0.4930, 2.1827) -- (1.7350, 0.4930, 2.1813) -- (1.7350, 0.5460, 2.1864) -- (1.6880, 0.5460, 2.1879) -- cycle;
\fill[blue!21.1, opacity=0.7] (1.6880, 0.5460, 2.1879) -- (1.7350, 0.5460, 2.1864) -- (1.7350, 0.5990, 2.1914) -- (1.6880, 0.5990, 2.1929) -- cycle;
\fill[blue!33.7, opacity=0.7] (1.6880, 0.5990, 2.1929) -- (1.7350, 0.5990, 2.1914) -- (1.7350, 0.6520, 2.1962) -- (1.6880, 0.6520, 2.1977) -- cycle;
\fill[blue!76.5, opacity=0.7] (1.6880, 0.6520, 2.1977) -- (1.7350, 0.6520, 2.1962) -- (1.7350, 0.7050, 2.2008) -- (1.6880, 0.7050, 2.2022) -- cycle;
\fill[blue!77.7, opacity=0.7] (1.6880, 0.7050, 2.2022) -- (1.7350, 0.7050, 2.2008) -- (1.7350, 0.7580, 2.2051) -- (1.6880, 0.7580, 2.2066) -- cycle;
\fill[blue!46.2, opacity=0.7] (1.6880, 0.7580, 2.2066) -- (1.7350, 0.7580, 2.2051) -- (1.7350, 0.8110, 2.2092) -- (1.6880, 0.8110, 2.2106) -- cycle;
\fill[blue!41.0, opacity=0.7] (1.6880, 0.8110, 2.2106) -- (1.7350, 0.8110, 2.2092) -- (1.7350, 0.8640, 2.2130) -- (1.6880, 0.8640, 2.2145) -- cycle;
\fill[blue!59.9, opacity=0.7] (1.6880, 0.8640, 2.2145) -- (1.7350, 0.8640, 2.2130) -- (1.7350, 0.9170, 2.2166) -- (1.6880, 0.9170, 2.2180) -- cycle;
\fill[blue!86.1, opacity=0.7] (1.6880, 0.9170, 2.2180) -- (1.7350, 0.9170, 2.2166) -- (1.7350, 0.9700, 2.2198) -- (1.6880, 0.9700, 2.2213) -- cycle;
\fill[blue!79.4, opacity=0.7] (1.6880, 0.9700, 2.2213) -- (1.7350, 0.9700, 2.2198) -- (1.7350, 1.0230, 2.2228) -- (1.6880, 1.0230, 2.2243) -- cycle;
\fill[blue!64.9, opacity=0.7] (1.6880, 1.0230, 2.2243) -- (1.7350, 1.0230, 2.2228) -- (1.7350, 1.0760, 2.2255) -- (1.6880, 1.0760, 2.2270) -- cycle;
\fill[blue!68.5, opacity=0.7] (1.6880, 1.0760, 2.2270) -- (1.7350, 1.0760, 2.2255) -- (1.7350, 1.1290, 2.2279) -- (1.6880, 1.1290, 2.2294) -- cycle;
\fill[blue!84.2, opacity=0.7] (1.6880, 1.1290, 2.2294) -- (1.7350, 1.1290, 2.2279) -- (1.7350, 1.1820, 2.2300) -- (1.6880, 1.1820, 2.2315) -- cycle;
\fill[blue!84.0, opacity=0.7] (1.6880, 1.1820, 2.2315) -- (1.7350, 1.1820, 2.2300) -- (1.7350, 1.2350, 2.2318) -- (1.6880, 1.2350, 2.2333) -- cycle;
\fill[blue!60.8, opacity=0.7] (1.6880, 1.2350, 2.2333) -- (1.7350, 1.2350, 2.2318) -- (1.7350, 1.2880, 2.2333) -- (1.6880, 1.2880, 2.2348) -- cycle;
\fill[blue!40.8, opacity=0.7] (1.6880, 1.2880, 2.2348) -- (1.7350, 1.2880, 2.2333) -- (1.7350, 1.3410, 2.2344) -- (1.6880, 1.3410, 2.2359) -- cycle;
\fill[blue!32.7, opacity=0.7] (1.6880, 1.3410, 2.2359) -- (1.7350, 1.3410, 2.2344) -- (1.7350, 1.3940, 2.2353) -- (1.6880, 1.3940, 2.2367) -- cycle;
\fill[blue!33.9, opacity=0.7] (1.6880, 1.3940, 2.2367) -- (1.7350, 1.3940, 2.2353) -- (1.7350, 1.4470, 2.2357) -- (1.6880, 1.4470, 2.2372) -- cycle;
\fill[blue!46.9, opacity=0.7] (1.6880, 1.4470, 2.2372) -- (1.7350, 1.4470, 2.2357) -- (1.7350, 1.5000, 2.2359) -- (1.6880, 1.5000, 2.2374) -- cycle;
\fill[blue!74.3, opacity=0.7] (1.6880, 1.5000, 2.2374) -- (1.7350, 1.5000, 2.2359) -- (1.7350, 1.5530, 2.2357) -- (1.6880, 1.5530, 2.2372) -- cycle;
\fill[blue!87.8, opacity=0.7] (1.6880, 1.5530, 2.2372) -- (1.7350, 1.5530, 2.2357) -- (1.7350, 1.6060, 2.2353) -- (1.6880, 1.6060, 2.2367) -- cycle;
\fill[blue!85.5, opacity=0.7] (1.6880, 1.6060, 2.2367) -- (1.7350, 1.6060, 2.2353) -- (1.7350, 1.6590, 2.2344) -- (1.6880, 1.6590, 2.2359) -- cycle;
\fill[blue!86.5, opacity=0.7] (1.6880, 1.6590, 2.2359) -- (1.7350, 1.6590, 2.2344) -- (1.7350, 1.7120, 2.2333) -- (1.6880, 1.7120, 2.2348) -- cycle;
\fill[blue!52.7, opacity=0.7] (1.6880, 1.7120, 2.2348) -- (1.7350, 1.7120, 2.2333) -- (1.7350, 1.7650, 2.2318) -- (1.6880, 1.7650, 2.2333) -- cycle;
\fill[blue!23.9, opacity=0.7] (1.6880, 1.7650, 2.2333) -- (1.7350, 1.7650, 2.2318) -- (1.7350, 1.8180, 2.2300) -- (1.6880, 1.8180, 2.2315) -- cycle;
\fill[blue!19.7, opacity=0.7] (1.6880, 1.8180, 2.2315) -- (1.7350, 1.8180, 2.2300) -- (1.7350, 1.8710, 2.2279) -- (1.6880, 1.8710, 2.2294) -- cycle;
\fill[blue!29.8, opacity=0.7] (1.6880, 1.8710, 2.2294) -- (1.7350, 1.8710, 2.2279) -- (1.7350, 1.9240, 2.2255) -- (1.6880, 1.9240, 2.2270) -- cycle;
\fill[blue!74.8, opacity=0.7] (1.6880, 1.9240, 2.2270) -- (1.7350, 1.9240, 2.2255) -- (1.7350, 1.9770, 2.2228) -- (1.6880, 1.9770, 2.2243) -- cycle;
\fill[blue!77.0, opacity=0.7] (1.6880, 1.9770, 2.2243) -- (1.7350, 1.9770, 2.2228) -- (1.7350, 2.0300, 2.2198) -- (1.6880, 2.0300, 2.2213) -- cycle;
\fill[blue!49.7, opacity=0.7] (1.6880, 2.0300, 2.2213) -- (1.7350, 2.0300, 2.2198) -- (1.7350, 2.0830, 2.2166) -- (1.6880, 2.0830, 2.2180) -- cycle;
\fill[blue!57.2, opacity=0.7] (1.6880, 2.0830, 2.2180) -- (1.7350, 2.0830, 2.2166) -- (1.7350, 2.1360, 2.2130) -- (1.6880, 2.1360, 2.2145) -- cycle;
\fill[blue!86.1, opacity=0.7] (1.6880, 2.1360, 2.2145) -- (1.7350, 2.1360, 2.2130) -- (1.7350, 2.1890, 2.2092) -- (1.6880, 2.1890, 2.2106) -- cycle;
\fill[blue!71.0, opacity=0.7] (1.6880, 2.1890, 2.2106) -- (1.7350, 2.1890, 2.2092) -- (1.7350, 2.2420, 2.2051) -- (1.6880, 2.2420, 2.2066) -- cycle;
\fill[blue!50.4, opacity=0.7] (1.6880, 2.2420, 2.2066) -- (1.7350, 2.2420, 2.2051) -- (1.7350, 2.2950, 2.2008) -- (1.6880, 2.2950, 2.2022) -- cycle;
\fill[blue!64.9, opacity=0.7] (1.6880, 2.2950, 2.2022) -- (1.7350, 2.2950, 2.2008) -- (1.7350, 2.3480, 2.1962) -- (1.6880, 2.3480, 2.1977) -- cycle;
\fill[blue!86.9, opacity=0.7] (1.6880, 2.3480, 2.1977) -- (1.7350, 2.3480, 2.1962) -- (1.7350, 2.4010, 2.1914) -- (1.6880, 2.4010, 2.1929) -- cycle;
\fill[blue!41.6, opacity=0.7] (1.6880, 2.4010, 2.1929) -- (1.7350, 2.4010, 2.1914) -- (1.7350, 2.4540, 2.1864) -- (1.6880, 2.4540, 2.1879) -- cycle;
\fill[blue!20.2, opacity=0.7] (1.6880, 2.4540, 2.1879) -- (1.7350, 2.4540, 2.1864) -- (1.7350, 2.5070, 2.1813) -- (1.6880, 2.5070, 2.1827) -- cycle;
\fill[blue!19.1, opacity=0.7] (1.6880, 2.5070, 2.1827) -- (1.7350, 2.5070, 2.1813) -- (1.7350, 2.5600, 2.1759) -- (1.6880, 2.5600, 2.1774) -- cycle;
\fill[blue!30.8, opacity=0.7] (1.6880, 2.5600, 2.1774) -- (1.7350, 2.5600, 2.1759) -- (1.7350, 2.6130, 2.1704) -- (1.6880, 2.6130, 2.1719) -- cycle;
\fill[blue!68.6, opacity=0.7] (1.6880, 2.6130, 2.1719) -- (1.7350, 2.6130, 2.1704) -- (1.7350, 2.6660, 2.1647) -- (1.6880, 2.6660, 2.1662) -- cycle;
\fill[blue!87.5, opacity=0.7] (1.6880, 2.6660, 2.1662) -- (1.7350, 2.6660, 2.1647) -- (1.7350, 2.7190, 2.1589) -- (1.6880, 2.7190, 2.1604) -- cycle;
\fill[blue!87.0, opacity=0.7] (1.6880, 2.7190, 2.1604) -- (1.7350, 2.7190, 2.1589) -- (1.7350, 2.7720, 2.1530) -- (1.6880, 2.7720, 2.1545) -- cycle;
\fill[blue!59.7, opacity=0.7] (1.6880, 2.7720, 2.1545) -- (1.7350, 2.7720, 2.1530) -- (1.7350, 2.8250, 2.1470) -- (1.6880, 2.8250, 2.1484) -- cycle;
\fill[blue!21.0, opacity=0.7] (1.6880, 2.8250, 2.1484) -- (1.7350, 2.8250, 2.1470) -- (1.7350, 2.8780, 2.1409) -- (1.6880, 2.8780, 2.1423) -- cycle;
\fill[blue!15.3, opacity=0.7] (1.6880, 2.8780, 2.1423) -- (1.7350, 2.8780, 2.1409) -- (1.7350, 2.9310, 2.1347) -- (1.6880, 2.9310, 2.1361) -- cycle;
\fill[blue!15.1, opacity=0.7] (1.6880, 2.9310, 2.1361) -- (1.7350, 2.9310, 2.1347) -- (1.7350, 2.9840, 2.1285) -- (1.6880, 2.9840, 2.1299) -- cycle;
\fill[blue!15.3, opacity=0.7] (1.6880, 2.9840, 2.1299) -- (1.7350, 2.9840, 2.1285) -- (1.7350, 3.0370, 2.1222) -- (1.6880, 3.0370, 2.1237) -- cycle;
\fill[blue!19.4, opacity=0.7] (1.6880, 3.0370, 2.1237) -- (1.7350, 3.0370, 2.1222) -- (1.7350, 3.0900, 2.1159) -- (1.6880, 3.0900, 2.1174) -- cycle;
\fill[blue!44.1, opacity=0.7] (1.7350, -0.0900, 2.1159) -- (1.7820, -0.0900, 2.1141) -- (1.7820, -0.0370, 2.1204) -- (1.7350, -0.0370, 2.1222) -- cycle;
\fill[blue!22.4, opacity=0.7] (1.7350, -0.0370, 2.1222) -- (1.7820, -0.0370, 2.1204) -- (1.7820, 0.0160, 2.1267) -- (1.7350, 0.0160, 2.1285) -- cycle;
\fill[blue!15.7, opacity=0.7] (1.7350, 0.0160, 2.1285) -- (1.7820, 0.0160, 2.1267) -- (1.7820, 0.0690, 2.1329) -- (1.7350, 0.0690, 2.1347) -- cycle;
\fill[blue!15.1, opacity=0.7] (1.7350, 0.0690, 2.1347) -- (1.7820, 0.0690, 2.1329) -- (1.7820, 0.1220, 2.1391) -- (1.7350, 0.1220, 2.1409) -- cycle;
\fill[blue!15.3, opacity=0.7] (1.7350, 0.1220, 2.1409) -- (1.7820, 0.1220, 2.1391) -- (1.7820, 0.1750, 2.1452) -- (1.7350, 0.1750, 2.1470) -- cycle;
\fill[blue!19.0, opacity=0.7] (1.7350, 0.1750, 2.1470) -- (1.7820, 0.1750, 2.1452) -- (1.7820, 0.2280, 2.1512) -- (1.7350, 0.2280, 2.1530) -- cycle;
\fill[blue!50.2, opacity=0.7] (1.7350, 0.2280, 2.1530) -- (1.7820, 0.2280, 2.1512) -- (1.7820, 0.2810, 2.1571) -- (1.7350, 0.2810, 2.1589) -- cycle;
\fill[blue!86.2, opacity=0.7] (1.7350, 0.2810, 2.1589) -- (1.7820, 0.2810, 2.1571) -- (1.7820, 0.3340, 2.1629) -- (1.7350, 0.3340, 2.1647) -- cycle;
\fill[blue!86.9, opacity=0.7] (1.7350, 0.3340, 2.1647) -- (1.7820, 0.3340, 2.1629) -- (1.7820, 0.3870, 2.1686) -- (1.7350, 0.3870, 2.1704) -- cycle;
\fill[blue!86.7, opacity=0.7] (1.7350, 0.3870, 2.1704) -- (1.7820, 0.3870, 2.1686) -- (1.7820, 0.4400, 2.1741) -- (1.7350, 0.4400, 2.1759) -- cycle;
\fill[blue!59.5, opacity=0.7] (1.7350, 0.4400, 2.1759) -- (1.7820, 0.4400, 2.1741) -- (1.7820, 0.4930, 2.1795) -- (1.7350, 0.4930, 2.1813) -- cycle;
\fill[blue!28.7, opacity=0.7] (1.7350, 0.4930, 2.1813) -- (1.7820, 0.4930, 2.1795) -- (1.7820, 0.5460, 2.1847) -- (1.7350, 0.5460, 2.1864) -- cycle;
\fill[blue!21.0, opacity=0.7] (1.7350, 0.5460, 2.1864) -- (1.7820, 0.5460, 2.1847) -- (1.7820, 0.5990, 2.1896) -- (1.7350, 0.5990, 2.1914) -- cycle;
\fill[blue!25.3, opacity=0.7] (1.7350, 0.5990, 2.1914) -- (1.7820, 0.5990, 2.1896) -- (1.7820, 0.6520, 2.1944) -- (1.7350, 0.6520, 2.1962) -- cycle;
\fill[blue!53.4, opacity=0.7] (1.7350, 0.6520, 2.1962) -- (1.7820, 0.6520, 2.1944) -- (1.7820, 0.7050, 2.1990) -- (1.7350, 0.7050, 2.2008) -- cycle;
\fill[blue!87.9, opacity=0.7] (1.7350, 0.7050, 2.2008) -- (1.7820, 0.7050, 2.1990) -- (1.7820, 0.7580, 2.2033) -- (1.7350, 0.7580, 2.2051) -- cycle;
\fill[blue!61.0, opacity=0.7] (1.7350, 0.7580, 2.2051) -- (1.7820, 0.7580, 2.2033) -- (1.7820, 0.8110, 2.2074) -- (1.7350, 0.8110, 2.2092) -- cycle;
\fill[blue!40.0, opacity=0.7] (1.7350, 0.8110, 2.2092) -- (1.7820, 0.8110, 2.2074) -- (1.7820, 0.8640, 2.2112) -- (1.7350, 0.8640, 2.2130) -- cycle;
\fill[blue!42.9, opacity=0.7] (1.7350, 0.8640, 2.2130) -- (1.7820, 0.8640, 2.2112) -- (1.7820, 0.9170, 2.2148) -- (1.7350, 0.9170, 2.2166) -- cycle;
\fill[blue!65.5, opacity=0.7] (1.7350, 0.9170, 2.2166) -- (1.7820, 0.9170, 2.2148) -- (1.7820, 0.9700, 2.2180) -- (1.7350, 0.9700, 2.2198) -- cycle;
\fill[blue!87.1, opacity=0.7] (1.7350, 0.9700, 2.2198) -- (1.7820, 0.9700, 2.2180) -- (1.7820, 1.0230, 2.2210) -- (1.7350, 1.0230, 2.2228) -- cycle;
\fill[blue!80.2, opacity=0.7] (1.7350, 1.0230, 2.2228) -- (1.7820, 1.0230, 2.2210) -- (1.7820, 1.0760, 2.2238) -- (1.7350, 1.0760, 2.2255) -- cycle;
\fill[blue!68.2, opacity=0.7] (1.7350, 1.0760, 2.2255) -- (1.7820, 1.0760, 2.2238) -- (1.7820, 1.1290, 2.2262) -- (1.7350, 1.1290, 2.2279) -- cycle;
\fill[blue!68.7, opacity=0.7] (1.7350, 1.1290, 2.2279) -- (1.7820, 1.1290, 2.2262) -- (1.7820, 1.1820, 2.2283) -- (1.7350, 1.1820, 2.2300) -- cycle;
\fill[blue!78.5, opacity=0.7] (1.7350, 1.1820, 2.2300) -- (1.7820, 1.1820, 2.2283) -- (1.7820, 1.2350, 2.2300) -- (1.7350, 1.2350, 2.2318) -- cycle;
\fill[blue!86.9, opacity=0.7] (1.7350, 1.2350, 2.2318) -- (1.7820, 1.2350, 2.2300) -- (1.7820, 1.2880, 2.2315) -- (1.7350, 1.2880, 2.2333) -- cycle;
\fill[blue!86.9, opacity=0.7] (1.7350, 1.2880, 2.2333) -- (1.7820, 1.2880, 2.2315) -- (1.7820, 1.3410, 2.2326) -- (1.7350, 1.3410, 2.2344) -- cycle;
\fill[blue!84.0, opacity=0.7] (1.7350, 1.3410, 2.2344) -- (1.7820, 1.3410, 2.2326) -- (1.7820, 1.3940, 2.2335) -- (1.7350, 1.3940, 2.2353) -- cycle;
\fill[blue!84.7, opacity=0.7] (1.7350, 1.3940, 2.2353) -- (1.7820, 1.3940, 2.2335) -- (1.7820, 1.4470, 2.2340) -- (1.7350, 1.4470, 2.2357) -- cycle;
\fill[blue!87.6, opacity=0.7] (1.7350, 1.4470, 2.2357) -- (1.7820, 1.4470, 2.2340) -- (1.7820, 1.5000, 2.2341) -- (1.7350, 1.5000, 2.2359) -- cycle;
\fill[blue!86.1, opacity=0.7] (1.7350, 1.5000, 2.2359) -- (1.7820, 1.5000, 2.2341) -- (1.7820, 1.5530, 2.2340) -- (1.7350, 1.5530, 2.2357) -- cycle;
\fill[blue!83.5, opacity=0.7] (1.7350, 1.5530, 2.2357) -- (1.7820, 1.5530, 2.2340) -- (1.7820, 1.6060, 2.2335) -- (1.7350, 1.6060, 2.2353) -- cycle;
\fill[blue!87.6, opacity=0.7] (1.7350, 1.6060, 2.2353) -- (1.7820, 1.6060, 2.2335) -- (1.7820, 1.6590, 2.2326) -- (1.7350, 1.6590, 2.2344) -- cycle;
\fill[blue!74.3, opacity=0.7] (1.7350, 1.6590, 2.2344) -- (1.7820, 1.6590, 2.2326) -- (1.7820, 1.7120, 2.2315) -- (1.7350, 1.7120, 2.2333) -- cycle;
\fill[blue!36.6, opacity=0.7] (1.7350, 1.7120, 2.2333) -- (1.7820, 1.7120, 2.2315) -- (1.7820, 1.7650, 2.2300) -- (1.7350, 1.7650, 2.2318) -- cycle;
\fill[blue!21.4, opacity=0.7] (1.7350, 1.7650, 2.2318) -- (1.7820, 1.7650, 2.2300) -- (1.7820, 1.8180, 2.2283) -- (1.7350, 1.8180, 2.2300) -- cycle;
\fill[blue!21.6, opacity=0.7] (1.7350, 1.8180, 2.2300) -- (1.7820, 1.8180, 2.2283) -- (1.7820, 1.8710, 2.2262) -- (1.7350, 1.8710, 2.2279) -- cycle;
\fill[blue!41.1, opacity=0.7] (1.7350, 1.8710, 2.2279) -- (1.7820, 1.8710, 2.2262) -- (1.7820, 1.9240, 2.2238) -- (1.7350, 1.9240, 2.2255) -- cycle;
\fill[blue!85.7, opacity=0.7] (1.7350, 1.9240, 2.2255) -- (1.7820, 1.9240, 2.2238) -- (1.7820, 1.9770, 2.2210) -- (1.7350, 1.9770, 2.2228) -- cycle;
\fill[blue!66.6, opacity=0.7] (1.7350, 1.9770, 2.2228) -- (1.7820, 1.9770, 2.2210) -- (1.7820, 2.0300, 2.2180) -- (1.7350, 2.0300, 2.2198) -- cycle;
\fill[blue!46.8, opacity=0.7] (1.7350, 2.0300, 2.2198) -- (1.7820, 2.0300, 2.2180) -- (1.7820, 2.0830, 2.2148) -- (1.7350, 2.0830, 2.2166) -- cycle;
\fill[blue!61.4, opacity=0.7] (1.7350, 2.0830, 2.2166) -- (1.7820, 2.0830, 2.2148) -- (1.7820, 2.1360, 2.2112) -- (1.7350, 2.1360, 2.2130) -- cycle;
\fill[blue!87.7, opacity=0.7] (1.7350, 2.1360, 2.2130) -- (1.7820, 2.1360, 2.2112) -- (1.7820, 2.1890, 2.2074) -- (1.7350, 2.1890, 2.2092) -- cycle;
\fill[blue!66.8, opacity=0.7] (1.7350, 2.1890, 2.2092) -- (1.7820, 2.1890, 2.2074) -- (1.7820, 2.2420, 2.2033) -- (1.7350, 2.2420, 2.2051) -- cycle;
\fill[blue!51.3, opacity=0.7] (1.7350, 2.2420, 2.2051) -- (1.7820, 2.2420, 2.2033) -- (1.7820, 2.2950, 2.1990) -- (1.7350, 2.2950, 2.2008) -- cycle;
\fill[blue!70.9, opacity=0.7] (1.7350, 2.2950, 2.2008) -- (1.7820, 2.2950, 2.1990) -- (1.7820, 2.3480, 2.1944) -- (1.7350, 2.3480, 2.1962) -- cycle;
\fill[blue!83.2, opacity=0.7] (1.7350, 2.3480, 2.1962) -- (1.7820, 2.3480, 2.1944) -- (1.7820, 2.4010, 2.1896) -- (1.7350, 2.4010, 2.1914) -- cycle;
\fill[blue!35.5, opacity=0.7] (1.7350, 2.4010, 2.1914) -- (1.7820, 2.4010, 2.1896) -- (1.7820, 2.4540, 2.1847) -- (1.7350, 2.4540, 2.1864) -- cycle;
\fill[blue!19.3, opacity=0.7] (1.7350, 2.4540, 2.1864) -- (1.7820, 2.4540, 2.1847) -- (1.7820, 2.5070, 2.1795) -- (1.7350, 2.5070, 2.1813) -- cycle;
\fill[blue!19.4, opacity=0.7] (1.7350, 2.5070, 2.1813) -- (1.7820, 2.5070, 2.1795) -- (1.7820, 2.5600, 2.1741) -- (1.7350, 2.5600, 2.1759) -- cycle;
\fill[blue!33.3, opacity=0.7] (1.7350, 2.5600, 2.1759) -- (1.7820, 2.5600, 2.1741) -- (1.7820, 2.6130, 2.1686) -- (1.7350, 2.6130, 2.1704) -- cycle;
\fill[blue!71.9, opacity=0.7] (1.7350, 2.6130, 2.1704) -- (1.7820, 2.6130, 2.1686) -- (1.7820, 2.6660, 2.1629) -- (1.7350, 2.6660, 2.1647) -- cycle;
\fill[blue!87.6, opacity=0.7] (1.7350, 2.6660, 2.1647) -- (1.7820, 2.6660, 2.1629) -- (1.7820, 2.7190, 2.1571) -- (1.7350, 2.7190, 2.1589) -- cycle;
\fill[blue!86.1, opacity=0.7] (1.7350, 2.7190, 2.1589) -- (1.7820, 2.7190, 2.1571) -- (1.7820, 2.7720, 2.1512) -- (1.7350, 2.7720, 2.1530) -- cycle;
\fill[blue!54.2, opacity=0.7] (1.7350, 2.7720, 2.1530) -- (1.7820, 2.7720, 2.1512) -- (1.7820, 2.8250, 2.1452) -- (1.7350, 2.8250, 2.1470) -- cycle;
\fill[blue!19.4, opacity=0.7] (1.7350, 2.8250, 2.1470) -- (1.7820, 2.8250, 2.1452) -- (1.7820, 2.8780, 2.1391) -- (1.7350, 2.8780, 2.1409) -- cycle;
\fill[blue!15.2, opacity=0.7] (1.7350, 2.8780, 2.1409) -- (1.7820, 2.8780, 2.1391) -- (1.7820, 2.9310, 2.1329) -- (1.7350, 2.9310, 2.1347) -- cycle;
\fill[blue!15.1, opacity=0.7] (1.7350, 2.9310, 2.1347) -- (1.7820, 2.9310, 2.1329) -- (1.7820, 2.9840, 2.1267) -- (1.7350, 2.9840, 2.1285) -- cycle;
\fill[blue!15.4, opacity=0.7] (1.7350, 2.9840, 2.1285) -- (1.7820, 2.9840, 2.1267) -- (1.7820, 3.0370, 2.1204) -- (1.7350, 3.0370, 2.1222) -- cycle;
\fill[blue!20.1, opacity=0.7] (1.7350, 3.0370, 2.1222) -- (1.7820, 3.0370, 2.1204) -- (1.7820, 3.0900, 2.1141) -- (1.7350, 3.0900, 2.1159) -- cycle;
\fill[blue!50.1, opacity=0.7] (1.7820, -0.0900, 2.1141) -- (1.8290, -0.0900, 2.1120) -- (1.8290, -0.0370, 2.1183) -- (1.7820, -0.0370, 2.1204) -- cycle;
\fill[blue!28.3, opacity=0.7] (1.7820, -0.0370, 2.1204) -- (1.8290, -0.0370, 2.1183) -- (1.8290, 0.0160, 2.1246) -- (1.7820, 0.0160, 2.1267) -- cycle;
\fill[blue!16.6, opacity=0.7] (1.7820, 0.0160, 2.1267) -- (1.8290, 0.0160, 2.1246) -- (1.8290, 0.0690, 2.1308) -- (1.7820, 0.0690, 2.1329) -- cycle;
\fill[blue!15.2, opacity=0.7] (1.7820, 0.0690, 2.1329) -- (1.8290, 0.0690, 2.1308) -- (1.8290, 0.1220, 2.1370) -- (1.7820, 0.1220, 2.1391) -- cycle;
\fill[blue!15.2, opacity=0.7] (1.7820, 0.1220, 2.1391) -- (1.8290, 0.1220, 2.1370) -- (1.8290, 0.1750, 2.1431) -- (1.7820, 0.1750, 2.1452) -- cycle;
\fill[blue!16.6, opacity=0.7] (1.7820, 0.1750, 2.1452) -- (1.8290, 0.1750, 2.1431) -- (1.8290, 0.2280, 2.1491) -- (1.7820, 0.2280, 2.1512) -- cycle;
\fill[blue!34.0, opacity=0.7] (1.7820, 0.2280, 2.1512) -- (1.8290, 0.2280, 2.1491) -- (1.8290, 0.2810, 2.1550) -- (1.7820, 0.2810, 2.1571) -- cycle;
\fill[blue!78.4, opacity=0.7] (1.7820, 0.2810, 2.1571) -- (1.8290, 0.2810, 2.1550) -- (1.8290, 0.3340, 2.1608) -- (1.7820, 0.3340, 2.1629) -- cycle;
\fill[blue!87.4, opacity=0.7] (1.7820, 0.3340, 2.1629) -- (1.8290, 0.3340, 2.1608) -- (1.8290, 0.3870, 2.1665) -- (1.7820, 0.3870, 2.1686) -- cycle;
\fill[blue!87.7, opacity=0.7] (1.7820, 0.3870, 2.1686) -- (1.8290, 0.3870, 2.1665) -- (1.8290, 0.4400, 2.1720) -- (1.7820, 0.4400, 2.1741) -- cycle;
\fill[blue!76.9, opacity=0.7] (1.7820, 0.4400, 2.1741) -- (1.8290, 0.4400, 2.1720) -- (1.8290, 0.4930, 2.1774) -- (1.7820, 0.4930, 2.1795) -- cycle;
\fill[blue!41.0, opacity=0.7] (1.7820, 0.4930, 2.1795) -- (1.8290, 0.4930, 2.1774) -- (1.8290, 0.5460, 2.1826) -- (1.7820, 0.5460, 2.1847) -- cycle;
\fill[blue!23.4, opacity=0.7] (1.7820, 0.5460, 2.1847) -- (1.8290, 0.5460, 2.1826) -- (1.8290, 0.5990, 2.1875) -- (1.7820, 0.5990, 2.1896) -- cycle;
\fill[blue!21.8, opacity=0.7] (1.7820, 0.5990, 2.1896) -- (1.8290, 0.5990, 2.1875) -- (1.8290, 0.6520, 2.1923) -- (1.7820, 0.6520, 2.1944) -- cycle;
\fill[blue!33.7, opacity=0.7] (1.7820, 0.6520, 2.1944) -- (1.8290, 0.6520, 2.1923) -- (1.8290, 0.7050, 2.1969) -- (1.7820, 0.7050, 2.1990) -- cycle;
\fill[blue!72.8, opacity=0.7] (1.7820, 0.7050, 2.1990) -- (1.8290, 0.7050, 2.1969) -- (1.8290, 0.7580, 2.2012) -- (1.7820, 0.7580, 2.2033) -- cycle;
\fill[blue!82.7, opacity=0.7] (1.7820, 0.7580, 2.2033) -- (1.8290, 0.7580, 2.2012) -- (1.8290, 0.8110, 2.2053) -- (1.7820, 0.8110, 2.2074) -- cycle;
\fill[blue!50.8, opacity=0.7] (1.7820, 0.8110, 2.2074) -- (1.8290, 0.8110, 2.2053) -- (1.8290, 0.8640, 2.2091) -- (1.7820, 0.8640, 2.2112) -- cycle;
\fill[blue!37.6, opacity=0.7] (1.7820, 0.8640, 2.2112) -- (1.8290, 0.8640, 2.2091) -- (1.8290, 0.9170, 2.2127) -- (1.7820, 0.9170, 2.2148) -- cycle;
\fill[blue!43.2, opacity=0.7] (1.7820, 0.9170, 2.2148) -- (1.8290, 0.9170, 2.2127) -- (1.8290, 0.9700, 2.2160) -- (1.7820, 0.9700, 2.2180) -- cycle;
\fill[blue!64.6, opacity=0.7] (1.7820, 0.9700, 2.2180) -- (1.8290, 0.9700, 2.2160) -- (1.8290, 1.0230, 2.2190) -- (1.7820, 1.0230, 2.2210) -- cycle;
\fill[blue!85.5, opacity=0.7] (1.7820, 1.0230, 2.2210) -- (1.8290, 1.0230, 2.2190) -- (1.8290, 1.0760, 2.2217) -- (1.7820, 1.0760, 2.2238) -- cycle;
\fill[blue!85.1, opacity=0.7] (1.7820, 1.0760, 2.2238) -- (1.8290, 1.0760, 2.2217) -- (1.8290, 1.1290, 2.2241) -- (1.7820, 1.1290, 2.2262) -- cycle;
\fill[blue!75.4, opacity=0.7] (1.7820, 1.1290, 2.2262) -- (1.8290, 1.1290, 2.2241) -- (1.8290, 1.1820, 2.2262) -- (1.7820, 1.1820, 2.2283) -- cycle;
\fill[blue!70.8, opacity=0.7] (1.7820, 1.1820, 2.2283) -- (1.8290, 1.1820, 2.2262) -- (1.8290, 1.2350, 2.2279) -- (1.7820, 1.2350, 2.2300) -- cycle;
\fill[blue!72.3, opacity=0.7] (1.7820, 1.2350, 2.2300) -- (1.8290, 1.2350, 2.2279) -- (1.8290, 1.2880, 2.2294) -- (1.7820, 1.2880, 2.2315) -- cycle;
\fill[blue!76.1, opacity=0.7] (1.7820, 1.2880, 2.2315) -- (1.8290, 1.2880, 2.2294) -- (1.8290, 1.3410, 2.2306) -- (1.7820, 1.3410, 2.2326) -- cycle;
\fill[blue!78.9, opacity=0.7] (1.7820, 1.3410, 2.2326) -- (1.8290, 1.3410, 2.2306) -- (1.8290, 1.3940, 2.2314) -- (1.7820, 1.3940, 2.2335) -- cycle;
\fill[blue!79.8, opacity=0.7] (1.7820, 1.3940, 2.2335) -- (1.8290, 1.3940, 2.2314) -- (1.8290, 1.4470, 2.2319) -- (1.7820, 1.4470, 2.2340) -- cycle;
\fill[blue!79.9, opacity=0.7] (1.7820, 1.4470, 2.2340) -- (1.8290, 1.4470, 2.2319) -- (1.8290, 1.5000, 2.2320) -- (1.7820, 1.5000, 2.2341) -- cycle;
\fill[blue!82.2, opacity=0.7] (1.7820, 1.5000, 2.2341) -- (1.8290, 1.5000, 2.2320) -- (1.8290, 1.5530, 2.2319) -- (1.7820, 1.5530, 2.2340) -- cycle;
\fill[blue!87.5, opacity=0.7] (1.7820, 1.5530, 2.2340) -- (1.8290, 1.5530, 2.2319) -- (1.8290, 1.6060, 2.2314) -- (1.7820, 1.6060, 2.2335) -- cycle;
\fill[blue!79.9, opacity=0.7] (1.7820, 1.6060, 2.2335) -- (1.8290, 1.6060, 2.2314) -- (1.8290, 1.6590, 2.2306) -- (1.7820, 1.6590, 2.2326) -- cycle;
\fill[blue!47.8, opacity=0.7] (1.7820, 1.6590, 2.2326) -- (1.8290, 1.6590, 2.2306) -- (1.8290, 1.7120, 2.2294) -- (1.7820, 1.7120, 2.2315) -- cycle;
\fill[blue!25.5, opacity=0.7] (1.7820, 1.7120, 2.2315) -- (1.8290, 1.7120, 2.2294) -- (1.8290, 1.7650, 2.2279) -- (1.7820, 1.7650, 2.2300) -- cycle;
\fill[blue!20.9, opacity=0.7] (1.7820, 1.7650, 2.2300) -- (1.8290, 1.7650, 2.2279) -- (1.8290, 1.8180, 2.2262) -- (1.7820, 1.8180, 2.2283) -- cycle;
\fill[blue!27.7, opacity=0.7] (1.7820, 1.8180, 2.2283) -- (1.8290, 1.8180, 2.2262) -- (1.8290, 1.8710, 2.2241) -- (1.7820, 1.8710, 2.2262) -- cycle;
\fill[blue!62.8, opacity=0.7] (1.7820, 1.8710, 2.2262) -- (1.8290, 1.8710, 2.2241) -- (1.8290, 1.9240, 2.2217) -- (1.7820, 1.9240, 2.2238) -- cycle;
\fill[blue!85.8, opacity=0.7] (1.7820, 1.9240, 2.2238) -- (1.8290, 1.9240, 2.2217) -- (1.8290, 1.9770, 2.2190) -- (1.7820, 1.9770, 2.2210) -- cycle;
\fill[blue!54.5, opacity=0.7] (1.7820, 1.9770, 2.2210) -- (1.8290, 1.9770, 2.2190) -- (1.8290, 2.0300, 2.2160) -- (1.7820, 2.0300, 2.2180) -- cycle;
\fill[blue!46.3, opacity=0.7] (1.7820, 2.0300, 2.2180) -- (1.8290, 2.0300, 2.2160) -- (1.8290, 2.0830, 2.2127) -- (1.7820, 2.0830, 2.2148) -- cycle;
\fill[blue!69.2, opacity=0.7] (1.7820, 2.0830, 2.2148) -- (1.8290, 2.0830, 2.2127) -- (1.8290, 2.1360, 2.2091) -- (1.7820, 2.1360, 2.2112) -- cycle;
\fill[blue!86.9, opacity=0.7] (1.7820, 2.1360, 2.2112) -- (1.8290, 2.1360, 2.2091) -- (1.8290, 2.1890, 2.2053) -- (1.7820, 2.1890, 2.2074) -- cycle;
\fill[blue!61.3, opacity=0.7] (1.7820, 2.1890, 2.2074) -- (1.8290, 2.1890, 2.2053) -- (1.8290, 2.2420, 2.2012) -- (1.7820, 2.2420, 2.2033) -- cycle;
\fill[blue!53.8, opacity=0.7] (1.7820, 2.2420, 2.2033) -- (1.8290, 2.2420, 2.2012) -- (1.8290, 2.2950, 2.1969) -- (1.7820, 2.2950, 2.1990) -- cycle;
\fill[blue!78.9, opacity=0.7] (1.7820, 2.2950, 2.1990) -- (1.8290, 2.2950, 2.1969) -- (1.8290, 2.3480, 2.1923) -- (1.7820, 2.3480, 2.1944) -- cycle;
\fill[blue!74.3, opacity=0.7] (1.7820, 2.3480, 2.1944) -- (1.8290, 2.3480, 2.1923) -- (1.8290, 2.4010, 2.1875) -- (1.7820, 2.4010, 2.1896) -- cycle;
\fill[blue!29.1, opacity=0.7] (1.7820, 2.4010, 2.1896) -- (1.8290, 2.4010, 2.1875) -- (1.8290, 2.4540, 2.1826) -- (1.7820, 2.4540, 2.1847) -- cycle;
\fill[blue!18.6, opacity=0.7] (1.7820, 2.4540, 2.1847) -- (1.8290, 2.4540, 2.1826) -- (1.8290, 2.5070, 2.1774) -- (1.7820, 2.5070, 2.1795) -- cycle;
\fill[blue!20.0, opacity=0.7] (1.7820, 2.5070, 2.1795) -- (1.8290, 2.5070, 2.1774) -- (1.8290, 2.5600, 2.1720) -- (1.7820, 2.5600, 2.1741) -- cycle;
\fill[blue!37.7, opacity=0.7] (1.7820, 2.5600, 2.1741) -- (1.8290, 2.5600, 2.1720) -- (1.8290, 2.6130, 2.1665) -- (1.7820, 2.6130, 2.1686) -- cycle;
\fill[blue!76.2, opacity=0.7] (1.7820, 2.6130, 2.1686) -- (1.8290, 2.6130, 2.1665) -- (1.8290, 2.6660, 2.1608) -- (1.7820, 2.6660, 2.1629) -- cycle;
\fill[blue!87.7, opacity=0.7] (1.7820, 2.6660, 2.1629) -- (1.8290, 2.6660, 2.1608) -- (1.8290, 2.7190, 2.1550) -- (1.7820, 2.7190, 2.1571) -- cycle;
\fill[blue!84.2, opacity=0.7] (1.7820, 2.7190, 2.1571) -- (1.8290, 2.7190, 2.1550) -- (1.8290, 2.7720, 2.1491) -- (1.7820, 2.7720, 2.1512) -- cycle;
\fill[blue!46.7, opacity=0.7] (1.7820, 2.7720, 2.1512) -- (1.8290, 2.7720, 2.1491) -- (1.8290, 2.8250, 2.1431) -- (1.7820, 2.8250, 2.1452) -- cycle;
\fill[blue!17.9, opacity=0.7] (1.7820, 2.8250, 2.1452) -- (1.8290, 2.8250, 2.1431) -- (1.8290, 2.8780, 2.1370) -- (1.7820, 2.8780, 2.1391) -- cycle;
\fill[blue!15.2, opacity=0.7] (1.7820, 2.8780, 2.1391) -- (1.8290, 2.8780, 2.1370) -- (1.8290, 2.9310, 2.1308) -- (1.7820, 2.9310, 2.1329) -- cycle;
\fill[blue!15.1, opacity=0.7] (1.7820, 2.9310, 2.1329) -- (1.8290, 2.9310, 2.1308) -- (1.8290, 2.9840, 2.1246) -- (1.7820, 2.9840, 2.1267) -- cycle;
\fill[blue!15.5, opacity=0.7] (1.7820, 2.9840, 2.1267) -- (1.8290, 2.9840, 2.1246) -- (1.8290, 3.0370, 2.1183) -- (1.7820, 3.0370, 2.1204) -- cycle;
\fill[blue!21.3, opacity=0.7] (1.7820, 3.0370, 2.1204) -- (1.8290, 3.0370, 2.1183) -- (1.8290, 3.0900, 2.1120) -- (1.7820, 3.0900, 2.1141) -- cycle;
\fill[blue!54.0, opacity=0.7] (1.8290, -0.0900, 2.1120) -- (1.8760, -0.0900, 2.1096) -- (1.8760, -0.0370, 2.1159) -- (1.8290, -0.0370, 2.1183) -- cycle;
\fill[blue!36.7, opacity=0.7] (1.8290, -0.0370, 2.1183) -- (1.8760, -0.0370, 2.1159) -- (1.8760, 0.0160, 2.1222) -- (1.8290, 0.0160, 2.1246) -- cycle;
\fill[blue!18.9, opacity=0.7] (1.8290, 0.0160, 2.1246) -- (1.8760, 0.0160, 2.1222) -- (1.8760, 0.0690, 2.1284) -- (1.8290, 0.0690, 2.1308) -- cycle;
\fill[blue!15.4, opacity=0.7] (1.8290, 0.0690, 2.1308) -- (1.8760, 0.0690, 2.1284) -- (1.8760, 0.1220, 2.1346) -- (1.8290, 0.1220, 2.1370) -- cycle;
\fill[blue!15.2, opacity=0.7] (1.8290, 0.1220, 2.1370) -- (1.8760, 0.1220, 2.1346) -- (1.8760, 0.1750, 2.1407) -- (1.8290, 0.1750, 2.1431) -- cycle;
\fill[blue!15.6, opacity=0.7] (1.8290, 0.1750, 2.1431) -- (1.8760, 0.1750, 2.1407) -- (1.8760, 0.2280, 2.1467) -- (1.8290, 0.2280, 2.1491) -- cycle;
\fill[blue!22.7, opacity=0.7] (1.8290, 0.2280, 2.1491) -- (1.8760, 0.2280, 2.1467) -- (1.8760, 0.2810, 2.1526) -- (1.8290, 0.2810, 2.1550) -- cycle;
\fill[blue!61.0, opacity=0.7] (1.8290, 0.2810, 2.1550) -- (1.8760, 0.2810, 2.1526) -- (1.8760, 0.3340, 2.1584) -- (1.8290, 0.3340, 2.1608) -- cycle;
\fill[blue!87.6, opacity=0.7] (1.8290, 0.3340, 2.1608) -- (1.8760, 0.3340, 2.1584) -- (1.8760, 0.3870, 2.1641) -- (1.8290, 0.3870, 2.1665) -- cycle;
\fill[blue!86.5, opacity=0.7] (1.8290, 0.3870, 2.1665) -- (1.8760, 0.3870, 2.1641) -- (1.8760, 0.4400, 2.1696) -- (1.8290, 0.4400, 2.1720) -- cycle;
\fill[blue!86.8, opacity=0.7] (1.8290, 0.4400, 2.1720) -- (1.8760, 0.4400, 2.1696) -- (1.8760, 0.4930, 2.1750) -- (1.8290, 0.4930, 2.1774) -- cycle;
\fill[blue!61.3, opacity=0.7] (1.8290, 0.4930, 2.1774) -- (1.8760, 0.4930, 2.1750) -- (1.8760, 0.5460, 2.1802) -- (1.8290, 0.5460, 2.1826) -- cycle;
\fill[blue!30.9, opacity=0.7] (1.8290, 0.5460, 2.1826) -- (1.8760, 0.5460, 2.1802) -- (1.8760, 0.5990, 2.1851) -- (1.8290, 0.5990, 2.1875) -- cycle;
\fill[blue!21.9, opacity=0.7] (1.8290, 0.5990, 2.1875) -- (1.8760, 0.5990, 2.1851) -- (1.8760, 0.6520, 2.1899) -- (1.8290, 0.6520, 2.1923) -- cycle;
\fill[blue!24.3, opacity=0.7] (1.8290, 0.6520, 2.1923) -- (1.8760, 0.6520, 2.1899) -- (1.8760, 0.7050, 2.1945) -- (1.8290, 0.7050, 2.1969) -- cycle;
\fill[blue!44.4, opacity=0.7] (1.8290, 0.7050, 2.1969) -- (1.8760, 0.7050, 2.1945) -- (1.8760, 0.7580, 2.1988) -- (1.8290, 0.7580, 2.2012) -- cycle;
\fill[blue!83.1, opacity=0.7] (1.8290, 0.7580, 2.2012) -- (1.8760, 0.7580, 2.1988) -- (1.8760, 0.8110, 2.2029) -- (1.8290, 0.8110, 2.2053) -- cycle;
\fill[blue!75.6, opacity=0.7] (1.8290, 0.8110, 2.2053) -- (1.8760, 0.8110, 2.2029) -- (1.8760, 0.8640, 2.2067) -- (1.8290, 0.8640, 2.2091) -- cycle;
\fill[blue!46.3, opacity=0.7] (1.8290, 0.8640, 2.2091) -- (1.8760, 0.8640, 2.2067) -- (1.8760, 0.9170, 2.2103) -- (1.8290, 0.9170, 2.2127) -- cycle;
\fill[blue!36.1, opacity=0.7] (1.8290, 0.9170, 2.2127) -- (1.8760, 0.9170, 2.2103) -- (1.8760, 0.9700, 2.2135) -- (1.8290, 0.9700, 2.2160) -- cycle;
\fill[blue!40.4, opacity=0.7] (1.8290, 0.9700, 2.2160) -- (1.8760, 0.9700, 2.2135) -- (1.8760, 1.0230, 2.2165) -- (1.8290, 1.0230, 2.2190) -- cycle;
\fill[blue!56.6, opacity=0.7] (1.8290, 1.0230, 2.2190) -- (1.8760, 1.0230, 2.2165) -- (1.8760, 1.0760, 2.2193) -- (1.8290, 1.0760, 2.2217) -- cycle;
\fill[blue!77.1, opacity=0.7] (1.8290, 1.0760, 2.2217) -- (1.8760, 1.0760, 2.2193) -- (1.8760, 1.1290, 2.2217) -- (1.8290, 1.1290, 2.2241) -- cycle;
\fill[blue!87.4, opacity=0.7] (1.8290, 1.1290, 2.2241) -- (1.8760, 1.1290, 2.2217) -- (1.8760, 1.1820, 2.2238) -- (1.8290, 1.1820, 2.2262) -- cycle;
\fill[blue!86.3, opacity=0.7] (1.8290, 1.1820, 2.2262) -- (1.8760, 1.1820, 2.2238) -- (1.8760, 1.2350, 2.2255) -- (1.8290, 1.2350, 2.2279) -- cycle;
\fill[blue!82.4, opacity=0.7] (1.8290, 1.2350, 2.2279) -- (1.8760, 1.2350, 2.2255) -- (1.8760, 1.2880, 2.2270) -- (1.8290, 1.2880, 2.2294) -- cycle;
\fill[blue!80.4, opacity=0.7] (1.8290, 1.2880, 2.2294) -- (1.8760, 1.2880, 2.2270) -- (1.8760, 1.3410, 2.2281) -- (1.8290, 1.3410, 2.2306) -- cycle;
\fill[blue!80.8, opacity=0.7] (1.8290, 1.3410, 2.2306) -- (1.8760, 1.3410, 2.2281) -- (1.8760, 1.3940, 2.2290) -- (1.8290, 1.3940, 2.2314) -- cycle;
\fill[blue!83.2, opacity=0.7] (1.8290, 1.3940, 2.2314) -- (1.8760, 1.3940, 2.2290) -- (1.8760, 1.4470, 2.2295) -- (1.8290, 1.4470, 2.2319) -- cycle;
\fill[blue!86.6, opacity=0.7] (1.8290, 1.4470, 2.2319) -- (1.8760, 1.4470, 2.2295) -- (1.8760, 1.5000, 2.2296) -- (1.8290, 1.5000, 2.2320) -- cycle;
\fill[blue!87.2, opacity=0.7] (1.8290, 1.5000, 2.2320) -- (1.8760, 1.5000, 2.2296) -- (1.8760, 1.5530, 2.2295) -- (1.8290, 1.5530, 2.2319) -- cycle;
\fill[blue!74.7, opacity=0.7] (1.8290, 1.5530, 2.2319) -- (1.8760, 1.5530, 2.2295) -- (1.8760, 1.6060, 2.2290) -- (1.8290, 1.6060, 2.2314) -- cycle;
\fill[blue!48.1, opacity=0.7] (1.8290, 1.6060, 2.2314) -- (1.8760, 1.6060, 2.2290) -- (1.8760, 1.6590, 2.2281) -- (1.8290, 1.6590, 2.2306) -- cycle;
\fill[blue!28.1, opacity=0.7] (1.8290, 1.6590, 2.2306) -- (1.8760, 1.6590, 2.2281) -- (1.8760, 1.7120, 2.2270) -- (1.8290, 1.7120, 2.2294) -- cycle;
\fill[blue!21.8, opacity=0.7] (1.8290, 1.7120, 2.2294) -- (1.8760, 1.7120, 2.2270) -- (1.8760, 1.7650, 2.2255) -- (1.8290, 1.7650, 2.2279) -- cycle;
\fill[blue!24.6, opacity=0.7] (1.8290, 1.7650, 2.2279) -- (1.8760, 1.7650, 2.2255) -- (1.8760, 1.8180, 2.2238) -- (1.8290, 1.8180, 2.2262) -- cycle;
\fill[blue!45.8, opacity=0.7] (1.8290, 1.8180, 2.2262) -- (1.8760, 1.8180, 2.2238) -- (1.8760, 1.8710, 2.2217) -- (1.8290, 1.8710, 2.2241) -- cycle;
\fill[blue!85.4, opacity=0.7] (1.8290, 1.8710, 2.2241) -- (1.8760, 1.8710, 2.2217) -- (1.8760, 1.9240, 2.2193) -- (1.8290, 1.9240, 2.2217) -- cycle;
\fill[blue!69.8, opacity=0.7] (1.8290, 1.9240, 2.2217) -- (1.8760, 1.9240, 2.2193) -- (1.8760, 1.9770, 2.2165) -- (1.8290, 1.9770, 2.2190) -- cycle;
\fill[blue!45.5, opacity=0.7] (1.8290, 1.9770, 2.2190) -- (1.8760, 1.9770, 2.2165) -- (1.8760, 2.0300, 2.2135) -- (1.8290, 2.0300, 2.2160) -- cycle;
\fill[blue!50.5, opacity=0.7] (1.8290, 2.0300, 2.2160) -- (1.8760, 2.0300, 2.2135) -- (1.8760, 2.0830, 2.2103) -- (1.8290, 2.0830, 2.2127) -- cycle;
\fill[blue!79.8, opacity=0.7] (1.8290, 2.0830, 2.2127) -- (1.8760, 2.0830, 2.2103) -- (1.8760, 2.1360, 2.2067) -- (1.8290, 2.1360, 2.2091) -- cycle;
\fill[blue!81.1, opacity=0.7] (1.8290, 2.1360, 2.2091) -- (1.8760, 2.1360, 2.2067) -- (1.8760, 2.1890, 2.2029) -- (1.8290, 2.1890, 2.2053) -- cycle;
\fill[blue!56.6, opacity=0.7] (1.8290, 2.1890, 2.2053) -- (1.8760, 2.1890, 2.2029) -- (1.8760, 2.2420, 2.1988) -- (1.8290, 2.2420, 2.2012) -- cycle;
\fill[blue!59.3, opacity=0.7] (1.8290, 2.2420, 2.2012) -- (1.8760, 2.2420, 2.1988) -- (1.8760, 2.2950, 2.1945) -- (1.8290, 2.2950, 2.1969) -- cycle;
\fill[blue!86.3, opacity=0.7] (1.8290, 2.2950, 2.1969) -- (1.8760, 2.2950, 2.1945) -- (1.8760, 2.3480, 2.1899) -- (1.8290, 2.3480, 2.1923) -- cycle;
\fill[blue!59.7, opacity=0.7] (1.8290, 2.3480, 2.1923) -- (1.8760, 2.3480, 2.1899) -- (1.8760, 2.4010, 2.1851) -- (1.8290, 2.4010, 2.1875) -- cycle;
\fill[blue!23.8, opacity=0.7] (1.8290, 2.4010, 2.1875) -- (1.8760, 2.4010, 2.1851) -- (1.8760, 2.4540, 2.1802) -- (1.8290, 2.4540, 2.1826) -- cycle;
\fill[blue!18.1, opacity=0.7] (1.8290, 2.4540, 2.1826) -- (1.8760, 2.4540, 2.1802) -- (1.8760, 2.5070, 2.1750) -- (1.8290, 2.5070, 2.1774) -- cycle;
\fill[blue!21.4, opacity=0.7] (1.8290, 2.5070, 2.1774) -- (1.8760, 2.5070, 2.1750) -- (1.8760, 2.5600, 2.1696) -- (1.8290, 2.5600, 2.1720) -- cycle;
\fill[blue!44.5, opacity=0.7] (1.8290, 2.5600, 2.1720) -- (1.8760, 2.5600, 2.1696) -- (1.8760, 2.6130, 2.1641) -- (1.8290, 2.6130, 2.1665) -- cycle;
\fill[blue!80.7, opacity=0.7] (1.8290, 2.6130, 2.1665) -- (1.8760, 2.6130, 2.1641) -- (1.8760, 2.6660, 2.1584) -- (1.8290, 2.6660, 2.1608) -- cycle;
\fill[blue!87.7, opacity=0.7] (1.8290, 2.6660, 2.1608) -- (1.8760, 2.6660, 2.1584) -- (1.8760, 2.7190, 2.1526) -- (1.8290, 2.7190, 2.1550) -- cycle;
\fill[blue!80.2, opacity=0.7] (1.8290, 2.7190, 2.1550) -- (1.8760, 2.7190, 2.1526) -- (1.8760, 2.7720, 2.1467) -- (1.8290, 2.7720, 2.1491) -- cycle;
\fill[blue!37.9, opacity=0.7] (1.8290, 2.7720, 2.1491) -- (1.8760, 2.7720, 2.1467) -- (1.8760, 2.8250, 2.1407) -- (1.8290, 2.8250, 2.1431) -- cycle;
\fill[blue!16.7, opacity=0.7] (1.8290, 2.8250, 2.1431) -- (1.8760, 2.8250, 2.1407) -- (1.8760, 2.8780, 2.1346) -- (1.8290, 2.8780, 2.1370) -- cycle;
\fill[blue!15.1, opacity=0.7] (1.8290, 2.8780, 2.1370) -- (1.8760, 2.8780, 2.1346) -- (1.8760, 2.9310, 2.1284) -- (1.8290, 2.9310, 2.1308) -- cycle;
\fill[blue!15.1, opacity=0.7] (1.8290, 2.9310, 2.1308) -- (1.8760, 2.9310, 2.1284) -- (1.8760, 2.9840, 2.1222) -- (1.8290, 2.9840, 2.1246) -- cycle;
\fill[blue!15.7, opacity=0.7] (1.8290, 2.9840, 2.1246) -- (1.8760, 2.9840, 2.1222) -- (1.8760, 3.0370, 2.1159) -- (1.8290, 3.0370, 2.1183) -- cycle;
\fill[blue!23.3, opacity=0.7] (1.8290, 3.0370, 2.1183) -- (1.8760, 3.0370, 2.1159) -- (1.8760, 3.0900, 2.1096) -- (1.8290, 3.0900, 2.1120) -- cycle;
\fill[blue!54.1, opacity=0.7] (1.8760, -0.0900, 2.1096) -- (1.9230, -0.0900, 2.1069) -- (1.9230, -0.0370, 2.1132) -- (1.8760, -0.0370, 2.1159) -- cycle;
\fill[blue!46.1, opacity=0.7] (1.8760, -0.0370, 2.1159) -- (1.9230, -0.0370, 2.1132) -- (1.9230, 0.0160, 2.1195) -- (1.8760, 0.0160, 2.1222) -- cycle;
\fill[blue!24.0, opacity=0.7] (1.8760, 0.0160, 2.1222) -- (1.9230, 0.0160, 2.1195) -- (1.9230, 0.0690, 2.1257) -- (1.8760, 0.0690, 2.1284) -- cycle;
\fill[blue!16.0, opacity=0.7] (1.8760, 0.0690, 2.1284) -- (1.9230, 0.0690, 2.1257) -- (1.9230, 0.1220, 2.1319) -- (1.8760, 0.1220, 2.1346) -- cycle;
\fill[blue!15.2, opacity=0.7] (1.8760, 0.1220, 2.1346) -- (1.9230, 0.1220, 2.1319) -- (1.9230, 0.1750, 2.1380) -- (1.8760, 0.1750, 2.1407) -- cycle;
\fill[blue!15.3, opacity=0.7] (1.8760, 0.1750, 2.1407) -- (1.9230, 0.1750, 2.1380) -- (1.9230, 0.2280, 2.1440) -- (1.8760, 0.2280, 2.1467) -- cycle;
\fill[blue!17.4, opacity=0.7] (1.8760, 0.2280, 2.1467) -- (1.9230, 0.2280, 2.1440) -- (1.9230, 0.2810, 2.1499) -- (1.8760, 0.2810, 2.1526) -- cycle;
\fill[blue!39.0, opacity=0.7] (1.8760, 0.2810, 2.1526) -- (1.9230, 0.2810, 2.1499) -- (1.9230, 0.3340, 2.1557) -- (1.8760, 0.3340, 2.1584) -- cycle;
\fill[blue!81.1, opacity=0.7] (1.8760, 0.3340, 2.1584) -- (1.9230, 0.3340, 2.1557) -- (1.9230, 0.3870, 2.1614) -- (1.8760, 0.3870, 2.1641) -- cycle;
\fill[blue!87.0, opacity=0.7] (1.8760, 0.3870, 2.1641) -- (1.9230, 0.3870, 2.1614) -- (1.9230, 0.4400, 2.1669) -- (1.8760, 0.4400, 2.1696) -- cycle;
\fill[blue!87.3, opacity=0.7] (1.8760, 0.4400, 2.1696) -- (1.9230, 0.4400, 2.1669) -- (1.9230, 0.4930, 2.1723) -- (1.8760, 0.4930, 2.1750) -- cycle;
\fill[blue!81.1, opacity=0.7] (1.8760, 0.4930, 2.1750) -- (1.9230, 0.4930, 2.1723) -- (1.9230, 0.5460, 2.1775) -- (1.8760, 0.5460, 2.1802) -- cycle;
\fill[blue!48.5, opacity=0.7] (1.8760, 0.5460, 2.1802) -- (1.9230, 0.5460, 2.1775) -- (1.9230, 0.5990, 2.1824) -- (1.8760, 0.5990, 2.1851) -- cycle;
\fill[blue!26.6, opacity=0.7] (1.8760, 0.5990, 2.1851) -- (1.9230, 0.5990, 2.1824) -- (1.9230, 0.6520, 2.1872) -- (1.8760, 0.6520, 2.1899) -- cycle;
\fill[blue!22.0, opacity=0.7] (1.8760, 0.6520, 2.1899) -- (1.9230, 0.6520, 2.1872) -- (1.9230, 0.7050, 2.1918) -- (1.8760, 0.7050, 2.1945) -- cycle;
\fill[blue!27.5, opacity=0.7] (1.8760, 0.7050, 2.1945) -- (1.9230, 0.7050, 2.1918) -- (1.9230, 0.7580, 2.1961) -- (1.8760, 0.7580, 2.1988) -- cycle;
\fill[blue!52.5, opacity=0.7] (1.8760, 0.7580, 2.1988) -- (1.9230, 0.7580, 2.1961) -- (1.9230, 0.8110, 2.2002) -- (1.8760, 0.8110, 2.2029) -- cycle;
\fill[blue!86.2, opacity=0.7] (1.8760, 0.8110, 2.2029) -- (1.9230, 0.8110, 2.2002) -- (1.9230, 0.8640, 2.2040) -- (1.8760, 0.8640, 2.2067) -- cycle;
\fill[blue!73.3, opacity=0.7] (1.8760, 0.8640, 2.2067) -- (1.9230, 0.8640, 2.2040) -- (1.9230, 0.9170, 2.2076) -- (1.8760, 0.9170, 2.2103) -- cycle;
\fill[blue!46.1, opacity=0.7] (1.8760, 0.9170, 2.2103) -- (1.9230, 0.9170, 2.2076) -- (1.9230, 0.9700, 2.2108) -- (1.8760, 0.9700, 2.2135) -- cycle;
\fill[blue!35.1, opacity=0.7] (1.8760, 0.9700, 2.2135) -- (1.9230, 0.9700, 2.2108) -- (1.9230, 1.0230, 2.2138) -- (1.8760, 1.0230, 2.2165) -- cycle;
\fill[blue!35.6, opacity=0.7] (1.8760, 1.0230, 2.2165) -- (1.9230, 1.0230, 2.2138) -- (1.9230, 1.0760, 2.2165) -- (1.8760, 1.0760, 2.2193) -- cycle;
\fill[blue!43.9, opacity=0.7] (1.8760, 1.0760, 2.2193) -- (1.9230, 1.0760, 2.2165) -- (1.9230, 1.1290, 2.2190) -- (1.8760, 1.1290, 2.2217) -- cycle;
\fill[blue!57.4, opacity=0.7] (1.8760, 1.1290, 2.2217) -- (1.9230, 1.1290, 2.2190) -- (1.9230, 1.1820, 2.2210) -- (1.8760, 1.1820, 2.2238) -- cycle;
\fill[blue!70.8, opacity=0.7] (1.8760, 1.1820, 2.2238) -- (1.9230, 1.1820, 2.2210) -- (1.9230, 1.2350, 2.2228) -- (1.8760, 1.2350, 2.2255) -- cycle;
\fill[blue!79.4, opacity=0.7] (1.8760, 1.2350, 2.2255) -- (1.9230, 1.2350, 2.2228) -- (1.9230, 1.2880, 2.2243) -- (1.8760, 1.2880, 2.2270) -- cycle;
\fill[blue!83.0, opacity=0.7] (1.8760, 1.2880, 2.2270) -- (1.9230, 1.2880, 2.2243) -- (1.9230, 1.3410, 2.2254) -- (1.8760, 1.3410, 2.2281) -- cycle;
\fill[blue!83.2, opacity=0.7] (1.8760, 1.3410, 2.2281) -- (1.9230, 1.3410, 2.2254) -- (1.9230, 1.3940, 2.2263) -- (1.8760, 1.3940, 2.2290) -- cycle;
\fill[blue!80.2, opacity=0.7] (1.8760, 1.3940, 2.2290) -- (1.9230, 1.3940, 2.2263) -- (1.9230, 1.4470, 2.2268) -- (1.8760, 1.4470, 2.2295) -- cycle;
\fill[blue!71.7, opacity=0.7] (1.8760, 1.4470, 2.2295) -- (1.9230, 1.4470, 2.2268) -- (1.9230, 1.5000, 2.2269) -- (1.8760, 1.5000, 2.2296) -- cycle;
\fill[blue!56.5, opacity=0.7] (1.8760, 1.5000, 2.2296) -- (1.9230, 1.5000, 2.2269) -- (1.9230, 1.5530, 2.2268) -- (1.8760, 1.5530, 2.2295) -- cycle;
\fill[blue!39.0, opacity=0.7] (1.8760, 1.5530, 2.2295) -- (1.9230, 1.5530, 2.2268) -- (1.9230, 1.6060, 2.2263) -- (1.8760, 1.6060, 2.2290) -- cycle;
\fill[blue!27.1, opacity=0.7] (1.8760, 1.6060, 2.2290) -- (1.9230, 1.6060, 2.2263) -- (1.9230, 1.6590, 2.2254) -- (1.8760, 1.6590, 2.2281) -- cycle;
\fill[blue!22.7, opacity=0.7] (1.8760, 1.6590, 2.2281) -- (1.9230, 1.6590, 2.2254) -- (1.9230, 1.7120, 2.2243) -- (1.8760, 1.7120, 2.2270) -- cycle;
\fill[blue!25.0, opacity=0.7] (1.8760, 1.7120, 2.2270) -- (1.9230, 1.7120, 2.2243) -- (1.9230, 1.7650, 2.2228) -- (1.8760, 1.7650, 2.2255) -- cycle;
\fill[blue!40.7, opacity=0.7] (1.8760, 1.7650, 2.2255) -- (1.9230, 1.7650, 2.2228) -- (1.9230, 1.8180, 2.2210) -- (1.8760, 1.8180, 2.2238) -- cycle;
\fill[blue!77.9, opacity=0.7] (1.8760, 1.8180, 2.2238) -- (1.9230, 1.8180, 2.2210) -- (1.9230, 1.8710, 2.2190) -- (1.8760, 1.8710, 2.2217) -- cycle;
\fill[blue!80.7, opacity=0.7] (1.8760, 1.8710, 2.2217) -- (1.9230, 1.8710, 2.2190) -- (1.9230, 1.9240, 2.2165) -- (1.8760, 1.9240, 2.2193) -- cycle;
\fill[blue!50.8, opacity=0.7] (1.8760, 1.9240, 2.2193) -- (1.9230, 1.9240, 2.2165) -- (1.9230, 1.9770, 2.2138) -- (1.8760, 1.9770, 2.2165) -- cycle;
\fill[blue!43.1, opacity=0.7] (1.8760, 1.9770, 2.2165) -- (1.9230, 1.9770, 2.2138) -- (1.9230, 2.0300, 2.2108) -- (1.8760, 2.0300, 2.2135) -- cycle;
\fill[blue!61.5, opacity=0.7] (1.8760, 2.0300, 2.2135) -- (1.9230, 2.0300, 2.2108) -- (1.9230, 2.0830, 2.2076) -- (1.8760, 2.0830, 2.2103) -- cycle;
\fill[blue!87.5, opacity=0.7] (1.8760, 2.0830, 2.2103) -- (1.9230, 2.0830, 2.2076) -- (1.9230, 2.1360, 2.2040) -- (1.8760, 2.1360, 2.2067) -- cycle;
\fill[blue!70.9, opacity=0.7] (1.8760, 2.1360, 2.2067) -- (1.9230, 2.1360, 2.2040) -- (1.9230, 2.1890, 2.2002) -- (1.8760, 2.1890, 2.2029) -- cycle;
\fill[blue!54.8, opacity=0.7] (1.8760, 2.1890, 2.2029) -- (1.9230, 2.1890, 2.2002) -- (1.9230, 2.2420, 2.1961) -- (1.8760, 2.2420, 2.1988) -- cycle;
\fill[blue!69.4, opacity=0.7] (1.8760, 2.2420, 2.1988) -- (1.9230, 2.2420, 2.1961) -- (1.9230, 2.2950, 2.1918) -- (1.8760, 2.2950, 2.1945) -- cycle;
\fill[blue!86.5, opacity=0.7] (1.8760, 2.2950, 2.1945) -- (1.9230, 2.2950, 2.1918) -- (1.9230, 2.3480, 2.1872) -- (1.8760, 2.3480, 2.1899) -- cycle;
\fill[blue!42.9, opacity=0.7] (1.8760, 2.3480, 2.1899) -- (1.9230, 2.3480, 2.1872) -- (1.9230, 2.4010, 2.1824) -- (1.8760, 2.4010, 2.1851) -- cycle;
\fill[blue!20.3, opacity=0.7] (1.8760, 2.4010, 2.1851) -- (1.9230, 2.4010, 2.1824) -- (1.9230, 2.4540, 2.1775) -- (1.8760, 2.4540, 2.1802) -- cycle;
\fill[blue!18.0, opacity=0.7] (1.8760, 2.4540, 2.1802) -- (1.9230, 2.4540, 2.1775) -- (1.9230, 2.5070, 2.1723) -- (1.8760, 2.5070, 2.1750) -- cycle;
\fill[blue!24.3, opacity=0.7] (1.8760, 2.5070, 2.1750) -- (1.9230, 2.5070, 2.1723) -- (1.9230, 2.5600, 2.1669) -- (1.8760, 2.5600, 2.1696) -- cycle;
\fill[blue!54.1, opacity=0.7] (1.8760, 2.5600, 2.1696) -- (1.9230, 2.5600, 2.1669) -- (1.9230, 2.6130, 2.1614) -- (1.8760, 2.6130, 2.1641) -- cycle;
\fill[blue!84.3, opacity=0.7] (1.8760, 2.6130, 2.1641) -- (1.9230, 2.6130, 2.1614) -- (1.9230, 2.6660, 2.1557) -- (1.8760, 2.6660, 2.1584) -- cycle;
\fill[blue!87.4, opacity=0.7] (1.8760, 2.6660, 2.1584) -- (1.9230, 2.6660, 2.1557) -- (1.9230, 2.7190, 2.1499) -- (1.8760, 2.7190, 2.1526) -- cycle;
\fill[blue!72.9, opacity=0.7] (1.8760, 2.7190, 2.1526) -- (1.9230, 2.7190, 2.1499) -- (1.9230, 2.7720, 2.1440) -- (1.8760, 2.7720, 2.1467) -- cycle;
\fill[blue!29.3, opacity=0.7] (1.8760, 2.7720, 2.1467) -- (1.9230, 2.7720, 2.1440) -- (1.9230, 2.8250, 2.1380) -- (1.8760, 2.8250, 2.1407) -- cycle;
\fill[blue!15.8, opacity=0.7] (1.8760, 2.8250, 2.1407) -- (1.9230, 2.8250, 2.1380) -- (1.9230, 2.8780, 2.1319) -- (1.8760, 2.8780, 2.1346) -- cycle;
\fill[blue!15.1, opacity=0.7] (1.8760, 2.8780, 2.1346) -- (1.9230, 2.8780, 2.1319) -- (1.9230, 2.9310, 2.1257) -- (1.8760, 2.9310, 2.1284) -- cycle;
\fill[blue!15.1, opacity=0.7] (1.8760, 2.9310, 2.1284) -- (1.9230, 2.9310, 2.1257) -- (1.9230, 2.9840, 2.1195) -- (1.8760, 2.9840, 2.1222) -- cycle;
\fill[blue!16.1, opacity=0.7] (1.8760, 2.9840, 2.1222) -- (1.9230, 2.9840, 2.1195) -- (1.9230, 3.0370, 2.1132) -- (1.8760, 3.0370, 2.1159) -- cycle;
\fill[blue!26.1, opacity=0.7] (1.8760, 3.0370, 2.1159) -- (1.9230, 3.0370, 2.1132) -- (1.9230, 3.0900, 2.1069) -- (1.8760, 3.0900, 2.1096) -- cycle;
\fill[blue!49.2, opacity=0.7] (1.9230, -0.0900, 2.1069) -- (1.9700, -0.0900, 2.1039) -- (1.9700, -0.0370, 2.1102) -- (1.9230, -0.0370, 2.1132) -- cycle;
\fill[blue!53.5, opacity=0.7] (1.9230, -0.0370, 2.1132) -- (1.9700, -0.0370, 2.1102) -- (1.9700, 0.0160, 2.1165) -- (1.9230, 0.0160, 2.1195) -- cycle;
\fill[blue!33.1, opacity=0.7] (1.9230, 0.0160, 2.1195) -- (1.9700, 0.0160, 2.1165) -- (1.9700, 0.0690, 2.1227) -- (1.9230, 0.0690, 2.1257) -- cycle;
\fill[blue!18.0, opacity=0.7] (1.9230, 0.0690, 2.1257) -- (1.9700, 0.0690, 2.1227) -- (1.9700, 0.1220, 2.1289) -- (1.9230, 0.1220, 2.1319) -- cycle;
\fill[blue!15.4, opacity=0.7] (1.9230, 0.1220, 2.1319) -- (1.9700, 0.1220, 2.1289) -- (1.9700, 0.1750, 2.1350) -- (1.9230, 0.1750, 2.1380) -- cycle;
\fill[blue!15.2, opacity=0.7] (1.9230, 0.1750, 2.1380) -- (1.9700, 0.1750, 2.1350) -- (1.9700, 0.2280, 2.1410) -- (1.9230, 0.2280, 2.1440) -- cycle;
\fill[blue!15.7, opacity=0.7] (1.9230, 0.2280, 2.1440) -- (1.9700, 0.2280, 2.1410) -- (1.9700, 0.2810, 2.1469) -- (1.9230, 0.2810, 2.1499) -- cycle;
\fill[blue!23.4, opacity=0.7] (1.9230, 0.2810, 2.1499) -- (1.9700, 0.2810, 2.1469) -- (1.9700, 0.3340, 2.1527) -- (1.9230, 0.3340, 2.1557) -- cycle;
\fill[blue!60.6, opacity=0.7] (1.9230, 0.3340, 2.1557) -- (1.9700, 0.3340, 2.1527) -- (1.9700, 0.3870, 2.1584) -- (1.9230, 0.3870, 2.1614) -- cycle;
\fill[blue!87.5, opacity=0.7] (1.9230, 0.3870, 2.1614) -- (1.9700, 0.3870, 2.1584) -- (1.9700, 0.4400, 2.1639) -- (1.9230, 0.4400, 2.1669) -- cycle;
\fill[blue!85.8, opacity=0.7] (1.9230, 0.4400, 2.1669) -- (1.9700, 0.4400, 2.1639) -- (1.9700, 0.4930, 2.1693) -- (1.9230, 0.4930, 2.1723) -- cycle;
\fill[blue!87.9, opacity=0.7] (1.9230, 0.4930, 2.1723) -- (1.9700, 0.4930, 2.1693) -- (1.9700, 0.5460, 2.1745) -- (1.9230, 0.5460, 2.1775) -- cycle;
\fill[blue!73.4, opacity=0.7] (1.9230, 0.5460, 2.1775) -- (1.9700, 0.5460, 2.1745) -- (1.9700, 0.5990, 2.1794) -- (1.9230, 0.5990, 2.1824) -- cycle;
\fill[blue!41.1, opacity=0.7] (1.9230, 0.5990, 2.1824) -- (1.9700, 0.5990, 2.1794) -- (1.9700, 0.6520, 2.1842) -- (1.9230, 0.6520, 2.1872) -- cycle;
\fill[blue!25.1, opacity=0.7] (1.9230, 0.6520, 2.1872) -- (1.9700, 0.6520, 2.1842) -- (1.9700, 0.7050, 2.1888) -- (1.9230, 0.7050, 2.1918) -- cycle;
\fill[blue!22.7, opacity=0.7] (1.9230, 0.7050, 2.1918) -- (1.9700, 0.7050, 2.1888) -- (1.9700, 0.7580, 2.1931) -- (1.9230, 0.7580, 2.1961) -- cycle;
\fill[blue!29.7, opacity=0.7] (1.9230, 0.7580, 2.1961) -- (1.9700, 0.7580, 2.1931) -- (1.9700, 0.8110, 2.1972) -- (1.9230, 0.8110, 2.2002) -- cycle;
\fill[blue!54.9, opacity=0.7] (1.9230, 0.8110, 2.2002) -- (1.9700, 0.8110, 2.1972) -- (1.9700, 0.8640, 2.2010) -- (1.9230, 0.8640, 2.2040) -- cycle;
\fill[blue!85.7, opacity=0.7] (1.9230, 0.8640, 2.2040) -- (1.9700, 0.8640, 2.2010) -- (1.9700, 0.9170, 2.2046) -- (1.9230, 0.9170, 2.2076) -- cycle;
\fill[blue!77.0, opacity=0.7] (1.9230, 0.9170, 2.2076) -- (1.9700, 0.9170, 2.2046) -- (1.9700, 0.9700, 2.2078) -- (1.9230, 0.9700, 2.2108) -- cycle;
\fill[blue!50.9, opacity=0.7] (1.9230, 0.9700, 2.2108) -- (1.9700, 0.9700, 2.2078) -- (1.9700, 1.0230, 2.2108) -- (1.9230, 1.0230, 2.2138) -- cycle;
\fill[blue!36.6, opacity=0.7] (1.9230, 1.0230, 2.2138) -- (1.9700, 1.0230, 2.2108) -- (1.9700, 1.0760, 2.2135) -- (1.9230, 1.0760, 2.2165) -- cycle;
\fill[blue!32.4, opacity=0.7] (1.9230, 1.0760, 2.2165) -- (1.9700, 1.0760, 2.2135) -- (1.9700, 1.1290, 2.2160) -- (1.9230, 1.1290, 2.2190) -- cycle;
\fill[blue!33.4, opacity=0.7] (1.9230, 1.1290, 2.2190) -- (1.9700, 1.1290, 2.2160) -- (1.9700, 1.1820, 2.2180) -- (1.9230, 1.1820, 2.2210) -- cycle;
\fill[blue!37.1, opacity=0.7] (1.9230, 1.1820, 2.2210) -- (1.9700, 1.1820, 2.2180) -- (1.9700, 1.2350, 2.2198) -- (1.9230, 1.2350, 2.2228) -- cycle;
\fill[blue!41.5, opacity=0.7] (1.9230, 1.2350, 2.2228) -- (1.9700, 1.2350, 2.2198) -- (1.9700, 1.2880, 2.2213) -- (1.9230, 1.2880, 2.2243) -- cycle;
\fill[blue!44.4, opacity=0.7] (1.9230, 1.2880, 2.2243) -- (1.9700, 1.2880, 2.2213) -- (1.9700, 1.3410, 2.2224) -- (1.9230, 1.3410, 2.2254) -- cycle;
\fill[blue!44.3, opacity=0.7] (1.9230, 1.3410, 2.2254) -- (1.9700, 1.3410, 2.2224) -- (1.9700, 1.3940, 2.2233) -- (1.9230, 1.3940, 2.2263) -- cycle;
\fill[blue!40.9, opacity=0.7] (1.9230, 1.3940, 2.2263) -- (1.9700, 1.3940, 2.2233) -- (1.9700, 1.4470, 2.2238) -- (1.9230, 1.4470, 2.2268) -- cycle;
\fill[blue!35.0, opacity=0.7] (1.9230, 1.4470, 2.2268) -- (1.9700, 1.4470, 2.2238) -- (1.9700, 1.5000, 2.2239) -- (1.9230, 1.5000, 2.2269) -- cycle;
\fill[blue!29.0, opacity=0.7] (1.9230, 1.5000, 2.2269) -- (1.9700, 1.5000, 2.2239) -- (1.9700, 1.5530, 2.2238) -- (1.9230, 1.5530, 2.2268) -- cycle;
\fill[blue!24.9, opacity=0.7] (1.9230, 1.5530, 2.2268) -- (1.9700, 1.5530, 2.2238) -- (1.9700, 1.6060, 2.2233) -- (1.9230, 1.6060, 2.2263) -- cycle;
\fill[blue!24.0, opacity=0.7] (1.9230, 1.6060, 2.2263) -- (1.9700, 1.6060, 2.2233) -- (1.9700, 1.6590, 2.2224) -- (1.9230, 1.6590, 2.2254) -- cycle;
\fill[blue!28.1, opacity=0.7] (1.9230, 1.6590, 2.2254) -- (1.9700, 1.6590, 2.2224) -- (1.9700, 1.7120, 2.2213) -- (1.9230, 1.7120, 2.2243) -- cycle;
\fill[blue!44.2, opacity=0.7] (1.9230, 1.7120, 2.2243) -- (1.9700, 1.7120, 2.2213) -- (1.9700, 1.7650, 2.2198) -- (1.9230, 1.7650, 2.2228) -- cycle;
\fill[blue!76.9, opacity=0.7] (1.9230, 1.7650, 2.2228) -- (1.9700, 1.7650, 2.2198) -- (1.9700, 1.8180, 2.2180) -- (1.9230, 1.8180, 2.2210) -- cycle;
\fill[blue!84.1, opacity=0.7] (1.9230, 1.8180, 2.2210) -- (1.9700, 1.8180, 2.2180) -- (1.9700, 1.8710, 2.2160) -- (1.9230, 1.8710, 2.2190) -- cycle;
\fill[blue!55.7, opacity=0.7] (1.9230, 1.8710, 2.2190) -- (1.9700, 1.8710, 2.2160) -- (1.9700, 1.9240, 2.2135) -- (1.9230, 1.9240, 2.2165) -- cycle;
\fill[blue!41.4, opacity=0.7] (1.9230, 1.9240, 2.2165) -- (1.9700, 1.9240, 2.2135) -- (1.9700, 1.9770, 2.2108) -- (1.9230, 1.9770, 2.2138) -- cycle;
\fill[blue!49.6, opacity=0.7] (1.9230, 1.9770, 2.2138) -- (1.9700, 1.9770, 2.2108) -- (1.9700, 2.0300, 2.2078) -- (1.9230, 2.0300, 2.2108) -- cycle;
\fill[blue!78.0, opacity=0.7] (1.9230, 2.0300, 2.2108) -- (1.9700, 2.0300, 2.2078) -- (1.9700, 2.0830, 2.2046) -- (1.9230, 2.0830, 2.2076) -- cycle;
\fill[blue!84.1, opacity=0.7] (1.9230, 2.0830, 2.2076) -- (1.9700, 2.0830, 2.2046) -- (1.9700, 2.1360, 2.2010) -- (1.9230, 2.1360, 2.2040) -- cycle;
\fill[blue!61.1, opacity=0.7] (1.9230, 2.1360, 2.2040) -- (1.9700, 2.1360, 2.2010) -- (1.9700, 2.1890, 2.1972) -- (1.9230, 2.1890, 2.2002) -- cycle;
\fill[blue!58.5, opacity=0.7] (1.9230, 2.1890, 2.2002) -- (1.9700, 2.1890, 2.1972) -- (1.9700, 2.2420, 2.1931) -- (1.9230, 2.2420, 2.1961) -- cycle;
\fill[blue!82.3, opacity=0.7] (1.9230, 2.2420, 2.1961) -- (1.9700, 2.2420, 2.1931) -- (1.9700, 2.2950, 2.1888) -- (1.9230, 2.2950, 2.1918) -- cycle;
\fill[blue!72.5, opacity=0.7] (1.9230, 2.2950, 2.1918) -- (1.9700, 2.2950, 2.1888) -- (1.9700, 2.3480, 2.1842) -- (1.9230, 2.3480, 2.1872) -- cycle;
\fill[blue!29.3, opacity=0.7] (1.9230, 2.3480, 2.1872) -- (1.9700, 2.3480, 2.1842) -- (1.9700, 2.4010, 2.1794) -- (1.9230, 2.4010, 2.1824) -- cycle;
\fill[blue!18.3, opacity=0.7] (1.9230, 2.4010, 2.1824) -- (1.9700, 2.4010, 2.1794) -- (1.9700, 2.4540, 2.1745) -- (1.9230, 2.4540, 2.1775) -- cycle;
\fill[blue!18.6, opacity=0.7] (1.9230, 2.4540, 2.1775) -- (1.9700, 2.4540, 2.1745) -- (1.9700, 2.5070, 2.1693) -- (1.9230, 2.5070, 2.1723) -- cycle;
\fill[blue!29.9, opacity=0.7] (1.9230, 2.5070, 2.1723) -- (1.9700, 2.5070, 2.1693) -- (1.9700, 2.5600, 2.1639) -- (1.9230, 2.5600, 2.1669) -- cycle;
\fill[blue!65.5, opacity=0.7] (1.9230, 2.5600, 2.1669) -- (1.9700, 2.5600, 2.1639) -- (1.9700, 2.6130, 2.1584) -- (1.9230, 2.6130, 2.1614) -- cycle;
\fill[blue!86.4, opacity=0.7] (1.9230, 2.6130, 2.1614) -- (1.9700, 2.6130, 2.1584) -- (1.9700, 2.6660, 2.1527) -- (1.9230, 2.6660, 2.1557) -- cycle;
\fill[blue!86.3, opacity=0.7] (1.9230, 2.6660, 2.1557) -- (1.9700, 2.6660, 2.1527) -- (1.9700, 2.7190, 2.1469) -- (1.9230, 2.7190, 2.1499) -- cycle;
\fill[blue!61.1, opacity=0.7] (1.9230, 2.7190, 2.1499) -- (1.9700, 2.7190, 2.1469) -- (1.9700, 2.7720, 2.1410) -- (1.9230, 2.7720, 2.1440) -- cycle;
\fill[blue!22.4, opacity=0.7] (1.9230, 2.7720, 2.1440) -- (1.9700, 2.7720, 2.1410) -- (1.9700, 2.8250, 2.1350) -- (1.9230, 2.8250, 2.1380) -- cycle;
\fill[blue!15.4, opacity=0.7] (1.9230, 2.8250, 2.1380) -- (1.9700, 2.8250, 2.1350) -- (1.9700, 2.8780, 2.1289) -- (1.9230, 2.8780, 2.1319) -- cycle;
\fill[blue!15.1, opacity=0.7] (1.9230, 2.8780, 2.1319) -- (1.9700, 2.8780, 2.1289) -- (1.9700, 2.9310, 2.1227) -- (1.9230, 2.9310, 2.1257) -- cycle;
\fill[blue!15.2, opacity=0.7] (1.9230, 2.9310, 2.1257) -- (1.9700, 2.9310, 2.1227) -- (1.9700, 2.9840, 2.1165) -- (1.9230, 2.9840, 2.1195) -- cycle;
\fill[blue!17.0, opacity=0.7] (1.9230, 2.9840, 2.1195) -- (1.9700, 2.9840, 2.1165) -- (1.9700, 3.0370, 2.1102) -- (1.9230, 3.0370, 2.1132) -- cycle;
\fill[blue!29.9, opacity=0.7] (1.9230, 3.0370, 2.1132) -- (1.9700, 3.0370, 2.1102) -- (1.9700, 3.0900, 2.1039) -- (1.9230, 3.0900, 2.1069) -- cycle;
\fill[blue!39.4, opacity=0.7] (1.9700, -0.0900, 2.1039) -- (2.0170, -0.0900, 2.1006) -- (2.0170, -0.0370, 2.1069) -- (1.9700, -0.0370, 2.1102) -- cycle;
\fill[blue!56.1, opacity=0.7] (1.9700, -0.0370, 2.1102) -- (2.0170, -0.0370, 2.1069) -- (2.0170, 0.0160, 2.1132) -- (1.9700, 0.0160, 2.1165) -- cycle;
\fill[blue!44.8, opacity=0.7] (1.9700, 0.0160, 2.1165) -- (2.0170, 0.0160, 2.1132) -- (2.0170, 0.0690, 2.1194) -- (1.9700, 0.0690, 2.1227) -- cycle;
\fill[blue!23.3, opacity=0.7] (1.9700, 0.0690, 2.1227) -- (2.0170, 0.0690, 2.1194) -- (2.0170, 0.1220, 2.1256) -- (1.9700, 0.1220, 2.1289) -- cycle;
\fill[blue!16.0, opacity=0.7] (1.9700, 0.1220, 2.1289) -- (2.0170, 0.1220, 2.1256) -- (2.0170, 0.1750, 2.1317) -- (1.9700, 0.1750, 2.1350) -- cycle;
\fill[blue!15.2, opacity=0.7] (1.9700, 0.1750, 2.1350) -- (2.0170, 0.1750, 2.1317) -- (2.0170, 0.2280, 2.1377) -- (1.9700, 0.2280, 2.1410) -- cycle;
\fill[blue!15.3, opacity=0.7] (1.9700, 0.2280, 2.1410) -- (2.0170, 0.2280, 2.1377) -- (2.0170, 0.2810, 2.1436) -- (1.9700, 0.2810, 2.1469) -- cycle;
\fill[blue!17.2, opacity=0.7] (1.9700, 0.2810, 2.1469) -- (2.0170, 0.2810, 2.1436) -- (2.0170, 0.3340, 2.1494) -- (1.9700, 0.3340, 2.1527) -- cycle;
\fill[blue!34.9, opacity=0.7] (1.9700, 0.3340, 2.1527) -- (2.0170, 0.3340, 2.1494) -- (2.0170, 0.3870, 2.1551) -- (1.9700, 0.3870, 2.1584) -- cycle;
\fill[blue!76.6, opacity=0.7] (1.9700, 0.3870, 2.1584) -- (2.0170, 0.3870, 2.1551) -- (2.0170, 0.4400, 2.1606) -- (1.9700, 0.4400, 2.1639) -- cycle;
\fill[blue!87.5, opacity=0.7] (1.9700, 0.4400, 2.1639) -- (2.0170, 0.4400, 2.1606) -- (2.0170, 0.4930, 2.1660) -- (1.9700, 0.4930, 2.1693) -- cycle;
\fill[blue!85.6, opacity=0.7] (1.9700, 0.4930, 2.1693) -- (2.0170, 0.4930, 2.1660) -- (2.0170, 0.5460, 2.1712) -- (1.9700, 0.5460, 2.1745) -- cycle;
\fill[blue!87.3, opacity=0.7] (1.9700, 0.5460, 2.1745) -- (2.0170, 0.5460, 2.1712) -- (2.0170, 0.5990, 2.1762) -- (1.9700, 0.5990, 2.1794) -- cycle;
\fill[blue!67.8, opacity=0.7] (1.9700, 0.5990, 2.1794) -- (2.0170, 0.5990, 2.1762) -- (2.0170, 0.6520, 2.1809) -- (1.9700, 0.6520, 2.1842) -- cycle;
\fill[blue!38.3, opacity=0.7] (1.9700, 0.6520, 2.1842) -- (2.0170, 0.6520, 2.1809) -- (2.0170, 0.7050, 2.1855) -- (1.9700, 0.7050, 2.1888) -- cycle;
\fill[blue!25.1, opacity=0.7] (1.9700, 0.7050, 2.1888) -- (2.0170, 0.7050, 2.1855) -- (2.0170, 0.7580, 2.1898) -- (1.9700, 0.7580, 2.1931) -- cycle;
\fill[blue!23.3, opacity=0.7] (1.9700, 0.7580, 2.1931) -- (2.0170, 0.7580, 2.1898) -- (2.0170, 0.8110, 2.1939) -- (1.9700, 0.8110, 2.1972) -- cycle;
\fill[blue!29.8, opacity=0.7] (1.9700, 0.8110, 2.1972) -- (2.0170, 0.8110, 2.1939) -- (2.0170, 0.8640, 2.1977) -- (1.9700, 0.8640, 2.2010) -- cycle;
\fill[blue!51.1, opacity=0.7] (1.9700, 0.8640, 2.2010) -- (2.0170, 0.8640, 2.1977) -- (2.0170, 0.9170, 2.2013) -- (1.9700, 0.9170, 2.2046) -- cycle;
\fill[blue!81.0, opacity=0.7] (1.9700, 0.9170, 2.2046) -- (2.0170, 0.9170, 2.2013) -- (2.0170, 0.9700, 2.2046) -- (1.9700, 0.9700, 2.2078) -- cycle;
\fill[blue!84.9, opacity=0.7] (1.9700, 0.9700, 2.2078) -- (2.0170, 0.9700, 2.2046) -- (2.0170, 1.0230, 2.2076) -- (1.9700, 1.0230, 2.2108) -- cycle;
\fill[blue!63.8, opacity=0.7] (1.9700, 1.0230, 2.2108) -- (2.0170, 1.0230, 2.2076) -- (2.0170, 1.0760, 2.2103) -- (1.9700, 1.0760, 2.2135) -- cycle;
\fill[blue!45.1, opacity=0.7] (1.9700, 1.0760, 2.2135) -- (2.0170, 1.0760, 2.2103) -- (2.0170, 1.1290, 2.2127) -- (1.9700, 1.1290, 2.2160) -- cycle;
\fill[blue!35.4, opacity=0.7] (1.9700, 1.1290, 2.2160) -- (2.0170, 1.1290, 2.2127) -- (2.0170, 1.1820, 2.2148) -- (1.9700, 1.1820, 2.2180) -- cycle;
\fill[blue!31.2, opacity=0.7] (1.9700, 1.1820, 2.2180) -- (2.0170, 1.1820, 2.2148) -- (2.0170, 1.2350, 2.2166) -- (1.9700, 1.2350, 2.2198) -- cycle;
\fill[blue!29.5, opacity=0.7] (1.9700, 1.2350, 2.2198) -- (2.0170, 1.2350, 2.2166) -- (2.0170, 1.2880, 2.2180) -- (1.9700, 1.2880, 2.2213) -- cycle;
\fill[blue!28.7, opacity=0.7] (1.9700, 1.2880, 2.2213) -- (2.0170, 1.2880, 2.2180) -- (2.0170, 1.3410, 2.2192) -- (1.9700, 1.3410, 2.2224) -- cycle;
\fill[blue!27.9, opacity=0.7] (1.9700, 1.3410, 2.2224) -- (2.0170, 1.3410, 2.2192) -- (2.0170, 1.3940, 2.2200) -- (1.9700, 1.3940, 2.2233) -- cycle;
\fill[blue!26.9, opacity=0.7] (1.9700, 1.3940, 2.2233) -- (2.0170, 1.3940, 2.2200) -- (2.0170, 1.4470, 2.2205) -- (1.9700, 1.4470, 2.2238) -- cycle;
\fill[blue!26.1, opacity=0.7] (1.9700, 1.4470, 2.2238) -- (2.0170, 1.4470, 2.2205) -- (2.0170, 1.5000, 2.2206) -- (1.9700, 1.5000, 2.2239) -- cycle;
\fill[blue!26.3, opacity=0.7] (1.9700, 1.5000, 2.2239) -- (2.0170, 1.5000, 2.2206) -- (2.0170, 1.5530, 2.2205) -- (1.9700, 1.5530, 2.2238) -- cycle;
\fill[blue!29.0, opacity=0.7] (1.9700, 1.5530, 2.2238) -- (2.0170, 1.5530, 2.2205) -- (2.0170, 1.6060, 2.2200) -- (1.9700, 1.6060, 2.2233) -- cycle;
\fill[blue!37.4, opacity=0.7] (1.9700, 1.6060, 2.2233) -- (2.0170, 1.6060, 2.2200) -- (2.0170, 1.6590, 2.2192) -- (1.9700, 1.6590, 2.2224) -- cycle;
\fill[blue!57.3, opacity=0.7] (1.9700, 1.6590, 2.2224) -- (2.0170, 1.6590, 2.2192) -- (2.0170, 1.7120, 2.2180) -- (1.9700, 1.7120, 2.2213) -- cycle;
\fill[blue!83.3, opacity=0.7] (1.9700, 1.7120, 2.2213) -- (2.0170, 1.7120, 2.2180) -- (2.0170, 1.7650, 2.2166) -- (1.9700, 1.7650, 2.2198) -- cycle;
\fill[blue!82.0, opacity=0.7] (1.9700, 1.7650, 2.2198) -- (2.0170, 1.7650, 2.2166) -- (2.0170, 1.8180, 2.2148) -- (1.9700, 1.8180, 2.2180) -- cycle;
\fill[blue!55.7, opacity=0.7] (1.9700, 1.8180, 2.2180) -- (2.0170, 1.8180, 2.2148) -- (2.0170, 1.8710, 2.2127) -- (1.9700, 1.8710, 2.2160) -- cycle;
\fill[blue!40.6, opacity=0.7] (1.9700, 1.8710, 2.2160) -- (2.0170, 1.8710, 2.2127) -- (2.0170, 1.9240, 2.2103) -- (1.9700, 1.9240, 2.2135) -- cycle;
\fill[blue!44.0, opacity=0.7] (1.9700, 1.9240, 2.2135) -- (2.0170, 1.9240, 2.2103) -- (2.0170, 1.9770, 2.2076) -- (1.9700, 1.9770, 2.2108) -- cycle;
\fill[blue!67.1, opacity=0.7] (1.9700, 1.9770, 2.2108) -- (2.0170, 1.9770, 2.2076) -- (2.0170, 2.0300, 2.2046) -- (1.9700, 2.0300, 2.2078) -- cycle;
\fill[blue!87.8, opacity=0.7] (1.9700, 2.0300, 2.2078) -- (2.0170, 2.0300, 2.2046) -- (2.0170, 2.0830, 2.2013) -- (1.9700, 2.0830, 2.2046) -- cycle;
\fill[blue!71.2, opacity=0.7] (1.9700, 2.0830, 2.2046) -- (2.0170, 2.0830, 2.2013) -- (2.0170, 2.1360, 2.1977) -- (1.9700, 2.1360, 2.2010) -- cycle;
\fill[blue!57.2, opacity=0.7] (1.9700, 2.1360, 2.2010) -- (2.0170, 2.1360, 2.1977) -- (2.0170, 2.1890, 2.1939) -- (1.9700, 2.1890, 2.1972) -- cycle;
\fill[blue!69.9, opacity=0.7] (1.9700, 2.1890, 2.1972) -- (2.0170, 2.1890, 2.1939) -- (2.0170, 2.2420, 2.1898) -- (1.9700, 2.2420, 2.1931) -- cycle;
\fill[blue!87.4, opacity=0.7] (1.9700, 2.2420, 2.1931) -- (2.0170, 2.2420, 2.1898) -- (2.0170, 2.2950, 2.1855) -- (1.9700, 2.2950, 2.1888) -- cycle;
\fill[blue!48.7, opacity=0.7] (1.9700, 2.2950, 2.1888) -- (2.0170, 2.2950, 2.1855) -- (2.0170, 2.3480, 2.1809) -- (1.9700, 2.3480, 2.1842) -- cycle;
\fill[blue!21.6, opacity=0.7] (1.9700, 2.3480, 2.1842) -- (2.0170, 2.3480, 2.1809) -- (2.0170, 2.4010, 2.1762) -- (1.9700, 2.4010, 2.1794) -- cycle;
\fill[blue!17.6, opacity=0.7] (1.9700, 2.4010, 2.1794) -- (2.0170, 2.4010, 2.1762) -- (2.0170, 2.4540, 2.1712) -- (1.9700, 2.4540, 2.1745) -- cycle;
\fill[blue!20.5, opacity=0.7] (1.9700, 2.4540, 2.1745) -- (2.0170, 2.4540, 2.1712) -- (2.0170, 2.5070, 2.1660) -- (1.9700, 2.5070, 2.1693) -- cycle;
\fill[blue!39.9, opacity=0.7] (1.9700, 2.5070, 2.1693) -- (2.0170, 2.5070, 2.1660) -- (2.0170, 2.5600, 2.1606) -- (1.9700, 2.5600, 2.1639) -- cycle;
\fill[blue!76.2, opacity=0.7] (1.9700, 2.5600, 2.1639) -- (2.0170, 2.5600, 2.1606) -- (2.0170, 2.6130, 2.1551) -- (1.9700, 2.6130, 2.1584) -- cycle;
\fill[blue!87.2, opacity=0.7] (1.9700, 2.6130, 2.1584) -- (2.0170, 2.6130, 2.1551) -- (2.0170, 2.6660, 2.1494) -- (1.9700, 2.6660, 2.1527) -- cycle;
\fill[blue!82.5, opacity=0.7] (1.9700, 2.6660, 2.1527) -- (2.0170, 2.6660, 2.1494) -- (2.0170, 2.7190, 2.1436) -- (1.9700, 2.7190, 2.1469) -- cycle;
\fill[blue!45.9, opacity=0.7] (1.9700, 2.7190, 2.1469) -- (2.0170, 2.7190, 2.1436) -- (2.0170, 2.7720, 2.1377) -- (1.9700, 2.7720, 2.1410) -- cycle;
\fill[blue!18.1, opacity=0.7] (1.9700, 2.7720, 2.1410) -- (2.0170, 2.7720, 2.1377) -- (2.0170, 2.8250, 2.1317) -- (1.9700, 2.8250, 2.1350) -- cycle;
\fill[blue!15.2, opacity=0.7] (1.9700, 2.8250, 2.1350) -- (2.0170, 2.8250, 2.1317) -- (2.0170, 2.8780, 2.1256) -- (1.9700, 2.8780, 2.1289) -- cycle;
\fill[blue!15.1, opacity=0.7] (1.9700, 2.8780, 2.1289) -- (2.0170, 2.8780, 2.1256) -- (2.0170, 2.9310, 2.1194) -- (1.9700, 2.9310, 2.1227) -- cycle;
\fill[blue!15.3, opacity=0.7] (1.9700, 2.9310, 2.1227) -- (2.0170, 2.9310, 2.1194) -- (2.0170, 2.9840, 2.1132) -- (1.9700, 2.9840, 2.1165) -- cycle;
\fill[blue!18.6, opacity=0.7] (1.9700, 2.9840, 2.1165) -- (2.0170, 2.9840, 2.1132) -- (2.0170, 3.0370, 2.1069) -- (1.9700, 3.0370, 2.1102) -- cycle;
\fill[blue!34.4, opacity=0.7] (1.9700, 3.0370, 2.1102) -- (2.0170, 3.0370, 2.1069) -- (2.0170, 3.0900, 2.1006) -- (1.9700, 3.0900, 2.1039) -- cycle;
\fill[blue!27.7, opacity=0.7] (2.0170, -0.0900, 2.1006) -- (2.0640, -0.0900, 2.0971) -- (2.0640, -0.0370, 2.1034) -- (2.0170, -0.0370, 2.1069) -- cycle;
\fill[blue!51.7, opacity=0.7] (2.0170, -0.0370, 2.1069) -- (2.0640, -0.0370, 2.1034) -- (2.0640, 0.0160, 2.1096) -- (2.0170, 0.0160, 2.1132) -- cycle;
\fill[blue!54.4, opacity=0.7] (2.0170, 0.0160, 2.1132) -- (2.0640, 0.0160, 2.1096) -- (2.0640, 0.0690, 2.1159) -- (2.0170, 0.0690, 2.1194) -- cycle;
\fill[blue!34.0, opacity=0.7] (2.0170, 0.0690, 2.1194) -- (2.0640, 0.0690, 2.1159) -- (2.0640, 0.1220, 2.1220) -- (2.0170, 0.1220, 2.1256) -- cycle;
\fill[blue!18.5, opacity=0.7] (2.0170, 0.1220, 2.1256) -- (2.0640, 0.1220, 2.1220) -- (2.0640, 0.1750, 2.1281) -- (2.0170, 0.1750, 2.1317) -- cycle;
\fill[blue!15.5, opacity=0.7] (2.0170, 0.1750, 2.1317) -- (2.0640, 0.1750, 2.1281) -- (2.0640, 0.2280, 2.1342) -- (2.0170, 0.2280, 2.1377) -- cycle;
\fill[blue!15.2, opacity=0.7] (2.0170, 0.2280, 2.1377) -- (2.0640, 0.2280, 2.1342) -- (2.0640, 0.2810, 2.1401) -- (2.0170, 0.2810, 2.1436) -- cycle;
\fill[blue!15.6, opacity=0.7] (2.0170, 0.2810, 2.1436) -- (2.0640, 0.2810, 2.1401) -- (2.0640, 0.3340, 2.1459) -- (2.0170, 0.3340, 2.1494) -- cycle;
\fill[blue!20.3, opacity=0.7] (2.0170, 0.3340, 2.1494) -- (2.0640, 0.3340, 2.1459) -- (2.0640, 0.3870, 2.1516) -- (2.0170, 0.3870, 2.1551) -- cycle;
\fill[blue!48.6, opacity=0.7] (2.0170, 0.3870, 2.1551) -- (2.0640, 0.3870, 2.1516) -- (2.0640, 0.4400, 2.1571) -- (2.0170, 0.4400, 2.1606) -- cycle;
\fill[blue!84.3, opacity=0.7] (2.0170, 0.4400, 2.1606) -- (2.0640, 0.4400, 2.1571) -- (2.0640, 0.4930, 2.1624) -- (2.0170, 0.4930, 2.1660) -- cycle;
\fill[blue!86.2, opacity=0.7] (2.0170, 0.4930, 2.1660) -- (2.0640, 0.4930, 2.1624) -- (2.0640, 0.5460, 2.1676) -- (2.0170, 0.5460, 2.1712) -- cycle;
\fill[blue!85.8, opacity=0.7] (2.0170, 0.5460, 2.1712) -- (2.0640, 0.5460, 2.1676) -- (2.0640, 0.5990, 2.1726) -- (2.0170, 0.5990, 2.1762) -- cycle;
\fill[blue!86.7, opacity=0.7] (2.0170, 0.5990, 2.1762) -- (2.0640, 0.5990, 2.1726) -- (2.0640, 0.6520, 2.1774) -- (2.0170, 0.6520, 2.1809) -- cycle;
\fill[blue!66.3, opacity=0.7] (2.0170, 0.6520, 2.1809) -- (2.0640, 0.6520, 2.1774) -- (2.0640, 0.7050, 2.1819) -- (2.0170, 0.7050, 2.1855) -- cycle;
\fill[blue!38.9, opacity=0.7] (2.0170, 0.7050, 2.1855) -- (2.0640, 0.7050, 2.1819) -- (2.0640, 0.7580, 2.1863) -- (2.0170, 0.7580, 2.1898) -- cycle;
\fill[blue!26.1, opacity=0.7] (2.0170, 0.7580, 2.1898) -- (2.0640, 0.7580, 2.1863) -- (2.0640, 0.8110, 2.1903) -- (2.0170, 0.8110, 2.1939) -- cycle;
\fill[blue!23.7, opacity=0.7] (2.0170, 0.8110, 2.1939) -- (2.0640, 0.8110, 2.1903) -- (2.0640, 0.8640, 2.1942) -- (2.0170, 0.8640, 2.1977) -- cycle;
\fill[blue!28.0, opacity=0.7] (2.0170, 0.8640, 2.1977) -- (2.0640, 0.8640, 2.1942) -- (2.0640, 0.9170, 2.1977) -- (2.0170, 0.9170, 2.2013) -- cycle;
\fill[blue!42.4, opacity=0.7] (2.0170, 0.9170, 2.2013) -- (2.0640, 0.9170, 2.1977) -- (2.0640, 0.9700, 2.2010) -- (2.0170, 0.9700, 2.2046) -- cycle;
\fill[blue!67.7, opacity=0.7] (2.0170, 0.9700, 2.2046) -- (2.0640, 0.9700, 2.2010) -- (2.0640, 1.0230, 2.2040) -- (2.0170, 1.0230, 2.2076) -- cycle;
\fill[blue!86.6, opacity=0.7] (2.0170, 1.0230, 2.2076) -- (2.0640, 1.0230, 2.2040) -- (2.0640, 1.0760, 2.2067) -- (2.0170, 1.0760, 2.2103) -- cycle;
\fill[blue!83.2, opacity=0.7] (2.0170, 1.0760, 2.2103) -- (2.0640, 1.0760, 2.2067) -- (2.0640, 1.1290, 2.2091) -- (2.0170, 1.1290, 2.2127) -- cycle;
\fill[blue!67.7, opacity=0.7] (2.0170, 1.1290, 2.2127) -- (2.0640, 1.1290, 2.2091) -- (2.0640, 1.1820, 2.2112) -- (2.0170, 1.1820, 2.2148) -- cycle;
\fill[blue!53.5, opacity=0.7] (2.0170, 1.1820, 2.2148) -- (2.0640, 1.1820, 2.2112) -- (2.0640, 1.2350, 2.2130) -- (2.0170, 1.2350, 2.2166) -- cycle;
\fill[blue!44.5, opacity=0.7] (2.0170, 1.2350, 2.2166) -- (2.0640, 1.2350, 2.2130) -- (2.0640, 1.2880, 2.2145) -- (2.0170, 1.2880, 2.2180) -- cycle;
\fill[blue!39.6, opacity=0.7] (2.0170, 1.2880, 2.2180) -- (2.0640, 1.2880, 2.2145) -- (2.0640, 1.3410, 2.2156) -- (2.0170, 1.3410, 2.2192) -- cycle;
\fill[blue!37.5, opacity=0.7] (2.0170, 1.3410, 2.2192) -- (2.0640, 1.3410, 2.2156) -- (2.0640, 1.3940, 2.2164) -- (2.0170, 1.3940, 2.2200) -- cycle;
\fill[blue!37.7, opacity=0.7] (2.0170, 1.3940, 2.2200) -- (2.0640, 1.3940, 2.2164) -- (2.0640, 1.4470, 2.2169) -- (2.0170, 1.4470, 2.2205) -- cycle;
\fill[blue!40.7, opacity=0.7] (2.0170, 1.4470, 2.2205) -- (2.0640, 1.4470, 2.2169) -- (2.0640, 1.5000, 2.2171) -- (2.0170, 1.5000, 2.2206) -- cycle;
\fill[blue!48.0, opacity=0.7] (2.0170, 1.5000, 2.2206) -- (2.0640, 1.5000, 2.2171) -- (2.0640, 1.5530, 2.2169) -- (2.0170, 1.5530, 2.2205) -- cycle;
\fill[blue!61.6, opacity=0.7] (2.0170, 1.5530, 2.2205) -- (2.0640, 1.5530, 2.2169) -- (2.0640, 1.6060, 2.2164) -- (2.0170, 1.6060, 2.2200) -- cycle;
\fill[blue!79.5, opacity=0.7] (2.0170, 1.6060, 2.2200) -- (2.0640, 1.6060, 2.2164) -- (2.0640, 1.6590, 2.2156) -- (2.0170, 1.6590, 2.2192) -- cycle;
\fill[blue!87.7, opacity=0.7] (2.0170, 1.6590, 2.2192) -- (2.0640, 1.6590, 2.2156) -- (2.0640, 1.7120, 2.2145) -- (2.0170, 1.7120, 2.2180) -- cycle;
\fill[blue!72.6, opacity=0.7] (2.0170, 1.7120, 2.2180) -- (2.0640, 1.7120, 2.2145) -- (2.0640, 1.7650, 2.2130) -- (2.0170, 1.7650, 2.2166) -- cycle;
\fill[blue!50.1, opacity=0.7] (2.0170, 1.7650, 2.2166) -- (2.0640, 1.7650, 2.2130) -- (2.0640, 1.8180, 2.2112) -- (2.0170, 1.8180, 2.2148) -- cycle;
\fill[blue!38.9, opacity=0.7] (2.0170, 1.8180, 2.2148) -- (2.0640, 1.8180, 2.2112) -- (2.0640, 1.8710, 2.2091) -- (2.0170, 1.8710, 2.2127) -- cycle;
\fill[blue!41.6, opacity=0.7] (2.0170, 1.8710, 2.2127) -- (2.0640, 1.8710, 2.2091) -- (2.0640, 1.9240, 2.2067) -- (2.0170, 1.9240, 2.2103) -- cycle;
\fill[blue!60.9, opacity=0.7] (2.0170, 1.9240, 2.2103) -- (2.0640, 1.9240, 2.2067) -- (2.0640, 1.9770, 2.2040) -- (2.0170, 1.9770, 2.2076) -- cycle;
\fill[blue!85.8, opacity=0.7] (2.0170, 1.9770, 2.2076) -- (2.0640, 1.9770, 2.2040) -- (2.0640, 2.0300, 2.2010) -- (2.0170, 2.0300, 2.2046) -- cycle;
\fill[blue!79.1, opacity=0.7] (2.0170, 2.0300, 2.2046) -- (2.0640, 2.0300, 2.2010) -- (2.0640, 2.0830, 2.1977) -- (2.0170, 2.0830, 2.2013) -- cycle;
\fill[blue!60.5, opacity=0.7] (2.0170, 2.0830, 2.2013) -- (2.0640, 2.0830, 2.1977) -- (2.0640, 2.1360, 2.1942) -- (2.0170, 2.1360, 2.1977) -- cycle;
\fill[blue!63.0, opacity=0.7] (2.0170, 2.1360, 2.1977) -- (2.0640, 2.1360, 2.1942) -- (2.0640, 2.1890, 2.1903) -- (2.0170, 2.1890, 2.1939) -- cycle;
\fill[blue!85.0, opacity=0.7] (2.0170, 2.1890, 2.1939) -- (2.0640, 2.1890, 2.1903) -- (2.0640, 2.2420, 2.1863) -- (2.0170, 2.2420, 2.1898) -- cycle;
\fill[blue!70.3, opacity=0.7] (2.0170, 2.2420, 2.1898) -- (2.0640, 2.2420, 2.1863) -- (2.0640, 2.2950, 2.1819) -- (2.0170, 2.2950, 2.1855) -- cycle;
\fill[blue!29.4, opacity=0.7] (2.0170, 2.2950, 2.1855) -- (2.0640, 2.2950, 2.1819) -- (2.0640, 2.3480, 2.1774) -- (2.0170, 2.3480, 2.1809) -- cycle;
\fill[blue!18.3, opacity=0.7] (2.0170, 2.3480, 2.1809) -- (2.0640, 2.3480, 2.1774) -- (2.0640, 2.4010, 2.1726) -- (2.0170, 2.4010, 2.1762) -- cycle;
\fill[blue!17.9, opacity=0.7] (2.0170, 2.4010, 2.1762) -- (2.0640, 2.4010, 2.1726) -- (2.0640, 2.4540, 2.1676) -- (2.0170, 2.4540, 2.1712) -- cycle;
\fill[blue!25.2, opacity=0.7] (2.0170, 2.4540, 2.1712) -- (2.0640, 2.4540, 2.1676) -- (2.0640, 2.5070, 2.1624) -- (2.0170, 2.5070, 2.1660) -- cycle;
\fill[blue!54.7, opacity=0.7] (2.0170, 2.5070, 2.1660) -- (2.0640, 2.5070, 2.1624) -- (2.0640, 2.5600, 2.1571) -- (2.0170, 2.5600, 2.1606) -- cycle;
\fill[blue!83.4, opacity=0.7] (2.0170, 2.5600, 2.1606) -- (2.0640, 2.5600, 2.1571) -- (2.0640, 2.6130, 2.1516) -- (2.0170, 2.6130, 2.1551) -- cycle;
\fill[blue!86.9, opacity=0.7] (2.0170, 2.6130, 2.1551) -- (2.0640, 2.6130, 2.1516) -- (2.0640, 2.6660, 2.1459) -- (2.0170, 2.6660, 2.1494) -- cycle;
\fill[blue!72.9, opacity=0.7] (2.0170, 2.6660, 2.1494) -- (2.0640, 2.6660, 2.1459) -- (2.0640, 2.7190, 2.1401) -- (2.0170, 2.7190, 2.1436) -- cycle;
\fill[blue!31.1, opacity=0.7] (2.0170, 2.7190, 2.1436) -- (2.0640, 2.7190, 2.1401) -- (2.0640, 2.7720, 2.1342) -- (2.0170, 2.7720, 2.1377) -- cycle;
\fill[blue!16.1, opacity=0.7] (2.0170, 2.7720, 2.1377) -- (2.0640, 2.7720, 2.1342) -- (2.0640, 2.8250, 2.1281) -- (2.0170, 2.8250, 2.1317) -- cycle;
\fill[blue!15.1, opacity=0.7] (2.0170, 2.8250, 2.1317) -- (2.0640, 2.8250, 2.1281) -- (2.0640, 2.8780, 2.1220) -- (2.0170, 2.8780, 2.1256) -- cycle;
\fill[blue!15.1, opacity=0.7] (2.0170, 2.8780, 2.1256) -- (2.0640, 2.8780, 2.1220) -- (2.0640, 2.9310, 2.1159) -- (2.0170, 2.9310, 2.1194) -- cycle;
\fill[blue!15.6, opacity=0.7] (2.0170, 2.9310, 2.1194) -- (2.0640, 2.9310, 2.1159) -- (2.0640, 2.9840, 2.1096) -- (2.0170, 2.9840, 2.1132) -- cycle;
\fill[blue!21.5, opacity=0.7] (2.0170, 2.9840, 2.1132) -- (2.0640, 2.9840, 2.1096) -- (2.0640, 3.0370, 2.1034) -- (2.0170, 3.0370, 2.1069) -- cycle;
\fill[blue!38.9, opacity=0.7] (2.0170, 3.0370, 2.1069) -- (2.0640, 3.0370, 2.1034) -- (2.0640, 3.0900, 2.0971) -- (2.0170, 3.0900, 2.1006) -- cycle;
\fill[blue!19.3, opacity=0.7] (2.0640, -0.0900, 2.0971) -- (2.1110, -0.0900, 2.0933) -- (2.1110, -0.0370, 2.0995) -- (2.0640, -0.0370, 2.1034) -- cycle;
\fill[blue!40.5, opacity=0.7] (2.0640, -0.0370, 2.1034) -- (2.1110, -0.0370, 2.0995) -- (2.1110, 0.0160, 2.1058) -- (2.0640, 0.0160, 2.1096) -- cycle;
\fill[blue!57.3, opacity=0.7] (2.0640, 0.0160, 2.1096) -- (2.1110, 0.0160, 2.1058) -- (2.1110, 0.0690, 2.1120) -- (2.0640, 0.0690, 2.1159) -- cycle;
\fill[blue!47.7, opacity=0.7] (2.0640, 0.0690, 2.1159) -- (2.1110, 0.0690, 2.1120) -- (2.1110, 0.1220, 2.1182) -- (2.0640, 0.1220, 2.1220) -- cycle;
\fill[blue!25.8, opacity=0.7] (2.0640, 0.1220, 2.1220) -- (2.1110, 0.1220, 2.1182) -- (2.1110, 0.1750, 2.1243) -- (2.0640, 0.1750, 2.1281) -- cycle;
\fill[blue!16.6, opacity=0.7] (2.0640, 0.1750, 2.1281) -- (2.1110, 0.1750, 2.1243) -- (2.1110, 0.2280, 2.1303) -- (2.0640, 0.2280, 2.1342) -- cycle;
\fill[blue!15.3, opacity=0.7] (2.0640, 0.2280, 2.1342) -- (2.1110, 0.2280, 2.1303) -- (2.1110, 0.2810, 2.1363) -- (2.0640, 0.2810, 2.1401) -- cycle;
\fill[blue!15.2, opacity=0.7] (2.0640, 0.2810, 2.1401) -- (2.1110, 0.2810, 2.1363) -- (2.1110, 0.3340, 2.1421) -- (2.0640, 0.3340, 2.1459) -- cycle;
\fill[blue!16.1, opacity=0.7] (2.0640, 0.3340, 2.1459) -- (2.1110, 0.3340, 2.1421) -- (2.1110, 0.3870, 2.1477) -- (2.0640, 0.3870, 2.1516) -- cycle;
\fill[blue!24.9, opacity=0.7] (2.0640, 0.3870, 2.1516) -- (2.1110, 0.3870, 2.1477) -- (2.1110, 0.4400, 2.1533) -- (2.0640, 0.4400, 2.1571) -- cycle;
\fill[blue!59.7, opacity=0.7] (2.0640, 0.4400, 2.1571) -- (2.1110, 0.4400, 2.1533) -- (2.1110, 0.4930, 2.1586) -- (2.0640, 0.4930, 2.1624) -- cycle;
\fill[blue!86.9, opacity=0.7] (2.0640, 0.4930, 2.1624) -- (2.1110, 0.4930, 2.1586) -- (2.1110, 0.5460, 2.1638) -- (2.0640, 0.5460, 2.1676) -- cycle;
\fill[blue!85.2, opacity=0.7] (2.0640, 0.5460, 2.1676) -- (2.1110, 0.5460, 2.1638) -- (2.1110, 0.5990, 2.1688) -- (2.0640, 0.5990, 2.1726) -- cycle;
\fill[blue!85.7, opacity=0.7] (2.0640, 0.5990, 2.1726) -- (2.1110, 0.5990, 2.1688) -- (2.1110, 0.6520, 2.1736) -- (2.0640, 0.6520, 2.1774) -- cycle;
\fill[blue!86.9, opacity=0.7] (2.0640, 0.6520, 2.1774) -- (2.1110, 0.6520, 2.1736) -- (2.1110, 0.7050, 2.1781) -- (2.0640, 0.7050, 2.1819) -- cycle;
\fill[blue!69.2, opacity=0.7] (2.0640, 0.7050, 2.1819) -- (2.1110, 0.7050, 2.1781) -- (2.1110, 0.7580, 2.1824) -- (2.0640, 0.7580, 2.1863) -- cycle;
\fill[blue!43.2, opacity=0.7] (2.0640, 0.7580, 2.1863) -- (2.1110, 0.7580, 2.1824) -- (2.1110, 0.8110, 2.1865) -- (2.0640, 0.8110, 2.1903) -- cycle;
\fill[blue!28.8, opacity=0.7] (2.0640, 0.8110, 2.1903) -- (2.1110, 0.8110, 2.1865) -- (2.1110, 0.8640, 2.1903) -- (2.0640, 0.8640, 2.1942) -- cycle;
\fill[blue!24.5, opacity=0.7] (2.0640, 0.8640, 2.1942) -- (2.1110, 0.8640, 2.1903) -- (2.1110, 0.9170, 2.1939) -- (2.0640, 0.9170, 2.1977) -- cycle;
\fill[blue!25.8, opacity=0.7] (2.0640, 0.9170, 2.1977) -- (2.1110, 0.9170, 2.1939) -- (2.1110, 0.9700, 2.1972) -- (2.0640, 0.9700, 2.2010) -- cycle;
\fill[blue!33.0, opacity=0.7] (2.0640, 0.9700, 2.2010) -- (2.1110, 0.9700, 2.1972) -- (2.1110, 1.0230, 2.2002) -- (2.0640, 1.0230, 2.2040) -- cycle;
\fill[blue!47.6, opacity=0.7] (2.0640, 1.0230, 2.2040) -- (2.1110, 1.0230, 2.2002) -- (2.1110, 1.0760, 2.2029) -- (2.0640, 1.0760, 2.2067) -- cycle;
\fill[blue!67.2, opacity=0.7] (2.0640, 1.0760, 2.2067) -- (2.1110, 1.0760, 2.2029) -- (2.1110, 1.1290, 2.2053) -- (2.0640, 1.1290, 2.2091) -- cycle;
\fill[blue!82.5, opacity=0.7] (2.0640, 1.1290, 2.2091) -- (2.1110, 1.1290, 2.2053) -- (2.1110, 1.1820, 2.2074) -- (2.0640, 1.1820, 2.2112) -- cycle;
\fill[blue!87.8, opacity=0.7] (2.0640, 1.1820, 2.2112) -- (2.1110, 1.1820, 2.2074) -- (2.1110, 1.2350, 2.2092) -- (2.0640, 1.2350, 2.2130) -- cycle;
\fill[blue!85.8, opacity=0.7] (2.0640, 1.2350, 2.2130) -- (2.1110, 1.2350, 2.2092) -- (2.1110, 1.2880, 2.2106) -- (2.0640, 1.2880, 2.2145) -- cycle;
\fill[blue!82.1, opacity=0.7] (2.0640, 1.2880, 2.2145) -- (2.1110, 1.2880, 2.2106) -- (2.1110, 1.3410, 2.2118) -- (2.0640, 1.3410, 2.2156) -- cycle;
\fill[blue!80.0, opacity=0.7] (2.0640, 1.3410, 2.2156) -- (2.1110, 1.3410, 2.2118) -- (2.1110, 1.3940, 2.2126) -- (2.0640, 1.3940, 2.2164) -- cycle;
\fill[blue!81.0, opacity=0.7] (2.0640, 1.3940, 2.2164) -- (2.1110, 1.3940, 2.2126) -- (2.1110, 1.4470, 2.2131) -- (2.0640, 1.4470, 2.2169) -- cycle;
\fill[blue!84.4, opacity=0.7] (2.0640, 1.4470, 2.2169) -- (2.1110, 1.4470, 2.2131) -- (2.1110, 1.5000, 2.2133) -- (2.0640, 1.5000, 2.2171) -- cycle;
\fill[blue!87.7, opacity=0.7] (2.0640, 1.5000, 2.2171) -- (2.1110, 1.5000, 2.2133) -- (2.1110, 1.5530, 2.2131) -- (2.0640, 1.5530, 2.2169) -- cycle;
\fill[blue!85.2, opacity=0.7] (2.0640, 1.5530, 2.2169) -- (2.1110, 1.5530, 2.2131) -- (2.1110, 1.6060, 2.2126) -- (2.0640, 1.6060, 2.2164) -- cycle;
\fill[blue!72.9, opacity=0.7] (2.0640, 1.6060, 2.2164) -- (2.1110, 1.6060, 2.2126) -- (2.1110, 1.6590, 2.2118) -- (2.0640, 1.6590, 2.2156) -- cycle;
\fill[blue!55.1, opacity=0.7] (2.0640, 1.6590, 2.2156) -- (2.1110, 1.6590, 2.2118) -- (2.1110, 1.7120, 2.2106) -- (2.0640, 1.7120, 2.2145) -- cycle;
\fill[blue!41.5, opacity=0.7] (2.0640, 1.7120, 2.2145) -- (2.1110, 1.7120, 2.2106) -- (2.1110, 1.7650, 2.2092) -- (2.0640, 1.7650, 2.2130) -- cycle;
\fill[blue!36.6, opacity=0.7] (2.0640, 1.7650, 2.2130) -- (2.1110, 1.7650, 2.2092) -- (2.1110, 1.8180, 2.2074) -- (2.0640, 1.8180, 2.2112) -- cycle;
\fill[blue!41.5, opacity=0.7] (2.0640, 1.8180, 2.2112) -- (2.1110, 1.8180, 2.2074) -- (2.1110, 1.8710, 2.2053) -- (2.0640, 1.8710, 2.2091) -- cycle;
\fill[blue!59.9, opacity=0.7] (2.0640, 1.8710, 2.2091) -- (2.1110, 1.8710, 2.2053) -- (2.1110, 1.9240, 2.2029) -- (2.0640, 1.9240, 2.2067) -- cycle;
\fill[blue!84.2, opacity=0.7] (2.0640, 1.9240, 2.2067) -- (2.1110, 1.9240, 2.2029) -- (2.1110, 1.9770, 2.2002) -- (2.0640, 1.9770, 2.2040) -- cycle;
\fill[blue!82.8, opacity=0.7] (2.0640, 1.9770, 2.2040) -- (2.1110, 1.9770, 2.2002) -- (2.1110, 2.0300, 2.1972) -- (2.0640, 2.0300, 2.2010) -- cycle;
\fill[blue!64.5, opacity=0.7] (2.0640, 2.0300, 2.2010) -- (2.1110, 2.0300, 2.1972) -- (2.1110, 2.0830, 2.1939) -- (2.0640, 2.0830, 2.1977) -- cycle;
\fill[blue!61.1, opacity=0.7] (2.0640, 2.0830, 2.1977) -- (2.1110, 2.0830, 2.1939) -- (2.1110, 2.1360, 2.1903) -- (2.0640, 2.1360, 2.1942) -- cycle;
\fill[blue!79.0, opacity=0.7] (2.0640, 2.1360, 2.1942) -- (2.1110, 2.1360, 2.1903) -- (2.1110, 2.1890, 2.1865) -- (2.0640, 2.1890, 2.1903) -- cycle;
\fill[blue!82.9, opacity=0.7] (2.0640, 2.1890, 2.1903) -- (2.1110, 2.1890, 2.1865) -- (2.1110, 2.2420, 2.1824) -- (2.0640, 2.2420, 2.1863) -- cycle;
\fill[blue!41.3, opacity=0.7] (2.0640, 2.2420, 2.1863) -- (2.1110, 2.2420, 2.1824) -- (2.1110, 2.2950, 2.1781) -- (2.0640, 2.2950, 2.1819) -- cycle;
\fill[blue!20.4, opacity=0.7] (2.0640, 2.2950, 2.1819) -- (2.1110, 2.2950, 2.1781) -- (2.1110, 2.3480, 2.1736) -- (2.0640, 2.3480, 2.1774) -- cycle;
\fill[blue!17.3, opacity=0.7] (2.0640, 2.3480, 2.1774) -- (2.1110, 2.3480, 2.1736) -- (2.1110, 2.4010, 2.1688) -- (2.0640, 2.4010, 2.1726) -- cycle;
\fill[blue!19.7, opacity=0.7] (2.0640, 2.4010, 2.1726) -- (2.1110, 2.4010, 2.1688) -- (2.1110, 2.4540, 2.1638) -- (2.0640, 2.4540, 2.1676) -- cycle;
\fill[blue!35.6, opacity=0.7] (2.0640, 2.4540, 2.1676) -- (2.1110, 2.4540, 2.1638) -- (2.1110, 2.5070, 2.1586) -- (2.0640, 2.5070, 2.1624) -- cycle;
\fill[blue!70.8, opacity=0.7] (2.0640, 2.5070, 2.1624) -- (2.1110, 2.5070, 2.1586) -- (2.1110, 2.5600, 2.1533) -- (2.0640, 2.5600, 2.1571) -- cycle;
\fill[blue!86.4, opacity=0.7] (2.0640, 2.5600, 2.1571) -- (2.1110, 2.5600, 2.1533) -- (2.1110, 2.6130, 2.1477) -- (2.0640, 2.6130, 2.1516) -- cycle;
\fill[blue!84.4, opacity=0.7] (2.0640, 2.6130, 2.1516) -- (2.1110, 2.6130, 2.1477) -- (2.1110, 2.6660, 2.1421) -- (2.0640, 2.6660, 2.1459) -- cycle;
\fill[blue!55.9, opacity=0.7] (2.0640, 2.6660, 2.1459) -- (2.1110, 2.6660, 2.1421) -- (2.1110, 2.7190, 2.1363) -- (2.0640, 2.7190, 2.1401) -- cycle;
\fill[blue!21.2, opacity=0.7] (2.0640, 2.7190, 2.1401) -- (2.1110, 2.7190, 2.1363) -- (2.1110, 2.7720, 2.1303) -- (2.0640, 2.7720, 2.1342) -- cycle;
\fill[blue!15.3, opacity=0.7] (2.0640, 2.7720, 2.1342) -- (2.1110, 2.7720, 2.1303) -- (2.1110, 2.8250, 2.1243) -- (2.0640, 2.8250, 2.1281) -- cycle;
\fill[blue!15.1, opacity=0.7] (2.0640, 2.8250, 2.1281) -- (2.1110, 2.8250, 2.1243) -- (2.1110, 2.8780, 2.1182) -- (2.0640, 2.8780, 2.1220) -- cycle;
\fill[blue!15.1, opacity=0.7] (2.0640, 2.8780, 2.1220) -- (2.1110, 2.8780, 2.1182) -- (2.1110, 2.9310, 2.1120) -- (2.0640, 2.9310, 2.1159) -- cycle;
\fill[blue!16.3, opacity=0.7] (2.0640, 2.9310, 2.1159) -- (2.1110, 2.9310, 2.1120) -- (2.1110, 2.9840, 2.1058) -- (2.0640, 2.9840, 2.1096) -- cycle;
\fill[blue!26.1, opacity=0.7] (2.0640, 2.9840, 2.1096) -- (2.1110, 2.9840, 2.1058) -- (2.1110, 3.0370, 2.0995) -- (2.0640, 3.0370, 2.1034) -- cycle;
\fill[blue!42.2, opacity=0.7] (2.0640, 3.0370, 2.1034) -- (2.1110, 3.0370, 2.0995) -- (2.1110, 3.0900, 2.0933) -- (2.0640, 3.0900, 2.0971) -- cycle;
\fill[blue!15.9, opacity=0.7] (2.1110, -0.0900, 2.0933) -- (2.1580, -0.0900, 2.0892) -- (2.1580, -0.0370, 2.0955) -- (2.1110, -0.0370, 2.0995) -- cycle;
\fill[blue!26.8, opacity=0.7] (2.1110, -0.0370, 2.0995) -- (2.1580, -0.0370, 2.0955) -- (2.1580, 0.0160, 2.1017) -- (2.1110, 0.0160, 2.1058) -- cycle;
\fill[blue!51.0, opacity=0.7] (2.1110, 0.0160, 2.1058) -- (2.1580, 0.0160, 2.1017) -- (2.1580, 0.0690, 2.1079) -- (2.1110, 0.0690, 2.1120) -- cycle;
\fill[blue!57.2, opacity=0.7] (2.1110, 0.0690, 2.1120) -- (2.1580, 0.0690, 2.1079) -- (2.1580, 0.1220, 2.1141) -- (2.1110, 0.1220, 2.1182) -- cycle;
\fill[blue!39.8, opacity=0.7] (2.1110, 0.1220, 2.1182) -- (2.1580, 0.1220, 2.1141) -- (2.1580, 0.1750, 2.1202) -- (2.1110, 0.1750, 2.1243) -- cycle;
\fill[blue!21.1, opacity=0.7] (2.1110, 0.1750, 2.1243) -- (2.1580, 0.1750, 2.1202) -- (2.1580, 0.2280, 2.1263) -- (2.1110, 0.2280, 2.1303) -- cycle;
\fill[blue!15.9, opacity=0.7] (2.1110, 0.2280, 2.1303) -- (2.1580, 0.2280, 2.1263) -- (2.1580, 0.2810, 2.1322) -- (2.1110, 0.2810, 2.1363) -- cycle;
\fill[blue!15.2, opacity=0.7] (2.1110, 0.2810, 2.1363) -- (2.1580, 0.2810, 2.1322) -- (2.1580, 0.3340, 2.1380) -- (2.1110, 0.3340, 2.1421) -- cycle;
\fill[blue!15.3, opacity=0.7] (2.1110, 0.3340, 2.1421) -- (2.1580, 0.3340, 2.1380) -- (2.1580, 0.3870, 2.1437) -- (2.1110, 0.3870, 2.1477) -- cycle;
\fill[blue!16.9, opacity=0.7] (2.1110, 0.3870, 2.1477) -- (2.1580, 0.3870, 2.1437) -- (2.1580, 0.4400, 2.1492) -- (2.1110, 0.4400, 2.1533) -- cycle;
\fill[blue!29.4, opacity=0.7] (2.1110, 0.4400, 2.1533) -- (2.1580, 0.4400, 2.1492) -- (2.1580, 0.4930, 2.1545) -- (2.1110, 0.4930, 2.1586) -- cycle;
\fill[blue!66.1, opacity=0.7] (2.1110, 0.4930, 2.1586) -- (2.1580, 0.4930, 2.1545) -- (2.1580, 0.5460, 2.1597) -- (2.1110, 0.5460, 2.1638) -- cycle;
\fill[blue!87.6, opacity=0.7] (2.1110, 0.5460, 2.1638) -- (2.1580, 0.5460, 2.1597) -- (2.1580, 0.5990, 2.1647) -- (2.1110, 0.5990, 2.1688) -- cycle;
\fill[blue!84.6, opacity=0.7] (2.1110, 0.5990, 2.1688) -- (2.1580, 0.5990, 2.1647) -- (2.1580, 0.6520, 2.1695) -- (2.1110, 0.6520, 2.1736) -- cycle;
\fill[blue!85.0, opacity=0.7] (2.1110, 0.6520, 2.1736) -- (2.1580, 0.6520, 2.1695) -- (2.1580, 0.7050, 2.1740) -- (2.1110, 0.7050, 2.1781) -- cycle;
\fill[blue!87.7, opacity=0.7] (2.1110, 0.7050, 2.1781) -- (2.1580, 0.7050, 2.1740) -- (2.1580, 0.7580, 2.1784) -- (2.1110, 0.7580, 2.1824) -- cycle;
\fill[blue!75.9, opacity=0.7] (2.1110, 0.7580, 2.1824) -- (2.1580, 0.7580, 2.1784) -- (2.1580, 0.8110, 2.1824) -- (2.1110, 0.8110, 2.1865) -- cycle;
\fill[blue!52.2, opacity=0.7] (2.1110, 0.8110, 2.1865) -- (2.1580, 0.8110, 2.1824) -- (2.1580, 0.8640, 2.1863) -- (2.1110, 0.8640, 2.1903) -- cycle;
\fill[blue!34.7, opacity=0.7] (2.1110, 0.8640, 2.1903) -- (2.1580, 0.8640, 2.1863) -- (2.1580, 0.9170, 2.1898) -- (2.1110, 0.9170, 2.1939) -- cycle;
\fill[blue!27.1, opacity=0.7] (2.1110, 0.9170, 2.1939) -- (2.1580, 0.9170, 2.1898) -- (2.1580, 0.9700, 2.1931) -- (2.1110, 0.9700, 2.1972) -- cycle;
\fill[blue!25.3, opacity=0.7] (2.1110, 0.9700, 2.1972) -- (2.1580, 0.9700, 2.1931) -- (2.1580, 1.0230, 2.1961) -- (2.1110, 1.0230, 2.2002) -- cycle;
\fill[blue!27.1, opacity=0.7] (2.1110, 1.0230, 2.2002) -- (2.1580, 1.0230, 2.1961) -- (2.1580, 1.0760, 2.1988) -- (2.1110, 1.0760, 2.2029) -- cycle;
\fill[blue!32.2, opacity=0.7] (2.1110, 1.0760, 2.2029) -- (2.1580, 1.0760, 2.1988) -- (2.1580, 1.1290, 2.2012) -- (2.1110, 1.1290, 2.2053) -- cycle;
\fill[blue!40.4, opacity=0.7] (2.1110, 1.1290, 2.2053) -- (2.1580, 1.1290, 2.2012) -- (2.1580, 1.1820, 2.2033) -- (2.1110, 1.1820, 2.2074) -- cycle;
\fill[blue!50.3, opacity=0.7] (2.1110, 1.1820, 2.2074) -- (2.1580, 1.1820, 2.2033) -- (2.1580, 1.2350, 2.2051) -- (2.1110, 1.2350, 2.2092) -- cycle;
\fill[blue!59.3, opacity=0.7] (2.1110, 1.2350, 2.2092) -- (2.1580, 1.2350, 2.2051) -- (2.1580, 1.2880, 2.2066) -- (2.1110, 1.2880, 2.2106) -- cycle;
\fill[blue!65.5, opacity=0.7] (2.1110, 1.2880, 2.2106) -- (2.1580, 1.2880, 2.2066) -- (2.1580, 1.3410, 2.2077) -- (2.1110, 1.3410, 2.2118) -- cycle;
\fill[blue!68.1, opacity=0.7] (2.1110, 1.3410, 2.2118) -- (2.1580, 1.3410, 2.2077) -- (2.1580, 1.3940, 2.2085) -- (2.1110, 1.3940, 2.2126) -- cycle;
\fill[blue!67.1, opacity=0.7] (2.1110, 1.3940, 2.2126) -- (2.1580, 1.3940, 2.2085) -- (2.1580, 1.4470, 2.2090) -- (2.1110, 1.4470, 2.2131) -- cycle;
\fill[blue!62.6, opacity=0.7] (2.1110, 1.4470, 2.2131) -- (2.1580, 1.4470, 2.2090) -- (2.1580, 1.5000, 2.2092) -- (2.1110, 1.5000, 2.2133) -- cycle;
\fill[blue!55.1, opacity=0.7] (2.1110, 1.5000, 2.2133) -- (2.1580, 1.5000, 2.2092) -- (2.1580, 1.5530, 2.2090) -- (2.1110, 1.5530, 2.2131) -- cycle;
\fill[blue!46.3, opacity=0.7] (2.1110, 1.5530, 2.2131) -- (2.1580, 1.5530, 2.2090) -- (2.1580, 1.6060, 2.2085) -- (2.1110, 1.6060, 2.2126) -- cycle;
\fill[blue!38.9, opacity=0.7] (2.1110, 1.6060, 2.2126) -- (2.1580, 1.6060, 2.2085) -- (2.1580, 1.6590, 2.2077) -- (2.1110, 1.6590, 2.2118) -- cycle;
\fill[blue!35.0, opacity=0.7] (2.1110, 1.6590, 2.2118) -- (2.1580, 1.6590, 2.2077) -- (2.1580, 1.7120, 2.2066) -- (2.1110, 1.7120, 2.2106) -- cycle;
\fill[blue!36.1, opacity=0.7] (2.1110, 1.7120, 2.2106) -- (2.1580, 1.7120, 2.2066) -- (2.1580, 1.7650, 2.2051) -- (2.1110, 1.7650, 2.2092) -- cycle;
\fill[blue!44.8, opacity=0.7] (2.1110, 1.7650, 2.2092) -- (2.1580, 1.7650, 2.2051) -- (2.1580, 1.8180, 2.2033) -- (2.1110, 1.8180, 2.2074) -- cycle;
\fill[blue!64.3, opacity=0.7] (2.1110, 1.8180, 2.2074) -- (2.1580, 1.8180, 2.2033) -- (2.1580, 1.8710, 2.2012) -- (2.1110, 1.8710, 2.2053) -- cycle;
\fill[blue!85.2, opacity=0.7] (2.1110, 1.8710, 2.2053) -- (2.1580, 1.8710, 2.2012) -- (2.1580, 1.9240, 2.1988) -- (2.1110, 1.9240, 2.2029) -- cycle;
\fill[blue!83.3, opacity=0.7] (2.1110, 1.9240, 2.2029) -- (2.1580, 1.9240, 2.1988) -- (2.1580, 1.9770, 2.1961) -- (2.1110, 1.9770, 2.2002) -- cycle;
\fill[blue!66.8, opacity=0.7] (2.1110, 1.9770, 2.2002) -- (2.1580, 1.9770, 2.1961) -- (2.1580, 2.0300, 2.1931) -- (2.1110, 2.0300, 2.1972) -- cycle;
\fill[blue!61.7, opacity=0.7] (2.1110, 2.0300, 2.1972) -- (2.1580, 2.0300, 2.1931) -- (2.1580, 2.0830, 2.1898) -- (2.1110, 2.0830, 2.1939) -- cycle;
\fill[blue!75.7, opacity=0.7] (2.1110, 2.0830, 2.1939) -- (2.1580, 2.0830, 2.1898) -- (2.1580, 2.1360, 2.1863) -- (2.1110, 2.1360, 2.1903) -- cycle;
\fill[blue!87.0, opacity=0.7] (2.1110, 2.1360, 2.1903) -- (2.1580, 2.1360, 2.1863) -- (2.1580, 2.1890, 2.1824) -- (2.1110, 2.1890, 2.1865) -- cycle;
\fill[blue!52.8, opacity=0.7] (2.1110, 2.1890, 2.1865) -- (2.1580, 2.1890, 2.1824) -- (2.1580, 2.2420, 2.1784) -- (2.1110, 2.2420, 2.1824) -- cycle;
\fill[blue!23.7, opacity=0.7] (2.1110, 2.2420, 2.1824) -- (2.1580, 2.2420, 2.1784) -- (2.1580, 2.2950, 2.1740) -- (2.1110, 2.2950, 2.1781) -- cycle;
\fill[blue!17.5, opacity=0.7] (2.1110, 2.2950, 2.1781) -- (2.1580, 2.2950, 2.1740) -- (2.1580, 2.3480, 2.1695) -- (2.1110, 2.3480, 2.1736) -- cycle;
\fill[blue!17.8, opacity=0.7] (2.1110, 2.3480, 2.1736) -- (2.1580, 2.3480, 2.1695) -- (2.1580, 2.4010, 2.1647) -- (2.1110, 2.4010, 2.1688) -- cycle;
\fill[blue!25.2, opacity=0.7] (2.1110, 2.4010, 2.1688) -- (2.1580, 2.4010, 2.1647) -- (2.1580, 2.4540, 2.1597) -- (2.1110, 2.4540, 2.1638) -- cycle;
\fill[blue!53.2, opacity=0.7] (2.1110, 2.4540, 2.1638) -- (2.1580, 2.4540, 2.1597) -- (2.1580, 2.5070, 2.1545) -- (2.1110, 2.5070, 2.1586) -- cycle;
\fill[blue!81.8, opacity=0.7] (2.1110, 2.5070, 2.1586) -- (2.1580, 2.5070, 2.1545) -- (2.1580, 2.5600, 2.1492) -- (2.1110, 2.5600, 2.1533) -- cycle;
\fill[blue!86.6, opacity=0.7] (2.1110, 2.5600, 2.1533) -- (2.1580, 2.5600, 2.1492) -- (2.1580, 2.6130, 2.1437) -- (2.1110, 2.6130, 2.1477) -- cycle;
\fill[blue!75.4, opacity=0.7] (2.1110, 2.6130, 2.1477) -- (2.1580, 2.6130, 2.1437) -- (2.1580, 2.6660, 2.1380) -- (2.1110, 2.6660, 2.1421) -- cycle;
\fill[blue!35.9, opacity=0.7] (2.1110, 2.6660, 2.1421) -- (2.1580, 2.6660, 2.1380) -- (2.1580, 2.7190, 2.1322) -- (2.1110, 2.7190, 2.1363) -- cycle;
\fill[blue!16.8, opacity=0.7] (2.1110, 2.7190, 2.1363) -- (2.1580, 2.7190, 2.1322) -- (2.1580, 2.7720, 2.1263) -- (2.1110, 2.7720, 2.1303) -- cycle;
\fill[blue!15.1, opacity=0.7] (2.1110, 2.7720, 2.1303) -- (2.1580, 2.7720, 2.1263) -- (2.1580, 2.8250, 2.1202) -- (2.1110, 2.8250, 2.1243) -- cycle;
\fill[blue!15.0, opacity=0.7] (2.1110, 2.8250, 2.1243) -- (2.1580, 2.8250, 2.1202) -- (2.1580, 2.8780, 2.1141) -- (2.1110, 2.8780, 2.1182) -- cycle;
\fill[blue!15.2, opacity=0.7] (2.1110, 2.8780, 2.1182) -- (2.1580, 2.8780, 2.1141) -- (2.1580, 2.9310, 2.1079) -- (2.1110, 2.9310, 2.1120) -- cycle;
\fill[blue!18.1, opacity=0.7] (2.1110, 2.9310, 2.1120) -- (2.1580, 2.9310, 2.1079) -- (2.1580, 2.9840, 2.1017) -- (2.1110, 2.9840, 2.1058) -- cycle;
\fill[blue!32.3, opacity=0.7] (2.1110, 2.9840, 2.1058) -- (2.1580, 2.9840, 2.1017) -- (2.1580, 3.0370, 2.0955) -- (2.1110, 3.0370, 2.0995) -- cycle;
\fill[blue!42.9, opacity=0.7] (2.1110, 3.0370, 2.0995) -- (2.1580, 3.0370, 2.0955) -- (2.1580, 3.0900, 2.0892) -- (2.1110, 3.0900, 2.0933) -- cycle;
\fill[blue!15.1, opacity=0.7] (2.1580, -0.0900, 2.0892) -- (2.2050, -0.0900, 2.0849) -- (2.2050, -0.0370, 2.0911) -- (2.1580, -0.0370, 2.0955) -- cycle;
\fill[blue!18.1, opacity=0.7] (2.1580, -0.0370, 2.0955) -- (2.2050, -0.0370, 2.0911) -- (2.2050, 0.0160, 2.0974) -- (2.1580, 0.0160, 2.1017) -- cycle;
\fill[blue!36.7, opacity=0.7] (2.1580, 0.0160, 2.1017) -- (2.2050, 0.0160, 2.0974) -- (2.2050, 0.0690, 2.1036) -- (2.1580, 0.0690, 2.1079) -- cycle;
\fill[blue!57.1, opacity=0.7] (2.1580, 0.0690, 2.1079) -- (2.2050, 0.0690, 2.1036) -- (2.2050, 0.1220, 2.1098) -- (2.1580, 0.1220, 2.1141) -- cycle;
\fill[blue!54.1, opacity=0.7] (2.1580, 0.1220, 2.1141) -- (2.2050, 0.1220, 2.1098) -- (2.2050, 0.1750, 2.1159) -- (2.1580, 0.1750, 2.1202) -- cycle;
\fill[blue!33.3, opacity=0.7] (2.1580, 0.1750, 2.1202) -- (2.2050, 0.1750, 2.1159) -- (2.2050, 0.2280, 2.1219) -- (2.1580, 0.2280, 2.1263) -- cycle;
\fill[blue!18.8, opacity=0.7] (2.1580, 0.2280, 2.1263) -- (2.2050, 0.2280, 2.1219) -- (2.2050, 0.2810, 2.1279) -- (2.1580, 0.2810, 2.1322) -- cycle;
\fill[blue!15.6, opacity=0.7] (2.1580, 0.2810, 2.1322) -- (2.2050, 0.2810, 2.1279) -- (2.2050, 0.3340, 2.1337) -- (2.1580, 0.3340, 2.1380) -- cycle;
\fill[blue!15.2, opacity=0.7] (2.1580, 0.3340, 2.1380) -- (2.2050, 0.3340, 2.1337) -- (2.2050, 0.3870, 2.1393) -- (2.1580, 0.3870, 2.1437) -- cycle;
\fill[blue!15.4, opacity=0.7] (2.1580, 0.3870, 2.1437) -- (2.2050, 0.3870, 2.1393) -- (2.2050, 0.4400, 2.1449) -- (2.1580, 0.4400, 2.1492) -- cycle;
\fill[blue!17.6, opacity=0.7] (2.1580, 0.4400, 2.1492) -- (2.2050, 0.4400, 2.1449) -- (2.2050, 0.4930, 2.1502) -- (2.1580, 0.4930, 2.1545) -- cycle;
\fill[blue!32.0, opacity=0.7] (2.1580, 0.4930, 2.1545) -- (2.2050, 0.4930, 2.1502) -- (2.2050, 0.5460, 2.1554) -- (2.1580, 0.5460, 2.1597) -- cycle;
\fill[blue!67.7, opacity=0.7] (2.1580, 0.5460, 2.1597) -- (2.2050, 0.5460, 2.1554) -- (2.2050, 0.5990, 2.1604) -- (2.1580, 0.5990, 2.1647) -- cycle;
\fill[blue!87.6, opacity=0.7] (2.1580, 0.5990, 2.1647) -- (2.2050, 0.5990, 2.1604) -- (2.2050, 0.6520, 2.1651) -- (2.1580, 0.6520, 2.1695) -- cycle;
\fill[blue!84.6, opacity=0.7] (2.1580, 0.6520, 2.1695) -- (2.2050, 0.6520, 2.1651) -- (2.2050, 0.7050, 2.1697) -- (2.1580, 0.7050, 2.1740) -- cycle;
\fill[blue!83.6, opacity=0.7] (2.1580, 0.7050, 2.1740) -- (2.2050, 0.7050, 2.1697) -- (2.2050, 0.7580, 2.1740) -- (2.1580, 0.7580, 2.1784) -- cycle;
\fill[blue!87.5, opacity=0.7] (2.1580, 0.7580, 2.1784) -- (2.2050, 0.7580, 2.1740) -- (2.2050, 0.8110, 2.1781) -- (2.1580, 0.8110, 2.1824) -- cycle;
\fill[blue!83.9, opacity=0.7] (2.1580, 0.8110, 2.1824) -- (2.2050, 0.8110, 2.1781) -- (2.2050, 0.8640, 2.1819) -- (2.1580, 0.8640, 2.1863) -- cycle;
\fill[blue!66.8, opacity=0.7] (2.1580, 0.8640, 2.1863) -- (2.2050, 0.8640, 2.1819) -- (2.2050, 0.9170, 2.1855) -- (2.1580, 0.9170, 2.1898) -- cycle;
\fill[blue!47.4, opacity=0.7] (2.1580, 0.9170, 2.1898) -- (2.2050, 0.9170, 2.1855) -- (2.2050, 0.9700, 2.1888) -- (2.1580, 0.9700, 2.1931) -- cycle;
\fill[blue!34.9, opacity=0.7] (2.1580, 0.9700, 2.1931) -- (2.2050, 0.9700, 2.1888) -- (2.2050, 1.0230, 2.1918) -- (2.1580, 1.0230, 2.1961) -- cycle;
\fill[blue!28.9, opacity=0.7] (2.1580, 1.0230, 2.1961) -- (2.2050, 1.0230, 2.1918) -- (2.2050, 1.0760, 2.1945) -- (2.1580, 1.0760, 2.1988) -- cycle;
\fill[blue!26.9, opacity=0.7] (2.1580, 1.0760, 2.1988) -- (2.2050, 1.0760, 2.1945) -- (2.2050, 1.1290, 2.1969) -- (2.1580, 1.1290, 2.2012) -- cycle;
\fill[blue!27.0, opacity=0.7] (2.1580, 1.1290, 2.2012) -- (2.2050, 1.1290, 2.1969) -- (2.2050, 1.1820, 2.1990) -- (2.1580, 1.1820, 2.2033) -- cycle;
\fill[blue!28.4, opacity=0.7] (2.1580, 1.1820, 2.2033) -- (2.2050, 1.1820, 2.1990) -- (2.2050, 1.2350, 2.2008) -- (2.1580, 1.2350, 2.2051) -- cycle;
\fill[blue!30.4, opacity=0.7] (2.1580, 1.2350, 2.2051) -- (2.2050, 1.2350, 2.2008) -- (2.2050, 1.2880, 2.2022) -- (2.1580, 1.2880, 2.2066) -- cycle;
\fill[blue!32.3, opacity=0.7] (2.1580, 1.2880, 2.2066) -- (2.2050, 1.2880, 2.2022) -- (2.2050, 1.3410, 2.2034) -- (2.1580, 1.3410, 2.2077) -- cycle;
\fill[blue!33.5, opacity=0.7] (2.1580, 1.3410, 2.2077) -- (2.2050, 1.3410, 2.2034) -- (2.2050, 1.3940, 2.2042) -- (2.1580, 1.3940, 2.2085) -- cycle;
\fill[blue!33.7, opacity=0.7] (2.1580, 1.3940, 2.2085) -- (2.2050, 1.3940, 2.2042) -- (2.2050, 1.4470, 2.2047) -- (2.1580, 1.4470, 2.2090) -- cycle;
\fill[blue!33.1, opacity=0.7] (2.1580, 1.4470, 2.2090) -- (2.2050, 1.4470, 2.2047) -- (2.2050, 1.5000, 2.2049) -- (2.1580, 1.5000, 2.2092) -- cycle;
\fill[blue!32.4, opacity=0.7] (2.1580, 1.5000, 2.2092) -- (2.2050, 1.5000, 2.2049) -- (2.2050, 1.5530, 2.2047) -- (2.1580, 1.5530, 2.2090) -- cycle;
\fill[blue!32.5, opacity=0.7] (2.1580, 1.5530, 2.2090) -- (2.2050, 1.5530, 2.2047) -- (2.2050, 1.6060, 2.2042) -- (2.1580, 1.6060, 2.2085) -- cycle;
\fill[blue!34.8, opacity=0.7] (2.1580, 1.6060, 2.2085) -- (2.2050, 1.6060, 2.2042) -- (2.2050, 1.6590, 2.2034) -- (2.1580, 1.6590, 2.2077) -- cycle;
\fill[blue!41.1, opacity=0.7] (2.1580, 1.6590, 2.2077) -- (2.2050, 1.6590, 2.2034) -- (2.2050, 1.7120, 2.2022) -- (2.1580, 1.7120, 2.2066) -- cycle;
\fill[blue!54.4, opacity=0.7] (2.1580, 1.7120, 2.2066) -- (2.2050, 1.7120, 2.2022) -- (2.2050, 1.7650, 2.2008) -- (2.1580, 1.7650, 2.2051) -- cycle;
\fill[blue!74.1, opacity=0.7] (2.1580, 1.7650, 2.2051) -- (2.2050, 1.7650, 2.2008) -- (2.2050, 1.8180, 2.1990) -- (2.1580, 1.8180, 2.2033) -- cycle;
\fill[blue!87.5, opacity=0.7] (2.1580, 1.8180, 2.2033) -- (2.2050, 1.8180, 2.1990) -- (2.2050, 1.8710, 2.1969) -- (2.1580, 1.8710, 2.2012) -- cycle;
\fill[blue!81.1, opacity=0.7] (2.1580, 1.8710, 2.2012) -- (2.2050, 1.8710, 2.1969) -- (2.2050, 1.9240, 2.1945) -- (2.1580, 1.9240, 2.1988) -- cycle;
\fill[blue!66.9, opacity=0.7] (2.1580, 1.9240, 2.1988) -- (2.2050, 1.9240, 2.1945) -- (2.2050, 1.9770, 2.1918) -- (2.1580, 1.9770, 2.1961) -- cycle;
\fill[blue!63.2, opacity=0.7] (2.1580, 1.9770, 2.1961) -- (2.2050, 1.9770, 2.1918) -- (2.2050, 2.0300, 2.1888) -- (2.1580, 2.0300, 2.1931) -- cycle;
\fill[blue!75.7, opacity=0.7] (2.1580, 2.0300, 2.1931) -- (2.2050, 2.0300, 2.1888) -- (2.2050, 2.0830, 2.1855) -- (2.1580, 2.0830, 2.1898) -- cycle;
\fill[blue!87.7, opacity=0.7] (2.1580, 2.0830, 2.1898) -- (2.2050, 2.0830, 2.1855) -- (2.2050, 2.1360, 2.1819) -- (2.1580, 2.1360, 2.1863) -- cycle;
\fill[blue!59.6, opacity=0.7] (2.1580, 2.1360, 2.1863) -- (2.2050, 2.1360, 2.1819) -- (2.2050, 2.1890, 2.1781) -- (2.1580, 2.1890, 2.1824) -- cycle;
\fill[blue!27.0, opacity=0.7] (2.1580, 2.1890, 2.1824) -- (2.2050, 2.1890, 2.1781) -- (2.2050, 2.2420, 2.1740) -- (2.1580, 2.2420, 2.1784) -- cycle;
\fill[blue!18.0, opacity=0.7] (2.1580, 2.2420, 2.1784) -- (2.2050, 2.2420, 2.1740) -- (2.2050, 2.2950, 2.1697) -- (2.1580, 2.2950, 2.1740) -- cycle;
\fill[blue!17.2, opacity=0.7] (2.1580, 2.2950, 2.1740) -- (2.2050, 2.2950, 2.1697) -- (2.2050, 2.3480, 2.1651) -- (2.1580, 2.3480, 2.1695) -- cycle;
\fill[blue!20.7, opacity=0.7] (2.1580, 2.3480, 2.1695) -- (2.2050, 2.3480, 2.1651) -- (2.2050, 2.4010, 2.1604) -- (2.1580, 2.4010, 2.1647) -- cycle;
\fill[blue!38.9, opacity=0.7] (2.1580, 2.4010, 2.1647) -- (2.2050, 2.4010, 2.1604) -- (2.2050, 2.4540, 2.1554) -- (2.1580, 2.4540, 2.1597) -- cycle;
\fill[blue!72.2, opacity=0.7] (2.1580, 2.4540, 2.1597) -- (2.2050, 2.4540, 2.1554) -- (2.2050, 2.5070, 2.1502) -- (2.1580, 2.5070, 2.1545) -- cycle;
\fill[blue!86.0, opacity=0.7] (2.1580, 2.5070, 2.1545) -- (2.2050, 2.5070, 2.1502) -- (2.2050, 2.5600, 2.1449) -- (2.1580, 2.5600, 2.1492) -- cycle;
\fill[blue!83.6, opacity=0.7] (2.1580, 2.5600, 2.1492) -- (2.2050, 2.5600, 2.1449) -- (2.2050, 2.6130, 2.1393) -- (2.1580, 2.6130, 2.1437) -- cycle;
\fill[blue!56.1, opacity=0.7] (2.1580, 2.6130, 2.1437) -- (2.2050, 2.6130, 2.1393) -- (2.2050, 2.6660, 2.1337) -- (2.1580, 2.6660, 2.1380) -- cycle;
\fill[blue!22.1, opacity=0.7] (2.1580, 2.6660, 2.1380) -- (2.2050, 2.6660, 2.1337) -- (2.2050, 2.7190, 2.1279) -- (2.1580, 2.7190, 2.1322) -- cycle;
\fill[blue!15.4, opacity=0.7] (2.1580, 2.7190, 2.1322) -- (2.2050, 2.7190, 2.1279) -- (2.2050, 2.7720, 2.1219) -- (2.1580, 2.7720, 2.1263) -- cycle;
\fill[blue!15.1, opacity=0.7] (2.1580, 2.7720, 2.1263) -- (2.2050, 2.7720, 2.1219) -- (2.2050, 2.8250, 2.1159) -- (2.1580, 2.8250, 2.1202) -- cycle;
\fill[blue!15.1, opacity=0.7] (2.1580, 2.8250, 2.1202) -- (2.2050, 2.8250, 2.1159) -- (2.2050, 2.8780, 2.1098) -- (2.1580, 2.8780, 2.1141) -- cycle;
\fill[blue!15.7, opacity=0.7] (2.1580, 2.8780, 2.1141) -- (2.2050, 2.8780, 2.1098) -- (2.2050, 2.9310, 2.1036) -- (2.1580, 2.9310, 2.1079) -- cycle;
\fill[blue!22.0, opacity=0.7] (2.1580, 2.9310, 2.1079) -- (2.2050, 2.9310, 2.1036) -- (2.2050, 2.9840, 2.0974) -- (2.1580, 2.9840, 2.1017) -- cycle;
\fill[blue!38.3, opacity=0.7] (2.1580, 2.9840, 2.1017) -- (2.2050, 2.9840, 2.0974) -- (2.2050, 3.0370, 2.0911) -- (2.1580, 3.0370, 2.0955) -- cycle;
\fill[blue!39.8, opacity=0.7] (2.1580, 3.0370, 2.0955) -- (2.2050, 3.0370, 2.0911) -- (2.2050, 3.0900, 2.0849) -- (2.1580, 3.0900, 2.0892) -- cycle;
\fill[blue!15.0, opacity=0.7] (2.2050, -0.0900, 2.0849) -- (2.2520, -0.0900, 2.0803) -- (2.2520, -0.0370, 2.0866) -- (2.2050, -0.0370, 2.0911) -- cycle;
\fill[blue!15.5, opacity=0.7] (2.2050, -0.0370, 2.0911) -- (2.2520, -0.0370, 2.0866) -- (2.2520, 0.0160, 2.0928) -- (2.2050, 0.0160, 2.0974) -- cycle;
\fill[blue!22.5, opacity=0.7] (2.2050, 0.0160, 2.0974) -- (2.2520, 0.0160, 2.0928) -- (2.2520, 0.0690, 2.0991) -- (2.2050, 0.0690, 2.1036) -- cycle;
\fill[blue!45.6, opacity=0.7] (2.2050, 0.0690, 2.1036) -- (2.2520, 0.0690, 2.0991) -- (2.2520, 0.1220, 2.1052) -- (2.2050, 0.1220, 2.1098) -- cycle;
\fill[blue!59.5, opacity=0.7] (2.2050, 0.1220, 2.1098) -- (2.2520, 0.1220, 2.1052) -- (2.2520, 0.1750, 2.1114) -- (2.2050, 0.1750, 2.1159) -- cycle;
\fill[blue!50.3, opacity=0.7] (2.2050, 0.1750, 2.1159) -- (2.2520, 0.1750, 2.1114) -- (2.2520, 0.2280, 2.1174) -- (2.2050, 0.2280, 2.1219) -- cycle;
\fill[blue!29.1, opacity=0.7] (2.2050, 0.2280, 2.1219) -- (2.2520, 0.2280, 2.1174) -- (2.2520, 0.2810, 2.1233) -- (2.2050, 0.2810, 2.1279) -- cycle;
\fill[blue!17.8, opacity=0.7] (2.2050, 0.2810, 2.1279) -- (2.2520, 0.2810, 2.1233) -- (2.2520, 0.3340, 2.1291) -- (2.2050, 0.3340, 2.1337) -- cycle;
\fill[blue!15.5, opacity=0.7] (2.2050, 0.3340, 2.1337) -- (2.2520, 0.3340, 2.1291) -- (2.2520, 0.3870, 2.1348) -- (2.2050, 0.3870, 2.1393) -- cycle;
\fill[blue!15.2, opacity=0.7] (2.2050, 0.3870, 2.1393) -- (2.2520, 0.3870, 2.1348) -- (2.2520, 0.4400, 2.1403) -- (2.2050, 0.4400, 2.1449) -- cycle;
\fill[blue!15.5, opacity=0.7] (2.2050, 0.4400, 2.1449) -- (2.2520, 0.4400, 2.1403) -- (2.2520, 0.4930, 2.1457) -- (2.2050, 0.4930, 2.1502) -- cycle;
\fill[blue!17.8, opacity=0.7] (2.2050, 0.4930, 2.1502) -- (2.2520, 0.4930, 2.1457) -- (2.2520, 0.5460, 2.1508) -- (2.2050, 0.5460, 2.1554) -- cycle;
\fill[blue!31.5, opacity=0.7] (2.2050, 0.5460, 2.1554) -- (2.2520, 0.5460, 2.1508) -- (2.2520, 0.5990, 2.1558) -- (2.2050, 0.5990, 2.1604) -- cycle;
\fill[blue!64.8, opacity=0.7] (2.2050, 0.5990, 2.1604) -- (2.2520, 0.5990, 2.1558) -- (2.2520, 0.6520, 2.1606) -- (2.2050, 0.6520, 2.1651) -- cycle;
\fill[blue!86.8, opacity=0.7] (2.2050, 0.6520, 2.1651) -- (2.2520, 0.6520, 2.1606) -- (2.2520, 0.7050, 2.1651) -- (2.2050, 0.7050, 2.1697) -- cycle;
\fill[blue!85.4, opacity=0.7] (2.2050, 0.7050, 2.1697) -- (2.2520, 0.7050, 2.1651) -- (2.2520, 0.7580, 2.1695) -- (2.2050, 0.7580, 2.1740) -- cycle;
\fill[blue!82.2, opacity=0.7] (2.2050, 0.7580, 2.1740) -- (2.2520, 0.7580, 2.1695) -- (2.2520, 0.8110, 2.1736) -- (2.2050, 0.8110, 2.1781) -- cycle;
\fill[blue!84.9, opacity=0.7] (2.2050, 0.8110, 2.1781) -- (2.2520, 0.8110, 2.1736) -- (2.2520, 0.8640, 2.1774) -- (2.2050, 0.8640, 2.1819) -- cycle;
\fill[blue!87.9, opacity=0.7] (2.2050, 0.8640, 2.1819) -- (2.2520, 0.8640, 2.1774) -- (2.2520, 0.9170, 2.1809) -- (2.2050, 0.9170, 2.1855) -- cycle;
\fill[blue!82.5, opacity=0.7] (2.2050, 0.9170, 2.1855) -- (2.2520, 0.9170, 2.1809) -- (2.2520, 0.9700, 2.1842) -- (2.2050, 0.9700, 2.1888) -- cycle;
\fill[blue!68.7, opacity=0.7] (2.2050, 0.9700, 2.1888) -- (2.2520, 0.9700, 2.1842) -- (2.2520, 1.0230, 2.1872) -- (2.2050, 1.0230, 2.1918) -- cycle;
\fill[blue!53.8, opacity=0.7] (2.2050, 1.0230, 2.1918) -- (2.2520, 1.0230, 2.1872) -- (2.2520, 1.0760, 2.1899) -- (2.2050, 1.0760, 2.1945) -- cycle;
\fill[blue!42.9, opacity=0.7] (2.2050, 1.0760, 2.1945) -- (2.2520, 1.0760, 2.1899) -- (2.2520, 1.1290, 2.1923) -- (2.2050, 1.1290, 2.1969) -- cycle;
\fill[blue!36.4, opacity=0.7] (2.2050, 1.1290, 2.1969) -- (2.2520, 1.1290, 2.1923) -- (2.2520, 1.1820, 2.1944) -- (2.2050, 1.1820, 2.1990) -- cycle;
\fill[blue!33.0, opacity=0.7] (2.2050, 1.1820, 2.1990) -- (2.2520, 1.1820, 2.1944) -- (2.2520, 1.2350, 2.1962) -- (2.2050, 1.2350, 2.2008) -- cycle;
\fill[blue!31.6, opacity=0.7] (2.2050, 1.2350, 2.2008) -- (2.2520, 1.2350, 2.1962) -- (2.2520, 1.2880, 2.1977) -- (2.2050, 1.2880, 2.2022) -- cycle;
\fill[blue!31.2, opacity=0.7] (2.2050, 1.2880, 2.2022) -- (2.2520, 1.2880, 2.1977) -- (2.2520, 1.3410, 2.1988) -- (2.2050, 1.3410, 2.2034) -- cycle;
\fill[blue!31.7, opacity=0.7] (2.2050, 1.3410, 2.2034) -- (2.2520, 1.3410, 2.1988) -- (2.2520, 1.3940, 2.1996) -- (2.2050, 1.3940, 2.2042) -- cycle;
\fill[blue!32.9, opacity=0.7] (2.2050, 1.3940, 2.2042) -- (2.2520, 1.3940, 2.1996) -- (2.2520, 1.4470, 2.2001) -- (2.2050, 1.4470, 2.2047) -- cycle;
\fill[blue!35.1, opacity=0.7] (2.2050, 1.4470, 2.2047) -- (2.2520, 1.4470, 2.2001) -- (2.2520, 1.5000, 2.2003) -- (2.2050, 1.5000, 2.2049) -- cycle;
\fill[blue!39.1, opacity=0.7] (2.2050, 1.5000, 2.2049) -- (2.2520, 1.5000, 2.2003) -- (2.2520, 1.5530, 2.2001) -- (2.2050, 1.5530, 2.2047) -- cycle;
\fill[blue!45.9, opacity=0.7] (2.2050, 1.5530, 2.2047) -- (2.2520, 1.5530, 2.2001) -- (2.2520, 1.6060, 2.1996) -- (2.2050, 1.6060, 2.2042) -- cycle;
\fill[blue!57.1, opacity=0.7] (2.2050, 1.6060, 2.2042) -- (2.2520, 1.6060, 2.1996) -- (2.2520, 1.6590, 2.1988) -- (2.2050, 1.6590, 2.2034) -- cycle;
\fill[blue!72.2, opacity=0.7] (2.2050, 1.6590, 2.2034) -- (2.2520, 1.6590, 2.1988) -- (2.2520, 1.7120, 2.1977) -- (2.2050, 1.7120, 2.2022) -- cycle;
\fill[blue!85.3, opacity=0.7] (2.2050, 1.7120, 2.2022) -- (2.2520, 1.7120, 2.1977) -- (2.2520, 1.7650, 2.1962) -- (2.2050, 1.7650, 2.2008) -- cycle;
\fill[blue!86.6, opacity=0.7] (2.2050, 1.7650, 2.2008) -- (2.2520, 1.7650, 2.1962) -- (2.2520, 1.8180, 2.1944) -- (2.2050, 1.8180, 2.1990) -- cycle;
\fill[blue!76.0, opacity=0.7] (2.2050, 1.8180, 2.1990) -- (2.2520, 1.8180, 2.1944) -- (2.2520, 1.8710, 2.1923) -- (2.2050, 1.8710, 2.1969) -- cycle;
\fill[blue!65.8, opacity=0.7] (2.2050, 1.8710, 2.1969) -- (2.2520, 1.8710, 2.1923) -- (2.2520, 1.9240, 2.1899) -- (2.2050, 1.9240, 2.1945) -- cycle;
\fill[blue!66.0, opacity=0.7] (2.2050, 1.9240, 2.1945) -- (2.2520, 1.9240, 2.1899) -- (2.2520, 1.9770, 2.1872) -- (2.2050, 1.9770, 2.1918) -- cycle;
\fill[blue!78.7, opacity=0.7] (2.2050, 1.9770, 2.1918) -- (2.2520, 1.9770, 2.1872) -- (2.2520, 2.0300, 2.1842) -- (2.2050, 2.0300, 2.1888) -- cycle;
\fill[blue!87.3, opacity=0.7] (2.2050, 2.0300, 2.1888) -- (2.2520, 2.0300, 2.1842) -- (2.2520, 2.0830, 2.1809) -- (2.2050, 2.0830, 2.1855) -- cycle;
\fill[blue!60.5, opacity=0.7] (2.2050, 2.0830, 2.1855) -- (2.2520, 2.0830, 2.1809) -- (2.2520, 2.1360, 2.1774) -- (2.2050, 2.1360, 2.1819) -- cycle;
\fill[blue!28.6, opacity=0.7] (2.2050, 2.1360, 2.1819) -- (2.2520, 2.1360, 2.1774) -- (2.2520, 2.1890, 2.1736) -- (2.2050, 2.1890, 2.1781) -- cycle;
\fill[blue!18.4, opacity=0.7] (2.2050, 2.1890, 2.1781) -- (2.2520, 2.1890, 2.1736) -- (2.2520, 2.2420, 2.1695) -- (2.2050, 2.2420, 2.1740) -- cycle;
\fill[blue!16.9, opacity=0.7] (2.2050, 2.2420, 2.1740) -- (2.2520, 2.2420, 2.1695) -- (2.2520, 2.2950, 2.1651) -- (2.2050, 2.2950, 2.1697) -- cycle;
\fill[blue!18.8, opacity=0.7] (2.2050, 2.2950, 2.1697) -- (2.2520, 2.2950, 2.1651) -- (2.2520, 2.3480, 2.1606) -- (2.2050, 2.3480, 2.1651) -- cycle;
\fill[blue!30.3, opacity=0.7] (2.2050, 2.3480, 2.1651) -- (2.2520, 2.3480, 2.1606) -- (2.2520, 2.4010, 2.1558) -- (2.2050, 2.4010, 2.1604) -- cycle;
\fill[blue!60.7, opacity=0.7] (2.2050, 2.4010, 2.1604) -- (2.2520, 2.4010, 2.1558) -- (2.2520, 2.4540, 2.1508) -- (2.2050, 2.4540, 2.1554) -- cycle;
\fill[blue!83.2, opacity=0.7] (2.2050, 2.4540, 2.1554) -- (2.2520, 2.4540, 2.1508) -- (2.2520, 2.5070, 2.1457) -- (2.2050, 2.5070, 2.1502) -- cycle;
\fill[blue!85.8, opacity=0.7] (2.2050, 2.5070, 2.1502) -- (2.2520, 2.5070, 2.1457) -- (2.2520, 2.5600, 2.1403) -- (2.2050, 2.5600, 2.1449) -- cycle;
\fill[blue!71.4, opacity=0.7] (2.2050, 2.5600, 2.1449) -- (2.2520, 2.5600, 2.1403) -- (2.2520, 2.6130, 2.1348) -- (2.2050, 2.6130, 2.1393) -- cycle;
\fill[blue!33.2, opacity=0.7] (2.2050, 2.6130, 2.1393) -- (2.2520, 2.6130, 2.1348) -- (2.2520, 2.6660, 2.1291) -- (2.2050, 2.6660, 2.1337) -- cycle;
\fill[blue!16.6, opacity=0.7] (2.2050, 2.6660, 2.1337) -- (2.2520, 2.6660, 2.1291) -- (2.2520, 2.7190, 2.1233) -- (2.2050, 2.7190, 2.1279) -- cycle;
\fill[blue!15.1, opacity=0.7] (2.2050, 2.7190, 2.1279) -- (2.2520, 2.7190, 2.1233) -- (2.2520, 2.7720, 2.1174) -- (2.2050, 2.7720, 2.1219) -- cycle;
\fill[blue!15.0, opacity=0.7] (2.2050, 2.7720, 2.1219) -- (2.2520, 2.7720, 2.1174) -- (2.2520, 2.8250, 2.1114) -- (2.2050, 2.8250, 2.1159) -- cycle;
\fill[blue!15.2, opacity=0.7] (2.2050, 2.8250, 2.1159) -- (2.2520, 2.8250, 2.1114) -- (2.2520, 2.8780, 2.1052) -- (2.2050, 2.8780, 2.1098) -- cycle;
\fill[blue!17.1, opacity=0.7] (2.2050, 2.8780, 2.1098) -- (2.2520, 2.8780, 2.1052) -- (2.2520, 2.9310, 2.0991) -- (2.2050, 2.9310, 2.1036) -- cycle;
\fill[blue!28.5, opacity=0.7] (2.2050, 2.9310, 2.1036) -- (2.2520, 2.9310, 2.0991) -- (2.2520, 2.9840, 2.0928) -- (2.2050, 2.9840, 2.0974) -- cycle;
\fill[blue!41.6, opacity=0.7] (2.2050, 2.9840, 2.0974) -- (2.2520, 2.9840, 2.0928) -- (2.2520, 3.0370, 2.0866) -- (2.2050, 3.0370, 2.0911) -- cycle;
\fill[blue!33.0, opacity=0.7] (2.2050, 3.0370, 2.0911) -- (2.2520, 3.0370, 2.0866) -- (2.2520, 3.0900, 2.0803) -- (2.2050, 3.0900, 2.0849) -- cycle;
\fill[blue!15.0, opacity=0.7] (2.2520, -0.0900, 2.0803) -- (2.2990, -0.0900, 2.0755) -- (2.2990, -0.0370, 2.0818) -- (2.2520, -0.0370, 2.0866) -- cycle;
\fill[blue!15.0, opacity=0.7] (2.2520, -0.0370, 2.0866) -- (2.2990, -0.0370, 2.0818) -- (2.2990, 0.0160, 2.0881) -- (2.2520, 0.0160, 2.0928) -- cycle;
\fill[blue!16.3, opacity=0.7] (2.2520, 0.0160, 2.0928) -- (2.2990, 0.0160, 2.0881) -- (2.2990, 0.0690, 2.0943) -- (2.2520, 0.0690, 2.0991) -- cycle;
\fill[blue!28.2, opacity=0.7] (2.2520, 0.0690, 2.0991) -- (2.2990, 0.0690, 2.0943) -- (2.2990, 0.1220, 2.1005) -- (2.2520, 0.1220, 2.1052) -- cycle;
\fill[blue!51.9, opacity=0.7] (2.2520, 0.1220, 2.1052) -- (2.2990, 0.1220, 2.1005) -- (2.2990, 0.1750, 2.1066) -- (2.2520, 0.1750, 2.1114) -- cycle;
\fill[blue!60.2, opacity=0.7] (2.2520, 0.1750, 2.1114) -- (2.2990, 0.1750, 2.1066) -- (2.2990, 0.2280, 2.1126) -- (2.2520, 0.2280, 2.1174) -- cycle;
\fill[blue!47.2, opacity=0.7] (2.2520, 0.2280, 2.1174) -- (2.2990, 0.2280, 2.1126) -- (2.2990, 0.2810, 2.1185) -- (2.2520, 0.2810, 2.1233) -- cycle;
\fill[blue!26.9, opacity=0.7] (2.2520, 0.2810, 2.1233) -- (2.2990, 0.2810, 2.1185) -- (2.2990, 0.3340, 2.1243) -- (2.2520, 0.3340, 2.1291) -- cycle;
\fill[blue!17.4, opacity=0.7] (2.2520, 0.3340, 2.1291) -- (2.2990, 0.3340, 2.1243) -- (2.2990, 0.3870, 2.1300) -- (2.2520, 0.3870, 2.1348) -- cycle;
\fill[blue!15.5, opacity=0.7] (2.2520, 0.3870, 2.1348) -- (2.2990, 0.3870, 2.1300) -- (2.2990, 0.4400, 2.1355) -- (2.2520, 0.4400, 2.1403) -- cycle;
\fill[blue!15.3, opacity=0.7] (2.2520, 0.4400, 2.1403) -- (2.2990, 0.4400, 2.1355) -- (2.2990, 0.4930, 2.1409) -- (2.2520, 0.4930, 2.1457) -- cycle;
\fill[blue!15.5, opacity=0.7] (2.2520, 0.4930, 2.1457) -- (2.2990, 0.4930, 2.1409) -- (2.2990, 0.5460, 2.1461) -- (2.2520, 0.5460, 2.1508) -- cycle;
\fill[blue!17.5, opacity=0.7] (2.2520, 0.5460, 2.1508) -- (2.2990, 0.5460, 2.1461) -- (2.2990, 0.5990, 2.1510) -- (2.2520, 0.5990, 2.1558) -- cycle;
\fill[blue!28.2, opacity=0.7] (2.2520, 0.5990, 2.1558) -- (2.2990, 0.5990, 2.1510) -- (2.2990, 0.6520, 2.1558) -- (2.2520, 0.6520, 2.1606) -- cycle;
\fill[blue!56.8, opacity=0.7] (2.2520, 0.6520, 2.1606) -- (2.2990, 0.6520, 2.1558) -- (2.2990, 0.7050, 2.1604) -- (2.2520, 0.7050, 2.1651) -- cycle;
\fill[blue!83.3, opacity=0.7] (2.2520, 0.7050, 2.1651) -- (2.2990, 0.7050, 2.1604) -- (2.2990, 0.7580, 2.1647) -- (2.2520, 0.7580, 2.1695) -- cycle;
\fill[blue!87.2, opacity=0.7] (2.2520, 0.7580, 2.1695) -- (2.2990, 0.7580, 2.1647) -- (2.2990, 0.8110, 2.1688) -- (2.2520, 0.8110, 2.1736) -- cycle;
\fill[blue!82.5, opacity=0.7] (2.2520, 0.8110, 2.1736) -- (2.2990, 0.8110, 2.1688) -- (2.2990, 0.8640, 2.1726) -- (2.2520, 0.8640, 2.1774) -- cycle;
\fill[blue!81.3, opacity=0.7] (2.2520, 0.8640, 2.1774) -- (2.2990, 0.8640, 2.1726) -- (2.2990, 0.9170, 2.1762) -- (2.2520, 0.9170, 2.1809) -- cycle;
\fill[blue!84.5, opacity=0.7] (2.2520, 0.9170, 2.1809) -- (2.2990, 0.9170, 2.1762) -- (2.2990, 0.9700, 2.1794) -- (2.2520, 0.9700, 2.1842) -- cycle;
\fill[blue!87.7, opacity=0.7] (2.2520, 0.9700, 2.1842) -- (2.2990, 0.9700, 2.1794) -- (2.2990, 1.0230, 2.1824) -- (2.2520, 1.0230, 2.1872) -- cycle;
\fill[blue!86.4, opacity=0.7] (2.2520, 1.0230, 2.1872) -- (2.2990, 1.0230, 2.1824) -- (2.2990, 1.0760, 2.1851) -- (2.2520, 1.0760, 2.1899) -- cycle;
\fill[blue!80.4, opacity=0.7] (2.2520, 1.0760, 2.1899) -- (2.2990, 1.0760, 2.1851) -- (2.2990, 1.1290, 2.1875) -- (2.2520, 1.1290, 2.1923) -- cycle;
\fill[blue!72.4, opacity=0.7] (2.2520, 1.1290, 2.1923) -- (2.2990, 1.1290, 2.1875) -- (2.2990, 1.1820, 2.1896) -- (2.2520, 1.1820, 2.1944) -- cycle;
\fill[blue!65.1, opacity=0.7] (2.2520, 1.1820, 2.1944) -- (2.2990, 1.1820, 2.1896) -- (2.2990, 1.2350, 2.1914) -- (2.2520, 1.2350, 2.1962) -- cycle;
\fill[blue!60.2, opacity=0.7] (2.2520, 1.2350, 2.1962) -- (2.2990, 1.2350, 2.1914) -- (2.2990, 1.2880, 2.1929) -- (2.2520, 1.2880, 2.1977) -- cycle;
\fill[blue!57.8, opacity=0.7] (2.2520, 1.2880, 2.1977) -- (2.2990, 1.2880, 2.1929) -- (2.2990, 1.3410, 2.1940) -- (2.2520, 1.3410, 2.1988) -- cycle;
\fill[blue!57.9, opacity=0.7] (2.2520, 1.3410, 2.1988) -- (2.2990, 1.3410, 2.1940) -- (2.2990, 1.3940, 2.1949) -- (2.2520, 1.3940, 2.1996) -- cycle;
\fill[blue!60.6, opacity=0.7] (2.2520, 1.3940, 2.1996) -- (2.2990, 1.3940, 2.1949) -- (2.2990, 1.4470, 2.1954) -- (2.2520, 1.4470, 2.2001) -- cycle;
\fill[blue!65.8, opacity=0.7] (2.2520, 1.4470, 2.2001) -- (2.2990, 1.4470, 2.1954) -- (2.2990, 1.5000, 2.1955) -- (2.2520, 1.5000, 2.2003) -- cycle;
\fill[blue!73.2, opacity=0.7] (2.2520, 1.5000, 2.2003) -- (2.2990, 1.5000, 2.1955) -- (2.2990, 1.5530, 2.1954) -- (2.2520, 1.5530, 2.2001) -- cycle;
\fill[blue!81.5, opacity=0.7] (2.2520, 1.5530, 2.2001) -- (2.2990, 1.5530, 2.1954) -- (2.2990, 1.6060, 2.1949) -- (2.2520, 1.6060, 2.1996) -- cycle;
\fill[blue!87.2, opacity=0.7] (2.2520, 1.6060, 2.1996) -- (2.2990, 1.6060, 2.1949) -- (2.2990, 1.6590, 2.1940) -- (2.2520, 1.6590, 2.1988) -- cycle;
\fill[blue!86.5, opacity=0.7] (2.2520, 1.6590, 2.1988) -- (2.2990, 1.6590, 2.1940) -- (2.2990, 1.7120, 2.1929) -- (2.2520, 1.7120, 2.1977) -- cycle;
\fill[blue!79.0, opacity=0.7] (2.2520, 1.7120, 2.1977) -- (2.2990, 1.7120, 2.1929) -- (2.2990, 1.7650, 2.1914) -- (2.2520, 1.7650, 2.1962) -- cycle;
\fill[blue!70.0, opacity=0.7] (2.2520, 1.7650, 2.1962) -- (2.2990, 1.7650, 2.1914) -- (2.2990, 1.8180, 2.1896) -- (2.2520, 1.8180, 2.1944) -- cycle;
\fill[blue!66.1, opacity=0.7] (2.2520, 1.8180, 2.1944) -- (2.2990, 1.8180, 2.1896) -- (2.2990, 1.8710, 2.1875) -- (2.2520, 1.8710, 2.1923) -- cycle;
\fill[blue!71.3, opacity=0.7] (2.2520, 1.8710, 2.1923) -- (2.2990, 1.8710, 2.1875) -- (2.2990, 1.9240, 2.1851) -- (2.2520, 1.9240, 2.1899) -- cycle;
\fill[blue!83.9, opacity=0.7] (2.2520, 1.9240, 2.1899) -- (2.2990, 1.9240, 2.1851) -- (2.2990, 1.9770, 2.1824) -- (2.2520, 1.9770, 2.1872) -- cycle;
\fill[blue!84.7, opacity=0.7] (2.2520, 1.9770, 2.1872) -- (2.2990, 1.9770, 2.1824) -- (2.2990, 2.0300, 2.1794) -- (2.2520, 2.0300, 2.1842) -- cycle;
\fill[blue!55.4, opacity=0.7] (2.2520, 2.0300, 2.1842) -- (2.2990, 2.0300, 2.1794) -- (2.2990, 2.0830, 2.1762) -- (2.2520, 2.0830, 2.1809) -- cycle;
\fill[blue!27.6, opacity=0.7] (2.2520, 2.0830, 2.1809) -- (2.2990, 2.0830, 2.1762) -- (2.2990, 2.1360, 2.1726) -- (2.2520, 2.1360, 2.1774) -- cycle;
\fill[blue!18.4, opacity=0.7] (2.2520, 2.1360, 2.1774) -- (2.2990, 2.1360, 2.1726) -- (2.2990, 2.1890, 2.1688) -- (2.2520, 2.1890, 2.1736) -- cycle;
\fill[blue!16.8, opacity=0.7] (2.2520, 2.1890, 2.1736) -- (2.2990, 2.1890, 2.1688) -- (2.2990, 2.2420, 2.1647) -- (2.2520, 2.2420, 2.1695) -- cycle;
\fill[blue!18.0, opacity=0.7] (2.2520, 2.2420, 2.1695) -- (2.2990, 2.2420, 2.1647) -- (2.2990, 2.2950, 2.1604) -- (2.2520, 2.2950, 2.1651) -- cycle;
\fill[blue!25.9, opacity=0.7] (2.2520, 2.2950, 2.1651) -- (2.2990, 2.2950, 2.1604) -- (2.2990, 2.3480, 2.1558) -- (2.2520, 2.3480, 2.1606) -- cycle;
\fill[blue!51.2, opacity=0.7] (2.2520, 2.3480, 2.1606) -- (2.2990, 2.3480, 2.1558) -- (2.2990, 2.4010, 2.1510) -- (2.2520, 2.4010, 2.1558) -- cycle;
\fill[blue!78.7, opacity=0.7] (2.2520, 2.4010, 2.1558) -- (2.2990, 2.4010, 2.1510) -- (2.2990, 2.4540, 2.1461) -- (2.2520, 2.4540, 2.1508) -- cycle;
\fill[blue!85.9, opacity=0.7] (2.2520, 2.4540, 2.1508) -- (2.2990, 2.4540, 2.1461) -- (2.2990, 2.5070, 2.1409) -- (2.2520, 2.5070, 2.1457) -- cycle;
\fill[blue!79.2, opacity=0.7] (2.2520, 2.5070, 2.1457) -- (2.2990, 2.5070, 2.1409) -- (2.2990, 2.5600, 2.1355) -- (2.2520, 2.5600, 2.1403) -- cycle;
\fill[blue!47.0, opacity=0.7] (2.2520, 2.5600, 2.1403) -- (2.2990, 2.5600, 2.1355) -- (2.2990, 2.6130, 2.1300) -- (2.2520, 2.6130, 2.1348) -- cycle;
\fill[blue!19.7, opacity=0.7] (2.2520, 2.6130, 2.1348) -- (2.2990, 2.6130, 2.1300) -- (2.2990, 2.6660, 2.1243) -- (2.2520, 2.6660, 2.1291) -- cycle;
\fill[blue!15.3, opacity=0.7] (2.2520, 2.6660, 2.1291) -- (2.2990, 2.6660, 2.1243) -- (2.2990, 2.7190, 2.1185) -- (2.2520, 2.7190, 2.1233) -- cycle;
\fill[blue!15.0, opacity=0.7] (2.2520, 2.7190, 2.1233) -- (2.2990, 2.7190, 2.1185) -- (2.2990, 2.7720, 2.1126) -- (2.2520, 2.7720, 2.1174) -- cycle;
\fill[blue!15.1, opacity=0.7] (2.2520, 2.7720, 2.1174) -- (2.2990, 2.7720, 2.1126) -- (2.2990, 2.8250, 2.1066) -- (2.2520, 2.8250, 2.1114) -- cycle;
\fill[blue!15.6, opacity=0.7] (2.2520, 2.8250, 2.1114) -- (2.2990, 2.8250, 2.1066) -- (2.2990, 2.8780, 2.1005) -- (2.2520, 2.8780, 2.1052) -- cycle;
\fill[blue!20.8, opacity=0.7] (2.2520, 2.8780, 2.1052) -- (2.2990, 2.8780, 2.1005) -- (2.2990, 2.9310, 2.0943) -- (2.2520, 2.9310, 2.0991) -- cycle;
\fill[blue!35.9, opacity=0.7] (2.2520, 2.9310, 2.0991) -- (2.2990, 2.9310, 2.0943) -- (2.2990, 2.9840, 2.0881) -- (2.2520, 2.9840, 2.0928) -- cycle;
\fill[blue!40.1, opacity=0.7] (2.2520, 2.9840, 2.0928) -- (2.2990, 2.9840, 2.0881) -- (2.2990, 3.0370, 2.0818) -- (2.2520, 3.0370, 2.0866) -- cycle;
\fill[blue!24.6, opacity=0.7] (2.2520, 3.0370, 2.0866) -- (2.2990, 3.0370, 2.0818) -- (2.2990, 3.0900, 2.0755) -- (2.2520, 3.0900, 2.0803) -- cycle;
\fill[blue!15.0, opacity=0.7] (2.2990, -0.0900, 2.0755) -- (2.3460, -0.0900, 2.0705) -- (2.3460, -0.0370, 2.0768) -- (2.2990, -0.0370, 2.0818) -- cycle;
\fill[blue!15.0, opacity=0.7] (2.2990, -0.0370, 2.0818) -- (2.3460, -0.0370, 2.0768) -- (2.3460, 0.0160, 2.0831) -- (2.2990, 0.0160, 2.0881) -- cycle;
\fill[blue!15.1, opacity=0.7] (2.2990, 0.0160, 2.0881) -- (2.3460, 0.0160, 2.0831) -- (2.3460, 0.0690, 2.0893) -- (2.2990, 0.0690, 2.0943) -- cycle;
\fill[blue!17.7, opacity=0.7] (2.2990, 0.0690, 2.0943) -- (2.3460, 0.0690, 2.0893) -- (2.3460, 0.1220, 2.0955) -- (2.2990, 0.1220, 2.1005) -- cycle;
\fill[blue!33.7, opacity=0.7] (2.2990, 0.1220, 2.1005) -- (2.3460, 0.1220, 2.0955) -- (2.3460, 0.1750, 2.1016) -- (2.2990, 0.1750, 2.1066) -- cycle;
\fill[blue!55.8, opacity=0.7] (2.2990, 0.1750, 2.1066) -- (2.3460, 0.1750, 2.1016) -- (2.3460, 0.2280, 2.1076) -- (2.2990, 0.2280, 2.1126) -- cycle;
\fill[blue!60.3, opacity=0.7] (2.2990, 0.2280, 2.1126) -- (2.3460, 0.2280, 2.1076) -- (2.3460, 0.2810, 2.1135) -- (2.2990, 0.2810, 2.1185) -- cycle;
\fill[blue!45.8, opacity=0.7] (2.2990, 0.2810, 2.1185) -- (2.3460, 0.2810, 2.1135) -- (2.3460, 0.3340, 2.1193) -- (2.2990, 0.3340, 2.1243) -- cycle;
\fill[blue!26.3, opacity=0.7] (2.2990, 0.3340, 2.1243) -- (2.3460, 0.3340, 2.1193) -- (2.3460, 0.3870, 2.1250) -- (2.2990, 0.3870, 2.1300) -- cycle;
\fill[blue!17.5, opacity=0.7] (2.2990, 0.3870, 2.1300) -- (2.3460, 0.3870, 2.1250) -- (2.3460, 0.4400, 2.1305) -- (2.2990, 0.4400, 2.1355) -- cycle;
\fill[blue!15.6, opacity=0.7] (2.2990, 0.4400, 2.1355) -- (2.3460, 0.4400, 2.1305) -- (2.3460, 0.4930, 2.1359) -- (2.2990, 0.4930, 2.1409) -- cycle;
\fill[blue!15.3, opacity=0.7] (2.2990, 0.4930, 2.1409) -- (2.3460, 0.4930, 2.1359) -- (2.3460, 0.5460, 2.1411) -- (2.2990, 0.5460, 2.1461) -- cycle;
\fill[blue!15.5, opacity=0.7] (2.2990, 0.5460, 2.1461) -- (2.3460, 0.5460, 2.1411) -- (2.3460, 0.5990, 2.1461) -- (2.2990, 0.5990, 2.1510) -- cycle;
\fill[blue!16.7, opacity=0.7] (2.2990, 0.5990, 2.1510) -- (2.3460, 0.5990, 2.1461) -- (2.3460, 0.6520, 2.1508) -- (2.2990, 0.6520, 2.1558) -- cycle;
\fill[blue!23.4, opacity=0.7] (2.2990, 0.6520, 2.1558) -- (2.3460, 0.6520, 2.1508) -- (2.3460, 0.7050, 2.1554) -- (2.2990, 0.7050, 2.1604) -- cycle;
\fill[blue!44.2, opacity=0.7] (2.2990, 0.7050, 2.1604) -- (2.3460, 0.7050, 2.1554) -- (2.3460, 0.7580, 2.1597) -- (2.2990, 0.7580, 2.1647) -- cycle;
\fill[blue!73.0, opacity=0.7] (2.2990, 0.7580, 2.1647) -- (2.3460, 0.7580, 2.1597) -- (2.3460, 0.8110, 2.1638) -- (2.2990, 0.8110, 2.1688) -- cycle;
\fill[blue!87.2, opacity=0.7] (2.2990, 0.8110, 2.1688) -- (2.3460, 0.8110, 2.1638) -- (2.3460, 0.8640, 2.1676) -- (2.2990, 0.8640, 2.1726) -- cycle;
\fill[blue!85.8, opacity=0.7] (2.2990, 0.8640, 2.1726) -- (2.3460, 0.8640, 2.1676) -- (2.3460, 0.9170, 2.1712) -- (2.2990, 0.9170, 2.1762) -- cycle;
\fill[blue!81.2, opacity=0.7] (2.2990, 0.9170, 2.1762) -- (2.3460, 0.9170, 2.1712) -- (2.3460, 0.9700, 2.1745) -- (2.2990, 0.9700, 2.1794) -- cycle;
\fill[blue!79.7, opacity=0.7] (2.2990, 0.9700, 2.1794) -- (2.3460, 0.9700, 2.1745) -- (2.3460, 1.0230, 2.1775) -- (2.2990, 1.0230, 2.1824) -- cycle;
\fill[blue!81.2, opacity=0.7] (2.2990, 1.0230, 2.1824) -- (2.3460, 1.0230, 2.1775) -- (2.3460, 1.0760, 2.1802) -- (2.2990, 1.0760, 2.1851) -- cycle;
\fill[blue!84.1, opacity=0.7] (2.2990, 1.0760, 2.1851) -- (2.3460, 1.0760, 2.1802) -- (2.3460, 1.1290, 2.1826) -- (2.2990, 1.1290, 2.1875) -- cycle;
\fill[blue!86.5, opacity=0.7] (2.2990, 1.1290, 2.1875) -- (2.3460, 1.1290, 2.1826) -- (2.3460, 1.1820, 2.1847) -- (2.2990, 1.1820, 2.1896) -- cycle;
\fill[blue!87.7, opacity=0.7] (2.2990, 1.1820, 2.1896) -- (2.3460, 1.1820, 2.1847) -- (2.3460, 1.2350, 2.1864) -- (2.2990, 1.2350, 2.1914) -- cycle;
\fill[blue!87.8, opacity=0.7] (2.2990, 1.2350, 2.1914) -- (2.3460, 1.2350, 2.1864) -- (2.3460, 1.2880, 2.1879) -- (2.2990, 1.2880, 2.1929) -- cycle;
\fill[blue!87.7, opacity=0.7] (2.2990, 1.2880, 2.1929) -- (2.3460, 1.2880, 2.1879) -- (2.3460, 1.3410, 2.1891) -- (2.2990, 1.3410, 2.1940) -- cycle;
\fill[blue!87.7, opacity=0.7] (2.2990, 1.3410, 2.1940) -- (2.3460, 1.3410, 2.1891) -- (2.3460, 1.3940, 2.1899) -- (2.2990, 1.3940, 2.1949) -- cycle;
\fill[blue!87.9, opacity=0.7] (2.2990, 1.3940, 2.1949) -- (2.3460, 1.3940, 2.1899) -- (2.3460, 1.4470, 2.1904) -- (2.2990, 1.4470, 2.1954) -- cycle;
\fill[blue!87.5, opacity=0.7] (2.2990, 1.4470, 2.1954) -- (2.3460, 1.4470, 2.1904) -- (2.3460, 1.5000, 2.1905) -- (2.2990, 1.5000, 2.1955) -- cycle;
\fill[blue!85.6, opacity=0.7] (2.2990, 1.5000, 2.1955) -- (2.3460, 1.5000, 2.1905) -- (2.3460, 1.5530, 2.1904) -- (2.2990, 1.5530, 2.1954) -- cycle;
\fill[blue!81.6, opacity=0.7] (2.2990, 1.5530, 2.1954) -- (2.3460, 1.5530, 2.1904) -- (2.3460, 1.6060, 2.1899) -- (2.2990, 1.6060, 2.1949) -- cycle;
\fill[blue!76.0, opacity=0.7] (2.2990, 1.6060, 2.1949) -- (2.3460, 1.6060, 2.1899) -- (2.3460, 1.6590, 2.1891) -- (2.2990, 1.6590, 2.1940) -- cycle;
\fill[blue!70.6, opacity=0.7] (2.2990, 1.6590, 2.1940) -- (2.3460, 1.6590, 2.1891) -- (2.3460, 1.7120, 2.1879) -- (2.2990, 1.7120, 2.1929) -- cycle;
\fill[blue!68.2, opacity=0.7] (2.2990, 1.7120, 2.1929) -- (2.3460, 1.7120, 2.1879) -- (2.3460, 1.7650, 2.1864) -- (2.2990, 1.7650, 2.1914) -- cycle;
\fill[blue!71.4, opacity=0.7] (2.2990, 1.7650, 2.1914) -- (2.3460, 1.7650, 2.1864) -- (2.3460, 1.8180, 2.1847) -- (2.2990, 1.8180, 2.1896) -- cycle;
\fill[blue!80.5, opacity=0.7] (2.2990, 1.8180, 2.1896) -- (2.3460, 1.8180, 2.1847) -- (2.3460, 1.8710, 2.1826) -- (2.2990, 1.8710, 2.1875) -- cycle;
\fill[blue!87.8, opacity=0.7] (2.2990, 1.8710, 2.1875) -- (2.3460, 1.8710, 2.1826) -- (2.3460, 1.9240, 2.1802) -- (2.2990, 1.9240, 2.1851) -- cycle;
\fill[blue!75.6, opacity=0.7] (2.2990, 1.9240, 2.1851) -- (2.3460, 1.9240, 2.1802) -- (2.3460, 1.9770, 2.1775) -- (2.2990, 1.9770, 2.1824) -- cycle;
\fill[blue!45.1, opacity=0.7] (2.2990, 1.9770, 2.1824) -- (2.3460, 1.9770, 2.1775) -- (2.3460, 2.0300, 2.1745) -- (2.2990, 2.0300, 2.1794) -- cycle;
\fill[blue!24.4, opacity=0.7] (2.2990, 2.0300, 2.1794) -- (2.3460, 2.0300, 2.1745) -- (2.3460, 2.0830, 2.1712) -- (2.2990, 2.0830, 2.1762) -- cycle;
\fill[blue!17.9, opacity=0.7] (2.2990, 2.0830, 2.1762) -- (2.3460, 2.0830, 2.1712) -- (2.3460, 2.1360, 2.1676) -- (2.2990, 2.1360, 2.1726) -- cycle;
\fill[blue!16.7, opacity=0.7] (2.2990, 2.1360, 2.1726) -- (2.3460, 2.1360, 2.1676) -- (2.3460, 2.1890, 2.1638) -- (2.2990, 2.1890, 2.1688) -- cycle;
\fill[blue!17.7, opacity=0.7] (2.2990, 2.1890, 2.1688) -- (2.3460, 2.1890, 2.1638) -- (2.3460, 2.2420, 2.1597) -- (2.2990, 2.2420, 2.1647) -- cycle;
\fill[blue!24.1, opacity=0.7] (2.2990, 2.2420, 2.1647) -- (2.3460, 2.2420, 2.1597) -- (2.3460, 2.2950, 2.1554) -- (2.2990, 2.2950, 2.1604) -- cycle;
\fill[blue!45.5, opacity=0.7] (2.2990, 2.2950, 2.1604) -- (2.3460, 2.2950, 2.1554) -- (2.3460, 2.3480, 2.1508) -- (2.2990, 2.3480, 2.1558) -- cycle;
\fill[blue!74.1, opacity=0.7] (2.2990, 2.3480, 2.1558) -- (2.3460, 2.3480, 2.1508) -- (2.3460, 2.4010, 2.1461) -- (2.2990, 2.4010, 2.1510) -- cycle;
\fill[blue!85.1, opacity=0.7] (2.2990, 2.4010, 2.1510) -- (2.3460, 2.4010, 2.1461) -- (2.3460, 2.4540, 2.1411) -- (2.2990, 2.4540, 2.1461) -- cycle;
\fill[blue!82.4, opacity=0.7] (2.2990, 2.4540, 2.1461) -- (2.3460, 2.4540, 2.1411) -- (2.3460, 2.5070, 2.1359) -- (2.2990, 2.5070, 2.1409) -- cycle;
\fill[blue!58.4, opacity=0.7] (2.2990, 2.5070, 2.1409) -- (2.3460, 2.5070, 2.1359) -- (2.3460, 2.5600, 2.1305) -- (2.2990, 2.5600, 2.1355) -- cycle;
\fill[blue!25.0, opacity=0.7] (2.2990, 2.5600, 2.1355) -- (2.3460, 2.5600, 2.1305) -- (2.3460, 2.6130, 2.1250) -- (2.2990, 2.6130, 2.1300) -- cycle;
\fill[blue!15.8, opacity=0.7] (2.2990, 2.6130, 2.1300) -- (2.3460, 2.6130, 2.1250) -- (2.3460, 2.6660, 2.1193) -- (2.2990, 2.6660, 2.1243) -- cycle;
\fill[blue!15.1, opacity=0.7] (2.2990, 2.6660, 2.1243) -- (2.3460, 2.6660, 2.1193) -- (2.3460, 2.7190, 2.1135) -- (2.2990, 2.7190, 2.1185) -- cycle;
\fill[blue!15.0, opacity=0.7] (2.2990, 2.7190, 2.1185) -- (2.3460, 2.7190, 2.1135) -- (2.3460, 2.7720, 2.1076) -- (2.2990, 2.7720, 2.1126) -- cycle;
\fill[blue!15.2, opacity=0.7] (2.2990, 2.7720, 2.1126) -- (2.3460, 2.7720, 2.1076) -- (2.3460, 2.8250, 2.1016) -- (2.2990, 2.8250, 2.1066) -- cycle;
\fill[blue!17.1, opacity=0.7] (2.2990, 2.8250, 2.1066) -- (2.3460, 2.8250, 2.1016) -- (2.3460, 2.8780, 2.0955) -- (2.2990, 2.8780, 2.1005) -- cycle;
\fill[blue!27.8, opacity=0.7] (2.2990, 2.8780, 2.1005) -- (2.3460, 2.8780, 2.0955) -- (2.3460, 2.9310, 2.0893) -- (2.2990, 2.9310, 2.0943) -- cycle;
\fill[blue!40.5, opacity=0.7] (2.2990, 2.9310, 2.0943) -- (2.3460, 2.9310, 2.0893) -- (2.3460, 2.9840, 2.0831) -- (2.2990, 2.9840, 2.0881) -- cycle;
\fill[blue!33.3, opacity=0.7] (2.2990, 2.9840, 2.0881) -- (2.3460, 2.9840, 2.0831) -- (2.3460, 3.0370, 2.0768) -- (2.2990, 3.0370, 2.0818) -- cycle;
\fill[blue!18.4, opacity=0.7] (2.2990, 3.0370, 2.0818) -- (2.3460, 3.0370, 2.0768) -- (2.3460, 3.0900, 2.0705) -- (2.2990, 3.0900, 2.0755) -- cycle;
\fill[blue!15.0, opacity=0.7] (2.3460, -0.0900, 2.0705) -- (2.3930, -0.0900, 2.0654) -- (2.3930, -0.0370, 2.0716) -- (2.3460, -0.0370, 2.0768) -- cycle;
\fill[blue!15.0, opacity=0.7] (2.3460, -0.0370, 2.0768) -- (2.3930, -0.0370, 2.0716) -- (2.3930, 0.0160, 2.0779) -- (2.3460, 0.0160, 2.0831) -- cycle;
\fill[blue!15.0, opacity=0.7] (2.3460, 0.0160, 2.0831) -- (2.3930, 0.0160, 2.0779) -- (2.3930, 0.0690, 2.0841) -- (2.3460, 0.0690, 2.0893) -- cycle;
\fill[blue!15.3, opacity=0.7] (2.3460, 0.0690, 2.0893) -- (2.3930, 0.0690, 2.0841) -- (2.3930, 0.1220, 2.0903) -- (2.3460, 0.1220, 2.0955) -- cycle;
\fill[blue!19.3, opacity=0.7] (2.3460, 0.1220, 2.0955) -- (2.3930, 0.1220, 2.0903) -- (2.3930, 0.1750, 2.0964) -- (2.3460, 0.1750, 2.1016) -- cycle;
\fill[blue!37.6, opacity=0.7] (2.3460, 0.1750, 2.1016) -- (2.3930, 0.1750, 2.0964) -- (2.3930, 0.2280, 2.1024) -- (2.3460, 0.2280, 2.1076) -- cycle;
\fill[blue!58.0, opacity=0.7] (2.3460, 0.2280, 2.1076) -- (2.3930, 0.2280, 2.1024) -- (2.3930, 0.2810, 2.1084) -- (2.3460, 0.2810, 2.1135) -- cycle;
\fill[blue!60.8, opacity=0.7] (2.3460, 0.2810, 2.1135) -- (2.3930, 0.2810, 2.1084) -- (2.3930, 0.3340, 2.1142) -- (2.3460, 0.3340, 2.1193) -- cycle;
\fill[blue!46.4, opacity=0.7] (2.3460, 0.3340, 2.1193) -- (2.3930, 0.3340, 2.1142) -- (2.3930, 0.3870, 2.1198) -- (2.3460, 0.3870, 2.1250) -- cycle;
\fill[blue!27.4, opacity=0.7] (2.3460, 0.3870, 2.1250) -- (2.3930, 0.3870, 2.1198) -- (2.3930, 0.4400, 2.1254) -- (2.3460, 0.4400, 2.1305) -- cycle;
\fill[blue!18.0, opacity=0.7] (2.3460, 0.4400, 2.1305) -- (2.3930, 0.4400, 2.1254) -- (2.3930, 0.4930, 2.1307) -- (2.3460, 0.4930, 2.1359) -- cycle;
\fill[blue!15.7, opacity=0.7] (2.3460, 0.4930, 2.1359) -- (2.3930, 0.4930, 2.1307) -- (2.3930, 0.5460, 2.1359) -- (2.3460, 0.5460, 2.1411) -- cycle;
\fill[blue!15.3, opacity=0.7] (2.3460, 0.5460, 2.1411) -- (2.3930, 0.5460, 2.1359) -- (2.3930, 0.5990, 2.1409) -- (2.3460, 0.5990, 2.1461) -- cycle;
\fill[blue!15.4, opacity=0.7] (2.3460, 0.5990, 2.1461) -- (2.3930, 0.5990, 2.1409) -- (2.3930, 0.6520, 2.1457) -- (2.3460, 0.6520, 2.1508) -- cycle;
\fill[blue!16.1, opacity=0.7] (2.3460, 0.6520, 2.1508) -- (2.3930, 0.6520, 2.1457) -- (2.3930, 0.7050, 2.1502) -- (2.3460, 0.7050, 2.1554) -- cycle;
\fill[blue!19.2, opacity=0.7] (2.3460, 0.7050, 2.1554) -- (2.3930, 0.7050, 2.1502) -- (2.3930, 0.7580, 2.1545) -- (2.3460, 0.7580, 2.1597) -- cycle;
\fill[blue!30.4, opacity=0.7] (2.3460, 0.7580, 2.1597) -- (2.3930, 0.7580, 2.1545) -- (2.3930, 0.8110, 2.1586) -- (2.3460, 0.8110, 2.1638) -- cycle;
\fill[blue!53.3, opacity=0.7] (2.3460, 0.8110, 2.1638) -- (2.3930, 0.8110, 2.1586) -- (2.3930, 0.8640, 2.1624) -- (2.3460, 0.8640, 2.1676) -- cycle;
\fill[blue!76.7, opacity=0.7] (2.3460, 0.8640, 2.1676) -- (2.3930, 0.8640, 2.1624) -- (2.3930, 0.9170, 2.1660) -- (2.3460, 0.9170, 2.1712) -- cycle;
\fill[blue!87.2, opacity=0.7] (2.3460, 0.9170, 2.1712) -- (2.3930, 0.9170, 2.1660) -- (2.3930, 0.9700, 2.1693) -- (2.3460, 0.9700, 2.1745) -- cycle;
\fill[blue!86.5, opacity=0.7] (2.3460, 0.9700, 2.1745) -- (2.3930, 0.9700, 2.1693) -- (2.3930, 1.0230, 2.1723) -- (2.3460, 1.0230, 2.1775) -- cycle;
\fill[blue!82.4, opacity=0.7] (2.3460, 1.0230, 2.1775) -- (2.3930, 1.0230, 2.1723) -- (2.3930, 1.0760, 2.1750) -- (2.3460, 1.0760, 2.1802) -- cycle;
\fill[blue!79.0, opacity=0.7] (2.3460, 1.0760, 2.1802) -- (2.3930, 1.0760, 2.1750) -- (2.3930, 1.1290, 2.1774) -- (2.3460, 1.1290, 2.1826) -- cycle;
\fill[blue!77.4, opacity=0.7] (2.3460, 1.1290, 2.1826) -- (2.3930, 1.1290, 2.1774) -- (2.3930, 1.1820, 2.1795) -- (2.3460, 1.1820, 2.1847) -- cycle;
\fill[blue!77.0, opacity=0.7] (2.3460, 1.1820, 2.1847) -- (2.3930, 1.1820, 2.1795) -- (2.3930, 1.2350, 2.1813) -- (2.3460, 1.2350, 2.1864) -- cycle;
\fill[blue!77.1, opacity=0.7] (2.3460, 1.2350, 2.1864) -- (2.3930, 1.2350, 2.1813) -- (2.3930, 1.2880, 2.1827) -- (2.3460, 1.2880, 2.1879) -- cycle;
\fill[blue!77.0, opacity=0.7] (2.3460, 1.2880, 2.1879) -- (2.3930, 1.2880, 2.1827) -- (2.3930, 1.3410, 2.1839) -- (2.3460, 1.3410, 2.1891) -- cycle;
\fill[blue!76.4, opacity=0.7] (2.3460, 1.3410, 2.1891) -- (2.3930, 1.3410, 2.1839) -- (2.3930, 1.3940, 2.1847) -- (2.3460, 1.3940, 2.1899) -- cycle;
\fill[blue!75.3, opacity=0.7] (2.3460, 1.3940, 2.1899) -- (2.3930, 1.3940, 2.1847) -- (2.3930, 1.4470, 2.1852) -- (2.3460, 1.4470, 2.1904) -- cycle;
\fill[blue!73.7, opacity=0.7] (2.3460, 1.4470, 2.1904) -- (2.3930, 1.4470, 2.1852) -- (2.3930, 1.5000, 2.1854) -- (2.3460, 1.5000, 2.1905) -- cycle;
\fill[blue!72.0, opacity=0.7] (2.3460, 1.5000, 2.1905) -- (2.3930, 1.5000, 2.1854) -- (2.3930, 1.5530, 2.1852) -- (2.3460, 1.5530, 2.1904) -- cycle;
\fill[blue!71.2, opacity=0.7] (2.3460, 1.5530, 2.1904) -- (2.3930, 1.5530, 2.1852) -- (2.3930, 1.6060, 2.1847) -- (2.3460, 1.6060, 2.1899) -- cycle;
\fill[blue!72.3, opacity=0.7] (2.3460, 1.6060, 2.1899) -- (2.3930, 1.6060, 2.1847) -- (2.3930, 1.6590, 2.1839) -- (2.3460, 1.6590, 2.1891) -- cycle;
\fill[blue!76.3, opacity=0.7] (2.3460, 1.6590, 2.1891) -- (2.3930, 1.6590, 2.1839) -- (2.3930, 1.7120, 2.1827) -- (2.3460, 1.7120, 2.1879) -- cycle;
\fill[blue!83.1, opacity=0.7] (2.3460, 1.7120, 2.1879) -- (2.3930, 1.7120, 2.1827) -- (2.3930, 1.7650, 2.1813) -- (2.3460, 1.7650, 2.1864) -- cycle;
\fill[blue!87.8, opacity=0.7] (2.3460, 1.7650, 2.1864) -- (2.3930, 1.7650, 2.1813) -- (2.3930, 1.8180, 2.1795) -- (2.3460, 1.8180, 2.1847) -- cycle;
\fill[blue!80.4, opacity=0.7] (2.3460, 1.8180, 2.1847) -- (2.3930, 1.8180, 2.1795) -- (2.3930, 1.8710, 2.1774) -- (2.3460, 1.8710, 2.1826) -- cycle;
\fill[blue!56.7, opacity=0.7] (2.3460, 1.8710, 2.1826) -- (2.3930, 1.8710, 2.1774) -- (2.3930, 1.9240, 2.1750) -- (2.3460, 1.9240, 2.1802) -- cycle;
\fill[blue!32.6, opacity=0.7] (2.3460, 1.9240, 2.1802) -- (2.3930, 1.9240, 2.1750) -- (2.3930, 1.9770, 2.1723) -- (2.3460, 1.9770, 2.1775) -- cycle;
\fill[blue!20.7, opacity=0.7] (2.3460, 1.9770, 2.1775) -- (2.3930, 1.9770, 2.1723) -- (2.3930, 2.0300, 2.1693) -- (2.3460, 2.0300, 2.1745) -- cycle;
\fill[blue!17.2, opacity=0.7] (2.3460, 2.0300, 2.1745) -- (2.3930, 2.0300, 2.1693) -- (2.3930, 2.0830, 2.1660) -- (2.3460, 2.0830, 2.1712) -- cycle;
\fill[blue!16.6, opacity=0.7] (2.3460, 2.0830, 2.1712) -- (2.3930, 2.0830, 2.1660) -- (2.3930, 2.1360, 2.1624) -- (2.3460, 2.1360, 2.1676) -- cycle;
\fill[blue!17.7, opacity=0.7] (2.3460, 2.1360, 2.1676) -- (2.3930, 2.1360, 2.1624) -- (2.3930, 2.1890, 2.1586) -- (2.3460, 2.1890, 2.1638) -- cycle;
\fill[blue!23.8, opacity=0.7] (2.3460, 2.1890, 2.1638) -- (2.3930, 2.1890, 2.1586) -- (2.3930, 2.2420, 2.1545) -- (2.3460, 2.2420, 2.1597) -- cycle;
\fill[blue!43.3, opacity=0.7] (2.3460, 2.2420, 2.1597) -- (2.3930, 2.2420, 2.1545) -- (2.3930, 2.2950, 2.1502) -- (2.3460, 2.2950, 2.1554) -- cycle;
\fill[blue!71.1, opacity=0.7] (2.3460, 2.2950, 2.1554) -- (2.3930, 2.2950, 2.1502) -- (2.3930, 2.3480, 2.1457) -- (2.3460, 2.3480, 2.1508) -- cycle;
\fill[blue!84.1, opacity=0.7] (2.3460, 2.3480, 2.1508) -- (2.3930, 2.3480, 2.1457) -- (2.3930, 2.4010, 2.1409) -- (2.3460, 2.4010, 2.1461) -- cycle;
\fill[blue!83.5, opacity=0.7] (2.3460, 2.4010, 2.1461) -- (2.3930, 2.4010, 2.1409) -- (2.3930, 2.4540, 2.1359) -- (2.3460, 2.4540, 2.1411) -- cycle;
\fill[blue!65.4, opacity=0.7] (2.3460, 2.4540, 2.1411) -- (2.3930, 2.4540, 2.1359) -- (2.3930, 2.5070, 2.1307) -- (2.3460, 2.5070, 2.1359) -- cycle;
\fill[blue!31.1, opacity=0.7] (2.3460, 2.5070, 2.1359) -- (2.3930, 2.5070, 2.1307) -- (2.3930, 2.5600, 2.1254) -- (2.3460, 2.5600, 2.1305) -- cycle;
\fill[blue!16.7, opacity=0.7] (2.3460, 2.5600, 2.1305) -- (2.3930, 2.5600, 2.1254) -- (2.3930, 2.6130, 2.1198) -- (2.3460, 2.6130, 2.1250) -- cycle;
\fill[blue!15.1, opacity=0.7] (2.3460, 2.6130, 2.1250) -- (2.3930, 2.6130, 2.1198) -- (2.3930, 2.6660, 2.1142) -- (2.3460, 2.6660, 2.1193) -- cycle;
\fill[blue!15.0, opacity=0.7] (2.3460, 2.6660, 2.1193) -- (2.3930, 2.6660, 2.1142) -- (2.3930, 2.7190, 2.1084) -- (2.3460, 2.7190, 2.1135) -- cycle;
\fill[blue!15.1, opacity=0.7] (2.3460, 2.7190, 2.1135) -- (2.3930, 2.7190, 2.1084) -- (2.3930, 2.7720, 2.1024) -- (2.3460, 2.7720, 2.1076) -- cycle;
\fill[blue!15.8, opacity=0.7] (2.3460, 2.7720, 2.1076) -- (2.3930, 2.7720, 2.1024) -- (2.3930, 2.8250, 2.0964) -- (2.3460, 2.8250, 2.1016) -- cycle;
\fill[blue!21.6, opacity=0.7] (2.3460, 2.8250, 2.1016) -- (2.3930, 2.8250, 2.0964) -- (2.3930, 2.8780, 2.0903) -- (2.3460, 2.8780, 2.0955) -- cycle;
\fill[blue!36.0, opacity=0.7] (2.3460, 2.8780, 2.0955) -- (2.3930, 2.8780, 2.0903) -- (2.3930, 2.9310, 2.0841) -- (2.3460, 2.9310, 2.0893) -- cycle;
\fill[blue!38.8, opacity=0.7] (2.3460, 2.9310, 2.0893) -- (2.3930, 2.9310, 2.0841) -- (2.3930, 2.9840, 2.0779) -- (2.3460, 2.9840, 2.0831) -- cycle;
\fill[blue!24.1, opacity=0.7] (2.3460, 2.9840, 2.0831) -- (2.3930, 2.9840, 2.0779) -- (2.3930, 3.0370, 2.0716) -- (2.3460, 3.0370, 2.0768) -- cycle;
\fill[blue!15.7, opacity=0.7] (2.3460, 3.0370, 2.0768) -- (2.3930, 3.0370, 2.0716) -- (2.3930, 3.0900, 2.0654) -- (2.3460, 3.0900, 2.0705) -- cycle;
\fill[blue!15.0, opacity=0.7] (2.3930, -0.0900, 2.0654) -- (2.4400, -0.0900, 2.0600) -- (2.4400, -0.0370, 2.0663) -- (2.3930, -0.0370, 2.0716) -- cycle;
\fill[blue!15.0, opacity=0.7] (2.3930, -0.0370, 2.0716) -- (2.4400, -0.0370, 2.0663) -- (2.4400, 0.0160, 2.0725) -- (2.3930, 0.0160, 2.0779) -- cycle;
\fill[blue!15.0, opacity=0.7] (2.3930, 0.0160, 2.0779) -- (2.4400, 0.0160, 2.0725) -- (2.4400, 0.0690, 2.0788) -- (2.3930, 0.0690, 2.0841) -- cycle;
\fill[blue!15.0, opacity=0.7] (2.3930, 0.0690, 2.0841) -- (2.4400, 0.0690, 2.0788) -- (2.4400, 0.1220, 2.0849) -- (2.3930, 0.1220, 2.0903) -- cycle;
\fill[blue!15.4, opacity=0.7] (2.3930, 0.1220, 2.0903) -- (2.4400, 0.1220, 2.0849) -- (2.4400, 0.1750, 2.0911) -- (2.3930, 0.1750, 2.0964) -- cycle;
\fill[blue!20.5, opacity=0.7] (2.3930, 0.1750, 2.0964) -- (2.4400, 0.1750, 2.0911) -- (2.4400, 0.2280, 2.0971) -- (2.3930, 0.2280, 2.1024) -- cycle;
\fill[blue!39.6, opacity=0.7] (2.3930, 0.2280, 2.1024) -- (2.4400, 0.2280, 2.0971) -- (2.4400, 0.2810, 2.1030) -- (2.3930, 0.2810, 2.1084) -- cycle;
\fill[blue!59.0, opacity=0.7] (2.3930, 0.2810, 2.1084) -- (2.4400, 0.2810, 2.1030) -- (2.4400, 0.3340, 2.1088) -- (2.3930, 0.3340, 2.1142) -- cycle;
\fill[blue!61.9, opacity=0.7] (2.3930, 0.3340, 2.1142) -- (2.4400, 0.3340, 2.1088) -- (2.4400, 0.3870, 2.1145) -- (2.3930, 0.3870, 2.1198) -- cycle;
\fill[blue!49.0, opacity=0.7] (2.3930, 0.3870, 2.1198) -- (2.4400, 0.3870, 2.1145) -- (2.4400, 0.4400, 2.1200) -- (2.3930, 0.4400, 2.1254) -- cycle;
\fill[blue!30.3, opacity=0.7] (2.3930, 0.4400, 2.1254) -- (2.4400, 0.4400, 2.1200) -- (2.4400, 0.4930, 2.1254) -- (2.3930, 0.4930, 2.1307) -- cycle;
\fill[blue!19.4, opacity=0.7] (2.3930, 0.4930, 2.1307) -- (2.4400, 0.4930, 2.1254) -- (2.4400, 0.5460, 2.1305) -- (2.3930, 0.5460, 2.1359) -- cycle;
\fill[blue!16.1, opacity=0.7] (2.3930, 0.5460, 2.1359) -- (2.4400, 0.5460, 2.1305) -- (2.4400, 0.5990, 2.1355) -- (2.3930, 0.5990, 2.1409) -- cycle;
\fill[blue!15.4, opacity=0.7] (2.3930, 0.5990, 2.1409) -- (2.4400, 0.5990, 2.1355) -- (2.4400, 0.6520, 2.1403) -- (2.3930, 0.6520, 2.1457) -- cycle;
\fill[blue!15.4, opacity=0.7] (2.3930, 0.6520, 2.1457) -- (2.4400, 0.6520, 2.1403) -- (2.4400, 0.7050, 2.1449) -- (2.3930, 0.7050, 2.1502) -- cycle;
\fill[blue!15.6, opacity=0.7] (2.3930, 0.7050, 2.1502) -- (2.4400, 0.7050, 2.1449) -- (2.4400, 0.7580, 2.1492) -- (2.3930, 0.7580, 2.1545) -- cycle;
\fill[blue!16.8, opacity=0.7] (2.3930, 0.7580, 2.1545) -- (2.4400, 0.7580, 2.1492) -- (2.4400, 0.8110, 2.1533) -- (2.3930, 0.8110, 2.1586) -- cycle;
\fill[blue!20.8, opacity=0.7] (2.3930, 0.8110, 2.1586) -- (2.4400, 0.8110, 2.1533) -- (2.4400, 0.8640, 2.1571) -- (2.3930, 0.8640, 2.1624) -- cycle;
\fill[blue!31.6, opacity=0.7] (2.3930, 0.8640, 2.1624) -- (2.4400, 0.8640, 2.1571) -- (2.4400, 0.9170, 2.1606) -- (2.3930, 0.9170, 2.1660) -- cycle;
\fill[blue!50.2, opacity=0.7] (2.3930, 0.9170, 2.1660) -- (2.4400, 0.9170, 2.1606) -- (2.4400, 0.9700, 2.1639) -- (2.3930, 0.9700, 2.1693) -- cycle;
\fill[blue!69.9, opacity=0.7] (2.3930, 0.9700, 2.1693) -- (2.4400, 0.9700, 2.1639) -- (2.4400, 1.0230, 2.1669) -- (2.3930, 1.0230, 2.1723) -- cycle;
\fill[blue!82.8, opacity=0.7] (2.3930, 1.0230, 2.1723) -- (2.4400, 1.0230, 2.1669) -- (2.4400, 1.0760, 2.1696) -- (2.3930, 1.0760, 2.1750) -- cycle;
\fill[blue!87.6, opacity=0.7] (2.3930, 1.0760, 2.1750) -- (2.4400, 1.0760, 2.1696) -- (2.4400, 1.1290, 2.1720) -- (2.3930, 1.1290, 2.1774) -- cycle;
\fill[blue!87.3, opacity=0.7] (2.3930, 1.1290, 2.1774) -- (2.4400, 1.1290, 2.1720) -- (2.4400, 1.1820, 2.1741) -- (2.3930, 1.1820, 2.1795) -- cycle;
\fill[blue!85.2, opacity=0.7] (2.3930, 1.1820, 2.1795) -- (2.4400, 1.1820, 2.1741) -- (2.4400, 1.2350, 2.1759) -- (2.3930, 1.2350, 2.1813) -- cycle;
\fill[blue!83.0, opacity=0.7] (2.3930, 1.2350, 2.1813) -- (2.4400, 1.2350, 2.1759) -- (2.4400, 1.2880, 2.1774) -- (2.3930, 1.2880, 2.1827) -- cycle;
\fill[blue!81.3, opacity=0.7] (2.3930, 1.2880, 2.1827) -- (2.4400, 1.2880, 2.1774) -- (2.4400, 1.3410, 2.1785) -- (2.3930, 1.3410, 2.1839) -- cycle;
\fill[blue!80.4, opacity=0.7] (2.3930, 1.3410, 2.1839) -- (2.4400, 1.3410, 2.1785) -- (2.4400, 1.3940, 2.1793) -- (2.3930, 1.3940, 2.1847) -- cycle;
\fill[blue!80.4, opacity=0.7] (2.3930, 1.3940, 2.1847) -- (2.4400, 1.3940, 2.1793) -- (2.4400, 1.4470, 2.1798) -- (2.3930, 1.4470, 2.1852) -- cycle;
\fill[blue!81.3, opacity=0.7] (2.3930, 1.4470, 2.1852) -- (2.4400, 1.4470, 2.1798) -- (2.4400, 1.5000, 2.1800) -- (2.3930, 1.5000, 2.1854) -- cycle;
\fill[blue!83.3, opacity=0.7] (2.3930, 1.5000, 2.1854) -- (2.4400, 1.5000, 2.1800) -- (2.4400, 1.5530, 2.1798) -- (2.3930, 1.5530, 2.1852) -- cycle;
\fill[blue!85.9, opacity=0.7] (2.3930, 1.5530, 2.1852) -- (2.4400, 1.5530, 2.1798) -- (2.4400, 1.6060, 2.1793) -- (2.3930, 1.6060, 2.1847) -- cycle;
\fill[blue!87.8, opacity=0.7] (2.3930, 1.6060, 2.1847) -- (2.4400, 1.6060, 2.1793) -- (2.4400, 1.6590, 2.1785) -- (2.3930, 1.6590, 2.1839) -- cycle;
\fill[blue!85.4, opacity=0.7] (2.3930, 1.6590, 2.1839) -- (2.4400, 1.6590, 2.1785) -- (2.4400, 1.7120, 2.1774) -- (2.3930, 1.7120, 2.1827) -- cycle;
\fill[blue!74.2, opacity=0.7] (2.3930, 1.7120, 2.1827) -- (2.4400, 1.7120, 2.1774) -- (2.4400, 1.7650, 2.1759) -- (2.3930, 1.7650, 2.1813) -- cycle;
\fill[blue!54.5, opacity=0.7] (2.3930, 1.7650, 2.1813) -- (2.4400, 1.7650, 2.1759) -- (2.4400, 1.8180, 2.1741) -- (2.3930, 1.8180, 2.1795) -- cycle;
\fill[blue!34.7, opacity=0.7] (2.3930, 1.8180, 2.1795) -- (2.4400, 1.8180, 2.1741) -- (2.4400, 1.8710, 2.1720) -- (2.3930, 1.8710, 2.1774) -- cycle;
\fill[blue!22.8, opacity=0.7] (2.3930, 1.8710, 2.1774) -- (2.4400, 1.8710, 2.1720) -- (2.4400, 1.9240, 2.1696) -- (2.3930, 1.9240, 2.1750) -- cycle;
\fill[blue!18.0, opacity=0.7] (2.3930, 1.9240, 2.1750) -- (2.4400, 1.9240, 2.1696) -- (2.4400, 1.9770, 2.1669) -- (2.3930, 1.9770, 2.1723) -- cycle;
\fill[blue!16.6, opacity=0.7] (2.3930, 1.9770, 2.1723) -- (2.4400, 1.9770, 2.1669) -- (2.4400, 2.0300, 2.1639) -- (2.3930, 2.0300, 2.1693) -- cycle;
\fill[blue!16.6, opacity=0.7] (2.3930, 2.0300, 2.1693) -- (2.4400, 2.0300, 2.1639) -- (2.4400, 2.0830, 2.1606) -- (2.3930, 2.0830, 2.1660) -- cycle;
\fill[blue!18.2, opacity=0.7] (2.3930, 2.0830, 2.1660) -- (2.4400, 2.0830, 2.1606) -- (2.4400, 2.1360, 2.1571) -- (2.3930, 2.1360, 2.1624) -- cycle;
\fill[blue!25.0, opacity=0.7] (2.3930, 2.1360, 2.1624) -- (2.4400, 2.1360, 2.1571) -- (2.4400, 2.1890, 2.1533) -- (2.3930, 2.1890, 2.1586) -- cycle;
\fill[blue!44.4, opacity=0.7] (2.3930, 2.1890, 2.1586) -- (2.4400, 2.1890, 2.1533) -- (2.4400, 2.2420, 2.1492) -- (2.3930, 2.2420, 2.1545) -- cycle;
\fill[blue!70.4, opacity=0.7] (2.3930, 2.2420, 2.1545) -- (2.4400, 2.2420, 2.1492) -- (2.4400, 2.2950, 2.1449) -- (2.3930, 2.2950, 2.1502) -- cycle;
\fill[blue!83.4, opacity=0.7] (2.3930, 2.2950, 2.1502) -- (2.4400, 2.2950, 2.1449) -- (2.4400, 2.3480, 2.1403) -- (2.3930, 2.3480, 2.1457) -- cycle;
\fill[blue!83.5, opacity=0.7] (2.3930, 2.3480, 2.1457) -- (2.4400, 2.3480, 2.1403) -- (2.4400, 2.4010, 2.1355) -- (2.3930, 2.4010, 2.1409) -- cycle;
\fill[blue!68.6, opacity=0.7] (2.3930, 2.4010, 2.1409) -- (2.4400, 2.4010, 2.1355) -- (2.4400, 2.4540, 2.1305) -- (2.3930, 2.4540, 2.1359) -- cycle;
\fill[blue!35.7, opacity=0.7] (2.3930, 2.4540, 2.1359) -- (2.4400, 2.4540, 2.1305) -- (2.4400, 2.5070, 2.1254) -- (2.3930, 2.5070, 2.1307) -- cycle;
\fill[blue!17.8, opacity=0.7] (2.3930, 2.5070, 2.1307) -- (2.4400, 2.5070, 2.1254) -- (2.4400, 2.5600, 2.1200) -- (2.3930, 2.5600, 2.1254) -- cycle;
\fill[blue!15.2, opacity=0.7] (2.3930, 2.5600, 2.1254) -- (2.4400, 2.5600, 2.1200) -- (2.4400, 2.6130, 2.1145) -- (2.3930, 2.6130, 2.1198) -- cycle;
\fill[blue!15.0, opacity=0.7] (2.3930, 2.6130, 2.1198) -- (2.4400, 2.6130, 2.1145) -- (2.4400, 2.6660, 2.1088) -- (2.3930, 2.6660, 2.1142) -- cycle;
\fill[blue!15.0, opacity=0.7] (2.3930, 2.6660, 2.1142) -- (2.4400, 2.6660, 2.1088) -- (2.4400, 2.7190, 2.1030) -- (2.3930, 2.7190, 2.1084) -- cycle;
\fill[blue!15.3, opacity=0.7] (2.3930, 2.7190, 2.1084) -- (2.4400, 2.7190, 2.1030) -- (2.4400, 2.7720, 2.0971) -- (2.3930, 2.7720, 2.1024) -- cycle;
\fill[blue!18.2, opacity=0.7] (2.3930, 2.7720, 2.1024) -- (2.4400, 2.7720, 2.0971) -- (2.4400, 2.8250, 2.0911) -- (2.3930, 2.8250, 2.0964) -- cycle;
\fill[blue!30.0, opacity=0.7] (2.3930, 2.8250, 2.0964) -- (2.4400, 2.8250, 2.0911) -- (2.4400, 2.8780, 2.0849) -- (2.3930, 2.8780, 2.0903) -- cycle;
\fill[blue!39.8, opacity=0.7] (2.3930, 2.8780, 2.0903) -- (2.4400, 2.8780, 2.0849) -- (2.4400, 2.9310, 2.0788) -- (2.3930, 2.9310, 2.0841) -- cycle;
\fill[blue!30.8, opacity=0.7] (2.3930, 2.9310, 2.0841) -- (2.4400, 2.9310, 2.0788) -- (2.4400, 2.9840, 2.0725) -- (2.3930, 2.9840, 2.0779) -- cycle;
\fill[blue!17.6, opacity=0.7] (2.3930, 2.9840, 2.0779) -- (2.4400, 2.9840, 2.0725) -- (2.4400, 3.0370, 2.0663) -- (2.3930, 3.0370, 2.0716) -- cycle;
\fill[blue!15.1, opacity=0.7] (2.3930, 3.0370, 2.0716) -- (2.4400, 3.0370, 2.0663) -- (2.4400, 3.0900, 2.0600) -- (2.3930, 3.0900, 2.0654) -- cycle;
\fill[blue!15.1, opacity=0.7] (2.4400, -0.0900, 2.0600) -- (2.4870, -0.0900, 2.0545) -- (2.4870, -0.0370, 2.0608) -- (2.4400, -0.0370, 2.0663) -- cycle;
\fill[blue!15.0, opacity=0.7] (2.4400, -0.0370, 2.0663) -- (2.4870, -0.0370, 2.0608) -- (2.4870, 0.0160, 2.0670) -- (2.4400, 0.0160, 2.0725) -- cycle;
\fill[blue!15.0, opacity=0.7] (2.4400, 0.0160, 2.0725) -- (2.4870, 0.0160, 2.0670) -- (2.4870, 0.0690, 2.0733) -- (2.4400, 0.0690, 2.0788) -- cycle;
\fill[blue!15.0, opacity=0.7] (2.4400, 0.0690, 2.0788) -- (2.4870, 0.0690, 2.0733) -- (2.4870, 0.1220, 2.0794) -- (2.4400, 0.1220, 2.0849) -- cycle;
\fill[blue!15.0, opacity=0.7] (2.4400, 0.1220, 2.0849) -- (2.4870, 0.1220, 2.0794) -- (2.4870, 0.1750, 2.0855) -- (2.4400, 0.1750, 2.0911) -- cycle;
\fill[blue!15.5, opacity=0.7] (2.4400, 0.1750, 2.0911) -- (2.4870, 0.1750, 2.0855) -- (2.4870, 0.2280, 2.0916) -- (2.4400, 0.2280, 2.0971) -- cycle;
\fill[blue!20.9, opacity=0.7] (2.4400, 0.2280, 2.0971) -- (2.4870, 0.2280, 2.0916) -- (2.4870, 0.2810, 2.0975) -- (2.4400, 0.2810, 2.1030) -- cycle;
\fill[blue!39.5, opacity=0.7] (2.4400, 0.2810, 2.1030) -- (2.4870, 0.2810, 2.0975) -- (2.4870, 0.3340, 2.1033) -- (2.4400, 0.3340, 2.1088) -- cycle;
\fill[blue!58.8, opacity=0.7] (2.4400, 0.3340, 2.1088) -- (2.4870, 0.3340, 2.1033) -- (2.4870, 0.3870, 2.1090) -- (2.4400, 0.3870, 2.1145) -- cycle;
\fill[blue!63.4, opacity=0.7] (2.4400, 0.3870, 2.1145) -- (2.4870, 0.3870, 2.1090) -- (2.4870, 0.4400, 2.1145) -- (2.4400, 0.4400, 2.1200) -- cycle;
\fill[blue!53.5, opacity=0.7] (2.4400, 0.4400, 2.1200) -- (2.4870, 0.4400, 2.1145) -- (2.4870, 0.4930, 2.1198) -- (2.4400, 0.4930, 2.1254) -- cycle;
\fill[blue!35.8, opacity=0.7] (2.4400, 0.4930, 2.1254) -- (2.4870, 0.4930, 2.1198) -- (2.4870, 0.5460, 2.1250) -- (2.4400, 0.5460, 2.1305) -- cycle;
\fill[blue!22.4, opacity=0.7] (2.4400, 0.5460, 2.1305) -- (2.4870, 0.5460, 2.1250) -- (2.4870, 0.5990, 2.1300) -- (2.4400, 0.5990, 2.1355) -- cycle;
\fill[blue!17.2, opacity=0.7] (2.4400, 0.5990, 2.1355) -- (2.4870, 0.5990, 2.1300) -- (2.4870, 0.6520, 2.1348) -- (2.4400, 0.6520, 2.1403) -- cycle;
\fill[blue!15.7, opacity=0.7] (2.4400, 0.6520, 2.1403) -- (2.4870, 0.6520, 2.1348) -- (2.4870, 0.7050, 2.1393) -- (2.4400, 0.7050, 2.1449) -- cycle;
\fill[blue!15.4, opacity=0.7] (2.4400, 0.7050, 2.1449) -- (2.4870, 0.7050, 2.1393) -- (2.4870, 0.7580, 2.1437) -- (2.4400, 0.7580, 2.1492) -- cycle;
\fill[blue!15.4, opacity=0.7] (2.4400, 0.7580, 2.1492) -- (2.4870, 0.7580, 2.1437) -- (2.4870, 0.8110, 2.1477) -- (2.4400, 0.8110, 2.1533) -- cycle;
\fill[blue!15.7, opacity=0.7] (2.4400, 0.8110, 2.1533) -- (2.4870, 0.8110, 2.1477) -- (2.4870, 0.8640, 2.1516) -- (2.4400, 0.8640, 2.1571) -- cycle;
\fill[blue!16.8, opacity=0.7] (2.4400, 0.8640, 2.1571) -- (2.4870, 0.8640, 2.1516) -- (2.4870, 0.9170, 2.1551) -- (2.4400, 0.9170, 2.1606) -- cycle;
\fill[blue!19.6, opacity=0.7] (2.4400, 0.9170, 2.1606) -- (2.4870, 0.9170, 2.1551) -- (2.4870, 0.9700, 2.1584) -- (2.4400, 0.9700, 2.1639) -- cycle;
\fill[blue!25.9, opacity=0.7] (2.4400, 0.9700, 2.1639) -- (2.4870, 0.9700, 2.1584) -- (2.4870, 1.0230, 2.1614) -- (2.4400, 1.0230, 2.1669) -- cycle;
\fill[blue!36.3, opacity=0.7] (2.4400, 1.0230, 2.1669) -- (2.4870, 1.0230, 2.1614) -- (2.4870, 1.0760, 2.1641) -- (2.4400, 1.0760, 2.1696) -- cycle;
\fill[blue!49.2, opacity=0.7] (2.4400, 1.0760, 2.1696) -- (2.4870, 1.0760, 2.1641) -- (2.4870, 1.1290, 2.1665) -- (2.4400, 1.1290, 2.1720) -- cycle;
\fill[blue!61.5, opacity=0.7] (2.4400, 1.1290, 2.1720) -- (2.4870, 1.1290, 2.1665) -- (2.4870, 1.1820, 2.1686) -- (2.4400, 1.1820, 2.1741) -- cycle;
\fill[blue!70.9, opacity=0.7] (2.4400, 1.1820, 2.1741) -- (2.4870, 1.1820, 2.1686) -- (2.4870, 1.2350, 2.1704) -- (2.4400, 1.2350, 2.1759) -- cycle;
\fill[blue!76.9, opacity=0.7] (2.4400, 1.2350, 2.1759) -- (2.4870, 1.2350, 2.1704) -- (2.4870, 1.2880, 2.1719) -- (2.4400, 1.2880, 2.1774) -- cycle;
\fill[blue!80.1, opacity=0.7] (2.4400, 1.2880, 2.1774) -- (2.4870, 1.2880, 2.1719) -- (2.4870, 1.3410, 2.1730) -- (2.4400, 1.3410, 2.1785) -- cycle;
\fill[blue!81.3, opacity=0.7] (2.4400, 1.3410, 2.1785) -- (2.4870, 1.3410, 2.1730) -- (2.4870, 1.3940, 2.1738) -- (2.4400, 1.3940, 2.1793) -- cycle;
\fill[blue!80.8, opacity=0.7] (2.4400, 1.3940, 2.1793) -- (2.4870, 1.3940, 2.1738) -- (2.4870, 1.4470, 2.1743) -- (2.4400, 1.4470, 2.1798) -- cycle;
\fill[blue!78.4, opacity=0.7] (2.4400, 1.4470, 2.1798) -- (2.4870, 1.4470, 2.1743) -- (2.4870, 1.5000, 2.1745) -- (2.4400, 1.5000, 2.1800) -- cycle;
\fill[blue!73.3, opacity=0.7] (2.4400, 1.5000, 2.1800) -- (2.4870, 1.5000, 2.1745) -- (2.4870, 1.5530, 2.1743) -- (2.4400, 1.5530, 2.1798) -- cycle;
\fill[blue!64.7, opacity=0.7] (2.4400, 1.5530, 2.1798) -- (2.4870, 1.5530, 2.1743) -- (2.4870, 1.6060, 2.1738) -- (2.4400, 1.6060, 2.1793) -- cycle;
\fill[blue!52.8, opacity=0.7] (2.4400, 1.6060, 2.1793) -- (2.4870, 1.6060, 2.1738) -- (2.4870, 1.6590, 2.1730) -- (2.4400, 1.6590, 2.1785) -- cycle;
\fill[blue!39.6, opacity=0.7] (2.4400, 1.6590, 2.1785) -- (2.4870, 1.6590, 2.1730) -- (2.4870, 1.7120, 2.1719) -- (2.4400, 1.7120, 2.1774) -- cycle;
\fill[blue!28.4, opacity=0.7] (2.4400, 1.7120, 2.1774) -- (2.4870, 1.7120, 2.1719) -- (2.4870, 1.7650, 2.1704) -- (2.4400, 1.7650, 2.1759) -- cycle;
\fill[blue!21.3, opacity=0.7] (2.4400, 1.7650, 2.1759) -- (2.4870, 1.7650, 2.1704) -- (2.4870, 1.8180, 2.1686) -- (2.4400, 1.8180, 2.1741) -- cycle;
\fill[blue!17.9, opacity=0.7] (2.4400, 1.8180, 2.1741) -- (2.4870, 1.8180, 2.1686) -- (2.4870, 1.8710, 2.1665) -- (2.4400, 1.8710, 2.1720) -- cycle;
\fill[blue!16.6, opacity=0.7] (2.4400, 1.8710, 2.1720) -- (2.4870, 1.8710, 2.1665) -- (2.4870, 1.9240, 2.1641) -- (2.4400, 1.9240, 2.1696) -- cycle;
\fill[blue!16.3, opacity=0.7] (2.4400, 1.9240, 2.1696) -- (2.4870, 1.9240, 2.1641) -- (2.4870, 1.9770, 2.1614) -- (2.4400, 1.9770, 2.1669) -- cycle;
\fill[blue!16.9, opacity=0.7] (2.4400, 1.9770, 2.1669) -- (2.4870, 1.9770, 2.1614) -- (2.4870, 2.0300, 2.1584) -- (2.4400, 2.0300, 2.1639) -- cycle;
\fill[blue!19.5, opacity=0.7] (2.4400, 2.0300, 2.1639) -- (2.4870, 2.0300, 2.1584) -- (2.4870, 2.0830, 2.1551) -- (2.4400, 2.0830, 2.1606) -- cycle;
\fill[blue!28.4, opacity=0.7] (2.4400, 2.0830, 2.1606) -- (2.4870, 2.0830, 2.1551) -- (2.4870, 2.1360, 2.1516) -- (2.4400, 2.1360, 2.1571) -- cycle;
\fill[blue!48.9, opacity=0.7] (2.4400, 2.1360, 2.1571) -- (2.4870, 2.1360, 2.1516) -- (2.4870, 2.1890, 2.1477) -- (2.4400, 2.1890, 2.1533) -- cycle;
\fill[blue!72.2, opacity=0.7] (2.4400, 2.1890, 2.1533) -- (2.4870, 2.1890, 2.1477) -- (2.4870, 2.2420, 2.1437) -- (2.4400, 2.2420, 2.1492) -- cycle;
\fill[blue!83.2, opacity=0.7] (2.4400, 2.2420, 2.1492) -- (2.4870, 2.2420, 2.1437) -- (2.4870, 2.2950, 2.1393) -- (2.4400, 2.2950, 2.1449) -- cycle;
\fill[blue!82.9, opacity=0.7] (2.4400, 2.2950, 2.1449) -- (2.4870, 2.2950, 2.1393) -- (2.4870, 2.3480, 2.1348) -- (2.4400, 2.3480, 2.1403) -- cycle;
\fill[blue!68.7, opacity=0.7] (2.4400, 2.3480, 2.1403) -- (2.4870, 2.3480, 2.1348) -- (2.4870, 2.4010, 2.1300) -- (2.4400, 2.4010, 2.1355) -- cycle;
\fill[blue!37.7, opacity=0.7] (2.4400, 2.4010, 2.1355) -- (2.4870, 2.4010, 2.1300) -- (2.4870, 2.4540, 2.1250) -- (2.4400, 2.4540, 2.1305) -- cycle;
\fill[blue!18.7, opacity=0.7] (2.4400, 2.4540, 2.1305) -- (2.4870, 2.4540, 2.1250) -- (2.4870, 2.5070, 2.1198) -- (2.4400, 2.5070, 2.1254) -- cycle;
\fill[blue!15.3, opacity=0.7] (2.4400, 2.5070, 2.1254) -- (2.4870, 2.5070, 2.1198) -- (2.4870, 2.5600, 2.1145) -- (2.4400, 2.5600, 2.1200) -- cycle;
\fill[blue!15.0, opacity=0.7] (2.4400, 2.5600, 2.1200) -- (2.4870, 2.5600, 2.1145) -- (2.4870, 2.6130, 2.1090) -- (2.4400, 2.6130, 2.1145) -- cycle;
\fill[blue!15.0, opacity=0.7] (2.4400, 2.6130, 2.1145) -- (2.4870, 2.6130, 2.1090) -- (2.4870, 2.6660, 2.1033) -- (2.4400, 2.6660, 2.1088) -- cycle;
\fill[blue!15.2, opacity=0.7] (2.4400, 2.6660, 2.1088) -- (2.4870, 2.6660, 2.1033) -- (2.4870, 2.7190, 2.0975) -- (2.4400, 2.7190, 2.1030) -- cycle;
\fill[blue!16.6, opacity=0.7] (2.4400, 2.7190, 2.1030) -- (2.4870, 2.7190, 2.0975) -- (2.4870, 2.7720, 2.0916) -- (2.4400, 2.7720, 2.0971) -- cycle;
\fill[blue!24.9, opacity=0.7] (2.4400, 2.7720, 2.0971) -- (2.4870, 2.7720, 2.0916) -- (2.4870, 2.8250, 2.0855) -- (2.4400, 2.8250, 2.0911) -- cycle;
\fill[blue!37.6, opacity=0.7] (2.4400, 2.8250, 2.0911) -- (2.4870, 2.8250, 2.0855) -- (2.4870, 2.8780, 2.0794) -- (2.4400, 2.8780, 2.0849) -- cycle;
\fill[blue!35.7, opacity=0.7] (2.4400, 2.8780, 2.0849) -- (2.4870, 2.8780, 2.0794) -- (2.4870, 2.9310, 2.0733) -- (2.4400, 2.9310, 2.0788) -- cycle;
\fill[blue!21.3, opacity=0.7] (2.4400, 2.9310, 2.0788) -- (2.4870, 2.9310, 2.0733) -- (2.4870, 2.9840, 2.0670) -- (2.4400, 2.9840, 2.0725) -- cycle;
\fill[blue!15.4, opacity=0.7] (2.4400, 2.9840, 2.0725) -- (2.4870, 2.9840, 2.0670) -- (2.4870, 3.0370, 2.0608) -- (2.4400, 3.0370, 2.0663) -- cycle;
\fill[blue!15.0, opacity=0.7] (2.4400, 3.0370, 2.0663) -- (2.4870, 3.0370, 2.0608) -- (2.4870, 3.0900, 2.0545) -- (2.4400, 3.0900, 2.0600) -- cycle;
\fill[blue!15.9, opacity=0.7] (2.4870, -0.0900, 2.0545) -- (2.5340, -0.0900, 2.0488) -- (2.5340, -0.0370, 2.0551) -- (2.4870, -0.0370, 2.0608) -- cycle;
\fill[blue!15.1, opacity=0.7] (2.4870, -0.0370, 2.0608) -- (2.5340, -0.0370, 2.0551) -- (2.5340, 0.0160, 2.0614) -- (2.4870, 0.0160, 2.0670) -- cycle;
\fill[blue!15.0, opacity=0.7] (2.4870, 0.0160, 2.0670) -- (2.5340, 0.0160, 2.0614) -- (2.5340, 0.0690, 2.0676) -- (2.4870, 0.0690, 2.0733) -- cycle;
\fill[blue!15.0, opacity=0.7] (2.4870, 0.0690, 2.0733) -- (2.5340, 0.0690, 2.0676) -- (2.5340, 0.1220, 2.0738) -- (2.4870, 0.1220, 2.0794) -- cycle;
\fill[blue!15.0, opacity=0.7] (2.4870, 0.1220, 2.0794) -- (2.5340, 0.1220, 2.0738) -- (2.5340, 0.1750, 2.0799) -- (2.4870, 0.1750, 2.0855) -- cycle;
\fill[blue!15.0, opacity=0.7] (2.4870, 0.1750, 2.0855) -- (2.5340, 0.1750, 2.0799) -- (2.5340, 0.2280, 2.0859) -- (2.4870, 0.2280, 2.0916) -- cycle;
\fill[blue!15.5, opacity=0.7] (2.4870, 0.2280, 2.0916) -- (2.5340, 0.2280, 2.0859) -- (2.5340, 0.2810, 2.0918) -- (2.4870, 0.2810, 2.0975) -- cycle;
\fill[blue!20.2, opacity=0.7] (2.4870, 0.2810, 2.0975) -- (2.5340, 0.2810, 2.0918) -- (2.5340, 0.3340, 2.0976) -- (2.4870, 0.3340, 2.1033) -- cycle;
\fill[blue!37.1, opacity=0.7] (2.4870, 0.3340, 2.1033) -- (2.5340, 0.3340, 2.0976) -- (2.5340, 0.3870, 2.1033) -- (2.4870, 0.3870, 2.1090) -- cycle;
\fill[blue!56.9, opacity=0.7] (2.4870, 0.3870, 2.1090) -- (2.5340, 0.3870, 2.1033) -- (2.5340, 0.4400, 2.1088) -- (2.4870, 0.4400, 2.1145) -- cycle;
\fill[blue!64.7, opacity=0.7] (2.4870, 0.4400, 2.1145) -- (2.5340, 0.4400, 2.1088) -- (2.5340, 0.4930, 2.1142) -- (2.4870, 0.4930, 2.1198) -- cycle;
\fill[blue!59.2, opacity=0.7] (2.4870, 0.4930, 2.1198) -- (2.5340, 0.4930, 2.1142) -- (2.5340, 0.5460, 2.1193) -- (2.4870, 0.5460, 2.1250) -- cycle;
\fill[blue!44.1, opacity=0.7] (2.4870, 0.5460, 2.1250) -- (2.5340, 0.5460, 2.1193) -- (2.5340, 0.5990, 2.1243) -- (2.4870, 0.5990, 2.1300) -- cycle;
\fill[blue!28.6, opacity=0.7] (2.4870, 0.5990, 2.1300) -- (2.5340, 0.5990, 2.1243) -- (2.5340, 0.6520, 2.1291) -- (2.4870, 0.6520, 2.1348) -- cycle;
\fill[blue!19.9, opacity=0.7] (2.4870, 0.6520, 2.1348) -- (2.5340, 0.6520, 2.1291) -- (2.5340, 0.7050, 2.1337) -- (2.4870, 0.7050, 2.1393) -- cycle;
\fill[blue!16.7, opacity=0.7] (2.4870, 0.7050, 2.1393) -- (2.5340, 0.7050, 2.1337) -- (2.5340, 0.7580, 2.1380) -- (2.4870, 0.7580, 2.1437) -- cycle;
\fill[blue!15.7, opacity=0.7] (2.4870, 0.7580, 2.1437) -- (2.5340, 0.7580, 2.1380) -- (2.5340, 0.8110, 2.1421) -- (2.4870, 0.8110, 2.1477) -- cycle;
\fill[blue!15.5, opacity=0.7] (2.4870, 0.8110, 2.1477) -- (2.5340, 0.8110, 2.1421) -- (2.5340, 0.8640, 2.1459) -- (2.4870, 0.8640, 2.1516) -- cycle;
\fill[blue!15.5, opacity=0.7] (2.4870, 0.8640, 2.1516) -- (2.5340, 0.8640, 2.1459) -- (2.5340, 0.9170, 2.1494) -- (2.4870, 0.9170, 2.1551) -- cycle;
\fill[blue!15.6, opacity=0.7] (2.4870, 0.9170, 2.1551) -- (2.5340, 0.9170, 2.1494) -- (2.5340, 0.9700, 2.1527) -- (2.4870, 0.9700, 2.1584) -- cycle;
\fill[blue!16.1, opacity=0.7] (2.4870, 0.9700, 2.1584) -- (2.5340, 0.9700, 2.1527) -- (2.5340, 1.0230, 2.1557) -- (2.4870, 1.0230, 2.1614) -- cycle;
\fill[blue!17.2, opacity=0.7] (2.4870, 1.0230, 2.1614) -- (2.5340, 1.0230, 2.1557) -- (2.5340, 1.0760, 2.1584) -- (2.4870, 1.0760, 2.1641) -- cycle;
\fill[blue!19.0, opacity=0.7] (2.4870, 1.0760, 2.1641) -- (2.5340, 1.0760, 2.1584) -- (2.5340, 1.1290, 2.1608) -- (2.4870, 1.1290, 2.1665) -- cycle;
\fill[blue!21.8, opacity=0.7] (2.4870, 1.1290, 2.1665) -- (2.5340, 1.1290, 2.1608) -- (2.5340, 1.1820, 2.1629) -- (2.4870, 1.1820, 2.1686) -- cycle;
\fill[blue!25.2, opacity=0.7] (2.4870, 1.1820, 2.1686) -- (2.5340, 1.1820, 2.1629) -- (2.5340, 1.2350, 2.1647) -- (2.4870, 1.2350, 2.1704) -- cycle;
\fill[blue!28.6, opacity=0.7] (2.4870, 1.2350, 2.1704) -- (2.5340, 1.2350, 2.1647) -- (2.5340, 1.2880, 2.1662) -- (2.4870, 1.2880, 2.1719) -- cycle;
\fill[blue!31.1, opacity=0.7] (2.4870, 1.2880, 2.1719) -- (2.5340, 1.2880, 2.1662) -- (2.5340, 1.3410, 2.1673) -- (2.4870, 1.3410, 2.1730) -- cycle;
\fill[blue!32.2, opacity=0.7] (2.4870, 1.3410, 2.1730) -- (2.5340, 1.3410, 2.1673) -- (2.5340, 1.3940, 2.1682) -- (2.4870, 1.3940, 2.1738) -- cycle;
\fill[blue!31.7, opacity=0.7] (2.4870, 1.3940, 2.1738) -- (2.5340, 1.3940, 2.1682) -- (2.5340, 1.4470, 2.1686) -- (2.4870, 1.4470, 2.1743) -- cycle;
\fill[blue!29.6, opacity=0.7] (2.4870, 1.4470, 2.1743) -- (2.5340, 1.4470, 2.1686) -- (2.5340, 1.5000, 2.1688) -- (2.4870, 1.5000, 2.1745) -- cycle;
\fill[blue!26.5, opacity=0.7] (2.4870, 1.5000, 2.1745) -- (2.5340, 1.5000, 2.1688) -- (2.5340, 1.5530, 2.1686) -- (2.4870, 1.5530, 2.1743) -- cycle;
\fill[blue!23.1, opacity=0.7] (2.4870, 1.5530, 2.1743) -- (2.5340, 1.5530, 2.1686) -- (2.5340, 1.6060, 2.1682) -- (2.4870, 1.6060, 2.1738) -- cycle;
\fill[blue!20.2, opacity=0.7] (2.4870, 1.6060, 2.1738) -- (2.5340, 1.6060, 2.1682) -- (2.5340, 1.6590, 2.1673) -- (2.4870, 1.6590, 2.1730) -- cycle;
\fill[blue!18.1, opacity=0.7] (2.4870, 1.6590, 2.1730) -- (2.5340, 1.6590, 2.1673) -- (2.5340, 1.7120, 2.1662) -- (2.4870, 1.7120, 2.1719) -- cycle;
\fill[blue!16.9, opacity=0.7] (2.4870, 1.7120, 2.1719) -- (2.5340, 1.7120, 2.1662) -- (2.5340, 1.7650, 2.1647) -- (2.4870, 1.7650, 2.1704) -- cycle;
\fill[blue!16.3, opacity=0.7] (2.4870, 1.7650, 2.1704) -- (2.5340, 1.7650, 2.1647) -- (2.5340, 1.8180, 2.1629) -- (2.4870, 1.8180, 2.1686) -- cycle;
\fill[blue!16.2, opacity=0.7] (2.4870, 1.8180, 2.1686) -- (2.5340, 1.8180, 2.1629) -- (2.5340, 1.8710, 2.1608) -- (2.4870, 1.8710, 2.1665) -- cycle;
\fill[blue!16.5, opacity=0.7] (2.4870, 1.8710, 2.1665) -- (2.5340, 1.8710, 2.1608) -- (2.5340, 1.9240, 2.1584) -- (2.4870, 1.9240, 2.1641) -- cycle;
\fill[blue!18.0, opacity=0.7] (2.4870, 1.9240, 2.1641) -- (2.5340, 1.9240, 2.1584) -- (2.5340, 1.9770, 2.1557) -- (2.4870, 1.9770, 2.1614) -- cycle;
\fill[blue!22.8, opacity=0.7] (2.4870, 1.9770, 2.1614) -- (2.5340, 1.9770, 2.1557) -- (2.5340, 2.0300, 2.1527) -- (2.4870, 2.0300, 2.1584) -- cycle;
\fill[blue!35.3, opacity=0.7] (2.4870, 2.0300, 2.1584) -- (2.5340, 2.0300, 2.1527) -- (2.5340, 2.0830, 2.1494) -- (2.4870, 2.0830, 2.1551) -- cycle;
\fill[blue!56.7, opacity=0.7] (2.4870, 2.0830, 2.1551) -- (2.5340, 2.0830, 2.1494) -- (2.5340, 2.1360, 2.1459) -- (2.4870, 2.1360, 2.1516) -- cycle;
\fill[blue!75.7, opacity=0.7] (2.4870, 2.1360, 2.1516) -- (2.5340, 2.1360, 2.1459) -- (2.5340, 2.1890, 2.1421) -- (2.4870, 2.1890, 2.1477) -- cycle;
\fill[blue!83.2, opacity=0.7] (2.4870, 2.1890, 2.1477) -- (2.5340, 2.1890, 2.1421) -- (2.5340, 2.2420, 2.1380) -- (2.4870, 2.2420, 2.1437) -- cycle;
\fill[blue!81.5, opacity=0.7] (2.4870, 2.2420, 2.1437) -- (2.5340, 2.2420, 2.1380) -- (2.5340, 2.2950, 2.1337) -- (2.4870, 2.2950, 2.1393) -- cycle;
\fill[blue!66.0, opacity=0.7] (2.4870, 2.2950, 2.1393) -- (2.5340, 2.2950, 2.1337) -- (2.5340, 2.3480, 2.1291) -- (2.4870, 2.3480, 2.1348) -- cycle;
\fill[blue!36.6, opacity=0.7] (2.4870, 2.3480, 2.1348) -- (2.5340, 2.3480, 2.1291) -- (2.5340, 2.4010, 2.1243) -- (2.4870, 2.4010, 2.1300) -- cycle;
\fill[blue!18.8, opacity=0.7] (2.4870, 2.4010, 2.1300) -- (2.5340, 2.4010, 2.1243) -- (2.5340, 2.4540, 2.1193) -- (2.4870, 2.4540, 2.1250) -- cycle;
\fill[blue!15.4, opacity=0.7] (2.4870, 2.4540, 2.1250) -- (2.5340, 2.4540, 2.1193) -- (2.5340, 2.5070, 2.1142) -- (2.4870, 2.5070, 2.1198) -- cycle;
\fill[blue!15.1, opacity=0.7] (2.4870, 2.5070, 2.1198) -- (2.5340, 2.5070, 2.1142) -- (2.5340, 2.5600, 2.1088) -- (2.4870, 2.5600, 2.1145) -- cycle;
\fill[blue!15.0, opacity=0.7] (2.4870, 2.5600, 2.1145) -- (2.5340, 2.5600, 2.1088) -- (2.5340, 2.6130, 2.1033) -- (2.4870, 2.6130, 2.1090) -- cycle;
\fill[blue!15.1, opacity=0.7] (2.4870, 2.6130, 2.1090) -- (2.5340, 2.6130, 2.1033) -- (2.5340, 2.6660, 2.0976) -- (2.4870, 2.6660, 2.1033) -- cycle;
\fill[blue!15.9, opacity=0.7] (2.4870, 2.6660, 2.1033) -- (2.5340, 2.6660, 2.0976) -- (2.5340, 2.7190, 2.0918) -- (2.4870, 2.7190, 2.0975) -- cycle;
\fill[blue!21.6, opacity=0.7] (2.4870, 2.7190, 2.0975) -- (2.5340, 2.7190, 2.0918) -- (2.5340, 2.7720, 2.0859) -- (2.4870, 2.7720, 2.0916) -- cycle;
\fill[blue!34.3, opacity=0.7] (2.4870, 2.7720, 2.0916) -- (2.5340, 2.7720, 2.0859) -- (2.5340, 2.8250, 2.0799) -- (2.4870, 2.8250, 2.0855) -- cycle;
\fill[blue!37.9, opacity=0.7] (2.4870, 2.8250, 2.0855) -- (2.5340, 2.8250, 2.0799) -- (2.5340, 2.8780, 2.0738) -- (2.4870, 2.8780, 2.0794) -- cycle;
\fill[blue!25.6, opacity=0.7] (2.4870, 2.8780, 2.0794) -- (2.5340, 2.8780, 2.0738) -- (2.5340, 2.9310, 2.0676) -- (2.4870, 2.9310, 2.0733) -- cycle;
\fill[blue!16.3, opacity=0.7] (2.4870, 2.9310, 2.0733) -- (2.5340, 2.9310, 2.0676) -- (2.5340, 2.9840, 2.0614) -- (2.4870, 2.9840, 2.0670) -- cycle;
\fill[blue!15.0, opacity=0.7] (2.4870, 2.9840, 2.0670) -- (2.5340, 2.9840, 2.0614) -- (2.5340, 3.0370, 2.0551) -- (2.4870, 3.0370, 2.0608) -- cycle;
\fill[blue!15.0, opacity=0.7] (2.4870, 3.0370, 2.0608) -- (2.5340, 3.0370, 2.0551) -- (2.5340, 3.0900, 2.0488) -- (2.4870, 3.0900, 2.0545) -- cycle;
\fill[blue!18.5, opacity=0.7] (2.5340, -0.0900, 2.0488) -- (2.5810, -0.0900, 2.0430) -- (2.5810, -0.0370, 2.0493) -- (2.5340, -0.0370, 2.0551) -- cycle;
\fill[blue!15.7, opacity=0.7] (2.5340, -0.0370, 2.0551) -- (2.5810, -0.0370, 2.0493) -- (2.5810, 0.0160, 2.0555) -- (2.5340, 0.0160, 2.0614) -- cycle;
\fill[blue!15.1, opacity=0.7] (2.5340, 0.0160, 2.0614) -- (2.5810, 0.0160, 2.0555) -- (2.5810, 0.0690, 2.0618) -- (2.5340, 0.0690, 2.0676) -- cycle;
\fill[blue!15.0, opacity=0.7] (2.5340, 0.0690, 2.0676) -- (2.5810, 0.0690, 2.0618) -- (2.5810, 0.1220, 2.0680) -- (2.5340, 0.1220, 2.0738) -- cycle;
\fill[blue!15.0, opacity=0.7] (2.5340, 0.1220, 2.0738) -- (2.5810, 0.1220, 2.0680) -- (2.5810, 0.1750, 2.0741) -- (2.5340, 0.1750, 2.0799) -- cycle;
\fill[blue!15.0, opacity=0.7] (2.5340, 0.1750, 2.0799) -- (2.5810, 0.1750, 2.0741) -- (2.5810, 0.2280, 2.0801) -- (2.5340, 0.2280, 2.0859) -- cycle;
\fill[blue!15.0, opacity=0.7] (2.5340, 0.2280, 2.0859) -- (2.5810, 0.2280, 2.0801) -- (2.5810, 0.2810, 2.0860) -- (2.5340, 0.2810, 2.0918) -- cycle;
\fill[blue!15.4, opacity=0.7] (2.5340, 0.2810, 2.0918) -- (2.5810, 0.2810, 2.0860) -- (2.5810, 0.3340, 2.0918) -- (2.5340, 0.3340, 2.0976) -- cycle;
\fill[blue!18.8, opacity=0.7] (2.5340, 0.3340, 2.0976) -- (2.5810, 0.3340, 2.0918) -- (2.5810, 0.3870, 2.0975) -- (2.5340, 0.3870, 2.1033) -- cycle;
\fill[blue!32.6, opacity=0.7] (2.5340, 0.3870, 2.1033) -- (2.5810, 0.3870, 2.0975) -- (2.5810, 0.4400, 2.1030) -- (2.5340, 0.4400, 2.1088) -- cycle;
\fill[blue!52.7, opacity=0.7] (2.5340, 0.4400, 2.1088) -- (2.5810, 0.4400, 2.1030) -- (2.5810, 0.4930, 2.1084) -- (2.5340, 0.4930, 2.1142) -- cycle;
\fill[blue!64.5, opacity=0.7] (2.5340, 0.4930, 2.1142) -- (2.5810, 0.4930, 2.1084) -- (2.5810, 0.5460, 2.1135) -- (2.5340, 0.5460, 2.1193) -- cycle;
\fill[blue!64.3, opacity=0.7] (2.5340, 0.5460, 2.1193) -- (2.5810, 0.5460, 2.1135) -- (2.5810, 0.5990, 2.1185) -- (2.5340, 0.5990, 2.1243) -- cycle;
\fill[blue!54.4, opacity=0.7] (2.5340, 0.5990, 2.1243) -- (2.5810, 0.5990, 2.1185) -- (2.5810, 0.6520, 2.1233) -- (2.5340, 0.6520, 2.1291) -- cycle;
\fill[blue!39.4, opacity=0.7] (2.5340, 0.6520, 2.1291) -- (2.5810, 0.6520, 2.1233) -- (2.5810, 0.7050, 2.1279) -- (2.5340, 0.7050, 2.1337) -- cycle;
\fill[blue!26.7, opacity=0.7] (2.5340, 0.7050, 2.1337) -- (2.5810, 0.7050, 2.1279) -- (2.5810, 0.7580, 2.1322) -- (2.5340, 0.7580, 2.1380) -- cycle;
\fill[blue!19.8, opacity=0.7] (2.5340, 0.7580, 2.1380) -- (2.5810, 0.7580, 2.1322) -- (2.5810, 0.8110, 2.1363) -- (2.5340, 0.8110, 2.1421) -- cycle;
\fill[blue!17.0, opacity=0.7] (2.5340, 0.8110, 2.1421) -- (2.5810, 0.8110, 2.1363) -- (2.5810, 0.8640, 2.1401) -- (2.5340, 0.8640, 2.1459) -- cycle;
\fill[blue!16.0, opacity=0.7] (2.5340, 0.8640, 2.1459) -- (2.5810, 0.8640, 2.1401) -- (2.5810, 0.9170, 2.1436) -- (2.5340, 0.9170, 2.1494) -- cycle;
\fill[blue!15.6, opacity=0.7] (2.5340, 0.9170, 2.1494) -- (2.5810, 0.9170, 2.1436) -- (2.5810, 0.9700, 2.1469) -- (2.5340, 0.9700, 2.1527) -- cycle;
\fill[blue!15.5, opacity=0.7] (2.5340, 0.9700, 2.1527) -- (2.5810, 0.9700, 2.1469) -- (2.5810, 1.0230, 2.1499) -- (2.5340, 1.0230, 2.1557) -- cycle;
\fill[blue!15.5, opacity=0.7] (2.5340, 1.0230, 2.1557) -- (2.5810, 1.0230, 2.1499) -- (2.5810, 1.0760, 2.1526) -- (2.5340, 1.0760, 2.1584) -- cycle;
\fill[blue!15.6, opacity=0.7] (2.5340, 1.0760, 2.1584) -- (2.5810, 1.0760, 2.1526) -- (2.5810, 1.1290, 2.1550) -- (2.5340, 1.1290, 2.1608) -- cycle;
\fill[blue!15.8, opacity=0.7] (2.5340, 1.1290, 2.1608) -- (2.5810, 1.1290, 2.1550) -- (2.5810, 1.1820, 2.1571) -- (2.5340, 1.1820, 2.1629) -- cycle;
\fill[blue!16.1, opacity=0.7] (2.5340, 1.1820, 2.1629) -- (2.5810, 1.1820, 2.1571) -- (2.5810, 1.2350, 2.1589) -- (2.5340, 1.2350, 2.1647) -- cycle;
\fill[blue!16.4, opacity=0.7] (2.5340, 1.2350, 2.1647) -- (2.5810, 1.2350, 2.1589) -- (2.5810, 1.2880, 2.1604) -- (2.5340, 1.2880, 2.1662) -- cycle;
\fill[blue!16.6, opacity=0.7] (2.5340, 1.2880, 2.1662) -- (2.5810, 1.2880, 2.1604) -- (2.5810, 1.3410, 2.1615) -- (2.5340, 1.3410, 2.1673) -- cycle;
\fill[blue!16.8, opacity=0.7] (2.5340, 1.3410, 2.1673) -- (2.5810, 1.3410, 2.1615) -- (2.5810, 1.3940, 2.1623) -- (2.5340, 1.3940, 2.1682) -- cycle;
\fill[blue!16.8, opacity=0.7] (2.5340, 1.3940, 2.1682) -- (2.5810, 1.3940, 2.1623) -- (2.5810, 1.4470, 2.1628) -- (2.5340, 1.4470, 2.1686) -- cycle;
\fill[blue!16.6, opacity=0.7] (2.5340, 1.4470, 2.1686) -- (2.5810, 1.4470, 2.1628) -- (2.5810, 1.5000, 2.1630) -- (2.5340, 1.5000, 2.1688) -- cycle;
\fill[blue!16.4, opacity=0.7] (2.5340, 1.5000, 2.1688) -- (2.5810, 1.5000, 2.1630) -- (2.5810, 1.5530, 2.1628) -- (2.5340, 1.5530, 2.1686) -- cycle;
\fill[blue!16.2, opacity=0.7] (2.5340, 1.5530, 2.1686) -- (2.5810, 1.5530, 2.1628) -- (2.5810, 1.6060, 2.1623) -- (2.5340, 1.6060, 2.1682) -- cycle;
\fill[blue!16.0, opacity=0.7] (2.5340, 1.6060, 2.1682) -- (2.5810, 1.6060, 2.1623) -- (2.5810, 1.6590, 2.1615) -- (2.5340, 1.6590, 2.1673) -- cycle;
\fill[blue!16.0, opacity=0.7] (2.5340, 1.6590, 2.1673) -- (2.5810, 1.6590, 2.1615) -- (2.5810, 1.7120, 2.1604) -- (2.5340, 1.7120, 2.1662) -- cycle;
\fill[blue!16.1, opacity=0.7] (2.5340, 1.7120, 2.1662) -- (2.5810, 1.7120, 2.1604) -- (2.5810, 1.7650, 2.1589) -- (2.5340, 1.7650, 2.1647) -- cycle;
\fill[blue!16.6, opacity=0.7] (2.5340, 1.7650, 2.1647) -- (2.5810, 1.7650, 2.1589) -- (2.5810, 1.8180, 2.1571) -- (2.5340, 1.8180, 2.1629) -- cycle;
\fill[blue!18.0, opacity=0.7] (2.5340, 1.8180, 2.1629) -- (2.5810, 1.8180, 2.1571) -- (2.5810, 1.8710, 2.1550) -- (2.5340, 1.8710, 2.1608) -- cycle;
\fill[blue!21.6, opacity=0.7] (2.5340, 1.8710, 2.1608) -- (2.5810, 1.8710, 2.1550) -- (2.5810, 1.9240, 2.1526) -- (2.5340, 1.9240, 2.1584) -- cycle;
\fill[blue!30.6, opacity=0.7] (2.5340, 1.9240, 2.1584) -- (2.5810, 1.9240, 2.1526) -- (2.5810, 1.9770, 2.1499) -- (2.5340, 1.9770, 2.1557) -- cycle;
\fill[blue!47.1, opacity=0.7] (2.5340, 1.9770, 2.1557) -- (2.5810, 1.9770, 2.1499) -- (2.5810, 2.0300, 2.1469) -- (2.5340, 2.0300, 2.1527) -- cycle;
\fill[blue!66.6, opacity=0.7] (2.5340, 2.0300, 2.1527) -- (2.5810, 2.0300, 2.1469) -- (2.5810, 2.0830, 2.1436) -- (2.5340, 2.0830, 2.1494) -- cycle;
\fill[blue!79.4, opacity=0.7] (2.5340, 2.0830, 2.1494) -- (2.5810, 2.0830, 2.1436) -- (2.5810, 2.1360, 2.1401) -- (2.5340, 2.1360, 2.1459) -- cycle;
\fill[blue!83.0, opacity=0.7] (2.5340, 2.1360, 2.1459) -- (2.5810, 2.1360, 2.1401) -- (2.5810, 2.1890, 2.1363) -- (2.5340, 2.1890, 2.1421) -- cycle;
\fill[blue!78.4, opacity=0.7] (2.5340, 2.1890, 2.1421) -- (2.5810, 2.1890, 2.1363) -- (2.5810, 2.2420, 2.1322) -- (2.5340, 2.2420, 2.1380) -- cycle;
\fill[blue!60.1, opacity=0.7] (2.5340, 2.2420, 2.1380) -- (2.5810, 2.2420, 2.1322) -- (2.5810, 2.2950, 2.1279) -- (2.5340, 2.2950, 2.1337) -- cycle;
\fill[blue!32.7, opacity=0.7] (2.5340, 2.2950, 2.1337) -- (2.5810, 2.2950, 2.1279) -- (2.5810, 2.3480, 2.1233) -- (2.5340, 2.3480, 2.1291) -- cycle;
\fill[blue!18.1, opacity=0.7] (2.5340, 2.3480, 2.1291) -- (2.5810, 2.3480, 2.1233) -- (2.5810, 2.4010, 2.1185) -- (2.5340, 2.4010, 2.1243) -- cycle;
\fill[blue!15.3, opacity=0.7] (2.5340, 2.4010, 2.1243) -- (2.5810, 2.4010, 2.1185) -- (2.5810, 2.4540, 2.1135) -- (2.5340, 2.4540, 2.1193) -- cycle;
\fill[blue!15.0, opacity=0.7] (2.5340, 2.4540, 2.1193) -- (2.5810, 2.4540, 2.1135) -- (2.5810, 2.5070, 2.1084) -- (2.5340, 2.5070, 2.1142) -- cycle;
\fill[blue!15.0, opacity=0.7] (2.5340, 2.5070, 2.1142) -- (2.5810, 2.5070, 2.1084) -- (2.5810, 2.5600, 2.1030) -- (2.5340, 2.5600, 2.1088) -- cycle;
\fill[blue!15.1, opacity=0.7] (2.5340, 2.5600, 2.1088) -- (2.5810, 2.5600, 2.1030) -- (2.5810, 2.6130, 2.0975) -- (2.5340, 2.6130, 2.1033) -- cycle;
\fill[blue!15.6, opacity=0.7] (2.5340, 2.6130, 2.1033) -- (2.5810, 2.6130, 2.0975) -- (2.5810, 2.6660, 2.0918) -- (2.5340, 2.6660, 2.0976) -- cycle;
\fill[blue!19.6, opacity=0.7] (2.5340, 2.6660, 2.0976) -- (2.5810, 2.6660, 2.0918) -- (2.5810, 2.7190, 2.0860) -- (2.5340, 2.7190, 2.0918) -- cycle;
\fill[blue!31.1, opacity=0.7] (2.5340, 2.7190, 2.0918) -- (2.5810, 2.7190, 2.0860) -- (2.5810, 2.7720, 2.0801) -- (2.5340, 2.7720, 2.0859) -- cycle;
\fill[blue!38.2, opacity=0.7] (2.5340, 2.7720, 2.0859) -- (2.5810, 2.7720, 2.0801) -- (2.5810, 2.8250, 2.0741) -- (2.5340, 2.8250, 2.0799) -- cycle;
\fill[blue!29.4, opacity=0.7] (2.5340, 2.8250, 2.0799) -- (2.5810, 2.8250, 2.0741) -- (2.5810, 2.8780, 2.0680) -- (2.5340, 2.8780, 2.0738) -- cycle;
\fill[blue!17.7, opacity=0.7] (2.5340, 2.8780, 2.0738) -- (2.5810, 2.8780, 2.0680) -- (2.5810, 2.9310, 2.0618) -- (2.5340, 2.9310, 2.0676) -- cycle;
\fill[blue!15.1, opacity=0.7] (2.5340, 2.9310, 2.0676) -- (2.5810, 2.9310, 2.0618) -- (2.5810, 2.9840, 2.0555) -- (2.5340, 2.9840, 2.0614) -- cycle;
\fill[blue!15.0, opacity=0.7] (2.5340, 2.9840, 2.0614) -- (2.5810, 2.9840, 2.0555) -- (2.5810, 3.0370, 2.0493) -- (2.5340, 3.0370, 2.0551) -- cycle;
\fill[blue!15.0, opacity=0.7] (2.5340, 3.0370, 2.0551) -- (2.5810, 3.0370, 2.0493) -- (2.5810, 3.0900, 2.0430) -- (2.5340, 3.0900, 2.0488) -- cycle;
\fill[blue!21.3, opacity=0.7] (2.5810, -0.0900, 2.0430) -- (2.6280, -0.0900, 2.0371) -- (2.6280, -0.0370, 2.0434) -- (2.5810, -0.0370, 2.0493) -- cycle;
\fill[blue!18.4, opacity=0.7] (2.5810, -0.0370, 2.0493) -- (2.6280, -0.0370, 2.0434) -- (2.6280, 0.0160, 2.0496) -- (2.5810, 0.0160, 2.0555) -- cycle;
\fill[blue!15.7, opacity=0.7] (2.5810, 0.0160, 2.0555) -- (2.6280, 0.0160, 2.0496) -- (2.6280, 0.0690, 2.0559) -- (2.5810, 0.0690, 2.0618) -- cycle;
\fill[blue!15.1, opacity=0.7] (2.5810, 0.0690, 2.0618) -- (2.6280, 0.0690, 2.0559) -- (2.6280, 0.1220, 2.0620) -- (2.5810, 0.1220, 2.0680) -- cycle;
\fill[blue!15.0, opacity=0.7] (2.5810, 0.1220, 2.0680) -- (2.6280, 0.1220, 2.0620) -- (2.6280, 0.1750, 2.0681) -- (2.5810, 0.1750, 2.0741) -- cycle;
\fill[blue!15.0, opacity=0.7] (2.5810, 0.1750, 2.0741) -- (2.6280, 0.1750, 2.0681) -- (2.6280, 0.2280, 2.0742) -- (2.5810, 0.2280, 2.0801) -- cycle;
\fill[blue!15.0, opacity=0.7] (2.5810, 0.2280, 2.0801) -- (2.6280, 0.2280, 2.0742) -- (2.6280, 0.2810, 2.0801) -- (2.5810, 0.2810, 2.0860) -- cycle;
\fill[blue!15.0, opacity=0.7] (2.5810, 0.2810, 2.0860) -- (2.6280, 0.2810, 2.0801) -- (2.6280, 0.3340, 2.0859) -- (2.5810, 0.3340, 2.0918) -- cycle;
\fill[blue!15.2, opacity=0.7] (2.5810, 0.3340, 2.0918) -- (2.6280, 0.3340, 2.0859) -- (2.6280, 0.3870, 2.0916) -- (2.5810, 0.3870, 2.0975) -- cycle;
\fill[blue!17.2, opacity=0.7] (2.5810, 0.3870, 2.0975) -- (2.6280, 0.3870, 2.0916) -- (2.6280, 0.4400, 2.0971) -- (2.5810, 0.4400, 2.1030) -- cycle;
\fill[blue!26.6, opacity=0.7] (2.5810, 0.4400, 2.1030) -- (2.6280, 0.4400, 2.0971) -- (2.6280, 0.4930, 2.1024) -- (2.5810, 0.4930, 2.1084) -- cycle;
\fill[blue!45.1, opacity=0.7] (2.5810, 0.4930, 2.1084) -- (2.6280, 0.4930, 2.1024) -- (2.6280, 0.5460, 2.1076) -- (2.5810, 0.5460, 2.1135) -- cycle;
\fill[blue!60.9, opacity=0.7] (2.5810, 0.5460, 2.1135) -- (2.6280, 0.5460, 2.1076) -- (2.6280, 0.5990, 2.1126) -- (2.5810, 0.5990, 2.1185) -- cycle;
\fill[blue!66.9, opacity=0.7] (2.5810, 0.5990, 2.1185) -- (2.6280, 0.5990, 2.1126) -- (2.6280, 0.6520, 2.1174) -- (2.5810, 0.6520, 2.1233) -- cycle;
\fill[blue!63.8, opacity=0.7] (2.5810, 0.6520, 2.1233) -- (2.6280, 0.6520, 2.1174) -- (2.6280, 0.7050, 2.1219) -- (2.5810, 0.7050, 2.1279) -- cycle;
\fill[blue!53.7, opacity=0.7] (2.5810, 0.7050, 2.1279) -- (2.6280, 0.7050, 2.1219) -- (2.6280, 0.7580, 2.1263) -- (2.5810, 0.7580, 2.1322) -- cycle;
\fill[blue!40.5, opacity=0.7] (2.5810, 0.7580, 2.1322) -- (2.6280, 0.7580, 2.1263) -- (2.6280, 0.8110, 2.1303) -- (2.5810, 0.8110, 2.1363) -- cycle;
\fill[blue!29.2, opacity=0.7] (2.5810, 0.8110, 2.1363) -- (2.6280, 0.8110, 2.1303) -- (2.6280, 0.8640, 2.1342) -- (2.5810, 0.8640, 2.1401) -- cycle;
\fill[blue!22.2, opacity=0.7] (2.5810, 0.8640, 2.1401) -- (2.6280, 0.8640, 2.1342) -- (2.6280, 0.9170, 2.1377) -- (2.5810, 0.9170, 2.1436) -- cycle;
\fill[blue!18.6, opacity=0.7] (2.5810, 0.9170, 2.1436) -- (2.6280, 0.9170, 2.1377) -- (2.6280, 0.9700, 2.1410) -- (2.5810, 0.9700, 2.1469) -- cycle;
\fill[blue!17.0, opacity=0.7] (2.5810, 0.9700, 2.1469) -- (2.6280, 0.9700, 2.1410) -- (2.6280, 1.0230, 2.1440) -- (2.5810, 1.0230, 2.1499) -- cycle;
\fill[blue!16.2, opacity=0.7] (2.5810, 1.0230, 2.1499) -- (2.6280, 1.0230, 2.1440) -- (2.6280, 1.0760, 2.1467) -- (2.5810, 1.0760, 2.1526) -- cycle;
\fill[blue!15.9, opacity=0.7] (2.5810, 1.0760, 2.1526) -- (2.6280, 1.0760, 2.1467) -- (2.6280, 1.1290, 2.1491) -- (2.5810, 1.1290, 2.1550) -- cycle;
\fill[blue!15.7, opacity=0.7] (2.5810, 1.1290, 2.1550) -- (2.6280, 1.1290, 2.1491) -- (2.6280, 1.1820, 2.1512) -- (2.5810, 1.1820, 2.1571) -- cycle;
\fill[blue!15.7, opacity=0.7] (2.5810, 1.1820, 2.1571) -- (2.6280, 1.1820, 2.1512) -- (2.6280, 1.2350, 2.1530) -- (2.5810, 1.2350, 2.1589) -- cycle;
\fill[blue!15.7, opacity=0.7] (2.5810, 1.2350, 2.1589) -- (2.6280, 1.2350, 2.1530) -- (2.6280, 1.2880, 2.1545) -- (2.5810, 1.2880, 2.1604) -- cycle;
\fill[blue!15.7, opacity=0.7] (2.5810, 1.2880, 2.1604) -- (2.6280, 1.2880, 2.1545) -- (2.6280, 1.3410, 2.1556) -- (2.5810, 1.3410, 2.1615) -- cycle;
\fill[blue!15.7, opacity=0.7] (2.5810, 1.3410, 2.1615) -- (2.6280, 1.3410, 2.1556) -- (2.6280, 1.3940, 2.1564) -- (2.5810, 1.3940, 2.1623) -- cycle;
\fill[blue!15.8, opacity=0.7] (2.5810, 1.3940, 2.1623) -- (2.6280, 1.3940, 2.1564) -- (2.6280, 1.4470, 2.1569) -- (2.5810, 1.4470, 2.1628) -- cycle;
\fill[blue!15.8, opacity=0.7] (2.5810, 1.4470, 2.1628) -- (2.6280, 1.4470, 2.1569) -- (2.6280, 1.5000, 2.1571) -- (2.5810, 1.5000, 2.1630) -- cycle;
\fill[blue!16.0, opacity=0.7] (2.5810, 1.5000, 2.1630) -- (2.6280, 1.5000, 2.1571) -- (2.6280, 1.5530, 2.1569) -- (2.5810, 1.5530, 2.1628) -- cycle;
\fill[blue!16.2, opacity=0.7] (2.5810, 1.5530, 2.1628) -- (2.6280, 1.5530, 2.1569) -- (2.6280, 1.6060, 2.1564) -- (2.5810, 1.6060, 2.1623) -- cycle;
\fill[blue!16.6, opacity=0.7] (2.5810, 1.6060, 2.1623) -- (2.6280, 1.6060, 2.1564) -- (2.6280, 1.6590, 2.1556) -- (2.5810, 1.6590, 2.1615) -- cycle;
\fill[blue!17.5, opacity=0.7] (2.5810, 1.6590, 2.1615) -- (2.6280, 1.6590, 2.1556) -- (2.6280, 1.7120, 2.1545) -- (2.5810, 1.7120, 2.1604) -- cycle;
\fill[blue!19.4, opacity=0.7] (2.5810, 1.7120, 2.1604) -- (2.6280, 1.7120, 2.1545) -- (2.6280, 1.7650, 2.1530) -- (2.5810, 1.7650, 2.1589) -- cycle;
\fill[blue!23.6, opacity=0.7] (2.5810, 1.7650, 2.1589) -- (2.6280, 1.7650, 2.1530) -- (2.6280, 1.8180, 2.1512) -- (2.5810, 1.8180, 2.1571) -- cycle;
\fill[blue!31.9, opacity=0.7] (2.5810, 1.8180, 2.1571) -- (2.6280, 1.8180, 2.1512) -- (2.6280, 1.8710, 2.1491) -- (2.5810, 1.8710, 2.1550) -- cycle;
\fill[blue!45.7, opacity=0.7] (2.5810, 1.8710, 2.1550) -- (2.6280, 1.8710, 2.1491) -- (2.6280, 1.9240, 2.1467) -- (2.5810, 1.9240, 2.1526) -- cycle;
\fill[blue!62.4, opacity=0.7] (2.5810, 1.9240, 2.1526) -- (2.6280, 1.9240, 2.1467) -- (2.6280, 1.9770, 2.1440) -- (2.5810, 1.9770, 2.1499) -- cycle;
\fill[blue!75.6, opacity=0.7] (2.5810, 1.9770, 2.1499) -- (2.6280, 1.9770, 2.1440) -- (2.6280, 2.0300, 2.1410) -- (2.5810, 2.0300, 2.1469) -- cycle;
\fill[blue!81.7, opacity=0.7] (2.5810, 2.0300, 2.1469) -- (2.6280, 2.0300, 2.1410) -- (2.6280, 2.0830, 2.1377) -- (2.5810, 2.0830, 2.1436) -- cycle;
\fill[blue!81.2, opacity=0.7] (2.5810, 2.0830, 2.1436) -- (2.6280, 2.0830, 2.1377) -- (2.6280, 2.1360, 2.1342) -- (2.5810, 2.1360, 2.1401) -- cycle;
\fill[blue!72.1, opacity=0.7] (2.5810, 2.1360, 2.1401) -- (2.6280, 2.1360, 2.1342) -- (2.6280, 2.1890, 2.1303) -- (2.5810, 2.1890, 2.1363) -- cycle;
\fill[blue!50.3, opacity=0.7] (2.5810, 2.1890, 2.1363) -- (2.6280, 2.1890, 2.1303) -- (2.6280, 2.2420, 2.1263) -- (2.5810, 2.2420, 2.1322) -- cycle;
\fill[blue!27.0, opacity=0.7] (2.5810, 2.2420, 2.1322) -- (2.6280, 2.2420, 2.1263) -- (2.6280, 2.2950, 2.1219) -- (2.5810, 2.2950, 2.1279) -- cycle;
\fill[blue!17.0, opacity=0.7] (2.5810, 2.2950, 2.1279) -- (2.6280, 2.2950, 2.1219) -- (2.6280, 2.3480, 2.1174) -- (2.5810, 2.3480, 2.1233) -- cycle;
\fill[blue!15.2, opacity=0.7] (2.5810, 2.3480, 2.1233) -- (2.6280, 2.3480, 2.1174) -- (2.6280, 2.4010, 2.1126) -- (2.5810, 2.4010, 2.1185) -- cycle;
\fill[blue!15.0, opacity=0.7] (2.5810, 2.4010, 2.1185) -- (2.6280, 2.4010, 2.1126) -- (2.6280, 2.4540, 2.1076) -- (2.5810, 2.4540, 2.1135) -- cycle;
\fill[blue!15.0, opacity=0.7] (2.5810, 2.4540, 2.1135) -- (2.6280, 2.4540, 2.1076) -- (2.6280, 2.5070, 2.1024) -- (2.5810, 2.5070, 2.1084) -- cycle;
\fill[blue!15.1, opacity=0.7] (2.5810, 2.5070, 2.1084) -- (2.6280, 2.5070, 2.1024) -- (2.6280, 2.5600, 2.0971) -- (2.5810, 2.5600, 2.1030) -- cycle;
\fill[blue!15.5, opacity=0.7] (2.5810, 2.5600, 2.1030) -- (2.6280, 2.5600, 2.0971) -- (2.6280, 2.6130, 2.0916) -- (2.5810, 2.6130, 2.0975) -- cycle;
\fill[blue!18.6, opacity=0.7] (2.5810, 2.6130, 2.0975) -- (2.6280, 2.6130, 2.0916) -- (2.6280, 2.6660, 2.0859) -- (2.5810, 2.6660, 2.0918) -- cycle;
\fill[blue!28.7, opacity=0.7] (2.5810, 2.6660, 2.0918) -- (2.6280, 2.6660, 2.0859) -- (2.6280, 2.7190, 2.0801) -- (2.5810, 2.7190, 2.0860) -- cycle;
\fill[blue!37.5, opacity=0.7] (2.5810, 2.7190, 2.0860) -- (2.6280, 2.7190, 2.0801) -- (2.6280, 2.7720, 2.0742) -- (2.5810, 2.7720, 2.0801) -- cycle;
\fill[blue!31.9, opacity=0.7] (2.5810, 2.7720, 2.0801) -- (2.6280, 2.7720, 2.0742) -- (2.6280, 2.8250, 2.0681) -- (2.5810, 2.8250, 2.0741) -- cycle;
\fill[blue!19.5, opacity=0.7] (2.5810, 2.8250, 2.0741) -- (2.6280, 2.8250, 2.0681) -- (2.6280, 2.8780, 2.0620) -- (2.5810, 2.8780, 2.0680) -- cycle;
\fill[blue!15.3, opacity=0.7] (2.5810, 2.8780, 2.0680) -- (2.6280, 2.8780, 2.0620) -- (2.6280, 2.9310, 2.0559) -- (2.5810, 2.9310, 2.0618) -- cycle;
\fill[blue!15.0, opacity=0.7] (2.5810, 2.9310, 2.0618) -- (2.6280, 2.9310, 2.0559) -- (2.6280, 2.9840, 2.0496) -- (2.5810, 2.9840, 2.0555) -- cycle;
\fill[blue!15.0, opacity=0.7] (2.5810, 2.9840, 2.0555) -- (2.6280, 2.9840, 2.0496) -- (2.6280, 3.0370, 2.0434) -- (2.5810, 3.0370, 2.0493) -- cycle;
\fill[blue!15.0, opacity=0.7] (2.5810, 3.0370, 2.0493) -- (2.6280, 3.0370, 2.0434) -- (2.6280, 3.0900, 2.0371) -- (2.5810, 3.0900, 2.0430) -- cycle;
\fill[blue!19.7, opacity=0.7] (2.6280, -0.0900, 2.0371) -- (2.6750, -0.0900, 2.0311) -- (2.6750, -0.0370, 2.0373) -- (2.6280, -0.0370, 2.0434) -- cycle;
\fill[blue!21.5, opacity=0.7] (2.6280, -0.0370, 2.0434) -- (2.6750, -0.0370, 2.0373) -- (2.6750, 0.0160, 2.0436) -- (2.6280, 0.0160, 2.0496) -- cycle;
\fill[blue!18.5, opacity=0.7] (2.6280, 0.0160, 2.0496) -- (2.6750, 0.0160, 2.0436) -- (2.6750, 0.0690, 2.0498) -- (2.6280, 0.0690, 2.0559) -- cycle;
\fill[blue!15.8, opacity=0.7] (2.6280, 0.0690, 2.0559) -- (2.6750, 0.0690, 2.0498) -- (2.6750, 0.1220, 2.0560) -- (2.6280, 0.1220, 2.0620) -- cycle;
\fill[blue!15.1, opacity=0.7] (2.6280, 0.1220, 2.0620) -- (2.6750, 0.1220, 2.0560) -- (2.6750, 0.1750, 2.0621) -- (2.6280, 0.1750, 2.0681) -- cycle;
\fill[blue!15.0, opacity=0.7] (2.6280, 0.1750, 2.0681) -- (2.6750, 0.1750, 2.0621) -- (2.6750, 0.2280, 2.0681) -- (2.6280, 0.2280, 2.0742) -- cycle;
\fill[blue!15.0, opacity=0.7] (2.6280, 0.2280, 2.0742) -- (2.6750, 0.2280, 2.0681) -- (2.6750, 0.2810, 2.0741) -- (2.6280, 0.2810, 2.0801) -- cycle;
\fill[blue!15.0, opacity=0.7] (2.6280, 0.2810, 2.0801) -- (2.6750, 0.2810, 2.0741) -- (2.6750, 0.3340, 2.0799) -- (2.6280, 0.3340, 2.0859) -- cycle;
\fill[blue!15.0, opacity=0.7] (2.6280, 0.3340, 2.0859) -- (2.6750, 0.3340, 2.0799) -- (2.6750, 0.3870, 2.0855) -- (2.6280, 0.3870, 2.0916) -- cycle;
\fill[blue!15.1, opacity=0.7] (2.6280, 0.3870, 2.0916) -- (2.6750, 0.3870, 2.0855) -- (2.6750, 0.4400, 2.0911) -- (2.6280, 0.4400, 2.0971) -- cycle;
\fill[blue!15.9, opacity=0.7] (2.6280, 0.4400, 2.0971) -- (2.6750, 0.4400, 2.0911) -- (2.6750, 0.4930, 2.0964) -- (2.6280, 0.4930, 2.1024) -- cycle;
\fill[blue!20.9, opacity=0.7] (2.6280, 0.4930, 2.1024) -- (2.6750, 0.4930, 2.0964) -- (2.6750, 0.5460, 2.1016) -- (2.6280, 0.5460, 2.1076) -- cycle;
\fill[blue!34.5, opacity=0.7] (2.6280, 0.5460, 2.1076) -- (2.6750, 0.5460, 2.1016) -- (2.6750, 0.5990, 2.1066) -- (2.6280, 0.5990, 2.1126) -- cycle;
\fill[blue!52.1, opacity=0.7] (2.6280, 0.5990, 2.1126) -- (2.6750, 0.5990, 2.1066) -- (2.6750, 0.6520, 2.1114) -- (2.6280, 0.6520, 2.1174) -- cycle;
\fill[blue!64.2, opacity=0.7] (2.6280, 0.6520, 2.1174) -- (2.6750, 0.6520, 2.1114) -- (2.6750, 0.7050, 2.1159) -- (2.6280, 0.7050, 2.1219) -- cycle;
\fill[blue!68.2, opacity=0.7] (2.6280, 0.7050, 2.1219) -- (2.6750, 0.7050, 2.1159) -- (2.6750, 0.7580, 2.1202) -- (2.6280, 0.7580, 2.1263) -- cycle;
\fill[blue!65.7, opacity=0.7] (2.6280, 0.7580, 2.1263) -- (2.6750, 0.7580, 2.1202) -- (2.6750, 0.8110, 2.1243) -- (2.6280, 0.8110, 2.1303) -- cycle;
\fill[blue!58.1, opacity=0.7] (2.6280, 0.8110, 2.1303) -- (2.6750, 0.8110, 2.1243) -- (2.6750, 0.8640, 2.1281) -- (2.6280, 0.8640, 2.1342) -- cycle;
\fill[blue!47.9, opacity=0.7] (2.6280, 0.8640, 2.1342) -- (2.6750, 0.8640, 2.1281) -- (2.6750, 0.9170, 2.1317) -- (2.6280, 0.9170, 2.1377) -- cycle;
\fill[blue!37.8, opacity=0.7] (2.6280, 0.9170, 2.1377) -- (2.6750, 0.9170, 2.1317) -- (2.6750, 0.9700, 2.1350) -- (2.6280, 0.9700, 2.1410) -- cycle;
\fill[blue!29.9, opacity=0.7] (2.6280, 0.9700, 2.1410) -- (2.6750, 0.9700, 2.1350) -- (2.6750, 1.0230, 2.1380) -- (2.6280, 1.0230, 2.1440) -- cycle;
\fill[blue!24.6, opacity=0.7] (2.6280, 1.0230, 2.1440) -- (2.6750, 1.0230, 2.1380) -- (2.6750, 1.0760, 2.1407) -- (2.6280, 1.0760, 2.1467) -- cycle;
\fill[blue!21.4, opacity=0.7] (2.6280, 1.0760, 2.1467) -- (2.6750, 1.0760, 2.1407) -- (2.6750, 1.1290, 2.1431) -- (2.6280, 1.1290, 2.1491) -- cycle;
\fill[blue!19.6, opacity=0.7] (2.6280, 1.1290, 2.1491) -- (2.6750, 1.1290, 2.1431) -- (2.6750, 1.1820, 2.1452) -- (2.6280, 1.1820, 2.1512) -- cycle;
\fill[blue!18.6, opacity=0.7] (2.6280, 1.1820, 2.1512) -- (2.6750, 1.1820, 2.1452) -- (2.6750, 1.2350, 2.1470) -- (2.6280, 1.2350, 2.1530) -- cycle;
\fill[blue!18.0, opacity=0.7] (2.6280, 1.2350, 2.1530) -- (2.6750, 1.2350, 2.1470) -- (2.6750, 1.2880, 2.1484) -- (2.6280, 1.2880, 2.1545) -- cycle;
\fill[blue!17.8, opacity=0.7] (2.6280, 1.2880, 2.1545) -- (2.6750, 1.2880, 2.1484) -- (2.6750, 1.3410, 2.1496) -- (2.6280, 1.3410, 2.1556) -- cycle;
\fill[blue!17.9, opacity=0.7] (2.6280, 1.3410, 2.1556) -- (2.6750, 1.3410, 2.1496) -- (2.6750, 1.3940, 2.1504) -- (2.6280, 1.3940, 2.1564) -- cycle;
\fill[blue!18.1, opacity=0.7] (2.6280, 1.3940, 2.1564) -- (2.6750, 1.3940, 2.1504) -- (2.6750, 1.4470, 2.1509) -- (2.6280, 1.4470, 2.1569) -- cycle;
\fill[blue!18.8, opacity=0.7] (2.6280, 1.4470, 2.1569) -- (2.6750, 1.4470, 2.1509) -- (2.6750, 1.5000, 2.1511) -- (2.6280, 1.5000, 2.1571) -- cycle;
\fill[blue!19.9, opacity=0.7] (2.6280, 1.5000, 2.1571) -- (2.6750, 1.5000, 2.1511) -- (2.6750, 1.5530, 2.1509) -- (2.6280, 1.5530, 2.1569) -- cycle;
\fill[blue!21.9, opacity=0.7] (2.6280, 1.5530, 2.1569) -- (2.6750, 1.5530, 2.1509) -- (2.6750, 1.6060, 2.1504) -- (2.6280, 1.6060, 2.1564) -- cycle;
\fill[blue!25.3, opacity=0.7] (2.6280, 1.6060, 2.1564) -- (2.6750, 1.6060, 2.1504) -- (2.6750, 1.6590, 2.1496) -- (2.6280, 1.6590, 2.1556) -- cycle;
\fill[blue!31.1, opacity=0.7] (2.6280, 1.6590, 2.1556) -- (2.6750, 1.6590, 2.1496) -- (2.6750, 1.7120, 2.1484) -- (2.6280, 1.7120, 2.1545) -- cycle;
\fill[blue!40.1, opacity=0.7] (2.6280, 1.7120, 2.1545) -- (2.6750, 1.7120, 2.1484) -- (2.6750, 1.7650, 2.1470) -- (2.6280, 1.7650, 2.1530) -- cycle;
\fill[blue!52.2, opacity=0.7] (2.6280, 1.7650, 2.1530) -- (2.6750, 1.7650, 2.1470) -- (2.6750, 1.8180, 2.1452) -- (2.6280, 1.8180, 2.1512) -- cycle;
\fill[blue!65.0, opacity=0.7] (2.6280, 1.8180, 2.1512) -- (2.6750, 1.8180, 2.1452) -- (2.6750, 1.8710, 2.1431) -- (2.6280, 1.8710, 2.1491) -- cycle;
\fill[blue!75.1, opacity=0.7] (2.6280, 1.8710, 2.1491) -- (2.6750, 1.8710, 2.1431) -- (2.6750, 1.9240, 2.1407) -- (2.6280, 1.9240, 2.1467) -- cycle;
\fill[blue!80.5, opacity=0.7] (2.6280, 1.9240, 2.1467) -- (2.6750, 1.9240, 2.1407) -- (2.6750, 1.9770, 2.1380) -- (2.6280, 1.9770, 2.1440) -- cycle;
\fill[blue!80.9, opacity=0.7] (2.6280, 1.9770, 2.1440) -- (2.6750, 1.9770, 2.1380) -- (2.6750, 2.0300, 2.1350) -- (2.6280, 2.0300, 2.1410) -- cycle;
\fill[blue!75.4, opacity=0.7] (2.6280, 2.0300, 2.1410) -- (2.6750, 2.0300, 2.1350) -- (2.6750, 2.0830, 2.1317) -- (2.6280, 2.0830, 2.1377) -- cycle;
\fill[blue!60.2, opacity=0.7] (2.6280, 2.0830, 2.1377) -- (2.6750, 2.0830, 2.1317) -- (2.6750, 2.1360, 2.1281) -- (2.6280, 2.1360, 2.1342) -- cycle;
\fill[blue!37.6, opacity=0.7] (2.6280, 2.1360, 2.1342) -- (2.6750, 2.1360, 2.1281) -- (2.6750, 2.1890, 2.1243) -- (2.6280, 2.1890, 2.1303) -- cycle;
\fill[blue!21.3, opacity=0.7] (2.6280, 2.1890, 2.1303) -- (2.6750, 2.1890, 2.1243) -- (2.6750, 2.2420, 2.1202) -- (2.6280, 2.2420, 2.1263) -- cycle;
\fill[blue!16.0, opacity=0.7] (2.6280, 2.2420, 2.1263) -- (2.6750, 2.2420, 2.1202) -- (2.6750, 2.2950, 2.1159) -- (2.6280, 2.2950, 2.1219) -- cycle;
\fill[blue!15.1, opacity=0.7] (2.6280, 2.2950, 2.1219) -- (2.6750, 2.2950, 2.1159) -- (2.6750, 2.3480, 2.1114) -- (2.6280, 2.3480, 2.1174) -- cycle;
\fill[blue!15.0, opacity=0.7] (2.6280, 2.3480, 2.1174) -- (2.6750, 2.3480, 2.1114) -- (2.6750, 2.4010, 2.1066) -- (2.6280, 2.4010, 2.1126) -- cycle;
\fill[blue!15.0, opacity=0.7] (2.6280, 2.4010, 2.1126) -- (2.6750, 2.4010, 2.1066) -- (2.6750, 2.4540, 2.1016) -- (2.6280, 2.4540, 2.1076) -- cycle;
\fill[blue!15.1, opacity=0.7] (2.6280, 2.4540, 2.1076) -- (2.6750, 2.4540, 2.1016) -- (2.6750, 2.5070, 2.0964) -- (2.6280, 2.5070, 2.1024) -- cycle;
\fill[blue!15.5, opacity=0.7] (2.6280, 2.5070, 2.1024) -- (2.6750, 2.5070, 2.0964) -- (2.6750, 2.5600, 2.0911) -- (2.6280, 2.5600, 2.0971) -- cycle;
\fill[blue!18.3, opacity=0.7] (2.6280, 2.5600, 2.0971) -- (2.6750, 2.5600, 2.0911) -- (2.6750, 2.6130, 2.0855) -- (2.6280, 2.6130, 2.0916) -- cycle;
\fill[blue!27.4, opacity=0.7] (2.6280, 2.6130, 2.0916) -- (2.6750, 2.6130, 2.0855) -- (2.6750, 2.6660, 2.0799) -- (2.6280, 2.6660, 2.0859) -- cycle;
\fill[blue!36.6, opacity=0.7] (2.6280, 2.6660, 2.0859) -- (2.6750, 2.6660, 2.0799) -- (2.6750, 2.7190, 2.0741) -- (2.6280, 2.7190, 2.0801) -- cycle;
\fill[blue!33.1, opacity=0.7] (2.6280, 2.7190, 2.0801) -- (2.6750, 2.7190, 2.0741) -- (2.6750, 2.7720, 2.0681) -- (2.6280, 2.7720, 2.0742) -- cycle;
\fill[blue!21.0, opacity=0.7] (2.6280, 2.7720, 2.0742) -- (2.6750, 2.7720, 2.0681) -- (2.6750, 2.8250, 2.0621) -- (2.6280, 2.8250, 2.0681) -- cycle;
\fill[blue!15.6, opacity=0.7] (2.6280, 2.8250, 2.0681) -- (2.6750, 2.8250, 2.0621) -- (2.6750, 2.8780, 2.0560) -- (2.6280, 2.8780, 2.0620) -- cycle;
\fill[blue!15.0, opacity=0.7] (2.6280, 2.8780, 2.0620) -- (2.6750, 2.8780, 2.0560) -- (2.6750, 2.9310, 2.0498) -- (2.6280, 2.9310, 2.0559) -- cycle;
\fill[blue!15.0, opacity=0.7] (2.6280, 2.9310, 2.0559) -- (2.6750, 2.9310, 2.0498) -- (2.6750, 2.9840, 2.0436) -- (2.6280, 2.9840, 2.0496) -- cycle;
\fill[blue!15.0, opacity=0.7] (2.6280, 2.9840, 2.0496) -- (2.6750, 2.9840, 2.0436) -- (2.6750, 3.0370, 2.0373) -- (2.6280, 3.0370, 2.0434) -- cycle;
\fill[blue!15.0, opacity=0.7] (2.6280, 3.0370, 2.0434) -- (2.6750, 3.0370, 2.0373) -- (2.6750, 3.0900, 2.0311) -- (2.6280, 3.0900, 2.0371) -- cycle;
\fill[blue!16.2, opacity=0.7] (2.6750, -0.0900, 2.0311) -- (2.7220, -0.0900, 2.0249) -- (2.7220, -0.0370, 2.0312) -- (2.6750, -0.0370, 2.0373) -- cycle;
\fill[blue!19.9, opacity=0.7] (2.6750, -0.0370, 2.0373) -- (2.7220, -0.0370, 2.0312) -- (2.7220, 0.0160, 2.0375) -- (2.6750, 0.0160, 2.0436) -- cycle;
\fill[blue!21.7, opacity=0.7] (2.6750, 0.0160, 2.0436) -- (2.7220, 0.0160, 2.0375) -- (2.7220, 0.0690, 2.0437) -- (2.6750, 0.0690, 2.0498) -- cycle;
\fill[blue!18.9, opacity=0.7] (2.6750, 0.0690, 2.0498) -- (2.7220, 0.0690, 2.0437) -- (2.7220, 0.1220, 2.0499) -- (2.6750, 0.1220, 2.0560) -- cycle;
\fill[blue!16.0, opacity=0.7] (2.6750, 0.1220, 2.0560) -- (2.7220, 0.1220, 2.0499) -- (2.7220, 0.1750, 2.0560) -- (2.6750, 0.1750, 2.0621) -- cycle;
\fill[blue!15.1, opacity=0.7] (2.6750, 0.1750, 2.0621) -- (2.7220, 0.1750, 2.0560) -- (2.7220, 0.2280, 2.0620) -- (2.6750, 0.2280, 2.0681) -- cycle;
\fill[blue!15.0, opacity=0.7] (2.6750, 0.2280, 2.0681) -- (2.7220, 0.2280, 2.0620) -- (2.7220, 0.2810, 2.0680) -- (2.6750, 0.2810, 2.0741) -- cycle;
\fill[blue!15.0, opacity=0.7] (2.6750, 0.2810, 2.0741) -- (2.7220, 0.2810, 2.0680) -- (2.7220, 0.3340, 2.0738) -- (2.6750, 0.3340, 2.0799) -- cycle;
\fill[blue!15.0, opacity=0.7] (2.6750, 0.3340, 2.0799) -- (2.7220, 0.3340, 2.0738) -- (2.7220, 0.3870, 2.0794) -- (2.6750, 0.3870, 2.0855) -- cycle;
\fill[blue!15.0, opacity=0.7] (2.6750, 0.3870, 2.0855) -- (2.7220, 0.3870, 2.0794) -- (2.7220, 0.4400, 2.0849) -- (2.6750, 0.4400, 2.0911) -- cycle;
\fill[blue!15.0, opacity=0.7] (2.6750, 0.4400, 2.0911) -- (2.7220, 0.4400, 2.0849) -- (2.7220, 0.4930, 2.0903) -- (2.6750, 0.4930, 2.0964) -- cycle;
\fill[blue!15.3, opacity=0.7] (2.6750, 0.4930, 2.0964) -- (2.7220, 0.4930, 2.0903) -- (2.7220, 0.5460, 2.0955) -- (2.6750, 0.5460, 2.1016) -- cycle;
\fill[blue!17.1, opacity=0.7] (2.6750, 0.5460, 2.1016) -- (2.7220, 0.5460, 2.0955) -- (2.7220, 0.5990, 2.1005) -- (2.6750, 0.5990, 2.1066) -- cycle;
\fill[blue!24.0, opacity=0.7] (2.6750, 0.5990, 2.1066) -- (2.7220, 0.5990, 2.1005) -- (2.7220, 0.6520, 2.1052) -- (2.6750, 0.6520, 2.1114) -- cycle;
\fill[blue!37.9, opacity=0.7] (2.6750, 0.6520, 2.1114) -- (2.7220, 0.6520, 2.1052) -- (2.7220, 0.7050, 2.1098) -- (2.6750, 0.7050, 2.1159) -- cycle;
\fill[blue!53.5, opacity=0.7] (2.6750, 0.7050, 2.1159) -- (2.7220, 0.7050, 2.1098) -- (2.7220, 0.7580, 2.1141) -- (2.6750, 0.7580, 2.1202) -- cycle;
\fill[blue!64.4, opacity=0.7] (2.6750, 0.7580, 2.1202) -- (2.7220, 0.7580, 2.1141) -- (2.7220, 0.8110, 2.1182) -- (2.6750, 0.8110, 2.1243) -- cycle;
\fill[blue!69.2, opacity=0.7] (2.6750, 0.8110, 2.1243) -- (2.7220, 0.8110, 2.1182) -- (2.7220, 0.8640, 2.1220) -- (2.6750, 0.8640, 2.1281) -- cycle;
\fill[blue!69.4, opacity=0.7] (2.6750, 0.8640, 2.1281) -- (2.7220, 0.8640, 2.1220) -- (2.7220, 0.9170, 2.1256) -- (2.6750, 0.9170, 2.1317) -- cycle;
\fill[blue!66.1, opacity=0.7] (2.6750, 0.9170, 2.1317) -- (2.7220, 0.9170, 2.1256) -- (2.7220, 0.9700, 2.1289) -- (2.6750, 0.9700, 2.1350) -- cycle;
\fill[blue!60.6, opacity=0.7] (2.6750, 0.9700, 2.1350) -- (2.7220, 0.9700, 2.1289) -- (2.7220, 1.0230, 2.1319) -- (2.6750, 1.0230, 2.1380) -- cycle;
\fill[blue!54.1, opacity=0.7] (2.6750, 1.0230, 2.1380) -- (2.7220, 1.0230, 2.1319) -- (2.7220, 1.0760, 2.1346) -- (2.6750, 1.0760, 2.1407) -- cycle;
\fill[blue!47.8, opacity=0.7] (2.6750, 1.0760, 2.1407) -- (2.7220, 1.0760, 2.1346) -- (2.7220, 1.1290, 2.1370) -- (2.6750, 1.1290, 2.1431) -- cycle;
\fill[blue!42.6, opacity=0.7] (2.6750, 1.1290, 2.1431) -- (2.7220, 1.1290, 2.1370) -- (2.7220, 1.1820, 2.1391) -- (2.6750, 1.1820, 2.1452) -- cycle;
\fill[blue!38.7, opacity=0.7] (2.6750, 1.1820, 2.1452) -- (2.7220, 1.1820, 2.1391) -- (2.7220, 1.2350, 2.1409) -- (2.6750, 1.2350, 2.1470) -- cycle;
\fill[blue!36.2, opacity=0.7] (2.6750, 1.2350, 2.1470) -- (2.7220, 1.2350, 2.1409) -- (2.7220, 1.2880, 2.1423) -- (2.6750, 1.2880, 2.1484) -- cycle;
\fill[blue!35.1, opacity=0.7] (2.6750, 1.2880, 2.1484) -- (2.7220, 1.2880, 2.1423) -- (2.7220, 1.3410, 2.1435) -- (2.6750, 1.3410, 2.1496) -- cycle;
\fill[blue!35.2, opacity=0.7] (2.6750, 1.3410, 2.1496) -- (2.7220, 1.3410, 2.1435) -- (2.7220, 1.3940, 2.1443) -- (2.6750, 1.3940, 2.1504) -- cycle;
\fill[blue!36.5, opacity=0.7] (2.6750, 1.3940, 2.1504) -- (2.7220, 1.3940, 2.1443) -- (2.7220, 1.4470, 2.1448) -- (2.6750, 1.4470, 2.1509) -- cycle;
\fill[blue!39.2, opacity=0.7] (2.6750, 1.4470, 2.1509) -- (2.7220, 1.4470, 2.1448) -- (2.7220, 1.5000, 2.1449) -- (2.6750, 1.5000, 2.1511) -- cycle;
\fill[blue!43.5, opacity=0.7] (2.6750, 1.5000, 2.1511) -- (2.7220, 1.5000, 2.1449) -- (2.7220, 1.5530, 2.1448) -- (2.6750, 1.5530, 2.1509) -- cycle;
\fill[blue!49.4, opacity=0.7] (2.6750, 1.5530, 2.1509) -- (2.7220, 1.5530, 2.1448) -- (2.7220, 1.6060, 2.1443) -- (2.6750, 1.6060, 2.1504) -- cycle;
\fill[blue!56.8, opacity=0.7] (2.6750, 1.6060, 2.1504) -- (2.7220, 1.6060, 2.1443) -- (2.7220, 1.6590, 2.1435) -- (2.6750, 1.6590, 2.1496) -- cycle;
\fill[blue!64.8, opacity=0.7] (2.6750, 1.6590, 2.1496) -- (2.7220, 1.6590, 2.1435) -- (2.7220, 1.7120, 2.1423) -- (2.6750, 1.7120, 2.1484) -- cycle;
\fill[blue!72.1, opacity=0.7] (2.6750, 1.7120, 2.1484) -- (2.7220, 1.7120, 2.1423) -- (2.7220, 1.7650, 2.1409) -- (2.6750, 1.7650, 2.1470) -- cycle;
\fill[blue!77.4, opacity=0.7] (2.6750, 1.7650, 2.1470) -- (2.7220, 1.7650, 2.1409) -- (2.7220, 1.8180, 2.1391) -- (2.6750, 1.8180, 2.1452) -- cycle;
\fill[blue!80.0, opacity=0.7] (2.6750, 1.8180, 2.1452) -- (2.7220, 1.8180, 2.1391) -- (2.7220, 1.8710, 2.1370) -- (2.6750, 1.8710, 2.1431) -- cycle;
\fill[blue!79.4, opacity=0.7] (2.6750, 1.8710, 2.1431) -- (2.7220, 1.8710, 2.1370) -- (2.7220, 1.9240, 2.1346) -- (2.6750, 1.9240, 2.1407) -- cycle;
\fill[blue!74.1, opacity=0.7] (2.6750, 1.9240, 2.1407) -- (2.7220, 1.9240, 2.1346) -- (2.7220, 1.9770, 2.1319) -- (2.6750, 1.9770, 2.1380) -- cycle;
\fill[blue!61.6, opacity=0.7] (2.6750, 1.9770, 2.1380) -- (2.7220, 1.9770, 2.1319) -- (2.7220, 2.0300, 2.1289) -- (2.6750, 2.0300, 2.1350) -- cycle;
\fill[blue!42.5, opacity=0.7] (2.6750, 2.0300, 2.1350) -- (2.7220, 2.0300, 2.1289) -- (2.7220, 2.0830, 2.1256) -- (2.6750, 2.0830, 2.1317) -- cycle;
\fill[blue!25.4, opacity=0.7] (2.6750, 2.0830, 2.1317) -- (2.7220, 2.0830, 2.1256) -- (2.7220, 2.1360, 2.1220) -- (2.6750, 2.1360, 2.1281) -- cycle;
\fill[blue!17.4, opacity=0.7] (2.6750, 2.1360, 2.1281) -- (2.7220, 2.1360, 2.1220) -- (2.7220, 2.1890, 2.1182) -- (2.6750, 2.1890, 2.1243) -- cycle;
\fill[blue!15.4, opacity=0.7] (2.6750, 2.1890, 2.1243) -- (2.7220, 2.1890, 2.1182) -- (2.7220, 2.2420, 2.1141) -- (2.6750, 2.2420, 2.1202) -- cycle;
\fill[blue!15.1, opacity=0.7] (2.6750, 2.2420, 2.1202) -- (2.7220, 2.2420, 2.1141) -- (2.7220, 2.2950, 2.1098) -- (2.6750, 2.2950, 2.1159) -- cycle;
\fill[blue!15.0, opacity=0.7] (2.6750, 2.2950, 2.1159) -- (2.7220, 2.2950, 2.1098) -- (2.7220, 2.3480, 2.1052) -- (2.6750, 2.3480, 2.1114) -- cycle;
\fill[blue!15.0, opacity=0.7] (2.6750, 2.3480, 2.1114) -- (2.7220, 2.3480, 2.1052) -- (2.7220, 2.4010, 2.1005) -- (2.6750, 2.4010, 2.1066) -- cycle;
\fill[blue!15.1, opacity=0.7] (2.6750, 2.4010, 2.1066) -- (2.7220, 2.4010, 2.1005) -- (2.7220, 2.4540, 2.0955) -- (2.6750, 2.4540, 2.1016) -- cycle;
\fill[blue!15.5, opacity=0.7] (2.6750, 2.4540, 2.1016) -- (2.7220, 2.4540, 2.0955) -- (2.7220, 2.5070, 2.0903) -- (2.6750, 2.5070, 2.0964) -- cycle;
\fill[blue!18.4, opacity=0.7] (2.6750, 2.5070, 2.0964) -- (2.7220, 2.5070, 2.0903) -- (2.7220, 2.5600, 2.0849) -- (2.6750, 2.5600, 2.0911) -- cycle;
\fill[blue!27.1, opacity=0.7] (2.6750, 2.5600, 2.0911) -- (2.7220, 2.5600, 2.0849) -- (2.7220, 2.6130, 2.0794) -- (2.6750, 2.6130, 2.0855) -- cycle;
\fill[blue!35.9, opacity=0.7] (2.6750, 2.6130, 2.0855) -- (2.7220, 2.6130, 2.0794) -- (2.7220, 2.6660, 2.0738) -- (2.6750, 2.6660, 2.0799) -- cycle;
\fill[blue!33.3, opacity=0.7] (2.6750, 2.6660, 2.0799) -- (2.7220, 2.6660, 2.0738) -- (2.7220, 2.7190, 2.0680) -- (2.6750, 2.7190, 2.0741) -- cycle;
\fill[blue!21.9, opacity=0.7] (2.6750, 2.7190, 2.0741) -- (2.7220, 2.7190, 2.0680) -- (2.7220, 2.7720, 2.0620) -- (2.6750, 2.7720, 2.0681) -- cycle;
\fill[blue!15.8, opacity=0.7] (2.6750, 2.7720, 2.0681) -- (2.7220, 2.7720, 2.0620) -- (2.7220, 2.8250, 2.0560) -- (2.6750, 2.8250, 2.0621) -- cycle;
\fill[blue!15.0, opacity=0.7] (2.6750, 2.8250, 2.0621) -- (2.7220, 2.8250, 2.0560) -- (2.7220, 2.8780, 2.0499) -- (2.6750, 2.8780, 2.0560) -- cycle;
\fill[blue!15.0, opacity=0.7] (2.6750, 2.8780, 2.0560) -- (2.7220, 2.8780, 2.0499) -- (2.7220, 2.9310, 2.0437) -- (2.6750, 2.9310, 2.0498) -- cycle;
\fill[blue!15.0, opacity=0.7] (2.6750, 2.9310, 2.0498) -- (2.7220, 2.9310, 2.0437) -- (2.7220, 2.9840, 2.0375) -- (2.6750, 2.9840, 2.0436) -- cycle;
\fill[blue!15.0, opacity=0.7] (2.6750, 2.9840, 2.0436) -- (2.7220, 2.9840, 2.0375) -- (2.7220, 3.0370, 2.0312) -- (2.6750, 3.0370, 2.0373) -- cycle;
\fill[blue!15.0, opacity=0.7] (2.6750, 3.0370, 2.0373) -- (2.7220, 3.0370, 2.0312) -- (2.7220, 3.0900, 2.0249) -- (2.6750, 3.0900, 2.0311) -- cycle;
\fill[blue!15.1, opacity=0.7] (2.7220, -0.0900, 2.0249) -- (2.7690, -0.0900, 2.0188) -- (2.7690, -0.0370, 2.0251) -- (2.7220, -0.0370, 2.0312) -- cycle;
\fill[blue!16.2, opacity=0.7] (2.7220, -0.0370, 2.0312) -- (2.7690, -0.0370, 2.0251) -- (2.7690, 0.0160, 2.0313) -- (2.7220, 0.0160, 2.0375) -- cycle;
\fill[blue!19.8, opacity=0.7] (2.7220, 0.0160, 2.0375) -- (2.7690, 0.0160, 2.0313) -- (2.7690, 0.0690, 2.0375) -- (2.7220, 0.0690, 2.0437) -- cycle;
\fill[blue!22.0, opacity=0.7] (2.7220, 0.0690, 2.0437) -- (2.7690, 0.0690, 2.0375) -- (2.7690, 0.1220, 2.0437) -- (2.7220, 0.1220, 2.0499) -- cycle;
\fill[blue!19.6, opacity=0.7] (2.7220, 0.1220, 2.0499) -- (2.7690, 0.1220, 2.0437) -- (2.7690, 0.1750, 2.0498) -- (2.7220, 0.1750, 2.0560) -- cycle;
\fill[blue!16.4, opacity=0.7] (2.7220, 0.1750, 2.0560) -- (2.7690, 0.1750, 2.0498) -- (2.7690, 0.2280, 2.0559) -- (2.7220, 0.2280, 2.0620) -- cycle;
\fill[blue!15.2, opacity=0.7] (2.7220, 0.2280, 2.0620) -- (2.7690, 0.2280, 2.0559) -- (2.7690, 0.2810, 2.0618) -- (2.7220, 0.2810, 2.0680) -- cycle;
\fill[blue!15.0, opacity=0.7] (2.7220, 0.2810, 2.0680) -- (2.7690, 0.2810, 2.0618) -- (2.7690, 0.3340, 2.0676) -- (2.7220, 0.3340, 2.0738) -- cycle;
\fill[blue!15.0, opacity=0.7] (2.7220, 0.3340, 2.0738) -- (2.7690, 0.3340, 2.0676) -- (2.7690, 0.3870, 2.0733) -- (2.7220, 0.3870, 2.0794) -- cycle;
\fill[blue!15.0, opacity=0.7] (2.7220, 0.3870, 2.0794) -- (2.7690, 0.3870, 2.0733) -- (2.7690, 0.4400, 2.0788) -- (2.7220, 0.4400, 2.0849) -- cycle;
\fill[blue!15.0, opacity=0.7] (2.7220, 0.4400, 2.0849) -- (2.7690, 0.4400, 2.0788) -- (2.7690, 0.4930, 2.0841) -- (2.7220, 0.4930, 2.0903) -- cycle;
\fill[blue!15.0, opacity=0.7] (2.7220, 0.4930, 2.0903) -- (2.7690, 0.4930, 2.0841) -- (2.7690, 0.5460, 2.0893) -- (2.7220, 0.5460, 2.0955) -- cycle;
\fill[blue!15.1, opacity=0.7] (2.7220, 0.5460, 2.0955) -- (2.7690, 0.5460, 2.0893) -- (2.7690, 0.5990, 2.0943) -- (2.7220, 0.5990, 2.1005) -- cycle;
\fill[blue!15.5, opacity=0.7] (2.7220, 0.5990, 2.1005) -- (2.7690, 0.5990, 2.0943) -- (2.7690, 0.6520, 2.0991) -- (2.7220, 0.6520, 2.1052) -- cycle;
\fill[blue!17.5, opacity=0.7] (2.7220, 0.6520, 2.1052) -- (2.7690, 0.6520, 2.0991) -- (2.7690, 0.7050, 2.1036) -- (2.7220, 0.7050, 2.1098) -- cycle;
\fill[blue!23.8, opacity=0.7] (2.7220, 0.7050, 2.1098) -- (2.7690, 0.7050, 2.1036) -- (2.7690, 0.7580, 2.1079) -- (2.7220, 0.7580, 2.1141) -- cycle;
\fill[blue!35.4, opacity=0.7] (2.7220, 0.7580, 2.1141) -- (2.7690, 0.7580, 2.1079) -- (2.7690, 0.8110, 2.1120) -- (2.7220, 0.8110, 2.1182) -- cycle;
\fill[blue!49.0, opacity=0.7] (2.7220, 0.8110, 2.1182) -- (2.7690, 0.8110, 2.1120) -- (2.7690, 0.8640, 2.1159) -- (2.7220, 0.8640, 2.1220) -- cycle;
\fill[blue!60.2, opacity=0.7] (2.7220, 0.8640, 2.1220) -- (2.7690, 0.8640, 2.1159) -- (2.7690, 0.9170, 2.1194) -- (2.7220, 0.9170, 2.1256) -- cycle;
\fill[blue!67.3, opacity=0.7] (2.7220, 0.9170, 2.1256) -- (2.7690, 0.9170, 2.1194) -- (2.7690, 0.9700, 2.1227) -- (2.7220, 0.9700, 2.1289) -- cycle;
\fill[blue!70.8, opacity=0.7] (2.7220, 0.9700, 2.1289) -- (2.7690, 0.9700, 2.1227) -- (2.7690, 1.0230, 2.1257) -- (2.7220, 1.0230, 2.1319) -- cycle;
\fill[blue!71.9, opacity=0.7] (2.7220, 1.0230, 2.1319) -- (2.7690, 1.0230, 2.1257) -- (2.7690, 1.0760, 2.1284) -- (2.7220, 1.0760, 2.1346) -- cycle;
\fill[blue!71.5, opacity=0.7] (2.7220, 1.0760, 2.1346) -- (2.7690, 1.0760, 2.1284) -- (2.7690, 1.1290, 2.1308) -- (2.7220, 1.1290, 2.1370) -- cycle;
\fill[blue!70.3, opacity=0.7] (2.7220, 1.1290, 2.1370) -- (2.7690, 1.1290, 2.1308) -- (2.7690, 1.1820, 2.1329) -- (2.7220, 1.1820, 2.1391) -- cycle;
\fill[blue!69.0, opacity=0.7] (2.7220, 1.1820, 2.1391) -- (2.7690, 1.1820, 2.1329) -- (2.7690, 1.2350, 2.1347) -- (2.7220, 1.2350, 2.1409) -- cycle;
\fill[blue!67.9, opacity=0.7] (2.7220, 1.2350, 2.1409) -- (2.7690, 1.2350, 2.1347) -- (2.7690, 1.2880, 2.1361) -- (2.7220, 1.2880, 2.1423) -- cycle;
\fill[blue!67.5, opacity=0.7] (2.7220, 1.2880, 2.1423) -- (2.7690, 1.2880, 2.1361) -- (2.7690, 1.3410, 2.1373) -- (2.7220, 1.3410, 2.1435) -- cycle;
\fill[blue!67.8, opacity=0.7] (2.7220, 1.3410, 2.1435) -- (2.7690, 1.3410, 2.1373) -- (2.7690, 1.3940, 2.1381) -- (2.7220, 1.3940, 2.1443) -- cycle;
\fill[blue!69.0, opacity=0.7] (2.7220, 1.3940, 2.1443) -- (2.7690, 1.3940, 2.1381) -- (2.7690, 1.4470, 2.1386) -- (2.7220, 1.4470, 2.1448) -- cycle;
\fill[blue!70.9, opacity=0.7] (2.7220, 1.4470, 2.1448) -- (2.7690, 1.4470, 2.1386) -- (2.7690, 1.5000, 2.1388) -- (2.7220, 1.5000, 2.1449) -- cycle;
\fill[blue!73.2, opacity=0.7] (2.7220, 1.5000, 2.1449) -- (2.7690, 1.5000, 2.1388) -- (2.7690, 1.5530, 2.1386) -- (2.7220, 1.5530, 2.1448) -- cycle;
\fill[blue!75.6, opacity=0.7] (2.7220, 1.5530, 2.1448) -- (2.7690, 1.5530, 2.1386) -- (2.7690, 1.6060, 2.1381) -- (2.7220, 1.6060, 2.1443) -- cycle;
\fill[blue!77.6, opacity=0.7] (2.7220, 1.6060, 2.1443) -- (2.7690, 1.6060, 2.1381) -- (2.7690, 1.6590, 2.1373) -- (2.7220, 1.6590, 2.1435) -- cycle;
\fill[blue!78.5, opacity=0.7] (2.7220, 1.6590, 2.1435) -- (2.7690, 1.6590, 2.1373) -- (2.7690, 1.7120, 2.1361) -- (2.7220, 1.7120, 2.1423) -- cycle;
\fill[blue!77.9, opacity=0.7] (2.7220, 1.7120, 2.1423) -- (2.7690, 1.7120, 2.1361) -- (2.7690, 1.7650, 2.1347) -- (2.7220, 1.7650, 2.1409) -- cycle;
\fill[blue!74.8, opacity=0.7] (2.7220, 1.7650, 2.1409) -- (2.7690, 1.7650, 2.1347) -- (2.7690, 1.8180, 2.1329) -- (2.7220, 1.8180, 2.1391) -- cycle;
\fill[blue!67.8, opacity=0.7] (2.7220, 1.8180, 2.1391) -- (2.7690, 1.8180, 2.1329) -- (2.7690, 1.8710, 2.1308) -- (2.7220, 1.8710, 2.1370) -- cycle;
\fill[blue!55.5, opacity=0.7] (2.7220, 1.8710, 2.1370) -- (2.7690, 1.8710, 2.1308) -- (2.7690, 1.9240, 2.1284) -- (2.7220, 1.9240, 2.1346) -- cycle;
\fill[blue!39.6, opacity=0.7] (2.7220, 1.9240, 2.1346) -- (2.7690, 1.9240, 2.1284) -- (2.7690, 1.9770, 2.1257) -- (2.7220, 1.9770, 2.1319) -- cycle;
\fill[blue!25.7, opacity=0.7] (2.7220, 1.9770, 2.1319) -- (2.7690, 1.9770, 2.1257) -- (2.7690, 2.0300, 2.1227) -- (2.7220, 2.0300, 2.1289) -- cycle;
\fill[blue!18.1, opacity=0.7] (2.7220, 2.0300, 2.1289) -- (2.7690, 2.0300, 2.1227) -- (2.7690, 2.0830, 2.1194) -- (2.7220, 2.0830, 2.1256) -- cycle;
\fill[blue!15.6, opacity=0.7] (2.7220, 2.0830, 2.1256) -- (2.7690, 2.0830, 2.1194) -- (2.7690, 2.1360, 2.1159) -- (2.7220, 2.1360, 2.1220) -- cycle;
\fill[blue!15.1, opacity=0.7] (2.7220, 2.1360, 2.1220) -- (2.7690, 2.1360, 2.1159) -- (2.7690, 2.1890, 2.1120) -- (2.7220, 2.1890, 2.1182) -- cycle;
\fill[blue!15.0, opacity=0.7] (2.7220, 2.1890, 2.1182) -- (2.7690, 2.1890, 2.1120) -- (2.7690, 2.2420, 2.1079) -- (2.7220, 2.2420, 2.1141) -- cycle;
\fill[blue!15.0, opacity=0.7] (2.7220, 2.2420, 2.1141) -- (2.7690, 2.2420, 2.1079) -- (2.7690, 2.2950, 2.1036) -- (2.7220, 2.2950, 2.1098) -- cycle;
\fill[blue!15.0, opacity=0.7] (2.7220, 2.2950, 2.1098) -- (2.7690, 2.2950, 2.1036) -- (2.7690, 2.3480, 2.0991) -- (2.7220, 2.3480, 2.1052) -- cycle;
\fill[blue!15.1, opacity=0.7] (2.7220, 2.3480, 2.1052) -- (2.7690, 2.3480, 2.0991) -- (2.7690, 2.4010, 2.0943) -- (2.7220, 2.4010, 2.1005) -- cycle;
\fill[blue!15.8, opacity=0.7] (2.7220, 2.4010, 2.1005) -- (2.7690, 2.4010, 2.0943) -- (2.7690, 2.4540, 2.0893) -- (2.7220, 2.4540, 2.0955) -- cycle;
\fill[blue!19.2, opacity=0.7] (2.7220, 2.4540, 2.0955) -- (2.7690, 2.4540, 2.0893) -- (2.7690, 2.5070, 2.0841) -- (2.7220, 2.5070, 2.0903) -- cycle;
\fill[blue!27.8, opacity=0.7] (2.7220, 2.5070, 2.0903) -- (2.7690, 2.5070, 2.0841) -- (2.7690, 2.5600, 2.0788) -- (2.7220, 2.5600, 2.0849) -- cycle;
\fill[blue!35.5, opacity=0.7] (2.7220, 2.5600, 2.0849) -- (2.7690, 2.5600, 2.0788) -- (2.7690, 2.6130, 2.0733) -- (2.7220, 2.6130, 2.0794) -- cycle;
\fill[blue!32.7, opacity=0.7] (2.7220, 2.6130, 2.0794) -- (2.7690, 2.6130, 2.0733) -- (2.7690, 2.6660, 2.0676) -- (2.7220, 2.6660, 2.0738) -- cycle;
\fill[blue!22.0, opacity=0.7] (2.7220, 2.6660, 2.0738) -- (2.7690, 2.6660, 2.0676) -- (2.7690, 2.7190, 2.0618) -- (2.7220, 2.7190, 2.0680) -- cycle;
\fill[blue!15.9, opacity=0.7] (2.7220, 2.7190, 2.0680) -- (2.7690, 2.7190, 2.0618) -- (2.7690, 2.7720, 2.0559) -- (2.7220, 2.7720, 2.0620) -- cycle;
\fill[blue!15.0, opacity=0.7] (2.7220, 2.7720, 2.0620) -- (2.7690, 2.7720, 2.0559) -- (2.7690, 2.8250, 2.0498) -- (2.7220, 2.8250, 2.0560) -- cycle;
\fill[blue!15.0, opacity=0.7] (2.7220, 2.8250, 2.0560) -- (2.7690, 2.8250, 2.0498) -- (2.7690, 2.8780, 2.0437) -- (2.7220, 2.8780, 2.0499) -- cycle;
\fill[blue!15.0, opacity=0.7] (2.7220, 2.8780, 2.0499) -- (2.7690, 2.8780, 2.0437) -- (2.7690, 2.9310, 2.0375) -- (2.7220, 2.9310, 2.0437) -- cycle;
\fill[blue!15.0, opacity=0.7] (2.7220, 2.9310, 2.0437) -- (2.7690, 2.9310, 2.0375) -- (2.7690, 2.9840, 2.0313) -- (2.7220, 2.9840, 2.0375) -- cycle;
\fill[blue!15.0, opacity=0.7] (2.7220, 2.9840, 2.0375) -- (2.7690, 2.9840, 2.0313) -- (2.7690, 3.0370, 2.0251) -- (2.7220, 3.0370, 2.0312) -- cycle;
\fill[blue!15.0, opacity=0.7] (2.7220, 3.0370, 2.0312) -- (2.7690, 3.0370, 2.0251) -- (2.7690, 3.0900, 2.0188) -- (2.7220, 3.0900, 2.0249) -- cycle;
\fill[blue!15.0, opacity=0.7] (2.7690, -0.0900, 2.0188) -- (2.8160, -0.0900, 2.0125) -- (2.8160, -0.0370, 2.0188) -- (2.7690, -0.0370, 2.0251) -- cycle;
\fill[blue!15.1, opacity=0.7] (2.7690, -0.0370, 2.0251) -- (2.8160, -0.0370, 2.0188) -- (2.8160, 0.0160, 2.0251) -- (2.7690, 0.0160, 2.0313) -- cycle;
\fill[blue!16.0, opacity=0.7] (2.7690, 0.0160, 2.0313) -- (2.8160, 0.0160, 2.0251) -- (2.8160, 0.0690, 2.0313) -- (2.7690, 0.0690, 2.0375) -- cycle;
\fill[blue!19.4, opacity=0.7] (2.7690, 0.0690, 2.0375) -- (2.8160, 0.0690, 2.0313) -- (2.8160, 0.1220, 2.0375) -- (2.7690, 0.1220, 2.0437) -- cycle;
\fill[blue!22.2, opacity=0.7] (2.7690, 0.1220, 2.0437) -- (2.8160, 0.1220, 2.0375) -- (2.8160, 0.1750, 2.0436) -- (2.7690, 0.1750, 2.0498) -- cycle;
\fill[blue!20.7, opacity=0.7] (2.7690, 0.1750, 2.0498) -- (2.8160, 0.1750, 2.0436) -- (2.8160, 0.2280, 2.0496) -- (2.7690, 0.2280, 2.0559) -- cycle;
\fill[blue!17.3, opacity=0.7] (2.7690, 0.2280, 2.0559) -- (2.8160, 0.2280, 2.0496) -- (2.8160, 0.2810, 2.0555) -- (2.7690, 0.2810, 2.0618) -- cycle;
\fill[blue!15.5, opacity=0.7] (2.7690, 0.2810, 2.0618) -- (2.8160, 0.2810, 2.0555) -- (2.8160, 0.3340, 2.0614) -- (2.7690, 0.3340, 2.0676) -- cycle;
\fill[blue!15.1, opacity=0.7] (2.7690, 0.3340, 2.0676) -- (2.8160, 0.3340, 2.0614) -- (2.8160, 0.3870, 2.0670) -- (2.7690, 0.3870, 2.0733) -- cycle;
\fill[blue!15.0, opacity=0.7] (2.7690, 0.3870, 2.0733) -- (2.8160, 0.3870, 2.0670) -- (2.8160, 0.4400, 2.0725) -- (2.7690, 0.4400, 2.0788) -- cycle;
\fill[blue!15.0, opacity=0.7] (2.7690, 0.4400, 2.0788) -- (2.8160, 0.4400, 2.0725) -- (2.8160, 0.4930, 2.0779) -- (2.7690, 0.4930, 2.0841) -- cycle;
\fill[blue!15.0, opacity=0.7] (2.7690, 0.4930, 2.0841) -- (2.8160, 0.4930, 2.0779) -- (2.8160, 0.5460, 2.0831) -- (2.7690, 0.5460, 2.0893) -- cycle;
\fill[blue!15.0, opacity=0.7] (2.7690, 0.5460, 2.0893) -- (2.8160, 0.5460, 2.0831) -- (2.8160, 0.5990, 2.0881) -- (2.7690, 0.5990, 2.0943) -- cycle;
\fill[blue!15.0, opacity=0.7] (2.7690, 0.5990, 2.0943) -- (2.8160, 0.5990, 2.0881) -- (2.8160, 0.6520, 2.0928) -- (2.7690, 0.6520, 2.0991) -- cycle;
\fill[blue!15.1, opacity=0.7] (2.7690, 0.6520, 2.0991) -- (2.8160, 0.6520, 2.0928) -- (2.8160, 0.7050, 2.0974) -- (2.7690, 0.7050, 2.1036) -- cycle;
\fill[blue!15.4, opacity=0.7] (2.7690, 0.7050, 2.1036) -- (2.8160, 0.7050, 2.0974) -- (2.8160, 0.7580, 2.1017) -- (2.7690, 0.7580, 2.1079) -- cycle;
\fill[blue!16.7, opacity=0.7] (2.7690, 0.7580, 2.1079) -- (2.8160, 0.7580, 2.1017) -- (2.8160, 0.8110, 2.1058) -- (2.7690, 0.8110, 2.1120) -- cycle;
\fill[blue!20.5, opacity=0.7] (2.7690, 0.8110, 2.1120) -- (2.8160, 0.8110, 2.1058) -- (2.8160, 0.8640, 2.1096) -- (2.7690, 0.8640, 2.1159) -- cycle;
\fill[blue!27.7, opacity=0.7] (2.7690, 0.8640, 2.1159) -- (2.8160, 0.8640, 2.1096) -- (2.8160, 0.9170, 2.1132) -- (2.7690, 0.9170, 2.1194) -- cycle;
\fill[blue!37.6, opacity=0.7] (2.7690, 0.9170, 2.1194) -- (2.8160, 0.9170, 2.1132) -- (2.8160, 0.9700, 2.1165) -- (2.7690, 0.9700, 2.1227) -- cycle;
\fill[blue!47.8, opacity=0.7] (2.7690, 0.9700, 2.1227) -- (2.8160, 0.9700, 2.1165) -- (2.8160, 1.0230, 2.1195) -- (2.7690, 1.0230, 2.1257) -- cycle;
\fill[blue!56.4, opacity=0.7] (2.7690, 1.0230, 2.1257) -- (2.8160, 1.0230, 2.1195) -- (2.8160, 1.0760, 2.1222) -- (2.7690, 1.0760, 2.1284) -- cycle;
\fill[blue!62.8, opacity=0.7] (2.7690, 1.0760, 2.1284) -- (2.8160, 1.0760, 2.1222) -- (2.8160, 1.1290, 2.1246) -- (2.7690, 1.1290, 2.1308) -- cycle;
\fill[blue!67.1, opacity=0.7] (2.7690, 1.1290, 2.1308) -- (2.8160, 1.1290, 2.1246) -- (2.8160, 1.1820, 2.1267) -- (2.7690, 1.1820, 2.1329) -- cycle;
\fill[blue!70.0, opacity=0.7] (2.7690, 1.1820, 2.1329) -- (2.8160, 1.1820, 2.1267) -- (2.8160, 1.2350, 2.1285) -- (2.7690, 1.2350, 2.1347) -- cycle;
\fill[blue!71.8, opacity=0.7] (2.7690, 1.2350, 2.1347) -- (2.8160, 1.2350, 2.1285) -- (2.8160, 1.2880, 2.1299) -- (2.7690, 1.2880, 2.1361) -- cycle;
\fill[blue!73.0, opacity=0.7] (2.7690, 1.2880, 2.1361) -- (2.8160, 1.2880, 2.1299) -- (2.8160, 1.3410, 2.1311) -- (2.7690, 1.3410, 2.1373) -- cycle;
\fill[blue!73.7, opacity=0.7] (2.7690, 1.3410, 2.1373) -- (2.8160, 1.3410, 2.1311) -- (2.8160, 1.3940, 2.1319) -- (2.7690, 1.3940, 2.1381) -- cycle;
\fill[blue!74.0, opacity=0.7] (2.7690, 1.3940, 2.1381) -- (2.8160, 1.3940, 2.1319) -- (2.8160, 1.4470, 2.1324) -- (2.7690, 1.4470, 2.1386) -- cycle;
\fill[blue!73.8, opacity=0.7] (2.7690, 1.4470, 2.1386) -- (2.8160, 1.4470, 2.1324) -- (2.8160, 1.5000, 2.1325) -- (2.7690, 1.5000, 2.1388) -- cycle;
\fill[blue!72.9, opacity=0.7] (2.7690, 1.5000, 2.1388) -- (2.8160, 1.5000, 2.1325) -- (2.8160, 1.5530, 2.1324) -- (2.7690, 1.5530, 2.1386) -- cycle;
\fill[blue!71.0, opacity=0.7] (2.7690, 1.5530, 2.1386) -- (2.8160, 1.5530, 2.1324) -- (2.8160, 1.6060, 2.1319) -- (2.7690, 1.6060, 2.1381) -- cycle;
\fill[blue!67.3, opacity=0.7] (2.7690, 1.6060, 2.1381) -- (2.8160, 1.6060, 2.1319) -- (2.8160, 1.6590, 2.1311) -- (2.7690, 1.6590, 2.1373) -- cycle;
\fill[blue!61.4, opacity=0.7] (2.7690, 1.6590, 2.1373) -- (2.8160, 1.6590, 2.1311) -- (2.8160, 1.7120, 2.1299) -- (2.7690, 1.7120, 2.1361) -- cycle;
\fill[blue!52.6, opacity=0.7] (2.7690, 1.7120, 2.1361) -- (2.8160, 1.7120, 2.1299) -- (2.8160, 1.7650, 2.1285) -- (2.7690, 1.7650, 2.1347) -- cycle;
\fill[blue!41.6, opacity=0.7] (2.7690, 1.7650, 2.1347) -- (2.8160, 1.7650, 2.1285) -- (2.8160, 1.8180, 2.1267) -- (2.7690, 1.8180, 2.1329) -- cycle;
\fill[blue!30.3, opacity=0.7] (2.7690, 1.8180, 2.1329) -- (2.8160, 1.8180, 2.1267) -- (2.8160, 1.8710, 2.1246) -- (2.7690, 1.8710, 2.1308) -- cycle;
\fill[blue!21.8, opacity=0.7] (2.7690, 1.8710, 2.1308) -- (2.8160, 1.8710, 2.1246) -- (2.8160, 1.9240, 2.1222) -- (2.7690, 1.9240, 2.1284) -- cycle;
\fill[blue!17.3, opacity=0.7] (2.7690, 1.9240, 2.1284) -- (2.8160, 1.9240, 2.1222) -- (2.8160, 1.9770, 2.1195) -- (2.7690, 1.9770, 2.1257) -- cycle;
\fill[blue!15.6, opacity=0.7] (2.7690, 1.9770, 2.1257) -- (2.8160, 1.9770, 2.1195) -- (2.8160, 2.0300, 2.1165) -- (2.7690, 2.0300, 2.1227) -- cycle;
\fill[blue!15.1, opacity=0.7] (2.7690, 2.0300, 2.1227) -- (2.8160, 2.0300, 2.1165) -- (2.8160, 2.0830, 2.1132) -- (2.7690, 2.0830, 2.1194) -- cycle;
\fill[blue!15.0, opacity=0.7] (2.7690, 2.0830, 2.1194) -- (2.8160, 2.0830, 2.1132) -- (2.8160, 2.1360, 2.1096) -- (2.7690, 2.1360, 2.1159) -- cycle;
\fill[blue!15.0, opacity=0.7] (2.7690, 2.1360, 2.1159) -- (2.8160, 2.1360, 2.1096) -- (2.8160, 2.1890, 2.1058) -- (2.7690, 2.1890, 2.1120) -- cycle;
\fill[blue!15.0, opacity=0.7] (2.7690, 2.1890, 2.1120) -- (2.8160, 2.1890, 2.1058) -- (2.8160, 2.2420, 2.1017) -- (2.7690, 2.2420, 2.1079) -- cycle;
\fill[blue!15.0, opacity=0.7] (2.7690, 2.2420, 2.1079) -- (2.8160, 2.2420, 2.1017) -- (2.8160, 2.2950, 2.0974) -- (2.7690, 2.2950, 2.1036) -- cycle;
\fill[blue!15.2, opacity=0.7] (2.7690, 2.2950, 2.1036) -- (2.8160, 2.2950, 2.0974) -- (2.8160, 2.3480, 2.0928) -- (2.7690, 2.3480, 2.0991) -- cycle;
\fill[blue!16.3, opacity=0.7] (2.7690, 2.3480, 2.0991) -- (2.8160, 2.3480, 2.0928) -- (2.8160, 2.4010, 2.0881) -- (2.7690, 2.4010, 2.0943) -- cycle;
\fill[blue!20.6, opacity=0.7] (2.7690, 2.4010, 2.0943) -- (2.8160, 2.4010, 2.0881) -- (2.8160, 2.4540, 2.0831) -- (2.7690, 2.4540, 2.0893) -- cycle;
\fill[blue!29.3, opacity=0.7] (2.7690, 2.4540, 2.0893) -- (2.8160, 2.4540, 2.0831) -- (2.8160, 2.5070, 2.0779) -- (2.7690, 2.5070, 2.0841) -- cycle;
\fill[blue!35.3, opacity=0.7] (2.7690, 2.5070, 2.0841) -- (2.8160, 2.5070, 2.0779) -- (2.8160, 2.5600, 2.0725) -- (2.7690, 2.5600, 2.0788) -- cycle;
\fill[blue!31.3, opacity=0.7] (2.7690, 2.5600, 2.0788) -- (2.8160, 2.5600, 2.0725) -- (2.8160, 2.6130, 2.0670) -- (2.7690, 2.6130, 2.0733) -- cycle;
\fill[blue!21.2, opacity=0.7] (2.7690, 2.6130, 2.0733) -- (2.8160, 2.6130, 2.0670) -- (2.8160, 2.6660, 2.0614) -- (2.7690, 2.6660, 2.0676) -- cycle;
\fill[blue!15.9, opacity=0.7] (2.7690, 2.6660, 2.0676) -- (2.8160, 2.6660, 2.0614) -- (2.8160, 2.7190, 2.0555) -- (2.7690, 2.7190, 2.0618) -- cycle;
\fill[blue!15.0, opacity=0.7] (2.7690, 2.7190, 2.0618) -- (2.8160, 2.7190, 2.0555) -- (2.8160, 2.7720, 2.0496) -- (2.7690, 2.7720, 2.0559) -- cycle;
\fill[blue!15.0, opacity=0.7] (2.7690, 2.7720, 2.0559) -- (2.8160, 2.7720, 2.0496) -- (2.8160, 2.8250, 2.0436) -- (2.7690, 2.8250, 2.0498) -- cycle;
\fill[blue!15.0, opacity=0.7] (2.7690, 2.8250, 2.0498) -- (2.8160, 2.8250, 2.0436) -- (2.8160, 2.8780, 2.0375) -- (2.7690, 2.8780, 2.0437) -- cycle;
\fill[blue!15.0, opacity=0.7] (2.7690, 2.8780, 2.0437) -- (2.8160, 2.8780, 2.0375) -- (2.8160, 2.9310, 2.0313) -- (2.7690, 2.9310, 2.0375) -- cycle;
\fill[blue!15.0, opacity=0.7] (2.7690, 2.9310, 2.0375) -- (2.8160, 2.9310, 2.0313) -- (2.8160, 2.9840, 2.0251) -- (2.7690, 2.9840, 2.0313) -- cycle;
\fill[blue!15.0, opacity=0.7] (2.7690, 2.9840, 2.0313) -- (2.8160, 2.9840, 2.0251) -- (2.8160, 3.0370, 2.0188) -- (2.7690, 3.0370, 2.0251) -- cycle;
\fill[blue!15.2, opacity=0.7] (2.7690, 3.0370, 2.0251) -- (2.8160, 3.0370, 2.0188) -- (2.8160, 3.0900, 2.0125) -- (2.7690, 3.0900, 2.0188) -- cycle;
\fill[blue!15.0, opacity=0.7] (2.8160, -0.0900, 2.0125) -- (2.8630, -0.0900, 2.0063) -- (2.8630, -0.0370, 2.0126) -- (2.8160, -0.0370, 2.0188) -- cycle;
\fill[blue!15.0, opacity=0.7] (2.8160, -0.0370, 2.0188) -- (2.8630, -0.0370, 2.0126) -- (2.8630, 0.0160, 2.0188) -- (2.8160, 0.0160, 2.0251) -- cycle;
\fill[blue!15.0, opacity=0.7] (2.8160, 0.0160, 2.0251) -- (2.8630, 0.0160, 2.0188) -- (2.8630, 0.0690, 2.0251) -- (2.8160, 0.0690, 2.0313) -- cycle;
\fill[blue!15.7, opacity=0.7] (2.8160, 0.0690, 2.0313) -- (2.8630, 0.0690, 2.0251) -- (2.8630, 0.1220, 2.0312) -- (2.8160, 0.1220, 2.0375) -- cycle;
\fill[blue!18.5, opacity=0.7] (2.8160, 0.1220, 2.0375) -- (2.8630, 0.1220, 2.0312) -- (2.8630, 0.1750, 2.0373) -- (2.8160, 0.1750, 2.0436) -- cycle;
\fill[blue!22.0, opacity=0.7] (2.8160, 0.1750, 2.0436) -- (2.8630, 0.1750, 2.0373) -- (2.8630, 0.2280, 2.0434) -- (2.8160, 0.2280, 2.0496) -- cycle;
\fill[blue!21.9, opacity=0.7] (2.8160, 0.2280, 2.0496) -- (2.8630, 0.2280, 2.0434) -- (2.8630, 0.2810, 2.0493) -- (2.8160, 0.2810, 2.0555) -- cycle;
\fill[blue!18.8, opacity=0.7] (2.8160, 0.2810, 2.0555) -- (2.8630, 0.2810, 2.0493) -- (2.8630, 0.3340, 2.0551) -- (2.8160, 0.3340, 2.0614) -- cycle;
\fill[blue!16.2, opacity=0.7] (2.8160, 0.3340, 2.0614) -- (2.8630, 0.3340, 2.0551) -- (2.8630, 0.3870, 2.0608) -- (2.8160, 0.3870, 2.0670) -- cycle;
\fill[blue!15.2, opacity=0.7] (2.8160, 0.3870, 2.0670) -- (2.8630, 0.3870, 2.0608) -- (2.8630, 0.4400, 2.0663) -- (2.8160, 0.4400, 2.0725) -- cycle;
\fill[blue!15.0, opacity=0.7] (2.8160, 0.4400, 2.0725) -- (2.8630, 0.4400, 2.0663) -- (2.8630, 0.4930, 2.0716) -- (2.8160, 0.4930, 2.0779) -- cycle;
\fill[blue!15.0, opacity=0.7] (2.8160, 0.4930, 2.0779) -- (2.8630, 0.4930, 2.0716) -- (2.8630, 0.5460, 2.0768) -- (2.8160, 0.5460, 2.0831) -- cycle;
\fill[blue!15.0, opacity=0.7] (2.8160, 0.5460, 2.0831) -- (2.8630, 0.5460, 2.0768) -- (2.8630, 0.5990, 2.0818) -- (2.8160, 0.5990, 2.0881) -- cycle;
\fill[blue!15.0, opacity=0.7] (2.8160, 0.5990, 2.0881) -- (2.8630, 0.5990, 2.0818) -- (2.8630, 0.6520, 2.0866) -- (2.8160, 0.6520, 2.0928) -- cycle;
\fill[blue!15.0, opacity=0.7] (2.8160, 0.6520, 2.0928) -- (2.8630, 0.6520, 2.0866) -- (2.8630, 0.7050, 2.0911) -- (2.8160, 0.7050, 2.0974) -- cycle;
\fill[blue!15.0, opacity=0.7] (2.8160, 0.7050, 2.0974) -- (2.8630, 0.7050, 2.0911) -- (2.8630, 0.7580, 2.0955) -- (2.8160, 0.7580, 2.1017) -- cycle;
\fill[blue!15.0, opacity=0.7] (2.8160, 0.7580, 2.1017) -- (2.8630, 0.7580, 2.0955) -- (2.8630, 0.8110, 2.0995) -- (2.8160, 0.8110, 2.1058) -- cycle;
\fill[blue!15.2, opacity=0.7] (2.8160, 0.8110, 2.1058) -- (2.8630, 0.8110, 2.0995) -- (2.8630, 0.8640, 2.1034) -- (2.8160, 0.8640, 2.1096) -- cycle;
\fill[blue!15.7, opacity=0.7] (2.8160, 0.8640, 2.1096) -- (2.8630, 0.8640, 2.1034) -- (2.8630, 0.9170, 2.1069) -- (2.8160, 0.9170, 2.1132) -- cycle;
\fill[blue!16.9, opacity=0.7] (2.8160, 0.9170, 2.1132) -- (2.8630, 0.9170, 2.1069) -- (2.8630, 0.9700, 2.1102) -- (2.8160, 0.9700, 2.1165) -- cycle;
\fill[blue!19.3, opacity=0.7] (2.8160, 0.9700, 2.1165) -- (2.8630, 0.9700, 2.1102) -- (2.8630, 1.0230, 2.1132) -- (2.8160, 1.0230, 2.1195) -- cycle;
\fill[blue!23.2, opacity=0.7] (2.8160, 1.0230, 2.1195) -- (2.8630, 1.0230, 2.1132) -- (2.8630, 1.0760, 2.1159) -- (2.8160, 1.0760, 2.1222) -- cycle;
\fill[blue!28.1, opacity=0.7] (2.8160, 1.0760, 2.1222) -- (2.8630, 1.0760, 2.1159) -- (2.8630, 1.1290, 2.1183) -- (2.8160, 1.1290, 2.1246) -- cycle;
\fill[blue!33.3, opacity=0.7] (2.8160, 1.1290, 2.1246) -- (2.8630, 1.1290, 2.1183) -- (2.8630, 1.1820, 2.1204) -- (2.8160, 1.1820, 2.1267) -- cycle;
\fill[blue!38.0, opacity=0.7] (2.8160, 1.1820, 2.1267) -- (2.8630, 1.1820, 2.1204) -- (2.8630, 1.2350, 2.1222) -- (2.8160, 1.2350, 2.1285) -- cycle;
\fill[blue!41.7, opacity=0.7] (2.8160, 1.2350, 2.1285) -- (2.8630, 1.2350, 2.1222) -- (2.8630, 1.2880, 2.1237) -- (2.8160, 1.2880, 2.1299) -- cycle;
\fill[blue!44.1, opacity=0.7] (2.8160, 1.2880, 2.1299) -- (2.8630, 1.2880, 2.1237) -- (2.8630, 1.3410, 2.1248) -- (2.8160, 1.3410, 2.1311) -- cycle;
\fill[blue!45.2, opacity=0.7] (2.8160, 1.3410, 2.1311) -- (2.8630, 1.3410, 2.1248) -- (2.8630, 1.3940, 2.1256) -- (2.8160, 1.3940, 2.1319) -- cycle;
\fill[blue!44.8, opacity=0.7] (2.8160, 1.3940, 2.1319) -- (2.8630, 1.3940, 2.1256) -- (2.8630, 1.4470, 2.1261) -- (2.8160, 1.4470, 2.1324) -- cycle;
\fill[blue!43.0, opacity=0.7] (2.8160, 1.4470, 2.1324) -- (2.8630, 1.4470, 2.1261) -- (2.8630, 1.5000, 2.1263) -- (2.8160, 1.5000, 2.1325) -- cycle;
\fill[blue!39.7, opacity=0.7] (2.8160, 1.5000, 2.1325) -- (2.8630, 1.5000, 2.1263) -- (2.8630, 1.5530, 2.1261) -- (2.8160, 1.5530, 2.1324) -- cycle;
\fill[blue!35.2, opacity=0.7] (2.8160, 1.5530, 2.1324) -- (2.8630, 1.5530, 2.1261) -- (2.8630, 1.6060, 2.1256) -- (2.8160, 1.6060, 2.1319) -- cycle;
\fill[blue!29.9, opacity=0.7] (2.8160, 1.6060, 2.1319) -- (2.8630, 1.6060, 2.1256) -- (2.8630, 1.6590, 2.1248) -- (2.8160, 1.6590, 2.1311) -- cycle;
\fill[blue!24.6, opacity=0.7] (2.8160, 1.6590, 2.1311) -- (2.8630, 1.6590, 2.1248) -- (2.8630, 1.7120, 2.1237) -- (2.8160, 1.7120, 2.1299) -- cycle;
\fill[blue!20.2, opacity=0.7] (2.8160, 1.7120, 2.1299) -- (2.8630, 1.7120, 2.1237) -- (2.8630, 1.7650, 2.1222) -- (2.8160, 1.7650, 2.1285) -- cycle;
\fill[blue!17.4, opacity=0.7] (2.8160, 1.7650, 2.1285) -- (2.8630, 1.7650, 2.1222) -- (2.8630, 1.8180, 2.1204) -- (2.8160, 1.8180, 2.1267) -- cycle;
\fill[blue!15.9, opacity=0.7] (2.8160, 1.8180, 2.1267) -- (2.8630, 1.8180, 2.1204) -- (2.8630, 1.8710, 2.1183) -- (2.8160, 1.8710, 2.1246) -- cycle;
\fill[blue!15.3, opacity=0.7] (2.8160, 1.8710, 2.1246) -- (2.8630, 1.8710, 2.1183) -- (2.8630, 1.9240, 2.1159) -- (2.8160, 1.9240, 2.1222) -- cycle;
\fill[blue!15.1, opacity=0.7] (2.8160, 1.9240, 2.1222) -- (2.8630, 1.9240, 2.1159) -- (2.8630, 1.9770, 2.1132) -- (2.8160, 1.9770, 2.1195) -- cycle;
\fill[blue!15.0, opacity=0.7] (2.8160, 1.9770, 2.1195) -- (2.8630, 1.9770, 2.1132) -- (2.8630, 2.0300, 2.1102) -- (2.8160, 2.0300, 2.1165) -- cycle;
\fill[blue!15.0, opacity=0.7] (2.8160, 2.0300, 2.1165) -- (2.8630, 2.0300, 2.1102) -- (2.8630, 2.0830, 2.1069) -- (2.8160, 2.0830, 2.1132) -- cycle;
\fill[blue!15.0, opacity=0.7] (2.8160, 2.0830, 2.1132) -- (2.8630, 2.0830, 2.1069) -- (2.8630, 2.1360, 2.1034) -- (2.8160, 2.1360, 2.1096) -- cycle;
\fill[blue!15.0, opacity=0.7] (2.8160, 2.1360, 2.1096) -- (2.8630, 2.1360, 2.1034) -- (2.8630, 2.1890, 2.0995) -- (2.8160, 2.1890, 2.1058) -- cycle;
\fill[blue!15.1, opacity=0.7] (2.8160, 2.1890, 2.1058) -- (2.8630, 2.1890, 2.0995) -- (2.8630, 2.2420, 2.0955) -- (2.8160, 2.2420, 2.1017) -- cycle;
\fill[blue!15.5, opacity=0.7] (2.8160, 2.2420, 2.1017) -- (2.8630, 2.2420, 2.0955) -- (2.8630, 2.2950, 2.0911) -- (2.8160, 2.2950, 2.0974) -- cycle;
\fill[blue!17.5, opacity=0.7] (2.8160, 2.2950, 2.0974) -- (2.8630, 2.2950, 2.0911) -- (2.8630, 2.3480, 2.0866) -- (2.8160, 2.3480, 2.0928) -- cycle;
\fill[blue!23.1, opacity=0.7] (2.8160, 2.3480, 2.0928) -- (2.8630, 2.3480, 2.0866) -- (2.8630, 2.4010, 2.0818) -- (2.8160, 2.4010, 2.0881) -- cycle;
\fill[blue!31.3, opacity=0.7] (2.8160, 2.4010, 2.0881) -- (2.8630, 2.4010, 2.0818) -- (2.8630, 2.4540, 2.0768) -- (2.8160, 2.4540, 2.0831) -- cycle;
\fill[blue!34.7, opacity=0.7] (2.8160, 2.4540, 2.0831) -- (2.8630, 2.4540, 2.0768) -- (2.8630, 2.5070, 2.0716) -- (2.8160, 2.5070, 2.0779) -- cycle;
\fill[blue!29.1, opacity=0.7] (2.8160, 2.5070, 2.0779) -- (2.8630, 2.5070, 2.0716) -- (2.8630, 2.5600, 2.0663) -- (2.8160, 2.5600, 2.0725) -- cycle;
\fill[blue!19.9, opacity=0.7] (2.8160, 2.5600, 2.0725) -- (2.8630, 2.5600, 2.0663) -- (2.8630, 2.6130, 2.0608) -- (2.8160, 2.6130, 2.0670) -- cycle;
\fill[blue!15.6, opacity=0.7] (2.8160, 2.6130, 2.0670) -- (2.8630, 2.6130, 2.0608) -- (2.8630, 2.6660, 2.0551) -- (2.8160, 2.6660, 2.0614) -- cycle;
\fill[blue!15.0, opacity=0.7] (2.8160, 2.6660, 2.0614) -- (2.8630, 2.6660, 2.0551) -- (2.8630, 2.7190, 2.0493) -- (2.8160, 2.7190, 2.0555) -- cycle;
\fill[blue!15.0, opacity=0.7] (2.8160, 2.7190, 2.0555) -- (2.8630, 2.7190, 2.0493) -- (2.8630, 2.7720, 2.0434) -- (2.8160, 2.7720, 2.0496) -- cycle;
\fill[blue!15.0, opacity=0.7] (2.8160, 2.7720, 2.0496) -- (2.8630, 2.7720, 2.0434) -- (2.8630, 2.8250, 2.0373) -- (2.8160, 2.8250, 2.0436) -- cycle;
\fill[blue!15.0, opacity=0.7] (2.8160, 2.8250, 2.0436) -- (2.8630, 2.8250, 2.0373) -- (2.8630, 2.8780, 2.0312) -- (2.8160, 2.8780, 2.0375) -- cycle;
\fill[blue!15.0, opacity=0.7] (2.8160, 2.8780, 2.0375) -- (2.8630, 2.8780, 2.0312) -- (2.8630, 2.9310, 2.0251) -- (2.8160, 2.9310, 2.0313) -- cycle;
\fill[blue!15.0, opacity=0.7] (2.8160, 2.9310, 2.0313) -- (2.8630, 2.9310, 2.0251) -- (2.8630, 2.9840, 2.0188) -- (2.8160, 2.9840, 2.0251) -- cycle;
\fill[blue!15.1, opacity=0.7] (2.8160, 2.9840, 2.0251) -- (2.8630, 2.9840, 2.0188) -- (2.8630, 3.0370, 2.0126) -- (2.8160, 3.0370, 2.0188) -- cycle;
\fill[blue!15.9, opacity=0.7] (2.8160, 3.0370, 2.0188) -- (2.8630, 3.0370, 2.0126) -- (2.8630, 3.0900, 2.0063) -- (2.8160, 3.0900, 2.0125) -- cycle;
\fill[blue!15.0, opacity=0.7] (2.8630, -0.0900, 2.0063) -- (2.9100, -0.0900, 2.0000) -- (2.9100, -0.0370, 2.0063) -- (2.8630, -0.0370, 2.0126) -- cycle;
\fill[blue!15.0, opacity=0.7] (2.8630, -0.0370, 2.0126) -- (2.9100, -0.0370, 2.0063) -- (2.9100, 0.0160, 2.0125) -- (2.8630, 0.0160, 2.0188) -- cycle;
\fill[blue!15.0, opacity=0.7] (2.8630, 0.0160, 2.0188) -- (2.9100, 0.0160, 2.0125) -- (2.9100, 0.0690, 2.0188) -- (2.8630, 0.0690, 2.0251) -- cycle;
\fill[blue!15.0, opacity=0.7] (2.8630, 0.0690, 2.0251) -- (2.9100, 0.0690, 2.0188) -- (2.9100, 0.1220, 2.0249) -- (2.8630, 0.1220, 2.0312) -- cycle;
\fill[blue!15.4, opacity=0.7] (2.8630, 0.1220, 2.0312) -- (2.9100, 0.1220, 2.0249) -- (2.9100, 0.1750, 2.0311) -- (2.8630, 0.1750, 2.0373) -- cycle;
\fill[blue!17.5, opacity=0.7] (2.8630, 0.1750, 2.0373) -- (2.9100, 0.1750, 2.0311) -- (2.9100, 0.2280, 2.0371) -- (2.8630, 0.2280, 2.0434) -- cycle;
\fill[blue!21.2, opacity=0.7] (2.8630, 0.2280, 2.0434) -- (2.9100, 0.2280, 2.0371) -- (2.9100, 0.2810, 2.0430) -- (2.8630, 0.2810, 2.0493) -- cycle;
\fill[blue!22.9, opacity=0.7] (2.8630, 0.2810, 2.0493) -- (2.9100, 0.2810, 2.0430) -- (2.9100, 0.3340, 2.0488) -- (2.8630, 0.3340, 2.0551) -- cycle;
\fill[blue!20.8, opacity=0.7] (2.8630, 0.3340, 2.0551) -- (2.9100, 0.3340, 2.0488) -- (2.9100, 0.3870, 2.0545) -- (2.8630, 0.3870, 2.0608) -- cycle;
\fill[blue!17.7, opacity=0.7] (2.8630, 0.3870, 2.0608) -- (2.9100, 0.3870, 2.0545) -- (2.9100, 0.4400, 2.0600) -- (2.8630, 0.4400, 2.0663) -- cycle;
\fill[blue!15.8, opacity=0.7] (2.8630, 0.4400, 2.0663) -- (2.9100, 0.4400, 2.0600) -- (2.9100, 0.4930, 2.0654) -- (2.8630, 0.4930, 2.0716) -- cycle;
\fill[blue!15.2, opacity=0.7] (2.8630, 0.4930, 2.0716) -- (2.9100, 0.4930, 2.0654) -- (2.9100, 0.5460, 2.0705) -- (2.8630, 0.5460, 2.0768) -- cycle;
\fill[blue!15.0, opacity=0.7] (2.8630, 0.5460, 2.0768) -- (2.9100, 0.5460, 2.0705) -- (2.9100, 0.5990, 2.0755) -- (2.8630, 0.5990, 2.0818) -- cycle;
\fill[blue!15.0, opacity=0.7] (2.8630, 0.5990, 2.0818) -- (2.9100, 0.5990, 2.0755) -- (2.9100, 0.6520, 2.0803) -- (2.8630, 0.6520, 2.0866) -- cycle;
\fill[blue!15.0, opacity=0.7] (2.8630, 0.6520, 2.0866) -- (2.9100, 0.6520, 2.0803) -- (2.9100, 0.7050, 2.0849) -- (2.8630, 0.7050, 2.0911) -- cycle;
\fill[blue!15.0, opacity=0.7] (2.8630, 0.7050, 2.0911) -- (2.9100, 0.7050, 2.0849) -- (2.9100, 0.7580, 2.0892) -- (2.8630, 0.7580, 2.0955) -- cycle;
\fill[blue!15.0, opacity=0.7] (2.8630, 0.7580, 2.0955) -- (2.9100, 0.7580, 2.0892) -- (2.9100, 0.8110, 2.0933) -- (2.8630, 0.8110, 2.0995) -- cycle;
\fill[blue!15.0, opacity=0.7] (2.8630, 0.8110, 2.0995) -- (2.9100, 0.8110, 2.0933) -- (2.9100, 0.8640, 2.0971) -- (2.8630, 0.8640, 2.1034) -- cycle;
\fill[blue!15.0, opacity=0.7] (2.8630, 0.8640, 2.1034) -- (2.9100, 0.8640, 2.0971) -- (2.9100, 0.9170, 2.1006) -- (2.8630, 0.9170, 2.1069) -- cycle;
\fill[blue!15.0, opacity=0.7] (2.8630, 0.9170, 2.1069) -- (2.9100, 0.9170, 2.1006) -- (2.9100, 0.9700, 2.1039) -- (2.8630, 0.9700, 2.1102) -- cycle;
\fill[blue!15.1, opacity=0.7] (2.8630, 0.9700, 2.1102) -- (2.9100, 0.9700, 2.1039) -- (2.9100, 1.0230, 2.1069) -- (2.8630, 1.0230, 2.1132) -- cycle;
\fill[blue!15.3, opacity=0.7] (2.8630, 1.0230, 2.1132) -- (2.9100, 1.0230, 2.1069) -- (2.9100, 1.0760, 2.1096) -- (2.8630, 1.0760, 2.1159) -- cycle;
\fill[blue!15.6, opacity=0.7] (2.8630, 1.0760, 2.1159) -- (2.9100, 1.0760, 2.1096) -- (2.9100, 1.1290, 2.1120) -- (2.8630, 1.1290, 2.1183) -- cycle;
\fill[blue!16.1, opacity=0.7] (2.8630, 1.1290, 2.1183) -- (2.9100, 1.1290, 2.1120) -- (2.9100, 1.1820, 2.1141) -- (2.8630, 1.1820, 2.1204) -- cycle;
\fill[blue!16.7, opacity=0.7] (2.8630, 1.1820, 2.1204) -- (2.9100, 1.1820, 2.1141) -- (2.9100, 1.2350, 2.1159) -- (2.8630, 1.2350, 2.1222) -- cycle;
\fill[blue!17.3, opacity=0.7] (2.8630, 1.2350, 2.1222) -- (2.9100, 1.2350, 2.1159) -- (2.9100, 1.2880, 2.1174) -- (2.8630, 1.2880, 2.1237) -- cycle;
\fill[blue!17.7, opacity=0.7] (2.8630, 1.2880, 2.1237) -- (2.9100, 1.2880, 2.1174) -- (2.9100, 1.3410, 2.1185) -- (2.8630, 1.3410, 2.1248) -- cycle;
\fill[blue!18.0, opacity=0.7] (2.8630, 1.3410, 2.1248) -- (2.9100, 1.3410, 2.1185) -- (2.9100, 1.3940, 2.1193) -- (2.8630, 1.3940, 2.1256) -- cycle;
\fill[blue!17.8, opacity=0.7] (2.8630, 1.3940, 2.1256) -- (2.9100, 1.3940, 2.1193) -- (2.9100, 1.4470, 2.1198) -- (2.8630, 1.4470, 2.1261) -- cycle;
\fill[blue!17.4, opacity=0.7] (2.8630, 1.4470, 2.1261) -- (2.9100, 1.4470, 2.1198) -- (2.9100, 1.5000, 2.1200) -- (2.8630, 1.5000, 2.1263) -- cycle;
\fill[blue!16.9, opacity=0.7] (2.8630, 1.5000, 2.1263) -- (2.9100, 1.5000, 2.1200) -- (2.9100, 1.5530, 2.1198) -- (2.8630, 1.5530, 2.1261) -- cycle;
\fill[blue!16.3, opacity=0.7] (2.8630, 1.5530, 2.1261) -- (2.9100, 1.5530, 2.1198) -- (2.9100, 1.6060, 2.1193) -- (2.8630, 1.6060, 2.1256) -- cycle;
\fill[blue!15.7, opacity=0.7] (2.8630, 1.6060, 2.1256) -- (2.9100, 1.6060, 2.1193) -- (2.9100, 1.6590, 2.1185) -- (2.8630, 1.6590, 2.1248) -- cycle;
\fill[blue!15.4, opacity=0.7] (2.8630, 1.6590, 2.1248) -- (2.9100, 1.6590, 2.1185) -- (2.9100, 1.7120, 2.1174) -- (2.8630, 1.7120, 2.1237) -- cycle;
\fill[blue!15.2, opacity=0.7] (2.8630, 1.7120, 2.1237) -- (2.9100, 1.7120, 2.1174) -- (2.9100, 1.7650, 2.1159) -- (2.8630, 1.7650, 2.1222) -- cycle;
\fill[blue!15.1, opacity=0.7] (2.8630, 1.7650, 2.1222) -- (2.9100, 1.7650, 2.1159) -- (2.9100, 1.8180, 2.1141) -- (2.8630, 1.8180, 2.1204) -- cycle;
\fill[blue!15.0, opacity=0.7] (2.8630, 1.8180, 2.1204) -- (2.9100, 1.8180, 2.1141) -- (2.9100, 1.8710, 2.1120) -- (2.8630, 1.8710, 2.1183) -- cycle;
\fill[blue!15.0, opacity=0.7] (2.8630, 1.8710, 2.1183) -- (2.9100, 1.8710, 2.1120) -- (2.9100, 1.9240, 2.1096) -- (2.8630, 1.9240, 2.1159) -- cycle;
\fill[blue!15.0, opacity=0.7] (2.8630, 1.9240, 2.1159) -- (2.9100, 1.9240, 2.1096) -- (2.9100, 1.9770, 2.1069) -- (2.8630, 1.9770, 2.1132) -- cycle;
\fill[blue!15.0, opacity=0.7] (2.8630, 1.9770, 2.1132) -- (2.9100, 1.9770, 2.1069) -- (2.9100, 2.0300, 2.1039) -- (2.8630, 2.0300, 2.1102) -- cycle;
\fill[blue!15.0, opacity=0.7] (2.8630, 2.0300, 2.1102) -- (2.9100, 2.0300, 2.1039) -- (2.9100, 2.0830, 2.1006) -- (2.8630, 2.0830, 2.1069) -- cycle;
\fill[blue!15.1, opacity=0.7] (2.8630, 2.0830, 2.1069) -- (2.9100, 2.0830, 2.1006) -- (2.9100, 2.1360, 2.0971) -- (2.8630, 2.1360, 2.1034) -- cycle;
\fill[blue!15.3, opacity=0.7] (2.8630, 2.1360, 2.1034) -- (2.9100, 2.1360, 2.0971) -- (2.9100, 2.1890, 2.0933) -- (2.8630, 2.1890, 2.0995) -- cycle;
\fill[blue!16.4, opacity=0.7] (2.8630, 2.1890, 2.0995) -- (2.9100, 2.1890, 2.0933) -- (2.9100, 2.2420, 2.0892) -- (2.8630, 2.2420, 2.0955) -- cycle;
\fill[blue!19.9, opacity=0.7] (2.8630, 2.2420, 2.0955) -- (2.9100, 2.2420, 2.0892) -- (2.9100, 2.2950, 2.0849) -- (2.8630, 2.2950, 2.0911) -- cycle;
\fill[blue!26.7, opacity=0.7] (2.8630, 2.2950, 2.0911) -- (2.9100, 2.2950, 2.0849) -- (2.9100, 2.3480, 2.0803) -- (2.8630, 2.3480, 2.0866) -- cycle;
\fill[blue!33.1, opacity=0.7] (2.8630, 2.3480, 2.0866) -- (2.9100, 2.3480, 2.0803) -- (2.9100, 2.4010, 2.0755) -- (2.8630, 2.4010, 2.0818) -- cycle;
\fill[blue!33.1, opacity=0.7] (2.8630, 2.4010, 2.0818) -- (2.9100, 2.4010, 2.0755) -- (2.9100, 2.4540, 2.0705) -- (2.8630, 2.4540, 2.0768) -- cycle;
\fill[blue!25.9, opacity=0.7] (2.8630, 2.4540, 2.0768) -- (2.9100, 2.4540, 2.0705) -- (2.9100, 2.5070, 2.0654) -- (2.8630, 2.5070, 2.0716) -- cycle;
\fill[blue!18.2, opacity=0.7] (2.8630, 2.5070, 2.0716) -- (2.9100, 2.5070, 2.0654) -- (2.9100, 2.5600, 2.0600) -- (2.8630, 2.5600, 2.0663) -- cycle;
\fill[blue!15.4, opacity=0.7] (2.8630, 2.5600, 2.0663) -- (2.9100, 2.5600, 2.0600) -- (2.9100, 2.6130, 2.0545) -- (2.8630, 2.6130, 2.0608) -- cycle;
\fill[blue!15.0, opacity=0.7] (2.8630, 2.6130, 2.0608) -- (2.9100, 2.6130, 2.0545) -- (2.9100, 2.6660, 2.0488) -- (2.8630, 2.6660, 2.0551) -- cycle;
\fill[blue!15.0, opacity=0.7] (2.8630, 2.6660, 2.0551) -- (2.9100, 2.6660, 2.0488) -- (2.9100, 2.7190, 2.0430) -- (2.8630, 2.7190, 2.0493) -- cycle;
\fill[blue!15.0, opacity=0.7] (2.8630, 2.7190, 2.0493) -- (2.9100, 2.7190, 2.0430) -- (2.9100, 2.7720, 2.0371) -- (2.8630, 2.7720, 2.0434) -- cycle;
\fill[blue!15.0, opacity=0.7] (2.8630, 2.7720, 2.0434) -- (2.9100, 2.7720, 2.0371) -- (2.9100, 2.8250, 2.0311) -- (2.8630, 2.8250, 2.0373) -- cycle;
\fill[blue!15.0, opacity=0.7] (2.8630, 2.8250, 2.0373) -- (2.9100, 2.8250, 2.0311) -- (2.9100, 2.8780, 2.0249) -- (2.8630, 2.8780, 2.0312) -- cycle;
\fill[blue!15.0, opacity=0.7] (2.8630, 2.8780, 2.0312) -- (2.9100, 2.8780, 2.0249) -- (2.9100, 2.9310, 2.0188) -- (2.8630, 2.9310, 2.0251) -- cycle;
\fill[blue!15.1, opacity=0.7] (2.8630, 2.9310, 2.0251) -- (2.9100, 2.9310, 2.0188) -- (2.9100, 2.9840, 2.0125) -- (2.8630, 2.9840, 2.0188) -- cycle;
\fill[blue!15.8, opacity=0.7] (2.8630, 2.9840, 2.0188) -- (2.9100, 2.9840, 2.0125) -- (2.9100, 3.0370, 2.0063) -- (2.8630, 3.0370, 2.0126) -- cycle;
\fill[blue!16.5, opacity=0.7] (2.8630, 3.0370, 2.0126) -- (2.9100, 3.0370, 2.0063) -- (2.9100, 3.0900, 2.0000) -- (2.8630, 3.0900, 2.0063) -- cycle;
% Slice 3 horizontal patches
\fill[blue!33.5, opacity=0.7] (0.1200, -0.1200, 3.0000) -- (0.1660, -0.1200, 3.0063) -- (0.1660, -0.0660, 3.0126) -- (0.1200, -0.0660, 3.0063) -- cycle;
\fill[blue!29.3, opacity=0.7] (0.1200, -0.0660, 3.0063) -- (0.1660, -0.0660, 3.0126) -- (0.1660, -0.0120, 3.0188) -- (0.1200, -0.0120, 3.0125) -- cycle;
\fill[blue!21.4, opacity=0.7] (0.1200, -0.0120, 3.0125) -- (0.1660, -0.0120, 3.0188) -- (0.1660, 0.0420, 3.0251) -- (0.1200, 0.0420, 3.0188) -- cycle;
\fill[blue!16.8, opacity=0.7] (0.1200, 0.0420, 3.0188) -- (0.1660, 0.0420, 3.0251) -- (0.1660, 0.0960, 3.0312) -- (0.1200, 0.0960, 3.0249) -- cycle;
\fill[blue!15.5, opacity=0.7] (0.1200, 0.0960, 3.0249) -- (0.1660, 0.0960, 3.0312) -- (0.1660, 0.1500, 3.0373) -- (0.1200, 0.1500, 3.0311) -- cycle;
\fill[blue!15.2, opacity=0.7] (0.1200, 0.1500, 3.0311) -- (0.1660, 0.1500, 3.0373) -- (0.1660, 0.2040, 3.0434) -- (0.1200, 0.2040, 3.0371) -- cycle;
\fill[blue!15.3, opacity=0.7] (0.1200, 0.2040, 3.0371) -- (0.1660, 0.2040, 3.0434) -- (0.1660, 0.2580, 3.0493) -- (0.1200, 0.2580, 3.0430) -- cycle;
\fill[blue!15.7, opacity=0.7] (0.1200, 0.2580, 3.0430) -- (0.1660, 0.2580, 3.0493) -- (0.1660, 0.3120, 3.0551) -- (0.1200, 0.3120, 3.0488) -- cycle;
\fill[blue!18.0, opacity=0.7] (0.1200, 0.3120, 3.0488) -- (0.1660, 0.3120, 3.0551) -- (0.1660, 0.3660, 3.0608) -- (0.1200, 0.3660, 3.0545) -- cycle;
\fill[blue!25.7, opacity=0.7] (0.1200, 0.3660, 3.0545) -- (0.1660, 0.3660, 3.0608) -- (0.1660, 0.4200, 3.0663) -- (0.1200, 0.4200, 3.0600) -- cycle;
\fill[blue!39.8, opacity=0.7] (0.1200, 0.4200, 3.0600) -- (0.1660, 0.4200, 3.0663) -- (0.1660, 0.4740, 3.0716) -- (0.1200, 0.4740, 3.0654) -- cycle;
\fill[blue!52.2, opacity=0.7] (0.1200, 0.4740, 3.0654) -- (0.1660, 0.4740, 3.0716) -- (0.1660, 0.5280, 3.0768) -- (0.1200, 0.5280, 3.0705) -- cycle;
\fill[blue!57.6, opacity=0.7] (0.1200, 0.5280, 3.0705) -- (0.1660, 0.5280, 3.0768) -- (0.1660, 0.5820, 3.0818) -- (0.1200, 0.5820, 3.0755) -- cycle;
\fill[blue!57.6, opacity=0.7] (0.1200, 0.5820, 3.0755) -- (0.1660, 0.5820, 3.0818) -- (0.1660, 0.6360, 3.0866) -- (0.1200, 0.6360, 3.0803) -- cycle;
\fill[blue!53.0, opacity=0.7] (0.1200, 0.6360, 3.0803) -- (0.1660, 0.6360, 3.0866) -- (0.1660, 0.6900, 3.0911) -- (0.1200, 0.6900, 3.0849) -- cycle;
\fill[blue!43.8, opacity=0.7] (0.1200, 0.6900, 3.0849) -- (0.1660, 0.6900, 3.0911) -- (0.1660, 0.7440, 3.0955) -- (0.1200, 0.7440, 3.0892) -- cycle;
\fill[blue!33.3, opacity=0.7] (0.1200, 0.7440, 3.0892) -- (0.1660, 0.7440, 3.0955) -- (0.1660, 0.7980, 3.0995) -- (0.1200, 0.7980, 3.0933) -- cycle;
\fill[blue!25.2, opacity=0.7] (0.1200, 0.7980, 3.0933) -- (0.1660, 0.7980, 3.0995) -- (0.1660, 0.8520, 3.1034) -- (0.1200, 0.8520, 3.0971) -- cycle;
\fill[blue!20.6, opacity=0.7] (0.1200, 0.8520, 3.0971) -- (0.1660, 0.8520, 3.1034) -- (0.1660, 0.9060, 3.1069) -- (0.1200, 0.9060, 3.1006) -- cycle;
\fill[blue!18.4, opacity=0.7] (0.1200, 0.9060, 3.1006) -- (0.1660, 0.9060, 3.1069) -- (0.1660, 0.9600, 3.1102) -- (0.1200, 0.9600, 3.1039) -- cycle;
\fill[blue!17.6, opacity=0.7] (0.1200, 0.9600, 3.1039) -- (0.1660, 0.9600, 3.1102) -- (0.1660, 1.0140, 3.1132) -- (0.1200, 1.0140, 3.1069) -- cycle;
\fill[blue!17.5, opacity=0.7] (0.1200, 1.0140, 3.1069) -- (0.1660, 1.0140, 3.1132) -- (0.1660, 1.0680, 3.1159) -- (0.1200, 1.0680, 3.1096) -- cycle;
\fill[blue!17.9, opacity=0.7] (0.1200, 1.0680, 3.1096) -- (0.1660, 1.0680, 3.1159) -- (0.1660, 1.1220, 3.1183) -- (0.1200, 1.1220, 3.1120) -- cycle;
\fill[blue!18.8, opacity=0.7] (0.1200, 1.1220, 3.1120) -- (0.1660, 1.1220, 3.1183) -- (0.1660, 1.1760, 3.1204) -- (0.1200, 1.1760, 3.1141) -- cycle;
\fill[blue!20.5, opacity=0.7] (0.1200, 1.1760, 3.1141) -- (0.1660, 1.1760, 3.1204) -- (0.1660, 1.2300, 3.1222) -- (0.1200, 1.2300, 3.1159) -- cycle;
\fill[blue!22.8, opacity=0.7] (0.1200, 1.2300, 3.1159) -- (0.1660, 1.2300, 3.1222) -- (0.1660, 1.2840, 3.1237) -- (0.1200, 1.2840, 3.1174) -- cycle;
\fill[blue!25.8, opacity=0.7] (0.1200, 1.2840, 3.1174) -- (0.1660, 1.2840, 3.1237) -- (0.1660, 1.3380, 3.1248) -- (0.1200, 1.3380, 3.1185) -- cycle;
\fill[blue!29.1, opacity=0.7] (0.1200, 1.3380, 3.1185) -- (0.1660, 1.3380, 3.1248) -- (0.1660, 1.3920, 3.1256) -- (0.1200, 1.3920, 3.1193) -- cycle;
\fill[blue!32.4, opacity=0.7] (0.1200, 1.3920, 3.1193) -- (0.1660, 1.3920, 3.1256) -- (0.1660, 1.4460, 3.1261) -- (0.1200, 1.4460, 3.1198) -- cycle;
\fill[blue!35.3, opacity=0.7] (0.1200, 1.4460, 3.1198) -- (0.1660, 1.4460, 3.1261) -- (0.1660, 1.5000, 3.1263) -- (0.1200, 1.5000, 3.1200) -- cycle;
\fill[blue!37.5, opacity=0.7] (0.1200, 1.5000, 3.1200) -- (0.1660, 1.5000, 3.1263) -- (0.1660, 1.5540, 3.1261) -- (0.1200, 1.5540, 3.1198) -- cycle;
\fill[blue!38.8, opacity=0.7] (0.1200, 1.5540, 3.1198) -- (0.1660, 1.5540, 3.1261) -- (0.1660, 1.6080, 3.1256) -- (0.1200, 1.6080, 3.1193) -- cycle;
\fill[blue!39.1, opacity=0.7] (0.1200, 1.6080, 3.1193) -- (0.1660, 1.6080, 3.1256) -- (0.1660, 1.6620, 3.1248) -- (0.1200, 1.6620, 3.1185) -- cycle;
\fill[blue!38.5, opacity=0.7] (0.1200, 1.6620, 3.1185) -- (0.1660, 1.6620, 3.1248) -- (0.1660, 1.7160, 3.1237) -- (0.1200, 1.7160, 3.1174) -- cycle;
\fill[blue!36.9, opacity=0.7] (0.1200, 1.7160, 3.1174) -- (0.1660, 1.7160, 3.1237) -- (0.1660, 1.7700, 3.1222) -- (0.1200, 1.7700, 3.1159) -- cycle;
\fill[blue!34.5, opacity=0.7] (0.1200, 1.7700, 3.1159) -- (0.1660, 1.7700, 3.1222) -- (0.1660, 1.8240, 3.1204) -- (0.1200, 1.8240, 3.1141) -- cycle;
\fill[blue!31.4, opacity=0.7] (0.1200, 1.8240, 3.1141) -- (0.1660, 1.8240, 3.1204) -- (0.1660, 1.8780, 3.1183) -- (0.1200, 1.8780, 3.1120) -- cycle;
\fill[blue!28.0, opacity=0.7] (0.1200, 1.8780, 3.1120) -- (0.1660, 1.8780, 3.1183) -- (0.1660, 1.9320, 3.1159) -- (0.1200, 1.9320, 3.1096) -- cycle;
\fill[blue!24.7, opacity=0.7] (0.1200, 1.9320, 3.1096) -- (0.1660, 1.9320, 3.1159) -- (0.1660, 1.9860, 3.1132) -- (0.1200, 1.9860, 3.1069) -- cycle;
\fill[blue!21.7, opacity=0.7] (0.1200, 1.9860, 3.1069) -- (0.1660, 1.9860, 3.1132) -- (0.1660, 2.0400, 3.1102) -- (0.1200, 2.0400, 3.1039) -- cycle;
\fill[blue!19.4, opacity=0.7] (0.1200, 2.0400, 3.1039) -- (0.1660, 2.0400, 3.1102) -- (0.1660, 2.0940, 3.1069) -- (0.1200, 2.0940, 3.1006) -- cycle;
\fill[blue!17.9, opacity=0.7] (0.1200, 2.0940, 3.1006) -- (0.1660, 2.0940, 3.1069) -- (0.1660, 2.1480, 3.1034) -- (0.1200, 2.1480, 3.0971) -- cycle;
\fill[blue!17.0, opacity=0.7] (0.1200, 2.1480, 3.0971) -- (0.1660, 2.1480, 3.1034) -- (0.1660, 2.2020, 3.0995) -- (0.1200, 2.2020, 3.0933) -- cycle;
\fill[blue!16.5, opacity=0.7] (0.1200, 2.2020, 3.0933) -- (0.1660, 2.2020, 3.0995) -- (0.1660, 2.2560, 3.0955) -- (0.1200, 2.2560, 3.0892) -- cycle;
\fill[blue!16.4, opacity=0.7] (0.1200, 2.2560, 3.0892) -- (0.1660, 2.2560, 3.0955) -- (0.1660, 2.3100, 3.0911) -- (0.1200, 2.3100, 3.0849) -- cycle;
\fill[blue!16.6, opacity=0.7] (0.1200, 2.3100, 3.0849) -- (0.1660, 2.3100, 3.0911) -- (0.1660, 2.3640, 3.0866) -- (0.1200, 2.3640, 3.0803) -- cycle;
\fill[blue!17.3, opacity=0.7] (0.1200, 2.3640, 3.0803) -- (0.1660, 2.3640, 3.0866) -- (0.1660, 2.4180, 3.0818) -- (0.1200, 2.4180, 3.0755) -- cycle;
\fill[blue!19.1, opacity=0.7] (0.1200, 2.4180, 3.0755) -- (0.1660, 2.4180, 3.0818) -- (0.1660, 2.4720, 3.0768) -- (0.1200, 2.4720, 3.0705) -- cycle;
\fill[blue!22.9, opacity=0.7] (0.1200, 2.4720, 3.0705) -- (0.1660, 2.4720, 3.0768) -- (0.1660, 2.5260, 3.0716) -- (0.1200, 2.5260, 3.0654) -- cycle;
\fill[blue!29.6, opacity=0.7] (0.1200, 2.5260, 3.0654) -- (0.1660, 2.5260, 3.0716) -- (0.1660, 2.5800, 3.0663) -- (0.1200, 2.5800, 3.0600) -- cycle;
\fill[blue!38.2, opacity=0.7] (0.1200, 2.5800, 3.0600) -- (0.1660, 2.5800, 3.0663) -- (0.1660, 2.6340, 3.0608) -- (0.1200, 2.6340, 3.0545) -- cycle;
\fill[blue!45.6, opacity=0.7] (0.1200, 2.6340, 3.0545) -- (0.1660, 2.6340, 3.0608) -- (0.1660, 2.6880, 3.0551) -- (0.1200, 2.6880, 3.0488) -- cycle;
\fill[blue!48.7, opacity=0.7] (0.1200, 2.6880, 3.0488) -- (0.1660, 2.6880, 3.0551) -- (0.1660, 2.7420, 3.0493) -- (0.1200, 2.7420, 3.0430) -- cycle;
\fill[blue!46.1, opacity=0.7] (0.1200, 2.7420, 3.0430) -- (0.1660, 2.7420, 3.0493) -- (0.1660, 2.7960, 3.0434) -- (0.1200, 2.7960, 3.0371) -- cycle;
\fill[blue!37.5, opacity=0.7] (0.1200, 2.7960, 3.0371) -- (0.1660, 2.7960, 3.0434) -- (0.1660, 2.8500, 3.0373) -- (0.1200, 2.8500, 3.0311) -- cycle;
\fill[blue!26.1, opacity=0.7] (0.1200, 2.8500, 3.0311) -- (0.1660, 2.8500, 3.0373) -- (0.1660, 2.9040, 3.0312) -- (0.1200, 2.9040, 3.0249) -- cycle;
\fill[blue!18.3, opacity=0.7] (0.1200, 2.9040, 3.0249) -- (0.1660, 2.9040, 3.0312) -- (0.1660, 2.9580, 3.0251) -- (0.1200, 2.9580, 3.0188) -- cycle;
\fill[blue!15.7, opacity=0.7] (0.1200, 2.9580, 3.0188) -- (0.1660, 2.9580, 3.0251) -- (0.1660, 3.0120, 3.0188) -- (0.1200, 3.0120, 3.0125) -- cycle;
\fill[blue!15.1, opacity=0.7] (0.1200, 3.0120, 3.0125) -- (0.1660, 3.0120, 3.0188) -- (0.1660, 3.0660, 3.0126) -- (0.1200, 3.0660, 3.0063) -- cycle;
\fill[blue!15.0, opacity=0.7] (0.1200, 3.0660, 3.0063) -- (0.1660, 3.0660, 3.0126) -- (0.1660, 3.1200, 3.0063) -- (0.1200, 3.1200, 3.0000) -- cycle;
\fill[blue!30.4, opacity=0.7] (0.1660, -0.1200, 3.0063) -- (0.2120, -0.1200, 3.0125) -- (0.2120, -0.0660, 3.0188) -- (0.1660, -0.0660, 3.0126) -- cycle;
\fill[blue!22.2, opacity=0.7] (0.1660, -0.0660, 3.0126) -- (0.2120, -0.0660, 3.0188) -- (0.2120, -0.0120, 3.0251) -- (0.1660, -0.0120, 3.0188) -- cycle;
\fill[blue!17.0, opacity=0.7] (0.1660, -0.0120, 3.0188) -- (0.2120, -0.0120, 3.0251) -- (0.2120, 0.0420, 3.0313) -- (0.1660, 0.0420, 3.0251) -- cycle;
\fill[blue!15.5, opacity=0.7] (0.1660, 0.0420, 3.0251) -- (0.2120, 0.0420, 3.0313) -- (0.2120, 0.0960, 3.0375) -- (0.1660, 0.0960, 3.0312) -- cycle;
\fill[blue!15.2, opacity=0.7] (0.1660, 0.0960, 3.0312) -- (0.2120, 0.0960, 3.0375) -- (0.2120, 0.1500, 3.0436) -- (0.1660, 0.1500, 3.0373) -- cycle;
\fill[blue!15.3, opacity=0.7] (0.1660, 0.1500, 3.0373) -- (0.2120, 0.1500, 3.0436) -- (0.2120, 0.2040, 3.0496) -- (0.1660, 0.2040, 3.0434) -- cycle;
\fill[blue!15.8, opacity=0.7] (0.1660, 0.2040, 3.0434) -- (0.2120, 0.2040, 3.0496) -- (0.2120, 0.2580, 3.0555) -- (0.1660, 0.2580, 3.0493) -- cycle;
\fill[blue!18.7, opacity=0.7] (0.1660, 0.2580, 3.0493) -- (0.2120, 0.2580, 3.0555) -- (0.2120, 0.3120, 3.0614) -- (0.1660, 0.3120, 3.0551) -- cycle;
\fill[blue!28.3, opacity=0.7] (0.1660, 0.3120, 3.0551) -- (0.2120, 0.3120, 3.0614) -- (0.2120, 0.3660, 3.0670) -- (0.1660, 0.3660, 3.0608) -- cycle;
\fill[blue!43.8, opacity=0.7] (0.1660, 0.3660, 3.0608) -- (0.2120, 0.3660, 3.0670) -- (0.2120, 0.4200, 3.0725) -- (0.1660, 0.4200, 3.0663) -- cycle;
\fill[blue!54.9, opacity=0.7] (0.1660, 0.4200, 3.0663) -- (0.2120, 0.4200, 3.0725) -- (0.2120, 0.4740, 3.0779) -- (0.1660, 0.4740, 3.0716) -- cycle;
\fill[blue!58.5, opacity=0.7] (0.1660, 0.4740, 3.0716) -- (0.2120, 0.4740, 3.0779) -- (0.2120, 0.5280, 3.0831) -- (0.1660, 0.5280, 3.0768) -- cycle;
\fill[blue!56.5, opacity=0.7] (0.1660, 0.5280, 3.0768) -- (0.2120, 0.5280, 3.0831) -- (0.2120, 0.5820, 3.0881) -- (0.1660, 0.5820, 3.0818) -- cycle;
\fill[blue!49.0, opacity=0.7] (0.1660, 0.5820, 3.0818) -- (0.2120, 0.5820, 3.0881) -- (0.2120, 0.6360, 3.0928) -- (0.1660, 0.6360, 3.0866) -- cycle;
\fill[blue!37.6, opacity=0.7] (0.1660, 0.6360, 3.0866) -- (0.2120, 0.6360, 3.0928) -- (0.2120, 0.6900, 3.0974) -- (0.1660, 0.6900, 3.0911) -- cycle;
\fill[blue!27.3, opacity=0.7] (0.1660, 0.6900, 3.0911) -- (0.2120, 0.6900, 3.0974) -- (0.2120, 0.7440, 3.1017) -- (0.1660, 0.7440, 3.0955) -- cycle;
\fill[blue!21.3, opacity=0.7] (0.1660, 0.7440, 3.0955) -- (0.2120, 0.7440, 3.1017) -- (0.2120, 0.7980, 3.1058) -- (0.1660, 0.7980, 3.0995) -- cycle;
\fill[blue!18.6, opacity=0.7] (0.1660, 0.7980, 3.0995) -- (0.2120, 0.7980, 3.1058) -- (0.2120, 0.8520, 3.1096) -- (0.1660, 0.8520, 3.1034) -- cycle;
\fill[blue!17.7, opacity=0.7] (0.1660, 0.8520, 3.1034) -- (0.2120, 0.8520, 3.1096) -- (0.2120, 0.9060, 3.1132) -- (0.1660, 0.9060, 3.1069) -- cycle;
\fill[blue!17.7, opacity=0.7] (0.1660, 0.9060, 3.1069) -- (0.2120, 0.9060, 3.1132) -- (0.2120, 0.9600, 3.1165) -- (0.1660, 0.9600, 3.1102) -- cycle;
\fill[blue!18.7, opacity=0.7] (0.1660, 0.9600, 3.1102) -- (0.2120, 0.9600, 3.1165) -- (0.2120, 1.0140, 3.1195) -- (0.1660, 1.0140, 3.1132) -- cycle;
\fill[blue!20.7, opacity=0.7] (0.1660, 1.0140, 3.1132) -- (0.2120, 1.0140, 3.1195) -- (0.2120, 1.0680, 3.1222) -- (0.1660, 1.0680, 3.1159) -- cycle;
\fill[blue!24.5, opacity=0.7] (0.1660, 1.0680, 3.1159) -- (0.2120, 1.0680, 3.1222) -- (0.2120, 1.1220, 3.1246) -- (0.1660, 1.1220, 3.1183) -- cycle;
\fill[blue!30.2, opacity=0.7] (0.1660, 1.1220, 3.1183) -- (0.2120, 1.1220, 3.1246) -- (0.2120, 1.1760, 3.1267) -- (0.1660, 1.1760, 3.1204) -- cycle;
\fill[blue!37.2, opacity=0.7] (0.1660, 1.1760, 3.1204) -- (0.2120, 1.1760, 3.1267) -- (0.2120, 1.2300, 3.1285) -- (0.1660, 1.2300, 3.1222) -- cycle;
\fill[blue!44.5, opacity=0.7] (0.1660, 1.2300, 3.1222) -- (0.2120, 1.2300, 3.1285) -- (0.2120, 1.2840, 3.1299) -- (0.1660, 1.2840, 3.1237) -- cycle;
\fill[blue!50.8, opacity=0.7] (0.1660, 1.2840, 3.1237) -- (0.2120, 1.2840, 3.1299) -- (0.2120, 1.3380, 3.1311) -- (0.1660, 1.3380, 3.1248) -- cycle;
\fill[blue!55.6, opacity=0.7] (0.1660, 1.3380, 3.1248) -- (0.2120, 1.3380, 3.1311) -- (0.2120, 1.3920, 3.1319) -- (0.1660, 1.3920, 3.1256) -- cycle;
\fill[blue!58.7, opacity=0.7] (0.1660, 1.3920, 3.1256) -- (0.2120, 1.3920, 3.1319) -- (0.2120, 1.4460, 3.1324) -- (0.1660, 1.4460, 3.1261) -- cycle;
\fill[blue!60.5, opacity=0.7] (0.1660, 1.4460, 3.1261) -- (0.2120, 1.4460, 3.1324) -- (0.2120, 1.5000, 3.1325) -- (0.1660, 1.5000, 3.1263) -- cycle;
\fill[blue!61.6, opacity=0.7] (0.1660, 1.5000, 3.1263) -- (0.2120, 1.5000, 3.1325) -- (0.2120, 1.5540, 3.1324) -- (0.1660, 1.5540, 3.1261) -- cycle;
\fill[blue!62.0, opacity=0.7] (0.1660, 1.5540, 3.1261) -- (0.2120, 1.5540, 3.1324) -- (0.2120, 1.6080, 3.1319) -- (0.1660, 1.6080, 3.1256) -- cycle;
\fill[blue!62.1, opacity=0.7] (0.1660, 1.6080, 3.1256) -- (0.2120, 1.6080, 3.1319) -- (0.2120, 1.6620, 3.1311) -- (0.1660, 1.6620, 3.1248) -- cycle;
\fill[blue!61.9, opacity=0.7] (0.1660, 1.6620, 3.1248) -- (0.2120, 1.6620, 3.1311) -- (0.2120, 1.7160, 3.1299) -- (0.1660, 1.7160, 3.1237) -- cycle;
\fill[blue!61.2, opacity=0.7] (0.1660, 1.7160, 3.1237) -- (0.2120, 1.7160, 3.1299) -- (0.2120, 1.7700, 3.1285) -- (0.1660, 1.7700, 3.1222) -- cycle;
\fill[blue!59.9, opacity=0.7] (0.1660, 1.7700, 3.1222) -- (0.2120, 1.7700, 3.1285) -- (0.2120, 1.8240, 3.1267) -- (0.1660, 1.8240, 3.1204) -- cycle;
\fill[blue!57.7, opacity=0.7] (0.1660, 1.8240, 3.1204) -- (0.2120, 1.8240, 3.1267) -- (0.2120, 1.8780, 3.1246) -- (0.1660, 1.8780, 3.1183) -- cycle;
\fill[blue!54.2, opacity=0.7] (0.1660, 1.8780, 3.1183) -- (0.2120, 1.8780, 3.1246) -- (0.2120, 1.9320, 3.1222) -- (0.1660, 1.9320, 3.1159) -- cycle;
\fill[blue!49.2, opacity=0.7] (0.1660, 1.9320, 3.1159) -- (0.2120, 1.9320, 3.1222) -- (0.2120, 1.9860, 3.1195) -- (0.1660, 1.9860, 3.1132) -- cycle;
\fill[blue!42.6, opacity=0.7] (0.1660, 1.9860, 3.1132) -- (0.2120, 1.9860, 3.1195) -- (0.2120, 2.0400, 3.1165) -- (0.1660, 2.0400, 3.1102) -- cycle;
\fill[blue!35.3, opacity=0.7] (0.1660, 2.0400, 3.1102) -- (0.2120, 2.0400, 3.1165) -- (0.2120, 2.0940, 3.1132) -- (0.1660, 2.0940, 3.1069) -- cycle;
\fill[blue!28.4, opacity=0.7] (0.1660, 2.0940, 3.1069) -- (0.2120, 2.0940, 3.1132) -- (0.2120, 2.1480, 3.1096) -- (0.1660, 2.1480, 3.1034) -- cycle;
\fill[blue!23.0, opacity=0.7] (0.1660, 2.1480, 3.1034) -- (0.2120, 2.1480, 3.1096) -- (0.2120, 2.2020, 3.1058) -- (0.1660, 2.2020, 3.0995) -- cycle;
\fill[blue!19.5, opacity=0.7] (0.1660, 2.2020, 3.0995) -- (0.2120, 2.2020, 3.1058) -- (0.2120, 2.2560, 3.1017) -- (0.1660, 2.2560, 3.0955) -- cycle;
\fill[blue!17.5, opacity=0.7] (0.1660, 2.2560, 3.0955) -- (0.2120, 2.2560, 3.1017) -- (0.2120, 2.3100, 3.0974) -- (0.1660, 2.3100, 3.0911) -- cycle;
\fill[blue!16.6, opacity=0.7] (0.1660, 2.3100, 3.0911) -- (0.2120, 2.3100, 3.0974) -- (0.2120, 2.3640, 3.0928) -- (0.1660, 2.3640, 3.0866) -- cycle;
\fill[blue!16.3, opacity=0.7] (0.1660, 2.3640, 3.0866) -- (0.2120, 2.3640, 3.0928) -- (0.2120, 2.4180, 3.0881) -- (0.1660, 2.4180, 3.0818) -- cycle;
\fill[blue!16.4, opacity=0.7] (0.1660, 2.4180, 3.0818) -- (0.2120, 2.4180, 3.0881) -- (0.2120, 2.4720, 3.0831) -- (0.1660, 2.4720, 3.0768) -- cycle;
\fill[blue!17.2, opacity=0.7] (0.1660, 2.4720, 3.0768) -- (0.2120, 2.4720, 3.0831) -- (0.2120, 2.5260, 3.0779) -- (0.1660, 2.5260, 3.0716) -- cycle;
\fill[blue!19.2, opacity=0.7] (0.1660, 2.5260, 3.0716) -- (0.2120, 2.5260, 3.0779) -- (0.2120, 2.5800, 3.0725) -- (0.1660, 2.5800, 3.0663) -- cycle;
\fill[blue!24.0, opacity=0.7] (0.1660, 2.5800, 3.0663) -- (0.2120, 2.5800, 3.0725) -- (0.2120, 2.6340, 3.0670) -- (0.1660, 2.6340, 3.0608) -- cycle;
\fill[blue!32.1, opacity=0.7] (0.1660, 2.6340, 3.0608) -- (0.2120, 2.6340, 3.0670) -- (0.2120, 2.6880, 3.0614) -- (0.1660, 2.6880, 3.0551) -- cycle;
\fill[blue!41.3, opacity=0.7] (0.1660, 2.6880, 3.0551) -- (0.2120, 2.6880, 3.0614) -- (0.2120, 2.7420, 3.0555) -- (0.1660, 2.7420, 3.0493) -- cycle;
\fill[blue!47.4, opacity=0.7] (0.1660, 2.7420, 3.0493) -- (0.2120, 2.7420, 3.0555) -- (0.2120, 2.7960, 3.0496) -- (0.1660, 2.7960, 3.0434) -- cycle;
\fill[blue!47.5, opacity=0.7] (0.1660, 2.7960, 3.0434) -- (0.2120, 2.7960, 3.0496) -- (0.2120, 2.8500, 3.0436) -- (0.1660, 2.8500, 3.0373) -- cycle;
\fill[blue!40.8, opacity=0.7] (0.1660, 2.8500, 3.0373) -- (0.2120, 2.8500, 3.0436) -- (0.2120, 2.9040, 3.0375) -- (0.1660, 2.9040, 3.0312) -- cycle;
\fill[blue!29.0, opacity=0.7] (0.1660, 2.9040, 3.0312) -- (0.2120, 2.9040, 3.0375) -- (0.2120, 2.9580, 3.0313) -- (0.1660, 2.9580, 3.0251) -- cycle;
\fill[blue!19.5, opacity=0.7] (0.1660, 2.9580, 3.0251) -- (0.2120, 2.9580, 3.0313) -- (0.2120, 3.0120, 3.0251) -- (0.1660, 3.0120, 3.0188) -- cycle;
\fill[blue!15.9, opacity=0.7] (0.1660, 3.0120, 3.0188) -- (0.2120, 3.0120, 3.0251) -- (0.2120, 3.0660, 3.0188) -- (0.1660, 3.0660, 3.0126) -- cycle;
\fill[blue!15.2, opacity=0.7] (0.1660, 3.0660, 3.0126) -- (0.2120, 3.0660, 3.0188) -- (0.2120, 3.1200, 3.0125) -- (0.1660, 3.1200, 3.0063) -- cycle;
\fill[blue!23.6, opacity=0.7] (0.2120, -0.1200, 3.0125) -- (0.2580, -0.1200, 3.0188) -- (0.2580, -0.0660, 3.0251) -- (0.2120, -0.0660, 3.0188) -- cycle;
\fill[blue!17.5, opacity=0.7] (0.2120, -0.0660, 3.0188) -- (0.2580, -0.0660, 3.0251) -- (0.2580, -0.0120, 3.0313) -- (0.2120, -0.0120, 3.0251) -- cycle;
\fill[blue!15.6, opacity=0.7] (0.2120, -0.0120, 3.0251) -- (0.2580, -0.0120, 3.0313) -- (0.2580, 0.0420, 3.0375) -- (0.2120, 0.0420, 3.0313) -- cycle;
\fill[blue!15.2, opacity=0.7] (0.2120, 0.0420, 3.0313) -- (0.2580, 0.0420, 3.0375) -- (0.2580, 0.0960, 3.0437) -- (0.2120, 0.0960, 3.0375) -- cycle;
\fill[blue!15.3, opacity=0.7] (0.2120, 0.0960, 3.0375) -- (0.2580, 0.0960, 3.0437) -- (0.2580, 0.1500, 3.0498) -- (0.2120, 0.1500, 3.0436) -- cycle;
\fill[blue!15.9, opacity=0.7] (0.2120, 0.1500, 3.0436) -- (0.2580, 0.1500, 3.0498) -- (0.2580, 0.2040, 3.0559) -- (0.2120, 0.2040, 3.0496) -- cycle;
\fill[blue!19.1, opacity=0.7] (0.2120, 0.2040, 3.0496) -- (0.2580, 0.2040, 3.0559) -- (0.2580, 0.2580, 3.0618) -- (0.2120, 0.2580, 3.0555) -- cycle;
\fill[blue!30.0, opacity=0.7] (0.2120, 0.2580, 3.0555) -- (0.2580, 0.2580, 3.0618) -- (0.2580, 0.3120, 3.0676) -- (0.2120, 0.3120, 3.0614) -- cycle;
\fill[blue!46.3, opacity=0.7] (0.2120, 0.3120, 3.0614) -- (0.2580, 0.3120, 3.0676) -- (0.2580, 0.3660, 3.0733) -- (0.2120, 0.3660, 3.0670) -- cycle;
\fill[blue!56.5, opacity=0.7] (0.2120, 0.3660, 3.0670) -- (0.2580, 0.3660, 3.0733) -- (0.2580, 0.4200, 3.0788) -- (0.2120, 0.4200, 3.0725) -- cycle;
\fill[blue!58.7, opacity=0.7] (0.2120, 0.4200, 3.0725) -- (0.2580, 0.4200, 3.0788) -- (0.2580, 0.4740, 3.0841) -- (0.2120, 0.4740, 3.0779) -- cycle;
\fill[blue!55.1, opacity=0.7] (0.2120, 0.4740, 3.0779) -- (0.2580, 0.4740, 3.0841) -- (0.2580, 0.5280, 3.0893) -- (0.2120, 0.5280, 3.0831) -- cycle;
\fill[blue!45.2, opacity=0.7] (0.2120, 0.5280, 3.0831) -- (0.2580, 0.5280, 3.0893) -- (0.2580, 0.5820, 3.0943) -- (0.2120, 0.5820, 3.0881) -- cycle;
\fill[blue!32.7, opacity=0.7] (0.2120, 0.5820, 3.0881) -- (0.2580, 0.5820, 3.0943) -- (0.2580, 0.6360, 3.0991) -- (0.2120, 0.6360, 3.0928) -- cycle;
\fill[blue!23.7, opacity=0.7] (0.2120, 0.6360, 3.0928) -- (0.2580, 0.6360, 3.0991) -- (0.2580, 0.6900, 3.1036) -- (0.2120, 0.6900, 3.0974) -- cycle;
\fill[blue!19.4, opacity=0.7] (0.2120, 0.6900, 3.0974) -- (0.2580, 0.6900, 3.1036) -- (0.2580, 0.7440, 3.1079) -- (0.2120, 0.7440, 3.1017) -- cycle;
\fill[blue!17.9, opacity=0.7] (0.2120, 0.7440, 3.1017) -- (0.2580, 0.7440, 3.1079) -- (0.2580, 0.7980, 3.1120) -- (0.2120, 0.7980, 3.1058) -- cycle;
\fill[blue!17.9, opacity=0.7] (0.2120, 0.7980, 3.1058) -- (0.2580, 0.7980, 3.1120) -- (0.2580, 0.8520, 3.1159) -- (0.2120, 0.8520, 3.1096) -- cycle;
\fill[blue!19.0, opacity=0.7] (0.2120, 0.8520, 3.1096) -- (0.2580, 0.8520, 3.1159) -- (0.2580, 0.9060, 3.1194) -- (0.2120, 0.9060, 3.1132) -- cycle;
\fill[blue!22.0, opacity=0.7] (0.2120, 0.9060, 3.1132) -- (0.2580, 0.9060, 3.1194) -- (0.2580, 0.9600, 3.1227) -- (0.2120, 0.9600, 3.1165) -- cycle;
\fill[blue!27.8, opacity=0.7] (0.2120, 0.9600, 3.1165) -- (0.2580, 0.9600, 3.1227) -- (0.2580, 1.0140, 3.1257) -- (0.2120, 1.0140, 3.1195) -- cycle;
\fill[blue!36.8, opacity=0.7] (0.2120, 1.0140, 3.1195) -- (0.2580, 1.0140, 3.1257) -- (0.2580, 1.0680, 3.1284) -- (0.2120, 1.0680, 3.1222) -- cycle;
\fill[blue!47.2, opacity=0.7] (0.2120, 1.0680, 3.1222) -- (0.2580, 1.0680, 3.1284) -- (0.2580, 1.1220, 3.1308) -- (0.2120, 1.1220, 3.1246) -- cycle;
\fill[blue!55.9, opacity=0.7] (0.2120, 1.1220, 3.1246) -- (0.2580, 1.1220, 3.1308) -- (0.2580, 1.1760, 3.1329) -- (0.2120, 1.1760, 3.1267) -- cycle;
\fill[blue!61.2, opacity=0.7] (0.2120, 1.1760, 3.1267) -- (0.2580, 1.1760, 3.1329) -- (0.2580, 1.2300, 3.1347) -- (0.2120, 1.2300, 3.1285) -- cycle;
\fill[blue!63.3, opacity=0.7] (0.2120, 1.2300, 3.1285) -- (0.2580, 1.2300, 3.1347) -- (0.2580, 1.2840, 3.1361) -- (0.2120, 1.2840, 3.1299) -- cycle;
\fill[blue!63.5, opacity=0.7] (0.2120, 1.2840, 3.1299) -- (0.2580, 1.2840, 3.1361) -- (0.2580, 1.3380, 3.1373) -- (0.2120, 1.3380, 3.1311) -- cycle;
\fill[blue!63.0, opacity=0.7] (0.2120, 1.3380, 3.1311) -- (0.2580, 1.3380, 3.1373) -- (0.2580, 1.3920, 3.1381) -- (0.2120, 1.3920, 3.1319) -- cycle;
\fill[blue!62.4, opacity=0.7] (0.2120, 1.3920, 3.1319) -- (0.2580, 1.3920, 3.1381) -- (0.2580, 1.4460, 3.1386) -- (0.2120, 1.4460, 3.1324) -- cycle;
\fill[blue!62.0, opacity=0.7] (0.2120, 1.4460, 3.1324) -- (0.2580, 1.4460, 3.1386) -- (0.2580, 1.5000, 3.1388) -- (0.2120, 1.5000, 3.1325) -- cycle;
\fill[blue!61.7, opacity=0.7] (0.2120, 1.5000, 3.1325) -- (0.2580, 1.5000, 3.1388) -- (0.2580, 1.5540, 3.1386) -- (0.2120, 1.5540, 3.1324) -- cycle;
\fill[blue!61.7, opacity=0.7] (0.2120, 1.5540, 3.1324) -- (0.2580, 1.5540, 3.1386) -- (0.2580, 1.6080, 3.1381) -- (0.2120, 1.6080, 3.1319) -- cycle;
\fill[blue!61.8, opacity=0.7] (0.2120, 1.6080, 3.1319) -- (0.2580, 1.6080, 3.1381) -- (0.2580, 1.6620, 3.1373) -- (0.2120, 1.6620, 3.1311) -- cycle;
\fill[blue!62.0, opacity=0.7] (0.2120, 1.6620, 3.1311) -- (0.2580, 1.6620, 3.1373) -- (0.2580, 1.7160, 3.1361) -- (0.2120, 1.7160, 3.1299) -- cycle;
\fill[blue!62.2, opacity=0.7] (0.2120, 1.7160, 3.1299) -- (0.2580, 1.7160, 3.1361) -- (0.2580, 1.7700, 3.1347) -- (0.2120, 1.7700, 3.1285) -- cycle;
\fill[blue!62.6, opacity=0.7] (0.2120, 1.7700, 3.1285) -- (0.2580, 1.7700, 3.1347) -- (0.2580, 1.8240, 3.1329) -- (0.2120, 1.8240, 3.1267) -- cycle;
\fill[blue!63.0, opacity=0.7] (0.2120, 1.8240, 3.1267) -- (0.2580, 1.8240, 3.1329) -- (0.2580, 1.8780, 3.1308) -- (0.2120, 1.8780, 3.1246) -- cycle;
\fill[blue!63.4, opacity=0.7] (0.2120, 1.8780, 3.1246) -- (0.2580, 1.8780, 3.1308) -- (0.2580, 1.9320, 3.1284) -- (0.2120, 1.9320, 3.1222) -- cycle;
\fill[blue!63.5, opacity=0.7] (0.2120, 1.9320, 3.1222) -- (0.2580, 1.9320, 3.1284) -- (0.2580, 1.9860, 3.1257) -- (0.2120, 1.9860, 3.1195) -- cycle;
\fill[blue!62.6, opacity=0.7] (0.2120, 1.9860, 3.1195) -- (0.2580, 1.9860, 3.1257) -- (0.2580, 2.0400, 3.1227) -- (0.2120, 2.0400, 3.1165) -- cycle;
\fill[blue!59.7, opacity=0.7] (0.2120, 2.0400, 3.1165) -- (0.2580, 2.0400, 3.1227) -- (0.2580, 2.0940, 3.1194) -- (0.2120, 2.0940, 3.1132) -- cycle;
\fill[blue!53.9, opacity=0.7] (0.2120, 2.0940, 3.1132) -- (0.2580, 2.0940, 3.1194) -- (0.2580, 2.1480, 3.1159) -- (0.2120, 2.1480, 3.1096) -- cycle;
\fill[blue!45.0, opacity=0.7] (0.2120, 2.1480, 3.1096) -- (0.2580, 2.1480, 3.1159) -- (0.2580, 2.2020, 3.1120) -- (0.2120, 2.2020, 3.1058) -- cycle;
\fill[blue!34.8, opacity=0.7] (0.2120, 2.2020, 3.1058) -- (0.2580, 2.2020, 3.1120) -- (0.2580, 2.2560, 3.1079) -- (0.2120, 2.2560, 3.1017) -- cycle;
\fill[blue!26.1, opacity=0.7] (0.2120, 2.2560, 3.1017) -- (0.2580, 2.2560, 3.1079) -- (0.2580, 2.3100, 3.1036) -- (0.2120, 2.3100, 3.0974) -- cycle;
\fill[blue!20.5, opacity=0.7] (0.2120, 2.3100, 3.0974) -- (0.2580, 2.3100, 3.1036) -- (0.2580, 2.3640, 3.0991) -- (0.2120, 2.3640, 3.0928) -- cycle;
\fill[blue!17.7, opacity=0.7] (0.2120, 2.3640, 3.0928) -- (0.2580, 2.3640, 3.0991) -- (0.2580, 2.4180, 3.0943) -- (0.2120, 2.4180, 3.0881) -- cycle;
\fill[blue!16.5, opacity=0.7] (0.2120, 2.4180, 3.0881) -- (0.2580, 2.4180, 3.0943) -- (0.2580, 2.4720, 3.0893) -- (0.2120, 2.4720, 3.0831) -- cycle;
\fill[blue!16.2, opacity=0.7] (0.2120, 2.4720, 3.0831) -- (0.2580, 2.4720, 3.0893) -- (0.2580, 2.5260, 3.0841) -- (0.2120, 2.5260, 3.0779) -- cycle;
\fill[blue!16.4, opacity=0.7] (0.2120, 2.5260, 3.0779) -- (0.2580, 2.5260, 3.0841) -- (0.2580, 2.5800, 3.0788) -- (0.2120, 2.5800, 3.0725) -- cycle;
\fill[blue!17.5, opacity=0.7] (0.2120, 2.5800, 3.0725) -- (0.2580, 2.5800, 3.0788) -- (0.2580, 2.6340, 3.0733) -- (0.2120, 2.6340, 3.0670) -- cycle;
\fill[blue!20.6, opacity=0.7] (0.2120, 2.6340, 3.0670) -- (0.2580, 2.6340, 3.0733) -- (0.2580, 2.6880, 3.0676) -- (0.2120, 2.6880, 3.0614) -- cycle;
\fill[blue!27.2, opacity=0.7] (0.2120, 2.6880, 3.0614) -- (0.2580, 2.6880, 3.0676) -- (0.2580, 2.7420, 3.0618) -- (0.2120, 2.7420, 3.0555) -- cycle;
\fill[blue!37.0, opacity=0.7] (0.2120, 2.7420, 3.0555) -- (0.2580, 2.7420, 3.0618) -- (0.2580, 2.7960, 3.0559) -- (0.2120, 2.7960, 3.0496) -- cycle;
\fill[blue!45.3, opacity=0.7] (0.2120, 2.7960, 3.0496) -- (0.2580, 2.7960, 3.0559) -- (0.2580, 2.8500, 3.0498) -- (0.2120, 2.8500, 3.0436) -- cycle;
\fill[blue!47.8, opacity=0.7] (0.2120, 2.8500, 3.0436) -- (0.2580, 2.8500, 3.0498) -- (0.2580, 2.9040, 3.0437) -- (0.2120, 2.9040, 3.0375) -- cycle;
\fill[blue!42.7, opacity=0.7] (0.2120, 2.9040, 3.0375) -- (0.2580, 2.9040, 3.0437) -- (0.2580, 2.9580, 3.0375) -- (0.2120, 2.9580, 3.0313) -- cycle;
\fill[blue!31.1, opacity=0.7] (0.2120, 2.9580, 3.0313) -- (0.2580, 2.9580, 3.0375) -- (0.2580, 3.0120, 3.0313) -- (0.2120, 3.0120, 3.0251) -- cycle;
\fill[blue!20.3, opacity=0.7] (0.2120, 3.0120, 3.0251) -- (0.2580, 3.0120, 3.0313) -- (0.2580, 3.0660, 3.0251) -- (0.2120, 3.0660, 3.0188) -- cycle;
\fill[blue!16.0, opacity=0.7] (0.2120, 3.0660, 3.0188) -- (0.2580, 3.0660, 3.0251) -- (0.2580, 3.1200, 3.0188) -- (0.2120, 3.1200, 3.0125) -- cycle;
\fill[blue!18.3, opacity=0.7] (0.2580, -0.1200, 3.0188) -- (0.3040, -0.1200, 3.0249) -- (0.3040, -0.0660, 3.0312) -- (0.2580, -0.0660, 3.0251) -- cycle;
\fill[blue!15.8, opacity=0.7] (0.2580, -0.0660, 3.0251) -- (0.3040, -0.0660, 3.0312) -- (0.3040, -0.0120, 3.0375) -- (0.2580, -0.0120, 3.0313) -- cycle;
\fill[blue!15.3, opacity=0.7] (0.2580, -0.0120, 3.0313) -- (0.3040, -0.0120, 3.0375) -- (0.3040, 0.0420, 3.0437) -- (0.2580, 0.0420, 3.0375) -- cycle;
\fill[blue!15.3, opacity=0.7] (0.2580, 0.0420, 3.0375) -- (0.3040, 0.0420, 3.0437) -- (0.3040, 0.0960, 3.0499) -- (0.2580, 0.0960, 3.0437) -- cycle;
\fill[blue!15.8, opacity=0.7] (0.2580, 0.0960, 3.0437) -- (0.3040, 0.0960, 3.0499) -- (0.3040, 0.1500, 3.0560) -- (0.2580, 0.1500, 3.0498) -- cycle;
\fill[blue!19.1, opacity=0.7] (0.2580, 0.1500, 3.0498) -- (0.3040, 0.1500, 3.0560) -- (0.3040, 0.2040, 3.0620) -- (0.2580, 0.2040, 3.0559) -- cycle;
\fill[blue!30.4, opacity=0.7] (0.2580, 0.2040, 3.0559) -- (0.3040, 0.2040, 3.0620) -- (0.3040, 0.2580, 3.0680) -- (0.2580, 0.2580, 3.0618) -- cycle;
\fill[blue!47.4, opacity=0.7] (0.2580, 0.2580, 3.0618) -- (0.3040, 0.2580, 3.0680) -- (0.3040, 0.3120, 3.0738) -- (0.2580, 0.3120, 3.0676) -- cycle;
\fill[blue!57.3, opacity=0.7] (0.2580, 0.3120, 3.0676) -- (0.3040, 0.3120, 3.0738) -- (0.3040, 0.3660, 3.0794) -- (0.2580, 0.3660, 3.0733) -- cycle;
\fill[blue!58.9, opacity=0.7] (0.2580, 0.3660, 3.0733) -- (0.3040, 0.3660, 3.0794) -- (0.3040, 0.4200, 3.0849) -- (0.2580, 0.4200, 3.0788) -- cycle;
\fill[blue!53.9, opacity=0.7] (0.2580, 0.4200, 3.0788) -- (0.3040, 0.4200, 3.0849) -- (0.3040, 0.4740, 3.0903) -- (0.2580, 0.4740, 3.0841) -- cycle;
\fill[blue!42.2, opacity=0.7] (0.2580, 0.4740, 3.0841) -- (0.3040, 0.4740, 3.0903) -- (0.3040, 0.5280, 3.0955) -- (0.2580, 0.5280, 3.0893) -- cycle;
\fill[blue!29.4, opacity=0.7] (0.2580, 0.5280, 3.0893) -- (0.3040, 0.5280, 3.0955) -- (0.3040, 0.5820, 3.1005) -- (0.2580, 0.5820, 3.0943) -- cycle;
\fill[blue!21.7, opacity=0.7] (0.2580, 0.5820, 3.0943) -- (0.3040, 0.5820, 3.1005) -- (0.3040, 0.6360, 3.1052) -- (0.2580, 0.6360, 3.0991) -- cycle;
\fill[blue!18.6, opacity=0.7] (0.2580, 0.6360, 3.0991) -- (0.3040, 0.6360, 3.1052) -- (0.3040, 0.6900, 3.1098) -- (0.2580, 0.6900, 3.1036) -- cycle;
\fill[blue!17.9, opacity=0.7] (0.2580, 0.6900, 3.1036) -- (0.3040, 0.6900, 3.1098) -- (0.3040, 0.7440, 3.1141) -- (0.2580, 0.7440, 3.1079) -- cycle;
\fill[blue!18.7, opacity=0.7] (0.2580, 0.7440, 3.1079) -- (0.3040, 0.7440, 3.1141) -- (0.3040, 0.7980, 3.1182) -- (0.2580, 0.7980, 3.1120) -- cycle;
\fill[blue!21.7, opacity=0.7] (0.2580, 0.7980, 3.1120) -- (0.3040, 0.7980, 3.1182) -- (0.3040, 0.8520, 3.1220) -- (0.2580, 0.8520, 3.1159) -- cycle;
\fill[blue!28.4, opacity=0.7] (0.2580, 0.8520, 3.1159) -- (0.3040, 0.8520, 3.1220) -- (0.3040, 0.9060, 3.1256) -- (0.2580, 0.9060, 3.1194) -- cycle;
\fill[blue!39.5, opacity=0.7] (0.2580, 0.9060, 3.1194) -- (0.3040, 0.9060, 3.1256) -- (0.3040, 0.9600, 3.1289) -- (0.2580, 0.9600, 3.1227) -- cycle;
\fill[blue!52.0, opacity=0.7] (0.2580, 0.9600, 3.1227) -- (0.3040, 0.9600, 3.1289) -- (0.3040, 1.0140, 3.1319) -- (0.2580, 1.0140, 3.1257) -- cycle;
\fill[blue!60.5, opacity=0.7] (0.2580, 1.0140, 3.1257) -- (0.3040, 1.0140, 3.1319) -- (0.3040, 1.0680, 3.1346) -- (0.2580, 1.0680, 3.1284) -- cycle;
\fill[blue!63.5, opacity=0.7] (0.2580, 1.0680, 3.1284) -- (0.3040, 1.0680, 3.1346) -- (0.3040, 1.1220, 3.1370) -- (0.2580, 1.1220, 3.1308) -- cycle;
\fill[blue!62.9, opacity=0.7] (0.2580, 1.1220, 3.1308) -- (0.3040, 1.1220, 3.1370) -- (0.3040, 1.1760, 3.1391) -- (0.2580, 1.1760, 3.1329) -- cycle;
\fill[blue!61.5, opacity=0.7] (0.2580, 1.1760, 3.1329) -- (0.3040, 1.1760, 3.1391) -- (0.3040, 1.2300, 3.1409) -- (0.2580, 1.2300, 3.1347) -- cycle;
\fill[blue!60.5, opacity=0.7] (0.2580, 1.2300, 3.1347) -- (0.3040, 1.2300, 3.1409) -- (0.3040, 1.2840, 3.1423) -- (0.2580, 1.2840, 3.1361) -- cycle;
\fill[blue!60.2, opacity=0.7] (0.2580, 1.2840, 3.1361) -- (0.3040, 1.2840, 3.1423) -- (0.3040, 1.3380, 3.1435) -- (0.2580, 1.3380, 3.1373) -- cycle;
\fill[blue!60.6, opacity=0.7] (0.2580, 1.3380, 3.1373) -- (0.3040, 1.3380, 3.1435) -- (0.3040, 1.3920, 3.1443) -- (0.2580, 1.3920, 3.1381) -- cycle;
\fill[blue!61.2, opacity=0.7] (0.2580, 1.3920, 3.1381) -- (0.3040, 1.3920, 3.1443) -- (0.3040, 1.4460, 3.1448) -- (0.2580, 1.4460, 3.1386) -- cycle;
\fill[blue!61.9, opacity=0.7] (0.2580, 1.4460, 3.1386) -- (0.3040, 1.4460, 3.1448) -- (0.3040, 1.5000, 3.1449) -- (0.2580, 1.5000, 3.1388) -- cycle;
\fill[blue!62.4, opacity=0.7] (0.2580, 1.5000, 3.1388) -- (0.3040, 1.5000, 3.1449) -- (0.3040, 1.5540, 3.1448) -- (0.2580, 1.5540, 3.1386) -- cycle;
\fill[blue!62.8, opacity=0.7] (0.2580, 1.5540, 3.1386) -- (0.3040, 1.5540, 3.1448) -- (0.3040, 1.6080, 3.1443) -- (0.2580, 1.6080, 3.1381) -- cycle;
\fill[blue!62.9, opacity=0.7] (0.2580, 1.6080, 3.1381) -- (0.3040, 1.6080, 3.1443) -- (0.3040, 1.6620, 3.1435) -- (0.2580, 1.6620, 3.1373) -- cycle;
\fill[blue!63.0, opacity=0.7] (0.2580, 1.6620, 3.1373) -- (0.3040, 1.6620, 3.1435) -- (0.3040, 1.7160, 3.1423) -- (0.2580, 1.7160, 3.1361) -- cycle;
\fill[blue!62.9, opacity=0.7] (0.2580, 1.7160, 3.1361) -- (0.3040, 1.7160, 3.1423) -- (0.3040, 1.7700, 3.1409) -- (0.2580, 1.7700, 3.1347) -- cycle;
\fill[blue!62.8, opacity=0.7] (0.2580, 1.7700, 3.1347) -- (0.3040, 1.7700, 3.1409) -- (0.3040, 1.8240, 3.1391) -- (0.2580, 1.8240, 3.1329) -- cycle;
\fill[blue!62.5, opacity=0.7] (0.2580, 1.8240, 3.1329) -- (0.3040, 1.8240, 3.1391) -- (0.3040, 1.8780, 3.1370) -- (0.2580, 1.8780, 3.1308) -- cycle;
\fill[blue!62.3, opacity=0.7] (0.2580, 1.8780, 3.1308) -- (0.3040, 1.8780, 3.1370) -- (0.3040, 1.9320, 3.1346) -- (0.2580, 1.9320, 3.1284) -- cycle;
\fill[blue!62.2, opacity=0.7] (0.2580, 1.9320, 3.1284) -- (0.3040, 1.9320, 3.1346) -- (0.3040, 1.9860, 3.1319) -- (0.2580, 1.9860, 3.1257) -- cycle;
\fill[blue!62.4, opacity=0.7] (0.2580, 1.9860, 3.1257) -- (0.3040, 1.9860, 3.1319) -- (0.3040, 2.0400, 3.1289) -- (0.2580, 2.0400, 3.1227) -- cycle;
\fill[blue!63.0, opacity=0.7] (0.2580, 2.0400, 3.1227) -- (0.3040, 2.0400, 3.1289) -- (0.3040, 2.0940, 3.1256) -- (0.2580, 2.0940, 3.1194) -- cycle;
\fill[blue!63.6, opacity=0.7] (0.2580, 2.0940, 3.1194) -- (0.3040, 2.0940, 3.1256) -- (0.3040, 2.1480, 3.1220) -- (0.2580, 2.1480, 3.1159) -- cycle;
\fill[blue!62.8, opacity=0.7] (0.2580, 2.1480, 3.1159) -- (0.3040, 2.1480, 3.1220) -- (0.3040, 2.2020, 3.1182) -- (0.2580, 2.2020, 3.1120) -- cycle;
\fill[blue!58.8, opacity=0.7] (0.2580, 2.2020, 3.1120) -- (0.3040, 2.2020, 3.1182) -- (0.3040, 2.2560, 3.1141) -- (0.2580, 2.2560, 3.1079) -- cycle;
\fill[blue!49.9, opacity=0.7] (0.2580, 2.2560, 3.1079) -- (0.3040, 2.2560, 3.1141) -- (0.3040, 2.3100, 3.1098) -- (0.2580, 2.3100, 3.1036) -- cycle;
\fill[blue!37.7, opacity=0.7] (0.2580, 2.3100, 3.1036) -- (0.3040, 2.3100, 3.1098) -- (0.3040, 2.3640, 3.1052) -- (0.2580, 2.3640, 3.0991) -- cycle;
\fill[blue!26.9, opacity=0.7] (0.2580, 2.3640, 3.0991) -- (0.3040, 2.3640, 3.1052) -- (0.3040, 2.4180, 3.1005) -- (0.2580, 2.4180, 3.0943) -- cycle;
\fill[blue!20.3, opacity=0.7] (0.2580, 2.4180, 3.0943) -- (0.3040, 2.4180, 3.1005) -- (0.3040, 2.4720, 3.0955) -- (0.2580, 2.4720, 3.0893) -- cycle;
\fill[blue!17.3, opacity=0.7] (0.2580, 2.4720, 3.0893) -- (0.3040, 2.4720, 3.0955) -- (0.3040, 2.5260, 3.0903) -- (0.2580, 2.5260, 3.0841) -- cycle;
\fill[blue!16.3, opacity=0.7] (0.2580, 2.5260, 3.0841) -- (0.3040, 2.5260, 3.0903) -- (0.3040, 2.5800, 3.0849) -- (0.2580, 2.5800, 3.0788) -- cycle;
\fill[blue!16.2, opacity=0.7] (0.2580, 2.5800, 3.0788) -- (0.3040, 2.5800, 3.0849) -- (0.3040, 2.6340, 3.0794) -- (0.2580, 2.6340, 3.0733) -- cycle;
\fill[blue!16.7, opacity=0.7] (0.2580, 2.6340, 3.0733) -- (0.3040, 2.6340, 3.0794) -- (0.3040, 2.6880, 3.0738) -- (0.2580, 2.6880, 3.0676) -- cycle;
\fill[blue!18.7, opacity=0.7] (0.2580, 2.6880, 3.0676) -- (0.3040, 2.6880, 3.0738) -- (0.3040, 2.7420, 3.0680) -- (0.2580, 2.7420, 3.0618) -- cycle;
\fill[blue!23.9, opacity=0.7] (0.2580, 2.7420, 3.0618) -- (0.3040, 2.7420, 3.0680) -- (0.3040, 2.7960, 3.0620) -- (0.2580, 2.7960, 3.0559) -- cycle;
\fill[blue!33.4, opacity=0.7] (0.2580, 2.7960, 3.0559) -- (0.3040, 2.7960, 3.0620) -- (0.3040, 2.8500, 3.0560) -- (0.2580, 2.8500, 3.0498) -- cycle;
\fill[blue!43.2, opacity=0.7] (0.2580, 2.8500, 3.0498) -- (0.3040, 2.8500, 3.0560) -- (0.3040, 2.9040, 3.0499) -- (0.2580, 2.9040, 3.0437) -- cycle;
\fill[blue!47.5, opacity=0.7] (0.2580, 2.9040, 3.0437) -- (0.3040, 2.9040, 3.0499) -- (0.3040, 2.9580, 3.0437) -- (0.2580, 2.9580, 3.0375) -- cycle;
\fill[blue!43.6, opacity=0.7] (0.2580, 2.9580, 3.0375) -- (0.3040, 2.9580, 3.0437) -- (0.3040, 3.0120, 3.0375) -- (0.2580, 3.0120, 3.0313) -- cycle;
\fill[blue!32.1, opacity=0.7] (0.2580, 3.0120, 3.0313) -- (0.3040, 3.0120, 3.0375) -- (0.3040, 3.0660, 3.0312) -- (0.2580, 3.0660, 3.0251) -- cycle;
\fill[blue!20.7, opacity=0.7] (0.2580, 3.0660, 3.0251) -- (0.3040, 3.0660, 3.0312) -- (0.3040, 3.1200, 3.0249) -- (0.2580, 3.1200, 3.0188) -- cycle;
\fill[blue!16.1, opacity=0.7] (0.3040, -0.1200, 3.0249) -- (0.3500, -0.1200, 3.0311) -- (0.3500, -0.0660, 3.0373) -- (0.3040, -0.0660, 3.0312) -- cycle;
\fill[blue!15.3, opacity=0.7] (0.3040, -0.0660, 3.0312) -- (0.3500, -0.0660, 3.0373) -- (0.3500, -0.0120, 3.0436) -- (0.3040, -0.0120, 3.0375) -- cycle;
\fill[blue!15.3, opacity=0.7] (0.3040, -0.0120, 3.0375) -- (0.3500, -0.0120, 3.0436) -- (0.3500, 0.0420, 3.0498) -- (0.3040, 0.0420, 3.0437) -- cycle;
\fill[blue!15.7, opacity=0.7] (0.3040, 0.0420, 3.0437) -- (0.3500, 0.0420, 3.0498) -- (0.3500, 0.0960, 3.0560) -- (0.3040, 0.0960, 3.0499) -- cycle;
\fill[blue!18.5, opacity=0.7] (0.3040, 0.0960, 3.0499) -- (0.3500, 0.0960, 3.0560) -- (0.3500, 0.1500, 3.0621) -- (0.3040, 0.1500, 3.0560) -- cycle;
\fill[blue!29.6, opacity=0.7] (0.3040, 0.1500, 3.0560) -- (0.3500, 0.1500, 3.0621) -- (0.3500, 0.2040, 3.0681) -- (0.3040, 0.2040, 3.0620) -- cycle;
\fill[blue!47.3, opacity=0.7] (0.3040, 0.2040, 3.0620) -- (0.3500, 0.2040, 3.0681) -- (0.3500, 0.2580, 3.0741) -- (0.3040, 0.2580, 3.0680) -- cycle;
\fill[blue!57.7, opacity=0.7] (0.3040, 0.2580, 3.0680) -- (0.3500, 0.2580, 3.0741) -- (0.3500, 0.3120, 3.0799) -- (0.3040, 0.3120, 3.0738) -- cycle;
\fill[blue!59.0, opacity=0.7] (0.3040, 0.3120, 3.0738) -- (0.3500, 0.3120, 3.0799) -- (0.3500, 0.3660, 3.0855) -- (0.3040, 0.3660, 3.0794) -- cycle;
\fill[blue!53.4, opacity=0.7] (0.3040, 0.3660, 3.0794) -- (0.3500, 0.3660, 3.0855) -- (0.3500, 0.4200, 3.0911) -- (0.3040, 0.4200, 3.0849) -- cycle;
\fill[blue!40.5, opacity=0.7] (0.3040, 0.4200, 3.0849) -- (0.3500, 0.4200, 3.0911) -- (0.3500, 0.4740, 3.0964) -- (0.3040, 0.4740, 3.0903) -- cycle;
\fill[blue!27.5, opacity=0.7] (0.3040, 0.4740, 3.0903) -- (0.3500, 0.4740, 3.0964) -- (0.3500, 0.5280, 3.1016) -- (0.3040, 0.5280, 3.0955) -- cycle;
\fill[blue!20.6, opacity=0.7] (0.3040, 0.5280, 3.0955) -- (0.3500, 0.5280, 3.1016) -- (0.3500, 0.5820, 3.1066) -- (0.3040, 0.5820, 3.1005) -- cycle;
\fill[blue!18.3, opacity=0.7] (0.3040, 0.5820, 3.1005) -- (0.3500, 0.5820, 3.1066) -- (0.3500, 0.6360, 3.1114) -- (0.3040, 0.6360, 3.1052) -- cycle;
\fill[blue!18.2, opacity=0.7] (0.3040, 0.6360, 3.1052) -- (0.3500, 0.6360, 3.1114) -- (0.3500, 0.6900, 3.1159) -- (0.3040, 0.6900, 3.1098) -- cycle;
\fill[blue!20.1, opacity=0.7] (0.3040, 0.6900, 3.1098) -- (0.3500, 0.6900, 3.1159) -- (0.3500, 0.7440, 3.1202) -- (0.3040, 0.7440, 3.1141) -- cycle;
\fill[blue!25.9, opacity=0.7] (0.3040, 0.7440, 3.1141) -- (0.3500, 0.7440, 3.1202) -- (0.3500, 0.7980, 3.1243) -- (0.3040, 0.7980, 3.1182) -- cycle;
\fill[blue!37.3, opacity=0.7] (0.3040, 0.7980, 3.1182) -- (0.3500, 0.7980, 3.1243) -- (0.3500, 0.8520, 3.1281) -- (0.3040, 0.8520, 3.1220) -- cycle;
\fill[blue!51.7, opacity=0.7] (0.3040, 0.8520, 3.1220) -- (0.3500, 0.8520, 3.1281) -- (0.3500, 0.9060, 3.1317) -- (0.3040, 0.9060, 3.1256) -- cycle;
\fill[blue!61.4, opacity=0.7] (0.3040, 0.9060, 3.1256) -- (0.3500, 0.9060, 3.1317) -- (0.3500, 0.9600, 3.1350) -- (0.3040, 0.9600, 3.1289) -- cycle;
\fill[blue!63.5, opacity=0.7] (0.3040, 0.9600, 3.1289) -- (0.3500, 0.9600, 3.1350) -- (0.3500, 1.0140, 3.1380) -- (0.3040, 1.0140, 3.1319) -- cycle;
\fill[blue!61.7, opacity=0.7] (0.3040, 1.0140, 3.1319) -- (0.3500, 1.0140, 3.1380) -- (0.3500, 1.0680, 3.1407) -- (0.3040, 1.0680, 3.1346) -- cycle;
\fill[blue!59.9, opacity=0.7] (0.3040, 1.0680, 3.1346) -- (0.3500, 1.0680, 3.1407) -- (0.3500, 1.1220, 3.1431) -- (0.3040, 1.1220, 3.1370) -- cycle;
\fill[blue!59.6, opacity=0.7] (0.3040, 1.1220, 3.1370) -- (0.3500, 1.1220, 3.1431) -- (0.3500, 1.1760, 3.1452) -- (0.3040, 1.1760, 3.1391) -- cycle;
\fill[blue!60.6, opacity=0.7] (0.3040, 1.1760, 3.1391) -- (0.3500, 1.1760, 3.1452) -- (0.3500, 1.2300, 3.1470) -- (0.3040, 1.2300, 3.1409) -- cycle;
\fill[blue!62.2, opacity=0.7] (0.3040, 1.2300, 3.1409) -- (0.3500, 1.2300, 3.1470) -- (0.3500, 1.2840, 3.1484) -- (0.3040, 1.2840, 3.1423) -- cycle;
\fill[blue!63.3, opacity=0.7] (0.3040, 1.2840, 3.1423) -- (0.3500, 1.2840, 3.1484) -- (0.3500, 1.3380, 3.1496) -- (0.3040, 1.3380, 3.1435) -- cycle;
\fill[blue!63.5, opacity=0.7] (0.3040, 1.3380, 3.1435) -- (0.3500, 1.3380, 3.1496) -- (0.3500, 1.3920, 3.1504) -- (0.3040, 1.3920, 3.1443) -- cycle;
\fill[blue!62.9, opacity=0.7] (0.3040, 1.3920, 3.1443) -- (0.3500, 1.3920, 3.1504) -- (0.3500, 1.4460, 3.1509) -- (0.3040, 1.4460, 3.1448) -- cycle;
\fill[blue!61.7, opacity=0.7] (0.3040, 1.4460, 3.1448) -- (0.3500, 1.4460, 3.1509) -- (0.3500, 1.5000, 3.1511) -- (0.3040, 1.5000, 3.1449) -- cycle;
\fill[blue!60.4, opacity=0.7] (0.3040, 1.5000, 3.1449) -- (0.3500, 1.5000, 3.1511) -- (0.3500, 1.5540, 3.1509) -- (0.3040, 1.5540, 3.1448) -- cycle;
\fill[blue!59.3, opacity=0.7] (0.3040, 1.5540, 3.1448) -- (0.3500, 1.5540, 3.1509) -- (0.3500, 1.6080, 3.1504) -- (0.3040, 1.6080, 3.1443) -- cycle;
\fill[blue!58.7, opacity=0.7] (0.3040, 1.6080, 3.1443) -- (0.3500, 1.6080, 3.1504) -- (0.3500, 1.6620, 3.1496) -- (0.3040, 1.6620, 3.1435) -- cycle;
\fill[blue!58.6, opacity=0.7] (0.3040, 1.6620, 3.1435) -- (0.3500, 1.6620, 3.1496) -- (0.3500, 1.7160, 3.1484) -- (0.3040, 1.7160, 3.1423) -- cycle;
\fill[blue!59.2, opacity=0.7] (0.3040, 1.7160, 3.1423) -- (0.3500, 1.7160, 3.1484) -- (0.3500, 1.7700, 3.1470) -- (0.3040, 1.7700, 3.1409) -- cycle;
\fill[blue!60.2, opacity=0.7] (0.3040, 1.7700, 3.1409) -- (0.3500, 1.7700, 3.1470) -- (0.3500, 1.8240, 3.1452) -- (0.3040, 1.8240, 3.1391) -- cycle;
\fill[blue!61.5, opacity=0.7] (0.3040, 1.8240, 3.1391) -- (0.3500, 1.8240, 3.1452) -- (0.3500, 1.8780, 3.1431) -- (0.3040, 1.8780, 3.1370) -- cycle;
\fill[blue!62.6, opacity=0.7] (0.3040, 1.8780, 3.1370) -- (0.3500, 1.8780, 3.1431) -- (0.3500, 1.9320, 3.1407) -- (0.3040, 1.9320, 3.1346) -- cycle;
\fill[blue!63.4, opacity=0.7] (0.3040, 1.9320, 3.1346) -- (0.3500, 1.9320, 3.1407) -- (0.3500, 1.9860, 3.1380) -- (0.3040, 1.9860, 3.1319) -- cycle;
\fill[blue!63.5, opacity=0.7] (0.3040, 1.9860, 3.1319) -- (0.3500, 1.9860, 3.1380) -- (0.3500, 2.0400, 3.1350) -- (0.3040, 2.0400, 3.1289) -- cycle;
\fill[blue!63.2, opacity=0.7] (0.3040, 2.0400, 3.1289) -- (0.3500, 2.0400, 3.1350) -- (0.3500, 2.0940, 3.1317) -- (0.3040, 2.0940, 3.1256) -- cycle;
\fill[blue!62.7, opacity=0.7] (0.3040, 2.0940, 3.1256) -- (0.3500, 2.0940, 3.1317) -- (0.3500, 2.1480, 3.1281) -- (0.3040, 2.1480, 3.1220) -- cycle;
\fill[blue!62.7, opacity=0.7] (0.3040, 2.1480, 3.1220) -- (0.3500, 2.1480, 3.1281) -- (0.3500, 2.2020, 3.1243) -- (0.3040, 2.2020, 3.1182) -- cycle;
\fill[blue!63.3, opacity=0.7] (0.3040, 2.2020, 3.1182) -- (0.3500, 2.2020, 3.1243) -- (0.3500, 2.2560, 3.1202) -- (0.3040, 2.2560, 3.1141) -- cycle;
\fill[blue!63.4, opacity=0.7] (0.3040, 2.2560, 3.1141) -- (0.3500, 2.2560, 3.1202) -- (0.3500, 2.3100, 3.1159) -- (0.3040, 2.3100, 3.1098) -- cycle;
\fill[blue!59.9, opacity=0.7] (0.3040, 2.3100, 3.1098) -- (0.3500, 2.3100, 3.1159) -- (0.3500, 2.3640, 3.1114) -- (0.3040, 2.3640, 3.1052) -- cycle;
\fill[blue!50.1, opacity=0.7] (0.3040, 2.3640, 3.1052) -- (0.3500, 2.3640, 3.1114) -- (0.3500, 2.4180, 3.1066) -- (0.3040, 2.4180, 3.1005) -- cycle;
\fill[blue!36.1, opacity=0.7] (0.3040, 2.4180, 3.1005) -- (0.3500, 2.4180, 3.1066) -- (0.3500, 2.4720, 3.1016) -- (0.3040, 2.4720, 3.0955) -- cycle;
\fill[blue!24.7, opacity=0.7] (0.3040, 2.4720, 3.0955) -- (0.3500, 2.4720, 3.1016) -- (0.3500, 2.5260, 3.0964) -- (0.3040, 2.5260, 3.0903) -- cycle;
\fill[blue!18.8, opacity=0.7] (0.3040, 2.5260, 3.0903) -- (0.3500, 2.5260, 3.0964) -- (0.3500, 2.5800, 3.0911) -- (0.3040, 2.5800, 3.0849) -- cycle;
\fill[blue!16.7, opacity=0.7] (0.3040, 2.5800, 3.0849) -- (0.3500, 2.5800, 3.0911) -- (0.3500, 2.6340, 3.0855) -- (0.3040, 2.6340, 3.0794) -- cycle;
\fill[blue!16.1, opacity=0.7] (0.3040, 2.6340, 3.0794) -- (0.3500, 2.6340, 3.0855) -- (0.3500, 2.6880, 3.0799) -- (0.3040, 2.6880, 3.0738) -- cycle;
\fill[blue!16.3, opacity=0.7] (0.3040, 2.6880, 3.0738) -- (0.3500, 2.6880, 3.0799) -- (0.3500, 2.7420, 3.0741) -- (0.3040, 2.7420, 3.0680) -- cycle;
\fill[blue!17.7, opacity=0.7] (0.3040, 2.7420, 3.0680) -- (0.3500, 2.7420, 3.0741) -- (0.3500, 2.7960, 3.0681) -- (0.3040, 2.7960, 3.0620) -- cycle;
\fill[blue!21.9, opacity=0.7] (0.3040, 2.7960, 3.0620) -- (0.3500, 2.7960, 3.0681) -- (0.3500, 2.8500, 3.0621) -- (0.3040, 2.8500, 3.0560) -- cycle;
\fill[blue!30.9, opacity=0.7] (0.3040, 2.8500, 3.0560) -- (0.3500, 2.8500, 3.0621) -- (0.3500, 2.9040, 3.0560) -- (0.3040, 2.9040, 3.0499) -- cycle;
\fill[blue!41.5, opacity=0.7] (0.3040, 2.9040, 3.0499) -- (0.3500, 2.9040, 3.0560) -- (0.3500, 2.9580, 3.0498) -- (0.3040, 2.9580, 3.0437) -- cycle;
\fill[blue!47.0, opacity=0.7] (0.3040, 2.9580, 3.0437) -- (0.3500, 2.9580, 3.0498) -- (0.3500, 3.0120, 3.0436) -- (0.3040, 3.0120, 3.0375) -- cycle;
\fill[blue!43.8, opacity=0.7] (0.3040, 3.0120, 3.0375) -- (0.3500, 3.0120, 3.0436) -- (0.3500, 3.0660, 3.0373) -- (0.3040, 3.0660, 3.0312) -- cycle;
\fill[blue!32.2, opacity=0.7] (0.3040, 3.0660, 3.0312) -- (0.3500, 3.0660, 3.0373) -- (0.3500, 3.1200, 3.0311) -- (0.3040, 3.1200, 3.0249) -- cycle;
\fill[blue!15.4, opacity=0.7] (0.3500, -0.1200, 3.0311) -- (0.3960, -0.1200, 3.0371) -- (0.3960, -0.0660, 3.0434) -- (0.3500, -0.0660, 3.0373) -- cycle;
\fill[blue!15.2, opacity=0.7] (0.3500, -0.0660, 3.0373) -- (0.3960, -0.0660, 3.0434) -- (0.3960, -0.0120, 3.0496) -- (0.3500, -0.0120, 3.0436) -- cycle;
\fill[blue!15.5, opacity=0.7] (0.3500, -0.0120, 3.0436) -- (0.3960, -0.0120, 3.0496) -- (0.3960, 0.0420, 3.0559) -- (0.3500, 0.0420, 3.0498) -- cycle;
\fill[blue!17.7, opacity=0.7] (0.3500, 0.0420, 3.0498) -- (0.3960, 0.0420, 3.0559) -- (0.3960, 0.0960, 3.0620) -- (0.3500, 0.0960, 3.0560) -- cycle;
\fill[blue!27.6, opacity=0.7] (0.3500, 0.0960, 3.0560) -- (0.3960, 0.0960, 3.0620) -- (0.3960, 0.1500, 3.0681) -- (0.3500, 0.1500, 3.0621) -- cycle;
\fill[blue!45.9, opacity=0.7] (0.3500, 0.1500, 3.0621) -- (0.3960, 0.1500, 3.0681) -- (0.3960, 0.2040, 3.0742) -- (0.3500, 0.2040, 3.0681) -- cycle;
\fill[blue!57.6, opacity=0.7] (0.3500, 0.2040, 3.0681) -- (0.3960, 0.2040, 3.0742) -- (0.3960, 0.2580, 3.0801) -- (0.3500, 0.2580, 3.0741) -- cycle;
\fill[blue!59.3, opacity=0.7] (0.3500, 0.2580, 3.0741) -- (0.3960, 0.2580, 3.0801) -- (0.3960, 0.3120, 3.0859) -- (0.3500, 0.3120, 3.0799) -- cycle;
\fill[blue!53.6, opacity=0.7] (0.3500, 0.3120, 3.0799) -- (0.3960, 0.3120, 3.0859) -- (0.3960, 0.3660, 3.0916) -- (0.3500, 0.3660, 3.0855) -- cycle;
\fill[blue!40.0, opacity=0.7] (0.3500, 0.3660, 3.0855) -- (0.3960, 0.3660, 3.0916) -- (0.3960, 0.4200, 3.0971) -- (0.3500, 0.4200, 3.0911) -- cycle;
\fill[blue!26.7, opacity=0.7] (0.3500, 0.4200, 3.0911) -- (0.3960, 0.4200, 3.0971) -- (0.3960, 0.4740, 3.1024) -- (0.3500, 0.4740, 3.0964) -- cycle;
\fill[blue!20.2, opacity=0.7] (0.3500, 0.4740, 3.0964) -- (0.3960, 0.4740, 3.1024) -- (0.3960, 0.5280, 3.1076) -- (0.3500, 0.5280, 3.1016) -- cycle;
\fill[blue!18.2, opacity=0.7] (0.3500, 0.5280, 3.1016) -- (0.3960, 0.5280, 3.1076) -- (0.3960, 0.5820, 3.1126) -- (0.3500, 0.5820, 3.1066) -- cycle;
\fill[blue!18.7, opacity=0.7] (0.3500, 0.5820, 3.1066) -- (0.3960, 0.5820, 3.1126) -- (0.3960, 0.6360, 3.1174) -- (0.3500, 0.6360, 3.1114) -- cycle;
\fill[blue!22.0, opacity=0.7] (0.3500, 0.6360, 3.1114) -- (0.3960, 0.6360, 3.1174) -- (0.3960, 0.6900, 3.1219) -- (0.3500, 0.6900, 3.1159) -- cycle;
\fill[blue!31.0, opacity=0.7] (0.3500, 0.6900, 3.1159) -- (0.3960, 0.6900, 3.1219) -- (0.3960, 0.7440, 3.1263) -- (0.3500, 0.7440, 3.1202) -- cycle;
\fill[blue!46.4, opacity=0.7] (0.3500, 0.7440, 3.1202) -- (0.3960, 0.7440, 3.1263) -- (0.3960, 0.7980, 3.1303) -- (0.3500, 0.7980, 3.1243) -- cycle;
\fill[blue!59.8, opacity=0.7] (0.3500, 0.7980, 3.1243) -- (0.3960, 0.7980, 3.1303) -- (0.3960, 0.8520, 3.1342) -- (0.3500, 0.8520, 3.1281) -- cycle;
\fill[blue!63.6, opacity=0.7] (0.3500, 0.8520, 3.1281) -- (0.3960, 0.8520, 3.1342) -- (0.3960, 0.9060, 3.1377) -- (0.3500, 0.9060, 3.1317) -- cycle;
\fill[blue!61.3, opacity=0.7] (0.3500, 0.9060, 3.1317) -- (0.3960, 0.9060, 3.1377) -- (0.3960, 0.9600, 3.1410) -- (0.3500, 0.9600, 3.1350) -- cycle;
\fill[blue!59.1, opacity=0.7] (0.3500, 0.9600, 3.1350) -- (0.3960, 0.9600, 3.1410) -- (0.3960, 1.0140, 3.1440) -- (0.3500, 1.0140, 3.1380) -- cycle;
\fill[blue!59.4, opacity=0.7] (0.3500, 1.0140, 3.1380) -- (0.3960, 1.0140, 3.1440) -- (0.3960, 1.0680, 3.1467) -- (0.3500, 1.0680, 3.1407) -- cycle;
\fill[blue!61.4, opacity=0.7] (0.3500, 1.0680, 3.1407) -- (0.3960, 1.0680, 3.1467) -- (0.3960, 1.1220, 3.1491) -- (0.3500, 1.1220, 3.1431) -- cycle;
\fill[blue!63.3, opacity=0.7] (0.3500, 1.1220, 3.1431) -- (0.3960, 1.1220, 3.1491) -- (0.3960, 1.1760, 3.1512) -- (0.3500, 1.1760, 3.1452) -- cycle;
\fill[blue!63.2, opacity=0.7] (0.3500, 1.1760, 3.1452) -- (0.3960, 1.1760, 3.1512) -- (0.3960, 1.2300, 3.1530) -- (0.3500, 1.2300, 3.1470) -- cycle;
\fill[blue!60.7, opacity=0.7] (0.3500, 1.2300, 3.1470) -- (0.3960, 1.2300, 3.1530) -- (0.3960, 1.2840, 3.1545) -- (0.3500, 1.2840, 3.1484) -- cycle;
\fill[blue!56.4, opacity=0.7] (0.3500, 1.2840, 3.1484) -- (0.3960, 1.2840, 3.1545) -- (0.3960, 1.3380, 3.1556) -- (0.3500, 1.3380, 3.1496) -- cycle;
\fill[blue!51.6, opacity=0.7] (0.3500, 1.3380, 3.1496) -- (0.3960, 1.3380, 3.1556) -- (0.3960, 1.3920, 3.1564) -- (0.3500, 1.3920, 3.1504) -- cycle;
\fill[blue!47.3, opacity=0.7] (0.3500, 1.3920, 3.1504) -- (0.3960, 1.3920, 3.1564) -- (0.3960, 1.4460, 3.1569) -- (0.3500, 1.4460, 3.1509) -- cycle;
\fill[blue!43.8, opacity=0.7] (0.3500, 1.4460, 3.1509) -- (0.3960, 1.4460, 3.1569) -- (0.3960, 1.5000, 3.1571) -- (0.3500, 1.5000, 3.1511) -- cycle;
\fill[blue!41.3, opacity=0.7] (0.3500, 1.5000, 3.1511) -- (0.3960, 1.5000, 3.1571) -- (0.3960, 1.5540, 3.1569) -- (0.3500, 1.5540, 3.1509) -- cycle;
\fill[blue!39.7, opacity=0.7] (0.3500, 1.5540, 3.1509) -- (0.3960, 1.5540, 3.1569) -- (0.3960, 1.6080, 3.1564) -- (0.3500, 1.6080, 3.1504) -- cycle;
\fill[blue!38.8, opacity=0.7] (0.3500, 1.6080, 3.1504) -- (0.3960, 1.6080, 3.1564) -- (0.3960, 1.6620, 3.1556) -- (0.3500, 1.6620, 3.1496) -- cycle;
\fill[blue!38.7, opacity=0.7] (0.3500, 1.6620, 3.1496) -- (0.3960, 1.6620, 3.1556) -- (0.3960, 1.7160, 3.1545) -- (0.3500, 1.7160, 3.1484) -- cycle;
\fill[blue!39.4, opacity=0.7] (0.3500, 1.7160, 3.1484) -- (0.3960, 1.7160, 3.1545) -- (0.3960, 1.7700, 3.1530) -- (0.3500, 1.7700, 3.1470) -- cycle;
\fill[blue!40.8, opacity=0.7] (0.3500, 1.7700, 3.1470) -- (0.3960, 1.7700, 3.1530) -- (0.3960, 1.8240, 3.1512) -- (0.3500, 1.8240, 3.1452) -- cycle;
\fill[blue!43.2, opacity=0.7] (0.3500, 1.8240, 3.1452) -- (0.3960, 1.8240, 3.1512) -- (0.3960, 1.8780, 3.1491) -- (0.3500, 1.8780, 3.1431) -- cycle;
\fill[blue!46.6, opacity=0.7] (0.3500, 1.8780, 3.1431) -- (0.3960, 1.8780, 3.1491) -- (0.3960, 1.9320, 3.1467) -- (0.3500, 1.9320, 3.1407) -- cycle;
\fill[blue!50.9, opacity=0.7] (0.3500, 1.9320, 3.1407) -- (0.3960, 1.9320, 3.1467) -- (0.3960, 1.9860, 3.1440) -- (0.3500, 1.9860, 3.1380) -- cycle;
\fill[blue!55.6, opacity=0.7] (0.3500, 1.9860, 3.1380) -- (0.3960, 1.9860, 3.1440) -- (0.3960, 2.0400, 3.1410) -- (0.3500, 2.0400, 3.1350) -- cycle;
\fill[blue!59.8, opacity=0.7] (0.3500, 2.0400, 3.1350) -- (0.3960, 2.0400, 3.1410) -- (0.3960, 2.0940, 3.1377) -- (0.3500, 2.0940, 3.1317) -- cycle;
\fill[blue!62.7, opacity=0.7] (0.3500, 2.0940, 3.1317) -- (0.3960, 2.0940, 3.1377) -- (0.3960, 2.1480, 3.1342) -- (0.3500, 2.1480, 3.1281) -- cycle;
\fill[blue!63.6, opacity=0.7] (0.3500, 2.1480, 3.1281) -- (0.3960, 2.1480, 3.1342) -- (0.3960, 2.2020, 3.1303) -- (0.3500, 2.2020, 3.1243) -- cycle;
\fill[blue!63.2, opacity=0.7] (0.3500, 2.2020, 3.1243) -- (0.3960, 2.2020, 3.1303) -- (0.3960, 2.2560, 3.1263) -- (0.3500, 2.2560, 3.1202) -- cycle;
\fill[blue!62.9, opacity=0.7] (0.3500, 2.2560, 3.1202) -- (0.3960, 2.2560, 3.1263) -- (0.3960, 2.3100, 3.1219) -- (0.3500, 2.3100, 3.1159) -- cycle;
\fill[blue!63.3, opacity=0.7] (0.3500, 2.3100, 3.1159) -- (0.3960, 2.3100, 3.1219) -- (0.3960, 2.3640, 3.1174) -- (0.3500, 2.3640, 3.1114) -- cycle;
\fill[blue!63.2, opacity=0.7] (0.3500, 2.3640, 3.1114) -- (0.3960, 2.3640, 3.1174) -- (0.3960, 2.4180, 3.1126) -- (0.3500, 2.4180, 3.1066) -- cycle;
\fill[blue!58.3, opacity=0.7] (0.3500, 2.4180, 3.1066) -- (0.3960, 2.4180, 3.1126) -- (0.3960, 2.4720, 3.1076) -- (0.3500, 2.4720, 3.1016) -- cycle;
\fill[blue!45.7, opacity=0.7] (0.3500, 2.4720, 3.1016) -- (0.3960, 2.4720, 3.1076) -- (0.3960, 2.5260, 3.1024) -- (0.3500, 2.5260, 3.0964) -- cycle;
\fill[blue!30.6, opacity=0.7] (0.3500, 2.5260, 3.0964) -- (0.3960, 2.5260, 3.1024) -- (0.3960, 2.5800, 3.0971) -- (0.3500, 2.5800, 3.0911) -- cycle;
\fill[blue!21.0, opacity=0.7] (0.3500, 2.5800, 3.0911) -- (0.3960, 2.5800, 3.0971) -- (0.3960, 2.6340, 3.0916) -- (0.3500, 2.6340, 3.0855) -- cycle;
\fill[blue!17.2, opacity=0.7] (0.3500, 2.6340, 3.0855) -- (0.3960, 2.6340, 3.0916) -- (0.3960, 2.6880, 3.0859) -- (0.3500, 2.6880, 3.0799) -- cycle;
\fill[blue!16.2, opacity=0.7] (0.3500, 2.6880, 3.0799) -- (0.3960, 2.6880, 3.0859) -- (0.3960, 2.7420, 3.0801) -- (0.3500, 2.7420, 3.0741) -- cycle;
\fill[blue!16.1, opacity=0.7] (0.3500, 2.7420, 3.0741) -- (0.3960, 2.7420, 3.0801) -- (0.3960, 2.7960, 3.0742) -- (0.3500, 2.7960, 3.0681) -- cycle;
\fill[blue!17.1, opacity=0.7] (0.3500, 2.7960, 3.0681) -- (0.3960, 2.7960, 3.0742) -- (0.3960, 2.8500, 3.0681) -- (0.3500, 2.8500, 3.0621) -- cycle;
\fill[blue!20.7, opacity=0.7] (0.3500, 2.8500, 3.0621) -- (0.3960, 2.8500, 3.0681) -- (0.3960, 2.9040, 3.0620) -- (0.3500, 2.9040, 3.0560) -- cycle;
\fill[blue!29.3, opacity=0.7] (0.3500, 2.9040, 3.0560) -- (0.3960, 2.9040, 3.0620) -- (0.3960, 2.9580, 3.0559) -- (0.3500, 2.9580, 3.0498) -- cycle;
\fill[blue!40.5, opacity=0.7] (0.3500, 2.9580, 3.0498) -- (0.3960, 2.9580, 3.0559) -- (0.3960, 3.0120, 3.0496) -- (0.3500, 3.0120, 3.0436) -- cycle;
\fill[blue!46.6, opacity=0.7] (0.3500, 3.0120, 3.0436) -- (0.3960, 3.0120, 3.0496) -- (0.3960, 3.0660, 3.0434) -- (0.3500, 3.0660, 3.0373) -- cycle;
\fill[blue!43.4, opacity=0.7] (0.3500, 3.0660, 3.0373) -- (0.3960, 3.0660, 3.0434) -- (0.3960, 3.1200, 3.0371) -- (0.3500, 3.1200, 3.0311) -- cycle;
\fill[blue!15.3, opacity=0.7] (0.3960, -0.1200, 3.0371) -- (0.4420, -0.1200, 3.0430) -- (0.4420, -0.0660, 3.0493) -- (0.3960, -0.0660, 3.0434) -- cycle;
\fill[blue!15.4, opacity=0.7] (0.3960, -0.0660, 3.0434) -- (0.4420, -0.0660, 3.0493) -- (0.4420, -0.0120, 3.0555) -- (0.3960, -0.0120, 3.0496) -- cycle;
\fill[blue!16.9, opacity=0.7] (0.3960, -0.0120, 3.0496) -- (0.4420, -0.0120, 3.0555) -- (0.4420, 0.0420, 3.0618) -- (0.3960, 0.0420, 3.0559) -- cycle;
\fill[blue!24.7, opacity=0.7] (0.3960, 0.0420, 3.0559) -- (0.4420, 0.0420, 3.0618) -- (0.4420, 0.0960, 3.0680) -- (0.3960, 0.0960, 3.0620) -- cycle;
\fill[blue!43.0, opacity=0.7] (0.3960, 0.0960, 3.0620) -- (0.4420, 0.0960, 3.0680) -- (0.4420, 0.1500, 3.0741) -- (0.3960, 0.1500, 3.0681) -- cycle;
\fill[blue!56.9, opacity=0.7] (0.3960, 0.1500, 3.0681) -- (0.4420, 0.1500, 3.0741) -- (0.4420, 0.2040, 3.0801) -- (0.3960, 0.2040, 3.0742) -- cycle;
\fill[blue!59.7, opacity=0.7] (0.3960, 0.2040, 3.0742) -- (0.4420, 0.2040, 3.0801) -- (0.4420, 0.2580, 3.0860) -- (0.3960, 0.2580, 3.0801) -- cycle;
\fill[blue!54.7, opacity=0.7] (0.3960, 0.2580, 3.0801) -- (0.4420, 0.2580, 3.0860) -- (0.4420, 0.3120, 3.0918) -- (0.3960, 0.3120, 3.0859) -- cycle;
\fill[blue!41.0, opacity=0.7] (0.3960, 0.3120, 3.0859) -- (0.4420, 0.3120, 3.0918) -- (0.4420, 0.3660, 3.0975) -- (0.3960, 0.3660, 3.0916) -- cycle;
\fill[blue!26.9, opacity=0.7] (0.3960, 0.3660, 3.0916) -- (0.4420, 0.3660, 3.0975) -- (0.4420, 0.4200, 3.1030) -- (0.3960, 0.4200, 3.0971) -- cycle;
\fill[blue!20.1, opacity=0.7] (0.3960, 0.4200, 3.0971) -- (0.4420, 0.4200, 3.1030) -- (0.4420, 0.4740, 3.1084) -- (0.3960, 0.4740, 3.1024) -- cycle;
\fill[blue!18.3, opacity=0.7] (0.3960, 0.4740, 3.1024) -- (0.4420, 0.4740, 3.1084) -- (0.4420, 0.5280, 3.1135) -- (0.3960, 0.5280, 3.1076) -- cycle;
\fill[blue!19.1, opacity=0.7] (0.3960, 0.5280, 3.1076) -- (0.4420, 0.5280, 3.1135) -- (0.4420, 0.5820, 3.1185) -- (0.3960, 0.5820, 3.1126) -- cycle;
\fill[blue!23.8, opacity=0.7] (0.3960, 0.5820, 3.1126) -- (0.4420, 0.5820, 3.1185) -- (0.4420, 0.6360, 3.1233) -- (0.3960, 0.6360, 3.1174) -- cycle;
\fill[blue!36.0, opacity=0.7] (0.3960, 0.6360, 3.1174) -- (0.4420, 0.6360, 3.1233) -- (0.4420, 0.6900, 3.1279) -- (0.3960, 0.6900, 3.1219) -- cycle;
\fill[blue!53.2, opacity=0.7] (0.3960, 0.6900, 3.1219) -- (0.4420, 0.6900, 3.1279) -- (0.4420, 0.7440, 3.1322) -- (0.3960, 0.7440, 3.1263) -- cycle;
\fill[blue!63.0, opacity=0.7] (0.3960, 0.7440, 3.1263) -- (0.4420, 0.7440, 3.1322) -- (0.4420, 0.7980, 3.1363) -- (0.3960, 0.7980, 3.1303) -- cycle;
\fill[blue!62.2, opacity=0.7] (0.3960, 0.7980, 3.1303) -- (0.4420, 0.7980, 3.1363) -- (0.4420, 0.8520, 3.1401) -- (0.3960, 0.8520, 3.1342) -- cycle;
\fill[blue!59.0, opacity=0.7] (0.3960, 0.8520, 3.1342) -- (0.4420, 0.8520, 3.1401) -- (0.4420, 0.9060, 3.1436) -- (0.3960, 0.9060, 3.1377) -- cycle;
\fill[blue!58.7, opacity=0.7] (0.3960, 0.9060, 3.1377) -- (0.4420, 0.9060, 3.1436) -- (0.4420, 0.9600, 3.1469) -- (0.3960, 0.9600, 3.1410) -- cycle;
\fill[blue!61.2, opacity=0.7] (0.3960, 0.9600, 3.1410) -- (0.4420, 0.9600, 3.1469) -- (0.4420, 1.0140, 3.1499) -- (0.3960, 1.0140, 3.1440) -- cycle;
\fill[blue!63.5, opacity=0.7] (0.3960, 1.0140, 3.1440) -- (0.4420, 1.0140, 3.1499) -- (0.4420, 1.0680, 3.1526) -- (0.3960, 1.0680, 3.1467) -- cycle;
\fill[blue!62.2, opacity=0.7] (0.3960, 1.0680, 3.1467) -- (0.4420, 1.0680, 3.1526) -- (0.4420, 1.1220, 3.1550) -- (0.3960, 1.1220, 3.1491) -- cycle;
\fill[blue!56.9, opacity=0.7] (0.3960, 1.1220, 3.1491) -- (0.4420, 1.1220, 3.1550) -- (0.4420, 1.1760, 3.1571) -- (0.3960, 1.1760, 3.1512) -- cycle;
\fill[blue!49.6, opacity=0.7] (0.3960, 1.1760, 3.1512) -- (0.4420, 1.1760, 3.1571) -- (0.4420, 1.2300, 3.1589) -- (0.3960, 1.2300, 3.1530) -- cycle;
\fill[blue!42.9, opacity=0.7] (0.3960, 1.2300, 3.1530) -- (0.4420, 1.2300, 3.1589) -- (0.4420, 1.2840, 3.1604) -- (0.3960, 1.2840, 3.1545) -- cycle;
\fill[blue!37.7, opacity=0.7] (0.3960, 1.2840, 3.1545) -- (0.4420, 1.2840, 3.1604) -- (0.4420, 1.3380, 3.1615) -- (0.3960, 1.3380, 3.1556) -- cycle;
\fill[blue!34.2, opacity=0.7] (0.3960, 1.3380, 3.1556) -- (0.4420, 1.3380, 3.1615) -- (0.4420, 1.3920, 3.1623) -- (0.3960, 1.3920, 3.1564) -- cycle;
\fill[blue!32.1, opacity=0.7] (0.3960, 1.3920, 3.1564) -- (0.4420, 1.3920, 3.1623) -- (0.4420, 1.4460, 3.1628) -- (0.3960, 1.4460, 3.1569) -- cycle;
\fill[blue!30.7, opacity=0.7] (0.3960, 1.4460, 3.1569) -- (0.4420, 1.4460, 3.1628) -- (0.4420, 1.5000, 3.1630) -- (0.3960, 1.5000, 3.1571) -- cycle;
\fill[blue!29.9, opacity=0.7] (0.3960, 1.5000, 3.1571) -- (0.4420, 1.5000, 3.1630) -- (0.4420, 1.5540, 3.1628) -- (0.3960, 1.5540, 3.1569) -- cycle;
\fill[blue!29.4, opacity=0.7] (0.3960, 1.5540, 3.1569) -- (0.4420, 1.5540, 3.1628) -- (0.4420, 1.6080, 3.1623) -- (0.3960, 1.6080, 3.1564) -- cycle;
\fill[blue!29.1, opacity=0.7] (0.3960, 1.6080, 3.1564) -- (0.4420, 1.6080, 3.1623) -- (0.4420, 1.6620, 3.1615) -- (0.3960, 1.6620, 3.1556) -- cycle;
\fill[blue!28.8, opacity=0.7] (0.3960, 1.6620, 3.1556) -- (0.4420, 1.6620, 3.1615) -- (0.4420, 1.7160, 3.1604) -- (0.3960, 1.7160, 3.1545) -- cycle;
\fill[blue!28.7, opacity=0.7] (0.3960, 1.7160, 3.1545) -- (0.4420, 1.7160, 3.1604) -- (0.4420, 1.7700, 3.1589) -- (0.3960, 1.7700, 3.1530) -- cycle;
\fill[blue!28.7, opacity=0.7] (0.3960, 1.7700, 3.1530) -- (0.4420, 1.7700, 3.1589) -- (0.4420, 1.8240, 3.1571) -- (0.3960, 1.8240, 3.1512) -- cycle;
\fill[blue!29.2, opacity=0.7] (0.3960, 1.8240, 3.1512) -- (0.4420, 1.8240, 3.1571) -- (0.4420, 1.8780, 3.1550) -- (0.3960, 1.8780, 3.1491) -- cycle;
\fill[blue!30.3, opacity=0.7] (0.3960, 1.8780, 3.1491) -- (0.4420, 1.8780, 3.1550) -- (0.4420, 1.9320, 3.1526) -- (0.3960, 1.9320, 3.1467) -- cycle;
\fill[blue!32.4, opacity=0.7] (0.3960, 1.9320, 3.1467) -- (0.4420, 1.9320, 3.1526) -- (0.4420, 1.9860, 3.1499) -- (0.3960, 1.9860, 3.1440) -- cycle;
\fill[blue!35.8, opacity=0.7] (0.3960, 1.9860, 3.1440) -- (0.4420, 1.9860, 3.1499) -- (0.4420, 2.0400, 3.1469) -- (0.3960, 2.0400, 3.1410) -- cycle;
\fill[blue!41.0, opacity=0.7] (0.3960, 2.0400, 3.1410) -- (0.4420, 2.0400, 3.1469) -- (0.4420, 2.0940, 3.1436) -- (0.3960, 2.0940, 3.1377) -- cycle;
\fill[blue!47.7, opacity=0.7] (0.3960, 2.0940, 3.1377) -- (0.4420, 2.0940, 3.1436) -- (0.4420, 2.1480, 3.1401) -- (0.3960, 2.1480, 3.1342) -- cycle;
\fill[blue!55.1, opacity=0.7] (0.3960, 2.1480, 3.1342) -- (0.4420, 2.1480, 3.1401) -- (0.4420, 2.2020, 3.1363) -- (0.3960, 2.2020, 3.1303) -- cycle;
\fill[blue!60.8, opacity=0.7] (0.3960, 2.2020, 3.1303) -- (0.4420, 2.2020, 3.1363) -- (0.4420, 2.2560, 3.1322) -- (0.3960, 2.2560, 3.1263) -- cycle;
\fill[blue!63.4, opacity=0.7] (0.3960, 2.2560, 3.1263) -- (0.4420, 2.2560, 3.1322) -- (0.4420, 2.3100, 3.1279) -- (0.3960, 2.3100, 3.1219) -- cycle;
\fill[blue!63.4, opacity=0.7] (0.3960, 2.3100, 3.1219) -- (0.4420, 2.3100, 3.1279) -- (0.4420, 2.3640, 3.1233) -- (0.3960, 2.3640, 3.1174) -- cycle;
\fill[blue!63.1, opacity=0.7] (0.3960, 2.3640, 3.1174) -- (0.4420, 2.3640, 3.1233) -- (0.4420, 2.4180, 3.1185) -- (0.3960, 2.4180, 3.1126) -- cycle;
\fill[blue!63.5, opacity=0.7] (0.3960, 2.4180, 3.1126) -- (0.4420, 2.4180, 3.1185) -- (0.4420, 2.4720, 3.1135) -- (0.3960, 2.4720, 3.1076) -- cycle;
\fill[blue!62.1, opacity=0.7] (0.3960, 2.4720, 3.1076) -- (0.4420, 2.4720, 3.1135) -- (0.4420, 2.5260, 3.1084) -- (0.3960, 2.5260, 3.1024) -- cycle;
\fill[blue!52.9, opacity=0.7] (0.3960, 2.5260, 3.1024) -- (0.4420, 2.5260, 3.1084) -- (0.4420, 2.5800, 3.1030) -- (0.3960, 2.5800, 3.0971) -- cycle;
\fill[blue!36.6, opacity=0.7] (0.3960, 2.5800, 3.0971) -- (0.4420, 2.5800, 3.1030) -- (0.4420, 2.6340, 3.0975) -- (0.3960, 2.6340, 3.0916) -- cycle;
\fill[blue!23.4, opacity=0.7] (0.3960, 2.6340, 3.0916) -- (0.4420, 2.6340, 3.0975) -- (0.4420, 2.6880, 3.0918) -- (0.3960, 2.6880, 3.0859) -- cycle;
\fill[blue!17.8, opacity=0.7] (0.3960, 2.6880, 3.0859) -- (0.4420, 2.6880, 3.0918) -- (0.4420, 2.7420, 3.0860) -- (0.3960, 2.7420, 3.0801) -- cycle;
\fill[blue!16.2, opacity=0.7] (0.3960, 2.7420, 3.0801) -- (0.4420, 2.7420, 3.0860) -- (0.4420, 2.7960, 3.0801) -- (0.3960, 2.7960, 3.0742) -- cycle;
\fill[blue!16.0, opacity=0.7] (0.3960, 2.7960, 3.0742) -- (0.4420, 2.7960, 3.0801) -- (0.4420, 2.8500, 3.0741) -- (0.3960, 2.8500, 3.0681) -- cycle;
\fill[blue!16.9, opacity=0.7] (0.3960, 2.8500, 3.0681) -- (0.4420, 2.8500, 3.0741) -- (0.4420, 2.9040, 3.0680) -- (0.3960, 2.9040, 3.0620) -- cycle;
\fill[blue!20.1, opacity=0.7] (0.3960, 2.9040, 3.0620) -- (0.4420, 2.9040, 3.0680) -- (0.4420, 2.9580, 3.0618) -- (0.3960, 2.9580, 3.0559) -- cycle;
\fill[blue!28.7, opacity=0.7] (0.3960, 2.9580, 3.0559) -- (0.4420, 2.9580, 3.0618) -- (0.4420, 3.0120, 3.0555) -- (0.3960, 3.0120, 3.0496) -- cycle;
\fill[blue!40.3, opacity=0.7] (0.3960, 3.0120, 3.0496) -- (0.4420, 3.0120, 3.0555) -- (0.4420, 3.0660, 3.0493) -- (0.3960, 3.0660, 3.0434) -- cycle;
\fill[blue!46.3, opacity=0.7] (0.3960, 3.0660, 3.0434) -- (0.4420, 3.0660, 3.0493) -- (0.4420, 3.1200, 3.0430) -- (0.3960, 3.1200, 3.0371) -- cycle;
\fill[blue!15.3, opacity=0.7] (0.4420, -0.1200, 3.0430) -- (0.4880, -0.1200, 3.0488) -- (0.4880, -0.0660, 3.0551) -- (0.4420, -0.0660, 3.0493) -- cycle;
\fill[blue!16.1, opacity=0.7] (0.4420, -0.0660, 3.0493) -- (0.4880, -0.0660, 3.0551) -- (0.4880, -0.0120, 3.0614) -- (0.4420, -0.0120, 3.0555) -- cycle;
\fill[blue!21.6, opacity=0.7] (0.4420, -0.0120, 3.0555) -- (0.4880, -0.0120, 3.0614) -- (0.4880, 0.0420, 3.0676) -- (0.4420, 0.0420, 3.0618) -- cycle;
\fill[blue!38.4, opacity=0.7] (0.4420, 0.0420, 3.0618) -- (0.4880, 0.0420, 3.0676) -- (0.4880, 0.0960, 3.0738) -- (0.4420, 0.0960, 3.0680) -- cycle;
\fill[blue!55.2, opacity=0.7] (0.4420, 0.0960, 3.0680) -- (0.4880, 0.0960, 3.0738) -- (0.4880, 0.1500, 3.0799) -- (0.4420, 0.1500, 3.0741) -- cycle;
\fill[blue!60.0, opacity=0.7] (0.4420, 0.1500, 3.0741) -- (0.4880, 0.1500, 3.0799) -- (0.4880, 0.2040, 3.0859) -- (0.4420, 0.2040, 3.0801) -- cycle;
\fill[blue!56.4, opacity=0.7] (0.4420, 0.2040, 3.0801) -- (0.4880, 0.2040, 3.0859) -- (0.4880, 0.2580, 3.0918) -- (0.4420, 0.2580, 3.0860) -- cycle;
\fill[blue!43.2, opacity=0.7] (0.4420, 0.2580, 3.0860) -- (0.4880, 0.2580, 3.0918) -- (0.4880, 0.3120, 3.0976) -- (0.4420, 0.3120, 3.0918) -- cycle;
\fill[blue!28.0, opacity=0.7] (0.4420, 0.3120, 3.0918) -- (0.4880, 0.3120, 3.0976) -- (0.4880, 0.3660, 3.1033) -- (0.4420, 0.3660, 3.0975) -- cycle;
\fill[blue!20.4, opacity=0.7] (0.4420, 0.3660, 3.0975) -- (0.4880, 0.3660, 3.1033) -- (0.4880, 0.4200, 3.1088) -- (0.4420, 0.4200, 3.1030) -- cycle;
\fill[blue!18.4, opacity=0.7] (0.4420, 0.4200, 3.1030) -- (0.4880, 0.4200, 3.1088) -- (0.4880, 0.4740, 3.1142) -- (0.4420, 0.4740, 3.1084) -- cycle;
\fill[blue!19.4, opacity=0.7] (0.4420, 0.4740, 3.1084) -- (0.4880, 0.4740, 3.1142) -- (0.4880, 0.5280, 3.1193) -- (0.4420, 0.5280, 3.1135) -- cycle;
\fill[blue!25.1, opacity=0.7] (0.4420, 0.5280, 3.1135) -- (0.4880, 0.5280, 3.1193) -- (0.4880, 0.5820, 3.1243) -- (0.4420, 0.5820, 3.1185) -- cycle;
\fill[blue!39.6, opacity=0.7] (0.4420, 0.5820, 3.1185) -- (0.4880, 0.5820, 3.1243) -- (0.4880, 0.6360, 3.1291) -- (0.4420, 0.6360, 3.1233) -- cycle;
\fill[blue!57.4, opacity=0.7] (0.4420, 0.6360, 3.1233) -- (0.4880, 0.6360, 3.1291) -- (0.4880, 0.6900, 3.1337) -- (0.4420, 0.6900, 3.1279) -- cycle;
\fill[blue!63.6, opacity=0.7] (0.4420, 0.6900, 3.1279) -- (0.4880, 0.6900, 3.1337) -- (0.4880, 0.7440, 3.1380) -- (0.4420, 0.7440, 3.1322) -- cycle;
\fill[blue!60.3, opacity=0.7] (0.4420, 0.7440, 3.1322) -- (0.4880, 0.7440, 3.1380) -- (0.4880, 0.7980, 3.1421) -- (0.4420, 0.7980, 3.1363) -- cycle;
\fill[blue!57.9, opacity=0.7] (0.4420, 0.7980, 3.1363) -- (0.4880, 0.7980, 3.1421) -- (0.4880, 0.8520, 3.1459) -- (0.4420, 0.8520, 3.1401) -- cycle;
\fill[blue!59.8, opacity=0.7] (0.4420, 0.8520, 3.1401) -- (0.4880, 0.8520, 3.1459) -- (0.4880, 0.9060, 3.1494) -- (0.4420, 0.9060, 3.1436) -- cycle;
\fill[blue!63.1, opacity=0.7] (0.4420, 0.9060, 3.1436) -- (0.4880, 0.9060, 3.1494) -- (0.4880, 0.9600, 3.1527) -- (0.4420, 0.9600, 3.1469) -- cycle;
\fill[blue!62.5, opacity=0.7] (0.4420, 0.9600, 3.1469) -- (0.4880, 0.9600, 3.1527) -- (0.4880, 1.0140, 3.1557) -- (0.4420, 1.0140, 3.1499) -- cycle;
\fill[blue!56.0, opacity=0.7] (0.4420, 1.0140, 3.1499) -- (0.4880, 1.0140, 3.1557) -- (0.4880, 1.0680, 3.1584) -- (0.4420, 1.0680, 3.1526) -- cycle;
\fill[blue!46.9, opacity=0.7] (0.4420, 1.0680, 3.1526) -- (0.4880, 1.0680, 3.1584) -- (0.4880, 1.1220, 3.1608) -- (0.4420, 1.1220, 3.1550) -- cycle;
\fill[blue!39.2, opacity=0.7] (0.4420, 1.1220, 3.1550) -- (0.4880, 1.1220, 3.1608) -- (0.4880, 1.1760, 3.1629) -- (0.4420, 1.1760, 3.1571) -- cycle;
\fill[blue!34.2, opacity=0.7] (0.4420, 1.1760, 3.1571) -- (0.4880, 1.1760, 3.1629) -- (0.4880, 1.2300, 3.1647) -- (0.4420, 1.2300, 3.1589) -- cycle;
\fill[blue!31.6, opacity=0.7] (0.4420, 1.2300, 3.1589) -- (0.4880, 1.2300, 3.1647) -- (0.4880, 1.2840, 3.1662) -- (0.4420, 1.2840, 3.1604) -- cycle;
\fill[blue!30.8, opacity=0.7] (0.4420, 1.2840, 3.1604) -- (0.4880, 1.2840, 3.1662) -- (0.4880, 1.3380, 3.1673) -- (0.4420, 1.3380, 3.1615) -- cycle;
\fill[blue!31.1, opacity=0.7] (0.4420, 1.3380, 3.1615) -- (0.4880, 1.3380, 3.1673) -- (0.4880, 1.3920, 3.1682) -- (0.4420, 1.3920, 3.1623) -- cycle;
\fill[blue!32.0, opacity=0.7] (0.4420, 1.3920, 3.1623) -- (0.4880, 1.3920, 3.1682) -- (0.4880, 1.4460, 3.1686) -- (0.4420, 1.4460, 3.1628) -- cycle;
\fill[blue!33.2, opacity=0.7] (0.4420, 1.4460, 3.1628) -- (0.4880, 1.4460, 3.1686) -- (0.4880, 1.5000, 3.1688) -- (0.4420, 1.5000, 3.1630) -- cycle;
\fill[blue!34.3, opacity=0.7] (0.4420, 1.5000, 3.1630) -- (0.4880, 1.5000, 3.1688) -- (0.4880, 1.5540, 3.1686) -- (0.4420, 1.5540, 3.1628) -- cycle;
\fill[blue!34.9, opacity=0.7] (0.4420, 1.5540, 3.1628) -- (0.4880, 1.5540, 3.1686) -- (0.4880, 1.6080, 3.1682) -- (0.4420, 1.6080, 3.1623) -- cycle;
\fill[blue!35.0, opacity=0.7] (0.4420, 1.6080, 3.1623) -- (0.4880, 1.6080, 3.1682) -- (0.4880, 1.6620, 3.1673) -- (0.4420, 1.6620, 3.1615) -- cycle;
\fill[blue!34.4, opacity=0.7] (0.4420, 1.6620, 3.1615) -- (0.4880, 1.6620, 3.1673) -- (0.4880, 1.7160, 3.1662) -- (0.4420, 1.7160, 3.1604) -- cycle;
\fill[blue!33.2, opacity=0.7] (0.4420, 1.7160, 3.1604) -- (0.4880, 1.7160, 3.1662) -- (0.4880, 1.7700, 3.1647) -- (0.4420, 1.7700, 3.1589) -- cycle;
\fill[blue!31.6, opacity=0.7] (0.4420, 1.7700, 3.1589) -- (0.4880, 1.7700, 3.1647) -- (0.4880, 1.8240, 3.1629) -- (0.4420, 1.8240, 3.1571) -- cycle;
\fill[blue!29.9, opacity=0.7] (0.4420, 1.8240, 3.1571) -- (0.4880, 1.8240, 3.1629) -- (0.4880, 1.8780, 3.1608) -- (0.4420, 1.8780, 3.1550) -- cycle;
\fill[blue!28.4, opacity=0.7] (0.4420, 1.8780, 3.1550) -- (0.4880, 1.8780, 3.1608) -- (0.4880, 1.9320, 3.1584) -- (0.4420, 1.9320, 3.1526) -- cycle;
\fill[blue!27.3, opacity=0.7] (0.4420, 1.9320, 3.1526) -- (0.4880, 1.9320, 3.1584) -- (0.4880, 1.9860, 3.1557) -- (0.4420, 1.9860, 3.1499) -- cycle;
\fill[blue!27.1, opacity=0.7] (0.4420, 1.9860, 3.1499) -- (0.4880, 1.9860, 3.1557) -- (0.4880, 2.0400, 3.1527) -- (0.4420, 2.0400, 3.1469) -- cycle;
\fill[blue!28.1, opacity=0.7] (0.4420, 2.0400, 3.1469) -- (0.4880, 2.0400, 3.1527) -- (0.4880, 2.0940, 3.1494) -- (0.4420, 2.0940, 3.1436) -- cycle;
\fill[blue!30.7, opacity=0.7] (0.4420, 2.0940, 3.1436) -- (0.4880, 2.0940, 3.1494) -- (0.4880, 2.1480, 3.1459) -- (0.4420, 2.1480, 3.1401) -- cycle;
\fill[blue!35.8, opacity=0.7] (0.4420, 2.1480, 3.1401) -- (0.4880, 2.1480, 3.1459) -- (0.4880, 2.2020, 3.1421) -- (0.4420, 2.2020, 3.1363) -- cycle;
\fill[blue!43.6, opacity=0.7] (0.4420, 2.2020, 3.1363) -- (0.4880, 2.2020, 3.1421) -- (0.4880, 2.2560, 3.1380) -- (0.4420, 2.2560, 3.1322) -- cycle;
\fill[blue!53.0, opacity=0.7] (0.4420, 2.2560, 3.1322) -- (0.4880, 2.2560, 3.1380) -- (0.4880, 2.3100, 3.1337) -- (0.4420, 2.3100, 3.1279) -- cycle;
\fill[blue!60.4, opacity=0.7] (0.4420, 2.3100, 3.1279) -- (0.4880, 2.3100, 3.1337) -- (0.4880, 2.3640, 3.1291) -- (0.4420, 2.3640, 3.1233) -- cycle;
\fill[blue!63.4, opacity=0.7] (0.4420, 2.3640, 3.1233) -- (0.4880, 2.3640, 3.1291) -- (0.4880, 2.4180, 3.1243) -- (0.4420, 2.4180, 3.1185) -- cycle;
\fill[blue!63.3, opacity=0.7] (0.4420, 2.4180, 3.1185) -- (0.4880, 2.4180, 3.1243) -- (0.4880, 2.4720, 3.1193) -- (0.4420, 2.4720, 3.1135) -- cycle;
\fill[blue!63.3, opacity=0.7] (0.4420, 2.4720, 3.1135) -- (0.4880, 2.4720, 3.1193) -- (0.4880, 2.5260, 3.1142) -- (0.4420, 2.5260, 3.1084) -- cycle;
\fill[blue!63.3, opacity=0.7] (0.4420, 2.5260, 3.1084) -- (0.4880, 2.5260, 3.1142) -- (0.4880, 2.5800, 3.1088) -- (0.4420, 2.5800, 3.1030) -- cycle;
\fill[blue!57.3, opacity=0.7] (0.4420, 2.5800, 3.1030) -- (0.4880, 2.5800, 3.1088) -- (0.4880, 2.6340, 3.1033) -- (0.4420, 2.6340, 3.0975) -- cycle;
\fill[blue!41.4, opacity=0.7] (0.4420, 2.6340, 3.0975) -- (0.4880, 2.6340, 3.1033) -- (0.4880, 2.6880, 3.0976) -- (0.4420, 2.6880, 3.0918) -- cycle;
\fill[blue!25.6, opacity=0.7] (0.4420, 2.6880, 3.0918) -- (0.4880, 2.6880, 3.0976) -- (0.4880, 2.7420, 3.0918) -- (0.4420, 2.7420, 3.0860) -- cycle;
\fill[blue!18.3, opacity=0.7] (0.4420, 2.7420, 3.0860) -- (0.4880, 2.7420, 3.0918) -- (0.4880, 2.7960, 3.0859) -- (0.4420, 2.7960, 3.0801) -- cycle;
\fill[blue!16.3, opacity=0.7] (0.4420, 2.7960, 3.0801) -- (0.4880, 2.7960, 3.0859) -- (0.4880, 2.8500, 3.0799) -- (0.4420, 2.8500, 3.0741) -- cycle;
\fill[blue!16.0, opacity=0.7] (0.4420, 2.8500, 3.0741) -- (0.4880, 2.8500, 3.0799) -- (0.4880, 2.9040, 3.0738) -- (0.4420, 2.9040, 3.0680) -- cycle;
\fill[blue!16.7, opacity=0.7] (0.4420, 2.9040, 3.0680) -- (0.4880, 2.9040, 3.0738) -- (0.4880, 2.9580, 3.0676) -- (0.4420, 2.9580, 3.0618) -- cycle;
\fill[blue!20.1, opacity=0.7] (0.4420, 2.9580, 3.0618) -- (0.4880, 2.9580, 3.0676) -- (0.4880, 3.0120, 3.0614) -- (0.4420, 3.0120, 3.0555) -- cycle;
\fill[blue!29.0, opacity=0.7] (0.4420, 3.0120, 3.0555) -- (0.4880, 3.0120, 3.0614) -- (0.4880, 3.0660, 3.0551) -- (0.4420, 3.0660, 3.0493) -- cycle;
\fill[blue!40.8, opacity=0.7] (0.4420, 3.0660, 3.0493) -- (0.4880, 3.0660, 3.0551) -- (0.4880, 3.1200, 3.0488) -- (0.4420, 3.1200, 3.0430) -- cycle;
\fill[blue!15.7, opacity=0.7] (0.4880, -0.1200, 3.0488) -- (0.5340, -0.1200, 3.0545) -- (0.5340, -0.0660, 3.0608) -- (0.4880, -0.0660, 3.0551) -- cycle;
\fill[blue!18.8, opacity=0.7] (0.4880, -0.0660, 3.0551) -- (0.5340, -0.0660, 3.0608) -- (0.5340, -0.0120, 3.0670) -- (0.4880, -0.0120, 3.0614) -- cycle;
\fill[blue!32.4, opacity=0.7] (0.4880, -0.0120, 3.0614) -- (0.5340, -0.0120, 3.0670) -- (0.5340, 0.0420, 3.0733) -- (0.4880, 0.0420, 3.0676) -- cycle;
\fill[blue!51.9, opacity=0.7] (0.4880, 0.0420, 3.0676) -- (0.5340, 0.0420, 3.0733) -- (0.5340, 0.0960, 3.0794) -- (0.4880, 0.0960, 3.0738) -- cycle;
\fill[blue!59.9, opacity=0.7] (0.4880, 0.0960, 3.0738) -- (0.5340, 0.0960, 3.0794) -- (0.5340, 0.1500, 3.0855) -- (0.4880, 0.1500, 3.0799) -- cycle;
\fill[blue!58.2, opacity=0.7] (0.4880, 0.1500, 3.0799) -- (0.5340, 0.1500, 3.0855) -- (0.5340, 0.2040, 3.0916) -- (0.4880, 0.2040, 3.0859) -- cycle;
\fill[blue!46.8, opacity=0.7] (0.4880, 0.2040, 3.0859) -- (0.5340, 0.2040, 3.0916) -- (0.5340, 0.2580, 3.0975) -- (0.4880, 0.2580, 3.0918) -- cycle;
\fill[blue!30.4, opacity=0.7] (0.4880, 0.2580, 3.0918) -- (0.5340, 0.2580, 3.0975) -- (0.5340, 0.3120, 3.1033) -- (0.4880, 0.3120, 3.0976) -- cycle;
\fill[blue!21.1, opacity=0.7] (0.4880, 0.3120, 3.0976) -- (0.5340, 0.3120, 3.1033) -- (0.5340, 0.3660, 3.1090) -- (0.4880, 0.3660, 3.1033) -- cycle;
\fill[blue!18.5, opacity=0.7] (0.4880, 0.3660, 3.1033) -- (0.5340, 0.3660, 3.1090) -- (0.5340, 0.4200, 3.1145) -- (0.4880, 0.4200, 3.1088) -- cycle;
\fill[blue!19.5, opacity=0.7] (0.4880, 0.4200, 3.1088) -- (0.5340, 0.4200, 3.1145) -- (0.5340, 0.4740, 3.1198) -- (0.4880, 0.4740, 3.1142) -- cycle;
\fill[blue!25.6, opacity=0.7] (0.4880, 0.4740, 3.1142) -- (0.5340, 0.4740, 3.1198) -- (0.5340, 0.5280, 3.1250) -- (0.4880, 0.5280, 3.1193) -- cycle;
\fill[blue!41.4, opacity=0.7] (0.4880, 0.5280, 3.1193) -- (0.5340, 0.5280, 3.1250) -- (0.5340, 0.5820, 3.1300) -- (0.4880, 0.5820, 3.1243) -- cycle;
\fill[blue!59.3, opacity=0.7] (0.4880, 0.5820, 3.1243) -- (0.5340, 0.5820, 3.1300) -- (0.5340, 0.6360, 3.1348) -- (0.4880, 0.6360, 3.1291) -- cycle;
\fill[blue!63.2, opacity=0.7] (0.4880, 0.6360, 3.1291) -- (0.5340, 0.6360, 3.1348) -- (0.5340, 0.6900, 3.1393) -- (0.4880, 0.6900, 3.1337) -- cycle;
\fill[blue!58.9, opacity=0.7] (0.4880, 0.6900, 3.1337) -- (0.5340, 0.6900, 3.1393) -- (0.5340, 0.7440, 3.1437) -- (0.4880, 0.7440, 3.1380) -- cycle;
\fill[blue!57.8, opacity=0.7] (0.4880, 0.7440, 3.1380) -- (0.5340, 0.7440, 3.1437) -- (0.5340, 0.7980, 3.1477) -- (0.4880, 0.7980, 3.1421) -- cycle;
\fill[blue!61.2, opacity=0.7] (0.4880, 0.7980, 3.1421) -- (0.5340, 0.7980, 3.1477) -- (0.5340, 0.8520, 3.1516) -- (0.4880, 0.8520, 3.1459) -- cycle;
\fill[blue!63.5, opacity=0.7] (0.4880, 0.8520, 3.1459) -- (0.5340, 0.8520, 3.1516) -- (0.5340, 0.9060, 3.1551) -- (0.4880, 0.9060, 3.1494) -- cycle;
\fill[blue!58.7, opacity=0.7] (0.4880, 0.9060, 3.1494) -- (0.5340, 0.9060, 3.1551) -- (0.5340, 0.9600, 3.1584) -- (0.4880, 0.9600, 3.1527) -- cycle;
\fill[blue!48.6, opacity=0.7] (0.4880, 0.9600, 3.1527) -- (0.5340, 0.9600, 3.1584) -- (0.5340, 1.0140, 3.1614) -- (0.4880, 1.0140, 3.1557) -- cycle;
\fill[blue!39.3, opacity=0.7] (0.4880, 1.0140, 3.1557) -- (0.5340, 1.0140, 3.1614) -- (0.5340, 1.0680, 3.1641) -- (0.4880, 1.0680, 3.1584) -- cycle;
\fill[blue!33.8, opacity=0.7] (0.4880, 1.0680, 3.1584) -- (0.5340, 1.0680, 3.1641) -- (0.5340, 1.1220, 3.1665) -- (0.4880, 1.1220, 3.1608) -- cycle;
\fill[blue!31.9, opacity=0.7] (0.4880, 1.1220, 3.1608) -- (0.5340, 1.1220, 3.1665) -- (0.5340, 1.1760, 3.1686) -- (0.4880, 1.1760, 3.1629) -- cycle;
\fill[blue!32.6, opacity=0.7] (0.4880, 1.1760, 3.1629) -- (0.5340, 1.1760, 3.1686) -- (0.5340, 1.2300, 3.1704) -- (0.4880, 1.2300, 3.1647) -- cycle;
\fill[blue!35.3, opacity=0.7] (0.4880, 1.2300, 3.1647) -- (0.5340, 1.2300, 3.1704) -- (0.5340, 1.2840, 3.1719) -- (0.4880, 1.2840, 3.1662) -- cycle;
\fill[blue!39.5, opacity=0.7] (0.4880, 1.2840, 3.1662) -- (0.5340, 1.2840, 3.1719) -- (0.5340, 1.3380, 3.1730) -- (0.4880, 1.3380, 3.1673) -- cycle;
\fill[blue!44.4, opacity=0.7] (0.4880, 1.3380, 3.1673) -- (0.5340, 1.3380, 3.1730) -- (0.5340, 1.3920, 3.1738) -- (0.4880, 1.3920, 3.1682) -- cycle;
\fill[blue!49.0, opacity=0.7] (0.4880, 1.3920, 3.1682) -- (0.5340, 1.3920, 3.1738) -- (0.5340, 1.4460, 3.1743) -- (0.4880, 1.4460, 3.1686) -- cycle;
\fill[blue!52.8, opacity=0.7] (0.4880, 1.4460, 3.1686) -- (0.5340, 1.4460, 3.1743) -- (0.5340, 1.5000, 3.1745) -- (0.4880, 1.5000, 3.1688) -- cycle;
\fill[blue!55.4, opacity=0.7] (0.4880, 1.5000, 3.1688) -- (0.5340, 1.5000, 3.1745) -- (0.5340, 1.5540, 3.1743) -- (0.4880, 1.5540, 3.1686) -- cycle;
\fill[blue!56.7, opacity=0.7] (0.4880, 1.5540, 3.1686) -- (0.5340, 1.5540, 3.1743) -- (0.5340, 1.6080, 3.1738) -- (0.4880, 1.6080, 3.1682) -- cycle;
\fill[blue!57.1, opacity=0.7] (0.4880, 1.6080, 3.1682) -- (0.5340, 1.6080, 3.1738) -- (0.5340, 1.6620, 3.1730) -- (0.4880, 1.6620, 3.1673) -- cycle;
\fill[blue!56.4, opacity=0.7] (0.4880, 1.6620, 3.1673) -- (0.5340, 1.6620, 3.1730) -- (0.5340, 1.7160, 3.1719) -- (0.4880, 1.7160, 3.1662) -- cycle;
\fill[blue!54.6, opacity=0.7] (0.4880, 1.7160, 3.1662) -- (0.5340, 1.7160, 3.1719) -- (0.5340, 1.7700, 3.1704) -- (0.4880, 1.7700, 3.1647) -- cycle;
\fill[blue!51.5, opacity=0.7] (0.4880, 1.7700, 3.1647) -- (0.5340, 1.7700, 3.1704) -- (0.5340, 1.8240, 3.1686) -- (0.4880, 1.8240, 3.1629) -- cycle;
\fill[blue!47.2, opacity=0.7] (0.4880, 1.8240, 3.1629) -- (0.5340, 1.8240, 3.1686) -- (0.5340, 1.8780, 3.1665) -- (0.4880, 1.8780, 3.1608) -- cycle;
\fill[blue!42.0, opacity=0.7] (0.4880, 1.8780, 3.1608) -- (0.5340, 1.8780, 3.1665) -- (0.5340, 1.9320, 3.1641) -- (0.4880, 1.9320, 3.1584) -- cycle;
\fill[blue!36.6, opacity=0.7] (0.4880, 1.9320, 3.1584) -- (0.5340, 1.9320, 3.1641) -- (0.5340, 1.9860, 3.1614) -- (0.4880, 1.9860, 3.1557) -- cycle;
\fill[blue!31.8, opacity=0.7] (0.4880, 1.9860, 3.1557) -- (0.5340, 1.9860, 3.1614) -- (0.5340, 2.0400, 3.1584) -- (0.4880, 2.0400, 3.1527) -- cycle;
\fill[blue!28.4, opacity=0.7] (0.4880, 2.0400, 3.1527) -- (0.5340, 2.0400, 3.1584) -- (0.5340, 2.0940, 3.1551) -- (0.4880, 2.0940, 3.1494) -- cycle;
\fill[blue!26.6, opacity=0.7] (0.4880, 2.0940, 3.1494) -- (0.5340, 2.0940, 3.1551) -- (0.5340, 2.1480, 3.1516) -- (0.4880, 2.1480, 3.1459) -- cycle;
\fill[blue!26.6, opacity=0.7] (0.4880, 2.1480, 3.1459) -- (0.5340, 2.1480, 3.1516) -- (0.5340, 2.2020, 3.1477) -- (0.4880, 2.2020, 3.1421) -- cycle;
\fill[blue!28.9, opacity=0.7] (0.4880, 2.2020, 3.1421) -- (0.5340, 2.2020, 3.1477) -- (0.5340, 2.2560, 3.1437) -- (0.4880, 2.2560, 3.1380) -- cycle;
\fill[blue!34.5, opacity=0.7] (0.4880, 2.2560, 3.1380) -- (0.5340, 2.2560, 3.1437) -- (0.5340, 2.3100, 3.1393) -- (0.4880, 2.3100, 3.1337) -- cycle;
\fill[blue!43.8, opacity=0.7] (0.4880, 2.3100, 3.1337) -- (0.5340, 2.3100, 3.1393) -- (0.5340, 2.3640, 3.1348) -- (0.4880, 2.3640, 3.1291) -- cycle;
\fill[blue!54.6, opacity=0.7] (0.4880, 2.3640, 3.1291) -- (0.5340, 2.3640, 3.1348) -- (0.5340, 2.4180, 3.1300) -- (0.4880, 2.4180, 3.1243) -- cycle;
\fill[blue!61.8, opacity=0.7] (0.4880, 2.4180, 3.1243) -- (0.5340, 2.4180, 3.1300) -- (0.5340, 2.4720, 3.1250) -- (0.4880, 2.4720, 3.1193) -- cycle;
\fill[blue!63.6, opacity=0.7] (0.4880, 2.4720, 3.1193) -- (0.5340, 2.4720, 3.1250) -- (0.5340, 2.5260, 3.1198) -- (0.4880, 2.5260, 3.1142) -- cycle;
\fill[blue!63.3, opacity=0.7] (0.4880, 2.5260, 3.1142) -- (0.5340, 2.5260, 3.1198) -- (0.5340, 2.5800, 3.1145) -- (0.4880, 2.5800, 3.1088) -- cycle;
\fill[blue!63.5, opacity=0.7] (0.4880, 2.5800, 3.1088) -- (0.5340, 2.5800, 3.1145) -- (0.5340, 2.6340, 3.1090) -- (0.4880, 2.6340, 3.1033) -- cycle;
\fill[blue!59.5, opacity=0.7] (0.4880, 2.6340, 3.1033) -- (0.5340, 2.6340, 3.1090) -- (0.5340, 2.6880, 3.1033) -- (0.4880, 2.6880, 3.0976) -- cycle;
\fill[blue!44.5, opacity=0.7] (0.4880, 2.6880, 3.0976) -- (0.5340, 2.6880, 3.1033) -- (0.5340, 2.7420, 3.0975) -- (0.4880, 2.7420, 3.0918) -- cycle;
\fill[blue!27.0, opacity=0.7] (0.4880, 2.7420, 3.0918) -- (0.5340, 2.7420, 3.0975) -- (0.5340, 2.7960, 3.0916) -- (0.4880, 2.7960, 3.0859) -- cycle;
\fill[blue!18.5, opacity=0.7] (0.4880, 2.7960, 3.0859) -- (0.5340, 2.7960, 3.0916) -- (0.5340, 2.8500, 3.0855) -- (0.4880, 2.8500, 3.0799) -- cycle;
\fill[blue!16.3, opacity=0.7] (0.4880, 2.8500, 3.0799) -- (0.5340, 2.8500, 3.0855) -- (0.5340, 2.9040, 3.0794) -- (0.4880, 2.9040, 3.0738) -- cycle;
\fill[blue!15.9, opacity=0.7] (0.4880, 2.9040, 3.0738) -- (0.5340, 2.9040, 3.0794) -- (0.5340, 2.9580, 3.0733) -- (0.4880, 2.9580, 3.0676) -- cycle;
\fill[blue!16.8, opacity=0.7] (0.4880, 2.9580, 3.0676) -- (0.5340, 2.9580, 3.0733) -- (0.5340, 3.0120, 3.0670) -- (0.4880, 3.0120, 3.0614) -- cycle;
\fill[blue!20.4, opacity=0.7] (0.4880, 3.0120, 3.0614) -- (0.5340, 3.0120, 3.0670) -- (0.5340, 3.0660, 3.0608) -- (0.4880, 3.0660, 3.0551) -- cycle;
\fill[blue!30.1, opacity=0.7] (0.4880, 3.0660, 3.0551) -- (0.5340, 3.0660, 3.0608) -- (0.5340, 3.1200, 3.0545) -- (0.4880, 3.1200, 3.0488) -- cycle;
\fill[blue!16.9, opacity=0.7] (0.5340, -0.1200, 3.0545) -- (0.5800, -0.1200, 3.0600) -- (0.5800, -0.0660, 3.0663) -- (0.5340, -0.0660, 3.0608) -- cycle;
\fill[blue!26.0, opacity=0.7] (0.5340, -0.0660, 3.0608) -- (0.5800, -0.0660, 3.0663) -- (0.5800, -0.0120, 3.0725) -- (0.5340, -0.0120, 3.0670) -- cycle;
\fill[blue!46.3, opacity=0.7] (0.5340, -0.0120, 3.0670) -- (0.5800, -0.0120, 3.0725) -- (0.5800, 0.0420, 3.0788) -- (0.5340, 0.0420, 3.0733) -- cycle;
\fill[blue!58.9, opacity=0.7] (0.5340, 0.0420, 3.0733) -- (0.5800, 0.0420, 3.0788) -- (0.5800, 0.0960, 3.0849) -- (0.5340, 0.0960, 3.0794) -- cycle;
\fill[blue!59.8, opacity=0.7] (0.5340, 0.0960, 3.0794) -- (0.5800, 0.0960, 3.0849) -- (0.5800, 0.1500, 3.0911) -- (0.5340, 0.1500, 3.0855) -- cycle;
\fill[blue!51.2, opacity=0.7] (0.5340, 0.1500, 3.0855) -- (0.5800, 0.1500, 3.0911) -- (0.5800, 0.2040, 3.0971) -- (0.5340, 0.2040, 3.0916) -- cycle;
\fill[blue!34.2, opacity=0.7] (0.5340, 0.2040, 3.0916) -- (0.5800, 0.2040, 3.0971) -- (0.5800, 0.2580, 3.1030) -- (0.5340, 0.2580, 3.0975) -- cycle;
\fill[blue!22.5, opacity=0.7] (0.5340, 0.2580, 3.0975) -- (0.5800, 0.2580, 3.1030) -- (0.5800, 0.3120, 3.1088) -- (0.5340, 0.3120, 3.1033) -- cycle;
\fill[blue!18.8, opacity=0.7] (0.5340, 0.3120, 3.1033) -- (0.5800, 0.3120, 3.1088) -- (0.5800, 0.3660, 3.1145) -- (0.5340, 0.3660, 3.1090) -- cycle;
\fill[blue!19.3, opacity=0.7] (0.5340, 0.3660, 3.1090) -- (0.5800, 0.3660, 3.1145) -- (0.5800, 0.4200, 3.1200) -- (0.5340, 0.4200, 3.1145) -- cycle;
\fill[blue!25.0, opacity=0.7] (0.5340, 0.4200, 3.1145) -- (0.5800, 0.4200, 3.1200) -- (0.5800, 0.4740, 3.1254) -- (0.5340, 0.4740, 3.1198) -- cycle;
\fill[blue!41.0, opacity=0.7] (0.5340, 0.4740, 3.1198) -- (0.5800, 0.4740, 3.1254) -- (0.5800, 0.5280, 3.1305) -- (0.5340, 0.5280, 3.1250) -- cycle;
\fill[blue!59.8, opacity=0.7] (0.5340, 0.5280, 3.1250) -- (0.5800, 0.5280, 3.1305) -- (0.5800, 0.5820, 3.1355) -- (0.5340, 0.5820, 3.1300) -- cycle;
\fill[blue!62.8, opacity=0.7] (0.5340, 0.5820, 3.1300) -- (0.5800, 0.5820, 3.1355) -- (0.5800, 0.6360, 3.1403) -- (0.5340, 0.6360, 3.1348) -- cycle;
\fill[blue!58.0, opacity=0.7] (0.5340, 0.6360, 3.1348) -- (0.5800, 0.6360, 3.1403) -- (0.5800, 0.6900, 3.1449) -- (0.5340, 0.6900, 3.1393) -- cycle;
\fill[blue!57.8, opacity=0.7] (0.5340, 0.6900, 3.1393) -- (0.5800, 0.6900, 3.1449) -- (0.5800, 0.7440, 3.1492) -- (0.5340, 0.7440, 3.1437) -- cycle;
\fill[blue!62.1, opacity=0.7] (0.5340, 0.7440, 3.1437) -- (0.5800, 0.7440, 3.1492) -- (0.5800, 0.7980, 3.1533) -- (0.5340, 0.7980, 3.1477) -- cycle;
\fill[blue!62.8, opacity=0.7] (0.5340, 0.7980, 3.1477) -- (0.5800, 0.7980, 3.1533) -- (0.5800, 0.8520, 3.1571) -- (0.5340, 0.8520, 3.1516) -- cycle;
\fill[blue!54.4, opacity=0.7] (0.5340, 0.8520, 3.1516) -- (0.5800, 0.8520, 3.1571) -- (0.5800, 0.9060, 3.1606) -- (0.5340, 0.9060, 3.1551) -- cycle;
\fill[blue!43.0, opacity=0.7] (0.5340, 0.9060, 3.1551) -- (0.5800, 0.9060, 3.1606) -- (0.5800, 0.9600, 3.1639) -- (0.5340, 0.9600, 3.1584) -- cycle;
\fill[blue!35.3, opacity=0.7] (0.5340, 0.9600, 3.1584) -- (0.5800, 0.9600, 3.1639) -- (0.5800, 1.0140, 3.1669) -- (0.5340, 1.0140, 3.1614) -- cycle;
\fill[blue!32.6, opacity=0.7] (0.5340, 1.0140, 3.1614) -- (0.5800, 1.0140, 3.1669) -- (0.5800, 1.0680, 3.1696) -- (0.5340, 1.0680, 3.1641) -- cycle;
\fill[blue!34.0, opacity=0.7] (0.5340, 1.0680, 3.1641) -- (0.5800, 1.0680, 3.1696) -- (0.5800, 1.1220, 3.1720) -- (0.5340, 1.1220, 3.1665) -- cycle;
\fill[blue!38.9, opacity=0.7] (0.5340, 1.1220, 3.1665) -- (0.5800, 1.1220, 3.1720) -- (0.5800, 1.1760, 3.1741) -- (0.5340, 1.1760, 3.1686) -- cycle;
\fill[blue!46.4, opacity=0.7] (0.5340, 1.1760, 3.1686) -- (0.5800, 1.1760, 3.1741) -- (0.5800, 1.2300, 3.1759) -- (0.5340, 1.2300, 3.1704) -- cycle;
\fill[blue!54.4, opacity=0.7] (0.5340, 1.2300, 3.1704) -- (0.5800, 1.2300, 3.1759) -- (0.5800, 1.2840, 3.1774) -- (0.5340, 1.2840, 3.1719) -- cycle;
\fill[blue!60.5, opacity=0.7] (0.5340, 1.2840, 3.1719) -- (0.5800, 1.2840, 3.1774) -- (0.5800, 1.3380, 3.1785) -- (0.5340, 1.3380, 3.1730) -- cycle;
\fill[blue!63.3, opacity=0.7] (0.5340, 1.3380, 3.1730) -- (0.5800, 1.3380, 3.1785) -- (0.5800, 1.3920, 3.1793) -- (0.5340, 1.3920, 3.1738) -- cycle;
\fill[blue!63.3, opacity=0.7] (0.5340, 1.3920, 3.1738) -- (0.5800, 1.3920, 3.1793) -- (0.5800, 1.4460, 3.1798) -- (0.5340, 1.4460, 3.1743) -- cycle;
\fill[blue!61.9, opacity=0.7] (0.5340, 1.4460, 3.1743) -- (0.5800, 1.4460, 3.1798) -- (0.5800, 1.5000, 3.1800) -- (0.5340, 1.5000, 3.1745) -- cycle;
\fill[blue!60.3, opacity=0.7] (0.5340, 1.5000, 3.1745) -- (0.5800, 1.5000, 3.1800) -- (0.5800, 1.5540, 3.1798) -- (0.5340, 1.5540, 3.1743) -- cycle;
\fill[blue!59.2, opacity=0.7] (0.5340, 1.5540, 3.1743) -- (0.5800, 1.5540, 3.1798) -- (0.5800, 1.6080, 3.1793) -- (0.5340, 1.6080, 3.1738) -- cycle;
\fill[blue!58.9, opacity=0.7] (0.5340, 1.6080, 3.1738) -- (0.5800, 1.6080, 3.1793) -- (0.5800, 1.6620, 3.1785) -- (0.5340, 1.6620, 3.1730) -- cycle;
\fill[blue!59.5, opacity=0.7] (0.5340, 1.6620, 3.1730) -- (0.5800, 1.6620, 3.1785) -- (0.5800, 1.7160, 3.1774) -- (0.5340, 1.7160, 3.1719) -- cycle;
\fill[blue!60.8, opacity=0.7] (0.5340, 1.7160, 3.1719) -- (0.5800, 1.7160, 3.1774) -- (0.5800, 1.7700, 3.1759) -- (0.5340, 1.7700, 3.1704) -- cycle;
\fill[blue!62.4, opacity=0.7] (0.5340, 1.7700, 3.1704) -- (0.5800, 1.7700, 3.1759) -- (0.5800, 1.8240, 3.1741) -- (0.5340, 1.8240, 3.1686) -- cycle;
\fill[blue!63.5, opacity=0.7] (0.5340, 1.8240, 3.1686) -- (0.5800, 1.8240, 3.1741) -- (0.5800, 1.8780, 3.1720) -- (0.5340, 1.8780, 3.1665) -- cycle;
\fill[blue!62.9, opacity=0.7] (0.5340, 1.8780, 3.1665) -- (0.5800, 1.8780, 3.1720) -- (0.5800, 1.9320, 3.1696) -- (0.5340, 1.9320, 3.1641) -- cycle;
\fill[blue!59.2, opacity=0.7] (0.5340, 1.9320, 3.1641) -- (0.5800, 1.9320, 3.1696) -- (0.5800, 1.9860, 3.1669) -- (0.5340, 1.9860, 3.1614) -- cycle;
\fill[blue!52.2, opacity=0.7] (0.5340, 1.9860, 3.1614) -- (0.5800, 1.9860, 3.1669) -- (0.5800, 2.0400, 3.1639) -- (0.5340, 2.0400, 3.1584) -- cycle;
\fill[blue!43.2, opacity=0.7] (0.5340, 2.0400, 3.1584) -- (0.5800, 2.0400, 3.1639) -- (0.5800, 2.0940, 3.1606) -- (0.5340, 2.0940, 3.1551) -- cycle;
\fill[blue!34.9, opacity=0.7] (0.5340, 2.0940, 3.1551) -- (0.5800, 2.0940, 3.1606) -- (0.5800, 2.1480, 3.1571) -- (0.5340, 2.1480, 3.1516) -- cycle;
\fill[blue!29.1, opacity=0.7] (0.5340, 2.1480, 3.1516) -- (0.5800, 2.1480, 3.1571) -- (0.5800, 2.2020, 3.1533) -- (0.5340, 2.2020, 3.1477) -- cycle;
\fill[blue!26.2, opacity=0.7] (0.5340, 2.2020, 3.1477) -- (0.5800, 2.2020, 3.1533) -- (0.5800, 2.2560, 3.1492) -- (0.5340, 2.2560, 3.1437) -- cycle;
\fill[blue!26.1, opacity=0.7] (0.5340, 2.2560, 3.1437) -- (0.5800, 2.2560, 3.1492) -- (0.5800, 2.3100, 3.1449) -- (0.5340, 2.3100, 3.1393) -- cycle;
\fill[blue!29.1, opacity=0.7] (0.5340, 2.3100, 3.1393) -- (0.5800, 2.3100, 3.1449) -- (0.5800, 2.3640, 3.1403) -- (0.5340, 2.3640, 3.1348) -- cycle;
\fill[blue!36.5, opacity=0.7] (0.5340, 2.3640, 3.1348) -- (0.5800, 2.3640, 3.1403) -- (0.5800, 2.4180, 3.1355) -- (0.5340, 2.4180, 3.1300) -- cycle;
\fill[blue!48.1, opacity=0.7] (0.5340, 2.4180, 3.1300) -- (0.5800, 2.4180, 3.1355) -- (0.5800, 2.4720, 3.1305) -- (0.5340, 2.4720, 3.1250) -- cycle;
\fill[blue!59.0, opacity=0.7] (0.5340, 2.4720, 3.1250) -- (0.5800, 2.4720, 3.1305) -- (0.5800, 2.5260, 3.1254) -- (0.5340, 2.5260, 3.1198) -- cycle;
\fill[blue!63.4, opacity=0.7] (0.5340, 2.5260, 3.1198) -- (0.5800, 2.5260, 3.1254) -- (0.5800, 2.5800, 3.1200) -- (0.5340, 2.5800, 3.1145) -- cycle;
\fill[blue!63.4, opacity=0.7] (0.5340, 2.5800, 3.1145) -- (0.5800, 2.5800, 3.1200) -- (0.5800, 2.6340, 3.1145) -- (0.5340, 2.6340, 3.1090) -- cycle;
\fill[blue!63.6, opacity=0.7] (0.5340, 2.6340, 3.1090) -- (0.5800, 2.6340, 3.1145) -- (0.5800, 2.6880, 3.1088) -- (0.5340, 2.6880, 3.1033) -- cycle;
\fill[blue!60.5, opacity=0.7] (0.5340, 2.6880, 3.1033) -- (0.5800, 2.6880, 3.1088) -- (0.5800, 2.7420, 3.1030) -- (0.5340, 2.7420, 3.0975) -- cycle;
\fill[blue!45.7, opacity=0.7] (0.5340, 2.7420, 3.0975) -- (0.5800, 2.7420, 3.1030) -- (0.5800, 2.7960, 3.0971) -- (0.5340, 2.7960, 3.0916) -- cycle;
\fill[blue!27.2, opacity=0.7] (0.5340, 2.7960, 3.0916) -- (0.5800, 2.7960, 3.0971) -- (0.5800, 2.8500, 3.0911) -- (0.5340, 2.8500, 3.0855) -- cycle;
\fill[blue!18.4, opacity=0.7] (0.5340, 2.8500, 3.0855) -- (0.5800, 2.8500, 3.0911) -- (0.5800, 2.9040, 3.0849) -- (0.5340, 2.9040, 3.0794) -- cycle;
\fill[blue!16.2, opacity=0.7] (0.5340, 2.9040, 3.0794) -- (0.5800, 2.9040, 3.0849) -- (0.5800, 2.9580, 3.0788) -- (0.5340, 2.9580, 3.0733) -- cycle;
\fill[blue!15.9, opacity=0.7] (0.5340, 2.9580, 3.0733) -- (0.5800, 2.9580, 3.0788) -- (0.5800, 3.0120, 3.0725) -- (0.5340, 3.0120, 3.0670) -- cycle;
\fill[blue!16.9, opacity=0.7] (0.5340, 3.0120, 3.0670) -- (0.5800, 3.0120, 3.0725) -- (0.5800, 3.0660, 3.0663) -- (0.5340, 3.0660, 3.0608) -- cycle;
\fill[blue!21.4, opacity=0.7] (0.5340, 3.0660, 3.0608) -- (0.5800, 3.0660, 3.0663) -- (0.5800, 3.1200, 3.0600) -- (0.5340, 3.1200, 3.0545) -- cycle;
\fill[blue!20.7, opacity=0.7] (0.5800, -0.1200, 3.0600) -- (0.6260, -0.1200, 3.0654) -- (0.6260, -0.0660, 3.0716) -- (0.5800, -0.0660, 3.0663) -- cycle;
\fill[blue!38.1, opacity=0.7] (0.5800, -0.0660, 3.0663) -- (0.6260, -0.0660, 3.0716) -- (0.6260, -0.0120, 3.0779) -- (0.5800, -0.0120, 3.0725) -- cycle;
\fill[blue!56.2, opacity=0.7] (0.5800, -0.0120, 3.0725) -- (0.6260, -0.0120, 3.0779) -- (0.6260, 0.0420, 3.0841) -- (0.5800, 0.0420, 3.0788) -- cycle;
\fill[blue!60.6, opacity=0.7] (0.5800, 0.0420, 3.0788) -- (0.6260, 0.0420, 3.0841) -- (0.6260, 0.0960, 3.0903) -- (0.5800, 0.0960, 3.0849) -- cycle;
\fill[blue!55.6, opacity=0.7] (0.5800, 0.0960, 3.0849) -- (0.6260, 0.0960, 3.0903) -- (0.6260, 0.1500, 3.0964) -- (0.5800, 0.1500, 3.0911) -- cycle;
\fill[blue!39.8, opacity=0.7] (0.5800, 0.1500, 3.0911) -- (0.6260, 0.1500, 3.0964) -- (0.6260, 0.2040, 3.1024) -- (0.5800, 0.2040, 3.0971) -- cycle;
\fill[blue!25.0, opacity=0.7] (0.5800, 0.2040, 3.0971) -- (0.6260, 0.2040, 3.1024) -- (0.6260, 0.2580, 3.1084) -- (0.5800, 0.2580, 3.1030) -- cycle;
\fill[blue!19.4, opacity=0.7] (0.5800, 0.2580, 3.1030) -- (0.6260, 0.2580, 3.1084) -- (0.6260, 0.3120, 3.1142) -- (0.5800, 0.3120, 3.1088) -- cycle;
\fill[blue!19.0, opacity=0.7] (0.5800, 0.3120, 3.1088) -- (0.6260, 0.3120, 3.1142) -- (0.6260, 0.3660, 3.1198) -- (0.5800, 0.3660, 3.1145) -- cycle;
\fill[blue!23.6, opacity=0.7] (0.5800, 0.3660, 3.1145) -- (0.6260, 0.3660, 3.1198) -- (0.6260, 0.4200, 3.1254) -- (0.5800, 0.4200, 3.1200) -- cycle;
\fill[blue!38.7, opacity=0.7] (0.5800, 0.4200, 3.1200) -- (0.6260, 0.4200, 3.1254) -- (0.6260, 0.4740, 3.1307) -- (0.5800, 0.4740, 3.1254) -- cycle;
\fill[blue!58.9, opacity=0.7] (0.5800, 0.4740, 3.1254) -- (0.6260, 0.4740, 3.1307) -- (0.6260, 0.5280, 3.1359) -- (0.5800, 0.5280, 3.1305) -- cycle;
\fill[blue!62.8, opacity=0.7] (0.5800, 0.5280, 3.1305) -- (0.6260, 0.5280, 3.1359) -- (0.6260, 0.5820, 3.1409) -- (0.5800, 0.5820, 3.1355) -- cycle;
\fill[blue!57.6, opacity=0.7] (0.5800, 0.5820, 3.1355) -- (0.6260, 0.5820, 3.1409) -- (0.6260, 0.6360, 3.1457) -- (0.5800, 0.6360, 3.1403) -- cycle;
\fill[blue!57.7, opacity=0.7] (0.5800, 0.6360, 3.1403) -- (0.6260, 0.6360, 3.1457) -- (0.6260, 0.6900, 3.1502) -- (0.5800, 0.6900, 3.1449) -- cycle;
\fill[blue!62.5, opacity=0.7] (0.5800, 0.6900, 3.1449) -- (0.6260, 0.6900, 3.1502) -- (0.6260, 0.7440, 3.1545) -- (0.5800, 0.7440, 3.1492) -- cycle;
\fill[blue!61.9, opacity=0.7] (0.5800, 0.7440, 3.1492) -- (0.6260, 0.7440, 3.1545) -- (0.6260, 0.7980, 3.1586) -- (0.5800, 0.7980, 3.1533) -- cycle;
\fill[blue!51.3, opacity=0.7] (0.5800, 0.7980, 3.1533) -- (0.6260, 0.7980, 3.1586) -- (0.6260, 0.8520, 3.1624) -- (0.5800, 0.8520, 3.1571) -- cycle;
\fill[blue!39.8, opacity=0.7] (0.5800, 0.8520, 3.1571) -- (0.6260, 0.8520, 3.1624) -- (0.6260, 0.9060, 3.1660) -- (0.5800, 0.9060, 3.1606) -- cycle;
\fill[blue!34.0, opacity=0.7] (0.5800, 0.9060, 3.1606) -- (0.6260, 0.9060, 3.1660) -- (0.6260, 0.9600, 3.1693) -- (0.5800, 0.9600, 3.1639) -- cycle;
\fill[blue!34.0, opacity=0.7] (0.5800, 0.9600, 3.1639) -- (0.6260, 0.9600, 3.1693) -- (0.6260, 1.0140, 3.1723) -- (0.5800, 1.0140, 3.1669) -- cycle;
\fill[blue!39.1, opacity=0.7] (0.5800, 1.0140, 3.1669) -- (0.6260, 1.0140, 3.1723) -- (0.6260, 1.0680, 3.1750) -- (0.5800, 1.0680, 3.1696) -- cycle;
\fill[blue!48.5, opacity=0.7] (0.5800, 1.0680, 3.1696) -- (0.6260, 1.0680, 3.1750) -- (0.6260, 1.1220, 3.1774) -- (0.5800, 1.1220, 3.1720) -- cycle;
\fill[blue!58.5, opacity=0.7] (0.5800, 1.1220, 3.1720) -- (0.6260, 1.1220, 3.1774) -- (0.6260, 1.1760, 3.1795) -- (0.5800, 1.1760, 3.1741) -- cycle;
\fill[blue!63.4, opacity=0.7] (0.5800, 1.1760, 3.1741) -- (0.6260, 1.1760, 3.1795) -- (0.6260, 1.2300, 3.1813) -- (0.5800, 1.2300, 3.1759) -- cycle;
\fill[blue!61.6, opacity=0.7] (0.5800, 1.2300, 3.1759) -- (0.6260, 1.2300, 3.1813) -- (0.6260, 1.2840, 3.1827) -- (0.5800, 1.2840, 3.1774) -- cycle;
\fill[blue!56.0, opacity=0.7] (0.5800, 1.2840, 3.1774) -- (0.6260, 1.2840, 3.1827) -- (0.6260, 1.3380, 3.1839) -- (0.5800, 1.3380, 3.1785) -- cycle;
\fill[blue!50.1, opacity=0.7] (0.5800, 1.3380, 3.1785) -- (0.6260, 1.3380, 3.1839) -- (0.6260, 1.3920, 3.1847) -- (0.5800, 1.3920, 3.1793) -- cycle;
\fill[blue!45.5, opacity=0.7] (0.5800, 1.3920, 3.1793) -- (0.6260, 1.3920, 3.1847) -- (0.6260, 1.4460, 3.1852) -- (0.5800, 1.4460, 3.1798) -- cycle;
\fill[blue!42.5, opacity=0.7] (0.5800, 1.4460, 3.1798) -- (0.6260, 1.4460, 3.1852) -- (0.6260, 1.5000, 3.1854) -- (0.5800, 1.5000, 3.1800) -- cycle;
\fill[blue!40.8, opacity=0.7] (0.5800, 1.5000, 3.1800) -- (0.6260, 1.5000, 3.1854) -- (0.6260, 1.5540, 3.1852) -- (0.5800, 1.5540, 3.1798) -- cycle;
\fill[blue!40.1, opacity=0.7] (0.5800, 1.5540, 3.1798) -- (0.6260, 1.5540, 3.1852) -- (0.6260, 1.6080, 3.1847) -- (0.5800, 1.6080, 3.1793) -- cycle;
\fill[blue!40.2, opacity=0.7] (0.5800, 1.6080, 3.1793) -- (0.6260, 1.6080, 3.1847) -- (0.6260, 1.6620, 3.1839) -- (0.5800, 1.6620, 3.1785) -- cycle;
\fill[blue!40.9, opacity=0.7] (0.5800, 1.6620, 3.1785) -- (0.6260, 1.6620, 3.1839) -- (0.6260, 1.7160, 3.1827) -- (0.5800, 1.7160, 3.1774) -- cycle;
\fill[blue!42.2, opacity=0.7] (0.5800, 1.7160, 3.1774) -- (0.6260, 1.7160, 3.1827) -- (0.6260, 1.7700, 3.1813) -- (0.5800, 1.7700, 3.1759) -- cycle;
\fill[blue!44.5, opacity=0.7] (0.5800, 1.7700, 3.1759) -- (0.6260, 1.7700, 3.1813) -- (0.6260, 1.8240, 3.1795) -- (0.5800, 1.8240, 3.1741) -- cycle;
\fill[blue!47.9, opacity=0.7] (0.5800, 1.8240, 3.1741) -- (0.6260, 1.8240, 3.1795) -- (0.6260, 1.8780, 3.1774) -- (0.5800, 1.8780, 3.1720) -- cycle;
\fill[blue!52.5, opacity=0.7] (0.5800, 1.8780, 3.1720) -- (0.6260, 1.8780, 3.1774) -- (0.6260, 1.9320, 3.1750) -- (0.5800, 1.9320, 3.1696) -- cycle;
\fill[blue!58.0, opacity=0.7] (0.5800, 1.9320, 3.1696) -- (0.6260, 1.9320, 3.1750) -- (0.6260, 1.9860, 3.1723) -- (0.5800, 1.9860, 3.1669) -- cycle;
\fill[blue!62.6, opacity=0.7] (0.5800, 1.9860, 3.1669) -- (0.6260, 1.9860, 3.1723) -- (0.6260, 2.0400, 3.1693) -- (0.5800, 2.0400, 3.1639) -- cycle;
\fill[blue!63.0, opacity=0.7] (0.5800, 2.0400, 3.1639) -- (0.6260, 2.0400, 3.1693) -- (0.6260, 2.0940, 3.1660) -- (0.5800, 2.0940, 3.1606) -- cycle;
\fill[blue!56.7, opacity=0.7] (0.5800, 2.0940, 3.1606) -- (0.6260, 2.0940, 3.1660) -- (0.6260, 2.1480, 3.1624) -- (0.5800, 2.1480, 3.1571) -- cycle;
\fill[blue!45.6, opacity=0.7] (0.5800, 2.1480, 3.1571) -- (0.6260, 2.1480, 3.1624) -- (0.6260, 2.2020, 3.1586) -- (0.5800, 2.2020, 3.1533) -- cycle;
\fill[blue!34.8, opacity=0.7] (0.5800, 2.2020, 3.1533) -- (0.6260, 2.2020, 3.1586) -- (0.6260, 2.2560, 3.1545) -- (0.5800, 2.2560, 3.1492) -- cycle;
\fill[blue!28.1, opacity=0.7] (0.5800, 2.2560, 3.1492) -- (0.6260, 2.2560, 3.1545) -- (0.6260, 2.3100, 3.1502) -- (0.5800, 2.3100, 3.1449) -- cycle;
\fill[blue!25.4, opacity=0.7] (0.5800, 2.3100, 3.1449) -- (0.6260, 2.3100, 3.1502) -- (0.6260, 2.3640, 3.1457) -- (0.5800, 2.3640, 3.1403) -- cycle;
\fill[blue!26.4, opacity=0.7] (0.5800, 2.3640, 3.1403) -- (0.6260, 2.3640, 3.1457) -- (0.6260, 2.4180, 3.1409) -- (0.5800, 2.4180, 3.1355) -- cycle;
\fill[blue!31.9, opacity=0.7] (0.5800, 2.4180, 3.1355) -- (0.6260, 2.4180, 3.1409) -- (0.6260, 2.4720, 3.1359) -- (0.5800, 2.4720, 3.1305) -- cycle;
\fill[blue!42.9, opacity=0.7] (0.5800, 2.4720, 3.1305) -- (0.6260, 2.4720, 3.1359) -- (0.6260, 2.5260, 3.1307) -- (0.5800, 2.5260, 3.1254) -- cycle;
\fill[blue!55.9, opacity=0.7] (0.5800, 2.5260, 3.1254) -- (0.6260, 2.5260, 3.1307) -- (0.6260, 2.5800, 3.1254) -- (0.5800, 2.5800, 3.1200) -- cycle;
\fill[blue!62.8, opacity=0.7] (0.5800, 2.5800, 3.1200) -- (0.6260, 2.5800, 3.1254) -- (0.6260, 2.6340, 3.1198) -- (0.5800, 2.6340, 3.1145) -- cycle;
\fill[blue!63.5, opacity=0.7] (0.5800, 2.6340, 3.1145) -- (0.6260, 2.6340, 3.1198) -- (0.6260, 2.6880, 3.1142) -- (0.5800, 2.6880, 3.1088) -- cycle;
\fill[blue!63.6, opacity=0.7] (0.5800, 2.6880, 3.1088) -- (0.6260, 2.6880, 3.1142) -- (0.6260, 2.7420, 3.1084) -- (0.5800, 2.7420, 3.1030) -- cycle;
\fill[blue!60.5, opacity=0.7] (0.5800, 2.7420, 3.1030) -- (0.6260, 2.7420, 3.1084) -- (0.6260, 2.7960, 3.1024) -- (0.5800, 2.7960, 3.0971) -- cycle;
\fill[blue!45.0, opacity=0.7] (0.5800, 2.7960, 3.0971) -- (0.6260, 2.7960, 3.1024) -- (0.6260, 2.8500, 3.0964) -- (0.5800, 2.8500, 3.0911) -- cycle;
\fill[blue!26.1, opacity=0.7] (0.5800, 2.8500, 3.0911) -- (0.6260, 2.8500, 3.0964) -- (0.6260, 2.9040, 3.0903) -- (0.5800, 2.9040, 3.0849) -- cycle;
\fill[blue!17.9, opacity=0.7] (0.5800, 2.9040, 3.0849) -- (0.6260, 2.9040, 3.0903) -- (0.6260, 2.9580, 3.0841) -- (0.5800, 2.9580, 3.0788) -- cycle;
\fill[blue!16.0, opacity=0.7] (0.5800, 2.9580, 3.0788) -- (0.6260, 2.9580, 3.0841) -- (0.6260, 3.0120, 3.0779) -- (0.5800, 3.0120, 3.0725) -- cycle;
\fill[blue!16.0, opacity=0.7] (0.5800, 3.0120, 3.0725) -- (0.6260, 3.0120, 3.0779) -- (0.6260, 3.0660, 3.0716) -- (0.5800, 3.0660, 3.0663) -- cycle;
\fill[blue!17.4, opacity=0.7] (0.5800, 3.0660, 3.0663) -- (0.6260, 3.0660, 3.0716) -- (0.6260, 3.1200, 3.0654) -- (0.5800, 3.1200, 3.0600) -- cycle;
\fill[blue!28.8, opacity=0.7] (0.6260, -0.1200, 3.0654) -- (0.6720, -0.1200, 3.0705) -- (0.6720, -0.0660, 3.0768) -- (0.6260, -0.0660, 3.0716) -- cycle;
\fill[blue!50.2, opacity=0.7] (0.6260, -0.0660, 3.0716) -- (0.6720, -0.0660, 3.0768) -- (0.6720, -0.0120, 3.0831) -- (0.6260, -0.0120, 3.0779) -- cycle;
\fill[blue!60.2, opacity=0.7] (0.6260, -0.0120, 3.0779) -- (0.6720, -0.0120, 3.0831) -- (0.6720, 0.0420, 3.0893) -- (0.6260, 0.0420, 3.0841) -- cycle;
\fill[blue!59.1, opacity=0.7] (0.6260, 0.0420, 3.0841) -- (0.6720, 0.0420, 3.0893) -- (0.6720, 0.0960, 3.0955) -- (0.6260, 0.0960, 3.0903) -- cycle;
\fill[blue!46.9, opacity=0.7] (0.6260, 0.0960, 3.0903) -- (0.6720, 0.0960, 3.0955) -- (0.6720, 0.1500, 3.1016) -- (0.6260, 0.1500, 3.0964) -- cycle;
\fill[blue!29.3, opacity=0.7] (0.6260, 0.1500, 3.0964) -- (0.6720, 0.1500, 3.1016) -- (0.6720, 0.2040, 3.1076) -- (0.6260, 0.2040, 3.1024) -- cycle;
\fill[blue!20.5, opacity=0.7] (0.6260, 0.2040, 3.1024) -- (0.6720, 0.2040, 3.1076) -- (0.6720, 0.2580, 3.1135) -- (0.6260, 0.2580, 3.1084) -- cycle;
\fill[blue!18.8, opacity=0.7] (0.6260, 0.2580, 3.1084) -- (0.6720, 0.2580, 3.1135) -- (0.6720, 0.3120, 3.1193) -- (0.6260, 0.3120, 3.1142) -- cycle;
\fill[blue!21.9, opacity=0.7] (0.6260, 0.3120, 3.1142) -- (0.6720, 0.3120, 3.1193) -- (0.6720, 0.3660, 3.1250) -- (0.6260, 0.3660, 3.1198) -- cycle;
\fill[blue!34.6, opacity=0.7] (0.6260, 0.3660, 3.1198) -- (0.6720, 0.3660, 3.1250) -- (0.6720, 0.4200, 3.1305) -- (0.6260, 0.4200, 3.1254) -- cycle;
\fill[blue!56.4, opacity=0.7] (0.6260, 0.4200, 3.1254) -- (0.6720, 0.4200, 3.1305) -- (0.6720, 0.4740, 3.1359) -- (0.6260, 0.4740, 3.1307) -- cycle;
\fill[blue!63.2, opacity=0.7] (0.6260, 0.4740, 3.1307) -- (0.6720, 0.4740, 3.1359) -- (0.6720, 0.5280, 3.1411) -- (0.6260, 0.5280, 3.1359) -- cycle;
\fill[blue!57.6, opacity=0.7] (0.6260, 0.5280, 3.1359) -- (0.6720, 0.5280, 3.1411) -- (0.6720, 0.5820, 3.1461) -- (0.6260, 0.5820, 3.1409) -- cycle;
\fill[blue!57.1, opacity=0.7] (0.6260, 0.5820, 3.1409) -- (0.6720, 0.5820, 3.1461) -- (0.6720, 0.6360, 3.1508) -- (0.6260, 0.6360, 3.1457) -- cycle;
\fill[blue!62.4, opacity=0.7] (0.6260, 0.6360, 3.1457) -- (0.6720, 0.6360, 3.1508) -- (0.6720, 0.6900, 3.1554) -- (0.6260, 0.6900, 3.1502) -- cycle;
\fill[blue!61.6, opacity=0.7] (0.6260, 0.6900, 3.1502) -- (0.6720, 0.6900, 3.1554) -- (0.6720, 0.7440, 3.1597) -- (0.6260, 0.7440, 3.1545) -- cycle;
\fill[blue!49.9, opacity=0.7] (0.6260, 0.7440, 3.1545) -- (0.6720, 0.7440, 3.1597) -- (0.6720, 0.7980, 3.1638) -- (0.6260, 0.7980, 3.1586) -- cycle;
\fill[blue!38.4, opacity=0.7] (0.6260, 0.7980, 3.1586) -- (0.6720, 0.7980, 3.1638) -- (0.6720, 0.8520, 3.1676) -- (0.6260, 0.8520, 3.1624) -- cycle;
\fill[blue!34.0, opacity=0.7] (0.6260, 0.8520, 3.1624) -- (0.6720, 0.8520, 3.1676) -- (0.6720, 0.9060, 3.1712) -- (0.6260, 0.9060, 3.1660) -- cycle;
\fill[blue!36.5, opacity=0.7] (0.6260, 0.9060, 3.1660) -- (0.6720, 0.9060, 3.1712) -- (0.6720, 0.9600, 3.1745) -- (0.6260, 0.9600, 3.1693) -- cycle;
\fill[blue!45.3, opacity=0.7] (0.6260, 0.9600, 3.1693) -- (0.6720, 0.9600, 3.1745) -- (0.6720, 1.0140, 3.1775) -- (0.6260, 1.0140, 3.1723) -- cycle;
\fill[blue!57.3, opacity=0.7] (0.6260, 1.0140, 3.1723) -- (0.6720, 1.0140, 3.1775) -- (0.6720, 1.0680, 3.1802) -- (0.6260, 1.0680, 3.1750) -- cycle;
\fill[blue!63.5, opacity=0.7] (0.6260, 1.0680, 3.1750) -- (0.6720, 1.0680, 3.1802) -- (0.6720, 1.1220, 3.1826) -- (0.6260, 1.1220, 3.1774) -- cycle;
\fill[blue!59.4, opacity=0.7] (0.6260, 1.1220, 3.1774) -- (0.6720, 1.1220, 3.1826) -- (0.6720, 1.1760, 3.1847) -- (0.6260, 1.1760, 3.1795) -- cycle;
\fill[blue!50.3, opacity=0.7] (0.6260, 1.1760, 3.1795) -- (0.6720, 1.1760, 3.1847) -- (0.6720, 1.2300, 3.1864) -- (0.6260, 1.2300, 3.1813) -- cycle;
\fill[blue!42.3, opacity=0.7] (0.6260, 1.2300, 3.1813) -- (0.6720, 1.2300, 3.1864) -- (0.6720, 1.2840, 3.1879) -- (0.6260, 1.2840, 3.1827) -- cycle;
\fill[blue!37.5, opacity=0.7] (0.6260, 1.2840, 3.1827) -- (0.6720, 1.2840, 3.1879) -- (0.6720, 1.3380, 3.1891) -- (0.6260, 1.3380, 3.1839) -- cycle;
\fill[blue!35.2, opacity=0.7] (0.6260, 1.3380, 3.1839) -- (0.6720, 1.3380, 3.1891) -- (0.6720, 1.3920, 3.1899) -- (0.6260, 1.3920, 3.1847) -- cycle;
\fill[blue!34.6, opacity=0.7] (0.6260, 1.3920, 3.1847) -- (0.6720, 1.3920, 3.1899) -- (0.6720, 1.4460, 3.1904) -- (0.6260, 1.4460, 3.1852) -- cycle;
\fill[blue!35.0, opacity=0.7] (0.6260, 1.4460, 3.1852) -- (0.6720, 1.4460, 3.1904) -- (0.6720, 1.5000, 3.1905) -- (0.6260, 1.5000, 3.1854) -- cycle;
\fill[blue!35.9, opacity=0.7] (0.6260, 1.5000, 3.1854) -- (0.6720, 1.5000, 3.1905) -- (0.6720, 1.5540, 3.1904) -- (0.6260, 1.5540, 3.1852) -- cycle;
\fill[blue!36.8, opacity=0.7] (0.6260, 1.5540, 3.1852) -- (0.6720, 1.5540, 3.1904) -- (0.6720, 1.6080, 3.1899) -- (0.6260, 1.6080, 3.1847) -- cycle;
\fill[blue!37.5, opacity=0.7] (0.6260, 1.6080, 3.1847) -- (0.6720, 1.6080, 3.1899) -- (0.6720, 1.6620, 3.1891) -- (0.6260, 1.6620, 3.1839) -- cycle;
\fill[blue!37.8, opacity=0.7] (0.6260, 1.6620, 3.1839) -- (0.6720, 1.6620, 3.1891) -- (0.6720, 1.7160, 3.1879) -- (0.6260, 1.7160, 3.1827) -- cycle;
\fill[blue!37.9, opacity=0.7] (0.6260, 1.7160, 3.1827) -- (0.6720, 1.7160, 3.1879) -- (0.6720, 1.7700, 3.1864) -- (0.6260, 1.7700, 3.1813) -- cycle;
\fill[blue!37.9, opacity=0.7] (0.6260, 1.7700, 3.1813) -- (0.6720, 1.7700, 3.1864) -- (0.6720, 1.8240, 3.1847) -- (0.6260, 1.8240, 3.1795) -- cycle;
\fill[blue!38.1, opacity=0.7] (0.6260, 1.8240, 3.1795) -- (0.6720, 1.8240, 3.1847) -- (0.6720, 1.8780, 3.1826) -- (0.6260, 1.8780, 3.1774) -- cycle;
\fill[blue!39.2, opacity=0.7] (0.6260, 1.8780, 3.1774) -- (0.6720, 1.8780, 3.1826) -- (0.6720, 1.9320, 3.1802) -- (0.6260, 1.9320, 3.1750) -- cycle;
\fill[blue!41.7, opacity=0.7] (0.6260, 1.9320, 3.1750) -- (0.6720, 1.9320, 3.1802) -- (0.6720, 1.9860, 3.1775) -- (0.6260, 1.9860, 3.1723) -- cycle;
\fill[blue!46.4, opacity=0.7] (0.6260, 1.9860, 3.1723) -- (0.6720, 1.9860, 3.1775) -- (0.6720, 2.0400, 3.1745) -- (0.6260, 2.0400, 3.1693) -- cycle;
\fill[blue!53.4, opacity=0.7] (0.6260, 2.0400, 3.1693) -- (0.6720, 2.0400, 3.1745) -- (0.6720, 2.0940, 3.1712) -- (0.6260, 2.0940, 3.1660) -- cycle;
\fill[blue!60.9, opacity=0.7] (0.6260, 2.0940, 3.1660) -- (0.6720, 2.0940, 3.1712) -- (0.6720, 2.1480, 3.1676) -- (0.6260, 2.1480, 3.1624) -- cycle;
\fill[blue!63.3, opacity=0.7] (0.6260, 2.1480, 3.1624) -- (0.6720, 2.1480, 3.1676) -- (0.6720, 2.2020, 3.1638) -- (0.6260, 2.2020, 3.1586) -- cycle;
\fill[blue!55.9, opacity=0.7] (0.6260, 2.2020, 3.1586) -- (0.6720, 2.2020, 3.1638) -- (0.6720, 2.2560, 3.1597) -- (0.6260, 2.2560, 3.1545) -- cycle;
\fill[blue!42.4, opacity=0.7] (0.6260, 2.2560, 3.1545) -- (0.6720, 2.2560, 3.1597) -- (0.6720, 2.3100, 3.1554) -- (0.6260, 2.3100, 3.1502) -- cycle;
\fill[blue!31.3, opacity=0.7] (0.6260, 2.3100, 3.1502) -- (0.6720, 2.3100, 3.1554) -- (0.6720, 2.3640, 3.1508) -- (0.6260, 2.3640, 3.1457) -- cycle;
\fill[blue!25.9, opacity=0.7] (0.6260, 2.3640, 3.1457) -- (0.6720, 2.3640, 3.1508) -- (0.6720, 2.4180, 3.1461) -- (0.6260, 2.4180, 3.1409) -- cycle;
\fill[blue!25.2, opacity=0.7] (0.6260, 2.4180, 3.1409) -- (0.6720, 2.4180, 3.1461) -- (0.6720, 2.4720, 3.1411) -- (0.6260, 2.4720, 3.1359) -- cycle;
\fill[blue!29.2, opacity=0.7] (0.6260, 2.4720, 3.1359) -- (0.6720, 2.4720, 3.1411) -- (0.6720, 2.5260, 3.1359) -- (0.6260, 2.5260, 3.1307) -- cycle;
\fill[blue!39.4, opacity=0.7] (0.6260, 2.5260, 3.1307) -- (0.6720, 2.5260, 3.1359) -- (0.6720, 2.5800, 3.1305) -- (0.6260, 2.5800, 3.1254) -- cycle;
\fill[blue!53.6, opacity=0.7] (0.6260, 2.5800, 3.1254) -- (0.6720, 2.5800, 3.1305) -- (0.6720, 2.6340, 3.1250) -- (0.6260, 2.6340, 3.1198) -- cycle;
\fill[blue!62.4, opacity=0.7] (0.6260, 2.6340, 3.1198) -- (0.6720, 2.6340, 3.1250) -- (0.6720, 2.6880, 3.1193) -- (0.6260, 2.6880, 3.1142) -- cycle;
\fill[blue!63.5, opacity=0.7] (0.6260, 2.6880, 3.1142) -- (0.6720, 2.6880, 3.1193) -- (0.6720, 2.7420, 3.1135) -- (0.6260, 2.7420, 3.1084) -- cycle;
\fill[blue!63.6, opacity=0.7] (0.6260, 2.7420, 3.1084) -- (0.6720, 2.7420, 3.1135) -- (0.6720, 2.7960, 3.1076) -- (0.6260, 2.7960, 3.1024) -- cycle;
\fill[blue!59.8, opacity=0.7] (0.6260, 2.7960, 3.1024) -- (0.6720, 2.7960, 3.1076) -- (0.6720, 2.8500, 3.1016) -- (0.6260, 2.8500, 3.0964) -- cycle;
\fill[blue!42.6, opacity=0.7] (0.6260, 2.8500, 3.0964) -- (0.6720, 2.8500, 3.1016) -- (0.6720, 2.9040, 3.0955) -- (0.6260, 2.9040, 3.0903) -- cycle;
\fill[blue!24.1, opacity=0.7] (0.6260, 2.9040, 3.0903) -- (0.6720, 2.9040, 3.0955) -- (0.6720, 2.9580, 3.0893) -- (0.6260, 2.9580, 3.0841) -- cycle;
\fill[blue!17.2, opacity=0.7] (0.6260, 2.9580, 3.0841) -- (0.6720, 2.9580, 3.0893) -- (0.6720, 3.0120, 3.0831) -- (0.6260, 3.0120, 3.0779) -- cycle;
\fill[blue!15.9, opacity=0.7] (0.6260, 3.0120, 3.0779) -- (0.6720, 3.0120, 3.0831) -- (0.6720, 3.0660, 3.0768) -- (0.6260, 3.0660, 3.0716) -- cycle;
\fill[blue!16.1, opacity=0.7] (0.6260, 3.0660, 3.0716) -- (0.6720, 3.0660, 3.0768) -- (0.6720, 3.1200, 3.0705) -- (0.6260, 3.1200, 3.0654) -- cycle;
\fill[blue!40.3, opacity=0.7] (0.6720, -0.1200, 3.0705) -- (0.7180, -0.1200, 3.0755) -- (0.7180, -0.0660, 3.0818) -- (0.6720, -0.0660, 3.0768) -- cycle;
\fill[blue!57.7, opacity=0.7] (0.6720, -0.0660, 3.0768) -- (0.7180, -0.0660, 3.0818) -- (0.7180, -0.0120, 3.0881) -- (0.6720, -0.0120, 3.0831) -- cycle;
\fill[blue!60.8, opacity=0.7] (0.6720, -0.0120, 3.0831) -- (0.7180, -0.0120, 3.0881) -- (0.7180, 0.0420, 3.0943) -- (0.6720, 0.0420, 3.0893) -- cycle;
\fill[blue!54.0, opacity=0.7] (0.6720, 0.0420, 3.0893) -- (0.7180, 0.0420, 3.0943) -- (0.7180, 0.0960, 3.1005) -- (0.6720, 0.0960, 3.0955) -- cycle;
\fill[blue!36.2, opacity=0.7] (0.6720, 0.0960, 3.0955) -- (0.7180, 0.0960, 3.1005) -- (0.7180, 0.1500, 3.1066) -- (0.6720, 0.1500, 3.1016) -- cycle;
\fill[blue!22.9, opacity=0.7] (0.6720, 0.1500, 3.1016) -- (0.7180, 0.1500, 3.1066) -- (0.7180, 0.2040, 3.1126) -- (0.6720, 0.2040, 3.1076) -- cycle;
\fill[blue!19.0, opacity=0.7] (0.6720, 0.2040, 3.1076) -- (0.7180, 0.2040, 3.1126) -- (0.7180, 0.2580, 3.1185) -- (0.6720, 0.2580, 3.1135) -- cycle;
\fill[blue!20.3, opacity=0.7] (0.6720, 0.2580, 3.1135) -- (0.7180, 0.2580, 3.1185) -- (0.7180, 0.3120, 3.1243) -- (0.6720, 0.3120, 3.1193) -- cycle;
\fill[blue!29.6, opacity=0.7] (0.6720, 0.3120, 3.1193) -- (0.7180, 0.3120, 3.1243) -- (0.7180, 0.3660, 3.1300) -- (0.6720, 0.3660, 3.1250) -- cycle;
\fill[blue!51.3, opacity=0.7] (0.6720, 0.3660, 3.1250) -- (0.7180, 0.3660, 3.1300) -- (0.7180, 0.4200, 3.1355) -- (0.6720, 0.4200, 3.1305) -- cycle;
\fill[blue!63.6, opacity=0.7] (0.6720, 0.4200, 3.1305) -- (0.7180, 0.4200, 3.1355) -- (0.7180, 0.4740, 3.1409) -- (0.6720, 0.4740, 3.1359) -- cycle;
\fill[blue!58.3, opacity=0.7] (0.6720, 0.4740, 3.1359) -- (0.7180, 0.4740, 3.1409) -- (0.7180, 0.5280, 3.1461) -- (0.6720, 0.5280, 3.1411) -- cycle;
\fill[blue!56.3, opacity=0.7] (0.6720, 0.5280, 3.1411) -- (0.7180, 0.5280, 3.1461) -- (0.7180, 0.5820, 3.1510) -- (0.6720, 0.5820, 3.1461) -- cycle;
\fill[blue!61.7, opacity=0.7] (0.6720, 0.5820, 3.1461) -- (0.7180, 0.5820, 3.1510) -- (0.7180, 0.6360, 3.1558) -- (0.6720, 0.6360, 3.1508) -- cycle;
\fill[blue!62.2, opacity=0.7] (0.6720, 0.6360, 3.1508) -- (0.7180, 0.6360, 3.1558) -- (0.7180, 0.6900, 3.1604) -- (0.6720, 0.6900, 3.1554) -- cycle;
\fill[blue!50.2, opacity=0.7] (0.6720, 0.6900, 3.1554) -- (0.7180, 0.6900, 3.1604) -- (0.7180, 0.7440, 3.1647) -- (0.6720, 0.7440, 3.1597) -- cycle;
\fill[blue!38.3, opacity=0.7] (0.6720, 0.7440, 3.1597) -- (0.7180, 0.7440, 3.1647) -- (0.7180, 0.7980, 3.1688) -- (0.6720, 0.7980, 3.1638) -- cycle;
\fill[blue!34.5, opacity=0.7] (0.6720, 0.7980, 3.1638) -- (0.7180, 0.7980, 3.1688) -- (0.7180, 0.8520, 3.1726) -- (0.6720, 0.8520, 3.1676) -- cycle;
\fill[blue!38.9, opacity=0.7] (0.6720, 0.8520, 3.1676) -- (0.7180, 0.8520, 3.1726) -- (0.7180, 0.9060, 3.1762) -- (0.6720, 0.9060, 3.1712) -- cycle;
\fill[blue!50.6, opacity=0.7] (0.6720, 0.9060, 3.1712) -- (0.7180, 0.9060, 3.1762) -- (0.7180, 0.9600, 3.1794) -- (0.6720, 0.9600, 3.1745) -- cycle;
\fill[blue!62.1, opacity=0.7] (0.6720, 0.9600, 3.1745) -- (0.7180, 0.9600, 3.1794) -- (0.7180, 1.0140, 3.1824) -- (0.6720, 1.0140, 3.1775) -- cycle;
\fill[blue!61.3, opacity=0.7] (0.6720, 1.0140, 3.1775) -- (0.7180, 1.0140, 3.1824) -- (0.7180, 1.0680, 3.1851) -- (0.6720, 1.0680, 3.1802) -- cycle;
\fill[blue!50.4, opacity=0.7] (0.6720, 1.0680, 3.1802) -- (0.7180, 1.0680, 3.1851) -- (0.7180, 1.1220, 3.1875) -- (0.6720, 1.1220, 3.1826) -- cycle;
\fill[blue!40.2, opacity=0.7] (0.6720, 1.1220, 3.1826) -- (0.7180, 1.1220, 3.1875) -- (0.7180, 1.1760, 3.1896) -- (0.6720, 1.1760, 3.1847) -- cycle;
\fill[blue!34.7, opacity=0.7] (0.6720, 1.1760, 3.1847) -- (0.7180, 1.1760, 3.1896) -- (0.7180, 1.2300, 3.1914) -- (0.6720, 1.2300, 3.1864) -- cycle;
\fill[blue!33.2, opacity=0.7] (0.6720, 1.2300, 3.1864) -- (0.7180, 1.2300, 3.1914) -- (0.7180, 1.2840, 3.1929) -- (0.6720, 1.2840, 3.1879) -- cycle;
\fill[blue!34.3, opacity=0.7] (0.6720, 1.2840, 3.1879) -- (0.7180, 1.2840, 3.1929) -- (0.7180, 1.3380, 3.1940) -- (0.6720, 1.3380, 3.1891) -- cycle;
\fill[blue!37.2, opacity=0.7] (0.6720, 1.3380, 3.1891) -- (0.7180, 1.3380, 3.1940) -- (0.7180, 1.3920, 3.1949) -- (0.6720, 1.3920, 3.1899) -- cycle;
\fill[blue!41.1, opacity=0.7] (0.6720, 1.3920, 3.1899) -- (0.7180, 1.3920, 3.1949) -- (0.7180, 1.4460, 3.1954) -- (0.6720, 1.4460, 3.1904) -- cycle;
\fill[blue!45.1, opacity=0.7] (0.6720, 1.4460, 3.1904) -- (0.7180, 1.4460, 3.1954) -- (0.7180, 1.5000, 3.1955) -- (0.6720, 1.5000, 3.1905) -- cycle;
\fill[blue!48.4, opacity=0.7] (0.6720, 1.5000, 3.1905) -- (0.7180, 1.5000, 3.1955) -- (0.7180, 1.5540, 3.1954) -- (0.6720, 1.5540, 3.1904) -- cycle;
\fill[blue!50.7, opacity=0.7] (0.6720, 1.5540, 3.1904) -- (0.7180, 1.5540, 3.1954) -- (0.7180, 1.6080, 3.1949) -- (0.6720, 1.6080, 3.1899) -- cycle;
\fill[blue!51.9, opacity=0.7] (0.6720, 1.6080, 3.1899) -- (0.7180, 1.6080, 3.1949) -- (0.7180, 1.6620, 3.1940) -- (0.6720, 1.6620, 3.1891) -- cycle;
\fill[blue!52.0, opacity=0.7] (0.6720, 1.6620, 3.1891) -- (0.7180, 1.6620, 3.1940) -- (0.7180, 1.7160, 3.1929) -- (0.6720, 1.7160, 3.1879) -- cycle;
\fill[blue!51.0, opacity=0.7] (0.6720, 1.7160, 3.1879) -- (0.7180, 1.7160, 3.1929) -- (0.7180, 1.7700, 3.1914) -- (0.6720, 1.7700, 3.1864) -- cycle;
\fill[blue!49.0, opacity=0.7] (0.6720, 1.7700, 3.1864) -- (0.7180, 1.7700, 3.1914) -- (0.7180, 1.8240, 3.1896) -- (0.6720, 1.8240, 3.1847) -- cycle;
\fill[blue!46.3, opacity=0.7] (0.6720, 1.8240, 3.1847) -- (0.7180, 1.8240, 3.1896) -- (0.7180, 1.8780, 3.1875) -- (0.6720, 1.8780, 3.1826) -- cycle;
\fill[blue!43.3, opacity=0.7] (0.6720, 1.8780, 3.1826) -- (0.7180, 1.8780, 3.1875) -- (0.7180, 1.9320, 3.1851) -- (0.6720, 1.9320, 3.1802) -- cycle;
\fill[blue!40.8, opacity=0.7] (0.6720, 1.9320, 3.1802) -- (0.7180, 1.9320, 3.1851) -- (0.7180, 1.9860, 3.1824) -- (0.6720, 1.9860, 3.1775) -- cycle;
\fill[blue!39.6, opacity=0.7] (0.6720, 1.9860, 3.1775) -- (0.7180, 1.9860, 3.1824) -- (0.7180, 2.0400, 3.1794) -- (0.6720, 2.0400, 3.1745) -- cycle;
\fill[blue!40.8, opacity=0.7] (0.6720, 2.0400, 3.1745) -- (0.7180, 2.0400, 3.1794) -- (0.7180, 2.0940, 3.1762) -- (0.6720, 2.0940, 3.1712) -- cycle;
\fill[blue!45.3, opacity=0.7] (0.6720, 2.0940, 3.1712) -- (0.7180, 2.0940, 3.1762) -- (0.7180, 2.1480, 3.1726) -- (0.6720, 2.1480, 3.1676) -- cycle;
\fill[blue!53.5, opacity=0.7] (0.6720, 2.1480, 3.1676) -- (0.7180, 2.1480, 3.1726) -- (0.7180, 2.2020, 3.1688) -- (0.6720, 2.2020, 3.1638) -- cycle;
\fill[blue!62.0, opacity=0.7] (0.6720, 2.2020, 3.1638) -- (0.7180, 2.2020, 3.1688) -- (0.7180, 2.2560, 3.1647) -- (0.6720, 2.2560, 3.1597) -- cycle;
\fill[blue!61.9, opacity=0.7] (0.6720, 2.2560, 3.1597) -- (0.7180, 2.2560, 3.1647) -- (0.7180, 2.3100, 3.1604) -- (0.6720, 2.3100, 3.1554) -- cycle;
\fill[blue!49.4, opacity=0.7] (0.6720, 2.3100, 3.1554) -- (0.7180, 2.3100, 3.1604) -- (0.7180, 2.3640, 3.1558) -- (0.6720, 2.3640, 3.1508) -- cycle;
\fill[blue!34.8, opacity=0.7] (0.6720, 2.3640, 3.1508) -- (0.7180, 2.3640, 3.1558) -- (0.7180, 2.4180, 3.1510) -- (0.6720, 2.4180, 3.1461) -- cycle;
\fill[blue!26.7, opacity=0.7] (0.6720, 2.4180, 3.1461) -- (0.7180, 2.4180, 3.1510) -- (0.7180, 2.4720, 3.1461) -- (0.6720, 2.4720, 3.1411) -- cycle;
\fill[blue!24.6, opacity=0.7] (0.6720, 2.4720, 3.1411) -- (0.7180, 2.4720, 3.1461) -- (0.7180, 2.5260, 3.1409) -- (0.6720, 2.5260, 3.1359) -- cycle;
\fill[blue!27.7, opacity=0.7] (0.6720, 2.5260, 3.1359) -- (0.7180, 2.5260, 3.1409) -- (0.7180, 2.5800, 3.1355) -- (0.6720, 2.5800, 3.1305) -- cycle;
\fill[blue!37.6, opacity=0.7] (0.6720, 2.5800, 3.1305) -- (0.7180, 2.5800, 3.1355) -- (0.7180, 2.6340, 3.1300) -- (0.6720, 2.6340, 3.1250) -- cycle;
\fill[blue!52.6, opacity=0.7] (0.6720, 2.6340, 3.1250) -- (0.7180, 2.6340, 3.1300) -- (0.7180, 2.6880, 3.1243) -- (0.6720, 2.6880, 3.1193) -- cycle;
\fill[blue!62.3, opacity=0.7] (0.6720, 2.6880, 3.1193) -- (0.7180, 2.6880, 3.1243) -- (0.7180, 2.7420, 3.1185) -- (0.6720, 2.7420, 3.1135) -- cycle;
\fill[blue!63.6, opacity=0.7] (0.6720, 2.7420, 3.1135) -- (0.7180, 2.7420, 3.1185) -- (0.7180, 2.7960, 3.1126) -- (0.6720, 2.7960, 3.1076) -- cycle;
\fill[blue!63.5, opacity=0.7] (0.6720, 2.7960, 3.1076) -- (0.7180, 2.7960, 3.1126) -- (0.7180, 2.8500, 3.1066) -- (0.6720, 2.8500, 3.1016) -- cycle;
\fill[blue!57.9, opacity=0.7] (0.6720, 2.8500, 3.1016) -- (0.7180, 2.8500, 3.1066) -- (0.7180, 2.9040, 3.1005) -- (0.6720, 2.9040, 3.0955) -- cycle;
\fill[blue!38.3, opacity=0.7] (0.6720, 2.9040, 3.0955) -- (0.7180, 2.9040, 3.1005) -- (0.7180, 2.9580, 3.0943) -- (0.6720, 2.9580, 3.0893) -- cycle;
\fill[blue!21.6, opacity=0.7] (0.6720, 2.9580, 3.0893) -- (0.7180, 2.9580, 3.0943) -- (0.7180, 3.0120, 3.0881) -- (0.6720, 3.0120, 3.0831) -- cycle;
\fill[blue!16.6, opacity=0.7] (0.6720, 3.0120, 3.0831) -- (0.7180, 3.0120, 3.0881) -- (0.7180, 3.0660, 3.0818) -- (0.6720, 3.0660, 3.0768) -- cycle;
\fill[blue!15.8, opacity=0.7] (0.6720, 3.0660, 3.0768) -- (0.7180, 3.0660, 3.0818) -- (0.7180, 3.1200, 3.0755) -- (0.6720, 3.1200, 3.0705) -- cycle;
\fill[blue!50.8, opacity=0.7] (0.7180, -0.1200, 3.0755) -- (0.7640, -0.1200, 3.0803) -- (0.7640, -0.0660, 3.0866) -- (0.7180, -0.0660, 3.0818) -- cycle;
\fill[blue!60.7, opacity=0.7] (0.7180, -0.0660, 3.0818) -- (0.7640, -0.0660, 3.0866) -- (0.7640, -0.0120, 3.0928) -- (0.7180, -0.0120, 3.0881) -- cycle;
\fill[blue!59.0, opacity=0.7] (0.7180, -0.0120, 3.0881) -- (0.7640, -0.0120, 3.0928) -- (0.7640, 0.0420, 3.0991) -- (0.7180, 0.0420, 3.0943) -- cycle;
\fill[blue!45.5, opacity=0.7] (0.7180, 0.0420, 3.0943) -- (0.7640, 0.0420, 3.0991) -- (0.7640, 0.0960, 3.1052) -- (0.7180, 0.0960, 3.1005) -- cycle;
\fill[blue!27.7, opacity=0.7] (0.7180, 0.0960, 3.1005) -- (0.7640, 0.0960, 3.1052) -- (0.7640, 0.1500, 3.1114) -- (0.7180, 0.1500, 3.1066) -- cycle;
\fill[blue!20.0, opacity=0.7] (0.7180, 0.1500, 3.1066) -- (0.7640, 0.1500, 3.1114) -- (0.7640, 0.2040, 3.1174) -- (0.7180, 0.2040, 3.1126) -- cycle;
\fill[blue!19.3, opacity=0.7] (0.7180, 0.2040, 3.1126) -- (0.7640, 0.2040, 3.1174) -- (0.7640, 0.2580, 3.1233) -- (0.7180, 0.2580, 3.1185) -- cycle;
\fill[blue!24.9, opacity=0.7] (0.7180, 0.2580, 3.1185) -- (0.7640, 0.2580, 3.1233) -- (0.7640, 0.3120, 3.1291) -- (0.7180, 0.3120, 3.1243) -- cycle;
\fill[blue!43.5, opacity=0.7] (0.7180, 0.3120, 3.1243) -- (0.7640, 0.3120, 3.1291) -- (0.7640, 0.3660, 3.1348) -- (0.7180, 0.3660, 3.1300) -- cycle;
\fill[blue!62.5, opacity=0.7] (0.7180, 0.3660, 3.1300) -- (0.7640, 0.3660, 3.1348) -- (0.7640, 0.4200, 3.1403) -- (0.7180, 0.4200, 3.1355) -- cycle;
\fill[blue!59.9, opacity=0.7] (0.7180, 0.4200, 3.1355) -- (0.7640, 0.4200, 3.1403) -- (0.7640, 0.4740, 3.1457) -- (0.7180, 0.4740, 3.1409) -- cycle;
\fill[blue!55.5, opacity=0.7] (0.7180, 0.4740, 3.1409) -- (0.7640, 0.4740, 3.1457) -- (0.7640, 0.5280, 3.1508) -- (0.7180, 0.5280, 3.1461) -- cycle;
\fill[blue!60.1, opacity=0.7] (0.7180, 0.5280, 3.1461) -- (0.7640, 0.5280, 3.1508) -- (0.7640, 0.5820, 3.1558) -- (0.7180, 0.5820, 3.1510) -- cycle;
\fill[blue!63.1, opacity=0.7] (0.7180, 0.5820, 3.1510) -- (0.7640, 0.5820, 3.1558) -- (0.7640, 0.6360, 3.1606) -- (0.7180, 0.6360, 3.1558) -- cycle;
\fill[blue!52.2, opacity=0.7] (0.7180, 0.6360, 3.1558) -- (0.7640, 0.6360, 3.1606) -- (0.7640, 0.6900, 3.1651) -- (0.7180, 0.6900, 3.1604) -- cycle;
\fill[blue!39.3, opacity=0.7] (0.7180, 0.6900, 3.1604) -- (0.7640, 0.6900, 3.1651) -- (0.7640, 0.7440, 3.1695) -- (0.7180, 0.7440, 3.1647) -- cycle;
\fill[blue!35.1, opacity=0.7] (0.7180, 0.7440, 3.1647) -- (0.7640, 0.7440, 3.1695) -- (0.7640, 0.7980, 3.1736) -- (0.7180, 0.7980, 3.1688) -- cycle;
\fill[blue!40.4, opacity=0.7] (0.7180, 0.7980, 3.1688) -- (0.7640, 0.7980, 3.1736) -- (0.7640, 0.8520, 3.1774) -- (0.7180, 0.8520, 3.1726) -- cycle;
\fill[blue!53.7, opacity=0.7] (0.7180, 0.8520, 3.1726) -- (0.7640, 0.8520, 3.1774) -- (0.7640, 0.9060, 3.1809) -- (0.7180, 0.9060, 3.1762) -- cycle;
\fill[blue!63.5, opacity=0.7] (0.7180, 0.9060, 3.1762) -- (0.7640, 0.9060, 3.1809) -- (0.7640, 0.9600, 3.1842) -- (0.7180, 0.9600, 3.1794) -- cycle;
\fill[blue!56.7, opacity=0.7] (0.7180, 0.9600, 3.1794) -- (0.7640, 0.9600, 3.1842) -- (0.7640, 1.0140, 3.1872) -- (0.7180, 1.0140, 3.1824) -- cycle;
\fill[blue!43.2, opacity=0.7] (0.7180, 1.0140, 3.1824) -- (0.7640, 1.0140, 3.1872) -- (0.7640, 1.0680, 3.1899) -- (0.7180, 1.0680, 3.1851) -- cycle;
\fill[blue!34.7, opacity=0.7] (0.7180, 1.0680, 3.1851) -- (0.7640, 1.0680, 3.1899) -- (0.7640, 1.1220, 3.1923) -- (0.7180, 1.1220, 3.1875) -- cycle;
\fill[blue!32.2, opacity=0.7] (0.7180, 1.1220, 3.1875) -- (0.7640, 1.1220, 3.1923) -- (0.7640, 1.1760, 3.1944) -- (0.7180, 1.1760, 3.1896) -- cycle;
\fill[blue!34.0, opacity=0.7] (0.7180, 1.1760, 3.1896) -- (0.7640, 1.1760, 3.1944) -- (0.7640, 1.2300, 3.1962) -- (0.7180, 1.2300, 3.1914) -- cycle;
\fill[blue!39.2, opacity=0.7] (0.7180, 1.2300, 3.1914) -- (0.7640, 1.2300, 3.1962) -- (0.7640, 1.2840, 3.1977) -- (0.7180, 1.2840, 3.1929) -- cycle;
\fill[blue!46.3, opacity=0.7] (0.7180, 1.2840, 3.1929) -- (0.7640, 1.2840, 3.1977) -- (0.7640, 1.3380, 3.1988) -- (0.7180, 1.3380, 3.1940) -- cycle;
\fill[blue!53.5, opacity=0.7] (0.7180, 1.3380, 3.1940) -- (0.7640, 1.3380, 3.1988) -- (0.7640, 1.3920, 3.1996) -- (0.7180, 1.3920, 3.1949) -- cycle;
\fill[blue!58.8, opacity=0.7] (0.7180, 1.3920, 3.1949) -- (0.7640, 1.3920, 3.1996) -- (0.7640, 1.4460, 3.2001) -- (0.7180, 1.4460, 3.1954) -- cycle;
\fill[blue!61.9, opacity=0.7] (0.7180, 1.4460, 3.1954) -- (0.7640, 1.4460, 3.2001) -- (0.7640, 1.5000, 3.2003) -- (0.7180, 1.5000, 3.1955) -- cycle;
\fill[blue!63.2, opacity=0.7] (0.7180, 1.5000, 3.1955) -- (0.7640, 1.5000, 3.2003) -- (0.7640, 1.5540, 3.2001) -- (0.7180, 1.5540, 3.1954) -- cycle;
\fill[blue!63.5, opacity=0.7] (0.7180, 1.5540, 3.1954) -- (0.7640, 1.5540, 3.2001) -- (0.7640, 1.6080, 3.1996) -- (0.7180, 1.6080, 3.1949) -- cycle;
\fill[blue!63.6, opacity=0.7] (0.7180, 1.6080, 3.1949) -- (0.7640, 1.6080, 3.1996) -- (0.7640, 1.6620, 3.1988) -- (0.7180, 1.6620, 3.1940) -- cycle;
\fill[blue!63.6, opacity=0.7] (0.7180, 1.6620, 3.1940) -- (0.7640, 1.6620, 3.1988) -- (0.7640, 1.7160, 3.1977) -- (0.7180, 1.7160, 3.1929) -- cycle;
\fill[blue!63.6, opacity=0.7] (0.7180, 1.7160, 3.1929) -- (0.7640, 1.7160, 3.1977) -- (0.7640, 1.7700, 3.1962) -- (0.7180, 1.7700, 3.1914) -- cycle;
\fill[blue!63.3, opacity=0.7] (0.7180, 1.7700, 3.1914) -- (0.7640, 1.7700, 3.1962) -- (0.7640, 1.8240, 3.1944) -- (0.7180, 1.8240, 3.1896) -- cycle;
\fill[blue!62.1, opacity=0.7] (0.7180, 1.8240, 3.1896) -- (0.7640, 1.8240, 3.1944) -- (0.7640, 1.8780, 3.1923) -- (0.7180, 1.8780, 3.1875) -- cycle;
\fill[blue!59.4, opacity=0.7] (0.7180, 1.8780, 3.1875) -- (0.7640, 1.8780, 3.1923) -- (0.7640, 1.9320, 3.1899) -- (0.7180, 1.9320, 3.1851) -- cycle;
\fill[blue!54.8, opacity=0.7] (0.7180, 1.9320, 3.1851) -- (0.7640, 1.9320, 3.1899) -- (0.7640, 1.9860, 3.1872) -- (0.7180, 1.9860, 3.1824) -- cycle;
\fill[blue!49.0, opacity=0.7] (0.7180, 1.9860, 3.1824) -- (0.7640, 1.9860, 3.1872) -- (0.7640, 2.0400, 3.1842) -- (0.7180, 2.0400, 3.1794) -- cycle;
\fill[blue!43.8, opacity=0.7] (0.7180, 2.0400, 3.1794) -- (0.7640, 2.0400, 3.1842) -- (0.7640, 2.0940, 3.1809) -- (0.7180, 2.0940, 3.1762) -- cycle;
\fill[blue!40.9, opacity=0.7] (0.7180, 2.0940, 3.1762) -- (0.7640, 2.0940, 3.1809) -- (0.7640, 2.1480, 3.1774) -- (0.7180, 2.1480, 3.1726) -- cycle;
\fill[blue!41.8, opacity=0.7] (0.7180, 2.1480, 3.1726) -- (0.7640, 2.1480, 3.1774) -- (0.7640, 2.2020, 3.1736) -- (0.7180, 2.2020, 3.1688) -- cycle;
\fill[blue!47.7, opacity=0.7] (0.7180, 2.2020, 3.1688) -- (0.7640, 2.2020, 3.1736) -- (0.7640, 2.2560, 3.1695) -- (0.7180, 2.2560, 3.1647) -- cycle;
\fill[blue!57.8, opacity=0.7] (0.7180, 2.2560, 3.1647) -- (0.7640, 2.2560, 3.1695) -- (0.7640, 2.3100, 3.1651) -- (0.7180, 2.3100, 3.1604) -- cycle;
\fill[blue!63.5, opacity=0.7] (0.7180, 2.3100, 3.1604) -- (0.7640, 2.3100, 3.1651) -- (0.7640, 2.3640, 3.1606) -- (0.7180, 2.3640, 3.1558) -- cycle;
\fill[blue!54.3, opacity=0.7] (0.7180, 2.3640, 3.1558) -- (0.7640, 2.3640, 3.1606) -- (0.7640, 2.4180, 3.1558) -- (0.7180, 2.4180, 3.1510) -- cycle;
\fill[blue!37.6, opacity=0.7] (0.7180, 2.4180, 3.1510) -- (0.7640, 2.4180, 3.1558) -- (0.7640, 2.4720, 3.1508) -- (0.7180, 2.4720, 3.1461) -- cycle;
\fill[blue!27.2, opacity=0.7] (0.7180, 2.4720, 3.1461) -- (0.7640, 2.4720, 3.1508) -- (0.7640, 2.5260, 3.1457) -- (0.7180, 2.5260, 3.1409) -- cycle;
\fill[blue!24.3, opacity=0.7] (0.7180, 2.5260, 3.1409) -- (0.7640, 2.5260, 3.1457) -- (0.7640, 2.5800, 3.1403) -- (0.7180, 2.5800, 3.1355) -- cycle;
\fill[blue!27.1, opacity=0.7] (0.7180, 2.5800, 3.1355) -- (0.7640, 2.5800, 3.1403) -- (0.7640, 2.6340, 3.1348) -- (0.7180, 2.6340, 3.1300) -- cycle;
\fill[blue!37.3, opacity=0.7] (0.7180, 2.6340, 3.1300) -- (0.7640, 2.6340, 3.1348) -- (0.7640, 2.6880, 3.1291) -- (0.7180, 2.6880, 3.1243) -- cycle;
\fill[blue!53.1, opacity=0.7] (0.7180, 2.6880, 3.1243) -- (0.7640, 2.6880, 3.1291) -- (0.7640, 2.7420, 3.1233) -- (0.7180, 2.7420, 3.1185) -- cycle;
\fill[blue!62.5, opacity=0.7] (0.7180, 2.7420, 3.1185) -- (0.7640, 2.7420, 3.1233) -- (0.7640, 2.7960, 3.1174) -- (0.7180, 2.7960, 3.1126) -- cycle;
\fill[blue!63.6, opacity=0.7] (0.7180, 2.7960, 3.1126) -- (0.7640, 2.7960, 3.1174) -- (0.7640, 2.8500, 3.1114) -- (0.7180, 2.8500, 3.1066) -- cycle;
\fill[blue!63.2, opacity=0.7] (0.7180, 2.8500, 3.1066) -- (0.7640, 2.8500, 3.1114) -- (0.7640, 2.9040, 3.1052) -- (0.7180, 2.9040, 3.1005) -- cycle;
\fill[blue!54.3, opacity=0.7] (0.7180, 2.9040, 3.1005) -- (0.7640, 2.9040, 3.1052) -- (0.7640, 2.9580, 3.0991) -- (0.7180, 2.9580, 3.0943) -- cycle;
\fill[blue!32.6, opacity=0.7] (0.7180, 2.9580, 3.0943) -- (0.7640, 2.9580, 3.0991) -- (0.7640, 3.0120, 3.0928) -- (0.7180, 3.0120, 3.0881) -- cycle;
\fill[blue!19.3, opacity=0.7] (0.7180, 3.0120, 3.0881) -- (0.7640, 3.0120, 3.0928) -- (0.7640, 3.0660, 3.0866) -- (0.7180, 3.0660, 3.0818) -- cycle;
\fill[blue!16.1, opacity=0.7] (0.7180, 3.0660, 3.0818) -- (0.7640, 3.0660, 3.0866) -- (0.7640, 3.1200, 3.0803) -- (0.7180, 3.1200, 3.0755) -- cycle;
\fill[blue!57.5, opacity=0.7] (0.7640, -0.1200, 3.0803) -- (0.8100, -0.1200, 3.0849) -- (0.8100, -0.0660, 3.0911) -- (0.7640, -0.0660, 3.0866) -- cycle;
\fill[blue!61.2, opacity=0.7] (0.7640, -0.0660, 3.0866) -- (0.8100, -0.0660, 3.0911) -- (0.8100, -0.0120, 3.0974) -- (0.7640, -0.0120, 3.0928) -- cycle;
\fill[blue!54.5, opacity=0.7] (0.7640, -0.0120, 3.0928) -- (0.8100, -0.0120, 3.0974) -- (0.8100, 0.0420, 3.1036) -- (0.7640, 0.0420, 3.0991) -- cycle;
\fill[blue!36.1, opacity=0.7] (0.7640, 0.0420, 3.0991) -- (0.8100, 0.0420, 3.1036) -- (0.8100, 0.0960, 3.1098) -- (0.7640, 0.0960, 3.1052) -- cycle;
\fill[blue!22.7, opacity=0.7] (0.7640, 0.0960, 3.1052) -- (0.8100, 0.0960, 3.1098) -- (0.8100, 0.1500, 3.1159) -- (0.7640, 0.1500, 3.1114) -- cycle;
\fill[blue!19.2, opacity=0.7] (0.7640, 0.1500, 3.1114) -- (0.8100, 0.1500, 3.1159) -- (0.8100, 0.2040, 3.1219) -- (0.7640, 0.2040, 3.1174) -- cycle;
\fill[blue!21.4, opacity=0.7] (0.7640, 0.2040, 3.1174) -- (0.8100, 0.2040, 3.1219) -- (0.8100, 0.2580, 3.1279) -- (0.7640, 0.2580, 3.1233) -- cycle;
\fill[blue!34.3, opacity=0.7] (0.7640, 0.2580, 3.1233) -- (0.8100, 0.2580, 3.1279) -- (0.8100, 0.3120, 3.1337) -- (0.7640, 0.3120, 3.1291) -- cycle;
\fill[blue!57.7, opacity=0.7] (0.7640, 0.3120, 3.1291) -- (0.8100, 0.3120, 3.1337) -- (0.8100, 0.3660, 3.1393) -- (0.7640, 0.3660, 3.1348) -- cycle;
\fill[blue!62.3, opacity=0.7] (0.7640, 0.3660, 3.1348) -- (0.8100, 0.3660, 3.1393) -- (0.8100, 0.4200, 3.1449) -- (0.7640, 0.4200, 3.1403) -- cycle;
\fill[blue!55.6, opacity=0.7] (0.7640, 0.4200, 3.1403) -- (0.8100, 0.4200, 3.1449) -- (0.8100, 0.4740, 3.1502) -- (0.7640, 0.4740, 3.1457) -- cycle;
\fill[blue!57.8, opacity=0.7] (0.7640, 0.4740, 3.1457) -- (0.8100, 0.4740, 3.1502) -- (0.8100, 0.5280, 3.1554) -- (0.7640, 0.5280, 3.1508) -- cycle;
\fill[blue!63.5, opacity=0.7] (0.7640, 0.5280, 3.1508) -- (0.8100, 0.5280, 3.1554) -- (0.8100, 0.5820, 3.1604) -- (0.7640, 0.5820, 3.1558) -- cycle;
\fill[blue!55.9, opacity=0.7] (0.7640, 0.5820, 3.1558) -- (0.8100, 0.5820, 3.1604) -- (0.8100, 0.6360, 3.1651) -- (0.7640, 0.6360, 3.1606) -- cycle;
\fill[blue!41.4, opacity=0.7] (0.7640, 0.6360, 3.1606) -- (0.8100, 0.6360, 3.1651) -- (0.8100, 0.6900, 3.1697) -- (0.7640, 0.6900, 3.1651) -- cycle;
\fill[blue!35.6, opacity=0.7] (0.7640, 0.6900, 3.1651) -- (0.8100, 0.6900, 3.1697) -- (0.8100, 0.7440, 3.1740) -- (0.7640, 0.7440, 3.1695) -- cycle;
\fill[blue!40.5, opacity=0.7] (0.7640, 0.7440, 3.1695) -- (0.8100, 0.7440, 3.1740) -- (0.8100, 0.7980, 3.1781) -- (0.7640, 0.7980, 3.1736) -- cycle;
\fill[blue!54.6, opacity=0.7] (0.7640, 0.7980, 3.1736) -- (0.8100, 0.7980, 3.1781) -- (0.8100, 0.8520, 3.1819) -- (0.7640, 0.8520, 3.1774) -- cycle;
\fill[blue!63.5, opacity=0.7] (0.7640, 0.8520, 3.1774) -- (0.8100, 0.8520, 3.1819) -- (0.8100, 0.9060, 3.1855) -- (0.7640, 0.9060, 3.1809) -- cycle;
\fill[blue!53.3, opacity=0.7] (0.7640, 0.9060, 3.1809) -- (0.8100, 0.9060, 3.1855) -- (0.8100, 0.9600, 3.1888) -- (0.7640, 0.9600, 3.1842) -- cycle;
\fill[blue!39.0, opacity=0.7] (0.7640, 0.9600, 3.1842) -- (0.8100, 0.9600, 3.1888) -- (0.8100, 1.0140, 3.1918) -- (0.7640, 1.0140, 3.1872) -- cycle;
\fill[blue!32.1, opacity=0.7] (0.7640, 1.0140, 3.1872) -- (0.8100, 1.0140, 3.1918) -- (0.8100, 1.0680, 3.1945) -- (0.7640, 1.0680, 3.1899) -- cycle;
\fill[blue!32.1, opacity=0.7] (0.7640, 1.0680, 3.1899) -- (0.8100, 1.0680, 3.1945) -- (0.8100, 1.1220, 3.1969) -- (0.7640, 1.1220, 3.1923) -- cycle;
\fill[blue!37.3, opacity=0.7] (0.7640, 1.1220, 3.1923) -- (0.8100, 1.1220, 3.1969) -- (0.8100, 1.1760, 3.1990) -- (0.7640, 1.1760, 3.1944) -- cycle;
\fill[blue!46.5, opacity=0.7] (0.7640, 1.1760, 3.1944) -- (0.8100, 1.1760, 3.1990) -- (0.8100, 1.2300, 3.2008) -- (0.7640, 1.2300, 3.1962) -- cycle;
\fill[blue!56.1, opacity=0.7] (0.7640, 1.2300, 3.1962) -- (0.8100, 1.2300, 3.2008) -- (0.8100, 1.2840, 3.2022) -- (0.7640, 1.2840, 3.1977) -- cycle;
\fill[blue!62.1, opacity=0.7] (0.7640, 1.2840, 3.1977) -- (0.8100, 1.2840, 3.2022) -- (0.8100, 1.3380, 3.2034) -- (0.7640, 1.3380, 3.1988) -- cycle;
\fill[blue!63.6, opacity=0.7] (0.7640, 1.3380, 3.1988) -- (0.8100, 1.3380, 3.2034) -- (0.8100, 1.3920, 3.2042) -- (0.7640, 1.3920, 3.1996) -- cycle;
\fill[blue!62.3, opacity=0.7] (0.7640, 1.3920, 3.1996) -- (0.8100, 1.3920, 3.2042) -- (0.8100, 1.4460, 3.2047) -- (0.7640, 1.4460, 3.2001) -- cycle;
\fill[blue!60.3, opacity=0.7] (0.7640, 1.4460, 3.2001) -- (0.8100, 1.4460, 3.2047) -- (0.8100, 1.5000, 3.2049) -- (0.7640, 1.5000, 3.2003) -- cycle;
\fill[blue!58.7, opacity=0.7] (0.7640, 1.5000, 3.2003) -- (0.8100, 1.5000, 3.2049) -- (0.8100, 1.5540, 3.2047) -- (0.7640, 1.5540, 3.2001) -- cycle;
\fill[blue!57.5, opacity=0.7] (0.7640, 1.5540, 3.2001) -- (0.8100, 1.5540, 3.2047) -- (0.8100, 1.6080, 3.2042) -- (0.7640, 1.6080, 3.1996) -- cycle;
\fill[blue!56.8, opacity=0.7] (0.7640, 1.6080, 3.1996) -- (0.8100, 1.6080, 3.2042) -- (0.8100, 1.6620, 3.2034) -- (0.7640, 1.6620, 3.1988) -- cycle;
\fill[blue!56.5, opacity=0.7] (0.7640, 1.6620, 3.1988) -- (0.8100, 1.6620, 3.2034) -- (0.8100, 1.7160, 3.2022) -- (0.7640, 1.7160, 3.1977) -- cycle;
\fill[blue!56.7, opacity=0.7] (0.7640, 1.7160, 3.1977) -- (0.8100, 1.7160, 3.2022) -- (0.8100, 1.7700, 3.2008) -- (0.7640, 1.7700, 3.1962) -- cycle;
\fill[blue!57.6, opacity=0.7] (0.7640, 1.7700, 3.1962) -- (0.8100, 1.7700, 3.2008) -- (0.8100, 1.8240, 3.1990) -- (0.7640, 1.8240, 3.1944) -- cycle;
\fill[blue!59.3, opacity=0.7] (0.7640, 1.8240, 3.1944) -- (0.8100, 1.8240, 3.1990) -- (0.8100, 1.8780, 3.1969) -- (0.7640, 1.8780, 3.1923) -- cycle;
\fill[blue!61.5, opacity=0.7] (0.7640, 1.8780, 3.1923) -- (0.8100, 1.8780, 3.1969) -- (0.8100, 1.9320, 3.1945) -- (0.7640, 1.9320, 3.1899) -- cycle;
\fill[blue!63.4, opacity=0.7] (0.7640, 1.9320, 3.1899) -- (0.8100, 1.9320, 3.1945) -- (0.8100, 1.9860, 3.1918) -- (0.7640, 1.9860, 3.1872) -- cycle;
\fill[blue!62.8, opacity=0.7] (0.7640, 1.9860, 3.1872) -- (0.8100, 1.9860, 3.1918) -- (0.8100, 2.0400, 3.1888) -- (0.7640, 2.0400, 3.1842) -- cycle;
\fill[blue!58.3, opacity=0.7] (0.7640, 2.0400, 3.1842) -- (0.8100, 2.0400, 3.1888) -- (0.8100, 2.0940, 3.1855) -- (0.7640, 2.0940, 3.1809) -- cycle;
\fill[blue!50.9, opacity=0.7] (0.7640, 2.0940, 3.1809) -- (0.8100, 2.0940, 3.1855) -- (0.8100, 2.1480, 3.1819) -- (0.7640, 2.1480, 3.1774) -- cycle;
\fill[blue!44.2, opacity=0.7] (0.7640, 2.1480, 3.1774) -- (0.8100, 2.1480, 3.1819) -- (0.8100, 2.2020, 3.1781) -- (0.7640, 2.2020, 3.1736) -- cycle;
\fill[blue!41.5, opacity=0.7] (0.7640, 2.2020, 3.1736) -- (0.8100, 2.2020, 3.1781) -- (0.8100, 2.2560, 3.1740) -- (0.7640, 2.2560, 3.1695) -- cycle;
\fill[blue!44.8, opacity=0.7] (0.7640, 2.2560, 3.1695) -- (0.8100, 2.2560, 3.1740) -- (0.8100, 2.3100, 3.1697) -- (0.7640, 2.3100, 3.1651) -- cycle;
\fill[blue!54.3, opacity=0.7] (0.7640, 2.3100, 3.1651) -- (0.8100, 2.3100, 3.1697) -- (0.8100, 2.3640, 3.1651) -- (0.7640, 2.3640, 3.1606) -- cycle;
\fill[blue!63.3, opacity=0.7] (0.7640, 2.3640, 3.1606) -- (0.8100, 2.3640, 3.1651) -- (0.8100, 2.4180, 3.1604) -- (0.7640, 2.4180, 3.1558) -- cycle;
\fill[blue!56.6, opacity=0.7] (0.7640, 2.4180, 3.1558) -- (0.8100, 2.4180, 3.1604) -- (0.8100, 2.4720, 3.1554) -- (0.7640, 2.4720, 3.1508) -- cycle;
\fill[blue!38.7, opacity=0.7] (0.7640, 2.4720, 3.1508) -- (0.8100, 2.4720, 3.1554) -- (0.8100, 2.5260, 3.1502) -- (0.7640, 2.5260, 3.1457) -- cycle;
\fill[blue!27.1, opacity=0.7] (0.7640, 2.5260, 3.1457) -- (0.8100, 2.5260, 3.1502) -- (0.8100, 2.5800, 3.1449) -- (0.7640, 2.5800, 3.1403) -- cycle;
\fill[blue!24.0, opacity=0.7] (0.7640, 2.5800, 3.1403) -- (0.8100, 2.5800, 3.1449) -- (0.8100, 2.6340, 3.1393) -- (0.7640, 2.6340, 3.1348) -- cycle;
\fill[blue!27.2, opacity=0.7] (0.7640, 2.6340, 3.1348) -- (0.8100, 2.6340, 3.1393) -- (0.8100, 2.6880, 3.1337) -- (0.7640, 2.6880, 3.1291) -- cycle;
\fill[blue!38.5, opacity=0.7] (0.7640, 2.6880, 3.1291) -- (0.8100, 2.6880, 3.1337) -- (0.8100, 2.7420, 3.1279) -- (0.7640, 2.7420, 3.1233) -- cycle;
\fill[blue!54.9, opacity=0.7] (0.7640, 2.7420, 3.1233) -- (0.8100, 2.7420, 3.1279) -- (0.8100, 2.7960, 3.1219) -- (0.7640, 2.7960, 3.1174) -- cycle;
\fill[blue!63.0, opacity=0.7] (0.7640, 2.7960, 3.1174) -- (0.8100, 2.7960, 3.1219) -- (0.8100, 2.8500, 3.1159) -- (0.7640, 2.8500, 3.1114) -- cycle;
\fill[blue!63.6, opacity=0.7] (0.7640, 2.8500, 3.1114) -- (0.8100, 2.8500, 3.1159) -- (0.8100, 2.9040, 3.1098) -- (0.7640, 2.9040, 3.1052) -- cycle;
\fill[blue!62.2, opacity=0.7] (0.7640, 2.9040, 3.1052) -- (0.8100, 2.9040, 3.1098) -- (0.8100, 2.9580, 3.1036) -- (0.7640, 2.9580, 3.0991) -- cycle;
\fill[blue!48.2, opacity=0.7] (0.7640, 2.9580, 3.0991) -- (0.8100, 2.9580, 3.1036) -- (0.8100, 3.0120, 3.0974) -- (0.7640, 3.0120, 3.0928) -- cycle;
\fill[blue!26.6, opacity=0.7] (0.7640, 3.0120, 3.0928) -- (0.8100, 3.0120, 3.0974) -- (0.8100, 3.0660, 3.0911) -- (0.7640, 3.0660, 3.0866) -- cycle;
\fill[blue!17.5, opacity=0.7] (0.7640, 3.0660, 3.0866) -- (0.8100, 3.0660, 3.0911) -- (0.8100, 3.1200, 3.0849) -- (0.7640, 3.1200, 3.0803) -- cycle;
\fill[blue!60.6, opacity=0.7] (0.8100, -0.1200, 3.0849) -- (0.8560, -0.1200, 3.0892) -- (0.8560, -0.0660, 3.0955) -- (0.8100, -0.0660, 3.0911) -- cycle;
\fill[blue!60.0, opacity=0.7] (0.8100, -0.0660, 3.0911) -- (0.8560, -0.0660, 3.0955) -- (0.8560, -0.0120, 3.1017) -- (0.8100, -0.0120, 3.0974) -- cycle;
\fill[blue!47.5, opacity=0.7] (0.8100, -0.0120, 3.0974) -- (0.8560, -0.0120, 3.1017) -- (0.8560, 0.0420, 3.1079) -- (0.8100, 0.0420, 3.1036) -- cycle;
\fill[blue!28.7, opacity=0.7] (0.8100, 0.0420, 3.1036) -- (0.8560, 0.0420, 3.1079) -- (0.8560, 0.0960, 3.1141) -- (0.8100, 0.0960, 3.1098) -- cycle;
\fill[blue!20.3, opacity=0.7] (0.8100, 0.0960, 3.1098) -- (0.8560, 0.0960, 3.1141) -- (0.8560, 0.1500, 3.1202) -- (0.8100, 0.1500, 3.1159) -- cycle;
\fill[blue!19.7, opacity=0.7] (0.8100, 0.1500, 3.1159) -- (0.8560, 0.1500, 3.1202) -- (0.8560, 0.2040, 3.1263) -- (0.8100, 0.2040, 3.1219) -- cycle;
\fill[blue!26.4, opacity=0.7] (0.8100, 0.2040, 3.1219) -- (0.8560, 0.2040, 3.1263) -- (0.8560, 0.2580, 3.1322) -- (0.8100, 0.2580, 3.1279) -- cycle;
\fill[blue!47.6, opacity=0.7] (0.8100, 0.2580, 3.1279) -- (0.8560, 0.2580, 3.1322) -- (0.8560, 0.3120, 3.1380) -- (0.8100, 0.3120, 3.1337) -- cycle;
\fill[blue!63.5, opacity=0.7] (0.8100, 0.3120, 3.1337) -- (0.8560, 0.3120, 3.1380) -- (0.8560, 0.3660, 3.1437) -- (0.8100, 0.3660, 3.1393) -- cycle;
\fill[blue!57.5, opacity=0.7] (0.8100, 0.3660, 3.1393) -- (0.8560, 0.3660, 3.1437) -- (0.8560, 0.4200, 3.1492) -- (0.8100, 0.4200, 3.1449) -- cycle;
\fill[blue!55.4, opacity=0.7] (0.8100, 0.4200, 3.1449) -- (0.8560, 0.4200, 3.1492) -- (0.8560, 0.4740, 3.1545) -- (0.8100, 0.4740, 3.1502) -- cycle;
\fill[blue!62.1, opacity=0.7] (0.8100, 0.4740, 3.1502) -- (0.8560, 0.4740, 3.1545) -- (0.8560, 0.5280, 3.1597) -- (0.8100, 0.5280, 3.1554) -- cycle;
\fill[blue!60.2, opacity=0.7] (0.8100, 0.5280, 3.1554) -- (0.8560, 0.5280, 3.1597) -- (0.8560, 0.5820, 3.1647) -- (0.8100, 0.5820, 3.1604) -- cycle;
\fill[blue!45.4, opacity=0.7] (0.8100, 0.5820, 3.1604) -- (0.8560, 0.5820, 3.1647) -- (0.8560, 0.6360, 3.1695) -- (0.8100, 0.6360, 3.1651) -- cycle;
\fill[blue!36.5, opacity=0.7] (0.8100, 0.6360, 3.1651) -- (0.8560, 0.6360, 3.1695) -- (0.8560, 0.6900, 3.1740) -- (0.8100, 0.6900, 3.1697) -- cycle;
\fill[blue!39.3, opacity=0.7] (0.8100, 0.6900, 3.1697) -- (0.8560, 0.6900, 3.1740) -- (0.8560, 0.7440, 3.1784) -- (0.8100, 0.7440, 3.1740) -- cycle;
\fill[blue!53.3, opacity=0.7] (0.8100, 0.7440, 3.1740) -- (0.8560, 0.7440, 3.1784) -- (0.8560, 0.7980, 3.1824) -- (0.8100, 0.7980, 3.1781) -- cycle;
\fill[blue!63.6, opacity=0.7] (0.8100, 0.7980, 3.1781) -- (0.8560, 0.7980, 3.1824) -- (0.8560, 0.8520, 3.1863) -- (0.8100, 0.8520, 3.1819) -- cycle;
\fill[blue!52.3, opacity=0.7] (0.8100, 0.8520, 3.1819) -- (0.8560, 0.8520, 3.1863) -- (0.8560, 0.9060, 3.1898) -- (0.8100, 0.9060, 3.1855) -- cycle;
\fill[blue!37.1, opacity=0.7] (0.8100, 0.9060, 3.1855) -- (0.8560, 0.9060, 3.1898) -- (0.8560, 0.9600, 3.1931) -- (0.8100, 0.9600, 3.1888) -- cycle;
\fill[blue!31.0, opacity=0.7] (0.8100, 0.9600, 3.1888) -- (0.8560, 0.9600, 3.1931) -- (0.8560, 1.0140, 3.1961) -- (0.8100, 1.0140, 3.1918) -- cycle;
\fill[blue!32.6, opacity=0.7] (0.8100, 1.0140, 3.1918) -- (0.8560, 1.0140, 3.1961) -- (0.8560, 1.0680, 3.1988) -- (0.8100, 1.0680, 3.1945) -- cycle;
\fill[blue!40.7, opacity=0.7] (0.8100, 1.0680, 3.1945) -- (0.8560, 1.0680, 3.1988) -- (0.8560, 1.1220, 3.2012) -- (0.8100, 1.1220, 3.1969) -- cycle;
\fill[blue!52.7, opacity=0.7] (0.8100, 1.1220, 3.1969) -- (0.8560, 1.1220, 3.2012) -- (0.8560, 1.1760, 3.2033) -- (0.8100, 1.1760, 3.1990) -- cycle;
\fill[blue!61.7, opacity=0.7] (0.8100, 1.1760, 3.1990) -- (0.8560, 1.1760, 3.2033) -- (0.8560, 1.2300, 3.2051) -- (0.8100, 1.2300, 3.2008) -- cycle;
\fill[blue!63.4, opacity=0.7] (0.8100, 1.2300, 3.2008) -- (0.8560, 1.2300, 3.2051) -- (0.8560, 1.2840, 3.2066) -- (0.8100, 1.2840, 3.2022) -- cycle;
\fill[blue!61.0, opacity=0.7] (0.8100, 1.2840, 3.2022) -- (0.8560, 1.2840, 3.2066) -- (0.8560, 1.3380, 3.2077) -- (0.8100, 1.3380, 3.2034) -- cycle;
\fill[blue!58.5, opacity=0.7] (0.8100, 1.3380, 3.2034) -- (0.8560, 1.3380, 3.2077) -- (0.8560, 1.3920, 3.2085) -- (0.8100, 1.3920, 3.2042) -- cycle;
\fill[blue!57.3, opacity=0.7] (0.8100, 1.3920, 3.2042) -- (0.8560, 1.3920, 3.2085) -- (0.8560, 1.4460, 3.2090) -- (0.8100, 1.4460, 3.2047) -- cycle;
\fill[blue!57.2, opacity=0.7] (0.8100, 1.4460, 3.2047) -- (0.8560, 1.4460, 3.2090) -- (0.8560, 1.5000, 3.2092) -- (0.8100, 1.5000, 3.2049) -- cycle;
\fill[blue!57.6, opacity=0.7] (0.8100, 1.5000, 3.2049) -- (0.8560, 1.5000, 3.2092) -- (0.8560, 1.5540, 3.2090) -- (0.8100, 1.5540, 3.2047) -- cycle;
\fill[blue!57.9, opacity=0.7] (0.8100, 1.5540, 3.2047) -- (0.8560, 1.5540, 3.2090) -- (0.8560, 1.6080, 3.2085) -- (0.8100, 1.6080, 3.2042) -- cycle;
\fill[blue!57.8, opacity=0.7] (0.8100, 1.6080, 3.2042) -- (0.8560, 1.6080, 3.2085) -- (0.8560, 1.6620, 3.2077) -- (0.8100, 1.6620, 3.2034) -- cycle;
\fill[blue!57.1, opacity=0.7] (0.8100, 1.6620, 3.2034) -- (0.8560, 1.6620, 3.2077) -- (0.8560, 1.7160, 3.2066) -- (0.8100, 1.7160, 3.2022) -- cycle;
\fill[blue!55.9, opacity=0.7] (0.8100, 1.7160, 3.2022) -- (0.8560, 1.7160, 3.2066) -- (0.8560, 1.7700, 3.2051) -- (0.8100, 1.7700, 3.2008) -- cycle;
\fill[blue!54.5, opacity=0.7] (0.8100, 1.7700, 3.2008) -- (0.8560, 1.7700, 3.2051) -- (0.8560, 1.8240, 3.2033) -- (0.8100, 1.8240, 3.1990) -- cycle;
\fill[blue!53.3, opacity=0.7] (0.8100, 1.8240, 3.1990) -- (0.8560, 1.8240, 3.2033) -- (0.8560, 1.8780, 3.2012) -- (0.8100, 1.8780, 3.1969) -- cycle;
\fill[blue!53.1, opacity=0.7] (0.8100, 1.8780, 3.1969) -- (0.8560, 1.8780, 3.2012) -- (0.8560, 1.9320, 3.1988) -- (0.8100, 1.9320, 3.1945) -- cycle;
\fill[blue!54.9, opacity=0.7] (0.8100, 1.9320, 3.1945) -- (0.8560, 1.9320, 3.1988) -- (0.8560, 1.9860, 3.1961) -- (0.8100, 1.9860, 3.1918) -- cycle;
\fill[blue!58.5, opacity=0.7] (0.8100, 1.9860, 3.1918) -- (0.8560, 1.9860, 3.1961) -- (0.8560, 2.0400, 3.1931) -- (0.8100, 2.0400, 3.1888) -- cycle;
\fill[blue!62.6, opacity=0.7] (0.8100, 2.0400, 3.1888) -- (0.8560, 2.0400, 3.1931) -- (0.8560, 2.0940, 3.1898) -- (0.8100, 2.0940, 3.1855) -- cycle;
\fill[blue!63.1, opacity=0.7] (0.8100, 2.0940, 3.1855) -- (0.8560, 2.0940, 3.1898) -- (0.8560, 2.1480, 3.1863) -- (0.8100, 2.1480, 3.1819) -- cycle;
\fill[blue!57.3, opacity=0.7] (0.8100, 2.1480, 3.1819) -- (0.8560, 2.1480, 3.1863) -- (0.8560, 2.2020, 3.1824) -- (0.8100, 2.2020, 3.1781) -- cycle;
\fill[blue!48.5, opacity=0.7] (0.8100, 2.2020, 3.1781) -- (0.8560, 2.2020, 3.1824) -- (0.8560, 2.2560, 3.1784) -- (0.8100, 2.2560, 3.1740) -- cycle;
\fill[blue!42.8, opacity=0.7] (0.8100, 2.2560, 3.1740) -- (0.8560, 2.2560, 3.1784) -- (0.8560, 2.3100, 3.1740) -- (0.8100, 2.3100, 3.1697) -- cycle;
\fill[blue!43.8, opacity=0.7] (0.8100, 2.3100, 3.1697) -- (0.8560, 2.3100, 3.1740) -- (0.8560, 2.3640, 3.1695) -- (0.8100, 2.3640, 3.1651) -- cycle;
\fill[blue!52.6, opacity=0.7] (0.8100, 2.3640, 3.1651) -- (0.8560, 2.3640, 3.1695) -- (0.8560, 2.4180, 3.1647) -- (0.8100, 2.4180, 3.1604) -- cycle;
\fill[blue!63.0, opacity=0.7] (0.8100, 2.4180, 3.1604) -- (0.8560, 2.4180, 3.1647) -- (0.8560, 2.4720, 3.1597) -- (0.8100, 2.4720, 3.1554) -- cycle;
\fill[blue!57.0, opacity=0.7] (0.8100, 2.4720, 3.1554) -- (0.8560, 2.4720, 3.1597) -- (0.8560, 2.5260, 3.1545) -- (0.8100, 2.5260, 3.1502) -- cycle;
\fill[blue!38.0, opacity=0.7] (0.8100, 2.5260, 3.1502) -- (0.8560, 2.5260, 3.1545) -- (0.8560, 2.5800, 3.1492) -- (0.8100, 2.5800, 3.1449) -- cycle;
\fill[blue!26.3, opacity=0.7] (0.8100, 2.5800, 3.1449) -- (0.8560, 2.5800, 3.1492) -- (0.8560, 2.6340, 3.1437) -- (0.8100, 2.6340, 3.1393) -- cycle;
\fill[blue!23.8, opacity=0.7] (0.8100, 2.6340, 3.1393) -- (0.8560, 2.6340, 3.1437) -- (0.8560, 2.6880, 3.1380) -- (0.8100, 2.6880, 3.1337) -- cycle;
\fill[blue!28.1, opacity=0.7] (0.8100, 2.6880, 3.1337) -- (0.8560, 2.6880, 3.1380) -- (0.8560, 2.7420, 3.1322) -- (0.8100, 2.7420, 3.1279) -- cycle;
\fill[blue!41.4, opacity=0.7] (0.8100, 2.7420, 3.1279) -- (0.8560, 2.7420, 3.1322) -- (0.8560, 2.7960, 3.1263) -- (0.8100, 2.7960, 3.1219) -- cycle;
\fill[blue!57.7, opacity=0.7] (0.8100, 2.7960, 3.1219) -- (0.8560, 2.7960, 3.1263) -- (0.8560, 2.8500, 3.1202) -- (0.8100, 2.8500, 3.1159) -- cycle;
\fill[blue!63.4, opacity=0.7] (0.8100, 2.8500, 3.1159) -- (0.8560, 2.8500, 3.1202) -- (0.8560, 2.9040, 3.1141) -- (0.8100, 2.9040, 3.1098) -- cycle;
\fill[blue!63.5, opacity=0.7] (0.8100, 2.9040, 3.1098) -- (0.8560, 2.9040, 3.1141) -- (0.8560, 2.9580, 3.1079) -- (0.8100, 2.9580, 3.1036) -- cycle;
\fill[blue!59.4, opacity=0.7] (0.8100, 2.9580, 3.1036) -- (0.8560, 2.9580, 3.1079) -- (0.8560, 3.0120, 3.1017) -- (0.8100, 3.0120, 3.0974) -- cycle;
\fill[blue!39.6, opacity=0.7] (0.8100, 3.0120, 3.0974) -- (0.8560, 3.0120, 3.1017) -- (0.8560, 3.0660, 3.0955) -- (0.8100, 3.0660, 3.0911) -- cycle;
\fill[blue!21.4, opacity=0.7] (0.8100, 3.0660, 3.0911) -- (0.8560, 3.0660, 3.0955) -- (0.8560, 3.1200, 3.0892) -- (0.8100, 3.1200, 3.0849) -- cycle;
\fill[blue!61.5, opacity=0.7] (0.8560, -0.1200, 3.0892) -- (0.9020, -0.1200, 3.0933) -- (0.9020, -0.0660, 3.0995) -- (0.8560, -0.0660, 3.0955) -- cycle;
\fill[blue!57.1, opacity=0.7] (0.8560, -0.0660, 3.0955) -- (0.9020, -0.0660, 3.0995) -- (0.9020, -0.0120, 3.1058) -- (0.8560, -0.0120, 3.1017) -- cycle;
\fill[blue!39.6, opacity=0.7] (0.8560, -0.0120, 3.1017) -- (0.9020, -0.0120, 3.1058) -- (0.9020, 0.0420, 3.1120) -- (0.8560, 0.0420, 3.1079) -- cycle;
\fill[blue!23.9, opacity=0.7] (0.8560, 0.0420, 3.1079) -- (0.9020, 0.0420, 3.1120) -- (0.9020, 0.0960, 3.1182) -- (0.8560, 0.0960, 3.1141) -- cycle;
\fill[blue!19.4, opacity=0.7] (0.8560, 0.0960, 3.1141) -- (0.9020, 0.0960, 3.1182) -- (0.9020, 0.1500, 3.1243) -- (0.8560, 0.1500, 3.1202) -- cycle;
\fill[blue!21.5, opacity=0.7] (0.8560, 0.1500, 3.1202) -- (0.9020, 0.1500, 3.1243) -- (0.9020, 0.2040, 3.1303) -- (0.8560, 0.2040, 3.1263) -- cycle;
\fill[blue!35.0, opacity=0.7] (0.8560, 0.2040, 3.1263) -- (0.9020, 0.2040, 3.1303) -- (0.9020, 0.2580, 3.1363) -- (0.8560, 0.2580, 3.1322) -- cycle;
\fill[blue!59.0, opacity=0.7] (0.8560, 0.2580, 3.1322) -- (0.9020, 0.2580, 3.1363) -- (0.9020, 0.3120, 3.1421) -- (0.8560, 0.3120, 3.1380) -- cycle;
\fill[blue!61.3, opacity=0.7] (0.8560, 0.3120, 3.1380) -- (0.9020, 0.3120, 3.1421) -- (0.9020, 0.3660, 3.1477) -- (0.8560, 0.3660, 3.1437) -- cycle;
\fill[blue!54.5, opacity=0.7] (0.8560, 0.3660, 3.1437) -- (0.9020, 0.3660, 3.1477) -- (0.9020, 0.4200, 3.1533) -- (0.8560, 0.4200, 3.1492) -- cycle;
\fill[blue!58.7, opacity=0.7] (0.8560, 0.4200, 3.1492) -- (0.9020, 0.4200, 3.1533) -- (0.9020, 0.4740, 3.1586) -- (0.8560, 0.4740, 3.1545) -- cycle;
\fill[blue!63.3, opacity=0.7] (0.8560, 0.4740, 3.1545) -- (0.9020, 0.4740, 3.1586) -- (0.9020, 0.5280, 3.1638) -- (0.8560, 0.5280, 3.1597) -- cycle;
\fill[blue!51.5, opacity=0.7] (0.8560, 0.5280, 3.1597) -- (0.9020, 0.5280, 3.1638) -- (0.9020, 0.5820, 3.1688) -- (0.8560, 0.5820, 3.1647) -- cycle;
\fill[blue!38.6, opacity=0.7] (0.8560, 0.5820, 3.1647) -- (0.9020, 0.5820, 3.1688) -- (0.9020, 0.6360, 3.1736) -- (0.8560, 0.6360, 3.1695) -- cycle;
\fill[blue!37.8, opacity=0.7] (0.8560, 0.6360, 3.1695) -- (0.9020, 0.6360, 3.1736) -- (0.9020, 0.6900, 3.1781) -- (0.8560, 0.6900, 3.1740) -- cycle;
\fill[blue!49.8, opacity=0.7] (0.8560, 0.6900, 3.1740) -- (0.9020, 0.6900, 3.1781) -- (0.9020, 0.7440, 3.1824) -- (0.8560, 0.7440, 3.1784) -- cycle;
\fill[blue!63.3, opacity=0.7] (0.8560, 0.7440, 3.1784) -- (0.9020, 0.7440, 3.1824) -- (0.9020, 0.7980, 3.1865) -- (0.8560, 0.7980, 3.1824) -- cycle;
\fill[blue!54.0, opacity=0.7] (0.8560, 0.7980, 3.1824) -- (0.9020, 0.7980, 3.1865) -- (0.9020, 0.8520, 3.1903) -- (0.8560, 0.8520, 3.1863) -- cycle;
\fill[blue!37.1, opacity=0.7] (0.8560, 0.8520, 3.1863) -- (0.9020, 0.8520, 3.1903) -- (0.9020, 0.9060, 3.1939) -- (0.8560, 0.9060, 3.1898) -- cycle;
\fill[blue!30.3, opacity=0.7] (0.8560, 0.9060, 3.1898) -- (0.9020, 0.9060, 3.1939) -- (0.9020, 0.9600, 3.1972) -- (0.8560, 0.9600, 3.1931) -- cycle;
\fill[blue!32.5, opacity=0.7] (0.8560, 0.9600, 3.1931) -- (0.9020, 0.9600, 3.1972) -- (0.9020, 1.0140, 3.2002) -- (0.8560, 1.0140, 3.1961) -- cycle;
\fill[blue!42.6, opacity=0.7] (0.8560, 1.0140, 3.1961) -- (0.9020, 1.0140, 3.2002) -- (0.9020, 1.0680, 3.2029) -- (0.8560, 1.0680, 3.1988) -- cycle;
\fill[blue!56.2, opacity=0.7] (0.8560, 1.0680, 3.1988) -- (0.9020, 1.0680, 3.2029) -- (0.9020, 1.1220, 3.2053) -- (0.8560, 1.1220, 3.2012) -- cycle;
\fill[blue!63.3, opacity=0.7] (0.8560, 1.1220, 3.2012) -- (0.9020, 1.1220, 3.2053) -- (0.9020, 1.1760, 3.2074) -- (0.8560, 1.1760, 3.2033) -- cycle;
\fill[blue!62.0, opacity=0.7] (0.8560, 1.1760, 3.2033) -- (0.9020, 1.1760, 3.2074) -- (0.9020, 1.2300, 3.2092) -- (0.8560, 1.2300, 3.2051) -- cycle;
\fill[blue!59.1, opacity=0.7] (0.8560, 1.2300, 3.2051) -- (0.9020, 1.2300, 3.2092) -- (0.9020, 1.2840, 3.2106) -- (0.8560, 1.2840, 3.2066) -- cycle;
\fill[blue!58.6, opacity=0.7] (0.8560, 1.2840, 3.2066) -- (0.9020, 1.2840, 3.2106) -- (0.9020, 1.3380, 3.2118) -- (0.8560, 1.3380, 3.2077) -- cycle;
\fill[blue!60.3, opacity=0.7] (0.8560, 1.3380, 3.2077) -- (0.9020, 1.3380, 3.2118) -- (0.9020, 1.3920, 3.2126) -- (0.8560, 1.3920, 3.2085) -- cycle;
\fill[blue!62.4, opacity=0.7] (0.8560, 1.3920, 3.2085) -- (0.9020, 1.3920, 3.2126) -- (0.9020, 1.4460, 3.2131) -- (0.8560, 1.4460, 3.2090) -- cycle;
\fill[blue!63.5, opacity=0.7] (0.8560, 1.4460, 3.2090) -- (0.9020, 1.4460, 3.2131) -- (0.9020, 1.5000, 3.2133) -- (0.8560, 1.5000, 3.2092) -- cycle;
\fill[blue!63.5, opacity=0.7] (0.8560, 1.5000, 3.2092) -- (0.9020, 1.5000, 3.2133) -- (0.9020, 1.5540, 3.2131) -- (0.8560, 1.5540, 3.2090) -- cycle;
\fill[blue!63.0, opacity=0.7] (0.8560, 1.5540, 3.2090) -- (0.9020, 1.5540, 3.2131) -- (0.9020, 1.6080, 3.2126) -- (0.8560, 1.6080, 3.2085) -- cycle;
\fill[blue!62.9, opacity=0.7] (0.8560, 1.6080, 3.2085) -- (0.9020, 1.6080, 3.2126) -- (0.9020, 1.6620, 3.2118) -- (0.8560, 1.6620, 3.2077) -- cycle;
\fill[blue!63.2, opacity=0.7] (0.8560, 1.6620, 3.2077) -- (0.9020, 1.6620, 3.2118) -- (0.9020, 1.7160, 3.2106) -- (0.8560, 1.7160, 3.2066) -- cycle;
\fill[blue!63.6, opacity=0.7] (0.8560, 1.7160, 3.2066) -- (0.9020, 1.7160, 3.2106) -- (0.9020, 1.7700, 3.2092) -- (0.8560, 1.7700, 3.2051) -- cycle;
\fill[blue!63.0, opacity=0.7] (0.8560, 1.7700, 3.2051) -- (0.9020, 1.7700, 3.2092) -- (0.9020, 1.8240, 3.2074) -- (0.8560, 1.8240, 3.2033) -- cycle;
\fill[blue!60.8, opacity=0.7] (0.8560, 1.8240, 3.2033) -- (0.9020, 1.8240, 3.2074) -- (0.9020, 1.8780, 3.2053) -- (0.8560, 1.8780, 3.2012) -- cycle;
\fill[blue!57.1, opacity=0.7] (0.8560, 1.8780, 3.2012) -- (0.9020, 1.8780, 3.2053) -- (0.9020, 1.9320, 3.2029) -- (0.8560, 1.9320, 3.1988) -- cycle;
\fill[blue!53.3, opacity=0.7] (0.8560, 1.9320, 3.1988) -- (0.9020, 1.9320, 3.2029) -- (0.9020, 1.9860, 3.2002) -- (0.8560, 1.9860, 3.1961) -- cycle;
\fill[blue!51.6, opacity=0.7] (0.8560, 1.9860, 3.1961) -- (0.9020, 1.9860, 3.2002) -- (0.9020, 2.0400, 3.1972) -- (0.8560, 2.0400, 3.1931) -- cycle;
\fill[blue!53.3, opacity=0.7] (0.8560, 2.0400, 3.1931) -- (0.9020, 2.0400, 3.1972) -- (0.9020, 2.0940, 3.1939) -- (0.8560, 2.0940, 3.1898) -- cycle;
\fill[blue!58.5, opacity=0.7] (0.8560, 2.0940, 3.1898) -- (0.9020, 2.0940, 3.1939) -- (0.9020, 2.1480, 3.1903) -- (0.8560, 2.1480, 3.1863) -- cycle;
\fill[blue!63.2, opacity=0.7] (0.8560, 2.1480, 3.1863) -- (0.9020, 2.1480, 3.1903) -- (0.9020, 2.2020, 3.1865) -- (0.8560, 2.2020, 3.1824) -- cycle;
\fill[blue!61.1, opacity=0.7] (0.8560, 2.2020, 3.1824) -- (0.9020, 2.2020, 3.1865) -- (0.9020, 2.2560, 3.1824) -- (0.8560, 2.2560, 3.1784) -- cycle;
\fill[blue!52.1, opacity=0.7] (0.8560, 2.2560, 3.1784) -- (0.9020, 2.2560, 3.1824) -- (0.9020, 2.3100, 3.1781) -- (0.8560, 2.3100, 3.1740) -- cycle;
\fill[blue!44.3, opacity=0.7] (0.8560, 2.3100, 3.1740) -- (0.9020, 2.3100, 3.1781) -- (0.9020, 2.3640, 3.1736) -- (0.8560, 2.3640, 3.1695) -- cycle;
\fill[blue!43.9, opacity=0.7] (0.8560, 2.3640, 3.1695) -- (0.9020, 2.3640, 3.1736) -- (0.9020, 2.4180, 3.1688) -- (0.8560, 2.4180, 3.1647) -- cycle;
\fill[blue!52.6, opacity=0.7] (0.8560, 2.4180, 3.1647) -- (0.9020, 2.4180, 3.1688) -- (0.9020, 2.4720, 3.1638) -- (0.8560, 2.4720, 3.1597) -- cycle;
\fill[blue!63.2, opacity=0.7] (0.8560, 2.4720, 3.1597) -- (0.9020, 2.4720, 3.1638) -- (0.9020, 2.5260, 3.1586) -- (0.8560, 2.5260, 3.1545) -- cycle;
\fill[blue!55.2, opacity=0.7] (0.8560, 2.5260, 3.1545) -- (0.9020, 2.5260, 3.1586) -- (0.9020, 2.5800, 3.1533) -- (0.8560, 2.5800, 3.1492) -- cycle;
\fill[blue!35.4, opacity=0.7] (0.8560, 2.5800, 3.1492) -- (0.9020, 2.5800, 3.1533) -- (0.9020, 2.6340, 3.1477) -- (0.8560, 2.6340, 3.1437) -- cycle;
\fill[blue!25.1, opacity=0.7] (0.8560, 2.6340, 3.1437) -- (0.9020, 2.6340, 3.1477) -- (0.9020, 2.6880, 3.1421) -- (0.8560, 2.6880, 3.1380) -- cycle;
\fill[blue!23.8, opacity=0.7] (0.8560, 2.6880, 3.1380) -- (0.9020, 2.6880, 3.1421) -- (0.9020, 2.7420, 3.1363) -- (0.8560, 2.7420, 3.1322) -- cycle;
\fill[blue!30.2, opacity=0.7] (0.8560, 2.7420, 3.1322) -- (0.9020, 2.7420, 3.1363) -- (0.9020, 2.7960, 3.1303) -- (0.8560, 2.7960, 3.1263) -- cycle;
\fill[blue!46.1, opacity=0.7] (0.8560, 2.7960, 3.1263) -- (0.9020, 2.7960, 3.1303) -- (0.9020, 2.8500, 3.1243) -- (0.8560, 2.8500, 3.1202) -- cycle;
\fill[blue!60.6, opacity=0.7] (0.8560, 2.8500, 3.1202) -- (0.9020, 2.8500, 3.1243) -- (0.9020, 2.9040, 3.1182) -- (0.8560, 2.9040, 3.1141) -- cycle;
\fill[blue!63.6, opacity=0.7] (0.8560, 2.9040, 3.1141) -- (0.9020, 2.9040, 3.1182) -- (0.9020, 2.9580, 3.1120) -- (0.8560, 2.9580, 3.1079) -- cycle;
\fill[blue!63.1, opacity=0.7] (0.8560, 2.9580, 3.1079) -- (0.9020, 2.9580, 3.1120) -- (0.9020, 3.0120, 3.1058) -- (0.8560, 3.0120, 3.1017) -- cycle;
\fill[blue!53.1, opacity=0.7] (0.8560, 3.0120, 3.1017) -- (0.9020, 3.0120, 3.1058) -- (0.9020, 3.0660, 3.0995) -- (0.8560, 3.0660, 3.0955) -- cycle;
\fill[blue!30.0, opacity=0.7] (0.8560, 3.0660, 3.0955) -- (0.9020, 3.0660, 3.0995) -- (0.9020, 3.1200, 3.0933) -- (0.8560, 3.1200, 3.0892) -- cycle;
\fill[blue!61.4, opacity=0.7] (0.9020, -0.1200, 3.0933) -- (0.9480, -0.1200, 3.0971) -- (0.9480, -0.0660, 3.1034) -- (0.9020, -0.0660, 3.0995) -- cycle;
\fill[blue!52.4, opacity=0.7] (0.9020, -0.0660, 3.0995) -- (0.9480, -0.0660, 3.1034) -- (0.9480, -0.0120, 3.1096) -- (0.9020, -0.0120, 3.1058) -- cycle;
\fill[blue!32.6, opacity=0.7] (0.9020, -0.0120, 3.1058) -- (0.9480, -0.0120, 3.1096) -- (0.9480, 0.0420, 3.1159) -- (0.9020, 0.0420, 3.1120) -- cycle;
\fill[blue!21.3, opacity=0.7] (0.9020, 0.0420, 3.1120) -- (0.9480, 0.0420, 3.1159) -- (0.9480, 0.0960, 3.1220) -- (0.9020, 0.0960, 3.1182) -- cycle;
\fill[blue!19.6, opacity=0.7] (0.9020, 0.0960, 3.1182) -- (0.9480, 0.0960, 3.1220) -- (0.9480, 0.1500, 3.1281) -- (0.9020, 0.1500, 3.1243) -- cycle;
\fill[blue!25.4, opacity=0.7] (0.9020, 0.1500, 3.1243) -- (0.9480, 0.1500, 3.1281) -- (0.9480, 0.2040, 3.1342) -- (0.9020, 0.2040, 3.1303) -- cycle;
\fill[blue!46.2, opacity=0.7] (0.9020, 0.2040, 3.1303) -- (0.9480, 0.2040, 3.1342) -- (0.9480, 0.2580, 3.1401) -- (0.9020, 0.2580, 3.1363) -- cycle;
\fill[blue!63.5, opacity=0.7] (0.9020, 0.2580, 3.1363) -- (0.9480, 0.2580, 3.1401) -- (0.9480, 0.3120, 3.1459) -- (0.9020, 0.3120, 3.1421) -- cycle;
\fill[blue!56.9, opacity=0.7] (0.9020, 0.3120, 3.1421) -- (0.9480, 0.3120, 3.1459) -- (0.9480, 0.3660, 3.1516) -- (0.9020, 0.3660, 3.1477) -- cycle;
\fill[blue!54.9, opacity=0.7] (0.9020, 0.3660, 3.1477) -- (0.9480, 0.3660, 3.1516) -- (0.9480, 0.4200, 3.1571) -- (0.9020, 0.4200, 3.1533) -- cycle;
\fill[blue!62.4, opacity=0.7] (0.9020, 0.4200, 3.1533) -- (0.9480, 0.4200, 3.1571) -- (0.9480, 0.4740, 3.1624) -- (0.9020, 0.4740, 3.1586) -- cycle;
\fill[blue!58.9, opacity=0.7] (0.9020, 0.4740, 3.1586) -- (0.9480, 0.4740, 3.1624) -- (0.9480, 0.5280, 3.1676) -- (0.9020, 0.5280, 3.1638) -- cycle;
\fill[blue!43.3, opacity=0.7] (0.9020, 0.5280, 3.1638) -- (0.9480, 0.5280, 3.1676) -- (0.9480, 0.5820, 3.1726) -- (0.9020, 0.5820, 3.1688) -- cycle;
\fill[blue!37.1, opacity=0.7] (0.9020, 0.5820, 3.1688) -- (0.9480, 0.5820, 3.1726) -- (0.9480, 0.6360, 3.1774) -- (0.9020, 0.6360, 3.1736) -- cycle;
\fill[blue!44.8, opacity=0.7] (0.9020, 0.6360, 3.1736) -- (0.9480, 0.6360, 3.1774) -- (0.9480, 0.6900, 3.1819) -- (0.9020, 0.6900, 3.1781) -- cycle;
\fill[blue!61.1, opacity=0.7] (0.9020, 0.6900, 3.1781) -- (0.9480, 0.6900, 3.1819) -- (0.9480, 0.7440, 3.1863) -- (0.9020, 0.7440, 3.1824) -- cycle;
\fill[blue!58.0, opacity=0.7] (0.9020, 0.7440, 3.1824) -- (0.9480, 0.7440, 3.1863) -- (0.9480, 0.7980, 3.1903) -- (0.9020, 0.7980, 3.1865) -- cycle;
\fill[blue!39.2, opacity=0.7] (0.9020, 0.7980, 3.1865) -- (0.9480, 0.7980, 3.1903) -- (0.9480, 0.8520, 3.1942) -- (0.9020, 0.8520, 3.1903) -- cycle;
\fill[blue!30.1, opacity=0.7] (0.9020, 0.8520, 3.1903) -- (0.9480, 0.8520, 3.1942) -- (0.9480, 0.9060, 3.1977) -- (0.9020, 0.9060, 3.1939) -- cycle;
\fill[blue!31.5, opacity=0.7] (0.9020, 0.9060, 3.1939) -- (0.9480, 0.9060, 3.1977) -- (0.9480, 0.9600, 3.2010) -- (0.9020, 0.9600, 3.1972) -- cycle;
\fill[blue!42.1, opacity=0.7] (0.9020, 0.9600, 3.1972) -- (0.9480, 0.9600, 3.2010) -- (0.9480, 1.0140, 3.2040) -- (0.9020, 1.0140, 3.2002) -- cycle;
\fill[blue!57.1, opacity=0.7] (0.9020, 1.0140, 3.2002) -- (0.9480, 1.0140, 3.2040) -- (0.9480, 1.0680, 3.2067) -- (0.9020, 1.0680, 3.2029) -- cycle;
\fill[blue!63.6, opacity=0.7] (0.9020, 1.0680, 3.2029) -- (0.9480, 1.0680, 3.2067) -- (0.9480, 1.1220, 3.2091) -- (0.9020, 1.1220, 3.2053) -- cycle;
\fill[blue!61.1, opacity=0.7] (0.9020, 1.1220, 3.2053) -- (0.9480, 1.1220, 3.2091) -- (0.9480, 1.1760, 3.2112) -- (0.9020, 1.1760, 3.2074) -- cycle;
\fill[blue!59.3, opacity=0.7] (0.9020, 1.1760, 3.2074) -- (0.9480, 1.1760, 3.2112) -- (0.9480, 1.2300, 3.2130) -- (0.9020, 1.2300, 3.2092) -- cycle;
\fill[blue!61.2, opacity=0.7] (0.9020, 1.2300, 3.2092) -- (0.9480, 1.2300, 3.2130) -- (0.9480, 1.2840, 3.2145) -- (0.9020, 1.2840, 3.2106) -- cycle;
\fill[blue!63.5, opacity=0.7] (0.9020, 1.2840, 3.2106) -- (0.9480, 1.2840, 3.2145) -- (0.9480, 1.3380, 3.2156) -- (0.9020, 1.3380, 3.2118) -- cycle;
\fill[blue!62.0, opacity=0.7] (0.9020, 1.3380, 3.2118) -- (0.9480, 1.3380, 3.2156) -- (0.9480, 1.3920, 3.2164) -- (0.9020, 1.3920, 3.2126) -- cycle;
\fill[blue!56.2, opacity=0.7] (0.9020, 1.3920, 3.2126) -- (0.9480, 1.3920, 3.2164) -- (0.9480, 1.4460, 3.2169) -- (0.9020, 1.4460, 3.2131) -- cycle;
\fill[blue!49.2, opacity=0.7] (0.9020, 1.4460, 3.2131) -- (0.9480, 1.4460, 3.2169) -- (0.9480, 1.5000, 3.2171) -- (0.9020, 1.5000, 3.2133) -- cycle;
\fill[blue!43.7, opacity=0.7] (0.9020, 1.5000, 3.2133) -- (0.9480, 1.5000, 3.2171) -- (0.9480, 1.5540, 3.2169) -- (0.9020, 1.5540, 3.2131) -- cycle;
\fill[blue!40.6, opacity=0.7] (0.9020, 1.5540, 3.2131) -- (0.9480, 1.5540, 3.2169) -- (0.9480, 1.6080, 3.2164) -- (0.9020, 1.6080, 3.2126) -- cycle;
\fill[blue!39.9, opacity=0.7] (0.9020, 1.6080, 3.2126) -- (0.9480, 1.6080, 3.2164) -- (0.9480, 1.6620, 3.2156) -- (0.9020, 1.6620, 3.2118) -- cycle;
\fill[blue!41.5, opacity=0.7] (0.9020, 1.6620, 3.2118) -- (0.9480, 1.6620, 3.2156) -- (0.9480, 1.7160, 3.2145) -- (0.9020, 1.7160, 3.2106) -- cycle;
\fill[blue!45.4, opacity=0.7] (0.9020, 1.7160, 3.2106) -- (0.9480, 1.7160, 3.2145) -- (0.9480, 1.7700, 3.2130) -- (0.9020, 1.7700, 3.2092) -- cycle;
\fill[blue!51.4, opacity=0.7] (0.9020, 1.7700, 3.2092) -- (0.9480, 1.7700, 3.2130) -- (0.9480, 1.8240, 3.2112) -- (0.9020, 1.8240, 3.2074) -- cycle;
\fill[blue!58.2, opacity=0.7] (0.9020, 1.8240, 3.2074) -- (0.9480, 1.8240, 3.2112) -- (0.9480, 1.8780, 3.2091) -- (0.9020, 1.8780, 3.2053) -- cycle;
\fill[blue!63.0, opacity=0.7] (0.9020, 1.8780, 3.2053) -- (0.9480, 1.8780, 3.2091) -- (0.9480, 1.9320, 3.2067) -- (0.9020, 1.9320, 3.2029) -- cycle;
\fill[blue!62.6, opacity=0.7] (0.9020, 1.9320, 3.2029) -- (0.9480, 1.9320, 3.2067) -- (0.9480, 1.9860, 3.2040) -- (0.9020, 1.9860, 3.2002) -- cycle;
\fill[blue!57.5, opacity=0.7] (0.9020, 1.9860, 3.2002) -- (0.9480, 1.9860, 3.2040) -- (0.9480, 2.0400, 3.2010) -- (0.9020, 2.0400, 3.1972) -- cycle;
\fill[blue!52.1, opacity=0.7] (0.9020, 2.0400, 3.1972) -- (0.9480, 2.0400, 3.2010) -- (0.9480, 2.0940, 3.1977) -- (0.9020, 2.0940, 3.1939) -- cycle;
\fill[blue!50.7, opacity=0.7] (0.9020, 2.0940, 3.1939) -- (0.9480, 2.0940, 3.1977) -- (0.9480, 2.1480, 3.1942) -- (0.9020, 2.1480, 3.1903) -- cycle;
\fill[blue!54.7, opacity=0.7] (0.9020, 2.1480, 3.1903) -- (0.9480, 2.1480, 3.1942) -- (0.9480, 2.2020, 3.1903) -- (0.9020, 2.2020, 3.1865) -- cycle;
\fill[blue!61.6, opacity=0.7] (0.9020, 2.2020, 3.1865) -- (0.9480, 2.2020, 3.1903) -- (0.9480, 2.2560, 3.1863) -- (0.9020, 2.2560, 3.1824) -- cycle;
\fill[blue!62.7, opacity=0.7] (0.9020, 2.2560, 3.1824) -- (0.9480, 2.2560, 3.1863) -- (0.9480, 2.3100, 3.1819) -- (0.9020, 2.3100, 3.1781) -- cycle;
\fill[blue!54.1, opacity=0.7] (0.9020, 2.3100, 3.1781) -- (0.9480, 2.3100, 3.1819) -- (0.9480, 2.3640, 3.1774) -- (0.9020, 2.3640, 3.1736) -- cycle;
\fill[blue!45.2, opacity=0.7] (0.9020, 2.3640, 3.1736) -- (0.9480, 2.3640, 3.1774) -- (0.9480, 2.4180, 3.1726) -- (0.9020, 2.4180, 3.1688) -- cycle;
\fill[blue!44.6, opacity=0.7] (0.9020, 2.4180, 3.1688) -- (0.9480, 2.4180, 3.1726) -- (0.9480, 2.4720, 3.1676) -- (0.9020, 2.4720, 3.1638) -- cycle;
\fill[blue!54.1, opacity=0.7] (0.9020, 2.4720, 3.1638) -- (0.9480, 2.4720, 3.1676) -- (0.9480, 2.5260, 3.1624) -- (0.9020, 2.5260, 3.1586) -- cycle;
\fill[blue!63.6, opacity=0.7] (0.9020, 2.5260, 3.1586) -- (0.9480, 2.5260, 3.1624) -- (0.9480, 2.5800, 3.1571) -- (0.9020, 2.5800, 3.1533) -- cycle;
\fill[blue!51.2, opacity=0.7] (0.9020, 2.5800, 3.1533) -- (0.9480, 2.5800, 3.1571) -- (0.9480, 2.6340, 3.1516) -- (0.9020, 2.6340, 3.1477) -- cycle;
\fill[blue!31.6, opacity=0.7] (0.9020, 2.6340, 3.1477) -- (0.9480, 2.6340, 3.1516) -- (0.9480, 2.6880, 3.1459) -- (0.9020, 2.6880, 3.1421) -- cycle;
\fill[blue!23.8, opacity=0.7] (0.9020, 2.6880, 3.1421) -- (0.9480, 2.6880, 3.1459) -- (0.9480, 2.7420, 3.1401) -- (0.9020, 2.7420, 3.1363) -- cycle;
\fill[blue!24.6, opacity=0.7] (0.9020, 2.7420, 3.1363) -- (0.9480, 2.7420, 3.1401) -- (0.9480, 2.7960, 3.1342) -- (0.9020, 2.7960, 3.1303) -- cycle;
\fill[blue!34.2, opacity=0.7] (0.9020, 2.7960, 3.1303) -- (0.9480, 2.7960, 3.1342) -- (0.9480, 2.8500, 3.1281) -- (0.9020, 2.8500, 3.1243) -- cycle;
\fill[blue!52.1, opacity=0.7] (0.9020, 2.8500, 3.1243) -- (0.9480, 2.8500, 3.1281) -- (0.9480, 2.9040, 3.1220) -- (0.9020, 2.9040, 3.1182) -- cycle;
\fill[blue!62.6, opacity=0.7] (0.9020, 2.9040, 3.1182) -- (0.9480, 2.9040, 3.1220) -- (0.9480, 2.9580, 3.1159) -- (0.9020, 2.9580, 3.1120) -- cycle;
\fill[blue!63.6, opacity=0.7] (0.9020, 2.9580, 3.1120) -- (0.9480, 2.9580, 3.1159) -- (0.9480, 3.0120, 3.1096) -- (0.9020, 3.0120, 3.1058) -- cycle;
\fill[blue!60.8, opacity=0.7] (0.9020, 3.0120, 3.1058) -- (0.9480, 3.0120, 3.1096) -- (0.9480, 3.0660, 3.1034) -- (0.9020, 3.0660, 3.0995) -- cycle;
\fill[blue!42.4, opacity=0.7] (0.9020, 3.0660, 3.0995) -- (0.9480, 3.0660, 3.1034) -- (0.9480, 3.1200, 3.0971) -- (0.9020, 3.1200, 3.0933) -- cycle;
\fill[blue!60.3, opacity=0.7] (0.9480, -0.1200, 3.0971) -- (0.9940, -0.1200, 3.1006) -- (0.9940, -0.0660, 3.1069) -- (0.9480, -0.0660, 3.1034) -- cycle;
\fill[blue!46.6, opacity=0.7] (0.9480, -0.0660, 3.1034) -- (0.9940, -0.0660, 3.1069) -- (0.9940, -0.0120, 3.1132) -- (0.9480, -0.0120, 3.1096) -- cycle;
\fill[blue!27.6, opacity=0.7] (0.9480, -0.0120, 3.1096) -- (0.9940, -0.0120, 3.1132) -- (0.9940, 0.0420, 3.1194) -- (0.9480, 0.0420, 3.1159) -- cycle;
\fill[blue!20.1, opacity=0.7] (0.9480, 0.0420, 3.1159) -- (0.9940, 0.0420, 3.1194) -- (0.9940, 0.0960, 3.1256) -- (0.9480, 0.0960, 3.1220) -- cycle;
\fill[blue!20.7, opacity=0.7] (0.9480, 0.0960, 3.1220) -- (0.9940, 0.0960, 3.1256) -- (0.9940, 0.1500, 3.1317) -- (0.9480, 0.1500, 3.1281) -- cycle;
\fill[blue!31.5, opacity=0.7] (0.9480, 0.1500, 3.1281) -- (0.9940, 0.1500, 3.1317) -- (0.9940, 0.2040, 3.1377) -- (0.9480, 0.2040, 3.1342) -- cycle;
\fill[blue!56.4, opacity=0.7] (0.9480, 0.2040, 3.1342) -- (0.9940, 0.2040, 3.1377) -- (0.9940, 0.2580, 3.1436) -- (0.9480, 0.2580, 3.1401) -- cycle;
\fill[blue!62.0, opacity=0.7] (0.9480, 0.2580, 3.1401) -- (0.9940, 0.2580, 3.1436) -- (0.9940, 0.3120, 3.1494) -- (0.9480, 0.3120, 3.1459) -- cycle;
\fill[blue!54.0, opacity=0.7] (0.9480, 0.3120, 3.1459) -- (0.9940, 0.3120, 3.1494) -- (0.9940, 0.3660, 3.1551) -- (0.9480, 0.3660, 3.1516) -- cycle;
\fill[blue!57.7, opacity=0.7] (0.9480, 0.3660, 3.1516) -- (0.9940, 0.3660, 3.1551) -- (0.9940, 0.4200, 3.1606) -- (0.9480, 0.4200, 3.1571) -- cycle;
\fill[blue!63.4, opacity=0.7] (0.9480, 0.4200, 3.1571) -- (0.9940, 0.4200, 3.1606) -- (0.9940, 0.4740, 3.1660) -- (0.9480, 0.4740, 3.1624) -- cycle;
\fill[blue!51.5, opacity=0.7] (0.9480, 0.4740, 3.1624) -- (0.9940, 0.4740, 3.1660) -- (0.9940, 0.5280, 3.1712) -- (0.9480, 0.5280, 3.1676) -- cycle;
\fill[blue!38.8, opacity=0.7] (0.9480, 0.5280, 3.1676) -- (0.9940, 0.5280, 3.1712) -- (0.9940, 0.5820, 3.1762) -- (0.9480, 0.5820, 3.1726) -- cycle;
\fill[blue!40.1, opacity=0.7] (0.9480, 0.5820, 3.1726) -- (0.9940, 0.5820, 3.1762) -- (0.9940, 0.6360, 3.1809) -- (0.9480, 0.6360, 3.1774) -- cycle;
\fill[blue!55.2, opacity=0.7] (0.9480, 0.6360, 3.1774) -- (0.9940, 0.6360, 3.1809) -- (0.9940, 0.6900, 3.1855) -- (0.9480, 0.6900, 3.1819) -- cycle;
\fill[blue!62.5, opacity=0.7] (0.9480, 0.6900, 3.1819) -- (0.9940, 0.6900, 3.1855) -- (0.9940, 0.7440, 3.1898) -- (0.9480, 0.7440, 3.1863) -- cycle;
\fill[blue!44.3, opacity=0.7] (0.9480, 0.7440, 3.1863) -- (0.9940, 0.7440, 3.1898) -- (0.9940, 0.7980, 3.1939) -- (0.9480, 0.7980, 3.1903) -- cycle;
\fill[blue!30.9, opacity=0.7] (0.9480, 0.7980, 3.1903) -- (0.9940, 0.7980, 3.1939) -- (0.9940, 0.8520, 3.1977) -- (0.9480, 0.8520, 3.1942) -- cycle;
\fill[blue!29.8, opacity=0.7] (0.9480, 0.8520, 3.1942) -- (0.9940, 0.8520, 3.1977) -- (0.9940, 0.9060, 3.2013) -- (0.9480, 0.9060, 3.1977) -- cycle;
\fill[blue!39.2, opacity=0.7] (0.9480, 0.9060, 3.1977) -- (0.9940, 0.9060, 3.2013) -- (0.9940, 0.9600, 3.2046) -- (0.9480, 0.9600, 3.2010) -- cycle;
\fill[blue!55.5, opacity=0.7] (0.9480, 0.9600, 3.2010) -- (0.9940, 0.9600, 3.2046) -- (0.9940, 1.0140, 3.2076) -- (0.9480, 1.0140, 3.2040) -- cycle;
\fill[blue!63.5, opacity=0.7] (0.9480, 1.0140, 3.2040) -- (0.9940, 1.0140, 3.2076) -- (0.9940, 1.0680, 3.2103) -- (0.9480, 1.0680, 3.2067) -- cycle;
\fill[blue!61.2, opacity=0.7] (0.9480, 1.0680, 3.2067) -- (0.9940, 1.0680, 3.2103) -- (0.9940, 1.1220, 3.2127) -- (0.9480, 1.1220, 3.2091) -- cycle;
\fill[blue!60.3, opacity=0.7] (0.9480, 1.1220, 3.2091) -- (0.9940, 1.1220, 3.2127) -- (0.9940, 1.1760, 3.2148) -- (0.9480, 1.1760, 3.2112) -- cycle;
\fill[blue!63.0, opacity=0.7] (0.9480, 1.1760, 3.2112) -- (0.9940, 1.1760, 3.2148) -- (0.9940, 1.2300, 3.2166) -- (0.9480, 1.2300, 3.2130) -- cycle;
\fill[blue!62.0, opacity=0.7] (0.9480, 1.2300, 3.2130) -- (0.9940, 1.2300, 3.2166) -- (0.9940, 1.2840, 3.2180) -- (0.9480, 1.2840, 3.2145) -- cycle;
\fill[blue!52.3, opacity=0.7] (0.9480, 1.2840, 3.2145) -- (0.9940, 1.2840, 3.2180) -- (0.9940, 1.3380, 3.2192) -- (0.9480, 1.3380, 3.2156) -- cycle;
\fill[blue!39.5, opacity=0.7] (0.9480, 1.3380, 3.2156) -- (0.9940, 1.3380, 3.2192) -- (0.9940, 1.3920, 3.2200) -- (0.9480, 1.3920, 3.2164) -- cycle;
\fill[blue!30.3, opacity=0.7] (0.9480, 1.3920, 3.2164) -- (0.9940, 1.3920, 3.2200) -- (0.9940, 1.4460, 3.2205) -- (0.9480, 1.4460, 3.2169) -- cycle;
\fill[blue!25.3, opacity=0.7] (0.9480, 1.4460, 3.2169) -- (0.9940, 1.4460, 3.2205) -- (0.9940, 1.5000, 3.2206) -- (0.9480, 1.5000, 3.2171) -- cycle;
\fill[blue!22.9, opacity=0.7] (0.9480, 1.5000, 3.2171) -- (0.9940, 1.5000, 3.2206) -- (0.9940, 1.5540, 3.2205) -- (0.9480, 1.5540, 3.2169) -- cycle;
\fill[blue!21.9, opacity=0.7] (0.9480, 1.5540, 3.2169) -- (0.9940, 1.5540, 3.2205) -- (0.9940, 1.6080, 3.2200) -- (0.9480, 1.6080, 3.2164) -- cycle;
\fill[blue!21.9, opacity=0.7] (0.9480, 1.6080, 3.2164) -- (0.9940, 1.6080, 3.2200) -- (0.9940, 1.6620, 3.2192) -- (0.9480, 1.6620, 3.2156) -- cycle;
\fill[blue!22.5, opacity=0.7] (0.9480, 1.6620, 3.2156) -- (0.9940, 1.6620, 3.2192) -- (0.9940, 1.7160, 3.2180) -- (0.9480, 1.7160, 3.2145) -- cycle;
\fill[blue!24.1, opacity=0.7] (0.9480, 1.7160, 3.2145) -- (0.9940, 1.7160, 3.2180) -- (0.9940, 1.7700, 3.2166) -- (0.9480, 1.7700, 3.2130) -- cycle;
\fill[blue!27.2, opacity=0.7] (0.9480, 1.7700, 3.2130) -- (0.9940, 1.7700, 3.2166) -- (0.9940, 1.8240, 3.2148) -- (0.9480, 1.8240, 3.2112) -- cycle;
\fill[blue!32.9, opacity=0.7] (0.9480, 1.8240, 3.2112) -- (0.9940, 1.8240, 3.2148) -- (0.9940, 1.8780, 3.2127) -- (0.9480, 1.8780, 3.2091) -- cycle;
\fill[blue!42.4, opacity=0.7] (0.9480, 1.8780, 3.2091) -- (0.9940, 1.8780, 3.2127) -- (0.9940, 1.9320, 3.2103) -- (0.9480, 1.9320, 3.2067) -- cycle;
\fill[blue!54.6, opacity=0.7] (0.9480, 1.9320, 3.2067) -- (0.9940, 1.9320, 3.2103) -- (0.9940, 1.9860, 3.2076) -- (0.9480, 1.9860, 3.2040) -- cycle;
\fill[blue!63.0, opacity=0.7] (0.9480, 1.9860, 3.2040) -- (0.9940, 1.9860, 3.2076) -- (0.9940, 2.0400, 3.2046) -- (0.9480, 2.0400, 3.2010) -- cycle;
\fill[blue!61.4, opacity=0.7] (0.9480, 2.0400, 3.2010) -- (0.9940, 2.0400, 3.2046) -- (0.9940, 2.0940, 3.2013) -- (0.9480, 2.0940, 3.1977) -- cycle;
\fill[blue!54.1, opacity=0.7] (0.9480, 2.0940, 3.1977) -- (0.9940, 2.0940, 3.2013) -- (0.9940, 2.1480, 3.1977) -- (0.9480, 2.1480, 3.1942) -- cycle;
\fill[blue!49.7, opacity=0.7] (0.9480, 2.1480, 3.1942) -- (0.9940, 2.1480, 3.1977) -- (0.9940, 2.2020, 3.1939) -- (0.9480, 2.2020, 3.1903) -- cycle;
\fill[blue!52.4, opacity=0.7] (0.9480, 2.2020, 3.1903) -- (0.9940, 2.2020, 3.1939) -- (0.9940, 2.2560, 3.1898) -- (0.9480, 2.2560, 3.1863) -- cycle;
\fill[blue!60.3, opacity=0.7] (0.9480, 2.2560, 3.1863) -- (0.9940, 2.2560, 3.1898) -- (0.9940, 2.3100, 3.1855) -- (0.9480, 2.3100, 3.1819) -- cycle;
\fill[blue!63.1, opacity=0.7] (0.9480, 2.3100, 3.1819) -- (0.9940, 2.3100, 3.1855) -- (0.9940, 2.3640, 3.1809) -- (0.9480, 2.3640, 3.1774) -- cycle;
\fill[blue!54.4, opacity=0.7] (0.9480, 2.3640, 3.1774) -- (0.9940, 2.3640, 3.1809) -- (0.9940, 2.4180, 3.1762) -- (0.9480, 2.4180, 3.1726) -- cycle;
\fill[blue!45.4, opacity=0.7] (0.9480, 2.4180, 3.1726) -- (0.9940, 2.4180, 3.1762) -- (0.9940, 2.4720, 3.1712) -- (0.9480, 2.4720, 3.1676) -- cycle;
\fill[blue!45.9, opacity=0.7] (0.9480, 2.4720, 3.1676) -- (0.9940, 2.4720, 3.1712) -- (0.9940, 2.5260, 3.1660) -- (0.9480, 2.5260, 3.1624) -- cycle;
\fill[blue!57.0, opacity=0.7] (0.9480, 2.5260, 3.1624) -- (0.9940, 2.5260, 3.1660) -- (0.9940, 2.5800, 3.1606) -- (0.9480, 2.5800, 3.1571) -- cycle;
\fill[blue!62.8, opacity=0.7] (0.9480, 2.5800, 3.1571) -- (0.9940, 2.5800, 3.1606) -- (0.9940, 2.6340, 3.1551) -- (0.9480, 2.6340, 3.1516) -- cycle;
\fill[blue!44.6, opacity=0.7] (0.9480, 2.6340, 3.1516) -- (0.9940, 2.6340, 3.1551) -- (0.9940, 2.6880, 3.1494) -- (0.9480, 2.6880, 3.1459) -- cycle;
\fill[blue!27.6, opacity=0.7] (0.9480, 2.6880, 3.1459) -- (0.9940, 2.6880, 3.1494) -- (0.9940, 2.7420, 3.1436) -- (0.9480, 2.7420, 3.1401) -- cycle;
\fill[blue!23.0, opacity=0.7] (0.9480, 2.7420, 3.1401) -- (0.9940, 2.7420, 3.1436) -- (0.9940, 2.7960, 3.1377) -- (0.9480, 2.7960, 3.1342) -- cycle;
\fill[blue!26.6, opacity=0.7] (0.9480, 2.7960, 3.1342) -- (0.9940, 2.7960, 3.1377) -- (0.9940, 2.8500, 3.1317) -- (0.9480, 2.8500, 3.1281) -- cycle;
\fill[blue!40.6, opacity=0.7] (0.9480, 2.8500, 3.1281) -- (0.9940, 2.8500, 3.1317) -- (0.9940, 2.9040, 3.1256) -- (0.9480, 2.9040, 3.1220) -- cycle;
\fill[blue!58.1, opacity=0.7] (0.9480, 2.9040, 3.1220) -- (0.9940, 2.9040, 3.1256) -- (0.9940, 2.9580, 3.1194) -- (0.9480, 2.9580, 3.1159) -- cycle;
\fill[blue!63.4, opacity=0.7] (0.9480, 2.9580, 3.1159) -- (0.9940, 2.9580, 3.1194) -- (0.9940, 3.0120, 3.1132) -- (0.9480, 3.0120, 3.1096) -- cycle;
\fill[blue!63.2, opacity=0.7] (0.9480, 3.0120, 3.1096) -- (0.9940, 3.0120, 3.1132) -- (0.9940, 3.0660, 3.1069) -- (0.9480, 3.0660, 3.1034) -- cycle;
\fill[blue!53.9, opacity=0.7] (0.9480, 3.0660, 3.1034) -- (0.9940, 3.0660, 3.1069) -- (0.9940, 3.1200, 3.1006) -- (0.9480, 3.1200, 3.0971) -- cycle;
\fill[blue!58.3, opacity=0.7] (0.9940, -0.1200, 3.1006) -- (1.0400, -0.1200, 3.1039) -- (1.0400, -0.0660, 3.1102) -- (0.9940, -0.0660, 3.1069) -- cycle;
\fill[blue!40.8, opacity=0.7] (0.9940, -0.0660, 3.1069) -- (1.0400, -0.0660, 3.1102) -- (1.0400, -0.0120, 3.1165) -- (0.9940, -0.0120, 3.1132) -- cycle;
\fill[blue!24.3, opacity=0.7] (0.9940, -0.0120, 3.1132) -- (1.0400, -0.0120, 3.1165) -- (1.0400, 0.0420, 3.1227) -- (0.9940, 0.0420, 3.1194) -- cycle;
\fill[blue!19.7, opacity=0.7] (0.9940, 0.0420, 3.1194) -- (1.0400, 0.0420, 3.1227) -- (1.0400, 0.0960, 3.1289) -- (0.9940, 0.0960, 3.1256) -- cycle;
\fill[blue!22.7, opacity=0.7] (0.9940, 0.0960, 3.1256) -- (1.0400, 0.0960, 3.1289) -- (1.0400, 0.1500, 3.1350) -- (0.9940, 0.1500, 3.1317) -- cycle;
\fill[blue!39.5, opacity=0.7] (0.9940, 0.1500, 3.1317) -- (1.0400, 0.1500, 3.1350) -- (1.0400, 0.2040, 3.1410) -- (0.9940, 0.2040, 3.1377) -- cycle;
\fill[blue!62.3, opacity=0.7] (0.9940, 0.2040, 3.1377) -- (1.0400, 0.2040, 3.1410) -- (1.0400, 0.2580, 3.1469) -- (0.9940, 0.2580, 3.1436) -- cycle;
\fill[blue!58.3, opacity=0.7] (0.9940, 0.2580, 3.1436) -- (1.0400, 0.2580, 3.1469) -- (1.0400, 0.3120, 3.1527) -- (0.9940, 0.3120, 3.1494) -- cycle;
\fill[blue!53.5, opacity=0.7] (0.9940, 0.3120, 3.1494) -- (1.0400, 0.3120, 3.1527) -- (1.0400, 0.3660, 3.1584) -- (0.9940, 0.3660, 3.1551) -- cycle;
\fill[blue!61.0, opacity=0.7] (0.9940, 0.3660, 3.1551) -- (1.0400, 0.3660, 3.1584) -- (1.0400, 0.4200, 3.1639) -- (0.9940, 0.4200, 3.1606) -- cycle;
\fill[blue!60.6, opacity=0.7] (0.9940, 0.4200, 3.1606) -- (1.0400, 0.4200, 3.1639) -- (1.0400, 0.4740, 3.1693) -- (0.9940, 0.4740, 3.1660) -- cycle;
\fill[blue!44.9, opacity=0.7] (0.9940, 0.4740, 3.1660) -- (1.0400, 0.4740, 3.1693) -- (1.0400, 0.5280, 3.1745) -- (0.9940, 0.5280, 3.1712) -- cycle;
\fill[blue!38.0, opacity=0.7] (0.9940, 0.5280, 3.1712) -- (1.0400, 0.5280, 3.1745) -- (1.0400, 0.5820, 3.1794) -- (0.9940, 0.5820, 3.1762) -- cycle;
\fill[blue!46.8, opacity=0.7] (0.9940, 0.5820, 3.1762) -- (1.0400, 0.5820, 3.1794) -- (1.0400, 0.6360, 3.1842) -- (0.9940, 0.6360, 3.1809) -- cycle;
\fill[blue!62.8, opacity=0.7] (0.9940, 0.6360, 3.1809) -- (1.0400, 0.6360, 3.1842) -- (1.0400, 0.6900, 3.1888) -- (0.9940, 0.6900, 3.1855) -- cycle;
\fill[blue!53.1, opacity=0.7] (0.9940, 0.6900, 3.1855) -- (1.0400, 0.6900, 3.1888) -- (1.0400, 0.7440, 3.1931) -- (0.9940, 0.7440, 3.1898) -- cycle;
\fill[blue!34.0, opacity=0.7] (0.9940, 0.7440, 3.1898) -- (1.0400, 0.7440, 3.1931) -- (1.0400, 0.7980, 3.1972) -- (0.9940, 0.7980, 3.1939) -- cycle;
\fill[blue!28.5, opacity=0.7] (0.9940, 0.7980, 3.1939) -- (1.0400, 0.7980, 3.1972) -- (1.0400, 0.8520, 3.2010) -- (0.9940, 0.8520, 3.1977) -- cycle;
\fill[blue!34.6, opacity=0.7] (0.9940, 0.8520, 3.1977) -- (1.0400, 0.8520, 3.2010) -- (1.0400, 0.9060, 3.2046) -- (0.9940, 0.9060, 3.2013) -- cycle;
\fill[blue!51.0, opacity=0.7] (0.9940, 0.9060, 3.2013) -- (1.0400, 0.9060, 3.2046) -- (1.0400, 0.9600, 3.2078) -- (0.9940, 0.9600, 3.2046) -- cycle;
\fill[blue!63.0, opacity=0.7] (0.9940, 0.9600, 3.2046) -- (1.0400, 0.9600, 3.2078) -- (1.0400, 1.0140, 3.2108) -- (0.9940, 1.0140, 3.2076) -- cycle;
\fill[blue!61.8, opacity=0.7] (0.9940, 1.0140, 3.2076) -- (1.0400, 1.0140, 3.2108) -- (1.0400, 1.0680, 3.2135) -- (0.9940, 1.0680, 3.2103) -- cycle;
\fill[blue!60.9, opacity=0.7] (0.9940, 1.0680, 3.2103) -- (1.0400, 1.0680, 3.2135) -- (1.0400, 1.1220, 3.2160) -- (0.9940, 1.1220, 3.2127) -- cycle;
\fill[blue!63.5, opacity=0.7] (0.9940, 1.1220, 3.2127) -- (1.0400, 1.1220, 3.2160) -- (1.0400, 1.1760, 3.2180) -- (0.9940, 1.1760, 3.2148) -- cycle;
\fill[blue!58.5, opacity=0.7] (0.9940, 1.1760, 3.2148) -- (1.0400, 1.1760, 3.2180) -- (1.0400, 1.2300, 3.2198) -- (0.9940, 1.2300, 3.2166) -- cycle;
\fill[blue!42.5, opacity=0.7] (0.9940, 1.2300, 3.2166) -- (1.0400, 1.2300, 3.2198) -- (1.0400, 1.2840, 3.2213) -- (0.9940, 1.2840, 3.2180) -- cycle;
\fill[blue!28.6, opacity=0.7] (0.9940, 1.2840, 3.2180) -- (1.0400, 1.2840, 3.2213) -- (1.0400, 1.3380, 3.2224) -- (0.9940, 1.3380, 3.2192) -- cycle;
\fill[blue!21.8, opacity=0.7] (0.9940, 1.3380, 3.2192) -- (1.0400, 1.3380, 3.2224) -- (1.0400, 1.3920, 3.2233) -- (0.9940, 1.3920, 3.2200) -- cycle;
\fill[blue!19.3, opacity=0.7] (0.9940, 1.3920, 3.2200) -- (1.0400, 1.3920, 3.2233) -- (1.0400, 1.4460, 3.2238) -- (0.9940, 1.4460, 3.2205) -- cycle;
\fill[blue!18.5, opacity=0.7] (0.9940, 1.4460, 3.2205) -- (1.0400, 1.4460, 3.2238) -- (1.0400, 1.5000, 3.2239) -- (0.9940, 1.5000, 3.2206) -- cycle;
\fill[blue!18.5, opacity=0.7] (0.9940, 1.5000, 3.2206) -- (1.0400, 1.5000, 3.2239) -- (1.0400, 1.5540, 3.2238) -- (0.9940, 1.5540, 3.2205) -- cycle;
\fill[blue!18.7, opacity=0.7] (0.9940, 1.5540, 3.2205) -- (1.0400, 1.5540, 3.2238) -- (1.0400, 1.6080, 3.2233) -- (0.9940, 1.6080, 3.2200) -- cycle;
\fill[blue!19.0, opacity=0.7] (0.9940, 1.6080, 3.2200) -- (1.0400, 1.6080, 3.2233) -- (1.0400, 1.6620, 3.2224) -- (0.9940, 1.6620, 3.2192) -- cycle;
\fill[blue!19.2, opacity=0.7] (0.9940, 1.6620, 3.2192) -- (1.0400, 1.6620, 3.2224) -- (1.0400, 1.7160, 3.2213) -- (0.9940, 1.7160, 3.2180) -- cycle;
\fill[blue!19.5, opacity=0.7] (0.9940, 1.7160, 3.2180) -- (1.0400, 1.7160, 3.2213) -- (1.0400, 1.7700, 3.2198) -- (0.9940, 1.7700, 3.2166) -- cycle;
\fill[blue!20.0, opacity=0.7] (0.9940, 1.7700, 3.2166) -- (1.0400, 1.7700, 3.2198) -- (1.0400, 1.8240, 3.2180) -- (0.9940, 1.8240, 3.2148) -- cycle;
\fill[blue!21.2, opacity=0.7] (0.9940, 1.8240, 3.2148) -- (1.0400, 1.8240, 3.2180) -- (1.0400, 1.8780, 3.2160) -- (0.9940, 1.8780, 3.2127) -- cycle;
\fill[blue!24.3, opacity=0.7] (0.9940, 1.8780, 3.2127) -- (1.0400, 1.8780, 3.2160) -- (1.0400, 1.9320, 3.2135) -- (0.9940, 1.9320, 3.2103) -- cycle;
\fill[blue!31.2, opacity=0.7] (0.9940, 1.9320, 3.2103) -- (1.0400, 1.9320, 3.2135) -- (1.0400, 1.9860, 3.2108) -- (0.9940, 1.9860, 3.2076) -- cycle;
\fill[blue!44.2, opacity=0.7] (0.9940, 1.9860, 3.2076) -- (1.0400, 1.9860, 3.2108) -- (1.0400, 2.0400, 3.2078) -- (0.9940, 2.0400, 3.2046) -- cycle;
\fill[blue!59.1, opacity=0.7] (0.9940, 2.0400, 3.2046) -- (1.0400, 2.0400, 3.2078) -- (1.0400, 2.0940, 3.2046) -- (0.9940, 2.0940, 3.2013) -- cycle;
\fill[blue!63.1, opacity=0.7] (0.9940, 2.0940, 3.2013) -- (1.0400, 2.0940, 3.2046) -- (1.0400, 2.1480, 3.2010) -- (0.9940, 2.1480, 3.1977) -- cycle;
\fill[blue!55.5, opacity=0.7] (0.9940, 2.1480, 3.1977) -- (1.0400, 2.1480, 3.2010) -- (1.0400, 2.2020, 3.1972) -- (0.9940, 2.2020, 3.1939) -- cycle;
\fill[blue!49.2, opacity=0.7] (0.9940, 2.2020, 3.1939) -- (1.0400, 2.2020, 3.1972) -- (1.0400, 2.2560, 3.1931) -- (0.9940, 2.2560, 3.1898) -- cycle;
\fill[blue!51.3, opacity=0.7] (0.9940, 2.2560, 3.1898) -- (1.0400, 2.2560, 3.1931) -- (1.0400, 2.3100, 3.1888) -- (0.9940, 2.3100, 3.1855) -- cycle;
\fill[blue!60.0, opacity=0.7] (0.9940, 2.3100, 3.1855) -- (1.0400, 2.3100, 3.1888) -- (1.0400, 2.3640, 3.1842) -- (0.9940, 2.3640, 3.1809) -- cycle;
\fill[blue!62.9, opacity=0.7] (0.9940, 2.3640, 3.1809) -- (1.0400, 2.3640, 3.1842) -- (1.0400, 2.4180, 3.1794) -- (0.9940, 2.4180, 3.1762) -- cycle;
\fill[blue!53.2, opacity=0.7] (0.9940, 2.4180, 3.1762) -- (1.0400, 2.4180, 3.1794) -- (1.0400, 2.4720, 3.1745) -- (0.9940, 2.4720, 3.1712) -- cycle;
\fill[blue!45.1, opacity=0.7] (0.9940, 2.4720, 3.1712) -- (1.0400, 2.4720, 3.1745) -- (1.0400, 2.5260, 3.1693) -- (0.9940, 2.5260, 3.1660) -- cycle;
\fill[blue!48.4, opacity=0.7] (0.9940, 2.5260, 3.1660) -- (1.0400, 2.5260, 3.1693) -- (1.0400, 2.5800, 3.1639) -- (0.9940, 2.5800, 3.1606) -- cycle;
\fill[blue!60.9, opacity=0.7] (0.9940, 2.5800, 3.1606) -- (1.0400, 2.5800, 3.1639) -- (1.0400, 2.6340, 3.1584) -- (0.9940, 2.6340, 3.1551) -- cycle;
\fill[blue!58.5, opacity=0.7] (0.9940, 2.6340, 3.1551) -- (1.0400, 2.6340, 3.1584) -- (1.0400, 2.6880, 3.1527) -- (0.9940, 2.6880, 3.1494) -- cycle;
\fill[blue!36.5, opacity=0.7] (0.9940, 2.6880, 3.1494) -- (1.0400, 2.6880, 3.1527) -- (1.0400, 2.7420, 3.1469) -- (0.9940, 2.7420, 3.1436) -- cycle;
\fill[blue!24.4, opacity=0.7] (0.9940, 2.7420, 3.1436) -- (1.0400, 2.7420, 3.1469) -- (1.0400, 2.7960, 3.1410) -- (0.9940, 2.7960, 3.1377) -- cycle;
\fill[blue!23.3, opacity=0.7] (0.9940, 2.7960, 3.1377) -- (1.0400, 2.7960, 3.1410) -- (1.0400, 2.8500, 3.1350) -- (0.9940, 2.8500, 3.1317) -- cycle;
\fill[blue!31.1, opacity=0.7] (0.9940, 2.8500, 3.1317) -- (1.0400, 2.8500, 3.1350) -- (1.0400, 2.9040, 3.1289) -- (0.9940, 2.9040, 3.1256) -- cycle;
\fill[blue!49.2, opacity=0.7] (0.9940, 2.9040, 3.1256) -- (1.0400, 2.9040, 3.1289) -- (1.0400, 2.9580, 3.1227) -- (0.9940, 2.9580, 3.1194) -- cycle;
\fill[blue!62.0, opacity=0.7] (0.9940, 2.9580, 3.1194) -- (1.0400, 2.9580, 3.1227) -- (1.0400, 3.0120, 3.1165) -- (0.9940, 3.0120, 3.1132) -- cycle;
\fill[blue!63.5, opacity=0.7] (0.9940, 3.0120, 3.1132) -- (1.0400, 3.0120, 3.1165) -- (1.0400, 3.0660, 3.1102) -- (0.9940, 3.0660, 3.1069) -- cycle;
\fill[blue!60.4, opacity=0.7] (0.9940, 3.0660, 3.1069) -- (1.0400, 3.0660, 3.1102) -- (1.0400, 3.1200, 3.1039) -- (0.9940, 3.1200, 3.1006) -- cycle;
\fill[blue!55.5, opacity=0.7] (1.0400, -0.1200, 3.1039) -- (1.0860, -0.1200, 3.1069) -- (1.0860, -0.0660, 3.1132) -- (1.0400, -0.0660, 3.1102) -- cycle;
\fill[blue!35.7, opacity=0.7] (1.0400, -0.0660, 3.1102) -- (1.0860, -0.0660, 3.1132) -- (1.0860, -0.0120, 3.1195) -- (1.0400, -0.0120, 3.1165) -- cycle;
\fill[blue!22.3, opacity=0.7] (1.0400, -0.0120, 3.1165) -- (1.0860, -0.0120, 3.1195) -- (1.0860, 0.0420, 3.1257) -- (1.0400, 0.0420, 3.1227) -- cycle;
\fill[blue!19.9, opacity=0.7] (1.0400, 0.0420, 3.1227) -- (1.0860, 0.0420, 3.1257) -- (1.0860, 0.0960, 3.1319) -- (1.0400, 0.0960, 3.1289) -- cycle;
\fill[blue!25.8, opacity=0.7] (1.0400, 0.0960, 3.1289) -- (1.0860, 0.0960, 3.1319) -- (1.0860, 0.1500, 3.1380) -- (1.0400, 0.1500, 3.1350) -- cycle;
\fill[blue!48.0, opacity=0.7] (1.0400, 0.1500, 3.1350) -- (1.0860, 0.1500, 3.1380) -- (1.0860, 0.2040, 3.1440) -- (1.0400, 0.2040, 3.1410) -- cycle;
\fill[blue!63.5, opacity=0.7] (1.0400, 0.2040, 3.1410) -- (1.0860, 0.2040, 3.1440) -- (1.0860, 0.2580, 3.1499) -- (1.0400, 0.2580, 3.1469) -- cycle;
\fill[blue!55.0, opacity=0.7] (1.0400, 0.2580, 3.1469) -- (1.0860, 0.2580, 3.1499) -- (1.0860, 0.3120, 3.1557) -- (1.0400, 0.3120, 3.1527) -- cycle;
\fill[blue!54.8, opacity=0.7] (1.0400, 0.3120, 3.1527) -- (1.0860, 0.3120, 3.1557) -- (1.0860, 0.3660, 3.1614) -- (1.0400, 0.3660, 3.1584) -- cycle;
\fill[blue!63.2, opacity=0.7] (1.0400, 0.3660, 3.1584) -- (1.0860, 0.3660, 3.1614) -- (1.0860, 0.4200, 3.1669) -- (1.0400, 0.4200, 3.1639) -- cycle;
\fill[blue!55.6, opacity=0.7] (1.0400, 0.4200, 3.1639) -- (1.0860, 0.4200, 3.1669) -- (1.0860, 0.4740, 3.1723) -- (1.0400, 0.4740, 3.1693) -- cycle;
\fill[blue!40.9, opacity=0.7] (1.0400, 0.4740, 3.1693) -- (1.0860, 0.4740, 3.1723) -- (1.0860, 0.5280, 3.1775) -- (1.0400, 0.5280, 3.1745) -- cycle;
\fill[blue!40.2, opacity=0.7] (1.0400, 0.5280, 3.1745) -- (1.0860, 0.5280, 3.1775) -- (1.0860, 0.5820, 3.1824) -- (1.0400, 0.5820, 3.1794) -- cycle;
\fill[blue!55.1, opacity=0.7] (1.0400, 0.5820, 3.1794) -- (1.0860, 0.5820, 3.1824) -- (1.0860, 0.6360, 3.1872) -- (1.0400, 0.6360, 3.1842) -- cycle;
\fill[blue!62.1, opacity=0.7] (1.0400, 0.6360, 3.1842) -- (1.0860, 0.6360, 3.1872) -- (1.0860, 0.6900, 3.1918) -- (1.0400, 0.6900, 3.1888) -- cycle;
\fill[blue!41.6, opacity=0.7] (1.0400, 0.6900, 3.1888) -- (1.0860, 0.6900, 3.1918) -- (1.0860, 0.7440, 3.1961) -- (1.0400, 0.7440, 3.1931) -- cycle;
\fill[blue!29.0, opacity=0.7] (1.0400, 0.7440, 3.1931) -- (1.0860, 0.7440, 3.1961) -- (1.0860, 0.7980, 3.2002) -- (1.0400, 0.7980, 3.1972) -- cycle;
\fill[blue!30.1, opacity=0.7] (1.0400, 0.7980, 3.1972) -- (1.0860, 0.7980, 3.2002) -- (1.0860, 0.8520, 3.2040) -- (1.0400, 0.8520, 3.2010) -- cycle;
\fill[blue!43.3, opacity=0.7] (1.0400, 0.8520, 3.2010) -- (1.0860, 0.8520, 3.2040) -- (1.0860, 0.9060, 3.2076) -- (1.0400, 0.9060, 3.2046) -- cycle;
\fill[blue!60.4, opacity=0.7] (1.0400, 0.9060, 3.2046) -- (1.0860, 0.9060, 3.2076) -- (1.0860, 0.9600, 3.2108) -- (1.0400, 0.9600, 3.2078) -- cycle;
\fill[blue!62.9, opacity=0.7] (1.0400, 0.9600, 3.2078) -- (1.0860, 0.9600, 3.2108) -- (1.0860, 1.0140, 3.2138) -- (1.0400, 1.0140, 3.2108) -- cycle;
\fill[blue!61.1, opacity=0.7] (1.0400, 1.0140, 3.2108) -- (1.0860, 1.0140, 3.2138) -- (1.0860, 1.0680, 3.2165) -- (1.0400, 1.0680, 3.2135) -- cycle;
\fill[blue!63.5, opacity=0.7] (1.0400, 1.0680, 3.2135) -- (1.0860, 1.0680, 3.2165) -- (1.0860, 1.1220, 3.2190) -- (1.0400, 1.1220, 3.2160) -- cycle;
\fill[blue!57.2, opacity=0.7] (1.0400, 1.1220, 3.2160) -- (1.0860, 1.1220, 3.2190) -- (1.0860, 1.1760, 3.2210) -- (1.0400, 1.1760, 3.2180) -- cycle;
\fill[blue!37.8, opacity=0.7] (1.0400, 1.1760, 3.2180) -- (1.0860, 1.1760, 3.2210) -- (1.0860, 1.2300, 3.2228) -- (1.0400, 1.2300, 3.2198) -- cycle;
\fill[blue!24.0, opacity=0.7] (1.0400, 1.2300, 3.2198) -- (1.0860, 1.2300, 3.2228) -- (1.0860, 1.2840, 3.2243) -- (1.0400, 1.2840, 3.2213) -- cycle;
\fill[blue!19.1, opacity=0.7] (1.0400, 1.2840, 3.2213) -- (1.0860, 1.2840, 3.2243) -- (1.0860, 1.3380, 3.2254) -- (1.0400, 1.3380, 3.2224) -- cycle;
\fill[blue!17.9, opacity=0.7] (1.0400, 1.3380, 3.2224) -- (1.0860, 1.3380, 3.2254) -- (1.0860, 1.3920, 3.2263) -- (1.0400, 1.3920, 3.2233) -- cycle;
\fill[blue!18.1, opacity=0.7] (1.0400, 1.3920, 3.2233) -- (1.0860, 1.3920, 3.2263) -- (1.0860, 1.4460, 3.2268) -- (1.0400, 1.4460, 3.2238) -- cycle;
\fill[blue!19.3, opacity=0.7] (1.0400, 1.4460, 3.2238) -- (1.0860, 1.4460, 3.2268) -- (1.0860, 1.5000, 3.2269) -- (1.0400, 1.5000, 3.2239) -- cycle;
\fill[blue!21.1, opacity=0.7] (1.0400, 1.5000, 3.2239) -- (1.0860, 1.5000, 3.2269) -- (1.0860, 1.5540, 3.2268) -- (1.0400, 1.5540, 3.2238) -- cycle;
\fill[blue!22.9, opacity=0.7] (1.0400, 1.5540, 3.2238) -- (1.0860, 1.5540, 3.2268) -- (1.0860, 1.6080, 3.2263) -- (1.0400, 1.6080, 3.2233) -- cycle;
\fill[blue!24.0, opacity=0.7] (1.0400, 1.6080, 3.2233) -- (1.0860, 1.6080, 3.2263) -- (1.0860, 1.6620, 3.2254) -- (1.0400, 1.6620, 3.2224) -- cycle;
\fill[blue!24.0, opacity=0.7] (1.0400, 1.6620, 3.2224) -- (1.0860, 1.6620, 3.2254) -- (1.0860, 1.7160, 3.2243) -- (1.0400, 1.7160, 3.2213) -- cycle;
\fill[blue!23.1, opacity=0.7] (1.0400, 1.7160, 3.2213) -- (1.0860, 1.7160, 3.2243) -- (1.0860, 1.7700, 3.2228) -- (1.0400, 1.7700, 3.2198) -- cycle;
\fill[blue!21.8, opacity=0.7] (1.0400, 1.7700, 3.2198) -- (1.0860, 1.7700, 3.2228) -- (1.0860, 1.8240, 3.2210) -- (1.0400, 1.8240, 3.2180) -- cycle;
\fill[blue!20.6, opacity=0.7] (1.0400, 1.8240, 3.2180) -- (1.0860, 1.8240, 3.2210) -- (1.0860, 1.8780, 3.2190) -- (1.0400, 1.8780, 3.2160) -- cycle;
\fill[blue!20.3, opacity=0.7] (1.0400, 1.8780, 3.2160) -- (1.0860, 1.8780, 3.2190) -- (1.0860, 1.9320, 3.2165) -- (1.0400, 1.9320, 3.2135) -- cycle;
\fill[blue!21.6, opacity=0.7] (1.0400, 1.9320, 3.2135) -- (1.0860, 1.9320, 3.2165) -- (1.0860, 1.9860, 3.2138) -- (1.0400, 1.9860, 3.2108) -- cycle;
\fill[blue!26.2, opacity=0.7] (1.0400, 1.9860, 3.2108) -- (1.0860, 1.9860, 3.2138) -- (1.0860, 2.0400, 3.2108) -- (1.0400, 2.0400, 3.2078) -- cycle;
\fill[blue!37.9, opacity=0.7] (1.0400, 2.0400, 3.2078) -- (1.0860, 2.0400, 3.2108) -- (1.0860, 2.0940, 3.2076) -- (1.0400, 2.0940, 3.2046) -- cycle;
\fill[blue!55.6, opacity=0.7] (1.0400, 2.0940, 3.2046) -- (1.0860, 2.0940, 3.2076) -- (1.0860, 2.1480, 3.2040) -- (1.0400, 2.1480, 3.2010) -- cycle;
\fill[blue!63.5, opacity=0.7] (1.0400, 2.1480, 3.2010) -- (1.0860, 2.1480, 3.2040) -- (1.0860, 2.2020, 3.2002) -- (1.0400, 2.2020, 3.1972) -- cycle;
\fill[blue!55.4, opacity=0.7] (1.0400, 2.2020, 3.1972) -- (1.0860, 2.2020, 3.2002) -- (1.0860, 2.2560, 3.1961) -- (1.0400, 2.2560, 3.1931) -- cycle;
\fill[blue!48.4, opacity=0.7] (1.0400, 2.2560, 3.1931) -- (1.0860, 2.2560, 3.1961) -- (1.0860, 2.3100, 3.1918) -- (1.0400, 2.3100, 3.1888) -- cycle;
\fill[blue!51.4, opacity=0.7] (1.0400, 2.3100, 3.1888) -- (1.0860, 2.3100, 3.1918) -- (1.0860, 2.3640, 3.1872) -- (1.0400, 2.3640, 3.1842) -- cycle;
\fill[blue!61.0, opacity=0.7] (1.0400, 2.3640, 3.1842) -- (1.0860, 2.3640, 3.1872) -- (1.0860, 2.4180, 3.1824) -- (1.0400, 2.4180, 3.1794) -- cycle;
\fill[blue!61.9, opacity=0.7] (1.0400, 2.4180, 3.1794) -- (1.0860, 2.4180, 3.1824) -- (1.0860, 2.4720, 3.1775) -- (1.0400, 2.4720, 3.1745) -- cycle;
\fill[blue!50.8, opacity=0.7] (1.0400, 2.4720, 3.1745) -- (1.0860, 2.4720, 3.1775) -- (1.0860, 2.5260, 3.1723) -- (1.0400, 2.5260, 3.1693) -- cycle;
\fill[blue!45.3, opacity=0.7] (1.0400, 2.5260, 3.1693) -- (1.0860, 2.5260, 3.1723) -- (1.0860, 2.5800, 3.1669) -- (1.0400, 2.5800, 3.1639) -- cycle;
\fill[blue!52.8, opacity=0.7] (1.0400, 2.5800, 3.1639) -- (1.0860, 2.5800, 3.1669) -- (1.0860, 2.6340, 3.1614) -- (1.0400, 2.6340, 3.1584) -- cycle;
\fill[blue!63.5, opacity=0.7] (1.0400, 2.6340, 3.1584) -- (1.0860, 2.6340, 3.1614) -- (1.0860, 2.6880, 3.1557) -- (1.0400, 2.6880, 3.1527) -- cycle;
\fill[blue!49.4, opacity=0.7] (1.0400, 2.6880, 3.1527) -- (1.0860, 2.6880, 3.1557) -- (1.0860, 2.7420, 3.1499) -- (1.0400, 2.7420, 3.1469) -- cycle;
\fill[blue!29.1, opacity=0.7] (1.0400, 2.7420, 3.1469) -- (1.0860, 2.7420, 3.1499) -- (1.0860, 2.7960, 3.1440) -- (1.0400, 2.7960, 3.1410) -- cycle;
\fill[blue!22.7, opacity=0.7] (1.0400, 2.7960, 3.1410) -- (1.0860, 2.7960, 3.1440) -- (1.0860, 2.8500, 3.1380) -- (1.0400, 2.8500, 3.1350) -- cycle;
\fill[blue!25.5, opacity=0.7] (1.0400, 2.8500, 3.1350) -- (1.0860, 2.8500, 3.1380) -- (1.0860, 2.9040, 3.1319) -- (1.0400, 2.9040, 3.1289) -- cycle;
\fill[blue!39.1, opacity=0.7] (1.0400, 2.9040, 3.1289) -- (1.0860, 2.9040, 3.1319) -- (1.0860, 2.9580, 3.1257) -- (1.0400, 2.9580, 3.1227) -- cycle;
\fill[blue!57.6, opacity=0.7] (1.0400, 2.9580, 3.1227) -- (1.0860, 2.9580, 3.1257) -- (1.0860, 3.0120, 3.1195) -- (1.0400, 3.0120, 3.1165) -- cycle;
\fill[blue!63.4, opacity=0.7] (1.0400, 3.0120, 3.1165) -- (1.0860, 3.0120, 3.1195) -- (1.0860, 3.0660, 3.1132) -- (1.0400, 3.0660, 3.1102) -- cycle;
\fill[blue!62.9, opacity=0.7] (1.0400, 3.0660, 3.1102) -- (1.0860, 3.0660, 3.1132) -- (1.0860, 3.1200, 3.1069) -- (1.0400, 3.1200, 3.1039) -- cycle;
\fill[blue!52.4, opacity=0.7] (1.0860, -0.1200, 3.1069) -- (1.1320, -0.1200, 3.1096) -- (1.1320, -0.0660, 3.1159) -- (1.0860, -0.0660, 3.1132) -- cycle;
\fill[blue!31.8, opacity=0.7] (1.0860, -0.0660, 3.1132) -- (1.1320, -0.0660, 3.1159) -- (1.1320, -0.0120, 3.1222) -- (1.0860, -0.0120, 3.1195) -- cycle;
\fill[blue!21.2, opacity=0.7] (1.0860, -0.0120, 3.1195) -- (1.1320, -0.0120, 3.1222) -- (1.1320, 0.0420, 3.1284) -- (1.0860, 0.0420, 3.1257) -- cycle;
\fill[blue!20.5, opacity=0.7] (1.0860, 0.0420, 3.1257) -- (1.1320, 0.0420, 3.1284) -- (1.1320, 0.0960, 3.1346) -- (1.0860, 0.0960, 3.1319) -- cycle;
\fill[blue!29.9, opacity=0.7] (1.0860, 0.0960, 3.1319) -- (1.1320, 0.0960, 3.1346) -- (1.1320, 0.1500, 3.1407) -- (1.0860, 0.1500, 3.1380) -- cycle;
\fill[blue!55.1, opacity=0.7] (1.0860, 0.1500, 3.1380) -- (1.1320, 0.1500, 3.1407) -- (1.1320, 0.2040, 3.1467) -- (1.0860, 0.2040, 3.1440) -- cycle;
\fill[blue!62.0, opacity=0.7] (1.0860, 0.2040, 3.1440) -- (1.1320, 0.2040, 3.1467) -- (1.1320, 0.2580, 3.1526) -- (1.0860, 0.2580, 3.1499) -- cycle;
\fill[blue!53.0, opacity=0.7] (1.0860, 0.2580, 3.1499) -- (1.1320, 0.2580, 3.1526) -- (1.1320, 0.3120, 3.1584) -- (1.0860, 0.3120, 3.1557) -- cycle;
\fill[blue!57.0, opacity=0.7] (1.0860, 0.3120, 3.1557) -- (1.1320, 0.3120, 3.1584) -- (1.1320, 0.3660, 3.1641) -- (1.0860, 0.3660, 3.1614) -- cycle;
\fill[blue!63.4, opacity=0.7] (1.0860, 0.3660, 3.1614) -- (1.1320, 0.3660, 3.1641) -- (1.1320, 0.4200, 3.1696) -- (1.0860, 0.4200, 3.1669) -- cycle;
\fill[blue!50.5, opacity=0.7] (1.0860, 0.4200, 3.1669) -- (1.1320, 0.4200, 3.1696) -- (1.1320, 0.4740, 3.1750) -- (1.0860, 0.4740, 3.1723) -- cycle;
\fill[blue!39.3, opacity=0.7] (1.0860, 0.4740, 3.1723) -- (1.1320, 0.4740, 3.1750) -- (1.1320, 0.5280, 3.1802) -- (1.0860, 0.5280, 3.1775) -- cycle;
\fill[blue!44.6, opacity=0.7] (1.0860, 0.5280, 3.1775) -- (1.1320, 0.5280, 3.1802) -- (1.1320, 0.5820, 3.1851) -- (1.0860, 0.5820, 3.1824) -- cycle;
\fill[blue!61.5, opacity=0.7] (1.0860, 0.5820, 3.1824) -- (1.1320, 0.5820, 3.1851) -- (1.1320, 0.6360, 3.1899) -- (1.0860, 0.6360, 3.1872) -- cycle;
\fill[blue!54.7, opacity=0.7] (1.0860, 0.6360, 3.1872) -- (1.1320, 0.6360, 3.1899) -- (1.1320, 0.6900, 3.1945) -- (1.0860, 0.6900, 3.1918) -- cycle;
\fill[blue!33.7, opacity=0.7] (1.0860, 0.6900, 3.1918) -- (1.1320, 0.6900, 3.1945) -- (1.1320, 0.7440, 3.1988) -- (1.0860, 0.7440, 3.1961) -- cycle;
\fill[blue!27.6, opacity=0.7] (1.0860, 0.7440, 3.1961) -- (1.1320, 0.7440, 3.1988) -- (1.1320, 0.7980, 3.2029) -- (1.0860, 0.7980, 3.2002) -- cycle;
\fill[blue!34.5, opacity=0.7] (1.0860, 0.7980, 3.2002) -- (1.1320, 0.7980, 3.2029) -- (1.1320, 0.8520, 3.2067) -- (1.0860, 0.8520, 3.2040) -- cycle;
\fill[blue!52.9, opacity=0.7] (1.0860, 0.8520, 3.2040) -- (1.1320, 0.8520, 3.2067) -- (1.1320, 0.9060, 3.2103) -- (1.0860, 0.9060, 3.2076) -- cycle;
\fill[blue!63.5, opacity=0.7] (1.0860, 0.9060, 3.2076) -- (1.1320, 0.9060, 3.2103) -- (1.1320, 0.9600, 3.2135) -- (1.0860, 0.9600, 3.2108) -- cycle;
\fill[blue!61.6, opacity=0.7] (1.0860, 0.9600, 3.2108) -- (1.1320, 0.9600, 3.2135) -- (1.1320, 1.0140, 3.2165) -- (1.0860, 1.0140, 3.2138) -- cycle;
\fill[blue!63.1, opacity=0.7] (1.0860, 1.0140, 3.2138) -- (1.1320, 1.0140, 3.2165) -- (1.1320, 1.0680, 3.2193) -- (1.0860, 1.0680, 3.2165) -- cycle;
\fill[blue!59.2, opacity=0.7] (1.0860, 1.0680, 3.2165) -- (1.1320, 1.0680, 3.2193) -- (1.1320, 1.1220, 3.2217) -- (1.0860, 1.1220, 3.2190) -- cycle;
\fill[blue!38.5, opacity=0.7] (1.0860, 1.1220, 3.2190) -- (1.1320, 1.1220, 3.2217) -- (1.1320, 1.1760, 3.2238) -- (1.0860, 1.1760, 3.2210) -- cycle;
\fill[blue!22.9, opacity=0.7] (1.0860, 1.1760, 3.2210) -- (1.1320, 1.1760, 3.2238) -- (1.1320, 1.2300, 3.2255) -- (1.0860, 1.2300, 3.2228) -- cycle;
\fill[blue!18.2, opacity=0.7] (1.0860, 1.2300, 3.2228) -- (1.1320, 1.2300, 3.2255) -- (1.1320, 1.2840, 3.2270) -- (1.0860, 1.2840, 3.2243) -- cycle;
\fill[blue!17.5, opacity=0.7] (1.0860, 1.2840, 3.2243) -- (1.1320, 1.2840, 3.2270) -- (1.1320, 1.3380, 3.2281) -- (1.0860, 1.3380, 3.2254) -- cycle;
\fill[blue!18.8, opacity=0.7] (1.0860, 1.3380, 3.2254) -- (1.1320, 1.3380, 3.2281) -- (1.1320, 1.3920, 3.2290) -- (1.0860, 1.3920, 3.2263) -- cycle;
\fill[blue!22.3, opacity=0.7] (1.0860, 1.3920, 3.2263) -- (1.1320, 1.3920, 3.2290) -- (1.1320, 1.4460, 3.2295) -- (1.0860, 1.4460, 3.2268) -- cycle;
\fill[blue!28.3, opacity=0.7] (1.0860, 1.4460, 3.2268) -- (1.1320, 1.4460, 3.2295) -- (1.1320, 1.5000, 3.2296) -- (1.0860, 1.5000, 3.2269) -- cycle;
\fill[blue!34.9, opacity=0.7] (1.0860, 1.5000, 3.2269) -- (1.1320, 1.5000, 3.2296) -- (1.1320, 1.5540, 3.2295) -- (1.0860, 1.5540, 3.2268) -- cycle;
\fill[blue!40.0, opacity=0.7] (1.0860, 1.5540, 3.2268) -- (1.1320, 1.5540, 3.2295) -- (1.1320, 1.6080, 3.2290) -- (1.0860, 1.6080, 3.2263) -- cycle;
\fill[blue!42.5, opacity=0.7] (1.0860, 1.6080, 3.2263) -- (1.1320, 1.6080, 3.2290) -- (1.1320, 1.6620, 3.2281) -- (1.0860, 1.6620, 3.2254) -- cycle;
\fill[blue!42.3, opacity=0.7] (1.0860, 1.6620, 3.2254) -- (1.1320, 1.6620, 3.2281) -- (1.1320, 1.7160, 3.2270) -- (1.0860, 1.7160, 3.2243) -- cycle;
\fill[blue!39.4, opacity=0.7] (1.0860, 1.7160, 3.2243) -- (1.1320, 1.7160, 3.2270) -- (1.1320, 1.7700, 3.2255) -- (1.0860, 1.7700, 3.2228) -- cycle;
\fill[blue!34.4, opacity=0.7] (1.0860, 1.7700, 3.2228) -- (1.1320, 1.7700, 3.2255) -- (1.1320, 1.8240, 3.2238) -- (1.0860, 1.8240, 3.2210) -- cycle;
\fill[blue!28.6, opacity=0.7] (1.0860, 1.8240, 3.2210) -- (1.1320, 1.8240, 3.2238) -- (1.1320, 1.8780, 3.2217) -- (1.0860, 1.8780, 3.2190) -- cycle;
\fill[blue!23.9, opacity=0.7] (1.0860, 1.8780, 3.2190) -- (1.1320, 1.8780, 3.2217) -- (1.1320, 1.9320, 3.2193) -- (1.0860, 1.9320, 3.2165) -- cycle;
\fill[blue!21.3, opacity=0.7] (1.0860, 1.9320, 3.2165) -- (1.1320, 1.9320, 3.2193) -- (1.1320, 1.9860, 3.2165) -- (1.0860, 1.9860, 3.2138) -- cycle;
\fill[blue!21.2, opacity=0.7] (1.0860, 1.9860, 3.2138) -- (1.1320, 1.9860, 3.2165) -- (1.1320, 2.0400, 3.2135) -- (1.0860, 2.0400, 3.2108) -- cycle;
\fill[blue!24.6, opacity=0.7] (1.0860, 2.0400, 3.2108) -- (1.1320, 2.0400, 3.2135) -- (1.1320, 2.0940, 3.2103) -- (1.0860, 2.0940, 3.2076) -- cycle;
\fill[blue!35.7, opacity=0.7] (1.0860, 2.0940, 3.2076) -- (1.1320, 2.0940, 3.2103) -- (1.1320, 2.1480, 3.2067) -- (1.0860, 2.1480, 3.2040) -- cycle;
\fill[blue!55.2, opacity=0.7] (1.0860, 2.1480, 3.2040) -- (1.1320, 2.1480, 3.2067) -- (1.1320, 2.2020, 3.2029) -- (1.0860, 2.2020, 3.2002) -- cycle;
\fill[blue!63.3, opacity=0.7] (1.0860, 2.2020, 3.2002) -- (1.1320, 2.2020, 3.2029) -- (1.1320, 2.2560, 3.1988) -- (1.0860, 2.2560, 3.1961) -- cycle;
\fill[blue!53.7, opacity=0.7] (1.0860, 2.2560, 3.1961) -- (1.1320, 2.2560, 3.1988) -- (1.1320, 2.3100, 3.1945) -- (1.0860, 2.3100, 3.1918) -- cycle;
\fill[blue!47.5, opacity=0.7] (1.0860, 2.3100, 3.1918) -- (1.1320, 2.3100, 3.1945) -- (1.1320, 2.3640, 3.1899) -- (1.0860, 2.3640, 3.1872) -- cycle;
\fill[blue!52.9, opacity=0.7] (1.0860, 2.3640, 3.1872) -- (1.1320, 2.3640, 3.1899) -- (1.1320, 2.4180, 3.1851) -- (1.0860, 2.4180, 3.1824) -- cycle;
\fill[blue!62.8, opacity=0.7] (1.0860, 2.4180, 3.1824) -- (1.1320, 2.4180, 3.1851) -- (1.1320, 2.4720, 3.1802) -- (1.0860, 2.4720, 3.1775) -- cycle;
\fill[blue!59.1, opacity=0.7] (1.0860, 2.4720, 3.1775) -- (1.1320, 2.4720, 3.1802) -- (1.1320, 2.5260, 3.1750) -- (1.0860, 2.5260, 3.1723) -- cycle;
\fill[blue!47.9, opacity=0.7] (1.0860, 2.5260, 3.1723) -- (1.1320, 2.5260, 3.1750) -- (1.1320, 2.5800, 3.1696) -- (1.0860, 2.5800, 3.1669) -- cycle;
\fill[blue!47.2, opacity=0.7] (1.0860, 2.5800, 3.1669) -- (1.1320, 2.5800, 3.1696) -- (1.1320, 2.6340, 3.1641) -- (1.0860, 2.6340, 3.1614) -- cycle;
\fill[blue!59.1, opacity=0.7] (1.0860, 2.6340, 3.1614) -- (1.1320, 2.6340, 3.1641) -- (1.1320, 2.6880, 3.1584) -- (1.0860, 2.6880, 3.1557) -- cycle;
\fill[blue!60.3, opacity=0.7] (1.0860, 2.6880, 3.1557) -- (1.1320, 2.6880, 3.1584) -- (1.1320, 2.7420, 3.1526) -- (1.0860, 2.7420, 3.1499) -- cycle;
\fill[blue!37.5, opacity=0.7] (1.0860, 2.7420, 3.1499) -- (1.1320, 2.7420, 3.1526) -- (1.1320, 2.7960, 3.1467) -- (1.0860, 2.7960, 3.1440) -- cycle;
\fill[blue!24.2, opacity=0.7] (1.0860, 2.7960, 3.1440) -- (1.1320, 2.7960, 3.1467) -- (1.1320, 2.8500, 3.1407) -- (1.0860, 2.8500, 3.1380) -- cycle;
\fill[blue!22.9, opacity=0.7] (1.0860, 2.8500, 3.1380) -- (1.1320, 2.8500, 3.1407) -- (1.1320, 2.9040, 3.1346) -- (1.0860, 2.9040, 3.1319) -- cycle;
\fill[blue!31.1, opacity=0.7] (1.0860, 2.9040, 3.1319) -- (1.1320, 2.9040, 3.1346) -- (1.1320, 2.9580, 3.1284) -- (1.0860, 2.9580, 3.1257) -- cycle;
\fill[blue!50.1, opacity=0.7] (1.0860, 2.9580, 3.1257) -- (1.1320, 2.9580, 3.1284) -- (1.1320, 3.0120, 3.1222) -- (1.0860, 3.0120, 3.1195) -- cycle;
\fill[blue!62.3, opacity=0.7] (1.0860, 3.0120, 3.1195) -- (1.1320, 3.0120, 3.1222) -- (1.1320, 3.0660, 3.1159) -- (1.0860, 3.0660, 3.1132) -- cycle;
\fill[blue!63.4, opacity=0.7] (1.0860, 3.0660, 3.1132) -- (1.1320, 3.0660, 3.1159) -- (1.1320, 3.1200, 3.1096) -- (1.0860, 3.1200, 3.1069) -- cycle;
\fill[blue!49.2, opacity=0.7] (1.1320, -0.1200, 3.1096) -- (1.1780, -0.1200, 3.1120) -- (1.1780, -0.0660, 3.1183) -- (1.1320, -0.0660, 3.1159) -- cycle;
\fill[blue!29.0, opacity=0.7] (1.1320, -0.0660, 3.1159) -- (1.1780, -0.0660, 3.1183) -- (1.1780, -0.0120, 3.1246) -- (1.1320, -0.0120, 3.1222) -- cycle;
\fill[blue!20.6, opacity=0.7] (1.1320, -0.0120, 3.1222) -- (1.1780, -0.0120, 3.1246) -- (1.1780, 0.0420, 3.1308) -- (1.1320, 0.0420, 3.1284) -- cycle;
\fill[blue!21.5, opacity=0.7] (1.1320, 0.0420, 3.1284) -- (1.1780, 0.0420, 3.1308) -- (1.1780, 0.0960, 3.1370) -- (1.1320, 0.0960, 3.1346) -- cycle;
\fill[blue!34.4, opacity=0.7] (1.1320, 0.0960, 3.1346) -- (1.1780, 0.0960, 3.1370) -- (1.1780, 0.1500, 3.1431) -- (1.1320, 0.1500, 3.1407) -- cycle;
\fill[blue!59.9, opacity=0.7] (1.1320, 0.1500, 3.1407) -- (1.1780, 0.1500, 3.1431) -- (1.1780, 0.2040, 3.1491) -- (1.1320, 0.2040, 3.1467) -- cycle;
\fill[blue!59.5, opacity=0.7] (1.1320, 0.2040, 3.1467) -- (1.1780, 0.2040, 3.1491) -- (1.1780, 0.2580, 3.1550) -- (1.1320, 0.2580, 3.1526) -- cycle;
\fill[blue!52.2, opacity=0.7] (1.1320, 0.2580, 3.1526) -- (1.1780, 0.2580, 3.1550) -- (1.1780, 0.3120, 3.1608) -- (1.1320, 0.3120, 3.1584) -- cycle;
\fill[blue!59.3, opacity=0.7] (1.1320, 0.3120, 3.1584) -- (1.1780, 0.3120, 3.1608) -- (1.1780, 0.3660, 3.1665) -- (1.1320, 0.3660, 3.1641) -- cycle;
\fill[blue!62.0, opacity=0.7] (1.1320, 0.3660, 3.1641) -- (1.1780, 0.3660, 3.1665) -- (1.1780, 0.4200, 3.1720) -- (1.1320, 0.4200, 3.1696) -- cycle;
\fill[blue!46.6, opacity=0.7] (1.1320, 0.4200, 3.1696) -- (1.1780, 0.4200, 3.1720) -- (1.1780, 0.4740, 3.1774) -- (1.1320, 0.4740, 3.1750) -- cycle;
\fill[blue!39.5, opacity=0.7] (1.1320, 0.4740, 3.1750) -- (1.1780, 0.4740, 3.1774) -- (1.1780, 0.5280, 3.1826) -- (1.1320, 0.5280, 3.1802) -- cycle;
\fill[blue!49.9, opacity=0.7] (1.1320, 0.5280, 3.1802) -- (1.1780, 0.5280, 3.1826) -- (1.1780, 0.5820, 3.1875) -- (1.1320, 0.5820, 3.1851) -- cycle;
\fill[blue!63.6, opacity=0.7] (1.1320, 0.5820, 3.1851) -- (1.1780, 0.5820, 3.1875) -- (1.1780, 0.6360, 3.1923) -- (1.1320, 0.6360, 3.1899) -- cycle;
\fill[blue!46.0, opacity=0.7] (1.1320, 0.6360, 3.1899) -- (1.1780, 0.6360, 3.1923) -- (1.1780, 0.6900, 3.1969) -- (1.1320, 0.6900, 3.1945) -- cycle;
\fill[blue!29.4, opacity=0.7] (1.1320, 0.6900, 3.1945) -- (1.1780, 0.6900, 3.1969) -- (1.1780, 0.7440, 3.2012) -- (1.1320, 0.7440, 3.1988) -- cycle;
\fill[blue!28.3, opacity=0.7] (1.1320, 0.7440, 3.1988) -- (1.1780, 0.7440, 3.2012) -- (1.1780, 0.7980, 3.2053) -- (1.1320, 0.7980, 3.2029) -- cycle;
\fill[blue!40.7, opacity=0.7] (1.1320, 0.7980, 3.2029) -- (1.1780, 0.7980, 3.2053) -- (1.1780, 0.8520, 3.2091) -- (1.1320, 0.8520, 3.2067) -- cycle;
\fill[blue!59.7, opacity=0.7] (1.1320, 0.8520, 3.2067) -- (1.1780, 0.8520, 3.2091) -- (1.1780, 0.9060, 3.2127) -- (1.1320, 0.9060, 3.2103) -- cycle;
\fill[blue!63.0, opacity=0.7] (1.1320, 0.9060, 3.2103) -- (1.1780, 0.9060, 3.2127) -- (1.1780, 0.9600, 3.2160) -- (1.1320, 0.9600, 3.2135) -- cycle;
\fill[blue!62.2, opacity=0.7] (1.1320, 0.9600, 3.2135) -- (1.1780, 0.9600, 3.2160) -- (1.1780, 1.0140, 3.2190) -- (1.1320, 1.0140, 3.2165) -- cycle;
\fill[blue!62.6, opacity=0.7] (1.1320, 1.0140, 3.2165) -- (1.1780, 1.0140, 3.2190) -- (1.1780, 1.0680, 3.2217) -- (1.1320, 1.0680, 3.2193) -- cycle;
\fill[blue!44.8, opacity=0.7] (1.1320, 1.0680, 3.2193) -- (1.1780, 1.0680, 3.2217) -- (1.1780, 1.1220, 3.2241) -- (1.1320, 1.1220, 3.2217) -- cycle;
\fill[blue!24.5, opacity=0.7] (1.1320, 1.1220, 3.2217) -- (1.1780, 1.1220, 3.2241) -- (1.1780, 1.1760, 3.2262) -- (1.1320, 1.1760, 3.2238) -- cycle;
\fill[blue!18.1, opacity=0.7] (1.1320, 1.1760, 3.2238) -- (1.1780, 1.1760, 3.2262) -- (1.1780, 1.2300, 3.2279) -- (1.1320, 1.2300, 3.2255) -- cycle;
\fill[blue!17.3, opacity=0.7] (1.1320, 1.2300, 3.2255) -- (1.1780, 1.2300, 3.2279) -- (1.1780, 1.2840, 3.2294) -- (1.1320, 1.2840, 3.2270) -- cycle;
\fill[blue!19.1, opacity=0.7] (1.1320, 1.2840, 3.2270) -- (1.1780, 1.2840, 3.2294) -- (1.1780, 1.3380, 3.2306) -- (1.1320, 1.3380, 3.2281) -- cycle;
\fill[blue!25.3, opacity=0.7] (1.1320, 1.3380, 3.2281) -- (1.1780, 1.3380, 3.2306) -- (1.1780, 1.3920, 3.2314) -- (1.1320, 1.3920, 3.2290) -- cycle;
\fill[blue!36.5, opacity=0.7] (1.1320, 1.3920, 3.2290) -- (1.1780, 1.3920, 3.2314) -- (1.1780, 1.4460, 3.2319) -- (1.1320, 1.4460, 3.2295) -- cycle;
\fill[blue!47.6, opacity=0.7] (1.1320, 1.4460, 3.2295) -- (1.1780, 1.4460, 3.2319) -- (1.1780, 1.5000, 3.2320) -- (1.1320, 1.5000, 3.2296) -- cycle;
\fill[blue!54.5, opacity=0.7] (1.1320, 1.5000, 3.2296) -- (1.1780, 1.5000, 3.2320) -- (1.1780, 1.5540, 3.2319) -- (1.1320, 1.5540, 3.2295) -- cycle;
\fill[blue!57.8, opacity=0.7] (1.1320, 1.5540, 3.2295) -- (1.1780, 1.5540, 3.2319) -- (1.1780, 1.6080, 3.2314) -- (1.1320, 1.6080, 3.2290) -- cycle;
\fill[blue!59.2, opacity=0.7] (1.1320, 1.6080, 3.2290) -- (1.1780, 1.6080, 3.2314) -- (1.1780, 1.6620, 3.2306) -- (1.1320, 1.6620, 3.2281) -- cycle;
\fill[blue!59.4, opacity=0.7] (1.1320, 1.6620, 3.2281) -- (1.1780, 1.6620, 3.2306) -- (1.1780, 1.7160, 3.2294) -- (1.1320, 1.7160, 3.2270) -- cycle;
\fill[blue!58.3, opacity=0.7] (1.1320, 1.7160, 3.2270) -- (1.1780, 1.7160, 3.2294) -- (1.1780, 1.7700, 3.2279) -- (1.1320, 1.7700, 3.2255) -- cycle;
\fill[blue!55.0, opacity=0.7] (1.1320, 1.7700, 3.2255) -- (1.1780, 1.7700, 3.2279) -- (1.1780, 1.8240, 3.2262) -- (1.1320, 1.8240, 3.2238) -- cycle;
\fill[blue!48.1, opacity=0.7] (1.1320, 1.8240, 3.2238) -- (1.1780, 1.8240, 3.2262) -- (1.1780, 1.8780, 3.2241) -- (1.1320, 1.8780, 3.2217) -- cycle;
\fill[blue!37.9, opacity=0.7] (1.1320, 1.8780, 3.2217) -- (1.1780, 1.8780, 3.2241) -- (1.1780, 1.9320, 3.2217) -- (1.1320, 1.9320, 3.2193) -- cycle;
\fill[blue!28.3, opacity=0.7] (1.1320, 1.9320, 3.2193) -- (1.1780, 1.9320, 3.2217) -- (1.1780, 1.9860, 3.2190) -- (1.1320, 1.9860, 3.2165) -- cycle;
\fill[blue!22.8, opacity=0.7] (1.1320, 1.9860, 3.2165) -- (1.1780, 1.9860, 3.2190) -- (1.1780, 2.0400, 3.2160) -- (1.1320, 2.0400, 3.2135) -- cycle;
\fill[blue!21.4, opacity=0.7] (1.1320, 2.0400, 3.2135) -- (1.1780, 2.0400, 3.2160) -- (1.1780, 2.0940, 3.2127) -- (1.1320, 2.0940, 3.2103) -- cycle;
\fill[blue!24.7, opacity=0.7] (1.1320, 2.0940, 3.2103) -- (1.1780, 2.0940, 3.2127) -- (1.1780, 2.1480, 3.2091) -- (1.1320, 2.1480, 3.2067) -- cycle;
\fill[blue!37.0, opacity=0.7] (1.1320, 2.1480, 3.2067) -- (1.1780, 2.1480, 3.2091) -- (1.1780, 2.2020, 3.2053) -- (1.1320, 2.2020, 3.2029) -- cycle;
\fill[blue!57.8, opacity=0.7] (1.1320, 2.2020, 3.2029) -- (1.1780, 2.2020, 3.2053) -- (1.1780, 2.2560, 3.2012) -- (1.1320, 2.2560, 3.1988) -- cycle;
\fill[blue!61.9, opacity=0.7] (1.1320, 2.2560, 3.1988) -- (1.1780, 2.2560, 3.2012) -- (1.1780, 2.3100, 3.1969) -- (1.1320, 2.3100, 3.1945) -- cycle;
\fill[blue!50.5, opacity=0.7] (1.1320, 2.3100, 3.1945) -- (1.1780, 2.3100, 3.1969) -- (1.1780, 2.3640, 3.1923) -- (1.1320, 2.3640, 3.1899) -- cycle;
\fill[blue!47.4, opacity=0.7] (1.1320, 2.3640, 3.1899) -- (1.1780, 2.3640, 3.1923) -- (1.1780, 2.4180, 3.1875) -- (1.1320, 2.4180, 3.1851) -- cycle;
\fill[blue!56.3, opacity=0.7] (1.1320, 2.4180, 3.1851) -- (1.1780, 2.4180, 3.1875) -- (1.1780, 2.4720, 3.1826) -- (1.1320, 2.4720, 3.1802) -- cycle;
\fill[blue!63.5, opacity=0.7] (1.1320, 2.4720, 3.1802) -- (1.1780, 2.4720, 3.1826) -- (1.1780, 2.5260, 3.1774) -- (1.1320, 2.5260, 3.1750) -- cycle;
\fill[blue!54.3, opacity=0.7] (1.1320, 2.5260, 3.1750) -- (1.1780, 2.5260, 3.1774) -- (1.1780, 2.5800, 3.1720) -- (1.1320, 2.5800, 3.1696) -- cycle;
\fill[blue!46.3, opacity=0.7] (1.1320, 2.5800, 3.1696) -- (1.1780, 2.5800, 3.1720) -- (1.1780, 2.6340, 3.1665) -- (1.1320, 2.6340, 3.1641) -- cycle;
\fill[blue!52.4, opacity=0.7] (1.1320, 2.6340, 3.1641) -- (1.1780, 2.6340, 3.1665) -- (1.1780, 2.6880, 3.1608) -- (1.1320, 2.6880, 3.1584) -- cycle;
\fill[blue!63.5, opacity=0.7] (1.1320, 2.6880, 3.1584) -- (1.1780, 2.6880, 3.1608) -- (1.1780, 2.7420, 3.1550) -- (1.1320, 2.7420, 3.1526) -- cycle;
\fill[blue!48.5, opacity=0.7] (1.1320, 2.7420, 3.1526) -- (1.1780, 2.7420, 3.1550) -- (1.1780, 2.7960, 3.1491) -- (1.1320, 2.7960, 3.1467) -- cycle;
\fill[blue!27.9, opacity=0.7] (1.1320, 2.7960, 3.1467) -- (1.1780, 2.7960, 3.1491) -- (1.1780, 2.8500, 3.1431) -- (1.1320, 2.8500, 3.1407) -- cycle;
\fill[blue!22.3, opacity=0.7] (1.1320, 2.8500, 3.1407) -- (1.1780, 2.8500, 3.1431) -- (1.1780, 2.9040, 3.1370) -- (1.1320, 2.9040, 3.1346) -- cycle;
\fill[blue!26.1, opacity=0.7] (1.1320, 2.9040, 3.1346) -- (1.1780, 2.9040, 3.1370) -- (1.1780, 2.9580, 3.1308) -- (1.1320, 2.9580, 3.1284) -- cycle;
\fill[blue!41.6, opacity=0.7] (1.1320, 2.9580, 3.1284) -- (1.1780, 2.9580, 3.1308) -- (1.1780, 3.0120, 3.1246) -- (1.1320, 3.0120, 3.1222) -- cycle;
\fill[blue!59.4, opacity=0.7] (1.1320, 3.0120, 3.1222) -- (1.1780, 3.0120, 3.1246) -- (1.1780, 3.0660, 3.1183) -- (1.1320, 3.0660, 3.1159) -- cycle;
\fill[blue!63.4, opacity=0.7] (1.1320, 3.0660, 3.1159) -- (1.1780, 3.0660, 3.1183) -- (1.1780, 3.1200, 3.1120) -- (1.1320, 3.1200, 3.1096) -- cycle;
\fill[blue!46.4, opacity=0.7] (1.1780, -0.1200, 3.1120) -- (1.2240, -0.1200, 3.1141) -- (1.2240, -0.0660, 3.1204) -- (1.1780, -0.0660, 3.1183) -- cycle;
\fill[blue!27.1, opacity=0.7] (1.1780, -0.0660, 3.1183) -- (1.2240, -0.0660, 3.1204) -- (1.2240, -0.0120, 3.1267) -- (1.1780, -0.0120, 3.1246) -- cycle;
\fill[blue!20.4, opacity=0.7] (1.1780, -0.0120, 3.1246) -- (1.2240, -0.0120, 3.1267) -- (1.2240, 0.0420, 3.1329) -- (1.1780, 0.0420, 3.1308) -- cycle;
\fill[blue!22.6, opacity=0.7] (1.1780, 0.0420, 3.1308) -- (1.2240, 0.0420, 3.1329) -- (1.2240, 0.0960, 3.1391) -- (1.1780, 0.0960, 3.1370) -- cycle;
\fill[blue!38.9, opacity=0.7] (1.1780, 0.0960, 3.1370) -- (1.2240, 0.0960, 3.1391) -- (1.2240, 0.1500, 3.1452) -- (1.1780, 0.1500, 3.1431) -- cycle;
\fill[blue!62.4, opacity=0.7] (1.1780, 0.1500, 3.1431) -- (1.2240, 0.1500, 3.1452) -- (1.2240, 0.2040, 3.1512) -- (1.1780, 0.2040, 3.1491) -- cycle;
\fill[blue!57.1, opacity=0.7] (1.1780, 0.2040, 3.1491) -- (1.2240, 0.2040, 3.1512) -- (1.2240, 0.2580, 3.1571) -- (1.1780, 0.2580, 3.1550) -- cycle;
\fill[blue!52.2, opacity=0.7] (1.1780, 0.2580, 3.1550) -- (1.2240, 0.2580, 3.1571) -- (1.2240, 0.3120, 3.1629) -- (1.1780, 0.3120, 3.1608) -- cycle;
\fill[blue!61.0, opacity=0.7] (1.1780, 0.3120, 3.1608) -- (1.2240, 0.3120, 3.1629) -- (1.2240, 0.3660, 3.1686) -- (1.1780, 0.3660, 3.1665) -- cycle;
\fill[blue!59.8, opacity=0.7] (1.1780, 0.3660, 3.1665) -- (1.2240, 0.3660, 3.1686) -- (1.2240, 0.4200, 3.1741) -- (1.1780, 0.4200, 3.1720) -- cycle;
\fill[blue!44.1, opacity=0.7] (1.1780, 0.4200, 3.1720) -- (1.2240, 0.4200, 3.1741) -- (1.2240, 0.4740, 3.1795) -- (1.1780, 0.4740, 3.1774) -- cycle;
\fill[blue!40.8, opacity=0.7] (1.1780, 0.4740, 3.1774) -- (1.2240, 0.4740, 3.1795) -- (1.2240, 0.5280, 3.1847) -- (1.1780, 0.5280, 3.1826) -- cycle;
\fill[blue!55.0, opacity=0.7] (1.1780, 0.5280, 3.1826) -- (1.2240, 0.5280, 3.1847) -- (1.2240, 0.5820, 3.1896) -- (1.1780, 0.5820, 3.1875) -- cycle;
\fill[blue!61.6, opacity=0.7] (1.1780, 0.5820, 3.1875) -- (1.2240, 0.5820, 3.1896) -- (1.2240, 0.6360, 3.1944) -- (1.1780, 0.6360, 3.1923) -- cycle;
\fill[blue!39.1, opacity=0.7] (1.1780, 0.6360, 3.1923) -- (1.2240, 0.6360, 3.1944) -- (1.2240, 0.6900, 3.1990) -- (1.1780, 0.6900, 3.1969) -- cycle;
\fill[blue!27.3, opacity=0.7] (1.1780, 0.6900, 3.1969) -- (1.2240, 0.6900, 3.1990) -- (1.2240, 0.7440, 3.2033) -- (1.1780, 0.7440, 3.2012) -- cycle;
\fill[blue!30.2, opacity=0.7] (1.1780, 0.7440, 3.2012) -- (1.2240, 0.7440, 3.2033) -- (1.2240, 0.7980, 3.2074) -- (1.1780, 0.7980, 3.2053) -- cycle;
\fill[blue!47.1, opacity=0.7] (1.1780, 0.7980, 3.2053) -- (1.2240, 0.7980, 3.2074) -- (1.2240, 0.8520, 3.2112) -- (1.1780, 0.8520, 3.2091) -- cycle;
\fill[blue!62.7, opacity=0.7] (1.1780, 0.8520, 3.2091) -- (1.2240, 0.8520, 3.2112) -- (1.2240, 0.9060, 3.2148) -- (1.1780, 0.9060, 3.2127) -- cycle;
\fill[blue!62.4, opacity=0.7] (1.1780, 0.9060, 3.2127) -- (1.2240, 0.9060, 3.2148) -- (1.2240, 0.9600, 3.2180) -- (1.1780, 0.9600, 3.2160) -- cycle;
\fill[blue!63.4, opacity=0.7] (1.1780, 0.9600, 3.2160) -- (1.2240, 0.9600, 3.2180) -- (1.2240, 1.0140, 3.2210) -- (1.1780, 1.0140, 3.2190) -- cycle;
\fill[blue!55.6, opacity=0.7] (1.1780, 1.0140, 3.2190) -- (1.2240, 1.0140, 3.2210) -- (1.2240, 1.0680, 3.2238) -- (1.1780, 1.0680, 3.2217) -- cycle;
\fill[blue!30.8, opacity=0.7] (1.1780, 1.0680, 3.2217) -- (1.2240, 1.0680, 3.2238) -- (1.2240, 1.1220, 3.2262) -- (1.1780, 1.1220, 3.2241) -- cycle;
\fill[blue!19.0, opacity=0.7] (1.1780, 1.1220, 3.2241) -- (1.2240, 1.1220, 3.2262) -- (1.2240, 1.1760, 3.2283) -- (1.1780, 1.1760, 3.2262) -- cycle;
\fill[blue!17.0, opacity=0.7] (1.1780, 1.1760, 3.2262) -- (1.2240, 1.1760, 3.2283) -- (1.2240, 1.2300, 3.2300) -- (1.1780, 1.2300, 3.2279) -- cycle;
\fill[blue!18.3, opacity=0.7] (1.1780, 1.2300, 3.2279) -- (1.2240, 1.2300, 3.2300) -- (1.2240, 1.2840, 3.2315) -- (1.1780, 1.2840, 3.2294) -- cycle;
\fill[blue!25.2, opacity=0.7] (1.1780, 1.2840, 3.2294) -- (1.2240, 1.2840, 3.2315) -- (1.2240, 1.3380, 3.2326) -- (1.1780, 1.3380, 3.2306) -- cycle;
\fill[blue!39.8, opacity=0.7] (1.1780, 1.3380, 3.2306) -- (1.2240, 1.3380, 3.2326) -- (1.2240, 1.3920, 3.2335) -- (1.1780, 1.3920, 3.2314) -- cycle;
\fill[blue!52.9, opacity=0.7] (1.1780, 1.3920, 3.2314) -- (1.2240, 1.3920, 3.2335) -- (1.2240, 1.4460, 3.2340) -- (1.1780, 1.4460, 3.2319) -- cycle;
\fill[blue!58.4, opacity=0.7] (1.1780, 1.4460, 3.2319) -- (1.2240, 1.4460, 3.2340) -- (1.2240, 1.5000, 3.2341) -- (1.1780, 1.5000, 3.2320) -- cycle;
\fill[blue!59.6, opacity=0.7] (1.1780, 1.5000, 3.2320) -- (1.2240, 1.5000, 3.2341) -- (1.2240, 1.5540, 3.2340) -- (1.1780, 1.5540, 3.2319) -- cycle;
\fill[blue!59.7, opacity=0.7] (1.1780, 1.5540, 3.2319) -- (1.2240, 1.5540, 3.2340) -- (1.2240, 1.6080, 3.2335) -- (1.1780, 1.6080, 3.2314) -- cycle;
\fill[blue!60.1, opacity=0.7] (1.1780, 1.6080, 3.2314) -- (1.2240, 1.6080, 3.2335) -- (1.2240, 1.6620, 3.2326) -- (1.1780, 1.6620, 3.2306) -- cycle;
\fill[blue!61.1, opacity=0.7] (1.1780, 1.6620, 3.2306) -- (1.2240, 1.6620, 3.2326) -- (1.2240, 1.7160, 3.2315) -- (1.1780, 1.7160, 3.2294) -- cycle;
\fill[blue!62.1, opacity=0.7] (1.1780, 1.7160, 3.2294) -- (1.2240, 1.7160, 3.2315) -- (1.2240, 1.7700, 3.2300) -- (1.1780, 1.7700, 3.2279) -- cycle;
\fill[blue!62.4, opacity=0.7] (1.1780, 1.7700, 3.2279) -- (1.2240, 1.7700, 3.2300) -- (1.2240, 1.8240, 3.2283) -- (1.1780, 1.8240, 3.2262) -- cycle;
\fill[blue!61.3, opacity=0.7] (1.1780, 1.8240, 3.2262) -- (1.2240, 1.8240, 3.2283) -- (1.2240, 1.8780, 3.2262) -- (1.1780, 1.8780, 3.2241) -- cycle;
\fill[blue!56.3, opacity=0.7] (1.1780, 1.8780, 3.2241) -- (1.2240, 1.8780, 3.2262) -- (1.2240, 1.9320, 3.2238) -- (1.1780, 1.9320, 3.2217) -- cycle;
\fill[blue!44.9, opacity=0.7] (1.1780, 1.9320, 3.2217) -- (1.2240, 1.9320, 3.2238) -- (1.2240, 1.9860, 3.2210) -- (1.1780, 1.9860, 3.2190) -- cycle;
\fill[blue!31.5, opacity=0.7] (1.1780, 1.9860, 3.2190) -- (1.2240, 1.9860, 3.2210) -- (1.2240, 2.0400, 3.2180) -- (1.1780, 2.0400, 3.2160) -- cycle;
\fill[blue!23.7, opacity=0.7] (1.1780, 2.0400, 3.2160) -- (1.2240, 2.0400, 3.2180) -- (1.2240, 2.0940, 3.2148) -- (1.1780, 2.0940, 3.2127) -- cycle;
\fill[blue!21.9, opacity=0.7] (1.1780, 2.0940, 3.2127) -- (1.2240, 2.0940, 3.2148) -- (1.2240, 2.1480, 3.2112) -- (1.1780, 2.1480, 3.2091) -- cycle;
\fill[blue!26.2, opacity=0.7] (1.1780, 2.1480, 3.2091) -- (1.2240, 2.1480, 3.2112) -- (1.2240, 2.2020, 3.2074) -- (1.1780, 2.2020, 3.2053) -- cycle;
\fill[blue!42.0, opacity=0.7] (1.1780, 2.2020, 3.2053) -- (1.2240, 2.2020, 3.2074) -- (1.2240, 2.2560, 3.2033) -- (1.1780, 2.2560, 3.2012) -- cycle;
\fill[blue!62.0, opacity=0.7] (1.1780, 2.2560, 3.2012) -- (1.2240, 2.2560, 3.2033) -- (1.2240, 2.3100, 3.1990) -- (1.1780, 2.3100, 3.1969) -- cycle;
\fill[blue!57.9, opacity=0.7] (1.1780, 2.3100, 3.1969) -- (1.2240, 2.3100, 3.1990) -- (1.2240, 2.3640, 3.1944) -- (1.1780, 2.3640, 3.1923) -- cycle;
\fill[blue!47.2, opacity=0.7] (1.1780, 2.3640, 3.1923) -- (1.2240, 2.3640, 3.1944) -- (1.2240, 2.4180, 3.1896) -- (1.1780, 2.4180, 3.1875) -- cycle;
\fill[blue!49.4, opacity=0.7] (1.1780, 2.4180, 3.1875) -- (1.2240, 2.4180, 3.1896) -- (1.2240, 2.4720, 3.1847) -- (1.1780, 2.4720, 3.1826) -- cycle;
\fill[blue!61.0, opacity=0.7] (1.1780, 2.4720, 3.1826) -- (1.2240, 2.4720, 3.1847) -- (1.2240, 2.5260, 3.1795) -- (1.1780, 2.5260, 3.1774) -- cycle;
\fill[blue!60.8, opacity=0.7] (1.1780, 2.5260, 3.1774) -- (1.2240, 2.5260, 3.1795) -- (1.2240, 2.5800, 3.1741) -- (1.1780, 2.5800, 3.1720) -- cycle;
\fill[blue!49.2, opacity=0.7] (1.1780, 2.5800, 3.1720) -- (1.2240, 2.5800, 3.1741) -- (1.2240, 2.6340, 3.1686) -- (1.1780, 2.6340, 3.1665) -- cycle;
\fill[blue!48.1, opacity=0.7] (1.1780, 2.6340, 3.1665) -- (1.2240, 2.6340, 3.1686) -- (1.2240, 2.6880, 3.1629) -- (1.1780, 2.6880, 3.1608) -- cycle;
\fill[blue!60.3, opacity=0.7] (1.1780, 2.6880, 3.1608) -- (1.2240, 2.6880, 3.1629) -- (1.2240, 2.7420, 3.1571) -- (1.1780, 2.7420, 3.1550) -- cycle;
\fill[blue!58.1, opacity=0.7] (1.1780, 2.7420, 3.1550) -- (1.2240, 2.7420, 3.1571) -- (1.2240, 2.7960, 3.1512) -- (1.1780, 2.7960, 3.1491) -- cycle;
\fill[blue!34.0, opacity=0.7] (1.1780, 2.7960, 3.1491) -- (1.2240, 2.7960, 3.1512) -- (1.2240, 2.8500, 3.1452) -- (1.1780, 2.8500, 3.1431) -- cycle;
\fill[blue!22.9, opacity=0.7] (1.1780, 2.8500, 3.1431) -- (1.2240, 2.8500, 3.1452) -- (1.2240, 2.9040, 3.1391) -- (1.1780, 2.9040, 3.1370) -- cycle;
\fill[blue!23.3, opacity=0.7] (1.1780, 2.9040, 3.1370) -- (1.2240, 2.9040, 3.1391) -- (1.2240, 2.9580, 3.1329) -- (1.1780, 2.9580, 3.1308) -- cycle;
\fill[blue!34.3, opacity=0.7] (1.1780, 2.9580, 3.1308) -- (1.2240, 2.9580, 3.1329) -- (1.2240, 3.0120, 3.1267) -- (1.1780, 3.0120, 3.1246) -- cycle;
\fill[blue!54.3, opacity=0.7] (1.1780, 3.0120, 3.1246) -- (1.2240, 3.0120, 3.1267) -- (1.2240, 3.0660, 3.1204) -- (1.1780, 3.0660, 3.1183) -- cycle;
\fill[blue!63.0, opacity=0.7] (1.1780, 3.0660, 3.1183) -- (1.2240, 3.0660, 3.1204) -- (1.2240, 3.1200, 3.1141) -- (1.1780, 3.1200, 3.1120) -- cycle;
\fill[blue!44.2, opacity=0.7] (1.2240, -0.1200, 3.1141) -- (1.2700, -0.1200, 3.1159) -- (1.2700, -0.0660, 3.1222) -- (1.2240, -0.0660, 3.1204) -- cycle;
\fill[blue!25.8, opacity=0.7] (1.2240, -0.0660, 3.1204) -- (1.2700, -0.0660, 3.1222) -- (1.2700, -0.0120, 3.1285) -- (1.2240, -0.0120, 3.1267) -- cycle;
\fill[blue!20.4, opacity=0.7] (1.2240, -0.0120, 3.1267) -- (1.2700, -0.0120, 3.1285) -- (1.2700, 0.0420, 3.1347) -- (1.2240, 0.0420, 3.1329) -- cycle;
\fill[blue!23.8, opacity=0.7] (1.2240, 0.0420, 3.1329) -- (1.2700, 0.0420, 3.1347) -- (1.2700, 0.0960, 3.1409) -- (1.2240, 0.0960, 3.1391) -- cycle;
\fill[blue!42.8, opacity=0.7] (1.2240, 0.0960, 3.1391) -- (1.2700, 0.0960, 3.1409) -- (1.2700, 0.1500, 3.1470) -- (1.2240, 0.1500, 3.1452) -- cycle;
\fill[blue!63.4, opacity=0.7] (1.2240, 0.1500, 3.1452) -- (1.2700, 0.1500, 3.1470) -- (1.2700, 0.2040, 3.1530) -- (1.2240, 0.2040, 3.1512) -- cycle;
\fill[blue!55.2, opacity=0.7] (1.2240, 0.2040, 3.1512) -- (1.2700, 0.2040, 3.1530) -- (1.2700, 0.2580, 3.1589) -- (1.2240, 0.2580, 3.1571) -- cycle;
\fill[blue!52.5, opacity=0.7] (1.2240, 0.2580, 3.1571) -- (1.2700, 0.2580, 3.1589) -- (1.2700, 0.3120, 3.1647) -- (1.2240, 0.3120, 3.1629) -- cycle;
\fill[blue!62.2, opacity=0.7] (1.2240, 0.3120, 3.1629) -- (1.2700, 0.3120, 3.1647) -- (1.2700, 0.3660, 3.1704) -- (1.2240, 0.3660, 3.1686) -- cycle;
\fill[blue!57.7, opacity=0.7] (1.2240, 0.3660, 3.1686) -- (1.2700, 0.3660, 3.1704) -- (1.2700, 0.4200, 3.1759) -- (1.2240, 0.4200, 3.1741) -- cycle;
\fill[blue!42.7, opacity=0.7] (1.2240, 0.4200, 3.1741) -- (1.2700, 0.4200, 3.1759) -- (1.2700, 0.4740, 3.1813) -- (1.2240, 0.4740, 3.1795) -- cycle;
\fill[blue!42.6, opacity=0.7] (1.2240, 0.4740, 3.1795) -- (1.2700, 0.4740, 3.1813) -- (1.2700, 0.5280, 3.1864) -- (1.2240, 0.5280, 3.1847) -- cycle;
\fill[blue!58.8, opacity=0.7] (1.2240, 0.5280, 3.1847) -- (1.2700, 0.5280, 3.1864) -- (1.2700, 0.5820, 3.1914) -- (1.2240, 0.5820, 3.1896) -- cycle;
\fill[blue!57.8, opacity=0.7] (1.2240, 0.5820, 3.1896) -- (1.2700, 0.5820, 3.1914) -- (1.2700, 0.6360, 3.1962) -- (1.2240, 0.6360, 3.1944) -- cycle;
\fill[blue!34.5, opacity=0.7] (1.2240, 0.6360, 3.1944) -- (1.2700, 0.6360, 3.1962) -- (1.2700, 0.6900, 3.2008) -- (1.2240, 0.6900, 3.1990) -- cycle;
\fill[blue!26.4, opacity=0.7] (1.2240, 0.6900, 3.1990) -- (1.2700, 0.6900, 3.2008) -- (1.2700, 0.7440, 3.2051) -- (1.2240, 0.7440, 3.2033) -- cycle;
\fill[blue!32.5, opacity=0.7] (1.2240, 0.7440, 3.2033) -- (1.2700, 0.7440, 3.2051) -- (1.2700, 0.7980, 3.2092) -- (1.2240, 0.7980, 3.2074) -- cycle;
\fill[blue!52.1, opacity=0.7] (1.2240, 0.7980, 3.2074) -- (1.2700, 0.7980, 3.2092) -- (1.2700, 0.8520, 3.2130) -- (1.2240, 0.8520, 3.2112) -- cycle;
\fill[blue!63.5, opacity=0.7] (1.2240, 0.8520, 3.2112) -- (1.2700, 0.8520, 3.2130) -- (1.2700, 0.9060, 3.2166) -- (1.2240, 0.9060, 3.2148) -- cycle;
\fill[blue!62.5, opacity=0.7] (1.2240, 0.9060, 3.2148) -- (1.2700, 0.9060, 3.2166) -- (1.2700, 0.9600, 3.2198) -- (1.2240, 0.9600, 3.2180) -- cycle;
\fill[blue!63.1, opacity=0.7] (1.2240, 0.9600, 3.2180) -- (1.2700, 0.9600, 3.2198) -- (1.2700, 1.0140, 3.2228) -- (1.2240, 1.0140, 3.2210) -- cycle;
\fill[blue!45.2, opacity=0.7] (1.2240, 1.0140, 3.2210) -- (1.2700, 1.0140, 3.2228) -- (1.2700, 1.0680, 3.2255) -- (1.2240, 1.0680, 3.2238) -- cycle;
\fill[blue!23.0, opacity=0.7] (1.2240, 1.0680, 3.2238) -- (1.2700, 1.0680, 3.2255) -- (1.2700, 1.1220, 3.2279) -- (1.2240, 1.1220, 3.2262) -- cycle;
\fill[blue!17.2, opacity=0.7] (1.2240, 1.1220, 3.2262) -- (1.2700, 1.1220, 3.2279) -- (1.2700, 1.1760, 3.2300) -- (1.2240, 1.1760, 3.2283) -- cycle;
\fill[blue!17.1, opacity=0.7] (1.2240, 1.1760, 3.2283) -- (1.2700, 1.1760, 3.2300) -- (1.2700, 1.2300, 3.2318) -- (1.2240, 1.2300, 3.2300) -- cycle;
\fill[blue!21.6, opacity=0.7] (1.2240, 1.2300, 3.2300) -- (1.2700, 1.2300, 3.2318) -- (1.2700, 1.2840, 3.2333) -- (1.2240, 1.2840, 3.2315) -- cycle;
\fill[blue!36.3, opacity=0.7] (1.2240, 1.2840, 3.2315) -- (1.2700, 1.2840, 3.2333) -- (1.2700, 1.3380, 3.2344) -- (1.2240, 1.3380, 3.2326) -- cycle;
\fill[blue!52.4, opacity=0.7] (1.2240, 1.3380, 3.2326) -- (1.2700, 1.3380, 3.2344) -- (1.2700, 1.3920, 3.2353) -- (1.2240, 1.3920, 3.2335) -- cycle;
\fill[blue!57.6, opacity=0.7] (1.2240, 1.3920, 3.2335) -- (1.2700, 1.3920, 3.2353) -- (1.2700, 1.4460, 3.2357) -- (1.2240, 1.4460, 3.2340) -- cycle;
\fill[blue!55.2, opacity=0.7] (1.2240, 1.4460, 3.2340) -- (1.2700, 1.4460, 3.2357) -- (1.2700, 1.5000, 3.2359) -- (1.2240, 1.5000, 3.2341) -- cycle;
\fill[blue!49.0, opacity=0.7] (1.2240, 1.5000, 3.2341) -- (1.2700, 1.5000, 3.2359) -- (1.2700, 1.5540, 3.2357) -- (1.2240, 1.5540, 3.2340) -- cycle;
\fill[blue!43.9, opacity=0.7] (1.2240, 1.5540, 3.2340) -- (1.2700, 1.5540, 3.2357) -- (1.2700, 1.6080, 3.2353) -- (1.2240, 1.6080, 3.2335) -- cycle;
\fill[blue!43.2, opacity=0.7] (1.2240, 1.6080, 3.2335) -- (1.2700, 1.6080, 3.2353) -- (1.2700, 1.6620, 3.2344) -- (1.2240, 1.6620, 3.2326) -- cycle;
\fill[blue!47.1, opacity=0.7] (1.2240, 1.6620, 3.2326) -- (1.2700, 1.6620, 3.2344) -- (1.2700, 1.7160, 3.2333) -- (1.2240, 1.7160, 3.2315) -- cycle;
\fill[blue!54.1, opacity=0.7] (1.2240, 1.7160, 3.2315) -- (1.2700, 1.7160, 3.2333) -- (1.2700, 1.7700, 3.2318) -- (1.2240, 1.7700, 3.2300) -- cycle;
\fill[blue!60.2, opacity=0.7] (1.2240, 1.7700, 3.2300) -- (1.2700, 1.7700, 3.2318) -- (1.2700, 1.8240, 3.2300) -- (1.2240, 1.8240, 3.2283) -- cycle;
\fill[blue!62.8, opacity=0.7] (1.2240, 1.8240, 3.2283) -- (1.2700, 1.8240, 3.2300) -- (1.2700, 1.8780, 3.2279) -- (1.2240, 1.8780, 3.2262) -- cycle;
\fill[blue!62.9, opacity=0.7] (1.2240, 1.8780, 3.2262) -- (1.2700, 1.8780, 3.2279) -- (1.2700, 1.9320, 3.2255) -- (1.2240, 1.9320, 3.2238) -- cycle;
\fill[blue!59.5, opacity=0.7] (1.2240, 1.9320, 3.2238) -- (1.2700, 1.9320, 3.2255) -- (1.2700, 1.9860, 3.2228) -- (1.2240, 1.9860, 3.2210) -- cycle;
\fill[blue!47.5, opacity=0.7] (1.2240, 1.9860, 3.2210) -- (1.2700, 1.9860, 3.2228) -- (1.2700, 2.0400, 3.2198) -- (1.2240, 2.0400, 3.2180) -- cycle;
\fill[blue!31.9, opacity=0.7] (1.2240, 2.0400, 3.2180) -- (1.2700, 2.0400, 3.2198) -- (1.2700, 2.0940, 3.2166) -- (1.2240, 2.0940, 3.2148) -- cycle;
\fill[blue!23.6, opacity=0.7] (1.2240, 2.0940, 3.2148) -- (1.2700, 2.0940, 3.2166) -- (1.2700, 2.1480, 3.2130) -- (1.2240, 2.1480, 3.2112) -- cycle;
\fill[blue!22.6, opacity=0.7] (1.2240, 2.1480, 3.2112) -- (1.2700, 2.1480, 3.2130) -- (1.2700, 2.2020, 3.2092) -- (1.2240, 2.2020, 3.2074) -- cycle;
\fill[blue!30.0, opacity=0.7] (1.2240, 2.2020, 3.2074) -- (1.2700, 2.2020, 3.2092) -- (1.2700, 2.2560, 3.2051) -- (1.2240, 2.2560, 3.2033) -- cycle;
\fill[blue!51.0, opacity=0.7] (1.2240, 2.2560, 3.2033) -- (1.2700, 2.2560, 3.2051) -- (1.2700, 2.3100, 3.2008) -- (1.2240, 2.3100, 3.1990) -- cycle;
\fill[blue!63.4, opacity=0.7] (1.2240, 2.3100, 3.1990) -- (1.2700, 2.3100, 3.2008) -- (1.2700, 2.3640, 3.1962) -- (1.2240, 2.3640, 3.1944) -- cycle;
\fill[blue!51.5, opacity=0.7] (1.2240, 2.3640, 3.1944) -- (1.2700, 2.3640, 3.1962) -- (1.2700, 2.4180, 3.1914) -- (1.2240, 2.4180, 3.1896) -- cycle;
\fill[blue!45.9, opacity=0.7] (1.2240, 2.4180, 3.1896) -- (1.2700, 2.4180, 3.1914) -- (1.2700, 2.4720, 3.1864) -- (1.2240, 2.4720, 3.1847) -- cycle;
\fill[blue!54.7, opacity=0.7] (1.2240, 2.4720, 3.1847) -- (1.2700, 2.4720, 3.1864) -- (1.2700, 2.5260, 3.1813) -- (1.2240, 2.5260, 3.1795) -- cycle;
\fill[blue!63.6, opacity=0.7] (1.2240, 2.5260, 3.1795) -- (1.2700, 2.5260, 3.1813) -- (1.2700, 2.5800, 3.1759) -- (1.2240, 2.5800, 3.1741) -- cycle;
\fill[blue!54.4, opacity=0.7] (1.2240, 2.5800, 3.1741) -- (1.2700, 2.5800, 3.1759) -- (1.2700, 2.6340, 3.1704) -- (1.2240, 2.6340, 3.1686) -- cycle;
\fill[blue!47.1, opacity=0.7] (1.2240, 2.6340, 3.1686) -- (1.2700, 2.6340, 3.1704) -- (1.2700, 2.6880, 3.1647) -- (1.2240, 2.6880, 3.1629) -- cycle;
\fill[blue!55.1, opacity=0.7] (1.2240, 2.6880, 3.1629) -- (1.2700, 2.6880, 3.1647) -- (1.2700, 2.7420, 3.1589) -- (1.2240, 2.7420, 3.1571) -- cycle;
\fill[blue!63.0, opacity=0.7] (1.2240, 2.7420, 3.1571) -- (1.2700, 2.7420, 3.1589) -- (1.2700, 2.7960, 3.1530) -- (1.2240, 2.7960, 3.1512) -- cycle;
\fill[blue!41.9, opacity=0.7] (1.2240, 2.7960, 3.1512) -- (1.2700, 2.7960, 3.1530) -- (1.2700, 2.8500, 3.1470) -- (1.2240, 2.8500, 3.1452) -- cycle;
\fill[blue!24.8, opacity=0.7] (1.2240, 2.8500, 3.1452) -- (1.2700, 2.8500, 3.1470) -- (1.2700, 2.9040, 3.1409) -- (1.2240, 2.9040, 3.1391) -- cycle;
\fill[blue!22.1, opacity=0.7] (1.2240, 2.9040, 3.1391) -- (1.2700, 2.9040, 3.1409) -- (1.2700, 2.9580, 3.1347) -- (1.2240, 2.9580, 3.1329) -- cycle;
\fill[blue!29.0, opacity=0.7] (1.2240, 2.9580, 3.1329) -- (1.2700, 2.9580, 3.1347) -- (1.2700, 3.0120, 3.1285) -- (1.2240, 3.0120, 3.1267) -- cycle;
\fill[blue!48.1, opacity=0.7] (1.2240, 3.0120, 3.1267) -- (1.2700, 3.0120, 3.1285) -- (1.2700, 3.0660, 3.1222) -- (1.2240, 3.0660, 3.1204) -- cycle;
\fill[blue!61.9, opacity=0.7] (1.2240, 3.0660, 3.1204) -- (1.2700, 3.0660, 3.1222) -- (1.2700, 3.1200, 3.1159) -- (1.2240, 3.1200, 3.1141) -- cycle;
\fill[blue!42.8, opacity=0.7] (1.2700, -0.1200, 3.1159) -- (1.3160, -0.1200, 3.1174) -- (1.3160, -0.0660, 3.1237) -- (1.2700, -0.0660, 3.1222) -- cycle;
\fill[blue!25.1, opacity=0.7] (1.2700, -0.0660, 3.1222) -- (1.3160, -0.0660, 3.1237) -- (1.3160, -0.0120, 3.1299) -- (1.2700, -0.0120, 3.1285) -- cycle;
\fill[blue!20.5, opacity=0.7] (1.2700, -0.0120, 3.1285) -- (1.3160, -0.0120, 3.1299) -- (1.3160, 0.0420, 3.1361) -- (1.2700, 0.0420, 3.1347) -- cycle;
\fill[blue!24.9, opacity=0.7] (1.2700, 0.0420, 3.1347) -- (1.3160, 0.0420, 3.1361) -- (1.3160, 0.0960, 3.1423) -- (1.2700, 0.0960, 3.1409) -- cycle;
\fill[blue!45.8, opacity=0.7] (1.2700, 0.0960, 3.1409) -- (1.3160, 0.0960, 3.1423) -- (1.3160, 0.1500, 3.1484) -- (1.2700, 0.1500, 3.1470) -- cycle;
\fill[blue!63.6, opacity=0.7] (1.2700, 0.1500, 3.1470) -- (1.3160, 0.1500, 3.1484) -- (1.3160, 0.2040, 3.1545) -- (1.2700, 0.2040, 3.1530) -- cycle;
\fill[blue!53.9, opacity=0.7] (1.2700, 0.2040, 3.1530) -- (1.3160, 0.2040, 3.1545) -- (1.3160, 0.2580, 3.1604) -- (1.2700, 0.2580, 3.1589) -- cycle;
\fill[blue!52.8, opacity=0.7] (1.2700, 0.2580, 3.1589) -- (1.3160, 0.2580, 3.1604) -- (1.3160, 0.3120, 3.1662) -- (1.2700, 0.3120, 3.1647) -- cycle;
\fill[blue!62.8, opacity=0.7] (1.2700, 0.3120, 3.1647) -- (1.3160, 0.3120, 3.1662) -- (1.3160, 0.3660, 3.1719) -- (1.2700, 0.3660, 3.1704) -- cycle;
\fill[blue!56.2, opacity=0.7] (1.2700, 0.3660, 3.1704) -- (1.3160, 0.3660, 3.1719) -- (1.3160, 0.4200, 3.1774) -- (1.2700, 0.4200, 3.1759) -- cycle;
\fill[blue!42.2, opacity=0.7] (1.2700, 0.4200, 3.1759) -- (1.3160, 0.4200, 3.1774) -- (1.3160, 0.4740, 3.1827) -- (1.2700, 0.4740, 3.1813) -- cycle;
\fill[blue!44.5, opacity=0.7] (1.2700, 0.4740, 3.1813) -- (1.3160, 0.4740, 3.1827) -- (1.3160, 0.5280, 3.1879) -- (1.2700, 0.5280, 3.1864) -- cycle;
\fill[blue!61.2, opacity=0.7] (1.2700, 0.5280, 3.1864) -- (1.3160, 0.5280, 3.1879) -- (1.3160, 0.5820, 3.1929) -- (1.2700, 0.5820, 3.1914) -- cycle;
\fill[blue!54.1, opacity=0.7] (1.2700, 0.5820, 3.1914) -- (1.3160, 0.5820, 3.1929) -- (1.3160, 0.6360, 3.1977) -- (1.2700, 0.6360, 3.1962) -- cycle;
\fill[blue!31.7, opacity=0.7] (1.2700, 0.6360, 3.1962) -- (1.3160, 0.6360, 3.1977) -- (1.3160, 0.6900, 3.2022) -- (1.2700, 0.6900, 3.2008) -- cycle;
\fill[blue!26.1, opacity=0.7] (1.2700, 0.6900, 3.2008) -- (1.3160, 0.6900, 3.2022) -- (1.3160, 0.7440, 3.2066) -- (1.2700, 0.7440, 3.2051) -- cycle;
\fill[blue!34.5, opacity=0.7] (1.2700, 0.7440, 3.2051) -- (1.3160, 0.7440, 3.2066) -- (1.3160, 0.7980, 3.2106) -- (1.2700, 0.7980, 3.2092) -- cycle;
\fill[blue!55.3, opacity=0.7] (1.2700, 0.7980, 3.2092) -- (1.3160, 0.7980, 3.2106) -- (1.3160, 0.8520, 3.2145) -- (1.2700, 0.8520, 3.2130) -- cycle;
\fill[blue!63.6, opacity=0.7] (1.2700, 0.8520, 3.2130) -- (1.3160, 0.8520, 3.2145) -- (1.3160, 0.9060, 3.2180) -- (1.2700, 0.9060, 3.2166) -- cycle;
\fill[blue!62.9, opacity=0.7] (1.2700, 0.9060, 3.2166) -- (1.3160, 0.9060, 3.2180) -- (1.3160, 0.9600, 3.2213) -- (1.2700, 0.9600, 3.2198) -- cycle;
\fill[blue!60.8, opacity=0.7] (1.2700, 0.9600, 3.2198) -- (1.3160, 0.9600, 3.2213) -- (1.3160, 1.0140, 3.2243) -- (1.2700, 1.0140, 3.2228) -- cycle;
\fill[blue!36.5, opacity=0.7] (1.2700, 1.0140, 3.2228) -- (1.3160, 1.0140, 3.2243) -- (1.3160, 1.0680, 3.2270) -- (1.2700, 1.0680, 3.2255) -- cycle;
\fill[blue!19.6, opacity=0.7] (1.2700, 1.0680, 3.2255) -- (1.3160, 1.0680, 3.2270) -- (1.3160, 1.1220, 3.2294) -- (1.2700, 1.1220, 3.2279) -- cycle;
\fill[blue!16.6, opacity=0.7] (1.2700, 1.1220, 3.2279) -- (1.3160, 1.1220, 3.2294) -- (1.3160, 1.1760, 3.2315) -- (1.2700, 1.1760, 3.2300) -- cycle;
\fill[blue!17.9, opacity=0.7] (1.2700, 1.1760, 3.2300) -- (1.3160, 1.1760, 3.2315) -- (1.3160, 1.2300, 3.2333) -- (1.2700, 1.2300, 3.2318) -- cycle;
\fill[blue!26.9, opacity=0.7] (1.2700, 1.2300, 3.2318) -- (1.3160, 1.2300, 3.2333) -- (1.3160, 1.2840, 3.2348) -- (1.2700, 1.2840, 3.2333) -- cycle;
\fill[blue!46.3, opacity=0.7] (1.2700, 1.2840, 3.2333) -- (1.3160, 1.2840, 3.2348) -- (1.3160, 1.3380, 3.2359) -- (1.2700, 1.3380, 3.2344) -- cycle;
\fill[blue!55.9, opacity=0.7] (1.2700, 1.3380, 3.2344) -- (1.3160, 1.3380, 3.2359) -- (1.3160, 1.3920, 3.2367) -- (1.2700, 1.3920, 3.2353) -- cycle;
\fill[blue!51.1, opacity=0.7] (1.2700, 1.3920, 3.2353) -- (1.3160, 1.3920, 3.2367) -- (1.3160, 1.4460, 3.2372) -- (1.2700, 1.4460, 3.2357) -- cycle;
\fill[blue!36.3, opacity=0.7] (1.2700, 1.4460, 3.2357) -- (1.3160, 1.4460, 3.2372) -- (1.3160, 1.5000, 3.2374) -- (1.2700, 1.5000, 3.2359) -- cycle;
\fill[blue!24.9, opacity=0.7] (1.2700, 1.5000, 3.2359) -- (1.3160, 1.5000, 3.2374) -- (1.3160, 1.5540, 3.2372) -- (1.2700, 1.5540, 3.2357) -- cycle;
\fill[blue!20.8, opacity=0.7] (1.2700, 1.5540, 3.2357) -- (1.3160, 1.5540, 3.2372) -- (1.3160, 1.6080, 3.2367) -- (1.2700, 1.6080, 3.2353) -- cycle;
\fill[blue!20.4, opacity=0.7] (1.2700, 1.6080, 3.2353) -- (1.3160, 1.6080, 3.2367) -- (1.3160, 1.6620, 3.2359) -- (1.2700, 1.6620, 3.2344) -- cycle;
\fill[blue!22.9, opacity=0.7] (1.2700, 1.6620, 3.2344) -- (1.3160, 1.6620, 3.2359) -- (1.3160, 1.7160, 3.2348) -- (1.2700, 1.7160, 3.2333) -- cycle;
\fill[blue!29.9, opacity=0.7] (1.2700, 1.7160, 3.2333) -- (1.3160, 1.7160, 3.2348) -- (1.3160, 1.7700, 3.2333) -- (1.2700, 1.7700, 3.2318) -- cycle;
\fill[blue!43.2, opacity=0.7] (1.2700, 1.7700, 3.2318) -- (1.3160, 1.7700, 3.2333) -- (1.3160, 1.8240, 3.2315) -- (1.2700, 1.8240, 3.2300) -- cycle;
\fill[blue!57.1, opacity=0.7] (1.2700, 1.8240, 3.2300) -- (1.3160, 1.8240, 3.2315) -- (1.3160, 1.8780, 3.2294) -- (1.2700, 1.8780, 3.2279) -- cycle;
\fill[blue!62.8, opacity=0.7] (1.2700, 1.8780, 3.2279) -- (1.3160, 1.8780, 3.2294) -- (1.3160, 1.9320, 3.2270) -- (1.2700, 1.9320, 3.2255) -- cycle;
\fill[blue!63.3, opacity=0.7] (1.2700, 1.9320, 3.2255) -- (1.3160, 1.9320, 3.2270) -- (1.3160, 1.9860, 3.2243) -- (1.2700, 1.9860, 3.2228) -- cycle;
\fill[blue!59.9, opacity=0.7] (1.2700, 1.9860, 3.2228) -- (1.3160, 1.9860, 3.2243) -- (1.3160, 2.0400, 3.2213) -- (1.2700, 2.0400, 3.2198) -- cycle;
\fill[blue!45.8, opacity=0.7] (1.2700, 2.0400, 3.2198) -- (1.3160, 2.0400, 3.2213) -- (1.3160, 2.0940, 3.2180) -- (1.2700, 2.0940, 3.2166) -- cycle;
\fill[blue!29.6, opacity=0.7] (1.2700, 2.0940, 3.2166) -- (1.3160, 2.0940, 3.2180) -- (1.3160, 2.1480, 3.2145) -- (1.2700, 2.1480, 3.2130) -- cycle;
\fill[blue!23.0, opacity=0.7] (1.2700, 2.1480, 3.2130) -- (1.3160, 2.1480, 3.2145) -- (1.3160, 2.2020, 3.2106) -- (1.2700, 2.2020, 3.2092) -- cycle;
\fill[blue!24.6, opacity=0.7] (1.2700, 2.2020, 3.2092) -- (1.3160, 2.2020, 3.2106) -- (1.3160, 2.2560, 3.2066) -- (1.2700, 2.2560, 3.2051) -- cycle;
\fill[blue!38.3, opacity=0.7] (1.2700, 2.2560, 3.2051) -- (1.3160, 2.2560, 3.2066) -- (1.3160, 2.3100, 3.2022) -- (1.2700, 2.3100, 3.2008) -- cycle;
\fill[blue!61.0, opacity=0.7] (1.2700, 2.3100, 3.2008) -- (1.3160, 2.3100, 3.2022) -- (1.3160, 2.3640, 3.1977) -- (1.2700, 2.3640, 3.1962) -- cycle;
\fill[blue!57.9, opacity=0.7] (1.2700, 2.3640, 3.1962) -- (1.3160, 2.3640, 3.1977) -- (1.3160, 2.4180, 3.1929) -- (1.2700, 2.4180, 3.1914) -- cycle;
\fill[blue!46.0, opacity=0.7] (1.2700, 2.4180, 3.1914) -- (1.3160, 2.4180, 3.1929) -- (1.3160, 2.4720, 3.1879) -- (1.2700, 2.4720, 3.1864) -- cycle;
\fill[blue!49.0, opacity=0.7] (1.2700, 2.4720, 3.1864) -- (1.3160, 2.4720, 3.1879) -- (1.3160, 2.5260, 3.1827) -- (1.2700, 2.5260, 3.1813) -- cycle;
\fill[blue!61.6, opacity=0.7] (1.2700, 2.5260, 3.1813) -- (1.3160, 2.5260, 3.1827) -- (1.3160, 2.5800, 3.1774) -- (1.2700, 2.5800, 3.1759) -- cycle;
\fill[blue!59.6, opacity=0.7] (1.2700, 2.5800, 3.1759) -- (1.3160, 2.5800, 3.1774) -- (1.3160, 2.6340, 3.1719) -- (1.2700, 2.6340, 3.1704) -- cycle;
\fill[blue!48.6, opacity=0.7] (1.2700, 2.6340, 3.1704) -- (1.3160, 2.6340, 3.1719) -- (1.3160, 2.6880, 3.1662) -- (1.2700, 2.6880, 3.1647) -- cycle;
\fill[blue!51.1, opacity=0.7] (1.2700, 2.6880, 3.1647) -- (1.3160, 2.6880, 3.1662) -- (1.3160, 2.7420, 3.1604) -- (1.2700, 2.7420, 3.1589) -- cycle;
\fill[blue!63.2, opacity=0.7] (1.2700, 2.7420, 3.1589) -- (1.3160, 2.7420, 3.1604) -- (1.3160, 2.7960, 3.1545) -- (1.2700, 2.7960, 3.1530) -- cycle;
\fill[blue!50.0, opacity=0.7] (1.2700, 2.7960, 3.1530) -- (1.3160, 2.7960, 3.1545) -- (1.3160, 2.8500, 3.1484) -- (1.2700, 2.8500, 3.1470) -- cycle;
\fill[blue!27.8, opacity=0.7] (1.2700, 2.8500, 3.1470) -- (1.3160, 2.8500, 3.1484) -- (1.3160, 2.9040, 3.1423) -- (1.2700, 2.9040, 3.1409) -- cycle;
\fill[blue!21.8, opacity=0.7] (1.2700, 2.9040, 3.1409) -- (1.3160, 2.9040, 3.1423) -- (1.3160, 2.9580, 3.1361) -- (1.2700, 2.9580, 3.1347) -- cycle;
\fill[blue!25.6, opacity=0.7] (1.2700, 2.9580, 3.1347) -- (1.3160, 2.9580, 3.1361) -- (1.3160, 3.0120, 3.1299) -- (1.2700, 3.0120, 3.1285) -- cycle;
\fill[blue!41.9, opacity=0.7] (1.2700, 3.0120, 3.1285) -- (1.3160, 3.0120, 3.1299) -- (1.3160, 3.0660, 3.1237) -- (1.2700, 3.0660, 3.1222) -- cycle;
\fill[blue!59.7, opacity=0.7] (1.2700, 3.0660, 3.1222) -- (1.3160, 3.0660, 3.1237) -- (1.3160, 3.1200, 3.1174) -- (1.2700, 3.1200, 3.1159) -- cycle;
\fill[blue!42.1, opacity=0.7] (1.3160, -0.1200, 3.1174) -- (1.3620, -0.1200, 3.1185) -- (1.3620, -0.0660, 3.1248) -- (1.3160, -0.0660, 3.1237) -- cycle;
\fill[blue!24.9, opacity=0.7] (1.3160, -0.0660, 3.1237) -- (1.3620, -0.0660, 3.1248) -- (1.3620, -0.0120, 3.1311) -- (1.3160, -0.0120, 3.1299) -- cycle;
\fill[blue!20.6, opacity=0.7] (1.3160, -0.0120, 3.1299) -- (1.3620, -0.0120, 3.1311) -- (1.3620, 0.0420, 3.1373) -- (1.3160, 0.0420, 3.1361) -- cycle;
\fill[blue!25.7, opacity=0.7] (1.3160, 0.0420, 3.1361) -- (1.3620, 0.0420, 3.1373) -- (1.3620, 0.0960, 3.1435) -- (1.3160, 0.0960, 3.1423) -- cycle;
\fill[blue!47.7, opacity=0.7] (1.3160, 0.0960, 3.1423) -- (1.3620, 0.0960, 3.1435) -- (1.3620, 0.1500, 3.1496) -- (1.3160, 0.1500, 3.1484) -- cycle;
\fill[blue!63.4, opacity=0.7] (1.3160, 0.1500, 3.1484) -- (1.3620, 0.1500, 3.1496) -- (1.3620, 0.2040, 3.1556) -- (1.3160, 0.2040, 3.1545) -- cycle;
\fill[blue!53.0, opacity=0.7] (1.3160, 0.2040, 3.1545) -- (1.3620, 0.2040, 3.1556) -- (1.3620, 0.2580, 3.1615) -- (1.3160, 0.2580, 3.1604) -- cycle;
\fill[blue!53.0, opacity=0.7] (1.3160, 0.2580, 3.1604) -- (1.3620, 0.2580, 3.1615) -- (1.3620, 0.3120, 3.1673) -- (1.3160, 0.3120, 3.1662) -- cycle;
\fill[blue!63.1, opacity=0.7] (1.3160, 0.3120, 3.1662) -- (1.3620, 0.3120, 3.1673) -- (1.3620, 0.3660, 3.1730) -- (1.3160, 0.3660, 3.1719) -- cycle;
\fill[blue!55.4, opacity=0.7] (1.3160, 0.3660, 3.1719) -- (1.3620, 0.3660, 3.1730) -- (1.3620, 0.4200, 3.1785) -- (1.3160, 0.4200, 3.1774) -- cycle;
\fill[blue!42.2, opacity=0.7] (1.3160, 0.4200, 3.1774) -- (1.3620, 0.4200, 3.1785) -- (1.3620, 0.4740, 3.1839) -- (1.3160, 0.4740, 3.1827) -- cycle;
\fill[blue!45.9, opacity=0.7] (1.3160, 0.4740, 3.1827) -- (1.3620, 0.4740, 3.1839) -- (1.3620, 0.5280, 3.1891) -- (1.3160, 0.5280, 3.1879) -- cycle;
\fill[blue!62.3, opacity=0.7] (1.3160, 0.5280, 3.1879) -- (1.3620, 0.5280, 3.1891) -- (1.3620, 0.5820, 3.1940) -- (1.3160, 0.5820, 3.1929) -- cycle;
\fill[blue!51.5, opacity=0.7] (1.3160, 0.5820, 3.1929) -- (1.3620, 0.5820, 3.1940) -- (1.3620, 0.6360, 3.1988) -- (1.3160, 0.6360, 3.1977) -- cycle;
\fill[blue!30.1, opacity=0.7] (1.3160, 0.6360, 3.1977) -- (1.3620, 0.6360, 3.1988) -- (1.3620, 0.6900, 3.2034) -- (1.3160, 0.6900, 3.2022) -- cycle;
\fill[blue!25.9, opacity=0.7] (1.3160, 0.6900, 3.2022) -- (1.3620, 0.6900, 3.2034) -- (1.3620, 0.7440, 3.2077) -- (1.3160, 0.7440, 3.2066) -- cycle;
\fill[blue!35.7, opacity=0.7] (1.3160, 0.7440, 3.2066) -- (1.3620, 0.7440, 3.2077) -- (1.3620, 0.7980, 3.2118) -- (1.3160, 0.7980, 3.2106) -- cycle;
\fill[blue!56.7, opacity=0.7] (1.3160, 0.7980, 3.2106) -- (1.3620, 0.7980, 3.2118) -- (1.3620, 0.8520, 3.2156) -- (1.3160, 0.8520, 3.2145) -- cycle;
\fill[blue!63.5, opacity=0.7] (1.3160, 0.8520, 3.2145) -- (1.3620, 0.8520, 3.2156) -- (1.3620, 0.9060, 3.2192) -- (1.3160, 0.9060, 3.2180) -- cycle;
\fill[blue!63.3, opacity=0.7] (1.3160, 0.9060, 3.2180) -- (1.3620, 0.9060, 3.2192) -- (1.3620, 0.9600, 3.2224) -- (1.3160, 0.9600, 3.2213) -- cycle;
\fill[blue!58.0, opacity=0.7] (1.3160, 0.9600, 3.2213) -- (1.3620, 0.9600, 3.2224) -- (1.3620, 1.0140, 3.2254) -- (1.3160, 1.0140, 3.2243) -- cycle;
\fill[blue!31.4, opacity=0.7] (1.3160, 1.0140, 3.2243) -- (1.3620, 1.0140, 3.2254) -- (1.3620, 1.0680, 3.2281) -- (1.3160, 1.0680, 3.2270) -- cycle;
\fill[blue!18.1, opacity=0.7] (1.3160, 1.0680, 3.2270) -- (1.3620, 1.0680, 3.2281) -- (1.3620, 1.1220, 3.2306) -- (1.3160, 1.1220, 3.2294) -- cycle;
\fill[blue!16.5, opacity=0.7] (1.3160, 1.1220, 3.2294) -- (1.3620, 1.1220, 3.2306) -- (1.3620, 1.1760, 3.2326) -- (1.3160, 1.1760, 3.2315) -- cycle;
\fill[blue!18.7, opacity=0.7] (1.3160, 1.1760, 3.2315) -- (1.3620, 1.1760, 3.2326) -- (1.3620, 1.2300, 3.2344) -- (1.3160, 1.2300, 3.2333) -- cycle;
\fill[blue!31.6, opacity=0.7] (1.3160, 1.2300, 3.2333) -- (1.3620, 1.2300, 3.2344) -- (1.3620, 1.2840, 3.2359) -- (1.3160, 1.2840, 3.2348) -- cycle;
\fill[blue!50.8, opacity=0.7] (1.3160, 1.2840, 3.2348) -- (1.3620, 1.2840, 3.2359) -- (1.3620, 1.3380, 3.2370) -- (1.3160, 1.3380, 3.2359) -- cycle;
\fill[blue!52.8, opacity=0.7] (1.3160, 1.3380, 3.2359) -- (1.3620, 1.3380, 3.2370) -- (1.3620, 1.3920, 3.2379) -- (1.3160, 1.3920, 3.2367) -- cycle;
\fill[blue!35.4, opacity=0.7] (1.3160, 1.3920, 3.2367) -- (1.3620, 1.3920, 3.2379) -- (1.3620, 1.4460, 3.2384) -- (1.3160, 1.4460, 3.2372) -- cycle;
\fill[blue!19.5, opacity=0.7] (1.3160, 1.4460, 3.2372) -- (1.3620, 1.4460, 3.2384) -- (1.3620, 1.5000, 3.2385) -- (1.3160, 1.5000, 3.2374) -- cycle;
\fill[blue!16.1, opacity=0.7] (1.3160, 1.5000, 3.2374) -- (1.3620, 1.5000, 3.2385) -- (1.3620, 1.5540, 3.2384) -- (1.3160, 1.5540, 3.2372) -- cycle;
\fill[blue!15.6, opacity=0.7] (1.3160, 1.5540, 3.2372) -- (1.3620, 1.5540, 3.2384) -- (1.3620, 1.6080, 3.2379) -- (1.3160, 1.6080, 3.2367) -- cycle;
\fill[blue!15.7, opacity=0.7] (1.3160, 1.6080, 3.2367) -- (1.3620, 1.6080, 3.2379) -- (1.3620, 1.6620, 3.2370) -- (1.3160, 1.6620, 3.2359) -- cycle;
\fill[blue!16.1, opacity=0.7] (1.3160, 1.6620, 3.2359) -- (1.3620, 1.6620, 3.2370) -- (1.3620, 1.7160, 3.2359) -- (1.3160, 1.7160, 3.2348) -- cycle;
\fill[blue!17.5, opacity=0.7] (1.3160, 1.7160, 3.2348) -- (1.3620, 1.7160, 3.2359) -- (1.3620, 1.7700, 3.2344) -- (1.3160, 1.7700, 3.2333) -- cycle;
\fill[blue!23.1, opacity=0.7] (1.3160, 1.7700, 3.2333) -- (1.3620, 1.7700, 3.2344) -- (1.3620, 1.8240, 3.2326) -- (1.3160, 1.8240, 3.2315) -- cycle;
\fill[blue!38.5, opacity=0.7] (1.3160, 1.8240, 3.2315) -- (1.3620, 1.8240, 3.2326) -- (1.3620, 1.8780, 3.2306) -- (1.3160, 1.8780, 3.2294) -- cycle;
\fill[blue!57.0, opacity=0.7] (1.3160, 1.8780, 3.2294) -- (1.3620, 1.8780, 3.2306) -- (1.3620, 1.9320, 3.2281) -- (1.3160, 1.9320, 3.2270) -- cycle;
\fill[blue!63.3, opacity=0.7] (1.3160, 1.9320, 3.2270) -- (1.3620, 1.9320, 3.2281) -- (1.3620, 1.9860, 3.2254) -- (1.3160, 1.9860, 3.2243) -- cycle;
\fill[blue!63.3, opacity=0.7] (1.3160, 1.9860, 3.2243) -- (1.3620, 1.9860, 3.2254) -- (1.3620, 2.0400, 3.2224) -- (1.3160, 2.0400, 3.2213) -- cycle;
\fill[blue!57.7, opacity=0.7] (1.3160, 2.0400, 3.2213) -- (1.3620, 2.0400, 3.2224) -- (1.3620, 2.0940, 3.2192) -- (1.3160, 2.0940, 3.2180) -- cycle;
\fill[blue!40.0, opacity=0.7] (1.3160, 2.0940, 3.2180) -- (1.3620, 2.0940, 3.2192) -- (1.3620, 2.1480, 3.2156) -- (1.3160, 2.1480, 3.2145) -- cycle;
\fill[blue!26.1, opacity=0.7] (1.3160, 2.1480, 3.2145) -- (1.3620, 2.1480, 3.2156) -- (1.3620, 2.2020, 3.2118) -- (1.3160, 2.2020, 3.2106) -- cycle;
\fill[blue!23.1, opacity=0.7] (1.3160, 2.2020, 3.2106) -- (1.3620, 2.2020, 3.2118) -- (1.3620, 2.2560, 3.2077) -- (1.3160, 2.2560, 3.2066) -- cycle;
\fill[blue!30.2, opacity=0.7] (1.3160, 2.2560, 3.2066) -- (1.3620, 2.2560, 3.2077) -- (1.3620, 2.3100, 3.2034) -- (1.3160, 2.3100, 3.2022) -- cycle;
\fill[blue!52.4, opacity=0.7] (1.3160, 2.3100, 3.2022) -- (1.3620, 2.3100, 3.2034) -- (1.3620, 2.3640, 3.1988) -- (1.3160, 2.3640, 3.1977) -- cycle;
\fill[blue!62.6, opacity=0.7] (1.3160, 2.3640, 3.1977) -- (1.3620, 2.3640, 3.1988) -- (1.3620, 2.4180, 3.1940) -- (1.3160, 2.4180, 3.1929) -- cycle;
\fill[blue!48.8, opacity=0.7] (1.3160, 2.4180, 3.1929) -- (1.3620, 2.4180, 3.1940) -- (1.3620, 2.4720, 3.1891) -- (1.3160, 2.4720, 3.1879) -- cycle;
\fill[blue!45.5, opacity=0.7] (1.3160, 2.4720, 3.1879) -- (1.3620, 2.4720, 3.1891) -- (1.3620, 2.5260, 3.1839) -- (1.3160, 2.5260, 3.1827) -- cycle;
\fill[blue!57.1, opacity=0.7] (1.3160, 2.5260, 3.1827) -- (1.3620, 2.5260, 3.1839) -- (1.3620, 2.5800, 3.1785) -- (1.3160, 2.5800, 3.1774) -- cycle;
\fill[blue!62.9, opacity=0.7] (1.3160, 2.5800, 3.1774) -- (1.3620, 2.5800, 3.1785) -- (1.3620, 2.6340, 3.1730) -- (1.3160, 2.6340, 3.1719) -- cycle;
\fill[blue!51.7, opacity=0.7] (1.3160, 2.6340, 3.1719) -- (1.3620, 2.6340, 3.1730) -- (1.3620, 2.6880, 3.1673) -- (1.3160, 2.6880, 3.1662) -- cycle;
\fill[blue!49.0, opacity=0.7] (1.3160, 2.6880, 3.1662) -- (1.3620, 2.6880, 3.1673) -- (1.3620, 2.7420, 3.1615) -- (1.3160, 2.7420, 3.1604) -- cycle;
\fill[blue!60.7, opacity=0.7] (1.3160, 2.7420, 3.1604) -- (1.3620, 2.7420, 3.1615) -- (1.3620, 2.7960, 3.1556) -- (1.3160, 2.7960, 3.1545) -- cycle;
\fill[blue!56.6, opacity=0.7] (1.3160, 2.7960, 3.1545) -- (1.3620, 2.7960, 3.1556) -- (1.3620, 2.8500, 3.1496) -- (1.3160, 2.8500, 3.1484) -- cycle;
\fill[blue!31.7, opacity=0.7] (1.3160, 2.8500, 3.1484) -- (1.3620, 2.8500, 3.1496) -- (1.3620, 2.9040, 3.1435) -- (1.3160, 2.9040, 3.1423) -- cycle;
\fill[blue!22.1, opacity=0.7] (1.3160, 2.9040, 3.1423) -- (1.3620, 2.9040, 3.1435) -- (1.3620, 2.9580, 3.1373) -- (1.3160, 2.9580, 3.1361) -- cycle;
\fill[blue!23.5, opacity=0.7] (1.3160, 2.9580, 3.1361) -- (1.3620, 2.9580, 3.1373) -- (1.3620, 3.0120, 3.1311) -- (1.3160, 3.0120, 3.1299) -- cycle;
\fill[blue!36.6, opacity=0.7] (1.3160, 3.0120, 3.1299) -- (1.3620, 3.0120, 3.1311) -- (1.3620, 3.0660, 3.1248) -- (1.3160, 3.0660, 3.1237) -- cycle;
\fill[blue!56.7, opacity=0.7] (1.3160, 3.0660, 3.1237) -- (1.3620, 3.0660, 3.1248) -- (1.3620, 3.1200, 3.1185) -- (1.3160, 3.1200, 3.1174) -- cycle;
\fill[blue!42.1, opacity=0.7] (1.3620, -0.1200, 3.1185) -- (1.4080, -0.1200, 3.1193) -- (1.4080, -0.0660, 3.1256) -- (1.3620, -0.0660, 3.1248) -- cycle;
\fill[blue!24.9, opacity=0.7] (1.3620, -0.0660, 3.1248) -- (1.4080, -0.0660, 3.1256) -- (1.4080, -0.0120, 3.1319) -- (1.3620, -0.0120, 3.1311) -- cycle;
\fill[blue!20.7, opacity=0.7] (1.3620, -0.0120, 3.1311) -- (1.4080, -0.0120, 3.1319) -- (1.4080, 0.0420, 3.1381) -- (1.3620, 0.0420, 3.1373) -- cycle;
\fill[blue!26.0, opacity=0.7] (1.3620, 0.0420, 3.1373) -- (1.4080, 0.0420, 3.1381) -- (1.4080, 0.0960, 3.1443) -- (1.3620, 0.0960, 3.1435) -- cycle;
\fill[blue!48.4, opacity=0.7] (1.3620, 0.0960, 3.1435) -- (1.4080, 0.0960, 3.1443) -- (1.4080, 0.1500, 3.1504) -- (1.3620, 0.1500, 3.1496) -- cycle;
\fill[blue!63.3, opacity=0.7] (1.3620, 0.1500, 3.1496) -- (1.4080, 0.1500, 3.1504) -- (1.4080, 0.2040, 3.1564) -- (1.3620, 0.2040, 3.1556) -- cycle;
\fill[blue!52.5, opacity=0.7] (1.3620, 0.2040, 3.1556) -- (1.4080, 0.2040, 3.1564) -- (1.4080, 0.2580, 3.1623) -- (1.3620, 0.2580, 3.1615) -- cycle;
\fill[blue!52.7, opacity=0.7] (1.3620, 0.2580, 3.1615) -- (1.4080, 0.2580, 3.1623) -- (1.4080, 0.3120, 3.1682) -- (1.3620, 0.3120, 3.1673) -- cycle;
\fill[blue!63.0, opacity=0.7] (1.3620, 0.3120, 3.1673) -- (1.4080, 0.3120, 3.1682) -- (1.4080, 0.3660, 3.1738) -- (1.3620, 0.3660, 3.1730) -- cycle;
\fill[blue!55.5, opacity=0.7] (1.3620, 0.3660, 3.1730) -- (1.4080, 0.3660, 3.1738) -- (1.4080, 0.4200, 3.1793) -- (1.3620, 0.4200, 3.1785) -- cycle;
\fill[blue!42.5, opacity=0.7] (1.3620, 0.4200, 3.1785) -- (1.4080, 0.4200, 3.1793) -- (1.4080, 0.4740, 3.1847) -- (1.3620, 0.4740, 3.1839) -- cycle;
\fill[blue!46.6, opacity=0.7] (1.3620, 0.4740, 3.1839) -- (1.4080, 0.4740, 3.1847) -- (1.4080, 0.5280, 3.1899) -- (1.3620, 0.5280, 3.1891) -- cycle;
\fill[blue!62.7, opacity=0.7] (1.3620, 0.5280, 3.1891) -- (1.4080, 0.5280, 3.1899) -- (1.4080, 0.5820, 3.1949) -- (1.3620, 0.5820, 3.1940) -- cycle;
\fill[blue!50.4, opacity=0.7] (1.3620, 0.5820, 3.1940) -- (1.4080, 0.5820, 3.1949) -- (1.4080, 0.6360, 3.1996) -- (1.3620, 0.6360, 3.1988) -- cycle;
\fill[blue!29.3, opacity=0.7] (1.3620, 0.6360, 3.1988) -- (1.4080, 0.6360, 3.1996) -- (1.4080, 0.6900, 3.2042) -- (1.3620, 0.6900, 3.2034) -- cycle;
\fill[blue!25.5, opacity=0.7] (1.3620, 0.6900, 3.2034) -- (1.4080, 0.6900, 3.2042) -- (1.4080, 0.7440, 3.2085) -- (1.3620, 0.7440, 3.2077) -- cycle;
\fill[blue!35.6, opacity=0.7] (1.3620, 0.7440, 3.2077) -- (1.4080, 0.7440, 3.2085) -- (1.4080, 0.7980, 3.2126) -- (1.3620, 0.7980, 3.2118) -- cycle;
\fill[blue!56.7, opacity=0.7] (1.3620, 0.7980, 3.2118) -- (1.4080, 0.7980, 3.2126) -- (1.4080, 0.8520, 3.2164) -- (1.3620, 0.8520, 3.2156) -- cycle;
\fill[blue!63.5, opacity=0.7] (1.3620, 0.8520, 3.2156) -- (1.4080, 0.8520, 3.2164) -- (1.4080, 0.9060, 3.2200) -- (1.3620, 0.9060, 3.2192) -- cycle;
\fill[blue!63.5, opacity=0.7] (1.3620, 0.9060, 3.2192) -- (1.4080, 0.9060, 3.2200) -- (1.4080, 0.9600, 3.2233) -- (1.3620, 0.9600, 3.2224) -- cycle;
\fill[blue!56.4, opacity=0.7] (1.3620, 0.9600, 3.2224) -- (1.4080, 0.9600, 3.2233) -- (1.4080, 1.0140, 3.2263) -- (1.3620, 1.0140, 3.2254) -- cycle;
\fill[blue!29.4, opacity=0.7] (1.3620, 1.0140, 3.2254) -- (1.4080, 1.0140, 3.2263) -- (1.4080, 1.0680, 3.2290) -- (1.3620, 1.0680, 3.2281) -- cycle;
\fill[blue!17.6, opacity=0.7] (1.3620, 1.0680, 3.2281) -- (1.4080, 1.0680, 3.2290) -- (1.4080, 1.1220, 3.2314) -- (1.3620, 1.1220, 3.2306) -- cycle;
\fill[blue!16.3, opacity=0.7] (1.3620, 1.1220, 3.2306) -- (1.4080, 1.1220, 3.2314) -- (1.4080, 1.1760, 3.2335) -- (1.3620, 1.1760, 3.2326) -- cycle;
\fill[blue!18.9, opacity=0.7] (1.3620, 1.1760, 3.2326) -- (1.4080, 1.1760, 3.2335) -- (1.4080, 1.2300, 3.2353) -- (1.3620, 1.2300, 3.2344) -- cycle;
\fill[blue!33.0, opacity=0.7] (1.3620, 1.2300, 3.2344) -- (1.4080, 1.2300, 3.2353) -- (1.4080, 1.2840, 3.2367) -- (1.3620, 1.2840, 3.2359) -- cycle;
\fill[blue!51.1, opacity=0.7] (1.3620, 1.2840, 3.2359) -- (1.4080, 1.2840, 3.2367) -- (1.4080, 1.3380, 3.2379) -- (1.3620, 1.3380, 3.2370) -- cycle;
\fill[blue!47.4, opacity=0.7] (1.3620, 1.3380, 3.2370) -- (1.4080, 1.3380, 3.2379) -- (1.4080, 1.3920, 3.2387) -- (1.3620, 1.3920, 3.2379) -- cycle;
\fill[blue!23.5, opacity=0.7] (1.3620, 1.3920, 3.2379) -- (1.4080, 1.3920, 3.2387) -- (1.4080, 1.4460, 3.2392) -- (1.3620, 1.4460, 3.2384) -- cycle;
\fill[blue!15.7, opacity=0.7] (1.3620, 1.4460, 3.2384) -- (1.4080, 1.4460, 3.2392) -- (1.4080, 1.5000, 3.2393) -- (1.3620, 1.5000, 3.2385) -- cycle;
\fill[blue!15.3, opacity=0.7] (1.3620, 1.5000, 3.2385) -- (1.4080, 1.5000, 3.2393) -- (1.4080, 1.5540, 3.2392) -- (1.3620, 1.5540, 3.2384) -- cycle;
\fill[blue!15.4, opacity=0.7] (1.3620, 1.5540, 3.2384) -- (1.4080, 1.5540, 3.2392) -- (1.4080, 1.6080, 3.2387) -- (1.3620, 1.6080, 3.2379) -- cycle;
\fill[blue!15.6, opacity=0.7] (1.3620, 1.6080, 3.2379) -- (1.4080, 1.6080, 3.2387) -- (1.4080, 1.6620, 3.2379) -- (1.3620, 1.6620, 3.2370) -- cycle;
\fill[blue!15.6, opacity=0.7] (1.3620, 1.6620, 3.2370) -- (1.4080, 1.6620, 3.2379) -- (1.4080, 1.7160, 3.2367) -- (1.3620, 1.7160, 3.2359) -- cycle;
\fill[blue!15.7, opacity=0.7] (1.3620, 1.7160, 3.2359) -- (1.4080, 1.7160, 3.2367) -- (1.4080, 1.7700, 3.2353) -- (1.3620, 1.7700, 3.2344) -- cycle;
\fill[blue!16.8, opacity=0.7] (1.3620, 1.7700, 3.2344) -- (1.4080, 1.7700, 3.2353) -- (1.4080, 1.8240, 3.2335) -- (1.3620, 1.8240, 3.2326) -- cycle;
\fill[blue!22.8, opacity=0.7] (1.3620, 1.8240, 3.2326) -- (1.4080, 1.8240, 3.2335) -- (1.4080, 1.8780, 3.2314) -- (1.3620, 1.8780, 3.2306) -- cycle;
\fill[blue!41.9, opacity=0.7] (1.3620, 1.8780, 3.2306) -- (1.4080, 1.8780, 3.2314) -- (1.4080, 1.9320, 3.2290) -- (1.3620, 1.9320, 3.2281) -- cycle;
\fill[blue!60.4, opacity=0.7] (1.3620, 1.9320, 3.2281) -- (1.4080, 1.9320, 3.2290) -- (1.4080, 1.9860, 3.2263) -- (1.3620, 1.9860, 3.2254) -- cycle;
\fill[blue!63.6, opacity=0.7] (1.3620, 1.9860, 3.2254) -- (1.4080, 1.9860, 3.2263) -- (1.4080, 2.0400, 3.2233) -- (1.3620, 2.0400, 3.2224) -- cycle;
\fill[blue!62.7, opacity=0.7] (1.3620, 2.0400, 3.2224) -- (1.4080, 2.0400, 3.2233) -- (1.4080, 2.0940, 3.2200) -- (1.3620, 2.0940, 3.2192) -- cycle;
\fill[blue!51.1, opacity=0.7] (1.3620, 2.0940, 3.2192) -- (1.4080, 2.0940, 3.2200) -- (1.4080, 2.1480, 3.2164) -- (1.3620, 2.1480, 3.2156) -- cycle;
\fill[blue!31.9, opacity=0.7] (1.3620, 2.1480, 3.2156) -- (1.4080, 2.1480, 3.2164) -- (1.4080, 2.2020, 3.2126) -- (1.3620, 2.2020, 3.2118) -- cycle;
\fill[blue!23.7, opacity=0.7] (1.3620, 2.2020, 3.2118) -- (1.4080, 2.2020, 3.2126) -- (1.4080, 2.2560, 3.2085) -- (1.3620, 2.2560, 3.2077) -- cycle;
\fill[blue!26.1, opacity=0.7] (1.3620, 2.2560, 3.2077) -- (1.4080, 2.2560, 3.2085) -- (1.4080, 2.3100, 3.2042) -- (1.3620, 2.3100, 3.2034) -- cycle;
\fill[blue!43.2, opacity=0.7] (1.3620, 2.3100, 3.2034) -- (1.4080, 2.3100, 3.2042) -- (1.4080, 2.3640, 3.1996) -- (1.3620, 2.3640, 3.1988) -- cycle;
\fill[blue!63.3, opacity=0.7] (1.3620, 2.3640, 3.1988) -- (1.4080, 2.3640, 3.1996) -- (1.4080, 2.4180, 3.1949) -- (1.3620, 2.4180, 3.1940) -- cycle;
\fill[blue!53.0, opacity=0.7] (1.3620, 2.4180, 3.1940) -- (1.4080, 2.4180, 3.1949) -- (1.4080, 2.4720, 3.1899) -- (1.3620, 2.4720, 3.1891) -- cycle;
\fill[blue!44.1, opacity=0.7] (1.3620, 2.4720, 3.1891) -- (1.4080, 2.4720, 3.1899) -- (1.4080, 2.5260, 3.1847) -- (1.3620, 2.5260, 3.1839) -- cycle;
\fill[blue!52.3, opacity=0.7] (1.3620, 2.5260, 3.1839) -- (1.4080, 2.5260, 3.1847) -- (1.4080, 2.5800, 3.1793) -- (1.3620, 2.5800, 3.1785) -- cycle;
\fill[blue!63.5, opacity=0.7] (1.3620, 2.5800, 3.1785) -- (1.4080, 2.5800, 3.1793) -- (1.4080, 2.6340, 3.1738) -- (1.3620, 2.6340, 3.1730) -- cycle;
\fill[blue!55.2, opacity=0.7] (1.3620, 2.6340, 3.1730) -- (1.4080, 2.6340, 3.1738) -- (1.4080, 2.6880, 3.1682) -- (1.3620, 2.6880, 3.1673) -- cycle;
\fill[blue!48.4, opacity=0.7] (1.3620, 2.6880, 3.1673) -- (1.4080, 2.6880, 3.1682) -- (1.4080, 2.7420, 3.1623) -- (1.3620, 2.7420, 3.1615) -- cycle;
\fill[blue!57.8, opacity=0.7] (1.3620, 2.7420, 3.1615) -- (1.4080, 2.7420, 3.1623) -- (1.4080, 2.7960, 3.1564) -- (1.3620, 2.7960, 3.1556) -- cycle;
\fill[blue!60.9, opacity=0.7] (1.3620, 2.7960, 3.1556) -- (1.4080, 2.7960, 3.1564) -- (1.4080, 2.8500, 3.1504) -- (1.3620, 2.8500, 3.1496) -- cycle;
\fill[blue!36.1, opacity=0.7] (1.3620, 2.8500, 3.1496) -- (1.4080, 2.8500, 3.1504) -- (1.4080, 2.9040, 3.1443) -- (1.3620, 2.9040, 3.1435) -- cycle;
\fill[blue!22.7, opacity=0.7] (1.3620, 2.9040, 3.1435) -- (1.4080, 2.9040, 3.1443) -- (1.4080, 2.9580, 3.1381) -- (1.3620, 2.9580, 3.1373) -- cycle;
\fill[blue!22.3, opacity=0.7] (1.3620, 2.9580, 3.1373) -- (1.4080, 2.9580, 3.1381) -- (1.4080, 3.0120, 3.1319) -- (1.3620, 3.0120, 3.1311) -- cycle;
\fill[blue!32.5, opacity=0.7] (1.3620, 3.0120, 3.1311) -- (1.4080, 3.0120, 3.1319) -- (1.4080, 3.0660, 3.1256) -- (1.3620, 3.0660, 3.1248) -- cycle;
\fill[blue!53.2, opacity=0.7] (1.3620, 3.0660, 3.1248) -- (1.4080, 3.0660, 3.1256) -- (1.4080, 3.1200, 3.1193) -- (1.3620, 3.1200, 3.1185) -- cycle;
\fill[blue!42.9, opacity=0.7] (1.4080, -0.1200, 3.1193) -- (1.4540, -0.1200, 3.1198) -- (1.4540, -0.0660, 3.1261) -- (1.4080, -0.0660, 3.1256) -- cycle;
\fill[blue!25.4, opacity=0.7] (1.4080, -0.0660, 3.1256) -- (1.4540, -0.0660, 3.1261) -- (1.4540, -0.0120, 3.1324) -- (1.4080, -0.0120, 3.1319) -- cycle;
\fill[blue!20.8, opacity=0.7] (1.4080, -0.0120, 3.1319) -- (1.4540, -0.0120, 3.1324) -- (1.4540, 0.0420, 3.1386) -- (1.4080, 0.0420, 3.1381) -- cycle;
\fill[blue!26.0, opacity=0.7] (1.4080, 0.0420, 3.1381) -- (1.4540, 0.0420, 3.1386) -- (1.4540, 0.0960, 3.1448) -- (1.4080, 0.0960, 3.1443) -- cycle;
\fill[blue!48.1, opacity=0.7] (1.4080, 0.0960, 3.1443) -- (1.4540, 0.0960, 3.1448) -- (1.4540, 0.1500, 3.1509) -- (1.4080, 0.1500, 3.1504) -- cycle;
\fill[blue!63.4, opacity=0.7] (1.4080, 0.1500, 3.1504) -- (1.4540, 0.1500, 3.1509) -- (1.4540, 0.2040, 3.1569) -- (1.4080, 0.2040, 3.1564) -- cycle;
\fill[blue!52.4, opacity=0.7] (1.4080, 0.2040, 3.1564) -- (1.4540, 0.2040, 3.1569) -- (1.4540, 0.2580, 3.1628) -- (1.4080, 0.2580, 3.1623) -- cycle;
\fill[blue!52.1, opacity=0.7] (1.4080, 0.2580, 3.1623) -- (1.4540, 0.2580, 3.1628) -- (1.4540, 0.3120, 3.1686) -- (1.4080, 0.3120, 3.1682) -- cycle;
\fill[blue!62.8, opacity=0.7] (1.4080, 0.3120, 3.1682) -- (1.4540, 0.3120, 3.1686) -- (1.4540, 0.3660, 3.1743) -- (1.4080, 0.3660, 3.1738) -- cycle;
\fill[blue!56.4, opacity=0.7] (1.4080, 0.3660, 3.1738) -- (1.4540, 0.3660, 3.1743) -- (1.4540, 0.4200, 3.1798) -- (1.4080, 0.4200, 3.1793) -- cycle;
\fill[blue!43.2, opacity=0.7] (1.4080, 0.4200, 3.1793) -- (1.4540, 0.4200, 3.1798) -- (1.4540, 0.4740, 3.1852) -- (1.4080, 0.4740, 3.1847) -- cycle;
\fill[blue!46.7, opacity=0.7] (1.4080, 0.4740, 3.1847) -- (1.4540, 0.4740, 3.1852) -- (1.4540, 0.5280, 3.1904) -- (1.4080, 0.5280, 3.1899) -- cycle;
\fill[blue!62.5, opacity=0.7] (1.4080, 0.5280, 3.1899) -- (1.4540, 0.5280, 3.1904) -- (1.4540, 0.5820, 3.1954) -- (1.4080, 0.5820, 3.1949) -- cycle;
\fill[blue!50.9, opacity=0.7] (1.4080, 0.5820, 3.1949) -- (1.4540, 0.5820, 3.1954) -- (1.4540, 0.6360, 3.2001) -- (1.4080, 0.6360, 3.1996) -- cycle;
\fill[blue!29.4, opacity=0.7] (1.4080, 0.6360, 3.1996) -- (1.4540, 0.6360, 3.2001) -- (1.4540, 0.6900, 3.2047) -- (1.4080, 0.6900, 3.2042) -- cycle;
\fill[blue!25.0, opacity=0.7] (1.4080, 0.6900, 3.2042) -- (1.4540, 0.6900, 3.2047) -- (1.4540, 0.7440, 3.2090) -- (1.4080, 0.7440, 3.2085) -- cycle;
\fill[blue!34.2, opacity=0.7] (1.4080, 0.7440, 3.2085) -- (1.4540, 0.7440, 3.2090) -- (1.4540, 0.7980, 3.2131) -- (1.4080, 0.7980, 3.2126) -- cycle;
\fill[blue!55.2, opacity=0.7] (1.4080, 0.7980, 3.2126) -- (1.4540, 0.7980, 3.2131) -- (1.4540, 0.8520, 3.2169) -- (1.4080, 0.8520, 3.2164) -- cycle;
\fill[blue!63.6, opacity=0.7] (1.4080, 0.8520, 3.2164) -- (1.4540, 0.8520, 3.2169) -- (1.4540, 0.9060, 3.2205) -- (1.4080, 0.9060, 3.2200) -- cycle;
\fill[blue!63.5, opacity=0.7] (1.4080, 0.9060, 3.2200) -- (1.4540, 0.9060, 3.2205) -- (1.4540, 0.9600, 3.2238) -- (1.4080, 0.9600, 3.2233) -- cycle;
\fill[blue!56.9, opacity=0.7] (1.4080, 0.9600, 3.2233) -- (1.4540, 0.9600, 3.2238) -- (1.4540, 1.0140, 3.2268) -- (1.4080, 1.0140, 3.2263) -- cycle;
\fill[blue!30.2, opacity=0.7] (1.4080, 1.0140, 3.2263) -- (1.4540, 1.0140, 3.2268) -- (1.4540, 1.0680, 3.2295) -- (1.4080, 1.0680, 3.2290) -- cycle;
\fill[blue!17.7, opacity=0.7] (1.4080, 1.0680, 3.2290) -- (1.4540, 1.0680, 3.2295) -- (1.4540, 1.1220, 3.2319) -- (1.4080, 1.1220, 3.2314) -- cycle;
\fill[blue!16.2, opacity=0.7] (1.4080, 1.1220, 3.2314) -- (1.4540, 1.1220, 3.2319) -- (1.4540, 1.1760, 3.2340) -- (1.4080, 1.1760, 3.2335) -- cycle;
\fill[blue!18.1, opacity=0.7] (1.4080, 1.1760, 3.2335) -- (1.4540, 1.1760, 3.2340) -- (1.4540, 1.2300, 3.2357) -- (1.4080, 1.2300, 3.2353) -- cycle;
\fill[blue!30.0, opacity=0.7] (1.4080, 1.2300, 3.2353) -- (1.4540, 1.2300, 3.2357) -- (1.4540, 1.2840, 3.2372) -- (1.4080, 1.2840, 3.2367) -- cycle;
\fill[blue!48.5, opacity=0.7] (1.4080, 1.2840, 3.2367) -- (1.4540, 1.2840, 3.2372) -- (1.4540, 1.3380, 3.2384) -- (1.4080, 1.3380, 3.2379) -- cycle;
\fill[blue!45.6, opacity=0.7] (1.4080, 1.3380, 3.2379) -- (1.4540, 1.3380, 3.2384) -- (1.4540, 1.3920, 3.2392) -- (1.4080, 1.3920, 3.2387) -- cycle;
\fill[blue!20.5, opacity=0.7] (1.4080, 1.3920, 3.2387) -- (1.4540, 1.3920, 3.2392) -- (1.4540, 1.4460, 3.2397) -- (1.4080, 1.4460, 3.2392) -- cycle;
\fill[blue!15.2, opacity=0.7] (1.4080, 1.4460, 3.2392) -- (1.4540, 1.4460, 3.2397) -- (1.4540, 1.5000, 3.2398) -- (1.4080, 1.5000, 3.2393) -- cycle;
\fill[blue!16.0, opacity=0.7] (1.4080, 1.5000, 3.2393) -- (1.4540, 1.5000, 3.2398) -- (1.4540, 1.5540, 3.2397) -- (1.4080, 1.5540, 3.2392) -- cycle;
\fill[blue!20.0, opacity=0.7] (1.4080, 1.5540, 3.2392) -- (1.4540, 1.5540, 3.2397) -- (1.4540, 1.6080, 3.2392) -- (1.4080, 1.6080, 3.2387) -- cycle;
\fill[blue!19.7, opacity=0.7] (1.4080, 1.6080, 3.2387) -- (1.4540, 1.6080, 3.2392) -- (1.4540, 1.6620, 3.2384) -- (1.4080, 1.6620, 3.2379) -- cycle;
\fill[blue!17.1, opacity=0.7] (1.4080, 1.6620, 3.2379) -- (1.4540, 1.6620, 3.2384) -- (1.4540, 1.7160, 3.2372) -- (1.4080, 1.7160, 3.2367) -- cycle;
\fill[blue!15.9, opacity=0.7] (1.4080, 1.7160, 3.2367) -- (1.4540, 1.7160, 3.2372) -- (1.4540, 1.7700, 3.2357) -- (1.4080, 1.7700, 3.2353) -- cycle;
\fill[blue!15.8, opacity=0.7] (1.4080, 1.7700, 3.2353) -- (1.4540, 1.7700, 3.2357) -- (1.4540, 1.8240, 3.2340) -- (1.4080, 1.8240, 3.2335) -- cycle;
\fill[blue!17.5, opacity=0.7] (1.4080, 1.8240, 3.2335) -- (1.4540, 1.8240, 3.2340) -- (1.4540, 1.8780, 3.2319) -- (1.4080, 1.8780, 3.2314) -- cycle;
\fill[blue!27.8, opacity=0.7] (1.4080, 1.8780, 3.2314) -- (1.4540, 1.8780, 3.2319) -- (1.4540, 1.9320, 3.2295) -- (1.4080, 1.9320, 3.2290) -- cycle;
\fill[blue!52.1, opacity=0.7] (1.4080, 1.9320, 3.2290) -- (1.4540, 1.9320, 3.2295) -- (1.4540, 1.9860, 3.2268) -- (1.4080, 1.9860, 3.2263) -- cycle;
\fill[blue!63.3, opacity=0.7] (1.4080, 1.9860, 3.2263) -- (1.4540, 1.9860, 3.2268) -- (1.4540, 2.0400, 3.2238) -- (1.4080, 2.0400, 3.2233) -- cycle;
\fill[blue!63.6, opacity=0.7] (1.4080, 2.0400, 3.2233) -- (1.4540, 2.0400, 3.2238) -- (1.4540, 2.0940, 3.2205) -- (1.4080, 2.0940, 3.2200) -- cycle;
\fill[blue!58.7, opacity=0.7] (1.4080, 2.0940, 3.2200) -- (1.4540, 2.0940, 3.2205) -- (1.4540, 2.1480, 3.2169) -- (1.4080, 2.1480, 3.2164) -- cycle;
\fill[blue!39.4, opacity=0.7] (1.4080, 2.1480, 3.2164) -- (1.4540, 2.1480, 3.2169) -- (1.4540, 2.2020, 3.2131) -- (1.4080, 2.2020, 3.2126) -- cycle;
\fill[blue!25.7, opacity=0.7] (1.4080, 2.2020, 3.2126) -- (1.4540, 2.2020, 3.2131) -- (1.4540, 2.2560, 3.2090) -- (1.4080, 2.2560, 3.2085) -- cycle;
\fill[blue!24.5, opacity=0.7] (1.4080, 2.2560, 3.2085) -- (1.4540, 2.2560, 3.2090) -- (1.4540, 2.3100, 3.2047) -- (1.4080, 2.3100, 3.2042) -- cycle;
\fill[blue!36.5, opacity=0.7] (1.4080, 2.3100, 3.2042) -- (1.4540, 2.3100, 3.2047) -- (1.4540, 2.3640, 3.2001) -- (1.4080, 2.3640, 3.1996) -- cycle;
\fill[blue!60.6, opacity=0.7] (1.4080, 2.3640, 3.1996) -- (1.4540, 2.3640, 3.2001) -- (1.4540, 2.4180, 3.1954) -- (1.4080, 2.4180, 3.1949) -- cycle;
\fill[blue!57.0, opacity=0.7] (1.4080, 2.4180, 3.1949) -- (1.4540, 2.4180, 3.1954) -- (1.4540, 2.4720, 3.1904) -- (1.4080, 2.4720, 3.1899) -- cycle;
\fill[blue!44.3, opacity=0.7] (1.4080, 2.4720, 3.1899) -- (1.4540, 2.4720, 3.1904) -- (1.4540, 2.5260, 3.1852) -- (1.4080, 2.5260, 3.1847) -- cycle;
\fill[blue!48.5, opacity=0.7] (1.4080, 2.5260, 3.1847) -- (1.4540, 2.5260, 3.1852) -- (1.4540, 2.5800, 3.1798) -- (1.4080, 2.5800, 3.1793) -- cycle;
\fill[blue!62.2, opacity=0.7] (1.4080, 2.5800, 3.1793) -- (1.4540, 2.5800, 3.1798) -- (1.4540, 2.6340, 3.1743) -- (1.4080, 2.6340, 3.1738) -- cycle;
\fill[blue!58.3, opacity=0.7] (1.4080, 2.6340, 3.1738) -- (1.4540, 2.6340, 3.1743) -- (1.4540, 2.6880, 3.1686) -- (1.4080, 2.6880, 3.1682) -- cycle;
\fill[blue!49.0, opacity=0.7] (1.4080, 2.6880, 3.1682) -- (1.4540, 2.6880, 3.1686) -- (1.4540, 2.7420, 3.1628) -- (1.4080, 2.7420, 3.1623) -- cycle;
\fill[blue!55.3, opacity=0.7] (1.4080, 2.7420, 3.1623) -- (1.4540, 2.7420, 3.1628) -- (1.4540, 2.7960, 3.1569) -- (1.4080, 2.7960, 3.1564) -- cycle;
\fill[blue!62.9, opacity=0.7] (1.4080, 2.7960, 3.1564) -- (1.4540, 2.7960, 3.1569) -- (1.4540, 2.8500, 3.1509) -- (1.4080, 2.8500, 3.1504) -- cycle;
\fill[blue!40.4, opacity=0.7] (1.4080, 2.8500, 3.1504) -- (1.4540, 2.8500, 3.1509) -- (1.4540, 2.9040, 3.1448) -- (1.4080, 2.9040, 3.1443) -- cycle;
\fill[blue!23.7, opacity=0.7] (1.4080, 2.9040, 3.1443) -- (1.4540, 2.9040, 3.1448) -- (1.4540, 2.9580, 3.1386) -- (1.4080, 2.9580, 3.1381) -- cycle;
\fill[blue!21.6, opacity=0.7] (1.4080, 2.9580, 3.1381) -- (1.4540, 2.9580, 3.1386) -- (1.4540, 3.0120, 3.1324) -- (1.4080, 3.0120, 3.1319) -- cycle;
\fill[blue!29.6, opacity=0.7] (1.4080, 3.0120, 3.1319) -- (1.4540, 3.0120, 3.1324) -- (1.4540, 3.0660, 3.1261) -- (1.4080, 3.0660, 3.1256) -- cycle;
\fill[blue!49.8, opacity=0.7] (1.4080, 3.0660, 3.1256) -- (1.4540, 3.0660, 3.1261) -- (1.4540, 3.1200, 3.1198) -- (1.4080, 3.1200, 3.1193) -- cycle;
\fill[blue!44.5, opacity=0.7] (1.4540, -0.1200, 3.1198) -- (1.5000, -0.1200, 3.1200) -- (1.5000, -0.0660, 3.1263) -- (1.4540, -0.0660, 3.1261) -- cycle;
\fill[blue!26.2, opacity=0.7] (1.4540, -0.0660, 3.1261) -- (1.5000, -0.0660, 3.1263) -- (1.5000, -0.0120, 3.1325) -- (1.4540, -0.0120, 3.1324) -- cycle;
\fill[blue!21.0, opacity=0.7] (1.4540, -0.0120, 3.1324) -- (1.5000, -0.0120, 3.1325) -- (1.5000, 0.0420, 3.1388) -- (1.4540, 0.0420, 3.1386) -- cycle;
\fill[blue!25.5, opacity=0.7] (1.4540, 0.0420, 3.1386) -- (1.5000, 0.0420, 3.1388) -- (1.5000, 0.0960, 3.1449) -- (1.4540, 0.0960, 3.1448) -- cycle;
\fill[blue!46.6, opacity=0.7] (1.4540, 0.0960, 3.1448) -- (1.5000, 0.0960, 3.1449) -- (1.5000, 0.1500, 3.1511) -- (1.4540, 0.1500, 3.1509) -- cycle;
\fill[blue!63.5, opacity=0.7] (1.4540, 0.1500, 3.1509) -- (1.5000, 0.1500, 3.1511) -- (1.5000, 0.2040, 3.1571) -- (1.4540, 0.2040, 3.1569) -- cycle;
\fill[blue!52.7, opacity=0.7] (1.4540, 0.2040, 3.1569) -- (1.5000, 0.2040, 3.1571) -- (1.5000, 0.2580, 3.1630) -- (1.4540, 0.2580, 3.1628) -- cycle;
\fill[blue!51.1, opacity=0.7] (1.4540, 0.2580, 3.1628) -- (1.5000, 0.2580, 3.1630) -- (1.5000, 0.3120, 3.1688) -- (1.4540, 0.3120, 3.1686) -- cycle;
\fill[blue!62.0, opacity=0.7] (1.4540, 0.3120, 3.1686) -- (1.5000, 0.3120, 3.1688) -- (1.5000, 0.3660, 3.1745) -- (1.4540, 0.3660, 3.1743) -- cycle;
\fill[blue!58.0, opacity=0.7] (1.4540, 0.3660, 3.1743) -- (1.5000, 0.3660, 3.1745) -- (1.5000, 0.4200, 3.1800) -- (1.4540, 0.4200, 3.1798) -- cycle;
\fill[blue!44.2, opacity=0.7] (1.4540, 0.4200, 3.1798) -- (1.5000, 0.4200, 3.1800) -- (1.5000, 0.4740, 3.1854) -- (1.4540, 0.4740, 3.1852) -- cycle;
\fill[blue!46.1, opacity=0.7] (1.4540, 0.4740, 3.1852) -- (1.5000, 0.4740, 3.1854) -- (1.5000, 0.5280, 3.1905) -- (1.4540, 0.5280, 3.1904) -- cycle;
\fill[blue!61.7, opacity=0.7] (1.4540, 0.5280, 3.1904) -- (1.5000, 0.5280, 3.1905) -- (1.5000, 0.5820, 3.1955) -- (1.4540, 0.5820, 3.1954) -- cycle;
\fill[blue!53.0, opacity=0.7] (1.4540, 0.5820, 3.1954) -- (1.5000, 0.5820, 3.1955) -- (1.5000, 0.6360, 3.2003) -- (1.4540, 0.6360, 3.2001) -- cycle;
\fill[blue!30.3, opacity=0.7] (1.4540, 0.6360, 3.2001) -- (1.5000, 0.6360, 3.2003) -- (1.5000, 0.6900, 3.2049) -- (1.4540, 0.6900, 3.2047) -- cycle;
\fill[blue!24.5, opacity=0.7] (1.4540, 0.6900, 3.2047) -- (1.5000, 0.6900, 3.2049) -- (1.5000, 0.7440, 3.2092) -- (1.4540, 0.7440, 3.2090) -- cycle;
\fill[blue!31.7, opacity=0.7] (1.4540, 0.7440, 3.2090) -- (1.5000, 0.7440, 3.2092) -- (1.5000, 0.7980, 3.2133) -- (1.4540, 0.7980, 3.2131) -- cycle;
\fill[blue!52.0, opacity=0.7] (1.4540, 0.7980, 3.2131) -- (1.5000, 0.7980, 3.2133) -- (1.5000, 0.8520, 3.2171) -- (1.4540, 0.8520, 3.2169) -- cycle;
\fill[blue!63.4, opacity=0.7] (1.4540, 0.8520, 3.2169) -- (1.5000, 0.8520, 3.2171) -- (1.5000, 0.9060, 3.2206) -- (1.4540, 0.9060, 3.2205) -- cycle;
\fill[blue!63.5, opacity=0.7] (1.4540, 0.9060, 3.2205) -- (1.5000, 0.9060, 3.2206) -- (1.5000, 0.9600, 3.2239) -- (1.4540, 0.9600, 3.2238) -- cycle;
\fill[blue!59.2, opacity=0.7] (1.4540, 0.9600, 3.2238) -- (1.5000, 0.9600, 3.2239) -- (1.5000, 1.0140, 3.2269) -- (1.4540, 1.0140, 3.2268) -- cycle;
\fill[blue!34.0, opacity=0.7] (1.4540, 1.0140, 3.2268) -- (1.5000, 1.0140, 3.2269) -- (1.5000, 1.0680, 3.2296) -- (1.4540, 1.0680, 3.2295) -- cycle;
\fill[blue!18.5, opacity=0.7] (1.4540, 1.0680, 3.2295) -- (1.5000, 1.0680, 3.2296) -- (1.5000, 1.1220, 3.2320) -- (1.4540, 1.1220, 3.2319) -- cycle;
\fill[blue!16.1, opacity=0.7] (1.4540, 1.1220, 3.2319) -- (1.5000, 1.1220, 3.2320) -- (1.5000, 1.1760, 3.2341) -- (1.4540, 1.1760, 3.2340) -- cycle;
\fill[blue!16.8, opacity=0.7] (1.4540, 1.1760, 3.2340) -- (1.5000, 1.1760, 3.2341) -- (1.5000, 1.2300, 3.2359) -- (1.4540, 1.2300, 3.2357) -- cycle;
\fill[blue!23.6, opacity=0.7] (1.4540, 1.2300, 3.2357) -- (1.5000, 1.2300, 3.2359) -- (1.5000, 1.2840, 3.2374) -- (1.4540, 1.2840, 3.2372) -- cycle;
\fill[blue!41.0, opacity=0.7] (1.4540, 1.2840, 3.2372) -- (1.5000, 1.2840, 3.2374) -- (1.5000, 1.3380, 3.2385) -- (1.4540, 1.3380, 3.2384) -- cycle;
\fill[blue!48.0, opacity=0.7] (1.4540, 1.3380, 3.2384) -- (1.5000, 1.3380, 3.2385) -- (1.5000, 1.3920, 3.2393) -- (1.4540, 1.3920, 3.2392) -- cycle;
\fill[blue!29.3, opacity=0.7] (1.4540, 1.3920, 3.2392) -- (1.5000, 1.3920, 3.2393) -- (1.5000, 1.4460, 3.2398) -- (1.4540, 1.4460, 3.2397) -- cycle;
\fill[blue!15.2, opacity=0.7] (1.4540, 1.4460, 3.2397) -- (1.5000, 1.4460, 3.2398) -- (1.5000, 1.5000, 3.2400) -- (1.4540, 1.5000, 3.2398) -- cycle;
\fill[blue!37.0, opacity=0.7] (1.4540, 1.5000, 3.2398) -- (1.5000, 1.5000, 3.2400) -- (1.5000, 1.5540, 3.2398) -- (1.4540, 1.5540, 3.2397) -- cycle;
\fill[blue!43.5, opacity=0.7] (1.4540, 1.5540, 3.2397) -- (1.5000, 1.5540, 3.2398) -- (1.5000, 1.6080, 3.2393) -- (1.4540, 1.6080, 3.2392) -- cycle;
\fill[blue!39.1, opacity=0.7] (1.4540, 1.6080, 3.2392) -- (1.5000, 1.6080, 3.2393) -- (1.5000, 1.6620, 3.2385) -- (1.4540, 1.6620, 3.2384) -- cycle;
\fill[blue!26.5, opacity=0.7] (1.4540, 1.6620, 3.2384) -- (1.5000, 1.6620, 3.2385) -- (1.5000, 1.7160, 3.2374) -- (1.4540, 1.7160, 3.2372) -- cycle;
\fill[blue!18.0, opacity=0.7] (1.4540, 1.7160, 3.2372) -- (1.5000, 1.7160, 3.2374) -- (1.5000, 1.7700, 3.2359) -- (1.4540, 1.7700, 3.2357) -- cycle;
\fill[blue!16.0, opacity=0.7] (1.4540, 1.7700, 3.2357) -- (1.5000, 1.7700, 3.2359) -- (1.5000, 1.8240, 3.2341) -- (1.4540, 1.8240, 3.2340) -- cycle;
\fill[blue!16.3, opacity=0.7] (1.4540, 1.8240, 3.2340) -- (1.5000, 1.8240, 3.2341) -- (1.5000, 1.8780, 3.2320) -- (1.4540, 1.8780, 3.2319) -- cycle;
\fill[blue!21.0, opacity=0.7] (1.4540, 1.8780, 3.2319) -- (1.5000, 1.8780, 3.2320) -- (1.5000, 1.9320, 3.2296) -- (1.4540, 1.9320, 3.2295) -- cycle;
\fill[blue!41.6, opacity=0.7] (1.4540, 1.9320, 3.2295) -- (1.5000, 1.9320, 3.2296) -- (1.5000, 1.9860, 3.2269) -- (1.4540, 1.9860, 3.2268) -- cycle;
\fill[blue!61.8, opacity=0.7] (1.4540, 1.9860, 3.2268) -- (1.5000, 1.9860, 3.2269) -- (1.5000, 2.0400, 3.2239) -- (1.4540, 2.0400, 3.2238) -- cycle;
\fill[blue!63.5, opacity=0.7] (1.4540, 2.0400, 3.2238) -- (1.5000, 2.0400, 3.2239) -- (1.5000, 2.0940, 3.2206) -- (1.4540, 2.0940, 3.2205) -- cycle;
\fill[blue!62.2, opacity=0.7] (1.4540, 2.0940, 3.2205) -- (1.5000, 2.0940, 3.2206) -- (1.5000, 2.1480, 3.2171) -- (1.4540, 2.1480, 3.2169) -- cycle;
\fill[blue!46.6, opacity=0.7] (1.4540, 2.1480, 3.2169) -- (1.5000, 2.1480, 3.2171) -- (1.5000, 2.2020, 3.2133) -- (1.4540, 2.2020, 3.2131) -- cycle;
\fill[blue!28.6, opacity=0.7] (1.4540, 2.2020, 3.2131) -- (1.5000, 2.2020, 3.2133) -- (1.5000, 2.2560, 3.2092) -- (1.4540, 2.2560, 3.2090) -- cycle;
\fill[blue!24.2, opacity=0.7] (1.4540, 2.2560, 3.2090) -- (1.5000, 2.2560, 3.2092) -- (1.5000, 2.3100, 3.2049) -- (1.4540, 2.3100, 3.2047) -- cycle;
\fill[blue!32.5, opacity=0.7] (1.4540, 2.3100, 3.2047) -- (1.5000, 2.3100, 3.2049) -- (1.5000, 2.3640, 3.2003) -- (1.4540, 2.3640, 3.2001) -- cycle;
\fill[blue!56.6, opacity=0.7] (1.4540, 2.3640, 3.2001) -- (1.5000, 2.3640, 3.2003) -- (1.5000, 2.4180, 3.1955) -- (1.4540, 2.4180, 3.1954) -- cycle;
\fill[blue!60.0, opacity=0.7] (1.4540, 2.4180, 3.1954) -- (1.5000, 2.4180, 3.1955) -- (1.5000, 2.4720, 3.1905) -- (1.4540, 2.4720, 3.1904) -- cycle;
\fill[blue!45.1, opacity=0.7] (1.4540, 2.4720, 3.1904) -- (1.5000, 2.4720, 3.1905) -- (1.5000, 2.5260, 3.1854) -- (1.4540, 2.5260, 3.1852) -- cycle;
\fill[blue!45.9, opacity=0.7] (1.4540, 2.5260, 3.1852) -- (1.5000, 2.5260, 3.1854) -- (1.5000, 2.5800, 3.1800) -- (1.4540, 2.5800, 3.1798) -- cycle;
\fill[blue!60.1, opacity=0.7] (1.4540, 2.5800, 3.1798) -- (1.5000, 2.5800, 3.1800) -- (1.5000, 2.6340, 3.1745) -- (1.4540, 2.6340, 3.1743) -- cycle;
\fill[blue!60.6, opacity=0.7] (1.4540, 2.6340, 3.1743) -- (1.5000, 2.6340, 3.1745) -- (1.5000, 2.6880, 3.1688) -- (1.4540, 2.6880, 3.1686) -- cycle;
\fill[blue!50.0, opacity=0.7] (1.4540, 2.6880, 3.1686) -- (1.5000, 2.6880, 3.1688) -- (1.5000, 2.7420, 3.1630) -- (1.4540, 2.7420, 3.1628) -- cycle;
\fill[blue!53.7, opacity=0.7] (1.4540, 2.7420, 3.1628) -- (1.5000, 2.7420, 3.1630) -- (1.5000, 2.7960, 3.1571) -- (1.4540, 2.7960, 3.1569) -- cycle;
\fill[blue!63.5, opacity=0.7] (1.4540, 2.7960, 3.1569) -- (1.5000, 2.7960, 3.1571) -- (1.5000, 2.8500, 3.1511) -- (1.4540, 2.8500, 3.1509) -- cycle;
\fill[blue!43.9, opacity=0.7] (1.4540, 2.8500, 3.1509) -- (1.5000, 2.8500, 3.1511) -- (1.5000, 2.9040, 3.1449) -- (1.4540, 2.9040, 3.1448) -- cycle;
\fill[blue!24.6, opacity=0.7] (1.4540, 2.9040, 3.1448) -- (1.5000, 2.9040, 3.1449) -- (1.5000, 2.9580, 3.1388) -- (1.4540, 2.9580, 3.1386) -- cycle;
\fill[blue!21.2, opacity=0.7] (1.4540, 2.9580, 3.1386) -- (1.5000, 2.9580, 3.1388) -- (1.5000, 3.0120, 3.1325) -- (1.4540, 3.0120, 3.1324) -- cycle;
\fill[blue!27.5, opacity=0.7] (1.4540, 3.0120, 3.1324) -- (1.5000, 3.0120, 3.1325) -- (1.5000, 3.0660, 3.1263) -- (1.4540, 3.0660, 3.1261) -- cycle;
\fill[blue!46.8, opacity=0.7] (1.4540, 3.0660, 3.1261) -- (1.5000, 3.0660, 3.1263) -- (1.5000, 3.1200, 3.1200) -- (1.4540, 3.1200, 3.1198) -- cycle;
\fill[blue!46.8, opacity=0.7] (1.5000, -0.1200, 3.1200) -- (1.5460, -0.1200, 3.1198) -- (1.5460, -0.0660, 3.1261) -- (1.5000, -0.0660, 3.1263) -- cycle;
\fill[blue!27.5, opacity=0.7] (1.5000, -0.0660, 3.1263) -- (1.5460, -0.0660, 3.1261) -- (1.5460, -0.0120, 3.1324) -- (1.5000, -0.0120, 3.1325) -- cycle;
\fill[blue!21.2, opacity=0.7] (1.5000, -0.0120, 3.1325) -- (1.5460, -0.0120, 3.1324) -- (1.5460, 0.0420, 3.1386) -- (1.5000, 0.0420, 3.1388) -- cycle;
\fill[blue!24.6, opacity=0.7] (1.5000, 0.0420, 3.1388) -- (1.5460, 0.0420, 3.1386) -- (1.5460, 0.0960, 3.1448) -- (1.5000, 0.0960, 3.1449) -- cycle;
\fill[blue!43.9, opacity=0.7] (1.5000, 0.0960, 3.1449) -- (1.5460, 0.0960, 3.1448) -- (1.5460, 0.1500, 3.1509) -- (1.5000, 0.1500, 3.1511) -- cycle;
\fill[blue!63.5, opacity=0.7] (1.5000, 0.1500, 3.1511) -- (1.5460, 0.1500, 3.1509) -- (1.5460, 0.2040, 3.1569) -- (1.5000, 0.2040, 3.1571) -- cycle;
\fill[blue!53.7, opacity=0.7] (1.5000, 0.2040, 3.1571) -- (1.5460, 0.2040, 3.1569) -- (1.5460, 0.2580, 3.1628) -- (1.5000, 0.2580, 3.1630) -- cycle;
\fill[blue!50.0, opacity=0.7] (1.5000, 0.2580, 3.1630) -- (1.5460, 0.2580, 3.1628) -- (1.5460, 0.3120, 3.1686) -- (1.5000, 0.3120, 3.1688) -- cycle;
\fill[blue!60.6, opacity=0.7] (1.5000, 0.3120, 3.1688) -- (1.5460, 0.3120, 3.1686) -- (1.5460, 0.3660, 3.1743) -- (1.5000, 0.3660, 3.1745) -- cycle;
\fill[blue!60.1, opacity=0.7] (1.5000, 0.3660, 3.1745) -- (1.5460, 0.3660, 3.1743) -- (1.5460, 0.4200, 3.1798) -- (1.5000, 0.4200, 3.1800) -- cycle;
\fill[blue!45.9, opacity=0.7] (1.5000, 0.4200, 3.1800) -- (1.5460, 0.4200, 3.1798) -- (1.5460, 0.4740, 3.1852) -- (1.5000, 0.4740, 3.1854) -- cycle;
\fill[blue!45.1, opacity=0.7] (1.5000, 0.4740, 3.1854) -- (1.5460, 0.4740, 3.1852) -- (1.5460, 0.5280, 3.1904) -- (1.5000, 0.5280, 3.1905) -- cycle;
\fill[blue!60.0, opacity=0.7] (1.5000, 0.5280, 3.1905) -- (1.5460, 0.5280, 3.1904) -- (1.5460, 0.5820, 3.1954) -- (1.5000, 0.5820, 3.1955) -- cycle;
\fill[blue!56.6, opacity=0.7] (1.5000, 0.5820, 3.1955) -- (1.5460, 0.5820, 3.1954) -- (1.5460, 0.6360, 3.2001) -- (1.5000, 0.6360, 3.2003) -- cycle;
\fill[blue!32.5, opacity=0.7] (1.5000, 0.6360, 3.2003) -- (1.5460, 0.6360, 3.2001) -- (1.5460, 0.6900, 3.2047) -- (1.5000, 0.6900, 3.2049) -- cycle;
\fill[blue!24.2, opacity=0.7] (1.5000, 0.6900, 3.2049) -- (1.5460, 0.6900, 3.2047) -- (1.5460, 0.7440, 3.2090) -- (1.5000, 0.7440, 3.2092) -- cycle;
\fill[blue!28.6, opacity=0.7] (1.5000, 0.7440, 3.2092) -- (1.5460, 0.7440, 3.2090) -- (1.5460, 0.7980, 3.2131) -- (1.5000, 0.7980, 3.2133) -- cycle;
\fill[blue!46.6, opacity=0.7] (1.5000, 0.7980, 3.2133) -- (1.5460, 0.7980, 3.2131) -- (1.5460, 0.8520, 3.2169) -- (1.5000, 0.8520, 3.2171) -- cycle;
\fill[blue!62.2, opacity=0.7] (1.5000, 0.8520, 3.2171) -- (1.5460, 0.8520, 3.2169) -- (1.5460, 0.9060, 3.2205) -- (1.5000, 0.9060, 3.2206) -- cycle;
\fill[blue!63.5, opacity=0.7] (1.5000, 0.9060, 3.2206) -- (1.5460, 0.9060, 3.2205) -- (1.5460, 0.9600, 3.2238) -- (1.5000, 0.9600, 3.2239) -- cycle;
\fill[blue!61.8, opacity=0.7] (1.5000, 0.9600, 3.2239) -- (1.5460, 0.9600, 3.2238) -- (1.5460, 1.0140, 3.2268) -- (1.5000, 1.0140, 3.2269) -- cycle;
\fill[blue!41.6, opacity=0.7] (1.5000, 1.0140, 3.2269) -- (1.5460, 1.0140, 3.2268) -- (1.5460, 1.0680, 3.2295) -- (1.5000, 1.0680, 3.2296) -- cycle;
\fill[blue!21.0, opacity=0.7] (1.5000, 1.0680, 3.2296) -- (1.5460, 1.0680, 3.2295) -- (1.5460, 1.1220, 3.2319) -- (1.5000, 1.1220, 3.2320) -- cycle;
\fill[blue!16.3, opacity=0.7] (1.5000, 1.1220, 3.2320) -- (1.5460, 1.1220, 3.2319) -- (1.5460, 1.1760, 3.2340) -- (1.5000, 1.1760, 3.2341) -- cycle;
\fill[blue!16.0, opacity=0.7] (1.5000, 1.1760, 3.2341) -- (1.5460, 1.1760, 3.2340) -- (1.5460, 1.2300, 3.2357) -- (1.5000, 1.2300, 3.2359) -- cycle;
\fill[blue!18.0, opacity=0.7] (1.5000, 1.2300, 3.2359) -- (1.5460, 1.2300, 3.2357) -- (1.5460, 1.2840, 3.2372) -- (1.5000, 1.2840, 3.2374) -- cycle;
\fill[blue!26.5, opacity=0.7] (1.5000, 1.2840, 3.2374) -- (1.5460, 1.2840, 3.2372) -- (1.5460, 1.3380, 3.2384) -- (1.5000, 1.3380, 3.2385) -- cycle;
\fill[blue!39.1, opacity=0.7] (1.5000, 1.3380, 3.2385) -- (1.5460, 1.3380, 3.2384) -- (1.5460, 1.3920, 3.2392) -- (1.5000, 1.3920, 3.2393) -- cycle;
\fill[blue!43.5, opacity=0.7] (1.5000, 1.3920, 3.2393) -- (1.5460, 1.3920, 3.2392) -- (1.5460, 1.4460, 3.2397) -- (1.5000, 1.4460, 3.2398) -- cycle;
\fill[blue!37.0, opacity=0.7] (1.5000, 1.4460, 3.2398) -- (1.5460, 1.4460, 3.2397) -- (1.5460, 1.5000, 3.2398) -- (1.5000, 1.5000, 3.2400) -- cycle;
\fill[blue!15.2, opacity=0.7] (1.5000, 1.5000, 3.2400) -- (1.5460, 1.5000, 3.2398) -- (1.5460, 1.5540, 3.2397) -- (1.5000, 1.5540, 3.2398) -- cycle;
\fill[blue!29.3, opacity=0.7] (1.5000, 1.5540, 3.2398) -- (1.5460, 1.5540, 3.2397) -- (1.5460, 1.6080, 3.2392) -- (1.5000, 1.6080, 3.2393) -- cycle;
\fill[blue!48.0, opacity=0.7] (1.5000, 1.6080, 3.2393) -- (1.5460, 1.6080, 3.2392) -- (1.5460, 1.6620, 3.2384) -- (1.5000, 1.6620, 3.2385) -- cycle;
\fill[blue!41.0, opacity=0.7] (1.5000, 1.6620, 3.2385) -- (1.5460, 1.6620, 3.2384) -- (1.5460, 1.7160, 3.2372) -- (1.5000, 1.7160, 3.2374) -- cycle;
\fill[blue!23.6, opacity=0.7] (1.5000, 1.7160, 3.2374) -- (1.5460, 1.7160, 3.2372) -- (1.5460, 1.7700, 3.2357) -- (1.5000, 1.7700, 3.2359) -- cycle;
\fill[blue!16.8, opacity=0.7] (1.5000, 1.7700, 3.2359) -- (1.5460, 1.7700, 3.2357) -- (1.5460, 1.8240, 3.2340) -- (1.5000, 1.8240, 3.2341) -- cycle;
\fill[blue!16.1, opacity=0.7] (1.5000, 1.8240, 3.2341) -- (1.5460, 1.8240, 3.2340) -- (1.5460, 1.8780, 3.2319) -- (1.5000, 1.8780, 3.2320) -- cycle;
\fill[blue!18.5, opacity=0.7] (1.5000, 1.8780, 3.2320) -- (1.5460, 1.8780, 3.2319) -- (1.5460, 1.9320, 3.2295) -- (1.5000, 1.9320, 3.2296) -- cycle;
\fill[blue!34.0, opacity=0.7] (1.5000, 1.9320, 3.2296) -- (1.5460, 1.9320, 3.2295) -- (1.5460, 1.9860, 3.2268) -- (1.5000, 1.9860, 3.2269) -- cycle;
\fill[blue!59.2, opacity=0.7] (1.5000, 1.9860, 3.2269) -- (1.5460, 1.9860, 3.2268) -- (1.5460, 2.0400, 3.2238) -- (1.5000, 2.0400, 3.2239) -- cycle;
\fill[blue!63.5, opacity=0.7] (1.5000, 2.0400, 3.2239) -- (1.5460, 2.0400, 3.2238) -- (1.5460, 2.0940, 3.2205) -- (1.5000, 2.0940, 3.2206) -- cycle;
\fill[blue!63.4, opacity=0.7] (1.5000, 2.0940, 3.2206) -- (1.5460, 2.0940, 3.2205) -- (1.5460, 2.1480, 3.2169) -- (1.5000, 2.1480, 3.2171) -- cycle;
\fill[blue!52.0, opacity=0.7] (1.5000, 2.1480, 3.2171) -- (1.5460, 2.1480, 3.2169) -- (1.5460, 2.2020, 3.2131) -- (1.5000, 2.2020, 3.2133) -- cycle;
\fill[blue!31.7, opacity=0.7] (1.5000, 2.2020, 3.2133) -- (1.5460, 2.2020, 3.2131) -- (1.5460, 2.2560, 3.2090) -- (1.5000, 2.2560, 3.2092) -- cycle;
\fill[blue!24.5, opacity=0.7] (1.5000, 2.2560, 3.2092) -- (1.5460, 2.2560, 3.2090) -- (1.5460, 2.3100, 3.2047) -- (1.5000, 2.3100, 3.2049) -- cycle;
\fill[blue!30.3, opacity=0.7] (1.5000, 2.3100, 3.2049) -- (1.5460, 2.3100, 3.2047) -- (1.5460, 2.3640, 3.2001) -- (1.5000, 2.3640, 3.2003) -- cycle;
\fill[blue!53.0, opacity=0.7] (1.5000, 2.3640, 3.2003) -- (1.5460, 2.3640, 3.2001) -- (1.5460, 2.4180, 3.1954) -- (1.5000, 2.4180, 3.1955) -- cycle;
\fill[blue!61.7, opacity=0.7] (1.5000, 2.4180, 3.1955) -- (1.5460, 2.4180, 3.1954) -- (1.5460, 2.4720, 3.1904) -- (1.5000, 2.4720, 3.1905) -- cycle;
\fill[blue!46.1, opacity=0.7] (1.5000, 2.4720, 3.1905) -- (1.5460, 2.4720, 3.1904) -- (1.5460, 2.5260, 3.1852) -- (1.5000, 2.5260, 3.1854) -- cycle;
\fill[blue!44.2, opacity=0.7] (1.5000, 2.5260, 3.1854) -- (1.5460, 2.5260, 3.1852) -- (1.5460, 2.5800, 3.1798) -- (1.5000, 2.5800, 3.1800) -- cycle;
\fill[blue!58.0, opacity=0.7] (1.5000, 2.5800, 3.1800) -- (1.5460, 2.5800, 3.1798) -- (1.5460, 2.6340, 3.1743) -- (1.5000, 2.6340, 3.1745) -- cycle;
\fill[blue!62.0, opacity=0.7] (1.5000, 2.6340, 3.1745) -- (1.5460, 2.6340, 3.1743) -- (1.5460, 2.6880, 3.1686) -- (1.5000, 2.6880, 3.1688) -- cycle;
\fill[blue!51.1, opacity=0.7] (1.5000, 2.6880, 3.1688) -- (1.5460, 2.6880, 3.1686) -- (1.5460, 2.7420, 3.1628) -- (1.5000, 2.7420, 3.1630) -- cycle;
\fill[blue!52.7, opacity=0.7] (1.5000, 2.7420, 3.1630) -- (1.5460, 2.7420, 3.1628) -- (1.5460, 2.7960, 3.1569) -- (1.5000, 2.7960, 3.1571) -- cycle;
\fill[blue!63.5, opacity=0.7] (1.5000, 2.7960, 3.1571) -- (1.5460, 2.7960, 3.1569) -- (1.5460, 2.8500, 3.1509) -- (1.5000, 2.8500, 3.1511) -- cycle;
\fill[blue!46.6, opacity=0.7] (1.5000, 2.8500, 3.1511) -- (1.5460, 2.8500, 3.1509) -- (1.5460, 2.9040, 3.1448) -- (1.5000, 2.9040, 3.1449) -- cycle;
\fill[blue!25.5, opacity=0.7] (1.5000, 2.9040, 3.1449) -- (1.5460, 2.9040, 3.1448) -- (1.5460, 2.9580, 3.1386) -- (1.5000, 2.9580, 3.1388) -- cycle;
\fill[blue!21.0, opacity=0.7] (1.5000, 2.9580, 3.1388) -- (1.5460, 2.9580, 3.1386) -- (1.5460, 3.0120, 3.1324) -- (1.5000, 3.0120, 3.1325) -- cycle;
\fill[blue!26.2, opacity=0.7] (1.5000, 3.0120, 3.1325) -- (1.5460, 3.0120, 3.1324) -- (1.5460, 3.0660, 3.1261) -- (1.5000, 3.0660, 3.1263) -- cycle;
\fill[blue!44.5, opacity=0.7] (1.5000, 3.0660, 3.1263) -- (1.5460, 3.0660, 3.1261) -- (1.5460, 3.1200, 3.1198) -- (1.5000, 3.1200, 3.1200) -- cycle;
\fill[blue!49.8, opacity=0.7] (1.5460, -0.1200, 3.1198) -- (1.5920, -0.1200, 3.1193) -- (1.5920, -0.0660, 3.1256) -- (1.5460, -0.0660, 3.1261) -- cycle;
\fill[blue!29.6, opacity=0.7] (1.5460, -0.0660, 3.1261) -- (1.5920, -0.0660, 3.1256) -- (1.5920, -0.0120, 3.1319) -- (1.5460, -0.0120, 3.1324) -- cycle;
\fill[blue!21.6, opacity=0.7] (1.5460, -0.0120, 3.1324) -- (1.5920, -0.0120, 3.1319) -- (1.5920, 0.0420, 3.1381) -- (1.5460, 0.0420, 3.1386) -- cycle;
\fill[blue!23.7, opacity=0.7] (1.5460, 0.0420, 3.1386) -- (1.5920, 0.0420, 3.1381) -- (1.5920, 0.0960, 3.1443) -- (1.5460, 0.0960, 3.1448) -- cycle;
\fill[blue!40.4, opacity=0.7] (1.5460, 0.0960, 3.1448) -- (1.5920, 0.0960, 3.1443) -- (1.5920, 0.1500, 3.1504) -- (1.5460, 0.1500, 3.1509) -- cycle;
\fill[blue!62.9, opacity=0.7] (1.5460, 0.1500, 3.1509) -- (1.5920, 0.1500, 3.1504) -- (1.5920, 0.2040, 3.1564) -- (1.5460, 0.2040, 3.1569) -- cycle;
\fill[blue!55.3, opacity=0.7] (1.5460, 0.2040, 3.1569) -- (1.5920, 0.2040, 3.1564) -- (1.5920, 0.2580, 3.1623) -- (1.5460, 0.2580, 3.1628) -- cycle;
\fill[blue!49.0, opacity=0.7] (1.5460, 0.2580, 3.1628) -- (1.5920, 0.2580, 3.1623) -- (1.5920, 0.3120, 3.1682) -- (1.5460, 0.3120, 3.1686) -- cycle;
\fill[blue!58.3, opacity=0.7] (1.5460, 0.3120, 3.1686) -- (1.5920, 0.3120, 3.1682) -- (1.5920, 0.3660, 3.1738) -- (1.5460, 0.3660, 3.1743) -- cycle;
\fill[blue!62.2, opacity=0.7] (1.5460, 0.3660, 3.1743) -- (1.5920, 0.3660, 3.1738) -- (1.5920, 0.4200, 3.1793) -- (1.5460, 0.4200, 3.1798) -- cycle;
\fill[blue!48.5, opacity=0.7] (1.5460, 0.4200, 3.1798) -- (1.5920, 0.4200, 3.1793) -- (1.5920, 0.4740, 3.1847) -- (1.5460, 0.4740, 3.1852) -- cycle;
\fill[blue!44.3, opacity=0.7] (1.5460, 0.4740, 3.1852) -- (1.5920, 0.4740, 3.1847) -- (1.5920, 0.5280, 3.1899) -- (1.5460, 0.5280, 3.1904) -- cycle;
\fill[blue!57.0, opacity=0.7] (1.5460, 0.5280, 3.1904) -- (1.5920, 0.5280, 3.1899) -- (1.5920, 0.5820, 3.1949) -- (1.5460, 0.5820, 3.1954) -- cycle;
\fill[blue!60.6, opacity=0.7] (1.5460, 0.5820, 3.1954) -- (1.5920, 0.5820, 3.1949) -- (1.5920, 0.6360, 3.1996) -- (1.5460, 0.6360, 3.2001) -- cycle;
\fill[blue!36.5, opacity=0.7] (1.5460, 0.6360, 3.2001) -- (1.5920, 0.6360, 3.1996) -- (1.5920, 0.6900, 3.2042) -- (1.5460, 0.6900, 3.2047) -- cycle;
\fill[blue!24.5, opacity=0.7] (1.5460, 0.6900, 3.2047) -- (1.5920, 0.6900, 3.2042) -- (1.5920, 0.7440, 3.2085) -- (1.5460, 0.7440, 3.2090) -- cycle;
\fill[blue!25.7, opacity=0.7] (1.5460, 0.7440, 3.2090) -- (1.5920, 0.7440, 3.2085) -- (1.5920, 0.7980, 3.2126) -- (1.5460, 0.7980, 3.2131) -- cycle;
\fill[blue!39.4, opacity=0.7] (1.5460, 0.7980, 3.2131) -- (1.5920, 0.7980, 3.2126) -- (1.5920, 0.8520, 3.2164) -- (1.5460, 0.8520, 3.2169) -- cycle;
\fill[blue!58.7, opacity=0.7] (1.5460, 0.8520, 3.2169) -- (1.5920, 0.8520, 3.2164) -- (1.5920, 0.9060, 3.2200) -- (1.5460, 0.9060, 3.2205) -- cycle;
\fill[blue!63.6, opacity=0.7] (1.5460, 0.9060, 3.2205) -- (1.5920, 0.9060, 3.2200) -- (1.5920, 0.9600, 3.2233) -- (1.5460, 0.9600, 3.2238) -- cycle;
\fill[blue!63.3, opacity=0.7] (1.5460, 0.9600, 3.2238) -- (1.5920, 0.9600, 3.2233) -- (1.5920, 1.0140, 3.2263) -- (1.5460, 1.0140, 3.2268) -- cycle;
\fill[blue!52.1, opacity=0.7] (1.5460, 1.0140, 3.2268) -- (1.5920, 1.0140, 3.2263) -- (1.5920, 1.0680, 3.2290) -- (1.5460, 1.0680, 3.2295) -- cycle;
\fill[blue!27.8, opacity=0.7] (1.5460, 1.0680, 3.2295) -- (1.5920, 1.0680, 3.2290) -- (1.5920, 1.1220, 3.2314) -- (1.5460, 1.1220, 3.2319) -- cycle;
\fill[blue!17.5, opacity=0.7] (1.5460, 1.1220, 3.2319) -- (1.5920, 1.1220, 3.2314) -- (1.5920, 1.1760, 3.2335) -- (1.5460, 1.1760, 3.2340) -- cycle;
\fill[blue!15.8, opacity=0.7] (1.5460, 1.1760, 3.2340) -- (1.5920, 1.1760, 3.2335) -- (1.5920, 1.2300, 3.2353) -- (1.5460, 1.2300, 3.2357) -- cycle;
\fill[blue!15.9, opacity=0.7] (1.5460, 1.2300, 3.2357) -- (1.5920, 1.2300, 3.2353) -- (1.5920, 1.2840, 3.2367) -- (1.5460, 1.2840, 3.2372) -- cycle;
\fill[blue!17.1, opacity=0.7] (1.5460, 1.2840, 3.2372) -- (1.5920, 1.2840, 3.2367) -- (1.5920, 1.3380, 3.2379) -- (1.5460, 1.3380, 3.2384) -- cycle;
\fill[blue!19.7, opacity=0.7] (1.5460, 1.3380, 3.2384) -- (1.5920, 1.3380, 3.2379) -- (1.5920, 1.3920, 3.2387) -- (1.5460, 1.3920, 3.2392) -- cycle;
\fill[blue!20.0, opacity=0.7] (1.5460, 1.3920, 3.2392) -- (1.5920, 1.3920, 3.2387) -- (1.5920, 1.4460, 3.2392) -- (1.5460, 1.4460, 3.2397) -- cycle;
\fill[blue!16.0, opacity=0.7] (1.5460, 1.4460, 3.2397) -- (1.5920, 1.4460, 3.2392) -- (1.5920, 1.5000, 3.2393) -- (1.5460, 1.5000, 3.2398) -- cycle;
\fill[blue!15.2, opacity=0.7] (1.5460, 1.5000, 3.2398) -- (1.5920, 1.5000, 3.2393) -- (1.5920, 1.5540, 3.2392) -- (1.5460, 1.5540, 3.2397) -- cycle;
\fill[blue!20.5, opacity=0.7] (1.5460, 1.5540, 3.2397) -- (1.5920, 1.5540, 3.2392) -- (1.5920, 1.6080, 3.2387) -- (1.5460, 1.6080, 3.2392) -- cycle;
\fill[blue!45.6, opacity=0.7] (1.5460, 1.6080, 3.2392) -- (1.5920, 1.6080, 3.2387) -- (1.5920, 1.6620, 3.2379) -- (1.5460, 1.6620, 3.2384) -- cycle;
\fill[blue!48.5, opacity=0.7] (1.5460, 1.6620, 3.2384) -- (1.5920, 1.6620, 3.2379) -- (1.5920, 1.7160, 3.2367) -- (1.5460, 1.7160, 3.2372) -- cycle;
\fill[blue!30.0, opacity=0.7] (1.5460, 1.7160, 3.2372) -- (1.5920, 1.7160, 3.2367) -- (1.5920, 1.7700, 3.2353) -- (1.5460, 1.7700, 3.2357) -- cycle;
\fill[blue!18.1, opacity=0.7] (1.5460, 1.7700, 3.2357) -- (1.5920, 1.7700, 3.2353) -- (1.5920, 1.8240, 3.2335) -- (1.5460, 1.8240, 3.2340) -- cycle;
\fill[blue!16.2, opacity=0.7] (1.5460, 1.8240, 3.2340) -- (1.5920, 1.8240, 3.2335) -- (1.5920, 1.8780, 3.2314) -- (1.5460, 1.8780, 3.2319) -- cycle;
\fill[blue!17.7, opacity=0.7] (1.5460, 1.8780, 3.2319) -- (1.5920, 1.8780, 3.2314) -- (1.5920, 1.9320, 3.2290) -- (1.5460, 1.9320, 3.2295) -- cycle;
\fill[blue!30.2, opacity=0.7] (1.5460, 1.9320, 3.2295) -- (1.5920, 1.9320, 3.2290) -- (1.5920, 1.9860, 3.2263) -- (1.5460, 1.9860, 3.2268) -- cycle;
\fill[blue!56.9, opacity=0.7] (1.5460, 1.9860, 3.2268) -- (1.5920, 1.9860, 3.2263) -- (1.5920, 2.0400, 3.2233) -- (1.5460, 2.0400, 3.2238) -- cycle;
\fill[blue!63.5, opacity=0.7] (1.5460, 2.0400, 3.2238) -- (1.5920, 2.0400, 3.2233) -- (1.5920, 2.0940, 3.2200) -- (1.5460, 2.0940, 3.2205) -- cycle;
\fill[blue!63.6, opacity=0.7] (1.5460, 2.0940, 3.2205) -- (1.5920, 2.0940, 3.2200) -- (1.5920, 2.1480, 3.2164) -- (1.5460, 2.1480, 3.2169) -- cycle;
\fill[blue!55.2, opacity=0.7] (1.5460, 2.1480, 3.2169) -- (1.5920, 2.1480, 3.2164) -- (1.5920, 2.2020, 3.2126) -- (1.5460, 2.2020, 3.2131) -- cycle;
\fill[blue!34.2, opacity=0.7] (1.5460, 2.2020, 3.2131) -- (1.5920, 2.2020, 3.2126) -- (1.5920, 2.2560, 3.2085) -- (1.5460, 2.2560, 3.2090) -- cycle;
\fill[blue!25.0, opacity=0.7] (1.5460, 2.2560, 3.2090) -- (1.5920, 2.2560, 3.2085) -- (1.5920, 2.3100, 3.2042) -- (1.5460, 2.3100, 3.2047) -- cycle;
\fill[blue!29.4, opacity=0.7] (1.5460, 2.3100, 3.2047) -- (1.5920, 2.3100, 3.2042) -- (1.5920, 2.3640, 3.1996) -- (1.5460, 2.3640, 3.2001) -- cycle;
\fill[blue!50.9, opacity=0.7] (1.5460, 2.3640, 3.2001) -- (1.5920, 2.3640, 3.1996) -- (1.5920, 2.4180, 3.1949) -- (1.5460, 2.4180, 3.1954) -- cycle;
\fill[blue!62.5, opacity=0.7] (1.5460, 2.4180, 3.1954) -- (1.5920, 2.4180, 3.1949) -- (1.5920, 2.4720, 3.1899) -- (1.5460, 2.4720, 3.1904) -- cycle;
\fill[blue!46.7, opacity=0.7] (1.5460, 2.4720, 3.1904) -- (1.5920, 2.4720, 3.1899) -- (1.5920, 2.5260, 3.1847) -- (1.5460, 2.5260, 3.1852) -- cycle;
\fill[blue!43.2, opacity=0.7] (1.5460, 2.5260, 3.1852) -- (1.5920, 2.5260, 3.1847) -- (1.5920, 2.5800, 3.1793) -- (1.5460, 2.5800, 3.1798) -- cycle;
\fill[blue!56.4, opacity=0.7] (1.5460, 2.5800, 3.1798) -- (1.5920, 2.5800, 3.1793) -- (1.5920, 2.6340, 3.1738) -- (1.5460, 2.6340, 3.1743) -- cycle;
\fill[blue!62.8, opacity=0.7] (1.5460, 2.6340, 3.1743) -- (1.5920, 2.6340, 3.1738) -- (1.5920, 2.6880, 3.1682) -- (1.5460, 2.6880, 3.1686) -- cycle;
\fill[blue!52.1, opacity=0.7] (1.5460, 2.6880, 3.1686) -- (1.5920, 2.6880, 3.1682) -- (1.5920, 2.7420, 3.1623) -- (1.5460, 2.7420, 3.1628) -- cycle;
\fill[blue!52.4, opacity=0.7] (1.5460, 2.7420, 3.1628) -- (1.5920, 2.7420, 3.1623) -- (1.5920, 2.7960, 3.1564) -- (1.5460, 2.7960, 3.1569) -- cycle;
\fill[blue!63.4, opacity=0.7] (1.5460, 2.7960, 3.1569) -- (1.5920, 2.7960, 3.1564) -- (1.5920, 2.8500, 3.1504) -- (1.5460, 2.8500, 3.1509) -- cycle;
\fill[blue!48.1, opacity=0.7] (1.5460, 2.8500, 3.1509) -- (1.5920, 2.8500, 3.1504) -- (1.5920, 2.9040, 3.1443) -- (1.5460, 2.9040, 3.1448) -- cycle;
\fill[blue!26.0, opacity=0.7] (1.5460, 2.9040, 3.1448) -- (1.5920, 2.9040, 3.1443) -- (1.5920, 2.9580, 3.1381) -- (1.5460, 2.9580, 3.1386) -- cycle;
\fill[blue!20.8, opacity=0.7] (1.5460, 2.9580, 3.1386) -- (1.5920, 2.9580, 3.1381) -- (1.5920, 3.0120, 3.1319) -- (1.5460, 3.0120, 3.1324) -- cycle;
\fill[blue!25.4, opacity=0.7] (1.5460, 3.0120, 3.1324) -- (1.5920, 3.0120, 3.1319) -- (1.5920, 3.0660, 3.1256) -- (1.5460, 3.0660, 3.1261) -- cycle;
\fill[blue!42.9, opacity=0.7] (1.5460, 3.0660, 3.1261) -- (1.5920, 3.0660, 3.1256) -- (1.5920, 3.1200, 3.1193) -- (1.5460, 3.1200, 3.1198) -- cycle;
\fill[blue!53.2, opacity=0.7] (1.5920, -0.1200, 3.1193) -- (1.6380, -0.1200, 3.1185) -- (1.6380, -0.0660, 3.1248) -- (1.5920, -0.0660, 3.1256) -- cycle;
\fill[blue!32.5, opacity=0.7] (1.5920, -0.0660, 3.1256) -- (1.6380, -0.0660, 3.1248) -- (1.6380, -0.0120, 3.1311) -- (1.5920, -0.0120, 3.1319) -- cycle;
\fill[blue!22.3, opacity=0.7] (1.5920, -0.0120, 3.1319) -- (1.6380, -0.0120, 3.1311) -- (1.6380, 0.0420, 3.1373) -- (1.5920, 0.0420, 3.1381) -- cycle;
\fill[blue!22.7, opacity=0.7] (1.5920, 0.0420, 3.1381) -- (1.6380, 0.0420, 3.1373) -- (1.6380, 0.0960, 3.1435) -- (1.5920, 0.0960, 3.1443) -- cycle;
\fill[blue!36.1, opacity=0.7] (1.5920, 0.0960, 3.1443) -- (1.6380, 0.0960, 3.1435) -- (1.6380, 0.1500, 3.1496) -- (1.5920, 0.1500, 3.1504) -- cycle;
\fill[blue!60.9, opacity=0.7] (1.5920, 0.1500, 3.1504) -- (1.6380, 0.1500, 3.1496) -- (1.6380, 0.2040, 3.1556) -- (1.5920, 0.2040, 3.1564) -- cycle;
\fill[blue!57.8, opacity=0.7] (1.5920, 0.2040, 3.1564) -- (1.6380, 0.2040, 3.1556) -- (1.6380, 0.2580, 3.1615) -- (1.5920, 0.2580, 3.1623) -- cycle;
\fill[blue!48.4, opacity=0.7] (1.5920, 0.2580, 3.1623) -- (1.6380, 0.2580, 3.1615) -- (1.6380, 0.3120, 3.1673) -- (1.5920, 0.3120, 3.1682) -- cycle;
\fill[blue!55.2, opacity=0.7] (1.5920, 0.3120, 3.1682) -- (1.6380, 0.3120, 3.1673) -- (1.6380, 0.3660, 3.1730) -- (1.5920, 0.3660, 3.1738) -- cycle;
\fill[blue!63.5, opacity=0.7] (1.5920, 0.3660, 3.1738) -- (1.6380, 0.3660, 3.1730) -- (1.6380, 0.4200, 3.1785) -- (1.5920, 0.4200, 3.1793) -- cycle;
\fill[blue!52.3, opacity=0.7] (1.5920, 0.4200, 3.1793) -- (1.6380, 0.4200, 3.1785) -- (1.6380, 0.4740, 3.1839) -- (1.5920, 0.4740, 3.1847) -- cycle;
\fill[blue!44.1, opacity=0.7] (1.5920, 0.4740, 3.1847) -- (1.6380, 0.4740, 3.1839) -- (1.6380, 0.5280, 3.1891) -- (1.5920, 0.5280, 3.1899) -- cycle;
\fill[blue!53.0, opacity=0.7] (1.5920, 0.5280, 3.1899) -- (1.6380, 0.5280, 3.1891) -- (1.6380, 0.5820, 3.1940) -- (1.5920, 0.5820, 3.1949) -- cycle;
\fill[blue!63.3, opacity=0.7] (1.5920, 0.5820, 3.1949) -- (1.6380, 0.5820, 3.1940) -- (1.6380, 0.6360, 3.1988) -- (1.5920, 0.6360, 3.1996) -- cycle;
\fill[blue!43.2, opacity=0.7] (1.5920, 0.6360, 3.1996) -- (1.6380, 0.6360, 3.1988) -- (1.6380, 0.6900, 3.2034) -- (1.5920, 0.6900, 3.2042) -- cycle;
\fill[blue!26.1, opacity=0.7] (1.5920, 0.6900, 3.2042) -- (1.6380, 0.6900, 3.2034) -- (1.6380, 0.7440, 3.2077) -- (1.5920, 0.7440, 3.2085) -- cycle;
\fill[blue!23.7, opacity=0.7] (1.5920, 0.7440, 3.2085) -- (1.6380, 0.7440, 3.2077) -- (1.6380, 0.7980, 3.2118) -- (1.5920, 0.7980, 3.2126) -- cycle;
\fill[blue!31.9, opacity=0.7] (1.5920, 0.7980, 3.2126) -- (1.6380, 0.7980, 3.2118) -- (1.6380, 0.8520, 3.2156) -- (1.5920, 0.8520, 3.2164) -- cycle;
\fill[blue!51.1, opacity=0.7] (1.5920, 0.8520, 3.2164) -- (1.6380, 0.8520, 3.2156) -- (1.6380, 0.9060, 3.2192) -- (1.5920, 0.9060, 3.2200) -- cycle;
\fill[blue!62.7, opacity=0.7] (1.5920, 0.9060, 3.2200) -- (1.6380, 0.9060, 3.2192) -- (1.6380, 0.9600, 3.2224) -- (1.5920, 0.9600, 3.2233) -- cycle;
\fill[blue!63.6, opacity=0.7] (1.5920, 0.9600, 3.2233) -- (1.6380, 0.9600, 3.2224) -- (1.6380, 1.0140, 3.2254) -- (1.5920, 1.0140, 3.2263) -- cycle;
\fill[blue!60.4, opacity=0.7] (1.5920, 1.0140, 3.2263) -- (1.6380, 1.0140, 3.2254) -- (1.6380, 1.0680, 3.2281) -- (1.5920, 1.0680, 3.2290) -- cycle;
\fill[blue!41.9, opacity=0.7] (1.5920, 1.0680, 3.2290) -- (1.6380, 1.0680, 3.2281) -- (1.6380, 1.1220, 3.2306) -- (1.5920, 1.1220, 3.2314) -- cycle;
\fill[blue!22.8, opacity=0.7] (1.5920, 1.1220, 3.2314) -- (1.6380, 1.1220, 3.2306) -- (1.6380, 1.1760, 3.2326) -- (1.5920, 1.1760, 3.2335) -- cycle;
\fill[blue!16.8, opacity=0.7] (1.5920, 1.1760, 3.2335) -- (1.6380, 1.1760, 3.2326) -- (1.6380, 1.2300, 3.2344) -- (1.5920, 1.2300, 3.2353) -- cycle;
\fill[blue!15.7, opacity=0.7] (1.5920, 1.2300, 3.2353) -- (1.6380, 1.2300, 3.2344) -- (1.6380, 1.2840, 3.2359) -- (1.5920, 1.2840, 3.2367) -- cycle;
\fill[blue!15.6, opacity=0.7] (1.5920, 1.2840, 3.2367) -- (1.6380, 1.2840, 3.2359) -- (1.6380, 1.3380, 3.2370) -- (1.5920, 1.3380, 3.2379) -- cycle;
\fill[blue!15.6, opacity=0.7] (1.5920, 1.3380, 3.2379) -- (1.6380, 1.3380, 3.2370) -- (1.6380, 1.3920, 3.2379) -- (1.5920, 1.3920, 3.2387) -- cycle;
\fill[blue!15.4, opacity=0.7] (1.5920, 1.3920, 3.2387) -- (1.6380, 1.3920, 3.2379) -- (1.6380, 1.4460, 3.2384) -- (1.5920, 1.4460, 3.2392) -- cycle;
\fill[blue!15.3, opacity=0.7] (1.5920, 1.4460, 3.2392) -- (1.6380, 1.4460, 3.2384) -- (1.6380, 1.5000, 3.2385) -- (1.5920, 1.5000, 3.2393) -- cycle;
\fill[blue!15.7, opacity=0.7] (1.5920, 1.5000, 3.2393) -- (1.6380, 1.5000, 3.2385) -- (1.6380, 1.5540, 3.2384) -- (1.5920, 1.5540, 3.2392) -- cycle;
\fill[blue!23.5, opacity=0.7] (1.5920, 1.5540, 3.2392) -- (1.6380, 1.5540, 3.2384) -- (1.6380, 1.6080, 3.2379) -- (1.5920, 1.6080, 3.2387) -- cycle;
\fill[blue!47.4, opacity=0.7] (1.5920, 1.6080, 3.2387) -- (1.6380, 1.6080, 3.2379) -- (1.6380, 1.6620, 3.2370) -- (1.5920, 1.6620, 3.2379) -- cycle;
\fill[blue!51.1, opacity=0.7] (1.5920, 1.6620, 3.2379) -- (1.6380, 1.6620, 3.2370) -- (1.6380, 1.7160, 3.2359) -- (1.5920, 1.7160, 3.2367) -- cycle;
\fill[blue!33.0, opacity=0.7] (1.5920, 1.7160, 3.2367) -- (1.6380, 1.7160, 3.2359) -- (1.6380, 1.7700, 3.2344) -- (1.5920, 1.7700, 3.2353) -- cycle;
\fill[blue!18.9, opacity=0.7] (1.5920, 1.7700, 3.2353) -- (1.6380, 1.7700, 3.2344) -- (1.6380, 1.8240, 3.2326) -- (1.5920, 1.8240, 3.2335) -- cycle;
\fill[blue!16.3, opacity=0.7] (1.5920, 1.8240, 3.2335) -- (1.6380, 1.8240, 3.2326) -- (1.6380, 1.8780, 3.2306) -- (1.5920, 1.8780, 3.2314) -- cycle;
\fill[blue!17.6, opacity=0.7] (1.5920, 1.8780, 3.2314) -- (1.6380, 1.8780, 3.2306) -- (1.6380, 1.9320, 3.2281) -- (1.5920, 1.9320, 3.2290) -- cycle;
\fill[blue!29.4, opacity=0.7] (1.5920, 1.9320, 3.2290) -- (1.6380, 1.9320, 3.2281) -- (1.6380, 1.9860, 3.2254) -- (1.5920, 1.9860, 3.2263) -- cycle;
\fill[blue!56.4, opacity=0.7] (1.5920, 1.9860, 3.2263) -- (1.6380, 1.9860, 3.2254) -- (1.6380, 2.0400, 3.2224) -- (1.5920, 2.0400, 3.2233) -- cycle;
\fill[blue!63.5, opacity=0.7] (1.5920, 2.0400, 3.2233) -- (1.6380, 2.0400, 3.2224) -- (1.6380, 2.0940, 3.2192) -- (1.5920, 2.0940, 3.2200) -- cycle;
\fill[blue!63.5, opacity=0.7] (1.5920, 2.0940, 3.2200) -- (1.6380, 2.0940, 3.2192) -- (1.6380, 2.1480, 3.2156) -- (1.5920, 2.1480, 3.2164) -- cycle;
\fill[blue!56.7, opacity=0.7] (1.5920, 2.1480, 3.2164) -- (1.6380, 2.1480, 3.2156) -- (1.6380, 2.2020, 3.2118) -- (1.5920, 2.2020, 3.2126) -- cycle;
\fill[blue!35.6, opacity=0.7] (1.5920, 2.2020, 3.2126) -- (1.6380, 2.2020, 3.2118) -- (1.6380, 2.2560, 3.2077) -- (1.5920, 2.2560, 3.2085) -- cycle;
\fill[blue!25.5, opacity=0.7] (1.5920, 2.2560, 3.2085) -- (1.6380, 2.2560, 3.2077) -- (1.6380, 2.3100, 3.2034) -- (1.5920, 2.3100, 3.2042) -- cycle;
\fill[blue!29.3, opacity=0.7] (1.5920, 2.3100, 3.2042) -- (1.6380, 2.3100, 3.2034) -- (1.6380, 2.3640, 3.1988) -- (1.5920, 2.3640, 3.1996) -- cycle;
\fill[blue!50.4, opacity=0.7] (1.5920, 2.3640, 3.1996) -- (1.6380, 2.3640, 3.1988) -- (1.6380, 2.4180, 3.1940) -- (1.5920, 2.4180, 3.1949) -- cycle;
\fill[blue!62.7, opacity=0.7] (1.5920, 2.4180, 3.1949) -- (1.6380, 2.4180, 3.1940) -- (1.6380, 2.4720, 3.1891) -- (1.5920, 2.4720, 3.1899) -- cycle;
\fill[blue!46.6, opacity=0.7] (1.5920, 2.4720, 3.1899) -- (1.6380, 2.4720, 3.1891) -- (1.6380, 2.5260, 3.1839) -- (1.5920, 2.5260, 3.1847) -- cycle;
\fill[blue!42.5, opacity=0.7] (1.5920, 2.5260, 3.1847) -- (1.6380, 2.5260, 3.1839) -- (1.6380, 2.5800, 3.1785) -- (1.5920, 2.5800, 3.1793) -- cycle;
\fill[blue!55.5, opacity=0.7] (1.5920, 2.5800, 3.1793) -- (1.6380, 2.5800, 3.1785) -- (1.6380, 2.6340, 3.1730) -- (1.5920, 2.6340, 3.1738) -- cycle;
\fill[blue!63.0, opacity=0.7] (1.5920, 2.6340, 3.1738) -- (1.6380, 2.6340, 3.1730) -- (1.6380, 2.6880, 3.1673) -- (1.5920, 2.6880, 3.1682) -- cycle;
\fill[blue!52.7, opacity=0.7] (1.5920, 2.6880, 3.1682) -- (1.6380, 2.6880, 3.1673) -- (1.6380, 2.7420, 3.1615) -- (1.5920, 2.7420, 3.1623) -- cycle;
\fill[blue!52.5, opacity=0.7] (1.5920, 2.7420, 3.1623) -- (1.6380, 2.7420, 3.1615) -- (1.6380, 2.7960, 3.1556) -- (1.5920, 2.7960, 3.1564) -- cycle;
\fill[blue!63.3, opacity=0.7] (1.5920, 2.7960, 3.1564) -- (1.6380, 2.7960, 3.1556) -- (1.6380, 2.8500, 3.1496) -- (1.5920, 2.8500, 3.1504) -- cycle;
\fill[blue!48.4, opacity=0.7] (1.5920, 2.8500, 3.1504) -- (1.6380, 2.8500, 3.1496) -- (1.6380, 2.9040, 3.1435) -- (1.5920, 2.9040, 3.1443) -- cycle;
\fill[blue!26.0, opacity=0.7] (1.5920, 2.9040, 3.1443) -- (1.6380, 2.9040, 3.1435) -- (1.6380, 2.9580, 3.1373) -- (1.5920, 2.9580, 3.1381) -- cycle;
\fill[blue!20.7, opacity=0.7] (1.5920, 2.9580, 3.1381) -- (1.6380, 2.9580, 3.1373) -- (1.6380, 3.0120, 3.1311) -- (1.5920, 3.0120, 3.1319) -- cycle;
\fill[blue!24.9, opacity=0.7] (1.5920, 3.0120, 3.1319) -- (1.6380, 3.0120, 3.1311) -- (1.6380, 3.0660, 3.1248) -- (1.5920, 3.0660, 3.1256) -- cycle;
\fill[blue!42.1, opacity=0.7] (1.5920, 3.0660, 3.1256) -- (1.6380, 3.0660, 3.1248) -- (1.6380, 3.1200, 3.1185) -- (1.5920, 3.1200, 3.1193) -- cycle;
\fill[blue!56.7, opacity=0.7] (1.6380, -0.1200, 3.1185) -- (1.6840, -0.1200, 3.1174) -- (1.6840, -0.0660, 3.1237) -- (1.6380, -0.0660, 3.1248) -- cycle;
\fill[blue!36.6, opacity=0.7] (1.6380, -0.0660, 3.1248) -- (1.6840, -0.0660, 3.1237) -- (1.6840, -0.0120, 3.1299) -- (1.6380, -0.0120, 3.1311) -- cycle;
\fill[blue!23.5, opacity=0.7] (1.6380, -0.0120, 3.1311) -- (1.6840, -0.0120, 3.1299) -- (1.6840, 0.0420, 3.1361) -- (1.6380, 0.0420, 3.1373) -- cycle;
\fill[blue!22.1, opacity=0.7] (1.6380, 0.0420, 3.1373) -- (1.6840, 0.0420, 3.1361) -- (1.6840, 0.0960, 3.1423) -- (1.6380, 0.0960, 3.1435) -- cycle;
\fill[blue!31.7, opacity=0.7] (1.6380, 0.0960, 3.1435) -- (1.6840, 0.0960, 3.1423) -- (1.6840, 0.1500, 3.1484) -- (1.6380, 0.1500, 3.1496) -- cycle;
\fill[blue!56.6, opacity=0.7] (1.6380, 0.1500, 3.1496) -- (1.6840, 0.1500, 3.1484) -- (1.6840, 0.2040, 3.1545) -- (1.6380, 0.2040, 3.1556) -- cycle;
\fill[blue!60.7, opacity=0.7] (1.6380, 0.2040, 3.1556) -- (1.6840, 0.2040, 3.1545) -- (1.6840, 0.2580, 3.1604) -- (1.6380, 0.2580, 3.1615) -- cycle;
\fill[blue!49.0, opacity=0.7] (1.6380, 0.2580, 3.1615) -- (1.6840, 0.2580, 3.1604) -- (1.6840, 0.3120, 3.1662) -- (1.6380, 0.3120, 3.1673) -- cycle;
\fill[blue!51.7, opacity=0.7] (1.6380, 0.3120, 3.1673) -- (1.6840, 0.3120, 3.1662) -- (1.6840, 0.3660, 3.1719) -- (1.6380, 0.3660, 3.1730) -- cycle;
\fill[blue!62.9, opacity=0.7] (1.6380, 0.3660, 3.1730) -- (1.6840, 0.3660, 3.1719) -- (1.6840, 0.4200, 3.1774) -- (1.6380, 0.4200, 3.1785) -- cycle;
\fill[blue!57.1, opacity=0.7] (1.6380, 0.4200, 3.1785) -- (1.6840, 0.4200, 3.1774) -- (1.6840, 0.4740, 3.1827) -- (1.6380, 0.4740, 3.1839) -- cycle;
\fill[blue!45.5, opacity=0.7] (1.6380, 0.4740, 3.1839) -- (1.6840, 0.4740, 3.1827) -- (1.6840, 0.5280, 3.1879) -- (1.6380, 0.5280, 3.1891) -- cycle;
\fill[blue!48.8, opacity=0.7] (1.6380, 0.5280, 3.1891) -- (1.6840, 0.5280, 3.1879) -- (1.6840, 0.5820, 3.1929) -- (1.6380, 0.5820, 3.1940) -- cycle;
\fill[blue!62.6, opacity=0.7] (1.6380, 0.5820, 3.1940) -- (1.6840, 0.5820, 3.1929) -- (1.6840, 0.6360, 3.1977) -- (1.6380, 0.6360, 3.1988) -- cycle;
\fill[blue!52.4, opacity=0.7] (1.6380, 0.6360, 3.1988) -- (1.6840, 0.6360, 3.1977) -- (1.6840, 0.6900, 3.2022) -- (1.6380, 0.6900, 3.2034) -- cycle;
\fill[blue!30.2, opacity=0.7] (1.6380, 0.6900, 3.2034) -- (1.6840, 0.6900, 3.2022) -- (1.6840, 0.7440, 3.2066) -- (1.6380, 0.7440, 3.2077) -- cycle;
\fill[blue!23.1, opacity=0.7] (1.6380, 0.7440, 3.2077) -- (1.6840, 0.7440, 3.2066) -- (1.6840, 0.7980, 3.2106) -- (1.6380, 0.7980, 3.2118) -- cycle;
\fill[blue!26.1, opacity=0.7] (1.6380, 0.7980, 3.2118) -- (1.6840, 0.7980, 3.2106) -- (1.6840, 0.8520, 3.2145) -- (1.6380, 0.8520, 3.2156) -- cycle;
\fill[blue!40.0, opacity=0.7] (1.6380, 0.8520, 3.2156) -- (1.6840, 0.8520, 3.2145) -- (1.6840, 0.9060, 3.2180) -- (1.6380, 0.9060, 3.2192) -- cycle;
\fill[blue!57.7, opacity=0.7] (1.6380, 0.9060, 3.2192) -- (1.6840, 0.9060, 3.2180) -- (1.6840, 0.9600, 3.2213) -- (1.6380, 0.9600, 3.2224) -- cycle;
\fill[blue!63.3, opacity=0.7] (1.6380, 0.9600, 3.2224) -- (1.6840, 0.9600, 3.2213) -- (1.6840, 1.0140, 3.2243) -- (1.6380, 1.0140, 3.2254) -- cycle;
\fill[blue!63.3, opacity=0.7] (1.6380, 1.0140, 3.2254) -- (1.6840, 1.0140, 3.2243) -- (1.6840, 1.0680, 3.2270) -- (1.6380, 1.0680, 3.2281) -- cycle;
\fill[blue!57.0, opacity=0.7] (1.6380, 1.0680, 3.2281) -- (1.6840, 1.0680, 3.2270) -- (1.6840, 1.1220, 3.2294) -- (1.6380, 1.1220, 3.2306) -- cycle;
\fill[blue!38.5, opacity=0.7] (1.6380, 1.1220, 3.2306) -- (1.6840, 1.1220, 3.2294) -- (1.6840, 1.1760, 3.2315) -- (1.6380, 1.1760, 3.2326) -- cycle;
\fill[blue!23.1, opacity=0.7] (1.6380, 1.1760, 3.2326) -- (1.6840, 1.1760, 3.2315) -- (1.6840, 1.2300, 3.2333) -- (1.6380, 1.2300, 3.2344) -- cycle;
\fill[blue!17.5, opacity=0.7] (1.6380, 1.2300, 3.2344) -- (1.6840, 1.2300, 3.2333) -- (1.6840, 1.2840, 3.2348) -- (1.6380, 1.2840, 3.2359) -- cycle;
\fill[blue!16.1, opacity=0.7] (1.6380, 1.2840, 3.2359) -- (1.6840, 1.2840, 3.2348) -- (1.6840, 1.3380, 3.2359) -- (1.6380, 1.3380, 3.2370) -- cycle;
\fill[blue!15.7, opacity=0.7] (1.6380, 1.3380, 3.2370) -- (1.6840, 1.3380, 3.2359) -- (1.6840, 1.3920, 3.2367) -- (1.6380, 1.3920, 3.2379) -- cycle;
\fill[blue!15.6, opacity=0.7] (1.6380, 1.3920, 3.2379) -- (1.6840, 1.3920, 3.2367) -- (1.6840, 1.4460, 3.2372) -- (1.6380, 1.4460, 3.2384) -- cycle;
\fill[blue!16.1, opacity=0.7] (1.6380, 1.4460, 3.2384) -- (1.6840, 1.4460, 3.2372) -- (1.6840, 1.5000, 3.2374) -- (1.6380, 1.5000, 3.2385) -- cycle;
\fill[blue!19.5, opacity=0.7] (1.6380, 1.5000, 3.2385) -- (1.6840, 1.5000, 3.2374) -- (1.6840, 1.5540, 3.2372) -- (1.6380, 1.5540, 3.2384) -- cycle;
\fill[blue!35.4, opacity=0.7] (1.6380, 1.5540, 3.2384) -- (1.6840, 1.5540, 3.2372) -- (1.6840, 1.6080, 3.2367) -- (1.6380, 1.6080, 3.2379) -- cycle;
\fill[blue!52.8, opacity=0.7] (1.6380, 1.6080, 3.2379) -- (1.6840, 1.6080, 3.2367) -- (1.6840, 1.6620, 3.2359) -- (1.6380, 1.6620, 3.2370) -- cycle;
\fill[blue!50.8, opacity=0.7] (1.6380, 1.6620, 3.2370) -- (1.6840, 1.6620, 3.2359) -- (1.6840, 1.7160, 3.2348) -- (1.6380, 1.7160, 3.2359) -- cycle;
\fill[blue!31.6, opacity=0.7] (1.6380, 1.7160, 3.2359) -- (1.6840, 1.7160, 3.2348) -- (1.6840, 1.7700, 3.2333) -- (1.6380, 1.7700, 3.2344) -- cycle;
\fill[blue!18.7, opacity=0.7] (1.6380, 1.7700, 3.2344) -- (1.6840, 1.7700, 3.2333) -- (1.6840, 1.8240, 3.2315) -- (1.6380, 1.8240, 3.2326) -- cycle;
\fill[blue!16.5, opacity=0.7] (1.6380, 1.8240, 3.2326) -- (1.6840, 1.8240, 3.2315) -- (1.6840, 1.8780, 3.2294) -- (1.6380, 1.8780, 3.2306) -- cycle;
\fill[blue!18.1, opacity=0.7] (1.6380, 1.8780, 3.2306) -- (1.6840, 1.8780, 3.2294) -- (1.6840, 1.9320, 3.2270) -- (1.6380, 1.9320, 3.2281) -- cycle;
\fill[blue!31.4, opacity=0.7] (1.6380, 1.9320, 3.2281) -- (1.6840, 1.9320, 3.2270) -- (1.6840, 1.9860, 3.2243) -- (1.6380, 1.9860, 3.2254) -- cycle;
\fill[blue!58.0, opacity=0.7] (1.6380, 1.9860, 3.2254) -- (1.6840, 1.9860, 3.2243) -- (1.6840, 2.0400, 3.2213) -- (1.6380, 2.0400, 3.2224) -- cycle;
\fill[blue!63.3, opacity=0.7] (1.6380, 2.0400, 3.2224) -- (1.6840, 2.0400, 3.2213) -- (1.6840, 2.0940, 3.2180) -- (1.6380, 2.0940, 3.2192) -- cycle;
\fill[blue!63.5, opacity=0.7] (1.6380, 2.0940, 3.2192) -- (1.6840, 2.0940, 3.2180) -- (1.6840, 2.1480, 3.2145) -- (1.6380, 2.1480, 3.2156) -- cycle;
\fill[blue!56.7, opacity=0.7] (1.6380, 2.1480, 3.2156) -- (1.6840, 2.1480, 3.2145) -- (1.6840, 2.2020, 3.2106) -- (1.6380, 2.2020, 3.2118) -- cycle;
\fill[blue!35.7, opacity=0.7] (1.6380, 2.2020, 3.2118) -- (1.6840, 2.2020, 3.2106) -- (1.6840, 2.2560, 3.2066) -- (1.6380, 2.2560, 3.2077) -- cycle;
\fill[blue!25.9, opacity=0.7] (1.6380, 2.2560, 3.2077) -- (1.6840, 2.2560, 3.2066) -- (1.6840, 2.3100, 3.2022) -- (1.6380, 2.3100, 3.2034) -- cycle;
\fill[blue!30.1, opacity=0.7] (1.6380, 2.3100, 3.2034) -- (1.6840, 2.3100, 3.2022) -- (1.6840, 2.3640, 3.1977) -- (1.6380, 2.3640, 3.1988) -- cycle;
\fill[blue!51.5, opacity=0.7] (1.6380, 2.3640, 3.1988) -- (1.6840, 2.3640, 3.1977) -- (1.6840, 2.4180, 3.1929) -- (1.6380, 2.4180, 3.1940) -- cycle;
\fill[blue!62.3, opacity=0.7] (1.6380, 2.4180, 3.1940) -- (1.6840, 2.4180, 3.1929) -- (1.6840, 2.4720, 3.1879) -- (1.6380, 2.4720, 3.1891) -- cycle;
\fill[blue!45.9, opacity=0.7] (1.6380, 2.4720, 3.1891) -- (1.6840, 2.4720, 3.1879) -- (1.6840, 2.5260, 3.1827) -- (1.6380, 2.5260, 3.1839) -- cycle;
\fill[blue!42.2, opacity=0.7] (1.6380, 2.5260, 3.1839) -- (1.6840, 2.5260, 3.1827) -- (1.6840, 2.5800, 3.1774) -- (1.6380, 2.5800, 3.1785) -- cycle;
\fill[blue!55.4, opacity=0.7] (1.6380, 2.5800, 3.1785) -- (1.6840, 2.5800, 3.1774) -- (1.6840, 2.6340, 3.1719) -- (1.6380, 2.6340, 3.1730) -- cycle;
\fill[blue!63.1, opacity=0.7] (1.6380, 2.6340, 3.1730) -- (1.6840, 2.6340, 3.1719) -- (1.6840, 2.6880, 3.1662) -- (1.6380, 2.6880, 3.1673) -- cycle;
\fill[blue!53.0, opacity=0.7] (1.6380, 2.6880, 3.1673) -- (1.6840, 2.6880, 3.1662) -- (1.6840, 2.7420, 3.1604) -- (1.6380, 2.7420, 3.1615) -- cycle;
\fill[blue!53.0, opacity=0.7] (1.6380, 2.7420, 3.1615) -- (1.6840, 2.7420, 3.1604) -- (1.6840, 2.7960, 3.1545) -- (1.6380, 2.7960, 3.1556) -- cycle;
\fill[blue!63.4, opacity=0.7] (1.6380, 2.7960, 3.1556) -- (1.6840, 2.7960, 3.1545) -- (1.6840, 2.8500, 3.1484) -- (1.6380, 2.8500, 3.1496) -- cycle;
\fill[blue!47.7, opacity=0.7] (1.6380, 2.8500, 3.1496) -- (1.6840, 2.8500, 3.1484) -- (1.6840, 2.9040, 3.1423) -- (1.6380, 2.9040, 3.1435) -- cycle;
\fill[blue!25.7, opacity=0.7] (1.6380, 2.9040, 3.1435) -- (1.6840, 2.9040, 3.1423) -- (1.6840, 2.9580, 3.1361) -- (1.6380, 2.9580, 3.1373) -- cycle;
\fill[blue!20.6, opacity=0.7] (1.6380, 2.9580, 3.1373) -- (1.6840, 2.9580, 3.1361) -- (1.6840, 3.0120, 3.1299) -- (1.6380, 3.0120, 3.1311) -- cycle;
\fill[blue!24.9, opacity=0.7] (1.6380, 3.0120, 3.1311) -- (1.6840, 3.0120, 3.1299) -- (1.6840, 3.0660, 3.1237) -- (1.6380, 3.0660, 3.1248) -- cycle;
\fill[blue!42.1, opacity=0.7] (1.6380, 3.0660, 3.1248) -- (1.6840, 3.0660, 3.1237) -- (1.6840, 3.1200, 3.1174) -- (1.6380, 3.1200, 3.1185) -- cycle;
\fill[blue!59.7, opacity=0.7] (1.6840, -0.1200, 3.1174) -- (1.7300, -0.1200, 3.1159) -- (1.7300, -0.0660, 3.1222) -- (1.6840, -0.0660, 3.1237) -- cycle;
\fill[blue!41.9, opacity=0.7] (1.6840, -0.0660, 3.1237) -- (1.7300, -0.0660, 3.1222) -- (1.7300, -0.0120, 3.1285) -- (1.6840, -0.0120, 3.1299) -- cycle;
\fill[blue!25.6, opacity=0.7] (1.6840, -0.0120, 3.1299) -- (1.7300, -0.0120, 3.1285) -- (1.7300, 0.0420, 3.1347) -- (1.6840, 0.0420, 3.1361) -- cycle;
\fill[blue!21.8, opacity=0.7] (1.6840, 0.0420, 3.1361) -- (1.7300, 0.0420, 3.1347) -- (1.7300, 0.0960, 3.1409) -- (1.6840, 0.0960, 3.1423) -- cycle;
\fill[blue!27.8, opacity=0.7] (1.6840, 0.0960, 3.1423) -- (1.7300, 0.0960, 3.1409) -- (1.7300, 0.1500, 3.1470) -- (1.6840, 0.1500, 3.1484) -- cycle;
\fill[blue!50.0, opacity=0.7] (1.6840, 0.1500, 3.1484) -- (1.7300, 0.1500, 3.1470) -- (1.7300, 0.2040, 3.1530) -- (1.6840, 0.2040, 3.1545) -- cycle;
\fill[blue!63.2, opacity=0.7] (1.6840, 0.2040, 3.1545) -- (1.7300, 0.2040, 3.1530) -- (1.7300, 0.2580, 3.1589) -- (1.6840, 0.2580, 3.1604) -- cycle;
\fill[blue!51.1, opacity=0.7] (1.6840, 0.2580, 3.1604) -- (1.7300, 0.2580, 3.1589) -- (1.7300, 0.3120, 3.1647) -- (1.6840, 0.3120, 3.1662) -- cycle;
\fill[blue!48.6, opacity=0.7] (1.6840, 0.3120, 3.1662) -- (1.7300, 0.3120, 3.1647) -- (1.7300, 0.3660, 3.1704) -- (1.6840, 0.3660, 3.1719) -- cycle;
\fill[blue!59.6, opacity=0.7] (1.6840, 0.3660, 3.1719) -- (1.7300, 0.3660, 3.1704) -- (1.7300, 0.4200, 3.1759) -- (1.6840, 0.4200, 3.1774) -- cycle;
\fill[blue!61.6, opacity=0.7] (1.6840, 0.4200, 3.1774) -- (1.7300, 0.4200, 3.1759) -- (1.7300, 0.4740, 3.1813) -- (1.6840, 0.4740, 3.1827) -- cycle;
\fill[blue!49.0, opacity=0.7] (1.6840, 0.4740, 3.1827) -- (1.7300, 0.4740, 3.1813) -- (1.7300, 0.5280, 3.1864) -- (1.6840, 0.5280, 3.1879) -- cycle;
\fill[blue!46.0, opacity=0.7] (1.6840, 0.5280, 3.1879) -- (1.7300, 0.5280, 3.1864) -- (1.7300, 0.5820, 3.1914) -- (1.6840, 0.5820, 3.1929) -- cycle;
\fill[blue!57.9, opacity=0.7] (1.6840, 0.5820, 3.1929) -- (1.7300, 0.5820, 3.1914) -- (1.7300, 0.6360, 3.1962) -- (1.6840, 0.6360, 3.1977) -- cycle;
\fill[blue!61.0, opacity=0.7] (1.6840, 0.6360, 3.1977) -- (1.7300, 0.6360, 3.1962) -- (1.7300, 0.6900, 3.2008) -- (1.6840, 0.6900, 3.2022) -- cycle;
\fill[blue!38.3, opacity=0.7] (1.6840, 0.6900, 3.2022) -- (1.7300, 0.6900, 3.2008) -- (1.7300, 0.7440, 3.2051) -- (1.6840, 0.7440, 3.2066) -- cycle;
\fill[blue!24.6, opacity=0.7] (1.6840, 0.7440, 3.2066) -- (1.7300, 0.7440, 3.2051) -- (1.7300, 0.7980, 3.2092) -- (1.6840, 0.7980, 3.2106) -- cycle;
\fill[blue!23.0, opacity=0.7] (1.6840, 0.7980, 3.2106) -- (1.7300, 0.7980, 3.2092) -- (1.7300, 0.8520, 3.2130) -- (1.6840, 0.8520, 3.2145) -- cycle;
\fill[blue!29.6, opacity=0.7] (1.6840, 0.8520, 3.2145) -- (1.7300, 0.8520, 3.2130) -- (1.7300, 0.9060, 3.2166) -- (1.6840, 0.9060, 3.2180) -- cycle;
\fill[blue!45.8, opacity=0.7] (1.6840, 0.9060, 3.2180) -- (1.7300, 0.9060, 3.2166) -- (1.7300, 0.9600, 3.2198) -- (1.6840, 0.9600, 3.2213) -- cycle;
\fill[blue!59.9, opacity=0.7] (1.6840, 0.9600, 3.2213) -- (1.7300, 0.9600, 3.2198) -- (1.7300, 1.0140, 3.2228) -- (1.6840, 1.0140, 3.2243) -- cycle;
\fill[blue!63.3, opacity=0.7] (1.6840, 1.0140, 3.2243) -- (1.7300, 1.0140, 3.2228) -- (1.7300, 1.0680, 3.2255) -- (1.6840, 1.0680, 3.2270) -- cycle;
\fill[blue!62.8, opacity=0.7] (1.6840, 1.0680, 3.2270) -- (1.7300, 1.0680, 3.2255) -- (1.7300, 1.1220, 3.2279) -- (1.6840, 1.1220, 3.2294) -- cycle;
\fill[blue!57.1, opacity=0.7] (1.6840, 1.1220, 3.2294) -- (1.7300, 1.1220, 3.2279) -- (1.7300, 1.1760, 3.2300) -- (1.6840, 1.1760, 3.2315) -- cycle;
\fill[blue!43.2, opacity=0.7] (1.6840, 1.1760, 3.2315) -- (1.7300, 1.1760, 3.2300) -- (1.7300, 1.2300, 3.2318) -- (1.6840, 1.2300, 3.2333) -- cycle;
\fill[blue!29.9, opacity=0.7] (1.6840, 1.2300, 3.2333) -- (1.7300, 1.2300, 3.2318) -- (1.7300, 1.2840, 3.2333) -- (1.6840, 1.2840, 3.2348) -- cycle;
\fill[blue!22.9, opacity=0.7] (1.6840, 1.2840, 3.2348) -- (1.7300, 1.2840, 3.2333) -- (1.7300, 1.3380, 3.2344) -- (1.6840, 1.3380, 3.2359) -- cycle;
\fill[blue!20.4, opacity=0.7] (1.6840, 1.3380, 3.2359) -- (1.7300, 1.3380, 3.2344) -- (1.7300, 1.3920, 3.2353) -- (1.6840, 1.3920, 3.2367) -- cycle;
\fill[blue!20.8, opacity=0.7] (1.6840, 1.3920, 3.2367) -- (1.7300, 1.3920, 3.2353) -- (1.7300, 1.4460, 3.2357) -- (1.6840, 1.4460, 3.2372) -- cycle;
\fill[blue!24.9, opacity=0.7] (1.6840, 1.4460, 3.2372) -- (1.7300, 1.4460, 3.2357) -- (1.7300, 1.5000, 3.2359) -- (1.6840, 1.5000, 3.2374) -- cycle;
\fill[blue!36.3, opacity=0.7] (1.6840, 1.5000, 3.2374) -- (1.7300, 1.5000, 3.2359) -- (1.7300, 1.5540, 3.2357) -- (1.6840, 1.5540, 3.2372) -- cycle;
\fill[blue!51.1, opacity=0.7] (1.6840, 1.5540, 3.2372) -- (1.7300, 1.5540, 3.2357) -- (1.7300, 1.6080, 3.2353) -- (1.6840, 1.6080, 3.2367) -- cycle;
\fill[blue!55.9, opacity=0.7] (1.6840, 1.6080, 3.2367) -- (1.7300, 1.6080, 3.2353) -- (1.7300, 1.6620, 3.2344) -- (1.6840, 1.6620, 3.2359) -- cycle;
\fill[blue!46.3, opacity=0.7] (1.6840, 1.6620, 3.2359) -- (1.7300, 1.6620, 3.2344) -- (1.7300, 1.7160, 3.2333) -- (1.6840, 1.7160, 3.2348) -- cycle;
\fill[blue!26.9, opacity=0.7] (1.6840, 1.7160, 3.2348) -- (1.7300, 1.7160, 3.2333) -- (1.7300, 1.7700, 3.2318) -- (1.6840, 1.7700, 3.2333) -- cycle;
\fill[blue!17.9, opacity=0.7] (1.6840, 1.7700, 3.2333) -- (1.7300, 1.7700, 3.2318) -- (1.7300, 1.8240, 3.2300) -- (1.6840, 1.8240, 3.2315) -- cycle;
\fill[blue!16.6, opacity=0.7] (1.6840, 1.8240, 3.2315) -- (1.7300, 1.8240, 3.2300) -- (1.7300, 1.8780, 3.2279) -- (1.6840, 1.8780, 3.2294) -- cycle;
\fill[blue!19.6, opacity=0.7] (1.6840, 1.8780, 3.2294) -- (1.7300, 1.8780, 3.2279) -- (1.7300, 1.9320, 3.2255) -- (1.6840, 1.9320, 3.2270) -- cycle;
\fill[blue!36.5, opacity=0.7] (1.6840, 1.9320, 3.2270) -- (1.7300, 1.9320, 3.2255) -- (1.7300, 1.9860, 3.2228) -- (1.6840, 1.9860, 3.2243) -- cycle;
\fill[blue!60.8, opacity=0.7] (1.6840, 1.9860, 3.2243) -- (1.7300, 1.9860, 3.2228) -- (1.7300, 2.0400, 3.2198) -- (1.6840, 2.0400, 3.2213) -- cycle;
\fill[blue!62.9, opacity=0.7] (1.6840, 2.0400, 3.2213) -- (1.7300, 2.0400, 3.2198) -- (1.7300, 2.0940, 3.2166) -- (1.6840, 2.0940, 3.2180) -- cycle;
\fill[blue!63.6, opacity=0.7] (1.6840, 2.0940, 3.2180) -- (1.7300, 2.0940, 3.2166) -- (1.7300, 2.1480, 3.2130) -- (1.6840, 2.1480, 3.2145) -- cycle;
\fill[blue!55.3, opacity=0.7] (1.6840, 2.1480, 3.2145) -- (1.7300, 2.1480, 3.2130) -- (1.7300, 2.2020, 3.2092) -- (1.6840, 2.2020, 3.2106) -- cycle;
\fill[blue!34.5, opacity=0.7] (1.6840, 2.2020, 3.2106) -- (1.7300, 2.2020, 3.2092) -- (1.7300, 2.2560, 3.2051) -- (1.6840, 2.2560, 3.2066) -- cycle;
\fill[blue!26.1, opacity=0.7] (1.6840, 2.2560, 3.2066) -- (1.7300, 2.2560, 3.2051) -- (1.7300, 2.3100, 3.2008) -- (1.6840, 2.3100, 3.2022) -- cycle;
\fill[blue!31.7, opacity=0.7] (1.6840, 2.3100, 3.2022) -- (1.7300, 2.3100, 3.2008) -- (1.7300, 2.3640, 3.1962) -- (1.6840, 2.3640, 3.1977) -- cycle;
\fill[blue!54.1, opacity=0.7] (1.6840, 2.3640, 3.1977) -- (1.7300, 2.3640, 3.1962) -- (1.7300, 2.4180, 3.1914) -- (1.6840, 2.4180, 3.1929) -- cycle;
\fill[blue!61.2, opacity=0.7] (1.6840, 2.4180, 3.1929) -- (1.7300, 2.4180, 3.1914) -- (1.7300, 2.4720, 3.1864) -- (1.6840, 2.4720, 3.1879) -- cycle;
\fill[blue!44.5, opacity=0.7] (1.6840, 2.4720, 3.1879) -- (1.7300, 2.4720, 3.1864) -- (1.7300, 2.5260, 3.1813) -- (1.6840, 2.5260, 3.1827) -- cycle;
\fill[blue!42.2, opacity=0.7] (1.6840, 2.5260, 3.1827) -- (1.7300, 2.5260, 3.1813) -- (1.7300, 2.5800, 3.1759) -- (1.6840, 2.5800, 3.1774) -- cycle;
\fill[blue!56.2, opacity=0.7] (1.6840, 2.5800, 3.1774) -- (1.7300, 2.5800, 3.1759) -- (1.7300, 2.6340, 3.1704) -- (1.6840, 2.6340, 3.1719) -- cycle;
\fill[blue!62.8, opacity=0.7] (1.6840, 2.6340, 3.1719) -- (1.7300, 2.6340, 3.1704) -- (1.7300, 2.6880, 3.1647) -- (1.6840, 2.6880, 3.1662) -- cycle;
\fill[blue!52.8, opacity=0.7] (1.6840, 2.6880, 3.1662) -- (1.7300, 2.6880, 3.1647) -- (1.7300, 2.7420, 3.1589) -- (1.6840, 2.7420, 3.1604) -- cycle;
\fill[blue!53.9, opacity=0.7] (1.6840, 2.7420, 3.1604) -- (1.7300, 2.7420, 3.1589) -- (1.7300, 2.7960, 3.1530) -- (1.6840, 2.7960, 3.1545) -- cycle;
\fill[blue!63.6, opacity=0.7] (1.6840, 2.7960, 3.1545) -- (1.7300, 2.7960, 3.1530) -- (1.7300, 2.8500, 3.1470) -- (1.6840, 2.8500, 3.1484) -- cycle;
\fill[blue!45.8, opacity=0.7] (1.6840, 2.8500, 3.1484) -- (1.7300, 2.8500, 3.1470) -- (1.7300, 2.9040, 3.1409) -- (1.6840, 2.9040, 3.1423) -- cycle;
\fill[blue!24.9, opacity=0.7] (1.6840, 2.9040, 3.1423) -- (1.7300, 2.9040, 3.1409) -- (1.7300, 2.9580, 3.1347) -- (1.6840, 2.9580, 3.1361) -- cycle;
\fill[blue!20.5, opacity=0.7] (1.6840, 2.9580, 3.1361) -- (1.7300, 2.9580, 3.1347) -- (1.7300, 3.0120, 3.1285) -- (1.6840, 3.0120, 3.1299) -- cycle;
\fill[blue!25.1, opacity=0.7] (1.6840, 3.0120, 3.1299) -- (1.7300, 3.0120, 3.1285) -- (1.7300, 3.0660, 3.1222) -- (1.6840, 3.0660, 3.1237) -- cycle;
\fill[blue!42.8, opacity=0.7] (1.6840, 3.0660, 3.1237) -- (1.7300, 3.0660, 3.1222) -- (1.7300, 3.1200, 3.1159) -- (1.6840, 3.1200, 3.1174) -- cycle;
\fill[blue!61.9, opacity=0.7] (1.7300, -0.1200, 3.1159) -- (1.7760, -0.1200, 3.1141) -- (1.7760, -0.0660, 3.1204) -- (1.7300, -0.0660, 3.1222) -- cycle;
\fill[blue!48.1, opacity=0.7] (1.7300, -0.0660, 3.1222) -- (1.7760, -0.0660, 3.1204) -- (1.7760, -0.0120, 3.1267) -- (1.7300, -0.0120, 3.1285) -- cycle;
\fill[blue!29.0, opacity=0.7] (1.7300, -0.0120, 3.1285) -- (1.7760, -0.0120, 3.1267) -- (1.7760, 0.0420, 3.1329) -- (1.7300, 0.0420, 3.1347) -- cycle;
\fill[blue!22.1, opacity=0.7] (1.7300, 0.0420, 3.1347) -- (1.7760, 0.0420, 3.1329) -- (1.7760, 0.0960, 3.1391) -- (1.7300, 0.0960, 3.1409) -- cycle;
\fill[blue!24.8, opacity=0.7] (1.7300, 0.0960, 3.1409) -- (1.7760, 0.0960, 3.1391) -- (1.7760, 0.1500, 3.1452) -- (1.7300, 0.1500, 3.1470) -- cycle;
\fill[blue!41.9, opacity=0.7] (1.7300, 0.1500, 3.1470) -- (1.7760, 0.1500, 3.1452) -- (1.7760, 0.2040, 3.1512) -- (1.7300, 0.2040, 3.1530) -- cycle;
\fill[blue!63.0, opacity=0.7] (1.7300, 0.2040, 3.1530) -- (1.7760, 0.2040, 3.1512) -- (1.7760, 0.2580, 3.1571) -- (1.7300, 0.2580, 3.1589) -- cycle;
\fill[blue!55.1, opacity=0.7] (1.7300, 0.2580, 3.1589) -- (1.7760, 0.2580, 3.1571) -- (1.7760, 0.3120, 3.1629) -- (1.7300, 0.3120, 3.1647) -- cycle;
\fill[blue!47.1, opacity=0.7] (1.7300, 0.3120, 3.1647) -- (1.7760, 0.3120, 3.1629) -- (1.7760, 0.3660, 3.1686) -- (1.7300, 0.3660, 3.1704) -- cycle;
\fill[blue!54.4, opacity=0.7] (1.7300, 0.3660, 3.1704) -- (1.7760, 0.3660, 3.1686) -- (1.7760, 0.4200, 3.1741) -- (1.7300, 0.4200, 3.1759) -- cycle;
\fill[blue!63.6, opacity=0.7] (1.7300, 0.4200, 3.1759) -- (1.7760, 0.4200, 3.1741) -- (1.7760, 0.4740, 3.1795) -- (1.7300, 0.4740, 3.1813) -- cycle;
\fill[blue!54.7, opacity=0.7] (1.7300, 0.4740, 3.1813) -- (1.7760, 0.4740, 3.1795) -- (1.7760, 0.5280, 3.1847) -- (1.7300, 0.5280, 3.1864) -- cycle;
\fill[blue!45.9, opacity=0.7] (1.7300, 0.5280, 3.1864) -- (1.7760, 0.5280, 3.1847) -- (1.7760, 0.5820, 3.1896) -- (1.7300, 0.5820, 3.1914) -- cycle;
\fill[blue!51.5, opacity=0.7] (1.7300, 0.5820, 3.1914) -- (1.7760, 0.5820, 3.1896) -- (1.7760, 0.6360, 3.1944) -- (1.7300, 0.6360, 3.1962) -- cycle;
\fill[blue!63.4, opacity=0.7] (1.7300, 0.6360, 3.1962) -- (1.7760, 0.6360, 3.1944) -- (1.7760, 0.6900, 3.1990) -- (1.7300, 0.6900, 3.2008) -- cycle;
\fill[blue!51.0, opacity=0.7] (1.7300, 0.6900, 3.2008) -- (1.7760, 0.6900, 3.1990) -- (1.7760, 0.7440, 3.2033) -- (1.7300, 0.7440, 3.2051) -- cycle;
\fill[blue!30.0, opacity=0.7] (1.7300, 0.7440, 3.2051) -- (1.7760, 0.7440, 3.2033) -- (1.7760, 0.7980, 3.2074) -- (1.7300, 0.7980, 3.2092) -- cycle;
\fill[blue!22.6, opacity=0.7] (1.7300, 0.7980, 3.2092) -- (1.7760, 0.7980, 3.2074) -- (1.7760, 0.8520, 3.2112) -- (1.7300, 0.8520, 3.2130) -- cycle;
\fill[blue!23.6, opacity=0.7] (1.7300, 0.8520, 3.2130) -- (1.7760, 0.8520, 3.2112) -- (1.7760, 0.9060, 3.2148) -- (1.7300, 0.9060, 3.2166) -- cycle;
\fill[blue!31.9, opacity=0.7] (1.7300, 0.9060, 3.2166) -- (1.7760, 0.9060, 3.2148) -- (1.7760, 0.9600, 3.2180) -- (1.7300, 0.9600, 3.2198) -- cycle;
\fill[blue!47.5, opacity=0.7] (1.7300, 0.9600, 3.2198) -- (1.7760, 0.9600, 3.2180) -- (1.7760, 1.0140, 3.2210) -- (1.7300, 1.0140, 3.2228) -- cycle;
\fill[blue!59.5, opacity=0.7] (1.7300, 1.0140, 3.2228) -- (1.7760, 1.0140, 3.2210) -- (1.7760, 1.0680, 3.2238) -- (1.7300, 1.0680, 3.2255) -- cycle;
\fill[blue!62.9, opacity=0.7] (1.7300, 1.0680, 3.2255) -- (1.7760, 1.0680, 3.2238) -- (1.7760, 1.1220, 3.2262) -- (1.7300, 1.1220, 3.2279) -- cycle;
\fill[blue!62.8, opacity=0.7] (1.7300, 1.1220, 3.2279) -- (1.7760, 1.1220, 3.2262) -- (1.7760, 1.1760, 3.2283) -- (1.7300, 1.1760, 3.2300) -- cycle;
\fill[blue!60.2, opacity=0.7] (1.7300, 1.1760, 3.2300) -- (1.7760, 1.1760, 3.2283) -- (1.7760, 1.2300, 3.2300) -- (1.7300, 1.2300, 3.2318) -- cycle;
\fill[blue!54.1, opacity=0.7] (1.7300, 1.2300, 3.2318) -- (1.7760, 1.2300, 3.2300) -- (1.7760, 1.2840, 3.2315) -- (1.7300, 1.2840, 3.2333) -- cycle;
\fill[blue!47.1, opacity=0.7] (1.7300, 1.2840, 3.2333) -- (1.7760, 1.2840, 3.2315) -- (1.7760, 1.3380, 3.2326) -- (1.7300, 1.3380, 3.2344) -- cycle;
\fill[blue!43.2, opacity=0.7] (1.7300, 1.3380, 3.2344) -- (1.7760, 1.3380, 3.2326) -- (1.7760, 1.3920, 3.2335) -- (1.7300, 1.3920, 3.2353) -- cycle;
\fill[blue!43.9, opacity=0.7] (1.7300, 1.3920, 3.2353) -- (1.7760, 1.3920, 3.2335) -- (1.7760, 1.4460, 3.2340) -- (1.7300, 1.4460, 3.2357) -- cycle;
\fill[blue!49.0, opacity=0.7] (1.7300, 1.4460, 3.2357) -- (1.7760, 1.4460, 3.2340) -- (1.7760, 1.5000, 3.2341) -- (1.7300, 1.5000, 3.2359) -- cycle;
\fill[blue!55.2, opacity=0.7] (1.7300, 1.5000, 3.2359) -- (1.7760, 1.5000, 3.2341) -- (1.7760, 1.5540, 3.2340) -- (1.7300, 1.5540, 3.2357) -- cycle;
\fill[blue!57.6, opacity=0.7] (1.7300, 1.5540, 3.2357) -- (1.7760, 1.5540, 3.2340) -- (1.7760, 1.6080, 3.2335) -- (1.7300, 1.6080, 3.2353) -- cycle;
\fill[blue!52.4, opacity=0.7] (1.7300, 1.6080, 3.2353) -- (1.7760, 1.6080, 3.2335) -- (1.7760, 1.6620, 3.2326) -- (1.7300, 1.6620, 3.2344) -- cycle;
\fill[blue!36.3, opacity=0.7] (1.7300, 1.6620, 3.2344) -- (1.7760, 1.6620, 3.2326) -- (1.7760, 1.7160, 3.2315) -- (1.7300, 1.7160, 3.2333) -- cycle;
\fill[blue!21.6, opacity=0.7] (1.7300, 1.7160, 3.2333) -- (1.7760, 1.7160, 3.2315) -- (1.7760, 1.7700, 3.2300) -- (1.7300, 1.7700, 3.2318) -- cycle;
\fill[blue!17.1, opacity=0.7] (1.7300, 1.7700, 3.2318) -- (1.7760, 1.7700, 3.2300) -- (1.7760, 1.8240, 3.2283) -- (1.7300, 1.8240, 3.2300) -- cycle;
\fill[blue!17.2, opacity=0.7] (1.7300, 1.8240, 3.2300) -- (1.7760, 1.8240, 3.2283) -- (1.7760, 1.8780, 3.2262) -- (1.7300, 1.8780, 3.2279) -- cycle;
\fill[blue!23.0, opacity=0.7] (1.7300, 1.8780, 3.2279) -- (1.7760, 1.8780, 3.2262) -- (1.7760, 1.9320, 3.2238) -- (1.7300, 1.9320, 3.2255) -- cycle;
\fill[blue!45.2, opacity=0.7] (1.7300, 1.9320, 3.2255) -- (1.7760, 1.9320, 3.2238) -- (1.7760, 1.9860, 3.2210) -- (1.7300, 1.9860, 3.2228) -- cycle;
\fill[blue!63.1, opacity=0.7] (1.7300, 1.9860, 3.2228) -- (1.7760, 1.9860, 3.2210) -- (1.7760, 2.0400, 3.2180) -- (1.7300, 2.0400, 3.2198) -- cycle;
\fill[blue!62.5, opacity=0.7] (1.7300, 2.0400, 3.2198) -- (1.7760, 2.0400, 3.2180) -- (1.7760, 2.0940, 3.2148) -- (1.7300, 2.0940, 3.2166) -- cycle;
\fill[blue!63.5, opacity=0.7] (1.7300, 2.0940, 3.2166) -- (1.7760, 2.0940, 3.2148) -- (1.7760, 2.1480, 3.2112) -- (1.7300, 2.1480, 3.2130) -- cycle;
\fill[blue!52.1, opacity=0.7] (1.7300, 2.1480, 3.2130) -- (1.7760, 2.1480, 3.2112) -- (1.7760, 2.2020, 3.2074) -- (1.7300, 2.2020, 3.2092) -- cycle;
\fill[blue!32.5, opacity=0.7] (1.7300, 2.2020, 3.2092) -- (1.7760, 2.2020, 3.2074) -- (1.7760, 2.2560, 3.2033) -- (1.7300, 2.2560, 3.2051) -- cycle;
\fill[blue!26.4, opacity=0.7] (1.7300, 2.2560, 3.2051) -- (1.7760, 2.2560, 3.2033) -- (1.7760, 2.3100, 3.1990) -- (1.7300, 2.3100, 3.2008) -- cycle;
\fill[blue!34.5, opacity=0.7] (1.7300, 2.3100, 3.2008) -- (1.7760, 2.3100, 3.1990) -- (1.7760, 2.3640, 3.1944) -- (1.7300, 2.3640, 3.1962) -- cycle;
\fill[blue!57.8, opacity=0.7] (1.7300, 2.3640, 3.1962) -- (1.7760, 2.3640, 3.1944) -- (1.7760, 2.4180, 3.1896) -- (1.7300, 2.4180, 3.1914) -- cycle;
\fill[blue!58.8, opacity=0.7] (1.7300, 2.4180, 3.1914) -- (1.7760, 2.4180, 3.1896) -- (1.7760, 2.4720, 3.1847) -- (1.7300, 2.4720, 3.1864) -- cycle;
\fill[blue!42.6, opacity=0.7] (1.7300, 2.4720, 3.1864) -- (1.7760, 2.4720, 3.1847) -- (1.7760, 2.5260, 3.1795) -- (1.7300, 2.5260, 3.1813) -- cycle;
\fill[blue!42.7, opacity=0.7] (1.7300, 2.5260, 3.1813) -- (1.7760, 2.5260, 3.1795) -- (1.7760, 2.5800, 3.1741) -- (1.7300, 2.5800, 3.1759) -- cycle;
\fill[blue!57.7, opacity=0.7] (1.7300, 2.5800, 3.1759) -- (1.7760, 2.5800, 3.1741) -- (1.7760, 2.6340, 3.1686) -- (1.7300, 2.6340, 3.1704) -- cycle;
\fill[blue!62.2, opacity=0.7] (1.7300, 2.6340, 3.1704) -- (1.7760, 2.6340, 3.1686) -- (1.7760, 2.6880, 3.1629) -- (1.7300, 2.6880, 3.1647) -- cycle;
\fill[blue!52.5, opacity=0.7] (1.7300, 2.6880, 3.1647) -- (1.7760, 2.6880, 3.1629) -- (1.7760, 2.7420, 3.1571) -- (1.7300, 2.7420, 3.1589) -- cycle;
\fill[blue!55.2, opacity=0.7] (1.7300, 2.7420, 3.1589) -- (1.7760, 2.7420, 3.1571) -- (1.7760, 2.7960, 3.1512) -- (1.7300, 2.7960, 3.1530) -- cycle;
\fill[blue!63.4, opacity=0.7] (1.7300, 2.7960, 3.1530) -- (1.7760, 2.7960, 3.1512) -- (1.7760, 2.8500, 3.1452) -- (1.7300, 2.8500, 3.1470) -- cycle;
\fill[blue!42.8, opacity=0.7] (1.7300, 2.8500, 3.1470) -- (1.7760, 2.8500, 3.1452) -- (1.7760, 2.9040, 3.1391) -- (1.7300, 2.9040, 3.1409) -- cycle;
\fill[blue!23.8, opacity=0.7] (1.7300, 2.9040, 3.1409) -- (1.7760, 2.9040, 3.1391) -- (1.7760, 2.9580, 3.1329) -- (1.7300, 2.9580, 3.1347) -- cycle;
\fill[blue!20.4, opacity=0.7] (1.7300, 2.9580, 3.1347) -- (1.7760, 2.9580, 3.1329) -- (1.7760, 3.0120, 3.1267) -- (1.7300, 3.0120, 3.1285) -- cycle;
\fill[blue!25.8, opacity=0.7] (1.7300, 3.0120, 3.1285) -- (1.7760, 3.0120, 3.1267) -- (1.7760, 3.0660, 3.1204) -- (1.7300, 3.0660, 3.1222) -- cycle;
\fill[blue!44.2, opacity=0.7] (1.7300, 3.0660, 3.1222) -- (1.7760, 3.0660, 3.1204) -- (1.7760, 3.1200, 3.1141) -- (1.7300, 3.1200, 3.1159) -- cycle;
\fill[blue!63.0, opacity=0.7] (1.7760, -0.1200, 3.1141) -- (1.8220, -0.1200, 3.1120) -- (1.8220, -0.0660, 3.1183) -- (1.7760, -0.0660, 3.1204) -- cycle;
\fill[blue!54.3, opacity=0.7] (1.7760, -0.0660, 3.1204) -- (1.8220, -0.0660, 3.1183) -- (1.8220, -0.0120, 3.1246) -- (1.7760, -0.0120, 3.1267) -- cycle;
\fill[blue!34.3, opacity=0.7] (1.7760, -0.0120, 3.1267) -- (1.8220, -0.0120, 3.1246) -- (1.8220, 0.0420, 3.1308) -- (1.7760, 0.0420, 3.1329) -- cycle;
\fill[blue!23.3, opacity=0.7] (1.7760, 0.0420, 3.1329) -- (1.8220, 0.0420, 3.1308) -- (1.8220, 0.0960, 3.1370) -- (1.7760, 0.0960, 3.1391) -- cycle;
\fill[blue!22.9, opacity=0.7] (1.7760, 0.0960, 3.1391) -- (1.8220, 0.0960, 3.1370) -- (1.8220, 0.1500, 3.1431) -- (1.7760, 0.1500, 3.1452) -- cycle;
\fill[blue!34.0, opacity=0.7] (1.7760, 0.1500, 3.1452) -- (1.8220, 0.1500, 3.1431) -- (1.8220, 0.2040, 3.1491) -- (1.7760, 0.2040, 3.1512) -- cycle;
\fill[blue!58.1, opacity=0.7] (1.7760, 0.2040, 3.1512) -- (1.8220, 0.2040, 3.1491) -- (1.8220, 0.2580, 3.1550) -- (1.7760, 0.2580, 3.1571) -- cycle;
\fill[blue!60.3, opacity=0.7] (1.7760, 0.2580, 3.1571) -- (1.8220, 0.2580, 3.1550) -- (1.8220, 0.3120, 3.1608) -- (1.7760, 0.3120, 3.1629) -- cycle;
\fill[blue!48.1, opacity=0.7] (1.7760, 0.3120, 3.1629) -- (1.8220, 0.3120, 3.1608) -- (1.8220, 0.3660, 3.1665) -- (1.7760, 0.3660, 3.1686) -- cycle;
\fill[blue!49.2, opacity=0.7] (1.7760, 0.3660, 3.1686) -- (1.8220, 0.3660, 3.1665) -- (1.8220, 0.4200, 3.1720) -- (1.7760, 0.4200, 3.1741) -- cycle;
\fill[blue!60.8, opacity=0.7] (1.7760, 0.4200, 3.1741) -- (1.8220, 0.4200, 3.1720) -- (1.8220, 0.4740, 3.1774) -- (1.7760, 0.4740, 3.1795) -- cycle;
\fill[blue!61.0, opacity=0.7] (1.7760, 0.4740, 3.1795) -- (1.8220, 0.4740, 3.1774) -- (1.8220, 0.5280, 3.1826) -- (1.7760, 0.5280, 3.1847) -- cycle;
\fill[blue!49.4, opacity=0.7] (1.7760, 0.5280, 3.1847) -- (1.8220, 0.5280, 3.1826) -- (1.8220, 0.5820, 3.1875) -- (1.7760, 0.5820, 3.1896) -- cycle;
\fill[blue!47.2, opacity=0.7] (1.7760, 0.5820, 3.1896) -- (1.8220, 0.5820, 3.1875) -- (1.8220, 0.6360, 3.1923) -- (1.7760, 0.6360, 3.1944) -- cycle;
\fill[blue!57.9, opacity=0.7] (1.7760, 0.6360, 3.1944) -- (1.8220, 0.6360, 3.1923) -- (1.8220, 0.6900, 3.1969) -- (1.7760, 0.6900, 3.1990) -- cycle;
\fill[blue!62.0, opacity=0.7] (1.7760, 0.6900, 3.1990) -- (1.8220, 0.6900, 3.1969) -- (1.8220, 0.7440, 3.2012) -- (1.7760, 0.7440, 3.2033) -- cycle;
\fill[blue!42.0, opacity=0.7] (1.7760, 0.7440, 3.2033) -- (1.8220, 0.7440, 3.2012) -- (1.8220, 0.7980, 3.2053) -- (1.7760, 0.7980, 3.2074) -- cycle;
\fill[blue!26.2, opacity=0.7] (1.7760, 0.7980, 3.2074) -- (1.8220, 0.7980, 3.2053) -- (1.8220, 0.8520, 3.2091) -- (1.7760, 0.8520, 3.2112) -- cycle;
\fill[blue!21.9, opacity=0.7] (1.7760, 0.8520, 3.2112) -- (1.8220, 0.8520, 3.2091) -- (1.8220, 0.9060, 3.2127) -- (1.7760, 0.9060, 3.2148) -- cycle;
\fill[blue!23.7, opacity=0.7] (1.7760, 0.9060, 3.2148) -- (1.8220, 0.9060, 3.2127) -- (1.8220, 0.9600, 3.2160) -- (1.7760, 0.9600, 3.2180) -- cycle;
\fill[blue!31.5, opacity=0.7] (1.7760, 0.9600, 3.2180) -- (1.8220, 0.9600, 3.2160) -- (1.8220, 1.0140, 3.2190) -- (1.7760, 1.0140, 3.2210) -- cycle;
\fill[blue!44.9, opacity=0.7] (1.7760, 1.0140, 3.2210) -- (1.8220, 1.0140, 3.2190) -- (1.8220, 1.0680, 3.2217) -- (1.7760, 1.0680, 3.2238) -- cycle;
\fill[blue!56.3, opacity=0.7] (1.7760, 1.0680, 3.2238) -- (1.8220, 1.0680, 3.2217) -- (1.8220, 1.1220, 3.2241) -- (1.7760, 1.1220, 3.2262) -- cycle;
\fill[blue!61.3, opacity=0.7] (1.7760, 1.1220, 3.2262) -- (1.8220, 1.1220, 3.2241) -- (1.8220, 1.1760, 3.2262) -- (1.7760, 1.1760, 3.2283) -- cycle;
\fill[blue!62.4, opacity=0.7] (1.7760, 1.1760, 3.2283) -- (1.8220, 1.1760, 3.2262) -- (1.8220, 1.2300, 3.2279) -- (1.7760, 1.2300, 3.2300) -- cycle;
\fill[blue!62.1, opacity=0.7] (1.7760, 1.2300, 3.2300) -- (1.8220, 1.2300, 3.2279) -- (1.8220, 1.2840, 3.2294) -- (1.7760, 1.2840, 3.2315) -- cycle;
\fill[blue!61.1, opacity=0.7] (1.7760, 1.2840, 3.2315) -- (1.8220, 1.2840, 3.2294) -- (1.8220, 1.3380, 3.2306) -- (1.7760, 1.3380, 3.2326) -- cycle;
\fill[blue!60.1, opacity=0.7] (1.7760, 1.3380, 3.2326) -- (1.8220, 1.3380, 3.2306) -- (1.8220, 1.3920, 3.2314) -- (1.7760, 1.3920, 3.2335) -- cycle;
\fill[blue!59.7, opacity=0.7] (1.7760, 1.3920, 3.2335) -- (1.8220, 1.3920, 3.2314) -- (1.8220, 1.4460, 3.2319) -- (1.7760, 1.4460, 3.2340) -- cycle;
\fill[blue!59.6, opacity=0.7] (1.7760, 1.4460, 3.2340) -- (1.8220, 1.4460, 3.2319) -- (1.8220, 1.5000, 3.2320) -- (1.7760, 1.5000, 3.2341) -- cycle;
\fill[blue!58.4, opacity=0.7] (1.7760, 1.5000, 3.2341) -- (1.8220, 1.5000, 3.2320) -- (1.8220, 1.5540, 3.2319) -- (1.7760, 1.5540, 3.2340) -- cycle;
\fill[blue!52.9, opacity=0.7] (1.7760, 1.5540, 3.2340) -- (1.8220, 1.5540, 3.2319) -- (1.8220, 1.6080, 3.2314) -- (1.7760, 1.6080, 3.2335) -- cycle;
\fill[blue!39.8, opacity=0.7] (1.7760, 1.6080, 3.2335) -- (1.8220, 1.6080, 3.2314) -- (1.8220, 1.6620, 3.2306) -- (1.7760, 1.6620, 3.2326) -- cycle;
\fill[blue!25.2, opacity=0.7] (1.7760, 1.6620, 3.2326) -- (1.8220, 1.6620, 3.2306) -- (1.8220, 1.7160, 3.2294) -- (1.7760, 1.7160, 3.2315) -- cycle;
\fill[blue!18.3, opacity=0.7] (1.7760, 1.7160, 3.2315) -- (1.8220, 1.7160, 3.2294) -- (1.8220, 1.7700, 3.2279) -- (1.7760, 1.7700, 3.2300) -- cycle;
\fill[blue!17.0, opacity=0.7] (1.7760, 1.7700, 3.2300) -- (1.8220, 1.7700, 3.2279) -- (1.8220, 1.8240, 3.2262) -- (1.7760, 1.8240, 3.2283) -- cycle;
\fill[blue!19.0, opacity=0.7] (1.7760, 1.8240, 3.2283) -- (1.8220, 1.8240, 3.2262) -- (1.8220, 1.8780, 3.2241) -- (1.7760, 1.8780, 3.2262) -- cycle;
\fill[blue!30.8, opacity=0.7] (1.7760, 1.8780, 3.2262) -- (1.8220, 1.8780, 3.2241) -- (1.8220, 1.9320, 3.2217) -- (1.7760, 1.9320, 3.2238) -- cycle;
\fill[blue!55.6, opacity=0.7] (1.7760, 1.9320, 3.2238) -- (1.8220, 1.9320, 3.2217) -- (1.8220, 1.9860, 3.2190) -- (1.7760, 1.9860, 3.2210) -- cycle;
\fill[blue!63.4, opacity=0.7] (1.7760, 1.9860, 3.2210) -- (1.8220, 1.9860, 3.2190) -- (1.8220, 2.0400, 3.2160) -- (1.7760, 2.0400, 3.2180) -- cycle;
\fill[blue!62.4, opacity=0.7] (1.7760, 2.0400, 3.2180) -- (1.8220, 2.0400, 3.2160) -- (1.8220, 2.0940, 3.2127) -- (1.7760, 2.0940, 3.2148) -- cycle;
\fill[blue!62.7, opacity=0.7] (1.7760, 2.0940, 3.2148) -- (1.8220, 2.0940, 3.2127) -- (1.8220, 2.1480, 3.2091) -- (1.7760, 2.1480, 3.2112) -- cycle;
\fill[blue!47.1, opacity=0.7] (1.7760, 2.1480, 3.2112) -- (1.8220, 2.1480, 3.2091) -- (1.8220, 2.2020, 3.2053) -- (1.7760, 2.2020, 3.2074) -- cycle;
\fill[blue!30.2, opacity=0.7] (1.7760, 2.2020, 3.2074) -- (1.8220, 2.2020, 3.2053) -- (1.8220, 2.2560, 3.2012) -- (1.7760, 2.2560, 3.2033) -- cycle;
\fill[blue!27.3, opacity=0.7] (1.7760, 2.2560, 3.2033) -- (1.8220, 2.2560, 3.2012) -- (1.8220, 2.3100, 3.1969) -- (1.7760, 2.3100, 3.1990) -- cycle;
\fill[blue!39.1, opacity=0.7] (1.7760, 2.3100, 3.1990) -- (1.8220, 2.3100, 3.1969) -- (1.8220, 2.3640, 3.1923) -- (1.7760, 2.3640, 3.1944) -- cycle;
\fill[blue!61.6, opacity=0.7] (1.7760, 2.3640, 3.1944) -- (1.8220, 2.3640, 3.1923) -- (1.8220, 2.4180, 3.1875) -- (1.7760, 2.4180, 3.1896) -- cycle;
\fill[blue!55.0, opacity=0.7] (1.7760, 2.4180, 3.1896) -- (1.8220, 2.4180, 3.1875) -- (1.8220, 2.4720, 3.1826) -- (1.7760, 2.4720, 3.1847) -- cycle;
\fill[blue!40.8, opacity=0.7] (1.7760, 2.4720, 3.1847) -- (1.8220, 2.4720, 3.1826) -- (1.8220, 2.5260, 3.1774) -- (1.7760, 2.5260, 3.1795) -- cycle;
\fill[blue!44.1, opacity=0.7] (1.7760, 2.5260, 3.1795) -- (1.8220, 2.5260, 3.1774) -- (1.8220, 2.5800, 3.1720) -- (1.7760, 2.5800, 3.1741) -- cycle;
\fill[blue!59.8, opacity=0.7] (1.7760, 2.5800, 3.1741) -- (1.8220, 2.5800, 3.1720) -- (1.8220, 2.6340, 3.1665) -- (1.7760, 2.6340, 3.1686) -- cycle;
\fill[blue!61.0, opacity=0.7] (1.7760, 2.6340, 3.1686) -- (1.8220, 2.6340, 3.1665) -- (1.8220, 2.6880, 3.1608) -- (1.7760, 2.6880, 3.1629) -- cycle;
\fill[blue!52.2, opacity=0.7] (1.7760, 2.6880, 3.1629) -- (1.8220, 2.6880, 3.1608) -- (1.8220, 2.7420, 3.1550) -- (1.7760, 2.7420, 3.1571) -- cycle;
\fill[blue!57.1, opacity=0.7] (1.7760, 2.7420, 3.1571) -- (1.8220, 2.7420, 3.1550) -- (1.8220, 2.7960, 3.1491) -- (1.7760, 2.7960, 3.1512) -- cycle;
\fill[blue!62.4, opacity=0.7] (1.7760, 2.7960, 3.1512) -- (1.8220, 2.7960, 3.1491) -- (1.8220, 2.8500, 3.1431) -- (1.7760, 2.8500, 3.1452) -- cycle;
\fill[blue!38.9, opacity=0.7] (1.7760, 2.8500, 3.1452) -- (1.8220, 2.8500, 3.1431) -- (1.8220, 2.9040, 3.1370) -- (1.7760, 2.9040, 3.1391) -- cycle;
\fill[blue!22.6, opacity=0.7] (1.7760, 2.9040, 3.1391) -- (1.8220, 2.9040, 3.1370) -- (1.8220, 2.9580, 3.1308) -- (1.7760, 2.9580, 3.1329) -- cycle;
\fill[blue!20.4, opacity=0.7] (1.7760, 2.9580, 3.1329) -- (1.8220, 2.9580, 3.1308) -- (1.8220, 3.0120, 3.1246) -- (1.7760, 3.0120, 3.1267) -- cycle;
\fill[blue!27.1, opacity=0.7] (1.7760, 3.0120, 3.1267) -- (1.8220, 3.0120, 3.1246) -- (1.8220, 3.0660, 3.1183) -- (1.7760, 3.0660, 3.1204) -- cycle;
\fill[blue!46.4, opacity=0.7] (1.7760, 3.0660, 3.1204) -- (1.8220, 3.0660, 3.1183) -- (1.8220, 3.1200, 3.1120) -- (1.7760, 3.1200, 3.1141) -- cycle;
\fill[blue!63.4, opacity=0.7] (1.8220, -0.1200, 3.1120) -- (1.8680, -0.1200, 3.1096) -- (1.8680, -0.0660, 3.1159) -- (1.8220, -0.0660, 3.1183) -- cycle;
\fill[blue!59.4, opacity=0.7] (1.8220, -0.0660, 3.1183) -- (1.8680, -0.0660, 3.1159) -- (1.8680, -0.0120, 3.1222) -- (1.8220, -0.0120, 3.1246) -- cycle;
\fill[blue!41.6, opacity=0.7] (1.8220, -0.0120, 3.1246) -- (1.8680, -0.0120, 3.1222) -- (1.8680, 0.0420, 3.1284) -- (1.8220, 0.0420, 3.1308) -- cycle;
\fill[blue!26.1, opacity=0.7] (1.8220, 0.0420, 3.1308) -- (1.8680, 0.0420, 3.1284) -- (1.8680, 0.0960, 3.1346) -- (1.8220, 0.0960, 3.1370) -- cycle;
\fill[blue!22.3, opacity=0.7] (1.8220, 0.0960, 3.1370) -- (1.8680, 0.0960, 3.1346) -- (1.8680, 0.1500, 3.1407) -- (1.8220, 0.1500, 3.1431) -- cycle;
\fill[blue!27.9, opacity=0.7] (1.8220, 0.1500, 3.1431) -- (1.8680, 0.1500, 3.1407) -- (1.8680, 0.2040, 3.1467) -- (1.8220, 0.2040, 3.1491) -- cycle;
\fill[blue!48.5, opacity=0.7] (1.8220, 0.2040, 3.1491) -- (1.8680, 0.2040, 3.1467) -- (1.8680, 0.2580, 3.1526) -- (1.8220, 0.2580, 3.1550) -- cycle;
\fill[blue!63.5, opacity=0.7] (1.8220, 0.2580, 3.1550) -- (1.8680, 0.2580, 3.1526) -- (1.8680, 0.3120, 3.1584) -- (1.8220, 0.3120, 3.1608) -- cycle;
\fill[blue!52.4, opacity=0.7] (1.8220, 0.3120, 3.1608) -- (1.8680, 0.3120, 3.1584) -- (1.8680, 0.3660, 3.1641) -- (1.8220, 0.3660, 3.1665) -- cycle;
\fill[blue!46.3, opacity=0.7] (1.8220, 0.3660, 3.1665) -- (1.8680, 0.3660, 3.1641) -- (1.8680, 0.4200, 3.1696) -- (1.8220, 0.4200, 3.1720) -- cycle;
\fill[blue!54.3, opacity=0.7] (1.8220, 0.4200, 3.1720) -- (1.8680, 0.4200, 3.1696) -- (1.8680, 0.4740, 3.1750) -- (1.8220, 0.4740, 3.1774) -- cycle;
\fill[blue!63.5, opacity=0.7] (1.8220, 0.4740, 3.1774) -- (1.8680, 0.4740, 3.1750) -- (1.8680, 0.5280, 3.1802) -- (1.8220, 0.5280, 3.1826) -- cycle;
\fill[blue!56.3, opacity=0.7] (1.8220, 0.5280, 3.1826) -- (1.8680, 0.5280, 3.1802) -- (1.8680, 0.5820, 3.1851) -- (1.8220, 0.5820, 3.1875) -- cycle;
\fill[blue!47.4, opacity=0.7] (1.8220, 0.5820, 3.1875) -- (1.8680, 0.5820, 3.1851) -- (1.8680, 0.6360, 3.1899) -- (1.8220, 0.6360, 3.1923) -- cycle;
\fill[blue!50.5, opacity=0.7] (1.8220, 0.6360, 3.1923) -- (1.8680, 0.6360, 3.1899) -- (1.8680, 0.6900, 3.1945) -- (1.8220, 0.6900, 3.1969) -- cycle;
\fill[blue!61.9, opacity=0.7] (1.8220, 0.6900, 3.1969) -- (1.8680, 0.6900, 3.1945) -- (1.8680, 0.7440, 3.1988) -- (1.8220, 0.7440, 3.2012) -- cycle;
\fill[blue!57.8, opacity=0.7] (1.8220, 0.7440, 3.2012) -- (1.8680, 0.7440, 3.1988) -- (1.8680, 0.7980, 3.2029) -- (1.8220, 0.7980, 3.2053) -- cycle;
\fill[blue!37.0, opacity=0.7] (1.8220, 0.7980, 3.2053) -- (1.8680, 0.7980, 3.2029) -- (1.8680, 0.8520, 3.2067) -- (1.8220, 0.8520, 3.2091) -- cycle;
\fill[blue!24.7, opacity=0.7] (1.8220, 0.8520, 3.2091) -- (1.8680, 0.8520, 3.2067) -- (1.8680, 0.9060, 3.2103) -- (1.8220, 0.9060, 3.2127) -- cycle;
\fill[blue!21.4, opacity=0.7] (1.8220, 0.9060, 3.2127) -- (1.8680, 0.9060, 3.2103) -- (1.8680, 0.9600, 3.2135) -- (1.8220, 0.9600, 3.2160) -- cycle;
\fill[blue!22.8, opacity=0.7] (1.8220, 0.9600, 3.2160) -- (1.8680, 0.9600, 3.2135) -- (1.8680, 1.0140, 3.2165) -- (1.8220, 1.0140, 3.2190) -- cycle;
\fill[blue!28.3, opacity=0.7] (1.8220, 1.0140, 3.2190) -- (1.8680, 1.0140, 3.2165) -- (1.8680, 1.0680, 3.2193) -- (1.8220, 1.0680, 3.2217) -- cycle;
\fill[blue!37.9, opacity=0.7] (1.8220, 1.0680, 3.2217) -- (1.8680, 1.0680, 3.2193) -- (1.8680, 1.1220, 3.2217) -- (1.8220, 1.1220, 3.2241) -- cycle;
\fill[blue!48.1, opacity=0.7] (1.8220, 1.1220, 3.2241) -- (1.8680, 1.1220, 3.2217) -- (1.8680, 1.1760, 3.2238) -- (1.8220, 1.1760, 3.2262) -- cycle;
\fill[blue!55.0, opacity=0.7] (1.8220, 1.1760, 3.2262) -- (1.8680, 1.1760, 3.2238) -- (1.8680, 1.2300, 3.2255) -- (1.8220, 1.2300, 3.2279) -- cycle;
\fill[blue!58.3, opacity=0.7] (1.8220, 1.2300, 3.2279) -- (1.8680, 1.2300, 3.2255) -- (1.8680, 1.2840, 3.2270) -- (1.8220, 1.2840, 3.2294) -- cycle;
\fill[blue!59.4, opacity=0.7] (1.8220, 1.2840, 3.2294) -- (1.8680, 1.2840, 3.2270) -- (1.8680, 1.3380, 3.2281) -- (1.8220, 1.3380, 3.2306) -- cycle;
\fill[blue!59.2, opacity=0.7] (1.8220, 1.3380, 3.2306) -- (1.8680, 1.3380, 3.2281) -- (1.8680, 1.3920, 3.2290) -- (1.8220, 1.3920, 3.2314) -- cycle;
\fill[blue!57.8, opacity=0.7] (1.8220, 1.3920, 3.2314) -- (1.8680, 1.3920, 3.2290) -- (1.8680, 1.4460, 3.2295) -- (1.8220, 1.4460, 3.2319) -- cycle;
\fill[blue!54.5, opacity=0.7] (1.8220, 1.4460, 3.2319) -- (1.8680, 1.4460, 3.2295) -- (1.8680, 1.5000, 3.2296) -- (1.8220, 1.5000, 3.2320) -- cycle;
\fill[blue!47.6, opacity=0.7] (1.8220, 1.5000, 3.2320) -- (1.8680, 1.5000, 3.2296) -- (1.8680, 1.5540, 3.2295) -- (1.8220, 1.5540, 3.2319) -- cycle;
\fill[blue!36.5, opacity=0.7] (1.8220, 1.5540, 3.2319) -- (1.8680, 1.5540, 3.2295) -- (1.8680, 1.6080, 3.2290) -- (1.8220, 1.6080, 3.2314) -- cycle;
\fill[blue!25.3, opacity=0.7] (1.8220, 1.6080, 3.2314) -- (1.8680, 1.6080, 3.2290) -- (1.8680, 1.6620, 3.2281) -- (1.8220, 1.6620, 3.2306) -- cycle;
\fill[blue!19.1, opacity=0.7] (1.8220, 1.6620, 3.2306) -- (1.8680, 1.6620, 3.2281) -- (1.8680, 1.7160, 3.2270) -- (1.8220, 1.7160, 3.2294) -- cycle;
\fill[blue!17.3, opacity=0.7] (1.8220, 1.7160, 3.2294) -- (1.8680, 1.7160, 3.2270) -- (1.8680, 1.7700, 3.2255) -- (1.8220, 1.7700, 3.2279) -- cycle;
\fill[blue!18.1, opacity=0.7] (1.8220, 1.7700, 3.2279) -- (1.8680, 1.7700, 3.2255) -- (1.8680, 1.8240, 3.2238) -- (1.8220, 1.8240, 3.2262) -- cycle;
\fill[blue!24.5, opacity=0.7] (1.8220, 1.8240, 3.2262) -- (1.8680, 1.8240, 3.2238) -- (1.8680, 1.8780, 3.2217) -- (1.8220, 1.8780, 3.2241) -- cycle;
\fill[blue!44.8, opacity=0.7] (1.8220, 1.8780, 3.2241) -- (1.8680, 1.8780, 3.2217) -- (1.8680, 1.9320, 3.2193) -- (1.8220, 1.9320, 3.2217) -- cycle;
\fill[blue!62.6, opacity=0.7] (1.8220, 1.9320, 3.2217) -- (1.8680, 1.9320, 3.2193) -- (1.8680, 1.9860, 3.2165) -- (1.8220, 1.9860, 3.2190) -- cycle;
\fill[blue!62.2, opacity=0.7] (1.8220, 1.9860, 3.2190) -- (1.8680, 1.9860, 3.2165) -- (1.8680, 2.0400, 3.2135) -- (1.8220, 2.0400, 3.2160) -- cycle;
\fill[blue!63.0, opacity=0.7] (1.8220, 2.0400, 3.2160) -- (1.8680, 2.0400, 3.2135) -- (1.8680, 2.0940, 3.2103) -- (1.8220, 2.0940, 3.2127) -- cycle;
\fill[blue!59.7, opacity=0.7] (1.8220, 2.0940, 3.2127) -- (1.8680, 2.0940, 3.2103) -- (1.8680, 2.1480, 3.2067) -- (1.8220, 2.1480, 3.2091) -- cycle;
\fill[blue!40.7, opacity=0.7] (1.8220, 2.1480, 3.2091) -- (1.8680, 2.1480, 3.2067) -- (1.8680, 2.2020, 3.2029) -- (1.8220, 2.2020, 3.2053) -- cycle;
\fill[blue!28.3, opacity=0.7] (1.8220, 2.2020, 3.2053) -- (1.8680, 2.2020, 3.2029) -- (1.8680, 2.2560, 3.1988) -- (1.8220, 2.2560, 3.2012) -- cycle;
\fill[blue!29.4, opacity=0.7] (1.8220, 2.2560, 3.2012) -- (1.8680, 2.2560, 3.1988) -- (1.8680, 2.3100, 3.1945) -- (1.8220, 2.3100, 3.1969) -- cycle;
\fill[blue!46.0, opacity=0.7] (1.8220, 2.3100, 3.1969) -- (1.8680, 2.3100, 3.1945) -- (1.8680, 2.3640, 3.1899) -- (1.8220, 2.3640, 3.1923) -- cycle;
\fill[blue!63.6, opacity=0.7] (1.8220, 2.3640, 3.1923) -- (1.8680, 2.3640, 3.1899) -- (1.8680, 2.4180, 3.1851) -- (1.8220, 2.4180, 3.1875) -- cycle;
\fill[blue!49.9, opacity=0.7] (1.8220, 2.4180, 3.1875) -- (1.8680, 2.4180, 3.1851) -- (1.8680, 2.4720, 3.1802) -- (1.8220, 2.4720, 3.1826) -- cycle;
\fill[blue!39.5, opacity=0.7] (1.8220, 2.4720, 3.1826) -- (1.8680, 2.4720, 3.1802) -- (1.8680, 2.5260, 3.1750) -- (1.8220, 2.5260, 3.1774) -- cycle;
\fill[blue!46.6, opacity=0.7] (1.8220, 2.5260, 3.1774) -- (1.8680, 2.5260, 3.1750) -- (1.8680, 2.5800, 3.1696) -- (1.8220, 2.5800, 3.1720) -- cycle;
\fill[blue!62.0, opacity=0.7] (1.8220, 2.5800, 3.1720) -- (1.8680, 2.5800, 3.1696) -- (1.8680, 2.6340, 3.1641) -- (1.8220, 2.6340, 3.1665) -- cycle;
\fill[blue!59.3, opacity=0.7] (1.8220, 2.6340, 3.1665) -- (1.8680, 2.6340, 3.1641) -- (1.8680, 2.6880, 3.1584) -- (1.8220, 2.6880, 3.1608) -- cycle;
\fill[blue!52.2, opacity=0.7] (1.8220, 2.6880, 3.1608) -- (1.8680, 2.6880, 3.1584) -- (1.8680, 2.7420, 3.1526) -- (1.8220, 2.7420, 3.1550) -- cycle;
\fill[blue!59.5, opacity=0.7] (1.8220, 2.7420, 3.1550) -- (1.8680, 2.7420, 3.1526) -- (1.8680, 2.7960, 3.1467) -- (1.8220, 2.7960, 3.1491) -- cycle;
\fill[blue!59.9, opacity=0.7] (1.8220, 2.7960, 3.1491) -- (1.8680, 2.7960, 3.1467) -- (1.8680, 2.8500, 3.1407) -- (1.8220, 2.8500, 3.1431) -- cycle;
\fill[blue!34.4, opacity=0.7] (1.8220, 2.8500, 3.1431) -- (1.8680, 2.8500, 3.1407) -- (1.8680, 2.9040, 3.1346) -- (1.8220, 2.9040, 3.1370) -- cycle;
\fill[blue!21.5, opacity=0.7] (1.8220, 2.9040, 3.1370) -- (1.8680, 2.9040, 3.1346) -- (1.8680, 2.9580, 3.1284) -- (1.8220, 2.9580, 3.1308) -- cycle;
\fill[blue!20.6, opacity=0.7] (1.8220, 2.9580, 3.1308) -- (1.8680, 2.9580, 3.1284) -- (1.8680, 3.0120, 3.1222) -- (1.8220, 3.0120, 3.1246) -- cycle;
\fill[blue!29.0, opacity=0.7] (1.8220, 3.0120, 3.1246) -- (1.8680, 3.0120, 3.1222) -- (1.8680, 3.0660, 3.1159) -- (1.8220, 3.0660, 3.1183) -- cycle;
\fill[blue!49.2, opacity=0.7] (1.8220, 3.0660, 3.1183) -- (1.8680, 3.0660, 3.1159) -- (1.8680, 3.1200, 3.1096) -- (1.8220, 3.1200, 3.1120) -- cycle;
\fill[blue!63.4, opacity=0.7] (1.8680, -0.1200, 3.1096) -- (1.9140, -0.1200, 3.1069) -- (1.9140, -0.0660, 3.1132) -- (1.8680, -0.0660, 3.1159) -- cycle;
\fill[blue!62.3, opacity=0.7] (1.8680, -0.0660, 3.1159) -- (1.9140, -0.0660, 3.1132) -- (1.9140, -0.0120, 3.1195) -- (1.8680, -0.0120, 3.1222) -- cycle;
\fill[blue!50.1, opacity=0.7] (1.8680, -0.0120, 3.1222) -- (1.9140, -0.0120, 3.1195) -- (1.9140, 0.0420, 3.1257) -- (1.8680, 0.0420, 3.1284) -- cycle;
\fill[blue!31.1, opacity=0.7] (1.8680, 0.0420, 3.1284) -- (1.9140, 0.0420, 3.1257) -- (1.9140, 0.0960, 3.1319) -- (1.8680, 0.0960, 3.1346) -- cycle;
\fill[blue!22.9, opacity=0.7] (1.8680, 0.0960, 3.1346) -- (1.9140, 0.0960, 3.1319) -- (1.9140, 0.1500, 3.1380) -- (1.8680, 0.1500, 3.1407) -- cycle;
\fill[blue!24.2, opacity=0.7] (1.8680, 0.1500, 3.1407) -- (1.9140, 0.1500, 3.1380) -- (1.9140, 0.2040, 3.1440) -- (1.8680, 0.2040, 3.1467) -- cycle;
\fill[blue!37.5, opacity=0.7] (1.8680, 0.2040, 3.1467) -- (1.9140, 0.2040, 3.1440) -- (1.9140, 0.2580, 3.1499) -- (1.8680, 0.2580, 3.1526) -- cycle;
\fill[blue!60.3, opacity=0.7] (1.8680, 0.2580, 3.1526) -- (1.9140, 0.2580, 3.1499) -- (1.9140, 0.3120, 3.1557) -- (1.8680, 0.3120, 3.1584) -- cycle;
\fill[blue!59.1, opacity=0.7] (1.8680, 0.3120, 3.1584) -- (1.9140, 0.3120, 3.1557) -- (1.9140, 0.3660, 3.1614) -- (1.8680, 0.3660, 3.1641) -- cycle;
\fill[blue!47.2, opacity=0.7] (1.8680, 0.3660, 3.1641) -- (1.9140, 0.3660, 3.1614) -- (1.9140, 0.4200, 3.1669) -- (1.8680, 0.4200, 3.1696) -- cycle;
\fill[blue!47.9, opacity=0.7] (1.8680, 0.4200, 3.1696) -- (1.9140, 0.4200, 3.1669) -- (1.9140, 0.4740, 3.1723) -- (1.8680, 0.4740, 3.1750) -- cycle;
\fill[blue!59.1, opacity=0.7] (1.8680, 0.4740, 3.1750) -- (1.9140, 0.4740, 3.1723) -- (1.9140, 0.5280, 3.1775) -- (1.8680, 0.5280, 3.1802) -- cycle;
\fill[blue!62.8, opacity=0.7] (1.8680, 0.5280, 3.1802) -- (1.9140, 0.5280, 3.1775) -- (1.9140, 0.5820, 3.1824) -- (1.8680, 0.5820, 3.1851) -- cycle;
\fill[blue!52.9, opacity=0.7] (1.8680, 0.5820, 3.1851) -- (1.9140, 0.5820, 3.1824) -- (1.9140, 0.6360, 3.1872) -- (1.8680, 0.6360, 3.1899) -- cycle;
\fill[blue!47.5, opacity=0.7] (1.8680, 0.6360, 3.1899) -- (1.9140, 0.6360, 3.1872) -- (1.9140, 0.6900, 3.1918) -- (1.8680, 0.6900, 3.1945) -- cycle;
\fill[blue!53.7, opacity=0.7] (1.8680, 0.6900, 3.1945) -- (1.9140, 0.6900, 3.1918) -- (1.9140, 0.7440, 3.1961) -- (1.8680, 0.7440, 3.1988) -- cycle;
\fill[blue!63.3, opacity=0.7] (1.8680, 0.7440, 3.1988) -- (1.9140, 0.7440, 3.1961) -- (1.9140, 0.7980, 3.2002) -- (1.8680, 0.7980, 3.2029) -- cycle;
\fill[blue!55.2, opacity=0.7] (1.8680, 0.7980, 3.2029) -- (1.9140, 0.7980, 3.2002) -- (1.9140, 0.8520, 3.2040) -- (1.8680, 0.8520, 3.2067) -- cycle;
\fill[blue!35.7, opacity=0.7] (1.8680, 0.8520, 3.2067) -- (1.9140, 0.8520, 3.2040) -- (1.9140, 0.9060, 3.2076) -- (1.8680, 0.9060, 3.2103) -- cycle;
\fill[blue!24.6, opacity=0.7] (1.8680, 0.9060, 3.2103) -- (1.9140, 0.9060, 3.2076) -- (1.9140, 0.9600, 3.2108) -- (1.8680, 0.9600, 3.2135) -- cycle;
\fill[blue!21.2, opacity=0.7] (1.8680, 0.9600, 3.2135) -- (1.9140, 0.9600, 3.2108) -- (1.9140, 1.0140, 3.2138) -- (1.8680, 1.0140, 3.2165) -- cycle;
\fill[blue!21.3, opacity=0.7] (1.8680, 1.0140, 3.2165) -- (1.9140, 1.0140, 3.2138) -- (1.9140, 1.0680, 3.2165) -- (1.8680, 1.0680, 3.2193) -- cycle;
\fill[blue!23.9, opacity=0.7] (1.8680, 1.0680, 3.2193) -- (1.9140, 1.0680, 3.2165) -- (1.9140, 1.1220, 3.2190) -- (1.8680, 1.1220, 3.2217) -- cycle;
\fill[blue!28.6, opacity=0.7] (1.8680, 1.1220, 3.2217) -- (1.9140, 1.1220, 3.2190) -- (1.9140, 1.1760, 3.2210) -- (1.8680, 1.1760, 3.2238) -- cycle;
\fill[blue!34.4, opacity=0.7] (1.8680, 1.1760, 3.2238) -- (1.9140, 1.1760, 3.2210) -- (1.9140, 1.2300, 3.2228) -- (1.8680, 1.2300, 3.2255) -- cycle;
\fill[blue!39.4, opacity=0.7] (1.8680, 1.2300, 3.2255) -- (1.9140, 1.2300, 3.2228) -- (1.9140, 1.2840, 3.2243) -- (1.8680, 1.2840, 3.2270) -- cycle;
\fill[blue!42.3, opacity=0.7] (1.8680, 1.2840, 3.2270) -- (1.9140, 1.2840, 3.2243) -- (1.9140, 1.3380, 3.2254) -- (1.8680, 1.3380, 3.2281) -- cycle;
\fill[blue!42.5, opacity=0.7] (1.8680, 1.3380, 3.2281) -- (1.9140, 1.3380, 3.2254) -- (1.9140, 1.3920, 3.2263) -- (1.8680, 1.3920, 3.2290) -- cycle;
\fill[blue!40.0, opacity=0.7] (1.8680, 1.3920, 3.2290) -- (1.9140, 1.3920, 3.2263) -- (1.9140, 1.4460, 3.2268) -- (1.8680, 1.4460, 3.2295) -- cycle;
\fill[blue!34.9, opacity=0.7] (1.8680, 1.4460, 3.2295) -- (1.9140, 1.4460, 3.2268) -- (1.9140, 1.5000, 3.2269) -- (1.8680, 1.5000, 3.2296) -- cycle;
\fill[blue!28.3, opacity=0.7] (1.8680, 1.5000, 3.2296) -- (1.9140, 1.5000, 3.2269) -- (1.9140, 1.5540, 3.2268) -- (1.8680, 1.5540, 3.2295) -- cycle;
\fill[blue!22.3, opacity=0.7] (1.8680, 1.5540, 3.2295) -- (1.9140, 1.5540, 3.2268) -- (1.9140, 1.6080, 3.2263) -- (1.8680, 1.6080, 3.2290) -- cycle;
\fill[blue!18.8, opacity=0.7] (1.8680, 1.6080, 3.2290) -- (1.9140, 1.6080, 3.2263) -- (1.9140, 1.6620, 3.2254) -- (1.8680, 1.6620, 3.2281) -- cycle;
\fill[blue!17.5, opacity=0.7] (1.8680, 1.6620, 3.2281) -- (1.9140, 1.6620, 3.2254) -- (1.9140, 1.7160, 3.2243) -- (1.8680, 1.7160, 3.2270) -- cycle;
\fill[blue!18.2, opacity=0.7] (1.8680, 1.7160, 3.2270) -- (1.9140, 1.7160, 3.2243) -- (1.9140, 1.7700, 3.2228) -- (1.8680, 1.7700, 3.2255) -- cycle;
\fill[blue!22.9, opacity=0.7] (1.8680, 1.7700, 3.2255) -- (1.9140, 1.7700, 3.2228) -- (1.9140, 1.8240, 3.2210) -- (1.8680, 1.8240, 3.2238) -- cycle;
\fill[blue!38.5, opacity=0.7] (1.8680, 1.8240, 3.2238) -- (1.9140, 1.8240, 3.2210) -- (1.9140, 1.8780, 3.2190) -- (1.8680, 1.8780, 3.2217) -- cycle;
\fill[blue!59.2, opacity=0.7] (1.8680, 1.8780, 3.2217) -- (1.9140, 1.8780, 3.2190) -- (1.9140, 1.9320, 3.2165) -- (1.8680, 1.9320, 3.2193) -- cycle;
\fill[blue!63.1, opacity=0.7] (1.8680, 1.9320, 3.2193) -- (1.9140, 1.9320, 3.2165) -- (1.9140, 1.9860, 3.2138) -- (1.8680, 1.9860, 3.2165) -- cycle;
\fill[blue!61.6, opacity=0.7] (1.8680, 1.9860, 3.2165) -- (1.9140, 1.9860, 3.2138) -- (1.9140, 2.0400, 3.2108) -- (1.8680, 2.0400, 3.2135) -- cycle;
\fill[blue!63.5, opacity=0.7] (1.8680, 2.0400, 3.2135) -- (1.9140, 2.0400, 3.2108) -- (1.9140, 2.0940, 3.2076) -- (1.8680, 2.0940, 3.2103) -- cycle;
\fill[blue!52.9, opacity=0.7] (1.8680, 2.0940, 3.2103) -- (1.9140, 2.0940, 3.2076) -- (1.9140, 2.1480, 3.2040) -- (1.8680, 2.1480, 3.2067) -- cycle;
\fill[blue!34.5, opacity=0.7] (1.8680, 2.1480, 3.2067) -- (1.9140, 2.1480, 3.2040) -- (1.9140, 2.2020, 3.2002) -- (1.8680, 2.2020, 3.2029) -- cycle;
\fill[blue!27.6, opacity=0.7] (1.8680, 2.2020, 3.2029) -- (1.9140, 2.2020, 3.2002) -- (1.9140, 2.2560, 3.1961) -- (1.8680, 2.2560, 3.1988) -- cycle;
\fill[blue!33.7, opacity=0.7] (1.8680, 2.2560, 3.1988) -- (1.9140, 2.2560, 3.1961) -- (1.9140, 2.3100, 3.1918) -- (1.8680, 2.3100, 3.1945) -- cycle;
\fill[blue!54.7, opacity=0.7] (1.8680, 2.3100, 3.1945) -- (1.9140, 2.3100, 3.1918) -- (1.9140, 2.3640, 3.1872) -- (1.8680, 2.3640, 3.1899) -- cycle;
\fill[blue!61.5, opacity=0.7] (1.8680, 2.3640, 3.1899) -- (1.9140, 2.3640, 3.1872) -- (1.9140, 2.4180, 3.1824) -- (1.8680, 2.4180, 3.1851) -- cycle;
\fill[blue!44.6, opacity=0.7] (1.8680, 2.4180, 3.1851) -- (1.9140, 2.4180, 3.1824) -- (1.9140, 2.4720, 3.1775) -- (1.8680, 2.4720, 3.1802) -- cycle;
\fill[blue!39.3, opacity=0.7] (1.8680, 2.4720, 3.1802) -- (1.9140, 2.4720, 3.1775) -- (1.9140, 2.5260, 3.1723) -- (1.8680, 2.5260, 3.1750) -- cycle;
\fill[blue!50.5, opacity=0.7] (1.8680, 2.5260, 3.1750) -- (1.9140, 2.5260, 3.1723) -- (1.9140, 2.5800, 3.1669) -- (1.8680, 2.5800, 3.1696) -- cycle;
\fill[blue!63.4, opacity=0.7] (1.8680, 2.5800, 3.1696) -- (1.9140, 2.5800, 3.1669) -- (1.9140, 2.6340, 3.1614) -- (1.8680, 2.6340, 3.1641) -- cycle;
\fill[blue!57.0, opacity=0.7] (1.8680, 2.6340, 3.1641) -- (1.9140, 2.6340, 3.1614) -- (1.9140, 2.6880, 3.1557) -- (1.8680, 2.6880, 3.1584) -- cycle;
\fill[blue!53.0, opacity=0.7] (1.8680, 2.6880, 3.1584) -- (1.9140, 2.6880, 3.1557) -- (1.9140, 2.7420, 3.1499) -- (1.8680, 2.7420, 3.1526) -- cycle;
\fill[blue!62.0, opacity=0.7] (1.8680, 2.7420, 3.1526) -- (1.9140, 2.7420, 3.1499) -- (1.9140, 2.7960, 3.1440) -- (1.8680, 2.7960, 3.1467) -- cycle;
\fill[blue!55.1, opacity=0.7] (1.8680, 2.7960, 3.1467) -- (1.9140, 2.7960, 3.1440) -- (1.9140, 2.8500, 3.1380) -- (1.8680, 2.8500, 3.1407) -- cycle;
\fill[blue!29.9, opacity=0.7] (1.8680, 2.8500, 3.1407) -- (1.9140, 2.8500, 3.1380) -- (1.9140, 2.9040, 3.1319) -- (1.8680, 2.9040, 3.1346) -- cycle;
\fill[blue!20.5, opacity=0.7] (1.8680, 2.9040, 3.1346) -- (1.9140, 2.9040, 3.1319) -- (1.9140, 2.9580, 3.1257) -- (1.8680, 2.9580, 3.1284) -- cycle;
\fill[blue!21.2, opacity=0.7] (1.8680, 2.9580, 3.1284) -- (1.9140, 2.9580, 3.1257) -- (1.9140, 3.0120, 3.1195) -- (1.8680, 3.0120, 3.1222) -- cycle;
\fill[blue!31.8, opacity=0.7] (1.8680, 3.0120, 3.1222) -- (1.9140, 3.0120, 3.1195) -- (1.9140, 3.0660, 3.1132) -- (1.8680, 3.0660, 3.1159) -- cycle;
\fill[blue!52.4, opacity=0.7] (1.8680, 3.0660, 3.1159) -- (1.9140, 3.0660, 3.1132) -- (1.9140, 3.1200, 3.1069) -- (1.8680, 3.1200, 3.1096) -- cycle;
\fill[blue!62.9, opacity=0.7] (1.9140, -0.1200, 3.1069) -- (1.9600, -0.1200, 3.1039) -- (1.9600, -0.0660, 3.1102) -- (1.9140, -0.0660, 3.1132) -- cycle;
\fill[blue!63.4, opacity=0.7] (1.9140, -0.0660, 3.1132) -- (1.9600, -0.0660, 3.1102) -- (1.9600, -0.0120, 3.1165) -- (1.9140, -0.0120, 3.1195) -- cycle;
\fill[blue!57.6, opacity=0.7] (1.9140, -0.0120, 3.1195) -- (1.9600, -0.0120, 3.1165) -- (1.9600, 0.0420, 3.1227) -- (1.9140, 0.0420, 3.1257) -- cycle;
\fill[blue!39.1, opacity=0.7] (1.9140, 0.0420, 3.1257) -- (1.9600, 0.0420, 3.1227) -- (1.9600, 0.0960, 3.1289) -- (1.9140, 0.0960, 3.1319) -- cycle;
\fill[blue!25.5, opacity=0.7] (1.9140, 0.0960, 3.1319) -- (1.9600, 0.0960, 3.1289) -- (1.9600, 0.1500, 3.1350) -- (1.9140, 0.1500, 3.1380) -- cycle;
\fill[blue!22.7, opacity=0.7] (1.9140, 0.1500, 3.1380) -- (1.9600, 0.1500, 3.1350) -- (1.9600, 0.2040, 3.1410) -- (1.9140, 0.2040, 3.1440) -- cycle;
\fill[blue!29.1, opacity=0.7] (1.9140, 0.2040, 3.1440) -- (1.9600, 0.2040, 3.1410) -- (1.9600, 0.2580, 3.1469) -- (1.9140, 0.2580, 3.1499) -- cycle;
\fill[blue!49.4, opacity=0.7] (1.9140, 0.2580, 3.1499) -- (1.9600, 0.2580, 3.1469) -- (1.9600, 0.3120, 3.1527) -- (1.9140, 0.3120, 3.1557) -- cycle;
\fill[blue!63.5, opacity=0.7] (1.9140, 0.3120, 3.1557) -- (1.9600, 0.3120, 3.1527) -- (1.9600, 0.3660, 3.1584) -- (1.9140, 0.3660, 3.1614) -- cycle;
\fill[blue!52.8, opacity=0.7] (1.9140, 0.3660, 3.1614) -- (1.9600, 0.3660, 3.1584) -- (1.9600, 0.4200, 3.1639) -- (1.9140, 0.4200, 3.1669) -- cycle;
\fill[blue!45.3, opacity=0.7] (1.9140, 0.4200, 3.1669) -- (1.9600, 0.4200, 3.1639) -- (1.9600, 0.4740, 3.1693) -- (1.9140, 0.4740, 3.1723) -- cycle;
\fill[blue!50.8, opacity=0.7] (1.9140, 0.4740, 3.1723) -- (1.9600, 0.4740, 3.1693) -- (1.9600, 0.5280, 3.1745) -- (1.9140, 0.5280, 3.1775) -- cycle;
\fill[blue!61.9, opacity=0.7] (1.9140, 0.5280, 3.1775) -- (1.9600, 0.5280, 3.1745) -- (1.9600, 0.5820, 3.1794) -- (1.9140, 0.5820, 3.1824) -- cycle;
\fill[blue!61.0, opacity=0.7] (1.9140, 0.5820, 3.1824) -- (1.9600, 0.5820, 3.1794) -- (1.9600, 0.6360, 3.1842) -- (1.9140, 0.6360, 3.1872) -- cycle;
\fill[blue!51.4, opacity=0.7] (1.9140, 0.6360, 3.1872) -- (1.9600, 0.6360, 3.1842) -- (1.9600, 0.6900, 3.1888) -- (1.9140, 0.6900, 3.1918) -- cycle;
\fill[blue!48.4, opacity=0.7] (1.9140, 0.6900, 3.1918) -- (1.9600, 0.6900, 3.1888) -- (1.9600, 0.7440, 3.1931) -- (1.9140, 0.7440, 3.1961) -- cycle;
\fill[blue!55.4, opacity=0.7] (1.9140, 0.7440, 3.1961) -- (1.9600, 0.7440, 3.1931) -- (1.9600, 0.7980, 3.1972) -- (1.9140, 0.7980, 3.2002) -- cycle;
\fill[blue!63.5, opacity=0.7] (1.9140, 0.7980, 3.2002) -- (1.9600, 0.7980, 3.1972) -- (1.9600, 0.8520, 3.2010) -- (1.9140, 0.8520, 3.2040) -- cycle;
\fill[blue!55.6, opacity=0.7] (1.9140, 0.8520, 3.2040) -- (1.9600, 0.8520, 3.2010) -- (1.9600, 0.9060, 3.2046) -- (1.9140, 0.9060, 3.2076) -- cycle;
\fill[blue!37.9, opacity=0.7] (1.9140, 0.9060, 3.2076) -- (1.9600, 0.9060, 3.2046) -- (1.9600, 0.9600, 3.2078) -- (1.9140, 0.9600, 3.2108) -- cycle;
\fill[blue!26.2, opacity=0.7] (1.9140, 0.9600, 3.2108) -- (1.9600, 0.9600, 3.2078) -- (1.9600, 1.0140, 3.2108) -- (1.9140, 1.0140, 3.2138) -- cycle;
\fill[blue!21.6, opacity=0.7] (1.9140, 1.0140, 3.2138) -- (1.9600, 1.0140, 3.2108) -- (1.9600, 1.0680, 3.2135) -- (1.9140, 1.0680, 3.2165) -- cycle;
\fill[blue!20.3, opacity=0.7] (1.9140, 1.0680, 3.2165) -- (1.9600, 1.0680, 3.2135) -- (1.9600, 1.1220, 3.2160) -- (1.9140, 1.1220, 3.2190) -- cycle;
\fill[blue!20.6, opacity=0.7] (1.9140, 1.1220, 3.2190) -- (1.9600, 1.1220, 3.2160) -- (1.9600, 1.1760, 3.2180) -- (1.9140, 1.1760, 3.2210) -- cycle;
\fill[blue!21.8, opacity=0.7] (1.9140, 1.1760, 3.2210) -- (1.9600, 1.1760, 3.2180) -- (1.9600, 1.2300, 3.2198) -- (1.9140, 1.2300, 3.2228) -- cycle;
\fill[blue!23.1, opacity=0.7] (1.9140, 1.2300, 3.2228) -- (1.9600, 1.2300, 3.2198) -- (1.9600, 1.2840, 3.2213) -- (1.9140, 1.2840, 3.2243) -- cycle;
\fill[blue!24.0, opacity=0.7] (1.9140, 1.2840, 3.2243) -- (1.9600, 1.2840, 3.2213) -- (1.9600, 1.3380, 3.2224) -- (1.9140, 1.3380, 3.2254) -- cycle;
\fill[blue!24.0, opacity=0.7] (1.9140, 1.3380, 3.2254) -- (1.9600, 1.3380, 3.2224) -- (1.9600, 1.3920, 3.2233) -- (1.9140, 1.3920, 3.2263) -- cycle;
\fill[blue!22.9, opacity=0.7] (1.9140, 1.3920, 3.2263) -- (1.9600, 1.3920, 3.2233) -- (1.9600, 1.4460, 3.2238) -- (1.9140, 1.4460, 3.2268) -- cycle;
\fill[blue!21.1, opacity=0.7] (1.9140, 1.4460, 3.2268) -- (1.9600, 1.4460, 3.2238) -- (1.9600, 1.5000, 3.2239) -- (1.9140, 1.5000, 3.2269) -- cycle;
\fill[blue!19.3, opacity=0.7] (1.9140, 1.5000, 3.2269) -- (1.9600, 1.5000, 3.2239) -- (1.9600, 1.5540, 3.2238) -- (1.9140, 1.5540, 3.2268) -- cycle;
\fill[blue!18.1, opacity=0.7] (1.9140, 1.5540, 3.2268) -- (1.9600, 1.5540, 3.2238) -- (1.9600, 1.6080, 3.2233) -- (1.9140, 1.6080, 3.2263) -- cycle;
\fill[blue!17.9, opacity=0.7] (1.9140, 1.6080, 3.2263) -- (1.9600, 1.6080, 3.2233) -- (1.9600, 1.6620, 3.2224) -- (1.9140, 1.6620, 3.2254) -- cycle;
\fill[blue!19.1, opacity=0.7] (1.9140, 1.6620, 3.2254) -- (1.9600, 1.6620, 3.2224) -- (1.9600, 1.7160, 3.2213) -- (1.9140, 1.7160, 3.2243) -- cycle;
\fill[blue!24.0, opacity=0.7] (1.9140, 1.7160, 3.2243) -- (1.9600, 1.7160, 3.2213) -- (1.9600, 1.7700, 3.2198) -- (1.9140, 1.7700, 3.2228) -- cycle;
\fill[blue!37.8, opacity=0.7] (1.9140, 1.7700, 3.2228) -- (1.9600, 1.7700, 3.2198) -- (1.9600, 1.8240, 3.2180) -- (1.9140, 1.8240, 3.2210) -- cycle;
\fill[blue!57.2, opacity=0.7] (1.9140, 1.8240, 3.2210) -- (1.9600, 1.8240, 3.2180) -- (1.9600, 1.8780, 3.2160) -- (1.9140, 1.8780, 3.2190) -- cycle;
\fill[blue!63.5, opacity=0.7] (1.9140, 1.8780, 3.2190) -- (1.9600, 1.8780, 3.2160) -- (1.9600, 1.9320, 3.2135) -- (1.9140, 1.9320, 3.2165) -- cycle;
\fill[blue!61.1, opacity=0.7] (1.9140, 1.9320, 3.2165) -- (1.9600, 1.9320, 3.2135) -- (1.9600, 1.9860, 3.2108) -- (1.9140, 1.9860, 3.2138) -- cycle;
\fill[blue!62.9, opacity=0.7] (1.9140, 1.9860, 3.2138) -- (1.9600, 1.9860, 3.2108) -- (1.9600, 2.0400, 3.2078) -- (1.9140, 2.0400, 3.2108) -- cycle;
\fill[blue!60.4, opacity=0.7] (1.9140, 2.0400, 3.2108) -- (1.9600, 2.0400, 3.2078) -- (1.9600, 2.0940, 3.2046) -- (1.9140, 2.0940, 3.2076) -- cycle;
\fill[blue!43.3, opacity=0.7] (1.9140, 2.0940, 3.2076) -- (1.9600, 2.0940, 3.2046) -- (1.9600, 2.1480, 3.2010) -- (1.9140, 2.1480, 3.2040) -- cycle;
\fill[blue!30.1, opacity=0.7] (1.9140, 2.1480, 3.2040) -- (1.9600, 2.1480, 3.2010) -- (1.9600, 2.2020, 3.1972) -- (1.9140, 2.2020, 3.2002) -- cycle;
\fill[blue!29.0, opacity=0.7] (1.9140, 2.2020, 3.2002) -- (1.9600, 2.2020, 3.1972) -- (1.9600, 2.2560, 3.1931) -- (1.9140, 2.2560, 3.1961) -- cycle;
\fill[blue!41.6, opacity=0.7] (1.9140, 2.2560, 3.1961) -- (1.9600, 2.2560, 3.1931) -- (1.9600, 2.3100, 3.1888) -- (1.9140, 2.3100, 3.1918) -- cycle;
\fill[blue!62.1, opacity=0.7] (1.9140, 2.3100, 3.1918) -- (1.9600, 2.3100, 3.1888) -- (1.9600, 2.3640, 3.1842) -- (1.9140, 2.3640, 3.1872) -- cycle;
\fill[blue!55.1, opacity=0.7] (1.9140, 2.3640, 3.1872) -- (1.9600, 2.3640, 3.1842) -- (1.9600, 2.4180, 3.1794) -- (1.9140, 2.4180, 3.1824) -- cycle;
\fill[blue!40.2, opacity=0.7] (1.9140, 2.4180, 3.1824) -- (1.9600, 2.4180, 3.1794) -- (1.9600, 2.4720, 3.1745) -- (1.9140, 2.4720, 3.1775) -- cycle;
\fill[blue!40.9, opacity=0.7] (1.9140, 2.4720, 3.1775) -- (1.9600, 2.4720, 3.1745) -- (1.9600, 2.5260, 3.1693) -- (1.9140, 2.5260, 3.1723) -- cycle;
\fill[blue!55.6, opacity=0.7] (1.9140, 2.5260, 3.1723) -- (1.9600, 2.5260, 3.1693) -- (1.9600, 2.5800, 3.1639) -- (1.9140, 2.5800, 3.1669) -- cycle;
\fill[blue!63.2, opacity=0.7] (1.9140, 2.5800, 3.1669) -- (1.9600, 2.5800, 3.1639) -- (1.9600, 2.6340, 3.1584) -- (1.9140, 2.6340, 3.1614) -- cycle;
\fill[blue!54.8, opacity=0.7] (1.9140, 2.6340, 3.1614) -- (1.9600, 2.6340, 3.1584) -- (1.9600, 2.6880, 3.1527) -- (1.9140, 2.6880, 3.1557) -- cycle;
\fill[blue!55.0, opacity=0.7] (1.9140, 2.6880, 3.1557) -- (1.9600, 2.6880, 3.1527) -- (1.9600, 2.7420, 3.1469) -- (1.9140, 2.7420, 3.1499) -- cycle;
\fill[blue!63.5, opacity=0.7] (1.9140, 2.7420, 3.1499) -- (1.9600, 2.7420, 3.1469) -- (1.9600, 2.7960, 3.1410) -- (1.9140, 2.7960, 3.1440) -- cycle;
\fill[blue!48.0, opacity=0.7] (1.9140, 2.7960, 3.1440) -- (1.9600, 2.7960, 3.1410) -- (1.9600, 2.8500, 3.1350) -- (1.9140, 2.8500, 3.1380) -- cycle;
\fill[blue!25.8, opacity=0.7] (1.9140, 2.8500, 3.1380) -- (1.9600, 2.8500, 3.1350) -- (1.9600, 2.9040, 3.1289) -- (1.9140, 2.9040, 3.1319) -- cycle;
\fill[blue!19.9, opacity=0.7] (1.9140, 2.9040, 3.1319) -- (1.9600, 2.9040, 3.1289) -- (1.9600, 2.9580, 3.1227) -- (1.9140, 2.9580, 3.1257) -- cycle;
\fill[blue!22.3, opacity=0.7] (1.9140, 2.9580, 3.1257) -- (1.9600, 2.9580, 3.1227) -- (1.9600, 3.0120, 3.1165) -- (1.9140, 3.0120, 3.1195) -- cycle;
\fill[blue!35.7, opacity=0.7] (1.9140, 3.0120, 3.1195) -- (1.9600, 3.0120, 3.1165) -- (1.9600, 3.0660, 3.1102) -- (1.9140, 3.0660, 3.1132) -- cycle;
\fill[blue!55.5, opacity=0.7] (1.9140, 3.0660, 3.1132) -- (1.9600, 3.0660, 3.1102) -- (1.9600, 3.1200, 3.1039) -- (1.9140, 3.1200, 3.1069) -- cycle;
\fill[blue!60.4, opacity=0.7] (1.9600, -0.1200, 3.1039) -- (2.0060, -0.1200, 3.1006) -- (2.0060, -0.0660, 3.1069) -- (1.9600, -0.0660, 3.1102) -- cycle;
\fill[blue!63.5, opacity=0.7] (1.9600, -0.0660, 3.1102) -- (2.0060, -0.0660, 3.1069) -- (2.0060, -0.0120, 3.1132) -- (1.9600, -0.0120, 3.1165) -- cycle;
\fill[blue!62.0, opacity=0.7] (1.9600, -0.0120, 3.1165) -- (2.0060, -0.0120, 3.1132) -- (2.0060, 0.0420, 3.1194) -- (1.9600, 0.0420, 3.1227) -- cycle;
\fill[blue!49.2, opacity=0.7] (1.9600, 0.0420, 3.1227) -- (2.0060, 0.0420, 3.1194) -- (2.0060, 0.0960, 3.1256) -- (1.9600, 0.0960, 3.1289) -- cycle;
\fill[blue!31.1, opacity=0.7] (1.9600, 0.0960, 3.1289) -- (2.0060, 0.0960, 3.1256) -- (2.0060, 0.1500, 3.1317) -- (1.9600, 0.1500, 3.1350) -- cycle;
\fill[blue!23.3, opacity=0.7] (1.9600, 0.1500, 3.1350) -- (2.0060, 0.1500, 3.1317) -- (2.0060, 0.2040, 3.1377) -- (1.9600, 0.2040, 3.1410) -- cycle;
\fill[blue!24.4, opacity=0.7] (1.9600, 0.2040, 3.1410) -- (2.0060, 0.2040, 3.1377) -- (2.0060, 0.2580, 3.1436) -- (1.9600, 0.2580, 3.1469) -- cycle;
\fill[blue!36.5, opacity=0.7] (1.9600, 0.2580, 3.1469) -- (2.0060, 0.2580, 3.1436) -- (2.0060, 0.3120, 3.1494) -- (1.9600, 0.3120, 3.1527) -- cycle;
\fill[blue!58.5, opacity=0.7] (1.9600, 0.3120, 3.1527) -- (2.0060, 0.3120, 3.1494) -- (2.0060, 0.3660, 3.1551) -- (1.9600, 0.3660, 3.1584) -- cycle;
\fill[blue!60.9, opacity=0.7] (1.9600, 0.3660, 3.1584) -- (2.0060, 0.3660, 3.1551) -- (2.0060, 0.4200, 3.1606) -- (1.9600, 0.4200, 3.1639) -- cycle;
\fill[blue!48.4, opacity=0.7] (1.9600, 0.4200, 3.1639) -- (2.0060, 0.4200, 3.1606) -- (2.0060, 0.4740, 3.1660) -- (1.9600, 0.4740, 3.1693) -- cycle;
\fill[blue!45.1, opacity=0.7] (1.9600, 0.4740, 3.1693) -- (2.0060, 0.4740, 3.1660) -- (2.0060, 0.5280, 3.1712) -- (1.9600, 0.5280, 3.1745) -- cycle;
\fill[blue!53.2, opacity=0.7] (1.9600, 0.5280, 3.1745) -- (2.0060, 0.5280, 3.1712) -- (2.0060, 0.5820, 3.1762) -- (1.9600, 0.5820, 3.1794) -- cycle;
\fill[blue!62.9, opacity=0.7] (1.9600, 0.5820, 3.1794) -- (2.0060, 0.5820, 3.1762) -- (2.0060, 0.6360, 3.1809) -- (1.9600, 0.6360, 3.1842) -- cycle;
\fill[blue!60.0, opacity=0.7] (1.9600, 0.6360, 3.1842) -- (2.0060, 0.6360, 3.1809) -- (2.0060, 0.6900, 3.1855) -- (1.9600, 0.6900, 3.1888) -- cycle;
\fill[blue!51.3, opacity=0.7] (1.9600, 0.6900, 3.1888) -- (2.0060, 0.6900, 3.1855) -- (2.0060, 0.7440, 3.1898) -- (1.9600, 0.7440, 3.1931) -- cycle;
\fill[blue!49.2, opacity=0.7] (1.9600, 0.7440, 3.1931) -- (2.0060, 0.7440, 3.1898) -- (2.0060, 0.7980, 3.1939) -- (1.9600, 0.7980, 3.1972) -- cycle;
\fill[blue!55.5, opacity=0.7] (1.9600, 0.7980, 3.1972) -- (2.0060, 0.7980, 3.1939) -- (2.0060, 0.8520, 3.1977) -- (1.9600, 0.8520, 3.2010) -- cycle;
\fill[blue!63.1, opacity=0.7] (1.9600, 0.8520, 3.2010) -- (2.0060, 0.8520, 3.1977) -- (2.0060, 0.9060, 3.2013) -- (1.9600, 0.9060, 3.2046) -- cycle;
\fill[blue!59.1, opacity=0.7] (1.9600, 0.9060, 3.2046) -- (2.0060, 0.9060, 3.2013) -- (2.0060, 0.9600, 3.2046) -- (1.9600, 0.9600, 3.2078) -- cycle;
\fill[blue!44.2, opacity=0.7] (1.9600, 0.9600, 3.2078) -- (2.0060, 0.9600, 3.2046) -- (2.0060, 1.0140, 3.2076) -- (1.9600, 1.0140, 3.2108) -- cycle;
\fill[blue!31.2, opacity=0.7] (1.9600, 1.0140, 3.2108) -- (2.0060, 1.0140, 3.2076) -- (2.0060, 1.0680, 3.2103) -- (1.9600, 1.0680, 3.2135) -- cycle;
\fill[blue!24.3, opacity=0.7] (1.9600, 1.0680, 3.2135) -- (2.0060, 1.0680, 3.2103) -- (2.0060, 1.1220, 3.2127) -- (1.9600, 1.1220, 3.2160) -- cycle;
\fill[blue!21.2, opacity=0.7] (1.9600, 1.1220, 3.2160) -- (2.0060, 1.1220, 3.2127) -- (2.0060, 1.1760, 3.2148) -- (1.9600, 1.1760, 3.2180) -- cycle;
\fill[blue!20.0, opacity=0.7] (1.9600, 1.1760, 3.2180) -- (2.0060, 1.1760, 3.2148) -- (2.0060, 1.2300, 3.2166) -- (1.9600, 1.2300, 3.2198) -- cycle;
\fill[blue!19.5, opacity=0.7] (1.9600, 1.2300, 3.2198) -- (2.0060, 1.2300, 3.2166) -- (2.0060, 1.2840, 3.2180) -- (1.9600, 1.2840, 3.2213) -- cycle;
\fill[blue!19.2, opacity=0.7] (1.9600, 1.2840, 3.2213) -- (2.0060, 1.2840, 3.2180) -- (2.0060, 1.3380, 3.2192) -- (1.9600, 1.3380, 3.2224) -- cycle;
\fill[blue!19.0, opacity=0.7] (1.9600, 1.3380, 3.2224) -- (2.0060, 1.3380, 3.2192) -- (2.0060, 1.3920, 3.2200) -- (1.9600, 1.3920, 3.2233) -- cycle;
\fill[blue!18.7, opacity=0.7] (1.9600, 1.3920, 3.2233) -- (2.0060, 1.3920, 3.2200) -- (2.0060, 1.4460, 3.2205) -- (1.9600, 1.4460, 3.2238) -- cycle;
\fill[blue!18.5, opacity=0.7] (1.9600, 1.4460, 3.2238) -- (2.0060, 1.4460, 3.2205) -- (2.0060, 1.5000, 3.2206) -- (1.9600, 1.5000, 3.2239) -- cycle;
\fill[blue!18.5, opacity=0.7] (1.9600, 1.5000, 3.2239) -- (2.0060, 1.5000, 3.2206) -- (2.0060, 1.5540, 3.2205) -- (1.9600, 1.5540, 3.2238) -- cycle;
\fill[blue!19.3, opacity=0.7] (1.9600, 1.5540, 3.2238) -- (2.0060, 1.5540, 3.2205) -- (2.0060, 1.6080, 3.2200) -- (1.9600, 1.6080, 3.2233) -- cycle;
\fill[blue!21.8, opacity=0.7] (1.9600, 1.6080, 3.2233) -- (2.0060, 1.6080, 3.2200) -- (2.0060, 1.6620, 3.2192) -- (1.9600, 1.6620, 3.2224) -- cycle;
\fill[blue!28.6, opacity=0.7] (1.9600, 1.6620, 3.2224) -- (2.0060, 1.6620, 3.2192) -- (2.0060, 1.7160, 3.2180) -- (1.9600, 1.7160, 3.2213) -- cycle;
\fill[blue!42.5, opacity=0.7] (1.9600, 1.7160, 3.2213) -- (2.0060, 1.7160, 3.2180) -- (2.0060, 1.7700, 3.2166) -- (1.9600, 1.7700, 3.2198) -- cycle;
\fill[blue!58.5, opacity=0.7] (1.9600, 1.7700, 3.2198) -- (2.0060, 1.7700, 3.2166) -- (2.0060, 1.8240, 3.2148) -- (1.9600, 1.8240, 3.2180) -- cycle;
\fill[blue!63.5, opacity=0.7] (1.9600, 1.8240, 3.2180) -- (2.0060, 1.8240, 3.2148) -- (2.0060, 1.8780, 3.2127) -- (1.9600, 1.8780, 3.2160) -- cycle;
\fill[blue!60.9, opacity=0.7] (1.9600, 1.8780, 3.2160) -- (2.0060, 1.8780, 3.2127) -- (2.0060, 1.9320, 3.2103) -- (1.9600, 1.9320, 3.2135) -- cycle;
\fill[blue!61.8, opacity=0.7] (1.9600, 1.9320, 3.2135) -- (2.0060, 1.9320, 3.2103) -- (2.0060, 1.9860, 3.2076) -- (1.9600, 1.9860, 3.2108) -- cycle;
\fill[blue!63.0, opacity=0.7] (1.9600, 1.9860, 3.2108) -- (2.0060, 1.9860, 3.2076) -- (2.0060, 2.0400, 3.2046) -- (1.9600, 2.0400, 3.2078) -- cycle;
\fill[blue!51.0, opacity=0.7] (1.9600, 2.0400, 3.2078) -- (2.0060, 2.0400, 3.2046) -- (2.0060, 2.0940, 3.2013) -- (1.9600, 2.0940, 3.2046) -- cycle;
\fill[blue!34.6, opacity=0.7] (1.9600, 2.0940, 3.2046) -- (2.0060, 2.0940, 3.2013) -- (2.0060, 2.1480, 3.1977) -- (1.9600, 2.1480, 3.2010) -- cycle;
\fill[blue!28.5, opacity=0.7] (1.9600, 2.1480, 3.2010) -- (2.0060, 2.1480, 3.1977) -- (2.0060, 2.2020, 3.1939) -- (1.9600, 2.2020, 3.1972) -- cycle;
\fill[blue!34.0, opacity=0.7] (1.9600, 2.2020, 3.1972) -- (2.0060, 2.2020, 3.1939) -- (2.0060, 2.2560, 3.1898) -- (1.9600, 2.2560, 3.1931) -- cycle;
\fill[blue!53.1, opacity=0.7] (1.9600, 2.2560, 3.1931) -- (2.0060, 2.2560, 3.1898) -- (2.0060, 2.3100, 3.1855) -- (1.9600, 2.3100, 3.1888) -- cycle;
\fill[blue!62.8, opacity=0.7] (1.9600, 2.3100, 3.1888) -- (2.0060, 2.3100, 3.1855) -- (2.0060, 2.3640, 3.1809) -- (1.9600, 2.3640, 3.1842) -- cycle;
\fill[blue!46.8, opacity=0.7] (1.9600, 2.3640, 3.1842) -- (2.0060, 2.3640, 3.1809) -- (2.0060, 2.4180, 3.1762) -- (1.9600, 2.4180, 3.1794) -- cycle;
\fill[blue!38.0, opacity=0.7] (1.9600, 2.4180, 3.1794) -- (2.0060, 2.4180, 3.1762) -- (2.0060, 2.4720, 3.1712) -- (1.9600, 2.4720, 3.1745) -- cycle;
\fill[blue!44.9, opacity=0.7] (1.9600, 2.4720, 3.1745) -- (2.0060, 2.4720, 3.1712) -- (2.0060, 2.5260, 3.1660) -- (1.9600, 2.5260, 3.1693) -- cycle;
\fill[blue!60.6, opacity=0.7] (1.9600, 2.5260, 3.1693) -- (2.0060, 2.5260, 3.1660) -- (2.0060, 2.5800, 3.1606) -- (1.9600, 2.5800, 3.1639) -- cycle;
\fill[blue!61.0, opacity=0.7] (1.9600, 2.5800, 3.1639) -- (2.0060, 2.5800, 3.1606) -- (2.0060, 2.6340, 3.1551) -- (1.9600, 2.6340, 3.1584) -- cycle;
\fill[blue!53.5, opacity=0.7] (1.9600, 2.6340, 3.1584) -- (2.0060, 2.6340, 3.1551) -- (2.0060, 2.6880, 3.1494) -- (1.9600, 2.6880, 3.1527) -- cycle;
\fill[blue!58.3, opacity=0.7] (1.9600, 2.6880, 3.1527) -- (2.0060, 2.6880, 3.1494) -- (2.0060, 2.7420, 3.1436) -- (1.9600, 2.7420, 3.1469) -- cycle;
\fill[blue!62.3, opacity=0.7] (1.9600, 2.7420, 3.1469) -- (2.0060, 2.7420, 3.1436) -- (2.0060, 2.7960, 3.1377) -- (1.9600, 2.7960, 3.1410) -- cycle;
\fill[blue!39.5, opacity=0.7] (1.9600, 2.7960, 3.1410) -- (2.0060, 2.7960, 3.1377) -- (2.0060, 2.8500, 3.1317) -- (1.9600, 2.8500, 3.1350) -- cycle;
\fill[blue!22.7, opacity=0.7] (1.9600, 2.8500, 3.1350) -- (2.0060, 2.8500, 3.1317) -- (2.0060, 2.9040, 3.1256) -- (1.9600, 2.9040, 3.1289) -- cycle;
\fill[blue!19.7, opacity=0.7] (1.9600, 2.9040, 3.1289) -- (2.0060, 2.9040, 3.1256) -- (2.0060, 2.9580, 3.1194) -- (1.9600, 2.9580, 3.1227) -- cycle;
\fill[blue!24.3, opacity=0.7] (1.9600, 2.9580, 3.1227) -- (2.0060, 2.9580, 3.1194) -- (2.0060, 3.0120, 3.1132) -- (1.9600, 3.0120, 3.1165) -- cycle;
\fill[blue!40.8, opacity=0.7] (1.9600, 3.0120, 3.1165) -- (2.0060, 3.0120, 3.1132) -- (2.0060, 3.0660, 3.1069) -- (1.9600, 3.0660, 3.1102) -- cycle;
\fill[blue!58.3, opacity=0.7] (1.9600, 3.0660, 3.1102) -- (2.0060, 3.0660, 3.1069) -- (2.0060, 3.1200, 3.1006) -- (1.9600, 3.1200, 3.1039) -- cycle;
\fill[blue!53.9, opacity=0.7] (2.0060, -0.1200, 3.1006) -- (2.0520, -0.1200, 3.0971) -- (2.0520, -0.0660, 3.1034) -- (2.0060, -0.0660, 3.1069) -- cycle;
\fill[blue!63.2, opacity=0.7] (2.0060, -0.0660, 3.1069) -- (2.0520, -0.0660, 3.1034) -- (2.0520, -0.0120, 3.1096) -- (2.0060, -0.0120, 3.1132) -- cycle;
\fill[blue!63.4, opacity=0.7] (2.0060, -0.0120, 3.1132) -- (2.0520, -0.0120, 3.1096) -- (2.0520, 0.0420, 3.1159) -- (2.0060, 0.0420, 3.1194) -- cycle;
\fill[blue!58.1, opacity=0.7] (2.0060, 0.0420, 3.1194) -- (2.0520, 0.0420, 3.1159) -- (2.0520, 0.0960, 3.1220) -- (2.0060, 0.0960, 3.1256) -- cycle;
\fill[blue!40.6, opacity=0.7] (2.0060, 0.0960, 3.1256) -- (2.0520, 0.0960, 3.1220) -- (2.0520, 0.1500, 3.1281) -- (2.0060, 0.1500, 3.1317) -- cycle;
\fill[blue!26.6, opacity=0.7] (2.0060, 0.1500, 3.1317) -- (2.0520, 0.1500, 3.1281) -- (2.0520, 0.2040, 3.1342) -- (2.0060, 0.2040, 3.1377) -- cycle;
\fill[blue!23.0, opacity=0.7] (2.0060, 0.2040, 3.1377) -- (2.0520, 0.2040, 3.1342) -- (2.0520, 0.2580, 3.1401) -- (2.0060, 0.2580, 3.1436) -- cycle;
\fill[blue!27.6, opacity=0.7] (2.0060, 0.2580, 3.1436) -- (2.0520, 0.2580, 3.1401) -- (2.0520, 0.3120, 3.1459) -- (2.0060, 0.3120, 3.1494) -- cycle;
\fill[blue!44.6, opacity=0.7] (2.0060, 0.3120, 3.1494) -- (2.0520, 0.3120, 3.1459) -- (2.0520, 0.3660, 3.1516) -- (2.0060, 0.3660, 3.1551) -- cycle;
\fill[blue!62.8, opacity=0.7] (2.0060, 0.3660, 3.1551) -- (2.0520, 0.3660, 3.1516) -- (2.0520, 0.4200, 3.1571) -- (2.0060, 0.4200, 3.1606) -- cycle;
\fill[blue!57.0, opacity=0.7] (2.0060, 0.4200, 3.1606) -- (2.0520, 0.4200, 3.1571) -- (2.0520, 0.4740, 3.1624) -- (2.0060, 0.4740, 3.1660) -- cycle;
\fill[blue!45.9, opacity=0.7] (2.0060, 0.4740, 3.1660) -- (2.0520, 0.4740, 3.1624) -- (2.0520, 0.5280, 3.1676) -- (2.0060, 0.5280, 3.1712) -- cycle;
\fill[blue!45.4, opacity=0.7] (2.0060, 0.5280, 3.1712) -- (2.0520, 0.5280, 3.1676) -- (2.0520, 0.5820, 3.1726) -- (2.0060, 0.5820, 3.1762) -- cycle;
\fill[blue!54.4, opacity=0.7] (2.0060, 0.5820, 3.1762) -- (2.0520, 0.5820, 3.1726) -- (2.0520, 0.6360, 3.1774) -- (2.0060, 0.6360, 3.1809) -- cycle;
\fill[blue!63.1, opacity=0.7] (2.0060, 0.6360, 3.1809) -- (2.0520, 0.6360, 3.1774) -- (2.0520, 0.6900, 3.1819) -- (2.0060, 0.6900, 3.1855) -- cycle;
\fill[blue!60.3, opacity=0.7] (2.0060, 0.6900, 3.1855) -- (2.0520, 0.6900, 3.1819) -- (2.0520, 0.7440, 3.1863) -- (2.0060, 0.7440, 3.1898) -- cycle;
\fill[blue!52.4, opacity=0.7] (2.0060, 0.7440, 3.1898) -- (2.0520, 0.7440, 3.1863) -- (2.0520, 0.7980, 3.1903) -- (2.0060, 0.7980, 3.1939) -- cycle;
\fill[blue!49.7, opacity=0.7] (2.0060, 0.7980, 3.1939) -- (2.0520, 0.7980, 3.1903) -- (2.0520, 0.8520, 3.1942) -- (2.0060, 0.8520, 3.1977) -- cycle;
\fill[blue!54.1, opacity=0.7] (2.0060, 0.8520, 3.1977) -- (2.0520, 0.8520, 3.1942) -- (2.0520, 0.9060, 3.1977) -- (2.0060, 0.9060, 3.2013) -- cycle;
\fill[blue!61.4, opacity=0.7] (2.0060, 0.9060, 3.2013) -- (2.0520, 0.9060, 3.1977) -- (2.0520, 0.9600, 3.2010) -- (2.0060, 0.9600, 3.2046) -- cycle;
\fill[blue!63.0, opacity=0.7] (2.0060, 0.9600, 3.2046) -- (2.0520, 0.9600, 3.2010) -- (2.0520, 1.0140, 3.2040) -- (2.0060, 1.0140, 3.2076) -- cycle;
\fill[blue!54.6, opacity=0.7] (2.0060, 1.0140, 3.2076) -- (2.0520, 1.0140, 3.2040) -- (2.0520, 1.0680, 3.2067) -- (2.0060, 1.0680, 3.2103) -- cycle;
\fill[blue!42.4, opacity=0.7] (2.0060, 1.0680, 3.2103) -- (2.0520, 1.0680, 3.2067) -- (2.0520, 1.1220, 3.2091) -- (2.0060, 1.1220, 3.2127) -- cycle;
\fill[blue!32.9, opacity=0.7] (2.0060, 1.1220, 3.2127) -- (2.0520, 1.1220, 3.2091) -- (2.0520, 1.1760, 3.2112) -- (2.0060, 1.1760, 3.2148) -- cycle;
\fill[blue!27.2, opacity=0.7] (2.0060, 1.1760, 3.2148) -- (2.0520, 1.1760, 3.2112) -- (2.0520, 1.2300, 3.2130) -- (2.0060, 1.2300, 3.2166) -- cycle;
\fill[blue!24.1, opacity=0.7] (2.0060, 1.2300, 3.2166) -- (2.0520, 1.2300, 3.2130) -- (2.0520, 1.2840, 3.2145) -- (2.0060, 1.2840, 3.2180) -- cycle;
\fill[blue!22.5, opacity=0.7] (2.0060, 1.2840, 3.2180) -- (2.0520, 1.2840, 3.2145) -- (2.0520, 1.3380, 3.2156) -- (2.0060, 1.3380, 3.2192) -- cycle;
\fill[blue!21.9, opacity=0.7] (2.0060, 1.3380, 3.2192) -- (2.0520, 1.3380, 3.2156) -- (2.0520, 1.3920, 3.2164) -- (2.0060, 1.3920, 3.2200) -- cycle;
\fill[blue!21.9, opacity=0.7] (2.0060, 1.3920, 3.2200) -- (2.0520, 1.3920, 3.2164) -- (2.0520, 1.4460, 3.2169) -- (2.0060, 1.4460, 3.2205) -- cycle;
\fill[blue!22.9, opacity=0.7] (2.0060, 1.4460, 3.2205) -- (2.0520, 1.4460, 3.2169) -- (2.0520, 1.5000, 3.2171) -- (2.0060, 1.5000, 3.2206) -- cycle;
\fill[blue!25.3, opacity=0.7] (2.0060, 1.5000, 3.2206) -- (2.0520, 1.5000, 3.2171) -- (2.0520, 1.5540, 3.2169) -- (2.0060, 1.5540, 3.2205) -- cycle;
\fill[blue!30.3, opacity=0.7] (2.0060, 1.5540, 3.2205) -- (2.0520, 1.5540, 3.2169) -- (2.0520, 1.6080, 3.2164) -- (2.0060, 1.6080, 3.2200) -- cycle;
\fill[blue!39.5, opacity=0.7] (2.0060, 1.6080, 3.2200) -- (2.0520, 1.6080, 3.2164) -- (2.0520, 1.6620, 3.2156) -- (2.0060, 1.6620, 3.2192) -- cycle;
\fill[blue!52.3, opacity=0.7] (2.0060, 1.6620, 3.2192) -- (2.0520, 1.6620, 3.2156) -- (2.0520, 1.7160, 3.2145) -- (2.0060, 1.7160, 3.2180) -- cycle;
\fill[blue!62.0, opacity=0.7] (2.0060, 1.7160, 3.2180) -- (2.0520, 1.7160, 3.2145) -- (2.0520, 1.7700, 3.2130) -- (2.0060, 1.7700, 3.2166) -- cycle;
\fill[blue!63.0, opacity=0.7] (2.0060, 1.7700, 3.2166) -- (2.0520, 1.7700, 3.2130) -- (2.0520, 1.8240, 3.2112) -- (2.0060, 1.8240, 3.2148) -- cycle;
\fill[blue!60.3, opacity=0.7] (2.0060, 1.8240, 3.2148) -- (2.0520, 1.8240, 3.2112) -- (2.0520, 1.8780, 3.2091) -- (2.0060, 1.8780, 3.2127) -- cycle;
\fill[blue!61.2, opacity=0.7] (2.0060, 1.8780, 3.2127) -- (2.0520, 1.8780, 3.2091) -- (2.0520, 1.9320, 3.2067) -- (2.0060, 1.9320, 3.2103) -- cycle;
\fill[blue!63.5, opacity=0.7] (2.0060, 1.9320, 3.2103) -- (2.0520, 1.9320, 3.2067) -- (2.0520, 1.9860, 3.2040) -- (2.0060, 1.9860, 3.2076) -- cycle;
\fill[blue!55.5, opacity=0.7] (2.0060, 1.9860, 3.2076) -- (2.0520, 1.9860, 3.2040) -- (2.0520, 2.0400, 3.2010) -- (2.0060, 2.0400, 3.2046) -- cycle;
\fill[blue!39.2, opacity=0.7] (2.0060, 2.0400, 3.2046) -- (2.0520, 2.0400, 3.2010) -- (2.0520, 2.0940, 3.1977) -- (2.0060, 2.0940, 3.2013) -- cycle;
\fill[blue!29.8, opacity=0.7] (2.0060, 2.0940, 3.2013) -- (2.0520, 2.0940, 3.1977) -- (2.0520, 2.1480, 3.1942) -- (2.0060, 2.1480, 3.1977) -- cycle;
\fill[blue!30.9, opacity=0.7] (2.0060, 2.1480, 3.1977) -- (2.0520, 2.1480, 3.1942) -- (2.0520, 2.2020, 3.1903) -- (2.0060, 2.2020, 3.1939) -- cycle;
\fill[blue!44.3, opacity=0.7] (2.0060, 2.2020, 3.1939) -- (2.0520, 2.2020, 3.1903) -- (2.0520, 2.2560, 3.1863) -- (2.0060, 2.2560, 3.1898) -- cycle;
\fill[blue!62.5, opacity=0.7] (2.0060, 2.2560, 3.1898) -- (2.0520, 2.2560, 3.1863) -- (2.0520, 2.3100, 3.1819) -- (2.0060, 2.3100, 3.1855) -- cycle;
\fill[blue!55.2, opacity=0.7] (2.0060, 2.3100, 3.1855) -- (2.0520, 2.3100, 3.1819) -- (2.0520, 2.3640, 3.1774) -- (2.0060, 2.3640, 3.1809) -- cycle;
\fill[blue!40.1, opacity=0.7] (2.0060, 2.3640, 3.1809) -- (2.0520, 2.3640, 3.1774) -- (2.0520, 2.4180, 3.1726) -- (2.0060, 2.4180, 3.1762) -- cycle;
\fill[blue!38.8, opacity=0.7] (2.0060, 2.4180, 3.1762) -- (2.0520, 2.4180, 3.1726) -- (2.0520, 2.4720, 3.1676) -- (2.0060, 2.4720, 3.1712) -- cycle;
\fill[blue!51.5, opacity=0.7] (2.0060, 2.4720, 3.1712) -- (2.0520, 2.4720, 3.1676) -- (2.0520, 2.5260, 3.1624) -- (2.0060, 2.5260, 3.1660) -- cycle;
\fill[blue!63.4, opacity=0.7] (2.0060, 2.5260, 3.1660) -- (2.0520, 2.5260, 3.1624) -- (2.0520, 2.5800, 3.1571) -- (2.0060, 2.5800, 3.1606) -- cycle;
\fill[blue!57.7, opacity=0.7] (2.0060, 2.5800, 3.1606) -- (2.0520, 2.5800, 3.1571) -- (2.0520, 2.6340, 3.1516) -- (2.0060, 2.6340, 3.1551) -- cycle;
\fill[blue!54.0, opacity=0.7] (2.0060, 2.6340, 3.1551) -- (2.0520, 2.6340, 3.1516) -- (2.0520, 2.6880, 3.1459) -- (2.0060, 2.6880, 3.1494) -- cycle;
\fill[blue!62.0, opacity=0.7] (2.0060, 2.6880, 3.1494) -- (2.0520, 2.6880, 3.1459) -- (2.0520, 2.7420, 3.1401) -- (2.0060, 2.7420, 3.1436) -- cycle;
\fill[blue!56.4, opacity=0.7] (2.0060, 2.7420, 3.1436) -- (2.0520, 2.7420, 3.1401) -- (2.0520, 2.7960, 3.1342) -- (2.0060, 2.7960, 3.1377) -- cycle;
\fill[blue!31.5, opacity=0.7] (2.0060, 2.7960, 3.1377) -- (2.0520, 2.7960, 3.1342) -- (2.0520, 2.8500, 3.1281) -- (2.0060, 2.8500, 3.1317) -- cycle;
\fill[blue!20.7, opacity=0.7] (2.0060, 2.8500, 3.1317) -- (2.0520, 2.8500, 3.1281) -- (2.0520, 2.9040, 3.1220) -- (2.0060, 2.9040, 3.1256) -- cycle;
\fill[blue!20.1, opacity=0.7] (2.0060, 2.9040, 3.1256) -- (2.0520, 2.9040, 3.1220) -- (2.0520, 2.9580, 3.1159) -- (2.0060, 2.9580, 3.1194) -- cycle;
\fill[blue!27.6, opacity=0.7] (2.0060, 2.9580, 3.1194) -- (2.0520, 2.9580, 3.1159) -- (2.0520, 3.0120, 3.1096) -- (2.0060, 3.0120, 3.1132) -- cycle;
\fill[blue!46.6, opacity=0.7] (2.0060, 3.0120, 3.1132) -- (2.0520, 3.0120, 3.1096) -- (2.0520, 3.0660, 3.1034) -- (2.0060, 3.0660, 3.1069) -- cycle;
\fill[blue!60.3, opacity=0.7] (2.0060, 3.0660, 3.1069) -- (2.0520, 3.0660, 3.1034) -- (2.0520, 3.1200, 3.0971) -- (2.0060, 3.1200, 3.1006) -- cycle;
\fill[blue!42.4, opacity=0.7] (2.0520, -0.1200, 3.0971) -- (2.0980, -0.1200, 3.0933) -- (2.0980, -0.0660, 3.0995) -- (2.0520, -0.0660, 3.1034) -- cycle;
\fill[blue!60.8, opacity=0.7] (2.0520, -0.0660, 3.1034) -- (2.0980, -0.0660, 3.0995) -- (2.0980, -0.0120, 3.1058) -- (2.0520, -0.0120, 3.1096) -- cycle;
\fill[blue!63.6, opacity=0.7] (2.0520, -0.0120, 3.1096) -- (2.0980, -0.0120, 3.1058) -- (2.0980, 0.0420, 3.1120) -- (2.0520, 0.0420, 3.1159) -- cycle;
\fill[blue!62.6, opacity=0.7] (2.0520, 0.0420, 3.1159) -- (2.0980, 0.0420, 3.1120) -- (2.0980, 0.0960, 3.1182) -- (2.0520, 0.0960, 3.1220) -- cycle;
\fill[blue!52.1, opacity=0.7] (2.0520, 0.0960, 3.1220) -- (2.0980, 0.0960, 3.1182) -- (2.0980, 0.1500, 3.1243) -- (2.0520, 0.1500, 3.1281) -- cycle;
\fill[blue!34.2, opacity=0.7] (2.0520, 0.1500, 3.1281) -- (2.0980, 0.1500, 3.1243) -- (2.0980, 0.2040, 3.1303) -- (2.0520, 0.2040, 3.1342) -- cycle;
\fill[blue!24.6, opacity=0.7] (2.0520, 0.2040, 3.1342) -- (2.0980, 0.2040, 3.1303) -- (2.0980, 0.2580, 3.1363) -- (2.0520, 0.2580, 3.1401) -- cycle;
\fill[blue!23.8, opacity=0.7] (2.0520, 0.2580, 3.1401) -- (2.0980, 0.2580, 3.1363) -- (2.0980, 0.3120, 3.1421) -- (2.0520, 0.3120, 3.1459) -- cycle;
\fill[blue!31.6, opacity=0.7] (2.0520, 0.3120, 3.1459) -- (2.0980, 0.3120, 3.1421) -- (2.0980, 0.3660, 3.1477) -- (2.0520, 0.3660, 3.1516) -- cycle;
\fill[blue!51.2, opacity=0.7] (2.0520, 0.3660, 3.1516) -- (2.0980, 0.3660, 3.1477) -- (2.0980, 0.4200, 3.1533) -- (2.0520, 0.4200, 3.1571) -- cycle;
\fill[blue!63.6, opacity=0.7] (2.0520, 0.4200, 3.1571) -- (2.0980, 0.4200, 3.1533) -- (2.0980, 0.4740, 3.1586) -- (2.0520, 0.4740, 3.1624) -- cycle;
\fill[blue!54.1, opacity=0.7] (2.0520, 0.4740, 3.1624) -- (2.0980, 0.4740, 3.1586) -- (2.0980, 0.5280, 3.1638) -- (2.0520, 0.5280, 3.1676) -- cycle;
\fill[blue!44.6, opacity=0.7] (2.0520, 0.5280, 3.1676) -- (2.0980, 0.5280, 3.1638) -- (2.0980, 0.5820, 3.1688) -- (2.0520, 0.5820, 3.1726) -- cycle;
\fill[blue!45.2, opacity=0.7] (2.0520, 0.5820, 3.1726) -- (2.0980, 0.5820, 3.1688) -- (2.0980, 0.6360, 3.1736) -- (2.0520, 0.6360, 3.1774) -- cycle;
\fill[blue!54.1, opacity=0.7] (2.0520, 0.6360, 3.1774) -- (2.0980, 0.6360, 3.1736) -- (2.0980, 0.6900, 3.1781) -- (2.0520, 0.6900, 3.1819) -- cycle;
\fill[blue!62.7, opacity=0.7] (2.0520, 0.6900, 3.1819) -- (2.0980, 0.6900, 3.1781) -- (2.0980, 0.7440, 3.1824) -- (2.0520, 0.7440, 3.1863) -- cycle;
\fill[blue!61.6, opacity=0.7] (2.0520, 0.7440, 3.1863) -- (2.0980, 0.7440, 3.1824) -- (2.0980, 0.7980, 3.1865) -- (2.0520, 0.7980, 3.1903) -- cycle;
\fill[blue!54.7, opacity=0.7] (2.0520, 0.7980, 3.1903) -- (2.0980, 0.7980, 3.1865) -- (2.0980, 0.8520, 3.1903) -- (2.0520, 0.8520, 3.1942) -- cycle;
\fill[blue!50.7, opacity=0.7] (2.0520, 0.8520, 3.1942) -- (2.0980, 0.8520, 3.1903) -- (2.0980, 0.9060, 3.1939) -- (2.0520, 0.9060, 3.1977) -- cycle;
\fill[blue!52.1, opacity=0.7] (2.0520, 0.9060, 3.1977) -- (2.0980, 0.9060, 3.1939) -- (2.0980, 0.9600, 3.1972) -- (2.0520, 0.9600, 3.2010) -- cycle;
\fill[blue!57.5, opacity=0.7] (2.0520, 0.9600, 3.2010) -- (2.0980, 0.9600, 3.1972) -- (2.0980, 1.0140, 3.2002) -- (2.0520, 1.0140, 3.2040) -- cycle;
\fill[blue!62.6, opacity=0.7] (2.0520, 1.0140, 3.2040) -- (2.0980, 1.0140, 3.2002) -- (2.0980, 1.0680, 3.2029) -- (2.0520, 1.0680, 3.2067) -- cycle;
\fill[blue!63.0, opacity=0.7] (2.0520, 1.0680, 3.2067) -- (2.0980, 1.0680, 3.2029) -- (2.0980, 1.1220, 3.2053) -- (2.0520, 1.1220, 3.2091) -- cycle;
\fill[blue!58.2, opacity=0.7] (2.0520, 1.1220, 3.2091) -- (2.0980, 1.1220, 3.2053) -- (2.0980, 1.1760, 3.2074) -- (2.0520, 1.1760, 3.2112) -- cycle;
\fill[blue!51.4, opacity=0.7] (2.0520, 1.1760, 3.2112) -- (2.0980, 1.1760, 3.2074) -- (2.0980, 1.2300, 3.2092) -- (2.0520, 1.2300, 3.2130) -- cycle;
\fill[blue!45.4, opacity=0.7] (2.0520, 1.2300, 3.2130) -- (2.0980, 1.2300, 3.2092) -- (2.0980, 1.2840, 3.2106) -- (2.0520, 1.2840, 3.2145) -- cycle;
\fill[blue!41.5, opacity=0.7] (2.0520, 1.2840, 3.2145) -- (2.0980, 1.2840, 3.2106) -- (2.0980, 1.3380, 3.2118) -- (2.0520, 1.3380, 3.2156) -- cycle;
\fill[blue!39.9, opacity=0.7] (2.0520, 1.3380, 3.2156) -- (2.0980, 1.3380, 3.2118) -- (2.0980, 1.3920, 3.2126) -- (2.0520, 1.3920, 3.2164) -- cycle;
\fill[blue!40.6, opacity=0.7] (2.0520, 1.3920, 3.2164) -- (2.0980, 1.3920, 3.2126) -- (2.0980, 1.4460, 3.2131) -- (2.0520, 1.4460, 3.2169) -- cycle;
\fill[blue!43.7, opacity=0.7] (2.0520, 1.4460, 3.2169) -- (2.0980, 1.4460, 3.2131) -- (2.0980, 1.5000, 3.2133) -- (2.0520, 1.5000, 3.2171) -- cycle;
\fill[blue!49.2, opacity=0.7] (2.0520, 1.5000, 3.2171) -- (2.0980, 1.5000, 3.2133) -- (2.0980, 1.5540, 3.2131) -- (2.0520, 1.5540, 3.2169) -- cycle;
\fill[blue!56.2, opacity=0.7] (2.0520, 1.5540, 3.2169) -- (2.0980, 1.5540, 3.2131) -- (2.0980, 1.6080, 3.2126) -- (2.0520, 1.6080, 3.2164) -- cycle;
\fill[blue!62.0, opacity=0.7] (2.0520, 1.6080, 3.2164) -- (2.0980, 1.6080, 3.2126) -- (2.0980, 1.6620, 3.2118) -- (2.0520, 1.6620, 3.2156) -- cycle;
\fill[blue!63.5, opacity=0.7] (2.0520, 1.6620, 3.2156) -- (2.0980, 1.6620, 3.2118) -- (2.0980, 1.7160, 3.2106) -- (2.0520, 1.7160, 3.2145) -- cycle;
\fill[blue!61.2, opacity=0.7] (2.0520, 1.7160, 3.2145) -- (2.0980, 1.7160, 3.2106) -- (2.0980, 1.7700, 3.2092) -- (2.0520, 1.7700, 3.2130) -- cycle;
\fill[blue!59.3, opacity=0.7] (2.0520, 1.7700, 3.2130) -- (2.0980, 1.7700, 3.2092) -- (2.0980, 1.8240, 3.2074) -- (2.0520, 1.8240, 3.2112) -- cycle;
\fill[blue!61.1, opacity=0.7] (2.0520, 1.8240, 3.2112) -- (2.0980, 1.8240, 3.2074) -- (2.0980, 1.8780, 3.2053) -- (2.0520, 1.8780, 3.2091) -- cycle;
\fill[blue!63.6, opacity=0.7] (2.0520, 1.8780, 3.2091) -- (2.0980, 1.8780, 3.2053) -- (2.0980, 1.9320, 3.2029) -- (2.0520, 1.9320, 3.2067) -- cycle;
\fill[blue!57.1, opacity=0.7] (2.0520, 1.9320, 3.2067) -- (2.0980, 1.9320, 3.2029) -- (2.0980, 1.9860, 3.2002) -- (2.0520, 1.9860, 3.2040) -- cycle;
\fill[blue!42.1, opacity=0.7] (2.0520, 1.9860, 3.2040) -- (2.0980, 1.9860, 3.2002) -- (2.0980, 2.0400, 3.1972) -- (2.0520, 2.0400, 3.2010) -- cycle;
\fill[blue!31.5, opacity=0.7] (2.0520, 2.0400, 3.2010) -- (2.0980, 2.0400, 3.1972) -- (2.0980, 2.0940, 3.1939) -- (2.0520, 2.0940, 3.1977) -- cycle;
\fill[blue!30.1, opacity=0.7] (2.0520, 2.0940, 3.1977) -- (2.0980, 2.0940, 3.1939) -- (2.0980, 2.1480, 3.1903) -- (2.0520, 2.1480, 3.1942) -- cycle;
\fill[blue!39.2, opacity=0.7] (2.0520, 2.1480, 3.1942) -- (2.0980, 2.1480, 3.1903) -- (2.0980, 2.2020, 3.1865) -- (2.0520, 2.2020, 3.1903) -- cycle;
\fill[blue!58.0, opacity=0.7] (2.0520, 2.2020, 3.1903) -- (2.0980, 2.2020, 3.1865) -- (2.0980, 2.2560, 3.1824) -- (2.0520, 2.2560, 3.1863) -- cycle;
\fill[blue!61.1, opacity=0.7] (2.0520, 2.2560, 3.1863) -- (2.0980, 2.2560, 3.1824) -- (2.0980, 2.3100, 3.1781) -- (2.0520, 2.3100, 3.1819) -- cycle;
\fill[blue!44.8, opacity=0.7] (2.0520, 2.3100, 3.1819) -- (2.0980, 2.3100, 3.1781) -- (2.0980, 2.3640, 3.1736) -- (2.0520, 2.3640, 3.1774) -- cycle;
\fill[blue!37.1, opacity=0.7] (2.0520, 2.3640, 3.1774) -- (2.0980, 2.3640, 3.1736) -- (2.0980, 2.4180, 3.1688) -- (2.0520, 2.4180, 3.1726) -- cycle;
\fill[blue!43.3, opacity=0.7] (2.0520, 2.4180, 3.1726) -- (2.0980, 2.4180, 3.1688) -- (2.0980, 2.4720, 3.1638) -- (2.0520, 2.4720, 3.1676) -- cycle;
\fill[blue!58.9, opacity=0.7] (2.0520, 2.4720, 3.1676) -- (2.0980, 2.4720, 3.1638) -- (2.0980, 2.5260, 3.1586) -- (2.0520, 2.5260, 3.1624) -- cycle;
\fill[blue!62.4, opacity=0.7] (2.0520, 2.5260, 3.1624) -- (2.0980, 2.5260, 3.1586) -- (2.0980, 2.5800, 3.1533) -- (2.0520, 2.5800, 3.1571) -- cycle;
\fill[blue!54.9, opacity=0.7] (2.0520, 2.5800, 3.1571) -- (2.0980, 2.5800, 3.1533) -- (2.0980, 2.6340, 3.1477) -- (2.0520, 2.6340, 3.1516) -- cycle;
\fill[blue!56.9, opacity=0.7] (2.0520, 2.6340, 3.1516) -- (2.0980, 2.6340, 3.1477) -- (2.0980, 2.6880, 3.1421) -- (2.0520, 2.6880, 3.1459) -- cycle;
\fill[blue!63.5, opacity=0.7] (2.0520, 2.6880, 3.1459) -- (2.0980, 2.6880, 3.1421) -- (2.0980, 2.7420, 3.1363) -- (2.0520, 2.7420, 3.1401) -- cycle;
\fill[blue!46.2, opacity=0.7] (2.0520, 2.7420, 3.1401) -- (2.0980, 2.7420, 3.1363) -- (2.0980, 2.7960, 3.1303) -- (2.0520, 2.7960, 3.1342) -- cycle;
\fill[blue!25.4, opacity=0.7] (2.0520, 2.7960, 3.1342) -- (2.0980, 2.7960, 3.1303) -- (2.0980, 2.8500, 3.1243) -- (2.0520, 2.8500, 3.1281) -- cycle;
\fill[blue!19.6, opacity=0.7] (2.0520, 2.8500, 3.1281) -- (2.0980, 2.8500, 3.1243) -- (2.0980, 2.9040, 3.1182) -- (2.0520, 2.9040, 3.1220) -- cycle;
\fill[blue!21.3, opacity=0.7] (2.0520, 2.9040, 3.1220) -- (2.0980, 2.9040, 3.1182) -- (2.0980, 2.9580, 3.1120) -- (2.0520, 2.9580, 3.1159) -- cycle;
\fill[blue!32.6, opacity=0.7] (2.0520, 2.9580, 3.1159) -- (2.0980, 2.9580, 3.1120) -- (2.0980, 3.0120, 3.1058) -- (2.0520, 3.0120, 3.1096) -- cycle;
\fill[blue!52.4, opacity=0.7] (2.0520, 3.0120, 3.1096) -- (2.0980, 3.0120, 3.1058) -- (2.0980, 3.0660, 3.0995) -- (2.0520, 3.0660, 3.1034) -- cycle;
\fill[blue!61.4, opacity=0.7] (2.0520, 3.0660, 3.1034) -- (2.0980, 3.0660, 3.0995) -- (2.0980, 3.1200, 3.0933) -- (2.0520, 3.1200, 3.0971) -- cycle;
\fill[blue!30.0, opacity=0.7] (2.0980, -0.1200, 3.0933) -- (2.1440, -0.1200, 3.0892) -- (2.1440, -0.0660, 3.0955) -- (2.0980, -0.0660, 3.0995) -- cycle;
\fill[blue!53.1, opacity=0.7] (2.0980, -0.0660, 3.0995) -- (2.1440, -0.0660, 3.0955) -- (2.1440, -0.0120, 3.1017) -- (2.0980, -0.0120, 3.1058) -- cycle;
\fill[blue!63.1, opacity=0.7] (2.0980, -0.0120, 3.1058) -- (2.1440, -0.0120, 3.1017) -- (2.1440, 0.0420, 3.1079) -- (2.0980, 0.0420, 3.1120) -- cycle;
\fill[blue!63.6, opacity=0.7] (2.0980, 0.0420, 3.1120) -- (2.1440, 0.0420, 3.1079) -- (2.1440, 0.0960, 3.1141) -- (2.0980, 0.0960, 3.1182) -- cycle;
\fill[blue!60.6, opacity=0.7] (2.0980, 0.0960, 3.1182) -- (2.1440, 0.0960, 3.1141) -- (2.1440, 0.1500, 3.1202) -- (2.0980, 0.1500, 3.1243) -- cycle;
\fill[blue!46.1, opacity=0.7] (2.0980, 0.1500, 3.1243) -- (2.1440, 0.1500, 3.1202) -- (2.1440, 0.2040, 3.1263) -- (2.0980, 0.2040, 3.1303) -- cycle;
\fill[blue!30.2, opacity=0.7] (2.0980, 0.2040, 3.1303) -- (2.1440, 0.2040, 3.1263) -- (2.1440, 0.2580, 3.1322) -- (2.0980, 0.2580, 3.1363) -- cycle;
\fill[blue!23.8, opacity=0.7] (2.0980, 0.2580, 3.1363) -- (2.1440, 0.2580, 3.1322) -- (2.1440, 0.3120, 3.1380) -- (2.0980, 0.3120, 3.1421) -- cycle;
\fill[blue!25.1, opacity=0.7] (2.0980, 0.3120, 3.1421) -- (2.1440, 0.3120, 3.1380) -- (2.1440, 0.3660, 3.1437) -- (2.0980, 0.3660, 3.1477) -- cycle;
\fill[blue!35.4, opacity=0.7] (2.0980, 0.3660, 3.1477) -- (2.1440, 0.3660, 3.1437) -- (2.1440, 0.4200, 3.1492) -- (2.0980, 0.4200, 3.1533) -- cycle;
\fill[blue!55.2, opacity=0.7] (2.0980, 0.4200, 3.1533) -- (2.1440, 0.4200, 3.1492) -- (2.1440, 0.4740, 3.1545) -- (2.0980, 0.4740, 3.1586) -- cycle;
\fill[blue!63.2, opacity=0.7] (2.0980, 0.4740, 3.1586) -- (2.1440, 0.4740, 3.1545) -- (2.1440, 0.5280, 3.1597) -- (2.0980, 0.5280, 3.1638) -- cycle;
\fill[blue!52.6, opacity=0.7] (2.0980, 0.5280, 3.1638) -- (2.1440, 0.5280, 3.1597) -- (2.1440, 0.5820, 3.1647) -- (2.0980, 0.5820, 3.1688) -- cycle;
\fill[blue!43.9, opacity=0.7] (2.0980, 0.5820, 3.1688) -- (2.1440, 0.5820, 3.1647) -- (2.1440, 0.6360, 3.1695) -- (2.0980, 0.6360, 3.1736) -- cycle;
\fill[blue!44.3, opacity=0.7] (2.0980, 0.6360, 3.1736) -- (2.1440, 0.6360, 3.1695) -- (2.1440, 0.6900, 3.1740) -- (2.0980, 0.6900, 3.1781) -- cycle;
\fill[blue!52.1, opacity=0.7] (2.0980, 0.6900, 3.1781) -- (2.1440, 0.6900, 3.1740) -- (2.1440, 0.7440, 3.1784) -- (2.0980, 0.7440, 3.1824) -- cycle;
\fill[blue!61.1, opacity=0.7] (2.0980, 0.7440, 3.1824) -- (2.1440, 0.7440, 3.1784) -- (2.1440, 0.7980, 3.1824) -- (2.0980, 0.7980, 3.1865) -- cycle;
\fill[blue!63.2, opacity=0.7] (2.0980, 0.7980, 3.1865) -- (2.1440, 0.7980, 3.1824) -- (2.1440, 0.8520, 3.1863) -- (2.0980, 0.8520, 3.1903) -- cycle;
\fill[blue!58.5, opacity=0.7] (2.0980, 0.8520, 3.1903) -- (2.1440, 0.8520, 3.1863) -- (2.1440, 0.9060, 3.1898) -- (2.0980, 0.9060, 3.1939) -- cycle;
\fill[blue!53.3, opacity=0.7] (2.0980, 0.9060, 3.1939) -- (2.1440, 0.9060, 3.1898) -- (2.1440, 0.9600, 3.1931) -- (2.0980, 0.9600, 3.1972) -- cycle;
\fill[blue!51.6, opacity=0.7] (2.0980, 0.9600, 3.1972) -- (2.1440, 0.9600, 3.1931) -- (2.1440, 1.0140, 3.1961) -- (2.0980, 1.0140, 3.2002) -- cycle;
\fill[blue!53.3, opacity=0.7] (2.0980, 1.0140, 3.2002) -- (2.1440, 1.0140, 3.1961) -- (2.1440, 1.0680, 3.1988) -- (2.0980, 1.0680, 3.2029) -- cycle;
\fill[blue!57.1, opacity=0.7] (2.0980, 1.0680, 3.2029) -- (2.1440, 1.0680, 3.1988) -- (2.1440, 1.1220, 3.2012) -- (2.0980, 1.1220, 3.2053) -- cycle;
\fill[blue!60.8, opacity=0.7] (2.0980, 1.1220, 3.2053) -- (2.1440, 1.1220, 3.2012) -- (2.1440, 1.1760, 3.2033) -- (2.0980, 1.1760, 3.2074) -- cycle;
\fill[blue!63.0, opacity=0.7] (2.0980, 1.1760, 3.2074) -- (2.1440, 1.1760, 3.2033) -- (2.1440, 1.2300, 3.2051) -- (2.0980, 1.2300, 3.2092) -- cycle;
\fill[blue!63.6, opacity=0.7] (2.0980, 1.2300, 3.2092) -- (2.1440, 1.2300, 3.2051) -- (2.1440, 1.2840, 3.2066) -- (2.0980, 1.2840, 3.2106) -- cycle;
\fill[blue!63.2, opacity=0.7] (2.0980, 1.2840, 3.2106) -- (2.1440, 1.2840, 3.2066) -- (2.1440, 1.3380, 3.2077) -- (2.0980, 1.3380, 3.2118) -- cycle;
\fill[blue!62.9, opacity=0.7] (2.0980, 1.3380, 3.2118) -- (2.1440, 1.3380, 3.2077) -- (2.1440, 1.3920, 3.2085) -- (2.0980, 1.3920, 3.2126) -- cycle;
\fill[blue!63.0, opacity=0.7] (2.0980, 1.3920, 3.2126) -- (2.1440, 1.3920, 3.2085) -- (2.1440, 1.4460, 3.2090) -- (2.0980, 1.4460, 3.2131) -- cycle;
\fill[blue!63.5, opacity=0.7] (2.0980, 1.4460, 3.2131) -- (2.1440, 1.4460, 3.2090) -- (2.1440, 1.5000, 3.2092) -- (2.0980, 1.5000, 3.2133) -- cycle;
\fill[blue!63.5, opacity=0.7] (2.0980, 1.5000, 3.2133) -- (2.1440, 1.5000, 3.2092) -- (2.1440, 1.5540, 3.2090) -- (2.0980, 1.5540, 3.2131) -- cycle;
\fill[blue!62.4, opacity=0.7] (2.0980, 1.5540, 3.2131) -- (2.1440, 1.5540, 3.2090) -- (2.1440, 1.6080, 3.2085) -- (2.0980, 1.6080, 3.2126) -- cycle;
\fill[blue!60.3, opacity=0.7] (2.0980, 1.6080, 3.2126) -- (2.1440, 1.6080, 3.2085) -- (2.1440, 1.6620, 3.2077) -- (2.0980, 1.6620, 3.2118) -- cycle;
\fill[blue!58.6, opacity=0.7] (2.0980, 1.6620, 3.2118) -- (2.1440, 1.6620, 3.2077) -- (2.1440, 1.7160, 3.2066) -- (2.0980, 1.7160, 3.2106) -- cycle;
\fill[blue!59.1, opacity=0.7] (2.0980, 1.7160, 3.2106) -- (2.1440, 1.7160, 3.2066) -- (2.1440, 1.7700, 3.2051) -- (2.0980, 1.7700, 3.2092) -- cycle;
\fill[blue!62.0, opacity=0.7] (2.0980, 1.7700, 3.2092) -- (2.1440, 1.7700, 3.2051) -- (2.1440, 1.8240, 3.2033) -- (2.0980, 1.8240, 3.2074) -- cycle;
\fill[blue!63.3, opacity=0.7] (2.0980, 1.8240, 3.2074) -- (2.1440, 1.8240, 3.2033) -- (2.1440, 1.8780, 3.2012) -- (2.0980, 1.8780, 3.2053) -- cycle;
\fill[blue!56.2, opacity=0.7] (2.0980, 1.8780, 3.2053) -- (2.1440, 1.8780, 3.2012) -- (2.1440, 1.9320, 3.1988) -- (2.0980, 1.9320, 3.2029) -- cycle;
\fill[blue!42.6, opacity=0.7] (2.0980, 1.9320, 3.2029) -- (2.1440, 1.9320, 3.1988) -- (2.1440, 1.9860, 3.1961) -- (2.0980, 1.9860, 3.2002) -- cycle;
\fill[blue!32.5, opacity=0.7] (2.0980, 1.9860, 3.2002) -- (2.1440, 1.9860, 3.1961) -- (2.1440, 2.0400, 3.1931) -- (2.0980, 2.0400, 3.1972) -- cycle;
\fill[blue!30.3, opacity=0.7] (2.0980, 2.0400, 3.1972) -- (2.1440, 2.0400, 3.1931) -- (2.1440, 2.0940, 3.1898) -- (2.0980, 2.0940, 3.1939) -- cycle;
\fill[blue!37.1, opacity=0.7] (2.0980, 2.0940, 3.1939) -- (2.1440, 2.0940, 3.1898) -- (2.1440, 2.1480, 3.1863) -- (2.0980, 2.1480, 3.1903) -- cycle;
\fill[blue!54.0, opacity=0.7] (2.0980, 2.1480, 3.1903) -- (2.1440, 2.1480, 3.1863) -- (2.1440, 2.2020, 3.1824) -- (2.0980, 2.2020, 3.1865) -- cycle;
\fill[blue!63.3, opacity=0.7] (2.0980, 2.2020, 3.1865) -- (2.1440, 2.2020, 3.1824) -- (2.1440, 2.2560, 3.1784) -- (2.0980, 2.2560, 3.1824) -- cycle;
\fill[blue!49.8, opacity=0.7] (2.0980, 2.2560, 3.1824) -- (2.1440, 2.2560, 3.1784) -- (2.1440, 2.3100, 3.1740) -- (2.0980, 2.3100, 3.1781) -- cycle;
\fill[blue!37.8, opacity=0.7] (2.0980, 2.3100, 3.1781) -- (2.1440, 2.3100, 3.1740) -- (2.1440, 2.3640, 3.1695) -- (2.0980, 2.3640, 3.1736) -- cycle;
\fill[blue!38.6, opacity=0.7] (2.0980, 2.3640, 3.1736) -- (2.1440, 2.3640, 3.1695) -- (2.1440, 2.4180, 3.1647) -- (2.0980, 2.4180, 3.1688) -- cycle;
\fill[blue!51.5, opacity=0.7] (2.0980, 2.4180, 3.1688) -- (2.1440, 2.4180, 3.1647) -- (2.1440, 2.4720, 3.1597) -- (2.0980, 2.4720, 3.1638) -- cycle;
\fill[blue!63.3, opacity=0.7] (2.0980, 2.4720, 3.1638) -- (2.1440, 2.4720, 3.1597) -- (2.1440, 2.5260, 3.1545) -- (2.0980, 2.5260, 3.1586) -- cycle;
\fill[blue!58.7, opacity=0.7] (2.0980, 2.5260, 3.1586) -- (2.1440, 2.5260, 3.1545) -- (2.1440, 2.5800, 3.1492) -- (2.0980, 2.5800, 3.1533) -- cycle;
\fill[blue!54.5, opacity=0.7] (2.0980, 2.5800, 3.1533) -- (2.1440, 2.5800, 3.1492) -- (2.1440, 2.6340, 3.1437) -- (2.0980, 2.6340, 3.1477) -- cycle;
\fill[blue!61.3, opacity=0.7] (2.0980, 2.6340, 3.1477) -- (2.1440, 2.6340, 3.1437) -- (2.1440, 2.6880, 3.1380) -- (2.0980, 2.6880, 3.1421) -- cycle;
\fill[blue!59.0, opacity=0.7] (2.0980, 2.6880, 3.1421) -- (2.1440, 2.6880, 3.1380) -- (2.1440, 2.7420, 3.1322) -- (2.0980, 2.7420, 3.1363) -- cycle;
\fill[blue!35.0, opacity=0.7] (2.0980, 2.7420, 3.1363) -- (2.1440, 2.7420, 3.1322) -- (2.1440, 2.7960, 3.1263) -- (2.0980, 2.7960, 3.1303) -- cycle;
\fill[blue!21.5, opacity=0.7] (2.0980, 2.7960, 3.1303) -- (2.1440, 2.7960, 3.1263) -- (2.1440, 2.8500, 3.1202) -- (2.0980, 2.8500, 3.1243) -- cycle;
\fill[blue!19.4, opacity=0.7] (2.0980, 2.8500, 3.1243) -- (2.1440, 2.8500, 3.1202) -- (2.1440, 2.9040, 3.1141) -- (2.0980, 2.9040, 3.1182) -- cycle;
\fill[blue!23.9, opacity=0.7] (2.0980, 2.9040, 3.1182) -- (2.1440, 2.9040, 3.1141) -- (2.1440, 2.9580, 3.1079) -- (2.0980, 2.9580, 3.1120) -- cycle;
\fill[blue!39.6, opacity=0.7] (2.0980, 2.9580, 3.1120) -- (2.1440, 2.9580, 3.1079) -- (2.1440, 3.0120, 3.1017) -- (2.0980, 3.0120, 3.1058) -- cycle;
\fill[blue!57.1, opacity=0.7] (2.0980, 3.0120, 3.1058) -- (2.1440, 3.0120, 3.1017) -- (2.1440, 3.0660, 3.0955) -- (2.0980, 3.0660, 3.0995) -- cycle;
\fill[blue!61.5, opacity=0.7] (2.0980, 3.0660, 3.0995) -- (2.1440, 3.0660, 3.0955) -- (2.1440, 3.1200, 3.0892) -- (2.0980, 3.1200, 3.0933) -- cycle;
\fill[blue!21.4, opacity=0.7] (2.1440, -0.1200, 3.0892) -- (2.1900, -0.1200, 3.0849) -- (2.1900, -0.0660, 3.0911) -- (2.1440, -0.0660, 3.0955) -- cycle;
\fill[blue!39.6, opacity=0.7] (2.1440, -0.0660, 3.0955) -- (2.1900, -0.0660, 3.0911) -- (2.1900, -0.0120, 3.0974) -- (2.1440, -0.0120, 3.1017) -- cycle;
\fill[blue!59.4, opacity=0.7] (2.1440, -0.0120, 3.1017) -- (2.1900, -0.0120, 3.0974) -- (2.1900, 0.0420, 3.1036) -- (2.1440, 0.0420, 3.1079) -- cycle;
\fill[blue!63.5, opacity=0.7] (2.1440, 0.0420, 3.1079) -- (2.1900, 0.0420, 3.1036) -- (2.1900, 0.0960, 3.1098) -- (2.1440, 0.0960, 3.1141) -- cycle;
\fill[blue!63.4, opacity=0.7] (2.1440, 0.0960, 3.1141) -- (2.1900, 0.0960, 3.1098) -- (2.1900, 0.1500, 3.1159) -- (2.1440, 0.1500, 3.1202) -- cycle;
\fill[blue!57.7, opacity=0.7] (2.1440, 0.1500, 3.1202) -- (2.1900, 0.1500, 3.1159) -- (2.1900, 0.2040, 3.1219) -- (2.1440, 0.2040, 3.1263) -- cycle;
\fill[blue!41.4, opacity=0.7] (2.1440, 0.2040, 3.1263) -- (2.1900, 0.2040, 3.1219) -- (2.1900, 0.2580, 3.1279) -- (2.1440, 0.2580, 3.1322) -- cycle;
\fill[blue!28.1, opacity=0.7] (2.1440, 0.2580, 3.1322) -- (2.1900, 0.2580, 3.1279) -- (2.1900, 0.3120, 3.1337) -- (2.1440, 0.3120, 3.1380) -- cycle;
\fill[blue!23.8, opacity=0.7] (2.1440, 0.3120, 3.1380) -- (2.1900, 0.3120, 3.1337) -- (2.1900, 0.3660, 3.1393) -- (2.1440, 0.3660, 3.1437) -- cycle;
\fill[blue!26.3, opacity=0.7] (2.1440, 0.3660, 3.1437) -- (2.1900, 0.3660, 3.1393) -- (2.1900, 0.4200, 3.1449) -- (2.1440, 0.4200, 3.1492) -- cycle;
\fill[blue!38.0, opacity=0.7] (2.1440, 0.4200, 3.1492) -- (2.1900, 0.4200, 3.1449) -- (2.1900, 0.4740, 3.1502) -- (2.1440, 0.4740, 3.1545) -- cycle;
\fill[blue!57.0, opacity=0.7] (2.1440, 0.4740, 3.1545) -- (2.1900, 0.4740, 3.1502) -- (2.1900, 0.5280, 3.1554) -- (2.1440, 0.5280, 3.1597) -- cycle;
\fill[blue!63.0, opacity=0.7] (2.1440, 0.5280, 3.1597) -- (2.1900, 0.5280, 3.1554) -- (2.1900, 0.5820, 3.1604) -- (2.1440, 0.5820, 3.1647) -- cycle;
\fill[blue!52.6, opacity=0.7] (2.1440, 0.5820, 3.1647) -- (2.1900, 0.5820, 3.1604) -- (2.1900, 0.6360, 3.1651) -- (2.1440, 0.6360, 3.1695) -- cycle;
\fill[blue!43.8, opacity=0.7] (2.1440, 0.6360, 3.1695) -- (2.1900, 0.6360, 3.1651) -- (2.1900, 0.6900, 3.1697) -- (2.1440, 0.6900, 3.1740) -- cycle;
\fill[blue!42.8, opacity=0.7] (2.1440, 0.6900, 3.1740) -- (2.1900, 0.6900, 3.1697) -- (2.1900, 0.7440, 3.1740) -- (2.1440, 0.7440, 3.1784) -- cycle;
\fill[blue!48.5, opacity=0.7] (2.1440, 0.7440, 3.1784) -- (2.1900, 0.7440, 3.1740) -- (2.1900, 0.7980, 3.1781) -- (2.1440, 0.7980, 3.1824) -- cycle;
\fill[blue!57.3, opacity=0.7] (2.1440, 0.7980, 3.1824) -- (2.1900, 0.7980, 3.1781) -- (2.1900, 0.8520, 3.1819) -- (2.1440, 0.8520, 3.1863) -- cycle;
\fill[blue!63.1, opacity=0.7] (2.1440, 0.8520, 3.1863) -- (2.1900, 0.8520, 3.1819) -- (2.1900, 0.9060, 3.1855) -- (2.1440, 0.9060, 3.1898) -- cycle;
\fill[blue!62.6, opacity=0.7] (2.1440, 0.9060, 3.1898) -- (2.1900, 0.9060, 3.1855) -- (2.1900, 0.9600, 3.1888) -- (2.1440, 0.9600, 3.1931) -- cycle;
\fill[blue!58.5, opacity=0.7] (2.1440, 0.9600, 3.1931) -- (2.1900, 0.9600, 3.1888) -- (2.1900, 1.0140, 3.1918) -- (2.1440, 1.0140, 3.1961) -- cycle;
\fill[blue!54.9, opacity=0.7] (2.1440, 1.0140, 3.1961) -- (2.1900, 1.0140, 3.1918) -- (2.1900, 1.0680, 3.1945) -- (2.1440, 1.0680, 3.1988) -- cycle;
\fill[blue!53.1, opacity=0.7] (2.1440, 1.0680, 3.1988) -- (2.1900, 1.0680, 3.1945) -- (2.1900, 1.1220, 3.1969) -- (2.1440, 1.1220, 3.2012) -- cycle;
\fill[blue!53.3, opacity=0.7] (2.1440, 1.1220, 3.2012) -- (2.1900, 1.1220, 3.1969) -- (2.1900, 1.1760, 3.1990) -- (2.1440, 1.1760, 3.2033) -- cycle;
\fill[blue!54.5, opacity=0.7] (2.1440, 1.1760, 3.2033) -- (2.1900, 1.1760, 3.1990) -- (2.1900, 1.2300, 3.2008) -- (2.1440, 1.2300, 3.2051) -- cycle;
\fill[blue!55.9, opacity=0.7] (2.1440, 1.2300, 3.2051) -- (2.1900, 1.2300, 3.2008) -- (2.1900, 1.2840, 3.2022) -- (2.1440, 1.2840, 3.2066) -- cycle;
\fill[blue!57.1, opacity=0.7] (2.1440, 1.2840, 3.2066) -- (2.1900, 1.2840, 3.2022) -- (2.1900, 1.3380, 3.2034) -- (2.1440, 1.3380, 3.2077) -- cycle;
\fill[blue!57.8, opacity=0.7] (2.1440, 1.3380, 3.2077) -- (2.1900, 1.3380, 3.2034) -- (2.1900, 1.3920, 3.2042) -- (2.1440, 1.3920, 3.2085) -- cycle;
\fill[blue!57.9, opacity=0.7] (2.1440, 1.3920, 3.2085) -- (2.1900, 1.3920, 3.2042) -- (2.1900, 1.4460, 3.2047) -- (2.1440, 1.4460, 3.2090) -- cycle;
\fill[blue!57.6, opacity=0.7] (2.1440, 1.4460, 3.2090) -- (2.1900, 1.4460, 3.2047) -- (2.1900, 1.5000, 3.2049) -- (2.1440, 1.5000, 3.2092) -- cycle;
\fill[blue!57.2, opacity=0.7] (2.1440, 1.5000, 3.2092) -- (2.1900, 1.5000, 3.2049) -- (2.1900, 1.5540, 3.2047) -- (2.1440, 1.5540, 3.2090) -- cycle;
\fill[blue!57.3, opacity=0.7] (2.1440, 1.5540, 3.2090) -- (2.1900, 1.5540, 3.2047) -- (2.1900, 1.6080, 3.2042) -- (2.1440, 1.6080, 3.2085) -- cycle;
\fill[blue!58.5, opacity=0.7] (2.1440, 1.6080, 3.2085) -- (2.1900, 1.6080, 3.2042) -- (2.1900, 1.6620, 3.2034) -- (2.1440, 1.6620, 3.2077) -- cycle;
\fill[blue!61.0, opacity=0.7] (2.1440, 1.6620, 3.2077) -- (2.1900, 1.6620, 3.2034) -- (2.1900, 1.7160, 3.2022) -- (2.1440, 1.7160, 3.2066) -- cycle;
\fill[blue!63.4, opacity=0.7] (2.1440, 1.7160, 3.2066) -- (2.1900, 1.7160, 3.2022) -- (2.1900, 1.7700, 3.2008) -- (2.1440, 1.7700, 3.2051) -- cycle;
\fill[blue!61.7, opacity=0.7] (2.1440, 1.7700, 3.2051) -- (2.1900, 1.7700, 3.2008) -- (2.1900, 1.8240, 3.1990) -- (2.1440, 1.8240, 3.2033) -- cycle;
\fill[blue!52.7, opacity=0.7] (2.1440, 1.8240, 3.2033) -- (2.1900, 1.8240, 3.1990) -- (2.1900, 1.8780, 3.1969) -- (2.1440, 1.8780, 3.2012) -- cycle;
\fill[blue!40.7, opacity=0.7] (2.1440, 1.8780, 3.2012) -- (2.1900, 1.8780, 3.1969) -- (2.1900, 1.9320, 3.1945) -- (2.1440, 1.9320, 3.1988) -- cycle;
\fill[blue!32.6, opacity=0.7] (2.1440, 1.9320, 3.1988) -- (2.1900, 1.9320, 3.1945) -- (2.1900, 1.9860, 3.1918) -- (2.1440, 1.9860, 3.1961) -- cycle;
\fill[blue!31.0, opacity=0.7] (2.1440, 1.9860, 3.1961) -- (2.1900, 1.9860, 3.1918) -- (2.1900, 2.0400, 3.1888) -- (2.1440, 2.0400, 3.1931) -- cycle;
\fill[blue!37.1, opacity=0.7] (2.1440, 2.0400, 3.1931) -- (2.1900, 2.0400, 3.1888) -- (2.1900, 2.0940, 3.1855) -- (2.1440, 2.0940, 3.1898) -- cycle;
\fill[blue!52.3, opacity=0.7] (2.1440, 2.0940, 3.1898) -- (2.1900, 2.0940, 3.1855) -- (2.1900, 2.1480, 3.1819) -- (2.1440, 2.1480, 3.1863) -- cycle;
\fill[blue!63.6, opacity=0.7] (2.1440, 2.1480, 3.1863) -- (2.1900, 2.1480, 3.1819) -- (2.1900, 2.2020, 3.1781) -- (2.1440, 2.2020, 3.1824) -- cycle;
\fill[blue!53.3, opacity=0.7] (2.1440, 2.2020, 3.1824) -- (2.1900, 2.2020, 3.1781) -- (2.1900, 2.2560, 3.1740) -- (2.1440, 2.2560, 3.1784) -- cycle;
\fill[blue!39.3, opacity=0.7] (2.1440, 2.2560, 3.1784) -- (2.1900, 2.2560, 3.1740) -- (2.1900, 2.3100, 3.1697) -- (2.1440, 2.3100, 3.1740) -- cycle;
\fill[blue!36.5, opacity=0.7] (2.1440, 2.3100, 3.1740) -- (2.1900, 2.3100, 3.1697) -- (2.1900, 2.3640, 3.1651) -- (2.1440, 2.3640, 3.1695) -- cycle;
\fill[blue!45.4, opacity=0.7] (2.1440, 2.3640, 3.1695) -- (2.1900, 2.3640, 3.1651) -- (2.1900, 2.4180, 3.1604) -- (2.1440, 2.4180, 3.1647) -- cycle;
\fill[blue!60.2, opacity=0.7] (2.1440, 2.4180, 3.1647) -- (2.1900, 2.4180, 3.1604) -- (2.1900, 2.4720, 3.1554) -- (2.1440, 2.4720, 3.1597) -- cycle;
\fill[blue!62.1, opacity=0.7] (2.1440, 2.4720, 3.1597) -- (2.1900, 2.4720, 3.1554) -- (2.1900, 2.5260, 3.1502) -- (2.1440, 2.5260, 3.1545) -- cycle;
\fill[blue!55.4, opacity=0.7] (2.1440, 2.5260, 3.1545) -- (2.1900, 2.5260, 3.1502) -- (2.1900, 2.5800, 3.1449) -- (2.1440, 2.5800, 3.1492) -- cycle;
\fill[blue!57.5, opacity=0.7] (2.1440, 2.5800, 3.1492) -- (2.1900, 2.5800, 3.1449) -- (2.1900, 2.6340, 3.1393) -- (2.1440, 2.6340, 3.1437) -- cycle;
\fill[blue!63.5, opacity=0.7] (2.1440, 2.6340, 3.1437) -- (2.1900, 2.6340, 3.1393) -- (2.1900, 2.6880, 3.1337) -- (2.1440, 2.6880, 3.1380) -- cycle;
\fill[blue!47.6, opacity=0.7] (2.1440, 2.6880, 3.1380) -- (2.1900, 2.6880, 3.1337) -- (2.1900, 2.7420, 3.1279) -- (2.1440, 2.7420, 3.1322) -- cycle;
\fill[blue!26.4, opacity=0.7] (2.1440, 2.7420, 3.1322) -- (2.1900, 2.7420, 3.1279) -- (2.1900, 2.7960, 3.1219) -- (2.1440, 2.7960, 3.1263) -- cycle;
\fill[blue!19.7, opacity=0.7] (2.1440, 2.7960, 3.1263) -- (2.1900, 2.7960, 3.1219) -- (2.1900, 2.8500, 3.1159) -- (2.1440, 2.8500, 3.1202) -- cycle;
\fill[blue!20.3, opacity=0.7] (2.1440, 2.8500, 3.1202) -- (2.1900, 2.8500, 3.1159) -- (2.1900, 2.9040, 3.1098) -- (2.1440, 2.9040, 3.1141) -- cycle;
\fill[blue!28.7, opacity=0.7] (2.1440, 2.9040, 3.1141) -- (2.1900, 2.9040, 3.1098) -- (2.1900, 2.9580, 3.1036) -- (2.1440, 2.9580, 3.1079) -- cycle;
\fill[blue!47.5, opacity=0.7] (2.1440, 2.9580, 3.1079) -- (2.1900, 2.9580, 3.1036) -- (2.1900, 3.0120, 3.0974) -- (2.1440, 3.0120, 3.1017) -- cycle;
\fill[blue!60.0, opacity=0.7] (2.1440, 3.0120, 3.1017) -- (2.1900, 3.0120, 3.0974) -- (2.1900, 3.0660, 3.0911) -- (2.1440, 3.0660, 3.0955) -- cycle;
\fill[blue!60.6, opacity=0.7] (2.1440, 3.0660, 3.0955) -- (2.1900, 3.0660, 3.0911) -- (2.1900, 3.1200, 3.0849) -- (2.1440, 3.1200, 3.0892) -- cycle;
\fill[blue!17.5, opacity=0.7] (2.1900, -0.1200, 3.0849) -- (2.2360, -0.1200, 3.0803) -- (2.2360, -0.0660, 3.0866) -- (2.1900, -0.0660, 3.0911) -- cycle;
\fill[blue!26.6, opacity=0.7] (2.1900, -0.0660, 3.0911) -- (2.2360, -0.0660, 3.0866) -- (2.2360, -0.0120, 3.0928) -- (2.1900, -0.0120, 3.0974) -- cycle;
\fill[blue!48.2, opacity=0.7] (2.1900, -0.0120, 3.0974) -- (2.2360, -0.0120, 3.0928) -- (2.2360, 0.0420, 3.0991) -- (2.1900, 0.0420, 3.1036) -- cycle;
\fill[blue!62.2, opacity=0.7] (2.1900, 0.0420, 3.1036) -- (2.2360, 0.0420, 3.0991) -- (2.2360, 0.0960, 3.1052) -- (2.1900, 0.0960, 3.1098) -- cycle;
\fill[blue!63.6, opacity=0.7] (2.1900, 0.0960, 3.1098) -- (2.2360, 0.0960, 3.1052) -- (2.2360, 0.1500, 3.1114) -- (2.1900, 0.1500, 3.1159) -- cycle;
\fill[blue!63.0, opacity=0.7] (2.1900, 0.1500, 3.1159) -- (2.2360, 0.1500, 3.1114) -- (2.2360, 0.2040, 3.1174) -- (2.1900, 0.2040, 3.1219) -- cycle;
\fill[blue!54.9, opacity=0.7] (2.1900, 0.2040, 3.1219) -- (2.2360, 0.2040, 3.1174) -- (2.2360, 0.2580, 3.1233) -- (2.1900, 0.2580, 3.1279) -- cycle;
\fill[blue!38.5, opacity=0.7] (2.1900, 0.2580, 3.1279) -- (2.2360, 0.2580, 3.1233) -- (2.2360, 0.3120, 3.1291) -- (2.1900, 0.3120, 3.1337) -- cycle;
\fill[blue!27.2, opacity=0.7] (2.1900, 0.3120, 3.1337) -- (2.2360, 0.3120, 3.1291) -- (2.2360, 0.3660, 3.1348) -- (2.1900, 0.3660, 3.1393) -- cycle;
\fill[blue!24.0, opacity=0.7] (2.1900, 0.3660, 3.1393) -- (2.2360, 0.3660, 3.1348) -- (2.2360, 0.4200, 3.1403) -- (2.1900, 0.4200, 3.1449) -- cycle;
\fill[blue!27.1, opacity=0.7] (2.1900, 0.4200, 3.1449) -- (2.2360, 0.4200, 3.1403) -- (2.2360, 0.4740, 3.1457) -- (2.1900, 0.4740, 3.1502) -- cycle;
\fill[blue!38.7, opacity=0.7] (2.1900, 0.4740, 3.1502) -- (2.2360, 0.4740, 3.1457) -- (2.2360, 0.5280, 3.1508) -- (2.1900, 0.5280, 3.1554) -- cycle;
\fill[blue!56.6, opacity=0.7] (2.1900, 0.5280, 3.1554) -- (2.2360, 0.5280, 3.1508) -- (2.2360, 0.5820, 3.1558) -- (2.1900, 0.5820, 3.1604) -- cycle;
\fill[blue!63.3, opacity=0.7] (2.1900, 0.5820, 3.1604) -- (2.2360, 0.5820, 3.1558) -- (2.2360, 0.6360, 3.1606) -- (2.1900, 0.6360, 3.1651) -- cycle;
\fill[blue!54.3, opacity=0.7] (2.1900, 0.6360, 3.1651) -- (2.2360, 0.6360, 3.1606) -- (2.2360, 0.6900, 3.1651) -- (2.1900, 0.6900, 3.1697) -- cycle;
\fill[blue!44.8, opacity=0.7] (2.1900, 0.6900, 3.1697) -- (2.2360, 0.6900, 3.1651) -- (2.2360, 0.7440, 3.1695) -- (2.1900, 0.7440, 3.1740) -- cycle;
\fill[blue!41.5, opacity=0.7] (2.1900, 0.7440, 3.1740) -- (2.2360, 0.7440, 3.1695) -- (2.2360, 0.7980, 3.1736) -- (2.1900, 0.7980, 3.1781) -- cycle;
\fill[blue!44.2, opacity=0.7] (2.1900, 0.7980, 3.1781) -- (2.2360, 0.7980, 3.1736) -- (2.2360, 0.8520, 3.1774) -- (2.1900, 0.8520, 3.1819) -- cycle;
\fill[blue!50.9, opacity=0.7] (2.1900, 0.8520, 3.1819) -- (2.2360, 0.8520, 3.1774) -- (2.2360, 0.9060, 3.1809) -- (2.1900, 0.9060, 3.1855) -- cycle;
\fill[blue!58.3, opacity=0.7] (2.1900, 0.9060, 3.1855) -- (2.2360, 0.9060, 3.1809) -- (2.2360, 0.9600, 3.1842) -- (2.1900, 0.9600, 3.1888) -- cycle;
\fill[blue!62.8, opacity=0.7] (2.1900, 0.9600, 3.1888) -- (2.2360, 0.9600, 3.1842) -- (2.2360, 1.0140, 3.1872) -- (2.1900, 1.0140, 3.1918) -- cycle;
\fill[blue!63.4, opacity=0.7] (2.1900, 1.0140, 3.1918) -- (2.2360, 1.0140, 3.1872) -- (2.2360, 1.0680, 3.1899) -- (2.1900, 1.0680, 3.1945) -- cycle;
\fill[blue!61.5, opacity=0.7] (2.1900, 1.0680, 3.1945) -- (2.2360, 1.0680, 3.1899) -- (2.2360, 1.1220, 3.1923) -- (2.1900, 1.1220, 3.1969) -- cycle;
\fill[blue!59.3, opacity=0.7] (2.1900, 1.1220, 3.1969) -- (2.2360, 1.1220, 3.1923) -- (2.2360, 1.1760, 3.1944) -- (2.1900, 1.1760, 3.1990) -- cycle;
\fill[blue!57.6, opacity=0.7] (2.1900, 1.1760, 3.1990) -- (2.2360, 1.1760, 3.1944) -- (2.2360, 1.2300, 3.1962) -- (2.1900, 1.2300, 3.2008) -- cycle;
\fill[blue!56.7, opacity=0.7] (2.1900, 1.2300, 3.2008) -- (2.2360, 1.2300, 3.1962) -- (2.2360, 1.2840, 3.1977) -- (2.1900, 1.2840, 3.2022) -- cycle;
\fill[blue!56.5, opacity=0.7] (2.1900, 1.2840, 3.2022) -- (2.2360, 1.2840, 3.1977) -- (2.2360, 1.3380, 3.1988) -- (2.1900, 1.3380, 3.2034) -- cycle;
\fill[blue!56.8, opacity=0.7] (2.1900, 1.3380, 3.2034) -- (2.2360, 1.3380, 3.1988) -- (2.2360, 1.3920, 3.1996) -- (2.1900, 1.3920, 3.2042) -- cycle;
\fill[blue!57.5, opacity=0.7] (2.1900, 1.3920, 3.2042) -- (2.2360, 1.3920, 3.1996) -- (2.2360, 1.4460, 3.2001) -- (2.1900, 1.4460, 3.2047) -- cycle;
\fill[blue!58.7, opacity=0.7] (2.1900, 1.4460, 3.2047) -- (2.2360, 1.4460, 3.2001) -- (2.2360, 1.5000, 3.2003) -- (2.1900, 1.5000, 3.2049) -- cycle;
\fill[blue!60.3, opacity=0.7] (2.1900, 1.5000, 3.2049) -- (2.2360, 1.5000, 3.2003) -- (2.2360, 1.5540, 3.2001) -- (2.1900, 1.5540, 3.2047) -- cycle;
\fill[blue!62.3, opacity=0.7] (2.1900, 1.5540, 3.2047) -- (2.2360, 1.5540, 3.2001) -- (2.2360, 1.6080, 3.1996) -- (2.1900, 1.6080, 3.2042) -- cycle;
\fill[blue!63.6, opacity=0.7] (2.1900, 1.6080, 3.2042) -- (2.2360, 1.6080, 3.1996) -- (2.2360, 1.6620, 3.1988) -- (2.1900, 1.6620, 3.2034) -- cycle;
\fill[blue!62.1, opacity=0.7] (2.1900, 1.6620, 3.2034) -- (2.2360, 1.6620, 3.1988) -- (2.2360, 1.7160, 3.1977) -- (2.1900, 1.7160, 3.2022) -- cycle;
\fill[blue!56.1, opacity=0.7] (2.1900, 1.7160, 3.2022) -- (2.2360, 1.7160, 3.1977) -- (2.2360, 1.7700, 3.1962) -- (2.1900, 1.7700, 3.2008) -- cycle;
\fill[blue!46.5, opacity=0.7] (2.1900, 1.7700, 3.2008) -- (2.2360, 1.7700, 3.1962) -- (2.2360, 1.8240, 3.1944) -- (2.1900, 1.8240, 3.1990) -- cycle;
\fill[blue!37.3, opacity=0.7] (2.1900, 1.8240, 3.1990) -- (2.2360, 1.8240, 3.1944) -- (2.2360, 1.8780, 3.1923) -- (2.1900, 1.8780, 3.1969) -- cycle;
\fill[blue!32.1, opacity=0.7] (2.1900, 1.8780, 3.1969) -- (2.2360, 1.8780, 3.1923) -- (2.2360, 1.9320, 3.1899) -- (2.1900, 1.9320, 3.1945) -- cycle;
\fill[blue!32.1, opacity=0.7] (2.1900, 1.9320, 3.1945) -- (2.2360, 1.9320, 3.1899) -- (2.2360, 1.9860, 3.1872) -- (2.1900, 1.9860, 3.1918) -- cycle;
\fill[blue!39.0, opacity=0.7] (2.1900, 1.9860, 3.1918) -- (2.2360, 1.9860, 3.1872) -- (2.2360, 2.0400, 3.1842) -- (2.1900, 2.0400, 3.1888) -- cycle;
\fill[blue!53.3, opacity=0.7] (2.1900, 2.0400, 3.1888) -- (2.2360, 2.0400, 3.1842) -- (2.2360, 2.0940, 3.1809) -- (2.1900, 2.0940, 3.1855) -- cycle;
\fill[blue!63.5, opacity=0.7] (2.1900, 2.0940, 3.1855) -- (2.2360, 2.0940, 3.1809) -- (2.2360, 2.1480, 3.1774) -- (2.1900, 2.1480, 3.1819) -- cycle;
\fill[blue!54.6, opacity=0.7] (2.1900, 2.1480, 3.1819) -- (2.2360, 2.1480, 3.1774) -- (2.2360, 2.2020, 3.1736) -- (2.1900, 2.2020, 3.1781) -- cycle;
\fill[blue!40.5, opacity=0.7] (2.1900, 2.2020, 3.1781) -- (2.2360, 2.2020, 3.1736) -- (2.2360, 2.2560, 3.1695) -- (2.1900, 2.2560, 3.1740) -- cycle;
\fill[blue!35.6, opacity=0.7] (2.1900, 2.2560, 3.1740) -- (2.2360, 2.2560, 3.1695) -- (2.2360, 2.3100, 3.1651) -- (2.1900, 2.3100, 3.1697) -- cycle;
\fill[blue!41.4, opacity=0.7] (2.1900, 2.3100, 3.1697) -- (2.2360, 2.3100, 3.1651) -- (2.2360, 2.3640, 3.1606) -- (2.1900, 2.3640, 3.1651) -- cycle;
\fill[blue!55.9, opacity=0.7] (2.1900, 2.3640, 3.1651) -- (2.2360, 2.3640, 3.1606) -- (2.2360, 2.4180, 3.1558) -- (2.1900, 2.4180, 3.1604) -- cycle;
\fill[blue!63.5, opacity=0.7] (2.1900, 2.4180, 3.1604) -- (2.2360, 2.4180, 3.1558) -- (2.2360, 2.4720, 3.1508) -- (2.1900, 2.4720, 3.1554) -- cycle;
\fill[blue!57.8, opacity=0.7] (2.1900, 2.4720, 3.1554) -- (2.2360, 2.4720, 3.1508) -- (2.2360, 2.5260, 3.1457) -- (2.1900, 2.5260, 3.1502) -- cycle;
\fill[blue!55.6, opacity=0.7] (2.1900, 2.5260, 3.1502) -- (2.2360, 2.5260, 3.1457) -- (2.2360, 2.5800, 3.1403) -- (2.1900, 2.5800, 3.1449) -- cycle;
\fill[blue!62.3, opacity=0.7] (2.1900, 2.5800, 3.1449) -- (2.2360, 2.5800, 3.1403) -- (2.2360, 2.6340, 3.1348) -- (2.1900, 2.6340, 3.1393) -- cycle;
\fill[blue!57.7, opacity=0.7] (2.1900, 2.6340, 3.1393) -- (2.2360, 2.6340, 3.1348) -- (2.2360, 2.6880, 3.1291) -- (2.1900, 2.6880, 3.1337) -- cycle;
\fill[blue!34.3, opacity=0.7] (2.1900, 2.6880, 3.1337) -- (2.2360, 2.6880, 3.1291) -- (2.2360, 2.7420, 3.1233) -- (2.1900, 2.7420, 3.1279) -- cycle;
\fill[blue!21.4, opacity=0.7] (2.1900, 2.7420, 3.1279) -- (2.2360, 2.7420, 3.1233) -- (2.2360, 2.7960, 3.1174) -- (2.1900, 2.7960, 3.1219) -- cycle;
\fill[blue!19.2, opacity=0.7] (2.1900, 2.7960, 3.1219) -- (2.2360, 2.7960, 3.1174) -- (2.2360, 2.8500, 3.1114) -- (2.1900, 2.8500, 3.1159) -- cycle;
\fill[blue!22.7, opacity=0.7] (2.1900, 2.8500, 3.1159) -- (2.2360, 2.8500, 3.1114) -- (2.2360, 2.9040, 3.1052) -- (2.1900, 2.9040, 3.1098) -- cycle;
\fill[blue!36.1, opacity=0.7] (2.1900, 2.9040, 3.1098) -- (2.2360, 2.9040, 3.1052) -- (2.2360, 2.9580, 3.0991) -- (2.1900, 2.9580, 3.1036) -- cycle;
\fill[blue!54.5, opacity=0.7] (2.1900, 2.9580, 3.1036) -- (2.2360, 2.9580, 3.0991) -- (2.2360, 3.0120, 3.0928) -- (2.1900, 3.0120, 3.0974) -- cycle;
\fill[blue!61.2, opacity=0.7] (2.1900, 3.0120, 3.0974) -- (2.2360, 3.0120, 3.0928) -- (2.2360, 3.0660, 3.0866) -- (2.1900, 3.0660, 3.0911) -- cycle;
\fill[blue!57.5, opacity=0.7] (2.1900, 3.0660, 3.0911) -- (2.2360, 3.0660, 3.0866) -- (2.2360, 3.1200, 3.0803) -- (2.1900, 3.1200, 3.0849) -- cycle;
\fill[blue!16.1, opacity=0.7] (2.2360, -0.1200, 3.0803) -- (2.2820, -0.1200, 3.0755) -- (2.2820, -0.0660, 3.0818) -- (2.2360, -0.0660, 3.0866) -- cycle;
\fill[blue!19.3, opacity=0.7] (2.2360, -0.0660, 3.0866) -- (2.2820, -0.0660, 3.0818) -- (2.2820, -0.0120, 3.0881) -- (2.2360, -0.0120, 3.0928) -- cycle;
\fill[blue!32.6, opacity=0.7] (2.2360, -0.0120, 3.0928) -- (2.2820, -0.0120, 3.0881) -- (2.2820, 0.0420, 3.0943) -- (2.2360, 0.0420, 3.0991) -- cycle;
\fill[blue!54.3, opacity=0.7] (2.2360, 0.0420, 3.0991) -- (2.2820, 0.0420, 3.0943) -- (2.2820, 0.0960, 3.1005) -- (2.2360, 0.0960, 3.1052) -- cycle;
\fill[blue!63.2, opacity=0.7] (2.2360, 0.0960, 3.1052) -- (2.2820, 0.0960, 3.1005) -- (2.2820, 0.1500, 3.1066) -- (2.2360, 0.1500, 3.1114) -- cycle;
\fill[blue!63.6, opacity=0.7] (2.2360, 0.1500, 3.1114) -- (2.2820, 0.1500, 3.1066) -- (2.2820, 0.2040, 3.1126) -- (2.2360, 0.2040, 3.1174) -- cycle;
\fill[blue!62.5, opacity=0.7] (2.2360, 0.2040, 3.1174) -- (2.2820, 0.2040, 3.1126) -- (2.2820, 0.2580, 3.1185) -- (2.2360, 0.2580, 3.1233) -- cycle;
\fill[blue!53.1, opacity=0.7] (2.2360, 0.2580, 3.1233) -- (2.2820, 0.2580, 3.1185) -- (2.2820, 0.3120, 3.1243) -- (2.2360, 0.3120, 3.1291) -- cycle;
\fill[blue!37.3, opacity=0.7] (2.2360, 0.3120, 3.1291) -- (2.2820, 0.3120, 3.1243) -- (2.2820, 0.3660, 3.1300) -- (2.2360, 0.3660, 3.1348) -- cycle;
\fill[blue!27.1, opacity=0.7] (2.2360, 0.3660, 3.1348) -- (2.2820, 0.3660, 3.1300) -- (2.2820, 0.4200, 3.1355) -- (2.2360, 0.4200, 3.1403) -- cycle;
\fill[blue!24.3, opacity=0.7] (2.2360, 0.4200, 3.1403) -- (2.2820, 0.4200, 3.1355) -- (2.2820, 0.4740, 3.1409) -- (2.2360, 0.4740, 3.1457) -- cycle;
\fill[blue!27.2, opacity=0.7] (2.2360, 0.4740, 3.1457) -- (2.2820, 0.4740, 3.1409) -- (2.2820, 0.5280, 3.1461) -- (2.2360, 0.5280, 3.1508) -- cycle;
\fill[blue!37.6, opacity=0.7] (2.2360, 0.5280, 3.1508) -- (2.2820, 0.5280, 3.1461) -- (2.2820, 0.5820, 3.1510) -- (2.2360, 0.5820, 3.1558) -- cycle;
\fill[blue!54.3, opacity=0.7] (2.2360, 0.5820, 3.1558) -- (2.2820, 0.5820, 3.1510) -- (2.2820, 0.6360, 3.1558) -- (2.2360, 0.6360, 3.1606) -- cycle;
\fill[blue!63.5, opacity=0.7] (2.2360, 0.6360, 3.1606) -- (2.2820, 0.6360, 3.1558) -- (2.2820, 0.6900, 3.1604) -- (2.2360, 0.6900, 3.1651) -- cycle;
\fill[blue!57.8, opacity=0.7] (2.2360, 0.6900, 3.1651) -- (2.2820, 0.6900, 3.1604) -- (2.2820, 0.7440, 3.1647) -- (2.2360, 0.7440, 3.1695) -- cycle;
\fill[blue!47.7, opacity=0.7] (2.2360, 0.7440, 3.1695) -- (2.2820, 0.7440, 3.1647) -- (2.2820, 0.7980, 3.1688) -- (2.2360, 0.7980, 3.1736) -- cycle;
\fill[blue!41.8, opacity=0.7] (2.2360, 0.7980, 3.1736) -- (2.2820, 0.7980, 3.1688) -- (2.2820, 0.8520, 3.1726) -- (2.2360, 0.8520, 3.1774) -- cycle;
\fill[blue!40.9, opacity=0.7] (2.2360, 0.8520, 3.1774) -- (2.2820, 0.8520, 3.1726) -- (2.2820, 0.9060, 3.1762) -- (2.2360, 0.9060, 3.1809) -- cycle;
\fill[blue!43.8, opacity=0.7] (2.2360, 0.9060, 3.1809) -- (2.2820, 0.9060, 3.1762) -- (2.2820, 0.9600, 3.1794) -- (2.2360, 0.9600, 3.1842) -- cycle;
\fill[blue!49.0, opacity=0.7] (2.2360, 0.9600, 3.1842) -- (2.2820, 0.9600, 3.1794) -- (2.2820, 1.0140, 3.1824) -- (2.2360, 1.0140, 3.1872) -- cycle;
\fill[blue!54.8, opacity=0.7] (2.2360, 1.0140, 3.1872) -- (2.2820, 1.0140, 3.1824) -- (2.2820, 1.0680, 3.1851) -- (2.2360, 1.0680, 3.1899) -- cycle;
\fill[blue!59.4, opacity=0.7] (2.2360, 1.0680, 3.1899) -- (2.2820, 1.0680, 3.1851) -- (2.2820, 1.1220, 3.1875) -- (2.2360, 1.1220, 3.1923) -- cycle;
\fill[blue!62.1, opacity=0.7] (2.2360, 1.1220, 3.1923) -- (2.2820, 1.1220, 3.1875) -- (2.2820, 1.1760, 3.1896) -- (2.2360, 1.1760, 3.1944) -- cycle;
\fill[blue!63.3, opacity=0.7] (2.2360, 1.1760, 3.1944) -- (2.2820, 1.1760, 3.1896) -- (2.2820, 1.2300, 3.1914) -- (2.2360, 1.2300, 3.1962) -- cycle;
\fill[blue!63.6, opacity=0.7] (2.2360, 1.2300, 3.1962) -- (2.2820, 1.2300, 3.1914) -- (2.2820, 1.2840, 3.1929) -- (2.2360, 1.2840, 3.1977) -- cycle;
\fill[blue!63.6, opacity=0.7] (2.2360, 1.2840, 3.1977) -- (2.2820, 1.2840, 3.1929) -- (2.2820, 1.3380, 3.1940) -- (2.2360, 1.3380, 3.1988) -- cycle;
\fill[blue!63.6, opacity=0.7] (2.2360, 1.3380, 3.1988) -- (2.2820, 1.3380, 3.1940) -- (2.2820, 1.3920, 3.1949) -- (2.2360, 1.3920, 3.1996) -- cycle;
\fill[blue!63.5, opacity=0.7] (2.2360, 1.3920, 3.1996) -- (2.2820, 1.3920, 3.1949) -- (2.2820, 1.4460, 3.1954) -- (2.2360, 1.4460, 3.2001) -- cycle;
\fill[blue!63.2, opacity=0.7] (2.2360, 1.4460, 3.2001) -- (2.2820, 1.4460, 3.1954) -- (2.2820, 1.5000, 3.1955) -- (2.2360, 1.5000, 3.2003) -- cycle;
\fill[blue!61.9, opacity=0.7] (2.2360, 1.5000, 3.2003) -- (2.2820, 1.5000, 3.1955) -- (2.2820, 1.5540, 3.1954) -- (2.2360, 1.5540, 3.2001) -- cycle;
\fill[blue!58.8, opacity=0.7] (2.2360, 1.5540, 3.2001) -- (2.2820, 1.5540, 3.1954) -- (2.2820, 1.6080, 3.1949) -- (2.2360, 1.6080, 3.1996) -- cycle;
\fill[blue!53.5, opacity=0.7] (2.2360, 1.6080, 3.1996) -- (2.2820, 1.6080, 3.1949) -- (2.2820, 1.6620, 3.1940) -- (2.2360, 1.6620, 3.1988) -- cycle;
\fill[blue!46.3, opacity=0.7] (2.2360, 1.6620, 3.1988) -- (2.2820, 1.6620, 3.1940) -- (2.2820, 1.7160, 3.1929) -- (2.2360, 1.7160, 3.1977) -- cycle;
\fill[blue!39.2, opacity=0.7] (2.2360, 1.7160, 3.1977) -- (2.2820, 1.7160, 3.1929) -- (2.2820, 1.7700, 3.1914) -- (2.2360, 1.7700, 3.1962) -- cycle;
\fill[blue!34.0, opacity=0.7] (2.2360, 1.7700, 3.1962) -- (2.2820, 1.7700, 3.1914) -- (2.2820, 1.8240, 3.1896) -- (2.2360, 1.8240, 3.1944) -- cycle;
\fill[blue!32.2, opacity=0.7] (2.2360, 1.8240, 3.1944) -- (2.2820, 1.8240, 3.1896) -- (2.2820, 1.8780, 3.1875) -- (2.2360, 1.8780, 3.1923) -- cycle;
\fill[blue!34.7, opacity=0.7] (2.2360, 1.8780, 3.1923) -- (2.2820, 1.8780, 3.1875) -- (2.2820, 1.9320, 3.1851) -- (2.2360, 1.9320, 3.1899) -- cycle;
\fill[blue!43.2, opacity=0.7] (2.2360, 1.9320, 3.1899) -- (2.2820, 1.9320, 3.1851) -- (2.2820, 1.9860, 3.1824) -- (2.2360, 1.9860, 3.1872) -- cycle;
\fill[blue!56.7, opacity=0.7] (2.2360, 1.9860, 3.1872) -- (2.2820, 1.9860, 3.1824) -- (2.2820, 2.0400, 3.1794) -- (2.2360, 2.0400, 3.1842) -- cycle;
\fill[blue!63.5, opacity=0.7] (2.2360, 2.0400, 3.1842) -- (2.2820, 2.0400, 3.1794) -- (2.2820, 2.0940, 3.1762) -- (2.2360, 2.0940, 3.1809) -- cycle;
\fill[blue!53.7, opacity=0.7] (2.2360, 2.0940, 3.1809) -- (2.2820, 2.0940, 3.1762) -- (2.2820, 2.1480, 3.1726) -- (2.2360, 2.1480, 3.1774) -- cycle;
\fill[blue!40.4, opacity=0.7] (2.2360, 2.1480, 3.1774) -- (2.2820, 2.1480, 3.1726) -- (2.2820, 2.2020, 3.1688) -- (2.2360, 2.2020, 3.1736) -- cycle;
\fill[blue!35.1, opacity=0.7] (2.2360, 2.2020, 3.1736) -- (2.2820, 2.2020, 3.1688) -- (2.2820, 2.2560, 3.1647) -- (2.2360, 2.2560, 3.1695) -- cycle;
\fill[blue!39.3, opacity=0.7] (2.2360, 2.2560, 3.1695) -- (2.2820, 2.2560, 3.1647) -- (2.2820, 2.3100, 3.1604) -- (2.2360, 2.3100, 3.1651) -- cycle;
\fill[blue!52.2, opacity=0.7] (2.2360, 2.3100, 3.1651) -- (2.2820, 2.3100, 3.1604) -- (2.2820, 2.3640, 3.1558) -- (2.2360, 2.3640, 3.1606) -- cycle;
\fill[blue!63.1, opacity=0.7] (2.2360, 2.3640, 3.1606) -- (2.2820, 2.3640, 3.1558) -- (2.2820, 2.4180, 3.1510) -- (2.2360, 2.4180, 3.1558) -- cycle;
\fill[blue!60.1, opacity=0.7] (2.2360, 2.4180, 3.1558) -- (2.2820, 2.4180, 3.1510) -- (2.2820, 2.4720, 3.1461) -- (2.2360, 2.4720, 3.1508) -- cycle;
\fill[blue!55.5, opacity=0.7] (2.2360, 2.4720, 3.1508) -- (2.2820, 2.4720, 3.1461) -- (2.2820, 2.5260, 3.1409) -- (2.2360, 2.5260, 3.1457) -- cycle;
\fill[blue!59.9, opacity=0.7] (2.2360, 2.5260, 3.1457) -- (2.2820, 2.5260, 3.1409) -- (2.2820, 2.5800, 3.1355) -- (2.2360, 2.5800, 3.1403) -- cycle;
\fill[blue!62.5, opacity=0.7] (2.2360, 2.5800, 3.1403) -- (2.2820, 2.5800, 3.1355) -- (2.2820, 2.6340, 3.1300) -- (2.2360, 2.6340, 3.1348) -- cycle;
\fill[blue!43.5, opacity=0.7] (2.2360, 2.6340, 3.1348) -- (2.2820, 2.6340, 3.1300) -- (2.2820, 2.6880, 3.1243) -- (2.2360, 2.6880, 3.1291) -- cycle;
\fill[blue!24.9, opacity=0.7] (2.2360, 2.6880, 3.1291) -- (2.2820, 2.6880, 3.1243) -- (2.2820, 2.7420, 3.1185) -- (2.2360, 2.7420, 3.1233) -- cycle;
\fill[blue!19.3, opacity=0.7] (2.2360, 2.7420, 3.1233) -- (2.2820, 2.7420, 3.1185) -- (2.2820, 2.7960, 3.1126) -- (2.2360, 2.7960, 3.1174) -- cycle;
\fill[blue!20.0, opacity=0.7] (2.2360, 2.7960, 3.1174) -- (2.2820, 2.7960, 3.1126) -- (2.2820, 2.8500, 3.1066) -- (2.2360, 2.8500, 3.1114) -- cycle;
\fill[blue!27.7, opacity=0.7] (2.2360, 2.8500, 3.1114) -- (2.2820, 2.8500, 3.1066) -- (2.2820, 2.9040, 3.1005) -- (2.2360, 2.9040, 3.1052) -- cycle;
\fill[blue!45.5, opacity=0.7] (2.2360, 2.9040, 3.1052) -- (2.2820, 2.9040, 3.1005) -- (2.2820, 2.9580, 3.0943) -- (2.2360, 2.9580, 3.0991) -- cycle;
\fill[blue!59.0, opacity=0.7] (2.2360, 2.9580, 3.0991) -- (2.2820, 2.9580, 3.0943) -- (2.2820, 3.0120, 3.0881) -- (2.2360, 3.0120, 3.0928) -- cycle;
\fill[blue!60.7, opacity=0.7] (2.2360, 3.0120, 3.0928) -- (2.2820, 3.0120, 3.0881) -- (2.2820, 3.0660, 3.0818) -- (2.2360, 3.0660, 3.0866) -- cycle;
\fill[blue!50.8, opacity=0.7] (2.2360, 3.0660, 3.0866) -- (2.2820, 3.0660, 3.0818) -- (2.2820, 3.1200, 3.0755) -- (2.2360, 3.1200, 3.0803) -- cycle;
\fill[blue!15.8, opacity=0.7] (2.2820, -0.1200, 3.0755) -- (2.3280, -0.1200, 3.0705) -- (2.3280, -0.0660, 3.0768) -- (2.2820, -0.0660, 3.0818) -- cycle;
\fill[blue!16.6, opacity=0.7] (2.2820, -0.0660, 3.0818) -- (2.3280, -0.0660, 3.0768) -- (2.3280, -0.0120, 3.0831) -- (2.2820, -0.0120, 3.0881) -- cycle;
\fill[blue!21.6, opacity=0.7] (2.2820, -0.0120, 3.0881) -- (2.3280, -0.0120, 3.0831) -- (2.3280, 0.0420, 3.0893) -- (2.2820, 0.0420, 3.0943) -- cycle;
\fill[blue!38.3, opacity=0.7] (2.2820, 0.0420, 3.0943) -- (2.3280, 0.0420, 3.0893) -- (2.3280, 0.0960, 3.0955) -- (2.2820, 0.0960, 3.1005) -- cycle;
\fill[blue!57.9, opacity=0.7] (2.2820, 0.0960, 3.1005) -- (2.3280, 0.0960, 3.0955) -- (2.3280, 0.1500, 3.1016) -- (2.2820, 0.1500, 3.1066) -- cycle;
\fill[blue!63.5, opacity=0.7] (2.2820, 0.1500, 3.1066) -- (2.3280, 0.1500, 3.1016) -- (2.3280, 0.2040, 3.1076) -- (2.2820, 0.2040, 3.1126) -- cycle;
\fill[blue!63.6, opacity=0.7] (2.2820, 0.2040, 3.1126) -- (2.3280, 0.2040, 3.1076) -- (2.3280, 0.2580, 3.1135) -- (2.2820, 0.2580, 3.1185) -- cycle;
\fill[blue!62.3, opacity=0.7] (2.2820, 0.2580, 3.1185) -- (2.3280, 0.2580, 3.1135) -- (2.3280, 0.3120, 3.1193) -- (2.2820, 0.3120, 3.1243) -- cycle;
\fill[blue!52.6, opacity=0.7] (2.2820, 0.3120, 3.1243) -- (2.3280, 0.3120, 3.1193) -- (2.3280, 0.3660, 3.1250) -- (2.2820, 0.3660, 3.1300) -- cycle;
\fill[blue!37.6, opacity=0.7] (2.2820, 0.3660, 3.1300) -- (2.3280, 0.3660, 3.1250) -- (2.3280, 0.4200, 3.1305) -- (2.2820, 0.4200, 3.1355) -- cycle;
\fill[blue!27.7, opacity=0.7] (2.2820, 0.4200, 3.1355) -- (2.3280, 0.4200, 3.1305) -- (2.3280, 0.4740, 3.1359) -- (2.2820, 0.4740, 3.1409) -- cycle;
\fill[blue!24.6, opacity=0.7] (2.2820, 0.4740, 3.1409) -- (2.3280, 0.4740, 3.1359) -- (2.3280, 0.5280, 3.1411) -- (2.2820, 0.5280, 3.1461) -- cycle;
\fill[blue!26.7, opacity=0.7] (2.2820, 0.5280, 3.1461) -- (2.3280, 0.5280, 3.1411) -- (2.3280, 0.5820, 3.1461) -- (2.2820, 0.5820, 3.1510) -- cycle;
\fill[blue!34.8, opacity=0.7] (2.2820, 0.5820, 3.1510) -- (2.3280, 0.5820, 3.1461) -- (2.3280, 0.6360, 3.1508) -- (2.2820, 0.6360, 3.1558) -- cycle;
\fill[blue!49.4, opacity=0.7] (2.2820, 0.6360, 3.1558) -- (2.3280, 0.6360, 3.1508) -- (2.3280, 0.6900, 3.1554) -- (2.2820, 0.6900, 3.1604) -- cycle;
\fill[blue!61.9, opacity=0.7] (2.2820, 0.6900, 3.1604) -- (2.3280, 0.6900, 3.1554) -- (2.3280, 0.7440, 3.1597) -- (2.2820, 0.7440, 3.1647) -- cycle;
\fill[blue!62.0, opacity=0.7] (2.2820, 0.7440, 3.1647) -- (2.3280, 0.7440, 3.1597) -- (2.3280, 0.7980, 3.1638) -- (2.2820, 0.7980, 3.1688) -- cycle;
\fill[blue!53.5, opacity=0.7] (2.2820, 0.7980, 3.1688) -- (2.3280, 0.7980, 3.1638) -- (2.3280, 0.8520, 3.1676) -- (2.2820, 0.8520, 3.1726) -- cycle;
\fill[blue!45.3, opacity=0.7] (2.2820, 0.8520, 3.1726) -- (2.3280, 0.8520, 3.1676) -- (2.3280, 0.9060, 3.1712) -- (2.2820, 0.9060, 3.1762) -- cycle;
\fill[blue!40.8, opacity=0.7] (2.2820, 0.9060, 3.1762) -- (2.3280, 0.9060, 3.1712) -- (2.3280, 0.9600, 3.1745) -- (2.2820, 0.9600, 3.1794) -- cycle;
\fill[blue!39.6, opacity=0.7] (2.2820, 0.9600, 3.1794) -- (2.3280, 0.9600, 3.1745) -- (2.3280, 1.0140, 3.1775) -- (2.2820, 1.0140, 3.1824) -- cycle;
\fill[blue!40.8, opacity=0.7] (2.2820, 1.0140, 3.1824) -- (2.3280, 1.0140, 3.1775) -- (2.3280, 1.0680, 3.1802) -- (2.2820, 1.0680, 3.1851) -- cycle;
\fill[blue!43.3, opacity=0.7] (2.2820, 1.0680, 3.1851) -- (2.3280, 1.0680, 3.1802) -- (2.3280, 1.1220, 3.1826) -- (2.2820, 1.1220, 3.1875) -- cycle;
\fill[blue!46.3, opacity=0.7] (2.2820, 1.1220, 3.1875) -- (2.3280, 1.1220, 3.1826) -- (2.3280, 1.1760, 3.1847) -- (2.2820, 1.1760, 3.1896) -- cycle;
\fill[blue!49.0, opacity=0.7] (2.2820, 1.1760, 3.1896) -- (2.3280, 1.1760, 3.1847) -- (2.3280, 1.2300, 3.1864) -- (2.2820, 1.2300, 3.1914) -- cycle;
\fill[blue!51.0, opacity=0.7] (2.2820, 1.2300, 3.1914) -- (2.3280, 1.2300, 3.1864) -- (2.3280, 1.2840, 3.1879) -- (2.2820, 1.2840, 3.1929) -- cycle;
\fill[blue!52.0, opacity=0.7] (2.2820, 1.2840, 3.1929) -- (2.3280, 1.2840, 3.1879) -- (2.3280, 1.3380, 3.1891) -- (2.2820, 1.3380, 3.1940) -- cycle;
\fill[blue!51.9, opacity=0.7] (2.2820, 1.3380, 3.1940) -- (2.3280, 1.3380, 3.1891) -- (2.3280, 1.3920, 3.1899) -- (2.2820, 1.3920, 3.1949) -- cycle;
\fill[blue!50.7, opacity=0.7] (2.2820, 1.3920, 3.1949) -- (2.3280, 1.3920, 3.1899) -- (2.3280, 1.4460, 3.1904) -- (2.2820, 1.4460, 3.1954) -- cycle;
\fill[blue!48.4, opacity=0.7] (2.2820, 1.4460, 3.1954) -- (2.3280, 1.4460, 3.1904) -- (2.3280, 1.5000, 3.1905) -- (2.2820, 1.5000, 3.1955) -- cycle;
\fill[blue!45.1, opacity=0.7] (2.2820, 1.5000, 3.1955) -- (2.3280, 1.5000, 3.1905) -- (2.3280, 1.5540, 3.1904) -- (2.2820, 1.5540, 3.1954) -- cycle;
\fill[blue!41.1, opacity=0.7] (2.2820, 1.5540, 3.1954) -- (2.3280, 1.5540, 3.1904) -- (2.3280, 1.6080, 3.1899) -- (2.2820, 1.6080, 3.1949) -- cycle;
\fill[blue!37.2, opacity=0.7] (2.2820, 1.6080, 3.1949) -- (2.3280, 1.6080, 3.1899) -- (2.3280, 1.6620, 3.1891) -- (2.2820, 1.6620, 3.1940) -- cycle;
\fill[blue!34.3, opacity=0.7] (2.2820, 1.6620, 3.1940) -- (2.3280, 1.6620, 3.1891) -- (2.3280, 1.7160, 3.1879) -- (2.2820, 1.7160, 3.1929) -- cycle;
\fill[blue!33.2, opacity=0.7] (2.2820, 1.7160, 3.1929) -- (2.3280, 1.7160, 3.1879) -- (2.3280, 1.7700, 3.1864) -- (2.2820, 1.7700, 3.1914) -- cycle;
\fill[blue!34.7, opacity=0.7] (2.2820, 1.7700, 3.1914) -- (2.3280, 1.7700, 3.1864) -- (2.3280, 1.8240, 3.1847) -- (2.2820, 1.8240, 3.1896) -- cycle;
\fill[blue!40.2, opacity=0.7] (2.2820, 1.8240, 3.1896) -- (2.3280, 1.8240, 3.1847) -- (2.3280, 1.8780, 3.1826) -- (2.2820, 1.8780, 3.1875) -- cycle;
\fill[blue!50.4, opacity=0.7] (2.2820, 1.8780, 3.1875) -- (2.3280, 1.8780, 3.1826) -- (2.3280, 1.9320, 3.1802) -- (2.2820, 1.9320, 3.1851) -- cycle;
\fill[blue!61.3, opacity=0.7] (2.2820, 1.9320, 3.1851) -- (2.3280, 1.9320, 3.1802) -- (2.3280, 1.9860, 3.1775) -- (2.2820, 1.9860, 3.1824) -- cycle;
\fill[blue!62.1, opacity=0.7] (2.2820, 1.9860, 3.1824) -- (2.3280, 1.9860, 3.1775) -- (2.3280, 2.0400, 3.1745) -- (2.2820, 2.0400, 3.1794) -- cycle;
\fill[blue!50.6, opacity=0.7] (2.2820, 2.0400, 3.1794) -- (2.3280, 2.0400, 3.1745) -- (2.3280, 2.0940, 3.1712) -- (2.2820, 2.0940, 3.1762) -- cycle;
\fill[blue!38.9, opacity=0.7] (2.2820, 2.0940, 3.1762) -- (2.3280, 2.0940, 3.1712) -- (2.3280, 2.1480, 3.1676) -- (2.2820, 2.1480, 3.1726) -- cycle;
\fill[blue!34.5, opacity=0.7] (2.2820, 2.1480, 3.1726) -- (2.3280, 2.1480, 3.1676) -- (2.3280, 2.2020, 3.1638) -- (2.2820, 2.2020, 3.1688) -- cycle;
\fill[blue!38.3, opacity=0.7] (2.2820, 2.2020, 3.1688) -- (2.3280, 2.2020, 3.1638) -- (2.3280, 2.2560, 3.1597) -- (2.2820, 2.2560, 3.1647) -- cycle;
\fill[blue!50.2, opacity=0.7] (2.2820, 2.2560, 3.1647) -- (2.3280, 2.2560, 3.1597) -- (2.3280, 2.3100, 3.1554) -- (2.2820, 2.3100, 3.1604) -- cycle;
\fill[blue!62.2, opacity=0.7] (2.2820, 2.3100, 3.1604) -- (2.3280, 2.3100, 3.1554) -- (2.3280, 2.3640, 3.1508) -- (2.2820, 2.3640, 3.1558) -- cycle;
\fill[blue!61.7, opacity=0.7] (2.2820, 2.3640, 3.1558) -- (2.3280, 2.3640, 3.1508) -- (2.3280, 2.4180, 3.1461) -- (2.2820, 2.4180, 3.1510) -- cycle;
\fill[blue!56.3, opacity=0.7] (2.2820, 2.4180, 3.1510) -- (2.3280, 2.4180, 3.1461) -- (2.3280, 2.4720, 3.1411) -- (2.2820, 2.4720, 3.1461) -- cycle;
\fill[blue!58.3, opacity=0.7] (2.2820, 2.4720, 3.1461) -- (2.3280, 2.4720, 3.1411) -- (2.3280, 2.5260, 3.1359) -- (2.2820, 2.5260, 3.1409) -- cycle;
\fill[blue!63.6, opacity=0.7] (2.2820, 2.5260, 3.1409) -- (2.3280, 2.5260, 3.1359) -- (2.3280, 2.5800, 3.1305) -- (2.2820, 2.5800, 3.1355) -- cycle;
\fill[blue!51.3, opacity=0.7] (2.2820, 2.5800, 3.1355) -- (2.3280, 2.5800, 3.1305) -- (2.3280, 2.6340, 3.1250) -- (2.2820, 2.6340, 3.1300) -- cycle;
\fill[blue!29.6, opacity=0.7] (2.2820, 2.6340, 3.1300) -- (2.3280, 2.6340, 3.1250) -- (2.3280, 2.6880, 3.1193) -- (2.2820, 2.6880, 3.1243) -- cycle;
\fill[blue!20.3, opacity=0.7] (2.2820, 2.6880, 3.1243) -- (2.3280, 2.6880, 3.1193) -- (2.3280, 2.7420, 3.1135) -- (2.2820, 2.7420, 3.1185) -- cycle;
\fill[blue!19.0, opacity=0.7] (2.2820, 2.7420, 3.1185) -- (2.3280, 2.7420, 3.1135) -- (2.3280, 2.7960, 3.1076) -- (2.2820, 2.7960, 3.1126) -- cycle;
\fill[blue!22.9, opacity=0.7] (2.2820, 2.7960, 3.1126) -- (2.3280, 2.7960, 3.1076) -- (2.3280, 2.8500, 3.1016) -- (2.2820, 2.8500, 3.1066) -- cycle;
\fill[blue!36.2, opacity=0.7] (2.2820, 2.8500, 3.1066) -- (2.3280, 2.8500, 3.1016) -- (2.3280, 2.9040, 3.0955) -- (2.2820, 2.9040, 3.1005) -- cycle;
\fill[blue!54.0, opacity=0.7] (2.2820, 2.9040, 3.1005) -- (2.3280, 2.9040, 3.0955) -- (2.3280, 2.9580, 3.0893) -- (2.2820, 2.9580, 3.0943) -- cycle;
\fill[blue!60.8, opacity=0.7] (2.2820, 2.9580, 3.0943) -- (2.3280, 2.9580, 3.0893) -- (2.3280, 3.0120, 3.0831) -- (2.2820, 3.0120, 3.0881) -- cycle;
\fill[blue!57.7, opacity=0.7] (2.2820, 3.0120, 3.0881) -- (2.3280, 3.0120, 3.0831) -- (2.3280, 3.0660, 3.0768) -- (2.2820, 3.0660, 3.0818) -- cycle;
\fill[blue!40.3, opacity=0.7] (2.2820, 3.0660, 3.0818) -- (2.3280, 3.0660, 3.0768) -- (2.3280, 3.1200, 3.0705) -- (2.2820, 3.1200, 3.0755) -- cycle;
\fill[blue!16.1, opacity=0.7] (2.3280, -0.1200, 3.0705) -- (2.3740, -0.1200, 3.0654) -- (2.3740, -0.0660, 3.0716) -- (2.3280, -0.0660, 3.0768) -- cycle;
\fill[blue!15.9, opacity=0.7] (2.3280, -0.0660, 3.0768) -- (2.3740, -0.0660, 3.0716) -- (2.3740, -0.0120, 3.0779) -- (2.3280, -0.0120, 3.0831) -- cycle;
\fill[blue!17.2, opacity=0.7] (2.3280, -0.0120, 3.0831) -- (2.3740, -0.0120, 3.0779) -- (2.3740, 0.0420, 3.0841) -- (2.3280, 0.0420, 3.0893) -- cycle;
\fill[blue!24.1, opacity=0.7] (2.3280, 0.0420, 3.0893) -- (2.3740, 0.0420, 3.0841) -- (2.3740, 0.0960, 3.0903) -- (2.3280, 0.0960, 3.0955) -- cycle;
\fill[blue!42.6, opacity=0.7] (2.3280, 0.0960, 3.0955) -- (2.3740, 0.0960, 3.0903) -- (2.3740, 0.1500, 3.0964) -- (2.3280, 0.1500, 3.1016) -- cycle;
\fill[blue!59.8, opacity=0.7] (2.3280, 0.1500, 3.1016) -- (2.3740, 0.1500, 3.0964) -- (2.3740, 0.2040, 3.1024) -- (2.3280, 0.2040, 3.1076) -- cycle;
\fill[blue!63.6, opacity=0.7] (2.3280, 0.2040, 3.1076) -- (2.3740, 0.2040, 3.1024) -- (2.3740, 0.2580, 3.1084) -- (2.3280, 0.2580, 3.1135) -- cycle;
\fill[blue!63.5, opacity=0.7] (2.3280, 0.2580, 3.1135) -- (2.3740, 0.2580, 3.1084) -- (2.3740, 0.3120, 3.1142) -- (2.3280, 0.3120, 3.1193) -- cycle;
\fill[blue!62.4, opacity=0.7] (2.3280, 0.3120, 3.1193) -- (2.3740, 0.3120, 3.1142) -- (2.3740, 0.3660, 3.1198) -- (2.3280, 0.3660, 3.1250) -- cycle;
\fill[blue!53.6, opacity=0.7] (2.3280, 0.3660, 3.1250) -- (2.3740, 0.3660, 3.1198) -- (2.3740, 0.4200, 3.1254) -- (2.3280, 0.4200, 3.1305) -- cycle;
\fill[blue!39.4, opacity=0.7] (2.3280, 0.4200, 3.1305) -- (2.3740, 0.4200, 3.1254) -- (2.3740, 0.4740, 3.1307) -- (2.3280, 0.4740, 3.1359) -- cycle;
\fill[blue!29.2, opacity=0.7] (2.3280, 0.4740, 3.1359) -- (2.3740, 0.4740, 3.1307) -- (2.3740, 0.5280, 3.1359) -- (2.3280, 0.5280, 3.1411) -- cycle;
\fill[blue!25.2, opacity=0.7] (2.3280, 0.5280, 3.1411) -- (2.3740, 0.5280, 3.1359) -- (2.3740, 0.5820, 3.1409) -- (2.3280, 0.5820, 3.1461) -- cycle;
\fill[blue!25.9, opacity=0.7] (2.3280, 0.5820, 3.1461) -- (2.3740, 0.5820, 3.1409) -- (2.3740, 0.6360, 3.1457) -- (2.3280, 0.6360, 3.1508) -- cycle;
\fill[blue!31.3, opacity=0.7] (2.3280, 0.6360, 3.1508) -- (2.3740, 0.6360, 3.1457) -- (2.3740, 0.6900, 3.1502) -- (2.3280, 0.6900, 3.1554) -- cycle;
\fill[blue!42.4, opacity=0.7] (2.3280, 0.6900, 3.1554) -- (2.3740, 0.6900, 3.1502) -- (2.3740, 0.7440, 3.1545) -- (2.3280, 0.7440, 3.1597) -- cycle;
\fill[blue!55.9, opacity=0.7] (2.3280, 0.7440, 3.1597) -- (2.3740, 0.7440, 3.1545) -- (2.3740, 0.7980, 3.1586) -- (2.3280, 0.7980, 3.1638) -- cycle;
\fill[blue!63.3, opacity=0.7] (2.3280, 0.7980, 3.1638) -- (2.3740, 0.7980, 3.1586) -- (2.3740, 0.8520, 3.1624) -- (2.3280, 0.8520, 3.1676) -- cycle;
\fill[blue!60.9, opacity=0.7] (2.3280, 0.8520, 3.1676) -- (2.3740, 0.8520, 3.1624) -- (2.3740, 0.9060, 3.1660) -- (2.3280, 0.9060, 3.1712) -- cycle;
\fill[blue!53.4, opacity=0.7] (2.3280, 0.9060, 3.1712) -- (2.3740, 0.9060, 3.1660) -- (2.3740, 0.9600, 3.1693) -- (2.3280, 0.9600, 3.1745) -- cycle;
\fill[blue!46.4, opacity=0.7] (2.3280, 0.9600, 3.1745) -- (2.3740, 0.9600, 3.1693) -- (2.3740, 1.0140, 3.1723) -- (2.3280, 1.0140, 3.1775) -- cycle;
\fill[blue!41.7, opacity=0.7] (2.3280, 1.0140, 3.1775) -- (2.3740, 1.0140, 3.1723) -- (2.3740, 1.0680, 3.1750) -- (2.3280, 1.0680, 3.1802) -- cycle;
\fill[blue!39.2, opacity=0.7] (2.3280, 1.0680, 3.1802) -- (2.3740, 1.0680, 3.1750) -- (2.3740, 1.1220, 3.1774) -- (2.3280, 1.1220, 3.1826) -- cycle;
\fill[blue!38.1, opacity=0.7] (2.3280, 1.1220, 3.1826) -- (2.3740, 1.1220, 3.1774) -- (2.3740, 1.1760, 3.1795) -- (2.3280, 1.1760, 3.1847) -- cycle;
\fill[blue!37.9, opacity=0.7] (2.3280, 1.1760, 3.1847) -- (2.3740, 1.1760, 3.1795) -- (2.3740, 1.2300, 3.1813) -- (2.3280, 1.2300, 3.1864) -- cycle;
\fill[blue!37.9, opacity=0.7] (2.3280, 1.2300, 3.1864) -- (2.3740, 1.2300, 3.1813) -- (2.3740, 1.2840, 3.1827) -- (2.3280, 1.2840, 3.1879) -- cycle;
\fill[blue!37.8, opacity=0.7] (2.3280, 1.2840, 3.1879) -- (2.3740, 1.2840, 3.1827) -- (2.3740, 1.3380, 3.1839) -- (2.3280, 1.3380, 3.1891) -- cycle;
\fill[blue!37.5, opacity=0.7] (2.3280, 1.3380, 3.1891) -- (2.3740, 1.3380, 3.1839) -- (2.3740, 1.3920, 3.1847) -- (2.3280, 1.3920, 3.1899) -- cycle;
\fill[blue!36.8, opacity=0.7] (2.3280, 1.3920, 3.1899) -- (2.3740, 1.3920, 3.1847) -- (2.3740, 1.4460, 3.1852) -- (2.3280, 1.4460, 3.1904) -- cycle;
\fill[blue!35.9, opacity=0.7] (2.3280, 1.4460, 3.1904) -- (2.3740, 1.4460, 3.1852) -- (2.3740, 1.5000, 3.1854) -- (2.3280, 1.5000, 3.1905) -- cycle;
\fill[blue!35.0, opacity=0.7] (2.3280, 1.5000, 3.1905) -- (2.3740, 1.5000, 3.1854) -- (2.3740, 1.5540, 3.1852) -- (2.3280, 1.5540, 3.1904) -- cycle;
\fill[blue!34.6, opacity=0.7] (2.3280, 1.5540, 3.1904) -- (2.3740, 1.5540, 3.1852) -- (2.3740, 1.6080, 3.1847) -- (2.3280, 1.6080, 3.1899) -- cycle;
\fill[blue!35.2, opacity=0.7] (2.3280, 1.6080, 3.1899) -- (2.3740, 1.6080, 3.1847) -- (2.3740, 1.6620, 3.1839) -- (2.3280, 1.6620, 3.1891) -- cycle;
\fill[blue!37.5, opacity=0.7] (2.3280, 1.6620, 3.1891) -- (2.3740, 1.6620, 3.1839) -- (2.3740, 1.7160, 3.1827) -- (2.3280, 1.7160, 3.1879) -- cycle;
\fill[blue!42.3, opacity=0.7] (2.3280, 1.7160, 3.1879) -- (2.3740, 1.7160, 3.1827) -- (2.3740, 1.7700, 3.1813) -- (2.3280, 1.7700, 3.1864) -- cycle;
\fill[blue!50.3, opacity=0.7] (2.3280, 1.7700, 3.1864) -- (2.3740, 1.7700, 3.1813) -- (2.3740, 1.8240, 3.1795) -- (2.3280, 1.8240, 3.1847) -- cycle;
\fill[blue!59.4, opacity=0.7] (2.3280, 1.8240, 3.1847) -- (2.3740, 1.8240, 3.1795) -- (2.3740, 1.8780, 3.1774) -- (2.3280, 1.8780, 3.1826) -- cycle;
\fill[blue!63.5, opacity=0.7] (2.3280, 1.8780, 3.1826) -- (2.3740, 1.8780, 3.1774) -- (2.3740, 1.9320, 3.1750) -- (2.3280, 1.9320, 3.1802) -- cycle;
\fill[blue!57.3, opacity=0.7] (2.3280, 1.9320, 3.1802) -- (2.3740, 1.9320, 3.1750) -- (2.3740, 1.9860, 3.1723) -- (2.3280, 1.9860, 3.1775) -- cycle;
\fill[blue!45.3, opacity=0.7] (2.3280, 1.9860, 3.1775) -- (2.3740, 1.9860, 3.1723) -- (2.3740, 2.0400, 3.1693) -- (2.3280, 2.0400, 3.1745) -- cycle;
\fill[blue!36.5, opacity=0.7] (2.3280, 2.0400, 3.1745) -- (2.3740, 2.0400, 3.1693) -- (2.3740, 2.0940, 3.1660) -- (2.3280, 2.0940, 3.1712) -- cycle;
\fill[blue!34.0, opacity=0.7] (2.3280, 2.0940, 3.1712) -- (2.3740, 2.0940, 3.1660) -- (2.3740, 2.1480, 3.1624) -- (2.3280, 2.1480, 3.1676) -- cycle;
\fill[blue!38.4, opacity=0.7] (2.3280, 2.1480, 3.1676) -- (2.3740, 2.1480, 3.1624) -- (2.3740, 2.2020, 3.1586) -- (2.3280, 2.2020, 3.1638) -- cycle;
\fill[blue!49.9, opacity=0.7] (2.3280, 2.2020, 3.1638) -- (2.3740, 2.2020, 3.1586) -- (2.3740, 2.2560, 3.1545) -- (2.3280, 2.2560, 3.1597) -- cycle;
\fill[blue!61.6, opacity=0.7] (2.3280, 2.2560, 3.1597) -- (2.3740, 2.2560, 3.1545) -- (2.3740, 2.3100, 3.1502) -- (2.3280, 2.3100, 3.1554) -- cycle;
\fill[blue!62.4, opacity=0.7] (2.3280, 2.3100, 3.1554) -- (2.3740, 2.3100, 3.1502) -- (2.3740, 2.3640, 3.1457) -- (2.3280, 2.3640, 3.1508) -- cycle;
\fill[blue!57.1, opacity=0.7] (2.3280, 2.3640, 3.1508) -- (2.3740, 2.3640, 3.1457) -- (2.3740, 2.4180, 3.1409) -- (2.3280, 2.4180, 3.1461) -- cycle;
\fill[blue!57.6, opacity=0.7] (2.3280, 2.4180, 3.1461) -- (2.3740, 2.4180, 3.1409) -- (2.3740, 2.4720, 3.1359) -- (2.3280, 2.4720, 3.1411) -- cycle;
\fill[blue!63.2, opacity=0.7] (2.3280, 2.4720, 3.1411) -- (2.3740, 2.4720, 3.1359) -- (2.3740, 2.5260, 3.1307) -- (2.3280, 2.5260, 3.1359) -- cycle;
\fill[blue!56.4, opacity=0.7] (2.3280, 2.5260, 3.1359) -- (2.3740, 2.5260, 3.1307) -- (2.3740, 2.5800, 3.1254) -- (2.3280, 2.5800, 3.1305) -- cycle;
\fill[blue!34.6, opacity=0.7] (2.3280, 2.5800, 3.1305) -- (2.3740, 2.5800, 3.1254) -- (2.3740, 2.6340, 3.1198) -- (2.3280, 2.6340, 3.1250) -- cycle;
\fill[blue!21.9, opacity=0.7] (2.3280, 2.6340, 3.1250) -- (2.3740, 2.6340, 3.1198) -- (2.3740, 2.6880, 3.1142) -- (2.3280, 2.6880, 3.1193) -- cycle;
\fill[blue!18.8, opacity=0.7] (2.3280, 2.6880, 3.1193) -- (2.3740, 2.6880, 3.1142) -- (2.3740, 2.7420, 3.1084) -- (2.3280, 2.7420, 3.1135) -- cycle;
\fill[blue!20.5, opacity=0.7] (2.3280, 2.7420, 3.1135) -- (2.3740, 2.7420, 3.1084) -- (2.3740, 2.7960, 3.1024) -- (2.3280, 2.7960, 3.1076) -- cycle;
\fill[blue!29.3, opacity=0.7] (2.3280, 2.7960, 3.1076) -- (2.3740, 2.7960, 3.1024) -- (2.3740, 2.8500, 3.0964) -- (2.3280, 2.8500, 3.1016) -- cycle;
\fill[blue!46.9, opacity=0.7] (2.3280, 2.8500, 3.1016) -- (2.3740, 2.8500, 3.0964) -- (2.3740, 2.9040, 3.0903) -- (2.3280, 2.9040, 3.0955) -- cycle;
\fill[blue!59.1, opacity=0.7] (2.3280, 2.9040, 3.0955) -- (2.3740, 2.9040, 3.0903) -- (2.3740, 2.9580, 3.0841) -- (2.3280, 2.9580, 3.0893) -- cycle;
\fill[blue!60.2, opacity=0.7] (2.3280, 2.9580, 3.0893) -- (2.3740, 2.9580, 3.0841) -- (2.3740, 3.0120, 3.0779) -- (2.3280, 3.0120, 3.0831) -- cycle;
\fill[blue!50.2, opacity=0.7] (2.3280, 3.0120, 3.0831) -- (2.3740, 3.0120, 3.0779) -- (2.3740, 3.0660, 3.0716) -- (2.3280, 3.0660, 3.0768) -- cycle;
\fill[blue!28.8, opacity=0.7] (2.3280, 3.0660, 3.0768) -- (2.3740, 3.0660, 3.0716) -- (2.3740, 3.1200, 3.0654) -- (2.3280, 3.1200, 3.0705) -- cycle;
\fill[blue!17.4, opacity=0.7] (2.3740, -0.1200, 3.0654) -- (2.4200, -0.1200, 3.0600) -- (2.4200, -0.0660, 3.0663) -- (2.3740, -0.0660, 3.0716) -- cycle;
\fill[blue!16.0, opacity=0.7] (2.3740, -0.0660, 3.0716) -- (2.4200, -0.0660, 3.0663) -- (2.4200, -0.0120, 3.0725) -- (2.3740, -0.0120, 3.0779) -- cycle;
\fill[blue!16.0, opacity=0.7] (2.3740, -0.0120, 3.0779) -- (2.4200, -0.0120, 3.0725) -- (2.4200, 0.0420, 3.0788) -- (2.3740, 0.0420, 3.0841) -- cycle;
\fill[blue!17.9, opacity=0.7] (2.3740, 0.0420, 3.0841) -- (2.4200, 0.0420, 3.0788) -- (2.4200, 0.0960, 3.0849) -- (2.3740, 0.0960, 3.0903) -- cycle;
\fill[blue!26.1, opacity=0.7] (2.3740, 0.0960, 3.0903) -- (2.4200, 0.0960, 3.0849) -- (2.4200, 0.1500, 3.0911) -- (2.3740, 0.1500, 3.0964) -- cycle;
\fill[blue!45.0, opacity=0.7] (2.3740, 0.1500, 3.0964) -- (2.4200, 0.1500, 3.0911) -- (2.4200, 0.2040, 3.0971) -- (2.3740, 0.2040, 3.1024) -- cycle;
\fill[blue!60.5, opacity=0.7] (2.3740, 0.2040, 3.1024) -- (2.4200, 0.2040, 3.0971) -- (2.4200, 0.2580, 3.1030) -- (2.3740, 0.2580, 3.1084) -- cycle;
\fill[blue!63.6, opacity=0.7] (2.3740, 0.2580, 3.1084) -- (2.4200, 0.2580, 3.1030) -- (2.4200, 0.3120, 3.1088) -- (2.3740, 0.3120, 3.1142) -- cycle;
\fill[blue!63.5, opacity=0.7] (2.3740, 0.3120, 3.1142) -- (2.4200, 0.3120, 3.1088) -- (2.4200, 0.3660, 3.1145) -- (2.3740, 0.3660, 3.1198) -- cycle;
\fill[blue!62.8, opacity=0.7] (2.3740, 0.3660, 3.1198) -- (2.4200, 0.3660, 3.1145) -- (2.4200, 0.4200, 3.1200) -- (2.3740, 0.4200, 3.1254) -- cycle;
\fill[blue!55.9, opacity=0.7] (2.3740, 0.4200, 3.1254) -- (2.4200, 0.4200, 3.1200) -- (2.4200, 0.4740, 3.1254) -- (2.3740, 0.4740, 3.1307) -- cycle;
\fill[blue!42.9, opacity=0.7] (2.3740, 0.4740, 3.1307) -- (2.4200, 0.4740, 3.1254) -- (2.4200, 0.5280, 3.1305) -- (2.3740, 0.5280, 3.1359) -- cycle;
\fill[blue!31.9, opacity=0.7] (2.3740, 0.5280, 3.1359) -- (2.4200, 0.5280, 3.1305) -- (2.4200, 0.5820, 3.1355) -- (2.3740, 0.5820, 3.1409) -- cycle;
\fill[blue!26.4, opacity=0.7] (2.3740, 0.5820, 3.1409) -- (2.4200, 0.5820, 3.1355) -- (2.4200, 0.6360, 3.1403) -- (2.3740, 0.6360, 3.1457) -- cycle;
\fill[blue!25.4, opacity=0.7] (2.3740, 0.6360, 3.1457) -- (2.4200, 0.6360, 3.1403) -- (2.4200, 0.6900, 3.1449) -- (2.3740, 0.6900, 3.1502) -- cycle;
\fill[blue!28.1, opacity=0.7] (2.3740, 0.6900, 3.1502) -- (2.4200, 0.6900, 3.1449) -- (2.4200, 0.7440, 3.1492) -- (2.3740, 0.7440, 3.1545) -- cycle;
\fill[blue!34.8, opacity=0.7] (2.3740, 0.7440, 3.1545) -- (2.4200, 0.7440, 3.1492) -- (2.4200, 0.7980, 3.1533) -- (2.3740, 0.7980, 3.1586) -- cycle;
\fill[blue!45.6, opacity=0.7] (2.3740, 0.7980, 3.1586) -- (2.4200, 0.7980, 3.1533) -- (2.4200, 0.8520, 3.1571) -- (2.3740, 0.8520, 3.1624) -- cycle;
\fill[blue!56.7, opacity=0.7] (2.3740, 0.8520, 3.1624) -- (2.4200, 0.8520, 3.1571) -- (2.4200, 0.9060, 3.1606) -- (2.3740, 0.9060, 3.1660) -- cycle;
\fill[blue!63.0, opacity=0.7] (2.3740, 0.9060, 3.1660) -- (2.4200, 0.9060, 3.1606) -- (2.4200, 0.9600, 3.1639) -- (2.3740, 0.9600, 3.1693) -- cycle;
\fill[blue!62.6, opacity=0.7] (2.3740, 0.9600, 3.1693) -- (2.4200, 0.9600, 3.1639) -- (2.4200, 1.0140, 3.1669) -- (2.3740, 1.0140, 3.1723) -- cycle;
\fill[blue!58.0, opacity=0.7] (2.3740, 1.0140, 3.1723) -- (2.4200, 1.0140, 3.1669) -- (2.4200, 1.0680, 3.1696) -- (2.3740, 1.0680, 3.1750) -- cycle;
\fill[blue!52.5, opacity=0.7] (2.3740, 1.0680, 3.1750) -- (2.4200, 1.0680, 3.1696) -- (2.4200, 1.1220, 3.1720) -- (2.3740, 1.1220, 3.1774) -- cycle;
\fill[blue!47.9, opacity=0.7] (2.3740, 1.1220, 3.1774) -- (2.4200, 1.1220, 3.1720) -- (2.4200, 1.1760, 3.1741) -- (2.3740, 1.1760, 3.1795) -- cycle;
\fill[blue!44.5, opacity=0.7] (2.3740, 1.1760, 3.1795) -- (2.4200, 1.1760, 3.1741) -- (2.4200, 1.2300, 3.1759) -- (2.3740, 1.2300, 3.1813) -- cycle;
\fill[blue!42.2, opacity=0.7] (2.3740, 1.2300, 3.1813) -- (2.4200, 1.2300, 3.1759) -- (2.4200, 1.2840, 3.1774) -- (2.3740, 1.2840, 3.1827) -- cycle;
\fill[blue!40.9, opacity=0.7] (2.3740, 1.2840, 3.1827) -- (2.4200, 1.2840, 3.1774) -- (2.4200, 1.3380, 3.1785) -- (2.3740, 1.3380, 3.1839) -- cycle;
\fill[blue!40.2, opacity=0.7] (2.3740, 1.3380, 3.1839) -- (2.4200, 1.3380, 3.1785) -- (2.4200, 1.3920, 3.1793) -- (2.3740, 1.3920, 3.1847) -- cycle;
\fill[blue!40.1, opacity=0.7] (2.3740, 1.3920, 3.1847) -- (2.4200, 1.3920, 3.1793) -- (2.4200, 1.4460, 3.1798) -- (2.3740, 1.4460, 3.1852) -- cycle;
\fill[blue!40.8, opacity=0.7] (2.3740, 1.4460, 3.1852) -- (2.4200, 1.4460, 3.1798) -- (2.4200, 1.5000, 3.1800) -- (2.3740, 1.5000, 3.1854) -- cycle;
\fill[blue!42.5, opacity=0.7] (2.3740, 1.5000, 3.1854) -- (2.4200, 1.5000, 3.1800) -- (2.4200, 1.5540, 3.1798) -- (2.3740, 1.5540, 3.1852) -- cycle;
\fill[blue!45.5, opacity=0.7] (2.3740, 1.5540, 3.1852) -- (2.4200, 1.5540, 3.1798) -- (2.4200, 1.6080, 3.1793) -- (2.3740, 1.6080, 3.1847) -- cycle;
\fill[blue!50.1, opacity=0.7] (2.3740, 1.6080, 3.1847) -- (2.4200, 1.6080, 3.1793) -- (2.4200, 1.6620, 3.1785) -- (2.3740, 1.6620, 3.1839) -- cycle;
\fill[blue!56.0, opacity=0.7] (2.3740, 1.6620, 3.1839) -- (2.4200, 1.6620, 3.1785) -- (2.4200, 1.7160, 3.1774) -- (2.3740, 1.7160, 3.1827) -- cycle;
\fill[blue!61.6, opacity=0.7] (2.3740, 1.7160, 3.1827) -- (2.4200, 1.7160, 3.1774) -- (2.4200, 1.7700, 3.1759) -- (2.3740, 1.7700, 3.1813) -- cycle;
\fill[blue!63.4, opacity=0.7] (2.3740, 1.7700, 3.1813) -- (2.4200, 1.7700, 3.1759) -- (2.4200, 1.8240, 3.1741) -- (2.3740, 1.8240, 3.1795) -- cycle;
\fill[blue!58.5, opacity=0.7] (2.3740, 1.8240, 3.1795) -- (2.4200, 1.8240, 3.1741) -- (2.4200, 1.8780, 3.1720) -- (2.3740, 1.8780, 3.1774) -- cycle;
\fill[blue!48.5, opacity=0.7] (2.3740, 1.8780, 3.1774) -- (2.4200, 1.8780, 3.1720) -- (2.4200, 1.9320, 3.1696) -- (2.3740, 1.9320, 3.1750) -- cycle;
\fill[blue!39.1, opacity=0.7] (2.3740, 1.9320, 3.1750) -- (2.4200, 1.9320, 3.1696) -- (2.4200, 1.9860, 3.1669) -- (2.3740, 1.9860, 3.1723) -- cycle;
\fill[blue!34.0, opacity=0.7] (2.3740, 1.9860, 3.1723) -- (2.4200, 1.9860, 3.1669) -- (2.4200, 2.0400, 3.1639) -- (2.3740, 2.0400, 3.1693) -- cycle;
\fill[blue!34.0, opacity=0.7] (2.3740, 2.0400, 3.1693) -- (2.4200, 2.0400, 3.1639) -- (2.4200, 2.0940, 3.1606) -- (2.3740, 2.0940, 3.1660) -- cycle;
\fill[blue!39.8, opacity=0.7] (2.3740, 2.0940, 3.1660) -- (2.4200, 2.0940, 3.1606) -- (2.4200, 2.1480, 3.1571) -- (2.3740, 2.1480, 3.1624) -- cycle;
\fill[blue!51.3, opacity=0.7] (2.3740, 2.1480, 3.1624) -- (2.4200, 2.1480, 3.1571) -- (2.4200, 2.2020, 3.1533) -- (2.3740, 2.2020, 3.1586) -- cycle;
\fill[blue!61.9, opacity=0.7] (2.3740, 2.2020, 3.1586) -- (2.4200, 2.2020, 3.1533) -- (2.4200, 2.2560, 3.1492) -- (2.3740, 2.2560, 3.1545) -- cycle;
\fill[blue!62.5, opacity=0.7] (2.3740, 2.2560, 3.1545) -- (2.4200, 2.2560, 3.1492) -- (2.4200, 2.3100, 3.1449) -- (2.3740, 2.3100, 3.1502) -- cycle;
\fill[blue!57.7, opacity=0.7] (2.3740, 2.3100, 3.1502) -- (2.4200, 2.3100, 3.1449) -- (2.4200, 2.3640, 3.1403) -- (2.3740, 2.3640, 3.1457) -- cycle;
\fill[blue!57.6, opacity=0.7] (2.3740, 2.3640, 3.1457) -- (2.4200, 2.3640, 3.1403) -- (2.4200, 2.4180, 3.1355) -- (2.3740, 2.4180, 3.1409) -- cycle;
\fill[blue!62.8, opacity=0.7] (2.3740, 2.4180, 3.1409) -- (2.4200, 2.4180, 3.1355) -- (2.4200, 2.4720, 3.1305) -- (2.3740, 2.4720, 3.1359) -- cycle;
\fill[blue!58.9, opacity=0.7] (2.3740, 2.4720, 3.1359) -- (2.4200, 2.4720, 3.1305) -- (2.4200, 2.5260, 3.1254) -- (2.3740, 2.5260, 3.1307) -- cycle;
\fill[blue!38.7, opacity=0.7] (2.3740, 2.5260, 3.1307) -- (2.4200, 2.5260, 3.1254) -- (2.4200, 2.5800, 3.1200) -- (2.3740, 2.5800, 3.1254) -- cycle;
\fill[blue!23.6, opacity=0.7] (2.3740, 2.5800, 3.1254) -- (2.4200, 2.5800, 3.1200) -- (2.4200, 2.6340, 3.1145) -- (2.3740, 2.6340, 3.1198) -- cycle;
\fill[blue!19.0, opacity=0.7] (2.3740, 2.6340, 3.1198) -- (2.4200, 2.6340, 3.1145) -- (2.4200, 2.6880, 3.1088) -- (2.3740, 2.6880, 3.1142) -- cycle;
\fill[blue!19.4, opacity=0.7] (2.3740, 2.6880, 3.1142) -- (2.4200, 2.6880, 3.1088) -- (2.4200, 2.7420, 3.1030) -- (2.3740, 2.7420, 3.1084) -- cycle;
\fill[blue!25.0, opacity=0.7] (2.3740, 2.7420, 3.1084) -- (2.4200, 2.7420, 3.1030) -- (2.4200, 2.7960, 3.0971) -- (2.3740, 2.7960, 3.1024) -- cycle;
\fill[blue!39.8, opacity=0.7] (2.3740, 2.7960, 3.1024) -- (2.4200, 2.7960, 3.0971) -- (2.4200, 2.8500, 3.0911) -- (2.3740, 2.8500, 3.0964) -- cycle;
\fill[blue!55.6, opacity=0.7] (2.3740, 2.8500, 3.0964) -- (2.4200, 2.8500, 3.0911) -- (2.4200, 2.9040, 3.0849) -- (2.3740, 2.9040, 3.0903) -- cycle;
\fill[blue!60.6, opacity=0.7] (2.3740, 2.9040, 3.0903) -- (2.4200, 2.9040, 3.0849) -- (2.4200, 2.9580, 3.0788) -- (2.3740, 2.9580, 3.0841) -- cycle;
\fill[blue!56.2, opacity=0.7] (2.3740, 2.9580, 3.0841) -- (2.4200, 2.9580, 3.0788) -- (2.4200, 3.0120, 3.0725) -- (2.3740, 3.0120, 3.0779) -- cycle;
\fill[blue!38.1, opacity=0.7] (2.3740, 3.0120, 3.0779) -- (2.4200, 3.0120, 3.0725) -- (2.4200, 3.0660, 3.0663) -- (2.3740, 3.0660, 3.0716) -- cycle;
\fill[blue!20.7, opacity=0.7] (2.3740, 3.0660, 3.0716) -- (2.4200, 3.0660, 3.0663) -- (2.4200, 3.1200, 3.0600) -- (2.3740, 3.1200, 3.0654) -- cycle;
\fill[blue!21.4, opacity=0.7] (2.4200, -0.1200, 3.0600) -- (2.4660, -0.1200, 3.0545) -- (2.4660, -0.0660, 3.0608) -- (2.4200, -0.0660, 3.0663) -- cycle;
\fill[blue!16.9, opacity=0.7] (2.4200, -0.0660, 3.0663) -- (2.4660, -0.0660, 3.0608) -- (2.4660, -0.0120, 3.0670) -- (2.4200, -0.0120, 3.0725) -- cycle;
\fill[blue!15.9, opacity=0.7] (2.4200, -0.0120, 3.0725) -- (2.4660, -0.0120, 3.0670) -- (2.4660, 0.0420, 3.0733) -- (2.4200, 0.0420, 3.0788) -- cycle;
\fill[blue!16.2, opacity=0.7] (2.4200, 0.0420, 3.0788) -- (2.4660, 0.0420, 3.0733) -- (2.4660, 0.0960, 3.0794) -- (2.4200, 0.0960, 3.0849) -- cycle;
\fill[blue!18.4, opacity=0.7] (2.4200, 0.0960, 3.0849) -- (2.4660, 0.0960, 3.0794) -- (2.4660, 0.1500, 3.0855) -- (2.4200, 0.1500, 3.0911) -- cycle;
\fill[blue!27.2, opacity=0.7] (2.4200, 0.1500, 3.0911) -- (2.4660, 0.1500, 3.0855) -- (2.4660, 0.2040, 3.0916) -- (2.4200, 0.2040, 3.0971) -- cycle;
\fill[blue!45.7, opacity=0.7] (2.4200, 0.2040, 3.0971) -- (2.4660, 0.2040, 3.0916) -- (2.4660, 0.2580, 3.0975) -- (2.4200, 0.2580, 3.1030) -- cycle;
\fill[blue!60.5, opacity=0.7] (2.4200, 0.2580, 3.1030) -- (2.4660, 0.2580, 3.0975) -- (2.4660, 0.3120, 3.1033) -- (2.4200, 0.3120, 3.1088) -- cycle;
\fill[blue!63.6, opacity=0.7] (2.4200, 0.3120, 3.1088) -- (2.4660, 0.3120, 3.1033) -- (2.4660, 0.3660, 3.1090) -- (2.4200, 0.3660, 3.1145) -- cycle;
\fill[blue!63.4, opacity=0.7] (2.4200, 0.3660, 3.1145) -- (2.4660, 0.3660, 3.1090) -- (2.4660, 0.4200, 3.1145) -- (2.4200, 0.4200, 3.1200) -- cycle;
\fill[blue!63.4, opacity=0.7] (2.4200, 0.4200, 3.1200) -- (2.4660, 0.4200, 3.1145) -- (2.4660, 0.4740, 3.1198) -- (2.4200, 0.4740, 3.1254) -- cycle;
\fill[blue!59.0, opacity=0.7] (2.4200, 0.4740, 3.1254) -- (2.4660, 0.4740, 3.1198) -- (2.4660, 0.5280, 3.1250) -- (2.4200, 0.5280, 3.1305) -- cycle;
\fill[blue!48.1, opacity=0.7] (2.4200, 0.5280, 3.1305) -- (2.4660, 0.5280, 3.1250) -- (2.4660, 0.5820, 3.1300) -- (2.4200, 0.5820, 3.1355) -- cycle;
\fill[blue!36.5, opacity=0.7] (2.4200, 0.5820, 3.1355) -- (2.4660, 0.5820, 3.1300) -- (2.4660, 0.6360, 3.1348) -- (2.4200, 0.6360, 3.1403) -- cycle;
\fill[blue!29.1, opacity=0.7] (2.4200, 0.6360, 3.1403) -- (2.4660, 0.6360, 3.1348) -- (2.4660, 0.6900, 3.1393) -- (2.4200, 0.6900, 3.1449) -- cycle;
\fill[blue!26.1, opacity=0.7] (2.4200, 0.6900, 3.1449) -- (2.4660, 0.6900, 3.1393) -- (2.4660, 0.7440, 3.1437) -- (2.4200, 0.7440, 3.1492) -- cycle;
\fill[blue!26.2, opacity=0.7] (2.4200, 0.7440, 3.1492) -- (2.4660, 0.7440, 3.1437) -- (2.4660, 0.7980, 3.1477) -- (2.4200, 0.7980, 3.1533) -- cycle;
\fill[blue!29.1, opacity=0.7] (2.4200, 0.7980, 3.1533) -- (2.4660, 0.7980, 3.1477) -- (2.4660, 0.8520, 3.1516) -- (2.4200, 0.8520, 3.1571) -- cycle;
\fill[blue!34.9, opacity=0.7] (2.4200, 0.8520, 3.1571) -- (2.4660, 0.8520, 3.1516) -- (2.4660, 0.9060, 3.1551) -- (2.4200, 0.9060, 3.1606) -- cycle;
\fill[blue!43.2, opacity=0.7] (2.4200, 0.9060, 3.1606) -- (2.4660, 0.9060, 3.1551) -- (2.4660, 0.9600, 3.1584) -- (2.4200, 0.9600, 3.1639) -- cycle;
\fill[blue!52.2, opacity=0.7] (2.4200, 0.9600, 3.1639) -- (2.4660, 0.9600, 3.1584) -- (2.4660, 1.0140, 3.1614) -- (2.4200, 1.0140, 3.1669) -- cycle;
\fill[blue!59.2, opacity=0.7] (2.4200, 1.0140, 3.1669) -- (2.4660, 1.0140, 3.1614) -- (2.4660, 1.0680, 3.1641) -- (2.4200, 1.0680, 3.1696) -- cycle;
\fill[blue!62.9, opacity=0.7] (2.4200, 1.0680, 3.1696) -- (2.4660, 1.0680, 3.1641) -- (2.4660, 1.1220, 3.1665) -- (2.4200, 1.1220, 3.1720) -- cycle;
\fill[blue!63.5, opacity=0.7] (2.4200, 1.1220, 3.1720) -- (2.4660, 1.1220, 3.1665) -- (2.4660, 1.1760, 3.1686) -- (2.4200, 1.1760, 3.1741) -- cycle;
\fill[blue!62.4, opacity=0.7] (2.4200, 1.1760, 3.1741) -- (2.4660, 1.1760, 3.1686) -- (2.4660, 1.2300, 3.1704) -- (2.4200, 1.2300, 3.1759) -- cycle;
\fill[blue!60.8, opacity=0.7] (2.4200, 1.2300, 3.1759) -- (2.4660, 1.2300, 3.1704) -- (2.4660, 1.2840, 3.1719) -- (2.4200, 1.2840, 3.1774) -- cycle;
\fill[blue!59.5, opacity=0.7] (2.4200, 1.2840, 3.1774) -- (2.4660, 1.2840, 3.1719) -- (2.4660, 1.3380, 3.1730) -- (2.4200, 1.3380, 3.1785) -- cycle;
\fill[blue!58.9, opacity=0.7] (2.4200, 1.3380, 3.1785) -- (2.4660, 1.3380, 3.1730) -- (2.4660, 1.3920, 3.1738) -- (2.4200, 1.3920, 3.1793) -- cycle;
\fill[blue!59.2, opacity=0.7] (2.4200, 1.3920, 3.1793) -- (2.4660, 1.3920, 3.1738) -- (2.4660, 1.4460, 3.1743) -- (2.4200, 1.4460, 3.1798) -- cycle;
\fill[blue!60.3, opacity=0.7] (2.4200, 1.4460, 3.1798) -- (2.4660, 1.4460, 3.1743) -- (2.4660, 1.5000, 3.1745) -- (2.4200, 1.5000, 3.1800) -- cycle;
\fill[blue!61.9, opacity=0.7] (2.4200, 1.5000, 3.1800) -- (2.4660, 1.5000, 3.1745) -- (2.4660, 1.5540, 3.1743) -- (2.4200, 1.5540, 3.1798) -- cycle;
\fill[blue!63.3, opacity=0.7] (2.4200, 1.5540, 3.1798) -- (2.4660, 1.5540, 3.1743) -- (2.4660, 1.6080, 3.1738) -- (2.4200, 1.6080, 3.1793) -- cycle;
\fill[blue!63.3, opacity=0.7] (2.4200, 1.6080, 3.1793) -- (2.4660, 1.6080, 3.1738) -- (2.4660, 1.6620, 3.1730) -- (2.4200, 1.6620, 3.1785) -- cycle;
\fill[blue!60.5, opacity=0.7] (2.4200, 1.6620, 3.1785) -- (2.4660, 1.6620, 3.1730) -- (2.4660, 1.7160, 3.1719) -- (2.4200, 1.7160, 3.1774) -- cycle;
\fill[blue!54.4, opacity=0.7] (2.4200, 1.7160, 3.1774) -- (2.4660, 1.7160, 3.1719) -- (2.4660, 1.7700, 3.1704) -- (2.4200, 1.7700, 3.1759) -- cycle;
\fill[blue!46.4, opacity=0.7] (2.4200, 1.7700, 3.1759) -- (2.4660, 1.7700, 3.1704) -- (2.4660, 1.8240, 3.1686) -- (2.4200, 1.8240, 3.1741) -- cycle;
\fill[blue!38.9, opacity=0.7] (2.4200, 1.8240, 3.1741) -- (2.4660, 1.8240, 3.1686) -- (2.4660, 1.8780, 3.1665) -- (2.4200, 1.8780, 3.1720) -- cycle;
\fill[blue!34.0, opacity=0.7] (2.4200, 1.8780, 3.1720) -- (2.4660, 1.8780, 3.1665) -- (2.4660, 1.9320, 3.1641) -- (2.4200, 1.9320, 3.1696) -- cycle;
\fill[blue!32.6, opacity=0.7] (2.4200, 1.9320, 3.1696) -- (2.4660, 1.9320, 3.1641) -- (2.4660, 1.9860, 3.1614) -- (2.4200, 1.9860, 3.1669) -- cycle;
\fill[blue!35.3, opacity=0.7] (2.4200, 1.9860, 3.1669) -- (2.4660, 1.9860, 3.1614) -- (2.4660, 2.0400, 3.1584) -- (2.4200, 2.0400, 3.1639) -- cycle;
\fill[blue!43.0, opacity=0.7] (2.4200, 2.0400, 3.1639) -- (2.4660, 2.0400, 3.1584) -- (2.4660, 2.0940, 3.1551) -- (2.4200, 2.0940, 3.1606) -- cycle;
\fill[blue!54.4, opacity=0.7] (2.4200, 2.0940, 3.1606) -- (2.4660, 2.0940, 3.1551) -- (2.4660, 2.1480, 3.1516) -- (2.4200, 2.1480, 3.1571) -- cycle;
\fill[blue!62.8, opacity=0.7] (2.4200, 2.1480, 3.1571) -- (2.4660, 2.1480, 3.1516) -- (2.4660, 2.2020, 3.1477) -- (2.4200, 2.2020, 3.1533) -- cycle;
\fill[blue!62.1, opacity=0.7] (2.4200, 2.2020, 3.1533) -- (2.4660, 2.2020, 3.1477) -- (2.4660, 2.2560, 3.1437) -- (2.4200, 2.2560, 3.1492) -- cycle;
\fill[blue!57.8, opacity=0.7] (2.4200, 2.2560, 3.1492) -- (2.4660, 2.2560, 3.1437) -- (2.4660, 2.3100, 3.1393) -- (2.4200, 2.3100, 3.1449) -- cycle;
\fill[blue!58.0, opacity=0.7] (2.4200, 2.3100, 3.1449) -- (2.4660, 2.3100, 3.1393) -- (2.4660, 2.3640, 3.1348) -- (2.4200, 2.3640, 3.1403) -- cycle;
\fill[blue!62.8, opacity=0.7] (2.4200, 2.3640, 3.1403) -- (2.4660, 2.3640, 3.1348) -- (2.4660, 2.4180, 3.1300) -- (2.4200, 2.4180, 3.1355) -- cycle;
\fill[blue!59.8, opacity=0.7] (2.4200, 2.4180, 3.1355) -- (2.4660, 2.4180, 3.1300) -- (2.4660, 2.4720, 3.1250) -- (2.4200, 2.4720, 3.1305) -- cycle;
\fill[blue!41.0, opacity=0.7] (2.4200, 2.4720, 3.1305) -- (2.4660, 2.4720, 3.1250) -- (2.4660, 2.5260, 3.1198) -- (2.4200, 2.5260, 3.1254) -- cycle;
\fill[blue!25.0, opacity=0.7] (2.4200, 2.5260, 3.1254) -- (2.4660, 2.5260, 3.1198) -- (2.4660, 2.5800, 3.1145) -- (2.4200, 2.5800, 3.1200) -- cycle;
\fill[blue!19.3, opacity=0.7] (2.4200, 2.5800, 3.1200) -- (2.4660, 2.5800, 3.1145) -- (2.4660, 2.6340, 3.1090) -- (2.4200, 2.6340, 3.1145) -- cycle;
\fill[blue!18.8, opacity=0.7] (2.4200, 2.6340, 3.1145) -- (2.4660, 2.6340, 3.1090) -- (2.4660, 2.6880, 3.1033) -- (2.4200, 2.6880, 3.1088) -- cycle;
\fill[blue!22.5, opacity=0.7] (2.4200, 2.6880, 3.1088) -- (2.4660, 2.6880, 3.1033) -- (2.4660, 2.7420, 3.0975) -- (2.4200, 2.7420, 3.1030) -- cycle;
\fill[blue!34.2, opacity=0.7] (2.4200, 2.7420, 3.1030) -- (2.4660, 2.7420, 3.0975) -- (2.4660, 2.7960, 3.0916) -- (2.4200, 2.7960, 3.0971) -- cycle;
\fill[blue!51.2, opacity=0.7] (2.4200, 2.7960, 3.0971) -- (2.4660, 2.7960, 3.0916) -- (2.4660, 2.8500, 3.0855) -- (2.4200, 2.8500, 3.0911) -- cycle;
\fill[blue!59.8, opacity=0.7] (2.4200, 2.8500, 3.0911) -- (2.4660, 2.8500, 3.0855) -- (2.4660, 2.9040, 3.0794) -- (2.4200, 2.9040, 3.0849) -- cycle;
\fill[blue!58.9, opacity=0.7] (2.4200, 2.9040, 3.0849) -- (2.4660, 2.9040, 3.0794) -- (2.4660, 2.9580, 3.0733) -- (2.4200, 2.9580, 3.0788) -- cycle;
\fill[blue!46.3, opacity=0.7] (2.4200, 2.9580, 3.0788) -- (2.4660, 2.9580, 3.0733) -- (2.4660, 3.0120, 3.0670) -- (2.4200, 3.0120, 3.0725) -- cycle;
\fill[blue!26.0, opacity=0.7] (2.4200, 3.0120, 3.0725) -- (2.4660, 3.0120, 3.0670) -- (2.4660, 3.0660, 3.0608) -- (2.4200, 3.0660, 3.0663) -- cycle;
\fill[blue!16.9, opacity=0.7] (2.4200, 3.0660, 3.0663) -- (2.4660, 3.0660, 3.0608) -- (2.4660, 3.1200, 3.0545) -- (2.4200, 3.1200, 3.0600) -- cycle;
\fill[blue!30.1, opacity=0.7] (2.4660, -0.1200, 3.0545) -- (2.5120, -0.1200, 3.0488) -- (2.5120, -0.0660, 3.0551) -- (2.4660, -0.0660, 3.0608) -- cycle;
\fill[blue!20.4, opacity=0.7] (2.4660, -0.0660, 3.0608) -- (2.5120, -0.0660, 3.0551) -- (2.5120, -0.0120, 3.0614) -- (2.4660, -0.0120, 3.0670) -- cycle;
\fill[blue!16.8, opacity=0.7] (2.4660, -0.0120, 3.0670) -- (2.5120, -0.0120, 3.0614) -- (2.5120, 0.0420, 3.0676) -- (2.4660, 0.0420, 3.0733) -- cycle;
\fill[blue!15.9, opacity=0.7] (2.4660, 0.0420, 3.0733) -- (2.5120, 0.0420, 3.0676) -- (2.5120, 0.0960, 3.0738) -- (2.4660, 0.0960, 3.0794) -- cycle;
\fill[blue!16.3, opacity=0.7] (2.4660, 0.0960, 3.0794) -- (2.5120, 0.0960, 3.0738) -- (2.5120, 0.1500, 3.0799) -- (2.4660, 0.1500, 3.0855) -- cycle;
\fill[blue!18.5, opacity=0.7] (2.4660, 0.1500, 3.0855) -- (2.5120, 0.1500, 3.0799) -- (2.5120, 0.2040, 3.0859) -- (2.4660, 0.2040, 3.0916) -- cycle;
\fill[blue!27.0, opacity=0.7] (2.4660, 0.2040, 3.0916) -- (2.5120, 0.2040, 3.0859) -- (2.5120, 0.2580, 3.0918) -- (2.4660, 0.2580, 3.0975) -- cycle;
\fill[blue!44.5, opacity=0.7] (2.4660, 0.2580, 3.0975) -- (2.5120, 0.2580, 3.0918) -- (2.5120, 0.3120, 3.0976) -- (2.4660, 0.3120, 3.1033) -- cycle;
\fill[blue!59.5, opacity=0.7] (2.4660, 0.3120, 3.1033) -- (2.5120, 0.3120, 3.0976) -- (2.5120, 0.3660, 3.1033) -- (2.4660, 0.3660, 3.1090) -- cycle;
\fill[blue!63.5, opacity=0.7] (2.4660, 0.3660, 3.1090) -- (2.5120, 0.3660, 3.1033) -- (2.5120, 0.4200, 3.1088) -- (2.4660, 0.4200, 3.1145) -- cycle;
\fill[blue!63.3, opacity=0.7] (2.4660, 0.4200, 3.1145) -- (2.5120, 0.4200, 3.1088) -- (2.5120, 0.4740, 3.1142) -- (2.4660, 0.4740, 3.1198) -- cycle;
\fill[blue!63.6, opacity=0.7] (2.4660, 0.4740, 3.1198) -- (2.5120, 0.4740, 3.1142) -- (2.5120, 0.5280, 3.1193) -- (2.4660, 0.5280, 3.1250) -- cycle;
\fill[blue!61.8, opacity=0.7] (2.4660, 0.5280, 3.1250) -- (2.5120, 0.5280, 3.1193) -- (2.5120, 0.5820, 3.1243) -- (2.4660, 0.5820, 3.1300) -- cycle;
\fill[blue!54.6, opacity=0.7] (2.4660, 0.5820, 3.1300) -- (2.5120, 0.5820, 3.1243) -- (2.5120, 0.6360, 3.1291) -- (2.4660, 0.6360, 3.1348) -- cycle;
\fill[blue!43.8, opacity=0.7] (2.4660, 0.6360, 3.1348) -- (2.5120, 0.6360, 3.1291) -- (2.5120, 0.6900, 3.1337) -- (2.4660, 0.6900, 3.1393) -- cycle;
\fill[blue!34.5, opacity=0.7] (2.4660, 0.6900, 3.1393) -- (2.5120, 0.6900, 3.1337) -- (2.5120, 0.7440, 3.1380) -- (2.4660, 0.7440, 3.1437) -- cycle;
\fill[blue!28.9, opacity=0.7] (2.4660, 0.7440, 3.1437) -- (2.5120, 0.7440, 3.1380) -- (2.5120, 0.7980, 3.1421) -- (2.4660, 0.7980, 3.1477) -- cycle;
\fill[blue!26.6, opacity=0.7] (2.4660, 0.7980, 3.1477) -- (2.5120, 0.7980, 3.1421) -- (2.5120, 0.8520, 3.1459) -- (2.4660, 0.8520, 3.1516) -- cycle;
\fill[blue!26.6, opacity=0.7] (2.4660, 0.8520, 3.1516) -- (2.5120, 0.8520, 3.1459) -- (2.5120, 0.9060, 3.1494) -- (2.4660, 0.9060, 3.1551) -- cycle;
\fill[blue!28.4, opacity=0.7] (2.4660, 0.9060, 3.1551) -- (2.5120, 0.9060, 3.1494) -- (2.5120, 0.9600, 3.1527) -- (2.4660, 0.9600, 3.1584) -- cycle;
\fill[blue!31.8, opacity=0.7] (2.4660, 0.9600, 3.1584) -- (2.5120, 0.9600, 3.1527) -- (2.5120, 1.0140, 3.1557) -- (2.4660, 1.0140, 3.1614) -- cycle;
\fill[blue!36.6, opacity=0.7] (2.4660, 1.0140, 3.1614) -- (2.5120, 1.0140, 3.1557) -- (2.5120, 1.0680, 3.1584) -- (2.4660, 1.0680, 3.1641) -- cycle;
\fill[blue!42.0, opacity=0.7] (2.4660, 1.0680, 3.1641) -- (2.5120, 1.0680, 3.1584) -- (2.5120, 1.1220, 3.1608) -- (2.4660, 1.1220, 3.1665) -- cycle;
\fill[blue!47.2, opacity=0.7] (2.4660, 1.1220, 3.1665) -- (2.5120, 1.1220, 3.1608) -- (2.5120, 1.1760, 3.1629) -- (2.4660, 1.1760, 3.1686) -- cycle;
\fill[blue!51.5, opacity=0.7] (2.4660, 1.1760, 3.1686) -- (2.5120, 1.1760, 3.1629) -- (2.5120, 1.2300, 3.1647) -- (2.4660, 1.2300, 3.1704) -- cycle;
\fill[blue!54.6, opacity=0.7] (2.4660, 1.2300, 3.1704) -- (2.5120, 1.2300, 3.1647) -- (2.5120, 1.2840, 3.1662) -- (2.4660, 1.2840, 3.1719) -- cycle;
\fill[blue!56.4, opacity=0.7] (2.4660, 1.2840, 3.1719) -- (2.5120, 1.2840, 3.1662) -- (2.5120, 1.3380, 3.1673) -- (2.4660, 1.3380, 3.1730) -- cycle;
\fill[blue!57.1, opacity=0.7] (2.4660, 1.3380, 3.1730) -- (2.5120, 1.3380, 3.1673) -- (2.5120, 1.3920, 3.1682) -- (2.4660, 1.3920, 3.1738) -- cycle;
\fill[blue!56.7, opacity=0.7] (2.4660, 1.3920, 3.1738) -- (2.5120, 1.3920, 3.1682) -- (2.5120, 1.4460, 3.1686) -- (2.4660, 1.4460, 3.1743) -- cycle;
\fill[blue!55.4, opacity=0.7] (2.4660, 1.4460, 3.1743) -- (2.5120, 1.4460, 3.1686) -- (2.5120, 1.5000, 3.1688) -- (2.4660, 1.5000, 3.1745) -- cycle;
\fill[blue!52.8, opacity=0.7] (2.4660, 1.5000, 3.1745) -- (2.5120, 1.5000, 3.1688) -- (2.5120, 1.5540, 3.1686) -- (2.4660, 1.5540, 3.1743) -- cycle;
\fill[blue!49.0, opacity=0.7] (2.4660, 1.5540, 3.1743) -- (2.5120, 1.5540, 3.1686) -- (2.5120, 1.6080, 3.1682) -- (2.4660, 1.6080, 3.1738) -- cycle;
\fill[blue!44.4, opacity=0.7] (2.4660, 1.6080, 3.1738) -- (2.5120, 1.6080, 3.1682) -- (2.5120, 1.6620, 3.1673) -- (2.4660, 1.6620, 3.1730) -- cycle;
\fill[blue!39.5, opacity=0.7] (2.4660, 1.6620, 3.1730) -- (2.5120, 1.6620, 3.1673) -- (2.5120, 1.7160, 3.1662) -- (2.4660, 1.7160, 3.1719) -- cycle;
\fill[blue!35.3, opacity=0.7] (2.4660, 1.7160, 3.1719) -- (2.5120, 1.7160, 3.1662) -- (2.5120, 1.7700, 3.1647) -- (2.4660, 1.7700, 3.1704) -- cycle;
\fill[blue!32.6, opacity=0.7] (2.4660, 1.7700, 3.1704) -- (2.5120, 1.7700, 3.1647) -- (2.5120, 1.8240, 3.1629) -- (2.4660, 1.8240, 3.1686) -- cycle;
\fill[blue!31.9, opacity=0.7] (2.4660, 1.8240, 3.1686) -- (2.5120, 1.8240, 3.1629) -- (2.5120, 1.8780, 3.1608) -- (2.4660, 1.8780, 3.1665) -- cycle;
\fill[blue!33.8, opacity=0.7] (2.4660, 1.8780, 3.1665) -- (2.5120, 1.8780, 3.1608) -- (2.5120, 1.9320, 3.1584) -- (2.4660, 1.9320, 3.1641) -- cycle;
\fill[blue!39.3, opacity=0.7] (2.4660, 1.9320, 3.1641) -- (2.5120, 1.9320, 3.1584) -- (2.5120, 1.9860, 3.1557) -- (2.4660, 1.9860, 3.1614) -- cycle;
\fill[blue!48.6, opacity=0.7] (2.4660, 1.9860, 3.1614) -- (2.5120, 1.9860, 3.1557) -- (2.5120, 2.0400, 3.1527) -- (2.4660, 2.0400, 3.1584) -- cycle;
\fill[blue!58.7, opacity=0.7] (2.4660, 2.0400, 3.1584) -- (2.5120, 2.0400, 3.1527) -- (2.5120, 2.0940, 3.1494) -- (2.4660, 2.0940, 3.1551) -- cycle;
\fill[blue!63.5, opacity=0.7] (2.4660, 2.0940, 3.1551) -- (2.5120, 2.0940, 3.1494) -- (2.5120, 2.1480, 3.1459) -- (2.4660, 2.1480, 3.1516) -- cycle;
\fill[blue!61.2, opacity=0.7] (2.4660, 2.1480, 3.1516) -- (2.5120, 2.1480, 3.1459) -- (2.5120, 2.2020, 3.1421) -- (2.4660, 2.2020, 3.1477) -- cycle;
\fill[blue!57.8, opacity=0.7] (2.4660, 2.2020, 3.1477) -- (2.5120, 2.2020, 3.1421) -- (2.5120, 2.2560, 3.1380) -- (2.4660, 2.2560, 3.1437) -- cycle;
\fill[blue!58.9, opacity=0.7] (2.4660, 2.2560, 3.1437) -- (2.5120, 2.2560, 3.1380) -- (2.5120, 2.3100, 3.1337) -- (2.4660, 2.3100, 3.1393) -- cycle;
\fill[blue!63.2, opacity=0.7] (2.4660, 2.3100, 3.1393) -- (2.5120, 2.3100, 3.1337) -- (2.5120, 2.3640, 3.1291) -- (2.4660, 2.3640, 3.1348) -- cycle;
\fill[blue!59.3, opacity=0.7] (2.4660, 2.3640, 3.1348) -- (2.5120, 2.3640, 3.1291) -- (2.5120, 2.4180, 3.1243) -- (2.4660, 2.4180, 3.1300) -- cycle;
\fill[blue!41.4, opacity=0.7] (2.4660, 2.4180, 3.1300) -- (2.5120, 2.4180, 3.1243) -- (2.5120, 2.4720, 3.1193) -- (2.4660, 2.4720, 3.1250) -- cycle;
\fill[blue!25.6, opacity=0.7] (2.4660, 2.4720, 3.1250) -- (2.5120, 2.4720, 3.1193) -- (2.5120, 2.5260, 3.1142) -- (2.4660, 2.5260, 3.1198) -- cycle;
\fill[blue!19.5, opacity=0.7] (2.4660, 2.5260, 3.1198) -- (2.5120, 2.5260, 3.1142) -- (2.5120, 2.5800, 3.1088) -- (2.4660, 2.5800, 3.1145) -- cycle;
\fill[blue!18.5, opacity=0.7] (2.4660, 2.5800, 3.1145) -- (2.5120, 2.5800, 3.1088) -- (2.5120, 2.6340, 3.1033) -- (2.4660, 2.6340, 3.1090) -- cycle;
\fill[blue!21.1, opacity=0.7] (2.4660, 2.6340, 3.1090) -- (2.5120, 2.6340, 3.1033) -- (2.5120, 2.6880, 3.0976) -- (2.4660, 2.6880, 3.1033) -- cycle;
\fill[blue!30.4, opacity=0.7] (2.4660, 2.6880, 3.1033) -- (2.5120, 2.6880, 3.0976) -- (2.5120, 2.7420, 3.0918) -- (2.4660, 2.7420, 3.0975) -- cycle;
\fill[blue!46.8, opacity=0.7] (2.4660, 2.7420, 3.0975) -- (2.5120, 2.7420, 3.0918) -- (2.5120, 2.7960, 3.0859) -- (2.4660, 2.7960, 3.0916) -- cycle;
\fill[blue!58.2, opacity=0.7] (2.4660, 2.7960, 3.0916) -- (2.5120, 2.7960, 3.0859) -- (2.5120, 2.8500, 3.0799) -- (2.4660, 2.8500, 3.0855) -- cycle;
\fill[blue!59.9, opacity=0.7] (2.4660, 2.8500, 3.0855) -- (2.5120, 2.8500, 3.0799) -- (2.5120, 2.9040, 3.0738) -- (2.4660, 2.9040, 3.0794) -- cycle;
\fill[blue!51.9, opacity=0.7] (2.4660, 2.9040, 3.0794) -- (2.5120, 2.9040, 3.0738) -- (2.5120, 2.9580, 3.0676) -- (2.4660, 2.9580, 3.0733) -- cycle;
\fill[blue!32.4, opacity=0.7] (2.4660, 2.9580, 3.0733) -- (2.5120, 2.9580, 3.0676) -- (2.5120, 3.0120, 3.0614) -- (2.4660, 3.0120, 3.0670) -- cycle;
\fill[blue!18.8, opacity=0.7] (2.4660, 3.0120, 3.0670) -- (2.5120, 3.0120, 3.0614) -- (2.5120, 3.0660, 3.0551) -- (2.4660, 3.0660, 3.0608) -- cycle;
\fill[blue!15.7, opacity=0.7] (2.4660, 3.0660, 3.0608) -- (2.5120, 3.0660, 3.0551) -- (2.5120, 3.1200, 3.0488) -- (2.4660, 3.1200, 3.0545) -- cycle;
\fill[blue!40.8, opacity=0.7] (2.5120, -0.1200, 3.0488) -- (2.5580, -0.1200, 3.0430) -- (2.5580, -0.0660, 3.0493) -- (2.5120, -0.0660, 3.0551) -- cycle;
\fill[blue!29.0, opacity=0.7] (2.5120, -0.0660, 3.0551) -- (2.5580, -0.0660, 3.0493) -- (2.5580, -0.0120, 3.0555) -- (2.5120, -0.0120, 3.0614) -- cycle;
\fill[blue!20.1, opacity=0.7] (2.5120, -0.0120, 3.0614) -- (2.5580, -0.0120, 3.0555) -- (2.5580, 0.0420, 3.0618) -- (2.5120, 0.0420, 3.0676) -- cycle;
\fill[blue!16.7, opacity=0.7] (2.5120, 0.0420, 3.0676) -- (2.5580, 0.0420, 3.0618) -- (2.5580, 0.0960, 3.0680) -- (2.5120, 0.0960, 3.0738) -- cycle;
\fill[blue!16.0, opacity=0.7] (2.5120, 0.0960, 3.0738) -- (2.5580, 0.0960, 3.0680) -- (2.5580, 0.1500, 3.0741) -- (2.5120, 0.1500, 3.0799) -- cycle;
\fill[blue!16.3, opacity=0.7] (2.5120, 0.1500, 3.0799) -- (2.5580, 0.1500, 3.0741) -- (2.5580, 0.2040, 3.0801) -- (2.5120, 0.2040, 3.0859) -- cycle;
\fill[blue!18.3, opacity=0.7] (2.5120, 0.2040, 3.0859) -- (2.5580, 0.2040, 3.0801) -- (2.5580, 0.2580, 3.0860) -- (2.5120, 0.2580, 3.0918) -- cycle;
\fill[blue!25.6, opacity=0.7] (2.5120, 0.2580, 3.0918) -- (2.5580, 0.2580, 3.0860) -- (2.5580, 0.3120, 3.0918) -- (2.5120, 0.3120, 3.0976) -- cycle;
\fill[blue!41.4, opacity=0.7] (2.5120, 0.3120, 3.0976) -- (2.5580, 0.3120, 3.0918) -- (2.5580, 0.3660, 3.0975) -- (2.5120, 0.3660, 3.1033) -- cycle;
\fill[blue!57.3, opacity=0.7] (2.5120, 0.3660, 3.1033) -- (2.5580, 0.3660, 3.0975) -- (2.5580, 0.4200, 3.1030) -- (2.5120, 0.4200, 3.1088) -- cycle;
\fill[blue!63.3, opacity=0.7] (2.5120, 0.4200, 3.1088) -- (2.5580, 0.4200, 3.1030) -- (2.5580, 0.4740, 3.1084) -- (2.5120, 0.4740, 3.1142) -- cycle;
\fill[blue!63.3, opacity=0.7] (2.5120, 0.4740, 3.1142) -- (2.5580, 0.4740, 3.1084) -- (2.5580, 0.5280, 3.1135) -- (2.5120, 0.5280, 3.1193) -- cycle;
\fill[blue!63.3, opacity=0.7] (2.5120, 0.5280, 3.1193) -- (2.5580, 0.5280, 3.1135) -- (2.5580, 0.5820, 3.1185) -- (2.5120, 0.5820, 3.1243) -- cycle;
\fill[blue!63.4, opacity=0.7] (2.5120, 0.5820, 3.1243) -- (2.5580, 0.5820, 3.1185) -- (2.5580, 0.6360, 3.1233) -- (2.5120, 0.6360, 3.1291) -- cycle;
\fill[blue!60.4, opacity=0.7] (2.5120, 0.6360, 3.1291) -- (2.5580, 0.6360, 3.1233) -- (2.5580, 0.6900, 3.1279) -- (2.5120, 0.6900, 3.1337) -- cycle;
\fill[blue!53.0, opacity=0.7] (2.5120, 0.6900, 3.1337) -- (2.5580, 0.6900, 3.1279) -- (2.5580, 0.7440, 3.1322) -- (2.5120, 0.7440, 3.1380) -- cycle;
\fill[blue!43.6, opacity=0.7] (2.5120, 0.7440, 3.1380) -- (2.5580, 0.7440, 3.1322) -- (2.5580, 0.7980, 3.1363) -- (2.5120, 0.7980, 3.1421) -- cycle;
\fill[blue!35.8, opacity=0.7] (2.5120, 0.7980, 3.1421) -- (2.5580, 0.7980, 3.1363) -- (2.5580, 0.8520, 3.1401) -- (2.5120, 0.8520, 3.1459) -- cycle;
\fill[blue!30.7, opacity=0.7] (2.5120, 0.8520, 3.1459) -- (2.5580, 0.8520, 3.1401) -- (2.5580, 0.9060, 3.1436) -- (2.5120, 0.9060, 3.1494) -- cycle;
\fill[blue!28.1, opacity=0.7] (2.5120, 0.9060, 3.1494) -- (2.5580, 0.9060, 3.1436) -- (2.5580, 0.9600, 3.1469) -- (2.5120, 0.9600, 3.1527) -- cycle;
\fill[blue!27.1, opacity=0.7] (2.5120, 0.9600, 3.1527) -- (2.5580, 0.9600, 3.1469) -- (2.5580, 1.0140, 3.1499) -- (2.5120, 1.0140, 3.1557) -- cycle;
\fill[blue!27.3, opacity=0.7] (2.5120, 1.0140, 3.1557) -- (2.5580, 1.0140, 3.1499) -- (2.5580, 1.0680, 3.1526) -- (2.5120, 1.0680, 3.1584) -- cycle;
\fill[blue!28.4, opacity=0.7] (2.5120, 1.0680, 3.1584) -- (2.5580, 1.0680, 3.1526) -- (2.5580, 1.1220, 3.1550) -- (2.5120, 1.1220, 3.1608) -- cycle;
\fill[blue!29.9, opacity=0.7] (2.5120, 1.1220, 3.1608) -- (2.5580, 1.1220, 3.1550) -- (2.5580, 1.1760, 3.1571) -- (2.5120, 1.1760, 3.1629) -- cycle;
\fill[blue!31.6, opacity=0.7] (2.5120, 1.1760, 3.1629) -- (2.5580, 1.1760, 3.1571) -- (2.5580, 1.2300, 3.1589) -- (2.5120, 1.2300, 3.1647) -- cycle;
\fill[blue!33.2, opacity=0.7] (2.5120, 1.2300, 3.1647) -- (2.5580, 1.2300, 3.1589) -- (2.5580, 1.2840, 3.1604) -- (2.5120, 1.2840, 3.1662) -- cycle;
\fill[blue!34.4, opacity=0.7] (2.5120, 1.2840, 3.1662) -- (2.5580, 1.2840, 3.1604) -- (2.5580, 1.3380, 3.1615) -- (2.5120, 1.3380, 3.1673) -- cycle;
\fill[blue!35.0, opacity=0.7] (2.5120, 1.3380, 3.1673) -- (2.5580, 1.3380, 3.1615) -- (2.5580, 1.3920, 3.1623) -- (2.5120, 1.3920, 3.1682) -- cycle;
\fill[blue!34.9, opacity=0.7] (2.5120, 1.3920, 3.1682) -- (2.5580, 1.3920, 3.1623) -- (2.5580, 1.4460, 3.1628) -- (2.5120, 1.4460, 3.1686) -- cycle;
\fill[blue!34.3, opacity=0.7] (2.5120, 1.4460, 3.1686) -- (2.5580, 1.4460, 3.1628) -- (2.5580, 1.5000, 3.1630) -- (2.5120, 1.5000, 3.1688) -- cycle;
\fill[blue!33.2, opacity=0.7] (2.5120, 1.5000, 3.1688) -- (2.5580, 1.5000, 3.1630) -- (2.5580, 1.5540, 3.1628) -- (2.5120, 1.5540, 3.1686) -- cycle;
\fill[blue!32.0, opacity=0.7] (2.5120, 1.5540, 3.1686) -- (2.5580, 1.5540, 3.1628) -- (2.5580, 1.6080, 3.1623) -- (2.5120, 1.6080, 3.1682) -- cycle;
\fill[blue!31.1, opacity=0.7] (2.5120, 1.6080, 3.1682) -- (2.5580, 1.6080, 3.1623) -- (2.5580, 1.6620, 3.1615) -- (2.5120, 1.6620, 3.1673) -- cycle;
\fill[blue!30.8, opacity=0.7] (2.5120, 1.6620, 3.1673) -- (2.5580, 1.6620, 3.1615) -- (2.5580, 1.7160, 3.1604) -- (2.5120, 1.7160, 3.1662) -- cycle;
\fill[blue!31.6, opacity=0.7] (2.5120, 1.7160, 3.1662) -- (2.5580, 1.7160, 3.1604) -- (2.5580, 1.7700, 3.1589) -- (2.5120, 1.7700, 3.1647) -- cycle;
\fill[blue!34.2, opacity=0.7] (2.5120, 1.7700, 3.1647) -- (2.5580, 1.7700, 3.1589) -- (2.5580, 1.8240, 3.1571) -- (2.5120, 1.8240, 3.1629) -- cycle;
\fill[blue!39.2, opacity=0.7] (2.5120, 1.8240, 3.1629) -- (2.5580, 1.8240, 3.1571) -- (2.5580, 1.8780, 3.1550) -- (2.5120, 1.8780, 3.1608) -- cycle;
\fill[blue!46.9, opacity=0.7] (2.5120, 1.8780, 3.1608) -- (2.5580, 1.8780, 3.1550) -- (2.5580, 1.9320, 3.1526) -- (2.5120, 1.9320, 3.1584) -- cycle;
\fill[blue!56.0, opacity=0.7] (2.5120, 1.9320, 3.1584) -- (2.5580, 1.9320, 3.1526) -- (2.5580, 1.9860, 3.1499) -- (2.5120, 1.9860, 3.1557) -- cycle;
\fill[blue!62.5, opacity=0.7] (2.5120, 1.9860, 3.1557) -- (2.5580, 1.9860, 3.1499) -- (2.5580, 2.0400, 3.1469) -- (2.5120, 2.0400, 3.1527) -- cycle;
\fill[blue!63.1, opacity=0.7] (2.5120, 2.0400, 3.1527) -- (2.5580, 2.0400, 3.1469) -- (2.5580, 2.0940, 3.1436) -- (2.5120, 2.0940, 3.1494) -- cycle;
\fill[blue!59.8, opacity=0.7] (2.5120, 2.0940, 3.1494) -- (2.5580, 2.0940, 3.1436) -- (2.5580, 2.1480, 3.1401) -- (2.5120, 2.1480, 3.1459) -- cycle;
\fill[blue!57.9, opacity=0.7] (2.5120, 2.1480, 3.1459) -- (2.5580, 2.1480, 3.1401) -- (2.5580, 2.2020, 3.1363) -- (2.5120, 2.2020, 3.1421) -- cycle;
\fill[blue!60.3, opacity=0.7] (2.5120, 2.2020, 3.1421) -- (2.5580, 2.2020, 3.1363) -- (2.5580, 2.2560, 3.1322) -- (2.5120, 2.2560, 3.1380) -- cycle;
\fill[blue!63.6, opacity=0.7] (2.5120, 2.2560, 3.1380) -- (2.5580, 2.2560, 3.1322) -- (2.5580, 2.3100, 3.1279) -- (2.5120, 2.3100, 3.1337) -- cycle;
\fill[blue!57.4, opacity=0.7] (2.5120, 2.3100, 3.1337) -- (2.5580, 2.3100, 3.1279) -- (2.5580, 2.3640, 3.1233) -- (2.5120, 2.3640, 3.1291) -- cycle;
\fill[blue!39.6, opacity=0.7] (2.5120, 2.3640, 3.1291) -- (2.5580, 2.3640, 3.1233) -- (2.5580, 2.4180, 3.1185) -- (2.5120, 2.4180, 3.1243) -- cycle;
\fill[blue!25.1, opacity=0.7] (2.5120, 2.4180, 3.1243) -- (2.5580, 2.4180, 3.1185) -- (2.5580, 2.4720, 3.1135) -- (2.5120, 2.4720, 3.1193) -- cycle;
\fill[blue!19.4, opacity=0.7] (2.5120, 2.4720, 3.1193) -- (2.5580, 2.4720, 3.1135) -- (2.5580, 2.5260, 3.1084) -- (2.5120, 2.5260, 3.1142) -- cycle;
\fill[blue!18.4, opacity=0.7] (2.5120, 2.5260, 3.1142) -- (2.5580, 2.5260, 3.1084) -- (2.5580, 2.5800, 3.1030) -- (2.5120, 2.5800, 3.1088) -- cycle;
\fill[blue!20.4, opacity=0.7] (2.5120, 2.5800, 3.1088) -- (2.5580, 2.5800, 3.1030) -- (2.5580, 2.6340, 3.0975) -- (2.5120, 2.6340, 3.1033) -- cycle;
\fill[blue!28.0, opacity=0.7] (2.5120, 2.6340, 3.1033) -- (2.5580, 2.6340, 3.0975) -- (2.5580, 2.6880, 3.0918) -- (2.5120, 2.6880, 3.0976) -- cycle;
\fill[blue!43.2, opacity=0.7] (2.5120, 2.6880, 3.0976) -- (2.5580, 2.6880, 3.0918) -- (2.5580, 2.7420, 3.0860) -- (2.5120, 2.7420, 3.0918) -- cycle;
\fill[blue!56.4, opacity=0.7] (2.5120, 2.7420, 3.0918) -- (2.5580, 2.7420, 3.0860) -- (2.5580, 2.7960, 3.0801) -- (2.5120, 2.7960, 3.0859) -- cycle;
\fill[blue!60.0, opacity=0.7] (2.5120, 2.7960, 3.0859) -- (2.5580, 2.7960, 3.0801) -- (2.5580, 2.8500, 3.0741) -- (2.5120, 2.8500, 3.0799) -- cycle;
\fill[blue!55.2, opacity=0.7] (2.5120, 2.8500, 3.0799) -- (2.5580, 2.8500, 3.0741) -- (2.5580, 2.9040, 3.0680) -- (2.5120, 2.9040, 3.0738) -- cycle;
\fill[blue!38.4, opacity=0.7] (2.5120, 2.9040, 3.0738) -- (2.5580, 2.9040, 3.0680) -- (2.5580, 2.9580, 3.0618) -- (2.5120, 2.9580, 3.0676) -- cycle;
\fill[blue!21.6, opacity=0.7] (2.5120, 2.9580, 3.0676) -- (2.5580, 2.9580, 3.0618) -- (2.5580, 3.0120, 3.0555) -- (2.5120, 3.0120, 3.0614) -- cycle;
\fill[blue!16.1, opacity=0.7] (2.5120, 3.0120, 3.0614) -- (2.5580, 3.0120, 3.0555) -- (2.5580, 3.0660, 3.0493) -- (2.5120, 3.0660, 3.0551) -- cycle;
\fill[blue!15.3, opacity=0.7] (2.5120, 3.0660, 3.0551) -- (2.5580, 3.0660, 3.0493) -- (2.5580, 3.1200, 3.0430) -- (2.5120, 3.1200, 3.0488) -- cycle;
\fill[blue!46.3, opacity=0.7] (2.5580, -0.1200, 3.0430) -- (2.6040, -0.1200, 3.0371) -- (2.6040, -0.0660, 3.0434) -- (2.5580, -0.0660, 3.0493) -- cycle;
\fill[blue!40.3, opacity=0.7] (2.5580, -0.0660, 3.0493) -- (2.6040, -0.0660, 3.0434) -- (2.6040, -0.0120, 3.0496) -- (2.5580, -0.0120, 3.0555) -- cycle;
\fill[blue!28.7, opacity=0.7] (2.5580, -0.0120, 3.0555) -- (2.6040, -0.0120, 3.0496) -- (2.6040, 0.0420, 3.0559) -- (2.5580, 0.0420, 3.0618) -- cycle;
\fill[blue!20.1, opacity=0.7] (2.5580, 0.0420, 3.0618) -- (2.6040, 0.0420, 3.0559) -- (2.6040, 0.0960, 3.0620) -- (2.5580, 0.0960, 3.0680) -- cycle;
\fill[blue!16.9, opacity=0.7] (2.5580, 0.0960, 3.0680) -- (2.6040, 0.0960, 3.0620) -- (2.6040, 0.1500, 3.0681) -- (2.5580, 0.1500, 3.0741) -- cycle;
\fill[blue!16.0, opacity=0.7] (2.5580, 0.1500, 3.0741) -- (2.6040, 0.1500, 3.0681) -- (2.6040, 0.2040, 3.0742) -- (2.5580, 0.2040, 3.0801) -- cycle;
\fill[blue!16.2, opacity=0.7] (2.5580, 0.2040, 3.0801) -- (2.6040, 0.2040, 3.0742) -- (2.6040, 0.2580, 3.0801) -- (2.5580, 0.2580, 3.0860) -- cycle;
\fill[blue!17.8, opacity=0.7] (2.5580, 0.2580, 3.0860) -- (2.6040, 0.2580, 3.0801) -- (2.6040, 0.3120, 3.0859) -- (2.5580, 0.3120, 3.0918) -- cycle;
\fill[blue!23.4, opacity=0.7] (2.5580, 0.3120, 3.0918) -- (2.6040, 0.3120, 3.0859) -- (2.6040, 0.3660, 3.0916) -- (2.5580, 0.3660, 3.0975) -- cycle;
\fill[blue!36.6, opacity=0.7] (2.5580, 0.3660, 3.0975) -- (2.6040, 0.3660, 3.0916) -- (2.6040, 0.4200, 3.0971) -- (2.5580, 0.4200, 3.1030) -- cycle;
\fill[blue!52.9, opacity=0.7] (2.5580, 0.4200, 3.1030) -- (2.6040, 0.4200, 3.0971) -- (2.6040, 0.4740, 3.1024) -- (2.5580, 0.4740, 3.1084) -- cycle;
\fill[blue!62.1, opacity=0.7] (2.5580, 0.4740, 3.1084) -- (2.6040, 0.4740, 3.1024) -- (2.6040, 0.5280, 3.1076) -- (2.5580, 0.5280, 3.1135) -- cycle;
\fill[blue!63.5, opacity=0.7] (2.5580, 0.5280, 3.1135) -- (2.6040, 0.5280, 3.1076) -- (2.6040, 0.5820, 3.1126) -- (2.5580, 0.5820, 3.1185) -- cycle;
\fill[blue!63.1, opacity=0.7] (2.5580, 0.5820, 3.1185) -- (2.6040, 0.5820, 3.1126) -- (2.6040, 0.6360, 3.1174) -- (2.5580, 0.6360, 3.1233) -- cycle;
\fill[blue!63.4, opacity=0.7] (2.5580, 0.6360, 3.1233) -- (2.6040, 0.6360, 3.1174) -- (2.6040, 0.6900, 3.1219) -- (2.5580, 0.6900, 3.1279) -- cycle;
\fill[blue!63.4, opacity=0.7] (2.5580, 0.6900, 3.1279) -- (2.6040, 0.6900, 3.1219) -- (2.6040, 0.7440, 3.1263) -- (2.5580, 0.7440, 3.1322) -- cycle;
\fill[blue!60.8, opacity=0.7] (2.5580, 0.7440, 3.1322) -- (2.6040, 0.7440, 3.1263) -- (2.6040, 0.7980, 3.1303) -- (2.5580, 0.7980, 3.1363) -- cycle;
\fill[blue!55.1, opacity=0.7] (2.5580, 0.7980, 3.1363) -- (2.6040, 0.7980, 3.1303) -- (2.6040, 0.8520, 3.1342) -- (2.5580, 0.8520, 3.1401) -- cycle;
\fill[blue!47.7, opacity=0.7] (2.5580, 0.8520, 3.1401) -- (2.6040, 0.8520, 3.1342) -- (2.6040, 0.9060, 3.1377) -- (2.5580, 0.9060, 3.1436) -- cycle;
\fill[blue!41.0, opacity=0.7] (2.5580, 0.9060, 3.1436) -- (2.6040, 0.9060, 3.1377) -- (2.6040, 0.9600, 3.1410) -- (2.5580, 0.9600, 3.1469) -- cycle;
\fill[blue!35.8, opacity=0.7] (2.5580, 0.9600, 3.1469) -- (2.6040, 0.9600, 3.1410) -- (2.6040, 1.0140, 3.1440) -- (2.5580, 1.0140, 3.1499) -- cycle;
\fill[blue!32.4, opacity=0.7] (2.5580, 1.0140, 3.1499) -- (2.6040, 1.0140, 3.1440) -- (2.6040, 1.0680, 3.1467) -- (2.5580, 1.0680, 3.1526) -- cycle;
\fill[blue!30.3, opacity=0.7] (2.5580, 1.0680, 3.1526) -- (2.6040, 1.0680, 3.1467) -- (2.6040, 1.1220, 3.1491) -- (2.5580, 1.1220, 3.1550) -- cycle;
\fill[blue!29.2, opacity=0.7] (2.5580, 1.1220, 3.1550) -- (2.6040, 1.1220, 3.1491) -- (2.6040, 1.1760, 3.1512) -- (2.5580, 1.1760, 3.1571) -- cycle;
\fill[blue!28.7, opacity=0.7] (2.5580, 1.1760, 3.1571) -- (2.6040, 1.1760, 3.1512) -- (2.6040, 1.2300, 3.1530) -- (2.5580, 1.2300, 3.1589) -- cycle;
\fill[blue!28.7, opacity=0.7] (2.5580, 1.2300, 3.1589) -- (2.6040, 1.2300, 3.1530) -- (2.6040, 1.2840, 3.1545) -- (2.5580, 1.2840, 3.1604) -- cycle;
\fill[blue!28.8, opacity=0.7] (2.5580, 1.2840, 3.1604) -- (2.6040, 1.2840, 3.1545) -- (2.6040, 1.3380, 3.1556) -- (2.5580, 1.3380, 3.1615) -- cycle;
\fill[blue!29.1, opacity=0.7] (2.5580, 1.3380, 3.1615) -- (2.6040, 1.3380, 3.1556) -- (2.6040, 1.3920, 3.1564) -- (2.5580, 1.3920, 3.1623) -- cycle;
\fill[blue!29.4, opacity=0.7] (2.5580, 1.3920, 3.1623) -- (2.6040, 1.3920, 3.1564) -- (2.6040, 1.4460, 3.1569) -- (2.5580, 1.4460, 3.1628) -- cycle;
\fill[blue!29.9, opacity=0.7] (2.5580, 1.4460, 3.1628) -- (2.6040, 1.4460, 3.1569) -- (2.6040, 1.5000, 3.1571) -- (2.5580, 1.5000, 3.1630) -- cycle;
\fill[blue!30.7, opacity=0.7] (2.5580, 1.5000, 3.1630) -- (2.6040, 1.5000, 3.1571) -- (2.6040, 1.5540, 3.1569) -- (2.5580, 1.5540, 3.1628) -- cycle;
\fill[blue!32.1, opacity=0.7] (2.5580, 1.5540, 3.1628) -- (2.6040, 1.5540, 3.1569) -- (2.6040, 1.6080, 3.1564) -- (2.5580, 1.6080, 3.1623) -- cycle;
\fill[blue!34.2, opacity=0.7] (2.5580, 1.6080, 3.1623) -- (2.6040, 1.6080, 3.1564) -- (2.6040, 1.6620, 3.1556) -- (2.5580, 1.6620, 3.1615) -- cycle;
\fill[blue!37.7, opacity=0.7] (2.5580, 1.6620, 3.1615) -- (2.6040, 1.6620, 3.1556) -- (2.6040, 1.7160, 3.1545) -- (2.5580, 1.7160, 3.1604) -- cycle;
\fill[blue!42.9, opacity=0.7] (2.5580, 1.7160, 3.1604) -- (2.6040, 1.7160, 3.1545) -- (2.6040, 1.7700, 3.1530) -- (2.5580, 1.7700, 3.1589) -- cycle;
\fill[blue!49.6, opacity=0.7] (2.5580, 1.7700, 3.1589) -- (2.6040, 1.7700, 3.1530) -- (2.6040, 1.8240, 3.1512) -- (2.5580, 1.8240, 3.1571) -- cycle;
\fill[blue!56.9, opacity=0.7] (2.5580, 1.8240, 3.1571) -- (2.6040, 1.8240, 3.1512) -- (2.6040, 1.8780, 3.1491) -- (2.5580, 1.8780, 3.1550) -- cycle;
\fill[blue!62.2, opacity=0.7] (2.5580, 1.8780, 3.1550) -- (2.6040, 1.8780, 3.1491) -- (2.6040, 1.9320, 3.1467) -- (2.5580, 1.9320, 3.1526) -- cycle;
\fill[blue!63.5, opacity=0.7] (2.5580, 1.9320, 3.1526) -- (2.6040, 1.9320, 3.1467) -- (2.6040, 1.9860, 3.1440) -- (2.5580, 1.9860, 3.1499) -- cycle;
\fill[blue!61.2, opacity=0.7] (2.5580, 1.9860, 3.1499) -- (2.6040, 1.9860, 3.1440) -- (2.6040, 2.0400, 3.1410) -- (2.5580, 2.0400, 3.1469) -- cycle;
\fill[blue!58.7, opacity=0.7] (2.5580, 2.0400, 3.1469) -- (2.6040, 2.0400, 3.1410) -- (2.6040, 2.0940, 3.1377) -- (2.5580, 2.0940, 3.1436) -- cycle;
\fill[blue!59.0, opacity=0.7] (2.5580, 2.0940, 3.1436) -- (2.6040, 2.0940, 3.1377) -- (2.6040, 2.1480, 3.1342) -- (2.5580, 2.1480, 3.1401) -- cycle;
\fill[blue!62.2, opacity=0.7] (2.5580, 2.1480, 3.1401) -- (2.6040, 2.1480, 3.1342) -- (2.6040, 2.2020, 3.1303) -- (2.5580, 2.2020, 3.1363) -- cycle;
\fill[blue!63.0, opacity=0.7] (2.5580, 2.2020, 3.1363) -- (2.6040, 2.2020, 3.1303) -- (2.6040, 2.2560, 3.1263) -- (2.5580, 2.2560, 3.1322) -- cycle;
\fill[blue!53.2, opacity=0.7] (2.5580, 2.2560, 3.1322) -- (2.6040, 2.2560, 3.1263) -- (2.6040, 2.3100, 3.1219) -- (2.5580, 2.3100, 3.1279) -- cycle;
\fill[blue!36.0, opacity=0.7] (2.5580, 2.3100, 3.1279) -- (2.6040, 2.3100, 3.1219) -- (2.6040, 2.3640, 3.1174) -- (2.5580, 2.3640, 3.1233) -- cycle;
\fill[blue!23.8, opacity=0.7] (2.5580, 2.3640, 3.1233) -- (2.6040, 2.3640, 3.1174) -- (2.6040, 2.4180, 3.1126) -- (2.5580, 2.4180, 3.1185) -- cycle;
\fill[blue!19.1, opacity=0.7] (2.5580, 2.4180, 3.1185) -- (2.6040, 2.4180, 3.1126) -- (2.6040, 2.4720, 3.1076) -- (2.5580, 2.4720, 3.1135) -- cycle;
\fill[blue!18.3, opacity=0.7] (2.5580, 2.4720, 3.1135) -- (2.6040, 2.4720, 3.1076) -- (2.6040, 2.5260, 3.1024) -- (2.5580, 2.5260, 3.1084) -- cycle;
\fill[blue!20.1, opacity=0.7] (2.5580, 2.5260, 3.1084) -- (2.6040, 2.5260, 3.1024) -- (2.6040, 2.5800, 3.0971) -- (2.5580, 2.5800, 3.1030) -- cycle;
\fill[blue!26.9, opacity=0.7] (2.5580, 2.5800, 3.1030) -- (2.6040, 2.5800, 3.0971) -- (2.6040, 2.6340, 3.0916) -- (2.5580, 2.6340, 3.0975) -- cycle;
\fill[blue!41.0, opacity=0.7] (2.5580, 2.6340, 3.0975) -- (2.6040, 2.6340, 3.0916) -- (2.6040, 2.6880, 3.0859) -- (2.5580, 2.6880, 3.0918) -- cycle;
\fill[blue!54.7, opacity=0.7] (2.5580, 2.6880, 3.0918) -- (2.6040, 2.6880, 3.0859) -- (2.6040, 2.7420, 3.0801) -- (2.5580, 2.7420, 3.0860) -- cycle;
\fill[blue!59.7, opacity=0.7] (2.5580, 2.7420, 3.0860) -- (2.6040, 2.7420, 3.0801) -- (2.6040, 2.7960, 3.0742) -- (2.5580, 2.7960, 3.0801) -- cycle;
\fill[blue!56.9, opacity=0.7] (2.5580, 2.7960, 3.0801) -- (2.6040, 2.7960, 3.0742) -- (2.6040, 2.8500, 3.0681) -- (2.5580, 2.8500, 3.0741) -- cycle;
\fill[blue!43.0, opacity=0.7] (2.5580, 2.8500, 3.0741) -- (2.6040, 2.8500, 3.0681) -- (2.6040, 2.9040, 3.0620) -- (2.5580, 2.9040, 3.0680) -- cycle;
\fill[blue!24.7, opacity=0.7] (2.5580, 2.9040, 3.0680) -- (2.6040, 2.9040, 3.0620) -- (2.6040, 2.9580, 3.0559) -- (2.5580, 2.9580, 3.0618) -- cycle;
\fill[blue!16.9, opacity=0.7] (2.5580, 2.9580, 3.0618) -- (2.6040, 2.9580, 3.0559) -- (2.6040, 3.0120, 3.0496) -- (2.5580, 3.0120, 3.0555) -- cycle;
\fill[blue!15.4, opacity=0.7] (2.5580, 3.0120, 3.0555) -- (2.6040, 3.0120, 3.0496) -- (2.6040, 3.0660, 3.0434) -- (2.5580, 3.0660, 3.0493) -- cycle;
\fill[blue!15.3, opacity=0.7] (2.5580, 3.0660, 3.0493) -- (2.6040, 3.0660, 3.0434) -- (2.6040, 3.1200, 3.0371) -- (2.5580, 3.1200, 3.0430) -- cycle;
\fill[blue!43.4, opacity=0.7] (2.6040, -0.1200, 3.0371) -- (2.6500, -0.1200, 3.0311) -- (2.6500, -0.0660, 3.0373) -- (2.6040, -0.0660, 3.0434) -- cycle;
\fill[blue!46.6, opacity=0.7] (2.6040, -0.0660, 3.0434) -- (2.6500, -0.0660, 3.0373) -- (2.6500, -0.0120, 3.0436) -- (2.6040, -0.0120, 3.0496) -- cycle;
\fill[blue!40.5, opacity=0.7] (2.6040, -0.0120, 3.0496) -- (2.6500, -0.0120, 3.0436) -- (2.6500, 0.0420, 3.0498) -- (2.6040, 0.0420, 3.0559) -- cycle;
\fill[blue!29.3, opacity=0.7] (2.6040, 0.0420, 3.0559) -- (2.6500, 0.0420, 3.0498) -- (2.6500, 0.0960, 3.0560) -- (2.6040, 0.0960, 3.0620) -- cycle;
\fill[blue!20.7, opacity=0.7] (2.6040, 0.0960, 3.0620) -- (2.6500, 0.0960, 3.0560) -- (2.6500, 0.1500, 3.0621) -- (2.6040, 0.1500, 3.0681) -- cycle;
\fill[blue!17.1, opacity=0.7] (2.6040, 0.1500, 3.0681) -- (2.6500, 0.1500, 3.0621) -- (2.6500, 0.2040, 3.0681) -- (2.6040, 0.2040, 3.0742) -- cycle;
\fill[blue!16.1, opacity=0.7] (2.6040, 0.2040, 3.0742) -- (2.6500, 0.2040, 3.0681) -- (2.6500, 0.2580, 3.0741) -- (2.6040, 0.2580, 3.0801) -- cycle;
\fill[blue!16.2, opacity=0.7] (2.6040, 0.2580, 3.0801) -- (2.6500, 0.2580, 3.0741) -- (2.6500, 0.3120, 3.0799) -- (2.6040, 0.3120, 3.0859) -- cycle;
\fill[blue!17.2, opacity=0.7] (2.6040, 0.3120, 3.0859) -- (2.6500, 0.3120, 3.0799) -- (2.6500, 0.3660, 3.0855) -- (2.6040, 0.3660, 3.0916) -- cycle;
\fill[blue!21.0, opacity=0.7] (2.6040, 0.3660, 3.0916) -- (2.6500, 0.3660, 3.0855) -- (2.6500, 0.4200, 3.0911) -- (2.6040, 0.4200, 3.0971) -- cycle;
\fill[blue!30.6, opacity=0.7] (2.6040, 0.4200, 3.0971) -- (2.6500, 0.4200, 3.0911) -- (2.6500, 0.4740, 3.0964) -- (2.6040, 0.4740, 3.1024) -- cycle;
\fill[blue!45.7, opacity=0.7] (2.6040, 0.4740, 3.1024) -- (2.6500, 0.4740, 3.0964) -- (2.6500, 0.5280, 3.1016) -- (2.6040, 0.5280, 3.1076) -- cycle;
\fill[blue!58.3, opacity=0.7] (2.6040, 0.5280, 3.1076) -- (2.6500, 0.5280, 3.1016) -- (2.6500, 0.5820, 3.1066) -- (2.6040, 0.5820, 3.1126) -- cycle;
\fill[blue!63.2, opacity=0.7] (2.6040, 0.5820, 3.1126) -- (2.6500, 0.5820, 3.1066) -- (2.6500, 0.6360, 3.1114) -- (2.6040, 0.6360, 3.1174) -- cycle;
\fill[blue!63.3, opacity=0.7] (2.6040, 0.6360, 3.1174) -- (2.6500, 0.6360, 3.1114) -- (2.6500, 0.6900, 3.1159) -- (2.6040, 0.6900, 3.1219) -- cycle;
\fill[blue!62.9, opacity=0.7] (2.6040, 0.6900, 3.1219) -- (2.6500, 0.6900, 3.1159) -- (2.6500, 0.7440, 3.1202) -- (2.6040, 0.7440, 3.1263) -- cycle;
\fill[blue!63.2, opacity=0.7] (2.6040, 0.7440, 3.1263) -- (2.6500, 0.7440, 3.1202) -- (2.6500, 0.7980, 3.1243) -- (2.6040, 0.7980, 3.1303) -- cycle;
\fill[blue!63.6, opacity=0.7] (2.6040, 0.7980, 3.1303) -- (2.6500, 0.7980, 3.1243) -- (2.6500, 0.8520, 3.1281) -- (2.6040, 0.8520, 3.1342) -- cycle;
\fill[blue!62.7, opacity=0.7] (2.6040, 0.8520, 3.1342) -- (2.6500, 0.8520, 3.1281) -- (2.6500, 0.9060, 3.1317) -- (2.6040, 0.9060, 3.1377) -- cycle;
\fill[blue!59.8, opacity=0.7] (2.6040, 0.9060, 3.1377) -- (2.6500, 0.9060, 3.1317) -- (2.6500, 0.9600, 3.1350) -- (2.6040, 0.9600, 3.1410) -- cycle;
\fill[blue!55.6, opacity=0.7] (2.6040, 0.9600, 3.1410) -- (2.6500, 0.9600, 3.1350) -- (2.6500, 1.0140, 3.1380) -- (2.6040, 1.0140, 3.1440) -- cycle;
\fill[blue!50.9, opacity=0.7] (2.6040, 1.0140, 3.1440) -- (2.6500, 1.0140, 3.1380) -- (2.6500, 1.0680, 3.1407) -- (2.6040, 1.0680, 3.1467) -- cycle;
\fill[blue!46.6, opacity=0.7] (2.6040, 1.0680, 3.1467) -- (2.6500, 1.0680, 3.1407) -- (2.6500, 1.1220, 3.1431) -- (2.6040, 1.1220, 3.1491) -- cycle;
\fill[blue!43.2, opacity=0.7] (2.6040, 1.1220, 3.1491) -- (2.6500, 1.1220, 3.1431) -- (2.6500, 1.1760, 3.1452) -- (2.6040, 1.1760, 3.1512) -- cycle;
\fill[blue!40.8, opacity=0.7] (2.6040, 1.1760, 3.1512) -- (2.6500, 1.1760, 3.1452) -- (2.6500, 1.2300, 3.1470) -- (2.6040, 1.2300, 3.1530) -- cycle;
\fill[blue!39.4, opacity=0.7] (2.6040, 1.2300, 3.1530) -- (2.6500, 1.2300, 3.1470) -- (2.6500, 1.2840, 3.1484) -- (2.6040, 1.2840, 3.1545) -- cycle;
\fill[blue!38.7, opacity=0.7] (2.6040, 1.2840, 3.1545) -- (2.6500, 1.2840, 3.1484) -- (2.6500, 1.3380, 3.1496) -- (2.6040, 1.3380, 3.1556) -- cycle;
\fill[blue!38.8, opacity=0.7] (2.6040, 1.3380, 3.1556) -- (2.6500, 1.3380, 3.1496) -- (2.6500, 1.3920, 3.1504) -- (2.6040, 1.3920, 3.1564) -- cycle;
\fill[blue!39.7, opacity=0.7] (2.6040, 1.3920, 3.1564) -- (2.6500, 1.3920, 3.1504) -- (2.6500, 1.4460, 3.1509) -- (2.6040, 1.4460, 3.1569) -- cycle;
\fill[blue!41.3, opacity=0.7] (2.6040, 1.4460, 3.1569) -- (2.6500, 1.4460, 3.1509) -- (2.6500, 1.5000, 3.1511) -- (2.6040, 1.5000, 3.1571) -- cycle;
\fill[blue!43.8, opacity=0.7] (2.6040, 1.5000, 3.1571) -- (2.6500, 1.5000, 3.1511) -- (2.6500, 1.5540, 3.1509) -- (2.6040, 1.5540, 3.1569) -- cycle;
\fill[blue!47.3, opacity=0.7] (2.6040, 1.5540, 3.1569) -- (2.6500, 1.5540, 3.1509) -- (2.6500, 1.6080, 3.1504) -- (2.6040, 1.6080, 3.1564) -- cycle;
\fill[blue!51.6, opacity=0.7] (2.6040, 1.6080, 3.1564) -- (2.6500, 1.6080, 3.1504) -- (2.6500, 1.6620, 3.1496) -- (2.6040, 1.6620, 3.1556) -- cycle;
\fill[blue!56.4, opacity=0.7] (2.6040, 1.6620, 3.1556) -- (2.6500, 1.6620, 3.1496) -- (2.6500, 1.7160, 3.1484) -- (2.6040, 1.7160, 3.1545) -- cycle;
\fill[blue!60.7, opacity=0.7] (2.6040, 1.7160, 3.1545) -- (2.6500, 1.7160, 3.1484) -- (2.6500, 1.7700, 3.1470) -- (2.6040, 1.7700, 3.1530) -- cycle;
\fill[blue!63.2, opacity=0.7] (2.6040, 1.7700, 3.1530) -- (2.6500, 1.7700, 3.1470) -- (2.6500, 1.8240, 3.1452) -- (2.6040, 1.8240, 3.1512) -- cycle;
\fill[blue!63.3, opacity=0.7] (2.6040, 1.8240, 3.1512) -- (2.6500, 1.8240, 3.1452) -- (2.6500, 1.8780, 3.1431) -- (2.6040, 1.8780, 3.1491) -- cycle;
\fill[blue!61.4, opacity=0.7] (2.6040, 1.8780, 3.1491) -- (2.6500, 1.8780, 3.1431) -- (2.6500, 1.9320, 3.1407) -- (2.6040, 1.9320, 3.1467) -- cycle;
\fill[blue!59.4, opacity=0.7] (2.6040, 1.9320, 3.1467) -- (2.6500, 1.9320, 3.1407) -- (2.6500, 1.9860, 3.1380) -- (2.6040, 1.9860, 3.1440) -- cycle;
\fill[blue!59.1, opacity=0.7] (2.6040, 1.9860, 3.1440) -- (2.6500, 1.9860, 3.1380) -- (2.6500, 2.0400, 3.1350) -- (2.6040, 2.0400, 3.1410) -- cycle;
\fill[blue!61.3, opacity=0.7] (2.6040, 2.0400, 3.1410) -- (2.6500, 2.0400, 3.1350) -- (2.6500, 2.0940, 3.1317) -- (2.6040, 2.0940, 3.1377) -- cycle;
\fill[blue!63.6, opacity=0.7] (2.6040, 2.0940, 3.1377) -- (2.6500, 2.0940, 3.1317) -- (2.6500, 2.1480, 3.1281) -- (2.6040, 2.1480, 3.1342) -- cycle;
\fill[blue!59.8, opacity=0.7] (2.6040, 2.1480, 3.1342) -- (2.6500, 2.1480, 3.1281) -- (2.6500, 2.2020, 3.1243) -- (2.6040, 2.2020, 3.1303) -- cycle;
\fill[blue!46.4, opacity=0.7] (2.6040, 2.2020, 3.1303) -- (2.6500, 2.2020, 3.1243) -- (2.6500, 2.2560, 3.1202) -- (2.6040, 2.2560, 3.1263) -- cycle;
\fill[blue!31.0, opacity=0.7] (2.6040, 2.2560, 3.1263) -- (2.6500, 2.2560, 3.1202) -- (2.6500, 2.3100, 3.1159) -- (2.6040, 2.3100, 3.1219) -- cycle;
\fill[blue!22.0, opacity=0.7] (2.6040, 2.3100, 3.1219) -- (2.6500, 2.3100, 3.1159) -- (2.6500, 2.3640, 3.1114) -- (2.6040, 2.3640, 3.1174) -- cycle;
\fill[blue!18.7, opacity=0.7] (2.6040, 2.3640, 3.1174) -- (2.6500, 2.3640, 3.1114) -- (2.6500, 2.4180, 3.1066) -- (2.6040, 2.4180, 3.1126) -- cycle;
\fill[blue!18.2, opacity=0.7] (2.6040, 2.4180, 3.1126) -- (2.6500, 2.4180, 3.1066) -- (2.6500, 2.4720, 3.1016) -- (2.6040, 2.4720, 3.1076) -- cycle;
\fill[blue!20.2, opacity=0.7] (2.6040, 2.4720, 3.1076) -- (2.6500, 2.4720, 3.1016) -- (2.6500, 2.5260, 3.0964) -- (2.6040, 2.5260, 3.1024) -- cycle;
\fill[blue!26.7, opacity=0.7] (2.6040, 2.5260, 3.1024) -- (2.6500, 2.5260, 3.0964) -- (2.6500, 2.5800, 3.0911) -- (2.6040, 2.5800, 3.0971) -- cycle;
\fill[blue!40.0, opacity=0.7] (2.6040, 2.5800, 3.0971) -- (2.6500, 2.5800, 3.0911) -- (2.6500, 2.6340, 3.0855) -- (2.6040, 2.6340, 3.0916) -- cycle;
\fill[blue!53.6, opacity=0.7] (2.6040, 2.6340, 3.0916) -- (2.6500, 2.6340, 3.0855) -- (2.6500, 2.6880, 3.0799) -- (2.6040, 2.6880, 3.0859) -- cycle;
\fill[blue!59.3, opacity=0.7] (2.6040, 2.6880, 3.0859) -- (2.6500, 2.6880, 3.0799) -- (2.6500, 2.7420, 3.0741) -- (2.6040, 2.7420, 3.0801) -- cycle;
\fill[blue!57.6, opacity=0.7] (2.6040, 2.7420, 3.0801) -- (2.6500, 2.7420, 3.0741) -- (2.6500, 2.7960, 3.0681) -- (2.6040, 2.7960, 3.0742) -- cycle;
\fill[blue!45.9, opacity=0.7] (2.6040, 2.7960, 3.0742) -- (2.6500, 2.7960, 3.0681) -- (2.6500, 2.8500, 3.0621) -- (2.6040, 2.8500, 3.0681) -- cycle;
\fill[blue!27.6, opacity=0.7] (2.6040, 2.8500, 3.0681) -- (2.6500, 2.8500, 3.0621) -- (2.6500, 2.9040, 3.0560) -- (2.6040, 2.9040, 3.0620) -- cycle;
\fill[blue!17.7, opacity=0.7] (2.6040, 2.9040, 3.0620) -- (2.6500, 2.9040, 3.0560) -- (2.6500, 2.9580, 3.0498) -- (2.6040, 2.9580, 3.0559) -- cycle;
\fill[blue!15.5, opacity=0.7] (2.6040, 2.9580, 3.0559) -- (2.6500, 2.9580, 3.0498) -- (2.6500, 3.0120, 3.0436) -- (2.6040, 3.0120, 3.0496) -- cycle;
\fill[blue!15.2, opacity=0.7] (2.6040, 3.0120, 3.0496) -- (2.6500, 3.0120, 3.0436) -- (2.6500, 3.0660, 3.0373) -- (2.6040, 3.0660, 3.0434) -- cycle;
\fill[blue!15.4, opacity=0.7] (2.6040, 3.0660, 3.0434) -- (2.6500, 3.0660, 3.0373) -- (2.6500, 3.1200, 3.0311) -- (2.6040, 3.1200, 3.0371) -- cycle;
\fill[blue!32.2, opacity=0.7] (2.6500, -0.1200, 3.0311) -- (2.6960, -0.1200, 3.0249) -- (2.6960, -0.0660, 3.0312) -- (2.6500, -0.0660, 3.0373) -- cycle;
\fill[blue!43.8, opacity=0.7] (2.6500, -0.0660, 3.0373) -- (2.6960, -0.0660, 3.0312) -- (2.6960, -0.0120, 3.0375) -- (2.6500, -0.0120, 3.0436) -- cycle;
\fill[blue!47.0, opacity=0.7] (2.6500, -0.0120, 3.0436) -- (2.6960, -0.0120, 3.0375) -- (2.6960, 0.0420, 3.0437) -- (2.6500, 0.0420, 3.0498) -- cycle;
\fill[blue!41.5, opacity=0.7] (2.6500, 0.0420, 3.0498) -- (2.6960, 0.0420, 3.0437) -- (2.6960, 0.0960, 3.0499) -- (2.6500, 0.0960, 3.0560) -- cycle;
\fill[blue!30.9, opacity=0.7] (2.6500, 0.0960, 3.0560) -- (2.6960, 0.0960, 3.0499) -- (2.6960, 0.1500, 3.0560) -- (2.6500, 0.1500, 3.0621) -- cycle;
\fill[blue!21.9, opacity=0.7] (2.6500, 0.1500, 3.0621) -- (2.6960, 0.1500, 3.0560) -- (2.6960, 0.2040, 3.0620) -- (2.6500, 0.2040, 3.0681) -- cycle;
\fill[blue!17.7, opacity=0.7] (2.6500, 0.2040, 3.0681) -- (2.6960, 0.2040, 3.0620) -- (2.6960, 0.2580, 3.0680) -- (2.6500, 0.2580, 3.0741) -- cycle;
\fill[blue!16.3, opacity=0.7] (2.6500, 0.2580, 3.0741) -- (2.6960, 0.2580, 3.0680) -- (2.6960, 0.3120, 3.0738) -- (2.6500, 0.3120, 3.0799) -- cycle;
\fill[blue!16.1, opacity=0.7] (2.6500, 0.3120, 3.0799) -- (2.6960, 0.3120, 3.0738) -- (2.6960, 0.3660, 3.0794) -- (2.6500, 0.3660, 3.0855) -- cycle;
\fill[blue!16.7, opacity=0.7] (2.6500, 0.3660, 3.0855) -- (2.6960, 0.3660, 3.0794) -- (2.6960, 0.4200, 3.0849) -- (2.6500, 0.4200, 3.0911) -- cycle;
\fill[blue!18.8, opacity=0.7] (2.6500, 0.4200, 3.0911) -- (2.6960, 0.4200, 3.0849) -- (2.6960, 0.4740, 3.0903) -- (2.6500, 0.4740, 3.0964) -- cycle;
\fill[blue!24.7, opacity=0.7] (2.6500, 0.4740, 3.0964) -- (2.6960, 0.4740, 3.0903) -- (2.6960, 0.5280, 3.0955) -- (2.6500, 0.5280, 3.1016) -- cycle;
\fill[blue!36.1, opacity=0.7] (2.6500, 0.5280, 3.1016) -- (2.6960, 0.5280, 3.0955) -- (2.6960, 0.5820, 3.1005) -- (2.6500, 0.5820, 3.1066) -- cycle;
\fill[blue!50.1, opacity=0.7] (2.6500, 0.5820, 3.1066) -- (2.6960, 0.5820, 3.1005) -- (2.6960, 0.6360, 3.1052) -- (2.6500, 0.6360, 3.1114) -- cycle;
\fill[blue!59.9, opacity=0.7] (2.6500, 0.6360, 3.1114) -- (2.6960, 0.6360, 3.1052) -- (2.6960, 0.6900, 3.1098) -- (2.6500, 0.6900, 3.1159) -- cycle;
\fill[blue!63.4, opacity=0.7] (2.6500, 0.6900, 3.1159) -- (2.6960, 0.6900, 3.1098) -- (2.6960, 0.7440, 3.1141) -- (2.6500, 0.7440, 3.1202) -- cycle;
\fill[blue!63.3, opacity=0.7] (2.6500, 0.7440, 3.1202) -- (2.6960, 0.7440, 3.1141) -- (2.6960, 0.7980, 3.1182) -- (2.6500, 0.7980, 3.1243) -- cycle;
\fill[blue!62.7, opacity=0.7] (2.6500, 0.7980, 3.1243) -- (2.6960, 0.7980, 3.1182) -- (2.6960, 0.8520, 3.1220) -- (2.6500, 0.8520, 3.1281) -- cycle;
\fill[blue!62.7, opacity=0.7] (2.6500, 0.8520, 3.1281) -- (2.6960, 0.8520, 3.1220) -- (2.6960, 0.9060, 3.1256) -- (2.6500, 0.9060, 3.1317) -- cycle;
\fill[blue!63.2, opacity=0.7] (2.6500, 0.9060, 3.1317) -- (2.6960, 0.9060, 3.1256) -- (2.6960, 0.9600, 3.1289) -- (2.6500, 0.9600, 3.1350) -- cycle;
\fill[blue!63.5, opacity=0.7] (2.6500, 0.9600, 3.1350) -- (2.6960, 0.9600, 3.1289) -- (2.6960, 1.0140, 3.1319) -- (2.6500, 1.0140, 3.1380) -- cycle;
\fill[blue!63.4, opacity=0.7] (2.6500, 1.0140, 3.1380) -- (2.6960, 1.0140, 3.1319) -- (2.6960, 1.0680, 3.1346) -- (2.6500, 1.0680, 3.1407) -- cycle;
\fill[blue!62.6, opacity=0.7] (2.6500, 1.0680, 3.1407) -- (2.6960, 1.0680, 3.1346) -- (2.6960, 1.1220, 3.1370) -- (2.6500, 1.1220, 3.1431) -- cycle;
\fill[blue!61.5, opacity=0.7] (2.6500, 1.1220, 3.1431) -- (2.6960, 1.1220, 3.1370) -- (2.6960, 1.1760, 3.1391) -- (2.6500, 1.1760, 3.1452) -- cycle;
\fill[blue!60.2, opacity=0.7] (2.6500, 1.1760, 3.1452) -- (2.6960, 1.1760, 3.1391) -- (2.6960, 1.2300, 3.1409) -- (2.6500, 1.2300, 3.1470) -- cycle;
\fill[blue!59.2, opacity=0.7] (2.6500, 1.2300, 3.1470) -- (2.6960, 1.2300, 3.1409) -- (2.6960, 1.2840, 3.1423) -- (2.6500, 1.2840, 3.1484) -- cycle;
\fill[blue!58.6, opacity=0.7] (2.6500, 1.2840, 3.1484) -- (2.6960, 1.2840, 3.1423) -- (2.6960, 1.3380, 3.1435) -- (2.6500, 1.3380, 3.1496) -- cycle;
\fill[blue!58.7, opacity=0.7] (2.6500, 1.3380, 3.1496) -- (2.6960, 1.3380, 3.1435) -- (2.6960, 1.3920, 3.1443) -- (2.6500, 1.3920, 3.1504) -- cycle;
\fill[blue!59.3, opacity=0.7] (2.6500, 1.3920, 3.1504) -- (2.6960, 1.3920, 3.1443) -- (2.6960, 1.4460, 3.1448) -- (2.6500, 1.4460, 3.1509) -- cycle;
\fill[blue!60.4, opacity=0.7] (2.6500, 1.4460, 3.1509) -- (2.6960, 1.4460, 3.1448) -- (2.6960, 1.5000, 3.1449) -- (2.6500, 1.5000, 3.1511) -- cycle;
\fill[blue!61.7, opacity=0.7] (2.6500, 1.5000, 3.1511) -- (2.6960, 1.5000, 3.1449) -- (2.6960, 1.5540, 3.1448) -- (2.6500, 1.5540, 3.1509) -- cycle;
\fill[blue!62.9, opacity=0.7] (2.6500, 1.5540, 3.1509) -- (2.6960, 1.5540, 3.1448) -- (2.6960, 1.6080, 3.1443) -- (2.6500, 1.6080, 3.1504) -- cycle;
\fill[blue!63.5, opacity=0.7] (2.6500, 1.6080, 3.1504) -- (2.6960, 1.6080, 3.1443) -- (2.6960, 1.6620, 3.1435) -- (2.6500, 1.6620, 3.1496) -- cycle;
\fill[blue!63.3, opacity=0.7] (2.6500, 1.6620, 3.1496) -- (2.6960, 1.6620, 3.1435) -- (2.6960, 1.7160, 3.1423) -- (2.6500, 1.7160, 3.1484) -- cycle;
\fill[blue!62.2, opacity=0.7] (2.6500, 1.7160, 3.1484) -- (2.6960, 1.7160, 3.1423) -- (2.6960, 1.7700, 3.1409) -- (2.6500, 1.7700, 3.1470) -- cycle;
\fill[blue!60.6, opacity=0.7] (2.6500, 1.7700, 3.1470) -- (2.6960, 1.7700, 3.1409) -- (2.6960, 1.8240, 3.1391) -- (2.6500, 1.8240, 3.1452) -- cycle;
\fill[blue!59.6, opacity=0.7] (2.6500, 1.8240, 3.1452) -- (2.6960, 1.8240, 3.1391) -- (2.6960, 1.8780, 3.1370) -- (2.6500, 1.8780, 3.1431) -- cycle;
\fill[blue!59.9, opacity=0.7] (2.6500, 1.8780, 3.1431) -- (2.6960, 1.8780, 3.1370) -- (2.6960, 1.9320, 3.1346) -- (2.6500, 1.9320, 3.1407) -- cycle;
\fill[blue!61.7, opacity=0.7] (2.6500, 1.9320, 3.1407) -- (2.6960, 1.9320, 3.1346) -- (2.6960, 1.9860, 3.1319) -- (2.6500, 1.9860, 3.1380) -- cycle;
\fill[blue!63.5, opacity=0.7] (2.6500, 1.9860, 3.1380) -- (2.6960, 1.9860, 3.1319) -- (2.6960, 2.0400, 3.1289) -- (2.6500, 2.0400, 3.1350) -- cycle;
\fill[blue!61.4, opacity=0.7] (2.6500, 2.0400, 3.1350) -- (2.6960, 2.0400, 3.1289) -- (2.6960, 2.0940, 3.1256) -- (2.6500, 2.0940, 3.1317) -- cycle;
\fill[blue!51.7, opacity=0.7] (2.6500, 2.0940, 3.1317) -- (2.6960, 2.0940, 3.1256) -- (2.6960, 2.1480, 3.1220) -- (2.6500, 2.1480, 3.1281) -- cycle;
\fill[blue!37.3, opacity=0.7] (2.6500, 2.1480, 3.1281) -- (2.6960, 2.1480, 3.1220) -- (2.6960, 2.2020, 3.1182) -- (2.6500, 2.2020, 3.1243) -- cycle;
\fill[blue!25.9, opacity=0.7] (2.6500, 2.2020, 3.1243) -- (2.6960, 2.2020, 3.1182) -- (2.6960, 2.2560, 3.1141) -- (2.6500, 2.2560, 3.1202) -- cycle;
\fill[blue!20.1, opacity=0.7] (2.6500, 2.2560, 3.1202) -- (2.6960, 2.2560, 3.1141) -- (2.6960, 2.3100, 3.1098) -- (2.6500, 2.3100, 3.1159) -- cycle;
\fill[blue!18.2, opacity=0.7] (2.6500, 2.3100, 3.1159) -- (2.6960, 2.3100, 3.1098) -- (2.6960, 2.3640, 3.1052) -- (2.6500, 2.3640, 3.1114) -- cycle;
\fill[blue!18.3, opacity=0.7] (2.6500, 2.3640, 3.1114) -- (2.6960, 2.3640, 3.1052) -- (2.6960, 2.4180, 3.1005) -- (2.6500, 2.4180, 3.1066) -- cycle;
\fill[blue!20.6, opacity=0.7] (2.6500, 2.4180, 3.1066) -- (2.6960, 2.4180, 3.1005) -- (2.6960, 2.4720, 3.0955) -- (2.6500, 2.4720, 3.1016) -- cycle;
\fill[blue!27.5, opacity=0.7] (2.6500, 2.4720, 3.1016) -- (2.6960, 2.4720, 3.0955) -- (2.6960, 2.5260, 3.0903) -- (2.6500, 2.5260, 3.0964) -- cycle;
\fill[blue!40.5, opacity=0.7] (2.6500, 2.5260, 3.0964) -- (2.6960, 2.5260, 3.0903) -- (2.6960, 2.5800, 3.0849) -- (2.6500, 2.5800, 3.0911) -- cycle;
\fill[blue!53.4, opacity=0.7] (2.6500, 2.5800, 3.0911) -- (2.6960, 2.5800, 3.0849) -- (2.6960, 2.6340, 3.0794) -- (2.6500, 2.6340, 3.0855) -- cycle;
\fill[blue!59.0, opacity=0.7] (2.6500, 2.6340, 3.0855) -- (2.6960, 2.6340, 3.0794) -- (2.6960, 2.6880, 3.0738) -- (2.6500, 2.6880, 3.0799) -- cycle;
\fill[blue!57.7, opacity=0.7] (2.6500, 2.6880, 3.0799) -- (2.6960, 2.6880, 3.0738) -- (2.6960, 2.7420, 3.0680) -- (2.6500, 2.7420, 3.0741) -- cycle;
\fill[blue!47.3, opacity=0.7] (2.6500, 2.7420, 3.0741) -- (2.6960, 2.7420, 3.0680) -- (2.6960, 2.7960, 3.0620) -- (2.6500, 2.7960, 3.0681) -- cycle;
\fill[blue!29.6, opacity=0.7] (2.6500, 2.7960, 3.0681) -- (2.6960, 2.7960, 3.0620) -- (2.6960, 2.8500, 3.0560) -- (2.6500, 2.8500, 3.0621) -- cycle;
\fill[blue!18.5, opacity=0.7] (2.6500, 2.8500, 3.0621) -- (2.6960, 2.8500, 3.0560) -- (2.6960, 2.9040, 3.0499) -- (2.6500, 2.9040, 3.0560) -- cycle;
\fill[blue!15.7, opacity=0.7] (2.6500, 2.9040, 3.0560) -- (2.6960, 2.9040, 3.0499) -- (2.6960, 2.9580, 3.0437) -- (2.6500, 2.9580, 3.0498) -- cycle;
\fill[blue!15.3, opacity=0.7] (2.6500, 2.9580, 3.0498) -- (2.6960, 2.9580, 3.0437) -- (2.6960, 3.0120, 3.0375) -- (2.6500, 3.0120, 3.0436) -- cycle;
\fill[blue!15.3, opacity=0.7] (2.6500, 3.0120, 3.0436) -- (2.6960, 3.0120, 3.0375) -- (2.6960, 3.0660, 3.0312) -- (2.6500, 3.0660, 3.0373) -- cycle;
\fill[blue!16.1, opacity=0.7] (2.6500, 3.0660, 3.0373) -- (2.6960, 3.0660, 3.0312) -- (2.6960, 3.1200, 3.0249) -- (2.6500, 3.1200, 3.0311) -- cycle;
\fill[blue!20.7, opacity=0.7] (2.6960, -0.1200, 3.0249) -- (2.7420, -0.1200, 3.0188) -- (2.7420, -0.0660, 3.0251) -- (2.6960, -0.0660, 3.0312) -- cycle;
\fill[blue!32.1, opacity=0.7] (2.6960, -0.0660, 3.0312) -- (2.7420, -0.0660, 3.0251) -- (2.7420, -0.0120, 3.0313) -- (2.6960, -0.0120, 3.0375) -- cycle;
\fill[blue!43.6, opacity=0.7] (2.6960, -0.0120, 3.0375) -- (2.7420, -0.0120, 3.0313) -- (2.7420, 0.0420, 3.0375) -- (2.6960, 0.0420, 3.0437) -- cycle;
\fill[blue!47.5, opacity=0.7] (2.6960, 0.0420, 3.0437) -- (2.7420, 0.0420, 3.0375) -- (2.7420, 0.0960, 3.0437) -- (2.6960, 0.0960, 3.0499) -- cycle;
\fill[blue!43.2, opacity=0.7] (2.6960, 0.0960, 3.0499) -- (2.7420, 0.0960, 3.0437) -- (2.7420, 0.1500, 3.0498) -- (2.6960, 0.1500, 3.0560) -- cycle;
\fill[blue!33.4, opacity=0.7] (2.6960, 0.1500, 3.0560) -- (2.7420, 0.1500, 3.0498) -- (2.7420, 0.2040, 3.0559) -- (2.6960, 0.2040, 3.0620) -- cycle;
\fill[blue!23.9, opacity=0.7] (2.6960, 0.2040, 3.0620) -- (2.7420, 0.2040, 3.0559) -- (2.7420, 0.2580, 3.0618) -- (2.6960, 0.2580, 3.0680) -- cycle;
\fill[blue!18.7, opacity=0.7] (2.6960, 0.2580, 3.0680) -- (2.7420, 0.2580, 3.0618) -- (2.7420, 0.3120, 3.0676) -- (2.6960, 0.3120, 3.0738) -- cycle;
\fill[blue!16.7, opacity=0.7] (2.6960, 0.3120, 3.0738) -- (2.7420, 0.3120, 3.0676) -- (2.7420, 0.3660, 3.0733) -- (2.6960, 0.3660, 3.0794) -- cycle;
\fill[blue!16.2, opacity=0.7] (2.6960, 0.3660, 3.0794) -- (2.7420, 0.3660, 3.0733) -- (2.7420, 0.4200, 3.0788) -- (2.6960, 0.4200, 3.0849) -- cycle;
\fill[blue!16.3, opacity=0.7] (2.6960, 0.4200, 3.0849) -- (2.7420, 0.4200, 3.0788) -- (2.7420, 0.4740, 3.0841) -- (2.6960, 0.4740, 3.0903) -- cycle;
\fill[blue!17.3, opacity=0.7] (2.6960, 0.4740, 3.0903) -- (2.7420, 0.4740, 3.0841) -- (2.7420, 0.5280, 3.0893) -- (2.6960, 0.5280, 3.0955) -- cycle;
\fill[blue!20.3, opacity=0.7] (2.6960, 0.5280, 3.0955) -- (2.7420, 0.5280, 3.0893) -- (2.7420, 0.5820, 3.0943) -- (2.6960, 0.5820, 3.1005) -- cycle;
\fill[blue!26.9, opacity=0.7] (2.6960, 0.5820, 3.1005) -- (2.7420, 0.5820, 3.0943) -- (2.7420, 0.6360, 3.0991) -- (2.6960, 0.6360, 3.1052) -- cycle;
\fill[blue!37.7, opacity=0.7] (2.6960, 0.6360, 3.1052) -- (2.7420, 0.6360, 3.0991) -- (2.7420, 0.6900, 3.1036) -- (2.6960, 0.6900, 3.1098) -- cycle;
\fill[blue!49.9, opacity=0.7] (2.6960, 0.6900, 3.1098) -- (2.7420, 0.6900, 3.1036) -- (2.7420, 0.7440, 3.1079) -- (2.6960, 0.7440, 3.1141) -- cycle;
\fill[blue!58.8, opacity=0.7] (2.6960, 0.7440, 3.1141) -- (2.7420, 0.7440, 3.1079) -- (2.7420, 0.7980, 3.1120) -- (2.6960, 0.7980, 3.1182) -- cycle;
\fill[blue!62.8, opacity=0.7] (2.6960, 0.7980, 3.1182) -- (2.7420, 0.7980, 3.1120) -- (2.7420, 0.8520, 3.1159) -- (2.6960, 0.8520, 3.1220) -- cycle;
\fill[blue!63.6, opacity=0.7] (2.6960, 0.8520, 3.1220) -- (2.7420, 0.8520, 3.1159) -- (2.7420, 0.9060, 3.1194) -- (2.6960, 0.9060, 3.1256) -- cycle;
\fill[blue!63.0, opacity=0.7] (2.6960, 0.9060, 3.1256) -- (2.7420, 0.9060, 3.1194) -- (2.7420, 0.9600, 3.1227) -- (2.6960, 0.9600, 3.1289) -- cycle;
\fill[blue!62.4, opacity=0.7] (2.6960, 0.9600, 3.1289) -- (2.7420, 0.9600, 3.1227) -- (2.7420, 1.0140, 3.1257) -- (2.6960, 1.0140, 3.1319) -- cycle;
\fill[blue!62.2, opacity=0.7] (2.6960, 1.0140, 3.1319) -- (2.7420, 1.0140, 3.1257) -- (2.7420, 1.0680, 3.1284) -- (2.6960, 1.0680, 3.1346) -- cycle;
\fill[blue!62.3, opacity=0.7] (2.6960, 1.0680, 3.1346) -- (2.7420, 1.0680, 3.1284) -- (2.7420, 1.1220, 3.1308) -- (2.6960, 1.1220, 3.1370) -- cycle;
\fill[blue!62.5, opacity=0.7] (2.6960, 1.1220, 3.1370) -- (2.7420, 1.1220, 3.1308) -- (2.7420, 1.1760, 3.1329) -- (2.6960, 1.1760, 3.1391) -- cycle;
\fill[blue!62.8, opacity=0.7] (2.6960, 1.1760, 3.1391) -- (2.7420, 1.1760, 3.1329) -- (2.7420, 1.2300, 3.1347) -- (2.6960, 1.2300, 3.1409) -- cycle;
\fill[blue!62.9, opacity=0.7] (2.6960, 1.2300, 3.1409) -- (2.7420, 1.2300, 3.1347) -- (2.7420, 1.2840, 3.1361) -- (2.6960, 1.2840, 3.1423) -- cycle;
\fill[blue!63.0, opacity=0.7] (2.6960, 1.2840, 3.1423) -- (2.7420, 1.2840, 3.1361) -- (2.7420, 1.3380, 3.1373) -- (2.6960, 1.3380, 3.1435) -- cycle;
\fill[blue!62.9, opacity=0.7] (2.6960, 1.3380, 3.1435) -- (2.7420, 1.3380, 3.1373) -- (2.7420, 1.3920, 3.1381) -- (2.6960, 1.3920, 3.1443) -- cycle;
\fill[blue!62.8, opacity=0.7] (2.6960, 1.3920, 3.1443) -- (2.7420, 1.3920, 3.1381) -- (2.7420, 1.4460, 3.1386) -- (2.6960, 1.4460, 3.1448) -- cycle;
\fill[blue!62.4, opacity=0.7] (2.6960, 1.4460, 3.1448) -- (2.7420, 1.4460, 3.1386) -- (2.7420, 1.5000, 3.1388) -- (2.6960, 1.5000, 3.1449) -- cycle;
\fill[blue!61.9, opacity=0.7] (2.6960, 1.5000, 3.1449) -- (2.7420, 1.5000, 3.1388) -- (2.7420, 1.5540, 3.1386) -- (2.6960, 1.5540, 3.1448) -- cycle;
\fill[blue!61.2, opacity=0.7] (2.6960, 1.5540, 3.1448) -- (2.7420, 1.5540, 3.1386) -- (2.7420, 1.6080, 3.1381) -- (2.6960, 1.6080, 3.1443) -- cycle;
\fill[blue!60.6, opacity=0.7] (2.6960, 1.6080, 3.1443) -- (2.7420, 1.6080, 3.1381) -- (2.7420, 1.6620, 3.1373) -- (2.6960, 1.6620, 3.1435) -- cycle;
\fill[blue!60.2, opacity=0.7] (2.6960, 1.6620, 3.1435) -- (2.7420, 1.6620, 3.1373) -- (2.7420, 1.7160, 3.1361) -- (2.6960, 1.7160, 3.1423) -- cycle;
\fill[blue!60.5, opacity=0.7] (2.6960, 1.7160, 3.1423) -- (2.7420, 1.7160, 3.1361) -- (2.7420, 1.7700, 3.1347) -- (2.6960, 1.7700, 3.1409) -- cycle;
\fill[blue!61.5, opacity=0.7] (2.6960, 1.7700, 3.1409) -- (2.7420, 1.7700, 3.1347) -- (2.7420, 1.8240, 3.1329) -- (2.6960, 1.8240, 3.1391) -- cycle;
\fill[blue!62.9, opacity=0.7] (2.6960, 1.8240, 3.1391) -- (2.7420, 1.8240, 3.1329) -- (2.7420, 1.8780, 3.1308) -- (2.6960, 1.8780, 3.1370) -- cycle;
\fill[blue!63.5, opacity=0.7] (2.6960, 1.8780, 3.1370) -- (2.7420, 1.8780, 3.1308) -- (2.7420, 1.9320, 3.1284) -- (2.6960, 1.9320, 3.1346) -- cycle;
\fill[blue!60.5, opacity=0.7] (2.6960, 1.9320, 3.1346) -- (2.7420, 1.9320, 3.1284) -- (2.7420, 1.9860, 3.1257) -- (2.6960, 1.9860, 3.1319) -- cycle;
\fill[blue!52.0, opacity=0.7] (2.6960, 1.9860, 3.1319) -- (2.7420, 1.9860, 3.1257) -- (2.7420, 2.0400, 3.1227) -- (2.6960, 2.0400, 3.1289) -- cycle;
\fill[blue!39.5, opacity=0.7] (2.6960, 2.0400, 3.1289) -- (2.7420, 2.0400, 3.1227) -- (2.7420, 2.0940, 3.1194) -- (2.6960, 2.0940, 3.1256) -- cycle;
\fill[blue!28.4, opacity=0.7] (2.6960, 2.0940, 3.1256) -- (2.7420, 2.0940, 3.1194) -- (2.7420, 2.1480, 3.1159) -- (2.6960, 2.1480, 3.1220) -- cycle;
\fill[blue!21.7, opacity=0.7] (2.6960, 2.1480, 3.1220) -- (2.7420, 2.1480, 3.1159) -- (2.7420, 2.2020, 3.1120) -- (2.6960, 2.2020, 3.1182) -- cycle;
\fill[blue!18.7, opacity=0.7] (2.6960, 2.2020, 3.1182) -- (2.7420, 2.2020, 3.1120) -- (2.7420, 2.2560, 3.1079) -- (2.6960, 2.2560, 3.1141) -- cycle;
\fill[blue!17.9, opacity=0.7] (2.6960, 2.2560, 3.1141) -- (2.7420, 2.2560, 3.1079) -- (2.7420, 2.3100, 3.1036) -- (2.6960, 2.3100, 3.1098) -- cycle;
\fill[blue!18.6, opacity=0.7] (2.6960, 2.3100, 3.1098) -- (2.7420, 2.3100, 3.1036) -- (2.7420, 2.3640, 3.0991) -- (2.6960, 2.3640, 3.1052) -- cycle;
\fill[blue!21.7, opacity=0.7] (2.6960, 2.3640, 3.1052) -- (2.7420, 2.3640, 3.0991) -- (2.7420, 2.4180, 3.0943) -- (2.6960, 2.4180, 3.1005) -- cycle;
\fill[blue!29.4, opacity=0.7] (2.6960, 2.4180, 3.1005) -- (2.7420, 2.4180, 3.0943) -- (2.7420, 2.4720, 3.0893) -- (2.6960, 2.4720, 3.0955) -- cycle;
\fill[blue!42.2, opacity=0.7] (2.6960, 2.4720, 3.0955) -- (2.7420, 2.4720, 3.0893) -- (2.7420, 2.5260, 3.0841) -- (2.6960, 2.5260, 3.0903) -- cycle;
\fill[blue!53.9, opacity=0.7] (2.6960, 2.5260, 3.0903) -- (2.7420, 2.5260, 3.0841) -- (2.7420, 2.5800, 3.0788) -- (2.6960, 2.5800, 3.0849) -- cycle;
\fill[blue!58.9, opacity=0.7] (2.6960, 2.5800, 3.0849) -- (2.7420, 2.5800, 3.0788) -- (2.7420, 2.6340, 3.0733) -- (2.6960, 2.6340, 3.0794) -- cycle;
\fill[blue!57.3, opacity=0.7] (2.6960, 2.6340, 3.0794) -- (2.7420, 2.6340, 3.0733) -- (2.7420, 2.6880, 3.0676) -- (2.6960, 2.6880, 3.0738) -- cycle;
\fill[blue!47.4, opacity=0.7] (2.6960, 2.6880, 3.0738) -- (2.7420, 2.6880, 3.0676) -- (2.7420, 2.7420, 3.0618) -- (2.6960, 2.7420, 3.0680) -- cycle;
\fill[blue!30.4, opacity=0.7] (2.6960, 2.7420, 3.0680) -- (2.7420, 2.7420, 3.0618) -- (2.7420, 2.7960, 3.0559) -- (2.6960, 2.7960, 3.0620) -- cycle;
\fill[blue!19.1, opacity=0.7] (2.6960, 2.7960, 3.0620) -- (2.7420, 2.7960, 3.0559) -- (2.7420, 2.8500, 3.0498) -- (2.6960, 2.8500, 3.0560) -- cycle;
\fill[blue!15.8, opacity=0.7] (2.6960, 2.8500, 3.0560) -- (2.7420, 2.8500, 3.0498) -- (2.7420, 2.9040, 3.0437) -- (2.6960, 2.9040, 3.0499) -- cycle;
\fill[blue!15.3, opacity=0.7] (2.6960, 2.9040, 3.0499) -- (2.7420, 2.9040, 3.0437) -- (2.7420, 2.9580, 3.0375) -- (2.6960, 2.9580, 3.0437) -- cycle;
\fill[blue!15.3, opacity=0.7] (2.6960, 2.9580, 3.0437) -- (2.7420, 2.9580, 3.0375) -- (2.7420, 3.0120, 3.0313) -- (2.6960, 3.0120, 3.0375) -- cycle;
\fill[blue!15.8, opacity=0.7] (2.6960, 3.0120, 3.0375) -- (2.7420, 3.0120, 3.0313) -- (2.7420, 3.0660, 3.0251) -- (2.6960, 3.0660, 3.0312) -- cycle;
\fill[blue!18.3, opacity=0.7] (2.6960, 3.0660, 3.0312) -- (2.7420, 3.0660, 3.0251) -- (2.7420, 3.1200, 3.0188) -- (2.6960, 3.1200, 3.0249) -- cycle;
\fill[blue!16.0, opacity=0.7] (2.7420, -0.1200, 3.0188) -- (2.7880, -0.1200, 3.0125) -- (2.7880, -0.0660, 3.0188) -- (2.7420, -0.0660, 3.0251) -- cycle;
\fill[blue!20.3, opacity=0.7] (2.7420, -0.0660, 3.0251) -- (2.7880, -0.0660, 3.0188) -- (2.7880, -0.0120, 3.0251) -- (2.7420, -0.0120, 3.0313) -- cycle;
\fill[blue!31.1, opacity=0.7] (2.7420, -0.0120, 3.0313) -- (2.7880, -0.0120, 3.0251) -- (2.7880, 0.0420, 3.0313) -- (2.7420, 0.0420, 3.0375) -- cycle;
\fill[blue!42.7, opacity=0.7] (2.7420, 0.0420, 3.0375) -- (2.7880, 0.0420, 3.0313) -- (2.7880, 0.0960, 3.0375) -- (2.7420, 0.0960, 3.0437) -- cycle;
\fill[blue!47.8, opacity=0.7] (2.7420, 0.0960, 3.0437) -- (2.7880, 0.0960, 3.0375) -- (2.7880, 0.1500, 3.0436) -- (2.7420, 0.1500, 3.0498) -- cycle;
\fill[blue!45.3, opacity=0.7] (2.7420, 0.1500, 3.0498) -- (2.7880, 0.1500, 3.0436) -- (2.7880, 0.2040, 3.0496) -- (2.7420, 0.2040, 3.0559) -- cycle;
\fill[blue!37.0, opacity=0.7] (2.7420, 0.2040, 3.0559) -- (2.7880, 0.2040, 3.0496) -- (2.7880, 0.2580, 3.0555) -- (2.7420, 0.2580, 3.0618) -- cycle;
\fill[blue!27.2, opacity=0.7] (2.7420, 0.2580, 3.0618) -- (2.7880, 0.2580, 3.0555) -- (2.7880, 0.3120, 3.0614) -- (2.7420, 0.3120, 3.0676) -- cycle;
\fill[blue!20.6, opacity=0.7] (2.7420, 0.3120, 3.0676) -- (2.7880, 0.3120, 3.0614) -- (2.7880, 0.3660, 3.0670) -- (2.7420, 0.3660, 3.0733) -- cycle;
\fill[blue!17.5, opacity=0.7] (2.7420, 0.3660, 3.0733) -- (2.7880, 0.3660, 3.0670) -- (2.7880, 0.4200, 3.0725) -- (2.7420, 0.4200, 3.0788) -- cycle;
\fill[blue!16.4, opacity=0.7] (2.7420, 0.4200, 3.0788) -- (2.7880, 0.4200, 3.0725) -- (2.7880, 0.4740, 3.0779) -- (2.7420, 0.4740, 3.0841) -- cycle;
\fill[blue!16.2, opacity=0.7] (2.7420, 0.4740, 3.0841) -- (2.7880, 0.4740, 3.0779) -- (2.7880, 0.5280, 3.0831) -- (2.7420, 0.5280, 3.0893) -- cycle;
\fill[blue!16.5, opacity=0.7] (2.7420, 0.5280, 3.0893) -- (2.7880, 0.5280, 3.0831) -- (2.7880, 0.5820, 3.0881) -- (2.7420, 0.5820, 3.0943) -- cycle;
\fill[blue!17.7, opacity=0.7] (2.7420, 0.5820, 3.0943) -- (2.7880, 0.5820, 3.0881) -- (2.7880, 0.6360, 3.0928) -- (2.7420, 0.6360, 3.0991) -- cycle;
\fill[blue!20.5, opacity=0.7] (2.7420, 0.6360, 3.0991) -- (2.7880, 0.6360, 3.0928) -- (2.7880, 0.6900, 3.0974) -- (2.7420, 0.6900, 3.1036) -- cycle;
\fill[blue!26.1, opacity=0.7] (2.7420, 0.6900, 3.1036) -- (2.7880, 0.6900, 3.0974) -- (2.7880, 0.7440, 3.1017) -- (2.7420, 0.7440, 3.1079) -- cycle;
\fill[blue!34.8, opacity=0.7] (2.7420, 0.7440, 3.1079) -- (2.7880, 0.7440, 3.1017) -- (2.7880, 0.7980, 3.1058) -- (2.7420, 0.7980, 3.1120) -- cycle;
\fill[blue!45.0, opacity=0.7] (2.7420, 0.7980, 3.1120) -- (2.7880, 0.7980, 3.1058) -- (2.7880, 0.8520, 3.1096) -- (2.7420, 0.8520, 3.1159) -- cycle;
\fill[blue!53.9, opacity=0.7] (2.7420, 0.8520, 3.1159) -- (2.7880, 0.8520, 3.1096) -- (2.7880, 0.9060, 3.1132) -- (2.7420, 0.9060, 3.1194) -- cycle;
\fill[blue!59.7, opacity=0.7] (2.7420, 0.9060, 3.1194) -- (2.7880, 0.9060, 3.1132) -- (2.7880, 0.9600, 3.1165) -- (2.7420, 0.9600, 3.1227) -- cycle;
\fill[blue!62.6, opacity=0.7] (2.7420, 0.9600, 3.1227) -- (2.7880, 0.9600, 3.1165) -- (2.7880, 1.0140, 3.1195) -- (2.7420, 1.0140, 3.1257) -- cycle;
\fill[blue!63.5, opacity=0.7] (2.7420, 1.0140, 3.1257) -- (2.7880, 1.0140, 3.1195) -- (2.7880, 1.0680, 3.1222) -- (2.7420, 1.0680, 3.1284) -- cycle;
\fill[blue!63.4, opacity=0.7] (2.7420, 1.0680, 3.1284) -- (2.7880, 1.0680, 3.1222) -- (2.7880, 1.1220, 3.1246) -- (2.7420, 1.1220, 3.1308) -- cycle;
\fill[blue!63.0, opacity=0.7] (2.7420, 1.1220, 3.1308) -- (2.7880, 1.1220, 3.1246) -- (2.7880, 1.1760, 3.1267) -- (2.7420, 1.1760, 3.1329) -- cycle;
\fill[blue!62.6, opacity=0.7] (2.7420, 1.1760, 3.1329) -- (2.7880, 1.1760, 3.1267) -- (2.7880, 1.2300, 3.1285) -- (2.7420, 1.2300, 3.1347) -- cycle;
\fill[blue!62.2, opacity=0.7] (2.7420, 1.2300, 3.1347) -- (2.7880, 1.2300, 3.1285) -- (2.7880, 1.2840, 3.1299) -- (2.7420, 1.2840, 3.1361) -- cycle;
\fill[blue!62.0, opacity=0.7] (2.7420, 1.2840, 3.1361) -- (2.7880, 1.2840, 3.1299) -- (2.7880, 1.3380, 3.1311) -- (2.7420, 1.3380, 3.1373) -- cycle;
\fill[blue!61.8, opacity=0.7] (2.7420, 1.3380, 3.1373) -- (2.7880, 1.3380, 3.1311) -- (2.7880, 1.3920, 3.1319) -- (2.7420, 1.3920, 3.1381) -- cycle;
\fill[blue!61.7, opacity=0.7] (2.7420, 1.3920, 3.1381) -- (2.7880, 1.3920, 3.1319) -- (2.7880, 1.4460, 3.1324) -- (2.7420, 1.4460, 3.1386) -- cycle;
\fill[blue!61.7, opacity=0.7] (2.7420, 1.4460, 3.1386) -- (2.7880, 1.4460, 3.1324) -- (2.7880, 1.5000, 3.1325) -- (2.7420, 1.5000, 3.1388) -- cycle;
\fill[blue!62.0, opacity=0.7] (2.7420, 1.5000, 3.1388) -- (2.7880, 1.5000, 3.1325) -- (2.7880, 1.5540, 3.1324) -- (2.7420, 1.5540, 3.1386) -- cycle;
\fill[blue!62.4, opacity=0.7] (2.7420, 1.5540, 3.1386) -- (2.7880, 1.5540, 3.1324) -- (2.7880, 1.6080, 3.1319) -- (2.7420, 1.6080, 3.1381) -- cycle;
\fill[blue!63.0, opacity=0.7] (2.7420, 1.6080, 3.1381) -- (2.7880, 1.6080, 3.1319) -- (2.7880, 1.6620, 3.1311) -- (2.7420, 1.6620, 3.1373) -- cycle;
\fill[blue!63.5, opacity=0.7] (2.7420, 1.6620, 3.1373) -- (2.7880, 1.6620, 3.1311) -- (2.7880, 1.7160, 3.1299) -- (2.7420, 1.7160, 3.1361) -- cycle;
\fill[blue!63.3, opacity=0.7] (2.7420, 1.7160, 3.1361) -- (2.7880, 1.7160, 3.1299) -- (2.7880, 1.7700, 3.1285) -- (2.7420, 1.7700, 3.1347) -- cycle;
\fill[blue!61.2, opacity=0.7] (2.7420, 1.7700, 3.1347) -- (2.7880, 1.7700, 3.1285) -- (2.7880, 1.8240, 3.1267) -- (2.7420, 1.8240, 3.1329) -- cycle;
\fill[blue!55.9, opacity=0.7] (2.7420, 1.8240, 3.1329) -- (2.7880, 1.8240, 3.1267) -- (2.7880, 1.8780, 3.1246) -- (2.7420, 1.8780, 3.1308) -- cycle;
\fill[blue!47.2, opacity=0.7] (2.7420, 1.8780, 3.1308) -- (2.7880, 1.8780, 3.1246) -- (2.7880, 1.9320, 3.1222) -- (2.7420, 1.9320, 3.1284) -- cycle;
\fill[blue!36.8, opacity=0.7] (2.7420, 1.9320, 3.1284) -- (2.7880, 1.9320, 3.1222) -- (2.7880, 1.9860, 3.1195) -- (2.7420, 1.9860, 3.1257) -- cycle;
\fill[blue!27.8, opacity=0.7] (2.7420, 1.9860, 3.1257) -- (2.7880, 1.9860, 3.1195) -- (2.7880, 2.0400, 3.1165) -- (2.7420, 2.0400, 3.1227) -- cycle;
\fill[blue!22.0, opacity=0.7] (2.7420, 2.0400, 3.1227) -- (2.7880, 2.0400, 3.1165) -- (2.7880, 2.0940, 3.1132) -- (2.7420, 2.0940, 3.1194) -- cycle;
\fill[blue!19.0, opacity=0.7] (2.7420, 2.0940, 3.1194) -- (2.7880, 2.0940, 3.1132) -- (2.7880, 2.1480, 3.1096) -- (2.7420, 2.1480, 3.1159) -- cycle;
\fill[blue!17.9, opacity=0.7] (2.7420, 2.1480, 3.1159) -- (2.7880, 2.1480, 3.1096) -- (2.7880, 2.2020, 3.1058) -- (2.7420, 2.2020, 3.1120) -- cycle;
\fill[blue!17.9, opacity=0.7] (2.7420, 2.2020, 3.1120) -- (2.7880, 2.2020, 3.1058) -- (2.7880, 2.2560, 3.1017) -- (2.7420, 2.2560, 3.1079) -- cycle;
\fill[blue!19.4, opacity=0.7] (2.7420, 2.2560, 3.1079) -- (2.7880, 2.2560, 3.1017) -- (2.7880, 2.3100, 3.0974) -- (2.7420, 2.3100, 3.1036) -- cycle;
\fill[blue!23.7, opacity=0.7] (2.7420, 2.3100, 3.1036) -- (2.7880, 2.3100, 3.0974) -- (2.7880, 2.3640, 3.0928) -- (2.7420, 2.3640, 3.0991) -- cycle;
\fill[blue!32.7, opacity=0.7] (2.7420, 2.3640, 3.0991) -- (2.7880, 2.3640, 3.0928) -- (2.7880, 2.4180, 3.0881) -- (2.7420, 2.4180, 3.0943) -- cycle;
\fill[blue!45.2, opacity=0.7] (2.7420, 2.4180, 3.0943) -- (2.7880, 2.4180, 3.0881) -- (2.7880, 2.4720, 3.0831) -- (2.7420, 2.4720, 3.0893) -- cycle;
\fill[blue!55.1, opacity=0.7] (2.7420, 2.4720, 3.0893) -- (2.7880, 2.4720, 3.0831) -- (2.7880, 2.5260, 3.0779) -- (2.7420, 2.5260, 3.0841) -- cycle;
\fill[blue!58.7, opacity=0.7] (2.7420, 2.5260, 3.0841) -- (2.7880, 2.5260, 3.0779) -- (2.7880, 2.5800, 3.0725) -- (2.7420, 2.5800, 3.0788) -- cycle;
\fill[blue!56.5, opacity=0.7] (2.7420, 2.5800, 3.0788) -- (2.7880, 2.5800, 3.0725) -- (2.7880, 2.6340, 3.0670) -- (2.7420, 2.6340, 3.0733) -- cycle;
\fill[blue!46.3, opacity=0.7] (2.7420, 2.6340, 3.0733) -- (2.7880, 2.6340, 3.0670) -- (2.7880, 2.6880, 3.0614) -- (2.7420, 2.6880, 3.0676) -- cycle;
\fill[blue!30.0, opacity=0.7] (2.7420, 2.6880, 3.0676) -- (2.7880, 2.6880, 3.0614) -- (2.7880, 2.7420, 3.0555) -- (2.7420, 2.7420, 3.0618) -- cycle;
\fill[blue!19.1, opacity=0.7] (2.7420, 2.7420, 3.0618) -- (2.7880, 2.7420, 3.0555) -- (2.7880, 2.7960, 3.0496) -- (2.7420, 2.7960, 3.0559) -- cycle;
\fill[blue!15.9, opacity=0.7] (2.7420, 2.7960, 3.0559) -- (2.7880, 2.7960, 3.0496) -- (2.7880, 2.8500, 3.0436) -- (2.7420, 2.8500, 3.0498) -- cycle;
\fill[blue!15.3, opacity=0.7] (2.7420, 2.8500, 3.0498) -- (2.7880, 2.8500, 3.0436) -- (2.7880, 2.9040, 3.0375) -- (2.7420, 2.9040, 3.0437) -- cycle;
\fill[blue!15.2, opacity=0.7] (2.7420, 2.9040, 3.0437) -- (2.7880, 2.9040, 3.0375) -- (2.7880, 2.9580, 3.0313) -- (2.7420, 2.9580, 3.0375) -- cycle;
\fill[blue!15.6, opacity=0.7] (2.7420, 2.9580, 3.0375) -- (2.7880, 2.9580, 3.0313) -- (2.7880, 3.0120, 3.0251) -- (2.7420, 3.0120, 3.0313) -- cycle;
\fill[blue!17.5, opacity=0.7] (2.7420, 3.0120, 3.0313) -- (2.7880, 3.0120, 3.0251) -- (2.7880, 3.0660, 3.0188) -- (2.7420, 3.0660, 3.0251) -- cycle;
\fill[blue!23.6, opacity=0.7] (2.7420, 3.0660, 3.0251) -- (2.7880, 3.0660, 3.0188) -- (2.7880, 3.1200, 3.0125) -- (2.7420, 3.1200, 3.0188) -- cycle;
\fill[blue!15.2, opacity=0.7] (2.7880, -0.1200, 3.0125) -- (2.8340, -0.1200, 3.0063) -- (2.8340, -0.0660, 3.0126) -- (2.7880, -0.0660, 3.0188) -- cycle;
\fill[blue!15.9, opacity=0.7] (2.7880, -0.0660, 3.0188) -- (2.8340, -0.0660, 3.0126) -- (2.8340, -0.0120, 3.0188) -- (2.7880, -0.0120, 3.0251) -- cycle;
\fill[blue!19.5, opacity=0.7] (2.7880, -0.0120, 3.0251) -- (2.8340, -0.0120, 3.0188) -- (2.8340, 0.0420, 3.0251) -- (2.7880, 0.0420, 3.0313) -- cycle;
\fill[blue!29.0, opacity=0.7] (2.7880, 0.0420, 3.0313) -- (2.8340, 0.0420, 3.0251) -- (2.8340, 0.0960, 3.0312) -- (2.7880, 0.0960, 3.0375) -- cycle;
\fill[blue!40.8, opacity=0.7] (2.7880, 0.0960, 3.0375) -- (2.8340, 0.0960, 3.0312) -- (2.8340, 0.1500, 3.0373) -- (2.7880, 0.1500, 3.0436) -- cycle;
\fill[blue!47.5, opacity=0.7] (2.7880, 0.1500, 3.0436) -- (2.8340, 0.1500, 3.0373) -- (2.8340, 0.2040, 3.0434) -- (2.7880, 0.2040, 3.0496) -- cycle;
\fill[blue!47.4, opacity=0.7] (2.7880, 0.2040, 3.0496) -- (2.8340, 0.2040, 3.0434) -- (2.8340, 0.2580, 3.0493) -- (2.7880, 0.2580, 3.0555) -- cycle;
\fill[blue!41.3, opacity=0.7] (2.7880, 0.2580, 3.0555) -- (2.8340, 0.2580, 3.0493) -- (2.8340, 0.3120, 3.0551) -- (2.7880, 0.3120, 3.0614) -- cycle;
\fill[blue!32.1, opacity=0.7] (2.7880, 0.3120, 3.0614) -- (2.8340, 0.3120, 3.0551) -- (2.8340, 0.3660, 3.0608) -- (2.7880, 0.3660, 3.0670) -- cycle;
\fill[blue!24.0, opacity=0.7] (2.7880, 0.3660, 3.0670) -- (2.8340, 0.3660, 3.0608) -- (2.8340, 0.4200, 3.0663) -- (2.7880, 0.4200, 3.0725) -- cycle;
\fill[blue!19.2, opacity=0.7] (2.7880, 0.4200, 3.0725) -- (2.8340, 0.4200, 3.0663) -- (2.8340, 0.4740, 3.0716) -- (2.7880, 0.4740, 3.0779) -- cycle;
\fill[blue!17.2, opacity=0.7] (2.7880, 0.4740, 3.0779) -- (2.8340, 0.4740, 3.0716) -- (2.8340, 0.5280, 3.0768) -- (2.7880, 0.5280, 3.0831) -- cycle;
\fill[blue!16.4, opacity=0.7] (2.7880, 0.5280, 3.0831) -- (2.8340, 0.5280, 3.0768) -- (2.8340, 0.5820, 3.0818) -- (2.7880, 0.5820, 3.0881) -- cycle;
\fill[blue!16.3, opacity=0.7] (2.7880, 0.5820, 3.0881) -- (2.8340, 0.5820, 3.0818) -- (2.8340, 0.6360, 3.0866) -- (2.7880, 0.6360, 3.0928) -- cycle;
\fill[blue!16.6, opacity=0.7] (2.7880, 0.6360, 3.0928) -- (2.8340, 0.6360, 3.0866) -- (2.8340, 0.6900, 3.0911) -- (2.7880, 0.6900, 3.0974) -- cycle;
\fill[blue!17.5, opacity=0.7] (2.7880, 0.6900, 3.0974) -- (2.8340, 0.6900, 3.0911) -- (2.8340, 0.7440, 3.0955) -- (2.7880, 0.7440, 3.1017) -- cycle;
\fill[blue!19.5, opacity=0.7] (2.7880, 0.7440, 3.1017) -- (2.8340, 0.7440, 3.0955) -- (2.8340, 0.7980, 3.0995) -- (2.7880, 0.7980, 3.1058) -- cycle;
\fill[blue!23.0, opacity=0.7] (2.7880, 0.7980, 3.1058) -- (2.8340, 0.7980, 3.0995) -- (2.8340, 0.8520, 3.1034) -- (2.7880, 0.8520, 3.1096) -- cycle;
\fill[blue!28.4, opacity=0.7] (2.7880, 0.8520, 3.1096) -- (2.8340, 0.8520, 3.1034) -- (2.8340, 0.9060, 3.1069) -- (2.7880, 0.9060, 3.1132) -- cycle;
\fill[blue!35.3, opacity=0.7] (2.7880, 0.9060, 3.1132) -- (2.8340, 0.9060, 3.1069) -- (2.8340, 0.9600, 3.1102) -- (2.7880, 0.9600, 3.1165) -- cycle;
\fill[blue!42.6, opacity=0.7] (2.7880, 0.9600, 3.1165) -- (2.8340, 0.9600, 3.1102) -- (2.8340, 1.0140, 3.1132) -- (2.7880, 1.0140, 3.1195) -- cycle;
\fill[blue!49.2, opacity=0.7] (2.7880, 1.0140, 3.1195) -- (2.8340, 1.0140, 3.1132) -- (2.8340, 1.0680, 3.1159) -- (2.7880, 1.0680, 3.1222) -- cycle;
\fill[blue!54.2, opacity=0.7] (2.7880, 1.0680, 3.1222) -- (2.8340, 1.0680, 3.1159) -- (2.8340, 1.1220, 3.1183) -- (2.7880, 1.1220, 3.1246) -- cycle;
\fill[blue!57.7, opacity=0.7] (2.7880, 1.1220, 3.1246) -- (2.8340, 1.1220, 3.1183) -- (2.8340, 1.1760, 3.1204) -- (2.7880, 1.1760, 3.1267) -- cycle;
\fill[blue!59.9, opacity=0.7] (2.7880, 1.1760, 3.1267) -- (2.8340, 1.1760, 3.1204) -- (2.8340, 1.2300, 3.1222) -- (2.7880, 1.2300, 3.1285) -- cycle;
\fill[blue!61.2, opacity=0.7] (2.7880, 1.2300, 3.1285) -- (2.8340, 1.2300, 3.1222) -- (2.8340, 1.2840, 3.1237) -- (2.7880, 1.2840, 3.1299) -- cycle;
\fill[blue!61.9, opacity=0.7] (2.7880, 1.2840, 3.1299) -- (2.8340, 1.2840, 3.1237) -- (2.8340, 1.3380, 3.1248) -- (2.7880, 1.3380, 3.1311) -- cycle;
\fill[blue!62.1, opacity=0.7] (2.7880, 1.3380, 3.1311) -- (2.8340, 1.3380, 3.1248) -- (2.8340, 1.3920, 3.1256) -- (2.7880, 1.3920, 3.1319) -- cycle;
\fill[blue!62.0, opacity=0.7] (2.7880, 1.3920, 3.1319) -- (2.8340, 1.3920, 3.1256) -- (2.8340, 1.4460, 3.1261) -- (2.7880, 1.4460, 3.1324) -- cycle;
\fill[blue!61.6, opacity=0.7] (2.7880, 1.4460, 3.1324) -- (2.8340, 1.4460, 3.1261) -- (2.8340, 1.5000, 3.1263) -- (2.7880, 1.5000, 3.1325) -- cycle;
\fill[blue!60.5, opacity=0.7] (2.7880, 1.5000, 3.1325) -- (2.8340, 1.5000, 3.1263) -- (2.8340, 1.5540, 3.1261) -- (2.7880, 1.5540, 3.1324) -- cycle;
\fill[blue!58.7, opacity=0.7] (2.7880, 1.5540, 3.1324) -- (2.8340, 1.5540, 3.1261) -- (2.8340, 1.6080, 3.1256) -- (2.7880, 1.6080, 3.1319) -- cycle;
\fill[blue!55.6, opacity=0.7] (2.7880, 1.6080, 3.1319) -- (2.8340, 1.6080, 3.1256) -- (2.8340, 1.6620, 3.1248) -- (2.7880, 1.6620, 3.1311) -- cycle;
\fill[blue!50.8, opacity=0.7] (2.7880, 1.6620, 3.1311) -- (2.8340, 1.6620, 3.1248) -- (2.8340, 1.7160, 3.1237) -- (2.7880, 1.7160, 3.1299) -- cycle;
\fill[blue!44.5, opacity=0.7] (2.7880, 1.7160, 3.1299) -- (2.8340, 1.7160, 3.1237) -- (2.8340, 1.7700, 3.1222) -- (2.7880, 1.7700, 3.1285) -- cycle;
\fill[blue!37.2, opacity=0.7] (2.7880, 1.7700, 3.1285) -- (2.8340, 1.7700, 3.1222) -- (2.8340, 1.8240, 3.1204) -- (2.7880, 1.8240, 3.1267) -- cycle;
\fill[blue!30.2, opacity=0.7] (2.7880, 1.8240, 3.1267) -- (2.8340, 1.8240, 3.1204) -- (2.8340, 1.8780, 3.1183) -- (2.7880, 1.8780, 3.1246) -- cycle;
\fill[blue!24.5, opacity=0.7] (2.7880, 1.8780, 3.1246) -- (2.8340, 1.8780, 3.1183) -- (2.8340, 1.9320, 3.1159) -- (2.7880, 1.9320, 3.1222) -- cycle;
\fill[blue!20.7, opacity=0.7] (2.7880, 1.9320, 3.1222) -- (2.8340, 1.9320, 3.1159) -- (2.8340, 1.9860, 3.1132) -- (2.7880, 1.9860, 3.1195) -- cycle;
\fill[blue!18.7, opacity=0.7] (2.7880, 1.9860, 3.1195) -- (2.8340, 1.9860, 3.1132) -- (2.8340, 2.0400, 3.1102) -- (2.7880, 2.0400, 3.1165) -- cycle;
\fill[blue!17.7, opacity=0.7] (2.7880, 2.0400, 3.1165) -- (2.8340, 2.0400, 3.1102) -- (2.8340, 2.0940, 3.1069) -- (2.7880, 2.0940, 3.1132) -- cycle;
\fill[blue!17.7, opacity=0.7] (2.7880, 2.0940, 3.1132) -- (2.8340, 2.0940, 3.1069) -- (2.8340, 2.1480, 3.1034) -- (2.7880, 2.1480, 3.1096) -- cycle;
\fill[blue!18.6, opacity=0.7] (2.7880, 2.1480, 3.1096) -- (2.8340, 2.1480, 3.1034) -- (2.8340, 2.2020, 3.0995) -- (2.7880, 2.2020, 3.1058) -- cycle;
\fill[blue!21.3, opacity=0.7] (2.7880, 2.2020, 3.1058) -- (2.8340, 2.2020, 3.0995) -- (2.8340, 2.2560, 3.0955) -- (2.7880, 2.2560, 3.1017) -- cycle;
\fill[blue!27.3, opacity=0.7] (2.7880, 2.2560, 3.1017) -- (2.8340, 2.2560, 3.0955) -- (2.8340, 2.3100, 3.0911) -- (2.7880, 2.3100, 3.0974) -- cycle;
\fill[blue!37.6, opacity=0.7] (2.7880, 2.3100, 3.0974) -- (2.8340, 2.3100, 3.0911) -- (2.8340, 2.3640, 3.0866) -- (2.7880, 2.3640, 3.0928) -- cycle;
\fill[blue!49.0, opacity=0.7] (2.7880, 2.3640, 3.0928) -- (2.8340, 2.3640, 3.0866) -- (2.8340, 2.4180, 3.0818) -- (2.7880, 2.4180, 3.0881) -- cycle;
\fill[blue!56.5, opacity=0.7] (2.7880, 2.4180, 3.0881) -- (2.8340, 2.4180, 3.0818) -- (2.8340, 2.4720, 3.0768) -- (2.7880, 2.4720, 3.0831) -- cycle;
\fill[blue!58.5, opacity=0.7] (2.7880, 2.4720, 3.0831) -- (2.8340, 2.4720, 3.0768) -- (2.8340, 2.5260, 3.0716) -- (2.7880, 2.5260, 3.0779) -- cycle;
\fill[blue!54.9, opacity=0.7] (2.7880, 2.5260, 3.0779) -- (2.8340, 2.5260, 3.0716) -- (2.8340, 2.5800, 3.0663) -- (2.7880, 2.5800, 3.0725) -- cycle;
\fill[blue!43.8, opacity=0.7] (2.7880, 2.5800, 3.0725) -- (2.8340, 2.5800, 3.0663) -- (2.8340, 2.6340, 3.0608) -- (2.7880, 2.6340, 3.0670) -- cycle;
\fill[blue!28.3, opacity=0.7] (2.7880, 2.6340, 3.0670) -- (2.8340, 2.6340, 3.0608) -- (2.8340, 2.6880, 3.0551) -- (2.7880, 2.6880, 3.0614) -- cycle;
\fill[blue!18.7, opacity=0.7] (2.7880, 2.6880, 3.0614) -- (2.8340, 2.6880, 3.0551) -- (2.8340, 2.7420, 3.0493) -- (2.7880, 2.7420, 3.0555) -- cycle;
\fill[blue!15.8, opacity=0.7] (2.7880, 2.7420, 3.0555) -- (2.8340, 2.7420, 3.0493) -- (2.8340, 2.7960, 3.0434) -- (2.7880, 2.7960, 3.0496) -- cycle;
\fill[blue!15.3, opacity=0.7] (2.7880, 2.7960, 3.0496) -- (2.8340, 2.7960, 3.0434) -- (2.8340, 2.8500, 3.0373) -- (2.7880, 2.8500, 3.0436) -- cycle;
\fill[blue!15.2, opacity=0.7] (2.7880, 2.8500, 3.0436) -- (2.8340, 2.8500, 3.0373) -- (2.8340, 2.9040, 3.0312) -- (2.7880, 2.9040, 3.0375) -- cycle;
\fill[blue!15.5, opacity=0.7] (2.7880, 2.9040, 3.0375) -- (2.8340, 2.9040, 3.0312) -- (2.8340, 2.9580, 3.0251) -- (2.7880, 2.9580, 3.0313) -- cycle;
\fill[blue!17.0, opacity=0.7] (2.7880, 2.9580, 3.0313) -- (2.8340, 2.9580, 3.0251) -- (2.8340, 3.0120, 3.0188) -- (2.7880, 3.0120, 3.0251) -- cycle;
\fill[blue!22.2, opacity=0.7] (2.7880, 3.0120, 3.0251) -- (2.8340, 3.0120, 3.0188) -- (2.8340, 3.0660, 3.0126) -- (2.7880, 3.0660, 3.0188) -- cycle;
\fill[blue!30.4, opacity=0.7] (2.7880, 3.0660, 3.0188) -- (2.8340, 3.0660, 3.0126) -- (2.8340, 3.1200, 3.0063) -- (2.7880, 3.1200, 3.0125) -- cycle;
\fill[blue!15.0, opacity=0.7] (2.8340, -0.1200, 3.0063) -- (2.8800, -0.1200, 3.0000) -- (2.8800, -0.0660, 3.0063) -- (2.8340, -0.0660, 3.0126) -- cycle;
\fill[blue!15.1, opacity=0.7] (2.8340, -0.0660, 3.0126) -- (2.8800, -0.0660, 3.0063) -- (2.8800, -0.0120, 3.0125) -- (2.8340, -0.0120, 3.0188) -- cycle;
\fill[blue!15.7, opacity=0.7] (2.8340, -0.0120, 3.0188) -- (2.8800, -0.0120, 3.0125) -- (2.8800, 0.0420, 3.0188) -- (2.8340, 0.0420, 3.0251) -- cycle;
\fill[blue!18.3, opacity=0.7] (2.8340, 0.0420, 3.0251) -- (2.8800, 0.0420, 3.0188) -- (2.8800, 0.0960, 3.0249) -- (2.8340, 0.0960, 3.0312) -- cycle;
\fill[blue!26.1, opacity=0.7] (2.8340, 0.0960, 3.0312) -- (2.8800, 0.0960, 3.0249) -- (2.8800, 0.1500, 3.0311) -- (2.8340, 0.1500, 3.0373) -- cycle;
\fill[blue!37.5, opacity=0.7] (2.8340, 0.1500, 3.0373) -- (2.8800, 0.1500, 3.0311) -- (2.8800, 0.2040, 3.0371) -- (2.8340, 0.2040, 3.0434) -- cycle;
\fill[blue!46.1, opacity=0.7] (2.8340, 0.2040, 3.0434) -- (2.8800, 0.2040, 3.0371) -- (2.8800, 0.2580, 3.0430) -- (2.8340, 0.2580, 3.0493) -- cycle;
\fill[blue!48.7, opacity=0.7] (2.8340, 0.2580, 3.0493) -- (2.8800, 0.2580, 3.0430) -- (2.8800, 0.3120, 3.0488) -- (2.8340, 0.3120, 3.0551) -- cycle;
\fill[blue!45.6, opacity=0.7] (2.8340, 0.3120, 3.0551) -- (2.8800, 0.3120, 3.0488) -- (2.8800, 0.3660, 3.0545) -- (2.8340, 0.3660, 3.0608) -- cycle;
\fill[blue!38.2, opacity=0.7] (2.8340, 0.3660, 3.0608) -- (2.8800, 0.3660, 3.0545) -- (2.8800, 0.4200, 3.0600) -- (2.8340, 0.4200, 3.0663) -- cycle;
\fill[blue!29.6, opacity=0.7] (2.8340, 0.4200, 3.0663) -- (2.8800, 0.4200, 3.0600) -- (2.8800, 0.4740, 3.0654) -- (2.8340, 0.4740, 3.0716) -- cycle;
\fill[blue!22.9, opacity=0.7] (2.8340, 0.4740, 3.0716) -- (2.8800, 0.4740, 3.0654) -- (2.8800, 0.5280, 3.0705) -- (2.8340, 0.5280, 3.0768) -- cycle;
\fill[blue!19.1, opacity=0.7] (2.8340, 0.5280, 3.0768) -- (2.8800, 0.5280, 3.0705) -- (2.8800, 0.5820, 3.0755) -- (2.8340, 0.5820, 3.0818) -- cycle;
\fill[blue!17.3, opacity=0.7] (2.8340, 0.5820, 3.0818) -- (2.8800, 0.5820, 3.0755) -- (2.8800, 0.6360, 3.0803) -- (2.8340, 0.6360, 3.0866) -- cycle;
\fill[blue!16.6, opacity=0.7] (2.8340, 0.6360, 3.0866) -- (2.8800, 0.6360, 3.0803) -- (2.8800, 0.6900, 3.0849) -- (2.8340, 0.6900, 3.0911) -- cycle;
\fill[blue!16.4, opacity=0.7] (2.8340, 0.6900, 3.0911) -- (2.8800, 0.6900, 3.0849) -- (2.8800, 0.7440, 3.0892) -- (2.8340, 0.7440, 3.0955) -- cycle;
\fill[blue!16.5, opacity=0.7] (2.8340, 0.7440, 3.0955) -- (2.8800, 0.7440, 3.0892) -- (2.8800, 0.7980, 3.0933) -- (2.8340, 0.7980, 3.0995) -- cycle;
\fill[blue!17.0, opacity=0.7] (2.8340, 0.7980, 3.0995) -- (2.8800, 0.7980, 3.0933) -- (2.8800, 0.8520, 3.0971) -- (2.8340, 0.8520, 3.1034) -- cycle;
\fill[blue!17.9, opacity=0.7] (2.8340, 0.8520, 3.1034) -- (2.8800, 0.8520, 3.0971) -- (2.8800, 0.9060, 3.1006) -- (2.8340, 0.9060, 3.1069) -- cycle;
\fill[blue!19.4, opacity=0.7] (2.8340, 0.9060, 3.1069) -- (2.8800, 0.9060, 3.1006) -- (2.8800, 0.9600, 3.1039) -- (2.8340, 0.9600, 3.1102) -- cycle;
\fill[blue!21.7, opacity=0.7] (2.8340, 0.9600, 3.1102) -- (2.8800, 0.9600, 3.1039) -- (2.8800, 1.0140, 3.1069) -- (2.8340, 1.0140, 3.1132) -- cycle;
\fill[blue!24.7, opacity=0.7] (2.8340, 1.0140, 3.1132) -- (2.8800, 1.0140, 3.1069) -- (2.8800, 1.0680, 3.1096) -- (2.8340, 1.0680, 3.1159) -- cycle;
\fill[blue!28.0, opacity=0.7] (2.8340, 1.0680, 3.1159) -- (2.8800, 1.0680, 3.1096) -- (2.8800, 1.1220, 3.1120) -- (2.8340, 1.1220, 3.1183) -- cycle;
\fill[blue!31.4, opacity=0.7] (2.8340, 1.1220, 3.1183) -- (2.8800, 1.1220, 3.1120) -- (2.8800, 1.1760, 3.1141) -- (2.8340, 1.1760, 3.1204) -- cycle;
\fill[blue!34.5, opacity=0.7] (2.8340, 1.1760, 3.1204) -- (2.8800, 1.1760, 3.1141) -- (2.8800, 1.2300, 3.1159) -- (2.8340, 1.2300, 3.1222) -- cycle;
\fill[blue!36.9, opacity=0.7] (2.8340, 1.2300, 3.1222) -- (2.8800, 1.2300, 3.1159) -- (2.8800, 1.2840, 3.1174) -- (2.8340, 1.2840, 3.1237) -- cycle;
\fill[blue!38.5, opacity=0.7] (2.8340, 1.2840, 3.1237) -- (2.8800, 1.2840, 3.1174) -- (2.8800, 1.3380, 3.1185) -- (2.8340, 1.3380, 3.1248) -- cycle;
\fill[blue!39.1, opacity=0.7] (2.8340, 1.3380, 3.1248) -- (2.8800, 1.3380, 3.1185) -- (2.8800, 1.3920, 3.1193) -- (2.8340, 1.3920, 3.1256) -- cycle;
\fill[blue!38.8, opacity=0.7] (2.8340, 1.3920, 3.1256) -- (2.8800, 1.3920, 3.1193) -- (2.8800, 1.4460, 3.1198) -- (2.8340, 1.4460, 3.1261) -- cycle;
\fill[blue!37.5, opacity=0.7] (2.8340, 1.4460, 3.1261) -- (2.8800, 1.4460, 3.1198) -- (2.8800, 1.5000, 3.1200) -- (2.8340, 1.5000, 3.1263) -- cycle;
\fill[blue!35.3, opacity=0.7] (2.8340, 1.5000, 3.1263) -- (2.8800, 1.5000, 3.1200) -- (2.8800, 1.5540, 3.1198) -- (2.8340, 1.5540, 3.1261) -- cycle;
\fill[blue!32.4, opacity=0.7] (2.8340, 1.5540, 3.1261) -- (2.8800, 1.5540, 3.1198) -- (2.8800, 1.6080, 3.1193) -- (2.8340, 1.6080, 3.1256) -- cycle;
\fill[blue!29.1, opacity=0.7] (2.8340, 1.6080, 3.1256) -- (2.8800, 1.6080, 3.1193) -- (2.8800, 1.6620, 3.1185) -- (2.8340, 1.6620, 3.1248) -- cycle;
\fill[blue!25.8, opacity=0.7] (2.8340, 1.6620, 3.1248) -- (2.8800, 1.6620, 3.1185) -- (2.8800, 1.7160, 3.1174) -- (2.8340, 1.7160, 3.1237) -- cycle;
\fill[blue!22.8, opacity=0.7] (2.8340, 1.7160, 3.1237) -- (2.8800, 1.7160, 3.1174) -- (2.8800, 1.7700, 3.1159) -- (2.8340, 1.7700, 3.1222) -- cycle;
\fill[blue!20.5, opacity=0.7] (2.8340, 1.7700, 3.1222) -- (2.8800, 1.7700, 3.1159) -- (2.8800, 1.8240, 3.1141) -- (2.8340, 1.8240, 3.1204) -- cycle;
\fill[blue!18.8, opacity=0.7] (2.8340, 1.8240, 3.1204) -- (2.8800, 1.8240, 3.1141) -- (2.8800, 1.8780, 3.1120) -- (2.8340, 1.8780, 3.1183) -- cycle;
\fill[blue!17.9, opacity=0.7] (2.8340, 1.8780, 3.1183) -- (2.8800, 1.8780, 3.1120) -- (2.8800, 1.9320, 3.1096) -- (2.8340, 1.9320, 3.1159) -- cycle;
\fill[blue!17.5, opacity=0.7] (2.8340, 1.9320, 3.1159) -- (2.8800, 1.9320, 3.1096) -- (2.8800, 1.9860, 3.1069) -- (2.8340, 1.9860, 3.1132) -- cycle;
\fill[blue!17.6, opacity=0.7] (2.8340, 1.9860, 3.1132) -- (2.8800, 1.9860, 3.1069) -- (2.8800, 2.0400, 3.1039) -- (2.8340, 2.0400, 3.1102) -- cycle;
\fill[blue!18.4, opacity=0.7] (2.8340, 2.0400, 3.1102) -- (2.8800, 2.0400, 3.1039) -- (2.8800, 2.0940, 3.1006) -- (2.8340, 2.0940, 3.1069) -- cycle;
\fill[blue!20.6, opacity=0.7] (2.8340, 2.0940, 3.1069) -- (2.8800, 2.0940, 3.1006) -- (2.8800, 2.1480, 3.0971) -- (2.8340, 2.1480, 3.1034) -- cycle;
\fill[blue!25.2, opacity=0.7] (2.8340, 2.1480, 3.1034) -- (2.8800, 2.1480, 3.0971) -- (2.8800, 2.2020, 3.0933) -- (2.8340, 2.2020, 3.0995) -- cycle;
\fill[blue!33.3, opacity=0.7] (2.8340, 2.2020, 3.0995) -- (2.8800, 2.2020, 3.0933) -- (2.8800, 2.2560, 3.0892) -- (2.8340, 2.2560, 3.0955) -- cycle;
\fill[blue!43.8, opacity=0.7] (2.8340, 2.2560, 3.0955) -- (2.8800, 2.2560, 3.0892) -- (2.8800, 2.3100, 3.0849) -- (2.8340, 2.3100, 3.0911) -- cycle;
\fill[blue!53.0, opacity=0.7] (2.8340, 2.3100, 3.0911) -- (2.8800, 2.3100, 3.0849) -- (2.8800, 2.3640, 3.0803) -- (2.8340, 2.3640, 3.0866) -- cycle;
\fill[blue!57.6, opacity=0.7] (2.8340, 2.3640, 3.0866) -- (2.8800, 2.3640, 3.0803) -- (2.8800, 2.4180, 3.0755) -- (2.8340, 2.4180, 3.0818) -- cycle;
\fill[blue!57.6, opacity=0.7] (2.8340, 2.4180, 3.0818) -- (2.8800, 2.4180, 3.0755) -- (2.8800, 2.4720, 3.0705) -- (2.8340, 2.4720, 3.0768) -- cycle;
\fill[blue!52.2, opacity=0.7] (2.8340, 2.4720, 3.0768) -- (2.8800, 2.4720, 3.0705) -- (2.8800, 2.5260, 3.0654) -- (2.8340, 2.5260, 3.0716) -- cycle;
\fill[blue!39.8, opacity=0.7] (2.8340, 2.5260, 3.0716) -- (2.8800, 2.5260, 3.0654) -- (2.8800, 2.5800, 3.0600) -- (2.8340, 2.5800, 3.0663) -- cycle;
\fill[blue!25.7, opacity=0.7] (2.8340, 2.5800, 3.0663) -- (2.8800, 2.5800, 3.0600) -- (2.8800, 2.6340, 3.0545) -- (2.8340, 2.6340, 3.0608) -- cycle;
\fill[blue!18.0, opacity=0.7] (2.8340, 2.6340, 3.0608) -- (2.8800, 2.6340, 3.0545) -- (2.8800, 2.6880, 3.0488) -- (2.8340, 2.6880, 3.0551) -- cycle;
\fill[blue!15.7, opacity=0.7] (2.8340, 2.6880, 3.0551) -- (2.8800, 2.6880, 3.0488) -- (2.8800, 2.7420, 3.0430) -- (2.8340, 2.7420, 3.0493) -- cycle;
\fill[blue!15.3, opacity=0.7] (2.8340, 2.7420, 3.0493) -- (2.8800, 2.7420, 3.0430) -- (2.8800, 2.7960, 3.0371) -- (2.8340, 2.7960, 3.0434) -- cycle;
\fill[blue!15.2, opacity=0.7] (2.8340, 2.7960, 3.0434) -- (2.8800, 2.7960, 3.0371) -- (2.8800, 2.8500, 3.0311) -- (2.8340, 2.8500, 3.0373) -- cycle;
\fill[blue!15.5, opacity=0.7] (2.8340, 2.8500, 3.0373) -- (2.8800, 2.8500, 3.0311) -- (2.8800, 2.9040, 3.0249) -- (2.8340, 2.9040, 3.0312) -- cycle;
\fill[blue!16.8, opacity=0.7] (2.8340, 2.9040, 3.0312) -- (2.8800, 2.9040, 3.0249) -- (2.8800, 2.9580, 3.0188) -- (2.8340, 2.9580, 3.0251) -- cycle;
\fill[blue!21.4, opacity=0.7] (2.8340, 2.9580, 3.0251) -- (2.8800, 2.9580, 3.0188) -- (2.8800, 3.0120, 3.0125) -- (2.8340, 3.0120, 3.0188) -- cycle;
\fill[blue!29.3, opacity=0.7] (2.8340, 3.0120, 3.0188) -- (2.8800, 3.0120, 3.0125) -- (2.8800, 3.0660, 3.0063) -- (2.8340, 3.0660, 3.0126) -- cycle;
\fill[blue!33.5, opacity=0.7] (2.8340, 3.0660, 3.0126) -- (2.8800, 3.0660, 3.0063) -- (2.8800, 3.1200, 3.0000) -- (2.8340, 3.1200, 3.0063) -- cycle;

\fi

% Draw slice edges - corners offset by vertical curve
\pgfmathtruncatemacro{\startslice}{\xoffset == 0 ? \nslices : 0}
\foreach \i in {\startslice,...,\nslices} {
    \pgfmathsetmacro{\zbase}{\i*\sliceheight}
    % Vertical curve offset at this height
    \pgfmathsetmacro{\vcurve}{0.5*\curvature*(1 - cos(\zbase*180/\zmax))}

    % Front edge (y=0): interpolate x offset from +vcurve at x=0 to -vcurve at x=xsize
    \draw[black!60, thick] ({0 + \vcurve}, {0 - \vcurve}, \zbase)
        \foreach \xx in {0,0.1,...,\xsize} {
            -- ({\xx + \vcurve*(1 - 2*\xx/\xsize)}, {0 - \vcurve}, {\zbase + \curvature*sin(\xx*180/\xsize)})
        };

    % Right edge (x=xsize): interpolate y offset from -vcurve at y=0 to +vcurve at y=ysize
    \draw[black!60, thick] ({\xsize - \vcurve}, {0 - \vcurve}, {\zbase + \curvature*sin(180)})
        \foreach \yy in {0,0.1,...,\ysize} {
            -- ({\xsize - \vcurve}, {\yy + \vcurve*(-1 + 2*\yy/\ysize)}, {\zbase + \curvature*sin(180) + \curvature*sin(\yy*180/\ysize)})
        };

    % Back edge (y=ysize): interpolate x offset from +vcurve at x=0 to -vcurve at x=xsize
    \draw[black!40, thick] ({0 + \vcurve}, {\ysize + \vcurve}, {\zbase + \curvature*sin(180)})
        \foreach \xx in {0,0.1,...,\xsize} {
            -- ({\xx + \vcurve*(1 - 2*\xx/\xsize)}, {\ysize + \vcurve}, {\zbase + \curvature*sin(\xx*180/\xsize) + \curvature*sin(180)})
        };

    % Left edge (x=0): interpolate y offset from -vcurve at y=0 to +vcurve at y=ysize
    \draw[black!40, thick] ({0 + \vcurve}, {0 - \vcurve}, \zbase)
        \foreach \yy in {0,0.1,...,\ysize} {
            -- ({0 + \vcurve}, {\yy + \vcurve*(-1 + 2*\yy/\ysize)}, {\zbase + \curvature*sin(\yy*180/\ysize)})
        };
}

% Circular orbit and black holes - positioned diagonally along corner-to-corner line
\pgfmathsetmacro{\orbitradius}{0.5}
\pgfmathsetmacro{\orbitz}{\zmax + 1.0}
% Diagonal angle: from (0,0) to (xsize,ysize) is 45 degrees in the xy plane
\pgfmathsetmacro{\diagangle}{135}
% Black hole positions on the circle at opposite ends of diagonal
\node[circle, fill=black, inner sep=3.5pt] (leftbh) at ({\xsize/2 + \orbitradius*cos(\diagangle+180)}, {\ysize/2 + \orbitradius*sin(\diagangle+180)}, \orbitz) {};
\node[circle, fill=black, inner sep=2.5pt] (rightbh) at ({\xsize/2 + \orbitradius*cos(\diagangle)}, {\ysize/2 + \orbitradius*sin(\diagangle)}, \orbitz) {};
% Draw the orbit circle
\draw[very thick, dotted] ({\xsize/2 + \orbitradius}, \ysize/2, \orbitz)
    \foreach \a in {10,20,...,360} {
        -- ({\xsize/2 + \orbitradius*cos(\a)}, {\ysize/2 + \orbitradius*sin(\a)}, \orbitz)
    };

% x and y axes on bottom edges (curved like foliation, smooth extension)
\pgfmathsetmacro{\xext}{\xsize + 0.6}
\pgfmathsetmacro{\yext}{\ysize + 1.0}
\draw[->, very thick] (0, 0, 0)
    \foreach \xx in {0,0.1,...,\xext} {
        -- (\xx, 0, {\curvature*sin(\xx*180/\xsize)})
    } node[right] {\sffamily\bfseries x};
\draw[->, very thick] (\xsize, 0, {\curvature*sin(180)})
    \foreach \yy in {0,0.1,...,\yext} {
        -- (\xsize, \yy, {\curvature*sin(180) + \curvature*sin(\yy*180/\ysize)})
    } node[above] {\sffamily\bfseries y};

% Time arrow (on back right vertical edge, slightly curved - starts vertical, curves inward)
\pgfmathsetmacro{\tarrowext}{\nslices*\sliceheight + 0.5}
\draw[->, very thick] (\xsize, \ysize, 0)
    \foreach \zz in {0,0.1,...,\tarrowext} {
        -- ({\xsize - 0.5*\curvature*(1 - cos(\zz*180/\zmax))}, {\ysize + 0.5*\curvature*(1 - cos(\zz*180/\zmax))}, \zz)
    } node[above] {\sffamily\bfseries t};

\end{scope}
}

% Hamilcar title in the center
\node at (8.5, 1.5, 2.2) {\scalebox{5}{\sffamily\bfseries Hamilcar}};

% Subtitle below - scaled to match Hamilcar width exactly
% Using tabular for natural width, then resizebox to target width
\node[anchor=north] at (8.5, 1.5, 1.4) {%
    \resizebox{7.2cm}{!}{%
        \begin{tabular}[t]{@{}l@{}}
        \sffamily\bfseries Hamiltonian (canonical)\\
        \sffamily\bfseries analysis toolkit for xAct
        \end{tabular}%
    }%
};

\end{tikzpicture}
\end{document}
